%%%%%%%%%%%%%%%%%%%%%%%%%%%%%%%%%
% LaTeX model https://hurlus.fr %
%%%%%%%%%%%%%%%%%%%%%%%%%%%%%%%%%

% Needed before document class
\RequirePackage{pdftexcmds} % needed for tests expressions
\RequirePackage{fix-cm} % correct units

% Define mode
\def\mode{a4}

\newif\ifaiv % a4
\newif\ifav % a5
\newif\ifbooklet % booklet
\newif\ifcover % cover for booklet

\ifnum \strcmp{\mode}{cover}=0
  \covertrue
\else\ifnum \strcmp{\mode}{booklet}=0
  \booklettrue
\else\ifnum \strcmp{\mode}{a5}=0
  \avtrue
\else
  \aivtrue
\fi\fi\fi

\ifbooklet % do not enclose with {}
  \documentclass[french,twoside]{book} % ,notitlepage
  \usepackage[%
    papersize={105mm, 297mm},
    inner=12mm,
    outer=12mm,
    top=20mm,
    bottom=15mm,
    marginparsep=0pt,
  ]{geometry}
  \usepackage[fontsize=9.5pt]{scrextend} % for Roboto
\else\ifav
  \documentclass[french,twoside]{book} % ,notitlepage
  \usepackage[%
    a5paper,
    inner=25mm,
    outer=15mm,
    top=15mm,
    bottom=15mm,
    marginparsep=0pt,
  ]{geometry}
  \usepackage[fontsize=12pt]{scrextend}
\else% A4 2 cols
  \documentclass[twocolumn]{report}
  \usepackage[%
    a4paper,
    inner=15mm,
    outer=10mm,
    top=25mm,
    bottom=18mm,
    marginparsep=0pt,
  ]{geometry}
  \setlength{\columnsep}{20mm}
  \usepackage[fontsize=9.5pt]{scrextend}
\fi\fi

%%%%%%%%%%%%%%
% Alignments %
%%%%%%%%%%%%%%
% before teinte macros

\setlength{\arrayrulewidth}{0.2pt}
\setlength{\columnseprule}{\arrayrulewidth} % twocol
\setlength{\parskip}{0pt} % classical para with no margin
\setlength{\parindent}{1.5em}

%%%%%%%%%%
% Colors %
%%%%%%%%%%
% before Teinte macros

\usepackage[dvipsnames]{xcolor}
\definecolor{rubric}{HTML}{800000} % the tonic 0c71c3
\def\columnseprulecolor{\color{rubric}}
\colorlet{borderline}{rubric!30!} % definecolor need exact code
\definecolor{shadecolor}{gray}{0.95}
\definecolor{bghi}{gray}{0.5}

%%%%%%%%%%%%%%%%%
% Teinte macros %
%%%%%%%%%%%%%%%%%
%%%%%%%%%%%%%%%%%%%%%%%%%%%%%%%%%%%%%%%%%%%%%%%%%%%
% <TEI> generic (LaTeX names generated by Teinte) %
%%%%%%%%%%%%%%%%%%%%%%%%%%%%%%%%%%%%%%%%%%%%%%%%%%%
% This template is inserted in a specific design
% It is XeLaTeX and otf fonts

\makeatletter % <@@@


\usepackage{blindtext} % generate text for testing
\usepackage[strict]{changepage} % for modulo 4
\usepackage{contour} % rounding words
\usepackage[nodayofweek]{datetime}
% \usepackage{DejaVuSans} % seems buggy for sffont font for symbols
\usepackage{enumitem} % <list>
\usepackage{etoolbox} % patch commands
\usepackage{fancyvrb}
\usepackage{fancyhdr}
\usepackage{float}
\usepackage{fontspec} % XeLaTeX mandatory for fonts
\usepackage{footnote} % used to capture notes in minipage (ex: quote)
\usepackage{framed} % bordering correct with footnote hack
\usepackage{graphicx}
\usepackage{lettrine} % drop caps
\usepackage{lipsum} % generate text for testing
\usepackage[framemethod=tikz,]{mdframed} % maybe used for frame with footnotes inside
\usepackage{pdftexcmds} % needed for tests expressions
\usepackage{polyglossia} % non-break space french punct, bug Warning: "Failed to patch part"
\usepackage[%
  indentfirst=false,
  vskip=1em,
  noorphanfirst=true,
  noorphanafter=true,
  leftmargin=\parindent,
  rightmargin=0pt,
]{quoting}
\usepackage{ragged2e}
\usepackage{setspace} % \setstretch for <quote>
\usepackage{tabularx} % <table>
\usepackage[explicit]{titlesec} % wear titles, !NO implicit
\usepackage{tikz} % ornaments
\usepackage{tocloft} % styling tocs
\usepackage[fit]{truncate} % used im runing titles
\usepackage{unicode-math}
\usepackage[normalem]{ulem} % breakable \uline, normalem is absolutely necessary to keep \emph
\usepackage{verse} % <l>
\usepackage{xcolor} % named colors
\usepackage{xparse} % @ifundefined
\XeTeXdefaultencoding "iso-8859-1" % bad encoding of xstring
\usepackage{xstring} % string tests
\XeTeXdefaultencoding "utf-8"
\PassOptionsToPackage{hyphens}{url} % before hyperref, which load url package

% TOTEST
% \usepackage{hypcap} % links in caption ?
% \usepackage{marginnote}
% TESTED
% \usepackage{background} % doesn’t work with xetek
% \usepackage{bookmark} % prefers the hyperref hack \phantomsection
% \usepackage[color, leftbars]{changebar} % 2 cols doc, impossible to keep bar left
% \usepackage[utf8x]{inputenc} % inputenc package ignored with utf8 based engines
% \usepackage[sfdefault,medium]{inter} % no small caps
% \usepackage{firamath} % choose firasans instead, firamath unavailable in Ubuntu 21-04
% \usepackage{flushend} % bad for last notes, supposed flush end of columns
% \usepackage[stable]{footmisc} % BAD for complex notes https://texfaq.org/FAQ-ftnsect
% \usepackage{helvet} % not for XeLaTeX
% \usepackage{multicol} % not compatible with too much packages (longtable, framed, memoir…)
% \usepackage[default,oldstyle,scale=0.95]{opensans} % no small caps
% \usepackage{sectsty} % \chapterfont OBSOLETE
% \usepackage{soul} % \ul for underline, OBSOLETE with XeTeX
% \usepackage[breakable]{tcolorbox} % text styling gone, footnote hack not kept with breakable


% Metadata inserted by a program, from the TEI source, for title page and runing heads
\title{\textbf{ Essai général de tactique (extraits) }}
\date{1772}
\author{Guibert, Jacques-Antoine-Hippolyte de (1744?-1790) }
\def\elbibl{Guibert, Jacques-Antoine-Hippolyte de (1744?-1790) . 1772. \emph{Essai général de tactique (extraits)}}
\def\elsource{Jacques-Antoine-Hippolyte de Guibert, {\itshape Essai général de tactique, précédé d’un discours sur l’état actuel de la politique et de la science militaire en Europe avec le plan d’un ouvrage intitulé : « La France politique et militaire »} [1772], Paris, Economica, 2004, XXXV-233 p.}

% Default metas
\newcommand{\colorprovide}[2]{\@ifundefinedcolor{#1}{\colorlet{#1}{#2}}{}}
\colorprovide{rubric}{red}
\colorprovide{silver}{lightgray}
\@ifundefined{syms}{\newfontfamily\syms{DejaVu Sans}}{}
\newif\ifdev
\@ifundefined{elbibl}{% No meta defined, maybe dev mode
  \newcommand{\elbibl}{Titre court ?}
  \newcommand{\elbook}{Titre du livre source ?}
  \newcommand{\elabstract}{Résumé\par}
  \newcommand{\elurl}{http://oeuvres.github.io/elbook/2}
  \author{Éric Lœchien}
  \title{Un titre de test assez long pour vérifier le comportement d’une maquette}
  \date{1566}
  \devtrue
}{}
\let\eltitle\@title
\let\elauthor\@author
\let\eldate\@date


\defaultfontfeatures{
  % Mapping=tex-text, % no effect seen
  Scale=MatchLowercase,
  Ligatures={TeX,Common},
}


% generic typo commands
\newcommand{\astermono}{\medskip\centerline{\color{rubric}\large\selectfont{\syms ✻}}\medskip\par}%
\newcommand{\astertri}{\medskip\par\centerline{\color{rubric}\large\selectfont{\syms ✻\,✻\,✻}}\medskip\par}%
\newcommand{\asterism}{\bigskip\par\noindent\parbox{\linewidth}{\centering\color{rubric}\large{\syms ✻}\\{\syms ✻}\hskip 0.75em{\syms ✻}}\bigskip\par}%

% lists
\newlength{\listmod}
\setlength{\listmod}{\parindent}
\setlist{
  itemindent=!,
  listparindent=\listmod,
  labelsep=0.2\listmod,
  parsep=0pt,
  % topsep=0.2em, % default topsep is best
}
\setlist[itemize]{
  label=—,
  leftmargin=0pt,
  labelindent=1.2em,
  labelwidth=0pt,
}
\setlist[enumerate]{
  label={\bf\color{rubric}\arabic*.},
  labelindent=0.8\listmod,
  leftmargin=\listmod,
  labelwidth=0pt,
}
\newlist{listalpha}{enumerate}{1}
\setlist[listalpha]{
  label={\bf\color{rubric}\alph*.},
  leftmargin=0pt,
  labelindent=0.8\listmod,
  labelwidth=0pt,
}
\newcommand{\listhead}[1]{\hspace{-1\listmod}\emph{#1}}

\renewcommand{\hrulefill}{%
  \leavevmode\leaders\hrule height 0.2pt\hfill\kern\z@}

% General typo
\DeclareTextFontCommand{\textlarge}{\large}
\DeclareTextFontCommand{\textsmall}{\small}

% commands, inlines
\newcommand{\anchor}[1]{\Hy@raisedlink{\hypertarget{#1}{}}} % link to top of an anchor (not baseline)
\newcommand\abbr[1]{#1}
\newcommand{\autour}[1]{\tikz[baseline=(X.base)]\node [draw=rubric,thin,rectangle,inner sep=1.5pt, rounded corners=3pt] (X) {\color{rubric}#1};}
\newcommand\corr[1]{#1}
\newcommand{\ed}[1]{ {\color{silver}\sffamily\footnotesize (#1)} } % <milestone ed="1688"/>
\newcommand\expan[1]{#1}
\newcommand\foreign[1]{\emph{#1}}
\newcommand\gap[1]{#1}
\renewcommand{\LettrineFontHook}{\color{rubric}}
\newcommand{\initial}[2]{\lettrine[lines=2, loversize=0.3, lhang=0.3]{#1}{#2}}
\newcommand{\initialiv}[2]{%
  \let\oldLFH\LettrineFontHook
  % \renewcommand{\LettrineFontHook}{\color{rubric}\ttfamily}
  \IfSubStr{QJ’}{#1}{
    \lettrine[lines=4, lhang=0.2, loversize=-0.1, lraise=0.2]{\smash{#1}}{#2}
  }{\IfSubStr{É}{#1}{
    \lettrine[lines=4, lhang=0.2, loversize=-0, lraise=0]{\smash{#1}}{#2}
  }{\IfSubStr{ÀÂ}{#1}{
    \lettrine[lines=4, lhang=0.2, loversize=-0, lraise=0, slope=0.6em]{\smash{#1}}{#2}
  }{\IfSubStr{A}{#1}{
    \lettrine[lines=4, lhang=0.2, loversize=0.2, slope=0.6em]{\smash{#1}}{#2}
  }{\IfSubStr{V}{#1}{
    \lettrine[lines=4, lhang=0.2, loversize=0.2, slope=-0.5em]{\smash{#1}}{#2}
  }{
    \lettrine[lines=4, lhang=0.2, loversize=0.2]{\smash{#1}}{#2}
  }}}}}
  \let\LettrineFontHook\oldLFH
}
\newcommand{\labelchar}[1]{\textbf{\color{rubric} #1}}
\newcommand{\milestone}[1]{\autour{\footnotesize\color{rubric} #1}} % <milestone n="4"/>
\newcommand\name[1]{#1}
\newcommand\orig[1]{#1}
\newcommand\orgName[1]{#1}
\newcommand\persName[1]{#1}
\newcommand\placeName[1]{#1}
\newcommand{\pn}[1]{\IfSubStr{-—–¶}{#1}% <p n="3"/>
  {\noindent{\bfseries\color{rubric}   ¶  }}
  {{\footnotesize\autour{ #1}  }}}
\newcommand\reg{}
% \newcommand\ref{} % already defined
\newcommand\sic[1]{#1}
\newcommand\surname[1]{\textsc{#1}}
\newcommand\term[1]{\textbf{#1}}

\def\mednobreak{\ifdim\lastskip<\medskipamount
  \removelastskip\nopagebreak\medskip\fi}
\def\bignobreak{\ifdim\lastskip<\bigskipamount
  \removelastskip\nopagebreak\bigskip\fi}

% commands, blocks
\newcommand{\byline}[1]{\bigskip{\RaggedLeft{#1}\par}\bigskip}
\newcommand{\bibl}[1]{{\RaggedLeft{#1}\par\bigskip}}
\newcommand{\biblitem}[1]{{\noindent\hangindent=\parindent   #1\par}}
\newcommand{\dateline}[1]{\medskip{\RaggedLeft{#1}\par}\bigskip}
\newcommand{\labelblock}[1]{\medbreak{\noindent\color{rubric}\bfseries #1}\par\mednobreak}
\newcommand{\salute}[1]{\bigbreak{#1}\par\medbreak}
\newcommand{\signed}[1]{\bigbreak\filbreak{\raggedleft #1\par}\medskip}

% environments for blocks (some may become commands)
\newenvironment{borderbox}{}{} % framing content
\newenvironment{citbibl}{\ifvmode\hfill\fi}{\ifvmode\par\fi }
\newenvironment{docAuthor}{\ifvmode\vskip4pt\fontsize{16pt}{18pt}\selectfont\fi\itshape}{\ifvmode\par\fi }
\newenvironment{docDate}{}{\ifvmode\par\fi }
\newenvironment{docImprint}{\vskip6pt}{\ifvmode\par\fi }
\newenvironment{docTitle}{\vskip6pt\bfseries\fontsize{18pt}{22pt}\selectfont}{\par }
\newenvironment{msHead}{\vskip6pt}{\par}
\newenvironment{msItem}{\vskip6pt}{\par}
\newenvironment{titlePart}{}{\par }


% environments for block containers
\newenvironment{argument}{\itshape\parindent0pt}{\vskip1.5em}
\newenvironment{biblfree}{}{\ifvmode\par\fi }
\newenvironment{bibitemlist}[1]{%
  \list{\@biblabel{\@arabic\c@enumiv}}%
  {%
    \settowidth\labelwidth{\@biblabel{#1}}%
    \leftmargin\labelwidth
    \advance\leftmargin\labelsep
    \@openbib@code
    \usecounter{enumiv}%
    \let\p@enumiv\@empty
    \renewcommand\theenumiv{\@arabic\c@enumiv}%
  }
  \sloppy
  \clubpenalty4000
  \@clubpenalty \clubpenalty
  \widowpenalty4000%
  \sfcode`\.\@m
}%
{\def\@noitemerr
  {\@latex@warning{Empty `bibitemlist' environment}}%
\endlist}
\newenvironment{quoteblock}% may be used for ornaments
  {\begin{quoting}}
  {\end{quoting}}

% table () is preceded and finished by custom command
\newcommand{\tableopen}[1]{%
  \ifnum\strcmp{#1}{wide}=0{%
    \begin{center}
  }
  \else\ifnum\strcmp{#1}{long}=0{%
    \begin{center}
  }
  \else{%
    \begin{center}
  }
  \fi\fi
}
\newcommand{\tableclose}[1]{%
  \ifnum\strcmp{#1}{wide}=0{%
    \end{center}
  }
  \else\ifnum\strcmp{#1}{long}=0{%
    \end{center}
  }
  \else{%
    \end{center}
  }
  \fi\fi
}


% text structure
\newcommand\chapteropen{} % before chapter title
\newcommand\chaptercont{} % after title, argument, epigraph…
\newcommand\chapterclose{} % maybe useful for multicol settings
\setcounter{secnumdepth}{-2} % no counters for hierarchy titles
\setcounter{tocdepth}{5} % deep toc
\markright{\@title} % ???
\markboth{\@title}{\@author} % ???
\renewcommand\tableofcontents{\@starttoc{toc}}
% toclof format
% \renewcommand{\@tocrmarg}{0.1em} % Useless command?
% \renewcommand{\@pnumwidth}{0.5em} % {1.75em}
\renewcommand{\@cftmaketoctitle}{}
\setlength{\cftbeforesecskip}{\z@ \@plus.2\p@}
\renewcommand{\cftchapfont}{}
\renewcommand{\cftchapdotsep}{\cftdotsep}
\renewcommand{\cftchapleader}{\normalfont\cftdotfill{\cftchapdotsep}}
\renewcommand{\cftchappagefont}{\bfseries}
\setlength{\cftbeforechapskip}{0em \@plus\p@}
% \renewcommand{\cftsecfont}{\small\relax}
\renewcommand{\cftsecpagefont}{\normalfont}
% \renewcommand{\cftsubsecfont}{\small\relax}
\renewcommand{\cftsecdotsep}{\cftdotsep}
\renewcommand{\cftsecpagefont}{\normalfont}
\renewcommand{\cftsecleader}{\normalfont\cftdotfill{\cftsecdotsep}}
\setlength{\cftsecindent}{1em}
\setlength{\cftsubsecindent}{2em}
\setlength{\cftsubsubsecindent}{3em}
\setlength{\cftchapnumwidth}{1em}
\setlength{\cftsecnumwidth}{1em}
\setlength{\cftsubsecnumwidth}{1em}
\setlength{\cftsubsubsecnumwidth}{1em}

% footnotes
\newif\ifheading
\newcommand*{\fnmarkscale}{\ifheading 0.70 \else 1 \fi}
\renewcommand\footnoterule{\vspace*{0.3cm}\hrule height \arrayrulewidth width 3cm \vspace*{0.3cm}}
\setlength\footnotesep{1.5\footnotesep} % footnote separator
\renewcommand\@makefntext[1]{\parindent 1.5em \noindent \hb@xt@1.8em{\hss{\normalfont\@thefnmark . }}#1} % no superscipt in foot
\patchcmd{\@footnotetext}{\footnotesize}{\footnotesize\sffamily}{}{} % before scrextend, hyperref


%   see https://tex.stackexchange.com/a/34449/5049
\def\truncdiv#1#2{((#1-(#2-1)/2)/#2)}
\def\moduloop#1#2{(#1-\truncdiv{#1}{#2}*#2)}
\def\modulo#1#2{\number\numexpr\moduloop{#1}{#2}\relax}

% orphans and widows
\clubpenalty=9996
\widowpenalty=9999
\brokenpenalty=4991
\predisplaypenalty=10000
\postdisplaypenalty=1549
\displaywidowpenalty=1602
\hyphenpenalty=400
% Copied from Rahtz but not understood
\def\@pnumwidth{1.55em}
\def\@tocrmarg {2.55em}
\def\@dotsep{4.5}
\emergencystretch 3em
\hbadness=4000
\pretolerance=750
\tolerance=2000
\vbadness=4000
\def\Gin@extensions{.pdf,.png,.jpg,.mps,.tif}
% \renewcommand{\@cite}[1]{#1} % biblio

\usepackage{hyperref} % supposed to be the last one, :o) except for the ones to follow
\urlstyle{same} % after hyperref
\hypersetup{
  % pdftex, % no effect
  pdftitle={\elbibl},
  % pdfauthor={Your name here},
  % pdfsubject={Your subject here},
  % pdfkeywords={keyword1, keyword2},
  bookmarksnumbered=true,
  bookmarksopen=true,
  bookmarksopenlevel=1,
  pdfstartview=Fit,
  breaklinks=true, % avoid long links
  pdfpagemode=UseOutlines,    % pdf toc
  hyperfootnotes=true,
  colorlinks=false,
  pdfborder=0 0 0,
  % pdfpagelayout=TwoPageRight,
  % linktocpage=true, % NO, toc, link only on page no
}

\makeatother % /@@@>
%%%%%%%%%%%%%%
% </TEI> end %
%%%%%%%%%%%%%%


%%%%%%%%%%%%%
% footnotes %
%%%%%%%%%%%%%
\renewcommand{\thefootnote}{\bfseries\textcolor{rubric}{\arabic{footnote}}} % color for footnote marks

%%%%%%%%%
% Fonts %
%%%%%%%%%
\usepackage[]{roboto} % SmallCaps, Regular is a bit bold
% \linespread{0.90} % too compact, keep font natural
\newfontfamily\fontrun[]{Roboto Condensed Light} % condensed runing heads
\ifav
  \setmainfont[
    ItalicFont={Roboto Light Italic},
  ]{Roboto}
\else\ifbooklet
  \setmainfont[
    ItalicFont={Roboto Light Italic},
  ]{Roboto}
\else
\setmainfont[
  ItalicFont={Roboto Italic},
]{Roboto Light}
\fi\fi
\renewcommand{\LettrineFontHook}{\bfseries\color{rubric}}
% \renewenvironment{labelblock}{\begin{center}\bfseries\color{rubric}}{\end{center}}

%%%%%%%%
% MISC %
%%%%%%%%

\setdefaultlanguage[frenchpart=false]{french} % bug on part


\newenvironment{quotebar}{%
    \def\FrameCommand{{\color{rubric!10!}\vrule width 0.5em} \hspace{0.9em}}%
    \def\OuterFrameSep{\itemsep} % séparateur vertical
    \MakeFramed {\advance\hsize-\width \FrameRestore}
  }%
  {%
    \endMakeFramed
  }
\renewenvironment{quoteblock}% may be used for ornaments
  {%
    \savenotes
    \setstretch{0.9}
    \normalfont
    \begin{quotebar}
  }
  {%
    \end{quotebar}
    \spewnotes
  }


\renewcommand{\headrulewidth}{\arrayrulewidth}
\renewcommand{\headrule}{{\color{rubric}\hrule}}

% delicate tuning, image has produce line-height problems in title on 2 lines
\titleformat{name=\chapter} % command
  [display] % shape
  {\vspace{1.5em}\centering} % format
  {} % label
  {0pt} % separator between n
  {}
[{\color{rubric}\huge\textbf{#1}}\bigskip] % after code
% \titlespacing{command}{left spacing}{before spacing}{after spacing}[right]
\titlespacing*{\chapter}{0pt}{-2em}{0pt}[0pt]

\titleformat{name=\section}
  [block]{}{}{}{}
  [\vbox{\color{rubric}\large\raggedleft\textbf{#1}}]
\titlespacing{\section}{0pt}{0pt plus 4pt minus 2pt}{\baselineskip}

\titleformat{name=\subsection}
  [block]
  {}
  {} % \thesection
  {} % separator \arrayrulewidth
  {}
[\vbox{\large\textbf{#1}}]
% \titlespacing{\subsection}{0pt}{0pt plus 4pt minus 2pt}{\baselineskip}

\ifaiv
  \fancypagestyle{main}{%
    \fancyhf{}
    \setlength{\headheight}{1.5em}
    \fancyhead{} % reset head
    \fancyfoot{} % reset foot
    \fancyhead[L]{\truncate{0.45\headwidth}{\fontrun\elbibl}} % book ref
    \fancyhead[R]{\truncate{0.45\headwidth}{ \fontrun\nouppercase\leftmark}} % Chapter title
    \fancyhead[C]{\thepage}
  }
  \fancypagestyle{plain}{% apply to chapter
    \fancyhf{}% clear all header and footer fields
    \setlength{\headheight}{1.5em}
    \fancyhead[L]{\truncate{0.9\headwidth}{\fontrun\elbibl}}
    \fancyhead[R]{\thepage}
  }
\else
  \fancypagestyle{main}{%
    \fancyhf{}
    \setlength{\headheight}{1.5em}
    \fancyhead{} % reset head
    \fancyfoot{} % reset foot
    \fancyhead[RE]{\truncate{0.9\headwidth}{\fontrun\elbibl}} % book ref
    \fancyhead[LO]{\truncate{0.9\headwidth}{\fontrun\nouppercase\leftmark}} % Chapter title, \nouppercase needed
    \fancyhead[RO,LE]{\thepage}
  }
  \fancypagestyle{plain}{% apply to chapter
    \fancyhf{}% clear all header and footer fields
    \setlength{\headheight}{1.5em}
    \fancyhead[L]{\truncate{0.9\headwidth}{\fontrun\elbibl}}
    \fancyhead[R]{\thepage}
  }
\fi

\ifav % a5 only
  \titleclass{\section}{top}
\fi

\newcommand\chapo{{%
  \vspace*{-3em}
  \centering % no vskip ()
  {\Large\addfontfeature{LetterSpace=25}\bfseries{\elauthor}}\par
  \smallskip
  {\large\eldate}\par
  \bigskip
  {\Large\selectfont{\eltitle}}\par
  \bigskip
  {\color{rubric}\hline\par}
  \bigskip
  {\Large TEXTE LIBRE À PARTICPATION LIBRE\par}
  \centerline{\small\color{rubric} {hurlus.fr, tiré le \today}}\par
  \bigskip
}}

\newcommand\cover{{%
  \thispagestyle{empty}
  \centering
  {\LARGE\bfseries{\elauthor}}\par
  \bigskip
  {\Large\eldate}\par
  \bigskip
  \bigskip
  {\LARGE\selectfont{\eltitle}}\par
  \vfill\null
  {\color{rubric}\setlength{\arrayrulewidth}{2pt}\hline\par}
  \vfill\null
  {\Large TEXTE LIBRE À PARTICPATION LIBRE\par}
  \centerline{{\href{https://hurlus.fr}{\dotuline{hurlus.fr}}, tiré le \today}}\par
}}

\begin{document}
\pagestyle{empty}
\ifbooklet{
  \cover\newpage
  \thispagestyle{empty}\hbox{}\newpage
  \cover\newpage\noindent Les voyages de la brochure\par
  \bigskip
  \begin{tabularx}{\textwidth}{l|X|X}
    \textbf{Date} & \textbf{Lieu}& \textbf{Nom/pseudo} \\ \hline
    \rule{0pt}{25cm} &  &   \\
  \end{tabularx}
  \newpage
  \addtocounter{page}{-4}
}\fi

\thispagestyle{empty}
\ifaiv
  \twocolumn[\chapo]
\else
  \chapo
\fi
{\it\elabstract}
\bigskip
\makeatletter\@starttoc{toc}\makeatother % toc without new page
\bigskip

\pagestyle{main} % after style

  \section[{À ma patrie}]{À ma patrie}\renewcommand{\leftmark}{À ma patrie}

\noindent Dédier mon ouvrage à ma Patrie, c’est le consacrer tout ensemble au roi qui en est le père ; aux ministres qui en sont les administrateurs ; à tous les ordres de l’État qui en sont les membres ; à tous les Français qui en sont les enfants. Eh ! puisse-t-on un jour rendre à ce Saint nom de Patrie toute sa signification et son énergie ; en faire le cri de la nation, le ralliement de tout ce qui compose l’État ! Puissent à la fois le maître et les sujets, les grands et les petits, s’honorer du titre de citoyens, s’unir, s’appuyer, s’aimer par lui ! Cette confédération de tous les cœurs et de toutes les forces rendra la France aussi heureuse que je le désire.\par
J’entreprends de tracer le tableau politique et militaire de l’Europe. Je m’attacherai plus particulièrement à l’examen des États qui intéressent ma nation ; je m’arrêterai ensuite sur elle ; je considérerai, sous ce double point de vue, sa constitution, ses moyens, son génie, la situation de son militaire qui sera mon objet principal. J’oserai parler de son administration, dévoiler ses abus, en chercher les remèdes, élever enfin l’édifice immense d’une constitution, à la fois politique et militaire ; d’une discipline nationale ; d’une tactique complète ; me servant pour cela, de tous les matériaux qui existent ; fouillant dans les débris de tous les siècles, et dans les connaissances actuelles de tous les peuples.\par
La vérité conduira ma plume. Sans la vérité, que seraient les hommes ? Elle est à l’univers moral, ce qu’est le soleil à l’univers physique. Elle le féconde et l’éclaire. Sans elle, le génie ne jette qu’une flamme incertaine et trompeuse. Sans elle, les rois, les ministres, les écrivains ne sont que de célèbres aveugles. Je lui dévoue mes travaux. Je parlerai avec la liberté qu’elle inspire ; et, si quelquefois je suis forcé de m’imposer silence sur elle, du moins je proteste de ne rien dire volontairement qui la blesse.\par
Loin de nous ce préjugé qui accuse la philosophie d’éteindre le patriotisme. Elle l’ennoblit. Elle l’empêche de dégénérer en orgueil. Éclairé par elle, le citoyen s’attache à sa nation sans fanatisme, et il ne hait, ou ne méprise pas les autres peuples. Il désire la prospérité de son pays ; et il gémirait de la voir s’élever sur l’esclavage et sur le malheur des pays voisins. Il chérit tous les hommes, comme ses semblables, et s’il porte à ses compatriotes un sentiment de prédilection, c’est celui qu’un frère a pour ses frères. Amour de la patrie, c’est ainsi que tu te fais sentir à mon cœur ! Je pourrai donc être utile à mes citoyens, et ne pas déplaire aux étrangers. Je pourrai écrire pour la France, et être lu du reste de l’Europe.\par
Je ne m’effraie, ni de l’immensité de mon projet, ni de mon âge, ni de la faiblesse de mes talents. Ainsi Colomb, partant pour découvrir un nouveau monde, ne recula point à la vue de l’océan et du frêle vaisseau qui devait le porter. J’ai sa hardiesse, je n’aurai peut-être pas son succès. Mais si je m’égare, si j’embrasse quelquefois la chimère du mieux impossible, qu’on me plaigne, et qu’on me pardonne. Le délire d’un citoyen, qui rêve au bonheur de sa patrie, a quelque chose de respectable.
\section[{Discours préliminaire}]{Discours préliminaire}\renewcommand{\leftmark}{Discours préliminaire}

\subsection[{I. - Tableau de la politique actuelle. Son parallèle avec celle des anciens : ses vices. Obstacles qu’elle apporte à la prospérité et à la grandeur des peuples.}]{I. - Tableau de la politique actuelle. Son parallèle avec celle des anciens : ses vices. Obstacles qu’elle apporte à la prospérité et à la grandeur des peuples.}
\noindent Si l’on entend par politique, l’art de négocier, ou plutôt d’intriguer ; celui de fomenter sourdement quelque révolution, de lier ou de rompre, dans l’obscurité des cabinets, quelques traités d’alliance, de paix, de mariage ou de commerce ; nous sommes sans doute, à cet égard, supérieurs aux anciens. Nous y apportons plus de finesse et plus d’esprit qu’eux. Mais si la politique est la science vaste et sublime de régir un État, au-dedans et au-dehors ; de diriger les intérêts particuliers vers l’intérêt général ; de rendre les peuples heureux, et de les attacher à leurs gouvernements ; convenons qu’elle est totalement inconnue à nos administrateurs modernes, que nos d’Ossat, nos d’Estrades, nos Richelieu, nos Colbert, ne peuvent se comparer aux Lycurgue, aux Périclès, aux Numa, aux grands hommes d’État de la Grèce et de Rome. Convenons que le Sénat Romain, dans le temps de sa splendeur, nous rappelle cet Atlas fabuleux, qui soutenait le fardeau du monde. Tandis que nos gouvernements ne sont que des machines frêles et compliquées, auxquelles la fortune et les circonstances impriment des mouvements irréguliers, incertains et passagers comme elles.\par
Je ne suis point admirateur aveugle des Anciens. Je sais ce qu’une longue suite de siècles, les ténèbres de l’ignorance, le prestige de l’histoire, la prévention de nos esprits, leur prêtent de colossal et de merveilleux. Je sais que, de même que les astres voisins de l’horizon se peignent plus grands à nos yeux, que quand, plus rapprochés de nous, ils s’élèvent sur nos têtes, les héros, les événements que nous apercevons dans le lointain de l’antiquité, acquièrent, à nos regards, une grandeur que n’ont jamais les objets contemporains. Fortifié contre cette illusion, je ne juge presque jamais les choses telles que l’histoire me les représente. Je ne me peins point des hommes au-dessus de l’humanité. Je rabaisse les héros à la mesure possible de perfection que le cœur humain comporte. Je cherche à démêler, dans les événements, l’influence que le hasard a pu avoir sur eux, les ressorts et quelquefois les fils imperceptibles qui en ont été les causes. Ainsi je n’ai point une vénération enthousiaste pour le gouvernement de l’ancienne Rome. Je ne prétends point qu’il ait été parfait. Il ne l’était point, puisqu’il a eu ses secousses, sa décadence et sa fin. Il ne pouvait pas l’être, puisqu’il était l’ouvrage des hommes. Mais si ce gouvernement imprima, pendant cinq cents ans, un caractère de vigueur et de majesté au peuple qui vécut sous lui ; s’il y fit germer plus de citoyens et de héros que le reste de la terre n’en a peut-être porté depuis ; si même, dans le temps de sa corruption, les vices de ce peuple eurent quelquefois une grandeur et une énergie qui forcent à l’étonnement ; si ce peuple enfin devint le maître du monde ; je dois alors attribuer des effets si grands, si soutenus, à des causes puissantes et confiantes. Je puis, sans me tromper, assurer que ce gouvernement était plus vigoureux ; que sa politique était plus vaste, plus profonde que celle de tous les États qui s’offrent à moi.\par
J’admire donc la politique des Romains dans leurs beaux jours, lorsque je la vois fondée sur un plan fixe ; lorsque ce plan a pour base le patriotisme et la vertu ; lorsque je vois Rome naissante, colonie faible et sans appui, devenir rapidement une ville ; s’agrandir sans cesse, vaincre tous ses voisins qui étaient ses ennemis ; s’en faire des citoyens ou des alliés ; se fortifier ainsi en s’étendant, comme un fleuve se grossit par les eaux qu’il reçoit dans son cours. J’admire cette politique, quand je vois Rome n’avoir jamais qu’une guerre à la fois ; ne jamais poser les armes que l’honneur du nom Romain ne soit satisfait ; ne pas s’aveugler par ses succès ; ne pas se laisser abattre par les revers ; devenir la proie des Gaulois et des flammes, et renaître de ses cendres. J’admire Rome enfin, quand j’examine sa constitution militaire, liée à sa constitution politique ; les lois de sa milice ; l’éducation de sa jeunesse ; ses grands hommes passant indifféremment par toutes les charges de l’État, parce qu’ils étaient propres à les remplir toutes ; ses citoyens fiers du nom de leur patrie et se croyant supérieurs aux rois qu’ils étaient accoutumés à vaincre. Je dis que peut-être il y a eu, dans quelque coin de l’univers, une nation obscure et paisible, dont les membres ont été plus heureux ; mais que certainement jamais peuple n’a eu autant de grandeur, autant de gloire, et n’en a autant mérité par son courage et par ses vertus.\par
Maintenant quel tableau offre, en opposition, l’Europe politique, au philosophe qui la contemple ? Des administrations tyranniques, ignorantes, ou faibles ; les forces des nations étouffées sous leurs vices ; les intérêts particuliers prévalant sur le bien public ; les mœurs, ce supplément des lois souvent plus efficace qu’elles, négligées ou corrompues ; l’oppression des peuples réduite en système ; les dépenses des administrations plus fortes que leurs recettes ; les impôts au-dessus des facultés des contribuables ; la population éparse et clairsemée ; les arts de premier besoin négligés pour les arts frivoles ; le luxe minant sourdement tous les États ; les gouvernements enfin indifférents au sort des peuples, et les peuples, par représailles, indifférents aux succès des gouvernements.\par
Fatigué de tant de maux, si le philosophe trouve à reposer sa vue sur des objets plus consolants, c’est sur quelques petits États qui ne sont que des points dans l’Europe, c’est sur quelques vérités morales et politiques qui, filtrant lentement à travers les erreurs, se développeront peu à peu, parviendront peut-être un jour aux hommes principaux des nations, s’assiéront sur les trônes, et rendront la postérité plus heureuse.\par
Tel est particulièrement l’état de malaise et d’anxiété des peuples, sous la plupart des gouvernements qu’ils y vivent avec dégoût et machinalement, et que, s’ils avaient la force de briser les liens qui les attachent, ils se donneraient d’autres lois et d’autres administrateurs. On verrait alors la moitié de l’Allemagne chasser les petits princes sous lesquels elle gémit ; la Castille, l’Aragon, l’Irlande rappeler ses rois ; la Toscane, ses ducs ; la Flandre, ses comtes ; tant d’autres États, leurs anciens souverains qui vivaient au milieu d’eux sans luxe, et du revenu de leurs domaines. On verrait presque toutes les provinces se séparer de leur métropole ; presque tous les gouvernements se dissoudre, ou changer de forme. Mais que dis-je ? Telle est en même temps la faiblesse des peuples, que, mécontents, ils murmurent et restent dans la même situation. Ils y sont enchaînés par l’habitude et par les vices.\par
Cette fermentation impuissante est une des plus grandes preuves de la mauvaise constitution de nos gouvernements. Car, d’une part, les peuples souffrent et se plaignent, de l’autre ils ont perdu toute espèce de ressort. Chacun vit pour soi, cherchant à se mettre à couvert des maux publics, à en profiter, ou à s’étourdir sur eux. Au milieu de cette faiblesse générale, les gouvernements, faibles eux-mêmes, mais par-là féconds en petits moyens, étendent leur autorité, et l’appesantissent. Ils semblent être en guerre secrète avec leurs sujets. Ils en corrompent une partie, pour dominer l’autre. Ils craignent que les lumières ne s’étendent, parce qu’ils savent qu’elles éclairent les peuples sur leurs droits et sur les fautes de ceux qui les gouvernent. Ils fomentent le luxe, parce qu’ils savent que le luxe énerve les courages. Comme ils ont dans leurs mains presque tout l’or des États, ils font de l’or le grand ressort de l’administration. Ils en font le moyen de la considération et de l’avancement des particuliers ; la solde du vice qu’il augmente ; la récompense de la vertu qu’il avilit ; l’objet de la cupidité de tous les citoyens. Ils repompent ensuite, par des opérations fiscales, cet or que leur prodigalité a répandu : circulation funeste, et dont l’effet est de ruiner une partie des nations, pour enchaîner l’autre. C’est enfin cet art malheureux de diviser, d’affaiblir, de dégrader, pour mieux dominer, d’opprimer sans révolter, qu’on appelle science de gouvernement, dans la plupart des cours.\par
Le philosophe sera-t-il plus satisfait, quand il jettera les yeux sur l’Europe militaire ? Il y verra toutes les constitutions servilement calquées les unes sur les autres ; les peuples du Midi ayant la même discipline que ceux du Nord ; le génie des nations en contradiction avec les lois de leur milice ; la profession de soldat abandonnée à la classe la plus vile et la plus misérable des citoyens ; le soldat, sous ses drapeaux continuant d’être malheureux et méprisé ; les armées plus nombreuses à proportion que les nations qui les entretiennent ; onéreuses à ces nations pendant la paix ; ne suffisant pas pour les rassurer à la guerre, parce que le reste du peuple n’est qu’une multitude timide et amollie. Il remarquera, en passant, qu’on a fait quelques progrès sur la tactique, et sur quelques autres branches de l’art militaire. Il admirera quelques morceaux de détail dans nos constitutions, le génie du roi de Prusse, l’essor momentané qu’il a donné à sa nation. Mais il se demandera, où est une milice constituée sur des principes solides ? Où est un peuple guerrier, ennemi du luxe, ami des travaux, et porté à la gloire par ses lois ?\par
N’attribuons en effet qu’en partie à la vigilance actuelle de tous les peuples sur les démarches de leurs voisins, à la correspondance de toutes les cours, au système d’équilibre établi en Europe, l’impossibilité où sont les nations de s’étendre et de conquérir. Elle provient bien plutôt de ce qu’aucune de ces nations n’est décisivement supérieure aux autres, par ses mœurs et sa constitution ; de ce qu’elles sont toutes contenues dans leur sphère, par la faiblesse et la ressemblance de leurs gouvernements.\par
Que peut-il résulter aujourd’hui de nos guerres ? Les États n’ont ni trésor, ni excédent de population. Leurs dépenses de paix sont déjà au-dessus de leurs recettes. Cependant on se déclare la guerre. On entre en campagne avec des armées qu’on ne peut ni recruter, ni payer. Vainqueur ou vaincu, on s’épuise à peu près également. La masse des dettes nationales s’accroît. Le crédit baisse. L’argent manque. Les flottes ne trouvent plus de matelots, ni les armées de soldats. Les ministres, de part et d’autre, sentent qu’il est temps de négocier. La paix se fait. Quelques colonies ou provinces changent de maître. Souvent la source des querelles n’est pas fermée, et chacun reste assis sur ses débris, occupé à payer ses dettes, et à aiguiser ses armes.\par
Mais supposons qu’il s’élevât en Europe, un peuple, qui joignît à des vertus austères et à une milice nationale, un plan fixe d’agrandissement, qui ne perdît pas de vue ce système, qui, sachant faire la guerre à peu de frais et subsister par ses victoires, ne fût pas réduit à poser les armes, par des calculs de finance. On verrait ce peuple subjuguer ses voisins, et renverser nos faibles constitutions, comme l’aquilon plie de frêles roseaux.\par
Ce peuple ne s’élèvera pas, parce qu’il ne reste en Europe aucune nation, à la fois puissante et neuve. Elles s’assimilent et se corrompent toutes, de proche en proche. Elles ont toutes des gouvernements destructifs de tout sentiment de patriotisme et de vertu. Lorsque la corruption a fait de tels progrès, lorsqu’elle a attaqué les principes des administrations, les administrateurs, les cours des souverains, les berceaux de leurs enfants, il est presque impossible d’espérer une régénération. Les lieux, d’où elle pourrait venir, sont le foyer du mal. Un seul peuple était, au commencement de ce siècle, en position de devenir redoutable. Son souverain, qui était un grand homme, mais qu’on admire peut-être trop, n’en a pas profité. Une fausse politique fut la base de son système. Il se hâta trop de polir sa nation. Il fit entrer dans ses États tous les arts de l’Europe ; et, avec les arts, il introduisit les vices. Il appela la Russie, dans l’Ingrie, dans la Livonie, et en rassemblant ainsi ses moyens à une des extrémités de ses États, il jeta dans la langueur, le reste de son empire. Il voulut jouir de son vivant. Il négligea les fruits pour les fleurs. S’il se fût moins pressé de prendre part à la politique de l’Europe, si, en attirant dans son pays les arts utiles, il eût repoussé ceux de luxe et de mollesse ; si, au lieu de bâtir des villes, il eût défriché des campagnes ; si, par trop de fréquentation avec les étrangers, il n’eût pas fait perdre à ses sujets cette âpreté sauvage, avec laquelle ils eussent fait de grandes choses ; s’il n’eût répandu sur sa nation que les lumières nécessaires pour augmenter sa force, et qu’il eût habilement éloigné celles qui pouvaient l’affaiblir ; si, avec un pareil plan, il eût vécu plus longtemps, et que ses successeurs se fussent conduits par les mêmes principes, la Russie serait aujourd’hui bien plus menaçante, et plus redoutable pour l’Europe. De ce vaste empire fussent peut-être sorties de nos jours, des peuplades endurcies et invincibles, qui auraient changé la face de nos contrées, ainsi que, des réservoirs du Nord, se répandirent autrefois ces flots de barbares qui inondèrent l’empire Romain. Ces peuplades eussent paru, avec un langage, des habillements, des armes, des mœurs une manière de faire la guerre, qui, en tout ou en partie, n’auraient pas été les nôtres. Cet appareil nouveau eût, sans doute, contribué à ses victoires.\par
Si l’Europe n’a plus à craindre ces torrents dévastateurs qui la couvrirent autrefois de sang et de ténèbres ; si les vices, qui minent tous ses gouvernements, semblent mettre une sorte d’équilibre entre eux, les nations de cette partie du monde, toutes faibles, toutes corrompues qu’elles font, n’en jouissent pas de plus de tranquillité. Car telle est leur misérable politique, que des haines nationales, des intérêts illusoires de commerce, ou d’ambition, les divisent sans cesse. Même, par les traités qui les pacifient, il reste toujours entre elles des germes de discussions qui, après une trêve périodique, les arment de nouveau l’une contre l’autre. Si leurs fantômes politiques ne leur fournissent pas d’occasions de rupture, les fantaisies des ministres, les vaines étiquettes, les petites intrigues, dans lesquelles consistent aujourd’hui les négociations, en font bientôt naître des prétextes. Tel est enfin le genre de guerre adopté par toutes ces nations, qui consume leurs forces et ne décide pas leurs querelles, que, vainqueur ou vaincu, chacun, à la paix, rentre à peu près dans ses anciennes limites. De là les guerres, effrayant moins les gouvernements, en deviennent plus fréquentes. Ce sont des athlètes timides, couverts de plaies, et toujours armés, qui s’épuisent à s’observer et à se craindre ; s’attaquent de temps en temps, pour s’en imposer mutuellement sur leurs forces ; rendent des combats faibles comme eux ; les suspendent quand leur sang coule ; et conviennent d’une trêve, pour essuyer leurs blessures.\par
Entre ces peuples, dont la faiblesse éternise les querelles, il se peut cependant qu’un jour il y ait des guerres plus décisives, et qui ébranlent les empires. La corruption, répandue sur la surface de l’Europe, ne fait pas partout des progrès égaux. Les différences, qui existent entre les gouvernements, font que, chez les uns, elle se développe plus lentement et chez les autres, avec plus de rapidité. Le mal devient ensuite plus ou moins dangereux, en raison des qualités des hommes qui gouvernent. Ici, de bonnes institutions, un souverain éclairé, un ministre vigoureux, servent de digue contre la corruption, remontent les ressorts du gouvernement, et font rétrograder l’État vers le haut de la roue. Là, gouvernement, souverain, ministres, tout est faible ou corrompu ; par conséquent tout se relâche, se détend, et l’État, entraîné avec une vitesse que sa masse multiplie, descend rapidement vers sa ruine. Supposons ces deux États voisins l’un de l’autre ; que le second ait successivement deux, ou trois souverains faibles ; que le règne de ces souverains, malheureux comme celui de Charles VI, soit long comme celui d’Auguste. Ce dernier État, chancelant, avili, démembré par son voisin, n’attendra plus qu’un orage qui détermine sa chute. Enfin, par une conséquence de la supposition établie ci-dessus, dans la décadence générale, où le luxe et les erreurs politiques mettent toutes les nations, celles qui parcourront le moins rapidement la ligne de leur déclinaison ; celles qui s’arrêteront, ou rétrograderont le plus souvent, dans cette funeste marche, auront sur les autres l’ascendant de vigueur, que la jeunesse a sur la maturité, la maturité sur la vieillesse, la vieillesse sur la décrépitude pour s’affaiblir à leur tour, décliner, et faire place à des États mieux constitués, ou parce que quelque révolution les aura régénérés, ou parce qu’ils seront moins avancés dans leur carrière, ou parce qu’enfin, formés récemment des débris de quelqu’État anéanti, ils auront, pour base, le courage et les vertus qui font prospérer les nouveaux empires.\par
Dans cette situation, quel devrait être le but de la politique des peuples ? Celui de se fortifier au-dedans, plutôt que de chercher à s’étendre au-dehors ; de se resserrer même, s’ils ont des possessions trop étendues ; et de faire, pour ainsi dire, en échange, des conquêtes sur eux-mêmes, en portant toutes les parties de leur administration au plus haut point de perfection ; celui d’augmenter la puissance publique, par les vertus des particuliers ; de travailler sur les lois, sur les mœurs, sur les opinions ; celui en un mot de changer, ou de ralentir le cours funeste qui les entraîne vers leur ruine.\par
S’il est une nation surtout à laquelle convienne cette sage politique et qui doive se hâter de l’embrasser, c’est la mienne, qui heureusement assise au milieu de l’Europe, sous la plus belle température, sur le sol le plus généralement fertile, entourée, presque partout, de limites que la nature semble avoir posées, peut être assez puissante pour ne rien craindre et pour ne rien désirer. C’est la mienne, parce que, si j’ose le dire, c’est elle qui déchoit maintenant avec le plus de rapidité. Son gouvernement ne la soutient pas et les vices, qui, partout ailleurs ne se répandent que par imitation, nés chez elle, y sont plus invétérés, plus destructifs et doivent la dévorer la première.\par
Comme le plan de cette régénération est le but de mon ouvrage, j’y reviendrai avec toute l’attention qu’il mérite. Achevons de peindre tout ce que la politique moderne a d’erroné, et de contraire à la prospérité des peuples.\par
Toutes les parties du gouvernement ont entre elles des rapports immédiats et nécessaires. Ce sont des rameaux du même tronc. Il s’en faut bien cependant, qu’elles soient conduites en conséquence. Dans presque tous les États de l’Europe, les différentes branches d’administration sont dirigées par des ministres particuliers, dont les vues et les intérêts se croisent et se nuisent. Chacun d’eux s’occupe exclusivement de son objet. On dirait que les autres départements appartiennent à une nation étrangère. Heureux encore les États, où ces ministres, jaloux l’un de l’autre, ne se traitent pas en ennemis.\par
Du peu de relation qui existe ainsi entre les différents départements d’une administration, s’ensuivent ces projets, avantageux sous une face, et désavantageux sous les autres ; ces encouragements de commerce, qui découragent l’agriculture ; ces édits financiers qui remplissent le fisc pendant quelques années, et ruinent les peuples pour un siècle ; ces systèmes morcelés ; ces édifices politiques qui n’ont qu’une façade et point de fondements ; ces demi-moyens, ces palliatifs, donc chaque ministre va, plâtrant les maux qu’il aperçoit dans son département, sans calculer si ces remèdes ne seront pas funestes aux autres branches.\par
Jetons les yeux sur l’Europe, et observons plus en détail ces effets funestes. Les ministres espagnols chassent les Maures. Ils oublient que ce sont des hommes, et que, sans une population nombreuse, un État ne peut prospérer. Ils envahissent le nouveau monde, y ouvrent des mines, et ne s’aperçoivent pas que l’Espagne reste en friche. Ils tyrannisent les Pays-Bas, et ne prévoient pas qu’ils vont les révolter, qu’ils ne pourront pas les remettre sous le joug. Faute de calculer qu’au-delà de certaines bornes la grandeur d’un État n’est que faiblesse, faute de savoir sagement se borner à ce qu’on peut vivifier et défendre, ils veulent tout embrasser, Pays-Bas, Franche-Comté, Roussillon, Italie, Portugal, et tout leur échappe.\par
Rapprochons-nous de nos temps, ils ne sont pas plus sages. Richelieu veut étendre le pouvoir de son maître, ou plutôt le sien. Il veut abattre les grands, et détruire ces prérogatives, qui en faisaient les vassaux, plutôt que les sujets des rois. Qu’il se fût servi, pour cela, de moyens vigoureux ; qu’il eût ouvertement attaqué ce que les prétentions de la noblesse pouvaient apporter d’entraves à la force et au bonheur de la monarchie ; qu’il eût étendu l’autorité par l’autorité même, j’admirerais, je bénirais son génie. Mais pour mieux détruire cette noblesse, il la corrompt, il la dégrade, il lui fait quitter ses châteaux. Parce qu’il sent que sa pauvreté et sa simplicité entretiennent sa vigueur, il l’attire à la cour, où il prévoit qu’elle se ruinera par le luxe, et qu’elle dépendra ensuite du souverain par les grâces qu’elle sera réduite à mendier. Ce funeste système est suivi par Louis XIV et par ses ministres. Les mœurs de la nation changent. La dégradation de la noblesse entraîne l’esclavage du peuple. Le fardeau de cette noblesse soudoyée et corrompue retombe sur ce peuple gémissant, autrefois soutenu par elle. Il ne reste bientôt plus, ni esprit national, ni énergie, ni vertus ; et c’est là ce Richelieu, dont le mausolée décore nos temples, dont le lycée de notre éloquence répète sans cesse l’éloge mensonger ; et l’histoire qui devrait être l’asile de la vérité, qui devrait prouver que les statues et les panégyriques sont presque toujours les monuments du préjugé ou de l’adulation, l’histoire éternise cette injuste réputation. Elle appelle sublime la politique de cet ambitieux qui énerva sa nation, croyant fortifier le gouvernement : comme si un bon gouvernement, au lieu d’abaisser sa nation, et de peser sur elle, ne devait pas au contraire chercher à l’élever, en s’élevant du même mouvement avec elle et au-dessus d’elle.\par
Colbert, avec du génie, s’égare sur les vrais intérêts de la France. Il en fait un État mercantile. Il a vu la Hollande s’élever du sein de ses marais et jouer un rôle en Europe. II se dit : « L’or et le commerce sont les mobiles de la prospérité publique. Je suis ministre des finances ; c’est à moi d’enrichir l’État ». Aussitôt les greniers se changent en manufactures, nos laboureurs en artisans. Une branche de l’administration se ranime et fleurit, tandis que le corps de l’arbre languit et se dessèche.\par
Louvois veut la guerre, parce que Colbert veut la paix ; parce que l’intérêt du ministre de la guerre est d’embarrasser le ministre des finances. Il échauffe l’ambition de son maître, il lui dit que la France n’a besoin que d’armées de terre ; qu’au moyen d’elles l’Europe pliera sous ses lois. Bientôt la marine est négligée, les ports se ferment. Toutes les autres parties de l’administration sont sacrifiées à la splendeur d’un seul département.\par
Louis XIV vient d’ajouter quelques provinces à la France. Il croit que, parce que son royaume a augmenté de surface, il s’est accru en puissance. Il prend pour signes d’abondance et de richesse, les étoffes de ses manufactures, et l’or de ses commerçants. Il s’élève à un luxe de puissance plus fort que ses moyens ; croit que, nouveau Cadmus, ses ordonnances d’augmentation font sortir de terre les hommes tout armés ; met tout son peuple en campagne ; épuise la France dans le temps de ses victoires ; la met à deux doigts de sa perte dans ses malheurs ; meurt, et ne laisse après lui que dettes et misère, avec un genre de guerre, moins décisif et plus ruineux.\par
Voyons à l’époque de ce prince, et comme entraînés par son exemple, tous les gouvernements de l’Europe, forcer de moyens ; grossir leurs armées ; augmenter leurs impôts ; étendre, à l’envi, leurs possessions ; appeler les campagnes dans les villes, les provinces dans les capitales, les capitales dans les cours ; prendre l’enflure pour la puissance, le luxe pour la richesse, l’éclat pour la gloire ; faire enfin gémir les peuples, pour atteindre à un agrandissement funeste : politique malheureuse, et qui rappelle ce chevalet sur lequel Busiris allongeait ses victimes, en leur brisant les membres.\par
Les puissances maritimes donnent dans une épidémie de commerce, qui n’est pas moins funeste. Elles veulent embrasser les deux pôles, naviguer sur toutes les mers, arborer leur pavillon sur toutes les côtes. Il s’élève entre elles une politique inconnue jusqu’alors, et digne d’un siècle barbare. Elles se ferment réciproquement leurs ports, ou ne les ouvrent qu’à de certaines denrées, et sous de certains droits. Elles oublient que le genre humain n’est qu’une vaste famille, subdivisée en plusieurs autres, appelées française, anglaise, hollandaise, espagnole, et dont aucune ne peut être pleinement heureuse et puissante, sans une libre et entière correspondance d’échanges, de secours, de bienfaits, et de lumières.\par
Ce serait un tableau bien intéressant et bien instructif, que celui de toutes les fautes qui ont été faites depuis quelques siècles, contre les principes de la saine politique. En s’accoutumant ainsi à examiner l’influence que ces fautes ont eu sur les événements, et les fautes nouvelles dont ces événements ont été la source à leur tour ; en apprenant à démêler la trame de cet enchaînement fatal, on trouverait la solution de la plupart des faits, si mal expliqués par les mots vagues de hasard et de fortune, trop prodigués dans nos histoires.\par
Une cause qui, dans la plupart des gouvernements contribue encore à rendre la politique si imparfaite, c’est la mobilité continuelle des ministères. Eh ! Comment les lumières politiques pourraient-elles s’y perpétuer et s’y étendre ? L’intrigue et le hasard placent et déplacent les ministres. Élevés à ces postes, ils songent plus à les conserver qu’à les remplir. Fatigués par la cabale et l’envie, il ne leur reste ni la force, ni le temps de corriger les vices de l’administration. Le système de leur prédécesseur n’est jamais le leur. Supposons même ces ministres avec du génie. Ils sont hommes, il faut qu’ils se forment des sous-ordres, des principes, un plan. Calculons donc : tant de fautes par leurs erreurs ; tant par leurs passions ; tant par les erreurs et les passions de leurs employés. Sont-ils sans génie ? Ils ne trouvent rien qui les instruise, ou les appuie. L’État n’ayant point de système, ils n’y savent pas suppléer. Ils gouvernent comme ils vivent, du jour à la journée. Au lieu de maîtriser les événements, ils sont maîtrisés par eux. Les détails les absorbent. Ils tiennent dans leurs mains quelques fils de l’administration, et en laissent aller les grands ressorts.\par
L’histoire nous fait voir des rois qui ont gouverné leurs États eux-mêmes, ou des ministres qui ont gouverné leurs maîtres, procurer à leurs nations quelques succès éphémères. Richelieu fit de grandes choses. Louis XIV eut ses éclairs de bonheur. Alberoni parut un moment ranimer l’Espagne. La Prusse élevée au-dessus de sa sphère, par les talents de son roi, étonne aujourd’hui l’Europe. Mais remarquons-le : jamais nation n’a eu de prospérité réelle et durable, que quand, par la nature de son gouvernement, il y a eu un corps permanent, chargé de recueillir les lumières, de réduire les intérêts de l’État en système, de prendre conseil du passé pour l’avenir, de faire, en un mot, sur le tillac de l’État, ce que fait le pilote à la poupe du vaisseau : observer la boussole, les nuages, les vents, les écueils, et tenir route en conséquence. C’est avec ce corps, que les dépositaires de la puissance exécutrice : rois, ministres, dictateurs, consuls, généraux, doivent venir se raccorder, consulter le système général de l’État, et prendre des délibérations. Ainsi était constituée l’ancienne Rome. Ainsi l’est, à quelques égards, l’Angleterre par son parlement : image bien imparfaite d’ailleurs de la majesté et des vertus du sénat romain.\par
Ceci me conduirait à examiner quelle est la forme de gouvernement la plus propre à l’exécution d’un plan de grande et saine politique. Mais c’est une question que je ne veux pas approfondir. Mes lecteurs jugeront suffisamment par l’exposé que je ferai ci-après de ce que devrait être la politique, si un plan, qui doit embrasser toutes les parties de l’administration, la gloire publique et la félicité particulière, le bonheur de la génération présente et celui des générations futures ; qui doit être conduit à sa fin, sans relâche et à travers les événements de plusieurs siècles, peut être raisonnablement confié à un gouvernement qui est entre les mains d’un seul, et dont par conséquent les principes doivent varier, non seulement à tous les changements de règne, mais même à tous les changements de ministère, à toutes les révolutions qui se font dans les caractères, les passions, l’esprit, l’âge, la santé, des souverains et de leurs ministres, à un gouvernement qui, par conséquent, tour à tour vigoureux, faible, éclairé, ignorant, doit tour à tour s’élever, s’abaisser, se relever, décliner, et finir enfin, dans toutes ses secousses convulsives et irrégulières, par perdre son ressort, se briser et s’anéantir.\par
La politique, telle qu’elle s’offre à mes idées, est l’art de gouverner les peuples et, envisagée sous ce vaste point de vue, elle est la science la plus intéressante qui existe. Elle doit avoir pour objet de rendre une nation heureuse au-dedans et de la faire respecter au-dehors. De là, elle se divise naturellement en deux parties : {\itshape politique intérieure} et {\itshape politique extérieure}.\par
La première sert de base à la seconde. Tout ce qui prépare le bonheur et la puissance d’une société est de son ressort : lois, mœurs, coutumes, préjugés, esprit national, justice, police, population, agriculture, commerce, revenus de la nation, dépenses du gouvernement, impôts, application de leur produit. Il faut qu’elle voit tous ces objets avec génie et réflexion ; qu’elle s’élève au-dessus d’eux, pour apercevoir les rapports généraux et l’influence qui les lient les uns aux autres ; qu’elle s’en rapproche ensuite, pour les observer et en suivre les détails ; qu’elle ne s’occupe d’aucun exclusivement aux autres, parce qu’en politique, ce qui fait fleurir trop, ou trop tôt, une branche, épuise souvent et fait languir le rameau voisin, ou une autre branche éloignée. Il faut, en un mot, qu’elle conduise de front toutes les parties de l’administration et, pour cela, qu’elle se forme un système général ; qu’elle l’ait sans cesse devant soi, portant tour à tour les yeux sur lui, pour déterminer les opérations qu’il exige, sur le produit de ces opérations, pour voir s’il concourt à l’exécution du plan général.\par
Tandis que la politique intérieure prépare ainsi et perfectionne tous les moyens du dedans, la politique extérieure examine ce que le résultat de ces moyens peut donner à l’État, de force et de considération au-dehors, et elle détermine, sur cela, son système. C’est à elle à connaître les rapports de toute espèce, qui lient sa nation avec les autres peuples ; à démêler les intérêts illusoires et apparents, d’avec les intérêts réels ; les alliances qui ne peuvent être que passagères et infructueuses, d’avec ces liaisons utiles et permanentes, que dictent la position topographique, ou les avantages respectifs des contractants. C’est à elle à calculer ensuite les forces militaires dont l’État a besoin pour en imposer à ses voisins, pour donner du poids à ses négociations. C’est à elle à constituer ces forces militaires relativement au génie et aux moyens de la nation ; à les constituer surtout, de manière qu’elles ne soient pas au-dessus de ces moyens ; parce qu’alors elles épuisent l’État, et ne lui donnent qu’une puissance factice et ruineuse. C’est à elle à y introduire le meilleur esprit, le plus grand courage, la plus savante discipline ; parce qu’alors elles peuvent être moins nombreuses, et que cette réduction de nombre est un soulagement pour les peuples. Il me semble enfin entendre la politique intérieure, quand elle a préparé le dedans de l’État, disant à la politique extérieure : « Je vous remets une nation heureuse et puissante : ses campagnes sont fécondes, ses denrées sont plus que suffisantes à ses besoins, la population y est nombreuse et encouragée, les lois y sont respectées, les mœurs y sont pures, le vice s’y cache, la vertu s’y montre et n’attend que d’être employée. Achevez mon ouvrage. Faites considérer au-dehors ce peuple que je rends heureux au-dedans. Mettez à profit ce patriotisme que j’ai fait naître dans tous les cœurs, ces vertus guerrières dont j’ai fécondé le germe. Formez des défenseurs à ces moissons ; que leur produit qui n’est point absorbé par mes impôts, ne soit point dévoré par des armées étrangères. Appelez les étrangers dans ses ports. Ouvrez des débouchés à son commerce. Rendez son alliance précieuse. Faites redouter ses armes, et jamais son ambition ».\par
La politique intérieure ayant ainsi préparé une nation, quelles facilités ne trouve pas la politique extérieure, à déterminer le système de ses intérêts vis-à-vis de l’étranger, à former une milice redoutable ! Qu’il est aisé d’avoir des armées invincibles, dans un État où les sujets sont citoyens, où ils chérissent le gouvernement, où ils aiment la gloire, où ils ne craignent point les travaux ! Qu’une nation devenue puissante par ses ressources intérieures, doit en retirer de considération au-dehors ! Qu’alors ses négociations diminuent de complication, et acquièrent de poids ! Que sa manière de les conduire, peut devenir franche et ouverte ! C’est la faiblesse de nos gouvernements qui met, dans leurs négociations, tant d’oblicité et de mauvaise foi. C’est elle qui fomente la division entre les peuples, qui tâche de corrompre réciproquement les membres des administrations. C’est elle qui fait que toutes les nations s’espionnent entre elles ; que les unes soudoient les autres ; qu’elles achètent la paix ; qu’elles se suscitent mutuellement des troubles et des embarras. C’est elle qui dicte ces rivalités en tout genre, basses et nuisibles ; cet empiètement perpétuel du commerce d’une nation sur le commerce de l’autre ; ces lois prohibitives ; ces droits qui repoussent l’étranger ; ces traités qui favorisent une nation, au préjudice des autres ; ces calculs chimériques de balance d’exportation et d’importation : moyens misérables et compliqués, qui, au bout d’un siècle, n’ont rien ajouté à la puissance du gouvernement qui les a le plus adroitement employés. C’est la faiblesse de nos gouvernements, en un mot, qui craint la prospérité des autres nations ; qui voudrait toutes les affaiblir, ou les corrompre : politique semblable à celle qui leur fait affaiblir ou corrompre leurs propres sujets : politique bien différente de celle d’un bon gouvernement qui, sans chercher à contrarier le bonheur et la puissance de ses voisins, tâcherait de s’élever au-dessus d’eux par sa vigueur et par ses vertus.\par
C’est de même la faiblesse de nos gouvernements qui rend nos constitutions militaires si imparfaites et si ruineuses. C’est elle qui, ne pouvant faire des armées citoyennes, les fait si nombreuses. C’est elle qui, ne sachant les récompenser par l’honneur, les paie avec de l’or. C’est elle qui, ne pouvant compter sur le courage et la fidélité des peuples, parce que les peuples sont énervés et mécontents, fait acheter au-dehors des milices stipendiaires. C’est elle qui hérisse les frontières de places. C’est elle enfin qui est occupée à éteindre les vertus guerrières dans les nations, à ne pas même les développer dans les troupes, parce qu’elle craindrait que de là elles ne se répandissent chez les citoyens, et ne les armassent un jour contre les abus qui les oppriment. Je reviendrai, dans l’instant, sur ce qui concerne les constitutions militaires, cette partie de la politique si importante et si négligée. Achevons de dire ce qui empêche nos gouvernements de se conduire d’après les principes de la science vaste et intéressante que je viens de définir.\par
Cette science, envisagée sous le point de vue que j’ai présenté, n’est traitée dans aucun ouvrage. Elle n’est l’objet de l’éducation d’aucun homme principal ; peut-être pas même celui des recherches d’aucun particulier. De là tous les hommes que la fortune porte à la tête des administrations, ne sont pas des hommes d’État. Ils ont tout au plus étudié quelques parties de l’administration. Les autres leur sont inconnues. Ils les dirigent au hasard, et selon la routine établie. L’étude, qu’ils ont faite de quelques parties de l’administration, devient même funeste aux autres parties ; parce qu’alors celles qu’ils connaissent sont à leurs yeux les seules importantes, les seules privilégiées. Ils s’en occupent, à l’exclusion de celles qu’ils ne connaissent pas ; et ces dernières sont abandonnées à des sous-ordres.\par
On objectera peut-être, qu’il est impossible que l’esprit d’un seul homme embrasse toutes les parties d’une science aussi vaste. Comment faisaient donc les Romains qui passaient successivement par toutes les charges de la république ? Comment faisaient ces hommes, tour à tour édiles, questeurs, censeurs, tribuns, pontifes, consuls, généraux ? Ayons des gouvernements qui le veuillent, qui le rendent nécessaire, qui dirigent en conséquence l’éducation publique. Nous aurons de ces esprits supérieurs et universels, qui font la gloire et les destins des empires. D’ailleurs est-ce un homme seul, qui doit conduire tous les détails de l’administration d’un peuple ? Plusieurs concourent à cet important ouvrage. Ils s’attachent chacun au détail d’une partie. Ils les approfondissent, ils les perfectionnent. Du concours des connaissances, répandues sur chaque branche, se forme ainsi, peu à peu, cette masse de lumières, qui éclaire toute l’administration. Au milieu de ces hommes il suffit qu’il s’élève, et il ne peut manquer de s’élever, quelque génie vaste. Celui-là s’empare, si je peux m’exprimer ainsi, des connaissances de tous, crée, ou perfectionne le système politique, se place au haut de la machine, et lui imprime le mouvement. Pour diriger l’ensemble de l’administration, il n’est pas nécessaire qu’il ait approfondi les détails de toutes les parties. Il suffit qu’il connaisse ceux des parties principales, le résultat des autres, la relation que chacune d’elles doit avoir avec le tout. Il suffit que, quand il aura besoin de descendre vers les détails d’une partie, pour éclairer les sous-ordres qui en sont chargés, ou pour la raccorder au système général, il ait ce tact subtil et précieux, qui voit et qui juge.\par
Ainsi, dans la vaste carrière des mathématiques, chacun s’attache à un objet, et poursuit la vérité par des chemins différents. Les Newton, les Leibnitz, les d’Alembert s’élèvent au faîte de la science, planent sur elle, se réservent l’étude des parties les plus difficiles. Mais, chemin faisant, ils voient les progrès des autres branches, ils fixent les opinions, ils répandent leur méthode et leur génie sur la science entière. Ainsi pour me servir d’une autre comparaison plus vaste, qui réponde mieux à l’importance de la science du gouvernement ; dans la hiérarchie de ces intelligences que la mythologie de quelques peuples fait veiller sur l’univers, il y a des génies inférieurs qui sont chargés chacun d’un élément ; et le grand Être, le génie universel, les domine et les dirige.\par
Il faut observer que la politique, en devenant plus parfaite, deviendrait moins difficile. L’imperfection d’une science ajoute presque toujours à sa difficulté. Les ténèbres de l’ignorance, les sophismes des préjugés en enveloppent alors les principes. On les complique, on les multiplie. On croît par la suppléer à leur insuffisance. La base de toutes les opérations étant fausse, les conséquences erronées s’accroissent chaque jour. Elles s’embranchent les unes sur les autres. Bientôt s’élève une théorie d’erreurs, mille fois plus compliquée et plus difficile à saisir, que ne le serait l’enchaînement des vérités, qui forme la science. C’est surtout dans la politique que les déviations ont ces suites rapides et funestes. Quand celte science sera redressée, quand elle portera sur des principes sûrs et immuables, comme la justice et la vertu, elle deviendra simple et lumineuse. Elle rejettera tous ces moyens de détail, ces suppléments, ces palliatifs, dont la faiblesse a surchargé et corrompu toutes les parties de l’administration. En proportion de ce qu’un État sera mieux constitué, de ce qu’il aura plus de puissance réelle, il deviendra plus facile à gouverner. Les États faibles et mal constitués, sont sans cesse le jouet des circonstances et de la fortune. Ils craignent les agitations du dedans et les attaques du dehors. Entraînés par la politique de leurs voisins, ils sont presque toujours obligés de se mouvoir en sens contraire à leurs véritables intérêts. Ce n’est qu’à force de tyrannie, d’adresse, de petits moyens, d’oblicité, de mauvaise foi, qu’ils conservent une existence précaire et languissante. Ils ressemblent à ces faibles bâtiments, hasardés sur le vaste sein des mers. Obligés sans cesse de louvoyer, de changer de manœuvre, de tenir une route opposée à leur but, de respecter tous les vaisseaux qu’ils rencontrent, recherchant leur compagnie, tâchant de se mettre dans leur sillage ; un nuage les alarme, une vague peut les couvrir, un écueil les briser.\par
Il n’en sera pas ainsi d’un État bien constitué et réellement puissant. Je dis, réellement ; parce qu’il faut bien distinguer la puissance véritable, fondée sur la bonne proportion et constitution d’un État, d’avec l’apparence de la puissance, fondée sur une trop grande extension de possessions, sur des triomphes momentanés, sur les talents d’un grand homme, en un mot, sur tout ce qui peut ne pas durer. Un tel État sera facile à gouverner. Sa politique extérieure pourra être uniforme et stable. Il ne craindra rien de ses voisins, il ne voudra rien entreprendre sur eux. Au-dehors il aura la considération qu’inspirent la modération et la force. Sur ces frontières veillera une milice redoutable et citoyenne. Au-dedans prospérera un peuple abondant et vertueux. Que lui importeront les intrigues des autres puissances, les passions des hommes qui les gouvernent, les guerres qui les déchirent ? II ne sera pas jaloux de leur richesse. Il ne le sera pas de leurs conquêtes. Il n’ira pas les troubler dans leurs possessions lointaines. Il sait que trop s’étendre, c’est s’affaiblir ; que des colonies éloignées, si elles fournissent à un commerce de luxe, entretiennent les vices de la métropole ; que, si plus heureuses elles peuvent tout tirer de leur sein, elles se fortifient et se détachent, tôt ou tard, de cette injuste métropole qui veut trop les asservir. Il n’empiétera pas sur leur commerce. Il n’aura besoin ni de règlement ni de traités, ni de calculs de prétendue balance. Il sait que les denrées appellent les échanges ; que, pourvu qu’on leur aplanisse des débouchés, elles s’y portent d’elles-mêmes, et sans avoir besoin d’encouragement. À l’entrée de ses ports, aux barrières de ses frontières seront inscrits ces mots qui formeront tout le code de son commerce : liberté, sûreté, protection. Ces avenues toujours ouvertes, ne se fermeront que pour le luxe et les vices ; et il ne craindra pas que ces poisons funestes s’introduisent en fraude. Il ne se fait de contrebande, que quand il y a des acheteurs ; que quand les objets sont prohibés par la tyrannie du gouvernement, ou par l’avarice du fisc ; que quand le gouvernement, inconséquent et faible, tonne contre elle, et la tolère, ou la favorise en secret. Mais ici la politique intérieure sera vigilante et ferme. Elle aura proscrit, dans l’opinion publique, le luxe et les vices. L’assentiment unanime de la nation les regardera comme les fléaux de sa prospérité. Où se cacheraient-ils dans cette terre qui leur est étrangère ? Dénoncés par tous les citoyens, poursuivis par le gouvernement, ils n’y trouveront point d’asile.\par
Cet État aura rarement à négocier ses voisins. Presque tous les intérêts des autres nations lui seront indifférents. Il aura eu l’art de rendre sa prospérité indépendante d’elles. Peut-être n’entretiendra-t-il point d’ambassadeurs. Mais en revanche, il fera voyager des hommes éclairés, non pour aller épier les moyens de nuire à ses voisins, pour lever le plan de leurs côtes et de leurs places, pour espionner leurs démarches, les secrets de leurs cours, pour corrompre les membres de leur gouvernement ; mais pour étudier, à visage découvert, les hommes, les sciences, les mœurs, les abus, le bien et le mal ; pour donner partout une idée avantageuse de la nation, pour s’y montrer simples, instruits, vertueux, pour rapporter ensuite, à la patrie, le produit de leurs connaissances, comme les abeilles ingénieuses rapportent le suc des fleurs à leur ruche. Il accueillera à son tour les étrangers, et il les recevra, sans jalousie, sans soupçon. Il ne craindra pas qu’ils visitent ses arsenaux, ses ports, ses places, ses troupes. Il n’y a que la faiblesse, ou l’ambition qui cache ses moyens. Un gouvernement, puissant et modéré, laisse voir les siens, sans méfiance et sans ostentation. Il les laisse voir, comme ses chemins, ses villes, ses campagnes, ses peuples : sûr que le spectacle de ses ressources fera désirer son amitié, et redouter ses armes.\par
L’État dont je parle, aura des possessions si rassemblées, si proportionnées à ses moyens de défense, qu’il ne craindra point l’inimitié de ses voisins. Dans un tel État, on ne distinguera ni le centre, ni les extrémités. Toutes les parties seront également florissantes et vigoureuses. Toutes auront entre elles une communication si facile, un rapport si grand d’intérêts, que, là où sera le danger, là se rassembleront bientôt toutes les forces. Il aura une milice nerveuse, supérieure à celle de ses voisins, des citoyens heureux, intéressés à la défense de cette prospérité. Est-ce avec des stipendiaires, avec des troupes constituées comme le sont aujourd’hui toutes celles de l’Europe, qu’on viendra attaquer de tels hommes ? Quelle différence les motifs et les préjugés apporteront dans le courage des deux partis !\par
Si enfin, malgré sa modération, il est offensé dans ses sujets, dans son territoire, dans son honneur, il fera la guerre. Mais lorsqu’il la fera, ce sera avec tous les efforts de sa puissance ; ce sera avec la ferme résolution de ne pas poser les armes, qu’on ne lui ait donné une réparation proportionnée à l’offense. Son genre de guerre ne sera pas même celui que tous les États ont adopté aujourd’hui. Il ne voudra pas conquérir, pour garder ses conquêtes. Il fera plutôt des expéditions, que des établissements. Terrible dans sa colère, il portera chez son ennemi la flamme et le fer. Il épouvantera, par ses vengeances, tous les peuples qui pourraient être tentés de troubler son repos. Et qu’on n’appelle pas barbarie, violation des prétendues lois de la guerre, ces représailles fondées sur les lois de la nature. On est venu insulter ce peuple heureux et pacifique. Il se soulève, il quitte ses foyers. Il périra, jusqu’au dernier, s’il le faut. Mais il obtiendra satisfaction, il se vengera, il assurera, par l’éclat de cette vengeance, son repos futur. Ainsi la justice, modérée, attentive à prévenir le crime, sait, quand le crime est commis, se rendre inexorable, poursuivre le coupable, appesantir sur lui le glaive des lois, et ôter, par l’exemple, aux méchants commencés, la tentation de devenir criminels.\par
Cet État, vigilant à réprimer ses injures, ne sera, par sa politique, l’allié d’aucun peuple ; mais il sera l’ami de tous. II leur portera, sans cesse, des paroles de paix. Il sera, s’il se peut, le médiateur de leurs querelles, non par des vues intéressées, non pour mettre à profit sa médiation, non par rapport à des calculs chimériques de balance de pouvoir. J’ai déjà dit combien toutes ces combinaisons de la politique moderne lui seraient indifférentes. Il offrira son arbitrage, parce que la paix est un bien, et qu’il en connaît le prix ; parce que la guerre interrompt la communication qui doit exister entre les peuples, et qu’à cet égard elle est nuisible aux États qu’elle avoisine. De même les tremblements de terre font sentir leurs contrecoups, hors des limites de leur foyer. Il dira à ses voisins : « Ô peuples ! ô mes frères ! Pourquoi vous déchirer ? Quelle fausse politique vous égare ? Les nations ne sont point nées ennemies. Elles sont les branches d’une même famille. Venez mettre à profit le spectacle de ma prospérité. Venez recueillir mes lumières, apportez-moi les vôtres. Je ne crains point que mes voisins deviennent heureux et puissants. Plus ils le deviendront, plus ils s’attacheront à leur repos. C’est de la félicité publique que naîtra la paix universelle ».\par
Enfin l’État que je peins, aura une administration simple, solide, facile à gouverner. Elle ressemblera à ces vastes machines qui, par des ressorts peu compliqués, produisent de grands effets. La force de cet État naîtra de sa force, sa prospérité de sa prospérité. Le temps, qui détruit tout, augmentera sa puissance. Il démentira ce préjugé vulgaire qui fait imaginer que les empires sont soumis à une loi impérieuse de décadence et de ruine. Si l’on jette les yeux sur l’histoire, cette loi semble exister. Elle est écrite sur les débris de tant de trônes, sur les tombeaux de tant de peuples ; mais elle n’est point irrésistible. Elle ne fait point partie de ce fatalisme qui sans cesse détruit et reproduit l’univers. Qu’un bon gouvernement soit la base d’un empire, qu’il sache maintenir ses principes, l’État s’élèvera toujours jusqu’à ce qu’il ait atteint le point de son ascendance, où est sa plus grande force. Si ce gouvernement est assez habile pour démêler ce point, par-delà lequel son élévation ne ferait que l’affaiblir ; s’il sait l’y arrêter, s’il sait toujours l’y soutenir, l’État fixé à ce faîte de puissance, et inébranlablement affermi sur la mer orageuse des destins, pourra voir les événements et les siècles se briser à ses pieds.\par
Ô ma patrie ! Ce tableau ne sera peut-être pas toujours un rêve fantastique. Tu peux le réaliser. Tu peux devenir cet État fortuné. Un jour peut-être, échappant aux vices de son siècle et placé dans des circonstances plus favorables, il s’élèvera sur ton trône un prince qui opérera cette grande révolution. Dans les écrits de quelques-uns de mes concitoyens, dans les miens peut-être, il en puisera le désir et les moyens. II changera nos mœurs, il retrempera nos âmes, il redonnera du ressort au gouvernement, il portera le flambeau de la vérité dans toutes les parties de l’administration. Il substituera, à notre politique étroite et compliquée, la science vaste et sublime que j’ai tenté de peindre. Alors s’évanouiront ces fausses lumières qui nous égarent ; ces petits talents que nous honorons du nom de génie ; ces préjugés que nous appelons des principes. Alors s’écroulera le système monstrueux et compliqué de nos lois, de nos finances, de notre milice. Alors s’anéantiront, devant cet homme supérieur, les réputations de ces souverains qu’on a encensés, de ces ministres qu’on a crus des hommes d’État. Il rendra la nation ce qu’elle peut devenir. Enfin ayant mis le comble à sa prospérité, ne pouvant plus y ajouter, qu’en la rendant durable, il changera lui-même la forme du gouvernement. Il appellera autour du trône ses peuples, devenus ses enfants ; il leur dira : « Je veux vous rendre heureux après moi. Je vous remets des droits, trop étendus, dont je n’ai point abusé, et dont je ne veux pas que mes successeurs abusent. Je vous appelle à partager avec moi le gouvernement. Je me réserve les honneurs de la couronne, le droit de vous proposer des lois sages, le pouvoir de les faire exécuter, quand vous les aurez ratifiées, l’autorité absolue, la dictature perpétuelle, dans toutes les crises qui menaceront l’État. Voici les statuts de ce gouvernement nouveau. Voici ses lois : je ne vais plus régner que selon elles et par elles. Que ma famille, qui va jurer avec moi, me succède à ces conditions. Recevez nos serments, comme nous allons recevoir les vôtres. Si de part ou d’autre il y a des infracteurs, les lois seront leurs juges ».\par
Quelle politique, que celle qui dicterait à un roi tout puissant cette résolution magnanime ! Eh ! croit-on que ce roi et ses successeurs en fussent moins heureux, en eussent moins d’autorité ? Ce premier créateur d’un peuple nouveau serait adoré de son ouvrage. Ses successeurs, tant qu’ils seraient vertueux, régneraient par le souvenir de leur ancêtre, par l’évidence du bien, par le despotisme des lois : le seul qui affermisse les trônes, qui ne dégrade pas les peuples ; le seul qui soit fait pour les jours de lumière et de philosophie, qui commencent à se lever sur nos têtes.
\subsection[{II. - Tableau de l’art de la guerre, depuis le commencement du monde. Situation actuelle de cette science en Europe. Son parallèle avec ce qu’elle fut autrefois. Nécessité du rapport des constitutions militaires avec les constitutions politiques. Vices de tous nos gouvernements modernes sur cet objet.}]{II. - Tableau de l’art de la guerre, depuis le commencement du monde. Situation actuelle de cette science en Europe. Son parallèle avec ce qu’elle fut autrefois. Nécessité du rapport des constitutions militaires avec les constitutions politiques. Vices de tous nos gouvernements modernes sur cet objet.}
\noindent Il est triste d’imaginer que le premier art qu’aient inventé les hommes, ait été celui de se nuire, et que, depuis le commencement des siècles, on ait combiné plus de moyens pour détruire l’humanité, que pour la rendre heureuse. C’est cependant une vérité bien prouvée par l’histoire. Les passions naquirent avec le monde. Elles enfantèrent la guerre. Celle-ci produisit le désir de vaincre et de se nuire avec plus de succès : l’art militaire enfin. D’abord faible à sa naissance, il ne fut, d’homme à homme, que le talent de tirer parti de son adresse et de sa force. II se borna, dans les premières familles, à la lutte, au pugilat, ou à l’escrime de quelques armes grossières. Bientôt il s’étendit avec les sociétés. Il combina plus de moyens et de forces. Il rassembla une plus grande quantité d’hommes. Il fut alors à peu près ce qu’il est aujourd’hui chez les peuples asiatiques, un amas de connaissances si informes, qu’on ne peut guère l’honorer du nom de science. Il s’éleva sur la terre des hommes ambitieux et cet art, perfectionné par eux, devint l’instrument de leur gloire. Il fit, dans leurs mains, le destin des nations. Il détruisit ou conserva les empires. Il précéda enfin, chez tous les peuples, les arts et les sciences, et y périt, à mesure que celles-ci s’étendirent.\par
Suivons l’art militaire dans ses révolutions. Nous le verrons parcourir successivement différentes parties du globe, portant tour à tour gloire et supériorité aux peuples qui le cultivèrent ; fuyant les nations riches et éclairées ; s’arrêtant de préférence chez les nations agrestes et pauvres, parce que les âmes y ont plus de courage et d’énergie. Nous remarquerons particulièrement cinq ou six grandes époques, qui sont, à proprement parler, ses âges, et les temps où il s’est fait de grands changements dans les principes.\par
C’est chez les peuples de l’Asie, chez les Perses surtout, que l’art de la guerre commença à prendre quelque consistance. Les Égyptiens, amis des sciences et de la paix, y firent toujours peu de progrès. Excepté sous Sésostris, ils ne furent jamais conquérants. Après la mort de Cyrus, le luxe lui fit quitter la Perse, et il passa chez les Grecs. Ce peuple, ingénieux et brave, le perfectionna, et le réduisit en principes. Alexandre vint, l’étendit encore, et conquit l’Asie qui en avait été le berceau. À cette époque, il parut au plus haut point de splendeur, et la phalange fut réputée la première ordonnance de l’univers.\par
Pendant ce temps-là quelques Troyens fugitifs et errants s’établissaient sur les côtes de l’Ausonie. Ils apportaient avec eux les principes de tactique échappés des ruines de Troie et ceux que leur avaient appris les funestes succès des Grecs. Les habitants du pays, repoussés par leurs armes, finissaient par s’unir avec eux. Des aventuriers, descendants de cette colonie, bâtissaient un hameau à quelques lieues d’elle. Des brigands se joignaient à eux et ce hameau devait un jour être la capitale de l’univers. En songeant aux ténèbres répandues sur l’origine de Rome, à ses étrangers fondateurs, à ses grandes destinées, on se rappelle ces fleuves qui ne sont quelquefois, à leur source, que des ruisseaux ignorés. Tullus Hostilius, un des souverains de cet État naissant, lui créait des lois, une milice, une tactique ; et ainsi, tandis que les Grecs se croyaient le premier peuple militaire du monde, il s’élevait à deux cents lieues d’eux une nation nouvelle, une ordonnance totalement opposée à la leur, qui devait enfin les vaincre et les faire oublier.\par
Les Romains, ambitieux et guerriers par leur constitution, profitant des lumières et des fautes de tous les siècles, durent bientôt prendre l’ascendant sur tous les peuples connus. L’Italie divisée plia sous le joug. Carthage lutta quelque temps. Mais les talents d’Annibal ne purent la défendre contre les vices de son gouvernement et la supériorité de celui de sa rivale. Elle eut le sort des nations riches et commerçantes. Elle fut vaincue. Les Grecs en éprouvèrent autant, et résistèrent encore moins. Amollis par le luxe et par les richesses, ils tendirent les mains aux fers des Romains. Contents pourvu qu’on les laissât écrire, peindre et sculpter, ils se consolaient bassement en régnant, par les arts, sur un peuple qui leur enlevait l’empire des armes.\par
Dans le dernier âge de la république, Rome se vit maîtresse du monde. Il n’y eut plus alors, dans l’univers connu, qu’une seule puissance, qu’une seule tactique. Toutes les institutions militaires étaient anéanties, ou fondues dans celles des Romains. L’art de la guerre parut donc, une seconde fois, au plus haut point de sa splendeur. Mais ce moment ne pouvait pas durer. Pour qu’une science et celle-là particulièrement, se soutienne et s’étende, il faut que plusieurs nations à la fois s’y attachent et la cultivent. Il faut qu’elles y soient excitées par l’ambition et la nécessité. Les Grecs étaient devenus guerriers par leurs divisions intestines, par l’ambition de leurs gouvernements, par le besoin d’opposer du courage et des principes aux invasions des Perses. Les Romains s’étaient de même formés en défendant leurs foyers, en attaquant leurs voisins quelquefois, comme les Samnites, pauvres et redoutables, en combattant surtout de grands hommes, Annibal et Pyrrhus, qui les instruisirent à force de les vaincre. Mais, quand Rome régna paisiblement sur l’univers, quand elle n’eut plus d’ennemis que ses richesses et ses vices, la discipline dégénéra, l’art militaire ne fut qu’une étude de théorie et de spéculation, abandonnée à quelques légionnaires obscurs et méprisés. Les Parthes, les Gaulois, les Germains attaquaient de toutes parts les frontières de l’empire. Les légions, jusqu’alors invincibles, étaient souvent vaincues. Mais ces guerres lointaines n’alarmaient pas encore l’Italie. Les empereurs, assoupis sur leur trône, portaient à peine leurs regards aux extrémités de l’empire. Ils ne voyaient pas l’abâtardissement de leur milice et le précipice qui se creusait sous leur grandeur.\par
Vespasien, Titus, Trajan et quelques autres princes remédièrent passagèrement à ces maux. Ils rétablirent la discipline dans les troupes. Ils firent la guerre eux-mêmes et ils la firent avec succès. Mais à ces grands hommes succédaient des princes faibles, ou des tyrans. Les ressorts du gouvernement se relâchaient de nouveau, les plaies politiques devenaient plus profondes et plus incurables. Les légions vendaient l’empire, au lieu de le défendre. Rome ne put survivre à tant de corruption. Des essaims de Goths, de Huns, de Vandales attaquèrent l’empire. Ils vinrent avec le nombre et le courage ; et on ne leur opposa, ni le courage qui supplée quelquefois à la discipline, ni la discipline qui peut suppléer au courage. L’empire ne fut plus, pendant un siècle et demi, qu’un colosse languissant et abattu, dont chacun s’arracha les dépouilles. Ce qu’il y eut de remarquable, c’est que ces Romains avilis appelaient barbares, les peuples qui les subjuguaient. Étrange aveuglement d’une nation qui n’avait conservé que l’orgueil de ses aïeux et qui faisait consister sa grandeur dans son luxe et ses théâtres !\par
Il ne resta plus bientôt à l’univers, que le souvenir de cette puissance qui l’avait enchaîné. Les papes s’assirent sur le trône de Rome ; les Turcs sur celui de Constantinople. L’art militaire, déjà presque ignoré dans la décadence du Bas-Empire, se perdit entièrement sous ses ruines et ne reparut en Europe que trois ou quatre siècles après. Pendant tout cet intervalle, et pendant les siècles qui le précédèrent, l’Europe fut sans tactique, sans discipline et presque sans troupes réglées. L’anarchie des gouvernements, la tyrannie des seigneurs féodaux, l’ignorance générale, l’oppression spirituelle qu’exerçait le clergé, empêchaient les arts de renaître. Tous les livres des anciens étaient entre les mains des prêtres ; et ces prêtres avaient intérêt de maintenir l’Europe dans les ténèbres. Elles faisaient leur grandeur.\par
Qu’offre à nos yeux l’histoire des premiers siècles de notre monarchie et de tous les États actuels ? Des émigrations de Goths, battues par Clovis, ou par Mérovée, qui allaient au-devant d’elles avec des laboureurs rassemblés pour quinze jours seulement ; des Germains et des Saxons subjugués par Charlemagne, parce qu’il était plus brave et plus puissant qu’eux ; les incursions des Normands héritiers du courage et de l’indiscipline des Vandales leurs aïeux. Partout des armées sans ordre et sans science ; des batailles gagnées par le hasard, ou par la valeur et jamais par la discipline ; des conquêtes rapides comme des torrents et surtout dévastatrices comme eux. Un prince qui aurait paru alors avec du génie et de bonnes troupes, aurait soumis l’Europe. On n’a qu’à voir ce que fit Gustave avec vingt-cinq mille Suédois, dans un temps où elle entrevoyait déjà le crépuscule de la renaissance des arts.\par
La découverte de la poudre ne perfectionna pas l’art militaire. Elle ne fit que fournir de nouveaux moyens de destruction et porter le dernier coup à la chevalerie : institution que nos siècles de lumière doivent envier à ces temps d’ignorance ! Les armes à feu retardèrent même vraisemblablement le progrès de la tactique ; parce qu’alors les armées s’approchèrent moins et qu’il entra encore plus de hasard et moins de combinaisons dans les batailles.\par
Gustave et Nassau parurent enfin. L’un combattait pour la liberté de son pays, l’autre pour l’amour de la gloire. Tous deux étudièrent l’antiquité. Tous deux cherchèrent, dans les débris des siècles, les vestiges épars de la tactique et de la discipline. Peut-être, admirateurs outrés des anciens, en appliquèrent-ils trop servilement les principes au temps où ils vécurent et aux armes en usage alors. Peut-être retardèrent-ils par là nos progrès, parce que leur autorité fut longtemps décisive pour le siècle suivant, parce qu’elle soutint longtemps le préjugé des piques et de l’ordre de profondeur. Mais ce qu’il y a de certain du moins, c’est que sous eux l’art militaire reprit naissance et que l’Europe étonnée dut crier au miracle, quand elle vit les troupes, le camp et les succès de Gustave.\par
Après sa mort, Bannier, Gassion, Veimar, Turenne, Montecucculi, combattirent d’après ses principes. L’art militaire fit, sur quelques points encore, de nouveaux progrès. Ce fut le temps des grands généraux, commandant de petites armées et faisant de grandes choses. Mais la tactique resta dans l’enfance. Il semblait qu’on n’osait perdre de vue les premières institutions. On craignit de s’égarer en s’écartant de l’ordonnance des anciens. On conserva les piques. On continua de croire que la force de l’infanterie consistait dans la densité de son ordre et dans son impulsion. On cita toujours les Anciens et on ne s’aperçut pas qu’il y avait deux mille ans entre les Anciens et nous, qu’il fallait d’autres principes, parce que les armes, les constitutions et surtout la trempe des âmes, n’étaient plus les mêmes.\par
Le dix-septième siècle et le commencement de celui-ci éclairèrent de plus en plus l’Europe sur quelques branches de la guerre ; sur d’autres ils la laissèrent, ou la rejetèrent dans les ténèbres. Cohorn et Vauban perfectionnèrent l’attaque des places. Nous fûmes créateurs en ce genre et, quoiqu’on en dise, bien supérieurs aux Anciens. L’art de la défense ne fit pas les mêmes progrès ; soit parce que le courage avait baissé et que c’est le courage qui est le véritable rempart des places ; soit parce qu’on ne réfléchit pas assez qu’il n’y a de bonne défense, que celle qui est offensive et qui multiplie les obstacles sur les pas des assiégeants. M. de Chamilli défendit Grave suivant ce principe et il a eu peu d’imitateurs.\par
Il se fit en même temps, à d’autres égards, des changements bien mal entendus, bien funestes à l’humanité et à la perfection de la science militaire. On eut, par exemple, des armées beaucoup plus nombreuses. On multiplia prodigieusement l’artillerie. Louis XIV, qui en donna l’exemple, n’y gagna rien. Il ne fit qu’engager l’Europe à l’imiter. Les armées, moins faciles à mouvoir et à nourrir, en devinrent plus difficiles à commander. Condé, Luxembourg, Eugène, Catinat, Vendôme, Villars, par l’ascendant de leur génie, surent remuer ces masses. Mais Villeroi, Marsin, Cumberland et tant d’autres restèrent écrasés sous elles. Eh ! comment les auraient-ils conduites ? Les grands hommes dont je viens de parler, n’introduisirent dans les armées, ni organisation, ni tactique. Ils ne laissèrent point de principes après eux. Peut-être même, j’ose le dire, agirent-ils souvent par instinct, plutôt que par méditation. De là, il ne pouvait pas se former de généraux sous eux. De là, quand le génie de ces hommes privilégiés ne marchait plus à la tête des armées, on tombait dans la nuit de l’ignorance. On accusait alors la fortune, la nature, la décadence du siècle, de la rareté des bons généraux. Il fallait bien qu’on s’en prit à ces causes chimériques. On regardait presque entièrement la science du commandement, comme un don inné, comme un présent du ciel. On imaginait à peine que l’éducation et l’étude fussent nécessaires. La science de la guerre n’était développée dans aucun ouvrage, d’une manière lumineuse. La tactique surtout était une routine étroite et bornée. Le maréchal de Puységur avait posé quelques principes, au milieu de beaucoup d’erreurs. Mais il s’était bientôt arrêté, ou égaré dans sa théorie. C’est au roi de Prusse qu’était réservée l’invention de l’art de diviser une armée, de simplifier les marches, de déployer les troupes, de manier cent mille hommes aussi facilement que dix mille.\par
Il y avait alors un grand schisme dans les opinions des militaires. La découverte des armes à feu devait-elle changer la tactique ? Devait-on rejeter l’ordonnance des Anciens, à cause de sa profondeur et de l’effet de l’artillerie ? Toute l’Europe fut divisée et flottante entre ces opinions. On écrivait de part et d’autre, et les discussions n’éclaircirent rien. Follard proposa les colonnes, il en faisait l’ordonnance fondamentale et presqu’exclusive de l’infanterie. Et telle était alors l’ignorance, qu’il eut beaucoup de partisans. On vit le moment que toute l’infanterie allait reprendre la pique et se former en phalange. La guerre de succession et celle de 1733 se firent dans cette incertitude, les bataillons combattant tantôt à quatre, tantôt à six rangs ; les anciens officiers réclamant toujours les piques que Vauban leur avait faites quitter ; la cavalerie n’ayant en France, que de la valeur et point d’ordre ; chez les étrangers, de l’ordre et point de légèreté ; combattant chez nous à la débandade, chez les autres, en masse ; incertaine si sa force était dans son choc, ou dans sa vitesse ; ayant cru pendant un temps, qu’elle devait aussi se servir de l’action du feu. Les généraux, plus indécis eux-mêmes, parce qu’ils avaient moins réfléchi sur ces discussions qu’ils regardaient comme oiseuses et subalternes, n’établirent de principes sur rien. La tactique ne les occupait pas. Ils semblaient la regarder comme indifférente aux succès de la guerre et ce vice ne s’apercevait pas, parce qu’alors personne en Europe n’était plus éclairé.\par
On touchait cependant au moment de sortir de ces ténèbres. Le Nord offrait une seconde fois le phénomène d’une armée aguerrie et disciplinée. Charles XII combattait à la tête des Suédois encore animés de l’esprit de Gustave. Son infanterie était presque aussi infatigable, aussi disciplinée, que celle des légions romaines ; chargeait, comme elles, l’épée à la main, avait d’excellents officiers généraux et quelque connaissance des déploiements modernes. Peut-être enfin Charles XII eût-il perfectionné l’art militaire, ainsi que son aïeul l’avait rétabli ; peut-être eût-il été le Frédéric de son temps. Mais il vécut trop peu. Avait-il au reste assez de connaissances et assez d’étendue dans le génie ? Ses premiers succès furent rapides, ainsi que le seront toujours ceux d’une armée disciplinée, sur une multitude ignorante. Il débuta comme Alexandre, se conduisit ensuite en aventurier et finit comme Gustave. Après sa mort les Suédois dégénérèrent et les Russes qui les avaient vaincus sans les égaler, ne devinrent pas plus éclairés.\par
Ce fut toujours le destin du Nord de faire les révolutions militaires de l’Europe, comme celui du Midi de faire celles de l’Europe savante. Un royaume venait de s’élever sur l’Oder et sur la Sprée. Ses nouveaux souverains, ne pouvant avoir ni commerce ni marine, s’attachèrent à former une armée et bientôt ils firent poids dans la balance générale par leurs prétentions et leurs soldats. Frédéric II parvint au trône et il acheva ce qu’avaient ébauché ses pères. Prince habile et plein de l’étude des Anciens, il y déploya le génie le plus vaste. Il doubla ses troupes par le nombre et plus encore par la discipline ; créa une tactique presque nouvelle ; se forma des généraux ; fut lui-même le plus habile de tous ; conquit une province, meilleure que son royaume ; lutta contre autant d’ennemis que Louis XIV, avec moins de moyens et plus de gloire ; et se fit enfin, avec peu de revenus, peu de population, peu de facultés dans ses sujets, la puissance la plus militaire et la plus surprenante de l’Europe. Le règne de ce prince sera un des âges remarquables de la science de la guerre, comme celui d’Auguste et celui de Louis XIV sont des âges principaux dans l’histoire des lettres.\par
Tel est l’empire de l’habitude et des préjugés chez les peuples, que le roi de Prusse formait des troupes et créait une tactique, sans qu’aucune autre nation songeât à se mettre à sa hauteur. Il avait cependant battu plusieurs fois les Autrichiens dans la guerre de 1740. Il leur avait enlevé la Silésie. Ces succès avaient été le fruit de ses travaux. Pendant la paix qui suivit cette guerre, il formait des camps à Spandau et à Magdebourg. Il y perfectionnait ce que l’expérience lui avait fait trouver de vicieux dans sa tactique. Il y introduisait ces déploiements savants et avantageux, cette célérité incroyable et décisive, devenus si nécessaires par rapport à nos armées nombreuses et à leur grand front. Mais personne ne réfléchissait autour de lui. L’Autriche restait assoupie dans sa routine. La France croyait que, parce qu’elle avait vaincu avec sa constitution, elle devait vaincre encore. Les victoires de Flandres entretenaient cette sécurité malheureuse. Tout le reste de l’Europe, moins militaire que la France et l’Autriche, parce qu’il a moins d’intérêt à l’être, était dans le même engourdissement. Ce fut dans cette situation que commença la dernière guerre.\par
Depuis la guerre de succession, on n’avait pas vu tant d’armées en campagne et réunies contre un seul prince. Sa science et leurs fautes furent le contrepoids de tant de forces. Jamais guerre ne fut plus instructive et plus féconde en événements. Il s’y fit des actions dignes des plus grands capitaines et des fautes dont les Marsin auraient rougi. On y vit quelquefois le génie aux prises avec le génie, mais plus souvent avec l’ignorance. Partout où le roi de Prusse put manœuvrer, il eut des succès. Presque partout où il fut réduit à se battre, il fut battu : événements qui prouvent combien ses troupes étaient supérieures en tactique, si elles ne l’étaient pas en valeur. Daun se conduisit avec lui en conséquence. Il évita les plaines, reçut les batailles dans des postes, n’en livra que lorsqu’il put surprendre, ou ne pas être obligé de manœuvrer. Il rétablit enfin les affaires de l’Autriche, comme Fabius rétablit celles de Rome vis-à-vis d’Annibal. Les Autrichiens disent de lui, comme les Romains disaient de Fabius, qu’il fut circonspect et timide. Mais pouvaient-ils l’un et l’autre se compromettre à manœuvrer avec des armées neuves et sans tactique, contre des ennemis que leurs chefs avaient rendus instruits et manœuvriers ?\par
On vit dans cette guerre la quantité d’artillerie s’accroître jusqu’à l’immensité. Les Russes en traînaient avec eux jusqu’à six cents pièces. Le roi de Prusse et les Autrichiens jusqu’à trois ou quatre cents. Mais on vit en même temps tomber le préjugé qui attachait le même honneur à la prise d’un canon, qu’à celle d’un drapeau. On vit, grande leçon pour les généraux, les armées du roi de Prusse ne pas être appesanties par cet attirail, faire des marches forcées, perdre des batailles, avec la plus grande partie de leurs canons et s’arrêter à deux lieues du terrain où elles les avaient perdues.\par
Le nombre des troupes légères s’accrut aussi prodigieusement. Il fallut à des armées si nombreuses, chargées de tant d’équipages de vivres et d’artillerie, des positions si étendues, des convois si fréquents, des établissements si hasardés, des communications si longues, qu’on augmenta, comme à l’envi de part et d’autre, l’espèce de troupes destinées à les attaquer et à les couvrir.\par
De ces deux changements, que toutes les puissances belligérantes ont adoptés en se calquant servilement les unes sur les autres et dont je pense qu’un général, homme de génie, pourrait avec avantage secouer les embarras, il s’ensuit qu’à la première guerre, les armées seront plus dispendieuses, plus dévastatrices, plus pesantes, que les accessoires y seront plus nombreux que le principal. J’entends par ce dernier, les troupes de ligne, celles qui gagnent les batailles. Il s’ensuit que les guerres seront encore moins décisives et pourtant plus funestes à la population et aux peuples ; car c’est toujours sur cette humanité malheureuse et gémissante que retombent les inventions nuisibles et tous les faux calculs, militaires ou politiques.\par
Tel est enfin aujourd’hui l’art militaire en Europe, qu’à le comparer à ce qu’il fut dans les siècles passés, dans le temps les plus éclairés de l’antiquité, il est devenu bien plus vaste et plus difficile. Chez les Anciens on ne connaissait ni la science de l’artillerie, ni celle des mines, sciences fondées sur des spéculations abstraites et profondes. La théorie de leur balistique, le feuillage des Beces et des Daces étaient, en comparaison, des arts informes et grossiers. La science de fortification des Anciens, celle de leurs sièges ne se mettra certainement point en parallèle avec les connaissances des Vauban et des Cohorn. Ces dernières sont fondées sur le concours réfléchi de presque toutes les branches des mathématiques. Les autres, dépourvues de géométrie, étaient de misérables routines. On n’avait pas chez les Anciens ces attirails prodigieux d’équipages d’artillerie, de vivres, si difficiles à mouvoir et à nourrir. On n’avait pas des armées aussi nombreuses. On connaissait peu les chicanes de la petite guerre. On ne s’embarrassait presque pas du choix des positions. On ne voit dans le récit des anciens historiens militaires, aucun détail topographique. Les armées ayant de très petits fronts, l’espèce des armes n’occasionnant ni fumée, ni tumulte, les batailles devaient être plus aisées à engager et à conduire. Je compare les guerres des Grecs et la plupart des guerres des Anciens, à celles de nos colonies dans l’autre continent. J’y vois cinq ou six mille hommes les uns contre les autres, des champs de bataille étroits, où l’œil du général peut tout embrasser, tout diriger, tout réparer. Un bon major conduirait aujourd’hui la manœuvre de Leuctres et de Mantinée, comme Épaminondas.\par
Je dis que la science de la guerre moderne, comparée avec celle des Anciens, est plus vaste et plus difficile. Ce n’est pas cependant qu’elle soit plus parfaite et plus lumineuse sur tous les points. Elle a fait des progrès à quelques égards. À d’autres, elle s’est étendue et compliquée, aux dépens de sa perfection. Nos armes à feu sont supérieures aux armes de jet des anciens. La science de l’artillerie l’emporte sur leur balistique, nos fortifications sur les leurs. Les places s’assiègent et se défendent avec plus d’art. Voilà les progrès modernes. Voilà l’effet des lumières mathématiques répandues sur la science de la guerre. Mais les armées sont devenues trop nombreuses. L’artillerie et les troupes légères se multiplient trop. Les frontières des États sont mal à propos hérissées de places, sur deux et sur trois lignes. Les places sont inutilement surchargées de pièces de fortifications. Les systèmes des ingénieurs sont la plupart trop exclusifs, trop méthodiques, trop peu combinés avec la tactique. Les armées, devenues immenses tant par l’augmentation des combattants, que par les attirails et les embarras qu’elles traînent à leur suite, sont difficiles à mouvoir. Les détails de leur subsistance forment une science dont les armées anciennes, moins nombreuses, plus sobres et bien mieux constituées, n’avaient point d’idée. Voilà les erreurs et les abus qui compliquent la science moderne, qui multiplient les connaissances qui la composent, qui rendent les grands généraux si rares. Tel homme, dont l’esprit eût embrassé toutes les parties de l’art militaire des Anciens, qui eût bien commandé quinze ou vingt mille Grecs ou Romains, tel homme, qui eût été alors un Xantippe, un Camille, ne suffit pas aujourd’hui à la moitié des connaissances qui composent la science moderne. Il est absorbé par les détails, aveuglé par l’immensité, étourdi par la multitude. Cent mille hommes dont il doit régler les mouvements ; le soin de pourvoir à leur subsistance ; tous les obstacles produits par nos mauvaises constitutions ; cent mille ennemis qui lui sont opposés : un plan de campagne à plusieurs branches ; les combinaisons sans nombre, qui résultent de la multiplicité des objets. Tant d’attentions réunies forment un fardeau au-dessus de ses forces. Il reste fatigué et accablé sous lui, ou du moins il ne se remue que péniblement et qu’avec une partie de ses facultés. Il n’est enfin qu’un général du second et du troisième ordre.\par
La science de la guerre moderne, en se perfectionnant, en se rapprochant des véritables principes, pourrait donc devenir plus simple et moins difficile. Alors les armées, mieux constituées et plus manœuvrières, seraient moins nombreuses. Les armes y seraient réparties, dans une proportion sagement combinée avec la nature du pays et l’espèce de guerre qu’on voudrait faire. Elles auraient des tactiques simples, analogues, susceptibles de se plier à tous les mouvements. De là l’officier d’une arme saurait commander l’autre arme. On ne verrait que des officiers généraux, ignorant les détails des corps dans lesquels ils n’ont pas servi, démentir le titre qu’ils portent, ce titre qui, en leur donnant le pouvoir de commander toutes les armes, leur suppose l’universalité des connaissances qui les dirigent. Les armées étant ainsi formées, elles seraient plus faciles à remuer et à conduire. On quitterait cette manière étroite et routinière, qui entrave et rapetisse les opérations. On ferait de grandes expéditions. On ferait des marches forcées. On saurait engager et gagner des batailles par manœuvres. On serait moins souvent sur la défensive. On ferait moins de cas de ce qu’on appelle des positions. Les détails topographiques n’auraient plus la même importance. Ils ne surchargeraient plus au même point la science militaire. Les embarras étant diminués, la sobriété ayant pris la place du luxe, les détails des subsistances deviendraient moins compliqués et moins gênants pour les opérations. La science du munitionnaire consisterait à traîner le moins d’attirails possible et à tâcher de vivre des moyens du pays. L’artillerie, les fortifications, s’éclaireraient de plus en plus. Elles suivraient, dans chaque siècle, les progrès des mathématiques qui leur servent de base. Mais elles n’élèveraient, ni l’une ni l’autre, des prétentions exclusives et dominantes, des systèmes qui multiplient les dépenses et les embarras. Elles ne tiendraient, dans les armées et dans les combinaisons militaires, que le rang qu’elles doivent y avoir. Elles ne seraient, dans les mains des généraux, que des accessoires utilement employés à fortifier les troupes et à les appuyer. Enfin toutes les branches de la science militaire formeraient un faisceau de rayons. C’est ce concours de lumières qui, réuni dans l’esprit d’un seul homme, le constituerait général, c’est-à-dire, capable de commander des armées.\par
Il serait intéressant de voir la science militaire se perfectionner ainsi en se simplifiant, en devenant moins difficile. J’ai dit ci-dessus comment la même révolution pourrait se faire dans la politique. Elle aurait lieu de même dans presque toutes les sciences, si on dépouillait leur théorie des erreurs qui les surchargent, des fausses méthodes qui les compliquent. Alors les hommes, arrivant plus promptement et en plus grand nombre au faîte de ces sciences, ils pourraient en reculer les bornes. Alors la brièveté de leur vie ne les empêcherait plus d’en embrasser plusieurs à la fois et de les étendre les unes par les autres. Alors l’encyclopédie des connaissances humaines, devenue un assemblage de vérités, s’élèverait et s’affirmerait au milieu des siècles, semblable à un arbre vigoureux qui n’a aucune branche inutile, aucune qui lui nuise et qui s’étendant et paraissant se fortifier sur sa base à mesure qu’il vieillit, répand l’ombre et les fruits sur ses heureux cultivateurs.\par
Mais pour achever le parallèle de l’art militaire chez les Anciens, avec ce qu’il est de nos jours, il y a des objets bien importants, qui sont à l’art militaire ce que les fondements sont à un édifice et sur lesquels les Grecs et les Romains nous étaient fort supérieurs. Ce sont les moyens continuels dont se servaient leurs gouvernements, pour former des citoyens, des soldats, des généraux. C’est la bonté de leur milice, la vigueur de leur discipline, l’éducation guerrière de leur jeunesse, l’espèce de leurs peines et de leurs récompensés. C’est ce rapport important qui liait leurs constitutions militaires à leurs constitutions politiques.\par
Aucun de ces objets ne semble intéresser les gouvernements modernes. Il n’y en a point qui ait calculé le nombre et la constitution de ses troupes, sur la population de ses États, sur la politique, sur le génie national. Il n’y en a point où la profession de soldat soit honorée ; où la jeunesse reçoive une éducation guerrière ; où les lois inspirent le courage et flétrissent la mollesse ; où la nation, en un mot, soit préparée par ses mœurs et ses préjugés à former une milice vigoureuse. Dans cet État même que nous appelons militaire, parce que son roi est un guerrier habile\phantomsection
\label{footnote1}\footnote{La Prusse.}, dans cet État qui s’est agrandi par les armes, qui n’existe et ne peut se flatter de conserver ses conquêtes que par elles, les troupes n’y sont pas plus vigoureusement constituées qu’ailleurs. Elles n’y sont point citoyennes. Elles y sont, plus qu’en aucun autre pays, un assemblage de stipendiaires, de vagabonds, d’étrangers, que l’inconstance ou la nécessité amène sous les drapeaux et que la discipline y retient. Cette discipline, ferme et vigilante sur quelques points, y est relâchée et méprisable sur beaucoup d’autres. Elle n’est, en comparaison de celle des Romains, qu’un enchaînement de choses, de formes, de demi-moyens, de correctifs, de suppléments vicieux. Ces troupes mal constituées ont eu des guerres heureuses, mais elles doivent ces succès à l’ignorance de leurs ennemis, à l’habileté de leur roi, à une science toute nouvelle de mouvements, dont il a été le créateur. Qu’après la mort de ce prince, dont le génie seul soutient l’édifice imparfait de sa constitution, il survienne un roi faible et sans talents, on verra dans peu d’années le militaire prussien dégénérer et déchoir : on verra cette puissance éphémère rentrer dans la sphère que ses moyens réels lui assignent et peut-être payer cher quelques années de gloire.\par
Si telle est la constitution militaire d’un État, dont le souverain est le plus grand homme de guerre de son siècle, qui instruit et qui commande lui-même ses armées, dont les armées forment, pour ainsi dire, toute la pompe et la cour. Que doit être celle de ces États, où le souverain n’est pas militaire ; où il ne voit pas ses troupes, où il semble dédaigner, ou ignorer tout ce qui y a rapport ; où la cour, qui suit toujours l’impression du souverain, n’est conséquemment point militaire ; où presque toutes les grandes récompenses sont surprises par l’intrigue ; où la plupart d’entre elles deviennent des apanages héréditaires ; où le mérite languit, quand il est suis appui ; où le crédit peut s’avancer sans talent ; où faire fortune ne signifie plus acquérir de la réputation, mais amasser des richesses ; où l’on peut en un mot être à la fois couvert de dignités et d’infamie, de grades et d’ignorance ; servit mal l’État et en posséder les premières charges ; avoir le blâme public et jouir de la faveur du souverain ?\par
Mais, sans parler des vices particuliers que le caractère des souverains et la corruption de leurs cours peuvent imprimer aux constitutions militaires de leurs États, comment calculer les abus sans nombre qui résultent du défaut de rapport entre l’administration militaire et les autres branches du gouvernement ? De là ces États exclusivement marchands, ou militaires, parce que le système momentané de leurs administrateurs fait mal à propos consister toute la force publique dans les richesses, ou dans les armes. De là ces directoires de guerre qui n’ont pas vu d’armées et règlent cependant le sort des armées ; ces ordonnances militaires, faites par des gens de plume ; ces ministres qui, n’étant pas généraux, contrarient toujours les demandes et les opérations des généraux ; ces généraux qui, n’étant pas ministres, ignorent l’influence qu’ont les opérations de la guerre sur la politique et ce qu’il en coûte à l’intérieur des États pour soutenir la guerre. De là toutes ces constitutions militaires mal calculées, s’imitant réciproquement au hasard et sans méditation ; le nombre des troupes disproportionné aux moyens des États ; les troupes, tantôt négligées et regardées comme un fardeau presqu’inutile, tantôt augmentées par-delà les bornes raisonnables, et attirant, aux dépens des autres branches, toute l’attention du gouvernement. De là ces troupes si étrangement constituées et employées par le gouvernement, qu’elles ruinent l’État dont elles devraient faire la prospérité, en même temps que la force ; qu’elles enlèvent à la population la plus belle espèce d’hommes ; que ces hommes y amollissent leurs mœurs et leurs bras, à un tel point que, quand ils quittent cette profession, ils ne sont plus capables que de travaux citadins et sédentaires ; que, pendant la paix, on ne les occupe presque que d’exercices puérils et étrangers à la guerre, qu’on les entasse dans des places, comme si l’ennemi était aux portes du royaume ; c’est-à-dire par conséquent sur les frontières, dans les pays où les vivres sont le plus chers, et ont le plus de débouchés, où les habitants ont le plus de ressources et d’industrie ; au lieu de les disperser dans les provinces intérieures qui manquent de vivification et d’espèces, qui ont plus de denrées que de consommateurs ; dans ces provinces qui sont en friche et que le soldat pourrait cultiver, qui manquent de chemins et que le soldat pourrait ouvrir. Dans le cours de mon ouvrage, je prouverai, par des détails, que ces abus existent et qu’on peut y remédier. Faire le tableau des abus, sans en fournir à la fois les preuves et les remèdes, c’est s’ériger en déclamateur. C’est ressembler à ces médecins barbares qui annoncent des maux qu’ils ne peuvent, ni expliquer ni guérir.\par
Il me reste à expliquer pourquoi l’histoire de l’univers nous représente toujours l’art militaire déclinant chez les peuples, à proportion que les autres arts y font des progrès. J’en ai moi-même fait l’observation, au commencement de ce chapitre. Mais ce n’est point aux arts ni aux sciences qu’il faut attribuer cette révolution. C’est à la maladresse des gouvernements. Ces effets ont été jusqu’ici contemporains, sans être nécessairement liés et dépendants. Les lumières ne peuvent nuire. Laissons ce préjugé funeste aux apologistes de l’ignorance. Les lumières chassent les erreurs, fixent les principes, amènent la vérité. Les siècles de lumières ne peuvent être des temps de malheur pour l’humanité, à moins qu’elles n’aient fait que des demi-progrès ; à moins qu’elles n’aient, comme chez les Anciens, porté sur les arts, plus que sur les sciences, sur les connaissances frivoles, plus que sur les connaissances utiles ; à moins que, comme alors, elles n’aient éclairé une partie du globe et laissé l’autre dans les ténèbres ; à moins que, comme aujourd’hui, elles ne soient le partage d’un petit nombre d’hommes et que rejetées par les gouvernements, elles ne mettent aux prises la vérité avec les préjugés, la philosophie avec l’ignorance, le despotisme avec les droits de la nature. Encore faudrait-il se consoler des malheurs passagers qui pourraient naître du choc des lumières et des ténèbres. Le crépuscule du matin éloigne la nuit, il fait espérer le jour. Enfin quand la propagation des connaissances sera générale, quand elle sera répandue à la fois sur les grands et sur les petits, sur les trônes et sur les peuples ; quand les gouvernements seront en même temps instruits et vigoureux, quand la lumière nous viendra d’eux, comme elle descend des astres qui sont sur nos têtes, la terre sera heureuse. Elle bénira ses gouvernements, comme ces astres bienfaisants qui la fécondent et qui l’éclairent.\par
Je reviens à mon objet. Ce ne sont pas les arts et les sciences qui ont fait déchoir l’art militaire chez les peuples de l’antiquité. Ce ne sont pas les arts et les sciences qui l’empêchent aujourd’hui de faire des progrès. Les lumières générales devraient au contraire perfectionner cet art avec tous les autres. Elles devraient rendre la tactique plus simple et plus savante, les troupes plus instruites, les généraux meilleurs. Elles devraient mettre la méthode à la place de la routine, les combinaisons à la place du hasard. Si, tandis que toutes les autres sciences se perfectionnent, celle de la guerre reste dans l’enfance, c’est la faute des gouvernements qui n’y attachent pas assez d’importance ; qui n’en font pas un objet d’éducation publique ; qui ne dirigent pas, vers cette profession, les hommes de génie ; qui leur laissent entrevoir plus de gloire et d’avantages dans des sciences frivoles ou moins utiles ; qui rendent la carrière des armes une carrière ingrate dans laquelle les talents sont devancés par l’intrigue et les prix distribués par la fortune.\par
Si enfin un peuple s’amollit, se corrompt, dédaigne la profession des armes, perd toute l’habitude des travaux qui y préparent ; si une nation étant dégradée à ce point, le nom de {\itshape patrie} n’y est plus qu’un mot vide de sens ; si ses défenseurs ne sont plus que des mercenaires, avilis, misérables, mal constitués, indifférents au succès, ou aux revers (c’est par ces vices de mœurs et de constitution, qu’ont déchu toutes les milices anciennes et que pèchent toutes nos milices modernes), c’est encore la faute du gouvernement. Car le gouvernement doit veiller sur les mœurs, sur les opinions, sur les préjugés, sur les courages. Avec la vertu, l’exemple, l’honneur, le châtiment, il peut être plus puissant que le luxe, que les abus, que les vices, que les passions, que la corruption la plus invétérée. Avec ces mêmes lumières qu’on croît la source de la décadence des empires, qu’il éclaire sa nation sur le précipice où elle se jette, qu’il se mette à sa tête, il l’entraînera. Elle le suivra avec d’autant plus de soumission, que plus instruite, elle sentira mieux le bien qu’on lui prépare, le mal auquel on l’arrache et la prospérité vers laquelle on veut la conduire. En général les gouvernements des grands peuples sont bien loin de faire et de connaître seulement tout ce qui est en leur pouvoir. Ils ne sentent pas assez l’étendue de leurs ressources. Ils se laissent décourager par le nombre et l’ancienneté des abus. Ils n’osent porter ni le fer, ni les remèdes, aux ulcères qui les rongent. Ils s’agitent sans succès, comme des mourants, dans les convulsions de l’agonie. Ne nous lassons donc pas de leur répéter que, si leurs vices sont sans nombre, leurs moyens sont immenses ; qu’ils n’ont qu’à perfectionner leur constitution, devenir justes, éclairés, nerveux, qu’alors ils relèveront bientôt les États ; que, si les vices corrompent rapidement, les vertus peuvent régénérer de même. Mettons sans cesse auprès du tableau effrayant de leurs maux, la possibilité encourageante de leur guérison. Peut-être il s’élèvera à la tête des nations, des hommes qui ne désespéreront pas de leur salut, qui désireront le bien, qui aimeront la gloire et à qui ces deux sentiments rendront tout facile. Le génie et la vertu peuvent naître sur les trônes.\par
Je n’ai offert ici qu’une ébauche imparfaite des révolutions de l’art militaire. Ce tableau mérite d’être l’objet d’une histoire complète. Qu’il serait intéressant d’y suivre les progrès de cet art, à travers le cours des siècles, de les suivre particulièrement chez les grands peuples ; d’y observer ce qu’il était aux différentes époques progressives de leur élévation, de leur décadence, de leur ruine ; et ce qu’il était en même temps chez les nations contemporaines aux dépens desquelles ils s’élevaient, ou qui s’élevaient sur leurs débris !\par
Ces recherches instructives ne se borneraient pas simplement à l’histoire de l’art, elles examineraient, aux mêmes époques, les constitutions des milices des différents peuples, les rapports qu’elles avaient avec leurs constitutions politiques, avec leurs mœurs. Car les succès militaires des nations dépendent, plus qu’on ne pense, de leur politique, de leurs mœurs surtout. C’est cet enchaînement que ne nous montrent jamais assez la plupart des historiens qui ne sont communément ni militaires, ni philosophes et encore moins l’un et l’autre à la fois. Il est digne de notre siècle de produire cet ouvrage intéressant. J’y encourage un de mes amis, qui le médite et le prépare depuis longtemps. Je dénonce ici son nom, son plan, ses talents\footnote{M. le chevalier d’Aguesseau, lieutenant-colonel du régiment de la Couronne. }. Je voudrais lui faire contracter vis-à-vis de ses citoyens, un engagement qu’il est en état de remplir, et dont l’exécution fera la gloire particulière, en même temps que l’instruction publique.
\section[{Plan d’un ouvrage intitulé : « La France politique et militaire »}]{Plan d’un ouvrage intitulé : « La France politique et militaire »}\renewcommand{\leftmark}{Plan d’un ouvrage intitulé : « La France politique et militaire »}

\noindent L’objet du discours précédent est de servir d’introduction à cet ouvrage. J’ai cru, pour en mieux développer le plan, devoir commencer par donner à mes lecteurs une idée de la manière dont j’envisage la politique et l’art militaire. Dans les grandes entreprises de toute espèce, les plans sont presque toujours trop négligés. On ne se remplit pas assez de son objet. On ne le médite pas assez sous toutes les faces. On s’engage avec un projet à demi conçu. On compte achever de l’asseoir en l’exécutant. On se promet que les idées feront naître les idées. On travaille par lambeaux. De là tant d’ouvrages qui ne remplissent pas leur but, ou qui démentent ce qu’annonce leur titre. Nos écrivains les plus profonds sont tombés dans cet inconvénient. Quand on ouvre l’{\itshape Esprit des Lois}, on s’attend à trouver le développement des principes qui ont servi de base à la législation ancienne et moderne. On espère que cet examen sera suivi d’un système de création et de réforme dans les lois actuelles de l’Europe, ou tout au moins dans celles de la nation. Mais, oserai-je le dire ? Faute de plan, cette espérance n’est pas remplie. Soit que l’immortel Montesquieu, tout occupé de la création de ses matériaux, ait dû dédaigner, dans la chaleur de cette création, de les assembler et de les polir : soit qu’écrivant de la hauteur de son génie, il laissât à ses pieds toutes les idées intermédiaires que nous demandons à sa cendre ; soit qu’il se proposât de descendre un jour vers les détails, de nous élever par eux jusqu’à lui, d’écrire en un mot pour le reste des hommes après avoir écrit pour lui-même, son ouvrage est resté un monument informe. On y trouve des pensées sublimes, des vérités éparses et à demi dévoilées, l’ébauche ou le germe de presque tous les principes politiques. Mais on sent que toutes ces matières ont besoin d’être accordées et de former un édifice. On éprouve enfin à la lecture de cet ouvrage ce mélange de plaisir et de regret qu’inspirent ces tableaux dont on admire les détails et qui, faute d’ordonnance, ne produisent point d’effet.\par
Cette observation devrait me rendre timide. Mais le pilote qui reconnaît un écueil sur sa route, qui le voit couvert des débris d’un grand naufrage, ne rentre pas dans le port. Il redouble de vigilance. Il tâche de ranger l’écueil où d’autres se sont brisés.\par
Mon objet est d’abord d’examiner la constitution politique et militaire de la France et, avant d’arriver à cet examen, de jeter les yeux sur celle de tous les états de l’Europe. En parcourant ainsi toutes les nations qui ont des rapports, soit prochains, soit éloignés, avec la mienne, je me préparerai à asseoir, avec plus de perfection, le plan de sa politique. Je recueillerai pour elle les lumières de tous les gouvernements ; je m’arrêterai particulièrement sur les choses utiles qu’elle peut imiter et sur les erreurs qu’elle partage. Cette manière indirecte de critiquer et de conseiller, ne m’empêchera pas d’être entendu, me donnera plus de liberté et n’indisposera pas le gouvernement pour lequel j’écris. Ainsi font ces instituteurs qui louent dans les autres, ce qu’ils veulent que leurs élèves adoptent et qui y blâment sans ménagement les vices qu’ils veulent éloigner d’eux.\par
Je vais parcourir l’Europe. Mais semblables à ce voyageur, qui, d’un point élevé, s’oriente et détermine la route qu’il doit suivre, jetons auparavant un regard sur l’Europe et traçons notre itinéraire. Un sentiment irrésistible m’entraîne vers l’Italie. Elle fut autrefois si célèbre ! Elle est un exemple si frappant des vicissitudes humaines ! Je commencerai par elle. J’examinerai successivement les deux Siciles, les États du pape, Lucques même et Saint-Marin, qui ne sont que des villages ; la Toscane, Gènes et la Corse qui lui échappe. De là, laissant la Savoie à ma gauche, pour revenir dans un autre temps vers elle, j’entrerai dans la Lombardie ; j’y verrai les possessions de la maison impériale ; le duché de Modène qui les grossira un jour ; celui de Parme qui fleurit à côté d’elles. Je rendrai compte ensuite de la république de Venise. Ses établissements le long du golfe Adriatique, me conduiront à Raguse, dans l’Archipel, et de là à Malte. Les habitants de cette île sont ennemis nés et perpétuels des États barbaresques. Je parcourrai donc ensuite Tunis, Tripoli, Alger, et le Maroc. Ces pays, quoique africains, ont rapport à l’Europe. Ils sont sous la protection du grand seigneur. Cela me conduira à Constantinople. J’examinerai ce colosse de puissance, que le despotisme détruit. Je passerai chez les Polonais que l’anarchie dévore, chez les Russes, leurs redoutables voisins. Les provinces que ces derniers ont conquises sur les Suédois, me mèneront en Suède, ensuite au Danemark. J’entrerai de là en Allemagne. Je détaillerai les deux États qui y dominent aujourd’hui ; l’un, par le génie de son roi ; l’autre, par sa puissance réelle : la Prusse et l’Autriche. Je parlerai de l’empire, ce corps si compliqué par le nombre de ses membres et par la diversité de leurs intérêts. Je m’étendrai particulièrement sur les États qui y tiennent un rang principal, comme la Saxe, la Bavière, etc. Je verrai les autres en masse, et pour dire seulement à quelle puissance leur intérêt les attache. Arrivé sur le Rhin, je touche à la France : je n’ai plus, et c’est là que je me suis ménagé exprès, qu’à décrire les États qui l’avoisinent. Je commencerai par la Hollande. J’examinerai la situation actuelle de la Flandre autrichienne. Je verrai, en passant, les pays qui bordent la Meuse et le Rhin. Je remonterai ce fleuve, pour aller en Suisse. La Savoie et les pays qui en dépendent, retrouveront ici leur place. De la Sardaigne, je passerai en Espagne ; d’Espagne, en Portugal ; de Portugal, en Angleterre ; pays depuis longtemps rival de la France, et dont, pour cette raison, j’ai voulu placer le tableau auprès du sien.\par
En rendant compte des constitutions politiques et militaires de tous ces États, je ne me propose point d’entrer sur tous, dans des détails également étendus. Je peindrai la plupart des objets, à grands traits, et planant sur eux. Je tâcherai d’imiter ces écrivains célèbres qui, s’élevant au-dessus de leur siècle, racontent d’un style philosophique et rapide, ce qu’ils voient autour du peuple, ou du héros dont ils projettent l’histoire.\par
C’est pour la France que j’écris. C’est relativement à elle que j’examine les autres nations. Ainsi celles qu’aucuns rapports ne peuvent lier à elle, ont peu de droits à mon attention. Je me contenterai de faire connaître le résultat de leur puissance, et ce qu’elles apportent de poids et d’intérêt dans la balance politique de l’Europe, par rapport aux autres États qu’elles avoisinent, et qui peuvent nous intéresser. Les nations que des rapports, de quelque espèce qu’ils soient, lient ou peuvent lier à la mienne, arrêteront de plus près et plus longtemps mes regards. Je pèserai leurs intérêts, leurs vertus, leurs vices, leurs moyens, leurs ressources, tout ce qui peut enfin déterminer la politique de mon pays à leur égard, et surtout éclairer son administration. En un mot, dans le vaste tableau que je vais tracer, la France sera le sujet dominant ; les États qui l’intéresseront, seront les figures secondaires, développées avec plus ou moins de soin, suivant le degré des rapports qui les lient au sujet principal. Les autres États seront, si je puis m’exprimer ainsi, les figures accessoires et lointaines du tableau.\par
Une chose importante dans l’exécution d’un tableau pareil, c’est de le bien ordonner ; de ne pas le surcharger de détails ; de les disposer de manière qu’ils n’embarrassent point la marche du plan, qu’ils n’en refroidissent point l’intérêt. Afin d’y parvenir, lorsque je parlerai des traités qui unissent une nation à l’autre, de ses finances, de son commerce, de son militaire, tout ce qui formera des preuves, ou des détails trop allongés pour être fondus dans le corps de l’ouvrage, sera mis en appendices. Par ce moyen, le tableau de chaque État sera, en quelque sorte, divisé en deux parties. L’une historique et philosophique, exposera les faits et les résultats ; l’autre justificative, et en forme de notes, contiendra les détails, et indiquera les sources, où on peut en puiser de plus étendus.\par
Arrivé à la France, je ferai d’elle l’objet d’un examen approfondi. Eh ! Comment, quand on porte un cœur, ne pas s’arrêter involontairement sur la situation de sa patrie, sur les moyens de remédier à ses maux et de relever sa gloire ? J’examinerai sa politique sous le même point de vue que j’ai examiné cette science dans mon discours préliminaire. Je tracerai d’abord tout ce qui a rapport à sa politique intérieure. J’examinerai, dans un chapitre particulier, la situation de chaque objet d’administration, les abus qui lui nuisent, les remèdes qui peuvent y être appliqués. Ensuite je traiterai de sa politique extérieure, de ses intérêts à l’égard des autres peuples, de son système de conduite vis-à-vis d’eux. Là, faisant de la France le centre de toutes mes combinaisons, je mènerai, s’il est permis de s’exprimer ainsi, des rayons vers tous les points de la circonférence de ses intérêts ; c’est-à-dire que j’examinerai successivement toutes les branches de rapports qui la lient, ou peuvent la lier aux autres peuples. Mon plan politique, si une fois la France était régénérée, me ferait sans doute supprimer beaucoup de ces rapports crus nécessaires, par la fausse opinion où l’on est qu’une grande puissance comme elle, doit avoir des colonies éloignées, un commerce considérable, ne doit jamais souffrir qu’il se fasse rien en Europe, sans y prendre part. Mais comme cette régénération est presqu’impossible à espérer, il faut examiner ces rapports, tout chimériques qu’ils sont. Il faut malheureusement dans un ouvrage, comme le mien, avoir deux sortes de plans ; l’un de création et de perfection, dans lequel il est nécessaire de renverser la plupart des idées reçues et qu’il faut par-là s’attendre à voir traiter de romanesque ; l’autre de réparation et de circonstances, dans lequel il faut se plier à la faiblesse de nos gouvernements, se traîner dans la routine de leurs préjugés et ne leur proposer que des remèdes doux et palliatifs. Tel un architecte procède par des moyens bien différents, quand au milieu d’une abondance de bons matériaux et sur des fondements solidement assis, il élève un édifice dont il a formé le plan ; ou, quand obligé de réparer un bâtiment antique, il a besoin de ménager les fondations, de suivre les anciennes coupes, d’avancer avec précaution et en étayant sans cesse.\par
Les intérêts politiques de la nation au-dehors étant déterminés, je passerai à ce qui les fait respecter, à ce qui les soutient, à la constitution militaire. Les moyens de la former nationale et vigoureuse, ayant été préparés à l’avance par la politique intérieure, il ne sera plus question que de l’asseoir relativement à ces moyens. Je lèverai les troupes, je les constituerai, je déterminerai leur nombre, soit sur pied de paix, soit sur celui de guerre, leur habillement, leur armement, leur solde, la manière de les recruter, de les remonter, de les entretenir, leur discipline, leur éducation, leur emplacement pendant la paix. Je dirai comment les officiers généraux doivent être constitués et employés. Je chercherai la meilleure forme à donner à l’administration du département de la guerre. Il est bien étrange, que tandis que le sort et l’esprit des troupes dépendent des officiers généraux et du ministère, on n’ait jamais fait mention, dans aucun ouvrage, de ces bases de la constitution militaire. Il semble qu’un faux respect, que la crainte d’attaquer des abus trop invétérés et trop puissants, aient empêché d’y jeter les yeux.\par
À la suite du plan de la constitution militaire, je donnerai un cours de tactique complet. Ouvrage bien important, si je réussis à y renfermer tout ce qui a été écrit d’utile sur cette science, tout ce que le roi de Prusse a mis en pratique et ce que l’étude peut y ajouter de découvertes. Ouvrage bien digne d’exciter mon attention, à titre de militaire et de philosophe, puisqu’en proportion de ce que l’art militaire fait des progrès et se perfectionne, la guerre, ce fléau que les passions politiques rendent inévitable, en devient moins funeste et moins ruineuse pour l’humanité.\par
La division de l’ouvrage, que je donne ci-après, servira à en développer encore plus parfaitement le plan. S’il est important pour un auteur, de bien asseoir son projet, si presque toujours, à l’exposition qu’il en fait, on peut juger de la manière dont il le remplira, il n’est pas moins intéressant pour les lecteurs, de pouvoir embrasser d’un coup d’œil le dessein et l’ensemble de l’ouvrage qu’on leur présente. Préparés par ce premier coup d’œil, ils doivent en suivre l’exécution avec plus d’intérêt et de facilité. Ainsi, pour mieux juger la construction d’un édifice, on en étudie auparavant le relief.\par
[…]\par
Voilà le plan immense que j’ai osé concevoir et auquel je travaille depuis plusieurs années. Mais tel est l’inconvénient attaché aux grandes entreprises, dans telle science que ce soit, que, si malheureusement elles ne sont pas poussées sans relâche, si quelqu’événement en suspend ou en ralentit l’exécution, la face des choses change. Des découvertes nouvelles remplacent les connaissances qui existent. Les renseignements amassés à grands frais, vieillissent sans être employés. L’auteur rencontré dans ses idées, se refroidit, se lasse et l’ouvrage reste abandonné. Ainsi dans ces vastes bâtiments, dont la construction est contrariée par des vues d’économie, ou par quelque projet plus nouveau, des matériaux épars et à demi rongés par le temps, des échafaudages inutiles, des parties d’édifice morcelées sans accord et se détruisant à mesure qu’elles s’élèvent, attestent la fragilité et l’inconstance des efforts humains.\par
Cet inconvénient était plus particulièrement attaché à mon ouvrage, qu’à tout autre. Qu’on songe combien de matériaux il faut amasser, pour son exécution ; et ensuite combien rapidement il serait nécessaire de mettre ces matériaux en œuvre. Pour peindre parfaitement la situation momentanée de l’Europe, il faudrait pouvoir arrêter le temps et les changements qu’il amène ; il faudrait, à défaut de ce miracle, pouvoir saisir cette situation, et en faire dans un an le vaste tableau. Sans cette activité, la mobilité des événements, des circonstances, des abus, de lumière, entraîne sans cesse les travaux commencés. J’ai une partie de ces matériaux, je rassemble les autres, je projette de vérifier les plus intéressants par des voyages, je désire ensuite un an de calme et de solitude pour les rédiger ; mais combien de circonstances m’ont déjà contrarié, et sans doute combien d’autres me contrarieront encore.\par
Cependant les années passent. Je vois dans mon pays une constitution militaire, neuve et mal affermie, les opinions flottantes et indécises, les troupes fatiguées de systèmes et d’innovations, aucune notion d’assurée, aucun ouvrage dogmatique qui puisse instruire. Je vois le temps précieux de la paix se perdre dans des minuties dangereuses ; les officiers généraux se circonscrire de plus en plus dans les détails. Je songe qu’une guerre peut nous surprendre dans cet état fâcheux. Je me hâte donc de présenter à ma nation les fruits de mes recherches sur la partie militaire. J’aime mieux les hasarder, détachées de mon grand plan, éloignées de la perfection à laquelle j’espérais les porter, que d’attendre encore quelques années pour les donner, au milieu d’un ouvrage, qui du moins, par la hardiesse de son projet, pourrait les faire valoir et les appuyer.\par
Je n’intitule ces recherches, que du titre {\itshape d’Essai}, parce que ce ne sont en effet que des observations rédigées rapidement et telles à peu près, que je les avais rassemblées dans mon portefeuille, pour leur donner ensuite place et forme dans mon ouvrage. Là, réunies un jour en corps, développées avec le plus grand détail, tenant à un plan de constitution, présentées avec l’enchaînement qui doit lier des vérités l’une à l’autre, j’oserai les appeler : un cours de tactique complet ; et espérer que le public les appellera de même.\par
On dira peut-être qu’il fallait me borner à donner cet {\itshape Essai de tactique}, qu’il y a de l’orgueil à afficher pompeusement un plan qu’on n’a pas rempli et qu’on n’est pas en état de remplir. Je donne ce plan afin que le public le juge, afin qu’il m’encourage ou qu’il m’arrête, afin que les hommes éclairés et qui, par conséquent, doivent s’intéresser au progrès des lumières, me communiquent les leurs et se servent de moi pour les répandre. Enfin en donnant ce plan, j’imite ces élèves des arts. C’est une grande étude que je présente au concours et que je soumets à mes juges. Puisse-t-il s’élever un homme plus capable que moi, auquel ce plan en fasse concevoir un meilleur, ou dont l’exécution du mien tente le génie.
\section[{Introduction à l’Essai Général de Tactique}]{Introduction à l’Essai Général de Tactique}\renewcommand{\leftmark}{Introduction à l’Essai Général de Tactique}

\subsection[{I. - Rareté des bons ouvrages militaires. Obstacles qui l’ont occasionnée jusqu’ici.}]{I. - Rareté des bons ouvrages militaires. Obstacles qui l’ont occasionnée jusqu’ici.}
\noindent De toutes les sciences qui exercent l’imagination des hommes, celle, sur laquelle on a peut-être le plus écrit et sur laquelle il existe le moins d’ouvrages qu’on puisse lire avec fruit, c’est sans contredit la science militaire et particulièrement la tactique qui est une de ses principales branches. Presque toutes les sciences ont des éléments certains, aussi anciens qu’elles et dont les siècles suivants n’ont fait qu’étendre et développer les conséquences. Au lieu que la tactique jusqu’ici incertaine, dépendante des temps, des armes, des mœurs, de toutes les qualités physiques et morales des peuples, a dû nécessairement varier sans cesse, et ne laisser, dans un siècle, que des principes désavoués et détruits par le siècle qui lui a succédé.\par
Supposons les premières vérités mathématiques enseignées à des peuples habitant les deux extrémités de la terre et n’ayant aucune communication entre eux. Ces peuples arriveront peut-être, à quelques années l’un de l’autre, mais arriveront certainement un jour aux mêmes résultats. Mais y a-t-il eu, en tactique, des vérités démontrées ? A-t-on déterminé les principes fondamentaux de cette science ? Un siècle a-t-il été d’accord sur ce point avec le siècle qui l’a précédé ? La tactique grecque n’était pas la même à Thèbes, qu’à Sparte, qu’à Athènes. Elle changeait sans cesse. À l’époque de l’institution de la phalange, elle paraissait à sa perfection. Bientôt l’ordonnance romaine prévalut sur la phalange. Les légions changèrent elles-mêmes vingt fois d’armes et d’ordonnance. La barbarie succéda à la décadence des légions. On retomba dans l’indiscipline. On en revint à l’ordre de profondeur, à la nombreuse cavalerie. Le seizième siècle débrouilla un peu ce chaos. Mais ce qu’il établit, fut détruit, à son tour, par le dix-septième. Aujourd’hui, que toutes les troupes de l’Europe ont les mêmes armes et la même ordonnance, on serait tenté d’imaginer que les principes de la tactique sont déterminés. Ils ne le sont pas davantage. Cette uniformité est une suite de l’esprit d’imitation, qui s’est répandu chez tous les peuples, plutôt que d’une conviction opérée par les lumières. Les militaires et surtout les auteurs militaires, ne sont d’accord sur presqu’aucun point. Celui-là croît l’invention de la poudre l’époque de la perfection de l’art militaire. Celui-ci la regarde comme une invention qui a nui aux progrès de l’art. L’un réclame les piques ; l’autre l’ordonnance de profondeur. L’ordre actuel n’est pas même approfondi. Enfin aucun ouvrage victorieux n’a paru au milieu de tant d’opinions diverses.\par
Pourquoi n’a-t-il paru aucun ouvrage victorieux et qui ait fixé les principes ? C’est que pendant longtemps les militaires n’ont su ni analyser, ni écrire ce qu’ils pensaient. Dans tous les arts, il y a eu des hommes qui ont écrit, avec succès, de leur art. Dans le nôtre, presque tous les grands hommes n’ont point écrit, ou, s’ils ont écrit, ils n’ont pas donné d’ouvrages dogmatiques. Presque toujours des commentateurs pénibles, des faiseurs de systèmes, des hommes sans génie\phantomsection
\label{footnote3}\footnote{Je suis loin de comprendre dans cette classe quelques auteurs respectables qui ont écrit sur différentes parties de la guerre, étrangères à la tactique, comme Vauban. Santa-Cruz, etc. Je n’y comprends certainement pas plusieurs auteurs estimés et vivants, dont les ouvrages ont développé mes connaissances et mon émulation, tels que M. le comte Turpin. M. de Maizerai. M. D. Mesnil Durand, etc. Je parle de ce nombre infini d’écrivains qui ont répandu les ténèbres, la complication et l’ennui sur une science qui peut être rendue intéressante, simple et lumineuse.} ont multiplié les ouvrages, sans étendre les connaissances. De là, l’opinion si triviale et si fausse, quand elle est absolue, que les écrits militaires sont inutiles, que la science ne s’apprend pas dans les livres. De là, le ridicule dont on cherche à couvrir les militaires qui écrivent et surtout ceux qui osent publier leurs recherches ; préjugé qui ne peut que rétrécir les talents et entretenir l’ignorance.\par
Quels livres de tactique peuvent aujourd’hui servir à l’instruction ? Sera-ce Puységur, dont les principes sont, ou faux, ou totalement détruits par la tactique actuelle ? Sera-ce Folard, dont le préjugé soutient la réputation\footnote{On me trouvera hardi de parler ainsi des deux premiers écrivains militaires de la nation. Mais pour encenser de froides cendres, faut-il trahir son opinion ? Faut-il par habitude continuer de regarder comme de bons livres dogmatiques, des ouvrages dont les principes sont pour la plupart faux ou inutiles ? En réfutant ces ouvrages, en les rejetant, je ne respecte pas moins leurs auteurs. Ils ont répandu quelques lumières dans un temps d’ignorance. Eh ! gardons-nous d’imaginer que des hommes qui éclairèrent leur siècle, tussent, s’ils revenaient à la vie, les partisans de leurs fanatiques admirateurs. Ils jetteraient les yeux sur l’état de la science qu’ils cultivèrent et avec les lumières qui les entoureraient à leur réveil, ils feraient de nouvelles découvertes. Quand ces hommes écrivirent, n’osèrent ils pas attaquer les erreurs de leur temps et les ouvrages que les autres siècles avaient honorés ?} ? Guichard, plus instructif que Folard, sur les faits de l’antiquité, mais n’enseignant rien de la tactique moderne ? Seront-ce ces dissertations sur l’ordre de profondeur ? Ces systèmes, tour à tour, détruits et renouvelés ? Seront-ce toutes ces controverses polémiques qui n’ont rien éclairci ? Au milieu de ces ouvrages, on peut trouver des idées utiles, des vues, de l’érudition. Mais avec du génie, avec des lumières, comment n’être pas rebuté de leur aridité, de leurs longueurs, de leur style ? Sans génie, sans lumières, comment y démêler ce petit nombre de vérités, perdu dans un abîme d’erreurs ?\par
Cette disette, en fait d’ouvrages didactiques, n’existe pas également, pour les ouvrages de maximes. César, Rohan, Montecucculi, Turenne, Saxe, le Roi de Prusse en offriront dans tous les temps à qui saura les entendre. Mais il faut remarquer que ces livres ne peuvent pas être mis entre les mains de tout le monde. Ils ne peuvent être médités, que par des généraux formés, ou par des officiers propres à le devenir. La manière dont ces grands hommes ont écrit, n’est ni assez détaillée, ni assez claire. Ils écrivaient, pour se rendre compte à eux-mêmes, plutôt que pour instruire. C’est ainsi que le génie écrit, toutes les fois qu’il ne s’est pas formé le plan bien décidé d’enseigner. Il traite les objets, comme il les a vus, c’est-à-dire rapidement et en planant sur eux. Il ne descend pas dans les détails. Il supprime toutes les idées intermédiaires, par lesquelles le commun des hommes se traîne, avec effort, d’une vérité à l’autre.\par
Un autre genre d’ouvrages militaires que nous possédons en grand nombre, ce sont les mémoires contemporains, les histoires des guerres. Mais combien peu d’hommes sont en état de démêler, dans des faits, les conséquences et les causes ? Combien peu d’hommes savent lire avec fruit ? D’ailleurs combien peu de ces ouvrages sont instructifs ? Combien peu sont faits par des gens de guerre ? Dans la plupart des histoires, je ne vois, en fait d’événements militaires, rien de certain, que le nom des généraux et l’époque des batailles. Ce sont les gazettes du temps, plus ou moins éloquemment rédigées. J’avance que dans le genre didactique, il n’y a presque pas d’ouvrages utiles sur la guerre, qu’il n’y en a surtout presque point d’utiles et d’intéressants à la fois. Oser ensuite en publier un, c’est me faire soupçonner d’orgueil, c’est peut-être prévenir contre moi. Mais dire que personne n’a écrit avec génie sur la science militaire, ou n’a plié son génie à écrire avec utilité, ce n’est pas assurer le public que je réussirai dans mon entreprise, c’est l’avertir seulement que j’en connais l’importance et la difficulté.
\subsection[{II. - Définition de la tactique. Sa division. Son état actuel.}]{II. - Définition de la tactique. Sa division. Son état actuel.}
\noindent Quand même l’histoire ne nous apprendrait pas que les Grecs sont les premiers qui aient réduit l’art d’ordonner les troupes en dogmes et en principes, nous serions forcés d’en convenir, en voyant le nom de cet art tirer son origine d’un mot grec. Ainsi l’Europe militaire voudrait en vain désavouer que les armes et les documents de la France lui ont donné le ton pendant près d’un siècle, presque tous les termes techniques de l’art de la guerre, tirés de notre langue, déposeraient contre elle.\par
Aux yeux de la plupart des militaires, la tactique n’est qu’une branche de la vaste science de la guerre. Aux miens, elle est la base de cette science. Elle est cette science elle-même ; puisqu’elle enseigne à constituer les troupes, à les ordonner, à les mouvoir, à les faire combattre. Elle est la ressource des petites armées et des armées nombreuses ; puisqu’elle seule peut suppléer au nombre et manier la multitude. Elle embrasse enfin la connaissance des hommes, des armes, des terrains, des circonstances ; puisque ce sont toutes ces connaissances réunies qui doivent déterminer ses mouvements.\par
Il faut diviser la tactique en deux parties : l’une élémentaire et bornée, l’autre composée et sublime.\par
La première renferme tous les détails de formation, d’instruction et d’exercice, d’un bataillon, d’un escadron, d’un régiment. C’est sur elle qu’il existe tant d’ordonnances des souverains, tant de systèmes subalternes, tant de contrariétés d’opinions. C’est elle qui agite maintenant nos esprits et qui les agitera longtemps, parce que les détails sont à la portée de tous les esprits ; parce que l’inconstance nationale, quand elle n’est pas contenue, varie sur les principes comme sur les modes ; et parce qu’enfin innover, ou s’attacher aux innovateurs, est devenu un moyen de réputation et de fortune.\par
La seconde partie est, à proprement parler, la science des généraux. Elle embrasse toutes les grandes parties de la guerre, comme mouvements d’armées, ordres de marche, ordres de bataille. Elle tient par là et s’identifie à la science du choix des positions et de la connaissance du pays, puisque ces deux parties n’ont pour but que de déterminer plus sûrement la disposition des troupes. Elle tient à la science des fortifications, puisque les ouvrages doivent être construits pour les troupes et relativement à elles. Elle tient à l’artillerie, puisque les mouvements et l’exécution de cette dernière doivent être combinés sur la position et les mouvements des troupes, puisqu’enfin cette dernière n’est qu’un accessoire, destiné à les seconder et à les soutenir. Elle est tout, en un mot, puisqu’elle est l’art de faire agir les troupes et que toutes les autres parties ne sont que des choses secondaires qui, sans elle, n’auraient point d’objet, ou ne produiraient que de l’embarras.\par
C’est sur cette seconde partie, embrassée sous ce vaste aspect, qu’il n’existe point d’écrits dogmatiques. Quelques auteurs ont traité une ou deux des branches qui la composent ; mais ils n’ont aperçu ni les autres branches, ni la liaison indispensable qu’elles ont toutes entre elles. De là, ces définitions si fausses de la tactique, quand on a cru qu’elle se bornait au seul mécanisme des mouvements des troupes. De là, l’art des tacticiens avili et presque ridiculisé dans l’opinion des ignorants. De là, chaque espèce d’armes se voyant la première et la plus importante. L’infanterie pensant être tout dans les armées, la cavalerie disant à son tour, qu’elle seule décide les batailles, l’artillerie s’imaginant qu’en elle résident la force et les grands moyens de destruction, les ingénieurs voyant toute la sublimité de la guerre dans leurs angles et dans leurs travaux, l’état-major de l’armée la voyant dans des reconnaissances de terrains et dans des supputations locales. De là, les troupes légères, devenues si nombreuses aujourd’hui et se voyant les seuls corps agissants et guerriers. Prétentions fondées sur ce que chacun ne voit que l’utilité dont il est dans sa sphère. Prétentions, toutes fausses, quand elles sont exclusives, toutes preuves d’ignorance et de la rareté des grandes vues. Prétentions qui rappellent cet apologue, où le bras, l’œil, la main disent : « C’est moi qui suis le corps, en moi résident tout le mouvement et toute l’utilité ».\par
Revenons à cette seconde partie. Négligée quand les beaux jours de Rome furent finis, entièrement perdue, sous les ruines de l’empire d’occident, inconnue depuis pendant plusieurs siècles, elle fut un moment relevée par Nassau, par Gustave et par les grands hommes qu’ils formèrent. Mais après eux elle ne fit aucun progrès. Les armées devinrent plus surchargées d’embarras. Il se fit de grandes innovations dans les armes et dans l’ordonnance. Les principes établis ne convinrent plus, on les abandonna et on n’en substitua pas de nouveaux. Depuis la fin du dernier siècle surtout, le hasard et la routine firent mouvoir les armées. Puységur posa quelques principes, au milieu de beaucoup d’erreurs. Saxe, dont on ne peut contester la gloire et la science, connaissait l’ignorance de son siècle. Il le dit dans son ouvrage. On y sent son génie entrevoyant l’art qu’il n’eut pas le temps de créer. Cette gloire était réservée au roi de Prusse. Il fit voir à l’Europe le phénomène d’une armée nombreuse et en même temps manœuvrière et disciplinée. Il fit voir que les mouvements de cent mille hommes sont assujettis à des calculs aussi simples, aussi certains, que ceux de dix mille ; que le ressort, qui fait mouvoir un bataillon, étant une fois trouvé, il ne s’agit plus que de combiner une plus grande quantité de ces ressorts et de savoir les manier. Ses victoires ont prouvé la bonté de ses découvertes. On s’est jeté en foule sur ses documents. On l’a imité au hasard et sans méditation. On a copié le costume de ses troupes, les dehors de la discipline et jusqu’aux vices de sa constitution. Mais ses grands principes n’ont pas été et ne sont pas encore aperçus.\par
Telle est enfin aujourd’hui la situation des esprits en France, par rapport à cette révolution de principes, que la plus grande partie des officiers, attachés aux vieux préjugés et rebutés par quelques innovations, peut-être trop peu réfléchies, rejettent tout et ne veulent pas même ouvrir les yeux pour examiner. L’autre partie et l’on ne sait laquelle des deux nuits le plus au progrès des lumières, dépasse le but du ministère, trompe la bonté de ses intentions, veut innover sans avoir calculé comment on remplacera, fatigue les troupes d’opinions mal digérées et prépare ainsi par le discrédit que sa conduite jette sur toutes les innovations futures, des difficultés plus grandes à la vente et au génie.\par
La tactique, divisée en deux parties et développée, comme je conçois qu’elle peut l’être, est simple et sublime. Elle devient la science de tous les temps, de tous les lieux et de toutes les armes ; c’est-à-dire que, si jamais, par quelque révolution qu’on ne peut pas prévoir dans l’espèce de nos armes, on voulait revenir à l’ordre de profondeur, il ne faudrait changer, pour y arriver, ni de manœuvres, ni de constitution. Elle est, en un mot, le résultat de tout ce que les siècles militaires ont pensé de bon, avant le nôtre et de ce que le nôtre a pu y ajouter.\par
Il serait hardi, il serait insensé à moi de parler ainsi, s’il était question d’une science dont je fusse le créateur. Mais ce sont, en partie, les principes du roi de Prusse, que je vais exposer. Ce sont les idées de plusieurs militaires, éclairés et studieux. Ce sont celles de mon père. Quarante ans de service et de travail, lui ont acquis le droit d’en avoir à lui. Ce sont les miennes, refroidies par son expérience. Je ne suis ici, en quelque sorte, que le rédacteur et le commentateur. Les principes m’ayant été donnés et prouvés, je n’ai fait qu’en développer et en rassembler les conséquences.\par
Abrégeons cette apologie, elle n’empêchera pas qu’on ne me critique. Elle n’empêchera pas que, si j’expose des opinions évidentes, beaucoup de gens ne s’y refusent. J’ai assez vécu, pour savoir que tout auteur encourt le blâme et que la vérité filtre à travers les préjugés, tandis que les erreurs se répandent en torrent.
\subsection[{III. - Influence que le génie des peuples, l’espèce de leur gouvernement et de leurs armes, ont sur la tactique.}]{III. - Influence que le génie des peuples, l’espèce de leur gouvernement et de leurs armes, ont sur la tactique.}
\noindent Autrefois chaque nation avait son armure, sa tactique, sa constitution particulière, parce que les peuples, plus séparés les uns des autres, avaient un génie, un gouvernement et des mœurs à eux. Ces différences d’armure et de génie durent nécessairement varier l’ordonnance de chaque peuple. Il fallait aux Grecs, armés de piques, un ordre condensé, qui les unît et qui favorisât leur impulsion. Ils étaient ingénieux, ils raffinèrent la tactique, ils en firent un art de complication et de calcul, où chaque homme, chaque file eut son nom. Les Romains, armés de pilums, d’épées et d’autres armes de main, eurent besoin de plus d’espace et de liberté, dans leurs rangs. Moins subtils et plus guerriers que les Grecs, ils créèrent un ordre plus simple, plus maniable, plus avantageux, en ce qu’il leur permettait de marcher plus rapidement et de s’entre secourir. La cavalerie numide et espagnole, armée de lances, dut ne combattre que sur un rang et avec de grands intervalles, afin de prendre librement carrière et de manier plus facilement cette arme. Ainsi fit la cavalerie thessalienne, qui était à demi-nue et armée de haches, tandis que la cavalerie grecque et la romaine, plus massives et armées d’épées, formèrent plusieurs rangs. Les Gaulois, robustes, ignorants et braves, méprisèrent toute espèce de tactique et s’armèrent d’épées. Les Francs, plus braves encore et plus impétueux, allaient à l’ennemi, avec de grands cris, et n’ayant pour armes, qu’une espèce de hache, appelée francisque, qu’ils lançaient, à dix pas de l’ennemi, se servant ensuite d’une épée courte et tranchante.\par
Jusqu’à l’époque de la découverte des armes à feu et même jusqu’à la fin du dernier siècle, le génie des peuples influa encore sur leur ordonnance et sur leur armure. Qu’on parcoure l’histoire, on verra la cavalerie allemande, toujours pesante, tenir aux lances, aux armures de toutes pièces, escadronner sur trois rangs et pouvoir, ainsi formée, en envoyer un à la charge et contenir les deux autres. L’infanterie de cette nation était toute composée de gens de traits et d’arquebusiers, la première de l’Europe pour les armes de jet et de feu, la plus molle pour les attaques et pour les combats de corps à corps. L’infanterie suisse, armée de piques, était propre à tous les ordres de consistance et de profondeur, à cause de son flegme et de l’ordre inaltérable qu’elle observait dans ses files. Il en était de même de l’infanterie espagnole. Il était alors à peine question en Europe des Russes et des Prussiens. Les Danois, constitués, à peu près, comme les peuples du nord de l’Allemagne, se modelaient sur eux. Il en était ainsi des Suédois, à l’exception de cette époque brillante et passagère, qu’ils eurent sous Gustave. Les Français étaient sans ordre et sans discipline, peu propres aux combats de feu et de plaine, redoutables dans toutes les attaques de postes et d’épée. Ils avaient alors, comme aujourd’hui, ce premier moment de vigueur et d’impétuosité, ce choc, qu’un jour rien n’arrête, et que, le lendemain, un léger obstacle rebute, ce mélange incroyable d’un courage quelquefois au-dessus de tout et d’une consternation portée quelquefois jusqu’à la faiblesse. Notre cavalerie fut la première à renoncer à la formation de profondeur, à cause de la difficulté qu’elle trouvait à observer ses files. Toute la cavalerie de l’Europe avait conservé ses armes défensives, faisait usage du feu, combattait sur trois rangs, en masse et au trot. La nôtre seule était nue, armée d’épées, formée sur deux rangs et allait à la charge, à toutes jambes et sans ordre. Les Anglais n’avaient point de tactique, rarement de grands généraux, mais un ordre, qui tient à la trempe de leurs armes, un courage peu capable d’offensive, mais difficile à faire reculer. Ils attendaient, dit un historien, en parlant des journées de Verneuil, de Crécy, d’Azincourt, que l’ignorance et l’impétuosité françaises vinssent se briser contre leur sang-froid et leurs palissades. Il est intéressant, pour la philosophie, de remarquer combien le caractère des nations se retrouvait ainsi dans leur milice et par quelle révolution il est devenu moins sensible et moins marqué dans les milices actuelles.\par
Maintenant tous les peuples de l’Europe étant, en quelque sorte, mêlés et confondus par la similitude des principes de leurs gouvernements, par celle de leurs mœurs, par la politique, par les voyages, par les lettres ; les préjugés nationaux, qui les séparaient autrefois, n’existent plus. Avec ces préjugés, s’effacent insensiblement ces traits caractéristiques, qui étaient imprimés sur chacun d’eux, ces traits, dans lesquels consiste le génie national et qui sont autant l’effet des mœurs et des gouvernements, que du physique et du climat.\par
Aujourd’hui donc toutes les nations de l’Europe se modèlent les unes sur les autres. Mais c’est dans les constitutions et les méthodes militaires que cette imitation est la plus marquée et la plus générale. Toutes les troupes de cette partie du monde, si j’en excepte les Turcs que leurs préjugés et leur religion en séparent, ont les mêmes armes et la même ordonnance. Les mêmes armes, soit parce que le même degré d’entendement et de lumières les éclairant presque toutes, elles ont senti la supériorité des armes à feu, sur les armes de jet des anciens, soit qu’étant toutes devenues molles, oisives, maladroites, inexpertes aux exercices de corps, elles ont dû préférer de concert une arme qui n’exige ni courage, ni force, ni adresse. La même ordonnance, parce qu’ainsi que je l’ai observé ci-dessus, c’est toujours l’espèce des armes qui détermine l’ordonnance des troupes.\par
Aujourd’hui toutes les troupes de l’Europe ont, à quelques légères différences près, les mêmes constitutions, c’est-à-dire des constitutions imparfaites, mal calculées sur leurs moyens et dont ni l’honneur ni le patriotisme ne sont la base. Toutes les armées sont composées de la portion la plus vile et la plus misérable des citoyens, d’étrangers, de vagabonds, d’hommes qui, pour le plus léger motif d’intérêt et de mécontentement, sont prêts à quitter leurs drapeaux. Ce sont les armées des gouvernements et non celles des nations. On ne peut excepter de ce nombre, qu’une partie des troupes de Suède\footnote{Cette partie est ce qu’on appelle en Suède : les règlements nationaux. Ils sont payés en fonds de terre, sur lesquels ils habitent.}, les milices de Suisse et celles d’Angleterre. Car pour les troupes réglées de cette dernière nation, toute républicaine, toute libre qu’elle se vante d’être, comme c’est la Cour qui dispose des emplois et des récompenses, on l’a vue, plus d’une fois, se servir de ces troupes contre le peuple et contre ses franchises.\par
La manière, dont les Anciens faisaient la guerre, était, il faut en convenir, plus propre à rendre les nations braves et belliqueuses. Un peuple, battu à la guerre, éprouvait les dernières misères. Souvent on tuait les vaincus, ou on les traînait en esclavage. La crainte de ce traitement, faisant une forte impression sur les esprits, devait nécessairement porter les peuples à s’occuper de discipline et d’exercices militaires. Elle devait rendre la guerre la première et la plus utile de toutes les professions. Aujourd’hui toute l’Europe est civilisée. Les guerres sont devenues moins cruelles. Hors des combats, on ne répand plus de sang. On respecte les prisonniers. On ne détruit plus les villes. On ne ravage plus les campagnes. Les peuples vaincus ne sont exposés qu’à quelques contributions, souvent moins fortes que les impôts qu’ils payaient à leur souverain. Conservés par le conquérant leur sort n’empire pas. Tous les États de l’Europe se gouvernent, à peu près, par les mêmes lois, et par les mêmes principes. De là nécessairement les nations prennent moins d’intérêt aux guerres. La querelle, qui s’agite, n’est pas la leur. Elles ne la regardent que comme celle du gouvernement. De là, le soutien de cette querelle abandonné à des mercenaires ; l’état militaire regardé comme un ordre onéreux et qui ne doit pas se compter parmi les autres ordres des citoyens. De là surtout l’extinction du patriotisme et le relâchement épidémique des courages. La moitié de l’Europe est habitée par des artistes, des rentiers, la plupart célibataires, gens qu’aucun lien n’attache au sol sur lequel ils vivent et qui affichent hautement cette maxime dangereuse : {\itshape ubi bene, ibi patria}. « La peste est en Provence. Eh bien, disent ces cosmopolites, j’irai habiter la Normandie. La guerre menace la Flandre ; j’abandonne cette frontière à qui voudra la défendre et je vais chercher la paix dans les provinces éloignées. Je porte avec moi mon existence, mon art, ma fortune. Partout la terre nourrit et le soleil éclaire. »\par
Ainsi, tandis que les arts et les lettres ont poli les nations, éclairé les esprits, rendu les mœurs plus douces, les gouvernements n’ont pas su empêcher que les vices des hommes ne tournassent en poisons une partie de ces remèdes salutaires. C’était du progrès des connaissances elles-mêmes, qu’ils devaient tirer des moyens de rendre les peuples plus forts et plus heureux. Il fallait veiller à ce qu’elles ne se portassent que sur les objets utiles, à ce qu’elles n’attaquassent point les préjugés nécessaires. Il fallait soutenir ces préjugés par toutes les ressources de la législation. En vain nos vices eussent tenté de détruire les vertus nationales. Le cri de la nature, l’amour propre, les récompenses, l’honneur, la honte, les peines et surtout l’amour qu’inspirent un bon gouvernement, l’auraient hautement emporté sur eux. Le patriotisme eût repris des forces. Il eût été, non ce fanatisme funeste que nous admirons trop chez les Anciens ; mais cet assentiment réfléchi de reconnaissance et de tendresse, qu’une famille heureuse a pour sa mère. Il fallait empêcher que l’industrie se portât vers les objets de luxe. Cela était facile. Car les arts frivoles ne sont que le produit des lumières humaines mal employées. Ils sont le résultat d’un bon levain, tourné en corruption. Les lettres, contre lesquelles on déclame tant, n’inspirent certainement, ni la soif des richesses, ni la mollesse, ni le goût des superfluités de la vie. Mais cela me ramènerait à ce que j’ai déjà traité dans le discours préliminaire de mon ouvrage et ce n’est point ici mon objet. J’ai voulu observer, relativement à l’influence qui en résulte sur la science de la guerre, quel était l’état actuel de nos mœurs et de nos âmes. Il est certain qu’elles se sont amollies et énervées. Il est certain que le sort des États est devenu dépendant de milices mercenaires, avilies, mal constituées, n’étant excitées au courage par aucun motif, ne gagnant rien à vaincre, ne perdant rien à se laisser battre. Puisque ces vices existent, et qu’ils ne pourraient être corrigés qu’en bouleversant la forme de nos gouvernements, cherchons donc, dans nos lumières, tous les remèdes qu’elles pourront nous procurer ; et tâchons de suppléer, par la perfection de l’art, à la décadence de nos constitutions et de nos courages.
\subsection[{IV. - Plan de cet Essai Général de Tactique}]{IV. - Plan de cet Essai Général de Tactique}
\noindent Lorsque, dans mon autre ouvrage, je donnerai un cours complet de tactique, ce cours sera précédé d’un plan de constitution militaire national, c’est-à-dire, d’un plan calculé sur les moyens, le génie et la puissance de ma nation. Ce plan sera contraire, à beaucoup d’égards, aux idées reçues. Car j’avoue que toutes les constitutions existantes en Europe, sont bien éloignées du point de perfection, soit réel, soit chimérique, que j’ose entrevoir.\par
Je ne présenterai ici que les matériaux de ce cours de tactique. Je les présenterai, sans m’assujettir strictement à l’ordre élémentaire et didactique, dans lequel je me propose de les ranger alors. Je ne parlerai de la constitution des troupes, qu’autant qu’elle m’offrira, sur mon chemin, des abus, ou des choses absolument contraires à l’exécution de mes principes. Disons seulement en passant, que, dans les changements qui s’y sont faits depuis la guerre dernière, on a considérablement gagné sur une infinité de points. Disons, en même temps, qu’on s’est trompé sur quelques autres\footnote{Cet ouvrage était fait avant les changements qui ont eu lieu dans le ministère et je le donne tel qu’il était alors. La vérité n’a pas deux langages, un pour la faveur et l’autre pour la disgrâce. Elle juge les choses et fait abstraction des personnes. Elle blâme sans fiel et loue sans adulation. La nouvelle constitution a des défauts. Elle a coûté à l’État beaucoup d’hommes et d’argent. On s’est dans son établissement trompé de moyens sur plusieurs objets. On a sur d’autres été par-delà le but ou manqué le but. Mais donnons de justes éloges à l’entreprise du ministre qui en est l’auteur, à l’étendue de ses vues. Ne fermons pas les yeux aux bons effets qui ont résulté d’une partie de ses opérations. Louons ce ministre d’avoir senti les vices de l’ancienne constitution, les avantages de la discipline, la nécessité de l’instruction. Louons-le d’avoir secoué le préjugé des vieilles erreurs, d’avoir cherché le bien. C’est avoir beaucoup fait, dans une nation qui est gouvernée par la routine et passionnée pour ses usages.}. Concluons ensuite que, pour refondre une constitution, chose plus difficile que de la créer, il faudrait être souverain ; puisqu’il s’agirait, à beaucoup d’égards, de changer les mœurs de la nation, et la routine de l’administration. Concluons que, fût-ce même un souverain et un souverain homme de génie, qui tentât cette importante entreprise, il faudrait qu’il y employât plusieurs années et qu’il revînt souvent sur ses pas, pour rectifier : car il n’y a que Dieu qui puisse créer d’un seul jet et ne pas retoucher son ouvrage.\par
Quelqu’avantageux qu’il fût que les idées de tactique, que je vais exposer, fussent adaptées au plan de constitution que je projette, elles en sont cependant indépendantes. Elles sont applicables à toutes les constitutions quelconques. Je vais les appliquer à celle de nos troupes. Je les appliquerais de même à celles d’Autriche, d’Angleterre, etc. et voilà en quoi j’ose croire que j’écris, avec plus d’utilité que n’ont fait tous les tacticiens, puisque ces derniers n’ont su autre chose, que fronder tout ce qui était établi et publier leurs vues sur des systèmes de formations qui n’existent pas et qui ne peuvent exister.\par
Je renvoie à l’ouvrage que j’ai annoncé, à parler des armes, de l’habillement et de la discipline intérieure des troupes. Tous ces détails, tenant au plan de constitution, y seront approfondis et sur ces deux derniers objets surtout, que de choses à changer. Que de choses à dire sur cet habillement bizarre, compliqué, pénétrable à toutes les injures de l’air, dont je vois les troupes couvertes ; sur cette manie de tenue, qui absorbe l’officier et désole le soldat ; sur notre prétendue discipline qui ne consiste presque qu’en minuties de forme ; qui, trop appesantie sur les grades subalternes, n’existe point assez dans les grades supérieurs et surtout parmi les officiers généraux, où cependant elle est bien autrement importante, parce que là les fautes de subordination font perdre les batailles et manquer les campagnes. Je m’arrête et je reviens à mon objet.\par
J’ai divisé la tactique en deux parties ; en tactique élémentaire et en grande tactique. C’est cette division que je vais suivre. Je traiterai dans la première partie, de toutes les armes qui entrent dans la composition d’une armée, (savoir, infanterie, cavalerie, artillerie, troupes légères). Je rassemblerai, dans la seconde, ces différentes armes. J’en composerai une armée. Je donnerai une théorie pratique de tous les mouvements qu’elle peut exécuter à la guerre. À la suite de cette théorie et afin que mon plan contienne tout ce qui appartient à la tactique, j’examinerai le rapport que la science des fortifications et la connaissance des terrains doivent avoir avec la tactique et particulièrement avec la guerre de campagne. Je parlerai de la manière dont nous faisons subsister nos armées et des changements avantageux qu’on pourrait faire à cet égard.
\section[{Première partie. Tactique élémentaire}]{Première partie. Tactique élémentaire}\renewcommand{\leftmark}{Première partie. Tactique élémentaire}

\subsection[{I. - Éducation des troupes}]{I. - Éducation des troupes}
\noindent C’est une chose bizarre que l’espèce d’instruction que l’on donne aujourd’hui aux troupes. Elle ne roule que sur un maniement d’armes et sur quelques manœuvres la plupart compliquées et inutiles à la guerre Qu’il y a loin de cette misérable routine, à un système d’éducation militaire, qui commencerait par fortifier et assouplir le corps du soldat, qui lui apprendrait ensuite à connaître ses armes, à les manier, à exécuter toutes les évolutions qu’il doit savoir : à se livrer, dans l’intervalle de ces exercices et comme par délassement, à des jeux propres à entretenir sa force et sa gaieté ! Après qu’on aurait ainsi dressé le soldat, on le familiariserait avec des représentations simulées de tout ce qu’il doit faire à la guerre. Il saurait porter des fardeaux, remuer la terre, faire des marches forcées, passer des rivières à la nage, travailler, avec adresse, à toutes les parties d’un retranchement. Passant une partie de sa vie dans des camps, il acquerrait l’habitude du service qu’il y doit faire, de la conduite qu’il doit tenir dans un poste avancé, en faction, en patrouille. Au moyen des grandes manœuvres qui se feraient dans ces camps, il s’accoutumerait à l’ordre qu’il doit observer dans les marches, au spectacle d’une armée, au bruit de l’artillerie, au concours des autres armes avec la sienne. Dans les exercices des places, on lui ferait contracter l’habitude machinale des travaux de tranchée et de défense. On lui apprendrait à couper une palissade, à la planter, à dresser une échelle, à attacher un pétard, ou à soutenir les gens qui l’attachent, à ouvrir un créneau, à savoir y placer, etc. Accoutumé dans toutes les circonstances à garder le silence, à obéir aux signaux et à la voix de ses officiers, à ne pas s’emporter au-delà du point attaqué, connaissant enfin toutes les situations que la guerre peut offrir ; le soldat la désirerait sans cesse. Au danger près, la paix elle-même serait pour lui une guerre continuelle.\par
Il y aurait, dans un système d’éducation pareil, une instruction progressive et relative à tous les grades ; car, où le soldat apprendrait les devoirs de soldat, l’officier subalterne apprendrait à conduire sa troupe, le capitaine sa compagnie, le colonel son régiment, l’officier général sa division, le général son armée.\par
Je ne parle pas de cette autre partie de l’éducation militaire, qui formerait le courage, les mœurs, les préjugés, partie si importante, mais si négligée, si inconnue à tous les généraux et à tous les gouvernements, que je ne vois dans l’histoire ancienne et moderne qu’un seul homme\footnote{C’est de Caton commandant les armées romaines en Espagne que l’histoire fait ce bel éloge.} dont on ait dit, il ne lui suffisait pas que ses soldats fussent braves, il voulait qu’ils fussent honnêtes gens.\par
Il faudrait donc que l’éducation du soldat embrassât trois objets, l’un les exercices du corps, le second les exercices d’armes et d’évolutions, le troisième la représentation des différentes situations, où l’on peut se trouver à la guerre. Ce sera là le plan que je suivrai dans mon cours de tactique.\par
Le premier de ces objets, enseigné même hors du service, devrait entrer dans l’éducation de toute la jeunesse du royaume. Qu’en France, où le prince est tout, où son exemple est législateur, où ses mœurs déterminent les mœurs publiques, un roi veuille ramener ses courtisans à une vie agissante et militaire ; que la sienne soit telle, qu’il fasse élever ses enfants dans ce principe, qu’il assiste à leurs exercices, qu’il flétrisse de son indifférence les jeunes gens oisifs, voluptueux, ignorants, qu’il distingue les autres. Bientôt on verra disparaître la mollesse, le libertinage, la débauche obscure et ruineuse et tous les petits vices qui dégradent les grands seigneurs. Bientôt à la génération actuelle succédera une génération propre à la guerre et à la gloire. Ce champ de Mars que l’herbe couvre et dont la Seine baigne inutilement les bords, ressemblera à ce champ fameux qu’arrosait le Tibre. On s’y exercera à vaincre. Les statues de Henri, de Condé, de Turenne, en décoreront l’enceinte, elles crieront à leurs descendants : « ces piédestaux t’attendent ». De la cour et de la capitale, l’esprit d’honneur et de courage passera dans les provinces étonnées. La noblesse, revenue des petites jouissances de luxe et de mollesse, abandonnera les villes pour rentrer dans ses châteaux. Là elle se trouvera, plus heureuse et moins confondue. Elle reprendra les mœurs de ses aïeux. En conservant ses lumières, elle redeviendra guerrière et galante. Le goût des armes et des exercices militaires ramené dans la noblesse, passera bientôt chez le peuple. La bourgeoisie ne regardera plus l’état de soldat comme un opprobre. La jeunesse des campagnes ne craindra plus de tomber à la milice. Elle s’assemblera les dimanches et fêtes pour disputer des prix de saut, de course et d’adresse. Ces prix que le gouvernement fonderait dans chaque paroisse, vaudraient mille fois mieux que la stérile et coûteuse assemblée annuelle des milices. Car ayez des paysans vigoureux, lestes, déjà accoutumés au bruit des armes et à les manier. Ayez en même temps une bonne discipline et des officiers, vous formerez bientôt des soldats. Qu’on ne croie pas au reste qu’une révolution pareille dans les esprits et dans les mœurs fût funeste ni à l’agriculture ni à la tranquillité du royaume. Une nation ainsi constituée n’en serait que plus portée et plus endurcie aux travaux. Ce sont les peuples laboureurs qui sont les plus guerriers. Qu’on se rappelle les Romains dans leurs beaux jours, qu’on voie les Suisses. L’état y gagnerait la réforme d’une partie de ces armées nombreuses qu’il entretient sur pied. Lorsqu’un pays entier est militaire, au premier signal tous ses habitants sont ses défenseurs. Quant à la tranquillité publique, elle n’en serait que plus assurée. L’histoire le prouve. Où se formèrent la fronde et la ligue ? Dans Paris, au milieu de cette populace lâche, corrompue, avide de nouveautés qui habite les villes. L’habitant de la campagne occupé de sa culture, attaché à l’espoir de sa récolte, chérit la paix et les lois qui la lui donnent. Disons-le enfin, jamais la crainte des révolutions ne doit en pareil cas arrêter les opérations de la saine et sage politique. Les gouvernements ne les redoutent que quand ils sentent leur faiblesse ou leur injustice.\par
Peut-être aurais-je dû réserver ce tableau pour mon grand ouvrage où le développement des objets qui y ont rapport, le rendront plus sensible ; mais les vérités de sentiment oppressent et forcent à parler.\par
Si enfin l’on ne veut pas que le royaume entier devienne une école de travaux et de guerre, il faudrait du moins que lorsque les soldats sont enrôlés, les exercices de corps fissent une partie considérable de leur instruction. Il est étrange qu’uniquement dressés à manier un fusil et à garder pendant trois heures des attitudes pénibles et contraires au mécanisme du corps, ils n’aient, quand la guerre arrive, aucune habitude des travaux qu’elle exige. Aussi une marche tant soit peu forcée les étonne. Un ruisseau les arrête. Quatre jours de terrassement les rebutent. Si l’on me dit que nos exercices actuels les occupent déjà assez, je répondrai que c’est parce que nos manœuvres sont compliquées. Nos méthodes d’instruction mal entendues. Notre prétention de précision et de perfection, sur beaucoup de points, minutieuse et ridicule. Je répondrai que la preuve, que nos soldats ne sont pas assez occupés, c’est que, pour remplir, dit-on, leur temps, on les surcharge de règles de discipline inquiétantes et odieuses. C’est qu’on a créé une tenue qui leur fait passer trois heures par jour à leur toilette, qui en fait des perruquiers, des polisseurs, des vernisseurs, tout, en un mot, hormis des gens de guerre\footnote{C’est l’excès de la tenue que j’attaque et non la tenue en elle-même. Portée jusqu’à un certain point, elle est nécessaire. Elle est une preuve de discipline. Elle contribue à la santé du soldat. Elle l’élève au-dessus du peuple. Elle le met dans la classe des citoyens aisés et heureux. Elle n’était pas négligée chez les Romains. Elle se portait particulièrement sur leurs armes ; mais elle ne les amollissait pas. Elle ne les empêchait pas de s’occuper de travaux durs et pénibles. Ces derniers faisaient la base et l’objet principal de leur éducation. Une armée romaine essuie des revers en Espagne. On envoie Caton pour la commander. Il la trouve répandue dans des quartiers, indisciplinée, amollie, chargée d’or et de honte. Les soldats étaient parés comme des femmes, ils prenaient des bains parfumés. Caton les fait camper, les exerce, les tient sans cesse en mouvement, les accable de travaux. « Romains indignes, leur disait-il, jusqu’à ce que vous sachiez vous laver dans le sang, je vous laverai dans la boue. » Il leur fit désirer les combats et on juge bien qu’ils les gagnèrent. Au reste cette manie de tenue, contre laquelle je m’élève avec force, parce qu’elle dégoûte le soldat, parce qu’elle l’amollit, parce qu’elle absorbe un temps qui pouvait être plus utilement employé, était peut-être inévitable dans un renouvellement de constitution. Il était presque impossible que, de l’extrême négligence, on ne passât point à l’extrême recherche. On est tombé dans le même inconvénient par rapport à nos méthodes de discipline, aux exercices de fusils, aux évolutions, aux écoles d’équitation. Nos têtes sont si légères ; elles fermentent avec tant d’activité ! Trop de récompenses, mal à propos distribuées aux officiers qui ont montré du zèle, dans l’établissement du nouveau système, de grandes fortunes faites par ces petits moyens, ont achevé d’emporter la plupart des inspecteurs et des chefs de corps au-delà du but. Il est malheureusement des points importants dont on ne s’est point occupé. On n’a point formé d’officiers généraux. On n’a point songé à la grande tactique, à l’organisation des armées, aux grandes parties de la guerre. La guerre arrivera. On éprouvera des malheurs. On les rejettera sur la constitution. On ne manquera pas de dire qu’il ne fallait point faire de changement ; que ce sont les changements qui ont tout perdu ; que tout allait bien auparavant ; qu’on battait les ennemis. Alors s’élèveront de toute part les mécontents, les envieux, les faiseurs de système, les anciens officiers opiniâtres dans leur routine. On renversera tout. On retombera dans le relâchement et ce relâchement sera d’autant plus grand, que l’excès contraire aura été porté plus loin. Car tel est le malheur de presque toutes les administrations et de celle de France surtout, qu’elles embrassent trop souvent des systèmes exclusifs : qu’elles négligent les objets, ou qu’elles s’en occupent trop, tour à tour, qu’enfin elles se balancent sans cesse d’un excès à l’excès opposé.}. Et que résulte-t-il de cette vie fainéante et pourtant pénible de ces travaux qui se font la plupart assis et à l’ombre ? C’est qu’un soldat, qui a servi dix ans, ayant perdu toute souplesse, toute aptitude aux travaux du corps, est contraint de se faire artiste, laquais, ou mendiant. Qu’arriverait-il de l’échange de ces occupations frivoles en travaux durs et pénibles ? C’est qu’un laboureur serait plus propre à être soldat : c’est qu’un soldat, quittant ses travaux, reprendrait sans peine la bêche et la charrue.\par
Mais pour terminer cet important chapitre, en vain formera-t-on des soldats, des soldats endurcis et guerriers, comme les anciens légionnaires, si on ne remet cette profession en honneur, si on n’attache le soldat à elle par des perspectives lucratives ; si on n’augmente sa paye\footnote{C’est là le plus grand de tous les vices actuels de notre constitution. La misère de nos soldats est une des principales causes de l’avilissement de cette profession. Dans la plupart des garnisons du royaume, ils n’ont pas de quoi se nourrir. Il est incroyable par quelle complication de petit détails, de moyens parcimonieux et abusifs, les chefs de corps sont obligés de suppléer à la modicité de la solde. C’est avec 6 livres et 8 deniers par jour, que le roi paie, habille, équipe et nourrit un soldat. C’est avec 3 francs restant après les retenues pour la masse d’habillement, pour celle de linge et de chaussure pour la livre et demie de pain, souvent d’une qualité très médiocre, qu’on lui donne, que ce soldat est obligé de pourvoir à la subsistance et à son entretien journalier. C’est avec cela qu’il faut qu’il soit poudré, ciré, vernissé, en un mot, sans trou ni tache. C’est ce soldat, attristé de son état, fatigué de ce qu’on exige de lui, enchaîné par la discipline, surchargé, dans ses casernes, d’une infinité de petites règles monastiques nécessaires sans doute, mais que son attachement à sa profession pourrait seul lui faire support. C’est cet homme, souvent exténué par une modique nourriture, toujours réduit à boire de l’eau, privé de toute espèce de divertissement, humilié par l’insolente fainéantise de la livrée, par le mépris du dernier bourgeois, par la dépense que le plus pauvre artisan fait pour sa récréation les jours de dimanches et de fêtes. C’est ce soldat, n’ayant au-dessous de lui, dans la classe des malheureux, que l’homme manquant de tout, ou ce journalier de nos campagnes, qui partage, avec sa famille, un pain trempé de sueur et de larmes. C’est lui qui doit défendre la patrie et verser son sang pour elle. C’est de lui qu’on a l’injustice d’exiger de l’honneur et des vertus. Nos constitutions militaires se bouleversent depuis un siècle, sans qu’on remédie à ce vice primitif, sans qu’on veuille sentir qu’avant de discipliner et d’instruire des troupes, il faudrait leur donner de la considération et les nourrir.}, cette paye immobile, depuis deux cents ans, tandis que les denrées et les salaires ont, de toute part, triplé et quadruplé autour d’elle ; si on ne lui fait désirer la guerre et trouver, à la guerre, des récompenses ; si enfin on n’assure des secours à sa vieillesse, à ses blessures, à ses infirmités, à sa femme, à ses enfants. Tous ces grands objets seront remplis dans mon plan de constitution. Ce plan sera peut-être regardé comme un rêve. Il sera si éloigné des principes actuels. Mais que m’importe ? Quelques-unes des vérités utiles qu’il renfermera, seront peut-être adoptées. Quelques autres germeront plus lentement et leur fruit sera recueilli un jour. En un mot, la totalité de mon ouvrage, jusque dans ses erreurs, sera un monument de mon amour pour le bien.
\subsection[{II. - Tactique de l’infanterie}]{II. - Tactique de l’infanterie}
\subsubsection[{1. Ordonnance de l’infanterie, sa formation. Principes qui doivent déterminer l’une et l’autre.}]{1. Ordonnance de l’infanterie, sa formation. Principes qui doivent déterminer l’une et l’autre.}
\noindent Je passe sur les définitions des premiers termes techniques de la tactique. Je n’écris point pour les commençants. Un jour je réduirai en forme de cours les idées que je vais exposer ici, et alors je tâcherai de les présenter d’une manière didactique, peut-être plus claire et moins rebutante, que celle dont on s’est servi jusqu’à présent.\par
Constituée et armée uniformément, comme elle l’est aujourd’hui, il n’y a plus qu’une sorte d’infanterie. De là plus qu’une ordonnance, variée, à la vérité, suivant les terrains, mais toujours la même dans sa base et dans son principe. Voilà un avantage de simplicité que je trouve à notre tactique par-dessus celle des Anciens. Ils avaient de l’infanterie pesante et de l’infanterie armée à la légère. Ils étaient par conséquent obligés d’avoir une ordonnance pour chacune d’elles. Notre infanterie au contraire réunit les deux propriétés, puisque le fusil armé de sa baïonnette est à la fois arme de jet et arme de main. Elle est propre par le fusil aux combats de jet et par la baïonnette aux combats de choc.\par
Je ne puis m’empêcher de remarquer ici combien le fusil, armé de sa baïonnette, me semble une arme supérieure à toutes celles des Anciens. Il pourrait cependant encore se perfectionner. On pourrait surtout tirer un plus grand parti de la baïonnette. Il y aurait une sorte d’escrime à apprendre, pour se servir de cette arme, pour la croiser, pour empêcher le plus fort de gagner etc. J’aurai l’occasion de revenir sur cela par la suite. Reprenons l’exposition de mes principes.\par
L’infanterie étant propre à l’action de feu et à l’action de choc, il lui faut une ordonnance qui lui permette l’usage de ces deux propriétés. Au cas que la même ordonnance ne puisse servir pour les deux objets, il faut que de celle qui sera déterminée devoir être l’ordonnance habituelle et primitive, elle puisse facilement et promptement passer à l’ordonnance accidentelle et momentanée, qui remplira le second objet. Mais laquelle sera l’ordonnance primitive et habituelle ? L’ordonnance de feu ou celle de choc ? C’est une question qui mérite d’être discutée avec quelques détails et examinée avec l’attention la plus réfléchie : car, comme on l’a dit avant moi, j’ignore l’art d’être clair pour qui ne veut pas être attentif.\par
Avant que d’être en mesure d’aborder l’ennemi, il faut se mettre en bataille, il faut arriver à lui, il faut ne pas être détruit, ou mis en désordre par l’effet de son feu. Il faut lui faire craindre du feu à son tour. Donc il est nécessaire que l’ordonnance primitive et habituelle soit l’ordonnance propre au feu, c’est-à-dire l’ordre mince. Je déterminerai ci-après quelle proportion cet ordre devra avoir.\par
La multiplicité de l’artillerie, la science du choix des postes, celle des retranchements, ont rendu aujourd’hui les actions de choc infiniment rares. Donc celles de feu étant plus communes, c’est une raison de plus pour que l’ordonnance propre au feu soit l’ordonnance primitive et habituelle.\par
Mais les circonstances, dira-t-on, la nature du terrain, la situation de l’ennemi peuvent exiger qu’on aille à lui sans tirer et qu’on engage une action de choc. D’accord ; je suis plus partisan que personne de cette manière d’attaquer. C’est celle du courage, c’est celle de la nation, c’est presque toujours celle de la victoire. Je vais prouver cependant que l’ordre mince, à quelques occasions près, est encore le plus avantageux et le plus favorable pour engager une action de choc.\par
Commençons pour cela par détruire l’ancien préjugé, d’après lequel on croyait augmenter la force d’une troupe, en augmentant sa profondeur. Toutes les lois physiques sur le mouvement et le choc des corps, deviennent des chimères, quand on veut les adapter à la tactique. Car premièrement, une troupe ne peut se comparer à une masse, puisqu’elle n’est pas un corps compact et sans interstices. Secondement, dans une troupe qui aborde l’ennemi, il n’y a que les hommes du rang qui le joint qui aient force de choc. Tous ceux qui sont derrière eux ne pouvant se serrer et s’unir avec l’adhérence et la pression qui existerait entre des corps physiques, ils sont inutiles et ne font souvent qu’occasionner du désordre et du tumulte. Troisièmement, ce prétendu choc pût-il avoir lieu de manière que tous les rangs y contribuassent, il existe dans une troupe composée d’individus qui machinalement du moins, calculent et sentent le danger, une sorte de mollesse et de désunion de volontés qui ralentit nécessairement la détermination de la marche et la mesure du pas. Donc, plus de quantités entières de mouvement, plus de produit de masses et de vitesse, plus de choc. Car le choc suppose que la vitesse une fois imprimée au corps mû par la cause motrice, continue jusqu’à la rencontre du corps choqué.\par
Il s’ensuit donc, me répliquera-t-on, que niant que la profondeur de l’ordonnance ajoute à sa force, vous voudriez que l’infanterie fût rangée sur un homme de hauteur. Non, je veux que la profondeur de l’ordonnance soit déterminée par l’espèce d’armes et par la protection que ces armes peuvent porter au premier rang. Or trois hommes l’un derrière l’autre et bien exercés peuvent tirer avec facilité. Les baïonnettes du second et troisième rangs peuvent, quand les rangs se serreront, former fraise et appui pour le premier. Donc je veux qu’on se forme sur trois, et jamais sur quatre ni sur six dans aucun cas, parce que, par-delà trois hommes de profondeur on ne tire ni feu ni augmentation de force, des rangs qui sont derrière eux.\par
S’il arrive enfin que la nature du terrain qui conduirait à couvert sur l’ennemi, ou l’attaque d’un retranchement, ou quelqu’autre circonstance qui doit être habilement et promptement jugée, rende la diminution du front nécessaire pour se renforcer sur un point, y attaquer et y percer, je dis qu’il faut forme l’infanterie en colonne. Mais ce ne sera pas pour avoir la pression exacte et chimérique dont ont parlé quelques tacticiens, ni pour augmenter la prétendue force de choc. Ce sera pour se procurer cette succession continue de mouvement qui fasse qu’une division entraînée par la division suivante, soit comme forcée d’arriver sur le point où l’on veut faire effort. Ce sera surtout, parce que cet ordre donne de la confiance au soldat et intimide l’ennemi. Car la plupart des hommes n’ayant pas les idées justes et ne voyant que par les yeux du corps, attribuent gain de cause à la troupe qui leur paraît la plus épaisse et qui rassemble le plus d’hommes sur un même point.\par
Voici le résumé de ma discussion. {\itshape L’ordonnance primitive, fondamentale et habituelle de l’infanterie sera sur trois de profondeur. L’ordonnance momentanée et accidentelle sera en colonne.} Il s’agit de trouver les moyens de passer de l’une à l’autre de ces ordonnances par des mouvements simples et rapides. C’est ce que je ferai par la suite. Mais enfin, restera-t-il à dire aux ennemis de l’ordre actuel, comment faire marcher une ligne ainsi mince et flottante ? Comment remuer un bataillon dont on a ainsi étendu le front aux dépens de sa profondeur ? Le voici : c’est en divisant une troupe nombreuse en plusieurs parties qu’on peut parvenir à la mouvoir avec facilité. Ce sont ces divisions, connues de tout temps dans la tactique, qu’on appelle régiment, bataillon, escadron, compagnie, division etc. […].\par
Le nombre impair est la base de toute ma formation. Trois divisions appelées tout naturellement divisions de droite, de gauche et de centre, et subdivisées chacune en trois compagnies dont une d’élite, forment mon bataillon trois bataillons mon régiment, trois grandes divisions, l’infanterie d’une armée […].\par
Les Grecs et, à leur exemple, Gustave, Charles XII avaient adopté le nombre trois, pour principe de leur formation. Ce n’est pas qu’ils attribuassent quelque vertu à ce nombre trinaire et merveilleux. C’est qu’ils avaient réfléchi qu’il apportait plus de commodité et de simplicité que tout autre, dans les calculs de tactique […].\par
Je veux tirer parti de ce qui existe, je prétends qu’avec quelques légers changements nos bataillons sont de toutes les formations la plus avantageuse. J’entreprends de prouver qu’au moyen de la tactique que je vais développer, nos bataillons réuniront les propriétés de feu, de choc, de simplicité, de légèreté de solidité et même, quand on le voudra, celle de profondeur.
\subsubsection[{2. École du soldat, maniement d’armes, formation des rangs et des files}]{2. École du soldat, maniement d’armes, formation des rangs et des files}
\noindent Je n’entrerai que le moins que je pourrai dans les détails didactiques de ces différents objets. Je veux me hâter d’arriver aux évolutions, à la partie intéressante de la tactique, parce que c’est sur elle que manquent les lumières et les principes.\par
Il n’est point indiffèrent en soi-même que le soldat soit dressé par telle ou telle méthode. S’il y a plusieurs moyens d’y procéder, il y en a un sans contredit, qui est le plus court, le plus simple et le plus conforme au mécanisme du corps […].\par
Le premier objet auquel on doit s’attacher quand on dresse un soldat, c’est de lui donner l’air et la démarche militaires. Il acquerra bientôt l’un et l’autre. Si les exercices de corps, remis en honneur et en coutume dans la nation, l’ont déjà occupé les dimanches et fêtes dans son village et si, arrivé dans les troupes, ils continuent de faire l’objet de ses jeux et de son émulation. Quel est l’effet des exercices de force et d’adresse ? C’est d’assouplir le corps, de placer tous les membres dans leur équilibre, de donner à chacun d’eux toute l’action dont il est susceptible. Que reste-t-il à ajouter à cela pour donner l’air soldat à un homme ainsi assoupli et forme ? C’est le port de tête, l’assurance du pas, la fierté du maintien. Il les prendra bientôt, si l’état de soldat est en honneur, si en élevant son âme on l’accoutume à estimer sa profession et à se croire ennobli par elle.\par
Il s’agit ensuite de donner au soldat la position de combat, c’est-à-dire, la position qu’il doit avoir dans le rang et dans la file, d’abord sans fusil, puis avec son fusil. Il la prendra bientôt, si elle n’est ni gênante ni forcée, c’est à dire, si elle n’est point contraire à la mécanique du corps. La nôtre ne lui est certainement pas conforme. Il n’est pas dans la nature d’avancer beaucoup la poitrine, ni de porter le ventre trop en arrière, ni de tourner la tête à droite ou à gauche quand on veut marcher devant soi, ni de rester tour à tour en équilibre sur un pied, ni enfin d’amener un homme à cette position en le mettant à la muraille, à la planche et à toutes les tortures inventées par nos tacticiens subalternes. Veut-on une preuve que notre position n’est ni simple, ni analogue au mécanisme de nos membres ? Qu’on entre dans la plupart de nos écoles d’exercices, on y verra tous ces malheureux soldats dans des attitudes contraintes et forcées. On verra tous leurs muscles en contraction, la circulation de leur sang interrompue. Ajoutez à cela la bizarrerie de notre habillement qui les oppresse, qui serre toutes les articulations, la plate routine des gens qui les enseignent, l’incertitude et l’inconstance des principes.\par
Étudions l’intention de la nature dans la construction du corps humain et nous trouverons la position et la contenance qu’elle prescrit clairement de donner au soldat.\par
Le soldat doit se tenir droit, les épaules effacées, la poitrine ouverte ; parce que ce n’est que dans cette situation que l’estomac et le diaphragme peuvent opérer bien à l’aise la digestion et la respiration. Les épaules étant effacées et la poitrine ouverte, ce poids porté en avant fait nécessairement rester le ventre un peu en arrière pour lui servir de contrepoids. Trop en avant ou trop en arrière, le ventre et la poitrine ne seraient plus dans cet équilibre qui peut seul produire l’aisance et la liberté des mouvements. Les muscles du bas-ventre feraient nécessairement quelque effort pour prendre cette position extraordinaire et aucune partie du corps ne doit agir sur elle-même, c’est-à-dire, faire effort et contraction quand le corps est dans son état de repos. Les reins, qui sont l’appui et l’arc-boutant du tronc humain, ne le soutiendraient plus carrément sur les hanches qui en sont comme le socle et la base.\par
Les mains doivent être pendantes sur le côté sans raideur, sans affectation, abandonnées à leur pesanteur, de manière en un mot qu’elles fassent deux balanciers abaissés par des poids égaux et qui maintiennent les épaules sur une ligne horizontale.\par
La tête doit être droite, dégagée hors des épaules et assise perpendiculairement au milieu d’elles. Elle doit n’être tournée ni à droite ni à gauche ; parce que, vu la correspondance qu’il y a entre les vertèbres du col et l’omoplate auxquelles elles sont attachées, aucune d’elles ne peut agir circulairement sans entraîner légèrement du même côté qu’elle agit, une des branches de l’épaule, et qu’alors le corps n’étant plus placé carrément, le soldat ne peut plus marcher droit devant lui, ni servir de point d’alignement.\par
Les genoux doivent être bien tendus, les deux talons sur une ligne droite à deux pouces l’un de l’autre et non exactement joints, les pieds légèrement tournés en dehors. Je propose de placer les talons à deux pouces et non exactement joints comme nous le pratiquons, parce que dans cette première position la ligne du centre de gravité du corps tombant sur un plan plus spacieux, le corps est plus terme et plus solidement établi.\par
Le soldat étant ainsi placé, il doit garder l’immobilité et le silence, et malgré cela ressembler, non à un automate, mais à une statue animée et prête à agir.\par
Lorsque le soldat sera bien accoutumé à prendre cette position de lui-même, sans efforts et moins comme exercice que comme le placement naturel de son corps, on lui fera porter en avant et ramener successivement et alternativement les deux jambes, de manière que le mouvement parte de la hanche et que le haut du corps ne chancelle pas. On le placera ensuite arme à l’épaule, c’est-à-dire, qu’on lui apprendra à porter son arme sans qu’elle dérange les principes de position établis ci-dessus. Cherchons encore dans la nature et nous allons voir la manière la moins gênante dont l’arme doit être portée.\par
Il faut que le soldat puisse porter son fusil de manière que son poids ne l’incommode que le moins qu’il est possible et ne nuise point à la précision de l’ordre de son rang et de sa file. Aucun de ces objets n’est rempli par notre port d’arme actuel. Le fusil est chancelant et comme dans une espèce d’équilibre […].\par
J’ai vu à la foire St-Germain des histrions faire avec aisance des tours d’équilibre et de force. Dira-t-on que ces tours sont dans la nature ?\par
Je veux donc que le soldat porte l’arme en allongeant le bras et cherchant à sa volonté et relativement à sa construction à quelle hauteur il trouvera le point d’appui le plus commode. Je veux que la main embrasse la crosse du fusil, cette crosse étant tournée à plat vers le corps et contenue par le poignet et par une partie de l’avant-bras. Il m’est indiffèrent, pourvu que le fusil soit ferme et droit, que la sous-garde soit à la hauteur du téton ou plus bas, que les chiens soient alignés et que les baïonnettes soient à la même hauteur. Cette position n’exige point d’apprentissage. L’avant-bras n’y étant pas plié, le coude se trouve nécessairement collé au corps. Enfin le soldat est droit devant lui et peut soutenir ainsi son fusil une heure ou deux sans souffrance.\par
Par préférence à cette position, j’aimerais peut-être encore mieux que le soldat portât son arme sur le bras droit, ainsi que le font aujourd’hui nos officiers et bas-officiers. Ce port d’arme a les mêmes avantages et est plus naturel et moins fatigant. J’ai interrogé là-dessus les soldats, les chasseurs et moi-même.\par
Voilà une bien longue discussion sur la première position du soldat et sur le port d’arme. Mais on a tant varié sur ces deux points. Ils ont tant fatigué le soldat et sa patience, ils sont si essentiellement la base de l’école élémentaire, que j’ai cru nécessaire de les approfondir et de les réduire en principes.\par
Quant au maniement d’armes, c’est un exercice si puéril, si indiffèrent en lui-même, que j’abrégerai ce qui le concerne. Il en faut un, parce qu’il convient que tous les mouvements du soldat sous les armes soient faits avec uniformité. Il faut aussi qu’il soit le plus simple, le plus court et le plus naturel possible, parce que c’est autant de diminué sur l’instruction. Il faut enfin qu’il ne se fasse jamais que dans les écoles et par compagnie. Je permettrais seulement quelques mouvements qui se feraient par bataillon et jamais par régiment comme :\par
{\itshape Présentez les armes} : Mouvement de parade qu’il est bon qu’un bataillon sache faire avec l’appareil de l’ensemble et de l’adresse.\par
{\itshape Reposez-vous sur les armes} : Dans le courant d’une manœuvre, ce mouvement se présente souvent à exécuter.\par
{\itshape Chargez les armes} : Il faut que ce mouvement s’exécute toujours au plus vite, l’homme d’aile, ou l’homme de la droite de chaque compagnie ou de chaque division marquant seulement le dernier temps, pour servir de ralliement et indiquer la cessation de mouvement. C’est ainsi à peu près que nous le pratiquons par le commandement d’armes plates, dont, par parenthèse, l’énoncé est bien ridicule.\par
{\itshape Apprêtez les armes, en joue, feu} : Ces trois mouvements, chacun en un temps, sont nécessaires pour les feux et pour accoutumer le soldat à ne tirer qu’au signal ou à la voix de son officier.\par
Enfin j’ajouterais à cela l’exercice de la baïonnette, consistant à la mettre au bout du canon, à la remettre dans le fourreau et à la présenter. Je ne voudrais pas que, comme nous le faisons, les troupes parussent aux exercices, aux parades, aux revues avec la baïonnette. Je voudrais qu’on ne la mît qu’au moment du combat, ou des mouvements simulés qui le représentent. On familiarise trop le soldat avec cette baïonnette toujours et inutilement armée. De là il s’accoutume à la regarder comme une arme sans usage. Autrefois il l’estimait sa dernière ressource. Un soldat et un soldat français surtout, disait « je n’ai plus de munition, mais ma baïonnette me reste ». Cet appareil de baïonnettes, réservé pour les occasions décisives, aurait quelque chose d’imposant et de terrible. Il ferait comme le drapeau rouge des Anciens, un signal de mort et de carnage. C’est de l’infanterie allemande que nous est venue la coutume de porter ainsi la baïonnette en tout temps et chose singulière, c’est que depuis qu’on la porte toujours on ne s’en sert jamais.\par
Une autre raison qui devrait déterminer à n’armer le fusil de sa baïonnette, qu’au moment du combat, c’est qu’elle fait au bout du fusil un poids incommode et fatigant pour le soldat, surtout notre port d’arme étant très élevé. Elle en fait un bien plus gênant encore si le soldat, étant en marché libre, veut pour sa commodité porter le fusil sur l’épaule.\par
Je voudrais enfin que notre baïonnette fût longue de dix-huit pouces, plate et tranchante des deux côtés, avec une arête au milieu de la lame et un ressort à sa douille pour l’adapter fermement au canon. Elle ferait alors de notre fusil une arme offensive et défensive, bien plus redoutable que le pilum des légionnaires, bien plus maniable, susceptible d’une espèce d’art d’escrime, qui enseignerait à la manier avec adresse et avec vigueur. On verra par la suite comment, avec le secours de cette arme et de quelques autres moyens, je mettrai l’infanterie en état de soutenir le choc de la cavalerie.
\subsubsection[{3. De la marche}]{3. De la marche}
\noindent C’est ici la partie essentielle et fondamentale de l’instruction du soldat, car ce n’est que par le moyen de la marche qu’une troupe est susceptible de manœuvre et d’action\par
Les principes de la marche des Anciens se sont perdus avec tous les détails intérieurs de leurs écoles de tactique. On ne peut douter seulement qu’elle ne fût assujettie à une mesure uniforme et cadencée. Les Grecs, ce peuple si ingénieux, si méthodique, si musicien, si nécessairement attaché, par son ordonnance, à l’exacte observation des rangs et des files, connurent presque de tout temps la mesure cadencée du pas. Homère, ce chantre de la Grèce héroïque et fabuleuse, nous apprend que c’était-là ce qui rendait leur marche si imposante et si majestueuse ; tandis que celle des Troyens et des autres peuples asiatiques était bruyante, inégale, semblable, dit-il, aux vagues de la mer en courroux. Il semblait au contraire, ajoute-t-il en parlant des Grecs, que Jupiter réglât leurs pas et leur eût ôté l’usage de la parole. Les Romains adoptèrent l’usage de cette marche cadencée, véloce, {\itshape sed æquo pede}, dit Tite-Live, en parlant des légions allant à la charge. Mais quel était le mécanisme, la mesure et la vitesse de leur pas ? Voilà ce que nous ignorons. Il en est de même de bien des arts, que nous ne pouvons douter que les Anciens n’aient connus et dont les principes ont péri faute des secours de l’imprimerie.\par
C’est de notre temps que l’usage de cette marche cadencée a été rétablie en Europe, on pourrait dire découverte, tant il y avait de siècles qu’elle était oubliée. Le maréchal de Saxe la regardait comme une chose bien intéressante et qui devait faire époque, pour la perfection de la tactique. Ce grand homme semblait deviner les révolutions qui allaient se faire dans les principes de cette science. Il prévoyait même que cette révolution porterait sur les marches et sur la formation des ordres de bataille, quand il écrivait : « Tout le secret de la tactique est dans les jambes »\footnote{Dans ses {\itshape Rêveries}.}.\par
Il faut considérer la marche sous deux points de vue : celui de manœuvrer et celui de faire route. Je vais traiter successivement chacun de ces objets. La première sorte de marche, exigeant de la précision et de l’ensemble, veut être enseignée méthodiquement. Elle oblige à différentes sortes de pas qui soient assujetties à une uniformité de mécanisme, d’étendue, d’accord et de vitesse. La seconde, rendant l’homme à sa liberté, n’a besoin d’aucune de ces règles.\par
Je diviserai encore la marche de manœuvre en deux parties : l’une aura pour objet de mouvoir les troupes en bataille, l’autre de mouvoir les troupes pour arriver à l’exécution de toutes sortes d’évolutions.\par
J’ai dit que la marche de manœuvre exigeait qu’on enseignât au soldat différentes sortes de pas. En effet les mouvements des troupes pouvant être faits avec plus ou moins de précision et de vitesse, ce n’est que par les différentes sortes de pas qu’on peut y parvenir. L’essentiel est que tous ces pas, différents de mesure et de vitesse, aient tous le même mécanisme. Je divise donc le pas en pas ordinaire, pas doublé, pas triplé ou de course. J’indiquerai ci-après les différences et l’objet de chacun de ces pas.\par
Ces trois sortes de pas doivent avoir un mécanisme uniforme et commun. J’appelle mécanisme les principes du mouvement des jambes qui sont les ressorts de la marche. C’est là le supplice des troupes, car chaque tacticien a cherché les règles de ce mécanisme et aucun n’en a encore donné d’assez palpables, d’assez évidentes pour ramener les autres à son opinion. Voici la mienne.\par
La mesure de toute espèce de pas, soit ordinaire, doublé ou triplé, sera de dix-huit à vingt pouces. À deux pieds, il est trop grand, trop peu conforme à la petite espèce de nos soldats et trop sujet par là à faire flotter et ouvrir le bataillon.\par
Quant à la vitesse, je voudrais que celle du pas ordinaire fût de quatre-vingt pas par minute. Je trouve qu’à soixante comme nous le pratiquons, la marche est trop lente, trop grave, trop pénible à soutenir. Celle du pas doublé serait conséquemment de cent soixante et celle du pas triple, qui serait, à proprement parler, un pas de course, irait depuis deux cents jusqu’à deux cent-cinquante suivant l’éloignement du but où l’on voudrait arriver et l’accélération qu’il serait nécessaire d’apporter aux mouvements.\par
À l’égard du mécanisme ou des principes sur lesquels on déterminerait la forme du pas, je pense que ce devraient être les suivants.\par
Le soldat étant arrêté et dans la position du port d’armes, c’est-à-dire bien droit devant lui, bien affermi dans son aplomb et dans l’habitude de porter la jambe en avant et en arrière, sans que le corps chancelle, on commencera par le former à l’usage du pas ordinaire et pour cet effet on divisera dans les premières leçons ce pas en deux espèces de temps qu’on lui fera marquer bien distinctement.\par
Au premier temps dont l’exécution aura lieu au commandement de {\itshape Marche}, le soldat portera vivement et sans secousse la jambe gauche en avant, la cuisse tournée un peu en dehors, le pied s’avançant à plat et parallèlement à la terre à deux pouces d’élévation et s’arrêtant quand le talon sera à hauteur de la pointe du pied droit. Ce mouvement se fera de la hanche, le jarret étant sans raideur et mollement plié et le corps restant bien perpendiculaire sur la jambe droite.\par
Au second temps qui sera déterminé par le commandement de {\itshape Deux}, le soldat avancera le pied gauche, le corps restant toujours droit et suivant le mouvement de la jambe. Quand par ce second temps le pied gauche se sera avancé de douze pouces, il se posera en terre. Le corps transporté en avant se trouvera presque entièrement appuyé sur le pied gauche, le pied droit sera appuyé légèrement sur sa pointe et le talon sera levé et prêt à commencer le second pas […].\par
Quelqu’un me dira : cett\phantomsection
\label{\_GoBack}e forme de pas est-elle naturelle à l’homme ? Pourquoi ne pas abandonner le soldat à son pas libre, au pas dont il marcherait hors de l’école ? Je répondrai qu’il est ici question d’un pas mesuré, cadencé, dont la forme et la vitesse soient tout à fait communes à toutes les jambes d’un bataillon ; qu’il faut pour cela des principes, une méthode, une espèce de pas à ressort, au moyen duquel on puisse dire avec certitude, une troupe parcourra tant de terrain en tant de minutes. Au reste les hommes ont-ils une forme de pas invariable et uniforme ? J’ai observé cela avec soin. Chaque classe d’hommes, chaque nation a sa démarche comme sa physionomie […].\par
J’ai été obligé d’assujettir le soldat pour la marche de manœuvre, à une forme de pas uniforme et à quelques égards artificielle, parce qu’il faut, dans cette marche, de l’ensemble et de la précision. Dans la marche de route, je lui rendrai la liberté de marcher à sa volonté, afin qu’il fasse chemin de la manière qui lui sera la plus commode et la moins pénible. Achevons ce qui concerne les principes de la marche de manœuvre.\par
Le soldat ayant parfaitement acquis l’habitude du pas ordinaire, on le fera passer au pas doublé et ensuite au pas triplé ou pas de course, observant de l’instruire d’abord seul, puis deux à deux, quatre à quatre et ainsi en multipliant successivement le nombre des élèves et les formant dans le commencement sur un rang puis sur une file, sur deux, sur plusieurs, les trois rangs couverts et faisant passer alternativement les soldats d’un rang à l’autre.\par
On s’attachera particulièrement à faire contracter aux soldats l’habitude de l’égalité du pas, soit par rapport à son étendue, soit par rapport à sa vitesse […].\par
Enfin ce sont des officiers, particulièrement sûrs de leurs pas et de leur coup d’œil, qui régleront la direction de la marche et l’alignement du bataillon.\par
Je propose sur ces deux derniers objets une école particulière pour les officiers. Là, indépendamment de la théorie et de la pratique de la marche, dans lesquelles j’aurai soin de les affermir encore plus que les soldats, ils apprendront à élever de l’œil une perpendiculaire ; à choisir une direction parallèle à telle autre direction ; à apprécier les distances, le temps qu’il faut pour les parcourir à tel ou tel pas ; à juger la force d’une troupe présentée à différents éloignements dans différents terrains et dans différents ordres : à marcher eux-mêmes, soit en troupe, soit à la tête d’une troupe ; à observer imperturbablement les distances d’une division, les hauteurs de deux têtes de colonnes, les intervalles que ces colonnes doivent garder entre elles, etc. Les colonels et les officiers supérieurs des régiments seront à la tête de ces écoles. Ils s’y instruiront eux-mêmes, ils s’y formeront de plus en plus le coup d’œil. Ce genre d’exercice, dont il sera possible de faire un amusement, développera l’intelligence des officiers […].\par
Les bataillons, une fois bien affermis à la marche sur des surfaces unies, il faudra les mener en plein champ, d’abord dans des plaines et ensuite dans des terrains inégaux et coupés. Là les exercices deviendront vraiment utiles et analogues à la guerre. L’œil des officiers s’accoutumera à l’immensité et au choix des points de vue. La marche perdra nécessairement un peu de cette perfection, de cet accord de mouvement qu’elle avait sur les esplanades. Mais elle deviendra plus militaire, plus décidée. Les officiers supérieurs apprendront à connaître combien l’inégalité et l’ondulation plus ou moins forte des terrains influe sur l’aspect du pays, sur le toisé de l’œil, sur la possibilité de l’alignement, sur l’ensemble et sur la vitesse de la marche. Là, les bataillons seront particulièrement exercés à parcourir jusqu’à trois ou quatre cents toises de suite au pas doublé, sorte de marche beaucoup trop négligée dans nos exercices actuels. Là, s’il s’offre à eux une haie, un fossé, un ravin, on verra lequel des bataillons présentés devant cet obstacle, le passera avec le plus de promptitude, d’ordre et de silence. Là, s’il s’offre une hauteur, qui militairement fût importante à occuper, on saura détacher un bataillon au pas de la course pour aller la couronner. On saura, pour mettre en activité l’émulation et donner plus de vérité à cet exercice, faire partir un bataillon d’un point également éloigné, pour tâcher de l’y prévenir. Toutes ces manœuvres ne seront que des jeux pour mes soldats, exercés par mon plan d’éducation au saut, à la course, à tout ce qui peut augmenter l’agilité et la force […].\par
C’est ici le lieu de parler de l’usage qu’on devrait faire des instruments militaires pour animer et pour soutenir la marche des troupes. Il semble qu’aujourd’hui nous n’en ayons plus que la forme. À la vérité, il est à propos qu’ils soient bannis des écoles, afin que le soldat s’accoutume de lui-même et sans secours, à l’accord de mouvement, d’étendue et de vitesse du pas. Mais toutes les fois que les bataillons seront réunis, toutes les fois qu’ils exerceront en terrain libre et ouvert : il faut que les tambours et la musique accompagnent leur marche et leurs mouvements. Il le faut à plus forte raison devant l’ennemi où l’âme du soldat a bien plus besoin d’être échauffée et soutenue. Je désirerais pour cela que nos instruments fussent plus sonores, plus éclatants, que le rythme de notre musique fût plus vif, plus serré, plus adapté à la différence des circonstances et des mouvements ; qu’il y eût, par exemple, des airs consacrés au combat, joués à la guerre et pendant les combats seulement, des airs dont alors les modulations fussent au plus haut degré de chaleur et de véhémence. Nos organes auraient-ils changé ? La musique aurait-elle dégénéré ? Se serait-elle amollie, affaiblie ? Ou bien doit-on traiter de fable ce que l’histoire rapporte de ce Timothée, de cet air phrygien qui forçait les peuples de la Grèce à courir aux armes ?\par
Il me reste à parler de la marche de route. Comme elle n’a pour objet, que de faire chemin à une troupe de la manière la moins pénible et la plus commode, il faut que le soldat y prenne son pas libre et naturel, bien entendu toutefois que ce pas ne pourra être de la même vitesse que celui d’un homme qui marcherait seul, parce qu’ici le soldat est charge de ses armes et de son bagage, fait partie d’une colonne nombreuse et est par conséquent assujetti à un certain ordre. Ce sera donc à l’officier qui conduira la colonne, à régler la vitesse du pas dont elle devra marcher relativement à la nature du chemin, à sa longueur, aux objets que l’infanterie devra remplir à son arrivée, à l’espèce d’hommes de cette infanterie, au poids dont ils sont chargés, à la profondeur de la colonne. On voit par là combien il est nécessaire que les officiers supérieurs soient habitués à conduire des colonnes d’infanterie, marchant eux-mêmes à pied à leur tête et réglant la vitesse de la marche. Cette habitude peut seule apprendre combien de temps une troupe doit employer à faire tant ou tant de chemin, dans tel ou tel pays, avec telle ou telle vitesse. On manque un grand nombre d’opérations à la guerre, parce qu’on n’a pas su combiner avec précision le temps, les distances, ou la nature des chemins.\par
Ces marches répétées fréquemment dans toutes sortes de chemin, en tout temps et en toute saison, les soldats étant chargés de tout ce qu’ils doivent porter en campagne, seraient un des exercices les plus utiles pour les troupes. Il en coûterait un peu plus de chaussure et d’habillement, la tenue en serait moins brillante et moins recherchée. Mais nous aurions des armées que les fatigues de la guerre n’étonneraient et ne détruiraient pas. Enfin je citerai sur cela l’exemple des Romains ; car toutes les fois qu’il sera question d’une milice nerveuse et solide, c’est chez eux qu’il faudra en puiser les institutions […].
\subsubsection[{4. Des feux}]{4. Des feux}
\noindent De toutes les parties de la tactique, c’est sur celle-ci peut-être, que nous avons les exercices les plus compliqués, les moins réfléchis et les moins relatifs à ce qui se passe à la guerre. Quand je dis nous, je parle de toutes les troupes de l’Europe, je parle des troupes allemandes qui ont tant accrédité le système du feu et qui regardent la supériorité de mousqueterie comme si décisive dans les combats.\par
On s’est attaché à l’envi à perfectionner le chargement du fusil, à tirer une plus grande quantité de coups par minute, c’est-à-dire à augmenter le bruit et la fumée. Mais on n’a travaillé ni à simplifier l’ordre dans lequel ces feux devaient être faits, ni à déterminer la meilleure posture du soldat pour bien ajuster, ni à augmenter son adresse sur ce point, ni à faire connaître aux troupes la différence des portées et des tirs, ni enfin à leur enseigner jusqu’à quel point il fallait compter, ou ne pas compter sur le feu ; comment il fallait l’employer et le ménager relativement au terrain, aux circonstances, à l’espèce d’arme, qu’on a vis-à-vis de soi. Quand, en un mot, il fallait cesser d’en faire usage, pour charger l’ennemi à la baïonnette.\par
Mais avant que d’entrer dans ces détails sur cet objet, disons un mot de nos armes à feu.\par
Il n’est pas douteux, je crois, que nos armes de jet, en considérant nos fusils comme tels, ne soient infiniment supérieures à celles des Anciens, soit pour la longueur de la portée, soit pour la justesse. Quelle différence, en effet, de ces traits, lourds, embarrassants, que chaque soldat ne pouvait porter qu’en petite quantité ; qui n’étaient chassés qu’à bras, ou par l’action d’une corde : moteurs faibles, incertains, sujets à inconvénient et à aberration ; avec ces petits globes de métal, que le soldat peut porter en grand nombre et qui sont forcés à suivre une direction presque certaine\footnote{Je dis presque certaine par comparaison et relativement aux armes de jet des Anciens. Car, dans le fait, une infinité de causes, soit connues, soit cachées, contribuent à jeter de l’incertitude et de la bizarrerie dans les tirs de nos fusils.}, par la forme de ces tubes cylindriques dans lesquels ils sont comprimés et par la force de ce fluide inflammable et élastique, que le débandement d’un ressort anime et met en action avec une vitesse incroyable ?\par
Veut-on une preuve de la supériorité de nos fusils sur toutes les armes de jet, comme frondes, arcs, javelots lancés à la main, etc., c’est l’empressement avec lequel tous les sauvages du nouveau monde ont quitté ces dernières pour adopter nos fusils malgré l’inconvénient du bruit, qui cependant en est un réel pour des hommes dont la chasse fait toute la nourriture et l’occupation.\par
Pour connaître ce qu’on doit appeler la portée d’une arme à feu, il faut considérer : 1) la ligne de mire c’est-à-dire la ligne droite par laquelle on voit l’objet vers lequel on veut porter la balle ; 2) la ligne de tir, autre ligne droite qui représente l’axe de l’arme ; 3) la trajectoire ou la ligne que décrit le globe qui est lancé par la poudre hors du calibre de l’arme vers le but qu’on se propose de frapper.\par
La ligne de tir et la ligne de mire ne sont point parallèles et elles forment entre elles, au-delà de la bouche du canon, un angle plus ou moins sensible, suivant l’épaisseur que le canon a à sa culasse et à son extrémité opposée. C’est le long de la ligne de mire que l’œil cherche sa visée et par conséquent à l’extérieur et au sommet du cylindre de l’arme : au lieu que c’est de l’intérieur et le long de la ligne de tir que le mobile est chassé. Donc la ligne de tir et la ligne de mire sont sécantes entre elles. Examinons maintenant à quel point elles le sont et quelle est la direction de la trajectoire.\par
À sa sortie du cylindre le boulet ou la balle décrit une courbe. C’est une loi que l’attrait de la pesanteur impose à tous les corps jetés obliquement. Cette ligne courbe, que décrit le mobile, coupe d’abord et à peu de distance de la bouche du canon, la ligne de mire, passe ensuite au-dessus d’elle. De là toujours ramené vers la terre par sa gravitation, à laquelle il est soumis, elle se rapproche de cette ligne, la recoupe une seconde fois et achève de décrire sa parabole jusqu’à la fin de sa chute. C’est ce second point d’intersection qu’on appelle la portée de l’arme de but en blanc et qui est plus ou moins éloigne de l’extrémité du cylindre, à proportion de l’ouverture de l’angle que forment entre elles la ligne de mire et la ligne de tir, ainsi qu’en raison de la force qui chasse le mobile, du volume de ce mobile, de sa densité, de celle du milieu qu’il traverse et de la longueur du calibre proportionnée avec son diamètre.\par
Ce que j’ai dit ci-dessus est certain et commun à toutes les armes à feu. Mais ce qui malheureusement fait problème encore (soit qu’on n’ait pas fait à cet égard des expériences assez exactes, soit qu’une infinité de raisons étrangères relatives aux effets de la poudre, à l’action de l’air, à la qualité des mobiles et à celle des moyens qui les chassent, les rendent extrêmement difficiles et incertaines), c’est la longueur des courbes que ces mobiles peuvent décrire, c’est la détermination exacte de la vitesse avec laquelle ils les parcourent et de leur déclinaison successive vers la terre.\par
Au milieu de ces incertitudes que des découvertes et des expériences plus heureuses rectifieront peut-être un jour, il existe cependant quelques vérités approximatives que je vais rassembler et qui doivent faire la base de la théorie des exercices à feu de l’infanterie.\par
Soit supposé un fusil de munition, tel que ceux dont nos troupes sont armées, chargé d’une balle de calibre avec la quantité de poudre accoutumée. Il est à peu près constant que la balle suivant sa trajectoire se trouvera, à 60 toises environ, à un pied et demi ou deux d’élévation au-dessus de la ligne de mire. Ce sera là le point où elle sera le plus élevée au-dessus de cette ligne. Ensuite continuant de décrire sa parabole et ramenée vers la ligne de mire, par l’attrait de sa pesanteur elle recoupera cette ligne environ à cent ou cent vingt toises et achèvera de parcourir sa trajectoire jusqu’à ce qu’elle rencontre la terre ou quelque autre obstacle qui diminue ou anéantisse la force qui la fait mouvoir\footnote{Ce que je dis ici est le résultat des épreuves qui ont été faites dans nos écoles d’artillerie. Mais on en pourrait faire de beaucoup plus précises. Celle, par exemple, qui déterminerait la vitesse initiale du mobile au moyen d’un pendule suspendu à différentes distances, dans lequel on tirerait successivement plusieurs balles, afin de juger par la force et la durée de la vibration que chacun de ces coups communiquerait au pendule, avec quelle vitesse la balle parcourt sa trajectoire et par conséquent quelle est la nature de cette courbe.}.\par
Je dis que jusqu’à la distance de 60 toises environ, la balle s’élèvera au-dessus de la ligne de mire. C’est là ce qui fait dire vulgairement que le coup relève. Dans le fait cependant la balle ne relève point. Elle suit, dès le moment de sa sortie du canon, une direction rectiligne ou, pour parler plus juste, une direction courbée toujours de plus en plus par la loi de la pesanteur.\par
Il suit de là, que pour qu’une balle de fusil atteigne au but que l’on veut frapper, il ne faut pas toujours précisément prendre sa visée vers ce but, et qu’il faut mirer au-dessus ou au-dessous de lui, suivant que ce but est plus ou moins éloigné […].\par
Ce principe de ne viser jamais précisément au but qu’on veut atteindre, est confirmé par l’expérience des chasseurs. Ceux qui tuent à tout coup ne tirent jamais en ayant parfaitement le gibier sur la ligne de mire de leur fusil. Non seulement ils tirent à l’endroit où sera la pièce de gibier, lorsque leur coup y arrivera ; mais ils visent plus au-dessous ou au-dessus, suivant l’éloignement du but qu’ils veulent frapper.\par
Concluons que le feu de mousqueterie des troupes peut être soumis à une théorie. Cependant, bien loin de l’être, il s’exécute au hasard et machinalement. C’est qu’il n’y a peut-être pas dix officiers d’infanterie qui connaissent la construction du fusil et qui aient réfléchi sur le jet des mobiles qu’il peut lancer. Aussi ne donne-t-on au soldat aucun principe sur la manière d’ajuster. Il tire comme il veut, quelles que soient la distance et la situation des objets. C’est particulièrement aux exercices de cible déjà beaucoup trop rares, que cette ignorance et ce défaut de principes sont bien sensibles. J’aurai occasion d’y revenir tout à l’heure.\par
À l’égard de la portée du fusil, toutes les expériences qui ont été faites pour en constater la longueur, n’ont rien déterminé de précis. On a vu souvent dans ces expériences deux balles tirées par deux fusils de même calibre, sous le même angle de projection et avec des charges égales, porter à des distances inégales, soit en raison de la force du bras qui les avait chargés, soit en raison de la densité plus ou moins grande de l’air, soit aussi par rapport à la qualité de la poudre, à son degré de siccité, à sa disposition dans le cylindre, à la promptitude de sa dilatation, etc.\par
Tout ce qu’on peut dire de certain, c’est que la portée des fusils, dont notre infanterie est armée, est, sous une direction à peu près horizontale d’environ 180 toises […].\par
Quoique la portée horizontale du fusil puisse être estimée jusqu’à 180 toises, ce n’est guère qu’à 80 que le feu de l’infanterie commence à avoir un grand effet. Je parle de l’infanterie rangée en bataille et dans le tumulte du combat. Par-delà cette distance, les coups deviennent incertains parce que le soldat charge et ajuste mal, vite, et avec trouble. Ces bataillons prussiens dont on a cru et dont quelques gens croient peut-être encore le feu si redoutable, sont ceux dont le feu est le moins meurtrier. Leur première décharge a de la portée et de l’effet, parce que ce premier coup, chargé hors du combat, l’est avec exactitude. Mais ensuite et dans le tumulte de l’action, ils chargent à la hâte et sans bourrer. On leur a dit que la plus grande perfection de l’exercice à feu était de tirer le plus grand nombre possible de coups par minute. En conséquence ils n’ajustent point. Une manière de mouvement machinal et comme de ressort, place leur arme contre l’épaule, au lieu de soutenir le fusil dans la direction horizontale, ce qui exigerait qu’il portât avec force sur la main gauche, à peu près comme les anciens mousquetons trouvaient leur appui sur la fourchette. Ils trouvent plus commode de ne pas se fatiguer et laissent tomber le fusil extrêmement bas. Ainsi le coup part sans que l’œil l’ait dirigé et la balle va mourir dans la poussière au quart de sa portée. Toutes les troupes de l’Europe cependant, séduites par la beauté des exercices à feu prussiens, par la célérité de leur chargement, par l’ensemble et la correspondance de leurs décharges, ont cherché à les imiter. Nos régiments allemands, dont la politique est d’introduire chez nous les pratiques étrangères et de les abandonner aussitôt que nous les avons adoptées, pour se donner le mérite de quelque autre invention nouvelle, y ont introduit la manie de ces exercices à feu et bientôt il n’a été question dans nos écoles que de la vitesse du chargement. On s’occupe de cette célérité aux dépens de la manière d’ajuster. On n’a aucune idée de la véritable théorie des tirs. On donne pour principes des lieux communs, vides de sens et de réflexion. Tirez vite, dit-on aux soldats, comme si le bruit tuait ; ajustez au milieu du corps, comme si ce principe pouvait être général, quelles que soient les distances et la situation des objets, comme si l’on ne devait pas chercher sa visée plus ou moins haut relativement à ces différences de distance et de situation et à la courbe que le mobile décrit. Ajustez bas, dit-on d’autres fois, le coup relèvera assez, comme si les balles pouvaient s’élever au-dessus de la ligne de tir, comme s’il n’y avait pas une loi de tendance et de pesanteur qui assujettît tous les corps en mouvement à retomber vers la terre. Faut-il s’étonner, après cela, si nos feux de mousqueterie sont si méprisables, si dans une bataille il y a cinq cent mille coups de fusil de tirés, sans qu’il reste deux mille morts sur le terrain du combat ?\par
Tant mieux pour l’humanité, dira-t-on, si les combats sont moins sanglants et s’ils décident également les querelles ! À cela je réponds que si l’on tirait mieux, ils n’en seraient pas plus sanglants. On tirerait moins longtemps, on serait plus impatient d’arriver à l’arme blanche, seul genre de combat favorable au courage et à l’adresse.\par
Qu’on ne donne donc plus pour la perfection de l’art ce qui en est la dégradation. Qu’on apprenne à se servir des armes actuelles.\par
Qu’on étudie la théorie de leurs effets. Qu’on ne cherche pas à en imaginer de nouvelles, si elles ne font que consommer plus de munitions, si elles ne portent pas plus droit et plus loin, si elles ne sont ni plus simples, ni plus solides, ni plus sûres […].\par
J’ai cru nécessaire de poser ces premiers principes sur l’effet et sur la portée de nos armes, afin d’appuyer quelques changements que je proposerai ci-après dans la théorie de nos exercices.\par
Le soldat ayant déjà acquis, par le maniement des armes, la parfaite habitude de charger et de tirer tant au blanc qu’à poudre, d’abord seul, puis dans une file, aux trois différentes places de cette file, sur plusieurs files et enfin par demi-compagnie et par compagnie, on le fera passer à l’exercice à balles en le conduisant par les mêmes gradations. Mais avant que d’aller plus loin, parlons de la position qu’on fera prendre aux soldats dans l’exercice à feu et de quelques autres règles qui seront observées dans les écoles […].\par
On accoutumera surtout le soldat à faire agir la détente sans remuer ni la tête, ni le corps, ni surtout le fusil que le moindre mouvement détournerait de la visée horizontale. Pour cet effet dans les écoles de principes, on fera rester le soldat après qu’il aura tiré, sur le temps d’en joue, pour voir, le coup étant parti, dans quelle direction est le fusil.\par
Je dis que dans la position d’en joue il faut que le canon du fusil soit parallèle au terrain où le soldat est placé. Ce doit être là la position habituelle lorsqu’il est question d’exercer les soldats au feu et sans les faire tirer sur des objets déterminés. Mais il en doit être autrement quand on assigne un but à leurs feux, comme des cibles ou des toiles tendues à hauteur d’homme. Alors on doit faire l’application de la théorie que j’ai exposée ci-dessus. On doit recommander aux soldats d’ajuster à telle ou telle partie de l’objet qu’il veut atteindre, suivant la situation de cet objet et la distance à laquelle il est placé. Il y a à cet égard des proportions qui doivent être regar\hyperref[bookmark11]{\dotuline{dées comme des axiomes et qu’on peut enseigner aux soldats sans}} \hyperref[bookmark12]{\dotuline{qu’il soit besoin de leur faire connaître la théorie sur laquelle elles}} sont fondées […].\par
\hyperref[bookmark13]{\dotuline{C’est surtout aux exercices très multipliés de la cible et des toilesqu’on fera l’application de cette théorie. Là, comme il faut au soldat}} des démonstrations palpables et simples, au lieu d’avoir pour but une cible informe élevée sur un piquet, on peindra sur des planches découpées un homme de grandeur naturelle et vêtu d’un uniforme de troupes étrangères. On aura la patience, quand le coup du soldat aura passé par-dessus ce but ou donné en terre, de lui montrer que cela est provenu, ou de ce qu’il n’a pas mis en joue suivant les principes établis, ou de ce qu’en appuyant sur la détente pour faire feu il n’a pas tenu son fusil bien ferme dans la visée où il l’avait place. On lui fera retirer sur-le-champ un autre coup, afin de mettre l’exemple à côté du précepte. On changera souvent le but de distance et d’emplacement, l’établissant tantôt dans un terrain en pente, tantôt sur une élévation, tantôt sur un terrain horizontal. On excitera enfin l’émulation et l’adresse du soldat par quelques prix […].\par
Il me reste à parler des différentes sortes de feux, c’est-à-dire, des différentes manières de faire tirer l’infanterie. Je serai court sur cet objet, car il ne faut que des feux simples, possibles à la guerre et que les soldats sachent exécuter dès le premier jour qu’on les rassemblera en bataillon.\par
J’ose d’abord avancer qu’il n’y a qu’une espèce de feu convenable à l’infanterie réglée, le feu de pied ferme. Cette assertion paraîtra bien hardie, quand on songera que le roi de Prusse a introduit et paraît faire cas de ce qu’on appelle dans ses troupes le feu de charge ; quand il a dit lui-même qu’on ne pouvait mener de l’infanterie à l’ennemi sans tirer […].\par
En un mot et j’en fais une maxime générale, il ne faut tirer que quand on ne peut pas marcher ; car soit qu’on attaque, soit qu’on se retire, soit qu’on suive un ennemi qui fuit, avancer est le premier objet et le seul qui puisse procurer quelque avantage.\par
Ce principe posé, qu’il n’y a que les feux de pied ferme qui soient praticables à la guerre […]. Je veux que dans tous les feux fractionnés de bataillon, il n’y ait d’autre règle que celle-ci : les deux parties couplées et voisines formant ou le peloton, ou la division, ou le bataillon, tireront comme si elles étaient seules et indépendantes du bataillon […], chaque partie observant seulement de ne pas tirer que l’autre n’ait finit de charger, afin que le feu soit le plus qu’il sera possible égal et continu sur toutes les parties du front.\par
Outre ce feu, j’exercerai encore les bataillons à tirer au commandement de l’officier par un, par deux et par trois rangs. Je les accoutumerai aussi à exécuter le feu à volonté, ou autrement appelé de billebaude […].\par
II est temps de parler de l’usage qu’on doit faire de ces différents feux et des circonstances auxquelles chacun d’eux est propre. Le feu par pelotons ou par divisions est celui où l’officier est le plus le maître de sa troupe, mais il ne convient guère qu’à de l’infanterie postée et voulant éloigner et contenir des attaques irrégulières et peu vives. Il est encore plus particulièrement propre aux postes d’infanterie retranchés, lorsqu’ils sont harcelés et qu’ils veulent ménager leurs munitions […].\par
Le feu par un ou par plusieurs rangs est, je crois, le seul propre contre la cavalerie et pour la défense d’un abattis, ou d’un poste que l’ennemi attaquerait décidément et la baïonnette au bout du fusil, parce que c’est le seul qui donne, si je peux m’exprimer ainsi, une masse de feux capable d’arrêter et de renverser de grands efforts. Mais il faut, comme je le dis, ne l’employer que quand l’ennemi est ébranlé pour une attaque de vive force et ménager la conduite de ce feu de manière que les deux derniers rangs fassent leur dernière décharge quand il est à vingt-cinq pas et que le premier rang réserve la sienne pour la faire à bout touchant […].\par
Le feu de billebaude est enfin le seul qui doive avoir lieu dans un combat de mousqueterie. Par-delà deux décharges essuyées et rendues, il n’y a pas d’effort de discipline qui puisse empêcher un feu compliqué et régulier de dégénérer en feu de volonté. Ce feu est le plus vif et le plus meurtrier de tous. Il échauffe la tête du soldat. Il l’étourdit sur le danger. Il convient particulièrement à la vivacité et à l’adresse françaises. L’essentiel est seulement d’accoutumer le soldat à le cesser au signal et à garder le silence.
\subsubsection[{5. Des évolutions}]{5. Des évolutions}
\noindent Il y a des militaires qui disent qu’il ne faut point d’évolutions et que toutes les évolutions sont impraticables devant l’ennemi. Il y a des tacticiens que la pratique n’a point éclairés, qui veulent multiplier les évolutions à l’infini, qui en fatiguent continuellement les troupes, soutenant que toutes les évolutions sont bonnes, qu’elles remplissent du moins l’objet utile d’assouplir et d’exercer le soldat. Cherchons un juste milieu entre ces extrêmes et faisons-en la base de nos principes.\par
Il faut des évolutions. Sans évolutions, une troupe ne serait qu’une masse sans mouvement, réduite à l’ordre primitif dans lequel on l’aurait placée et incapable d’agir au premier changement de terrain ou de circonstances. Les évolutions sont donc les mouvements par lesquels une troupe doit, relativement aux circonstances et au terrain, changer d’ordre et de situation.\par
Elles doivent être simples, faciles, en petit nombre et relatives à la guerre. Elles doivent surtout être promptes, parce que le mouvement qu’une troupe fait pour passer d’un ordre à un autre, la jette nécessairement dans un état de désunion et de faiblesse, d’où il est important qu’elle sorte le plus tôt possible. Toute évolution qui n’a pas à la fois toutes ces propriétés, doit être rejetée comme vicieuse, superflue et même dangereuse. Car dans un métier, où il y a beaucoup de choses nécessaires à apprendre, ce n’est qu’à leurs dépens qu’on s’occupe de celles qui sont inutiles.\par
Les évolutions les meilleures, les plus analogues aux armes, à la constitution des troupes, au génie national, étant une fois déterminées, elles doivent être exécutées par les mêmes principes. Elles doivent être invariables, ou du moins ne varier que par des ordres du gouvernement. C’est à lui à faire examiner, par des gens éclairés, les changements que les troupes des autres puissances font dans leur tactique, les ouvrages qui paraissent, les projets proposés, à ordonner des épreuves et à savoir à propos adopter, ou rejeter, se tenant également en garde contre la manie de l’innovation qui fait tout imiter sans réfléchir et contre l’aveuglement de l’habitude qui porte à refuser tout changement. Ce que je dis pour les évolutions, peut s’appliquer à toutes les branches de la constitution militaire. Imitons à cet égard les Romains. Ils savaient s’enrichir des connaissances et des découvertes de tous les peuples, mais sans cesser de s’estimer et de se croire supérieurs à eux.\par
La multiplicité des évolutions et des épreuves qui y sont relatives est funeste en ce qu’elle fatigue les troupes, surcharge leur entendement, et les détourne des autres travaux de leur éducation.\par
Il ne faut pas manœuvrer devant l’ennemi. Je vais analyser cette espèce d’axiome et chercher ce qu’il renferme d’erreurs et de ventes.\par
Toute évolution sous le feu et sous un feu vif de l’ennemi est impossible à tenter avec des troupes qui ne sont point aguerries et délicate avec des troupes qui le sont […].\par
C’est de l’espèce des troupes que dépend presque toujours la possibilité d’un mouvement […].\par
Il n’y a pas d’évolution proprement dangereuse en elle-même. Tout dépend de la circonstance à laquelle on l’applique et cet à-propos consiste dans la combinaison la plus précise et la plus sûre du temps qu’on emploiera à faire son mouvement avec celui qu’emploiera l’ennemi pour venir le troubler ; combinaison sur laquelle on ne peut être parfaitement affermi que par l’habitude de remuer des troupes des deux armes, dans toutes sortes de terrains et surtout à la guerre qui produit bien d’autres circonstances que les exercices de paix […].
\subsubsection[{6. Doublement des rangs. Ordonnance et moyens dont l’infanterie doit se servir pour combattre la cavalerie.}]{6. Doublement des rangs. Ordonnance et moyens dont l’infanterie doit se servir pour combattre la cavalerie.}
\noindent C’est relativement à tous ces principes que je voudrais exercer l’infanterie, ayant soin en même temps de parler au soldat, lui faisant connaître la force et les raisons de mes dispositions, les avantages immenses que l’infanterie ainsi disposée a sur la cavalerie, le nombre de baïonnettes et de coups de fusil qu’elle a à opposer à chaque cavalier ; l’effet prodigieux de son feu s’il est bien dirigé […] ; enfin le danger que court l’infanterie si elle s’effraie et se désunit et sa force invincible tant qu’elle reste intrépide et serrée. En général on ne raisonne pas assez avec le soldat et surtout avec le soldat français, que son intelligence met à portée de comprendre beaucoup de choses. Cependant la fermeté d’une troupe augmenterait en raison de ce que chaque individu serait plus persuadé de la bonté de l’ordonnance et de la disposition dans laquelle il est rangé.\par
Je voudrais enfin accoutumer l’infanterie à manœuvrer vis-à-vis de la cavalerie […]. Sans ces exercices, l’officier d’infanterie inexpert sur les mouvements de la cavalerie, sur son degré de vitesse, sur le temps qu’elle met à parcourir telle ou telle distance, ne saurait juger ni quand il devra s’arrêter ni quand il pourra remarcher ni sur quel point la cavalerie veut faire un effort ni comment il doit la repousser.\par
Mais diront les officiers de cavalerie, ces exercices simulés entre les deux armes ne peuvent avoir lieu. Ils n’aboutiraient qu’à former l’infanterie aux dépens de la cavalerie. Car si on ne nous mène contre de l’infanterie retranchée et ordonnée, comme ci-dessus que pour nous faire voir l’impossibilité de l’enfoncer, pour nous faire essuyer son feu, et nous faire demi-tour à droite avant que d’arriver à elle, l’infanterie seule s’aguerrira à cet exercice. Nos chevaux au contraire s’accoutumeront à ne jamais approcher les baïonnettes. Nos cavaliers ne pourront s’abandonner au baissement de main, et le résultat de ces exercices étant toujours pour eux de se retirer sans enfoncer l’infanterie, le préjuge de la supériorité restera entièrement en faveur de cette dernière.\par
Je réponds à cela que l’objet important est de former l’infanterie, laissée jusqu’ici beaucoup trop en arrière en moyens de défense contre la cavalerie. Ce n’est que depuis la décadence de la discipline militaire que la cavalerie charge avec succès l’infanterie et cette infanterie étant régénérée et ordonnée, comme je le propose, il faudra que la cavalerie s’abstienne de l’attaquer comme elle s’abstient d’attaquer un chemin ouvert ou un retranchement. Chacune des armes rentrera alors dans sa sphère et dans ses droits. L’infanterie, corps solide et pesant, redoutable par son feu, par les ressources de l’art et des terrains, ne pourra être attaquée que par de l’infanterie. La cavalerie attaquera la cavalerie, elle sera maîtresse des plaines, elle fera les détachements et les courses rapides. Elle couvrira les flancs de l’infanterie, parce que, par sa vélocité, elle peut mieux embrasser et envelopper. Elle soutiendra l’infanterie, parce qu’au moyen du même avantage elle peut en un clin d’œil tomber sur l’ennemi que sa victoire ou sa défaite aura mis en désordre. Elle pourra enfin attaquer toute infanterie qui n’aura pas eu le temps ou la prudence de prendre ma disposition, et toute infanterie qui, comme celle d’aujourd’hui, sera nue, faible, ignorante, maladroite, et mal ordonnée.\par
Je ne suis point comme quelques militaires, exclusivement partisan du corps dans lequel j’ai servi. Je crois les deux armes nécessaires l’une à l’autre. J’ai cherché à rendre à l’infanterie toute la force qu’elle peut avoir. Quand je parlerai de la cavalerie je chercherai de même tout ce qui peut augmenter la célérité et la simplicité de ses mouvements. Je prouverai qu’on n’en tire pas tout le parti dont elle est susceptible, que c’est elle qui devrait décider la moitié des batailles et compléter presque toutes les victoires.
\subsubsection[{7. Des mouvements de conversion}]{7. Des mouvements de conversion}
\noindent […]
\subsubsection[{8. Des formations en colonne}]{8. Des formations en colonne}
\noindent L’infanterie se forme en colonne pour attaquer l’ennemi dans cet ordre, ou pour parcourir plus promptement et plus commodément une longue étendue de terrain, soit au pas réglé, soit au pas de route. Dans l’un et l’autre objet la formation de la colonne doit s’opérer par le même mécanisme […].\par
\paragraph[{Colonne formée dans le dessein d’attaquer l’ennemi}]{Colonne formée dans le dessein d’attaquer l’ennemi}
\noindent Dans quel cas peut-il être nécessaire et avantageux d’attaquer l’ennemi en colonne ? C’est quand l’ennemi est derrière un retranchement ou dans tel autre poste, dont les flancs naturels ou artificiels réduisent nécessairement à attaquer les saillants et à ne pas se présenter sur les faces. C’est quand, ne pouvant déboucher sur l’ennemi que par un chemin, on est forcé de rassembler ses troupes sur ce débouché et d’arriver par ce seul passage. C’est enfin, quand, d’un retranchement ou d’un poste fermé, on veut faire une sortie sur l’ennemi attaquant et déjà mis en désordre par le mauvais succès de son attaque.\par
Quel est dans ces circonstances l’avantage de l’ordre en colonne ? Ce n’est point, comme bien des gens le croient, la force de choc produite par la pression exacte des rangs et des files ; puisqu’ainsi que je l’ai prouvé dans ma discussion sur l’ordre de profondeur, cette pression exacte ne saurait avoir lieu entre des individus actifs et pensants, au point de former un corps sans interstices et capable d’acquérir une force combinée sur la quantité de masse et de mouvement.\par
Cependant, soit qu’on se flatte d’approcher de cette pression chimérique, soit qu’on se laisse guider en cela par la routine, comme en tant d’autres choses, voici comment se forment toutes les attaques en colonne. On s’ébranle, on approche de l’ennemi, on crie aux soldats : {\itshape serrez, serrez}. L’instinct machinal et moutonnier qui porte tout l’homme à se jeter sur son voisin, parce qu’il croit par là se mettre à l’abri du danger, ne porte déjà que trop à l’exécution de ce commandement. Les soldats se pressent donc, les rangs se confondent. Bientôt au rang du front et aux files extérieures près qui conservent quelque liberté de mouvement, la colonne ne forme plus qu’une masse tumultueuse et incapable de manœuvre. Que la tête et les flancs de cette colonne soient battus d’un feu vif, que du premier effort elle ne surmonte pas les obstacles qu’elle rencontre, dès lors les officiers ne peuvent plus se faire entendre. Il n’y a plus de distance entre les divisions, le soldat étourdi se met à tirer en l’air, la masse tourbillonne, se disperse et ne peut se rallier qu’à une distance très éloignée. Quelques-unes de ces attaques réussissent cependant, parce que l’ennemi se défend mollement, parce qu’il s’effraie de cette masse d’hommes qui arrive à lui, parce que la tête des colonnes étant toujours composée de troupes d’élite, ces troupes pénètrent et frayent le chemin. Mais portée dans le retranchement, la masse étonnée de son succès, ne peut plus s’y débrouiller. Elle n’est plus en état de se déployer et de s’étendre. L’ennemi a-t-il des troupes fraîches à portée ? Il marche sur elle, la culbute et c’est à recommencer sur nouveaux frais. Je demande à tous les anciens officiers si ce n’est pas là le tableau de la plupart des attaques qu’ils ont vu faire en colonne […].\par
Les avantages de l’ordre en colonne consistent, je le répète, non dans la pression exacte des rangs et des files, mais dans la succession continue d’efforts que font les divisions rangées les unes derrière les autres et se succédant rapidement pour se porter à un point d’attaque, dont, couvertes par les divisions qui les précèdent, elles n’ont ni vu les obstacles ni presque essuyé les coups […].\par
Quand les colonnes auront battu l’ennemi et emporté le retranchement, elles se déploieront sur le champ pour être en état de pousser leur avantage et de présenter un front aux attaques que l’ennemi pourrait tenter. Les compagnies de chasseurs se jetteront en avant d’elles pour couvrir de déploiement et s’emparer promptement de tous les points avantageux, comme fossés, ravins, haies, ou maisons qui pourraient leur donner protection. Car je ne crois pouvoir assez le répéter, c’est de l’occupation des points qui peuvent donner des flancs ou des revers sur l’ennemi, que dépend le succès de presque toutes les affaires de poste
\paragraph[{Colonne formée pour manœuvrer à portée de l’ennemi}]{Colonne formée pour manœuvrer à portée de l’ennemi}
\noindent Il est avantageux dans plusieurs cas, de se former en colonne pour exécuter un mouvement avec plus de commodité, de rapidité ou de sûreté […].\par
Ce sera lorsque, l’ordre de bataille étant pris, il faudra porter des troupes d’un point ou d’une aile à l’autre et faire quelquefois des changements considérables dans ledit ordre. Dans la tactique qu’avaient, il y a trente ans, toutes les troupes de l’Europe et qu’une partie de ces troupes a encore aujourd’hui, les mouvements, qui mettaient une armée en colonne ou en bataille, étaient si lents et si compliqués, qu’il fallait des heures entières pour faire une disposition générale. Il fallait prendre son ordre de bataille très loin de l’ennemi. Une fois cet ordre formé, on n’osait, crainte de le bouleverser, y hasarder des changements. À présent ou pour mieux dire, dorénavant, si la tactique que j’expose est adoptée, les mouvements qui mettront les troupes en colonne ou en bataille, étant simples, rapides, applicables à tous les terrains, on prendra cet ordre de bataille le plus tard et le plus près de l’ennemi qu’il sera possible ; parce que des colonnes sont bien plus faciles à remuer que des lignes et parce qu’en ne démasquant sa disposition qu’au moment de l’attaque, l’ennemi n’aura pas le temps de la parer. Enfin l’ordre de bataille étant formé, on saura y exécuter des manœuvres intérieures, y apporter des changements et faire succéder à la disposition primitive des dispositions imprévues et si j’ose m’exprimer ainsi, des contre-manœuvres. J’appelle de ce nom tout mouvement occasionné par un mouvement de l’ennemi et ayant pour but d’en balancer ou d’en empêcher l’effet […].
\paragraph[{Colonne de marche}]{Colonne de marche}
\noindent La marche est l’objet pour lequel on forme le plus souvent des troupes en colonne […].\par
La colonne étant formée, elle se mettra en marche d’un pas libre et naturel, au commandement de : {\itshape Pas de route}. Les rangs observeront deux pas de distance entre eux pour donner aux soldats l’aisance et la liberté nécessaires […].
\subsubsection[{9. Des formations en bataille}]{9. Des formations en bataille}
\noindent […]\par
\paragraph[{Reformation de la colonne en bataille}]{Reformation de la colonne en bataille}
\noindent […]
\paragraph[{Déploiement de la colonne}]{Déploiement de la colonne}
\noindent […]
\paragraph[{Observations sur quelques autres manières de mettre un bataillon en bataille}]{Observations sur quelques autres manières de mettre un bataillon en bataille}
\noindent C’est la manie de l’exclusif qui perd tous les faiseurs de systèmes. C’est elle qui égare Folard et tous ses sectateurs. Une fois prévenus de leur opinion, ils ne veulent plus entendre à aucune autre. Quels que soient les lieux, les cas, les armes ; prenez l’ordre que je propose, disent-ils, il est propre à tout, c’est le bon unique, le bon absolu, le bon par excellence. Cela me rappelle le médecin de Molière conseillant ses pilules pour tous les maux. Je cherche à éviter cet écueil. Sur une infinité de circonstances locales, ou autres qui se présenteront à la guerre, les formations en bataille que j’ai exposées ci-dessus, peuvent s’appliquer au plus grand nombre. Il peut aussi y en avoir quelques-unes où elles ne puissent pas servir. Il est à propos de chercher quelles sont ces circonstances et d’indiquer les mouvements qu’elles pourront exiger […].
\paragraph[{Moyens qu’on peut employer, pour faire illusion à l’ennemi sur la force d’une colonne et sur l’objet qu‘elle doit remplir}]{Moyens qu’on peut employer, pour faire illusion à l’ennemi sur la force d’une colonne et sur l’objet qu‘elle doit remplir}
\noindent Ceci montrera combien la tactique que j’expose est peu exclusive, comment elle sait se plier aux terrains, aux circonstances et s’écarter, dans quelques occasions, des règles établies […].
\paragraph[{Principes généraux à observer, pour la formation en bataille des colonnes de plusieurs bataillons}]{Principes généraux à observer, pour la formation en bataille des colonnes de plusieurs bataillons}
\noindent […]\par
Les différents exemples (indiqués ci-dessus) suffiront pour faire sentir la nécessite et la manière d’exercer toujours les troupes relativement au terrain. Avec l’habitude de diriger de pareils exercices, combien les officiers supérieurs des régiments n’acquerront-ils pas d’intelligence et de vrais talents militaires ? Ils pourront, avec leur régiment, exécuter les mêmes combinaisons qu’un officier général avec un corps de troupes, les manœuvres d’une colonne nombreuse ou de plusieurs colonnes n’étant que la multiplication et le concept des mouvements d’un bataillon. Ces exercices deviendront intéressants pour les officiers particuliers, pour le soldat même. Il y a dans tous les hommes un instinct sûr et réfléchi qui leur fait goûter les choses utiles et quel heureux effet ne résulterait-il pas parmi les troupes, de la confiance qu’elles contracteraient dans l’art qui les fait mouvoir ? Apercevant l’objet de leurs travaux, elles cesseraient de s’en plaindre, sachant que leurs chefs sont instruits et qu’elles ont une bonne disposition à prendre dans tous les lieux et dans tous les cas ; elles verraient tout ; elles iraient partout, avec cette sécurité, gage de la victoire.
\paragraph[{Conclusion}]{Conclusion}
\noindent Il n’y a que les charlatans ou les enthousiastes qui proposent sans prouver. Je ne suis ni l’un ni l’autre. Il peut, il doit y avoir des gens qui ont des doutes. Je voudrais les dissiper entièrement. Je vais donc, pour terminer ce chapitre présenter le parallèle raisonné de la manœuvre de deux bataillons en colonne, dont l’un se mettra en bataille sur le front, suivant les anciennes méthodes et l’autre par le pas de flanc, suivant les principes que j’ai établis.
\paragraph[{Des changements de front}]{Des changements de front}
\noindent J’ai différé jusque ici de traiter des changements de front, parce que c’est par le moyen des déploiements que je propose de les exécuter pour la plupart. Ce n’est pas une chose indifférente en tactique que d’appliquer à plusieurs objets une manœuvre déjà nécessaire et de diminuer par là le nombre de celles que les troupes doivent apprendre […].\par
Je terminerai ici mes idées sur la tactique de l’infanterie. J’ai moins prétendu parcourir toutes les circonstances dans lesquelles un régiment pourra se mouvoir, que démontrer le mécanisme le plus simple et le plus rapide par lequel il pourrait se mouvoir dans tous les cas. Il ne reste, je crois rien à ajouter à la perfection d’un art, quand les instruments sont créés ; quand on a appris à l’artiste à les manier ; quand on a développé son génie ; quand enfin on l’a mis en état de n’être surpris par aucune circonstance.
\subsection[{III. - Essai sur la tactique de la cavalerie}]{III. - Essai sur la tactique de la cavalerie}
\noindent Je ne dois point m’engager à faire pour la cavalerie un travail aussi étendu que celui que j’ai fait pour l’infanterie. Accoutumé à manier les détails de cette dernière, j’ai pu en parler avec assurance. Ceux de la cavalerie me sont plus étrangers. Je ne les approfondirai donc pas. Je me bornerai à quelques résultats généraux d’après mes études particulières et ce que j’ai entendu dire à des officiers expérimentés. Après avoir établi les principes et le but de la tactique de cette arme, je tâcherai de prouver qu’elle doit être analogue à celle de l’infanterie, et que conséquemment tout officier qui a le génie de la guerre, doit savoir commander et connaître les manœuvres des deux armes.\par
Quand je dis que la tactique des deux corps n’est qu’une, je ne prétends pas qu’il n’y ait, dans les détails intérieurs des écoles et dans les principes d’instruction, des différences considérables. Il est bien sûr qu’il faut que ces différences existent, puisque les individus et les armes ne sont pas les mêmes. Mais le bataillon et l’escadron étant une fois dressés, les détails cessent. Leurs mouvements doivent arriver aux mêmes résultats. Il faut s’attacher à les combiner ensemble, à les rendre si intimement analogues les uns aux autres, que ceux de l’infanterie ne soient point étrangers à la cavalerie que ceux de la cavalerie ne le soient point non plus à l’infanterie et qu’enfin tout officier qui aura réfléchi et qui aura exercé son coup d’œil relativement à ces deux armes, puisse les manier habilement l’une et l’autre. Cette vue paraîtra sans doute un paradoxe. Je m’y attends. Je le soutiendrai. Je demande seulement qu’on m’écoute sans prévention.\par
\subsubsection[{1. Abus de la cavalerie trop nombreuse}]{1. Abus de la cavalerie trop nombreuse}
\noindent Je commencerai par quelques observations préliminaires sur la cavalerie. En logique, comme en trigonométrie, il faut pour première opération, commencer par établir sa base.\par
Chez les nations sans discipline et sans lumières, la cavalerie est la première arme des armées. Chez celles où la discipline et les lumières ont fait des progrès, elle devient la seconde. Mais la seconde regardée comme nécessaire, comme importante, comme souvent décisive et par conséquent comme devant être portée à la plus grande perfection possible. Elle n’y est que la seconde, parce que la perfection de l’art ouvre une carrière bien plus vaste aux opérations de l’infanterie, parce que l’infanterie propre aux travaux, aux sièges, aux combats, à toutes les natures de pays, est toujours le mobile principal et peut au besoin se suffire à elle-même, tandis que la cavalerie qui n’est presque propre qu’à une seule action et à un seul terrain, ne peut communément se passer de la protection de l’infanterie.\par
En ne considérant la cavalerie que comme la seconde arme, je dis qu’elle entre nécessairement dans la composition d’une armée bien ordonnée et que sa bonté peut beaucoup influer sur le sort de la guerre. En effet c’est la cavalerie qui décide souvent les combats et qui souvent en complète les succès. C’est elle qui protège l’infanterie dispersée et battue. C’est elle qui fait les courses, les avant-gardes, les expéditions rapides. C’est elle qui tient la campagne. Toutes ces opérations sont nécessairement du ressort de la cavalerie, à cause de la célérité de ses mouvements. Que les deux armes cessent donc de se jalouser, qu’elles se regardent comme intimement liées, comme nécessaires l’une à l’autre. L’infanterie pourrait agir et combattre sans la cavalerie ; mais elle n’avancerait qu’à pas de tortue. Elle serait sans cesse harcelée, sans cesse exposée à manquer de subsistance. Elle ne ferait rien de rapide. La cavalerie, sans l’infanterie, ne ferait rien de décisif, ne pourrait s’établir nulle part. Le moindre poste, le moindre obstacle l’arrêteraient. La nuit elle tremblerait pour sa sûreté.\par
Il faut de la cavalerie dans une constitution militaire. Mais il la faut très bonne, plutôt que très nombreuse. On sentira la vérité de cette maxime à mesure que la tactique fera des progrès. L’inverse sera une marque de la décadence de l’art militaire. Lorsque l’infanterie sera brave, bien armée, lorsqu’elle saura se suffire à elle-même, lorsqu’elle ne se croira pas battue quand elle n’est pas soutenue par de la cavalerie, on n’aura de la cavalerie, que dans une saine et raisonnable proportion avec les objets qu’elle doit remplir et on l’aura bonne et bien dressée. Lorsqu’au contraire l’infanterie sera l’opposé de ce que je viens de dire, lorsque la tactique sera dans l’enfance, lorsqu’elle ne saura fournir des ressources, ni à l’infanterie contre la cavalerie, ni à la cavalerie contre une cavalerie supérieure, il faudra une cavalerie immense ; parce qu’il en faudra pour couvrir les ailes pour appuyer partout l’infanterie et après cela pour être supérieure à la cavalerie ennemie. Car dans toutes nos fausses combinaisons actuelles de constitution, c’est toujours l’ennemi qui donne la loi. S’il met deux cents escadrons en campagne on se croit battu dès qu’on ne lui en oppose pas au moins deux cents.\par
Qu’arrive-t-il cependant de cette quantité de cavalerie accrue follement et par imitation, au-delà des bornes raisonnables ? C’est qu’elle est onéreuse pour l’État, s’il veut l’entretenir pendant la paix. C’est que si, trouvant le fardeau trop pesant, il ne l’augmente qu’à la guerre : voilà des régiments nouveaux, ou des compagnies nouvelles, ou des incorporations subites de jeunes cavaliers et de jeunes chevaux. Il faut entrer en campagne, tout cela ne se trouve ni ameuté ni amalgamé. Les travaux de la paix deviennent inutiles. Il n’en est pas de la cavalerie comme de l’infanterie. Un bataillon peut recevoir quelques recrues, sans que cela déroute et dérange absolument l’instruction du bataillon. Mais qu’on place, dans l’escadron le plus instruit, des cavaliers ou des chevaux non dressés, le faux mouvement d’un seul de ces individus suffit pour entraîner l’escadron et pour faire manquer ses manœuvres.\par
Qu’arrive-t-il encore de cette cavalerie si prodigieusement accrue dans les armées ? C’est qu’il n’y a presque pas une occasion où elle puisse s’employer en totalité. C’est que dans la plupart des pays, elle est embarrassante à mouvoir et à nourrir. De là magasins énormes, convois continuels, communications immenses, pour peu qu’on s’éloigne. De là ces grandes vues de la guerre subordonnées à des calculs de subsistance et les armées appesanties, au lieu que le vrai but d’une cavalerie raisonnablement nombreuse devrait être d’alléger et de faciliter les mouvements des armées.\par
Mais un changement dans la routine de nos opinions à cet égard, ne peut être que l’ouvrage du temps et d’un grand nombre de circonstances. Il faut qu’auparavant la tactique de l’infanterie ait été perfectionnée et que celle de la cavalerie soit mise à la même hauteur. Il faut qu’un général, homme de génie, soit frappé des ressources qu’offriraient de nouveaux mouvements plus rapides et plus raffinés ; que de là il ose se mettre en campagne avec une cavalerie excellente et peu nombreuse : que sa cavalerie, une fois combinée sur ses vues et sur la force de son infanterie, il voit l’ennemi augmenter la sienne et que non seulement il ne soit pas tenté de l’imiter ; mais même qu’il soit persuadé que la supériorité que son adversaire aura cru se donner, ne fait que l’affaiblir, parce qu’au-delà de certaines proportions, le nombre ne produit qu’embarras et lenteur. Ce que je vais dire ci-après, jettera peut-être les fondements de cette importante révolution et cet espoir doit m’encourager.
\subsubsection[{2. Armure et habillement de la cavalerie}]{2. Armure et habillement de la cavalerie}
\noindent La définition des propriétés de la cavalerie me conduira à déterminer, d’une manière plus précise, son ordonnance et sa constitution.\par
La cavalerie n’a qu’une manière de combattre, c’est par la charge ou par le choc. Toute action de feu en troupe lui est impropre. On n’a qu’à voir même combien est inutile et peu meurtrier le feu des troupes légères à cheval, quoique éparpillées et pouvant tirer avec liberté et sang-froid. Si on laisse des armes à feu à la cavalerie, ce n’est donc pas pour s’en servir à cheval, c’est pour en faire usage, dans la supposition ou faute d’infanterie, elle serait obligée de mettre en partie pied à terre, pour garnir la tête d’un défilé, ou pour occuper un poste. Je voudrais qu’elle fût, pour cet effet, armée d’une carabine et d’un seul pistolet.\par
Si la charge ou le choc est la seule action propre à la cavalerie, il faut chercher à rendre ce choc redoutable. Comment y parvenir ? En augmentant sa vitesse. En voici les raisons.\par
L’avantage principal et décisif de la cavalerie, c’est la vélocité des mouvements : 1) parce qu’elle ajoute à la force du choc dont je vais tout à l’heure démontrer l’action physique ; 2) parce que hors de l’action du choc elle fait que dans une disposition de combat, la cavalerie se transportant avec rapidité d’un point à l’autre, fait changer de face aux circonstances et à la fortune.\par
On doit entendre par la plus grande vitesse possible de la cavalerie, non la plus grande vitesse possible d’un cavalier seul, et abandonné à sa vélocité, mais la plus grande vitesse possible d’une troupe, conservant cependant toujours son ordre et proportionnant cette vitesse à la distance du but dont elle part, jusqu’au but où elle se porte et à l’objet qu’elle doit remplir en y arrivant. On avait faussement cru en France, que cette vitesse était incompatible avec l’ordre. De là, par exemple la cavalerie ne savait pas manœuvrer au galop. De là elle avait adopté une manière de charger l’ennemi, qu’on appelait charger en fourrageurs, parce que véritablement cette cavalerie ainsi à la débandade, ressemblait à une troupe de fourrageurs lâches et se dispersant dans l’enceinte de la chaîne. Il était véritablement plaisant que ce fussent-là l’image et l’étymologie de la seule manœuvre de combat, que sût exécuter la cavalerie française. Avec cela j’entends encore quelques anciens officiers réclamer cette manière de charger. C’était celle de la nation, disent-ils. Ainsi nos pères battirent l’ennemi à Fleurus, à Leuze, etc. En effet et c’est là sans doute ce qui a contribué à retarder chez nous les progrès des lumières, notre valeur s’est de temps en temps créé quelques époques de gloire au milieu de notre ignorance. Mais, peut-on répondre à ces anciens officiers, vos aïeux furent battus dans mille autres occasions : Crécy, Poitiers, Azincourt, Ramillies, Hochstadt, nous font rougir encore. Dans combien d’autres combats notre chevaleresque ignorance ne nous a-t-elle pas été funeste ? Je veux qu’elle soit redoutable dans son premier effort. Elle est incapable d’un second. Repoussée, elle ne sait point se rallier. Victorieuse, elle ne peut pas profiter de sa victoire. En voulez-vous une preuve ? Aucune nation n’a perdu de batailles aussi honteuses, aussi décisives, que la nôtre. Aucune n’en a gagné si peu de décisives et de complètes. Mais finissons cette incursion sur une erreur dont on commence à revenir et reprenons le fil de mes principes.\par
Pour que la cavalerie ait cette vélocité de mouvement si avantageuse, quand l’ordre y est joint, il faut qu’elle ne soit appesantie, ni par ses armes, ni par l’ordonnance sur laquelle elle est rangée. C’était donc contre toute espèce de principes que les Anciens formaient leurs turmes de cavalerie, de huit de front sur huit de profondeur, ou en losange, en trapèze, en coin. C’était par un reste de cette ignorance que dans des temps plus modernes la cavalerie combattait sur quatre ou sur six de profondeur. C’était par une ignorance tout aussi funeste, qu’elle était armée de pied en cap et couverte d’armes défensives.\par
On ne voit point dans l’histoire, sans pitié pour l’aveuglement de ce temps, la gendarmerie bardée de fer, allant à la charge au pas et au trot ; ne pouvant se mouvoir, si la pluie avait détrempé le terrain et périssant alors sous son inutile armure et sous les coups des archers, ou d’une cavalerie plus légère. Quelques siècles auparavant la cavalerie romaine, armée de même, essuyait les mêmes désastres. Ces exemples malheureux ont enfin fait renoncer à l’ordonnance de profondeur et aux armures pesantes. Mais cette révolution s’est faite lentement. Il en est ainsi de toutes les erreurs que le préjugé de plusieurs siècles à accréditées. Longtemps on a conservé les lances, les cuirasses, les plastrons, les bottes fortes, l’ordonnance sur quatre et sur trois rangs. Aujourd’hui enfin tous les anciens officiers se trouvent trop nus, trop désarmés, trop légers, disent-ils, comme si la cavalerie pouvait jamais trop acquérir une propriété, dans laquelle consistent tout son avantage et toute son utilité.\par
Il résulte de ce que je viens de dire :\par
1) Que si l’on veut se donner la peine d’étudier l’antiquité, on verra que les meilleures cavaleries, la thessalienne, la numide, l’espagnole étaient à demi-nues sur des chevaux presque nus aussi et armées ou de haches ou d’épées tranchantes et que ce ne fut qu’en se rapprochant de leur institution et en allégeant, que la cavalerie grecque et romaine parvinrent à lutter quelquefois plus également contre elles.\par
2) Qu’on doit à jamais abolir le mot de cavalerie pesante, parce que cette épithète est hétérogène à l’institution de la cavalerie. Toute la différence qui peut exister entre les différents corps de cavalerie, ne doit consister que dans des hommes et des chevaux plus ou moins élevés et qu’ainsi ce que nous appelons la cavalerie, étant destiné à combattre toujours en escadron et en ligne doit être composé des hommes les plus élevés et les plus robustes, tandis que les dragons et houzards, destinés à se mouvoir plus rapidement, à se disperser, à faire la guerre en détail, sans cependant ignorer comment elle se fait en masse, doivent être composés de chevaux plus petits et d’hommes proportionnés.\par
3) Qu’il faut que tout ce qui s’appelle combattant à cheval, renonce à tout ce qui appesantit et surcharge, comme cuirasses, plastrons et autres armes défensives à l’épreuve du coup de fusil. Voici mes raisons. La cavalerie en panne doit se tenir hors de la portée du feu de la mousqueterie. La cavalerie ne doit en tout s’attaquer à l’infanterie, que quand cette dernière est ébranlée, ou en mauvaise contenance de courage ou de disposition […].\par
4) Qu’à plus forte raison il faut que la cavalerie renonce à ces armes prétendues défensives contre le feu et dans le fait, nulles contre lui, comme plastrons, demi-plastrons […].\par
5) Que, si je désapprouve toute espèce d’armes défensives contre le feu, j’approuverais au contraire quelques précautions pour défendre le cavalier contre l’arme blanche, pourvu que ces précautions ne fissent ni poids, ni embarras. Je voudrais, par exemple, qu’on couvrit la tête du cavalier d’un casque à l’épreuve du coup de sabre et ses épaules de trois rangs de chaînes de maille attachées sur une épaulette de cuir\par
6) Que la lance et toutes les armes de longueur doivent être rejetées pour la cavalerie, parce qu’elles sont lourdes à porter hors de l’action et embarrassantes à manier pendant le combat ; parce qu’elles exigent qu’on combatte à files ouvertes et presque seul à seul, afin de pouvoir prendre champ pour les manier ; parce qu’au moyen de cela, il ne peut plus y avoir ni ordre, ni manœuvre, ni unanimité de choc […].\par
7) Qu’après un examen réfléchi, il paraît qu’il n’y a pas de meilleure arme pour la cavalerie que des épées tranchantes. Moins on les rendra longues, plus elles seront avantageuses et meurtrières\par
Enfin quand la valeur d’un peuple baisse, on allonge les armes, on prend des armes de jet on cherche à mettre le plus d’intervalle qu’on peut entre l’ennemi et soi […].\par
8) Qu’après avoir allégé le cavalier dans son armure et son habillement, après avoir de même cherché pour son cheval, la forme de harnais la plus simple, la plus commode et la plus légère, il faut donner à l’escadron la constitution et l’ordonnance la plus propre à favoriser la vélocité et l’ordre de ses mouvements, sans que cela nuise à la force de son choc. C’est ce que le chapitre suivant aura pour objet de déterminer.
\subsubsection[{3. Vélocité des mouvements de la cavalerie}]{3. Vélocité des mouvements de la cavalerie}
\noindent La cavalerie, allant à la charge, a sans contredit une force de choc, mais cette force de choc n’est produite que par la quantité de vitesse avec laquelle elle se meut et par la quantité de masse du premier rang seulement. Car la quantité de masse des rangs suivants n’ajoute rien à celle de ce premier, puisqu’il ne peut y avoir entre des chevaux, ni cette pression, ni cette adhérence sans interstices, par laquelle des corps rangés l’un derrière l’autre, se compriment et augmentent la force du corps qu’ils poussent.\par
Ainsi, pour procurer à la cavalerie une plus grande quantité de mouvement, ou une force de choc plus décisive, ce n’est donc point la profondeur de son ordonnance qu’il faut augmenter, c’est la quantité de vitesse.\par
Pour que cette quantité de vitesse produise tout l’effet qu’on doit en attendre, il faut qu’elle soit proportionnée à l’éloignement du but où l’on va frapper. Si ayant six cents pas à parcourir, on s’ébranlait avec la même vitesse que si l’on n’en avait que deux cents, les chevaux s’essouffleraient et le mouvement irait en se ralentissant vers la fin de la charge, tandis qu’au contraire il doit augmenter d’accélération. Il faut que cette quantité de vitesse soit graduelle et progressive, c’est-à-dire par exemple, que, si un corps de cavalerie allant à la charge, a six cents pas à parcourir, il doit s’ébranler au petit trot, faire ainsi deux cents pas et ensuite deux cents au grand trot. Cette mesure de mouvement ne manquera pas de s’accélérer presque d’elle-même, à proportion que les chevaux s’échaufferont et se mettront en haleine. Enfin les deux cents pas restants seront faits au galop, les cavaliers baissant la main et abandonnant leurs chevaux aux cinquante derniers, de manière que la plus grande quantité de vitesse possible existe en arrivant sur la troupe qu’on va charger et qu’ainsi cette impétuosité, rendue décisive par l’accélération du mouvement, étourdisse le cavalier sur le danger et entraîne sur l’ennemi le lâche comme le brave et les demi-volontés comme les volontés entières.\par
Je dois observer qu’il existe à cet égard une différence remarquable entre l’action de choc d’une troupe d’infanterie et celle d’une troupe de cavalerie. La première, ainsi que je l’ai dit en son lieu est souvent ralentie dans son mouvement, par l’instinct machinal qui fait hésiter le soldat à l’approche du danger. La troupe de cavalerie au contraire et cela lui donne une analogie plus parfaite avec les corps physiques, étant une fois déterminée, les chevaux s’animent à un tel point par l’accélération et par l’ensemble de mouvement, qu’ils entraînent la volonté du cavalier et le portent jusque sur l’ennemi, sans que la force motrice éprouve de ralentissement et d’altération […].\par
On vient de voir la vélocité de mouvement appliquée à l’action de charge. Elle doit de même avoir lieu dans toutes les manœuvres. Car ce principe, que j’ai posé en traitant des évolutions de l’infanterie, que presque toute manœuvre étant un moment de crise et de faiblesse pour une troupe, parce qu’elle y est désunie, il faut qu’elle en sorte le plutôt possible, est commun aux deux armes.
\subsubsection[{4. Ordonnance de la cavalerie}]{4. Ordonnance de la cavalerie}
\noindent L’ordonnance habituelle de la cavalerie doit être sur deux rangs. Ce n’est pas pour que son second augmente sa force de choc, car, comme je l’ai déjà dit, où il ne peut y avoir pression exacte, la quantité de masse ne saurait s’accroître; mais c’est pour que le second rang soit à portée de remplacer les pertes et les vides du premier. C’est pour qu’arrivé sur l’ennemi et étant mêlé avec lui, ce second rang augmente le nombre des combattants.\par
Mais, disent quelques officiers de cavalerie, il y aurait un avantage infini à avoir un troisième rang en entier, ou du moins aux ailes de l’escadron. Souvent en allant à la charge, les escadrons s’ouvrent et se désunissent. Souvent on aurait besoin d’étendre le front, soit pour déborder l’ennemi, soit pour n’être pas débordé par lui. Le troisième rang bien exercé servirait à remplir cet objet.\par
Un habile officier de cavalerie, avec qui je m’entretenais de cette opinion, m’en a proposé une qui me paraît meilleure et plus réfléchie. Ce serait d’avoir, au lieu de ce troisième rang, une petite troupe d’élite, montée sur des chevaux plus légers que ceux de l’escadron, formée sur deux rangs et placée à vingt pas en arrière, ou à côté de l’intervalle. Cette troupe, aux ordres d’un officier choisi, aurait pour objet de fermer l’intervalle, quand cela serait jugé nécessaire et quand cet intervalle s’ouvrirait au-delà de la distance ordonnée. Elle serait exercée à gagner à toutes jambes, par cet intervalle, le flanc de l’ennemi, quand on approcherait de lui. Quelquefois pour rendre son mouvement plus inopiné et plus décisif, cette troupe d’élite serait placée en arrière d’une des ailes de l’escadron et dérobée ainsi aux yeux de l’ennemi. Elle ne paraîtrait qu’au moment de la charge. Aurait-on besoin de tirailleurs en avant, ce serait elle qui les fournirait. L’ennemi battu, ce serait elle qui le poursuivrait. Elle agirait enfin dans le combat, pour le plus grand succès de l’escadron, suivant l’intelligence de l’officier qui la commanderait et serait tour à tour la réserve, le corps auxiliaire et le corps défensif de cet escadron […].
\subsubsection[{5. École du cavalier}]{5. École du cavalier}
\noindent II faut beaucoup de temps pour former un bon cavalier. Ce que j’entends par un bon cavalier, ce n’est point un homme exercé à manier son cheval avec grâce et adresse. Ce n’est point un écuyer. C’est un homme robuste, placé à cheval ainsi qu’il doit l’être, relativement à la structure de son corps et à la facilité la plus grande de le gouverner, le gouvernant et le dirigeant à son gré, mais plutôt par l’éperon et le poignet, plutôt par son étreinte et son assiette vigoureuse, que par les aides et toutes les finesses de l’équitation. C’est un homme intrépide à cheval et qui, moins instruit que brave, n’imagine rien d’impossible pour son cheval et pour lui. C’est avec cela un homme qui aime son cheval, qui le soigne, comme un fantassin doit soigner son fusil ; qui connaisse tous les détails journaliers nécessaires à sa conservation ; qui ait fait plusieurs campagnes ; et qui, par conséquent, familiarisé avec les combats, les fatigues, les accidents, ne soit étonné de rien […].\par
C’est, ce me semble, une étrange chose et qui porte bien l’empreinte du caractère national, que le système d’après lequel nous travaillons depuis six ans à former notre cavalerie. Elle était dans l’ignorance et enchaînée par les vices de sa constitution. Elle ne pouvait faire un pas pour en sortir. La paix de 1763 se fait. Le gouvernement change cette constitution et en substitue une, sinon parfaite, du moins propre à l’essai d’une instruction et a l’encouragement de l’émulation. On dit au gouvernement et on lui dit avec raison, que le grand vice de la cavalerie française est le défaut d’instruction ; qu’elle ne sait pas manier ses chevaux ; qu’avant de dresser l’escadron, il faut dresser le cavalier. Le gouvernement, frappé de cette vérité, ordonne qu’on construise des manèges, appelle des écuyers, jette un coup d’œil favorable sur tous ceux qui apportent du zèle et de l’aptitude aux institutions nouvelles. À l’instant toutes les têtes fermentent. Les villes de guerre, les quartiers se remplissent d’écoles d’équitation. Il n’y a plus de bons officiers que ceux qui manient un cheval avec adresse. Les vieux cavaliers n’ont ni la souplesse, ni la grâce qu’on exige. Il faut les renvoyer. Il faut en user de même à l’égard des anciens officiers. On dirait que toute la science de la cavalerie s’apprend dans la poussière des manèges. Cependant au milieu de cette effervescence, les principes de l’équitation ne sont ni posés, ni reconnus. On les discute, on les change. Deux systèmes différents partagent les opinions, sans compter nombre de petites éducations particulières, imaginées par les chefs des régiments. Les années passent, les chevaux se ruinent, les cavaliers sont excédés. On forme dans chaque régiment quelques officiers écuyers et dix ou douze cavaliers créais\footnote{Cavalier créât : sous écuyer dans une école d’équitation, un manège.}. Notez que ces derniers le sont à peine, qu’ils désirent leur congé, pour aller se faire piqueurs en France, ou chez l’étranger. Dans les régiments les plus avancés, on met cinquante ou soixante hommes par escadron en état de manœuvrer. On forme les autres successivement, mais successivement aussi l’engagement des hommes formés est à son terme. Des recrues leur succèdent. Des chevaux neufs remplacent de même les chevaux dressés et ruinés, chose devenue synonyme, par les travaux établis dans les manèges. Bref dans cette fluctuation continuelle d’individus et de principes, dans ces écoles outrées de détail et de précision, tout se consume, les hommes, les chevaux et ce qu’il y a de plus précieux encore, le temps de la paix, ce temps fugitif et irrévocable qui devrait être employé à rassembler de grands camps, à exécuter de grandes manœuvres et à étudier leur résultat.\par
Eh ! dirait la raison à tous ces instituteurs modernes, si la raison était appelée à leur conseil, quel est votre but ? Notre but est de sortir de l’ignorance, puisque toute l’Europe s’éclaire. Notre but est de rendre la cavalerie manœuvrière et pour cela d’établir des écoles. D’accord, mais avant que d’établir des écoles, cherchons la vérité, posons des principes. Vous avez, je pense, songé que vos cavaliers sont, ou doivent être en plus grande partie, des paysans bien épais, bien grossiers, et par conséquent bien sourds à toutes les recherches d’un art raffiné. Vous avez réfléchi, sans doute, que votre constitution vous oblige à congédier tous les ans le huitième de ces cavaliers : qu’il en meurt, qu’il en déserte tous les ans quelques-uns, qu’en temps de guerre ces deux branches de consommation s’accroissent considérablement. Vous avez fait le même calcul pour les chevaux : vous savez donc qu’il faut, pour vos cavaliers et pour vos chevaux, une instruction prompte, simple et qui les mette le plutôt possible en état d’entrer dans l’escadron. Maintenant, messieurs les instituteurs, vous prétendez que l’équitation est la base indispensable de cette instruction. Mais de quelle espèce d’équitation parlez-vous ? Si c’est de cet art qui, à force de vouloir rendre un cheval agréable et souple, lui fait la bouche délicate, les aides fines et les jarrets tremblants : si c’est de cet art, par le moyen duquel vos jeunes gens, placés de très bonne grâce, ne savent pas au bout de deux ans maîtriser un cheval, gardez ces leçons pour les manèges, elles ne conviennent ni à l’espèce de nos cavaliers, ni à celle de leurs chevaux, ni au temps qu’on peut employer à leur éducation […]. Enfin, messieurs, conclurait la raison, vous n’avez pas depuis six ans achevé l’éducation d’un régiment entier. La moitié de la cavalerie du royaume suit encore les talons et change de main dans la poussière des manèges. Portez ailleurs votre lente méthode, votre bonne grâce, votre théorie raffinée. Elles peuvent être le fruit de beaucoup de méditations, mais je ne m’en servirai pas : car je veux des cavaliers et non pas des écuyers.\par
Déterminer ensuite la méthode la plus prompte, la plus simple et la plus conforme au mécanisme du corps, pour placer un paysan à cheval et lui apprendre à le conduire. Ne point hérisser cette instruction des difficultés et des mots de l’art. Déterminer de même la meilleure et la plus courte manière de dresser un cheval et de le mettre en état d’entrer dans l’escadron, sans l’accoutumer à des aides trop recherchées, sans le ruiner pour vouloir l’assouplir. Voilà ce que la raison donnerait à résoudre aux officiers de cavalerie les plus habiles, donnant la préférence au système qui remplirait ces objets avec le plus de facilité et de promptitude et le mettant ensuite en exécution dans toutes les écoles du royaume. Il entrerait dans ce système, et ce serait un des principaux changements à exiger de celui qui le donnerait, que, passé les premières leçons de longe et d’assiette, les écoles se feraient en plein champ dans toutes sortes de terrain et non entre les murailles ou les barrières d’un manège et sur des surfaces battues et aplanies avec soin. Car que deviennent des cavaliers et des chevaux qui ont été élevés dans des enceintes et sur des terrains pareils, lorsqu’à la guerre ils se trouvent transportés dans les lieux vastes et difficiles ?
\subsubsection[{6. Analogie entre les mouvements de la cavalerie et ceux de l’infanterie}]{6. Analogie entre les mouvements de la cavalerie et ceux de l’infanterie}
\noindent Je suppose les cavaliers dressés et en état d’être rassemblés en escadron. Là commence l’analogie que j’ai annoncée devoir exister entre les mouvements de la cavalerie et ceux de l’infanterie. Je vais le prouver.\par
Ce n’est point par le nombre des mouvements que la tactique de la cavalerie est relative à celle de l’infanterie. Car, comme la cavalerie n’est propre qu’à l’action de choc, ses mouvements sont en bien moindre quantité. Ils se réduisent à savoir se mettre en ordre de marche, se remettre en bataille, marcher en ligne et à quelques autres mouvements indiqués par les circonstances.\par
Ainsi que l’infanterie, la cavalerie doit pouvoir se mettre en ordre de marche sur le front ou sur le flanc […].
\subsubsection[{7. Des formations en bataille}]{7. Des formations en bataille}
\noindent Dans cette manœuvre, dans la conduite de cette manœuvre consiste véritablement presque toute la science et l’instruction de la cavalerie : car la cavalerie n’a de force et d’action qu’autant qu’elle est en bataille. Dans tout autre ordre, elle est faible et sans défense. Elle ne peut enfin avoir de succès, qu’autant qu’elle sait se former en un clin d’œil, cacher sa force et se mettre rapidement en état d’en faire usage.\par
C’est surtout dans les formations en bataille, que se montre l’analogie annoncée entre la tactique de la cavalerie et celle de l’infanterie. Elle y est si sensible, que les détails dans lesquels je pourrais entrer sur les principes et sur la théorie des formations en bataille de la cavalerie, ne seraient, aux changements des termes près occasionnés par la différence des armes et des constitutions, qu’une répétition exacte de ce que j’ai exposé dans la tactique de l’infanterie […].
\subsubsection[{8. Mouvement de charge}]{8. Mouvement de charge}
\noindent C’est ici l’action de combat de la cavalerie et par conséquent son mouvement important et décisif. On ne saurait donc le figurer trop souvent dans les exercices, tant pour y accoutumer les chevaux et les cavaliers, que pour former le coup d’œil des officiers qui le conduisent et pour les habituer à saisir cet à-propos si précieux, de la connaissance et de l’emploi duquel dépendent presque tous les combats de cavalerie.\par
Je crois avoir démontré d’une manière sensible les principes et la théorie de l’action du choc, la manière de se procurer la plus grande quantité possible de vitesse, sans renoncer à l’ensemble de mouvement. Je crois encore avoir démontré sensiblement la nécessité de cet ensemble de mouvement. C’est lui qui produit l’unanimité d’efforts. C’est lui qui concourt, avec la vitesse, à augmenter la force du choc. C’est lui enfin qui en impose à l’ennemi, qui le renverse, qui fait trouée. Car la cavalerie bat plutôt en effrayant, en dispersant ce qui s’oppose à elle, qu’en répandant du sang et dans ce sens-là c’était un homme qui selon moi, connaissait bien la propriété de la cavalerie, qu’un officier qui me disait un jour, qu’il comptait plus pour le succès d’une charge, sur la quantité de vitesse et d’ordre de son escadron, que sur la trempe de ses armes.\par
Tous les mouvements de charge de la cavalerie doivent se faire en bataille. Il peut y avoir une ou deux occasions où il soit avantageux de charger en colonne. C’est le cas, par exemple, où il s’agirait d’attaquer une infanterie environnée, surtout si elle présentait maladroitement un flanc, ou des angles dégarnis de feu et sur la capitale desquels il fût possible d’arriver presque à couvert. Quelles doivent être alors ces colonnes ? Ce ne seront pas des troupes serrées et pressées les unes derrière les autres. Ce seront des demi-escadrons, ou des escadrons se suivant à trente, quarante, ou cinquante pas d’intervalle entre eux, se portant ainsi sur l’infanterie par une succession continue d’efforts et pouvant, au moyen de leurs intervalles, manœuvrer s’il en était besoin, soit pour changer la direction de leur attaque, soit pour n’être pas renversés par le mauvais succès des escadrons qui les précèdent. Ces sortes de colonnes ne devront pas être composées de beaucoup d’escadrons, parce qu’il vaudra mieux les multiplier et en attacher à tous les angles à la fois, que d’en former de considérables, qui, par leur profondeur, ne feraient que donner plus de prise au feu de l’ennemi, sans augmenter l’effet de la charge. Car supposé que le premier escadron ou demi-escadron de cette colonne soit battu, le second, le troisième, le quatrième et peut-être le cinquième et le sixième pourront renouveler des efforts décisifs […].\par
La seconde occasion, où il peut être convenable de charger en colonne, c’est quand, avec une cavalerie supérieure, on aura à charger un corps de cavalerie inférieur, occupant, je suppose, une trouée et si bien appuyée à ses deux ailes, qu’il serait impossible de le tourner ou de l’incommoder avec de l’infanterie glissée sur ses flancs, ou avec de la cavalerie mise sur pied à terre pour cet objet, comme dans le cas où il serait entre deux marais […].\par
Hors les deux occasions susdites, toutes les charges de cavalerie doivent se faire en bataille. Car le grand avantage de la cavalerie, quand elle est supérieure, c’est de déployer ses forces, de les étendre, de gagner le flanc ou les derrières de la disposition ennemie ; tout comme le grand art de la cavalerie qui est inférieure, doit être d’empêcher qu’on ne la déborde, en sachant pour cet effet appuyer ses ailes, ou en les renforçant par des crochets, par des obliques, par des escadrons aboutés aux ailes, ou cachés à la faveur de quelque éminence, lesquels crochets, obliques, ou escadrons de réserve laissent engager la pointe de l’aile de l’ennemi qui s’avance avec confiance, croyant prendre en flanc l’ennemi qu’elle déborde et se voit au contraire au moment de la charge, prise en flanc elle-même par ces corps, qui doivent s’abattre sur elle tête baissée et sans considération du nombre […].\par
Cette tactique sera peut-être un jour mise en œuvre par un général qui voudra alléger son armée, diminuer le grand nombre de la cavalerie et n’en avoir qu’une quantité raisonnablement proportionnée à son infanterie, sans s’embarrasser du nombre supérieur d’escadrons que l’ennemi pourra lui opposer. On verra alors combien le génie et la science des manœuvres l’emportent aisément sur la multitude. On verra ce général, si la supériorité des manœuvres de sa cavalerie ne suffit pas pour contrebalancer son infériorité du côté du nombre, savoir la renforcer par d’autres moyens, par de l’infanterie, par de l’artillerie, par des ouvrages de fortification, qui seront les bastions et les contreforts de sa position, tandis qu’il placera sa cavalerie en arrière des courtines, pour qu’elle puisse prendre carrière sur l’ennemi qui tenterait d’y pénétrer. On le verra savoir se passer entièrement de cavalerie à une aile, pour la réunir toute par des manœuvres très rapides, sur un point où il prévoira en tirer parti. Car tel est l’avantage qu’on peut retirer de la tactique exposée dans cet ouvrage, que l’homme de génie n’ayant, je suppose, que quatre-vingts escadrons dans une armée, contre cent escadrons dans l’armée ennemie, saura par la combinaison de ses déploiements et des dispositions de son ordre de bataille, porter soixante escadrons où l’ennemi n’en aura que cinquante et battre par conséquent ces cinquante escadrons, avant qu’ils aient reçu du renfort, tandis que les vingt qui lui resteront, seront à l’abri, ou par leur éloignement, ou par le terrain où ils seront placés, ou par l’appui que leur fourniront les autres armes, de craindre les efforts de l’ennemi.\par
Tout ce que j’ai dit ci-dessus, est en plus grande partie relatif à la grande tactique, puisqu’il y est question de manœuvres en ligne et de mouvements d’armée. Mais j’éprouve toujours que, dans des discussions pareilles, il est impossible de s’arrêter. Comment parler des détails, sans jeter un coup d’œil sur leur résultat ? Comment expliquer le jeu particulier des ressorts sans faire apercevoir l’influence qu’ils doivent avoir sur l’ensemble de la machine […].\par
Une vérité souvent dite et trop peu méditée, c’est que la science du coup d’œil est essentiellement ce qui constitue le bon officier de cavalerie. L’infanterie se remuant avec plus de lenteur, l’œil a plus de temps pour mesurer et pour comparer. Dans la cavalerie au contraire, les mouvements étant très rapides, il faut que les déterminations soient prises avec la même rapidité. Les points de vue sont plus difficiles à saisir, les moindres erreurs de coup d’œil produisent bientôt des déviations considérables. Enfin la même vitesse avec laquelle on fait un faux mouvement, employée plus utilement par un ennemi habile, lui donne des ailes pour profiter des fautes. De là il s’ensuit que les officiers supérieurs ne sauraient trop s’attacher à former le coup d’œil des officiers qui sont à leurs ordres, à exercer eux-mêmes le leur, à le fortifier comme les illusions que les différences de terrain produisent, à manier en conséquence leurs régiments, tantôt sur des surfaces unies, tantôt dans des terrains inégaux et onduleux, quelquefois même dans des bois clairs, dans des pays coupés d’obstacles surmontables. De là il s’ensuit que le gouvernement devrait souvent rassembler de gros corps de cavalerie, leur faire exécuter de grandes manœuvres, puis former des camps composés de toutes les armes et là les amalgamer, les accoutumer l’une à l’autre, et leur faire étudier ce que j’appelle la grande tactique.
\subsubsection[{9. Conclusion}]{9. Conclusion}
\noindent C’est en traitant cette grande tactique que je démontrerai, par les avantages qu’on peut tirer de la cavalerie, le peu de parti qu’on en a tiré jusqu’à présent. Perfectionner la tactique particulière de cette arme, indiquer la meilleure manière de l’employer, soit seule, soit combinée avec les autres armes, prouver qu’au-delà d’une certaine proportion, l’accroissement du nombre de la cavalerie ne fait qu’appesantir les armées et mettre des entraves à la perfection de l’art militaire, voilà les trois objets que j’ai en vue. Je viens de commencer à les remplir, en cherchant les principes sur lesquels doivent être fondés la constitution, l’ordonnance et les mouvements de la cavalerie en simplifiant ces manœuvres, en les rendant plus rapides, plus décisives et presque entièrement analogues à celles de l’infanterie. Je souhaite que cette ébauche imparfaite engage des militaires plus habiles que moi, à rectifier mes idées, ou à les étendre. Cela peut, cela doit être. Aujourd’hui le bandeau est levé. Beaucoup d’officiers de cavalerie s’appliquent et s’éclairent. J’en connais, dont j’ai emprunté les lumières et qui étaient plus faits que moi pour les répandre.
\subsection[{IV. - Des troupes légères}]{IV. - Des troupes légères}
\subsubsection[{1. Origine et accroissement des troupes légères}]{1. Origine et accroissement des troupes légères}
\noindent On a longtemps fait la guerre sans cette espèce de troupes que nous appelions aujourd’hui troupes légères. Car les armes à la légère des Anciens ne leur ressemblaient en rien, ni par leur constitution, ni par l’usage qu’on en faisait. Ils étaient vêtus plus légèrement que les autres troupes. Ils étaient armés différemment. Ils étaient composés d’une autre espèce d’hommes. Ils faisaient cependant corps avec les pesamment armés. Ils marchaient avec eux, combattaient avec eux, faisaient en un mot partie de l’ordonnance de combat. Nos troupes légères au contraire sont armées et habillées comme nos autres troupes. Elles sont composées de la même espèce d’hommes : mais elles ne font point corps avec elles. Elles ont un genre de guerre et des fonctions séparées. Un jour de bataille elles ne se mettent point en ligne. Elles ne sont presque comptées que comme hors-d’œuvre dans la disposition générale. Les Parthes, les Numides, les Thessaliens, cette cavalerie si légère et si vantée dans l’histoire, ne peuvent pas non plus se comparer à nos troupes légères, puisque c’étaient des nations entières ainsi constituées, habituées à ce genre de guerre, de vitesse et de désordre, et n’ayant point de troupes d’une autre espèce. Tels sont encore aujourd’hui les Tartares de Crimée et quelques peuples de la côte d’Afrique.\par
Comment faisaient donc les Anciens pour avoir des nouvelles, pour faire des courses, pour se garder contre les surprises, pour remplir tous les objets dont nous avons aujourd’hui assigné l’exécution aux troupes légères ? Cette question est trop intéressante, trop propre à jeter du jour sur la grande partie de la guerre, pour que je ne cherche pas à la résoudre.\par
Les Anciens avaient un autre genre de guerre que nous. Ils faisaient en général moins de marches et de mouvements. Ils étaient retranchés dans tous leurs camps. Ils avaient pour principe de se tenir toujours le plus près possible de l’ennemi. En étaient-ils éloignés ? Comme leurs camps étaient des citadelles, ils avaient moins besoin de postes extérieurs. Dans ces camps étaient à la fois leurs arsenaux, leurs magasins, leurs ateliers de toute espèce. Ils avaient soin de les asseoir à la portée de la mer, d’une rivière, d’une ville, ou d’un grand entrepôt fortifié. Voyons, pour nous donner une idée de leur conduite à cet égard, la belle campagne de César en Afrique. Il n’avait que des légions et il faisait la guerre contre une multitude d’Africains, bien autrement habiles que nos troupes légères, à harceler, à inquiéter, à couper des subsistances. Les Anciens se mettaient-ils en marche ? Ils détachaient à leur avant-garde, c’est-à-dire, à un quart de lieue ordinairement ou à quelques stades tout au plus dans les pays ouverts, ce qu’ils appelaient des coureurs. C’étaient des armés à la légère, tirés des légions et propres à ce service. Cela suffisait, parce que leurs armées peu nombreuses et rangées sur une ordonnance à lignes redoublées, passaient promptement de l’ordre de marche à celui de combat. Étaient-ils dans le cas de faire un détachement ? Ce détachement était composé ou de gens tirés des légions, ou même d’une ou plusieurs légions […].\par
Quand Gustave et Nassau rétablirent l’art militaire en Europe, il ne leur vint pas dans l’idée de créer une espèce de troupes particulières, pour faire la guerre en avant d’eux et pour veiller à la sûreté de leurs armées. Ils se conduisirent comme les Anciens. Ils n’eurent point d’armées nombreuses. Ils eurent peu d’attirails de guerre et d’équipages, par conséquent moins de magasins, moins de convois, des communications moins longues et moins difficiles. Ces principes subsistaient à beaucoup d’égards du temps de Turenne. Ce grand homme préférait de commander de petites armées. Il avait l’excellente maxime de se tenir le plus qu’il pouvait à la portée et à la vue de l’ennemi. Il faisait peu de détachements. Il ne morcelait point son armée. Il la faisait remuer en entier. Aussi ne voit-on pas qu’il ait imaginé de créer des troupes légères. On ne commença à en voir qu’après lui. Alors les armées devinrent prodigieusement plus nombreuses et plus chargées d’embarras. La manière de faire la guerre changea. On chercha, soit pour profiter de cette immensité de troupes, soit pour trouver plus de facilité à la nourrir, à embrasser, par les opérations militaires, une plus grande étendue de pays. On fit beaucoup de détachements, on eut de grosses réserves, des corps particuliers. De là, longues et difficiles communications ; magasins emplacés sur plusieurs points ; nécessité, au milieu de ce morcellement, d’être éclairés au loin pour avoir le temps de se rassembler et d’opposer comme aux échecs mouvement à mouvement et pièce à pièce ; nécessité de couvrir ces longues communications et d’inquiéter celles de l’ennemi. Ces objets firent naître l’idée d’avoir des corps de troupes spécialement destinées à les remplir. Quelques officiers, revenus des guerres de Hongrie, avaient vu les troupes irrégulières turques et hongroises. Ils avaient amené quelques cavaliers de cette dernière nation. Ce fut ce qui donna au maréchal de Luxembourg l’idée de lever en 1692 le premier régiment d’houzards qui ait paru en France. Ce régiment se nommait Mortagni. Ensuite le maréchal de Villars en fit lever un second et l’électeur de Bavière en donna un troisième au roi. Ainsi dans le siècle précédent, le maréchal de Brissac, faisant la guerre en Piémont, avait imaginé les premiers dragons\footnote{Les Espagnols furent les premiers qui imitèrent les Français et bientôt toutes les autres puissances levèrent successivement des dragons. Ces dragons du maréchal de Brissac étaient proprement de l’infanterie à cheval. Ils conservèrent pendant quelque temps le mousquet et la pique. On leur donnait de mauvais chevaux afin que la perte fût moins grande quand ils seraient obligés de les abandonner. Ils ne portaient ni bottes ni éperons et lorsqu’ils mettaient pied à terre pour combattre, ils attachaient leurs chevaux deux à deux.}. Je cite ce qui s’est fait en France, parce qu’alors la France combattait contre l’Europe et que, malgré ses malheurs, dans la guerre de 1700, c’étaient les règlements et les institutions de son militaire, qui donnaient le ton à l’Europe. À ces houzards et dragons se joignit bientôt l’usage des compagnies franches. Louis XIV en entretenait un assez grand nombre. C’étaient des compagnies, levées par des officiers suisses et non avoués par les Cantons, qui faisaient cette sorte de service et l’on voit dans l’histoire de ce temps-là que ces compagnies, peut-être plus utiles que nos corps de troupes légères actuels, faisaient des coups bien plus hardis. Il eût sans doute été heureux qu’on s’en fût tenu là. On s’y tint pendant la guerre de 1733, mais il n’en fut pas de même dans celle de 1740. L’héritière de Charles VI fut obligée de se jeter entre les bras des Hongrois. Alors parurent en Allemagne les peuples de ce royaume, les Transylvains, Croates et autres, milices irrégulières et indisciplinées que la maison d’Autriche n’avait jamais tenté d’appeler dans ses armées, soit par politique, soit parce qu’elle ne s’en sentait pas aimée. Les généraux de Thérèse en disciplinèrent une partie, ils laissèrent l’autre servir suivant son génie et sa coutume. Thérèse remonta sur le trône de ses ancêtres et elle conserva sur pied ses fidèles Hongrois. La guerre suivante ils vinrent pour la première fois en Flandre et sur le Rhin. Les ignorants ne manquèrent pas de dire en France que c’était cette quantité de milice harcelante qui avait détruit nos armées de Bohême et de Bavière, tandis qu’en effet elles avaient bien plutôt été les victimes du climat et de nos fautes. On dit qu’il fallait leur opposer des troupes à peu près semblables. Le maréchal de Saxe fit des uhlans. On leva des régiments qu’on appela de troupes légères. À l’autre bout de l’Europe, le roi de Prusse augmentait aussi dans le même temps ses houzards et ses dragons pour faire face aux arrière-bans de Hongrie. Ainsi se termina la guerre de 1740. Dans celle de 1756 cette augmentation réciproque de troupes légères a été poussée plus loin encore. Car dans toutes nos constitutions sans principes, tout se fait par imitation et par engouement. Telle est enfin aujourd’hui en France la situation des opinions sur cet objet, que beaucoup d’officiers osent avancer que les troupes légères sont les corps les plus importants et les plus utiles d’une armée, qu’il faut les multiplier, les rendre supérieurs en nombre et en bonté, à celles de l’ennemi. Il semble, à les entendre, que ces corps soient l’école de la guerre, que ce ne soient qu’eux qui la fassent, ou la doivent faire. Étrange prévention, que celle qui peut confondre ainsi la pratique de manier quelques troupes, d’éclairer un pays, de faire quelques expéditions hardies, d’engager et de conduire un petit combat, avec la science immense et plus qu’humaine de remuer une armée, de donner une bataille, de créer et de diriger le plan d’une campagne : prévention, dont les suites pourront former quelques bons chefs d’avant-garde, peut-être même quelques bons lieutenants de généraux, mais certainement jamais des hommes du premier genre, comme les Turenne et les Luxembourg.\par
Sans doute il faut qu’une armée s’éclaire, couvre ses communications, harcèle l’ennemi. Mais n’y aurait-il pas un système de guerre, par lequel on rendrait toutes ces opérations moins compliquées ? Ne pourrait-on pas employer à la plus grande partie de ces opérations ce que nous appelons des troupes régulières ? Enfin, en admettant qu’il faille entretenir des corps de troupes, spécialement destinées à les remplir, la constitution qu’on donne à ces corps et particulièrement celle qu’on leur donne en France est-elle la meilleure et la plus avantageuse ? Voilà trois points que je vais examiner.
\subsubsection[{2. Il est possible de créer un système de guerre qui rende les troupes légères peu nécessaires}]{2. Il est possible de créer un système de guerre qui rende les troupes légères peu nécessaires}
\noindent Si les armées étaient moins nombreuses, moins chargées d’embarras, plus sobres, si elles avaient d’autres méthodes de subsistance, certainement elles occuperaient des positions moins étendues. Elles se remueraient plus rapidement. Elles craindraient moins les surprises, les dérobements de marche, les échecs à leurs convois, ou à leurs magasins. Elles seraient moins forcées de se diviser en réserves, en corps détachés ; car ce sont tous ces objets qui imposent la nécessité de ce morcellement, de cette guerre par pièces, qui, étant plus compliquée et remplissant la tête du général de plus de détails, est bien moins décisive que le système de guerre des Anciens. Les armées se tenant plus ensemble, plus rapprochées l’une de l’autre, faisant la guerre en masse au lieu de la faire par détachement, il faudrait moins de précautions, moins de postes, pour être éclairé ou couvert, donc moins d’occasions d’employer les troupes légères. Aujourd’hui ces occasions sont multipliées à l’infini. Partout il faut des troupes légères. Les avant-gardes en sont composées. Il en faut aux réserves, aux détachements. Il en faut sur les communications, dans les postes intermédiaires. Cependant ces troupes légères ainsi répandues et formant le cinquième des armées, ne remplissent point d’objet décisif. L’ennemi s’avance-t-il en force ? Il faut les soutenir, ou elles se replient. Se donne-t-il des combats entre les armées ? Elles n’y prennent point part. Le préjugé même semble les en avoir dispensées.\par
Mais le moyen, me dira-t-on, de reformer la constitution actuelle des armées, quand cette constitution est générale en Europe ? Le moyen par conséquent de changer le système de guerre qui existe ? J’avoue que ce premier changement est impossible dans les circonstances où sont actuellement toutes les nations. Il faudrait, pour l’exécuter, un peuple vigoureux, supérieur aux autres par son gouvernement, par son courage, un peuple qui n’eût pas nos vices et nos fausses lumières. Mais avec nos constitutions, avec nos armées, telles qu’elles sont, le changement du système de guerre ne serait pas de même impossible […].
\subsubsection[{3. Les troupes de ligne peuvent faire avec avantage le service, ou au moins une partie du service confié particulièrement aux troupes légères}]{3. Les troupes de ligne peuvent faire avec avantage le service, ou au moins une partie du service confié particulièrement aux troupes légères}
\noindent S’il est possible de créer un système de guerre qui rende le grand nombre de troupes légères beaucoup moins nécessaire, il l’est encore plus de remplir par des troupes de ligne, les objets aujourd’hui particulièrement confiés à ces premières. Car quelle différence y a-t-il entre l’infanterie d’un bataillon et celle d’un corps de troupes légères ? Ne sont-ce pas des hommes de la même espèce, vêtus de même, armés de même, assujettis à la même discipline ? Cette infanterie des troupes légères a-t-elle seulement reçu une éducation relative à ses fonctions ? Sait-elle nager, courir, supporter la faim, résister plus longtemps aux fatigues ? Ses officiers ont-ils une instruction qui soit le moins du monde analogue à ce qu’ils doivent remplir ? J’en dirai autant de la cavalerie attachée aux corps de troupes légères, par comparaison à la cavalerie de ligne. Enfin non seulement les troupes de ligne peuvent remplir une partie des fonctions assignées aux troupes légères, mais il sera avantageux de les leur faire remplir, non par piquets, non par détachements, comme cela se faisait autrefois en France, comme nous le pratiquions dans les premières campagnes de la guerre dernière, ce qui était la source de nos échecs journaliers et de l’ascendant que l’ennemi avait pris sur nous, ce qui faisait échouer presque toutes nos expéditions ; mais comme le fit M. le maréchal de Broglie en 1760. Ce général forma des bataillons de grenadiers, il fit servir des régiments hors de ligne, il régénéra nos dragons, troupe supérieure par sa composition, troupe vraiment d’élite qui n’attendait qu’un homme qui sût la manier. Il les employa tour à tour, ou à la guerre de détail, ou à la guerre de masse. Il les accoutuma à sortir de ligne pour le service journalier et à y rentrer un jour de combat. Les événements justifièrent la bonté de sa méthode. Un nouvel esprit naquit dans l’armée. On eut des succès. On s’acquitta dans une campagne de huit mille prisonniers qu’on devait à l’ennemi. Après un exemple pareil on pourrait s’épargner de nouvelles discussions : mais poursuivons. Tant de gens ont les yeux fermés à la lumière !\par
En employant ainsi des troupes de ligne aux avant-gardes et aux objets importants et par préférence l’élite de ces troupes, on fait essuyer plus d’échecs à l’ennemi, on en essuie moins, chose bien importante. Car c’est de la supériorité journalière que naissent la vigueur et la confiance qui animent une armée. Les corps, qu’on porte en avant, sont plus solides, moins sujets à être pliés, plus propres à attendre des renforts ou de nouvelles dispositions. Le fond de l’armée s’habitue à voir l’ennemi, s’aguerrit, s’instruit. Si au contraire, comme le veulent bien des gens, on multiplie prodigieusement les troupes légères, si on les emploie journellement à la guerre de détail, l’armée ne fait plus de service extérieur. Elle s’abâtardit dans ses camps. Elle ne voit l’ennemi que les jours de bataille. Ces jours arrivent et alors, chose inconcevable, chose bien digne de cette contradiction perpétuelle qu’on trouve entre la raison et nos principes, ces troupes légères qu’on a aguerries, qu’on a menées sur l’ennemi toute la campagne, se retirent pour laisser décider le sort de l’action, celui de l’État aux troupes de ligne, bien neuves et bien étonnées de tout le spectacle qui s’offre à elles, parce qu’on les a tenues constamment éloignées des occasions de voir et d’agir.
\subsubsection[{4. De la constitution des troupes légères}]{4. De la constitution des troupes légères}
\noindent Je viens de faire voir par quelle manie d’imitation désordonnée et peu réfléchie, le nombre des troupes légères s’est si prodigieusement accru et paraît vouloir s’accroître encore. J’ai démontré qu’il faudrait le diminuer considérablement, que celles d’infanterie surtout sont absolument inutiles, qu’on suppléerait avantageusement aux unes et aux autres, par les troupes de ligne. Examinons maintenant, dans la supposition qu’on en veuille conserver, la constitution qu’il serait nécessaire de leur donner pour en tirer un parti utile.\par
Ce ne serait certainement pas de les former en corps de deux ou trois mille hommes, comme on dit qu’on veut le faire en France. Car outre qu’il n’est pas aisé de trouver des chefs qui soient en état de commander tous les jours des corps aussi nombreux, ces corps ainsi constitués deviennent moins mobiles, moins agissants, moins audacieux. Ils ont la prétention d’être de petites réserves et nous n’avons déjà que trop dans nos armées de ces corps détachés, animés d’un esprit particulier qui n’est presque jamais celui de l’armée, occupés de se conserver bien entiers, bien indépendants, combinant exclusivement pour eux et indifférents aux succès et aux échecs qui ne sont pas les leurs.\par
Je préférerais donc des corps de troupes légères de mille ou douze cents hommes seulement, dont les deux tiers de cavalerie. À quoi un corps de troupes légères est-il destiné ? C’est à faire une course rapide, c’est à découvrir, c’est à harceler, c’est à être ce soir sur un point, demain sur un autre. Or si, comme aujourd’hui, on les compose en plus grande partie d’infanterie ; ou ils sont obligés, pour faire ce genre de guerre, d’abandonner leur infanterie et elle ne leur est qu’embarrassante ; ou ce qu’ils font plus communément, ne voulant pas se morceler, craignant de se compromettre, ils ne hasardent rien et font pesamment l’office de troupes de ligne. Ayant au contraire un tiers d’infanterie seulement, ils peuvent tout entreprendre. Ils peuvent, quand il en sera besoin, porter cette infanterie en croupe : la relayant tour à tour par leurs dragons mis pied à terre, ils auront assez de cette infanterie pour se garder la nuit, pour assurer un défilé, un pont, pour en jeter dans quelques maisons. En cas de nécessité même tout le corps deviendra infanterie, car je suppose qu’il sera exercé à manier tour à tour les deux armes. Faudra-t-il qu’il tienne ferme ? On le fera soutenir par des troupes de ligne. S’engagera-t-il un combat sérieux ? Il prendra rang avec elles et se battra comme elles. Exercé aux mêmes mouvements, il doit savoir se battre en masse comme en détail. Je dois ajouter que ces corps seraient composés de soldats choisis et aguerris. Qu’il n’y serait pour cet effet jamais admis en temps de guerre ni déserteurs, ni gens douteux. À qui en effet confier les têtes d’avant-gardes, les patrouilles, les découvertes, les chaînes qui à la veille d’un moment intéressant, doivent arrêter les transfuges et les émissaires, si ce n’est à ce qu’il y a de plus brave et de plus fidèle dans l’armée ? Le même choix serait fait pour les officiers de ces corps. On en donnerait le commandement à des officiers hardis, intelligents dont le mérite fût connu et dont la fortune ne fût que commencée ; à des hommes qui sussent que de pareils corps sont faits pour se morceler, pour se compromettre, pour être sacrifiés au besoin, enfin pour ne pas craindre d’être battus quand ils remplissent, en l’étant, un objet utile à l’armée.\par
J’ai dit que ces corps seraient habitués à combattre en détail et en masse. On les instruirait en conséquence. On les exercerait de plus à nager, à courir, à tout ce qui peut augmenter l’agilité et la force. On leur ferait faire pendant la paix des exercices simulés de toutes les opérations dont ils peuvent être chargés à la guerre. On montrerait aux officiers et aux bas-officiers comment on fait une patrouille, une reconnaissance, un rapport ; comment on établit un poste à pied, ou à cheval : comment on retranche l’un, comment on assure l’autre par les positions des vedettes et par des patrouilles poussées sur tous les rayons. On montrerait aux officiers comment on fait les dispositions pour surprendre, enlever ou attaquer un poste, pour défendre ou attaquer un village, comment on crénelle des maisons, comment on attache un pétard, instrument dont les troupes légères devraient toujours être pourvues, etc. On leur apprendrait comme on s’oriente dans un pays, comme on prend une idée juste de ce pays vu sous différents aspects, comme il faut s’accoutumer à le voir ainsi, afin de le bien connaître, comme on juge des distances de la force des troupes qu’on aperçoit, de leurs dispositions, de leurs manœuvres. On leur ferait connaître par quelles illusions l’art ou le terrain font paraître des troupes plus ou moins nombreuses ; et en leur fortifiant l’œil contre ces illusions, on leur montrerait à les employer contre l’ennemi. On formerait même une école de stratagèmes et de ruses, ressources tant employées par les Anciens et si inconnues aujourd’hui. On accoutumerait enfin les officiers de ces corps à être vrais dans leurs rapports, à ne pas exagérer le nombre des ennemis qu’ils ont vus et combattus, à ne pas consommer des munitions inutilement et pour se faire croire plus souvent aux prises ; et pour cela il faudrait leur faire entendre, leur bien graver dans l’esprit que tous ces mensonges, malheureusement trop reçus aujourd’hui, peuvent avoir les conséquences les plus fâcheuses ; qu’en écrivant, je suppose au général qu’on a six mille hommes devant soi et qu’on les a combattus, qu’on a vu telle chose dans tel ou tel point, on lui fait faire une fausse combinaison si l’ennemi est moins nombreux, ou si la chose avancée n’est pas exacte ; qu’une fois ces mensonges reconnus en deux ou trois occasions, le général ne sait plus comment démêler la vérité et les attaques véritables d’avec celles sont jouées : qu’en un mot, pour quelques officiers que ces mensonges ont fait un moment valoir, il en est bien plus qu’ils ont flétris, ou auxquels ils n’ont fait qu’une réputation passagère que les grandes occasions ont détruite. J’attaque cet abus, parce qu’il est plus funeste qu’on ne le pense : parce qu’il existe dans toutes les troupes légères de l’Europe, parce que s’il est des corps dans l’armée où il soit essentiel de trouver clairvoyance et vérité, c’est dans ceux qui sont le plus en avant, puisque c’est d’après leur rapport que la masse se meut et se dirige. Les Romains punissaient d’ignominie les sentinelles et les postes avancés qui faisaient un faux signal. César dit dans ses commentaires, qu’il ne se servit plus d’un certain Publius, officier brave, intelligent, parce qu’il s’aperçut qu’ou la vanité ou quelque motif particulier dictaient toujours les comptes qu’il rendait.\par
J’avais écrit ce morceau avant que d’être placé dans un corps de troupes légères. Y servant maintenant, ce n’est pas une raison pour changer de sentiment ni pour le faire. Honte soit à l’écrivain et surtout à l’écrivain militaire, qui vend son opinion aux circonstances ou à la fortune !
\subsection[{V. - Essai sur la tactique de l’artillerie}]{V. - Essai sur la tactique de l’artillerie}
\subsubsection[{1. De l’artillerie en général. Ses avantages trop élevés par les uns, et trop abaissés par les autres. Son utilité réelle}]{1. De l’artillerie en général. Ses avantages trop élevés par les uns, et trop abaissés par les autres. Son utilité réelle}
\noindent L’artillerie est la troisième arme des armées ; ou, pour parler plus juste, elle est un accessoire utile et important à la force des troupes qui composent les armées. Cette distinction entre arme et accessoire paraîtra peut-être un peu sophistiquée. Elle est cependant nécessaire pour donner une idée précise de l’objet de l’artillerie ; car par le mot d’arme on ne peut précisément entendre que l’infanterie ou la cavalerie, qui sont deux mobiles principaux et constituants d’une armée. Tandis que celui d’accessoire convient parfaitement à ces moyens étrangers dont l’imagination humaine a cherché, dans tous les siècles, à augmenter la force des combattants ; moyens qui ne peuvent pas combattre seuls et par eux-mêmes et qui ont varié fréquemment ; puisqu’on a eu successivement des éléphants, des chariots armés de faux, des catapultes, des balistes, des onagres, etc., et enfin, de nos jours, toutes ces grosses armes de jet comprises sous le nom générique d’artillerie.\par
Les machines de guerre des Anciens étaient incommodes et de peu d’effet. Notre artillerie est plus simple, plus ingénieuse, plus facile à mouvoir. Son exécution est plus certaine et plus meurtrière. Quelques militaires ne sont pas de cet avis. Mais comment oser comparer des machines qu’on ne pouvait mettre en jeu qu’à force de vérins, de treuils, de moufles, de cordages, à des armes d’une manœuvre aisée et qui par l’inflammation subite de la poudre, chassent des mobiles plus pesants et plus destructifs ; des machines dont les montants et les bras donnaient tant de prise aux batteries opposées, à des armes que l’on peut rendre presque inaccessibles aux coups de l’ennemi : des machines dont le tir n’était pas horizontal, dont la plus grande étendue de portée était au-dessous de la moyenne portée des nôtres, dont la rectitude de portée était bien plus imparfaite ; des machines qui permettaient qu’une place se défendît plusieurs années et que des tours de charpente d’une élévation prodigieuse subsistassent devant elles plusieurs jours, à des armes qui, tantôt sous des angles de projection élevés, lancent leurs mobiles à des portées inouïes, qui, tantôt sous des angles moins sensibles, chassent ces mobiles horizontalement, battent de but en blanc des terrasses énormes, les détruisent en peu de jours, enfilent des prolongements, les ricochent, empêchent l’ennemi de s’y maintenir et finissent enfin par détruire toutes les places qui ne sont pas délivrées par des secours du dehors, ou par les fautes de ceux qui les assiègent.\par
Qu’on ne conclue pas de là, que la science de l’artillerie soit arrivée au point de perfection où elle peut atteindre. Dimensions des pièces, construction des affûts, effets de la poudre, jet des mobiles, portée de ces mobiles, presque tout sur ces différents objets, est encore système ou erreur. Il y a peu de principes dans cette science qui ne soient contestés. Plusieurs points de première importance sont encore problème et le seront peut-être longtemps. On ignore quels sont les effets de la poudre, jusqu’à quel point elle agit sur les mobiles qu’elle chasse, soit relativement à sa qualité, à sa quantité, à la manière dont elle est employée, aux impressions que l’air fait sur elle ; soit relativement au métal, à la longueur et à l’épaisseur des pièces. On ignore la quantité de force motrice, par laquelle les mobiles sont chassés et la diminution successive de vitesse qu’ils éprouvent par la résistance plus ou moins forte de l’air. La théorie de la balistique est encore plus incertaine. On a cherché en vain jusqu’ici une équation générale qui dans tous les cas, déterminât la courbe décrite par le centre de gravité d’un corps sphérique projeté en l’air, etc. On n’a que des tables approximatives des portées de but en blanc primitif. Là, où le pointement du but en blanc primitif n’a point lieu, il faut le faire par estime et par tâtonnement, ainsi que c’était l’ancienne méthode, ou avec le coin de mire, ou bien par le moyen des hausses et des visières mobiles, nouvelle invention trop compliquée, trop peu solide peut-être et qui exige une théorie pratique et des précautions qu’on ne doit pas attendre du soldat, surtout au milieu du tumulte et du danger d’un combat. On voit qu’il y a loin de tout cela à la perfection de l’art. Il est donc apparent que le temps, que les connaissances mathématiques qui se répandent et font de plus en plus fermenter les esprits chaque jour, produiront des découvertes nouvelles et que ces découvertes amèneront de nouveaux principes. Puisse seulement le gouvernement exciter le génie sur cette importante branche du militaire comme sur toutes les autres et en même temps contenir l’inquiétude des novateurs, ne pas rejeter sans examen et ne pas adopter sans épreuves ! Puissent les épreuves qu’il ordonnera, n’être pas ce que j’ai ouï dire qu’elles étaient trop souvent, des assemblées dont le résultat est connu avant qu’elles ne se tiennent ; soit parce que l’autorité des officiers qui y président, entraîne et couvre toutes les opinions, soit parce que chacun y apporte sa prévention plutôt que son jugement et l’avis qu’il veut conserver plutôt que l’impartialité qui fait qu’on veut voir avant que de juger.\par
Cette digression sur les avantages de l’artillerie et sur les progrès qui lui restent à faire, servira à fixer plus précisément l’opinion qu’on doit avoir de son utilité. Se persuader, comme l’ont fait quelques tacticiens, que l’artillerie est un accessoire plus embarrassant qu’utile, plus bruyant que meurtrier ; en conséquence ne pas parler de l’artillerie, ne la faire entrer pour rien dans les combinaisons de la tactique, c’est une erreur que l’expérience et la raison condamnent. Dire, avec quelques officiers d’artillerie, qui l’ont avancé dans un manuscrit estimable d’ailleurs\footnote{Ce manuscrit est intitulé {\itshape Du service de l’artillerie dans la guerre de campagne} : il est le résultat des conférences tenues par les capitaines du régiment d’Auxonne.}, que l’artillerie est l’âme des armées, que la supériorité d’artillerie doit décider la victoire, c’est une autre erreur qui est ou l’effet d’une prévention de corps, ou celui de l’amour de l’art qu’on cultive. Tel serait l’aveuglement extrême et également déraisonnable de deux hommes qui croiraient, l’un, que tous les mobiles, lancés par les bouches à feu, atteignent leur but, que l’exécution de l’artillerie est certaine et terrible ; et l’autre, que le hasard seul dirige ces mobiles et qu’en conséquence l’effet du canon ne doit être compté pour rien dans la combinaison d’une disposition.\par
Mais qu’importe d’où proviennent les erreurs, dès qu’elles sont réelles ? Trop vanter l’artillerie et trop croire à ses effets, la déprimer trop et faire trop peu de fonds sur elle, ce sont deux extrêmes également préjudiciables. Je vais chercher le juste milieu entre ces extrêmes. Je vais le chercher surtout relativement à la propriété et aux effets de l’artillerie, dans la guerre de campagne, puisque c’est à elle principalement que la tactique a rapport.\par
L’artillerie est aux troupes, ce que sont les flancs aux ouvrages de fortification. Elle est faite pour les appuyer, pour les soutenir, pour prendre des revers et des prolongements sur les lignes qu’elles occupent. Elle doit, dans un ordre de bataille, occuper les saillants, les points qui font contrefort, les parties faibles, ou par le nombre, ou par l’espèce des troupes, ou par la nature du terrain. Elle doit éloigner l’ennemi, le tenir en échec, l’empêcher de déboucher. L’artillerie bien employée relativement à ces différents objets, est un accessoire utile, et un moyen de plus pour l’homme de génie. Donc la tactique de l’artillerie doit être analogue à celle des troupes. Donc il faut que les commandants des troupes connaissent du moins le résultat qu’on peut attendre des différentes dispositions, ou exécutions des bouches à feu, afin de combiner ce résultat dans leur disposition générale.\par
Machines, agents, poudres, mobiles, milieux, circonstances, tout, en un mot, contribue à rendre les portées des bouches à feu incertaines, soit pour la justesse, soit pour l’étendue. Pointez à la portée du but en blanc, une pièce sur un objet isolé qui présente peu de surface. Il faudra peut-être dix, peut être cent coups avant que de toucher cet objet. Je le suppose atteint. Le coup suivant tiré sous le même angle de projection, par les mêmes canonniers, avec la même charge, la même qualité de poudre en apparence, s’écartera plus ou moins sensiblement du même but. Que conclure de cette incertitude ? Que le canon, considéré dans son effet individuel et pointé vers un objet isolé et présentant peu de surface, est une machine peu, ou point du tout redoutable. Mais ce n’est point ainsi qu’on l’emploie dans les combats. Il n’y est pas question d’un point unique. Ce sont des lignes, des masses de troupes. Là, si l’on entend l’usage de l’artillerie, on forme de grosses batteries, on bat non des points déterminés, mais des espaces, des débouches. On fait usage du ricochet. On prend des prolongements. On s’attache uniquement à porter ses mobiles dans le plan vertical de l’ordonnance ennemie. On remplit, non le petit objet de démonter un canon, ou de tuer quelques hommes, mais le grand objet, l’objet décisif, qui doit être de couvrir, de traverser de feux le terrain qu’occupe l’ennemi et celui par lequel l’ennemi voudrait s’avancer. L’artillerie ainsi emplacée, ainsi exécutée, fait beaucoup de mal et encore plus de frayeur.\par
Voilà les effets avantageux qu’on peut se promettre de l’artillerie. Ils deviendront moins décisifs et moins redoutés, à proportion que les troupes seront plus aguerries, mieux ordonnées et plus manœuvrières. Bien aguerries, elles ne s’exagéreront pas le ravage que peut causer l’artillerie ennemie. Elles ne prendront pas la quantité de bruit, pour la quantité de danger. Elles sauront que, pour dix lignes de direction qui peuvent conduire les boulets vers elles, il y en a cent d’aberration où ils ne peuvent leur nuire. Elles sauront la nécessité d’essuyer le feu du canon une fois admis, que si l’on est en panne, ou si l’on combat de pied ferme, la frayeur ne garantit pas ; que si l’on marche pour attaquer, le moyen de faire cesser, ou du moins de diminuer le danger, est d’arriver sur l’ennemi, parce qu’alors l’ennemi s’étonne, chancelle et pointe avec moins de justesse. Bien ordonnées et habilement manœuvrières, elles s’en tiendront, devant le canon, à une ordonnance mince et qui offre à ses coups le moins de prise possible. Si elles sont en colonne, elles sauront promptement quitter cet ordre de profondeur, pour se mettre en bataille par des mouvements simples, rapides, qui ne pourront occasionner ni désordre ni confusion. Elles sauront, au moyen de la discipline et de l’habitude de manœuvres, qu’elles auront contractée, se mettre à l’abri du feu de l’artillerie, par tous les moyens qu’offrira le terrain. Là, si elles sont en panne, mettre devant elles une petite éminence, se couvrir d’un ravin, se rassembler en colonne derrière un rideau, se placer derrière un terrain mou et marécageux où le ricochet ne puisse point faire effet. Ici, rompues en colonnes par division, ou par demi-bataillon, présenter ainsi à l’ennemi, au lieu d’une ligne continue, de minces divisions, avec de grands intervalles, vues par le flanc et offrant seulement trois files au pointement de l’ennemi. D’autres fois elles se mettront ventre à terre, ayant seulement en avant d’elles quelques hommes intelligents pour les avertir de ce qui se passe. Elles ne regarderont pas, ainsi qu’on l’a fait dans un siècle de préjugés et d’ignorance, ces précautions comme déshonorantes ; car la première loi de la guerre est de ne pas exposer le soldat, quand cela n’est pas nécessaire, pour l’exposer ensuite sans ménagement, quand la nécessité l’exige. Si elles doivent marcher à l’ennemi, elles sauront encore profiter de toutes les ressources du terrain, déboucher en colonne par des points qui ne seront pas vus de l’artillerie ennemie, si ces points conduisent très à portée d’elle, ou s’il n’y a point de débouchés pareils, marcher rapidement à l’ennemi, jetant en avant de leur marche des compagnies de chasseurs éparpillées pour attirer son attention, le harceler de coups de fusil et s’attacher principalement aux canonniers des batteries.\par
On voudra bien me pardonner ces détails qui, s’ils n’ont rien que de connu, n’en étaient pas moins nécessaires pour présenter l’ensemble des avantages que les troupes peuvent retirer de leur artillerie, des moyens de la rendre redoutable et de l’art précieux d’empêcher, ou au moins de diminuer les effets de celle de l’ennemi. J’ai tâché de balancer les opinions contraires des partisans indiscrets de l’artillerie et de ses adversaires outrés.
\subsubsection[{2. Système d’artillerie. Parallèle de l’ancien avec le nouveau}]{2. Système d’artillerie. Parallèle de l’ancien avec le nouveau}
\noindent Mon projet n’est point d’entrer ici dans la discussion des sentiments qui partagent aujourd’hui les artilleurs sur les détails intérieurs de leur art, comme proportion des bouches à feu, construction des affûts, théorie des tirs, etc. Je ne suis point assez instruit sur ces matières pour avoir une opinion à moi, et à quoi servent les discussions quand elles ne répandent pas de lumières sur les objets qu’on discute ?\par
S’il y a eu jusqu’ici tant de révolutions dans les systèmes d’artillerie, s’il y a encore de nos jours un partage de sentiments sur une infinité d’objets, c’est que, dans un corps où il y a nécessairement de l’étude et un travail habituel, il faut que les esprits fermentent et s’agitent. Heureuse fermentation ! Ce serait un grand malheur, que les idées fussent stables et uniformes, tant que l’art n’est pas à sa perfection. Ce serait un présage fâcheux qu’il n’y parviendrait jamais.\par
La révolution, qui s’est faite à la paix, a bouleversé l’artillerie encore plus que les autres parties de notre constitution militaire. Ce bouleversement a produit du bien et du mal. C’est le sort commun des opérations humaines. Mais lequel a prévalu ? C’est ce qui mérite un examen réfléchi. Passons les détails : arrêtons-nous à quelques résultats.\par
On a changé la proportion des pièces et la construction des affûts. Un nouveau système d’artillerie de campagne et de siège s’est élevé sur les débris de l’ancien. Ses adversaires prétendent que ces grands changements ont coûté des sommes énormes. Je sais, moi, de science certaine, qu’elles n’ont pas été telles. J’en ai vu les détails. Eh ! l’eussent-elles été, si le nouveau système est meilleur, s’il rend l’artillerie française supérieure à celle de l’ennemi, si par là il influe sur le gain d’une bataille, la dépense est plus que compensée. En politique il n’y a que les erreurs qui coûtent ; les dépenses utiles sont économie.\par
En changeant les proportions et les affûts de l’artillerie de campagne, on l’a considérablement allégée […].\par
Les partisans du nouveau système prétendent que les pièces n’y ont perdu ni du côté de l’étendue, ni du côté de la rectitude de la portée. Ils disent qu’avec l’artillerie qu’on mènera en campagne ils auront des portées proportionnées aux objets et au but de la guerre de campagne. On leur objecte néanmoins qu’en raccourcissant et atténuant les pièces, pour les alléger, on a perdu sur la longueur et la justesse des portées ; que les inconvénients du recul ont prodigieusement augmenté. On regrette les pièces longues et la solidité moins ingénieuse et moins compliquée des anciens affûts. On soutient qu’il ne fallait pas que les affûts de campagne fussent différents des affûts de siège, que c’est une complication de moyens et de dépense, qui privera de la facilité de reverser tour à tour l’artillerie des armées dans les places et celle des places dans les armées. Les épreuves auraient pu faire découvrir le vrai sur quelques-uns de ces objets, par exemple, sur la longueur et la justesse des portées. Mais, comme je l’ai déjà observé, la plupart des épreuves qui se font dans les écoles d’artillerie, ne décident rien et leur résultat est toujours conforme à l’opinion dominante. Les officiers d’artillerie qui ne sont ni de l’un ni de l’autre parti, ceux qui aiment le vrai et le bon, quelles que soient ses livrées, conviennent que l’ancienne artillerie de campagne était trop pesante, que les mouvements de tactique des troupes étant devenus plus rapides et plus savants, il fallait que l’artillerie s’y accommodât ; qu’en conséquence on a bien fait d’alléger les pièces, que leur raccourcissement peut bien leur avoir fait perdre de leur portée, mais qu’au-delà de celle qui leur reste, les coups étaient si incertains, que cette perte, plus apparente que réelle, ne doit point laisser de regrets. Ils avouent qu’elles ont peut-être aussi perdu de leur justesse, mais que cette différence est si peu sensible, qu’elle ne peut donner de désavantage, parce que dans la guerre de campagne, il s’agit de battre de grands espaces et non des points ; et que si par hasard on veut battre des points, comme des retranchements, ou d’autres obstacles, qu’il est à propos de détruire, on rapproche l’artillerie à des distances qui ne permettent plus que l’aberration des mobiles soit sensible. Ils soutiennent que les anciens affûts de place avaient besoin d’être changés, qu’ils étaient trop difficiles à manœuvrer, à dérober au feu de l’ennemi et à réparer au milieu des embarras d’un siège. Jusque-là tout serait bien dans les changements qui ont été faits. Mais ils blâment les masses énormes et maladroites, substituées à ces derniers affûts. Ils regrettent qu’on paraisse vouloir renoncer, pour la guerre de campagne, aux pièces de 16. Ils demandent avec quoi on battra des maisons, des abattis, des retranchements tant soit peu épais et tels que la main des hommes peut en quatre jours en élever en rase campagne. Ils se plaignent de la trop grande quantité de pièces de 4 qu’on se propose d’attacher, soit aux régiments, soit au parc. Ils proposent un plus grand nombre et un usage plus fréquent des obusiers. Ils blâment la complication d’avoir deux espèces de cartouche à balle, l’invention ingénieuse et compliquée des visières mobiles, celle des vis de pointage, et quelques autres détails, soit dans les affûts, soit dans la manœuvre des pièces, qu’il serait trop long de rapporter ici. Après tout, ils approuvent plus qu’ils ne blâment, en convenant tous, que le génie de l’auteur du nouveau système\phantomsection
\label{footnote8}\footnote{M. de Gribeauval. C’est le même qui s’est fait tant d’honneur par sa défense de Schweidnitz, étant alors au service de l’impératrice reine. Ce serait une histoire bien intéressante et bien instructive que celle de ce siège publiée par lui-même. On y reconnaîtrait toute l’opiniâtreté et toute l’habileté que lit paraître autrefois. M. de Chamilly à Grave ; et bien plus de génie encore dans les moyens de défense, outre cette différence sensible que M. de Chamilly commandait à sa nation, au lieu que M. de Gribeauval était au milieu d’une nation étrangère, qu’il ne commandait pas dans la place, et qu’il dut s’acquérir peu à peu l’autorité et la prépondérance par sa conduite et ses lumières.} est digne de sa fortune.
\subsubsection[{3. Inconvénients d’une artillerie trop nombreuse}]{3. Inconvénients d’une artillerie trop nombreuse}
\noindent Ce n’est pas moi qui ai parlé jusqu’ici. Je n’ai fait qu’exposer les opinions établies. Oserai-je maintenant m’élever contre un abus épidémique venu du nord de l’Europe et adopté dans le nouveau système, sans doute parce qu’on a cru ne pouvoir se dispenser d’imiter trois grandes puissances qui nous ont donné l’exemple sur ce point ? Je veux parler de l’immense quantité d’artillerie : abus que nous tenons de la Russie, de la Prusse et de l’Autriche.\par
Combien l’histoire de tous les siècles se ressemble ! et qu’il est étonnant que cette similitude d’événements n’instruise pas les hommes. Dans la haute antiquité on n’eut d’abord que quelques chariots armés en guerre, pour garnir les ailes et pour entamer le combat […].\par
II en fut de même pour les machines de jet qui succédèrent à l’usage des chariots armés. Les Romains aguerris et disciplinés, pour tout dire en un mot, les Romains de la république n’en avaient point à la suite de leurs légions. Peu à peu on en eut quelques-unes pour battre les retranchements, pour occuper les points principaux dans les ordres de bataille. Cette petite quantité relative et suffisante à l’objet proposé, pouvait être regardée comme un progrès de l’art militaire. On en accrût successivement le nombre. La tactique déchut : les courages dégénérèrent […].\par
Suivons l’histoire de nos siècles. Nous y verrons pareillement les nations placer leur confiance dans la quantité de leur artillerie, en raison de la diminution du courage et de l’ignorance des vrais principes de la guerre. Les Suisses qui humilièrent la maison de Bourgogne, ces Suisses dont François I\textsuperscript{er} et Charles V se disputaient l’alliance, dédaignaient le canon. Ils se seraient crus déshonorés de s’en servir. C’était une étrange prévention, effet de leur ignorance, qui causa leur défaite à Marignan. Encore cet excès valait-il mieux que celui où l’on a donné depuis. Il supposait du courage et celui, dans lequel nous sommes tombés, ne fait honneur, ni à notre courage, ni à nos lumières […].\par
Les Autrichiens eurent à l’instar des Russes dans la guerre dernière une artillerie nombreuse et formidable. Ils firent la guerre relativement à cette quantité. Ils tachèrent de réduire tous leurs combats à des affaires de poste. On ne vit de leur côté ni les grands mouvements, ni les marches forcées, ni la supériorité de manœuvre.\par
Le roi de Prusse, dira-t-on, n’avait-il pas aussi une artillerie immense ? Sans doute ; mais outre qu’il en eut moins que les Autrichiens, elle était emplacée, ou en réserve, dans ses villes de guerre plutôt que dans ses armées. C’était de là qu’il la tirait pour réparer ses désastres. C’était de là qu’il en faisait arriver des renforts sur ses positions défensives. Sa tactique en diminua l’embarras. Il sut la perdre et la remplacer […].\par
Parlons de nous. À l’époque de la paix de 1762, la quantité prodigieuse d’artillerie introduite dans les armées des autres puissances, l’influence qu’on supposa qu’elle avait eue dans les combats, firent juger nécessaire de changer entièrement la constitution de notre artillerie. On reprochait particulièrement à nos pièces de campagne d’être trop pesantes, trop difficiles à manœuvrer. J’ai rendu compte des mesures qu’on a prises pour les alléger. Mais cet objet rempli, pourquoi vouloir mener à la guerre un plus grand nombre de bouches à feu ? Ne sera-ce pas perdre l’avantage qu’on a voulu acquérir et changer les embarras de l’espèce de l’artillerie contre ceux de la quantité ?\par
Je ne vois pas, sans frémir, les dispositions de notre nouveau système d’artillerie relativement à la formation de l’équipage de campagne d’une armée. Il est réglé que chaque bataillon aura à sa suite deux pièces de canon de 4 et qu’indépendamment de cela, le parc de l’artillerie sera composé sur le pied de deux pièces de canon par bataillon ; donc une armée de 100 bataillons traînera à sa suite 400 pièces de canon. Ces 400 pièces de canon exigeront 2 000 voitures pour le transport des munitions, outils, effets de rechange, pontons, et autres attirails nécessaires. Voilà 2 400 attelages faisant au moins 9 600 chevaux. Voilà 2 000 et tant de charretiers conducteurs, gardes d’artillerie, capitaines de charrois, etc. […].\par
Si l’artillerie s’augmente si prodigieusement dans les armées, elle s’accroîtra de même partout. Partout on mettra en elle sa confiance unique. On n’attaquera plus, on ne défendra plus les places que par le canon. On ne croira plus ses côtés en sûreté, que quand elles seront couvertes de batteries. Il en sera sur mer comme sur terre. Les vaisseaux ne se joindront plus, ils ne se battront que par leur artillerie […].\par
Quel fruit retirera-t-on de cette énorme quantité d’artillerie ? Si l’ennemi en a dans la même proportion, voilà de part et d’autre les armées difficiles à mouvoir et à nourrir. Voilà toutes les actions de guerre réduites à des affaires de poste et d’artillerie ; les marches, à quelques transports lourds et rares d’une position à une autre position peu éloignée ; toutes les opérations subordonnées à des calculs de subsistance. Dès lors plus rien de grand, plus de science militaire. Si l’un des adversaires, plus habile, ose s’écarter de l’opinion reçue et n’avoir que cent cinquante pièces de canon avec une armée égale de cent bataillons, tous les avantages seront de son côté […].\par
Les opérations de sa campagne seront calculées d’après la constitution de son armée à cet égard et d’après celle de l’ennemi. Il fera vis-à-vis de lui une guerre de mouvement, il le désolera par des marches forcées, auxquelles l’ennemi sera contraint d’opposer des contremarches qui seront lentes, destructives pour les attirails prodigieux et attelés avec économie qu’il traînera à sa suite, ou bien qui l’obligeront à laisser en arrière la plus grande partie de ces embarras. Alors ils seront à armes égales et il aura pour lui la perfection et la supériorité de manœuvre des siennes. Enfin fût-il obligé d’attaquer l’ennemi, ou de recevoir son attaque, il ne se croira pas battu, parce qu’il aura moins de canons à lui opposer. Ses batteries mieux disposées, mieux emplacées, mieux exécutées, des pièces d’un calibre plus décisif, des prolongements plus habilement pris, lui donneront encore l’avantage. Eh ! quelles batailles ont été perdues parce que l’artillerie a manqué à l’armée vaincue ? Je vois partout que peu de pièces ont agi et que beaucoup sont restées dans l’inaction, ou faute d’emplacement, ou faute de pouvoir atteindre à l’objet, ou faute de savoir les porter rapidement au point d’attaque.\par
Je serre, je presse mes idées. C’est ainsi qu’il faut présenter des doutes. S’ils contiennent des vérités, on en dit assez pour les faire apercevoir. S’ils n’en contiennent pas, on épargne au lecteur l’ennui d’une erreur pesamment détaillée. Tel est en deux mots le résumé de ce que j’ai avancé ci-dessus : diminuer la quantité d’artillerie et faire consister la perfection de l’art à tirer un grand parti d’un petit nombre de pièces, à former la meilleure artillerie possible plutôt qu’à se procurer la plus nombreuse.\par
Je vais maintenant parler de la tactique de l’artillerie, car il en existe une pour l’artillerie comme pour les troupes, une qui tient à celle des troupes, qui doit être calculée sur elle et qui, à beaucoup d’égards, peut lui être rendue analogue. Cette tactique se divise naturellement en deux parties, mouvement, exécution.
\subsubsection[{4. Mouvements de l’artillerie}]{4. Mouvements de l’artillerie}
\noindent La science des mouvements de l’artillerie embrasse toutes les dispositions par lesquelles l’artillerie peut, dans un ordre de marche, marcher avec les troupes et ensuite, dans un ordre de bataille, se mettre en position d’appuyer ces troupes par son feu.\par
Les mouvements des troupes doivent absolument régler ceux de l’artillerie. J’ai tâché de donner aux premiers toute la simplicité et la rapidité dont ils sont susceptibles. II faut que l’artillerie s’y conforme, autant que la différence de ses moyens le lui permet […].
\subsubsection[{5. Exécution de l’artillerie}]{5. Exécution de l’artillerie}
\noindent J’ai pu proposer mes idées particulières sur la partie que je viens de traiter. Les manœuvres de l’artillerie tiennent à celles des troupes, elles doivent en dériver. Ayant donc tâché de perfectionner les mouvements des troupes, j’ai été conduit nécessairement à parler de ceux de l’artillerie. Il n’en est pas de même de l’exécution des bouches à feu. Elle est proprement du ressort des officiers d’artillerie. Ce sont eux par conséquent qui doivent donner des leçons sur cet objet. Ce sont eux qui m’en ont donné et c’est presque toujours d’après eux que je vais parler dans ce que je dirai de cette branche de la science de l’artillerie.\par
Ce que j’appelle exécution de l’artillerie, c’est non seulement l’art de se servir des bouches à feu et de calculer leurs effets, c’est encore celui de les emplacer et de diriger leurs coups, de manière que le résultat de ces attentions combinées soit, en faisant le plus de mal possible à l’ennemi, de donner la plus grande protection possible aux troupes pour lesquelles elles agissent. Les troupes et l’artillerie étant unies ensemble par une protection réciproque, il faut que, pour tirer le parti le plus utile des machines qui sont sous sa conduite, l’officier d’artillerie connaisse la tactique des troupes, sinon les détails intérieurs de cette tactique, au moins le résultat des principaux mouvements, les changements qu’ils apportent dans l’ordonnance des troupes, le dommage ou l’appui que les troupes, dans telle ou telle occasion, peuvent recevoir de l’artillerie exécutée ou emplacée de telle ou telle manière. Il faut pareillement et à plus forte raison, que l’officier d’infanterie et de cavalerie, lui qui, commandant les armes, commande nécessairement l’artillerie qui n’est qu’un accessoire des armes : il faut, dis-je, que cet officier connaisse, sinon les détails intérieurs de construction, d’attirail et d’exécution de l’artillerie, au moins le résultat de tous ces détails, les portées des différentes bouches à feu, emplacées ou exécutées de telle ou telle manière, le dommage ou l’appui que les troupes peuvent en recevoir. Faute de ces connaissances, ou il ne saura pas employer l’artillerie, avec intelligence, dans sa disposition générale : ou il sera obligé de s’en rapporter aveuglément, pour toutes les manœuvres de cette artillerie à un officier de ce corps, qui peut-être à son tour, faute d’avoir porté ses vues au-delà de la conduite mécanique de son canon, ne le disposera pas de manière à remplir l’objet général : ou enfin il contrariera, par ignorance, les dispositions de cet officier d’artillerie qui peut-être en aurait fait de bonnes.\par
L’artillerie pourra, j’espère, par la lecture de mon ouvrage, se faire une idée nette et précise de la tactique des troupes. Faisons connaître aux troupes les effets de l’artillerie suivant les différentes manières dont elle peut être disposée et exécutée […].\par
Il suffit à l’officier commandant les armes de savoir qu’il peut, en telle ou telle position, demander à l’officier d’artillerie de lui procurer des feux qui remplissent tel ou tel objet.\par
Mais une chose dont, pour cet effet, l’officier commandant les armes doit avoir l’intelligence, comme l’officier d’artillerie, c’est l’art de choisir les emplacements, de disposer les pièces, de diriger les feux, de les ménager […].\par
La disposition la plus avantageuse de l’artillerie, considérée soit du côté de l’emplacement, soit du côté de l’exécution, est sans contredit celle qui rend ses effets les plus meurtriers et les plus nuisibles à l’ennemi.\par
Les coups les plus meurtriers étant indubitablement ceux qui parcourent la plus grande longueur sur le terrain occupé par les troupes ennemies, il est certain que leur effet augmentera à mesure que ces troupes seront rangées sur une plus grande profondeur, puisqu’alors le boulet ne cessera de détruire que quand il aura perdu sa force et que quand même il n’aurait pas touche les premiers rangs, il aura son effet de plongée ou de ricochet sur les derniers.\par
Pour obvier à ce prodigieux et meurtrier effet de l’artillerie, toutes les troupes de l’Europe ont abandonné l’ordonnance de profondeur, pour prendre avec raison un ordre plus mince et qui donne moins de prise aux tirs du canon […].\par
Afin de tâcher de faire parcourir à ces mobiles la trajectoire, sur laquelle ils peuvent rencontrer le plus d’ennemis, seule manière de remédier à l’irrégularité et au hasard des portées, l’artillerie doit donc chercher à prendre des prolongements, des revers et des ricochets sur la troupe qu’elle veut battre […].\par
Il faut donc, toutes les fois que cela est praticable, ne pas placer ses batteries vis-à-vis des points que l’on veut battre, à moins que, dans le cas où l’on ne pourrait pas s’approcher assez, l’oblicité ne fit trop perdre sur la longueur de la portée Si l’on doit battre plusieurs points à la fois, comme cela arrive ordinairement quand on dispose des batteries vis-à-vis une ligne de troupes, il faut les placer de manière que les coups de l’une aillent frapper vis-à-vis de l’autre. Ces batteries, qu’on nomme croisées, se protègent et se défendent réciproquement.\par
Indépendamment de la protection mutuelle que les batteries doivent tâcher de se donner, il faut les faire fortes. Alors elles procurent des effets décisifs. Elles font trouée, elles préparent la victoire. Au contraire la même quantité de pièces dispersée est plus propre à irriter l’ennemi qu’à le détruire. L’objet de l’artillerie enfin ne doit point être de tuer des hommes sur la totalité du front de l’ennemi. Il doit être de renverser, de détruire les parties de ce front, soit vers les points où il peut venir attaquer le plus avantageusement, soit vers ceux où il peut être attaqué avec le plus d’avantage.\par
Il ne s’ensuit pas de la maxime posée ci-dessus, qu’on doive réunir trop d’artillerie dans une seule et même batterie. Ce serait tomber dans un autre inconvénient : celui de donner trop de prise à l’ennemi. Il convient seulement de réunir, sur le même objet, plusieurs batteries peu distantes l’une de l’autre, et il faut y joindre l’attention, si le terrain le permet, de ne pas placer ses batteries sur la même ligne, afin que, si l’ennemi peut se ménager des prolongements sur elles, ces prolongements ne traversent pas toutes les batteries à la fois […].\par
Ce n’est point le bruit qui tue. Comme l’incertitude des portées augmente en raison de l’éloignement des points qu’on veut battre, ou du peu d’attention que l’on donne au pointage, il faut s’attacher à pointer avec exactitude, plutôt qu’à tirer avec vitesse. Il faut pointer surtout avec beaucoup d’attention, quand les portées sont éloignées et augmenter la vivacité de son feu progressivement à la diminution des distances, parce qu’en proportion de cette diminution les coups s’assurent toujours davantage.\par
Ce principe n’est pas assez connu des troupes. Leur grand grief contre l’artillerie est toujours qu’elle ne fait pas assez de feu. La mesure de leur contenance dans une canonnade semble être la quantité de bruit que font les batteries qui les soutiennent. Faute de connaissances, les officiers supérieurs eux-mêmes entretiennent ce préjugé. Ils sont les premiers à se plaindre de ce que le canon ne tire pas sans relâche. Qu’arrive-t-il de là ? C’est que souvent l’officier d’artillerie se laisse entraîner à ces clameurs, perd de vue le principe exposé ci-dessus, tire trop vite et à des portées trop incertaines, fait peu de mal à l’ennemi, le rend par là plus audacieux, consomme inutilement des munitions et finit par s’en trouver dépourvu, dans le moment où son feu aurait besoin de devenir le plus vif […].\par
On ne doit pas abandonner mal à propos l’artillerie, ni craindre mal à propos de la perdre. Cette maxime est si importante, si faussement entendue, si peu mise en pratique, qu’elle a besoin d’être développée. Il faut que les troupes contractent l’habitude de ne pas abandonner trop légèrement le canon et qu’elles attachent une sorte de point d’honneur à ne pas le perdre, parce qu’alors l’artillerie ayant confiance dans les troupes qui la soutiennent, se comportera avec plus de vigueur et se croira en quelque sorte obligée, par reconnaissance, à se comporter ainsi. Il faut que l’artillerie de son côté s’accoutume à manœuvrer avec hardiesse, à se hasarder et à se soutenir dans des emplacements avancés, à ne pas regarder si on la soutient, quand ses effets sont décisifs et meurtriers, à n’abandonner ses pièces que quand l’ennemi est, pour ainsi dire, dans sa batterie, puisque c’est l’exécution de ses dernières décharges, qui est la plus terrible. Il faut qu’elle attache son point d’honneur, non à conserver ses machines qui ne sont au bout du compte que des engins faciles à remplacer, mais à les faire jouer le plus efficacement et le plus longtemps possible. Si ces pièces sont prises, ce n’était pas aux soldats d’artillerie, qui n’en sont que les agents à les défendre, c’est aux troupes à les reprendre, ou dans une autre occasion, à remplacer leur perte. En un mot, c’est à l’officier général qui commande, à cet homme qui doit tout voir de sang-froid et sans erreur, de se servir des préjugés des troupes, de ceux de l’artillerie, de son autorité enfin, pour, suivant les circonstances, exposer le canon, le sacrifier ou le conserver. C’est à lui de calculer qu’en telle occasion il faut ramener le canon, soit pour aller prendre ailleurs une position meilleure, soit pour que le soldat découragé ne prenne pas la retraite pour une fuite ; qu’en telle occasion, il faut l’exposer pour qu’il nuise plus longtemps et plus efficacement à l’ennemi ; qu’en telle autre aussi il faut le laisser prendre ; parce qu’il en coûterait trop de sang, ou un temps trop précieux pour le défendre ; et parce qu’après tout, à la guerre, il n’y a pas de honte à faire ce qu’il est impossible d’éviter.\par
Me voici à la fin de mon essai sur la tactique de l’artillerie. Il me reste à dire dans quelles sources j’ai puisé mes connaissances sur cet objet. C’est dans le corps de l’artillerie. C’est dans d’excellents mémoires manuscrits, faits par des officiers de ce corps. C’est en approfondissant avec ces officiers les principes de leur art. Partout mon but est le même et ce sont les lumières d’autrui, bien plus souvent que mes opinions, que je cherche à répandre […].
\section[{Seconde partie. Grande tactique}]{Seconde partie. Grande tactique}\renewcommand{\leftmark}{Seconde partie. Grande tactique}

\subsection[{Avant-propos}]{Avant-propos}
\noindent J’ai essayé, dans la partie précédente, de tracer les principes sur lesquels doivent être constitués et instruits les différents corps destinés à composer une armée. Ici, la carrière s’ouvre et s’étend. Il s’agit de rassembler ces corps, de les amalgamer, de les faire concourir à l’exécution des grandes manœuvres de la guerre. C’est l’art d’enseigner cette exécution, de la combiner, de la diriger, qu’on appelle grande tactique. C’est cette grande tactique qui est proprement la science des généraux, puisqu’elle est le résumé et la combinaison de toutes les connaissances militaires ; puisque, par général, on doit entendre un homme qui les possède toutes, qui est de toutes les armes, qui sait les conduire toutes, soit en particulier, soit réunies. Quels mots, que ceux de général et d’armée ! Et pour peu qu’on les médite, quelle immensité d’idées ils présentent à l’imagination !\par
Que la plume tombe des mains du philosophe qui cherche à régler les devoirs de l’homme, à peser ses préjugés, à déterminer ses opinions : cela doit être. Il doit, s’il est vertueux, frémir des conséquences que peut avoir son ouvrage. Il doit prévoir qu’il peut se tromper, qu’il ne mettra souvent que des sophismes à la place des ténèbres, et des vérités funestes à celle des erreurs utiles. Mais moi j’écris sur mon art, sur un art malheureux, devenu nécessaire et important à perfectionner. Je n’écris point d’imagination, je ne fais point de systèmes, j’examine ceux qui existent. Je mets en ordre ce que j’ai médité. Je crois pouvoir démontrer les principes de la grande tactique, comme j’ai démontré ceux de la tactique élémentaire. J’entreprends de l’exécuter. Tout homme a le droit de publier ses réflexions sur l’art qu’il cultive. Tel est l’avantage des arts et de toutes les sciences exactes, c’est que leurs progrès naissent de la discussion et même des erreurs ; tandis qu’en fait de morale, de métaphysique, et de toutes les sciences d’opinion, les écrits ne font qu’augmenter les doutes et l’ignorance.\par
Ainsi que la tactique élémentaire a pour objet de mouvoir un régiment dans toutes les circonstances que la guerre peut offrir, de même la grande tactique a celui de mouvoir une armée d’après toutes les données possibles.\par
Cet essai n’étant que l’esquisse d’un ouvrage plus complet et plus didactique, je n’entrerai pas dans les détails de principes et de méthode auxquels ce dernier m’assujettira. Je demande seulement qu’avant que de lire cette seconde partie, on se donne la peine de lire avec soin celle qui la précède. Car cette dernière sert de base à l’autre, et je veux, autant que je le pourrai, éviter les répétitions.\par
Marcher ou combattre : c’est à l’un, ou à l’autre de ces objets qu’ont rapport tous les mouvements d’une armée. Je vais donc commencer par poser les principes sur lesquels est fondée la théorie des marches et des ordres de bataille. Cette théorie déduite, je rassemblerai les différentes armes, j’en formerai une armée, je ferai exécuter à cette armée toutes les combinaisons possibles de marche et de combat. C’est-à-dire qu’après avoir établi les grandes et principales règles de la théorie, j’en développerai la pratique, en traçant, à cet effet, le plan en forme de journal d’un camp d’instruction, qui sera l’école de la grande tactique et de toutes les opérations relatives à la guerre de campagne.
\subsection[{I. - Marches d’armée}]{I. - Marches d’armée}
\noindent Par marche d’armée, il faut entendre tous les mouvements quelconques que peut faire une armée. La chose envisagée sous ce vaste point de vue, elle devient une des plus grandes et des plus importantes parties de la science militaire. C’est par les marches qu’une armée agit, se transporte d’une position à une autre, envahit, ou couvre de grandes étendues de pays. C’est par les marches qu’elle surprend l’ennemi, qu’elle le prévient dans un point intéressant. Ce sont les marches qui la conduisent à la formation de tous les ordres de bataille et de toutes les dispositions offensives.\par
À mesure que la science de la guerre se perfectionne, en proportion de ce que les armées sont commandées par des généraux plus habiles, les marches deviennent plus importantes à bien combiner et à bien exécuter, plus fréquentes, plus décisives. Elles deviennent plus décisives en ce qu’elles ont toujours alors un objet prochain, ou éloigné, comme de faire une diversion et de porter la guerre sur un point inattendu : ou de conduire à une action offensive, ou d’engager l’ennemi à un contre-mouvement qui le mette en prise, soit en tout, soit en partie. Elles deviennent plus fréquentes en ce que l’homme de génie peut rarement rester dans l’inaction. Son esprit aperçoit plus d’objets, embrasse plus de combinaisons et là, par conséquent, où le général médiocre ne voit que sa position à garder, ou l’impossibilité d’agir, il se présente, à l’imagination de ce premier, un mouvement avantageux qu’il exécute. Elles deviennent plus importantes à bien combiner et à bien exécuter, parce que leur succès dépend de leur combinaison et de leur exécution, tant dans l’ensemble que dans les détails : parce que des fautes dans leur combinaison ou dans l’exécution, soit générale, soit intérieure, peuvent être adroitement saisies par l’ennemi, faire manquer ce succès et mettre l’armée en prise. Je vais faire voir, par des exemples, la vérité de ce que j’ai dit ci-dessus et comment au contraire les marches sont à peine des opérations dans des armées ignorantes et mal commandées.\par
Chez les Romains (car, parmi les peuples de l’antiquité, il faut chercher celui dont l’histoire militaire est la moins douteuse) jusqu’au milieu de la seconde guerre punique, la science des marches n’était pas connue. J’entends particulièrement la science des marches-manœuvres, terme que je fais parce qu’il exprime mon idée. Une armée sortait de Rome, allait au-devant de l’ennemi, marchant sur une seule colonne et suivant le chemin qui conduisait vers lui. L’ennemi en faisait autant de son côté. Les deux armées se rencontraient, se mettaient en ordre de bataille, s’attaquaient, ou bien se campaient l’une vis-à-vis de l’autre. Là, pendant quelques jours on se harcelait, on cherchait mutuellement à s’attirer dans un champ de bataille désavantageux entre les deux camps. Enfin le combat s’engageait. Vainqueur on assiégeait la capitale, ou une des principales villes. Si on ne pouvait s’emparer, on ravageait le pays et on se retirait. L’année suivante les armées se rassemblaient de nouveau pour recommencer des hostilités dans le même genre. Telles furent les guerres de Rome avec les Samnites, les Fidénates, les Volsques et tous les peuples du Latium. Telles furent celles de tous ces petits états de la Grèce, sur les événements militaires desquels l’histoire a jeté trop de merveilleux et de célébrité.\par
Ce ne fut que dans les guerres puniques que les armées commencèrent à faire la guerre avec plus de méthode et de combinaison. Amilcar, le père du fameux Annibal, fut particulièrement celui qui imagina le premier de mettre un certain ordre dans les marches, de diviser son armée, de la mouvoir sur plusieurs colonnes, afin que la marche fût plus prompte et que l’ordre de bataille fût plus rapidement formé. Annibal ajouta à ce qu’avait imaginé son père et ce furent ces ennemis redoutables qui, à force de vaincre les Romains, leur apprirent la science des marches : comme Pyrrhus, en les battant, leur enseigna à camper, à se retrancher, à perfectionner leur ordonnance. Pourquoi appela-t-on si justement Fabius le bouclier des Romains ? Ce fut à cause de cette campagne de marches et de mouvements qu’il fit vis-à-vis d’Annibal. Ce genre de guerre leur parut si nouveau, que, quoiqu’il sauvât la patrie, ils blâmaient cette défensive dont ils ne connaissaient pas la sublimité.\par
Rapprochons-nous de nos siècles, nous y verrons de même que ce n’est que quand l’art de la guerre s’est perfectionné, que les armées ont commencé à marcher avec quelque combinaison. Nous y verrons, ce que j’ai avancé ci-dessus, que les généraux ont toujours fait plus d’usage de la guerre de marches et de mouvements, en raison de ce qu’ils ont été plus habiles et qu’ils ont eu devant eux des ennemis plus éclairés. Jusqu’à l’époque de Gustave et de Nassau, qui furent les restaurateurs de l’art militaire en Europe, il n’y avait dans les armées, ni mouvements, ni marches combinées. On se joignait, on se battait. Il se faisait peut-être plus d’actions personnelles, plus d’actions d’héroïsme. Mais il n’y avait ni plans de campagne, ni vues, ni projets à plusieurs branches. Qu’on lise toutes les guerres entre l’Angleterre et la France, celles de Charles V et de François I\textsuperscript{er}, les deux princes de l’Europe qui étaient les plus puissants alors, et qui avaient les meilleures troupes. Qu’on lise le récit des batailles de Bouvines, de Poitiers, de Crécy, d’Azincourt, celui des Croisades, on verra comment se remuaient les armées de ces temps, comment elles combattaient, quels étaient leurs ordres de marche. Ils se faisaient sur une seule colonne, les armées étant partagées en trois corps, dont la tête s’appelait l’avant-garde, le centre le corps de bataille, la queue l’arrière-garde. Fallait-il se mettre en ordre de bataille ? Un jour ne suffisait pas pour débrouiller cette masse et pour former la disposition de combat. Le plus souvent l’avant-garde, composée de gens de trait, des enfants perdus et quelquefois de gendarmerie, engageait l’action, tandis que le corps de bataille, composé de la gendarmerie et de la noblesse, s’avançait pour le soutenir. La méprisable infanterie des communes arrivait ensuite, ou plutôt n’arrivait qu’après le combat et pour fuir ou pour piller. Telle est à peu près encore aujourd’hui la disposition des armées ottomanes restées, heureusement pour l’Europe, dans l’ignorance où l’on était alors.\par
Sous Gustave et sous Nassau, on commença à s’éclairer sur les ordres de marche, on en sentit la conséquence. Gustave en exécuta quelques-uns sur plusieurs colonnes et il faut lire, dans son histoire, quel ordre et quelles précautions il y recommandait à ses troupes. Le duc de Rohan, dans son {\itshape Parfait Capitaine}, conseille aussi de les exécuter sur plusieurs colonnes, pour rendre, dit-il, les mouvements moins fatigants et plus prompts. Il s’en fallait bien cependant que cette multiplication de colonnes fut nécessaire alors, comme aujourd’hui que les armées sont plus nombreuses et que l’ordonnance étant plus mince, leur front est plus étendu. Mais ce n’est pas sur les détails intérieurs, sur le mécanisme des marches, que se firent les plus grands progrès. Ce qu’il faut remarquer et méditer dans la conduite de Gustave et des grands généraux de son siècle, c’est la conduite de leurs campagnes, la hardiesse de leurs expéditions, le parti qu’ils savaient tirer de leurs petites armées, la grandeur de leurs projets, la rapidité avec laquelle ils portaient la guerre d’une province à l’autre. C’est ce nouveau genre de guerre, plus en mouvements et en science qu’en combats, dont ils furent les créateurs. Gustave, et après lui ses généraux, se soutenant en Allemagne avec une poignée de Suédois, rappellent Annibal au milieu de l’Italie.\par
Mais rapprochons-nous encore plus de notre temps. Voyons Turenne, voyons Montecucculi, les deux derniers grands hommes qui commandèrent et préférèrent de commander de petites armées. Quel fut leur genre de guerre ? Celui dont je viens de parler. Comment se passa cette fameuse campagne qui termina la vie de l’un et la carrière militaire de l’autre ? En marches et en contremarches, les deux armées étant sans cesse en mouvement, se côtoyant, se tenant sans cesse en mesure de s’attaquer et cela dans un espace de pays de dix ou douze lieues de long sur quatre ou cinq de large, dans un pays couvert et coupé, où des généraux médiocres ne manqueraient pas de faire une guerre de positions. J’aurai occasion de revenir, dans la suite de cet ouvrage, sur la différence bien grande d’une défensive de positions, à une défensive de mouvements. Ce sont les marches qui sont l’objet de mon examen actuel.\par
Après la mort de M. de Turenne, il n’y eut plus de petites armées, chargées de grandes opérations. L’ambition de Louis XIV, voulant envahir à la fois plusieurs pays, il avait déjà commencé quelque temps auparavant, dans la guerre de Hollande, à former plusieurs corps d’armée. Cela ne fit dès lors qu’augmenter et toute l’Europe à l’envi leva des armées plus nombreuses. Avec le nombre des troupes on accrût celui de l’artillerie. Il fallut des équipages de vivres proportionnés. Il aurait été nécessaire qu’en raison de ces accroissements énormes d’hommes et d’embarras, la tactique fît des progrès, qu’elle en fît particulièrement sur la partie des marches. Elle n’en fît pas. Des généraux médiocres se trouvèrent chargés de plus grandes masses et alors le genre de guerre changea. Ne pouvant et ne sachant pas les remuer, étant la plupart du temps embarrassés de les nourrir, ils firent moins de marches, ils renoncèrent à la guerre de mouvements. Ils introduisirent celle de positions. Se trouvèrent-ils inférieurs ? Ils s’enfermèrent dans des lignes, dans des camps retranchés. En un mot, il ne se fit plus rien de hardi, rien de décisif, on ne fit plus ce que j’appelle la Grande Guerre.\par
Au milieu de cette quantité de généraux qui ont commandé les armées françaises depuis cette époque, s’il en a paru quelques-uns de plus heureux, c’est parce qu’ils se sont rapprochés des anciens principes. Ce fut par des marches hardies et rapides que Vendôme conserva la couronne d’Espagne à Philippe V. Ce fut une marche offensive qui sauva la France à Denain. Ce fut par une campagne de marches et de mouvements, que Créqui s’immortalisa sur la Sarre et sur la Moselle. Mais pour parler du général de Louis XIV qui, commandant de grandes armées, sût le mieux les remuer, le maréchal de Luxembourg, il faut voir dans ses campagnes, il faut lire dans ses dépêches combien il croyait les marches importantes, combien il leur a dû de succès. Ce fut sous lui, ce fut étant maréchal général des logis de son armée, que le maréchal de Puységur jeta le plan d’une partie de ces combinaisons de marche qu’il développa depuis dans son traité sur l’art de la guerre. C’est une théorie bien imparfaite, bien compliquée, que celle du maréchal. Mais alors elle n’était pas sans mérite, elle apportait quelques lumières au milieu des ténèbres.\par
Il eût même été heureux qu’elle eût paru plus tôt et qu’elle eût été méditée. C’était enfin le premier ouvrage dogmatique sur la grande tactique des armées. Les ignorants le regardèrent comme un chef-d’œuvre. Les gens instruits, les gens de guerre y virent beaucoup d’erreurs à côté d’un petit nombre de vérités. Ils sentirent que le maréchal n’avait pas touché le but, que notre tactique était vicieuse, qu’elle avait besoin d’être changée, d’être refondue par un homme de génie. Le maréchal de Saxe, qui aurait pu faire cette révolution, qui l’aurait peut-être faite s’il avait vécu, s’il avait joint plus d’amour du travail à ses grandes qualités pour la guerre, sentait ce besoin. Il le disait souvent, il écrivait en 1750 à M. d’Argenson, que toutes les troupes de l’Europe, aux prussiennes près, qui commençaient à s’éclairer, étaient mal constituées et incapables d’exécuter de grandes manœuvres. Il le répète dans ses {\itshape Rêveries}. On y voit combien il est indigné de la lenteur de nos marches, de notre ignorance, de notre maladresse à prendre un ordre de bataille. C’est à ce sujet qu’il dit : « Tout le secret de l’exercice, tout celui de la guerre est dans les jambes ». J’ai déjà cité ailleurs cette expression obscure en apparence, mais qui me semble renfermer un sens bien profond et bien juste.\par
Ce vice, que le maréchal de Saxe sentait être dans la constitution de nos troupes, dans la théorie de nos marches et de nos ordres de bataille, M. le maréchal de Broglie le sentit de même, quand il parvint au commandement de l’armée. Il y établit en conséquence un ordre nouveau. Il la partagea en plusieurs divisions. De cette organisation, dont la plupart des gens n’aperçurent pas l’objet, on vit résulter plus de célérité dans les marches, moins de fatigues pour les troupes, plus de discipline dans les camps. Mais le maréchal n’eut pas le temps d’achever son ouvrage. Ce n’est pas d’ailleurs en deux campagnes et au milieu du tumulte des opérations de la guerre, qu’on peut changer la tactique d’une armée et former des officiers généraux. Ce n’est pas surtout en France, qu’une pareille révolution peut se faire. Il n’y aurait qu’un roi général qui pourrait l’exécuter.\par
La Prusse en avait un de cette espèce et c’est là ce qui la fit lutter avec avantage contre la ligue qui la menaçait. Les ventes qu’on entrevoyait ailleurs, sans faire de pas décisifs vers elles, le roi de Prusse les avait vues à son arrivée au trône, et il avait en conséquence profité de la paix pour instruire ses troupes. Elles étaient les mieux ordonnées et les plus manœuvrières de l’Europe. Elles avaient une tactique particulière de marches et de déploiements. Dans son armée seule, étaient des officiers généraux qui sussent conduire une colonne, manier des troupes et concourir à l’exécution d’un ordre de bataille. On en a vu le résultat. Par son armée seule, ont été faits de grands et hardis mouvements. On l’a vu, à la tête de cette armée, voler de l’Elbe en Silésie, de la Silésie vers les Russes. On l’a vu, à la tête de cette armée surprise et battue, s’arrêter à deux lieues du champ de bataille et présenter un nouveau combat. On l’a vu, à Torgau, perdre la bataille, s’apercevoir, en se retirant, d’un faux mouvement fait par les Autrichiens, d’une hauteur mal-à-propos dégarnie, y faire remarcher des troupes, s’en emparer, reporter toute son armée à l’appui et forcer les Autrichiens à se retirer derrière l’Elbe. Enfin, partout où il a fallu manœuvrer, partout où le succès a dépendu de l’intelligence et de la rapidité des marches, le succès a été pour lui. Sans doute il n’aurait pas tant osé, il ne l’aurait pu, s’il avait eu des troupes moins manœuvrières, des officiers généraux moins en état de le seconder. Car quelle action tirer d’une machine dont les ressorts ne seraient susceptibles, ni de jeu ni de combinaison ?\par
J’ose avancer cependant que le roi de Prusse n’a pas épuisé toutes les combinaisons de l’art et que sur la grande tactique, sur la partie des marches principalement, il reste beaucoup de progrès à faire. Cette assertion est bien hardie, mais elle ne sera pas dénuée de preuves. Je demande qu’avant de la juger, on prenne la peine de lire mon ouvrage jusqu’au bout.\par
Il faut d’abord distinguer les différentes espèces de marche qu’une armée est dans le cas d’exécuter.\par
Ce peuvent être des marches simples et faites hors de la portée de l’ennemi, avec l’objet de se porter commodément vers un point. Dans cette espèce de marches, qu’on peut appeler marches de route et que les armées sont quelquefois, dans le cas de faire, soit au commencement des campagnes, en se rapprochant de l’ennemi ; soit à la fin des campagnes, en se séparant mutuellement ; soit dans ces moments où les opérations ont éloigné les armées les unes des autres et les ont mises respectivement hors de mesure, il n’entre que des combinaisons simples et qui doivent être uniquement relatives à la moindre fatigue et à la plus grande commodité des troupes.\par
Ce peuvent être des marches hors de la portée de l’ennemi, mais dont le but soit de la prévenir sur un point, ou de s’emparer rapidement d’un poste, ou de porter des secours à un objet menacé, ou de changer, sans qu’il s’y attende, le théâtre de la guerre. Dans ce cas, il faut que les marches soient combinées de manière à se procurer toute la célérité possible, à faire forcer des journées, s’il est nécessaire, à la totalité de l’armée, ou du moins à un corps de troupe, à l’appui duquel on puisse arriver à temps avec le reste de ses forces. Il faut savoir s’écarter des principes ordinaires et pour cet effet, si cela peut rendre la marche plus rapide et plus commode, séparer l’armée en plusieurs corps qui se réunissent sur le point, ou à portée du point médité. Il faut enfin calculer qu’étant hors de la portée de l’ennemi et avant pour objet d’arriver, il faut gagner en vitesse ce dont on se relâche en méthode, faire de la célérité, l’objet principal et unique de ses combinaisons.\par
Ce peuvent être des marches-manœuvres, c’est-à-dire des marches faites à portée de l’ennemi et par conséquent, dans l’objet de prendre, s’il est besoin, un ordre de bataille. Cette dernière espèce de marches, sur laquelle je vais entrer dans des détails étendus, est la plus importante et celle qui exige le plus de combinaisons. Il s’agit à la fois d’y calculer la nature du pays qu’on traverse, celle du pays où l’on doit aboutir ; l’espèce d’arme dans laquelle on est supérieur ; la qualité des troupes de l’armée ; la disposition qu’on veut prendre, soit qu’on doive attaquer, soit qu’on doive se défendre ; le plus ou le moins d’habileté de l’ennemi ; sa position ; ses vues ; la distance à laquelle il est ; la dextérité, plus ou moins grande, de ses troupes à prendre un ordre de bataille. Cette espèce de marches, en un mot, est la préparation à la plus grande opération militaire qu’il y ait, à la formation des ordres de bataille, aux batailles qui en sont la suite. Les mouvements, par lesquels l’armée passe de l’ordre de marche à l’ordre de bataille, sont tellement liés aux combinaisons de l’ordre de marche, qu’on doit les regarder comme une seule et même opération. Je ferai voir comment cet enchaînement existe. Commençons par établir des règles sur le premier objet dont on doit s’occuper pour mettre une armée en marche. Je veux dire l’ouverture des chemins. Nous en déduirons la manière dont l’armée doit se former en ordre de marche. Cela nous mènera à poser les principes des mouvements qui doivent la mettre en ordre de bataille. Cette théorie est particulièrement relative à la troisième espèce de marche, dont j’ai parlé ci-dessus et ne peut, qu’en quelques points, être appliquée aux deux autres, puisqu’elles n’ont pas pour but de conduire l’armée à une disposition de combat.
\subsection[{II. - Ouverture des marches}]{II. - Ouverture des marches}
\noindent Une armée rangée dans l’ordonnance actuelle et surtout une armée composée, comme le sont aujourd’hui les nôtres, de beaucoup d’hommes, de chevaux, d’attirails et d’embarras ne peut se mouvoir et à plus forte raison, exécuter une marche en ligne, trouvât-elle même des plaines assez vastes et assez continues pour la recevoir. L’étendue de son front rendrait les mouvements si lourds et si lents, qu’ils seraient impraticables.\par
Elle ne peut se mouvoir sur une seule colonne, parce que l’immense allongement de cette colonne ralentirait la marche, augmenterait la fatigue des troupes, et mettrait en danger d’être battue et renversée, avant qu’elle pût se former.\par
Il faut donc que, pour exécuter une marche, l’armée se partage en plusieurs corps ou colonnes, qui, suivant chacun des chemins différents, arrivent sur la même direction et soient en mesure de pouvoir, par des mouvements combinés entre eux, prendre une disposition générale de combat. Quand je dis sur la même direction, c’est-à-dire vers le même objet ; car la disposition de la marche peut être telle qu’on veuille porter une partie de l’armée sur le flanc de l’ennemi, tandis qu’on en portera le reste sur son front ; et alors, quoique la direction de toutes les colonnes ne soit pas précisément la même, toutes cependant concourent au même objet, qui est de prendre un ordre de bataille et d’attaquer l’ennemi.\par
Les marches d’armée devant s’exécuter sur plusieurs colonnes, il faut qu’en conséquence chacune de ces colonnes ait un chemin ouvert ou reconnu, ou du moins une direction sur laquelle elle puisse s’avancer à l’aide des tirailleurs qui sont à sa tête. J’établis ces différences ; parce qu’il est possible que le défaut de temps ou l’ennemi, n’aient pas permis d’ouvrir le chemin à l’avance, ou qu’ils aient permis de le reconnaître seulement et non de l’ouvrir ; ou qu’enfin, le chemin n’ayant pu être ni reconnu ni ouvert, il faille s’avancer sur une direction projetée, en reconnaissant et préparant, chemin faisant, son débouché. Ce dernier cas arrive communément lorsqu’on marche pour donner bataille à l’ennemi et que cet ennemi a en avant de lui des corps et portes détachés qu’il faut attaquer et replier successivement.\par
Le nombre de colonnes, sur lequel une armée doit marcher et par conséquent celui des débouchés qu’il faut ouvrir, doit être en proportion de la force de cette armée et du nombre de divisions dans lequel le général l’aura partagée. Je dirai dans le chapitre suivant, pourquoi il faut partager une armée en plusieurs divisions, quelle est la proportion qu’il faut observer à cet égard, comment, cette base prise, il faut autant que les circonstances le permettent, combiner sur elle ses ordres de marche, former une colonne de chaque division et se tenir, en formant par là toutes les parties de sa disposition de la même force, en mesure de pouvoir prendre tel ordre de bataille et de se renforcer sur tel ou tel point de cet ordre de bataille que l’on juge à propos.\par
Pour qu’une armée ne soit que le moins qu’il est possible, dans le cas de faire des marches, sans que ses débouchés soient préparés, il faut, quand cette armée arrive dans une position, que le maréchal général des logis s’occupe d’abord de faire ouvrir des marches sur toutes les directions que les circonstances ultérieures pourraient l’obliger à suivre. Cette méthode remplit à la fois l’objet de parer à l’avenir et celui de cacher à l’ennemi le mouvement qu’on projette. Si au lieu de cela on ne fait ouvrir des marches que sur la direction indiquée par la circonstance momentanée, ou prévue, on découvre les vues qu’on peut avoir et ces vues n’ayant qu’à changer, on se trouve obligé de faire une marche incommode et pénible. Ce principe au reste est soumis aux événements ; car quelquefois on ne séjourne pas dans une position et à peine a-t-on le temps d’ouvrir la marche du lendemain. Quelquefois l’ennemi occupe, par des postes considérables, ou par des corps détachés, le pays où la marche doit s’exécuter, et alors elle ne peut se faire qu’en s’avançant à lui, et en combattant s’il résiste. Quelquefois il est avantageux de donner le change à l’ennemi, en faisant ouvrir une marche sur un point vers lequel on ne veut pas se porter, tandis que l’on en fait reconnaître secrètement une autre vers celui sur lequel on veut lui dérober un mouvement. D’autres fois les circonstances et les positions respectives des armées sont telles, que si elles remuent, ce ne peut être que pour se porter vers un objet indiqué. Alors il est inutile de se fatiguer à ouvrir des marches vers les autres points. D’autres fois on est en défensive absolue et déterminée, de manière à ne pouvoir ou à ne vouloir faire de mouvements qu’en arrière de soi. Alors il est certainement inutile d’ouvrir des marches en avant, puisque ce ne serait que fournir à l’ennemi des débouchés défensifs. Pour l’éclaircissement du principe, que j’ai exposé au commencement de cet article, il faut enfin conclure qu’il peut y avoir des occasions où il est inutile et même impossible d’ouvrir des marches sur toutes les directions. Du moins il est important que toutes les directions soient reconnues par les officiers de l’état-major, ou si cela ne se peut, par des renseignements pris avec les gens du pays. Il faut conclure que le maréchal général des logis doit se faire un tableau exact de ces itinéraires et reconnaissances, de façon à ne jamais quitter une position sans avoir poussé des rayons par ses reconnaissances ou par les renseignements qu’il prendra sur toutes les directions environnantes et à parvenir, par ce moyen, à la plus parfaite connaissance possible du théâtre de la guerre […].
\subsection[{III. - De la disposition des ordres de marches}]{III. - De la disposition des ordres de marches}
\noindent Je viens de dire comment les débouchés des marches d’armée doivent être reconnus et ouverts. II s’agit maintenant d’exposer les différentes manières dont une armée peut être disposée en ordre de marche. Commençons par expliquer, ainsi que je l’ai promis, comment cette armée doit être divisée, quel est l’objet de cette division et sur quelle proportion il faut la déterminer.\par
Pour qu’une grande masse puisse être mue avec plus de facilité, il faut la partager, s’il est possible, en plusieurs parties. Alors chacune de ces parties est susceptible de recevoir plus de mouvement et d’action. Alors on peut, par des forces combinées et multipliées, agir sur toutes ces parties à la fois. Il en est ainsi d’une armée. Qu’on veuille la mouvoir en masse, elle sera maladroite, lente, incapable de grandes manœuvres. Qu’on la partage en plusieurs corps, chacun d’eux agira séparément avec plus d’ordre et de célérité. Tous pourront agir à la fois et concourir à l’exécution d’un mouvement général. Je parlerai ailleurs des avantages qu’une division pareille procure, soit relativement à la simplicité du service et à la discipline, soit par rapport à la diminution de détails et d’embarras qui en résulte pour le général.\par
À l’égard de la proportion sur laquelle le partage d’une armée en divisions doit être fondé, la mécanique va encore me fournir une démonstration sensible. Si voulant organiser une machine on en fait les mobiles trop nombreux et trop faibles, on complique les détails de cette machine et on diminue sa force. Si on les fait trop peu nombreux et trop solides, ils deviennent susceptibles de trop peu d’action, soit en force, soit en vitesse. Il en est de même d’une armée. Si en composant ses divisions d’un trop petit nombre de troupes, on la forme d’un trop grand nombre de divisions, on tombe dans la complication. On manque d’hommes capables de conduire ces divisions. On a de la peine à combiner tant de mouvements séparés. Si en composant les divisions d’un trop grand nombre de troupes on forme l’armée d’un trop petit nombre de divisions, chacune d’elles restant encore trop massive et trop pesante, l’opération ne remplit pas son objet, qui est d’alléger et de mettre en état d’agir. Il existe au milieu de ces extrêmes une moyenne qu’un raisonnement simple va faire trouver […].\par
Les divisions ont pour objet principal de simplifier les ordres de marche et de faciliter, ainsi que d’accélérer les mouvements par lesquels l’armée peut prendre un ordre de bataille. Il faut par conséquent que, destinées à former ordinairement chacune une colonne, elles ne soient pas composées d’un trop grand nombre de troupes. Alors elles seraient trop pesantes, trop lentes à se déployer, et à concourir à l’exécution de l’ordre de bataille. Il ne faut pas non plus qu’elles soient composées d’un trop petit nombre de troupes. Alors celui des divisions serait trop multiplié et il serait pénible et souvent impraticable d’ouvrir assez de débouchés pour faire mouvoir l’armée. Je me répète peut-être, mais dans une discussion de principes, cela est quelquefois inévitable.\par
La véritable proportion du partage d’une armée, combinée sur les manœuvres de déploiement et sur les vues de la grande tactique, est de trois et au plus quatre divisions pour l’infanterie, indépendamment des ailes de cavalerie qui en formeront chacune une. Quant à la force des divisions, elle doit être au plus de vingt-quatre bataillons dont moitié de première et moitié de seconde ligne et jamais au-dessous de douze. Chaque aile de cavalerie devant former une division, il n’y a pas de bornes précises à lui assigner. Je donnerai dans la suite de cet ouvrage un exemple d’une armée divisée d’après ce principe et exécutant tous les ordres de marche et de bataille possibles.\par
Expliquons maintenant avec le détail nécessaire, pourquoi il faut que les divisions soient égales, pourquoi il faut, autant qu’il est possible, que les dispositions de l’ordre de marche soient immuables. Ce sont ces divisions de la même force, ce sont, au moyen de cela, toutes les dispositions de l’ordre de marche égale entre elles, c’est cet ordre de marche, autant que les circonstances le permettent toujours le même, qui donnent au général la facilité de combiner dans le moment et à la vue de l’ennemi, tel ou tel ordre de bataille qu’il juge à propos. J’ai dit dans tout cela, autant que les circonstances le permettent, car quelquefois il peut arriver que la nature du pays ne comportant pas d’ouvrir le nombre de débouchés nécessaire, on soit obligé de former les colonnes de plus d’une division. Quelquefois il s’agit d’un mouvement offensif dans un pays absolument ouvert et dont le but est de prendre rapidement une disposition de combat. Alors il n’y a nul inconvénient à multiplier les colonnes, à en former deux de la même division, puisque cela peut conduire à un déploiement plus prompt. Quelquefois la marche doit s’exécuter à travers un pays où il faut nécessairement, pour la sûreté du mouvement, mettre de l’infanterie aux colonnes des ailes et alors s’écarter de l’ordre habituel. D’autres fois elle doit conduire à une position dont tout le centre soit une plaine et dont les ailes soient un pays couvert. Alors si l’on n’a pas à craindre d’être attaqué dans la marche, on s’écarte encore de l’ordre habituel, on forme les colonnes des ailes d’infanterie et les colonnes du centre de cavalerie. Quelquefois enfin, il est question d’aller attaquer l’ennemi dans une position connue, et où il ne peut pas faire de changements imprévus dans sa disposition. Alors l’ordre de marche peut être combiné à l’avance sur l’ordre de bataille qu’on projette. Alors conséquemment il peut s’écarter de la disposition habituelle. Les colonnes peuvent être d’inégale force et différemment combinées, suivant qu’on se propose d’attaquer avec une partie et de refuser les autres, de mettre là ou là telle ou telle espèce d’arme, en telle ou telle proportion. Voilà, comme on voit, plusieurs exceptions au principe ordinaire. Exceptions qui, loin de le détruire, servent à l’éclaircir en annonçant les circonstances où il faut s’en écarter.\par
Le principe fondamental et habituel, je le répète, doit donc être reformer autant de colonnes qu’il y a de divisions, afin que de cette égalité de toutes les parties de l’ordre de marche, il puisse en résulter la possibilité d’exécuter dans le moment un ordre de bataille renforcé ou dégarni dans tous les points que les circonstances indiqueront. Je démontrerai, en traitant des ordres de bataille, combien est immense l’avantage de ne faire ses dispositions qu’à la vue du terrain où l’on doit combattre et des dispositions de l’ennemi. Je démontrerai combien il a été perdu de batailles, ou faute d’avoir eu ce principe, ou, quand même on l’aurait eu, faute d’une tactique de déploiements et d’une habitude de manœuvres qui permissent de l’exécuter. Terminons ce qui concerne les ordres de marche.\par
Il s’ensuit de la théorie et particulièrement du principe que je viens d’exposer, que les dispositions des ordres de marche doivent être presque toujours les mêmes et que c’est au génie à tirer de ces dispositions un ordre de bataille combiné sur la nature des terrains et des circonstances. Voilà ce qui n’est certainement ni dans la théorie de M. de Puységur, ni dans la routine actuelle de toutes nos armées. Les ordres de marche y sont toujours combinés sur l’ordre de bataille qu’on veut prendre. On forme le projet d’attaquer l’ennemi sur tel ou tel point de sa disposition. On règle en conséquence l’ordre de marche. On renforce telle ou telle colonne. Arrivé vis-à-vis de l’ennemi on se met en bataille comme le dicte l’ordre de marche. Je dirai, au chapitre des ordres de bataille, les erreurs et les mauvais succès qui en résultent.\par
Il me reste à examiner quelle doit être la disposition intérieure des ordres de marche, relativement à la nature de la marche qu’on exécute et aux différents mobiles qui composent les colonnes. Ces mobiles sont les troupes, l’artillerie, les équipages. Je dirai donc successivement quelles doivent être les dispositions des troupes, de l’artillerie et des équipages dans les colonnes de marche.
\subsection[{IV. - Disposition des troupes dans les ordres de marche}]{IV. - Disposition des troupes dans les ordres de marche}
\noindent Je renverrai pour tout ce qui concerne la formation des troupes en colonne de marche et les mouvements par lesquels elles doivent passer de l’ordre de marche à l’ordre de bataille, à ce que j’ai dit à ce sujet dans la tactique de l’infanterie au chapitre des formations en colonne et des formations en bataille et dans l’essai sur la tactique de la cavalerie. J’y ai établi les différences de marche de front et de marche de flanc, les différentes manières dont les troupes des deux armes devaient se mettre en colonne relativement à ces différences, l’ordre dans lequel elles devaient marcher, les mouvements préliminaires par lesquels elles devaient se préparer à se mettre en bataille et enfin ceux par lesquels elles devaient se former et concourir à l’exécution d’une manœuvre combinée […].
\subsection[{V. - Disposition de l’artillerie dans les ordres de marche}]{V. - Disposition de l’artillerie dans les ordres de marche}
\noindent On a vu, dans la première partie de cet ouvrage, au chapitre de l’artillerie, comment cet accessoire devait manœuvrer pour se mettre en colonne de marche et se former en batterie. J’ai démontré que ces mouvements pouvaient être absolument analogues à ceux des troupes. Il me reste à dire ici comment la quantité d’artillerie que les armées traînent aujourd’hui à leur suite, doit être disposée pour ne pas embarrasser les marches et pour concourir à la formation d’un ordre de bataille.\par
Les mêmes raisons, qui m’ont fait soutenir le système du partage de l’armée en plusieurs divisions, exigent que l’artillerie soit divisée dans la même proportion. Si l’infanterie de l’armée forme trois ou quatre divisions, l’artillerie formera de même trois ou quatre divisions d’égale force, dont chacune sera attachée à une division d’infanterie, pour camper, marcher et combattre avec elle. Indépendamment de cela, il y aura une autre division appelée division de réserve, composée de gros calibre et d’obusiers. De celle-là qui marchera à la tête du parc, seront tirés les renforts qu’on voudra porter sur un point, les détachements qui seront nécessaires, l’artillerie que, suivant mon opinion, on devra quelquefois employer à l’appui de la cavalerie et pour cela faire marcher avec elle. Enfin il y aura une autre petite division composée de deux, quatre, ou au plus six pièces de gros calibre. Cette division sera appelée division d’avant-garde, campera en avant de l’armée et marchera avec l’avant-garde. Je la forme de pièces de gros calibre, parce que c’est de l’avant-garde que doivent être faits les signaux qui régleront les mouvements de l’armée et parce que, si cette avant-garde trouve sur son chemin quelque poste retranché il lui faut du gros calibre pour le battre et l’emporter. Je ne parle ici que de l’artillerie de parc. Car pour celle des régiments, si l’on continue de leur en donner, elle campe et marche avec eux. Ainsi elle est tout naturellement divisée.\par
Cette grande division de l’artillerie est indépendante des subdivisions intérieures qu’on doit y former de manière, par exemple, que chaque subdivision soit composée de six pièces du même calibre et que chaque division le soit d’un nombre égal de subdivisions composées de calibres différents.\par
Je donnerai dans la formation de l’armée que je rassemblerai pour l’exécution des principes que je développe ici un exemple d’un équipage d’artillerie de campagne divisé suivant ce que j’ai dit ci-dessus et l’on verra la facilité et la simplicité de mouvements qui en sont le résultat.\par
C’est la nature et l’objet de la marche qui doivent déterminer l’ordre dans lequel l’artillerie doit marcher. Ainsi il faut se rappeler à ce sujet les distinctions que j’ai posées entre marches de route et marches-manœuvres, marches de front et marches de flanc, puisque relativement à chacune de ces sortes de marches, l’artillerie doit observer un ordre différent.\par
S’il s’agit d’une marche de route, comme son but unique est la plus grande commodité des troupes, ou la plus grande célérité possible, comme il n’y est point question d’arriver à un ordre de bataille et conséquemment d’avoir de l’artillerie à portée de le protéger, l’artillerie marchera tout simplement à la queue des troupes afin de ne pas les retarder dans leur marche et de ne pas gâter les chemins ; c’est-à-dire que si l’espèce des débouchés le permet, chaque division marchera à la suite de la division d’infanterie à laquelle elle est attachée et que, si elle ne le permet pas, on mettra l’artillerie à telle colonne que l’on jugera à propos. Dans les deux cas la division de réserve et le gros parc marcheront à la colonne dont le débouché sera le meilleur et le plus facile.\par
S’il s’agit d’une marche-manœuvre et par conséquent d’une marche faite à portée de l’ennemi et avec le but de prendre un ordre de bataille, il faut que l’artillerie soit disposée de manière à ne gêner les troupes et à ne ralentir la marche que le moins qu’il sera possible et en même temps à pouvoir entrer dans la combinaison de l’ordre de bataille, à protéger son exécution. D’après cela on doit voir si la marche est de front ou de flanc et faire ses dispositions en conséquence.\par
Si la marche est de front, voici comment marcheront les divisions d’artillerie. À la tête de chaque colonne et précédées seulement d’un bataillon de grenadiers, seront placées une ou deux subdivisions de gros calibre débarrassées de toutes leurs voitures d’attirails et ayant une vingtaine de coups par pièce pour commencer le combat. Le reste de chaque division d’artillerie suivra la division d’infanterie à laquelle elle est attachée, de manière que le canon soit immédiatement à la queue des troupes et que toutes les voitures d’attirails et de munitions soient derrière lui. Par le moyen de cette disposition on aura à la tête des colonnes l’artillerie nécessaire pour protéger le déploiement. Les troupes n’étant point embarrassées se mettront rapidement en bataille et l’on disposera ensuite, ainsi que l’on voudra, du reste de l’artillerie ; soit en la faisant arriver à l’appui de celle qui sera déjà postée, soit en lui faisant prendre des emplacements collatéraux à la disposition des troupes, soit enfin en la laissant derrière les lignes si l’on veut entrer sur le champ en action décidée et pour cela ne pas embarrasser le front. L’artillerie de réserve marchera derrière les colonnes du centre. Elle sera toujours renforcée d’attelages afin de pouvoir, à toutes jambes, se porter aux points de l’ordre de bataille où elle sera jugée nécessaire.\par
Voilà dans les marches-manœuvres de front quelle sera la disposition habituelle, mais les circonstances pourront y occasionner différents changements. Quelquefois, par exemple, les points d’attaque étant connus à l’avance, on saura que telle colonne doit s’emparer d’un village ou d’un retranchement qu’il sera nécessaire de battre auparavant par un grand feu d’artillerie. Alors on mettra à la tête de cette colonne un plus grand nombre de subdivisions et toutes de gros calibre. Quelquefois on voudra appuyer et soutenir une aile de cavalerie. L’on joindra en conséquence à la colonne qui doit la former, une ou plusieurs subdivisions de bouches à feu et particulièrement d’obusiers. Cette artillerie, pourvue de vingt coups par pièce, marchera à la tête de la colonne couverte par quelques escadrons et ses voitures de munitions marcheront à la queue. J’ai parlé, dans l’essai sur la tactique de l’artillerie, des services que pourrait rendre cette méthode peu, ou pour mieux dire, point connue, d’employer de l’artillerie avec la cavalerie.\par
Viennent ensuite toutes les dispositions intérieures qu’il sera à propos de faire dans les divisions d’artillerie, lorsque, marchant pour attaquer l’ennemi, l’on aura connaissance des parties de l’ordre de bataille avec lesquelles on veut faire effort et de celles qu’on veut lui refuser. Ces dispositions auront pour but de renforcer d’attelages l’artillerie des colonnes destinées à agir, de mettre dans les divisions d’artillerie de ces colonnes plus de pièces de petit calibre et d’attacher les calibres les plus forts aux colonnes qui doivent former les parties de la disposition les plus éloignées de l’ennemi et où par conséquent les plus longues portées seraient le plus nécessaires, etc. Il faut lire, dans la tactique de l’artillerie, les principes posés à cet égard.\par
Quant aux pièces de canon de régiment, elles marcheront avec leurs bataillons et en suivront les mouvements. Mais je ne puis m’empêcher de répéter que telles qu’elles sont par leur espèce multipliées au point où elles le sont, elles donnent plus d’embarras qu’elles ne rendent de service.\par
Je dois maintenant parler des marches de liane. Si elles ne sont pas faites dans des circonstances qui fassent craindre que l’ennemi, longeant une parallèle à la direction du mouvement de l’armée, ne cherche à l’attaquer pendant sa marche, l’artillerie pourra marcher à la queue des troupes de chaque colonne, ou par une colonne séparée sur le flanc intérieur de l’ordre de marche. Si l’on peut craindre d’être attaqué par l’ennemi, chaque division d’artillerie marchera à la tête et à la queue des troupes de chaque division de troupes, n’ayant toutefois lesdites divisions d’artillerie avec elles que les caissons de munitions nécessaires pour le premier moment et tout le reste marchant, ainsi qu’il est dit ci-dessus, par une colonne séparée en dedans de l’ordre de marche.\par
Je laisse à juger aux gens de guerre si cette théorie des dispositions de l’artillerie dans les marches n’est pas supérieure à la routine qu’on a suivie jusqu’à présent. Je laisse à juger si ayant de l’artillerie divisée et manœuvrée ainsi, on ne pourrait pas en avoir une bien moins grande quantité qu’on n’en a aujourd’hui, une bien moins nombreuse que celle de l’armée ignorante que l’on aurait vis-à-vis de soi, et avec cela ne jamais manquer d’artillerie et se procurer sur les points nécessaires, des feux supérieurs aux siens.
\subsection[{VI. - De la disposition des équipages dans les marches}]{VI. - De la disposition des équipages dans les marches}
\noindent Il y a peu de chose à dire sur cet objet. On ne doit jamais mêler les équipages avec les troupes. On les fait marcher à la suite des colonnes dans le même rang que les troupes y tiennent et les mulets ou chariots de campement des soldats ayant toujours la tête. Si l’on a à craindre pour les flancs de la marche, on n’en place point aux colonnes extérieures de l’ordre de marche et on prend des précautions pour les couvrir. S’il est question d’une marche-manœuvre ou d’une marche forcée, on prend le parti de laisser les équipages en arrière, choisissant pour cet effet un lieu sûr et qui à tout événement puisse être couvert par l’armée. Toutes ces règles sont connues. Mais ce sur quoi on est dans la routine la plus mal entendue, c’est l’ordre individuel de ces équipages. Je vois que dans les chaussées de Flandre, dans les pays les plus ouverts, sur les mêmes débouchés où les troupes ont marché par pelotons, les mulets ou chevaux d’équipages y marchent sur une seule file, comme si l’on était dans les défilés des Alpes. Il serait certainement possible qu’au lieu de cela ils marchassent sur deux ou sur trois files. J’en dirai autant pour les voitures, qui souvent pourraient marcher sur deux de front. Enfin il devrait y avoir une règle de proportion établie, d’après laquelle le maréchal général des logis dirait : les troupes de telle colonne peuvent marcher sur tel front, par conséquent les équipages de cette colonne marcheront de telle manière.\par
Mais à quoi servira toute l’intelligence possible dans la disposition des marches, si nous ne cherchons à diminuer cette quantité énorme d’attirails, d’équipages, de valets, si bien nommée par les Anciens « impedimenta ». Si pour cela nous ne devenons plus sobres, moins amoureux de nos aises, plus endurcis aux travaux ? Je ne m’étendrai pas là-dessus. Car une pareille révolution ne peut s’opérer qu’en changeant l’esprit et les mœurs actuelles. Or changer l’esprit et les mœurs d’une nation ne peut être l’ouvrage d’un écrivain quel qu’il soit. Ce ne peut être que celui du souverain, ou d’un homme de génie dans les mains duquel de grands malheurs et le cri public, plus fort que les cabales, remettront pendant quelques années de suite, le timon de la machine.
\subsection[{VII. - Des ordres de bataille}]{VII. - Des ordres de bataille}
\noindent Ordre de bataille dans la tactique actuelle, peut s’entendre de deux manières. Il signifie d’abord l’ordre primitif et fondamental dans lequel une armée se dispose pour camper et pour combattre, étant mises à part toutes circonstances de manœuvre et de terrain. Il doit signifier ensuite toute disposition quelconque dérivant de cet ordre primitif, avec telles ou telles différences quelconques occasionnées par ces circonstances. Je vais éclaircir cette double définition. Cela répandra du jour sur la théorie qui en sera la suite.\par
Considéré comme disposition primitive et fondamentale, l’ordre de bataille d’une armée est le tableau qu’on forme au commencement de la campagne pour régler l’emplacement et la disposition des différents corps qui composent l’armée. C’est d’après lui que les troupes sont disposées sur deux lignes, l’infanterie au centre et la cavalerie sur les ailes. Ce premier arrangement est fondé en raison, quand il n’est que la disposition préparatoire et, si je peux m’exprimer ainsi, la disposition d’attente et d’organisation. Mais il devient abus et erreur quand il dégénère en routine, quand on le prend indifféremment dans toutes les circonstances et dans tous les terrains, quand surtout on en fait sa disposition de combat.\par
Je dis que cet arrangement est fondé en raison, quand il n’est que la disposition de campement et d’organisation. En effet quand on rassemble une armée, il faut bien y établir un ordre primitif et habituel, un ordre qui soit la base d’après laquelle on puisse partir pour opérer et à laquelle on puisse revenir quand les circonstances qui en éloignaient, n’existent plus. Je dis que cet arrangement devient abus et erreur quand on ne sait pas s’en écarter suivant les circonstances, quand on en fait aveuglément sa disposition de combat. En effet il est aisé de sentir que des incidents, des circonstances et des vues sans nombre doivent obliger de faire des changements à l’ordre primitif. Il est aisé de sentir, par exemple, que, quoique suivant cet ordre l’armée doive être formée sur deux lignes, l’infanterie étant au centre, la cavalerie sur les ailes et tous les corps qui composent chaque ligne étant contigus l’un à l’autre, les circonstances à la guerre peuvent exiger que là il faille mettre de la cavalerie au centre et de l’infanterie sur les ailes, ici combattre sur une ligne, plus loin se former sur trois, ailleurs séparer l’armée en plusieurs corps pour les faire agir chacun sur différents points. Toutes ces dérogations à l’ordre primitif n’empêchant cependant point que la totalité de la disposition ne soit un ordre de bataille, puisqu’elle a pour but de combattre. Concluons de là qu’ainsi que la postérité se tromperait, si elle imaginait, que telle était la disposition sur laquelle combattaient toujours nos armées, nous nous trompons sans doute étrangement, lorsqu’en trouvant dans l’histoire la disposition d’une armée grecque ou consulaire, nous croyons qu’elles combattaient toujours dans cet ordre. Car vraisemblablement cet ordre n’était qu’une disposition fondamentale et primitive à laquelle ils faisaient des changements, suivant ce qu’exigeaient la nature du terrain et les mouvements de l’ennemi. Revenons à mon objet qui est d’éclaircir la définition que j’ai avancée.\par
Considérée comme disposition dictée par le terrain et par les circonstances, l’ordre de bataille d’une armée est l’ordre quelconque dans lequel elle se range pour combattre, c’est-à-dire, que ce n’est jamais et ne peut presque jamais être l’ordre de méthode. Car rarement se trouve-t-on dans des plaines où l’armée puisse être formée sur des lignes droites et continues : aussi rarement dans des pays où l’on doive composer tout le centre d’infanterie et les ailes de cavalerie. Souvent on affaiblit et on met hors de prise une partie de sa disposition pour en renforcer une autre avec laquelle on veut combattre. Dans ces différentes circonstances, on se conduit ainsi que je l’ai dit ci-dessus. On s’écarte de l’ordre de méthode, on prend une disposition qui y a quelquefois très peu de rapport. Il y a même plus. C’est que plus un général est habile, plus son armée est manœuvrière et plus il s’écarte de la routine établie, afin de porter à son ennemi des coups imprévus et décisifs. J’aurai occasion de développer cela dans la suite de cette théorie. Je crois en avoir dit assez pour faire sentir la différence qu’il y a entre la disposition de méthode et la disposition de circonstance, qui toutes deux cependant peuvent s’appeler ordre de bataille, avec cette différence que la première n’a lieu que dans les camps et dans les rêves des tacticiens, tandis que la seconde est celle dans laquelle on donne les batailles et surtout celle qui les fait gagner.\par
C’est cette dernière qu’on parviendra à exécuter facilement au moyen de la tactique exposée dans cet ouvrage, et qu’on ne pouvait exécuter de même avec les anciens principes et les mouvements pratiqués jusqu’à présent dans nos armées. Outre qu’on n’avait aucune idée de la grande tactique, outre que les armées n’étaient ni divisées ni constituées de façon à pouvoir être manœuvrières, les différents corps qui la composaient, ne se remuaient individuellement que par des méthodes lentes, lourdes et dont elles n’avaient pas même l’habitude. Les officiers généraux n’avaient point d’usage de manier les troupes. De cette ignorance et de cette maladresse générale, tant de la part des agents que des conducteurs, il résultait qu’il fallait plusieurs heures pour mettre une armée en bataille ; qu’une fois cette armée en bataille on n’osait, de peur de tout confondre, de tout perdre, faire le moindre changement dans la disposition. Il résultait qu’il fallait toujours combiner l’ordre de marche sur la disposition qu’on voulait prendre. Ainsi, par exemple, on se mettait en marche avec l’objet d’attaquer l’ennemi sur tel ou tel point. On renforçait en conséquence telles ou telles colonnes. Arrivait-on en présence de l’ennemi, l’ordre de bataille était dicté par l’ordre de marche et se prenait en conséquence. Qu’arrivait-il cependant ? C’est que souvent cet ordre de bataille se trouvait vicieux, ou parce qu’on avait eu de fausses connaissances du terrain et de la position de l’ennemi, ou parce que l’ennemi avait fait des changements dans sa disposition. Comment y remédier ? Le moyen de changer sa disposition primitive dans des armées sans tactique ? Quand le général se serait senti le génie de l’entreprendre, comment oser le tenter avec des troupes et des officiers généraux incapables d’aucune grande manœuvre ? C’était une si difficile, une si lente opération alors que celle de mettre une armée en bataille ! Qu’arrivait-il encore ? C’est que l’armée employant un temps infini à passer de l’ordre de marche à l’ordre de bataille, l’ennemi pouvait à loisir juger la force des colonnes, le point vers lequel elles se dirigeaient, l’objet qu’on avait en vue et faire ses dispositions en conséquence. S’il fallait des exemples pour appuyer ce que j’avance, j’en pourrais citer en foule, et la dernière guerre m’en fournirait plusieurs.\par
Dans la tactique que j’expose, on arrive d’une manière toute différente à la formation des ordres de bataille. Veut-on, par exemple, aller attaquer l’ennemi ? Comme on peut ne pas connaître précisément la position qu’il occupe, comme quand même on la connaîtrait, on ne peut pas être sûr qu’instruit du mouvement qu’on fait sur lui, il ne fera pas quelques changements dans cette position ou dans la disposition par laquelle il compte la défendre, on met l’armée en marche dans l’ordre habituel, les colonnes étant toutes égales et formées chacune d’une division. Ainsi disposée, l’armée s’avance, le général étant en avant d’elle à la tête de l’avant-garde. On arrive à portée de l’ennemi, et alors le général détermine son ordre de bataille conséquemment à la nature du terrain, à la position qu’occupe l’ennemi, et à la disposition qu’il a prise. Il renforce ou affaiblit à cet effet telles ou telles colonnes qu’il juge à propos, fait avancer l’une, laisse l’autre en arrière, dirige celle-là vers un point, celle-ci vers un autre, donne le signal pour que l’ordre de bataille se prenne. À l’instant toutes ses troupes qui sont accoutumées à l’exécution des grandes manœuvres, qui ont des méthodes rapides de déploiements, se mettent en bataille et l’attaque commence avant que l’ennemi ait eu le temps de démêler où l’on veut le frapper, ou que, s’il l’a démêlé, il ait eu le plaisir de changer sa disposition pour y parer. Mais que ne peut point encore le général, ayant derrière lui toutes ses colonnes, les tenant pour ainsi dire dans sa main, et prêtes à prendre les dispositions qu’il leur indiquera ! Arrivé à la vue de l’ennemi, et ne trouvant pas que celui-ci soit en posture désavantageuse, il manœuvre vis-à-vis de lui, il cherche à lui donner le change. Il emploie toutes les ressources de terrain et de la tactique pour lui faire illusion sur son projet. Il feint un mouvement offensif sur sa gauche pour former son attaque réelle sur sa droite. Là il lui présente des colonnes à distances ouvertes. Ici il lui en présente à distances serrées. Il fait tant, en un mot, que si cet ennemi n’est pas aussi habile que lui, il prend le change, abandonne ou occupe un poste qui le met en prise, ou bien s’affaiblit sur un point, soit en y laissant trop peu de troupes, soit en y laissant trop peu de l’arme propre à le défendre, soit en y laissant les troupes les moins bonnes de son armée. Alors cette faute est saisie, le général habile et manœuvrier porte sur le champ ses efforts sur cette partie faible. Si pourtant l’ennemi ne se met en prise, ni par sa position ni par sa disposition, alors ce général se trouve n’avoir rien engagé. Il se retire, prend une position et attend une occasion plus favorable. Voilà quelle est la véritable science de la formation des ordres de bataille. Voilà celle qui a fait gagner au roi de Prusse les batailles de Lissa, de Hohenfriedberg et plusieurs autres. Voilà la science dont je vais développer les principes, en montrant les grandes combinaisons et le mécanisme intérieur qui doivent faire passer une armée de son ordre de marche à une disposition de combat.\par
Il ne peut y avoir que deux manières de donner bataille à l’ennemi. La première, en engageant ou en se mettant à portée d’engager à la fois le combat sur toutes les parties de son front. La seconde, en attaquant seulement sur un ou plusieurs points. Je crois d’après cela pouvoir réduire les sept ordres dont Végèce a parlé et dont tous les tacticiens ont parlé après lui, à deux, le parallèle et l’oblique. Je vais, en traitant séparément de chacun de ces ordres, les définir, assigner leurs principes, leurs objets et démontrer comment toutes les dispositions quelconques tiennent à ces deux dispositions principales et n’en sont que des conséquences et des modifications.
\subsection[{VIII. - Ordre parallèle}]{VIII. - Ordre parallèle}
\noindent Il faut appeler ainsi une disposition de bataille dont le front développé parallèlement à celui de l’ennemi, peut entrer en action à la fois de toutes les parties qui la composent. Quand je dis parallèlement, on ne doit pas entendre ce mot dans la précision géométrique, car il y a peu de pays qui puissent permettre à deux armées de s’étendre sur des fronts exactement parallèles l’un à l’autre. Le nom d’ordre parallèle appartient donc à toute disposition qui place tous les corps de deux armées les uns vis-à-vis des autres, en mesure et à portée de combattre.\par
Voilà certainement comme ont dû se disposer les armées dans les premiers âges de la science militaire. Alors elles n’étaient pas si nombreuses qu’aujourd’hui. Elles se formaient sur une ordonnance moins étendue. On était armé de manière à avoir besoin de s’approcher pour se nuire. On ne connaissait point toutes les finesses de la tactique. En raison de ce qu’on était moins éclairé, on était plus courageux peut-être. Chacun voulait combattre. Chacun voulait avoir part au danger et à la gloire. De là ces batailles si terribles et si sanglantes, que nos combats actuels, qui ne sont que des jeux auprès d’elles nous les font presque regarder comme fabuleuses […].\par
On voit que l’ordre parallèle étant le plus naturel et le plus simple, a dû être la plus ancienne disposition connue. Ce ne sont pas les mots qui font les choses et quoique des sauvages ne connaissent peut-être ni le mot ordre ni certainement le terme parallèle, c’est cette disposition informe et d’instincts qu’ils prennent pour aborder tous à la fois l’ennemi et le combattre. C’est elle qui se perfectionnant peu à peu et les mots naissant avec les idées, est devenue et a été nommée ordre parallèle.\par
À mesure que les hommes s’éclairèrent, les armées supérieures en nombre durent chercher à tirer parti de leur supériorité et pour cet effet à envelopper l’ennemi et à embrasser ses flancs. De là cette disposition en forme de croissant qui subsiste encore aujourd’hui dans les armées turques et asiatiques. D’un autre côté, des généraux habiles, se trouvant à la tête d’armées inférieures, durent chercher les moyens de suppléer à cette infériorité par la perfection de la tactique. Ils durent sentir qu’en se présentant parallèlement à un ennemi supérieur en nombre, ils s’exposaient à être enveloppés et battus : qu’il y avait telle autre sorte de disposition, telle science de mouvements au moyen de laquelle ils pouvaient porter l’élite de leurs forces à un des points de l’ordre de bataille, n’engager le combat que sur ce point et mettre hors de prise toutes les autres parties de leur disposition. De là l’ordre oblique et toutes les autres dérogations à l’ordre parallèle. Enfin entre les généraux tant soit peu éclairés de part et d’autre, l’ordre parallèle cessa d’avoir lieu dans les batailles, parce que, supérieurs ou inférieurs en nombre, ils calculèrent avec raison qu’il y avait d’autres dispositions plus avantageuses […].\par
Ce qui contribue maintenant à faire rejeter généralement l’ordre parallèle, c’est, outre l’immense front des armées et la difficulté de se joindre, la nécessité où sont tous les États de ne pas compromettre au hasard d’une action générale des armées qui sont toutes leurs forces et leurs destinées. Aujourd’hui qu’aucune nation n’est guerrière ni par ses mœurs ni par sa constitution. Aujourd’hui que les peuples n’ont pour défense qu’un certain nombre de troupes, que hors ces troupes tout le reste des citoyens n’est qu’une multitude lâche, sans aucune idée de guerre et de discipline, prête par conséquent à subir le joug du vainqueur : la politique respective des gouvernements veut que les généraux ne donnent rien au hasard. On vient de voir comment le résultat des ordres parallèles mis en exécution un jour de bataille, était de rendre l’action générale, comment elle devenait plus terrible, plus décisive, plus sanglante, comment il se pouvait qu’elle entraînât la destruction totale des vaincus. Qu’on se peigne la détresse d’une de nos nations prétendues policées, si on venait lui dire, comme on le dit aux Romains après la journée de Cannes : « L’ennemi arrive, l’armée qui couvrait la capitale a engagé une bataille générale et cette armée n’est plus ».\par
L’ordre parallèle pris dans la signification que je lui ai donnée au commencement de ce chapitre, n’est donc plus aujourd’hui mis à exécution dans les batailles. Mais ce nom peut rester à la disposition primitive et habituelle d’organisation et de campement d’une armée, puisque toutes les parties de cette disposition se trouvent d’égale force et prêtes (les obstacles du terrain mis à part) à entrer en action avec l’ennemi s’il venait attaquer à la fois tout le front. Je vais démontrer dans le chapitre suivant quelle espèce d’ordre de bataille a remplacé l’ordre parallèle et les changements les plus avantageux qui peuvent encore être faits à cet égard.\par
Cependant je dois dire, avant que de quitter cet article, qu’il pourrait y avoir des occasions où une armée supérieure en courage et sûre de ne pas être prise par ses flancs, pourrait se servir de l’ordre parallèle. Ce qu’il y a de certain du moins, c’est que les batailles qu’une pareille armée gagnerait dans cette disposition, ruineraient l’armée qui lui serait opposée, tandis que les batailles actuelles entre deux généraux habiles, ne peuvent jamais avoir de grands résultats.
\subsection[{IX. - Ordre oblique}]{IX. - Ordre oblique}
\noindent On vient de voir comment la science militaire a substitué l’ordre oblique à l’ordre parallèle, et a rendu les batailles plus savantes et moins sanglantes. C’est un jeu de calcul et de combinaison qui a succédé à un jeu de hasard et de ruine. Il est heureux que la science militaire, qui est la science de la destruction, rende la guerre moins destructive en se perfectionnant, il est heureux que ce puisse être l’habileté des généraux qui décide le sort des batailles, plutôt que la quantité de sang répandu. Enfin, dans un siècle où tous les arts ont fait des progrès, il est honorable, il est encourageant pour les militaires, que celui de la guerre se ressente de la propagation générale des lumières.\par
L’ordre oblique est l’ordre de bataille le plus usité, le plus savant, le plus susceptible de combinaisons, l’ordre dont se serviront toujours les armées inférieures commandées par de bons généraux. C’est cet ordre si fameux chez les Anciens, mais dont aucun de leurs tacticiens ne nous a fait connaître le mécanisme intérieur. Le roi de Prusse est le premier moderne qui l’a exécuté par principes, et qui l’a adapté à la tactique actuelle.\par
Pour qu’un ordre de bataille soit oblique, il n’est pas nécessaire que le front de cet ordre dessine exactement une ligne oblique par rapport au front de l’ennemi, car rarement les terrains et les circonstances permettent qu’une pareille régularité puisse avoir lieu. J’appelle donc oblique toute disposition où l’on porte sur l’ennemi une partie et l’élite de ses forces et où l’on tient le reste hors de portée de lui. Toute disposition en un mot, où l’on attaque avec avantage un ou plusieurs points de l’ordre de bataille ennemi, tandis qu’on donne le change aux autres points et qu’on se met hors de mesure de pouvoir être attaqué par eux.\par
La chose ainsi entendue, presque toutes les batailles qui se sont données depuis un siècle, ont été dans l’ordre oblique, car elles se sont toutes réduites à des points d’attaque. Mais cet ordre était pris au hasard et dicté par les circonstances ou par la nature du terrain. On n’avait point approfondi ses avantages, on ne connaissait pas ses finesses, on ignorait la manière de le prendre rapidement sur un point indiqué par les circonstances du moment et non prévu dans l’ordre de marche. Ainsi dans un art qui est au berceau, il arrive qu’on se sert machinalement d’un instrument dont on ne connaît ni les propriétés ni l’usage.\par
Pour développer parfaitement la théorie de l’ordre oblique, il faut entrer dans des détails qui fassent concevoir pas à pas ses principes, leur objet et leur application. Je suis, plus que personne, ennemi des longueurs. Mais dans les ouvrages dogmatiques il faut convaincre et pour convaincre il faut quelquefois savoir s’appesantir.\par
Je distinguerai d’abord deux différentes espèces d’ordre oblique : l’une est l’ordre oblique de principe, l’oblique proprement dit, c’est-à-dire l’ordre dans lequel l’armée est disposée véritablement obliquement au front de l’ennemi. L’autre est l’ordre oblique de circonstance, c’est-à-dire l’ordre dans lequel l’armée, quoique n’étant point disposée obliquement au front de l’ennemi, se trouve cependant, soit par la nature du terrain, soit par l’habileté de ses mouvements dans le cas de l’attaquer sur un ou plusieurs points et d’être elle-même hors de prise sur les autres. Je vais parler successivement de chacune de ces espèces et faire sentir leur différence.\par
L’ordre oblique, proprement dit, peut s’exécuter de deux manières, par ligne ou par échelons […].\par
De ces deux manières de prendre l’ordre oblique par lignes, ou par échelons, la première est élémentaire et purement de méthode.\par
Il est bon de l’exécuter dans un camp d’instruction, afin de commencer à faire connaître aux officiers généraux ce que c’est que l’ordre oblique et quel est son objet. La seconde, qui n’est qu’une suite de la première, est plus simple, plus facile dans son déploiement, plus applicable à tous les terrains, plus susceptible de manœuvre et d’action lorsque l’ordre est formé. C’est celle dont il faut se servir à la guerre, surtout quand on forme par brigades ou par gros corps, les échelons destinés à se refuser à l’ennemi ou à lui faire illusion […].\par
Ainsi j’appelle la disposition de la bataille de Lissa une disposition oblique, quoique certainement l’armée […]du roi de Prusse ne fût pas rangée obliquement au front des Autrichiens. Mais il attaqua leur aile gauche avec l’élite de ses forces, la prit à revers et la culbuta, tandis qu’il profitait d’une lisière de hauteurs qui était vis-à-vis de leur droite et de leur centre, pour leur faire illusion les tenir en échec et y placer, dans une excellente défensive, le reste de son armée affaibli par les renforts qu’il avait portés à sa droite […].\par
L’ordre oblique de la seconde espèce, étant celui qui s’adapte le plus facilement aux terrains et aux circonstances, c’est donc de celui-là particulièrement que les généraux doivent faire un objet d’étude et de méditation. Où cette étude peut-elle se faire avec succès ? C’est dans des camps d’instruction, c’est à la guerre, c’est, si je peux m’exprimer ainsi, à force de manier les troupes et les circonstances. J’ai posé quelques principes où il n’y en avait aucun, c’est au génie à en faire l’application. Il ne me reste maintenant qu’à prouver la vérité de ces principes. C’est ce que je vais faire, en traitant ci-après de la formation des armées et du rassemblement d’un camp d’instruction, dans lequel s’exécuteront tous les ordres de marche et de bataille, relatifs à la théorie que j’ai établie.
\subsection[{X. - Formation des armées. Nécessité d’en rassembler en temps de paix dans des camps destinés à être les écoles de la grande tactique}]{X. - Formation des armées. Nécessité d’en rassembler en temps de paix dans des camps destinés à être les écoles de la grande tactique}
\noindent Si une nation avait des troupes et des généraux tels que je me les imagine, ses armées pourraient être bien moins nombreuses que ne le sont celles qu’on a aujourd’hui et avec cela valoir mieux et exécuter de plus grandes choses. Elle aurait dans ses armées moins de cavalerie, moins de troupes légères, moins d’artillerie. Son infanterie serait mieux armée, plus aguerrie, mieux disciplinée, plus manœuvrière. Elle saurait se suffire à elle-même comme l’ancienne infanterie des légions romaines. Sa cavalerie serait peu nombreuse, mais sa bonté, sa vélocité, sa science de mouvements suppléeraient à son petit nombre. Ses troupes légères feraient en même temps le service de ligne et ses troupes de ligne feraient au besoin le service de troupes légères. Par conséquent point de double emploi, point de corps inutilement et dispendieusement employés à un seul objet. Son artillerie serait peu nombreuse, mais elle n’aurait que des calibres utiles et propres à produire de grands effets. Elle serait bien constituée, bien allégée, bien attelée, bien disposée dans ses emplacements, bien exécutée dans l’action. Tous les corps qui composeraient ses armées auraient une tactique simple, analogue l’une à l’autre et prête à servir les combinaisons des généraux. De pareilles armées ne seraient point embarrassées par une quantité immense d’équipages. Elles seraient sobres, infatigables, plus amoureuses de gloire que de commodité. Elles sauraient vivre des denrées du pays et ne seraient pas subordonnées aux calculs étroits et routiniers d’un entrepreneur de subsistances. Enfin de semblables armées, commandées par de grands hommes, renouvelleraient les prodiges opérés autrefois par de petites armées contre des multitudes ignorantes. Elles feraient encore de grandes conquêtes et des révolutions dans les empires.\par
Dans mes essais particuliers sur la tactique de l’infanterie, de la cavalerie et de l’artillerie, dans mon chapitre sur les troupes légères, j’ai déjà exposé une partie de mes idées sur les changements qui pourraient être faits dans la manière actuelle de faire la guerre. Je développerai et appuierai de plus en plus mes opinions à cet égard. On vient de voir dans le commencement de cette seconde partie, quelle doit être la théorie de la grande tactique. À l’aperçu des nouveaux procédés d’ordres de marche et d’ordres de bataille qui y sont déduits, on peut commencer à juger que la tactique est une science, une grande science, que c’est la supériorité de nombre qui doit décider les succès. La grande tactique mise en action, ainsi qu’elle va l’être ci-après, présentera cette vérité dans tout son jour.\par
C’est une chose bien étrange que la manière dont on forme aujourd’hui les armées. La guerre se déclare. On résout dans le cabinet des ministres, qu’il faut attaquer l’ennemi sur tel point et se défendre sur tel autre. Voilà par conséquent des armées à former, les généraux à choisir. Comment cela se fait-il ? Le département de a guerre, si c’est ce département qui a la prépondérance du crédit dans le conseil du souverain, propose une armée en Allemagne et ne en Flandre. On observera que souvent le ministre, qui est à la tête de ce département, ne sait pas ce que c’est qu’une armée, ou que, s’il est militaire, rarement il arrive qu’il ait commandé des armées, encore plus rarement qu’il les ait bien commandées. Par conséquent il ne peut asseoir un plan de guerre avec connaissance de cause. Cependant ce plan se fait. On se résout à former deux armées. On décide, je suppose, qu’on agira offensivement en Flandre et qu’on restera sur la défensive en Allemagne. Comment se détermine la force de ces deux armées ? On spécule quelle sera la quantité de troupes que l’ennemi pourra opposer dans chacun de ces points. On dit, […]l’ennemi aura une armée de soixante mille hommes en Flandre, faisons-en une de quatre-vingts et agissons offensivement dans cette partie. Il en a une de soixante en Allemagne, formons-y en une de quarante et tenons-nous y sur la défensive. On nomme ensuite les corps qui doivent composer les armées. Une méchante règle de proportion ou plutôt de routine, veut que l’armée étant de tant de milliers d’hommes, il y ait tant d’infanterie, tant de cavalerie, tant de troupes légères, tant d’artillerie. On choisit les généraux. On entre en campagne. Les généraux la plupart du temps, comptant sur le nombre bien plus que sur la science, n’ont ni paix ni relâche qu’ils n’aient obtenu des renforts. C’est aujourd’hui pour couvrir un point à la protection duquel leur armée ne peut atteindre. Demain ce sera pour s’opposer à une diversion, qui souvent n’aurait pas eu lieu s’ils avaient serré la mesure à l’ennemi. Cette fois c’est parce que l’ennemi a trois cents pièces de canon et qu’ils n’en ont que deux cents. Une autre fois, c’est parce qu’il a quinze mille hommes de troupes légères et qu’ils n’en ont que dix. Ils ne sentent pas qu’ayant moins d’artillerie ils ont moins d’embarras, que leurs deux cents bouches à feu bien employées équivaudraient facilement aux trois cents de l’ennemi, que pour rendre ces dernières inutiles il n’y a qu’à faire vis-à-vis de lui une guerre de marches et de mouvements. Ils ne sentent pas que l’ennemi ayant quinze mille hommes de troupes légères et ces troupes légères étant constituées comme elles le sont aujourd’hui, il est affaibli par cette espèce de troupes, qu’il n’y a qu’à, pour lui ôter cet avantage apparent, éviter la guerre de détails et la faire toujours en masse. Ils ne sentent pas enfin que le grand art de la guerre, c’est de suppléer au nombre plutôt ne de l’augmenter, d’engager les actions avec l’arme dans laquelle on est supérieur et d’appuyer ou de refuser celle dans laquelle on est le plus faible. Réciproquement et en se modelant les unes sur les autres, les armées s’augmentent donc à un tel point, que les généraux ne savent plus comment les manier, les pays comment les nourrir, les gouvernements comment les entretenir et les payer. Des généraux plus éclairés seraient même obligés de se conformer à la routine établie et de demander des armées nombreuses. Car est-il en Europe des troupes citoyennes, des troupes qui, par leur constitution, leur esprit, leur valeur, leur sobriété, leur aptitude aux travaux, leur science de manœuvres, soient si décisivement supérieures à celles des États voisins, qu’on puisse dire, avec quarante mille hommes j’oserai tenir campagne et campagne offensive contre soixante mille ! Y a-t-il des troupes qui aient assez de confiance dans leur courage, dans leur tactique, dans leurs généraux pour regarder comme un embarras et un affaiblissement tout nombre au-delà des proportions raisonnables, pour ne pas être étonné d’entrer en campagne vis-à-vis d’une armée supérieure ? Y a-t-il en Europe des généraux auxquels les gouvernements abandonnent assez d’autorité pour qu’ils puissent à l’avance acquérir cette confiance et l’inspirer, en formant à cet effet des troupes pendant la paix, en les faisant, si je peux m’exprimer ainsi, à leur système et à leur main ? Si par hasard ii s’élève dans une nation un bon général, la politique des ministres et les intrigues des courtisans ont soin de le tenir éloigné des troupes pendant la paix. On aime mieux confier ces troupes à des hommes médiocres, incapables de les former, mais passifs, dociles à toutes les volontés et à tous les systèmes, plutôt qu’à cet homme supérieur qui pourrait acquérir trop de crédit, résister aux opinions qu’on aurait adoptées, se rendre le canal des grâces militaires du souverain et devenir enfin l’homme des troupes, le général né. On veut pouvoir donner des armées à commander à ses créatures. On veut accoutumer les troupes à recevoir aveuglément tel homme que ce soit, que l’on voudra mettre à leur tête, je dis tel homme que ce soit, pourvu qu’il ait le brevet du souverain. La guerre arrive, les malheurs seuls peuvent ramener le choix sur le général habile. On l’emploie, mais en même temps on le contrarie, on le traverse. Le dirai-je ? On voudrait (si un tel partage était possible) que la besogne réussît et que le général échouât. Ce général parvient à réparer les affaires, à les soutenir. Bientôt on craint sa réputation, on est importuné de sa gloire. On fait la paix. Le général déjà formé ou qui commençait à se former, n’est plus consulté, plus employé. Ses talents se rouillent ou n’achèvent pas de se perfectionner. Les troupes qu’il connaissait, changent, se renouvellent, prennent d’autres institutions, d’autres principes. Ainsi quand des malheurs nouveaux le replacent à la tête des armées, il se trouve étranger à ces armées et ces armées lui sont étrangères. Ce tableau est l’histoire de presque tous les États dans presque tous les temps. Qu’on ne m’accuse pas d’avoir voulu en désigner aucun.\par
Quelle différence de cette manière de former les armées à celle dont les Grecs, les Romains, dont tous les grands conquérants ont formé les leurs ! Miltiade, Thémistocle, Épaminondas comptaient-ils les forces de l’ennemi ? Alexandre compara-t-il les siennes avec celles de l’Asie, quand il voulut la conquérir ? Il partit avec une armée de cinquante mille hommes pour aller détrôner un roi qui pouvait en armer des millions. Annibal partit avec soixante mille hommes pour la conquête de l’Italie. Scipion avec cinquante mille pour attaquer Carthage. César avec quelques légions, soumit les Gaules, l’Afrique et une partie de l’Asie. Et, […]pour citer un seul moderne. Gustave, avec vingt-cinq mille Suédois, fut la terreur de l’Empire. Ces grands hommes savaient bien qu’ils allaient attaquer des armées supérieures. Ils savaient qu’on leur opposerait et plus de troupes qu’ils n’en avaient, et quelquefois des armes et des manières de combattre inconnues à leurs soldats. Mais ils avaient leur plan, leur tactique, leurs armées élevées par eux et pleines de confiance en eux. Dans la tête du petit nombre d’hommes qui les suivaient, était profondément gravé, que c’est la science et le courage qui donnent la victoire et non la multitude […].\par
Mais pour qu’un général puisse oser s’écarter de la routine établie et introduire un nouveau genre de guerre, il faut, je le répète, qu’il ait d’excellentes troupes. Il faut que, si elles ne sont pas composées de l’élite des citoyens et que la constitution de l’État soit telle que le gouvernement n’y puisse et n’y veuille rien changer, elles réparent du moins ce vice primitif par toute la perfection possible dans leur constitution intérieure, dans leur discipline et dans leur tactique. Il faut que le temps de la paix soit mis à profit pour les former, pour instruire elles et les hommes qui doivent les commander. Les camps, que je vais proposer, rempliront, je crois, cet important objet.\par
C’est une idée bien ancienne que celle de former des camps de paix. Les Romains étaient dans cet usage. Leurs légions campaient presque toute l’année. Au moyen de cette institution, la discipline de ces légions survécut quelque temps à la corruption de l’empire. Mais peu à peu le luxe pénétra ces camps. Il y relâcha la discipline. Il les peupla d’histrions, de courtisanes, d’ouvriers, de marchands, de toutes les professions nécessaires à la mollesse et à la débauche. Il en fit des villes et alors les vertus guerrières n’ayant plus d’asile, c’en fut fait d’elles et de l’empire.\par
Aucune nation n’a imité les Romains. Aussi aucune milice n’a égalé la leur, Louis XIV et Auguste I\textsuperscript{er} ont formé des camps de paix. Mais c’étaient uniquement des camps de parade. Ces princes cherchaient l’occasion de donner des fêtes d’un nouveau genre. Ils faisaient ostentation de leurs troupes comme des dorures de leurs palais. Le roi de Prusse est le premier moderne qui ait formé des camps d’instruction, qui ait fait servir ces camps à exécuter des marches, des ordres de bataille et à former des généraux. On voit le fruit qu’il en a retiré et cependant quelle différence de ces camps de quinze jours et exclusivement destinés à rendre des troupes manœuvrières, à ces camps stables où les Romains bravaient les saisons, remuaient la terre, pliaient à la guerre leurs corps et leurs esprits !\par
Pendant la paix dernière on a formé aussi des camps en France. Mais on n’avait pas alors les premières notions de la tactique. On faisait bonne chère, on manœuvrait pour les dames, on se séparait sans avoir rien appris. Pendant cette paix nous formons tous les ans des camps, et ils ne sont guère plus utiles. Le temps s’y passe en revues et en exercices de détail. C’est à qui y paraîtra avec les armes les plus brillantes, les soldats les mieux tenus. C’est à qui y surprendra le plus adroitement de petits suffrages et de grosses pensions. On n’y exécute point de manœuvres du grand genre et propres à former des officiers généraux. On brigue pour y venir ou pour y revenir l’année suivante. Si, au milieu de ces futilités, quelques officiers plus éclairés élèvent la voix pour dire que ces camps ne remplissent pas l’objet, qu’il faut rassembler une armée et l’instruire aux grandes opérations de la tactique, on leur répond, ou qu’il n’en est pas encore temps, ou que les officiers généraux ne sont pas faits pour venir à l’école.
\subsection[{XI. - Projet d’un camp d’instruction : Composition et division de l’armée qu’on propose d’y rassembler}]{XI. - Projet d’un camp d’instruction : Composition et division de l’armée qu’on propose d’y rassembler}
\noindent Si les troupes étaient constituées ainsi qu’elles devraient l’être, je parlerais de former des camps à l’instar de ceux des Romains. Des camps où éloignées des villes et des vices, elles fussent dans l’exercice continuel des travaux de la guerre et où l’on pût leur faire faire un cours complet d’éducation. Car c’est cet entrelacement de vie citadine et de vie militaire qui rend nos troupes molles et peu propres aux grandes choses. C’est lui qui détourne nos officiers de l’étude. C’est Paris surtout qui est le tombeau des talents. Là les caractères s’atténuent, les courages s’énervent, les mœurs se corrompent, l’application se relâche. Là on ne prend que des idées de fortune au lieu d’idées de gloire, et c’en est fait de l’honneur, de toutes les vertus, de l’État par conséquent, quand l’ambition des particuliers a pris ce cours funeste.\par
Mais en attendant qu’une révolution presque miraculeuse opère ce changement, il faut proposer des choses qu’on puisse et qu’on veuille exécuter. Il faut, ne pouvant faire des troupes citoyennes et parfaites, les rendre du moins disciplinées et instruites. Je propose donc de former tous les ans des camps de trois mois seulement et d’y rassembler des armées composées, divisées, organisées comme elles devront l’être à la guerre. Celle dont je donne le tableau ci-après et que je vais employer à l’exécution de toutes les opérations de la grande tactique, sera une armée de cinquante mille hommes et par conséquent une armée du second ordre. L’instruction de celle-là pourra facilement s’appliquer à une moins nombreuse ou à une plus considérable, à une de soixante ou soixante-dix mille hommes que je regarde comme une armée du premier ordre et comme le nombre qu’il ne faut jamais passer […].\par
Les quatre-vingt bataillons seront partagés en trois divisions appelées divisions de droite, de gauche et de centre. Chaque division, composée de vingt-quatre bataillons, dont douze en première et douze en seconde ligne, sera commandée par un lieutenant général qui aura sous lui un second lieutenant général et trois maréchaux de camp.\par
Les huit bataillons restants seront divisés en deux brigades de quatre bataillons, appelées brigades de flanc. Chacune d’elles, commandée par un maréchal de camp, campera en potence sur le flanc de la cavalerie et dans la disposition de combat se placera où le général la jugera le plus nécessaire.\par
Les deux ailes de cavalerie formeront chacune une division de quarante escadrons, dont vingt en première et vingt en seconde. Cette division sera commandée par un lieutenant général et quatre maréchaux de camps.\par
Oui pourra croire que les Anciens aient connu cet ordre de division et que nous ayons été si longtemps à l’appliquer à nos armées devenues cependant si compliquées et si nombreuses ? On lit dans Quinte-Curce que l’armée d’Alexandre était partagée en plusieurs divisions. Il nous dit leur nombre, leur force, et les généraux qui les commandaient. Cela prouve que beaucoup de gens lisent sans fruit et que les choses simples et grandes ne frappent pas la plupart des hommes.\par
Les troupes légères seront campées en avant et sur les ailes de l’armée. Elles auront l’avant-garde des mouvements que fera l’armée ou couvriront ses flancs. Elles tiendront rang dans la disposition de combat, et serviront ordinairement à renforcer les ailes et à menacer les flancs et les derrières des ennemis. On les fera soutenir au besoin par les dragons tirés de la ligne et par des corps d’infanterie.\par
L’artillerie sera partagée en trois divisions, chacune attachée à une division d’infanterie et composée de trente-six bouches à feu, formant six subdivisions de six pièces chacune, de manière que chaque division soit composée d’un nombre égal de pièces de même calibre. Une autre subdivision de six pièces sera parquée en avant de l’armée et sera appelée subdivision d’avant-garde. Le reste de l’artillerie, dont les canons de seize et la moitié des obusiers, formera la division de réserve et la tête du gros parc, où seront toutes les voitures d’attirail et de dépôt […].\par
Obligé de me conformer ici au plan de notre constitution actuelle, je subdiviserai l’infanterie en brigades de quatre bataillons chacune et la cavalerie en brigades de huit escadrons ou de deux régiments. Si le plan de constitution, que je proposerai dans mon grand ouvrage, avait lieu, cette division serait encore plus simple. Chaque régiment d’infanterie\footnote{Les régiments étaient alors composés de deux bataillons.}, composé de trois bataillons, formerait une brigade et ainsi chaque régiment de cavalerie, que je composerais de sept escadrons. Mais c’est au reste une chose bien peu importante en elle-même que cette subdivision, et quelle qu’elle fût, la grande tactique, que je vais exposer, saurait en tirer parti.\par
Ce qui est plus important et ce sur quoi il faut que je revienne, c’est le nombre des officiers généraux. On voit combien je m’écarte de l’usage où nous sommes d’en surcharger les armées. Je trouve qu’il ne faut, pour commander les divisions d’une armée comme celle du camp d’instruction, que dix lieutenants généraux et vingt maréchaux de camp. Suivant cette proportion, chacun commande un nombre de troupes convenable à son grade. Le lieutenant général commandant une division et celui qui est sous lui, ont à leurs ordres vingt-quatre bataillons. Les maréchaux de camp, qui sont sous eux, en ont chacun huit. Dans la cavalerie, deux lieutenants généraux ont quarante escadrons à leurs ordres et les maréchaux de camp, qui sont sous eux, en ont chacun dix. Je veux maintenant que pour les occasions extraordinaires, comme commandement d’avant-garde, de détachement, commandement dans les places, commission particulière, remplacement des officiers généraux qui viendraient à manquer, on porte, en temps de guerre, ce nombre à douze lieutenants généraux et vingt-quatre maréchaux de camp.\par
En n’employant ainsi strictement que le nombre nécessaire d’officiers généraux, on diminue la quantité immense d’équipages et d’embarras qui est à la suite de nos armées. On se trouve forcé de choisir avec plus d’attention les officiers généraux qu’on emploie. Ceux qu’on choisit, ayant des commandements plus étendus et des occasions plus fréquentes, s’instruisent plus facilement. De là ces grades éminents prennent la considération qu’ils doivent avoir et les troupes s’accoutument à les respecter. Aujourd’hui, à peine les regardent-elles. Tant de gens en sont revêtus. Il y en a tant à la suite des armées et partout ! Tant de gens, en un mot, traînent ces grades à la honte du militaire, ou à son détriment.\par
Il y aurait bien des réflexions à faire à cet égard. Il y en aurait bien à faire sur ce que nous appelons états-majors d’armée. Je pourrais prouver que ces derniers, tels que nous sommes dans l’usage de les former, sont compliqués, contraires au secret des opérations et à la simplicité du service. Je pourrais prouver que la plupart du temps ils sont composés des créatures des généraux et des ministres, plutôt que des hommes de métier. J’aurais à parler surtout de la classe de ces états-majors à laquelle sont confiés les détails des marches, des reconnaissances, des subsistances, etc. On est, sur ces grands objets, sans principes et sans théorie. On procède par routine. Il faudrait que le maréchal général des logis d’une armée et ses principaux aides fussent des officiers consommés dans la grande tactique et qui à l’ensemble des vues générales joignissent la connaissance des détails. Il faudrait que ces sortes d’emplois fussent non des places à jeunes gens et des débouchés pour arriver, mais des places de perspective auxquelles des talents prouvés pussent conduire. Il faudrait enfin que ces emplois subsistassent en temps de paix, qu’ils fussent mis en activité dans les camps d’instruction ; que pendant le reste de l’année, on leur assignat des commissions et des courses relatives à leurs fonctions. Tout cela vaudrait la peine d’être approfondi et cela le sera dans la suite de cet ouvrage.\par
J’ai parlé, dans la théorie des ordres de bataille, d’une avant-garde qui devait marcher à la tête de l’armée et à la faveur de laquelle le général devait déterminer les mouvements de ses colonnes et la disposition qui serait à prendre. Je dois expliquer ici plus particulièrement quel est l’objet de cette avant-garde et comment elle doit être composée.\par
On appelle généralement avant-garde tout corps placé en avant de l’armée et destiné à précéder ses mouvements. Quelquefois on a des avant-gardes détachées de l’armée et qui la précèdent de quelques lieues. Il y a des opérations où ces corps détachés peuvent être utiles. Mais en général il faut éviter de morceler ainsi les armées. Car est-on supérieur à l’ennemi ? On se remet, par ce morcellement, de niveau avec lui. On s’expose à faire battre ces corps détachés et à perdre en détail l’avantage qu’on aurait eu, si l’on fût resté en masse. Est-on inférieur à l’ennemi ? On doit à plus forte raison faire la guerre sans se morceler. En se divisant ainsi, on se réduit à être partout sur la défensive, partout dans l’inquiétude, partout exposé aux échecs et aux coups de main. Est-on enfin en offensive décidée et généralement dans quelque opération manœuvre que ce soit ? Il faut rappeler à soi tous ses corps détachés, ses troupes légères même et se tenir ensemble. En effet, si l’on veut attaquer, pourquoi se découvrir, s’annoncer, se mettre en prise sur quelque point ? Il serait à désirer que, comme la foudre a déjà frappé lorsqu’on voit l’éclair, quand l’ennemi voit arriver la tête de l’armée, toute l’armée fût là, et qu’il ne fût plus à temps de parer la disposition qu’elle va prendre. Si l’on est en défensive, si l’on craint d’être attaqué, est-il une meilleure disposition que celle d’être réuni et prêt à faire résistance où l’ennemi voudra faire effort ? […].\par
Il me reste à parler du choix des emplacements et de la dépense des camps d’instruction. Les premiers peuvent se trouver facilement. Il n’y a point de province du royaume qui n’en offre. Il y en a peu, où il n’y ait de grands terrains incultes. Mais il faudrait de préférence choisir les provinces de l’intérieur, les provinces qui, regorgeant de denrées, manquent de vivification et d’argent. Il faudrait surtout éloigner ces camps de la cour et de la capitale. Quant à la dépense, elle ne sera pas considérable. Elle ne le sera guère plus que celle des camps inutilement rassemblés à Compiègne. Elle le serait même bien moins, si l’on voulait réprimer le luxe des tables et l’énorme profusion des grâces pécuniaires. Si l’on voulait ramener le militaire à l’esprit de désintéressement, à l’austérité de mœurs, qui devrait faire la base de sa constitution. Enfin, en attendant que cette révolution s’opère de pareils camps coûtassent-ils deux millions chaque année, à quel avantageux intérêt ce capital ne serait-il pas placé, si la gloire de nos armes, si quelques batailles gagnées en étaient le fruit !
\subsection[{XII. - Manœuvres qui devront être exécutées dans le camp d’instruction}]{XII. - Manœuvres qui devront être exécutées dans le camp d’instruction}
\noindent J’en ai dit assez pour pouvoir maintenant passer, sans préambule, à l’exécution de tous les ordres de marche et de bataille. Je vais donc l’exposer ici en forme de journal, et dans le rang, parce qu’il faut passer de l’un à l’autre suivant la connexion qu’ils ont entre eux, exécutant d’abord ceux qui sont les plus simples et ensuite ceux qui sont les plus composés et s’attachant à répéter le même jusqu’à ce qu’il ait été parfaitement conçu par les troupes et par les officiers généraux.\par
J’expliquerai chaque manœuvre par le moyen de l’ordre de marche et de disposition, tel que je suppose qu’il devrait être donné à l’armée. Cet ordre sera accompagné d’une planche qui représentera les différents mouvements qui devront être exécutés. Il me semble que la meilleure manière d’expliquer une manœuvre, est celle de donner l’instruction dont les troupes auraient besoin pour l’exécuter […].
\subsection[{XIII. - Application des manœuvres précédentes aux terrains et aux circonstances}]{XIII. - Application des manœuvres précédentes aux terrains et aux circonstances}
\noindent De toutes les manœuvres que je viens de décrire, il n’y en aura peut-être pas une qu’on soit dans le cas d’exécuter à la guerre par des combinaisons exactement semblables à celles qui y sont détaillées. Les terrains et les circonstances changent absolument les données et à la guerre la nature des terrains et des circonstances ne pouvant souvent pas être prévue, les mouvements ne sont point prémédités, et c’est ordinairement le moment qui les détermine.\par
Quelque infinies, quelque variées que soient les combinaisons qu’on peut former, c’est cependant par le même mécanisme qu’on les exécute. J’ai dû d’abord enseigner quel était ce mécanisme isolé et sans aucune relation avec les terrains et les circonstances. J’ai dû, par conséquent, indiquer toujours à l’avance dans les ordres de marche l’espèce d’ordre de bataille auquel ils doivent conduire. Maintenant l’objet primitif et les principes des ordres de bataille étant conçus, les officiers généraux et les troupes s’étant formé le coup d’œil et l’intelligence par des manœuvres simples et toutes supputées, la sphère de l’instruction s’étendra et deviendra de plus en plus intéressante.\par
On peut supposer que les manœuvres précédentes se sont faites dans des terrains absolument nus et uniformes, qui par conséquent n’obligeraient à aucune supputation locale. Ici les exemples vont prendre plus de vraisemblance. On manœuvrera toujours relativement aux terrains et à des terrains variés, tels que le pays les offrira.\par
L’armée se mettra en marche comme à la guerre, pour se porter sur tel ou tel point et ce ne sera que de l’avant-garde et relativement à la nature du pays, que le général déterminera l’ordre de bataille qu’elle devra prendre. Car, je dois le répéter, tel est l’avantage de cette organisation de l’armée et de la disposition de ses ordres de marche, que l’armée peut, rapidement et suivant les circonstances, prendre un ordre de bataille quelconque, et renforcer ou refuser telle ou telle partie de cet ordre. Le général, marchant à la tête de son avant-garde, a derrière lui toutes ses colonnes qu’il dirige, avance, retarde, arrête et déploie suivant ses projets. Que les batailles de Condé et de Turenne eussent été savantes, s’ils avaient connu la simplicité et les ressources de ce mécanisme ! […].\par
C’est un avantage bien grand et bien peu connu dans nos armées, que celui de se tenir en colonnes jusqu’à ce que l’ordre de bataille qu’on veut prendre, soit déterminé. Par là on tient parfaitement son armée dans la main. On peut la manier rapidement, faire des mouvements intérieurs qui échappent à l’ennemi, lui faire illusion, le menacer tantôt sur un point, tantôt sur un autre, l’induire en erreur, et cependant ne jamais se mettre en prise. J’ai déjà parlé de cet avantage au chapitre des ordres de bataille. Je vais, au hasard de me répéter en partie, citer quelques exemples qui le développeront et le feront mieux sentir […].\par
Nous n’avons pas, il faut en convenir, la moindre idée de ce genre de guerre, de cette manière de reconnaître l’ennemi avec toutes les forces d’une armée, de lui présenter le combat, de l’induire à une fausse manœuvre et d’en profiter avec rapidité. Nous ne savons guère prendre que des ordres de bataille momentanés et combinés sur la circonstance. Nous ignorons, pour tout dire en un mot, l’art de manœuvrer les armées. Si nous l’avions connu, que de batailles nous avons perdues qui ne se fussent seulement pas données ! […].\par
Tous les exemples exposés ci-dessus, ou d’autres différemment combinés relativement à d’autres natures de terrains, mais tendant à enseigner les mêmes résultats, pourront être mis à exécution dans le camp d’instruction. Le général pourra, pour en exécuter quelques-uns avec plus de vraisemblance, partager l’armée en deux corps et les faire agir l’un contre l’autre d’après telle ou telle circonstance donnée. Charger, par exemple, l’officier général, commandant l’un de ces corps, d’aller choisir et occuper une position qui remplisse tel ou tel objet et l’officier général, qui commandera l’autre corps, de l’attaquer ou de le déposter.\par
Ces deux officiers généraux se conduiront chacun de leur côté suivant leurs lumières, le général se bornant, pendant tout le temps que durera l’opération, à être spectateur de leurs mouvements pour ensuite discuter avec eux ce qu’ils auront fait et ce qu’ils auraient dû faire. Mais pour cela quel homme ce devrait être que ce général ! Il faudrait dans de certains États que ce fût le souverain lui-même, afin qu’il n’éprouvât point de contradiction. Il faut dans tous qu’il soit d’une habileté assez universellement reconnue pour commander aux opinions.\par
C’est sous un tel homme et dans un camp d’instruction pareil que les officiers généraux apprendront à remuer des troupes, à calculer les distances, à saisir d’un coup d’œil l’analogie du terrain avec les différentes armes et bien d’autres principes encore qui naissent des circonstances et des situations et qu’on ne peut pas indiquer ici.\par
C’est là qu’ils apprendront qu’après que le général leur a indiqué en gros la position qu’ils doivent occuper dans l’ordre de bataille avec une division ou un corps de troupes, il reste, dans la manière d’occuper cette position, une infinité de détails qui les regardent. Qu’ils doivent savoir occuper une hauteur un peu plus avantageuse en avant ou en arrière des points donnés, mettre devant les troupes un rideau ou un ravin pour les abriter du feu de l’artillerie ennemie quand elles sont en panne, faire quelques légers changements dans l’alignement donné quand ces changements peuvent être avantageux, prendre en un mot sur eux tout ce qui, en procurant quelque avantage, ne fait pas contre sens à l’ordre de bataille, et concourt à remplir plus parfaitement l’objet de la disposition générale.\par
Dans toutes les manœuvres qui se feront dans le camp d’instruction les troupes ne s’approcheront jamais à la portée du fusil et l’on évitera toutes ces tirailleries qui ne servent qu’à mettre du tumulte et de l’invraisemblance dans les manœuvres. L’avantage sera censé demeurer à celui qui, par le choix de sa position, aura le mieux suppléé au petit nombre de ses troupes, ou qui, par ses déploiements et ses manœuvres, présentera sur un de ses points d’attaque et de défense, des moyens supérieurs à ceux de l’ennemi. Car il faut et c’est un principe bien important pour ne pas décréditer l’instruction, que dans ces camps de paix les manœuvres s’arrêtent où elles cessent de devenir vraisemblables.
\subsection[{XIV. - Application de la tactique exposée ci-dessus, aux ordres de bataille défensifs. Nécessité de faire connaître cette application aux troupes et aux officiers généraux}]{XIV. - Application de la tactique exposée ci-dessus, aux ordres de bataille défensifs. Nécessité de faire connaître cette application aux troupes et aux officiers généraux}
\noindent En considérant tous les ordres de bataille, relativement à l’objet offensif, j’ai démontré les avantages qui pouvaient résulter de la combinaison des marches et des déploiements, soit pour tromper l’ennemi sur la force des colonnes et sur le point d’attaque, soit pour prendre rapidement une disposition. J’ai fait voir que ces avantages devenaient immenses, lorsque l’armée attaquée faisait, suivant la routine ordinaire, sa disposition à l’avance, et étalait ses lignes sur la disposition qu’elle devait défendre. Alors le général attaquant arrive avec son avant-garde reconnaît cette disposition, compte le nombre et l’espèce de troupes qui défendent chaque point et détermine son ordre de bataille en conséquence.\par
Ce serait une science fort imparfaite que celle de la tactique, si elle n’offrait pas à l’armée qui est sur la défensive, le moyen de balancer ces avantages. Elle les offre et elle est en cela comme l’art des mines, comme celui de l’attaque et de la défense des places. Également susceptible d’être employée par les deux partis, c’est à celui qui la possède et qui l’applique le mieux, qu’elle rend les servies les plus décisifs.\par
Supposons un général habile et tacticien dans la nécessité de recevoir une bataille. Il ne démasquera sa disposition de défense qu’après qu’il aura reconnu les points où l’ennemi veut faire effort. Il tiendra son armée en colonnes sur le champ de bataille qu’il devra occuper, afin de ne déterminer la répartition de ses troupes que sur celle des troupes de l’ennemi. Il opposera enfin finesse à finesse et manœuvre à manœuvre. C’est-à-dire qu’il sera continuellement en mouvement devant l’ennemi, qu’il cherchera à le jeter dans l’irrésolution, à l’induire en erreur, à lui faire illusion sur le nombre et sur la disposition de ses troupes, à lui présenter un point dégarni en apparence, afin de l’engager à diriger son attaque sur ce point, c’est-à-dire même, qu’il ne se bornera pas toujours à une simple disposition défensive et que, si l’ennemi se met en prise sur quelque point, il faudra faire sur lui un contre mouvement offensif.\par
Il n’est pas question ici de ces positions défensives, tellement avantageuses que le terrain y réduise nécessairement l’attaque à un point. Car alors, comme il ne peut y avoir d’incertitude sur la partie où il est nécessaire de porter ses plus grandes forces, il n’y a pas d’inconvénient à déterminer son ordre de bataille à l’avance. Mais il n’en est pas de même dans les positions qui sont susceptibles d’être attaquées sur plusieurs points. Là, pour qu’il n’y ait pas un de ces points dégarni, dans le temps que les autres seront inutilement occupés par un trop grand nombre de troupes, pour que l’ennemi ne puisse pas engager sa partie forte contre une partie faible, il faut ne déterminer sa disposition que sur celle de l’ennemi. Il faut occuper les points d’attaque par des têtes de troupes, et tenir derrière et entre eux le reste de son armée en colonnes, afin de porter ses forces où l’ennemi portera ses efforts, et quelquefois où il se mettra en prise et se rendra susceptible d’être attaqué lui-même. Il faut à plus forte raison dans les positions de plaine, ne déterminer son ordre de bataille que sur celui de l’ennemi, puisque dans ces positions, c’est le nombre des troupes, c’est une aile plus ou moins forte, c’est telle ou telle arme rendue supérieure dans une partie de l’ordre de bataille, qui décident du succès de l’action.\par
Que fera cependant le général ennemi ? Il verra des têtes de troupes dans les principaux points de la position qu’il veut attaquer et au lieu d’une armée de bataille et disposée pour se laisser compter et battre, cette armée partagée en colonnes dont il ne pourra juger ni la profondeur ni l’objet. Manœuvrera-t-il ? Cette armée manœuvrera aussi. Cherchera-t-il à lui donner le change ? Elle se tiendra en garde contre lui, elle cherchera à lui taire illusion à son tour. Se décidera-t-il à attaquer un point et réunira-t-il ses forces pour l’emporter ? Les forces de cette armée se réuniront pour le défendre. Entre deux armées pareilles, ce sera enfin à qui l’emportera de génie et de célérité dans les manœuvres.\par
Cette application de la tactique à la défensive, est encore plus inconnue et cependant non moins importante que l’application aux ordres de bataille offensifs. J’ai dit combien, faute de cette dernière, les armées attaquantes avaient perdu de batailles […] On doit donc s’occuper essentiellement de l’ordre défensif dans le camp d’instruction. Il faut qu’on y familiarise les troupes et les officiers généraux. C’est surtout lorsqu’on partagera l’armée en deux corps, qu’on peut donner à cet égard des leçons très vraisemblables.\par
Quand je dis qu’il faut familiariser les troupes avec cette manière de prendre les dispositions défensives, c’est qu’elle les étonnerait beaucoup, si on venait à l’employer à la guerre sans que des exemples raisonnés en eussent fait connaître les avantages à la paix. On est généralement dans l’opinion qu’une armée qui doit être attaquée, ne saurait être trop tôt disposée en bataille. On est en conséquence dans la routine de former ses lignes sur la position qu’on a choisie, avant que l’ennemi ait fait sa disposition. Or ne choquerait-on pas les opinions et les usages, si l’ennemi débouchant pour attaquer une armée, le général qui la commande la portait en plusieurs colonnes sur le champ de bataille reconnu. Si là, loin de se mettre en bataille, il attendait le parti que prendra l’ennemi. Si enfin il combinait habilement qu’il lui suffit d’avoir achevé sa disposition au moment que l’ennemi arrivera sur lui ? Il n’y a que l’habitude réfléchie de ces sortes de manœuvres, qui, vu le préjugé actuel, pût rassurer les troupes contre cette contenance, qu’elles croiraient incertitude et danger.
\subsection[{XV. - Suite des objets dont on devra s’occuper dans le camp d’instruction}]{XV. - Suite des objets dont on devra s’occuper dans le camp d’instruction}
\noindent Mais combien d’autres objets d’instruction se présenteront encore dans les camps de paix ? Quelquefois l’armée exécutera de simples marches pour aller occuper une position reconnue par les officiers de l’état-major. Indépendamment de l’avantage qui en résultera pour accoutumer les troupes à la pratique et à la fatigue des marches, elles seront pour l’état-major de l’armée l’instruction la plus utile et la plus importante. Le général s’y occupera particulièrement d’examiner si les colonnes sont ouvertes et disposées, relativement à la nature du pays et aux principes établis, si la position reconnue remplit bien l’objet qu’il avait indiqué, etc. C’est ainsi qu’en appliquant la théorie aux terrains et en faisant travailler les officiers de l’état-major sous les yeux, d’un général, on peut espérer d’en former, plutôt que par des courses infructueuses, tantôt sur une frontière, tantôt sur une autre : courses où personne ne les dirige, où ils ne voient jamais que des terrains nus, où conséquemment les faits ne peuvent pas rectifier leurs idées et d’où enfin ils ne rapportent que des connaissances purement topographiques, ou des mémoires tout en suppositions, qu’on ne peut pas vérifier et qui souvent ne sont pas leur ouvrage.\par
La reconnaissance d’un pays, cette partie intéressante des fonctions de l’état-major de l’armée, ayant nécessairement relation avec la grande tactique, je rassemblerai dans un article particulier quelques idées à ce sujet. Achevons de dire ce dont on devra s’occuper dans le camp d’instruction.\par
Ce sera de figurer des attaques et des défenses de différents genres, comme de retranchements, de postes, de villages, de rivières et de défilés, de fourrages, etc., opérations toutes importantes, toutes faites pour instruire les troupes, pour développer et pour étendre les idées de ceux qui les commandent, mais sur lesquels il serait superflu de raisonner ici à l’avance, parce qu’elles dépendent entièrement de la nature du terrain et du génie du général.\par
J’ai dit sur la tactique, considérée en elle-même, tout ce que je crois qu’il y a à en dire. Je me flatte d’avoir présenté cette science sous des rapports vastes et nouveaux. Je pense qu’il serait utile et intéressant de l’enseigner dans des cours publics, telle que j’ai essayé de l’exposer.\par
Lorsque toutes les autres sciences s’étendent et se perfectionnent par des théories lumineuses, la science de la guerre sera-t-elle donc la seule qu’on abandonne à la routine ? La croit-on si vague, si dénuée de principes positifs, qu’elle ne doive pas être enseignée ? Est-ce l’indignation d’Annibal quand il entendit le rhéteur d’Éphèse donner des leçons sur l’art militaire, qui a à jamais ridiculisé le projet de le démontrer dans des écoles ? Annibal vit en pitié un rhéteur obscur et ignorant, oser parler devant lui des devoirs du général. Il eut aimé à entendre un homme de guerre, un Xanthippe, un Épaminondas raisonner de la théorie de son art. Il eût senti que dans un pays où de grands hommes commanderaient les armées pendant la guerre, il faudrait encore que pendant la paix ils prissent la peine de se former des troupes et des successeurs.\par
La nature de nos constitutions et de nos préjugés nous défend d’espérer un pareil spectacle. Regrettons du moins qu’il ne puisse avoir lieu et offrons, en attendant, à nos concitoyens le tribut de nos faibles travaux. Le cours complet de tactique que je donnerai dans mon grand ouvrage, les présentera dans un ordre plus didactique et plus instructif. J’essaierai d’y faire l’ébauche de ce qui devrait être l’objet d’un cours public. Je montrerai que cette science peut être enseignée par des procédés simples et attachants. Elle me l’a été ainsi. Qu’il me soit permis de rendre ici hommage à mon père ! Mon entendement était à peine ouvert qu’il me donnait les premières leçons de la tactique. Il me la démontrait successivement par paroles, par figures et sur le terrain. Lorsque j’eus une fois bien conçu les éléments de la tactique, il fit taire en carton découpé, des espèces de plans figurés et mobiles avec lesquels on représentait toutes sortes de terrains. Sur ces plans il m’expliquait, avec des figures de bois, tout le mécanisme des armées. Il me représentait des batailles qui pouvaient fournir des exemples qui y étaient relatifs. Il me représentait particulièrement celles de la guerre qui se faisait alors et dont les événements et les détails avaient le plus frappé mon attention. Il allait ensuite dans toutes sortes de terrains exercer mon coup d’œil et mon jugement. Nous revenions et nous reprenions notre amusement. Il me permettait des objections. Il laissait mon imagination s’essayer. Insensiblement elle acquérait plus de développement et de rectitude. Alors nous formions deux armées et nous prenions chacun le commandement de l’une des deux. Puis dans différents pays représentés au hasard par l’assemblage de nos cartons, nous faisions manœuvrer nos armées. Nous leur faisions exécuter des marches, nous choisissions des positions. Nous exécutions l’un contre l’autre des ordres de bataille. Nous raisonnions ensuite sur ce que nous avions fait. Il aimait mes doutes et jusqu’à mes contrariétés. Souvent les nuits se passaient dans cette occupation, tant cette étude nous attachait, tant mon instituteur avait su lui prêter de charmes.
\section[{Premier mémoire. Rapport de la science des fortifications avec la tactique et avec la guerre en général}]{Premier mémoire. Rapport de la science des fortifications avec la tactique et avec la guerre en général}\renewcommand{\leftmark}{Premier mémoire. Rapport de la science des fortifications avec la tactique et avec la guerre en général}

\noindent La science des fortifications et celle de la tactique sont intimement liées l’une à l’autre. C’est de la science des fortifications que la tactique défensive emprunte quelques-uns de ses principes, comme la nécessité d’appuyer les flancs d’une disposition, la nécessité d’ordonner toutes les parties de cette disposition, de manière qu’elles se protègent mutuellement : la nécessité par conséquent de réunir sur les points principaux, sur les parties les plus menacées, la plus grande quantité de feux et de forces. C’est, à son tour, sur la tactique que sont fondés les bons et véritables principes de la science des fortifications, puisque les ouvrages doivent être assis et combinés relativement à la nature du terrain, à l’espèce des troupes, à leur nombre, à leur ordonnance, à l’esprit qui les anime, à ces différents objets supputés tant du côté de celui qui défend que de celui qui attaque.\par
Il résulte de là que, pour être tacticien, il faut connaître la science des fortifications et que, pour être ingénieur, il faut être tacticien. La première partie de cette conséquence est admise et reconnue dans le militaire, sans que cependant les officiers s’éclairent en conséquence. La seconde semble ne pas l’être parmi les ingénieurs. Généralement ils ne savent ni comment les troupes manœuvrent, ni comment on doit les conduire. Ils ne veulent pas même le savoir. Regardant leur art comme le premier des arts, ils dédaignent toutes les autres branches de la science militaire. Si ce préjugé est entre- tenu chez eux par le beau nom de génie dont on a honoré leur corps et les connaissances qu’ils cultivent, je dois les avertir que cette pompeuse dénomination est de création nouvelle, que, du temps de Vauban on disait tout simplement le corps des ingénieurs, et qu’ingénieur\footnote{Autrefois engineur en français, encore aujourd’hui engineer en anglais, kunftabler en allemand, ce qui revient proprement à artificier.}, dans l’institut de cette profession, et dans toutes les langues de l’Europe, dérive, non du mot génie, mais du mot engin, parce qu’alors les ingénieurs étaient les constructeurs et les directeurs de toutes les machines de guerre et particulièrement de celles de siège.\par
C’est surtout dans la détermination des fortifications de campagnes qu’on doit sentir combien il est important que la tactique dirige les idées. Faute de cela, on procède avec lenteur, on n’ose s’écarter de la routine de méthode. On voit l’effet d’une pièce de fortification, le rapport qu’elle aura avec la pièce voisine, mais on ne s’occupe point de l’ensemble général de la position, de l’objet qu’elle doit remplir. On remue de la terre, on multiplie les ouvrages et l’on ne calcule, ni qui défendra cette immensité d’ouvrages, ni que des troupes, enfermées dans des retranchements pareils, perdent tout l’avantage que pourraient donner la manœuvre et la science.\par
Qu’on parcoure l’histoire militaire depuis un siècle, on verra toutes les erreurs dans lesquelles on est tombé, faute de n’avoir pas combiné les fortifications avec la tactique. C’est sous les généraux médiocres, c’est dans le temps où toutes les troupes de l’Europe n’avaient ni discipline ni tactique, que s’introduisit l’usage des lignes. Absurdité qui rappelle cette fameuse et inutile muraille que l’ignorance chinoise a bâtie à six mille lieues de nous. À l’usage des lignes succéda celui des grandes positions retranchées qui n’étaient, à vrai dire, que des lignes plus courtes et proportionnées au front de l’armée qui devait les occuper. Second genre de défensive moins mauvais que le premier, mais toujours funeste aux généraux qui n’en ont pas connu d’autre. Tel était alors le préjugé, qu’on ne croyait une position bien retranchée que quand les ouvrages qui la défendaient, étaient continus. Dans la correspondance des généraux de Louis XIV avec ce prince et avec ses ministres, on lit en propres termes ceci et souvent l’équivalent : « Notre position est déjà couverte de redoutes et pourvu que l’ennemi nous donne le temps de les lier, tout ira bien ». Aujourd’hui on est revenu de ce préjugé. On regarde qu’en fait de fortifications de campagne, les courtines sont inutiles et qu’elles ne doivent être formées que de troupes. C’est déjà un grand pas de fait vers la lumière, vers le véritable emploi qu’une armée doit faire des fortifications. Mais il reste encore à réduire à une plus juste valeur le mérite des retranchements d’armée et à se bien persuader que la défensive sublime consiste, non à aller toujours cherchant des positions et recevant les combats, mais à faire sans cesse craindre l’offensive à l’ennemi et pour cela à manœuvrer, à le forcer d’en faire de même et à épier le moment où quelques fautes le divisent, le retardent, le mettent en prise, pour ensuite agir offensivement sur lui. Ceci est un point trop intéressant pour que je ne le développe pas avec plus de détail.\par
Quel est le but des fortifications ? C’est de mettre une troupe inférieure par le nombre, par le courage ou par la science des mouvements en état de résister à une troupe qui lui est supérieure en quelqu’un de ces points. Donc toute fortification suppose des vues défensives et n’est par conséquent que le pis-aller de la troupe qui s’y renferme. Donc toutes les fois qu’un général se sentira la supériorité du génie et qu’il verra ses troupes plus nombreuses et plus aguerries et plus manœuvrières, il se gardera bien de mettre des retranchements devant lui. Il prendra l’offensive, il manœuvrera, il attaquera. Ou, si quelquefois il reçoit le combat, ce ne sera que parce qu’il aura mis l’ennemi dans la nécessité de le donner avec désavantage, ou parce qu’il préméditera un mouvement qui, avant le combat ou pendant le combat même, lui rendra l’offensive qu’il aura paru abandonner.\par
Voyons ce qui arrivera à un général qui, se trouvant inférieur à l’ennemi, se conduira différemment et suivant les principes usités dans les armées modernes. S’il prend le parti de construire des lignes et de se mettre derrière elles, pour peu que son ennemi sache manœuvrer, elles seront tournées, surprises, percées et je n’ai pas besoin de dire pourquoi elles le seront, tant d’exemples et de raisons reconnues rendent cette conséquence sensible. S’il se jette dans une position excellente et dont tout le front soit couvert par une continuité de retranchements, il se liera les mains, il ne sera plus en mesure de faire craindre l’offensive à l’ennemi, il jettera dans son armée l’esprit de timidité et de découragement. Il n’osera se compromettre hors de sa position. Je veux que l’ennemi ne puisse attaquer de vive force dans sa citadelle ; il le désolera par des courses sur ses flancs, sur ses communications, sur le pays qui l’intéresse, il s’approchera de lui, il le resserrera, il l’assiégera. Offensif et mobile, il prendra sur cette armée ainsi retranchée, tous les avantages que l’assiégeant a sur l’assiégé et sur des ouvrages qui sont immobiles et défensifs. Il ira à elle par tranchée, il réunira sur quelques points de cette position, tous ses feux et ses efforts. Il l’obligera, ou à l’extrémité fâcheuse d’abattre ses retranchements et de venir présenter un combat désavantageux, ou à celle de mettre bas les armes, ainsi que l’ont fait les Saxons à Pirna, ainsi que l’aurait fait Pierre, sur le Pruth sans l’adresse de la Czarine.\par
Mais je veux que, revenu avec son siècle du préjugé qui existait autrefois en faveur des lignes et des camps retranchés, il ne prenne ni l’une ni l’autre de ces défensives. S’attachera-t-il à ne se présenter à l’ennemi que dans des positions couvertes par plusieurs points retranchés, comme redoutes, batteries, villages, abattis etc., faisant en quelque sorte de son armée la courtine de ces bastions ? C’est aujourd’hui la grande routine de la défensive moderne, routine sans doute préférable à celle qu’elle a remplacée, mais sujette elle-même à beaucoup d’inconvénients. 1) En ce qu’elle réduit l’armée qui s’en sert, à la défensive et que c’est déjà une espèce d’échec, de recevoir la loi des dispositions de l’ennemi, d’être sans cesse occupé à parer et de n’être pas en mesure de lui porter coup à son tour. 2) En ce que l’ennemi ne court jamais aucun risque décisif en attaquant une armée ainsi postée. Battu, il se retire et il est rare qu’avec des précautions bien prises il craigne la poursuite. Vainqueur, il peut rendre sa journée complète, parce qu’il déborde et prend de revers les postes occupés. Ainsi fut pris Hochstadt. Ainsi l’aurait peut-être été Autoin et une partie de l’armée du maréchal de Saxe, si les bonnes dispositions des ennemis avaient soutenu ce que le hasard leur fit entreprendre. 3) En ce que de deux choses l’une : si les points fortifiés sont trop éloignés l’un de l’autre, comme à Fontenoy, à Lauffelt, à Rocoux, l’ennemi passe entre deux, ou bien fait, vis-à-vis chacun de ces points, une disposition qui les enveloppe à demi de batteries et de forces supérieures, les emporte, met à découvert l’armée qui les soutient et gagne la bataille. Si ces points fortifiés sont rapprochés au point de se protéger et de se flanquer mutuellement, cette position retombe dans l’inconvénient des camps retranchés. Toute l’armée se trouve emplacée dans des points où elle est réduite à la défensive la plus passive et la plus inégale. Si l’un de ces points est forcé, comment rétablir le combat ? Il ne reste point assez de troupes, point d’assez grands efforts à employer pour chasser l’ennemi du bastion où il s’est établi et de la courtine sur laquelle il se sera bientôt étendu. Que deviennent alors toutes les troupes emplacées dans des postes où elles sont débordées, prises à revers et d’où elles ne peuvent plus se retirer qu’avec peine ? 4) Cette défensive, fondée sur des positions retranchées, est enfin contraire à toutes les grandes vues de la guerre. Elle n’a du moins certainement jamais été la manière des grands hommes. On n’a qu’à récapituler les batailles qu’ils ont données. Ils ont presque toujours attaqué ; et s’ils ont reçu des combats, ce n’a presque jamais été derrière des retranchements.\par
Il ne résulte pas de là qu’il n’existe quelques occasions où une armée puisse se retrancher. Je blâme l’abus qu’on fait des positions retranchées et non l’usage qu’il est quelquefois à propos d’en faire. Si, par exemple, une armée inférieure occupe une position importante et par laquelle elle traverse absolument les projets de l’ennemi. Si, voulant couvrir un siège, un pays, une opération, elle trouve une de ces positions uniques, qui ne laissent à l’ennemi ni la ressource des manœuvres, ni celle des diversions et qui l’obligent nécessairement à venir attaquer dans cette position. Si enfin l’avantage qu’on trouvera à y recevoir la bataille, est plus grand que celui qu’on se procurerait en allant au-devant de l’ennemi, il n’y a pas à balancer à augmenter la force d’une position pareille par des retranchements. Encore faut-il qu’ils soient disposés tellement qu’on conserve la possibilité d’agir offensivement sur l’ennemi, si ses dispositions d’attaque où les mouvements du combat donnaient lieu d’espérer de tirer de ce parti une victoire plus certaine ou plus complète. Il faut, en un mot, que ces retranchements soient tels que l’armée, qui est derrière eux, ne puisse être réduite au rôle d’assiégée et qu’il reste une entière liberté de mouvement au génie de l’homme qui la commande, ainsi qu’au courage et à la science de manœuvres des troupes qui la composent.\par
Voici donc comme je pense qu’une armée devait se retrancher en pareil cas. Ce serait, non par des retranchements continus, ou, ce qui reviendrait au même, par des points retranchés, distribués symétriquement de distance en distance, de manière à se flanquer et à se protéger mutuellement. Ce serait en retranchant quelques points de sa position seulement, comme ceux qui sont vis-à-vis des débouchés si l’ennemi est réduit à déboucher, ceux où l’on ne peut disposer qu’un petit nombre de troupes et les troupes sur le courage et les manœuvres desquelles on compte le moins. Ce serait en se retranchant ainsi sur quelques points et en les mettant à l’abri d’être emportés, tandis qu’en réunissant sur d’autres points nus et ouverts, l’élite et le plus grand nombre de ses troupes, on y préparerait contre l’ennemi une disposition vigoureuse et prête à devenir offensive au moindre faux mouvement qu’on lui verrait faire.\par
Qu’on prenne la peine d’y réfléchir, cette manière de défendre une position absolument opposée à la routine actuelle, serait cependant conforme à tous les grands et véritables principes de la guerre. 1) Elle serait offensive, qualité primordialement constitutive de toute défensive d’armée. 2) Les retranchements y seraient ramenés à leur véritable usage, qui est de suppléer au nombre inférieur ou à la mauvaise espèce de troupes et de mettre à couvert des parties faibles et dégarnies. Ils ne seraient qu’un accessoire combiné et employé dans la disposition générale, de manière à fortifier là quelques points, pour laisser porter ailleurs l’élite et la majeure partie des troupes ; c’est-à-dire, à donner là une somme de résistance supérieure aux efforts de l’ennemi, pour procurer ailleurs une somme d’efforts supérieurs à son offensive.\par
Autant je pense que les retranchements doivent être rarement employés par une armée, autant je crois au reste que tous les postes et corps détachés doivent en faire usage, surtout si ces postes ou corps détachés occupent des points où il soit nécessaire qu’ils résistent, s’ils couvrent une opération, s’ils gardent un entrepôt, un magasin, un débouché. Dans ces occasions il s’agit de tenir ferme, d’attendre du secours et le petit nombre, quelque brave, quelque bien posté qu’il soit, peut être accablé par la multitude. Or de bons retranchements suppléant à l’infériorité du nombre et mettant en état d’attendre des renforts, c’est le cas d’en construire. Ils sont, dans cette circonstance, le moyen principal et primitif de la défensive.\par
Par une conséquence du raisonnement posé ci-dessus, il ne faut pas que les postes ou corps détachés s’occupent de se retrancher, lorsqu’ils sont simplement destinés à servir de masque, à couvrir une plus grande étendue de pays que celle qu’ils peuvent occuper. Dans le premier cas, leur but n’est pas de combattre, mais d’avertir. Dans le second, il est inutile de retrancher quelques points, puisqu’ils ne pourraient tout défendre et qu’ils ne serviraient qu’à indiquer à l’ennemi où il faut qu’il cherche à percer. C’est en manœuvrant, en se tenant sans cesse en mouvement le long de la ligne de défense qu’on a choisie, qu’on peut espérer de s’opposer à lui. Dans l’un et dans l’autre cas enfin, tout poste ou corps de troupes, qui prendra le parti de se retrancher, c’est-à-dire, celui de s’établir dans une position et de s’y arrêter plusieurs jours, s’exposera à s’y faire attaquer avec avantage, parce qu’il donnera à l’ennemi la tentation et le temps de combiner un mouvement offensif sur lui. Ceci n’exclut pas l’excellente maxime de passer les nuits dans la meilleure position possible, de rendre, si l’on est à portée de l’ennemi, cette position encore meilleure par quelques retranchements, placés non de manière à faire de la position une position de combat, puisqu’on ne veut pas en recevoir, mais à donner le temps de se rassembler, de rappeler ses postes, à couvrir et à faciliter la retraite.\par
Enfin, savoir à propos se retrancher, ou ne pas se retrancher, distinguer les occasions où des retranchements peuvent être utiles, inutiles ou funestes, les combiner, quand on a résolu d’en construire, avec l’objet qu’on se propose, avec ce qu’on peut faire de ses troupes, avec ce que peut l’ennemi et pour cela ne pas en abandonner la détermination à un ingénieur si cet ingénieur n’est en même temps homme de guerre et tacticien : voilà le devoir des officiers qui commandent des troupes à la guerre. Il faut pour cela qu’ils aient les connaissances nécessaires. Il faut qu’à cet effet il soit établi dans les troupes des écoles et des écoles de pratique, bien plus que de théorie, pour la construction, l’attaque et la défense des fortifications de campagne. Je dirai dans mon grand ouvrage sur quel plan devront être formées ces écoles, quels objets l’instruction devra y embrasser. Je prouverai qu’en six mois un officier pourra acquérir les connaissances indispensables sur cette partie de la science militaire. Ce seront ensuite, s’il est né homme de guerre, l’expérience, les occasions, la fermentation d’esprit qui naît toujours de la vue des choses et des événements, quand on a quelques lumières acquises, qui l’affermiront dans ces connaissances et lui apprendront à en faire usage.\par
Je viens de chercher à établir le véritable rapport que les fortifications de campagne doivent avoir avec la tactique et avec les opérations militaires. Examinons maintenant l’influence que la grande fortification, la fortification permanente, c’est-à-dire, les places de guerre, ont eu sur le système militaire de l’Europe. Cela nous conduira à chercher jusqu’à quel point cette influence devrait exister et nous trouverons qu’il s’en faut que ce point soit celui où elle existe.\par
L’esprit d’imitation et de manie, qui fait aujourd’hui si prodigieusement augmenter l’artillerie et les troupes légères, semblait sur la fin du dernier siècle vouloir convertir toutes les villes en place de guerre. Vauban et Cohorn donnaient une si grande célébrité à leur art et presque toute l’Europe militaire était si ignorante alors, qu’il n’est pas étonnant que ces deux hommes, avec du génie et des principes, aient entraîné toutes les opinions. Cohorn fortifia la Hollande ; Vauban fortifia la Flandre, le Rhin et une partie des frontières du royaume. Il bâtit ou répara près de cent forteresses. On vit en Flandre surtout s’élever des chaînes de places sur deux ou trois lignes. On vit en même temps, car les erreurs partant du même principe sont ordinairement contemporaines, des provinces entières couvertes par des lignes. Ces lignes étaient, à le bien prendre, des polygones multipliés et ajoutés l’un à l’autre sur un développement immense. Tel est enfin le reste du préjugé répandu alors, que la plupart des calculateurs politiques, en pesant les forces de la France avec celles des États voisins, font encore entrer aujourd’hui pour beaucoup trop dans la balance, cette quantité de places dont quelques-unes de ses frontières sont garnies, comme si des bastions pouvaient défendre à eux seuls les villes qu’ils enveloppent ; comme si la destinée de ces villes, quelque bien fortifiées qu’elles soient, ne dépendait pas de la bonté et de la vigueur des troupes qui les défendent et les soutiennent, comme si enfin des places mal défendues ne tournaient pas à l’épuisement, à la honte et à l’esclavage certain des peuples vaincus qui en ont été les constructeurs et les maîtres.\par
Qu’est-il cependant résulté de cette multiplication énorme de forteresses ? Les guerres en sont devenues plus ruineuses et moins savantes. Plus ruineuses du côté de l’argent et des hommes, parce qu’elles en ont consommé alors une bien plus grande quantité. Il a fallu construire ces places, il faut les entretenir. Mais ce ne serait encore rien que cette mise de dépenses primitives et annuelles. Ces places construites, il faut les approvisionner, il faut les garder même en temps de paix. Il faut en temps de guerre les couvrir, les défendre, attaquer celles de l’ennemi. D’où il a fallu augmenter de part et d’autre le nombre des troupes et de tous les attirails relatifs, entretenir ces troupes et ces attirails en temps de paix, par conséquent être perpétuellement dans un état de guerre qui ne laisse jamais respirer les peuples […].\par
Les armées étant devenues plus nombreuses et traînant à leur suite une beaucoup plus grande quantité d’attirails, il eût fallu que la tactique eût fait, en raison de cet accroissement, des progrès relatifs. Elle n’en fit pas et par conséquent les armées ne furent que des masses plus compliquées, plus pesantes, plus difficiles à mouvoir et à nourrir. Il y eut moins de grands mouvements en jeu de part et d’autre, moins de manœuvres, moins d’habileté. Dans les pays couverts de places comme la Flandre, la guerre prit un caractère de routine et de mollesse, qui n’est certainement pas celui du génie. On put à peu près calculer ce que chaque campagne devait produire. Une ou deux batailles, la plupart du temps conduites et décidées par le hasard, s’y donnent ou s’y reçoivent, soit pour couvrir des places, soit pour préparer ou couvrir des sièges. Celui qui les perd, se retire derrière ces places et celui qui les gagne, fait ou finit tranquillement quelques sièges. La campagne suivante, c’est la même chose et ainsi des autres jusqu’à ce qu’un des deux partis se sentant à ses dernières ressources, se hâte de conclure la paix. C’est-à-dire, pour peindre d’un seul trait cette manière de guerre, que deux cents mille hommes de part ou d’autre vont pendant quelques années sur la frontière répandre beaucoup de sang et d’argent, sans qu’il en résulte ordinairement d’autre effet décisif que celui de la prise de quelques places et de l’épuisement à peu près égal des vainqueurs et des vaincus.\par
À voir cela sous le point de vue de la philosophie et de l’humanité, il peut être heureux que, soit l’effet des places, soit celui de la routine établie, les guerres se passent ainsi en petites opérations, en alternatives de places prises et reprises, au lieu de conquérir et de ravager comme elles faisaient autrefois. Mais à envisager l’objet militaire, l’art de la guerre y a sans doute perdu, puisque ses effets sont moins grands, puisqu’enfin ils ne remplissent pas le premier et le malheureux but qu’ils doivent avoir, celui de faire le plus de mal possible à l’ennemi et de décider promptement les querelles des nations.\par
Il ne s’ensuit pas de là que l’art de construire de bonnes places, de les attaquer, de les défendre, porté au point où il l’est sur quelques parties et perfectionné comme il pourrait l’être sur beaucoup d’autres, ne fasse honneur à l’esprit humain et ne soit une branche intéressante de la vaste science de la guerre. Mais on lui a fait jouer un trop grand rôle, on a oublié qu’il n’était qu’un accessoire et que la grande tactique, la tactique des mouvements, celle qui fait gagner les combats, était le principal. On a trop compté sur les places de guerre, on les a trop multipliées. On s’est livré aveuglement à une infinité d’erreurs et de préjugés que les ingénieurs, malheureusement un peu instruits, tandis que le reste du militaire ne l’était point, mais malheureusement circonscrits dans la sphère de leur art seulement et par conséquent enthousiastes, exclusifs et portant rarement leurs vues au-delà de leurs fortifications, n’ont pas manqué d’accréditer et d’étendre […].\par
Les places sont, dit-on, la force d’un État. Cela exige une grande modification, car les places en elles-mêmes n’ajoutent pas plus à la force d’un État, que ses arsenaux et tous ses attirails de guerre, qui ne deviennent des moyens que quand on a des armées en état de s’en servir. Il n’y a dans un État de force réelle et existante par elle-même, que des troupes et des troupes portées au plus haut point d’instruction et de discipline. Ayez des places de guerre et les meilleures possibles, si en même temps vous n’avez pas d’armée, ou si cette armée est mauvaise, ces places, quelque multipliées, quelque fortes qu’elles soient, ne serviront qu’à faire faire des garnisons prisonnières et à affermir les conquêtes de l’ennemi. On voit la Hollande hérissée de places et défendue ordinairement par des troupes mercenaires et sans vigueur. En 1692, elle fut envahie presqu’en totalité dans six semaines. Elle ne fut sauvée que par ses inondations, par le parti qu’elle prit de mettre à la tête du peu de troupes qui lui restaient, le prince d’Orange qui leur rendit le courage et reprit l’offensive sur les Français dispersés et affaiblis par la garde de ces mêmes places qu’ils avaient conquises et mal à propos gardées au lieu de les détruire. Dans l’avant-dernière guerre on l’a vue de même prête à être envahie. Le maréchal de Saxe, supérieur en génie et en habileté aux généraux ennemis, avait donné l’ascendant à nos armées, il gagnait les batailles. De là toutes les places mollement défendues ouvraient leurs portes. Tant il est vrai que le destin des places est toujours réglé par celui des combats, que les places ne sont qu’un accessoire et que l’important, la chose à laquelle on doit s’attacher, c’est à avoir une armée supérieure en manœuvre et maîtresse de la campagne !\par
Sans les places, les guerres seraient plus dévastatrices, l’intérieur des États courrait plus de risque. Voilà de toutes les objections la plus fondée et celle qui milite le plus fortement en faveur des places. Approfondissons là soigneusement. De la manière dont se fait la guerre aujourd’hui, il est constant qu’elles empêchent les incursions et retardent l’invasion d‘un pays. Il reste à savoir seulement, si les places seraient des obstacles pour des armées autrement constituées que les nôtres. Si une cavalerie infatigable et facile à nourrir, comme celle des Numides et des Tartares, craindrait de passer entre elles pour aller faire des courses dans le pays et rentrer par une province opposée. Reste à savoir si un général, homme de génie, à la tête d’une armée accoutumée à la patience, à la sobriété, aux choses grandes et fortes, n’oserait pas laisser derrière lui toutes ces prétendues barrières et porter la guerre dans l’intérieur des États, aux capitales même. Les doutes, que je propose ici, serviront peut-être à faire voir que si les places retiennent l’ennemi sur les frontières et éloignent la guerre du cœur des États, c’est plutôt à cause de l’espèce et de la similitude de nos constitutions, à cause de la routine de guerre que nous avons adoptée, que par rapport aux obstacles réels qu’elles opposent.\par
Mais il ne s’agit pas de la manière dont la guerre pourrait se faire. Il s’agit de celle dont elle se fait et relativement à cette dernière, relativement à nos constitutions militaires et bien plus encore à nos constitutions politiques. Les places ont une utilité dont je vais parler et qui me les ferait conseiller à la plupart des États de l’Europe. Cette utilité n’a peut-être pas été aperçue sous le même point de vue par leurs plus zélés partisans.\par
Dans la plupart des pays de l’Europe, les intérêts du peuple et ceux du gouvernement sont très séparés. Le patriotisme n’est qu’un mot. Les citoyens ne sont pas soldats. Les soldats ne sont pas citoyens. Les guerres ne sont pas les querelles de la nation. Elles sont celles du ministère ou du souverain. Cependant elles ne se soutiennent qu’à prix d’argent et au moyen des impôts. Ajoutez que dans quelques-uns de ces États ces impôts sont excessifs, que le peuple y est mécontent, misérable et dans une situation qu’aucune révolution ne peut empirer. Cela posé, je dis que dans les États de cette nature, les places sont utiles. Car indépendamment des services qu’elles rendent contre les troubles du dedans, il est important pour eux que les guerres avec l’étranger se fassent toujours hors des frontières. Si elles pénétraient dans l’intérieur, il n’y aurait nulle ressource vigoureuse à attendre de la part des peuples. Indifférents et sans courage ils baisseraient la tête sous le nouveau joug. Les malheurs pourraient amener de grands troubles et des secousses dans le gouvernement. Tout au moins ils occasionneraient des révolutions dans le ministère. Mais qu’il existe un État libre, un peuple qui ait des mœurs, des vertus, du courage, du patriotisme ; un peuple qui fasse la guerre à peu de frais, parce que tous les citoyens s’armeront pour la défense commune, sans exiger de salaire ; un peuple qui se gouverne par lui-même et qui par conséquent, dans les temps de crise, mette nécessairement à sa tête l’homme le plus éclairé et le plus digne. Je dirai qu’un tel pays peut se passer de places, qu’il doit même s’en passer, afin de conserver sa liberté. Qu’en n’ayant point de places, il ne court aucun risque d’être subjugué. Premièrement il y a à parier que ses armées plus braves, mieux constituées, mieux commandées, arrêteront l’ennemi sur la frontière. Si le contraire arrive, l’État ne sera pas en danger pour la perte de quelques lieues de pays. Ses citoyens se rassembleront de toutes parts contre l’ennemi commun. Plus l’ennemi aura de succès, plus il faudra qu’il s’étende et qu’il s’affaiblisse. Où sera l’ennemi, là sera la frontière, parce que, si je peux m’exprimer ainsi, l’État ne fera que se replier sur lui-même et que partout où il restera de la terre et des hommes, l’État subsistera encore. Ainsi les campagnes de Rome étaient inondées par les Gaulois, Rome était détruite ; mais ses chevaliers, son nom, ses destinées s’étaient retirés sur la colline du Capitole, en attendant qu’un citoyen rassemblât les débris de la nation et vint chasser les vainqueurs.\par
Résumons, le plus brièvement possible, ce que je pense sur les places de guerre. Elles se sont trop multipliées. Elles sont comptées pour beaucoup trop dans la balance des forces des États et dans le système actuel de guerre. Elles ont rendu les guerres plus ruineuses en ce qu’elles ont obligé de renforcer et de multiplier les armées. Elles les ont rendu moins savantes et moins décisives, en ce qu’elles ont fait négliger la grande tactique, l’art des batailles, en ce qu’elles ont, en général, rétréci les vues et les opérations militaires. D’un autre côté, elles ont rendu les guerres plus douces. Elles empêchent les incursions, les dévastations. Elles peuvent, bien défendues, empêcher ou retarder les conquêtes. Elles ne procureraient peut-être pas ces derniers avantages, si les armées étaient différemment constituées, si un nouveau genre de guerre était substitué à la routine adoptée. Mais cela n’étant pas, il faut compter les effets qui existent. Enfin, politiquement parlant, les places sont nécessaires à la plupart de nos gouvernements. Elles le seraient moins en proportion de ce qu’ils seraient plus libres, plus vigoureux, plus vertueux, plus aimés des peuples. Elles le sont davantage en raison de ce qu’ils s’éloignent plus de ces qualités.\par
Il me reste à dire comment les places peuvent être le plus avantageuses à un État. C’est quand, par exemple, les débouchés sur la frontière se réduisant à quelques points, ces places les occupent et les défendent. C’est quand la frontière se trouvant, par la nature du pays, ouverte et sans obstacle, elles sont assises sur quelques points principaux, comme rivières, confluents de rivières etc. C’est quand, quelque part qu’elles soient assises, elles sont grandes, capables de contenir des magasins, des arsenaux, des entrepôts d’armée. C’est quand étant ainsi, elles sont fortifiées de manière à recevoir de grosses garnisons, des débris d’armée et cependant, au besoin, à pouvoir se défendre avec peu de troupes. C’est, en un mot, quand elles sont des places d’armes, des points d’entrepôt et d’appui, des bastions dont une armée bonne et manœuvrière est la courtine, ou en avant desquels cette armée peut agir offensivement avec la sûreté d’en retrouver l’appui en cas d’échec, ou enfin que cette armée peut abandonner à leur propre force, en attendant les circonstances favorables d’attaquer l’ennemi qui les assiège.\par
Je reviens à dire qu’il faut que les places soient en petit nombre. Si elles sont multipliées, il faut se consumer en grosses garnisons pour les garder, ce qui oblige à ne pas tenir la campagne et à prendre la défensive ; ou si l’on n’en met que de faibles, l’ennemi les menace toutes, manœuvre, dérobe un mouvement et finit par investir celles qui se trouvent dépourvues. Au lieu de cela, si l’on n’a qu’une ou deux places à couvrir, on peut ne pas les perdre de vue, primer toujours l’ennemi sur chacune d’elles et lui tenir tête avec toutes ses forces rassemblées. Ceci tient à l’opinion que j’ai établie ci-dessus, que la guerre en grand, la guerre de campagne doit toujours être l’objet principal, parce que c’est le sort des armées qui règle celui des places.\par
Je reviens de même à dire qu’il faut que les places soient grandes, de manière à pouvoir servir d’entrepôt et d’appui aux armées. Si elles sont petites, si elles sont comme toutes nos places du second et troisième ordre, elles sont inutiles. Elles ne sont pour les armées ni des points de retraite, ni des points de ralliement, ni des points d’établissement. Que l’ennemi les assiège, elles ne peuvent manquer d’être prises. Qu’il ne veuille pas les assiéger, il lui est aisé de les masquer. Il peut souvent même, sans inconvénient, les laisser derrière lui. Met-on de faibles garnisons dans des places de cette nature ? Les ouvrages abandonnés à la défensive de méthode sont bientôt accablés par la supériorité de l’assiégeant. Y met-on des garnisons nombreuses ? Elles n’en sont souvent que plutôt prises, parce que ce nombre y devient embarras, parce que la plupart des commandants ignorent l’art de se créer des dehors sous les approches de l’ennemi et de profiter de la force de leur garnison pour rendre leur défense offensive. Enfin la grande et la décisive raison qu’on peut donner contre cette sorte de places, c’est qu’il est tout au moins inutile de les construire à l’avance et à grands frais, c’est qu’à la guerre il serait possible de suppléer à l’objet momentané qu’elles remplissent, par des postes fortifiés momentanément. A-t-on besoin d’un entrepôt, d’une tête de quartier, d’un point pour soutenir une communication, pour défendre un débouché ? Qu’on choisisse une ville, un village, une hauteur, un terrain avantageux, qu’on emploie beaucoup de bras à s’y retrancher. Dans peu de jours on va en faire un poste où de bonnes troupes et un homme de tête se soutiendront assez pour donner le temps à l’armée de les secourir. Quels services rendent de plus les petites places construites et entretenues à grands frais ? On peut être un siècle sans avoir occasion de faire usage d’elles et alors on les laisse dégrader : ou si on les entretient, voilà, pendant un siècle, des dépenses annuelles qui forment des sommes considérables et qui auraient pu être bien plus utilement employées. Cependant quand elles sont assiégées, si une armée ne vient à leur secours, elles finissent par être prises. Ainsi, à quelques jours près, les postes élevés en terre, tels que je les propose, rempliraient l’objet qu’elles remplissent. En un mot, ces derniers ont des avantages que les places ne peuvent pas avoir. C’est que la circonstance détermine leur position et par conséquent la détermine toujours bien plus convenablement à l’objet du moment. C’est que la circonstance ayant changé, on abandonne le poste, ou on le rase pour en aller faire un autre ailleurs. C’est qu’on fait son poste proportionnément à l’objet qu’on veut qu’il remplisse, au nombre de jours qu’on veut qu’il tienne, au nombre et à l’habileté de l’ennemi qu’on a devant soi. C’est qu’enfin l’officier qu’on charge de la défense du poste, préside en même temps à sa construction, la dirige suivant ses vues de défensive et ses moyens, tandis qu’au contraire, dans la plupart des places, ses vues et ses moyens se trouvent souvent en contradiction avec l’espèce, la disposition et l’extension trop grande ou trop bornée de leurs ouvrages. Expliquons cette dernière idée. Pour prouver, il faut, malgré soi, s’étendre en raisonnement.\par
L’inconvénient de toutes les places et un inconvénient qui devient plus sensible à proportion qu’elles sont plus multipliées, c’est qu’en général les circonstances qui en ont déterminé l’assiette et le système de construction, venant à changer par la révolution des événements, ces places se trouvent ou inutiles, ou mal emplacées, ou sans rapport avec les circonstances du moment. Pour nous convaincre de cette vérité, jetons les yeux sur deux cents places qu’on compte en France. Un homme qui n’aurait pas réfléchi sur leur position, serait porté à croire qu’avec cette quantité de forteresse, toutes les provinces du royaume sont couvertes et nous avons des frontières qui en sont absolument dégarnies. Il n’y en a presque point dans nos provinces maritimes. Nos plus grands ports, nos établissements de marine sont à peine, du côté de terre, à l’abri d’un coup de main.\par
Ailleurs nous avons deux ou trois lignes de places. Nous en avons dans des points où elles ne couvrent rien, où elles ne défendent rien. C’est que la frontière, dans de certaines parties, s’est avancée et que, dans d’autres, elle a reculé. C’est qu’autrefois on avait pour système d’opposer place à place et que celles de l’ennemi ne subsistant plus, les nôtres sont, dans quelques points, devenues inutiles. C’est qu’alors on avait la manie de tout fortifier. C’est qu’aujourd’hui cette branche de l’administration n’est pas conduite sur un plan plus déterminé. On n’a ni le courage de raser ou d’abandonner totalement une partie des places, ni assez d’argent pour les entretenir. On les répare à demi. Il en est d’inutiles qu’on conserve par respect pour Vauban, ou par d’autres préjugés de routine. Il en est qu’on augmente, parce que les villes ont des octrois qui sont les fonds annuels de leur entretien et qu’il est de règle établie que ces fonds doivent être employés aux fortifications de ces villes, ces villes n’en eussent-elles pas besoin. Il en est que les directeurs ou ingénieurs en chef se plaisent à bouleverser ou à surcharger de pièces inutiles, afin de contrecarrer l’opinion de leurs prédécesseurs, ou de suivre la leur. Il en est autour desquelles on fait des enceintes d’ouvrages qu’une armée seule pourra défendre, travaux sur l’immensité, sur l’inutilité et sur la cherté desquels on ne peut s’empêcher de gémir, quand on songe que, si on n’a pas une armée à jeter dans ces ouvrages, ils ne sauveront pas la place ; que si l’on a une armée, il vaudrait mieux qu’elle tînt la campagne, qu’elle couvrît la place par une position bien prise ou par une guerre de mouvements et qu’enfin, en cas de malheur ou d’infériorité trop grande, elle se retranchât sous la place et se soutint dans ses dehors jusqu’à ce que la fortune eût changé.\par
On voit combien il serait important que le gouvernement s’occupât de cet objet, qu’il format à cet égard un plan combiné sur la situation actuelle du royaume et sur les véritables principes de la guerre […].\par
Mais pour exécuter avec fruit un si grand changement, il faut auparavant avoir perfectionné toutes les parties de notre constitution militaire. Il faut avoir des troupes à l’épreuve et manœuvrières, des généraux qui sachent les conduire et qui osant s’écarter de la routine établie, adoptent, pour ainsi dire, un nouveau genre de guerre. Il faut des troupes infatigables, accoutumées aux travaux et qui puissent au besoin et promptement créer les places mobiles dont j’ai parlé, comme les légions Romaines construisaient les castrums qu’elles employaient au même usage. Il faut exercer ces troupes à la construction et à la défense de ces postes, avoir sur cet objet des écoles continuelles et bien dirigées. Il faut enfin former les bras et le courage des soldats, la tête et les préjuges des officiers. Car les troupes étant une fois parvenues à ce point de perfection, avec de la terre et des hommes, on fait aisément des postes qui remplissent l’objet des places […].\par
Ayant osé avancer mes opinions sur le véritable usage qu’on devrait faire des fortifications, je peux, bien à propos du corps qui les dirige, oser dire que dans cette révolution de système, il y aurait une autre constitution à lui donner, une constitution qui le rapprocherait davantage des troupes et de la connaissance de toutes les autres parties de la guerre : qui lui donnerait même sur l’art qu’il cultive, des écoles plus instructives et plus militaires ; qui enfin détruisant beaucoup de préjugés, suite de sa constitution actuelle et de la manière dont on le fait servir, le rendrait propre à plus et de plus grands objets […].
\section[{Second mémoire. Rapport de la connaissance des terrains avec la tactique}]{Second mémoire. Rapport de la connaissance des terrains avec la tactique}\renewcommand{\leftmark}{Second mémoire. Rapport de la connaissance des terrains avec la tactique}

\noindent À en juger par tous les détails militaires qui nous restent des Anciens, la science de la reconnaissance des terrains devait être pour eux bien moins importante que pour nous. Leurs ordres de bataille, plus profonds et plus raccourcis que les nôtres, n’avaient pas besoin de position d’un grand développement. À peine voit-on même que le choix de ces positions les occupât. Dans le récit de toutes les batailles de l’antiquité, on ne voit aucun détail topographique. Il semble que leurs combats se donnaient toujours dans les plaines et que les armées recherchaient cette espèce de terrain par préférence. C’est qu’alors toute la ressource des troupes était dans les manœuvres. Une arme appuyait l’autre. C’était ordinairement la cavalerie qui formait les ailes. À Pharsale, César sut disposer une partie de son aile gauche en échelon oblique et ce fut là ce qui lui valut le gain de la bataille. Rarement voit-on dans l’histoire qu’il soit question d’une aile qui ait cherché protection dans la nature du terrain. Pour des affaires de postes, il n’en est certainement jamais fait mention. L’espèce des armes et de la tactique des Anciens ne les y rendait pas propres. La phalange n’avait de force que dans les plaines. Une légion romaine avait toute sa confiance en elle-même. Tant que l’infanterie fut brave et bien armée, tant que les machines de guerre ne se multiplièrent pas, tant qu’on se battit corps à corps, il en fut ainsi. Mais lorsque les légions dégénérèrent, lorsqu’elles quittèrent les armes défensives, lorsqu’elles devinrent timides et tremblantes dans les plaines, lorsque les catapultes et les balistes se multiplièrent dans les armées, comme les canons se multiplient aujourd’hui dans les nôtres, on commença à avoir recours aux ressources du terrain. On chercha les hauteurs, on espéra augmenter par elles l’effet des machines de jet. On tâcha de mettre des obstacles entre l’ennemi et soi […].\par
Quand les armes à feu eurent acquis quelque perfection, le terrain dut commencer nécessairement à prendre de l’influence sur les opérations de la guerre. L’infanterie chercha les pays coupés. Elle occupa par préférence les villages, les bois, les hauteurs. Ces points devinrent des postes et des appuis intéressants à se procurer. Ils entrèrent par conséquent dans les combinaisons de la castramétation et de la tactique. Ce fut, sans doute, une nouvelle ressource pour le génie et un pas de plus vers la perfection de l’art. Mais, comme presque partout l’abus suit la vérité, peu à peu cette influence des terrains sur les opérations est devenue trop absolue. La science du mouvement des troupes a été négligée. On a cru qu’il était inutile de manœuvrer, que toute la science de la guerre consistait à choisir des positions avantageuses. De là se sont élevés tant d’officiers topographes, réels ou prétendus, qui remplissent les états-majors de l’armée et les cabinets des ministres ; officiers qui, pour la plupart, n’ont aucune connaissance de la tactique, aucune habitude de manier les troupes, qui regardent même cette connaissance et cette habitude au-dessous d’eux. Cette manie de topographe, cette prévention outrée des états-majors d’armée en faveur des détails dont ils sont chargés, étaient faites pour s’accréditer en France, plutôt qu’ailleurs ; parce que tous les officiers y sont portés à raisonner et à se croire relevés par des fonctions qui, revêtues de quelques apparences d’importance, initient aux mystères des opérations.\par
Sans doute la science de la reconnaissance des terrains est importante. Il faut qu’elle soit cultivée et que ses résultats entrent dans les combinaisons journalières de la guerre. Mais il faut qu’elle ne soit regardée que comme une branche de la tactique qui est, je le répète, la science mère. Il faut donc que les officiers de l’état-major de l’armée soient tacticiens. Il faut qu’ils sachent disposer et manier les troupes. Il faut que, dans leurs supputations, ils n’oublient pas que les troupes défendent encore plus les positions, qu’elles ne sont défendues par elles, que le terrain n’est jamais que l’accessoire et que l’arme est toujours le principal. Il faut enfin qu’ils n’aient point la prétention aveugle de croire que toute la science de la guerre et la sublimité du métier résident dans leur travail de cabinet.\par
Pour que cela fût ainsi, comment faudrait-il choisir les officiers de l’état-major ? Ce devrait être parmi des hommes qui eussent l’habitude des détails et des mouvements de toutes les armes ; parmi les officiers majors, ou supérieurs des corps ; parmi ceux d’entre eux, qui ont le plus d’intelligence, le plus d’activité, le plus de sagacité et de justesse dans le coup d’œil. Comme ensuite c’est un talent que de bien reconnaître un pays et que ce talent est fondé sur une théorie dont il est important d’acquérir la pratique, ces officiers formeraient, en temps de paix comme en temps de guerre, un corps d’état-major permanent. Ce corps serait sous la direction d’un officier général qui lui-même joindrait aux talents les plus décidés pour la grande partie de la guerre, la science et l’habitude de remuer toutes les armes qui entrent dans la composition d’une armée, qui par conséquent ne regarderait pas la tactique comme une science minutieuse et subalterne ; sous la direction d’un officier général, en un mot ; car la dénomination de ce grade, qui trop souvent ne tient presque rien de ce qu’elle promet, signifie un homme qui, par son étude et son expérience, a embrassé toutes les parties de la guerre et qui connaît l’analogie qu’elles doivent toutes avoir entre elles.\par
Où se tiendraient les écoles d’instruction de cet état-major ? Ce serait au milieu des troupes, dans les grandes garnisons, dans les camps de paix. Là, plus de supputations idéales et que la pratique ne peut pas éclairer. Là, les grandes opérations de la guerre comme marches, ordres de bataille, seraient mises à exécution et combinées avec le terrain. Là, conséquemment la tactique serait enseignée, c’est-à-dire, la tactique, telle que je l’ai définie : la science de toutes les parties de la guerre. Là, les officiers d’état-major acquerraient de plus en plus l’habitude de manier les troupes. Se fortifierait le coup d’œil contre les illusions que produit la multitude, contre les différences d’un terrain nu, ou couvert de troupes. Là enfin, ces officiers se familiariseraient de plus en plus avec les troupes, au lieu de tendre à s’en séparer, au lieu de les regarder comme des ressorts purement mécaniques, ainsi qu’ils le font aujourd’hui.\par
Qu’on compare une telle école d’état-major, à celle que, depuis la paix, nous tentons de former. Dans cette dernière, ce sont quelques officiers, dont la plupart ignorent et dédaignent les premiers éléments de la tactique, qu’on envoie faire des reconnaissances sur les frontières. Ceux, dans lesquels on a le plus de confiance, ont la permission de joindre aux rapports qu’ils font de ces reconnaissances, des mémoires militaires, des systèmes d’opérations offensives ou défensives. Les autres sont de jeunes officiers désignés sous le nom d’élèves. À ces derniers on ne demande que des mémoires purement topographiques, c’est-à-dire, le travail d’un ingénieur géographe. Cependant les uns et les autres perdent de vue les troupes […].\par
Il y a certainement une théorie et des principes pour reconnaître un pays, pour en démêler les détails, les saisir et les calquer dans sa mémoire. En étudiant la direction des chemins et le cours des eaux, on se fait d’un pays une idée plus nette et plus militaire. Aux points où un officier se portera, en entrant dans un pays qui lui sera inconnu, pour en mieux saisir l’aspect, aux points de repère et de signalement qu’il se choisira, aux triangles et aux rayons que son œil projettera, il sera facile de juger s’il a, ou non, le talent de reconnaître. Dans les pays montagneux particulièrement, c’est un art que de bien démêler les chaînes principales d’avec les sommités, ou contreforts qui en dérivent : les points où naissent et coulent les eaux, les entrées des gorges ; l’espèce des pendants, la profondeur des vallons, les distances des lieux. Il existe à cet égard, une théorie excellente, créée par M. de Bourcet, dont le militaire doit désirer la publication.\par
Mais le grand moyen pour devenir habile dans la science de reconnaître les terrains, c’est la pratique journalière, c’est dans sa jeunesse, de voyager, de chasser, de se promener souvent militairement. Ainsi feront tous les officiers qui voudront s’élever aux grandes parties de la guerre. Car, dans quelque arme qu’on serve, la science du coup d’œil est de la plus grande importance. Dans ma tactique élémentaire, j’ai proposé, à cet égard, des écoles pour les officiers. De ces écoles sortiraient de bons officiers-majors et de ces officiers-majors d’excellents sujets pour les états-majors des armées.\par
Lorsqu’on a le coup d’œil formé, lorsqu’on sait juger parfaitement un terrain, mesurer les distances jugées sous différents aspects, lorsqu’on s’est affermi la vue contre les illusions sans nombre que peuvent produire la différence des terrains, la quantité et la complication de troupes de différentes armes, vues sous différents aspects, les manœuvres de ces troupes, les ruses de tactique dont elles se servent, si elles sont habilement maniées, l’horizon plus ou moins serein et mille autres causes accidentelles ou locales, il s’agit d’apprendre à voir un pays militairement, c’est-à-dire à démêler promptement et sûrement quelle influence ce pays peut avoir sur les opérations militaires ; quelle portion il offre, dans tel ou tel cas, à l’armée ou au corps de troupes dont on suppute les mouvements ; quels y seraient les débouchés et l’ensemble d’une marche sur tel ou tel point ; enfin les rapports généraux et de détails que cette masse de pays pourrait avoir avec les armées qui y agiraient. Mais ce talent-là peut s’augmenter et non s’acquérir, par l’habitude. Il est un présent de la nature et l’instinct du génie […].\par
C’est là cette sagacité de coup d’œil et de jugement qui gagne les batailles et que la nature ne donne, dans l’espace d’un siècle, qu’à quelques hommes privilégiés.\par
La science du coup d’œil et la connaissance des terrains étant intimement liées avec la tactique, on voit combien de fausses et d’inutiles lumières donneront les écoles d’état-major, qui ne seront pas constituées d’après ce principe fondamental. Je vais le faire sentir encore davantage. Il s’agit de choisir une position pour une armée. Si celui qui la détermine n’est pas tacticien, comment saura-t-il combiner, relativement à la force de cette armée, l’étendue que cette position devra avoir ? Comment aura-t-il égard, dans le choix de cette position, à l’espèce d’arme dans laquelle l’armée est la plus forte ou la plus faible ; à l’espèce d’ordre de bataille dans lequel il peut être le plus avantageux de l’occuper ? Faute de cette combinaison, on prend des positions bonnes en elles-mêmes, mais qui se trouvent défectueuses, relativement au nombre et à l’espèce de troupes qui les garnissent. On prend des positions dont le front est redoutable et où l’armée ne peut pas manœuvrer, faute de fonds. On en prend d’autres qui sont formidables de toutes parts, mais dans lesquelles l’armée, réduite à la défensive, perd l’avantage de pouvoir manœuvrer et profiter des fautes de l’ennemi. On en prend enfin que, par un mouvement qu’on n’a pas su prévoir, l’ennemi parvient, ou à tourner, ou à percer, ou à faire abandonner, sans qu’on ait le pouvoir de lui résister.\par
Mais, après qu’une position est déterminée, après même qu’elle est reconnue avantageuse, soit relativement aux vues d’offensive et de défensive, soit par rapport au nombre et à l’espèce des troupes qui doivent l’occuper, il reste une manière d’y disposer les différentes armes, dans laquelle il faut encore que la tactique de combinaisons soit un art qui ait aussi ses principes. Soit, par exemple, une lisière de hauteurs, déterminée pour être le front de la position que l’armée doit occuper. Si, suivant la routine ordinaire, on y ordonne la disposition des troupes, étant sur le terrain même et en parcourant le front de la position, on court risque de ne pas distribuer les armes dans les emplacements qui peuvent leur être le plus avantageux et de ne pas tirer de la position tout le parti dont elle est susceptible. En se portant au contraire en avant de la position et aux points par où l’ennemi pourrait arriver sur elle, on en découvrira plus parfaitement l’ensemble et les détails. On verra d’abord le terrain qui est en avant d’elle, l’aspect qu’il présente à l’ennemi, la disposition d’offensive qu’il peut lui indiquer. Se supposant ensuite à la place de l’ennemi, on cherchera quels sont les moyens par lesquels il pourrait attaquer cette position et, partant de là, quels sont les contre moyens qu’on pourra lui opposer. En voyant la position de face, on jugera mieux l’emplacement qu’il faut y donner à chaque espèce d’armes, les saillants avantageux à occuper par des batteries, l’effet que le feu de ces batteries doit faire sur les débouchés par où peut arriver l’ennemi, le point des hauteurs le plus convenable à occuper, pour que le feu de l’infanterie ne soit pas trop plongeant, les rideaux derrière lesquels on peut mettre une partie de ses troupes à l’abri du feu des batteries de l’ennemi, ou faire illusion à l’ennemi sur le nombre de ses forces et sur la véritable disposition qu’on lui oppose.\par
La science de la reconnaissance des terrains, combinée avec la tactique, pourrait être le sujet d’un ouvrage intéressant et il naîtrait, sans doute, d’une école d’état-major, constituée sur le plan que je propose. Sur cette branche de l’art militaire, comme sur tant d’autres, il n’y a rien de réduit en principes. L’on croit que cela est inutile. D’une part, parce que le talent de connaître un pays est, dit-on, un don inné et que, sur les choses de génie, il n’y a point de préceptes à poser. De l’autre, parce qu’à peine croit-on que la tactique soit une science et encore moins qu’elle ait une liaison indispensable avec la connaissance des terrains.\par
En supposant que les officiers-majors devinssent tacticiens, j’admets, à plus forte raison, que les généraux le seraient et alors les armées seraient plus manœuvrières. Quand je dis les armées, j’entends des armées en masse et non morcelées par corps et par détachements. Elles sauraient exécuter des marches à portée de l’ennemi, prendre des ordres de bataille et gagner des batailles par manœuvres. À mesure qu’on ferait davantage la guerre de mouvement, on s’écarterait de la routine actuelle. On reviendrait aux armées moins nombreuses et moins surchargées d’embarras. On rechercherait moins ce qu’on appelle des positions : car les positions ne doivent jamais être que la dernière ressource d’une armée manœuvrière et bien commandée. Quand une armée sait manœuvrer et qu’elle veut combattre, il est peu de positions qu’elle ne puisse attaquer de revers, ou faire abandonner à l’ennemi. Les positions, en un mot, ne sont bonnes à prendre que quand on a des raisons pour ne pas chercher à agir, ou qu’elles sont de telle nature qu’elles réduisent l’ennemi à les attaquer avec désavantage, ou à manquer ses opérations. Ceci mérite d’être développé avec plus de détail.\par
Qu’est-ce en effet qu’une bonne position ? C’est un vaste développement de terrain, dont le front et les flancs fournissent des emplacements avantageux à l’armée qui doit les occuper et présentent à l’ennemi qui voudrait l’en déposter, des obstacles difficiles à vaincre. Mais que fera cette position, quelque bonne qu’elle soit, à un ennemi habile et manœuvrier ? Ne peut-elle pas être tournée de loin, si ce n’est de près ; alors l’armée, qui l’occupe, n’est-elle pas obligée de l’abandonner ? Cette position, formidable par devant, l’est-elle par-derrière et attaquée par ce dernier côté, ne peut-elle pas devenir désavantageuse ? Il est rare que la nature présente de ces positions à double front, dans lesquelles une armée puisse être également bien postée sur l’une ou sur l’autre face. Telle est même la routine des idées reçues, que, comme on n’a pas encore vu d’armée attaquée par-derrière, on ne songe pas que cela puisse être. Rien n’est cependant plus possible.\par
Supposons, d’un côté, une armée surchargée d’embarras, malhabile à manœuvrer, telle enfin que sont les nôtres et de l’autre, une armée bien constituée, manœuvrière, commandée par un général qui ait médité toutes les ressources de la tactique. L’une cherchera des positions, y mettra toute sa confiance ; se remuera difficilement et avec lenteur ; sera enchaînée par ses méthodes de subsistance ; se croira perdue, si elle n’a pas toujours ses établissements bien exactement derrière elle. L’autre sera légère et maniable, capable de mouvements hardis, de marches rapides et forcées. Elle sera toujours sur l’offensive, ne s’enfermera presque jamais dans des positions et méprisera celles qu’on voudrait lui opposer. L’ennemi croira-t-il l’arrêter par une de ces positions prétendues inexpugnables ? Elle saura lui dérober un mouvement, ou même, sans le lui dérober, se porter, à sa vue, sur son flanc ou derrière lui. Pour exécuter ce mouvement, elle portera, s’il le faut, des vivres pour huit jours et se passera de ses établissements. Que fera l’ennemi, étonné de ce genre de guerre nouveau ? Attendra-t-il qu’une armée habile à se remuer, à fondre rapidement sur la partie faible d’une disposition, à passer, en un moment, de l’ordre de marche à l’ordre de combat, se trouve en mesure d’attaquer le flanc ou le derrière de sa position ?\par
Cette inaction lui deviendrait funeste. Changera-t-il de position ? Alors il perdra les avantages de terrain sur lesquels il avait compté et il sera obligé de recevoir la bataille où il pourra. Peut-être son mouvement lourd et lent donnera-t-il prise sur lui. Il sera embarrassé de ses attirails et de ses moyens de subsistance. Il craindra d’être séparé de ses établissements, dont il ne saura se passer, parce qu’il aura contracté l’habitude d’y être assujetti et parce que ses troupes, plus nombreuses que celles de l’autre armée, seront, indépendamment de cela, moins sobres et moins patientes.\par
Enfin je dis qu’une armée, bien constituée et bien commandée, ne doit jamais trouver devant elle de position qui l’arrête ou qui la force d’y attaquer avec désavantage l’armée qui est y établie ; à moins que ce ne soit de ces positions rares qui, touchant à l’objet qu’elles veulent couvrir, ne laissent la ressource de manœuvrer, ni sur leur derrière, ni sur leur flanc […].
\section[{Troisième mémoire. Rapport de la science des subsistances avec la guerre et particulièrement avec la guerre de campagne. Examen de la manière dont nous faisons subsister nos armées}]{Troisième mémoire. Rapport de la science des subsistances avec la guerre et particulièrement avec la guerre de campagne. Examen de la manière dont nous faisons subsister nos armées}\renewcommand{\leftmark}{Troisième mémoire. Rapport de la science des subsistances avec la guerre et particulièrement avec la guerre de campagne. Examen de la manière dont nous faisons subsister nos armées}

\noindent C’est une branche importante de la vaste science de la guerre, que l’art de pourvoir à la subsistance des armées. Cet art, comme tous les autres, a eu ses révolutions. Il a, suivant les temps, varié dans ses détails et dans ses principes. Je vais faire ici l’examen intéressant de ce qu’il a été dans les principaux âges de l’antiquité et de ce qu’il est dans le nôtre.\par
On ne voit pas dans l’histoire, mais il est aisé de concevoir, comment pouvaient et devaient subsister ces petites armées des républiques grecques, faisant la guerre à quelques lieues de leur territoire et quelle espèce de guerre ! Des incursions de quelques jours, faites pendant la saison des récoltes et terminées ordinairement par une bataille, à la suite de laquelle les deux parties allaient réparer leurs pertes et cultiver leurs champs.\par
L’histoire nous laisse également sans lumière, et il est plus difficile d’y suppléer, sur la manière dont subsistèrent ces armées, quand l’ambition des États de la Grèce, augmentée avec leur puissance, les fit plus nombreuses et les porta à la conquête des îles voisines et de quelques parties de la côte d’Asie. On voit seulement qu’alors le soldat, qui combattait auparavant gratuitement, eut une solde réglée. L’histoire dit que cette solde était toute en argent et elle en marque le montant. Le soldat était-il chargé ensuite, au moyen de cette paye, de pourvoir à sa nourriture ? Comment y pourvoyait-il ? L’armée formait-elle des magasins ? Voilà ce que nous ignorons. Je pourrais donner des conjectures sur tous ces objets. Mais il est inutile de hasarder des conjectures où manquent les lumières.\par
On sait bien moins encore comment subsistaient ces multitudes presque fabuleuses avec lesquelles les rois de Perse tentèrent d’envahir la Grèce. Elles étaient si nombreuses, elles traînaient à leur suite une si grande quantité d’attirails et de bêtes de charge, qu’elles mettaient à sec, dit l’hyperbolique Hérodote, les rivières auprès desquelles elles séjournaient ; et que la disette et la peste s’établissaient après elles dans le pays où elles avaient passé. On peut conclure de là que ces armées vivaient, au hasard et sans méthode, des moyens que leur offrait le pays et ce qui le confirme, c’est que leurs expéditions n’étaient que des incursions. Ces inondations armées avaient le cours des torrents et s’écoulaient comme eux.\par
Au reste, ce n’est pas le cas de regretter que l’histoire ne nous dise point comment ces armées de barbares subsistaient dans leurs expéditions. Elles y périssaient, comme dans les combats qu’elles donnaient, victimes de leur immensité et de leur ignorance. Mais on doit regretter, en revanche, de n’avoir pas plus de détails sur les procédés de subsistance employés par des conquérants heureux et habiles, tels que Cyrus, Alexandre, Annibal. L’histoire ne nous en transmet aucun. Nulle part seulement nous ne voyons leurs armées arrêtées par des formations de magasins et par des calculs de subsistance. Sans doute elles vivaient dans les pays où elles faisaient la guerre et des denrées de ce pays. Sans doute elles étaient sobres et endurcies. Sans doute aussi avaient-elles des combinaisons de subsistance moins compliquées, moins timides, moins financières que les nôtres. Qu’on songe aux expéditions de ces armées. Qu’on voie Alexandre partant de la Macédoine pour aller conquérir l’Asie. Qu’on suive Annibal partant d’Espagne pour aller porter la guerre à Rome, passant les Pyrénées, traversant les Gaules, ayant, à chaque pas, des peuples inconnus à se concilier ou à combattre, s’ouvrant ensuite un chemin à travers les Alpes, descendant en Italie et s’y soutenant neuf ans victorieux et sans tirer aucun secours de Carthage. Qu’on mette ces campagnes-là en parallèle avec les nôtres. Qu’on transporte ces vastes opérations sur l’échelle actuelle de nos combinaisons militaires, on sera forcé de révoquer l’histoire en doute, ou de convenir du rétrécissement de nos génies.\par
Les guerres des Romains ne nous instruisent pas davantage sur les détails de la science des subsistances chez les Anciens. On conçoit qu’ils durent être simples et faciles, tant que les armées romaines curent affaires aux peuples du Latium. Mais quels ils furent, quand Rome entreprit des guerres étrangères et lointaines, voilà ce qu’aucun historien ne nous apprend. Quelques traits, épars çà et là, forment là-dessus toutes nos lumières. Il est quelquefois mention dans Tite-Live, des distributions de vinaigre, de vin et de grains. On y voit des légions qu’on voulait punir, condamnées au pain d’orge ; preuve, sans doute, qu’il s’en distribuait d’une autre espèce au reste de l’armée. On lit dans Végèce, que les préfets du camp, ce qui était un office purement militaire, étaient chargés du détail des subsistances. On y lit que les centuries romaines avaient des moulins à bras, qu’on leur distribuait du grain en nature. Ailleurs il est dit que, dans les expéditions, chaque soldat portait sa portion de farine pour quinze jours et qu’ensuite arrivé au camp, il faisait, avec cette farine détrempée, une manière de gâteau qui servait à sa subsistance. Cet usage de moulins à bras et des distributions de grain ou de farine aux troupes, a été proposé plusieurs fois de notre temps et traité de chimère. Un exemple instructif qu’on doit enfin recueillir de l’étude de la constitution des légions romaines, dans le temps de leur vigueur et du résultat de leurs opérations, c’est la tempérance, l’austérité, la patience infatigable qui en était la base. De telles troupes savaient s’accommoder à toute espèce de nourriture et au besoin, endurer la faim et la soif. Aussi nulle part, dans l’histoire du bel âge militaire de cette nation, on ne voit les opérations arrêtées par des calculs de subsistance. Dans nos histoires modernes, on verra, à chaque pas, les combinaisons de subsistance faire séjourner les armées et commander aux généraux.\par
Une autre vérité importante qu’on peut retirer de l’étude des guerres romaines, vérité dont le résultat contrarie bien nos systèmes de subsistance actuels, c’est que les armées vivaient dans le pays et aux dépens du pays. Il faut que la guerre nourrisse la guerre, disait Caton dans le Sénat et cette maxime de Caton était, chez les Romains, une maxime d’État. Dès qu’une armée avait mis le pied chez l’ennemi, c’était au général, qui la commandait, à la faire subsister et celui-là avait le plus utilement servi la République qui, en faisant la campagne la plus glorieuse, avait le mieux entretenu son armée et rapporté, après la campagne, plus d’argent au trésor public. De là la solution de cet état de guerre presque continuel, au milieu duquel fleurissait la République. Elle recevait de la guerre accroissement et richesse, comme nos États d’aujourd’hui, par la constitution désordonnée de leurs systèmes militaires, en reçoivent affaiblissement et misère. Scipion portait la guerre en Afrique et bien loin d’épuiser Rome pour nourrir son armée, les greniers de Rome se remplissaient des blés d’Afrique. César allait conquérir les Gaules et Rome n’entendait plus parler de lui que par le bruit de ses victoires. Non seulement son armée n’était point à charge à l’État, mais il enrichissait cette armée, il faisait passer des fonds au trésor public. Il en réservait pour ses vastes desseins. Il embellissait les Gaules, après les avoir soumises. Il y changeait la face des villes. Il y construisait des chemins qui sont encore aujourd’hui des monuments. Avec l’or des Gaules, il préparait des fers à la Germanie, à sa patrie elle-même. Les Gaules cependant aimaient sa domination. Nous n’avons pas l’art de conduire des guerres ainsi. Mais revenons à celui qui fait l’objet de mes recherches […].\par
Les temps de décadence, qui minèrent l’Empire romain et les siècles de barbarie qui suivirent sa chute, n’offrent rien d’instructif sur aucune branche de la guerre. Jusqu’à l’époque de Nassau et de Gustave, les armées se battirent sans combinaison et subsistèrent à peu près de même. Les campagnes étaient des espèces d’incursions. On se répandait dans le pays. On marchait par corps et en cantonnant. Si l’on se rassemblait, c’était pour quelques jours seulement et afin de livrer le combat. Le pays subvenait, comme il pouvait, à la subsistance des gens de guerre ; et il n’y pouvait pas fournir longtemps, à cause de l’extrême indiscipline qui régnait parmi eux.\par
Sous Nassau et sous Gustave, un nouvel ordre naquit dans les armées. Les troupes apprirent à camper, à marcher, à combattre. Avec l’austère discipline que ces grands hommes établirent, il fallut d’autres procédés de subsistances. Les armées, rassemblées dans des camps, eurent besoin de magasins. Gustave faisait faire des distributions journalières de pain et de viande à ses soldats. Dans les opérations forcées, elles savaient vivre plus sobrement. Il les avait élevées à se nourrir de tout et à jeûner sans murmure. Cet esprit subsista encore longtemps après lui dans les troupes suédoises. Les nouvelles méthodes de subsistances n’entravaient cependant point les opérations de Gustave et des généraux habiles qui lui succédèrent. Alors les armées étaient peu nombreuses. Elles ne traînaient pas à leur suite une énorme quantité d’artillerie et d’équipages. Le luxe n’avait pas énervé les mœurs et augmenté les besoins. Avec ces petites armées, on pouvait faire de grandes conquêtes. Les généraux faisaient eux-mêmes l’office de munitionnaires. Le duc de Rohan, dans son {\itshape Parfait Capitaine}, en détaille les fonctions. Il s’élève contre quelques-uns qui avaient proposé de confier ces détails à des personnes non militaires. Comme si, dit-il, pourvoir à ce que l’armée vive, ne faisait pas partie de l’art de la conduire.\par
Ce fut sous la fin du règne de Louis XIII et sous Louis XIV, que, les armées françaises s’organisant avec plus de perfection, les subsistances commencèrent à y être délivrées régulièrement aux troupes. Les détails des subsistances cessèrent en même temps d’être dans les mains des militaires. Si les généraux eurent la maladresse de s’estimer heureux d’en être débarrassés, les ministres les virent, sans doute avec plaisir, entrer dans leur département ; parce que cela leur assujettit, en quelque sorte, les opérations et les généraux.\par
Les subsistances de nos armées ont été depuis, tour à tour, administrées par entreprise et par régie. M. de Louvois fut le premier ministre qui commença à donner de l’extension et de l’importance à cette branche de détails, jusque-là regardée comme très subalterne. Elle le devenait moins en effet par le changement qui s’était fait dans le système de guerre, par l’augmentation prodigieuse des armées et de leurs attirails, par l’espèce de la plupart des campagnes qui se passaient toutes en sièges. J’ai dit ailleurs comment dès lors il ne se fit presque plus, de part et d’autre, ce que j’appelle la Grande Guerre. La science parut consister à opposer place à place, magasin à magasin. L’amas des approvisionnements précaution sage, quand elle a ses bornes, était dégénéré en manie chez M. de Louvois. Il en avait sur toutes les frontières. Il prétendait par là tenir dans sa main tous les moyens des opérations et décider les plans de campagne. Il les décidait en effet. Ses adulateurs l’appelaient le général des généraux. Je ne prétends pas dire que M. de Louvois n’eût du génie, qu’il n’ait rendu de grands services aux armes de Louis XIV ; mais pour quelques succès passagers, auxquels contribua, pendant sa vie, sa prépondérance de génie et sa supériorité à manier le nouveau système de guerre, sur les cabinets des autres puissances, il occasionna par la suite de grands maux. Il trompa Louis XIV sur sa puissance réelle. Il introduisit un genre de guerre désastreux pour la population et pour les finances. Il augmenta les armées, les dépenses et n’ayant pas sur cela des moyens supérieurs au reste de l’Europe, il n’y gagna rien. Il força seulement les autres princes à se liguer contre Louis XIV et à ruiner leurs États comme lui.\par
Après la mort de M. de Louvois, Louis XIV eut de mauvais ministres et des généraux plus mauvais encore. Cependant la routine était prise et adoptée par toute l’Europe. Il n’était plus possible d’y rien changer. Obligée de faire face partout, la France se trouva accablée sous une défensive malheureuse. Il est inouï, ce que les nouveaux systèmes de subsistances, introduits par M. de Louvois, coûtèrent alors de millions au royaume. Il n’y avait pas de bataille perdue, ou de ville prise, qui n’entraînât des pertes de magasins immenses. Les malheurs accessoires devenaient plus destructifs que le malheur principal […].\par
À cela on ne peut pas objecter que ces magasins étaient formés aux dépens de l’ennemi. Ils l’étaient aux frais de la France. Presque toute la partie de l’Allemagne, où nous faisions la guerre, était notre alliée et les achats, qu’y faisait le Roi, s’y payaient comptant. En Piémont, des ménagements pour la duchesse de Bourgogne faisaient payer, sous main, les livraisons qu’on demandait hautement au pays à titre de contributions. Le royaume était obéré de dettes. Toutes les fournitures de subsistances se faisaient par entreprises. Les marchés des entrepreneurs augmentaient toutes les campagnes. C’était l’usure qui vendait ses services à la nécessité.\par
Notre système de subsistances ne s’est point amélioré depuis la guerre de 1700. Il est devenu de plus en plus financier et ruineux. Le désordre des finances et la routine ont toujours fait recourir aux entreprises […].\par
On semble encore mettre en problème aujourd’hui, s’il vaut mieux administrer la subsistance des armées par régie ou par entreprise. C’est être incertain si l’administration des pays d’État est plus avantageuse que l’administration financière. C’est mettre en doute s’il vaut mieux affermer son champ que de le cultiver soi-même.\par
Tout marché par entreprise fait nécessairement supposer à la société qui contracte (ou cette société est une compagnie de dupes) la convention tacite de gagner sur le marché et la sûreté calculée de ce gain. Il se pourra que, par des malheurs extraordinaires, suivis de beaucoup de désintéressement, la société gagne peu. Mais pour cette chance unique, il y en a mille qui porteront le gain au-delà des espérances supputées. Toute entreprise calculée et conduite par des gens de tête, doit donc leur prospérer. Leurs gains seront moins considérables, en raison de ce qu’ils seront moins avides, plus honnêtes, plus exacts dans leurs fournitures […].\par
Mais si ces associations d’entrepreneurs sont mal composées, alors les gains deviennent illicites et immenses. Alors s’ensuivent les fournitures de mauvais aloi, les déprédations, les pertes supposées ou exagérées aux dépens du Roi, les faux procès-verbaux etc. Alors accourent de toute part, attirés par l’appât de la fortune, le protégé, l’intrigant, l’usurier. Ils se réunissent, ils pénètrent dans les bureaux et dans les antichambres de la cour. Ils proposent des parts, des intérêts. Ils trouvent des appuis. Tant de gens sont avides dans un siècle de luxe et d’intrigues ! Le ministre est séduit par l’offre d’un marché à plus bas prix, il consent à l’entreprise. Cette entreprise se sous-ferme, passe en deux ou trois mains et finit enfin par tomber dans celles d’un homme qui, pour ne pas se ruiner, pour suffire à toutes les rétributions qu’on lui a imposées, est forcé de mal remplir le service dont il est chargé.\par
Frappé de la vérité de ce qu’on vient d’exposer, quand il n’y aurait pas d’autres raisons qu’on donnera ci-après, un gouvernement éclairé devrait donc s’abstenir de toute sorte de marchés par entreprise. Il le devrait, afin de faire, pour l’État, le profit que les entrepreneurs font pour eux-mêmes. Afin d’ôter à ses alentours toute tentation de corruption. Afin d’éviter au public l’éclat de ces fortunes indécentes, élevées par la voie des entreprises. Afin d’empêcher la gangrené que l’exemple de ces fortunes apporte aux mœurs publiques […].\par
On s’est assoupi sur cet objet important et quand on voudra sortir de cet assoupissement, il sera trop tard. La guerre nous menace, l’argent nous manque, dix objets plus pressants occuperont à la fois le ministère. Il en est à peu près d’un État, comme d’un particulier et l’on peut appliquer à l’un, comme à l’autre, ce vieux proverbe : Richesse fait richesse. Un État est-il riche et surtout bien ordonné ? Il peut améliorer sa constitution, il peut exécuter des projets utiles. Est-il à un certain point de désordre et de ruine ? Tout augmente sa triste situation. On ne peut presque point y proposer de changement qui soit bon, ou du moins qui soit facile. On le peut bien moins encore, si, pour dernier malheur, on n’y a point de plan de régénération ; si tous les départements des ministres s’y croisent et s’y nuisent ; s’ils n’ont pas la ressource de venir se raccorder à la volonté générale du souverain ; si tel y est, en un mot, le nombre et l’entrelacement des abus, qu’on ait abandonné le mal à ses progrès. Cette réflexion m’a un peu éloigné du but.\par
Mais ce n’est encore rien, que les inconvénients pécuniaires attachés à notre système de subsistances. Il faut voir comment ce système contrarie les opérations de nos armées. Ce dernier désavantage, au reste, ne tient pas seulement à la maladresse de nos méthodes de subsistance. Il tient à la constitution de nos troupes, à nos mœurs, aux idées reçues parmi nos généraux. Tous ces objets sont liés ensemble par des rapports dont il va être intéressant de démêler l’enchaînement et les abus.\par
Depuis qu’en France les détails de subsistances des armées ne sont plus entre les mains des militaires et qu’ils forment, en quelque sorte, une branche particulière de connaissances, les militaires ne les étudient pas. À peine nommerait-on dix officiers qui connussent les ouvrages qui en traitent. Pourquoi les étudier dit-on ? N’y a-t-il pas de munitionnaires ? D’un autre côté, ces derniers, flattés en secret de se voir initiés aux mystères des opérations et les faisant, à quelques égards, dépendre d’eux, ne manquent pas de jeter des ténèbres sur tous ces détails. La pratique et la combinaison de ces détails composent sans doute une science. Mais ils en exagèrent l’importance et la difficulté. Ils la surchargent de calculs. Ils s’environnent d’écritures. Tout cet appareil en impose aux hommes qui ne percent pas la surface des choses.\par
Un officier général cependant arrive au commandement des armées. Il croit ce qu’il n’a pas étudié, un labyrinthe. Il demande au munitionnaire des résultats relatifs aux opérations qu’il médite.\par
Mais dans le fond, celui-ci restant maître des détails, y étant seul initié, demeure despotique dans sa partie. Il demande à la cour la moitié plus d’équipages, de vivres qu’il n’en faudrait, afin de mieux assurer son service. Peu lui importe que cette multiplicité d’attirails double les embarras et appesantisse l’armée\phantomsection
\label{footnote9}\footnote{Dans les différents marchés d’entreprise, que le gouvernement a passés avec des compagnies de vivres, l’achat des attirails et équipages a toujours été au compte du roi. Il en a été de même des pertes de magasins, des enlèvements de convois, des déchets, ou accidents des matières brutes ou employées, quand ces déchets ont été occasionnés par les marchés de l’armée. Cela posé, il faut que les compagnies soient bien mal administrées, si elles ne font pas des gains considérables. L’on voit ce que le roi gagnerait à faire la fourniture entière des subsistances à son compte, puisqu’il a déjà à sa charge toutes les dépenses de formations d’équipages et d’établissement, de non valeurs, d’accidents et de déchets.}. Il multiplie, à chaque pas, les magasins, les établissements. Tous ces établissements ne se font pas aux frais de la compagnie. S’ils sont pris, c’est au compte du roi. S’ils ne le sont pas, il a des précautions sur tous les points. Il ne peut être fait par l’armée aucun mouvement qui le prenne au dépourvu et c’est ce qu’on appelle alors un service brillant et que le général comble d’éloges. Ici, il supposera des difficultés, afin de se donner le mérite de les vaincre. Là, il fera pencher le général vers une opération dont le résultat sera commode et avantageux à ses propres dispositions. Presque toujours, faute de calculer, l’ensemble des opérations, faute, à cet égard, de lumières qu’il ne peut pas avoir, il regardera ses vivres comme le principal et ils ne sont que l’accessoire […].\par
On voit où je tends. C’est à regretter que nous ayons séparé la science des subsistances de la science de la guerre. Qu’elle ne fasse pas un des objets de notre étude. Que nous en ayons abandonné les détails à des mains étrangères. Si les vivres étaient administrés au compte du roi, pourquoi le général ne serait-il pas lui-même le munitionnaire de son armée ? Nourrir l’armée, est-il donc un objet moins important, moins lié aux opérations, que celui de la camper et de la faire marcher. Le général a, pour ces derniers détails, un maréchal général des logis. Un officier général, habile et de confiance, pourrait de même être chargé par lui des détails des vivres. Cet officier général travaillerait dans son cabinet. Il aurait, sous lui, les employés nécessaires, enfin une administration montée, mais avec le moins de frais et d’appareil possibles. Je donnerai, dans mon grand ouvrage, un plan de cette nouvelle administration, développée et comparée à celle qui a eu lieu jusqu’à présent dans nos armées.\par
Mais pour remettre ainsi l’administration des vivres entre les mains des militaires, il faut que les militaires s’instruisent. Les détails des subsistances ne peuvent pas être maniés par des personnes, qui n’en aient ni la théorie, ni l’habitude. Il existe des sources, dans lesquelles on peut en puiser la connaissance […].\par
Sur la science des subsistances, comme sur toutes les branches de l’art militaire, il faudrait enfin qu’il y eût, en temps de paix, une école bien dirigée. Là, s’essaieraient, se compareraient, se perfectionneraient les différentes méthodes pratiquées, tant chez nous que chez les étrangers. Là, pourraient s’imaginer des moyens de simplifier les détails d’emmagasinement, de fabrications, de transports, de comptabilité etc. Là, des officiers choisis se familiariseraient avec la connaissance et l’inspection de ces détails. Comme dans l’école d’état-major ils s’accoutumeraient aux détails des marches et des reconnaissances. De cette école, en un mot, résulterait l’avantage infini de réunir à la science de la guerre, une branche qui n’aurait jamais dû en être séparée […].\par
Je vais maintenant examiner combien d’autres vices rendent notre système de subsistances dispendieux, routinier et contraire à tous les principes de la guerre.\par
Nos équipages de vivres sont presque toujours trop nombreux. Il n’en faut pas faire un crime aux munitionnaires qui, pour l’ordinaire, aveuglement crus dans leur partie, en règlent la formation. Ils les demandent nombreux, pour assurer le succès de leur service. Ils sont forcés à les demander tels, parce que, dans nos armées, on ne sait pas tirer de ressources des pays où l’on fait la guerre. Parce qu’on n’y est, ni sobre, ni patient. Parce qu’on y murmure, si la distribution est retardée de quelques heures, si le pain n’y est pas toujours de la meilleure qualité. Parce qu’on y murmurerait bien plus, si l’on en manquait un seul jour, ou si l’espèce de la nourriture était changée. Dans le cas que les munitionnaires n’eussent, pour leur service, que les moyens strictement nécessaires, la plupart des généraux, sans calculer ces moyens, leur demanderaient des résultats qui y seraient disproportionnés et ne sauraient ni se prêter à leur situation, ni augmenter leurs moyens, en mettant à profit les ressources du pays, ni changer l’esprit des troupes. Mais, qu’à la première guerre, un bon général soit lui-même régisseur de ses subsistances, ou qu’il ait avec lui un habile régisseur, ils sentiront que tout ce qui allège une armée, la rend plus maniable, plus aisée à faire subsister, plus propre à de grandes opérations. Ils trouveront ensemble cette juste proportion qui doit régler la formation des équipages des vivres, cette proportion avec laquelle on peut atteindre à nourrir l’armée et en même temps ne la pas surcharger. Ils supputeront que, par-delà cette proportion qui doit être relative aux opérations simples et journalières, les moyens des opérations extraordinaires doivent se trouver dans l’industrie, dans les ressources du pays, dans l’esprit de sobriété et de patience qu’il faut donner aux troupes. Attentifs même à diminuer cette proportion le plus qu’ils le pourront, ils la calculeront toujours relativement au pays où doit être le théâtre de la guerre. Il faut certainement des équipages de vivres moins nombreux pour faire la guerre en Flandre, dans un pays couvert de places et de chaussées, que pour la faire en Allemagne, où il y a peu de points d’entrepôt, où les chemins ne sont presque tous que des sentiers tracés dans les terres. Ainsi il faudrait des équipages bien moins nombreux dans le Palatinat, pays qui abonde en denrées, en voitures et en habitants, que dans ces déserts de l’Ukraine, qui furent le tombeau de l’armée de Charles XII. On voit que le premier principe de ce qui est, dans mes idées, la science des subsistances, est de diminuer les attirails et de remplir le plus d’objets avec le moins de moyens possibles.\par
Mais c’est dans la formation de nos magasins qu’il existe des abus bien préjudiciables. Cette partie, absolument indépendante du général, est entre les mains des entrepreneurs. Force est, justice même, qu’en pensant le plus honnêtement possible, ils se conduisent suivant leurs plus grands intérêts. C’est sur l’achat des matières, fait à propos, que les entrepreneurs ont leur gain assuré. Sur tout le reste de la manutention, quand ils servent bien, il y a aussi souvent à perdre qu’à gagner. Or le bon prix des achats est donc, comme on peut se l’imaginer, le but principal de toutes leurs combinaisons. Ils achètent dans les bonnes saisons. Ils ont leurs magasins d’évidence et leurs magasins secrets. Ils ont leurs courtiers, leurs agioteurs ; ils arrhent les denrées sous main et à l’avance dans les pays voisins du théâtre de la guerre et par là trahissent quelquefois le secret des opérations. On me dira que ces achats, ou arrhes feints à propos, peuvent devenir une ruse contre l’ennemi. J’en conviens. Mais, dans la main du général, ce moyen existera tout de même et il ne sera employé que quand il le jugera utile à ses projets. Jusque-là le mal n’est pas encore grand. C’est à l’emplacement des matières que les inconvénients deviennent dangereux. On juge bien qu’à moins que le général ne sache et ne veuille entrer dans ce détail, ces emplacements sont déterminés à la volonté des entrepreneurs et que les entrepreneurs les déterminent le plus souvent relativement aux spéculations bornées et exclusives de leur art. Souvent ces matières se trouvent placées dans des points peu militaires, ou sans relation avec les opérations. Presque toujours ces matières trop dispersées forment une infinité de petits magasins, dont chacun a sa garde, ses employés, ses déchets, ses accidents de guerre ou autres. Souvent tels magasins se trouvent engorgés de matières et alors l’entrepreneur penche vers le parti qui lui en procure la consommation. Quelquefois de même il résiste à un parti qui attirerait trop de consommation sur un point où les matières lui manquent et où il n’en pourrait rassembler qu’en les achetant à haut prix. D’autres fois des magasins trop pleins, parce qu’on s’est hâté d’acheter à des prix favorables ; ou des magasins qu’on veut vider parce que les spéculations apprennent qu’on est au moment de faire de bons achats, exigent que les entrepreneurs fassent consommer. Alors on se garde bien de faire vivre l’armée aux dépens des contributions exigibles du pays, on trouve des difficultés à ce système, on gagne du temps et les denrées de l’entreprise se consomment. De ce dédale inconnu et que je viens de parcourir un moment, émanent, pour l’ordinaire, toutes les combinaisons de subsistances que les entrepreneurs mettent en avant dans le cabinet des généraux. Quand il est question de supputer les moyens de telle ou telle opération, le général s’aperçoit bien d’une résistance ou d’une inclinaison secrète, mais il n’en démêle pas la cause. De grands calculs lui font illusion, communément il n’est ni assez ferme pour résister, ni assez éclaire pour fournir des moyens. Son projet ne s’exécute pas et il se trouve que, loin de commander aux subsistances, ce sont les subsistances qui ont commandé aux opérations. Je le répète, je n’ai personne en vue. Si je peins des abus qui n’ont pas existé, ce sont des abus qu’on peut craindre.\par
Si un général régissait lui-même ses subsistances, ou qu’il eût sous lui un régisseur habile et qui partît avec lui du même système, ils combineraient ensemble le moment des achats, l’espèce et la quantité des matières à se procurer, les lieux où doivent se faire les achats, ceux où il faut les déposer etc. Une infinité de vues économiques ou militaires doivent influer sur ces objets étrangers à la guerre en apparence […].\par
Je ne suis pas exclusif ni outré dans mes opinions. Je ne dirais pas à une armée : « N’ayez point d’équipages de vivres, de magasins, de moyens de transport. Vivez toujours du pays. Avancez, s’il le faut, dans les déserts de l’Ukraine, la Providence vous nourrira ». Je veux, je crois l’avoir déjà dit, qu’une armée ait un équipage de vivres, mais le moins nombreux possible, proportionné à sa force, à la nature du pays où elle doit agir et aux moyens qu’exigent les opérations ordinaires. Je veux que, partant d’un fleuve, d’une frontière, elle ait, sur cette base, des magasins et des entrepôts bien disposés relativement à leur sûreté et au plan de ses opérations. Je veux que, si elle est dans le pays ennemi, ses magasins soient formés aux dépens du pays et par les soins du pays. Je veux, autant qu’on le pourra, que le pays soit chargé de la manutention, comptabilité, conservation, reversement d’un heu à l’autre, afin de n’avoir, au moyen de cela, ni dommages, ni événements, ni employés, ni procès-verbaux à payer. Je veux que, soit en pays ami, soit en pays ennemi, les magasins soient formés des matières qui sont la nourriture habituelle des gens du pays, parce qu’alors on les aura à meilleur compte et en plus grande abondance. Par conséquent, si les habitants se nourrissent de seigle, les troupes s’en nourriront et l’on ne s’assujettira point, parce qu’un règlement de bureau aura déterminé, il y a 80 ans, l’espèce et la forme du pain qui doit être délivré au soldat, à ne leur en distribuer que de cette forme et de cette espèce. Je veux que, tant que les opérations seront simples, faciles, à portée des établissements qu’on aura formés, que le pain se fasse et se délivre dans la règle accoutumée, que la régie remplisse son service avec le plus d’ordre et d’exactitude possible. J’entends que les moyens de transport, qu’on pourra se procurer dans le pays, soient employés aux détails intérieurs de cette manutention, afin que par là les équipages des vivres soient soulagés d’autant, dépérissent moins, soient à portée de l’armée et prêts à la servir efficacement dans une opération extraordinaire. Les mouvements viennent-ils à se multiplier et à se succéder, est-il nécessaire de faire une opération hardie, des marches forcées ? Il faut alors que la régie force de moyens, il faut qu’elle sache s’écarter de ses méthodes de routine et de précision […].\par
Il faut que l’ennemi me voit marcher, quand il me croira enchaîné par des calculs de subsistances. Il faut que ce genre de guerre nouveau l’étonne, ne lui laisse le temps de respirer nulle part et fasse voir, à ses dépens, cette vérité constante, qu’il n’y a presque pas de position tenable devant une armée bien constituée, sobre, patiente et manœuvrière. Les moments de crise passés, mon mouvement ayant rempli son objet, alors les subsistances rentrent dans le système accoutumé d’ordre et de précision. On tient compte aux troupes des efforts qu’elles ont faits, du mal qu’elles ont souffert. C’est par cette alternative bien ménagée, de douceurs et de travaux, qu’on éloigne d’elles le dégoût, l’ennui, l’indiscipline, les maladies. C’est par elle qu’on leur fait faire, dans l’occasion, des choses au-dessus des forces humaines. Enfin, si je suis dans un pays ennemi et que ce pays soit abondant, je suspends les dépenses de la régie pour tout le temps qu’il peut y fournir. Je vis à ses frais. Je les suspends à plus forte raison, si j’y entre en quartier d’hiver. Je fais faire les livraisons par le pays, ainsi que les emmagasinements, les fournitures, les comptabilités. Là, je veux que les troupes soient dédommagées de la fatigue de la campagne, qu’elles vivent chez l’habitant, qu’elles mettent leur solde en réserve. Je règle ce qu’elles peuvent exiger, sur un pied raisonnable et dans l’espèce de denrées que le pays consomme. En même temps que je procure ces douceurs aux troupes, j’établis une discipline de fer pour réprimer les moindres désordres. Pendant cet intervalle de repos, les équipages des vivres se réparent, se remontent, et la régie prépare, dans le silence, ses moyens pour la campagne suivante.\par
Ceci me conduit à une vérité politique importante, qui n’est pas assez sentie par notre gouvernement. C’est qu’à un royaume constitué et puissant, comme la France devrait l’être, il faudrait rarement de grands alliés et jamais de petits. Il devrait surtout éviter d’en avoir dans le pays, ou aux environs du pays où il porte le théâtre de la guerre. C’était une maxime d’État chez les Romains. Ceux qu’ils appelaient leurs alliés, étaient des espèces de vassaux. Ils contribuaient aux frais de la guerre. Ils nourrissaient l’armée, si elle était sur leur territoire. Notre politique de ménagements, de considérations, de subsides secrets, est petite et ruineuse pour un grand peuple. Elle est surtout funeste aux opérations militaires. Elle embarrasse les généraux et met les armées mal à l’aise. La France, au point de splendeur et de prépondérance où devrait la porter un plan de régénération qu’il faut malheureusement désespérer de voir, devrait au milieu de l’Europe dont elle est le centre, se soutenir seule et par son propre poids. Elle devrait avec cette manière franche, large, hardie, qui convient aux grands empires, dire à ses voisins « Je ne veux point m’étendre. Je tâcherai de ne pas me faire d’ennemis et je ne veux point d’alliés ».\par
Je n’ai fait que jeter ici des vues générales sur la nécessité de donner une forme nouvelle à notre système de subsistances. Ces vues veulent être développées et appuyées par des détails ; elles le seront dans mon grand ouvrage […].\par
Mais, pour conclure cet important article, une refonte aussi entière, que celle que je propose dans nos méthodes de subsistances, ne peut avoir lieu, tant qu’on ne changera rien à la constitution de nos troupes et à celle de nos mœurs. Nos troupes ne sont pas constituées militairement. Nos mœurs ne sont pas militaires. Nos soldats et nos officiers encore moins, n’ont ni la frugalité, ni la patience, ni la force de corps, qui sont les qualités primordiales et constitutives des gens de guerre. Ces qualités ne sont pas honorées dans notre siècle, elles y sont affaiblies et tournées en ridicule par le luxe et par l’esprit qui domine. Nous sommes des sybarites. Telle est cependant l’influence de l’exemple et de la mode sur notre nation, à la fois faible et forte, légère et capable de réfléchir, que, si le souverain voulait en changer les mœurs, lui donner l’esprit militaire, apprendre à commander ses armées, les commander, en bannir le luxe, être lui-même frugal et patient à souffrir, avant peu d’années les vertus guerrières y deviendraient communes et respectées, autant qu’elles le sont peu aujourd’hui. L’honneur si facile de régénérer la nation ne tentera-t-il donc jamais un de nos princes ?
\section[{Conclusion}]{Conclusion}\renewcommand{\leftmark}{Conclusion}

\noindent Je termine ici mon essai général de tactique. J’ai examiné cette science dans toutes ses branches et sous tous ses rapports. S’il y a quelques parties que je n’ai pas approfondies, c’est parce que cet essai n’est que l’ébauche d’un plus grand ouvrage. Je veux que le travail, le temps, l’expérience, la critique m’éclairent et donnent plus d’autorité à mes opinions. Je ressemble à ces architectes qui, chargés d’une grande construction, en exposent le relief et attendent, observateurs attentifs, les remarques que fera le public, pour en profiter, avant que d’élever leur édifice.\par
Il est des objets importants, sur lesquels je n’ai presque jeté que des doutes. Tels sont l’examen de notre système actuel, relativement à la formation des armées et à la conduite de la guerre de campagne et les changements que je pense qu’il serait avantageux d’y faire. J’ai appuyé ces doutes d’assez de détails et de raisons, pour mettre sur la voie, quiconque pourra m’entendre. En les donnant, comme assertions, je n’aurais pas mieux convaincu et j’aurais indisposé davantage. Tout ce qui tient au génie et aux généraux, est si délicat à traiter ! Tout ce qui tient au génie porte nécessairement une empreinte systématique. Annoncer des systèmes, c’est déjà mettre contre soi une infinité de gens qui les condamnent, sans les lire. Il est encore plus dangereux de raisonner sur ce qui regarde les généraux. Newton, sans crainte de blasphème, osa attaquer Descartes et l’Europe entière devenue cartésienne. Un autre géomètre pourra élever un système contre Newton. Il ne courra que le danger de la honte, s’il le fait sans génie. Mais, dans la profession militaire, à peine convient-on que l’étude puisse vieillir l’esprit, que le génie puisse devancer les années. Pour avoir le droit de parler de la science des généraux, il faut avoir commandé des armées. La subordination d’âge et de grade est étendue jusque sur les pensées. Hors du service et dans les discussions littéraires, si je peux donner ce nom aux controverses militaires que l’on fait imprimer, cette servitude peut nuire aux progrès de l’art. Mais elle est nécessaire et en ayant cherché peut-être moi-même à m’en affranchir, je ne pense pas qu’il en faille avancer qu’on en dût relâcher les liens, dans un siècle et dans une nation, où des portions de connaissances, répandues dans presque toutes les têtes, donnent à presque toutes la prétention d’opiner et l’apparence des talents.\par
Dans le cours de cet essai, attentif à éviter les personnalités, je me suis quelquefois élevé avec force contre des abus régnants. On m’avait conseillé d’adoucir ces passages. Je l’ai essayé. Mais, sans doute, le fond de mes pensées tenait à mes expressions. Car les mots, que je substituais, ne rendaient plus ce que je voulais dire. Ce serait un talent bien nécessaire, mais je le crois le fruit des années, que de parler froidement des vérités qu’on sent avec chaleur.\par
C’est une encyclopédie, elle seule, que la science militaire. C’est la plus intéressante des sciences, soit qu’on la considère relativement à la variété de ses détails, ou à l’importance de son objet, ou à la gloire et aux grands intérêts qui y tiennent. Puisse cette vérité, sentie par les hommes qui sont destinés à commander les armées, leur faire apercevoir l’immensité de leurs obligations ! Car ce n’est encore rien, que l’acquisition des connaissances qui composent la science militaire. Il faut, pour être un général du premier ordre, savoir employer ces connaissances. Il faut avoir le génie que rien ne peut faire acquérir, le coup d’œil que l’habitude perfectionne et ne peut donner. Il faut réunir un assemblage plus qu’humain, de qualités physiques et morales. Aussi doit-on rester confondu d’étonnement et de respect, à la vue du petit nombre des généraux, que la postérité honore du nom de grands. Il semble que la nature ne les produise çà et là, au milieu des siècles, que pour servir d’époques à la grandeur de l’esprit humain.\par
Les gens de lettres n’ont pas, en général, cette haute idée de la science de la guerre. Ils la croient vague et dénuée de principes positifs. Ce malheureux préjugé est répandu même chez beaucoup de militaires. Faute de n’avoir pas assez étudié leur art, ils ne font point assez de cas de leur profession. Je retirerai de mon travail un grand prix, si j’ouvre les yeux à quelques-uns d’eux. Il est si encourageant, quand on cultive une science, de la voir acquérir de l’estime et de l’importance dans l’opinion des hommes.
 


% at least one empty page at end (for booklet couv)
\ifbooklet
  \pagestyle{empty}
  \clearpage
  % 2 empty pages maybe needed for 4e cover
  \ifnum\modulo{\value{page}}{4}=0 \hbox{}\newpage\hbox{}\newpage\fi
  \ifnum\modulo{\value{page}}{4}=1 \hbox{}\newpage\hbox{}\newpage\fi


  \hbox{}\newpage
  \ifodd\value{page}\hbox{}\newpage\fi
  {\centering\color{rubric}\bfseries\noindent\large
    Hurlus ? Qu’est-ce.\par
    \bigskip
  }
  \noindent Des bouquinistes électroniques, pour du texte libre à participation libre,
  téléchargeable gratuitement sur \href{https://hurlus.fr}{\dotuline{hurlus.fr}}.\par
  \bigskip
  \noindent Cette brochure a été produite par des éditeurs bénévoles.
  Elle n’est pas faîte pour être possédée, mais pour être lue, et puis donnée.
  Que circule le texte !
  En page de garde, on peut ajouter une date, un lieu, un nom ; pour suivre le voyage des idées.
  \par

  Ce texte a été choisi parce qu’une personne l’a aimé,
  ou haï, elle a en tous cas pensé qu’il partipait à la formation de notre présent ;
  sans le souci de plaire, vendre, ou militer pour une cause.
  \par

  L’édition électronique est soigneuse, tant sur la technique
  que sur l’établissement du texte ; mais sans aucune prétention scolaire, au contraire.
  Le but est de s’adresser à tous, sans distinction de science ou de diplôme.
  Au plus direct ! (possible)
  \par

  Cet exemplaire en papier a été tiré sur une imprimante personnelle
   ou une photocopieuse. Tout le monde peut le faire.
  Il suffit de
  télécharger un fichier sur \href{https://hurlus.fr}{\dotuline{hurlus.fr}},
  d’imprimer, et agrafer ; puis de lire et donner.\par

  \bigskip

  \noindent PS : Les hurlus furent aussi des rebelles protestants qui cassaient les statues dans les églises catholiques. En 1566 démarra la révolte des gueux dans le pays de Lille. L’insurrection enflamma la région jusqu’à Anvers où les gueux de mer bloquèrent les bateaux espagnols.
  Ce fut une rare guerre de libération dont naquit un pays toujours libre : les Pays-Bas.
  En plat pays francophone, par contre, restèrent des bandes de huguenots, les hurlus, progressivement réprimés par la très catholique Espagne.
  Cette mémoire d’une défaite est éteinte, rallumons-la. Sortons les livres du culte universitaire, cherchons les idoles de l’époque, pour les briser.
\fi

\ifdev % autotext in dev mode
\fontname\font — \textsc{Les règles du jeu}\par
(\hyperref[utopie]{\underline{Lien}})\par
\noindent \initialiv{A}{lors là}\blindtext\par
\noindent \initialiv{À}{ la bonheur des dames}\blindtext\par
\noindent \initialiv{É}{tonnez-le}\blindtext\par
\noindent \initialiv{Q}{ualitativement}\blindtext\par
\noindent \initialiv{V}{aloriser}\blindtext\par
\Blindtext
\phantomsection
\label{utopie}
\Blinddocument
\fi
\end{document}
