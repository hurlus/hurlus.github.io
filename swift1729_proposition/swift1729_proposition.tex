%%%%%%%%%%%%%%%%%%%%%%%%%%%%%%%%%
% LaTeX model https://hurlus.fr %
%%%%%%%%%%%%%%%%%%%%%%%%%%%%%%%%%

% Needed before document class
\RequirePackage{pdftexcmds} % needed for tests expressions
\RequirePackage{fix-cm} % correct units

% Define mode
\def\mode{a4}

\newif\ifaiv % a4
\newif\ifav % a5
\newif\ifbooklet % booklet
\newif\ifcover % cover for booklet

\ifnum \strcmp{\mode}{cover}=0
  \covertrue
\else\ifnum \strcmp{\mode}{booklet}=0
  \booklettrue
\else\ifnum \strcmp{\mode}{a5}=0
  \avtrue
\else
  \aivtrue
\fi\fi\fi

\ifbooklet % do not enclose with {}
  \documentclass[french,twoside]{book} % ,notitlepage
  \usepackage[%
    papersize={105mm, 297mm},
    inner=12mm,
    outer=12mm,
    top=20mm,
    bottom=15mm,
    marginparsep=0pt,
  ]{geometry}
  \usepackage[fontsize=9.5pt]{scrextend} % for Roboto
\else\ifav
  \documentclass[french,twoside]{book} % ,notitlepage
  \usepackage[%
    a5paper,
    inner=25mm,
    outer=15mm,
    top=15mm,
    bottom=15mm,
    marginparsep=0pt,
  ]{geometry}
  \usepackage[fontsize=12pt]{scrextend}
\else% A4 2 cols
  \documentclass[twocolumn]{report}
  \usepackage[%
    a4paper,
    inner=15mm,
    outer=10mm,
    top=25mm,
    bottom=18mm,
    marginparsep=0pt,
  ]{geometry}
  \setlength{\columnsep}{20mm}
  \usepackage[fontsize=9.5pt]{scrextend}
\fi\fi

%%%%%%%%%%%%%%
% Alignments %
%%%%%%%%%%%%%%
% before teinte macros

\setlength{\arrayrulewidth}{0.2pt}
\setlength{\columnseprule}{\arrayrulewidth} % twocol
\setlength{\parskip}{0pt} % classical para with no margin
\setlength{\parindent}{1.5em}

%%%%%%%%%%
% Colors %
%%%%%%%%%%
% before Teinte macros

\usepackage[dvipsnames]{xcolor}
\definecolor{rubric}{HTML}{800000} % the tonic 0c71c3
\def\columnseprulecolor{\color{rubric}}
\colorlet{borderline}{rubric!30!} % definecolor need exact code
\definecolor{shadecolor}{gray}{0.95}
\definecolor{bghi}{gray}{0.5}

%%%%%%%%%%%%%%%%%
% Teinte macros %
%%%%%%%%%%%%%%%%%
%%%%%%%%%%%%%%%%%%%%%%%%%%%%%%%%%%%%%%%%%%%%%%%%%%%
% <TEI> generic (LaTeX names generated by Teinte) %
%%%%%%%%%%%%%%%%%%%%%%%%%%%%%%%%%%%%%%%%%%%%%%%%%%%
% This template is inserted in a specific design
% It is XeLaTeX and otf fonts

\makeatletter % <@@@


\usepackage{blindtext} % generate text for testing
\usepackage[strict]{changepage} % for modulo 4
\usepackage{contour} % rounding words
\usepackage[nodayofweek]{datetime}
% \usepackage{DejaVuSans} % seems buggy for sffont font for symbols
\usepackage{enumitem} % <list>
\usepackage{etoolbox} % patch commands
\usepackage{fancyvrb}
\usepackage{fancyhdr}
\usepackage{float}
\usepackage{fontspec} % XeLaTeX mandatory for fonts
\usepackage{footnote} % used to capture notes in minipage (ex: quote)
\usepackage{framed} % bordering correct with footnote hack
\usepackage{graphicx}
\usepackage{lettrine} % drop caps
\usepackage{lipsum} % generate text for testing
\usepackage[framemethod=tikz,]{mdframed} % maybe used for frame with footnotes inside
\usepackage{pdftexcmds} % needed for tests expressions
\usepackage{polyglossia} % non-break space french punct, bug Warning: "Failed to patch part"
\usepackage[%
  indentfirst=false,
  vskip=1em,
  noorphanfirst=true,
  noorphanafter=true,
  leftmargin=\parindent,
  rightmargin=0pt,
]{quoting}
\usepackage{ragged2e}
\usepackage{setspace} % \setstretch for <quote>
\usepackage{tabularx} % <table>
\usepackage[explicit]{titlesec} % wear titles, !NO implicit
\usepackage{tikz} % ornaments
\usepackage{tocloft} % styling tocs
\usepackage[fit]{truncate} % used im runing titles
\usepackage{unicode-math}
\usepackage[normalem]{ulem} % breakable \uline, normalem is absolutely necessary to keep \emph
\usepackage{verse} % <l>
\usepackage{xcolor} % named colors
\usepackage{xparse} % @ifundefined
\XeTeXdefaultencoding "iso-8859-1" % bad encoding of xstring
\usepackage{xstring} % string tests
\XeTeXdefaultencoding "utf-8"
\PassOptionsToPackage{hyphens}{url} % before hyperref, which load url package

% TOTEST
% \usepackage{hypcap} % links in caption ?
% \usepackage{marginnote}
% TESTED
% \usepackage{background} % doesn’t work with xetek
% \usepackage{bookmark} % prefers the hyperref hack \phantomsection
% \usepackage[color, leftbars]{changebar} % 2 cols doc, impossible to keep bar left
% \usepackage[utf8x]{inputenc} % inputenc package ignored with utf8 based engines
% \usepackage[sfdefault,medium]{inter} % no small caps
% \usepackage{firamath} % choose firasans instead, firamath unavailable in Ubuntu 21-04
% \usepackage{flushend} % bad for last notes, supposed flush end of columns
% \usepackage[stable]{footmisc} % BAD for complex notes https://texfaq.org/FAQ-ftnsect
% \usepackage{helvet} % not for XeLaTeX
% \usepackage{multicol} % not compatible with too much packages (longtable, framed, memoir…)
% \usepackage[default,oldstyle,scale=0.95]{opensans} % no small caps
% \usepackage{sectsty} % \chapterfont OBSOLETE
% \usepackage{soul} % \ul for underline, OBSOLETE with XeTeX
% \usepackage[breakable]{tcolorbox} % text styling gone, footnote hack not kept with breakable


% Metadata inserted by a program, from the TEI source, for title page and runing heads
\title{\textbf{ Modeste proposition pour empêcher les enfants des pauvres d’être à la charge de leurs parents ou de leur pays et pour les rendre utiles au public }}
\date{1729}
\author{Swift, Jonathan}
\def\elbibl{Swift, Jonathan. 1729. \emph{Modeste proposition pour empêcher les enfants des pauvres d’être à la charge de leurs parents ou de leur pays et pour les rendre utiles au public}}
\def\elsource{\href{http://kropot.free.fr/Swift-proposition.htm}{\dotuline{http://kropot.free.fr/Swift-proposition.htm}}\footnote{\href{http://kropot.free.fr/Swift-proposition.htm}{\url{http://kropot.free.fr/Swift-proposition.htm}}}}

% Default metas
\newcommand{\colorprovide}[2]{\@ifundefinedcolor{#1}{\colorlet{#1}{#2}}{}}
\colorprovide{rubric}{red}
\colorprovide{silver}{lightgray}
\@ifundefined{syms}{\newfontfamily\syms{DejaVu Sans}}{}
\newif\ifdev
\@ifundefined{elbibl}{% No meta defined, maybe dev mode
  \newcommand{\elbibl}{Titre court ?}
  \newcommand{\elbook}{Titre du livre source ?}
  \newcommand{\elabstract}{Résumé\par}
  \newcommand{\elurl}{http://oeuvres.github.io/elbook/2}
  \author{Éric Lœchien}
  \title{Un titre de test assez long pour vérifier le comportement d’une maquette}
  \date{1566}
  \devtrue
}{}
\let\eltitle\@title
\let\elauthor\@author
\let\eldate\@date


\defaultfontfeatures{
  % Mapping=tex-text, % no effect seen
  Scale=MatchLowercase,
  Ligatures={TeX,Common},
}


% generic typo commands
\newcommand{\astermono}{\medskip\centerline{\color{rubric}\large\selectfont{\syms ✻}}\medskip\par}%
\newcommand{\astertri}{\medskip\par\centerline{\color{rubric}\large\selectfont{\syms ✻\,✻\,✻}}\medskip\par}%
\newcommand{\asterism}{\bigskip\par\noindent\parbox{\linewidth}{\centering\color{rubric}\large{\syms ✻}\\{\syms ✻}\hskip 0.75em{\syms ✻}}\bigskip\par}%

% lists
\newlength{\listmod}
\setlength{\listmod}{\parindent}
\setlist{
  itemindent=!,
  listparindent=\listmod,
  labelsep=0.2\listmod,
  parsep=0pt,
  % topsep=0.2em, % default topsep is best
}
\setlist[itemize]{
  label=—,
  leftmargin=0pt,
  labelindent=1.2em,
  labelwidth=0pt,
}
\setlist[enumerate]{
  label={\bf\color{rubric}\arabic*.},
  labelindent=0.8\listmod,
  leftmargin=\listmod,
  labelwidth=0pt,
}
\newlist{listalpha}{enumerate}{1}
\setlist[listalpha]{
  label={\bf\color{rubric}\alph*.},
  leftmargin=0pt,
  labelindent=0.8\listmod,
  labelwidth=0pt,
}
\newcommand{\listhead}[1]{\hspace{-1\listmod}\emph{#1}}

\renewcommand{\hrulefill}{%
  \leavevmode\leaders\hrule height 0.2pt\hfill\kern\z@}

% General typo
\DeclareTextFontCommand{\textlarge}{\large}
\DeclareTextFontCommand{\textsmall}{\small}

% commands, inlines
\newcommand{\anchor}[1]{\Hy@raisedlink{\hypertarget{#1}{}}} % link to top of an anchor (not baseline)
\newcommand\abbr[1]{#1}
\newcommand{\autour}[1]{\tikz[baseline=(X.base)]\node [draw=rubric,thin,rectangle,inner sep=1.5pt, rounded corners=3pt] (X) {\color{rubric}#1};}
\newcommand\corr[1]{#1}
\newcommand{\ed}[1]{ {\color{silver}\sffamily\footnotesize (#1)} } % <milestone ed="1688"/>
\newcommand\expan[1]{#1}
\newcommand\foreign[1]{\emph{#1}}
\newcommand\gap[1]{#1}
\renewcommand{\LettrineFontHook}{\color{rubric}}
\newcommand{\initial}[2]{\lettrine[lines=2, loversize=0.3, lhang=0.3]{#1}{#2}}
\newcommand{\initialiv}[2]{%
  \let\oldLFH\LettrineFontHook
  % \renewcommand{\LettrineFontHook}{\color{rubric}\ttfamily}
  \IfSubStr{QJ’}{#1}{
    \lettrine[lines=4, lhang=0.2, loversize=-0.1, lraise=0.2]{\smash{#1}}{#2}
  }{\IfSubStr{É}{#1}{
    \lettrine[lines=4, lhang=0.2, loversize=-0, lraise=0]{\smash{#1}}{#2}
  }{\IfSubStr{ÀÂ}{#1}{
    \lettrine[lines=4, lhang=0.2, loversize=-0, lraise=0, slope=0.6em]{\smash{#1}}{#2}
  }{\IfSubStr{A}{#1}{
    \lettrine[lines=4, lhang=0.2, loversize=0.2, slope=0.6em]{\smash{#1}}{#2}
  }{\IfSubStr{V}{#1}{
    \lettrine[lines=4, lhang=0.2, loversize=0.2, slope=-0.5em]{\smash{#1}}{#2}
  }{
    \lettrine[lines=4, lhang=0.2, loversize=0.2]{\smash{#1}}{#2}
  }}}}}
  \let\LettrineFontHook\oldLFH
}
\newcommand{\labelchar}[1]{\textbf{\color{rubric} #1}}
\newcommand{\milestone}[1]{\autour{\footnotesize\color{rubric} #1}} % <milestone n="4"/>
\newcommand\name[1]{#1}
\newcommand\orig[1]{#1}
\newcommand\orgName[1]{#1}
\newcommand\persName[1]{#1}
\newcommand\placeName[1]{#1}
\newcommand{\pn}[1]{\IfSubStr{-—–¶}{#1}% <p n="3"/>
  {\noindent{\bfseries\color{rubric}   ¶  }}
  {{\footnotesize\autour{ #1}  }}}
\newcommand\reg{}
% \newcommand\ref{} % already defined
\newcommand\sic[1]{#1}
\newcommand\surname[1]{\textsc{#1}}
\newcommand\term[1]{\textbf{#1}}

\def\mednobreak{\ifdim\lastskip<\medskipamount
  \removelastskip\nopagebreak\medskip\fi}
\def\bignobreak{\ifdim\lastskip<\bigskipamount
  \removelastskip\nopagebreak\bigskip\fi}

% commands, blocks
\newcommand{\byline}[1]{\bigskip{\RaggedLeft{#1}\par}\bigskip}
\newcommand{\bibl}[1]{{\RaggedLeft{#1}\par\bigskip}}
\newcommand{\biblitem}[1]{{\noindent\hangindent=\parindent   #1\par}}
\newcommand{\dateline}[1]{\medskip{\RaggedLeft{#1}\par}\bigskip}
\newcommand{\labelblock}[1]{\medbreak{\noindent\color{rubric}\bfseries #1}\par\mednobreak}
\newcommand{\salute}[1]{\bigbreak{#1}\par\medbreak}
\newcommand{\signed}[1]{\bigbreak\filbreak{\raggedleft #1\par}\medskip}

% environments for blocks (some may become commands)
\newenvironment{borderbox}{}{} % framing content
\newenvironment{citbibl}{\ifvmode\hfill\fi}{\ifvmode\par\fi }
\newenvironment{docAuthor}{\ifvmode\vskip4pt\fontsize{16pt}{18pt}\selectfont\fi\itshape}{\ifvmode\par\fi }
\newenvironment{docDate}{}{\ifvmode\par\fi }
\newenvironment{docImprint}{\vskip6pt}{\ifvmode\par\fi }
\newenvironment{docTitle}{\vskip6pt\bfseries\fontsize{18pt}{22pt}\selectfont}{\par }
\newenvironment{msHead}{\vskip6pt}{\par}
\newenvironment{msItem}{\vskip6pt}{\par}
\newenvironment{titlePart}{}{\par }


% environments for block containers
\newenvironment{argument}{\itshape\parindent0pt}{\vskip1.5em}
\newenvironment{biblfree}{}{\ifvmode\par\fi }
\newenvironment{bibitemlist}[1]{%
  \list{\@biblabel{\@arabic\c@enumiv}}%
  {%
    \settowidth\labelwidth{\@biblabel{#1}}%
    \leftmargin\labelwidth
    \advance\leftmargin\labelsep
    \@openbib@code
    \usecounter{enumiv}%
    \let\p@enumiv\@empty
    \renewcommand\theenumiv{\@arabic\c@enumiv}%
  }
  \sloppy
  \clubpenalty4000
  \@clubpenalty \clubpenalty
  \widowpenalty4000%
  \sfcode`\.\@m
}%
{\def\@noitemerr
  {\@latex@warning{Empty `bibitemlist' environment}}%
\endlist}
\newenvironment{quoteblock}% may be used for ornaments
  {\begin{quoting}}
  {\end{quoting}}

% table () is preceded and finished by custom command
\newcommand{\tableopen}[1]{%
  \ifnum\strcmp{#1}{wide}=0{%
    \begin{center}
  }
  \else\ifnum\strcmp{#1}{long}=0{%
    \begin{center}
  }
  \else{%
    \begin{center}
  }
  \fi\fi
}
\newcommand{\tableclose}[1]{%
  \ifnum\strcmp{#1}{wide}=0{%
    \end{center}
  }
  \else\ifnum\strcmp{#1}{long}=0{%
    \end{center}
  }
  \else{%
    \end{center}
  }
  \fi\fi
}


% text structure
\newcommand\chapteropen{} % before chapter title
\newcommand\chaptercont{} % after title, argument, epigraph…
\newcommand\chapterclose{} % maybe useful for multicol settings
\setcounter{secnumdepth}{-2} % no counters for hierarchy titles
\setcounter{tocdepth}{5} % deep toc
\markright{\@title} % ???
\markboth{\@title}{\@author} % ???
\renewcommand\tableofcontents{\@starttoc{toc}}
% toclof format
% \renewcommand{\@tocrmarg}{0.1em} % Useless command?
% \renewcommand{\@pnumwidth}{0.5em} % {1.75em}
\renewcommand{\@cftmaketoctitle}{}
\setlength{\cftbeforesecskip}{\z@ \@plus.2\p@}
\renewcommand{\cftchapfont}{}
\renewcommand{\cftchapdotsep}{\cftdotsep}
\renewcommand{\cftchapleader}{\normalfont\cftdotfill{\cftchapdotsep}}
\renewcommand{\cftchappagefont}{\bfseries}
\setlength{\cftbeforechapskip}{0em \@plus\p@}
% \renewcommand{\cftsecfont}{\small\relax}
\renewcommand{\cftsecpagefont}{\normalfont}
% \renewcommand{\cftsubsecfont}{\small\relax}
\renewcommand{\cftsecdotsep}{\cftdotsep}
\renewcommand{\cftsecpagefont}{\normalfont}
\renewcommand{\cftsecleader}{\normalfont\cftdotfill{\cftsecdotsep}}
\setlength{\cftsecindent}{1em}
\setlength{\cftsubsecindent}{2em}
\setlength{\cftsubsubsecindent}{3em}
\setlength{\cftchapnumwidth}{1em}
\setlength{\cftsecnumwidth}{1em}
\setlength{\cftsubsecnumwidth}{1em}
\setlength{\cftsubsubsecnumwidth}{1em}

% footnotes
\newif\ifheading
\newcommand*{\fnmarkscale}{\ifheading 0.70 \else 1 \fi}
\renewcommand\footnoterule{\vspace*{0.3cm}\hrule height \arrayrulewidth width 3cm \vspace*{0.3cm}}
\setlength\footnotesep{1.5\footnotesep} % footnote separator
\renewcommand\@makefntext[1]{\parindent 1.5em \noindent \hb@xt@1.8em{\hss{\normalfont\@thefnmark . }}#1} % no superscipt in foot
\patchcmd{\@footnotetext}{\footnotesize}{\footnotesize\sffamily}{}{} % before scrextend, hyperref


%   see https://tex.stackexchange.com/a/34449/5049
\def\truncdiv#1#2{((#1-(#2-1)/2)/#2)}
\def\moduloop#1#2{(#1-\truncdiv{#1}{#2}*#2)}
\def\modulo#1#2{\number\numexpr\moduloop{#1}{#2}\relax}

% orphans and widows
\clubpenalty=9996
\widowpenalty=9999
\brokenpenalty=4991
\predisplaypenalty=10000
\postdisplaypenalty=1549
\displaywidowpenalty=1602
\hyphenpenalty=400
% Copied from Rahtz but not understood
\def\@pnumwidth{1.55em}
\def\@tocrmarg {2.55em}
\def\@dotsep{4.5}
\emergencystretch 3em
\hbadness=4000
\pretolerance=750
\tolerance=2000
\vbadness=4000
\def\Gin@extensions{.pdf,.png,.jpg,.mps,.tif}
% \renewcommand{\@cite}[1]{#1} % biblio

\usepackage{hyperref} % supposed to be the last one, :o) except for the ones to follow
\urlstyle{same} % after hyperref
\hypersetup{
  % pdftex, % no effect
  pdftitle={\elbibl},
  % pdfauthor={Your name here},
  % pdfsubject={Your subject here},
  % pdfkeywords={keyword1, keyword2},
  bookmarksnumbered=true,
  bookmarksopen=true,
  bookmarksopenlevel=1,
  pdfstartview=Fit,
  breaklinks=true, % avoid long links
  pdfpagemode=UseOutlines,    % pdf toc
  hyperfootnotes=true,
  colorlinks=false,
  pdfborder=0 0 0,
  % pdfpagelayout=TwoPageRight,
  % linktocpage=true, % NO, toc, link only on page no
}

\makeatother % /@@@>
%%%%%%%%%%%%%%
% </TEI> end %
%%%%%%%%%%%%%%


%%%%%%%%%%%%%
% footnotes %
%%%%%%%%%%%%%
\renewcommand{\thefootnote}{\bfseries\textcolor{rubric}{\arabic{footnote}}} % color for footnote marks

%%%%%%%%%
% Fonts %
%%%%%%%%%
\usepackage[]{roboto} % SmallCaps, Regular is a bit bold
% \linespread{0.90} % too compact, keep font natural
\newfontfamily\fontrun[]{Roboto Condensed Light} % condensed runing heads
\ifav
  \setmainfont[
    ItalicFont={Roboto Light Italic},
  ]{Roboto}
\else\ifbooklet
  \setmainfont[
    ItalicFont={Roboto Light Italic},
  ]{Roboto}
\else
\setmainfont[
  ItalicFont={Roboto Italic},
]{Roboto Light}
\fi\fi
\renewcommand{\LettrineFontHook}{\bfseries\color{rubric}}
% \renewenvironment{labelblock}{\begin{center}\bfseries\color{rubric}}{\end{center}}

%%%%%%%%
% MISC %
%%%%%%%%

\setdefaultlanguage[frenchpart=false]{french} % bug on part


\newenvironment{quotebar}{%
    \def\FrameCommand{{\color{rubric!10!}\vrule width 0.5em} \hspace{0.9em}}%
    \def\OuterFrameSep{\itemsep} % séparateur vertical
    \MakeFramed {\advance\hsize-\width \FrameRestore}
  }%
  {%
    \endMakeFramed
  }
\renewenvironment{quoteblock}% may be used for ornaments
  {%
    \savenotes
    \setstretch{0.9}
    \normalfont
    \begin{quotebar}
  }
  {%
    \end{quotebar}
    \spewnotes
  }


\renewcommand{\headrulewidth}{\arrayrulewidth}
\renewcommand{\headrule}{{\color{rubric}\hrule}}

% delicate tuning, image has produce line-height problems in title on 2 lines
\titleformat{name=\chapter} % command
  [display] % shape
  {\vspace{1.5em}\centering} % format
  {} % label
  {0pt} % separator between n
  {}
[{\color{rubric}\huge\textbf{#1}}\bigskip] % after code
% \titlespacing{command}{left spacing}{before spacing}{after spacing}[right]
\titlespacing*{\chapter}{0pt}{-2em}{0pt}[0pt]

\titleformat{name=\section}
  [block]{}{}{}{}
  [\vbox{\color{rubric}\large\raggedleft\textbf{#1}}]
\titlespacing{\section}{0pt}{0pt plus 4pt minus 2pt}{\baselineskip}

\titleformat{name=\subsection}
  [block]
  {}
  {} % \thesection
  {} % separator \arrayrulewidth
  {}
[\vbox{\large\textbf{#1}}]
% \titlespacing{\subsection}{0pt}{0pt plus 4pt minus 2pt}{\baselineskip}

\ifaiv
  \fancypagestyle{main}{%
    \fancyhf{}
    \setlength{\headheight}{1.5em}
    \fancyhead{} % reset head
    \fancyfoot{} % reset foot
    \fancyhead[L]{\truncate{0.45\headwidth}{\fontrun\elbibl}} % book ref
    \fancyhead[R]{\truncate{0.45\headwidth}{ \fontrun\nouppercase\leftmark}} % Chapter title
    \fancyhead[C]{\thepage}
  }
  \fancypagestyle{plain}{% apply to chapter
    \fancyhf{}% clear all header and footer fields
    \setlength{\headheight}{1.5em}
    \fancyhead[L]{\truncate{0.9\headwidth}{\fontrun\elbibl}}
    \fancyhead[R]{\thepage}
  }
\else
  \fancypagestyle{main}{%
    \fancyhf{}
    \setlength{\headheight}{1.5em}
    \fancyhead{} % reset head
    \fancyfoot{} % reset foot
    \fancyhead[RE]{\truncate{0.9\headwidth}{\fontrun\elbibl}} % book ref
    \fancyhead[LO]{\truncate{0.9\headwidth}{\fontrun\nouppercase\leftmark}} % Chapter title, \nouppercase needed
    \fancyhead[RO,LE]{\thepage}
  }
  \fancypagestyle{plain}{% apply to chapter
    \fancyhf{}% clear all header and footer fields
    \setlength{\headheight}{1.5em}
    \fancyhead[L]{\truncate{0.9\headwidth}{\fontrun\elbibl}}
    \fancyhead[R]{\thepage}
  }
\fi

\ifav % a5 only
  \titleclass{\section}{top}
\fi

\newcommand\chapo{{%
  \vspace*{-3em}
  \centering % no vskip ()
  {\Large\addfontfeature{LetterSpace=25}\bfseries{\elauthor}}\par
  \smallskip
  {\large\eldate}\par
  \bigskip
  {\Large\selectfont{\eltitle}}\par
  \bigskip
  {\color{rubric}\hline\par}
  \bigskip
  {\Large TEXTE LIBRE À PARTICPATION LIBRE\par}
  \centerline{\small\color{rubric} {hurlus.fr, tiré le \today}}\par
  \bigskip
}}

\newcommand\cover{{%
  \thispagestyle{empty}
  \centering
  {\LARGE\bfseries{\elauthor}}\par
  \bigskip
  {\Large\eldate}\par
  \bigskip
  \bigskip
  {\LARGE\selectfont{\eltitle}}\par
  \vfill\null
  {\color{rubric}\setlength{\arrayrulewidth}{2pt}\hline\par}
  \vfill\null
  {\Large TEXTE LIBRE À PARTICPATION LIBRE\par}
  \centerline{{\href{https://hurlus.fr}{\dotuline{hurlus.fr}}, tiré le \today}}\par
}}

\begin{document}
\pagestyle{empty}
\ifbooklet{
  \cover\newpage
  \thispagestyle{empty}\hbox{}\newpage
  \cover\newpage\noindent Les voyages de la brochure\par
  \bigskip
  \begin{tabularx}{\textwidth}{l|X|X}
    \textbf{Date} & \textbf{Lieu}& \textbf{Nom/pseudo} \\ \hline
    \rule{0pt}{25cm} &  &   \\
  \end{tabularx}
  \newpage
  \addtocounter{page}{-4}
}\fi

\thispagestyle{empty}
\ifaiv
  \twocolumn[\chapo]
\else
  \chapo
\fi
{\it\elabstract}
\bigskip
\makeatletter\@starttoc{toc}\makeatother % toc without new page
\bigskip

\pagestyle{main} % after style

  \noindent C’est un objet de tristesse, pour celui qui traverse cette grande ville ou voyage dans les campagnes, que de voir les rues, les routes et le seuil des masures encombrés de mendiantes, suivies de trois, quatre ou six enfants, tous en guenilles, importunant le passant de leurs mains tendues. Ces mères, plutôt que de travailler pour gagner honnêtement leur vie, sont forcées de passer leur temps à arpenter le pavé, à mendier la pitance de leurs nourrissons sans défense qui, en grandissant, deviendront voleurs faute de trouver du travail, quitteront leur cher Pays natal afin d’aller combattre pour le prétendant d’Espagne, ou partiront encore se vendre aux îles Barbades.\par
Je pense que chacun s’accorde à reconnaître que ce nombre phénoménal d’enfants pendus aux bras, au dos ou aux talons de leur mère, et fréquemment de leur père, constitue dans le déplorable état présent du royaume une très grande charge supplémentaire ; par conséquent, celui qui trouverait un moyen équitable, simple et peu onéreux de faire participer ces enfants à la richesse commune mériterait si bien de l’intérêt public qu’on lui élèverait pour le moins une statue comme bienfaiteur de la nation.\par
Mais mon intention n’est pas, loin de là, de m’en tenir aux seuls enfants des mendiants avérés ; mon projet se conçoit à une bien plus vaste échelle et se propose d’englober tous les enfants d’un âge donné dont les parents sont en vérité aussi incapables d’assurer la subsistance que ceux qui nous demandent la charité dans les rues.\par
Pour ma part, j’ai consacré plusieurs années à réfléchir à ce sujet capital, à examiner avec attention les différents projets des autres penseurs, et y ai toujours trouvé de grossières erreurs de calcul. Il est vrai qu’une mère peut sustenter son nouveau-né de son lait durant toute une année solaire sans recours ou presque à une autre nourriture, du moins avec un complément alimentaire dont le coût ne dépasse pas deux shillings, somme qu’elle pourra aisément se procurer, ou l’équivalent en reliefs de table, par la mendicité, et c’est précisément à l’âge d’un an que je me propose de prendre en charge ces enfants, de sorte qu’au lieu d’être un fardeau pour leurs parents ou leur paroisse et de manquer de pain et de vêtements, ils puissent contribuer à nourrir et, partiellement, à vêtir des multitudes.\par
Mon projet comporte encore cet autre avantage de faire cesser les avortements volontaires et cette horrible pratique des femmes, hélas trop fréquente dans notre société, qui assassinent leurs bâtards, sacrifiant, me semble-t-il, ces bébés innocents pour s’éviter les dépenses plus que la honte, pratique qui tirerait des larmes de compassion du cœur le plus sauvage et le plus inhumain.\par
Étant généralement admis que la population de ce royaume s’élève à un million et demi d’âmes, je déduis qu’il y a environ deux cent mille couples dont la femme est reproductrice, chiffre duquel je retranche environ trente mille couples qui sont capables de subvenir aux besoins de leurs enfants, bien que je craigne qu’il n’y en ait guère autant, compte tenu de la détresse actuelle du royaume, mais cela posé, il nous reste cent soixante-dix mille reproductrices. J’en retranche encore cinquante mille pour tenir compte des fausses couches ou des enfants qui meurent de maladie ou d’accident au cours de la première année. Il reste donc cent vingt mille enfants nés chaque année de parents pauvres. Comment élever et assurer l’avenir de ces multitudes, telle est donc la question puisque, ainsi que je l’ai déjà dit, dans l’état actuel des choses, toutes les méthodes proposées à ce jour se sont révélées totalement impossibles à appliquer, du fait qu’on ne peut trouver d’emploi pour ces gens ni dans l’artisanat ni dans l’agriculture ; que nous ne construisons pas de nouveaux bâtiments (du moins dans les campagnes), pas plus que nous ne cultivons la terre ; il est rare que ces enfants puissent vivre de rapines avant l’âge de six ans, à l’exception de sujets particulièrement doués, bien qu’ils apprennent les rudiments du métier, je dois le reconnaître, beaucoup plus tôt : durant cette période, néanmoins, ils ne peuvent être tenus que pour des apprentis délinquants, ainsi que me l’a rapporté une importante personnalité du comté de Cavan qui m’a assuré ne pas connaître plus d’un ou deux voleurs qualifiés de moins de six ans, dans une région du royaume pourtant renommée pour la pratique compétente et précoce de cet art.\par
Nos marchands m’assurent qu’en dessous de douze ans, les filles pas plus que les garçons ne font de satisfaisants produits négociables, et que même à cet âge, on n’en tire pas plus de trois livres, ou au mieux trois livres et demie à la Bourse, ce qui n’est profitable ni aux parents ni au royaume, les frais de nourriture et de haillons s’élevant au moins à quatre fois cette somme.\par
J’en viens donc à exposer humblement mes propres idées qui, je l’espère, ne soulèveront pas la moindre objection.\par
Un Américain très avisé que j’ai connu à Londres m’a assuré qu’un jeune enfant en bonne santé et bien nourri constitue à l’âge d’un an un mets délicieux, nutritif et sain, qu’il soit cuit en daube, au pot, rôti à la broche ou au four, et j’ai tout lieu de croire qu’il s’accommode aussi bien en fricassée ou en ragoût.\par
Je porte donc humblement à l’attention du public cette proposition : sur ce chiffre estimé de cent vingt mille enfants, on en garderait vingt mille pour la reproduction, dont un quart seulement de mâles – ce qui est plus que nous n’en accordons aux moutons, aux bovins et aux porcs – la raison en étant que ces enfants sont rarement le fruit du mariage, formalité peu prisée de nos sauvages, et qu’en conséquence, un seul mâle suffira à servir quatre femelles. On mettrait en vente les cent mille autres à l’âge d’un an, pour les proposer aux personnes de bien et de qualité à travers le royaume, non sans recommander à la mère de les laisser téter à satiété pendant le dernier mois, de manière à les rendre dodus, et gras à souhait pour une bonne table. Si l’on reçoit, on pourra faire deux plats d’un enfant, et si l’on dîne en famille, on pourra se contenter d’un quartier, épaule ou gigot, qui, assaisonné d’un peu de sel et de poivre, sera excellent cuit au pot le quatrième jour, particulièrement en hiver.\par
J’ai calculé qu’un nouveau-né pèse en moyenne douze livres, et qu’il peut, en une année solaire, s’il est convenablement nourri, atteindre vingt-huit livres.\par
Je reconnais que ce comestible se révélera quelque peu onéreux, en quoi il conviendra parfaitement aux propriétaires terriens qui, ayant déjà sucé la moelle des pères, semblent les mieux qualifiés pour manger la chair des enfants.\par
On trouvera de la chair de nourrisson toute l’année, mais elle sera plus abondante en mars, ainsi qu’un peu avant et après, car un auteur sérieux, un éminent médecin français, nous assure que grâce aux effets prolifiques du régime à base de poisson, il naît, neuf mois environ après le Carême, plus d’enfants dans les pays catholiques qu’en toute saison ; c’est donc à compter d’un an après le Carême que les marchés seront le mieux fournis, étant donné que la proportion de nourrissons papistes dans le royaume est au moins de trois pour un ; par conséquent, mon projet aura l’avantage supplémentaire de réduire le nombre de papistes parmi nous.\par
Ainsi que je l’ai précisé plus haut, subvenir aux besoins d’un enfant de mendiant (catégorie dans laquelle j’inclus les métayers, les journalistes et les quatre cinquièmes des fermiers) revient à deux shillings par an, haillons inclus, et je crois que pas un gentleman ne rechignera à débourser dix shillings pour un nourrisson de boucherie engraissé à point qui, je le répète, fournira quatre plats d’une viande excellente et nourrissante, que l’on traite un ami ou que l’on dîne en famille. Ainsi, les hobereaux apprendront à être de bons propriétaires et verront leur popularité croître parmi leurs métayers, les mères feront un bénéfice net de huit shillings et seront aptes au travail jusqu’à ce qu’elles produisent un autre enfant.\par
Ceux qui sont économes (ce que réclame, je dois bien l’avouer, notre époque) pourront écorcher la pièce avant de la dépecer ; la peau, traitée comme il convient, fera d’admirables gants pour dames et des bottes d’été pour messieurs raffinés.\par
Quant à notre ville de Dublin, on pourrait y aménager des abattoirs, dans les quartiers les plus appropriés, et qu’on en soit assuré, les bouchers ne manqueront pas, bien que je recommande d’acheter plutôt les nourrissons vivants et de les préparer « au sang » comme les cochons à rôtir.\par
Une personne de qualité, un véritable patriote dont je tiens les vertus en haute estime, se fit un plaisir, comme nous discutions récemment de mon projet, d’y apporter le perfectionnement qui suit. De nombreux gentilshommes du royaume ayant, disait-il, exterminé leurs cervidés, leur appétit de gibier pourrait être comblé par les corps de garçonnets et de fillettes entre douze et quatorze ans, ni plus jeunes ni plus âgés, ceux-ci étant de toute façon destinés à mourir de faim en grand nombre dans toutes les provinces, aussi bien les femmes que les hommes, parce qu’ils ne trouveront pas d’emploi : à charge pour leurs parents, s’ils sont vivants, d’en disposer, à défaut la décision reviendrait à leur plus proche famille. Avec tout le respect que je dois à cet excellent ami et patriote méritant, je ne puis tout à fait me ranger à son avis ; car, mon ami américain me l’assure d’expérience, trop d’exercice rend la viande de garçon généralement coriace et maigre, comme celle de nos écoliers, et lui donne un goût désagréable; les engraisser ne serait pas rentable. Quant aux filles, ce serait, à mon humble avis, une perte pour le public parce qu’elles sont à cet âge sur le point de devenir reproductrices. De plus, il n’est pas improbable que certaines personnes scrupuleuses en viennent (ce qui est fort injuste) à censurer cette pratique, au prétexte qu’elle frôle la cruauté, chose qui, je le confesse, a toujours été pour moi l’objection majeure à tout projet, aussi bien intentionné fût-il.\par
Mais à la décharge de mon ami, j’ajoute qu’il m’a fait cet aveu : l’idée lui a été mise en tête par le fameux Sallmanazor, un indigène de l’île de Formose qui vint à Londres voilà vingt ans et qui, dans le cours de la conversation, lui raconta que dans son pays, lorsque le condamné à mort se trouve être une jeune personne, le bourreau vend le corps à des gens de qualité, comme morceau de choix, et que de son temps, la carcasse dodue d’une jeune fille de quatorze années qui avait été crucifiée pour avoir tenté d’empoisonner l’empereur, fut débitée au pied du gibet et vendue au Premier Ministre de sa Majesté Impériale, ainsi qu’à d’autres mandarins de la cour, pour quatre cents couronnes. Et je ne peux vraiment pas nier que si le même usage était fait de certaines jeunes filles dodues de la ville qui, sans un sou vaillant, ne sortent qu’en chaise et se montrent au théâtre et aux assemblées dans des atours d’importation qu’elles ne paieront jamais, le royaume ne s’en porterait pas plus mal.\par
Certains esprits chagrins s’inquiéteront du grand nombre de pauvres qui sont âgés, malades ou infirmes, et l’on m’a invité à réfléchir aux mesures qui permettraient de délivrer la nation de ce fardeau si pénible. Mais je ne vois pas là le moindre problème, car il est bien connu que chaque jour apporte son lot de mort et de corruption, par le froid, la faim, la crasse et la vermine, à un rythme aussi rapide qu’on peut raisonnablement l’espérer. Quant aux ouvriers plus jeunes, ils sont à présent dans une situation presque aussi prometteuse. Ils ne parviennent pas à trouver d’emploi et dépérissent par manque de nourriture, de sorte que si par accident ils sont embauchés comme journaliers, ils n’ont plus la force de travailler ; ainsi sont-ils, de même que leur pays, bien heureusement délivrés des maux à venir.\par
Je me suis trop longtemps écarté de mon sujet, et me propose par conséquent d’y revenir. Je pense que les avantages de ma proposition sont nombreux et évidents, tout autant que de la plus haute importance.\par
D’abord, comme je l’ai déjà fait remarquer, elle réduirait considérablement le nombre des papistes qui se font chaque jour plus envahissants, puisqu’ils sont les principaux reproducteurs de ce pays ainsi que nos plus dangereux ennemis, et restent dans le royaume avec l’intention bien arrêtée de le livrer au Prétendant, dans l’espoir de tirer avantage de l’absence de tant de bons protestants qui ont choisi de s’exiler plutôt que de demeurer sur le sol natal et de payer, contre leur conscience, la dîme au desservant épiscopal.\par
Deuxièmement. Les fermiers les plus pauvres posséderont enfin quelque chose de valeur, un bien saisissable qui les aidera à payer leur loyer au propriétaire, puisque leurs bêtes et leur grain sont déjà saisis et que l’argent est inconnu chez eux.\par
Troisièmement. Attendu que le coût de l’entretien de cent mille enfants de deux ans et plus ne peut être abaissé en dessous du seuil de dix shillings par tête et \emph{per annum}, la richesse publique se trouvera grossie de cinquante mille livres par année, sans compter les bénéfices d’un nouvel aliment introduit à la table de tous les riches gentilshommes du royaume qui jouissent d’un goût un tant soit peu raffiné, et l’argent circulera dans notre pays, les biens consommés étant entièrement d’origine et de manufacture locale.\par
Quatrièmement. En vendant leurs enfants, les reproducteurs permanents, en plus du gain de huit shillings \emph{per annum}, seront débarrassés des frais d’entretien après la première année.\par
Cinquièmement. Nul doute que cet aliment attirerait de nombreux clients dans les auberges dont les patrons ne manqueraient pas de mettre au point les meilleures recettes pour le préparer à la perfection, et leurs établissements seraient ainsi fréquentés par les gentilshommes les plus distingués qui s’enorgueillissent à juste titre de leur science gastronomique ; un cuisinier habile, sachant obliger ses hôtes, trouvera la façon de l’accommoder en plats aussi fastueux qu’ils les affectionnent.\par
Sixièmement. Ce projet constituerait une forte incitation au mariage, que toutes les nations sages ont soit encouragé par des récompenses, soit imposé par des lois et des sanctions. Il accentuerait le dévouement et la tendresse des mères envers leurs enfants, sachant qu’ils ne sont plus là pour toute la vie, ces pauvres bébés dont l’intervention de la société ferait pour elles, d’une certaine façon, une source de profits et non plus de dépenses. Nous devrions voir naître une saine émulation chez les femmes mariées – à celle qui apportera au marché le bébé le plus gras – les hommes deviendraient aussi attentionnés que leurs épouses, durant le temps de leur grossesse, qu’ils le sont aujourd’hui envers leurs juments ou leurs vaches pleines, envers leur truie prête à mettre bas, et la crainte d’une fausse couche les empêcherait de distribuer (ainsi qu’ils le font trop fréquemment) coups de poing ou de pied.\par
On pourrait énumérer beaucoup d’autres avantages : par exemple, la réintégration de quelque mille pièces de bœuf qui viendraient grossir nos exportations de viande salée ; la réintroduction sur le marché de la viande de porc et le perfectionnement de l’art de faire du bon bacon, denrée rendue précieuse à nos palais par la grande destruction du cochon, trop souvent servi frais à nos tables, alors que sa chair ne peut rivaliser, tant en saveur qu’en magnificence, avec celle d’un bébé d’un an, gras à souhait, qui, rôti d’une pièce, fera grande impression au banquet du Lord Maire ou à toute autre réjouissance publique. Mais, dans un souci de concision, je ne m’attarderai ni sur ce point, ni sur beaucoup d’autres.\par
En supposant que mille familles de cette ville deviennent des acheteurs réguliers de viande de nourrisson, sans parler de ceux qui pourraient en consommer à l’occasion d’agapes familiales, mariages et baptêmes en particulier, j’ai calculé que Dublin offrirait un débouché annuel d’environ vingt mille pièces tandis que les vingt mille autres s’écouleraient dans le reste du royaume (où elles se vendraient sans doute à un prix un peu inférieur).\par
Je ne vois aucune objection possible à cette proposition, si ce n’est qu’on pourra faire valoir qu’elle réduira considérablement le nombre d’habitants du royaume. Je revendique ouvertement ce point, qui était en fait mon intention déclarée en offrant ce projet au public. Je désire faire remarquer au lecteur que j’ai conçu ce remède pour le seul Royaume d’Irlande et pour nul autre Etat au monde, passé, présent, et sans doute à venir. u’on ne vienne donc pas me parler d’autres expédients : d’imposer une taxe de cinq shillings par livre de revenus aux non-résidents ; de refuser l’usage des vêtements et des meubles qui ne sont pas d’origine et de fabrication irlandaise ; de rejeter rigoureusement les articles et ustensiles encourageant au luxe venu de l’étranger ; de remédier à l’expansion de l’orgueil, de la vanité, de la paresse et de la futilité chez nos femmes ; d’implanter un esprit d’économie, de prudence et de tempérance ; d’apprendre à aimer notre Pays, matière en laquelle nous surpassent même les Lapons et les habitants de Topinambou ; d’abandonner nos querelles et nos divisions, de cesser de nous comporter comme les Juifs qui s’égorgeaient entre eux pendant qu’on prenait leur ville, de faire preuve d’un minimum de scrupules avant de brader notre pays et nos consciences ; d’apprendre à nos propriétaires terriens à montrer un peu de pitié envers leurs métayers. Enfin, d’insuffler l’esprit d’honnêteté, de zèle et de compétence à nos commerçants qui, si l’on parvenait aujourd’hui à imposer la décision de n’acheter que les produits irlandais, s’uniraient immédiatement pour tricher et nous escroquer sur la valeur, la mesure et la qualité, et ne pourraient être convaincus de faire ne serait-ce qu’une proposition équitable de juste prix, en dépit d’exhortations ferventes et répétées.\par
Par conséquent, je le redis, qu’on ne vienne pas me parler de ces expédients, ni d’autres mesures du même ordre, tant qu’il n’existe pas le moindre espoir qu’on puisse tenter un jour, avec vaillance et sincérité, de les mettre en pratique.\par
En ce qui me concerne, je me suis épuisé des années durant à proposer des théories vaines, futiles et utopiques, et j’avais perdu tout espoir de succès quand, par bonheur, je suis tombé sur ce plan qui, bien qu’étant complètement nouveau, possède quelque chose e solide et de réel, n’exige que peu d’efforts et aucune dépense, peut être entièrement exécuté par nous-même et grâce auquel nous ne courrons pas le moindre risque de mécontenter l’Angleterre. Car ce type de produit ne peut être exporté, la viande d’enfant tant trop tendre pour supporter un long séjour dans le sel, encore que je pourrai nommer un pays qui se ferait un plaisir de dévorer notre nation, même sans sel.\par
Après tout, je ne suis pas si farouchement accroché à mon opinion que j’en réfuterais toute autre proposition, émise par des hommes sages, qui se révélerait aussi innocente, bon marché, facile et efficace. Mais avant qu’un projet de cette sorte soit avancé pour contredire le mien et offrir une meilleure solution, je conjure l’auteur, ou les auteurs, de bien vouloir considérer avec mûre attention ces deux points. Premièrement, en l’état actuel des choses, comment ils espèrent parvenir à nourrir cent mille bouches inutiles et à vêtir cent mille dos. Deuxièmement, tenir compte de l’existence à travers ce royaume d’un bon million de créatures apparemment humaines dont tous les moyens de subsistance mis en commun laisseraient un déficit de deux millions de livres sterling ; adjoindre les mendiants par profession à la masse des fermiers, métayers et ouvriers agricoles, avec femmes et enfants, qui sont mendiants de fait. Je conjure les hommes d’état qui sont opposés à ma proposition, et assez hardis peut-être pour tenter d’apporter une autre réponse, d’aller auparavant demander aux parents de ces mortels s’ils ne regarderaient pas aujourd’hui comme un grand bonheur d’avoir été vendus comme viande de boucherie à l’âge de un an, de la manière que je prescris, et ; d’avoir évité ainsi toute la série d’infortunes par lesquelles ils ont passé jusqu’ici, l’oppression des propriétaires, l’impossibilité de régler leurs termes sans argent ni travail, les privations de toutes sortes, sans toit ni vêtement pour les protéger des rigueurs de l’hiver, et la perspective inévitable de léguer pareille misère, ou pire encore, à leur progéniture, génération après génération.\par
D’un cœur sincère, j’affirme n’avoir pas le moindre intérêt personnel à tenter de promouvoir cette œuvre nécessaire, je n’ai pour seule motivation que le bien de mon pays, je ne cherche qu’à développer notre commerce, à assurer le bien-être de nos enfants, à soulager les pauvres et à procurer un peu d’agrément aux riches. Je n’ai pas d’enfants dont la vente puisse me rapporter le moindre penny ; le plus jeune a neuf ans et ma femme a passé l’âge d’être mère.
 


% at least one empty page at end (for booklet couv)
\ifbooklet
  \pagestyle{empty}
  \clearpage
  % 2 empty pages maybe needed for 4e cover
  \ifnum\modulo{\value{page}}{4}=0 \hbox{}\newpage\hbox{}\newpage\fi
  \ifnum\modulo{\value{page}}{4}=1 \hbox{}\newpage\hbox{}\newpage\fi


  \hbox{}\newpage
  \ifodd\value{page}\hbox{}\newpage\fi
  {\centering\color{rubric}\bfseries\noindent\large
    Hurlus ? Qu’est-ce.\par
    \bigskip
  }
  \noindent Des bouquinistes électroniques, pour du texte libre à participation libre,
  téléchargeable gratuitement sur \href{https://hurlus.fr}{\dotuline{hurlus.fr}}.\par
  \bigskip
  \noindent Cette brochure a été produite par des éditeurs bénévoles.
  Elle n’est pas faîte pour être possédée, mais pour être lue, et puis donnée.
  Que circule le texte !
  En page de garde, on peut ajouter une date, un lieu, un nom ; pour suivre le voyage des idées.
  \par

  Ce texte a été choisi parce qu’une personne l’a aimé,
  ou haï, elle a en tous cas pensé qu’il partipait à la formation de notre présent ;
  sans le souci de plaire, vendre, ou militer pour une cause.
  \par

  L’édition électronique est soigneuse, tant sur la technique
  que sur l’établissement du texte ; mais sans aucune prétention scolaire, au contraire.
  Le but est de s’adresser à tous, sans distinction de science ou de diplôme.
  Au plus direct ! (possible)
  \par

  Cet exemplaire en papier a été tiré sur une imprimante personnelle
   ou une photocopieuse. Tout le monde peut le faire.
  Il suffit de
  télécharger un fichier sur \href{https://hurlus.fr}{\dotuline{hurlus.fr}},
  d’imprimer, et agrafer ; puis de lire et donner.\par

  \bigskip

  \noindent PS : Les hurlus furent aussi des rebelles protestants qui cassaient les statues dans les églises catholiques. En 1566 démarra la révolte des gueux dans le pays de Lille. L’insurrection enflamma la région jusqu’à Anvers où les gueux de mer bloquèrent les bateaux espagnols.
  Ce fut une rare guerre de libération dont naquit un pays toujours libre : les Pays-Bas.
  En plat pays francophone, par contre, restèrent des bandes de huguenots, les hurlus, progressivement réprimés par la très catholique Espagne.
  Cette mémoire d’une défaite est éteinte, rallumons-la. Sortons les livres du culte universitaire, cherchons les idoles de l’époque, pour les briser.
\fi

\ifdev % autotext in dev mode
\fontname\font — \textsc{Les règles du jeu}\par
(\hyperref[utopie]{\underline{Lien}})\par
\noindent \initialiv{A}{lors là}\blindtext\par
\noindent \initialiv{À}{ la bonheur des dames}\blindtext\par
\noindent \initialiv{É}{tonnez-le}\blindtext\par
\noindent \initialiv{Q}{ualitativement}\blindtext\par
\noindent \initialiv{V}{aloriser}\blindtext\par
\Blindtext
\phantomsection
\label{utopie}
\Blinddocument
\fi
\end{document}
