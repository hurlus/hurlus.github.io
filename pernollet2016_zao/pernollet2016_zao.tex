%%%%%%%%%%%%%%%%%%%%%%%%%%%%%%%%%
% LaTeX model https://hurlus.fr %
%%%%%%%%%%%%%%%%%%%%%%%%%%%%%%%%%

% Needed before document class
\RequirePackage{pdftexcmds} % needed for tests expressions
\RequirePackage{fix-cm} % correct units

% Define mode
\def\mode{a4}

\newif\ifaiv % a4
\newif\ifav % a5
\newif\ifbooklet % booklet
\newif\ifcover % cover for booklet

\ifnum \strcmp{\mode}{cover}=0
  \covertrue
\else\ifnum \strcmp{\mode}{booklet}=0
  \booklettrue
\else\ifnum \strcmp{\mode}{a5}=0
  \avtrue
\else
  \aivtrue
\fi\fi\fi

\ifbooklet % do not enclose with {}
  \documentclass[twoside]{book} % ,notitlepage
  \usepackage[%
    papersize={105mm, 297mm},
    inner=12mm,
    outer=12mm,
    top=20mm,
    bottom=15mm,
    marginparsep=3pt,
    marginpar=7mm,
  ]{geometry}
  \usepackage[fontsize=9.5pt]{scrextend} % for Roboto
\else\ifav % A5
  \documentclass[twoside]{book} % ,notitlepage
  \usepackage[%
    a5paper
  ]{geometry}
  \usepackage[fontsize=12pt]{scrextend}
\else% A4 2 cols
  \documentclass[twocolumn]{report}
  \usepackage[%
    a4paper,
    inner=15mm,
    outer=10mm,
    top=25mm,
    bottom=18mm,
    marginparsep=0pt,
  ]{geometry}
  \setlength{\columnsep}{20mm}
  \usepackage[fontsize=9.5pt]{scrextend}
\fi\fi

%%%%%%%%%%%%%%
% Alignments %
%%%%%%%%%%%%%%
% before teinte macros

\setlength{\arrayrulewidth}{0.2pt}
\setlength{\columnseprule}{\arrayrulewidth} % twocol

%%%%%%%%%%
% Colors %
%%%%%%%%%%
% before Teinte macros

\usepackage[dvipsnames]{xcolor}
\definecolor{rubric}{HTML}{0c71c3} % the tonic
\def\columnseprulecolor{\color{rubric}}
\colorlet{borderline}{rubric!30!} % definecolor need exact code
\definecolor{shadecolor}{gray}{0.95}
\definecolor{bghi}{gray}{0.5}

%%%%%%%%%%%%%%%%%
% Teinte macros %
%%%%%%%%%%%%%%%%%
%%%%%%%%%%%%%%%%%%%%%%%%%%%%%%%%%%%%%%%%%%%%%%%%%%%
% <TEI> generic (LaTeX names generated by Teinte) %
%%%%%%%%%%%%%%%%%%%%%%%%%%%%%%%%%%%%%%%%%%%%%%%%%%%
% This template is inserted in a specific design
% It is XeLaTeX and otf fonts

\makeatletter % <@@@

\setlength{\parskip}{0pt} % 1pt allow better vertical justification
\setlength{\parindent}{1.5em}

\usepackage{alphalph} % for alph couter z, aa, ab…
\usepackage{blindtext} % generate text for testing
\usepackage{booktabs} % for tables: \toprule, \midrule…
\usepackage[strict]{changepage} % for modulo 4
\usepackage{contour} % rounding words
\usepackage[nodayofweek]{datetime}
\usepackage{enumitem} % <list>
\usepackage{etoolbox} % patch commands
\usepackage{fancyvrb}
\usepackage{fancyhdr}
\usepackage{float}
\usepackage{fontspec} % XeLaTeX mandatory for fonts
\usepackage{footnote} % used to capture notes in minipage (ex: quote)
\usepackage{graphicx}
\usepackage{lettrine} % drop caps
\usepackage{lipsum} % generate text for testing
\usepackage{relsize} % \smaller \larger (ex: quotes in body and footnotes)
\usepackage{manyfoot} % for parallel footnote numerotation
\usepackage[framemethod=tikz,]{mdframed} % maybe used for frame with footnotes inside
\usepackage[defaultlines=2,all]{nowidow} % at least 2 lines by par (works well!)
\usepackage{pdftexcmds} % needed for tests expressions
\usepackage{poetry} % <l>, bad for theater
\usepackage{polyglossia} % bug Warning: "Failed to patch part"
\usepackage[%
  indentfirst=false,
  vskip=1em,
  noorphanfirst=true,
  noorphanafter=true,
  leftmargin=\parindent,
  rightmargin=0pt,
]{quoting}
\usepackage{ragged2e}
\usepackage{setspace} % \setstretch for <quote>
\usepackage{scrextend} % KOMA-common, used for addmargin
\usepackage{tabularx} % <table>
\usepackage[explicit]{titlesec} % wear titles, !NO implicit
\usepackage{tikz} % ornaments
\usepackage{tocloft} % styling tocs
\usepackage[fit]{truncate} % used im runing titles
\usepackage{unicode-math}
\usepackage[normalem]{ulem} % breakable \uline, normalem is absolutely necessary to keep \emph
\usepackage{xcolor} % named colors
\usepackage{xparse} % @ifundefined
\XeTeXdefaultencoding "iso-8859-1" % bad encoding of xstring
\usepackage{xstring} % string tests
\XeTeXdefaultencoding "utf-8"

\defaultfontfeatures{
  % Mapping=tex-text, % no effect seen
  Scale=MatchLowercase,
  Ligatures={TeX,Common},
}
\newfontfamily\zhfont{Noto Sans CJK SC}

% Metadata inserted by a program, from the TEI source, for title page and runing heads
\title{Zao\par
\medskip
\emph{Histoire courte}}
\date{2016}
\author{Richard Pernollet}
\def\elbibl{Richard Pernollet. 2016. \emph{Zao}}
\def\elabstract{%
 
\labelblock{Chapeau de l’auteur}

 \noindent Encore une histoire de fin du monde, des enfants survivent dans un bois au milieu de cabanes. Ils se rendent compte soudain que leur maître Zao a disparu. Un des grands est désigné pour aller le chercher au lac, mais il se fait surprendre par la nuit. En fait, tout le monde le pressent, Zao avait annoncé qu’un jour il disparaîtrait.\par
 
\labelblock{Avertissement de l’édition}

 \noindent L’auteur étant bien vivant, ce texte est protégé par le droit d’auteur. Il a exprimé l’intention que l’“œuvre“ de son esprit (ce sont les termes de la loi, pas du tout les siens) soit librement communicable et partageable, mais sans profit commercial, en gardant le nom de l’auteur, et sans modification. Le choix de cette \href{https://creativecommons.org/licenses/by-nc-nd/4.0/deed.fr}{\dotuline{license CC BY-NC-ND}}\footnote{\href{https://creativecommons.org/licenses/by-nc-nd/4.0/deed.fr}{\url{https://creativecommons.org/licenses/by-nc-nd/4.0/deed.fr}}} n’est pas seulement l’effet d’opinions politiques, elle est aussi conséquente avec son esthétique. Son texte a été longtemps mûri, puis jeté tout d’un trait, il témoigne d’un moment, signé, daté. Ce texte est un morceau de temps libre qui ne peut plus être modifié ; de son temps à lui et personne d’autre. Ce temps, il ne le vend pas, il le partage.
 \vfill

}
\def\elsource{ \href{https://labarresymbolique.files.wordpress.com/2016/10/zao-octobre-20161.pdf}{\dotuline{LaBarreSymbolique}}\footnote{\href{https://labarresymbolique.files.wordpress.com/2016/10/zao-octobre-20161.pdf}{\url{https://labarresymbolique.files.wordpress.com/2016/10/zao-octobre-20161.pdf}}} }
\def\eltitlepage{%
{\centering\parindent0pt
  {\LARGE\addfontfeature{LetterSpace=25}\bfseries Richard Pernollet\par}\bigskip
  {\Large 2016\par}\bigskip
  {\LARGE
\bigskip\textbf{Zao}\par
\bigskip\emph{Histoire courte}\par

  }
}

}

% Default metas
\newcommand{\colorprovide}[2]{\@ifundefinedcolor{#1}{\colorlet{#1}{#2}}{}}
\colorprovide{rubric}{red}
\colorprovide{silver}{lightgray}
\@ifundefined{syms}{\newfontfamily\syms{DejaVu Sans}}{}
\newif\ifdev
\@ifundefined{elbibl}{% No meta defined, maybe dev mode
  \newcommand{\elbibl}{Titre court ?}
  \newcommand{\elbook}{Titre du livre source ?}
  \newcommand{\elabstract}{Résumé\par}
  \newcommand{\elurl}{http://oeuvres.github.io/elbook/2}
  \author{Éric Lœchien}
  \title{Un titre de test assez long pour vérifier le comportement d’une maquette}
  \date{1566}
  \devtrue
}{}
\let\eltitle\@title
\let\elauthor\@author
\let\eldate\@date




% generic typo commands
\newcommand{\astermono}{\medskip\centerline{\color{rubric}\large\selectfont{\syms ✻}}\medskip\par}%
\newcommand{\astertri}{\medskip\par\centerline{\color{rubric}\large\selectfont{\syms ✻\,✻\,✻}}\medskip\par}%
\newcommand{\asterism}{\bigskip\par\noindent\parbox{\linewidth}{\centering\color{rubric}\large{\syms ✻}\\{\syms ✻}\hskip 0.75em{\syms ✻}}\bigskip\par}%

% lists
\newlength{\listmod}
\setlength{\listmod}{\parindent}
\setlist{
  itemindent=!,
  listparindent=\listmod,
  labelsep=0.2\listmod,
  parsep=0pt,
  % topsep=0.2em, % default topsep is best
}
\setlist[itemize]{
  label=—,
  leftmargin=0pt,
  labelindent=1.2em,
  labelwidth=0pt,
}
\setlist[enumerate]{
  label={\arabic*°},
  labelindent=0.8\listmod,
  leftmargin=\listmod,
  labelwidth=0pt,
}
% list for big items
\newlist{decbig}{enumerate}{1}
\setlist[decbig]{
  label={\bf\color{rubric}\arabic*.},
  labelindent=0.8\listmod,
  leftmargin=\listmod,
  labelwidth=0pt,
}
\newlist{listalpha}{enumerate}{1}
\setlist[listalpha]{
  label={\bf\color{rubric}\alph*.},
  leftmargin=0pt,
  labelindent=0.8\listmod,
  labelwidth=0pt,
}
\newcommand{\listhead}[1]{\hspace{-1\listmod}\emph{#1}}

\renewcommand{\hrulefill}{%
  \leavevmode\leaders\hrule height 0.2pt\hfill\kern\z@}

% General typo
\DeclareTextFontCommand{\textlarge}{\large}
\DeclareTextFontCommand{\textsmall}{\small}

% commands, inlines
\newcommand{\anchor}[1]{\Hy@raisedlink{\hypertarget{#1}{}}} % link to top of an anchor (not baseline)
\newcommand\abbr[1]{#1}
\newcommand{\autour}[1]{\tikz[baseline=(X.base)]\node [draw=rubric,thin,rectangle,inner sep=1.5pt, rounded corners=3pt] (X) {\color{rubric}#1};}
\newcommand\corr[1]{#1}
\newcommand{\ed}[1]{ {\color{silver}\sffamily\footnotesize (#1)} } % <milestone ed="1688"/>
\newcommand\expan[1]{#1}
\newcommand\foreign[1]{\emph{#1}}
\newcommand\gap[1]{#1}
\renewcommand{\LettrineFontHook}{\color{rubric}}
\newcommand{\initial}[2]{\lettrine[lines=2, loversize=0.3, lhang=0.3]{#1}{#2}}
\newcommand{\initialiv}[2]{%
  \let\oldLFH\LettrineFontHook
  % \renewcommand{\LettrineFontHook}{\color{rubric}\ttfamily}
  \IfSubStr{QJ’}{#1}{
    \lettrine[lines=4, lhang=0.2, loversize=-0.1, lraise=0.2]{\smash{#1}}{#2}
  }{\IfSubStr{É}{#1}{
    \lettrine[lines=4, lhang=0.2, loversize=-0, lraise=0]{\smash{#1}}{#2}
  }{\IfSubStr{ÀÂ}{#1}{
    \lettrine[lines=4, lhang=0.2, loversize=-0, lraise=0, slope=0.6em]{\smash{#1}}{#2}
  }{\IfSubStr{A}{#1}{
    \lettrine[lines=4, lhang=0.2, loversize=0.2, slope=0.6em]{\smash{#1}}{#2}
  }{\IfSubStr{V}{#1}{
    \lettrine[lines=4, lhang=0.2, loversize=0.2, slope=-0.5em]{\smash{#1}}{#2}
  }{
    \lettrine[lines=4, lhang=0.2, loversize=0.2]{\smash{#1}}{#2}
  }}}}}
  \let\LettrineFontHook\oldLFH
}
\newcommand{\labelchar}[1]{\textbf{\color{rubric} #1}}
\newcommand{\lnatt}[1]{\reversemarginpar\marginpar[\sffamily\scriptsize #1]{}}
\newcommand{\milestone}[1]{\autour{\footnotesize\color{rubric} #1}} % <milestone n="4"/>
\newcommand\name[1]{#1}
\newcommand\orig[1]{#1}
\newcommand\orgName[1]{#1}
\newcommand\persName[1]{#1}
\newcommand\placeName[1]{#1}
\newcommand{\pn}[1]{\IfSubStr{-—–¶}{#1}% <p n="3"/>
  {\noindent{\bfseries\color{rubric}   ¶  }}
  {{\footnotesize\autour{#1}}}}
\newcommand\reg{}
% \newcommand\ref{} % already defined
\newcommand\sic[1]{#1}
\newcommand\surname[1]{\textsc{#1}}
\newcommand\term[1]{\textbf{#1}}
\newcommand\zh[1]{{\zhfont #1}}


\def\mednobreak{\ifdim\lastskip<\medskipamount
  \removelastskip\nopagebreak\medskip\fi}
\def\bignobreak{\ifdim\lastskip<\bigskipamount
  \removelastskip\nopagebreak\bigskip\fi}

% commands, blocks

\newcommand{\byline}[1]{\bigskip{\RaggedLeft{#1}\par}\bigskip}
% \setlength{\RaggedLeftLeftskip}{2em plus \leftskip}
\newcommand{\bibl}[1]{{\RaggedLeft\normalfont #1\par}}
\newcommand{\biblitem}[1]{{\noindent\hangindent=\parindent   #1\par}}
\newcommand{\castItem}[1]{{\noindent\hangindent=\parindent #1\par}}
\newcommand{\dateline}[1]{\medskip{\RaggedLeft{#1}\par}\bigskip}
\newcommand{\docAuthor}[1]{{\large\bigskip #1 \par\bigskip}}
\newcommand{\docDate}[1]{#1 \ifvmode\par\fi }
\newcommand{\docImprint}[1]{\ifvmode\medskip\fi #1 \ifvmode\par\fi }
\newcommand{\labelblock}[1]{\medbreak{\noindent\color{rubric}\bfseries #1}\par\mednobreak}
\newcommand{\question}[1]{\bigbreak{\RaggedRight\noindent\emph{#1}\par}\mednobreak}
\newcommand{\salute}[1]{\bigbreak{#1}\par\medbreak}
\newcommand{\signed}[1]{\medskip{\RaggedLeft #1\par}\bigbreak} % supposed bottom
\newcommand{\speaker}[1]{\medskip{\Centering\sffamily #1 \par\nopagebreak}} % supposed bottom
\newcommand{\stagescene}[1]{{\Centering\sffamily\textsf{#1}\par}\bigskip}
\newcommand{\stageblock}[1]{\begingroup\leftskip\parindent\noindent\it\sffamily\footnotesize #1\par\endgroup} % left margin, better than list envs
\newcommand{\lpar}[1]{\noindent\hangindent=2\parindent  #1\par} % sp/l
\newcommand{\trailer}[1]{{\Centering\bigskip #1\par}} % sp/l

% environments for blocks (some may become commands)
\newenvironment{borderbox}{}{} % framing content
\newenvironment{citbibl}{\ifvmode\hfill\fi}{\ifvmode\par\fi }
\newenvironment{msHead}{\vskip6pt}{\par}
\newenvironment{msItem}{\vskip6pt}{\par}


% environments for block containers
\newenvironment{argument}{\itshape\parindent0pt}{\bigskip}
\newenvironment{biblfree}{}{\ifvmode\par\fi }
\newenvironment{bibitemlist}[1]{%
  \list{\@biblabel{\@arabic\c@enumiv}}%
  {%
    \settowidth\labelwidth{\@biblabel{#1}}%
    \leftmargin\labelwidth
    \advance\leftmargin\labelsep
    \@openbib@code
    \usecounter{enumiv}%
    \let\p@enumiv\@empty
    \renewcommand\theenumiv{\@arabic\c@enumiv}%
  }
  \sloppy
  \clubpenalty4000
  \@clubpenalty \clubpenalty
  \widowpenalty4000%
  \sfcode`\.\@m
}%
{\def\@noitemerr
  {\@latex@warning{Empty `bibitemlist' environment}}%
\endlist}
\newenvironment{docTitle}{\LARGE\bigskip\bfseries\onehalfspacing}{\bigskip}
% leftskip makes big bugs in Lexmark printing \sffamily
\newenvironment{epigraph}{\begin{addmargin}[2\parindent]{0em}\sffamily\large\setstretch{0.95}}{\end{addmargin}\bigskip}
\newenvironment{quoteblock}
  {\begin{quoting}\setstretch{0.9}} %
  {\end{quoting}}
\newenvironment{frametext}
  {\begin{mdframed}[default]} %
  {\end{mdframed}}

\quotingsetup{vskip=0pt}
\newcommand{\quoteskip}{\medskip}
\newenvironment{titlePage}
  {\Centering}
  {}






% table () is preceded and finished by custom command
\renewcommand\tabularxcolumn[1]{m{#1}}% for vertical centering text in X column
\newcommand{\tableopen}[1]{%
  \ifnum\strcmp{#1}{wide}=0{%
    \begin{center}
  }
  \else\ifnum\strcmp{#1}{long}=0{%
    \begin{center}
  }
  \else{%
    \begin{center}
  }
  \fi\fi
}
\newcommand{\tableclose}[1]{%
  \ifnum\strcmp{#1}{wide}=0{%
    \end{center}
  }
  \else\ifnum\strcmp{#1}{long}=0{%
    \end{center}
  }
  \else{%
    \end{center}
  }
  \fi\fi
}


% text structure
\newcommand\chapteropen{} % before chapter title
\newcommand\chaptercont{} % after title, argument, epigraph…
\newcommand\chapterclose{} % maybe useful for multicol settings
\setcounter{secnumdepth}{-2} % no counters for hierarchy titles
\setcounter{tocdepth}{5} % deep toc
\renewcommand\tableofcontents{\@starttoc{toc}}
% toclof format
% \renewcommand{\@tocrmarg}{0.1em} % Useless command?
% \renewcommand{\@pnumwidth}{0.5em} % {1.75em}
\renewcommand{\@cftmaketoctitle}{}
\setlength{\cftbeforesecskip}{\z@ \@plus.2\p@}

\@ifclassloaded{article}{%
  \typeout{class: article}%
}{%
  \renewcommand{\cftchapfont}{}
  \renewcommand{\cftchapdotsep}{\cftdotsep}
  \renewcommand{\cftchapleader}{\normalfont\cftdotfill{\cftchapdotsep}}
  \renewcommand{\cftchappagefont}{\bfseries}
  \setlength{\cftbeforechapskip}{0pt}
  \setlength{\cftchapnumwidth}{1em}
}
\renewcommand{\cftsecfont}{\normalfont}
\renewcommand{\cftsecpagefont}{\normalfont}
% \renewcommand{\cftsubsecfont}{\small\relax}
\renewcommand{\cftsecdotsep}{\cftdotsep}
\renewcommand{\cftsecpagefont}{\normalfont}
\renewcommand{\cftsecleader}{\normalfont\cftdotfill{\cftsecdotsep}}
\setlength{\cftsecindent}{1em}
\setlength{\cftsubsecindent}{2em}
\setlength{\cftsubsubsecindent}{3em}
\setlength{\cftsecnumwidth}{1em}
\setlength{\cftsubsecnumwidth}{1em}
\setlength{\cftsubsubsecnumwidth}{1em}

% footnotes
\newif\ifheading
\newcommand*{\fnmarkscale}{\ifheading 0.70 \else 1 \fi}
\renewcommand\footnoterule{\vspace*{0.3cm}\hrule height \arrayrulewidth width 3cm \vspace*{0.3cm}}
\setlength\footnotesep{1.5\footnotesep} % footnote separator
\renewcommand\@makefntext[1]{\parindent 1.5em \noindent \hb@xt@1.8em{\hss{\normalfont\@thefnmark . }}#1} % no superscipt in foot
\patchcmd{\@footnotetext}{\footnotesize}{\footnotesize\sffamily}{}{} % before scrextend, hyperref
\DeclareNewFootnote{A}[alph] % for editor notes
\renewcommand*{\thefootnoteA}{\alphalph{\value{footnoteA}}} % z, aa, ab…

% poem
\setlength{\poembotskip}{0pt}
\setlength{\poemtopskip}{0pt}
\setlength{\poemindent}{0pt}
\setlength{\poemmaxlinewd}{\linewidth}
\poemlinenumsfalse

%   see https://tex.stackexchange.com/a/34449/5049
\def\truncdiv#1#2{((#1-(#2-1)/2)/#2)}
\def\moduloop#1#2{(#1-\truncdiv{#1}{#2}*#2)}
\def\modulo#1#2{\number\numexpr\moduloop{#1}{#2}\relax}

% orphans and widows, nowidow package in test
% from memoir package
\clubpenalty=9996
\widowpenalty=9999
\brokenpenalty=4991
\predisplaypenalty=10000
\postdisplaypenalty=1549
\displaywidowpenalty=1602
\hyphenpenalty=400
% report h or v overfull ?
\hbadness=4000
\vbadness=4000
% good to avoid lines too wide
\emergencystretch 3em
\pretolerance=750
\tolerance=2000
\def\Gin@extensions{.pdf,.png,.jpg,.mps,.tif}

\PassOptionsToPackage{hyphens}{url} % before hyperref and biblatex, which load url package
\usepackage{hyperref} % supposed to be the last one, :o) except for the ones to follow
\hypersetup{
  % pdftex, % no effect
  pdftitle={\elbibl},
  % pdfauthor={Your name here},
  % pdfsubject={Your subject here},
  % pdfkeywords={keyword1, keyword2},
  bookmarksnumbered=true,
  bookmarksopen=true,
  bookmarksopenlevel=1,
  pdfstartview=Fit,
  breaklinks=true, % avoid long links, overrided by url package
  pdfpagemode=UseOutlines,    % pdf toc
  hyperfootnotes=true,
  colorlinks=false,
  pdfborder=0 0 0,
  % pdfpagelayout=TwoPageRight,
  % linktocpage=true, % NO, toc, link only on page no
}
\urlstyle{same} % after hyperref



\makeatother % /@@@>
%%%%%%%%%%%%%%
% </TEI> end %
%%%%%%%%%%%%%%

\setmainlanguage{french}
%%%%%%%%%%%%%
% footnotes %
%%%%%%%%%%%%%
\renewcommand{\thefootnote}{\bfseries\textcolor{rubric}{\arabic{footnote}}} % color for footnote marks

%%%%%%%%%
% Fonts %
%%%%%%%%%
% \linespread{0.90} % too compact, keep font natural
\ifav % A5
  \usepackage{DejaVuSans} % correct
  \setsansfont{DejaVuSans} % seen, if not set, problem with printer
\else\ifbooklet
  \usepackage[]{roboto} % SmallCaps, Regular is a bit bold
  \setmainfont[
    ItalicFont={Roboto Light Italic},
  ]{Roboto}
  \setsansfont{Roboto Light} % seen, if not set, problem with printer
  \newfontfamily\fontrun[]{Roboto Condensed Light} % condensed runing heads
\else
  \usepackage[]{roboto} % SmallCaps, Regular is a bit bold
  \setmainfont[
    ItalicFont={Roboto Italic},
  ]{Roboto Light}
  \setsansfont{Roboto Light} % seen, if not set, problem with printer
  \newfontfamily\fontrun[]{Roboto Condensed Light} % condensed runing heads
\fi\fi
\renewcommand{\LettrineFontHook}{\bfseries\color{rubric}}
% \renewenvironment{labelblock}{\begin{center}\bfseries\color{rubric}}{\end{center}}

%%%%%%%%
% MISC %
%%%%%%%%

\setdefaultlanguage[frenchpart=false]{french} % bug on part


\newenvironment{quotebar}{%
    \def\FrameCommand{{\color{rubric!10!}\vrule width 0.5em} \hspace{0.9em}}%
    \def\OuterFrameSep{0pt} % séparateur vertical
    \MakeFramed {\advance\hsize-\width \FrameRestore}
  }%
  {%
    \endMakeFramed
  }
\renewenvironment{quoteblock}% may be used for ornaments
  {%
    \savenotes
    \setstretch{0.9}
    \begin{quotebar}
    \smallskip
  }
  {%
    \smallskip
    \end{quotebar}
    \spewnotes
  }


\renewcommand{\headrulewidth}{\arrayrulewidth}
\renewcommand{\headrule}{{\color{rubric}\hrule}}
\renewcommand{\lnatt}[1]{\marginpar{\sffamily\scriptsize #1}}

\titleformat{name=\chapter} % command
  [display] % shape
  {\vspace{1.5em}\centering} % format
  {} % label
  {0pt} % separator between n
  {}
[{\color{rubric}\huge\textbf{#1}}\bigskip] % after code
% \titlespacing{command}{left spacing}{before spacing}{after spacing}[right]
\titlespacing*{\chapter}{0pt}{-2em}{0pt}[0pt]

\titleformat{name=\section}
  [display]{}{}{}{}
  [\vbox{\color{rubric}\large\centering\textbf{#1}}]
\titlespacing{\section}{0pt}{0pt plus 4pt minus 2pt}{\baselineskip}

\titleformat{name=\subsection}
  [block]
  {}
  {} % \thesection
  {} % separator \arrayrulewidth
  {}
[\vbox{\large\textbf{#1}}]
% \titlespacing{\subsection}{0pt}{0pt plus 4pt minus 2pt}{\baselineskip}

\ifaiv
  \fancypagestyle{main}{%
    \fancyhf{}
    \setlength{\headheight}{1.5em}
    \fancyhead{} % reset head
    \fancyfoot{} % reset foot
    \fancyhead[L]{\truncate{0.45\headwidth}{\fontrun\elbibl}} % book ref
    \fancyhead[R]{\truncate{0.45\headwidth}{ \fontrun\nouppercase\leftmark}} % Chapter title
    \fancyhead[C]{\thepage}
  }
  \fancypagestyle{plain}{% apply to chapter
    \fancyhf{}% clear all header and footer fields
    \setlength{\headheight}{1.5em}
    \fancyhead[L]{\truncate{0.9\headwidth}{\fontrun\elbibl}}
    \fancyhead[R]{\thepage}
  }
\else
  \fancypagestyle{main}{%
    \fancyhf{}
    \setlength{\headheight}{1.5em}
    \fancyhead{} % reset head
    \fancyfoot{} % reset foot
    \fancyhead[RE]{\truncate{0.9\headwidth}{\fontrun\elbibl}} % book ref
    \fancyhead[LO]{\truncate{0.9\headwidth}{\fontrun\nouppercase\leftmark}} % Chapter title, \nouppercase needed
    \fancyhead[RO,LE]{\thepage}
  }
  \fancypagestyle{plain}{% apply to chapter
    \fancyhf{}% clear all header and footer fields
    \setlength{\headheight}{1.5em}
    \fancyhead[L]{\truncate{0.9\headwidth}{\fontrun\elbibl}}
    \fancyhead[R]{\thepage}
  }
\fi

\ifav % a5 only
  \titleclass{\section}{top}
\fi

\newcommand\chapo{{%
  \vspace*{-3em}
  \centering\parindent0pt % no vskip ()
  \eltitlepage
  \bigskip
  {\color{rubric}\hline}
  \bigskip
  {\Large TEXTE LIBRE À PARTICIPATIONS LIBRES\par}
  \centerline{\small\color{rubric} {\href{https://hurlus.fr}{\dotuline{hurlus.fr}}}, tiré le \today}\par
  \bigskip
}}

\newcommand\cover{{%
  \thispagestyle{empty}
  \centering\parindent0pt
  \eltitlepage
  \vfill\null
  {\color{rubric}\setlength{\arrayrulewidth}{2pt}\hline}
  \vfill\null
  {\Large TEXTE LIBRE À PARTICIPATIONS LIBRES\par}
  \centerline{\href{https://hurlus.fr}{\dotuline{hurlus.fr}}, tiré le \today}\par
}}

\begin{document}
\pagestyle{empty}
\ifbooklet{
  \cover\newpage
  \thispagestyle{empty}\hbox{}\newpage
  \cover\newpage\noindent Les voyages de la brochure\par
  \bigskip
  \begin{tabularx}{\textwidth}{l|X|X}
    \textbf{Date} & \textbf{Lieu}& \textbf{Nom/pseudo} \\ \hline
    \rule{0pt}{25cm} &  &   \\
  \end{tabularx}
  \newpage
  \addtocounter{page}{-4}
}\fi

\thispagestyle{empty}
\ifaiv
  \twocolumn[\chapo]
\else
  \chapo
\fi
{\it\elabstract}
\bigskip
\makeatletter\@starttoc{toc}\makeatother % toc without new page
\bigskip

\pagestyle{main} % after style
\setcounter{footnote}{0}
\setcounter{footnoteA}{0}
  
\section[{— 1 —}]{— 1 —}
\renewcommand{\leftmark}{— 1 —}

\noindent Ce soir je cours après ZAO et ne le trouve pas, non, ce soir je courais dans la lande après ZAO et ne le trouvais pas, non, comment qu’on dit déjà ? \emph{Hé, demande à Zao, demande à Zao} !\par
— ZAO ! — MAIS OÙ ES-TU BON SANG ?\par
BORDEL DE NOM DE DIEU je suis parti trop tard, je sais, on part toujours trop tard. Quand on est au camp, on glande, et d’un seul coup y en a un qui dit \emph{on fait quelque chose} ? – et on y va.\par
Faut deux heures pour aller au lac, une demi-heure en courant, trente minutes, mais ce soir j’ai pas envie de courir, je suis triste. On a perdu Zao. Et moi je cours enfin je trottine et il fait déjà nuit.\par
J’arriverai trop tard au lac. Trop tard.\par
Pourquoi es-tu parti, Zao — OÙ — où es-tu parti ?\par
Devant moi la lune éclaire les prés, les champs de chaque côté, leurs herbes folles en paraissent toutes grises. Elle est presque pleine, la lune, encore un petit \emph{chouia} et c’est la pleine lune. \emph{C’est la lune gibbeuse}, comme dit Zao, je n’ai jamais bien compris ce terme de \emph{gibbeuse}. Enfin c’est quand il reste encore un bout d’ombre de la Terre, un tout petit bout d’ombre sur un bord du rond ce qui fait qu’elle n’est pas encore totalement de totalement ronde.\par
Devant moi les herbes se couchent.\par
Passe une brise légère.\par
Laquelle poursuit sa vague dans le champ tout gris.\par
— JE COURS APRÈS ZAO !

\section[{— 2 —}]{— 2 —}
\renewcommand{\leftmark}{— 2 —}

\noindent C’est ce matin en fin de matinée que Yasmine et Émeline nous ont trouvés. Nous étions sous un arbre à glander, à fumer.\par
Elles sont arrivées affolées. Selon elles, Zao avait disparu.\par
Cela fait une semaine qu’on l’a pas vu !\par
Ah, Zao ! Mais Zao, il a l’habitude de foutre le camp !\par
D’abord, on les a pas crues, c’est vrai, on les a chambrées, et elles ont dit \emph{non non}, c’est pas vrai qu’on couche avec lui, vous le savez très bien. On dort avec lui, c’est pas la même chose.\par
Émeline : si on dort avec lui c’est que ça nous fait du bien aussi, moi de toute façon je crois que c’est la fin. Il nous a dit qu’il partait. Que ce serait pas la peine de le chercher. Il partait.\par
Yasmine : il a dit aussi que si dans une semaine je ne suis pas là, alors vous pouvez commencer à me chercher ! Et il est parti en gueulant. Ça ne vous rappelle rien ?\par
Émeline : si, les enseignements de Zao.\par
Yasmine : oui, il a toujours dit que quand il se sentirait vieux il irait dans les bois et se mettrait sous un arbre et se couperait la respiration comme les Anciens. On a fouillé les bois autour et il n’est pas là.\par
— Reste le lac, c’est une option, a dit alors Lol dans le clan des garçons. Est-ce que vous êtes allées au lac ?\par
— Non, vous savez très bien que Zao n’aime pas qu’on aille au lac, pourquoi il irait, il a peur des moustiques.\par
— Et bien on peut y aller, fit Lol. Qui y va ? C’est peut-être pas la peine d’y aller tous. On tire à la courte-paille ?\par
— Et bien ça tombe sur toi, Brisbur, a fait Jamel dit Karis, depuis le temps que tu nous les casses, Brisbur. Allez !\par
— Hé, j’ai dit, hé, les gars, coulosse, je vais d’abord attendre qu’il fasse un peu moins chaud.\par
— Il faut une demi-heure pour aller au lac en courant.\par
— Oui et deux heures à pied, c’est bon, je vais pas courir, OK, je vais juste attendre un peu qu’il fasse un peu moins chaud !

\section[{— 3 —}]{— 3 —}
\renewcommand{\leftmark}{— 3 —}

\noindent L’histoire de Zao, c’est l’histoire d’un vieux, il est arrivé un jour dans la cité et nous on l’a toujours connu vieux. Il avait une couverture marron sur le dos, marchait pieds nus, un bâton à la main. Nous, les gosses, on l’a pris pour un mendiant. Mais, par la suite, on s’est vite rendu compte qu’il visitait nos parents. Ils blablataient ensemble dans le salon comme s’ils se connaissaient depuis longtemps. Il racontait ses voyages.\par
Non, il n’était pas allé à pied jusqu’en Chine, mais il y était allé par d’autres moyens. Il avait vécu dans des tas de contrées. Il disait que les gens lui avaient appris à vivre dans les bois.\par
Il avait vu des éléphants, des singes, des crocodiles.\par
Un jour, j’ai dit à mon père, tu sais, Zao est venu à pied par la Chine, à l’école on dit \emph{à chier par la pine, ah ah ah} !\par
J’ai de la chance, mon père ne m’a jamais donné une baffe.\par
Donc Zao !\par
Zao chez nous.\par
Zao, il revenait, il devenait comme un grand tonton, il avait beau s’habiller avec des habits normaux et des chaussures, il restait toujours aussi maigre et nous rappelait le personnage d’un clown flottant dans ses vêtements, Il parlait d’une voix douce, assez lent, avec beaucoup de gestes, prenant le temps de finir ses phrases, et nos parents l’écoutaient. Bref, au bout de quelque temps, Zao a commencé à nous emmener le week-end dans les bois. La première fois, on a construit des cabanes. On a voulu revenir le week-end suivant. Puis, de fil en aiguille, on a démonté les cabanes et on est allé plus loin, de plus en plus loin. Un jour, nos parents nous ont dit au-revoir en nous saluant de la main au bas des immeubles. On ne les a jamais revus. Quelques-uns d’entre nous sont retournés en ville. Ils racontent qu’elle est détruite. Il ne reste rien. Les immeubles sont noirs. Comme bombardés. Ils ont trouvé des caves remplies de bouffe. Des fois, ils vont en chercher. Mais Zao nous a appris à faire du feu, à trouver des baies, des légumes sauvages, à chercher du miel dans les arbres, à manger les bonnes racines. Quant à Zao, on le sait, il s’en va des fois longtemps. C’est ce qu’il dit :\par
{\itshape Il faut apprendre à vous débrouiller !}

\section[{— 4 —}]{— 4 —}
\renewcommand{\leftmark}{— 4 —}

\noindent Zao, t’es con, c’est pas le moment de nous quitter ! Je suis au lac et je ne peux rien voir. La lune se reflète dans l’eau, autour les arbres sont noirs. C’est noir. J’entrevois tout juste les rochers.\par
Je lève les yeux au ciel et je me couche, wouf !\par
Et je vois la Voie Lactée.\par
— ZAO !\par
\bigbreak
\noindent Zao n’a jamais voulu qu’on s’installe au lac, d’abord à cause des moustiques, \emph{tic tic tic}, dit-il, il déteste les moustiques. Les garçons et les filles, écoutez bien il faut fuir les endroits dégagés, il y a aussi des drones qui rodent, restez sous les arbres.\par
Au lac. On vient parfois se baigner au lac. On plonge du gros rocher là-bas. En bas l’eau est claire comme de l’eau de roche, il y a quelques algues qui bougent dans le fond et plouf tu plonges, et tu remontes à la surface avec plein de bulles autour de toi, c’est magique. Au camp, on a la rivière, c’est moins loin, c’est la rivière qui vient du lac. On vient au lac quand on a envie de plonger.\par
Pour le reste, c’est vrai qu’on sait tout faire. Depuis qu’on est avec Zao, il nous a tout appris, à reconnaître les herbes, les racines, les baies et, l’année dernière, nous avons eu notre première récolte.\par
Zao a dit que l’année prochaine, peut-être, si ça se trouve, il n’y aura plus besoin d’aller en ville chercher des provisions.\par
Zao disait aussi qu’il nous voyait grandir et qu’un jour, il faudra nous y faire, il disparaîtrait. Le jour où il se sentirait trop vieux, il partirait, il partirait de lui-même au royaume des morts sans laisser de traces. Nous n’aurons pas besoin de le chercher, mais si nous le trouvons, il dit qu’il faudra laisser son corps dehors.\par
« \emph{Vous le laissez exposé aux vautours, comme les Indiens} ! »

\section[{— 5 —}]{— 5 —}
\renewcommand{\leftmark}{— 5 —}

{\itshape ZAO, OÙ ES-TU ?\par}
{\itshape ZAO, TU ME FAIS CHIER !\par}
{\itshape MAIS OÙ ES-TU PASSÉ, ZAO ?\par}
\bigbreak
\noindent Au camp, il y a des histoires. On est trois garçons à faire bande à part, Lol, Jamel dit Karis et moi Brisbur, des noms de guerre. Les filles se moquent, elles disent qu’on est trop sales.\par
Zao, quand il est là, rigole.\par
Il en profite pour nous rappeler qu’il faut nous laver le zizi, les fesses, les pieds, les mains, le visage et tout le haut du corps le soir avant d’aller se coucher. Puis il s’en va. Il est comme ça Zao. Il n’est pas toujours là. Pas toujours avec nous. Peut-être les filles n’ont pas tort, il est réellement parti, Zao, et c’est peut-être aussi ce que je ne veux pas voir. C’est tout. Zao est vieux et s’il est parti, c’est fou.\par
\bigbreak
\noindent En fait on a tous pensé la même chose en même temps.\par
L’effet Zao, a dit Yasmine.\par
Elle, elle est vraiment trop belle. Vraiment.\par
Zao, je m’excuse, je m’endors, je suis dans l’herbe, je ne vois plus le lac, j’ai pas envie de chercher, je me suis mis à l’écart, je suis tes enseignements, Zao, sur la plage vous avez l’humidité qui remonte, c’est pas bon de coucher sur la plage, à quelques mètres au-dessus vous avez des herbes sèches, faites-en un lit et couchez-vous dessus. Vaut mieux mettre des fois une couverture par terre, ça isole. \emph{Mais dans les herbes, hé hé, il y a aussi les moustiques et c’est là où ça se complique.} Et PANG ! j’en tape un. La vache !\par
\bigbreak
Zao, j’ai sommeil, où es-tu passé, Zao ?\par
Es-tu vraiment au lac ?

\section[{— 6 —}]{— 6 —}
\renewcommand{\leftmark}{— 6 —}

\noindent Je me réveille avec le soleil dans le pif, la plage est paisible, déserte. L’eau du lac est rose et bleue, tranquille. Et je vois Zao, assis là-bas sur le rocher où on a l’habitude de plonger.\par
Je dévale la pente et je crie ZAO !\par
Il ne se retourne pas.\par
Il regarde le lac.\par
J’avance jusqu’à lui sur les rochers et viens doucement derrière lui et je dis doucement – ZAO ! Il redresse la tête.\par
Ses grands yeux tout bleus, son visage tout maigre et sans dents.\par
{\itshape Ah, c’est toi, Brisbur, dit-il d’un air égaré, qu’est-ce que tu viens faire là ?\par}
On te cherche partout, je dis, au camp, Zao.\par
Pourquoi ça vous inquiète ? Je vous ai dit qu’un jour je partirai, je suis parti. Je voyage en ce moment sur mon point de non-retour.\par
\bigbreak
\noindent J’hésite un instant.\par
J’hésite à dire à Zao quand même tu déconnes et je dis quand même Zao tu déconnes ! Il me sourit.\par
Ça a l’air de ne lui faire ni chaud ni froid.\par
Il est enveloppé dans une couverture orange et sur son crâne tanné deux ou trois mèches de ses cheveux blancs flottent.\par
« Tu le sais très bien, Brisbur ! » dit-il en revenant regarder le lac.\par
J’acquiesce.\par
Je regarde le lac aussi.\par
Les rives en face sont remplies de roseaux. Des oiseaux passent. Deux pies noires et blanches se suivent. Du côté de la plage un héron avance en levant la patte. Le ciel est bleu clair. Des canards s’approchent du rocher.

\section[{— 7 —}]{— 7 —}
\renewcommand{\leftmark}{— 7 —}

{\itshape \noindent Les moustiques, ils peuvent avoir des dards gros comme mon pouce, je vous jure, des fois, et ils foncent sur vous, bzz bzz, et pic et pic, et vous, vous vous en sortez avec des boutons gros comme mon poing, hein, et vous viendrez tous pleurer dans mes bras, ah yayaille, ça pique, Zao !\par}
\bigbreak
\noindent Écoute, petit, a dit Zao, je vais te dire quelque chose, mais avant tu peux t’asseoir ! Il se retourne sur lui-même, pivote, me fait face en croisant les jambes. Et, levant le doigt avec le lac dans le dos, il dit :\par
Écoute bien les enseignements de Zao !\par
J’acquiesce.\par
Zao, il me parle longtemps, je l’entends dire qu’on est des cueilleurs, il me dit Brisbur, ce n’est pas à toi que je préférerais confier la direction du camp, tu le sais, j’ai dit aux filles que ce serait elles. Yasmine et Émeline, elles sont plus mûres que vous, elles feront régner la paix. Alors que vous, au camp, vous vous bataillez tout le temps. Pourquoi vous faites-vous la guerre ? Vous parents ne l’ont pas assez faite ? Pourquoi ?\par
Je ne sais quoi dire. Zao parle longtemps.\par
Et soudain je crie Ah !\par
J’ai le soleil en pleine poire, je regarde la plage et je crie Zao et il n’y a naturellement plus de Zao. Je cligne des yeux et je le revois de nouveau au bout du rocher regarder le lac. Et moi je me recouche, je suis fatigué, je vois Zao qui me tourne le dos et j’avance un coude à l’ombre d’un arbre et je pose ma tête sur ma main. \emph{— ZAO ! — je vois un peintre chinois qui dessine d’un trait la vague grise que j’ai vue cette nuit.}\par
Il peint.

\section[{— 8 —}]{— 8 —}
\renewcommand{\leftmark}{— 8 —}

\noindent Quand je fus enfin dans le bon tempo et que, mes songes s’étant dissipés, je me suis levé, je suis descendu sur la plage et j’ai regardé le rocher. Il n’y avait pas plus de Zao que de couilles en barre.\par
J’avais rêvé, c’est tout. Je m’étais endormi.\par
Et je n’avais pas dû vouloir me réveiller tout de suite, c’est ça.\par
J’ai regardé le lac.\par
Le soleil au-dessus des arbres se levait tranquille, le lac respirait la paix, l’immensité calme, j’entendais les oiseaux.\par
Zao nous dit de bien remarquer qu’ils ne chantent pas tous en même temps. Écoutez. Ils se répondent ! Ils font des chorus !\par
Je suis allé finalement jusqu’au rocher, histoire de voir s’il n’était pas derrière en train de nager et, sur la plage, j’ai vu des pas qui partaient dans l’eau. Je les ai suivis.\par
Ils reviennent sur le sable. Je piste. Zao, Lol, Karis et moi, on est les plus forts pour attraper les mulots, les rats d’eau. Les filles, ce sont les canards, les lapins. Mais dans l’ensemble on mange pas trop de viande. Ça va comme ça. Karis est allé au supermarché, Zao. Il dit que l’entrepôt à conserves sera bientôt vide.\par
On n’est pas les seuls, Zao.\par
Tu nous a entraînés dans les bois, mais pourquoi faire ?\par
Nous vivons dans les bois.\par
Et maintenant je sens que nous allons te perdre, Zao.\par
Hé, OÙ ES-TU ZAO ?\par
Je fais presque un demi-tour de lac, je retrouve la trace.\par
\bigbreak
\noindent Si c’est lui, il a marché dans l’eau.

\section[{— 9 —}]{— 9 —}
\renewcommand{\leftmark}{— 9 —}

\noindent Je ne suis pas arrivé à pied par la Chine comme disent vos parents, ils enjolivent les choses, je suis allé deux fois en Chine, et j’y suis allé en avion, c’est très loin. Par contre, c’est vrai, j’ai traversé beaucoup de frontières à pied. Je me moque des frontières. Des fois j’avais pas de visa. Je me démerdais. Si tu vas au devant des gens, en général tout s’arrange. Si vous allez au devant des gens, Brisbur, et toi aussi Nath, Émeline ou Yasmine, Jamel Karis ou Lol, les autres vous en sauront gré, ils vous accueilleront. On n’est pas obligé d’être partout en guerre, d’ailleurs la grande majorité des gens dans le monde ne la veulent pas, la guerre. J’ai beaucoup appris en regardant les autres faire. Au départ, c’est forcé, on ne parle pas la langue, mais au bout de quelques jours on sait trente mots et on commence à communiquer. Ce que je sais des plantes, des racines, comment survivre, je l’ai appris des autres, auprès de gens qu’on a souvent bombardés. Quand ton pays est en guerre depuis si longtemps et que toi-même tu es né dans la guerre, tu ne sais même pas si elle va cesser, boum boum, c’est toujours dans des terres lointaines. Et pourtant, vous ne me croirez pas, il y a dix ans je ne savais pas planter un haricot. Ou une tomate. Ou un poireau. C’est incroyable je me suis dit qu’un intello ne sache rien faire de ses dix doigts. J’ai eu des amis en route qui sont partis manuels et sont devenus des gros cerveaux. Voilà, ce que je vous ai appris c’est vivre, j’y suis finalement pour pas grand-chose, maintenant vous vivez. Vous n’avez plus besoin de moi.\par
— Mais Zao, tu dis des conneries !\par
— Mais non, Brisbur, sache qu’un sage ne dit jamais de conneries. Enfin pour lui c’est jamais des conneries. Méfie-toi du regard des autres ! N’hésite pas !\par
— Merci Zao !\par
— Y a pas à dire merci, en plus, si tu penses que c’est des conneries, t’as pas à dire merci !

\section[{— 10 —}]{— 10 —}
\renewcommand{\leftmark}{— 10 —}

\noindent MON FRÈRE ! Écoute-moi bien, mon frère, je vais t’emmener dans la forêt, nous allons y rester sept jours, je vais te dire là-bas comment il faut s’y prendre avec une femme, la famille, les autres, la Terre, oui, surtout la Terre, et quand tu chasseras un chevreuil, tu le ramèneras sur ton dos. Tu t’apercevras que c’est lourd, tu n’en chasseras qu’un. Les animaux ne tuent jamais plus d’un animal à la fois. Comme les lionnes. Elles, elles se mettent à deux pour chasser. L’homme des débuts mangeait ce que laissaient les fauves ou les loups derrière eux, mais tous n’étaient pas forcément des charognards, on a retrouvé dans le monde des tribus presque entièrement végétariennes. Toi aussi rogne tes os. Dans ton cerveau il y a le bas qui est ton cerveau reptilien. C’est lui qui t’apporte la mémoire de millions d’années. C’était bien avant l’homme. Quand les os sont secs, l’os ça met du temps à disparaître, mais à la fin il finit quand même en poussière. Nous sommes dans un état mou. L’eau c’est notre corps. Sans eau nous mourons. La leçon de cette histoire c’est que si nous tuons beaucoup trop d’êtres vivants, nous finirons certainement encore plus vite à l’état d’os.

\section[{— 11 —}]{— 11 —}
\renewcommand{\leftmark}{— 11 —}

\noindent Je retrouve la trace, elle s’enfonce dans les roseaux, c’est un passage où sont passées plusieurs personnes, c’est peut-être nous, il y a plusieurs personnes qui sont passées par là, je suis, et je débouche sur le sentier. Il est à l’ombre, avec les arbres au-dessus qui font voûte. Ce qui fait qu’il est presque comme il était avant, comme il a été pendant longtemps, quand des chars à bœufs devaient passer par là, nous a dit Zao, pour tirer les arbres de la forêt. Il fait presque entièrement le tour du lac. Parfois on le perd quand il manque des arbres. Au milieu, on débouche sur la cascade, le déversoir du lac. J’entends de loin son jet à mesure que j’approche. J’y suis. La cascade tombe depuis un mur de ciment vertigineux. Parfois on voit des chèvres venir s’accrocher là-dessus pour venir brouter les pouces d’herbes.\par
Et là-dessus il y a mon Zao pendu !\par
\bigbreak
\noindent MERDE Zao !\par
Il s’est pendu !\par
Il pend à une branche au-dessus de cet abîme avec la tête presque à l’équerre et sa couverture orange qui lui fait comme un poncho. Les oiseaux lui ont déjà piqué les yeux. Ses bras, ses joues, ses jambes portent la marque de coups de becs. Depuis quand il est là ? Je ne réfléchis pas, je monte, grimpe dans l’arbre, avise la branche, me glisse dessus, sous moi je vois le bouillonnement des eaux. J’essaye d’attraper la corde mais dès que je la touche celle-ci balance et d’un seul coup le corps lâche, il se détache et plonge, laissant la tête fixée au nœud (\emph{MERDE} !), et en bas le corps fait plouf en disparaissant dans les remous.

\section[{— 12 —}]{— 12 —}
\renewcommand{\leftmark}{— 12 —}

\noindent J’AI RAMENÉ LA TÊTE À ZAO AU CAMP !\par
\bigbreak
\noindent Je suis parti avec elle sous le bras et j’ai ramené sa tête au camp. J’ai mis du temps à la détacher, je vous assure, c’était très serré et, n’osant trop la regarder car elle était toute marron, j’ai pensé à son corps. Le courant devait l’emporter. Et je me suis dit, si ça se trouve, avec la rivière, dans dix minutes il est au camp. Comment vont-ils l’exposer sans sa tête coupée ? Et dans mon for intérieur, je ruminais les enseignements de Zao, vous savez je préfère que ce soit les filles qui assurent la direction du camp, ça sera mieux, je vous promets, et en même temps je me disais : est-ce que je le leur dirai ?\par
J’étais triste.\par
\bigbreak
\noindent JE RAMENAIS LA TÊTE DE ZAO AU CAMP !\par

\dateline{Lille, octobre 2016}
 


% at least one empty page at end (for booklet couv)
\ifbooklet
  \pagestyle{empty}
  \clearpage
  % 2 empty pages maybe needed for 4e cover
  \ifnum\modulo{\value{page}}{4}=0 \hbox{}\newpage\hbox{}\newpage\fi
  \ifnum\modulo{\value{page}}{4}=1 \hbox{}\newpage\hbox{}\newpage\fi


  \hbox{}\newpage
  \ifodd\value{page}\hbox{}\newpage\fi
  {\centering\color{rubric}\bfseries\noindent\large
    Hurlus ? Qu’est-ce.\par
    \bigskip
  }
  \noindent Des bouquinistes électroniques, pour du texte libre à participations libres,
  téléchargeable gratuitement sur \href{https://hurlus.fr}{\dotuline{hurlus.fr}}.\par
  \bigskip
  \noindent Cette brochure a été produite par des éditeurs bénévoles.
  Elle n’est pas faite pour être possédée, mais pour être lue, et puis donnée, ou déposée dans une boîte à livres.
  En page de garde, on peut ajouter une date, un lieu, un nom ;
  comme une fiche de bibliothèque en papier qui enregistre \emph{les voyages de la brochure}.
  \par

  Ce texte a été choisi parce qu’une personne l’a aimé,
  ou haï, elle a pensé qu’il partipait à la formation de notre présent ;
  sans le souci de plaire, vendre, ou militer pour une cause.
  \par

  L’édition électronique est soigneuse, tant sur la technique
  que sur l’établissement du texte ; mais sans aucune prétention scolaire, au contraire.
  Le but est de s’adresser à tous, sans distinction de science ou de diplôme.
  \par

  Cet exemplaire en papier a été tiré sur une imprimante personnelle
   ou une photocopieuse. Tout le monde peut le faire.
  Il suffit de
  télécharger un fichier sur \href{https://hurlus.fr}{\dotuline{hurlus.fr}},
  d’imprimer, et agrafer (puis lire et donner).\par

  \bigskip

  \noindent PS : Les hurlus furent aussi des rebelles protestants qui cassaient les statues dans les églises catholiques. En 1566 démarra la révolte des gueux dans le pays de Lille. L’insurrection enflamma la région jusqu’à Anvers où les gueux de mer bloquèrent les bateaux espagnols.
  Ce fut une rare guerre de libération dont naquit un pays toujours libre : les Pays-Bas.
  En plat pays francophone, par contre, restèrent des bandes de huguenots, les hurlus, progressivement réprimés par la très catholique Espagne.
  Cette mémoire d’une défaite est éteinte, rallumons-la. Sortons les livres du culte universitaire, débusquons les idoles de l’époque, pour les démonter.
\fi

\end{document}
