%%%%%%%%%%%%%%%%%%%%%%%%%%%%%%%%%
% LaTeX model https://hurlus.fr %
%%%%%%%%%%%%%%%%%%%%%%%%%%%%%%%%%

% Needed before document class
\RequirePackage{pdftexcmds} % needed for tests expressions
\RequirePackage{fix-cm} % correct units

% Define mode
\def\mode{a4}

\newif\ifaiv % a4
\newif\ifav % a5
\newif\ifbooklet % booklet
\newif\ifcover % cover for booklet

\ifnum \strcmp{\mode}{cover}=0
  \covertrue
\else\ifnum \strcmp{\mode}{booklet}=0
  \booklettrue
\else\ifnum \strcmp{\mode}{a5}=0
  \avtrue
\else
  \aivtrue
\fi\fi\fi

\ifbooklet % do not enclose with {}
  \documentclass[french,twoside]{book} % ,notitlepage
  \usepackage[%
    papersize={105mm, 297mm},
    inner=12mm,
    outer=12mm,
    top=20mm,
    bottom=15mm,
    marginparsep=0pt,
  ]{geometry}
  \usepackage[fontsize=9.5pt]{scrextend} % for Roboto
\else\ifav
  \documentclass[french,twoside]{book} % ,notitlepage
  \usepackage[%
    a5paper,
    inner=25mm,
    outer=15mm,
    top=15mm,
    bottom=15mm,
    marginparsep=0pt,
  ]{geometry}
  \usepackage[fontsize=12pt]{scrextend}
\else% A4 2 cols
  \documentclass[twocolumn]{report}
  \usepackage[%
    a4paper,
    inner=15mm,
    outer=10mm,
    top=25mm,
    bottom=18mm,
    marginparsep=0pt,
  ]{geometry}
  \setlength{\columnsep}{20mm}
  \usepackage[fontsize=9.5pt]{scrextend}
\fi\fi

%%%%%%%%%%%%%%
% Alignments %
%%%%%%%%%%%%%%
% before teinte macros

\setlength{\arrayrulewidth}{0.2pt}
\setlength{\columnseprule}{\arrayrulewidth} % twocol
\setlength{\parskip}{0pt} % classical para with no margin
\setlength{\parindent}{1.5em}

%%%%%%%%%%
% Colors %
%%%%%%%%%%
% before Teinte macros

\usepackage[dvipsnames]{xcolor}
\definecolor{rubric}{HTML}{800000} % the tonic 0c71c3
\def\columnseprulecolor{\color{rubric}}
\colorlet{borderline}{rubric!30!} % definecolor need exact code
\definecolor{shadecolor}{gray}{0.95}
\definecolor{bghi}{gray}{0.5}

%%%%%%%%%%%%%%%%%
% Teinte macros %
%%%%%%%%%%%%%%%%%
%%%%%%%%%%%%%%%%%%%%%%%%%%%%%%%%%%%%%%%%%%%%%%%%%%%
% <TEI> generic (LaTeX names generated by Teinte) %
%%%%%%%%%%%%%%%%%%%%%%%%%%%%%%%%%%%%%%%%%%%%%%%%%%%
% This template is inserted in a specific design
% It is XeLaTeX and otf fonts

\makeatletter % <@@@


\usepackage{blindtext} % generate text for testing
\usepackage[strict]{changepage} % for modulo 4
\usepackage{contour} % rounding words
\usepackage[nodayofweek]{datetime}
% \usepackage{DejaVuSans} % seems buggy for sffont font for symbols
\usepackage{enumitem} % <list>
\usepackage{etoolbox} % patch commands
\usepackage{fancyvrb}
\usepackage{fancyhdr}
\usepackage{float}
\usepackage{fontspec} % XeLaTeX mandatory for fonts
\usepackage{footnote} % used to capture notes in minipage (ex: quote)
\usepackage{framed} % bordering correct with footnote hack
\usepackage{graphicx}
\usepackage{lettrine} % drop caps
\usepackage{lipsum} % generate text for testing
\usepackage[framemethod=tikz,]{mdframed} % maybe used for frame with footnotes inside
\usepackage{pdftexcmds} % needed for tests expressions
\usepackage{polyglossia} % non-break space french punct, bug Warning: "Failed to patch part"
\usepackage[%
  indentfirst=false,
  vskip=1em,
  noorphanfirst=true,
  noorphanafter=true,
  leftmargin=\parindent,
  rightmargin=0pt,
]{quoting}
\usepackage{ragged2e}
\usepackage{setspace} % \setstretch for <quote>
\usepackage{tabularx} % <table>
\usepackage[explicit]{titlesec} % wear titles, !NO implicit
\usepackage{tikz} % ornaments
\usepackage{tocloft} % styling tocs
\usepackage[fit]{truncate} % used im runing titles
\usepackage{unicode-math}
\usepackage[normalem]{ulem} % breakable \uline, normalem is absolutely necessary to keep \emph
\usepackage{verse} % <l>
\usepackage{xcolor} % named colors
\usepackage{xparse} % @ifundefined
\XeTeXdefaultencoding "iso-8859-1" % bad encoding of xstring
\usepackage{xstring} % string tests
\XeTeXdefaultencoding "utf-8"
\PassOptionsToPackage{hyphens}{url} % before hyperref, which load url package

% TOTEST
% \usepackage{hypcap} % links in caption ?
% \usepackage{marginnote}
% TESTED
% \usepackage{background} % doesn’t work with xetek
% \usepackage{bookmark} % prefers the hyperref hack \phantomsection
% \usepackage[color, leftbars]{changebar} % 2 cols doc, impossible to keep bar left
% \usepackage[utf8x]{inputenc} % inputenc package ignored with utf8 based engines
% \usepackage[sfdefault,medium]{inter} % no small caps
% \usepackage{firamath} % choose firasans instead, firamath unavailable in Ubuntu 21-04
% \usepackage{flushend} % bad for last notes, supposed flush end of columns
% \usepackage[stable]{footmisc} % BAD for complex notes https://texfaq.org/FAQ-ftnsect
% \usepackage{helvet} % not for XeLaTeX
% \usepackage{multicol} % not compatible with too much packages (longtable, framed, memoir…)
% \usepackage[default,oldstyle,scale=0.95]{opensans} % no small caps
% \usepackage{sectsty} % \chapterfont OBSOLETE
% \usepackage{soul} % \ul for underline, OBSOLETE with XeTeX
% \usepackage[breakable]{tcolorbox} % text styling gone, footnote hack not kept with breakable


% Metadata inserted by a program, from the TEI source, for title page and runing heads
\title{\textbf{ Discours et conférences }}
\date{1887}
\author{Ernest Renan}
\def\elbibl{Ernest Renan. 1887. \emph{Discours et conférences}}
\def\elsource{Ernest Renan, \emph{Discours et conférences}, Paris : C. Lévy, 1887. Fichier source : \href{https://archive.org/details/discoursetconf00renauoft}{\dotuline{Internet Archive}}\footnote{\href{https://archive.org/details/discoursetconf00renauoft}{\url{https://archive.org/details/discoursetconf00renauoft}}}.}

% Default metas
\newcommand{\colorprovide}[2]{\@ifundefinedcolor{#1}{\colorlet{#1}{#2}}{}}
\colorprovide{rubric}{red}
\colorprovide{silver}{lightgray}
\@ifundefined{syms}{\newfontfamily\syms{DejaVu Sans}}{}
\newif\ifdev
\@ifundefined{elbibl}{% No meta defined, maybe dev mode
  \newcommand{\elbibl}{Titre court ?}
  \newcommand{\elbook}{Titre du livre source ?}
  \newcommand{\elabstract}{Résumé\par}
  \newcommand{\elurl}{http://oeuvres.github.io/elbook/2}
  \author{Éric Lœchien}
  \title{Un titre de test assez long pour vérifier le comportement d’une maquette}
  \date{1566}
  \devtrue
}{}
\let\eltitle\@title
\let\elauthor\@author
\let\eldate\@date


\defaultfontfeatures{
  % Mapping=tex-text, % no effect seen
  Scale=MatchLowercase,
  Ligatures={TeX,Common},
}


% generic typo commands
\newcommand{\astermono}{\medskip\centerline{\color{rubric}\large\selectfont{\syms ✻}}\medskip\par}%
\newcommand{\astertri}{\medskip\par\centerline{\color{rubric}\large\selectfont{\syms ✻\,✻\,✻}}\medskip\par}%
\newcommand{\asterism}{\bigskip\par\noindent\parbox{\linewidth}{\centering\color{rubric}\large{\syms ✻}\\{\syms ✻}\hskip 0.75em{\syms ✻}}\bigskip\par}%

% lists
\newlength{\listmod}
\setlength{\listmod}{\parindent}
\setlist{
  itemindent=!,
  listparindent=\listmod,
  labelsep=0.2\listmod,
  parsep=0pt,
  % topsep=0.2em, % default topsep is best
}
\setlist[itemize]{
  label=—,
  leftmargin=0pt,
  labelindent=1.2em,
  labelwidth=0pt,
}
\setlist[enumerate]{
  label={\bf\color{rubric}\arabic*.},
  labelindent=0.8\listmod,
  leftmargin=\listmod,
  labelwidth=0pt,
}
\newlist{listalpha}{enumerate}{1}
\setlist[listalpha]{
  label={\bf\color{rubric}\alph*.},
  leftmargin=0pt,
  labelindent=0.8\listmod,
  labelwidth=0pt,
}
\newcommand{\listhead}[1]{\hspace{-1\listmod}\emph{#1}}

\renewcommand{\hrulefill}{%
  \leavevmode\leaders\hrule height 0.2pt\hfill\kern\z@}

% General typo
\DeclareTextFontCommand{\textlarge}{\large}
\DeclareTextFontCommand{\textsmall}{\small}

% commands, inlines
\newcommand{\anchor}[1]{\Hy@raisedlink{\hypertarget{#1}{}}} % link to top of an anchor (not baseline)
\newcommand\abbr[1]{#1}
\newcommand{\autour}[1]{\tikz[baseline=(X.base)]\node [draw=rubric,thin,rectangle,inner sep=1.5pt, rounded corners=3pt] (X) {\color{rubric}#1};}
\newcommand\corr[1]{#1}
\newcommand{\ed}[1]{ {\color{silver}\sffamily\footnotesize (#1)} } % <milestone ed="1688"/>
\newcommand\expan[1]{#1}
\newcommand\foreign[1]{\emph{#1}}
\newcommand\gap[1]{#1}
\renewcommand{\LettrineFontHook}{\color{rubric}}
\newcommand{\initial}[2]{\lettrine[lines=2, loversize=0.3, lhang=0.3]{#1}{#2}}
\newcommand{\initialiv}[2]{%
  \let\oldLFH\LettrineFontHook
  % \renewcommand{\LettrineFontHook}{\color{rubric}\ttfamily}
  \IfSubStr{QJ’}{#1}{
    \lettrine[lines=4, lhang=0.2, loversize=-0.1, lraise=0.2]{\smash{#1}}{#2}
  }{\IfSubStr{É}{#1}{
    \lettrine[lines=4, lhang=0.2, loversize=-0, lraise=0]{\smash{#1}}{#2}
  }{\IfSubStr{ÀÂ}{#1}{
    \lettrine[lines=4, lhang=0.2, loversize=-0, lraise=0, slope=0.6em]{\smash{#1}}{#2}
  }{\IfSubStr{A}{#1}{
    \lettrine[lines=4, lhang=0.2, loversize=0.2, slope=0.6em]{\smash{#1}}{#2}
  }{\IfSubStr{V}{#1}{
    \lettrine[lines=4, lhang=0.2, loversize=0.2, slope=-0.5em]{\smash{#1}}{#2}
  }{
    \lettrine[lines=4, lhang=0.2, loversize=0.2]{\smash{#1}}{#2}
  }}}}}
  \let\LettrineFontHook\oldLFH
}
\newcommand{\labelchar}[1]{\textbf{\color{rubric} #1}}
\newcommand{\milestone}[1]{\autour{\footnotesize\color{rubric} #1}} % <milestone n="4"/>
\newcommand\name[1]{#1}
\newcommand\orig[1]{#1}
\newcommand\orgName[1]{#1}
\newcommand\persName[1]{#1}
\newcommand\placeName[1]{#1}
\newcommand{\pn}[1]{\IfSubStr{-—–¶}{#1}% <p n="3"/>
  {\noindent{\bfseries\color{rubric}   ¶  }}
  {{\footnotesize\autour{ #1}  }}}
\newcommand\reg{}
% \newcommand\ref{} % already defined
\newcommand\sic[1]{#1}
\newcommand\surname[1]{\textsc{#1}}
\newcommand\term[1]{\textbf{#1}}

\def\mednobreak{\ifdim\lastskip<\medskipamount
  \removelastskip\nopagebreak\medskip\fi}
\def\bignobreak{\ifdim\lastskip<\bigskipamount
  \removelastskip\nopagebreak\bigskip\fi}

% commands, blocks
\newcommand{\byline}[1]{\bigskip{\RaggedLeft{#1}\par}\bigskip}
\newcommand{\bibl}[1]{{\RaggedLeft{#1}\par\bigskip}}
\newcommand{\biblitem}[1]{{\noindent\hangindent=\parindent   #1\par}}
\newcommand{\dateline}[1]{\medskip{\RaggedLeft{#1}\par}\bigskip}
\newcommand{\labelblock}[1]{\medbreak{\noindent\color{rubric}\bfseries #1}\par\mednobreak}
\newcommand{\salute}[1]{\bigbreak{#1}\par\medbreak}
\newcommand{\signed}[1]{\bigbreak\filbreak{\raggedleft #1\par}\medskip}

% environments for blocks (some may become commands)
\newenvironment{borderbox}{}{} % framing content
\newenvironment{citbibl}{\ifvmode\hfill\fi}{\ifvmode\par\fi }
\newenvironment{docAuthor}{\ifvmode\vskip4pt\fontsize{16pt}{18pt}\selectfont\fi\itshape}{\ifvmode\par\fi }
\newenvironment{docDate}{}{\ifvmode\par\fi }
\newenvironment{docImprint}{\vskip6pt}{\ifvmode\par\fi }
\newenvironment{docTitle}{\vskip6pt\bfseries\fontsize{18pt}{22pt}\selectfont}{\par }
\newenvironment{msHead}{\vskip6pt}{\par}
\newenvironment{msItem}{\vskip6pt}{\par}
\newenvironment{titlePart}{}{\par }


% environments for block containers
\newenvironment{argument}{\itshape\parindent0pt}{\vskip1.5em}
\newenvironment{biblfree}{}{\ifvmode\par\fi }
\newenvironment{bibitemlist}[1]{%
  \list{\@biblabel{\@arabic\c@enumiv}}%
  {%
    \settowidth\labelwidth{\@biblabel{#1}}%
    \leftmargin\labelwidth
    \advance\leftmargin\labelsep
    \@openbib@code
    \usecounter{enumiv}%
    \let\p@enumiv\@empty
    \renewcommand\theenumiv{\@arabic\c@enumiv}%
  }
  \sloppy
  \clubpenalty4000
  \@clubpenalty \clubpenalty
  \widowpenalty4000%
  \sfcode`\.\@m
}%
{\def\@noitemerr
  {\@latex@warning{Empty `bibitemlist' environment}}%
\endlist}
\newenvironment{quoteblock}% may be used for ornaments
  {\begin{quoting}}
  {\end{quoting}}

% table () is preceded and finished by custom command
\newcommand{\tableopen}[1]{%
  \ifnum\strcmp{#1}{wide}=0{%
    \begin{center}
  }
  \else\ifnum\strcmp{#1}{long}=0{%
    \begin{center}
  }
  \else{%
    \begin{center}
  }
  \fi\fi
}
\newcommand{\tableclose}[1]{%
  \ifnum\strcmp{#1}{wide}=0{%
    \end{center}
  }
  \else\ifnum\strcmp{#1}{long}=0{%
    \end{center}
  }
  \else{%
    \end{center}
  }
  \fi\fi
}


% text structure
\newcommand\chapteropen{} % before chapter title
\newcommand\chaptercont{} % after title, argument, epigraph…
\newcommand\chapterclose{} % maybe useful for multicol settings
\setcounter{secnumdepth}{-2} % no counters for hierarchy titles
\setcounter{tocdepth}{5} % deep toc
\markright{\@title} % ???
\markboth{\@title}{\@author} % ???
\renewcommand\tableofcontents{\@starttoc{toc}}
% toclof format
% \renewcommand{\@tocrmarg}{0.1em} % Useless command?
% \renewcommand{\@pnumwidth}{0.5em} % {1.75em}
\renewcommand{\@cftmaketoctitle}{}
\setlength{\cftbeforesecskip}{\z@ \@plus.2\p@}
\renewcommand{\cftchapfont}{}
\renewcommand{\cftchapdotsep}{\cftdotsep}
\renewcommand{\cftchapleader}{\normalfont\cftdotfill{\cftchapdotsep}}
\renewcommand{\cftchappagefont}{\bfseries}
\setlength{\cftbeforechapskip}{0em \@plus\p@}
% \renewcommand{\cftsecfont}{\small\relax}
\renewcommand{\cftsecpagefont}{\normalfont}
% \renewcommand{\cftsubsecfont}{\small\relax}
\renewcommand{\cftsecdotsep}{\cftdotsep}
\renewcommand{\cftsecpagefont}{\normalfont}
\renewcommand{\cftsecleader}{\normalfont\cftdotfill{\cftsecdotsep}}
\setlength{\cftsecindent}{1em}
\setlength{\cftsubsecindent}{2em}
\setlength{\cftsubsubsecindent}{3em}
\setlength{\cftchapnumwidth}{1em}
\setlength{\cftsecnumwidth}{1em}
\setlength{\cftsubsecnumwidth}{1em}
\setlength{\cftsubsubsecnumwidth}{1em}

% footnotes
\newif\ifheading
\newcommand*{\fnmarkscale}{\ifheading 0.70 \else 1 \fi}
\renewcommand\footnoterule{\vspace*{0.3cm}\hrule height \arrayrulewidth width 3cm \vspace*{0.3cm}}
\setlength\footnotesep{1.5\footnotesep} % footnote separator
\renewcommand\@makefntext[1]{\parindent 1.5em \noindent \hb@xt@1.8em{\hss{\normalfont\@thefnmark . }}#1} % no superscipt in foot
\patchcmd{\@footnotetext}{\footnotesize}{\footnotesize\sffamily}{}{} % before scrextend, hyperref


%   see https://tex.stackexchange.com/a/34449/5049
\def\truncdiv#1#2{((#1-(#2-1)/2)/#2)}
\def\moduloop#1#2{(#1-\truncdiv{#1}{#2}*#2)}
\def\modulo#1#2{\number\numexpr\moduloop{#1}{#2}\relax}

% orphans and widows
\clubpenalty=9996
\widowpenalty=9999
\brokenpenalty=4991
\predisplaypenalty=10000
\postdisplaypenalty=1549
\displaywidowpenalty=1602
\hyphenpenalty=400
% Copied from Rahtz but not understood
\def\@pnumwidth{1.55em}
\def\@tocrmarg {2.55em}
\def\@dotsep{4.5}
\emergencystretch 3em
\hbadness=4000
\pretolerance=750
\tolerance=2000
\vbadness=4000
\def\Gin@extensions{.pdf,.png,.jpg,.mps,.tif}
% \renewcommand{\@cite}[1]{#1} % biblio

\usepackage{hyperref} % supposed to be the last one, :o) except for the ones to follow
\urlstyle{same} % after hyperref
\hypersetup{
  % pdftex, % no effect
  pdftitle={\elbibl},
  % pdfauthor={Your name here},
  % pdfsubject={Your subject here},
  % pdfkeywords={keyword1, keyword2},
  bookmarksnumbered=true,
  bookmarksopen=true,
  bookmarksopenlevel=1,
  pdfstartview=Fit,
  breaklinks=true, % avoid long links
  pdfpagemode=UseOutlines,    % pdf toc
  hyperfootnotes=true,
  colorlinks=false,
  pdfborder=0 0 0,
  % pdfpagelayout=TwoPageRight,
  % linktocpage=true, % NO, toc, link only on page no
}

\makeatother % /@@@>
%%%%%%%%%%%%%%
% </TEI> end %
%%%%%%%%%%%%%%


%%%%%%%%%%%%%
% footnotes %
%%%%%%%%%%%%%
\renewcommand{\thefootnote}{\bfseries\textcolor{rubric}{\arabic{footnote}}} % color for footnote marks

%%%%%%%%%
% Fonts %
%%%%%%%%%
\usepackage[]{roboto} % SmallCaps, Regular is a bit bold
% \linespread{0.90} % too compact, keep font natural
\newfontfamily\fontrun[]{Roboto Condensed Light} % condensed runing heads
\ifav
  \setmainfont[
    ItalicFont={Roboto Light Italic},
  ]{Roboto}
\else\ifbooklet
  \setmainfont[
    ItalicFont={Roboto Light Italic},
  ]{Roboto}
\else
\setmainfont[
  ItalicFont={Roboto Italic},
]{Roboto Light}
\fi\fi
\renewcommand{\LettrineFontHook}{\bfseries\color{rubric}}
% \renewenvironment{labelblock}{\begin{center}\bfseries\color{rubric}}{\end{center}}

%%%%%%%%
% MISC %
%%%%%%%%

\setdefaultlanguage[frenchpart=false]{french} % bug on part


\newenvironment{quotebar}{%
    \def\FrameCommand{{\color{rubric!10!}\vrule width 0.5em} \hspace{0.9em}}%
    \def\OuterFrameSep{\itemsep} % séparateur vertical
    \MakeFramed {\advance\hsize-\width \FrameRestore}
  }%
  {%
    \endMakeFramed
  }
\renewenvironment{quoteblock}% may be used for ornaments
  {%
    \savenotes
    \setstretch{0.9}
    \normalfont
    \begin{quotebar}
  }
  {%
    \end{quotebar}
    \spewnotes
  }


\renewcommand{\headrulewidth}{\arrayrulewidth}
\renewcommand{\headrule}{{\color{rubric}\hrule}}

% delicate tuning, image has produce line-height problems in title on 2 lines
\titleformat{name=\chapter} % command
  [display] % shape
  {\vspace{1.5em}\centering} % format
  {} % label
  {0pt} % separator between n
  {}
[{\color{rubric}\huge\textbf{#1}}\bigskip] % after code
% \titlespacing{command}{left spacing}{before spacing}{after spacing}[right]
\titlespacing*{\chapter}{0pt}{-2em}{0pt}[0pt]

\titleformat{name=\section}
  [block]{}{}{}{}
  [\vbox{\color{rubric}\large\raggedleft\textbf{#1}}]
\titlespacing{\section}{0pt}{0pt plus 4pt minus 2pt}{\baselineskip}

\titleformat{name=\subsection}
  [block]
  {}
  {} % \thesection
  {} % separator \arrayrulewidth
  {}
[\vbox{\large\textbf{#1}}]
% \titlespacing{\subsection}{0pt}{0pt plus 4pt minus 2pt}{\baselineskip}

\ifaiv
  \fancypagestyle{main}{%
    \fancyhf{}
    \setlength{\headheight}{1.5em}
    \fancyhead{} % reset head
    \fancyfoot{} % reset foot
    \fancyhead[L]{\truncate{0.45\headwidth}{\fontrun\elbibl}} % book ref
    \fancyhead[R]{\truncate{0.45\headwidth}{ \fontrun\nouppercase\leftmark}} % Chapter title
    \fancyhead[C]{\thepage}
  }
  \fancypagestyle{plain}{% apply to chapter
    \fancyhf{}% clear all header and footer fields
    \setlength{\headheight}{1.5em}
    \fancyhead[L]{\truncate{0.9\headwidth}{\fontrun\elbibl}}
    \fancyhead[R]{\thepage}
  }
\else
  \fancypagestyle{main}{%
    \fancyhf{}
    \setlength{\headheight}{1.5em}
    \fancyhead{} % reset head
    \fancyfoot{} % reset foot
    \fancyhead[RE]{\truncate{0.9\headwidth}{\fontrun\elbibl}} % book ref
    \fancyhead[LO]{\truncate{0.9\headwidth}{\fontrun\nouppercase\leftmark}} % Chapter title, \nouppercase needed
    \fancyhead[RO,LE]{\thepage}
  }
  \fancypagestyle{plain}{% apply to chapter
    \fancyhf{}% clear all header and footer fields
    \setlength{\headheight}{1.5em}
    \fancyhead[L]{\truncate{0.9\headwidth}{\fontrun\elbibl}}
    \fancyhead[R]{\thepage}
  }
\fi

\ifav % a5 only
  \titleclass{\section}{top}
\fi

\newcommand\chapo{{%
  \vspace*{-3em}
  \centering % no vskip ()
  {\Large\addfontfeature{LetterSpace=25}\bfseries{\elauthor}}\par
  \smallskip
  {\large\eldate}\par
  \bigskip
  {\Large\selectfont{\eltitle}}\par
  \bigskip
  {\color{rubric}\hline\par}
  \bigskip
  {\Large TEXTE LIBRE À PARTICPATION LIBRE\par}
  \centerline{\small\color{rubric} {hurlus.fr, tiré le \today}}\par
  \bigskip
}}

\newcommand\cover{{%
  \thispagestyle{empty}
  \centering
  {\LARGE\bfseries{\elauthor}}\par
  \bigskip
  {\Large\eldate}\par
  \bigskip
  \bigskip
  {\LARGE\selectfont{\eltitle}}\par
  \vfill\null
  {\color{rubric}\setlength{\arrayrulewidth}{2pt}\hline\par}
  \vfill\null
  {\Large TEXTE LIBRE À PARTICPATION LIBRE\par}
  \centerline{{\href{https://hurlus.fr}{\dotuline{hurlus.fr}}, tiré le \today}}\par
}}

\begin{document}
\pagestyle{empty}
\ifbooklet{
  \cover\newpage
  \thispagestyle{empty}\hbox{}\newpage
  \cover\newpage\noindent Les voyages de la brochure\par
  \bigskip
  \begin{tabularx}{\textwidth}{l|X|X}
    \textbf{Date} & \textbf{Lieu}& \textbf{Nom/pseudo} \\ \hline
    \rule{0pt}{25cm} &  &   \\
  \end{tabularx}
  \newpage
  \addtocounter{page}{-4}
}\fi

\thispagestyle{empty}
\ifaiv
  \twocolumn[\chapo]
\else
  \chapo
\fi
{\it\elabstract}
\bigskip
\makeatletter\@starttoc{toc}\makeatother % toc without new page
\bigskip

\pagestyle{main} % after style

  \section[{Préface}]{Préface}\renewcommand{\leftmark}{Préface}

\noindent Les discours qu’on a recueillis dans le présent volume ont été pour la plupart imprimés séparément au moment où ils furent prononcés. Les brochures qui les contenaient étant épuisées, on a jugé à propos de réunir ici les moins oubliées de ces paroles jetées au vent. Quelques répétitions ont été la conséquence d’un pareil rapprochement, fortuit à certains égards. L’indulgence du lecteur admettra facilement que de brèves allocutions, parties du cœur sans nul apprêt, à peu d’intervalle les unes des autres offrent des pensées qui se ressemblent. On a mieux aimé compter sur cette indulgence que de pratiquer des suppressions qui eussent changé le caractère de petites pièces marquées au coin d’une inspiration toute spontanée. La parole une fois émise est un fait qui a une date. Il n’y faut rien changer, même quand, en se relisant à huit et dix ans de distance, on peut trouver que certaines assertions seraient à confirmer, d’autres à atténuer, quelques-unes à présenter sous un jour différent.\par
Le morceau de ce volume auquel j’attache le plus d’importance et sur lequel je me permets d’appeler l’attention du lecteur, est la conférence : \emph{Qu’est-ce qu’une nation} ? J’en ai pesé chaque mot avec le plus grand soin ; c’est ma profession de foi en ce qui touche les choses humaines, et, quand la civilisation moderne aura sombré par suite de l’équivoque funeste de ces mots : nation, nationalité, race, je désire qu’on se souvienne de ces vingt pages-là. Je les crois tout à fait correctes. On va aux guerres d’extermination, parce qu’on abandonne le principe salutaire de l’adhésion libre, parce qu’on accorde aux nations, comme on accordait autrefois aux dynasties, le droit de s’annexer des provinces malgré elles. Des politiques transcendants se raillent de notre principe français, que, pour disposer des populations, il faut préalablement avoir leur avis. Laissons-les triompher à leur aise. C’est nous qui avons raison. Ces façons de prendre les gens à la gorge et de leur dire : « Tu parles la même langue que nous, donc tu nous appartiens », ces façons-là sont mauvaises ; la pauvre humanité, qu’on traite un peu trop comme un troupeau de moutons, finira par s’en lasser.\par
L’homme n’appartient ni à sa langue, ni à sa race : il n’appartient qu’à lui-même, car c’est un être libre, c’est un être moral. On n’admet plus qu’il soit permis de persécuter les gens pour leur faire changer de religion ; les persécuter pour leur faire changer de langue ou de patrie nous paraît tout aussi mal. Nous pensons qu’on peut sentir noblement dans toutes les langues et, en parlant des idiomes divers, poursuivre le même idéal. Au-dessus de la langue, de la race, des frontières naturelles, de la géographie, nous plaçons le consentement des populations, quels que soit leur langue, leur race, leur culte. La {\placeName Suisse} est peut-être la nation de l’{\persName Europe} la plus légitimement composée. Or elle compte dans son sein trois ou quatre langues, deux ou trois religions et Dieu sait combien de races. Une nation, c’est pour nous une âme, un esprit, une famille spirituelle, résultant, dans le passé, de souvenirs, de sacrifices, de gloires, souvent de deuils et de regrets communs ; dans le présent, du désir de continuer à vivre ensemble. Ce qui constitue une nation, ce n’est pas de parler la même langue ou d’appartenir au même groupe ethnographique, c’est d’avoir fait ensemble de grandes choses dans le passé et de vouloir en faire encore dans l’avenir. Le droit des populations à décider de leur sort est la seule solution aux difficultés de l’heure présente que peuvent rêver les sages ; c’est dire qu’elle n’a aucune chance d’être adoptée. Les grands hommes qui, en ce moment, gouvernent les affaires des peuples (avec quel succès ? l’avenir le dira) n’ont pour de telles naïvetés que le dédain. Mais il y a une raison, je l’avoue, qui m’a rendu insensible au dédain des politiques sûrs d’eux-mêmes. Depuis que j’ai pu observer les choses humaines, j’ai vu huit ou dix écoles d’hommes d’État qui se sont crues en possession de la sagesse et ont traité ceux qui doutaient d’elles avec la dernière ironie. Une ironie supérieure, celle du sort, a donné successivement de cruels démentis à ces infaillibles d’un jour. Et cela n’a pas rendu les autres modestes !… Ah ! quel profond penseur était ce juif du \textsc{vi}\textsuperscript{e} siècle avant Jésus-Christ, qui, à la vue des écroulements d’empires de son temps, s’écriait : \emph{« Et voilà comme les nations se fatiguent pour le néant, s’exténuent au profit du feu !\footnote{\emph{Jérémie}, LI, 58.} »}\par

\dateline{Dimanche, 8 mai 1887.}
\section[{Discours de réception à l’Académie française}]{Discours de réception à l’{\orgName Académie française}}\renewcommand{\leftmark}{Discours de réception à l’{\orgName Académie française}}


\dateline{3 avril 1879}

\salute{Messieurs,}
\noindent Ce grand {\persName cardinal de Richelieu}, comme tous les hommes qui ont laissé dans l’histoire la marque de leur passage, se trouve avoir fondé bien des choses auxquelles il ne pensait guère, certaines même qu’il ne voulait qu’à demi. Je ne sais, par exemple, s’il se souciait beaucoup de ce que nous appelons aujourd’hui tolérance réciproque et liberté de penser. La déférence pour les idées contraires aux siennes n’était pas sa vertu dominante, et, quant à la liberté, on ne voit pas qu’elle eût sa place indiquée dans le plan de l’édifice qu’il bâtissait. Et pourtant, voici qu’à deux cent cinquante ans de distance, l’âpre fondateur de l’unité française se trouve, dans un sens très réel, avoir été le fauteur de principes qu’il eût peut-être vivement combattus, s’il les eût vus éclore de son vivant. Cette compagnie, qui est après tout la plus durable de ses créations (depuis deux siècles et demi, elle vit sans avoir modifié un seul article de son règlement !), qu’est-elle, Messieurs, si ce n’est une grande leçon de liberté, puisque ici toutes les opinions politiques, philosophiques, religieuses, littéraires, toutes les façons de comprendre la vie, tous les genres de talent, tous les mérites, s’assoient côte à côte avec un droit égal ? La règle de la maison de Mécène, vous l’observez :\par

… Nil mi officit unquam  \\
Ditior hic aut est quia doctior ; est locus uni \\
Cuique suus…\\

\noindent Réunir les hommes, c’est être bien près de les réconcilier, c’est au moins rendre à l’esprit humain le plus signalé des services, puisque l’œuvre pacifique de la civilisation résulte d’éléments contradictoires, maintenus face à face, obligés de se tolérer, amenés à se comprendre et presque à s’aimer.\par
Que vit, en effet, Messieurs, avec une admirable sagacité, votre grand fondateur ? Une chose qu’on a exprimée depuis avec beaucoup de prétention, mais qu’il fit mieux que de proclamer en paroles, qu’il appliqua ; je veux dire ce principe qu’à un certain degré d’élévation, toutes les grandes fonctions de la vie raisonnable sont sœurs ; que, dans une société bien organisée, tous ceux qui se consacrent aux belles et bonnes choses sont collaborateurs ; que tout devient littérature quand on le fait avec talent ; en d’autres termes, que les lettres sont en quelque sorte l’Olympe où s’éteignent toutes les luttes, toutes les inégalités, où s’opèrent toutes les réconciliations. Séparées en leurs applications spéciales, souvent opposées, ennemies même, les maîtrises diverses du monde des esprits se rencontrent sur les sommets où elles aspirent. La paix n’habite que les hauteurs. C’est en montant, montant toujours, que la lutte devient harmonie, et que l’apparente incohérence des efforts de l’homme aboutit à cette grande lumière, la gloire, qui est encore, quoi que l’on dise, ce qui a le plus de chance de n’être pas tout à fait une vanité.\par
C’est là l’idée mère de votre {\orgName Compagnie}, Messieurs. Elle repose avant tout sur ce que je serais tenté d’appeler le grand dogme français, l’unité de la gloire, la communauté de l’esprit humain, l’assimilation de tous les ordres de services sociaux en une légion unique, créée, maintenue, sanctionnée, couronnée par la patrie. Le génie de la {\placeName France} avait déjà donné la mesure de sa largeur en créant {\placeName Paris}, ce centre incomparable, où se rencontrent et se croisent toutes les excitations, tous les éveils, le monde, la science, l’art, la littérature, la politique, les hautes pensées et les instincts populaires, l’héroïsme du bien, par moments la fièvre du mal. Le {\persName cardinal de Richelieu}, en fondant votre {\orgName Compagnie} « sur des fondements assez forts (ce sont ses propres paroles) pour durer autant que la monarchie », la {\orgName Convention nationale}, en décrétant l’{\orgName Institut}, le {\orgName premier Consul}, en établissant la Légion d’honneur, furent conduits par la même pensée : c’est que l’{\orgName État}, fondé sur la raison, croit au bien et au vrai et en voit la suprême unité. Toutes les noblesses leur apparurent comme égales. La gloire est quelque chose d’homogène et d’identique. Tout ce qui vibre la produit. Il n’y a pas plusieurs espèces de gloire, pas plus qu’il n’y a plusieurs espèces de lumière. À un degré inférieur, il y a les mérites divers ; mais la gloire de {\persName Descartes}, celle de {\persName Pascal}, celle de {\persName Molière}, sont composées des mêmes rayons.\par
La plupart des pays civilisés, depuis le XVI\textsuperscript{e} siècle, ont eu des académies, et la science a tiré le plus grand profit de ces associations, où, de la discussion et de la confrontation des idées, naît parfois la vérité. Votre principe va plus loin et plonge plus profondément dans l’intime de l’esprit humain. Vous trouvez que le poète, l’orateur, le philosophe, le savant, le politique, l’homme qui représente éminemment la civilité d’une nation, celui qui porte dignement un de ces noms qui sont synonymes d’honneur et de patrie, que tous ces hommes-là, dis-je, sont confrères, qu’ils travaillent à une œuvre commune, à constituer une société grande et libérale. Rien ne vous est indifférent : le charme mondain, le goût, le tact, sont pour vous de la bonne littérature. Ceux qui parlent bien, ceux qui pensent bien, ceux qui sentent bien, le savant qui a fait de profondes découvertes, l’homme éloquent qui a dirigé sa patrie dans la glorieuse voie du gouvernement libre, le méditatif solitaire qui a consacré sa vie à la vérité, tout ce qui a de l’éclat, tout ce qui produit de la lumière et de la chaleur, tout ce dont l’opinion éclairée s’occupe et s’entretient, tout cela vous appartient ; car vous repoussez également et l’étroite conception de la vie qui renferme chaque homme dans sa spécialité comme dans une espèce de besogne obscure dont il ne doit pas sortir, et la fade rhétorique où l’art de bien dire est confiné dans les écoles, séparé du monde et de la vie.\par
Cet esprit de votre fondation, vous le conservez admirablement, Messieurs ; et m’en faut-il d’autre preuve que ce que je vois en venant occuper aujourd’hui le siège où votre indulgence a bien voulu m’appeler ? Pour ne rien dire de pertes récentes et si cruelles que seule votre {\orgName Compagnie} pouvait les endurer sans être amoindrie, quelle variété je trouve en cette enceinte, quels hommes, quels caractères, quels cœurs ! Vous, cher et illustre maître, dont le génie, comme le timbre des cymbales de {\persName Bivar}, a sonné chaque heure de notre siècle, donné un corps à chacun de nos rêves, des ailes à chacune de nos pensées. Vous, bien-aimé confrère, qui trouvez dans une noble philosophie la conciliation du devoir et de la liberté. Ici je vois la poésie souveraine, qui nous impose le monde qu’elle crée, nous entraîne, nous dompte sous le coup impérieux de son archet magique ; là (ces contrastes sont votre gloire) le sens droit et ferme de la vie, l’art charmant du romancier, l’esprit du moraliste, et, ce que notre pays seul connaît encore, le rire aimable, l’ironie légère. Ici la foi sincère, l’art excellent de tirer d’un culte bien entendu pour le passé la dignité de toute une vie, le repos dans des doctrines qu’il n’est pas permis de qualifier d’étroites, puisque de grands génies s’y sont trouvés à l’aise ; là une négation réfléchie, calme, sûre d’elle-même, et donnant à l’âme forte qui s’y complaît le même repos, au caractère d’acier qui s’y plie la même grandeur que la foi. Ici la politique sincère, qui, dans nos jours troublés, a cru, pour sauver le pays, devoir revenir aux maximes qui l’ont fondé ; là une politique non moins sincère, qui s’est tournée résolument vers l’avenir et a conçu la possibilité d’une société vivante et forte sans les conditions qui autrefois paraissaient pour cela de nécessité absolue. Et dans l’appréciation du plus grand évènement de l’histoire moderne, de cette Révolution qui est devenue comme la croix de chemin où l’on se divise, le symbole sur lequel on se compte, que de pacifiques dissentiments ! Ici la foi dans le signe qui une fois a vaincu, l’enthousiasme des jours sublimes où un souffle étrange courut dans cette foule et la fit penser et parler pour l’humanité, la hardie assurance de cœurs virils, disant à leurs aînés, comme les jeunes gens de {\placeName Sparte} : \emph{« Nous serons ce que vous fûtes »} ; là un loyal effort pour peindre dans toute leur vérité des scènes funestes et dont on voudrait dire, comme L’Hôpital de la Saint-Barthélemy : \emph{Nocte tegi nostræ patiamur crimina gentis.}\par
Où est donc votre unité, Messieurs ? Elle est dans l’amour de la vérité, dans le génie qui la trouve, dans l’art savant qui la fait valoir. Vous ne couronnez pas telle ou telle opinion ; vous couronnez la sincérité et le talent. Vous admettez pleinement que, dans toutes les écoles, dans tous les systèmes, dans tous les partis, il y a place pour l’éloquence et la droiture du cœur. Tout ce qui peut s’exprimer en bon français, tout ce qui fait le grand homme ou l’homme aimable, a chez vous ses entrées. Il y a une source commune d’où dérivent le bon style et la bonne vie, le bien-dire et le noble caractère. Vous enseignez la chose dont l’humanité a le plus besoin, la concorde, l’union des contrastes. Ah ! si le monde pouvait vous imiter ! L’homme vit quatre jours ici-bas ; quoi de plus fou que de les passer à haïr, quand il est clair que l’avenir nous jugera comme nous jugeons le passé et que, dans cinquante ans, on traitera d’enfantillage les batailles où nous sacrifions le meilleur de notre vie !\par
Voilà le secret de votre éternelle jeunesse ; voilà pourquoi votre institution verdoie, quand le monde vieillit. Tout s’embrasse dans votre sein. Ailleurs la littérature et la société sont choses distinctes, profondément divisées. Dans notre pays, grâce à vous, elles se pénètrent. Vous vous inquiétez peu d’entendre annoncer pompeusement l’avènement de ce qu’on appelle une autre {\itshape culture}, qui saura se passer du talent. Vous vous défiez d’une {\itshape culture} qui ne rend l’homme ni plus aimable ni meilleur. Je crains fort que des races, bien sérieuses sans doute, puisqu’elles nous reprochent notre légèreté, n’éprouvent quelque mécompte dans l’espérance qu’elles ont de gagner la faveur du monde par de tout autres procédés que ceux qui ont réussi jusqu’ici. Une science pédantesque en sa solitude, une littérature sans gaieté, une politique maussade, une haute société sans éclat, une noblesse sans esprit, des gentilshommes sans politesse, de grands capitaines sans mots sonores, ne détrôneront pas, je crois, de sitôt le souvenir de cette vieille société française, si brillante, si polie, si jalouse de plaire. Quand une nation, par ce qu’elle appelle son sérieux et son application, aura produit ce que nous avons fait avec notre frivolité, des écrivains supérieurs à {\persName Pascal} et à {\persName Voltaire}, de meilleures têtes scientifiques que {\persName d’Alembert} et {\persName Lavoisier}, une noblesse mieux élevée que la nôtre au XVII\textsuperscript{e} et au XVIII\textsuperscript{e} siècle, des femmes plus charmantes que celles qui ont souri à notre philosophie, un élan plus extraordinaire que celui de notre Révolution, plus de facilité à embrasser les nobles chimères, plus de courage, plus de savoir-vivre, plus de bonne humeur pour affronter la mort, une société, en un mot, plus sympathique et plus spirituelle que celle de nos pères, alors nous serons vaincus. Nous ne le sommes pas encore. Nous n’avons pas perdu l’audience du monde. Créer un grand homme, frapper des médaillons pour la postérité, n’est pas donné à tous. Il y faut votre collaboration. Ce qui se fait sans les {\orgName Athéniens} est perdu pour la gloire ; longtemps encore vous saurez seuls décerner une louange qui fasse vivre éternellement.\par
Ainsi, en conservant votre vieil esprit, vous conservez la meilleure des choses. Vous admettez tous les changements, tous les progrès dans les idées ; les cadres, vous les maintenez, et, de tous les cadres, le plus essentiel, c’est la langue. Une langue bien faite n’a plus besoin de changer. Le français, tel que l’a créé le XVII\textsuperscript{e} siècle, peut servir à l’expression d’idées que n’avait pas le XVII\textsuperscript{e} siècle. Assurément, quelques modifications de nuances sont nécessaires. Même le {\persName cardinal de Retz} aurait besoin d’un moment de réflexion pour comprendre certaines phrases de {\persName Turgot} et de {\persName Condorcet}. {\persName Turgot} et {\persName Condorcet} remarqueraient, s’ils pouvaient nous lire, que, chez les meilleurs écrivains de notre temps, le sens de quelques mots, tels que {\itshape révolution, agitation, développement, mouvement, apparition}, a pris une extension répondant à certaines idées philosophiques. Mais la langue est bien la même ; on ne la trouve pauvre, cette vieille et admirable langue, que quand on ne la sait pas ; on ne prétend l’enrichir que quand on ne veut pas se donner la peine de connaître sa richesse. Toutes les hardiesses sont permises, excepté les hardiesses contre vous, Messieurs. On ne vous brave jamais impunément. J’ai remarqué que cela portait malheur. Dans mes plus grandes libertés, la crainte de l’{\orgName Académie} a toujours été au fond de mon cœur, et je m’en suis bien trouvé.\par
Merci donc, Messieurs, de m’avoir associé à votre {\orgName Compagnie} et à votre œuvre. Comptez sur moi pour vous aider à étonner les personnes qui n’ont pas le secret de vos choix et n’en comprennent pas toute la philosophie. Vous n’êtes pas une distribution de prix. L’hérésie la plus dangereuse en ce monde est de réclamer en tout une justice rigoureuse, que la nature n’a pas voulue. Justes, vous l’êtes jusque dans vos délais. On arrive à votre cénacle à l’âge de l’Ecclésiaste, âge charmant, le plus propre à la sereine gaieté, où l’on commence à voir, après une jeunesse laborieuse, que tout est vanité, mais aussi qu’une foule de choses vaines sont dignes d’être longuement savourées. Mes confrères de l’{\orgName Académie des Inscriptions et Belles-Lettres}, qui me connaissent depuis vingt-deux ans, vous rendront ce témoignage que je suis bon académicien, bien exact dans l’accomplissement de mes devoirs. Comptez sur mon assiduité et mon application ; moi, je compte sur de charmantes heures à passer parmi vous.\par
Ces maximes fondamentales que j’essayais d’esquisser tout à l’heure, vous les avez admirablement appliquées, Messieurs, le jour où vous choisissiez pour confrère l’homme illustre auquel vous m’avez appelé à succéder parmi vous. {\persName Claude Bernard} fut le plus grand physiologiste de notre siècle. L’{\orgName Académie des Sciences} fera son éloge ; elle exposera ces découvertes surprenantes qui ont porté la lumière sur les opérations les plus intimes des êtres organisés. Ce n’est pas le physiologiste que vous avez nommé, Messieurs ; dans les élections de savants illustres, c’est l’homme même, ou, en d’autres termes, l’écrivain que vous prenez. L’intelligence humaine est un ensemble si bien lié dans toutes ses parties qu’un grand esprit est toujours un bon écrivain. La vraie méthode d’investigation, supposant un jugement ferme et sain, entraîne les solides qualités du style. Tel mémoire de {\persName Letronne} et d’{\persName Eugène Burnouf}, en apparence étranger à tout souci de la forme, est un chef-d’œuvre à sa manière. La règle du bon style scientifique, c’est la clarté, la parfaite adaptation au sujet, le complet oubli de soi-même, l’abnégation absolue. Mais c’est là aussi la règle pour bien écrire en quelque matière que ce soit. Le meilleur écrivain est celui qui traite un grand sujet, et s’oublie lui-même, pour laisser parler son sujet. \emph{« Il se sert de la parole, écrivait {\persName M. de Cambrai} à votre secrétaire perpétuel, comme un homme modeste de son habit pour se couvrir. Il pense, il sent, la parole suit. »} Principe admirablement vrai ! Le beau est hors de nous, notre tâche est de nous mettre à son service et d’en être les dignes interprètes. Avoir quelque chose à dire, ne pas gâter la beauté naturelle d’un sujet noble, d’une pensée vraie, par le désordre, l’obscurité, l’incorrection, le faux goût, telle est la condition essentielle de cet art du bon langage, que certaines personnes, bien à tort, se figurent distinct de l’art même de penser et de trouver le vrai.\par
C’est en vous souvenant de ces principes que votre attention se porta sur un homme voué aux travaux en apparence les plus éloignés de ce qu’on peut appeler la littérature. Il passait sa vie dans un laboratoire obscur au {\orgName Collège de France} ; et là, au milieu des spectacles les plus repoussants, respirant l’atmosphère de la mort, la main dans le sang, il trouvait les plus intimes secrets de la vie, et les vérités qui sortaient de ce triste réduit éblouissaient tous ceux qui savaient les voir. Écrivain, certes il l’était, et écrivain excellent ; car il ne pensa jamais à l’être. Il eut la première qualité de l’écrivain, qui est de ne pas songer à écrire. Son style, c’est sa pensée elle-même ; et, comme cette pensée est toujours grande et forte, son style aussi est toujours grand, solide et fort. Rhétorique excellente que celle du savant ! Car elle repose sur la justesse d’un style vrai, sobre, proportionné à ce qu’il s’agit d’exprimer, ou plutôt sur la logique, base unique, base éternelle du bon style. Rhétorique au fond identique à celle de l’orateur, « qui ne se sert de la parole que pour la pensée et de la pensée que pour la vérité ! » Rhétorique au fond identique à celle du grand poète ! Car il y a une logique dans une tragédie en cinq actes comme dans un mémoire de physiologie, et la règle des ouvrages de l’esprit est toujours la même : être égal à la vérité, ne pas l’affaiblir en s’y mêlant, se mettre tout entier à son service, s’immoler à elle pour la montrer seule, dans sa haute et sereine beauté.\par
Telle est la raison qui fait que, depuis votre fondation, vous avez eu pour confrères {\persName Mairan}, {\persName Buffon}, {\persName d’Alembert}, {\persName Vicq d’Azyr}, {\persName Cuvier}, {\persName Claude Bernard} et le chimiste illustre qui continue à l’heure qu’il est dans votre sein cette glorieuse tradition. Vous représentez l’esprit humain. Comment le plus beau fleuron de l’esprit humain, la science, vous serait-elle étrangère ? Vous ne voyez, il est vrai, que le résultat ; l’œuvre pénible du laboratoire n’est pas votre domaine. De même que, le soir, en admirant l’éclairage de nos grandes cités, nous jouissons de l’éblouissante lumière sans songer au récipient obscur où elle se prépare, de même vous assistez à ces éclosions merveilleuses sans vous préoccuper du travail matériel qui les amène. Vous acceptez les conquêtes définitives ; vous constatez les transformations que ces merveilleuses découvertes introduisent dans toute la discipline de l’esprit. Qui ne voit que {\persName Galilée}, {\persName Descartes}, {\persName Newton}, {\persName Lavoisier}, {\persName Laplace} ont changé la base de la pensée humaine, en modifiant totalement l’idée de l’univers et de ses lois, en substituant aux enfantines imaginations des âges non scientifiques la notion d’un ordre éternel, où le caprice, la volonté particulière, n’ont plus de part ? Ont-ils diminué l’univers, comme le pensent quelques personnes ? Pour moi, j’estime tout le contraire. Le ciel, tel qu’on le voit avec les données de l’astronomie moderne, est bien supérieur à cette voûte solide, constellée de points brillants, portée sur des piliers, à quelques lieues de distance en l’air, dont les siècles naïfs se contentèrent. Je ne regrette pas beaucoup les petits génies qui autrefois dirigeaient les planètes dans leur orbite ; la gravitation s’acquitte beaucoup mieux de cette besogne, et, si par moments j’ai quelques mélancoliques souvenirs pour les neuf chœurs d’anges qui embrassaient les orbes des sept planètes, et pour cette mer cristalline qui se déroulait aux pieds de {\persName l’Éternel}, je me console en songeant que l’infini où notre œil plonge est un infini réel, mille fois plus sublime aux yeux du vrai contemplateur que tous les cercles d’azur des paradis d’{\persName Angelico de Fiésole}. L’homme d’État illustre dont la mort a produit un si grand vide dans votre {\orgName Compagnie} laissait rarement passer une belle nuit sans jeter un regard sur cet océan sans limites. \emph{« C’est là ma messe »}, disait-il. Combien les vues profondes du chimiste et du cristallographe sur l’atome dépassent la vague notion de la matière dont vivait la philosophie scolastique ! Et quant à l’âme, qui venait, à un moment donné avant la naissance, s’adjoindre à une masse qui jusque-là ne méritait aucun nom, mon Dieu ! parfois je la regrette, je l’avoue ; car il était facile de démontrer qu’une telle âme, créée tout exprès, se détachait sans peine du corps qu’elle avait cessé d’animer ; mais, en y réfléchissant, je retrouve plus d’âme encore dans ce mystère sans fond de la vie, où nous voyons la conscience émerger de l’abîme, comme un rameau d’or prédestiné, et l’œuvre divine se poursuivre par un effort sans fin, où la personne de chacun de nous laissera une trace éternelle. Le triomphe de la science est en réalité le triomphe de l’idéalisme. Heureuse génération que la nôtre ! Combien de martyrs de la science ont voulu voir ces merveilles et n’en ont eu que l’incomplète divination ! Jouissons de ces connaissances que tant d’hommes illustres n’ont fait qu’entrevoir, et, quand l’horizon se charge de nuages passagers, quand nous serions tentés de médire de notre siècle, songeons que ces héros du passé, un {\persName Jordano Bruno}, un {\persName Galilée}, donneraient dix fois encore leur vie pour savoir le dixième de ce que nous savons, et qu’ils estimeraient de telles conquêtes trop peu achetées de leurs larmes, de leurs angoisses et de leur sang.\par
Et quant à la noblesse des caractères, comment reprocher à la science d’y porter atteinte, quand on voit les âmes qu’elle forme, ce désintéressement, ce dévouement absolu à l’œuvre, cet oubli de soi-même, qu’elle inspire et entretient ? Ici encore, nous n’avons rien à envier au passé. Aux saints, aux héros, aux grands hommes de tous les âges, nous comparerons sans crainte ces caractères scientifiques, attachés uniquement à la recherche de la vérité, indifférents à la fortune, souvent fiers de leur pauvreté, souriant des honneurs qu’on leur offre, aussi indifférents à la louange qu’au dénigrement, sûrs de la valeur de ce qu’ils font, et heureux, car ils ont la vérité. Grandes assurément sont les joies que donne une croyance assurée sur les choses divines ; mais le bonheur intime du savant les égale ; car il sent qu’il travaille à une œuvre d’éternité, et qu’il appartient à la phalange de ceux dont on peut dire : \emph{Opera eorum sequuntur illos.}\par
{\persName Claude Bernard}, Messieurs, fut de ceux-là. Sa vie, toute consacrée au vrai, est le modèle que nous pouvons opposer à ceux qui prétendent que, de notre temps, la source des grandes vertus est tarie. Il naquit au petit {\placeName village de Saint-Julien}, près {\placeName Villefranche}, dans une maison de vignerons, qui lui resta toujours chère, et où il passa, jusqu’aux derniers temps, ses moments les plus doux. \emph{« J’habite, écrivait-il, sur les {\placeName coteaux du Beaujolais}, qui font face à la {\placeName Dombe}. J’ai pour horizon les {\placeName Alpes}, dont j’aperçois les cimes blanches, quand le ciel est clair. En tout temps, je vois se dérouler à deux lieues devant moi les prairies de la {\placeName vallée de la Saône}. Sur les coteaux où je demeure, je suis noyé à la lettre dans des étendues sans bornes de vignes, qui donneraient au pays un aspect monotone, s’il n’était coupé par des vallées ombragées et par des ruisseaux qui descendent des montagnes vers la Saône. Ma maison, quoique située sur une hauteur, est comme un nid de verdure, grâce à un petit bois qui l’ombrage sur la droite et à un verger qui s’y appuie sur la gauche : haute rareté dans un pays où l’on défriche même les buissons pour planter de la vigne ! »}\par
{\persName Bernard} perdit son père de bonne heure ; dans ses premières années, comme au début de la vie de presque tous les grands hommes, se plaça l’amour d’une mère, qu’il adorait et dont il était adoré. Comme il apprenait bien à l’école, le curé le choisit pour enfant de chœur et lui fit commencer le latin. Il continua ses études au {\placeName collège de Villefranche}, tenu par des ecclésiastiques ; et, la situation de sa famille ne lui permettant pas les années de loisirs, il vint le plus tôt qu’il put à {\placeName Lyon}, où il trouva, chez un pharmacien du {\placeName faubourg de Vaise}, un emploi qui lui donnait la nourriture et le logement. Cette pharmacie desservait l’école vétérinaire située près de là, et c’était {\persName Bernard} qui portait les médicaments aux bêtes malades. Déjà il jetait plus d’un regard curieux sur ce qu’il voyait, et il y avait dans « {\persName Monsieur Claude} », comme l’appelait son patron, bien des choses qui étonnaient ce dernier. C’était surtout à propos de la thériaque qu’ils ne se comprenaient pas. Toutes les fois que {\persName Bernard} apportait à l’apothicaire des produits gâtés : \emph{« Gardez cela pour la thériaque, lui répondait ce digne homme ; ce sera bon pour faire de la thériaque. »} Telle fut l’origine première des doutes de notre confrère sur l’efficacité de l’art de guérir. Cette drogue infecte, fabriquée avec toutes les substances avariées de l’officine, quelle que fût leur nature, et qui guérissait tout de même, lui causait de profonds étonnements.\par
Il était jeune, et sa voie était encore obscure devant lui. Il essayait toute chose, eut un petit succès sur un théâtre de {\placeName Lyon} avec un vaudeville, dont il ne voulait jamais dire le titre, vint à {\placeName Paris}, ayant dans sa valise une tragédie en cinq actes et une lettre. Il tenait naturellement plus à la tragédie qu’à la lettre ; mais le fait est que la lettre valut pour lui mille fois plus que la tragédie. Elle était adressée à notre regretté confrère {\persName M. Saint-Marc Girardin}. L’honnête homme que nous avons connu se montra bien dans cette circonstance. Il lut la tragédie, fut très-net et conseilla au jeune homme d’apprendre un métier pour vivre, quitte à faire ensuite de la poésie à ses heures. {\persName Claude Bernard} suivit cette précieuse indication, et combien cela fut heureux, Messieurs ! Auteur dramatique, il eût ajouté quelques tragédies de plus au tas énorme de celles qui attendent à l’{\orgName Odéon} les réparations de la postérité ; il est douteux qu’il fût devenu votre confrère. Ainsi, en tournant le dos à la littérature, il prit le droit chemin qui devait le mener parmi vous. En réalité, sa vocation était scientifique. La médecine, qui est à la fois le plus honorable des états et la plus passionnante des sciences, fut l’occupation de son choix.\par
Les facilités qu’on a créées depuis aux abords des carrières scientifiques n’existaient point alors. La société humaine a été jusqu’ici ainsi faite que la recherche pure de la vérité ne rapporte rien à celui qui s’y livre. Le nombre de ceux qui s’intéressent à la vérité étant imperceptible, le savant vit, non de la science, mais des applications de la science ; or, de toutes les applications de la science, la plus indispensable a toujours été la médecine. Aux siècles barbares, la science n’en connut guère d’autre ; presque tous les savants du moyen âge, {\orgName musulmans} ou {\orgName chrétiens}, ont trouvé l’appui nécessaire à la vie en se disant médecins ; car l’homme le plus brutal et le plus fanatique, quand il est malade, veut être guéri. On peut dire que, si l’humanité s’était toujours bien portée, la science et la philosophie seraient vingt fois mortes de faim. {\persName Claude Bernard}, déjà invinciblement attiré par les problèmes de la nature vivante, embrassa la profession qui se trouvait en quelque sorte à sa portée ; mais, des deux grandes parties de la médecine, l’art de guérir et la connaissance du sujet à guérir, la seconde eut toutes ses préférences. Disons-le, {\persName Bernard} était aussi peu médecin que possible. Il était sceptique à l’égard de l’autel qu’il desservait. Le médecin, comme le magistrat, applique des règles qu’il sait n’être pas parfaites, et, de même que le meilleur magistrat fait souvent faire peu de progrès à la législation, de même le meilleur praticien n’est pas toujours un savant. Sa tâche est presque aussi difficile que celle de l’horloger à qui on demanderait de corriger les irrégularités d’une montre qu’il lui serait défendu d’ouvrir. Or, ce que cherchait {\persName Bernard}, c’était le secret même des rouages intérieurs ; cette montre, il la brisait, l’ouvrait violemment, plutôt que d’admettre qu’il fût permis de la manier à l’aveugle et sans savoir clairement ce que l’on fait.\par
Il expia comme il convient sa supériorité et ses dons exceptionnels. La physiologie, quand il débuta, n’avait guère de place dans l’enseignement. Lors de la division des sections dans le sein de l’{\orgName Académie des Sciences}, en 1795, division qui, par un privilège singulier, est venue jusqu’à nos jours presque sans modifications, on ne conçut la science de la vie que sous le nom de médecine. {\persName Claude Bernard} paya cher sa gloire d’être créateur. Il n’y avait pas de cadre pour lui. Le temps était plus favorable à une littérature souvent de médiocre aloi qu’à des recherches qui ne prêtaient pas à de jolies phrases. De son entre-sol de la {\placeName cour du Commerce}, {\persName Bernard} lutta seul. Il y avait dans la vie pauvre, ardente, du {\placeName quartier Latin} d’alors, tant de foi, d’espérance, de loyale et généreuse fraternité. Nulle épreuve ne l’arrêta. Avec son ami le {\persName Dr Lasègue}, il essaya, vers 1845, d’établir un laboratoire de physiologie. Cela se passait {\placeName rue Saint-Jacques}, près du {\placeName Panthéon}, avant que des trouées, désolantes pour ceux dont elles dérangent les souvenirs, eussent fait pénétrer l’air et le jour dans ces sombres ruelles qui n’avaient point changé depuis le XIV\textsuperscript{e} siècle. Le laboratoire n’eut pas plus de cinq ou six élèves, et l’établissement ne fit jamais les frais du hangar qui l’abritait ni des lapins qu’on y sacrifiait. Mais {\persName Claude Bernard} y conçut l’idée de ses expériences sur la corde du tympan, sur le suc gastrique. Il essaya les concours, et y échoua complètement ; il n’avait pas les qualités superficielles qui font réussir en des épreuves où c’est un défaut d’avoir des idées, et où l’on est perdu si un moment on se laisse aller à suivre sa propre pensée. Son air était gauche et embarrassé, et les brillants sujets qui croyaient se partager l’avenir ne lui prédisaient qu’une carrière médicale des plus modestes.\par
Quelqu’un qui ne s’y laissa point tromper, ce fut {\persName M. Magendie}. Le sort, on serait tenté de dire une harmonie préétablie, avait attaché {\persName Claude Bernard} au service de cet homme éminent, à l’{\orgName Hôtel-Dieu}. Jamais le hasard n’opéra un rapprochement plus judicieux. {\persName Bernard} et {\persName Magendie} étaient en quelque sorte créés pour se joindre, se compléter et se continuer. Si {\persName Magendie} n’eût pas eu {\persName Bernard} pour élève, sa gloire ne serait pas le quart de ce qu’elle est. Si {\persName Bernard} n’eût pas trouvé la direction de {\persName Magendie}, il est douteux qu’il eût pu surmonter les énormes difficultés matérielles que la fortune, par un jeu malin, semblait avoir semées devant lui, comme pour lui rendre méritoires les brillantes faveurs qu’elle lui réservait.\par
Chose singulière ! Le premier abord de l’homme qui devait être son initiateur à la vie scientifique lui fut désagréable, presque pénible. {\persName Magendie}, avec ses rares qualités, était peu aimable. Son accueil rude déconcerta le jeune interne, et un moment {\persName Bernard} méconnut la rare chance qui lui était échue. {\persName Magendie}, lui, n’hésita pas longtemps. Au bout de quelques jours, sachant à peine le nom de son jeune élève, ayant remarqué ses yeux et sa main pendant une dissection : \emph{« Dites donc, lui cria-t-il d’un bout de la table à l’autre, je vous prends pour mon préparateur au {\orgName Collège de France}. »} À partir de ce jour, la carrière de {\persName Claude Bernard} était tracée. Il avait trouvé l’établissement qui seul pouvait convenir au développement de son génie.\par
Grâce, en effet, à la complète liberté dont jouit le professeur dans cette école unique, {\persName Magendie}, suivant les traces de {\persName Laënnec}, faisait sous le titre de « Médecine » un cours de recherches originales sur les phénomènes physiques de la vie. {\persName Magendie} n’était pas l’idéal du médecin ; il était trop critique envers lui-même pour pratiquer un art qui consiste aussi souvent à consoler le malade qu’à le guérir. Mais c’était l’idéal du professeur au {\orgName Collège de France}, toujours cherchant le nouveau, ne visant en rien au cours complet, uniquement attentif à éveiller chez ses auditeurs l’esprit d’investigation. Comme le vrai professeur au {\orgName Collège de France}, il ne préparait pas son cours et donnait à ses élèves le spectacle de ses doutes, de ses perplexités. Bien différent de ceux qui prennent d’avance leurs précautions pour éviter l’embarras que leur causerait un entretien trop immédiat avec une réalité qui leur est peu familière, il interrogeait directement la nature, souvent sans savoir ce qu’elle répondrait. Quelquefois, quand il se hasardait à prédire le résultat, l’expérience disait juste le contraire. {\persName Magendie} alors s’associait à l’hilarité de son auditoire. Il était enchanté ; car, si son système, auquel il ne tenait pas, sortait ébréché de l’expérience, son scepticisme, auquel il tenait, en était confirmé. Avec ce caractère, il devait laisser à son préparateur une part considérable dans la direction du cours. {\persName Claude Bernard} faisait l’expérience de chaque leçon avec sa prodigieuse habileté d’opérateur, et, à la troisième ou quatrième séance, {\persName Magendie} sortait de la salle en disant du ton bourru qui lui était habituel : \emph{« Eh bien, tu es plus fort que moi. »}\par
Ce que {\persName Magendie}, en effet, avait voulu, prêché, désiré durant quarante ans, {\persName Claude Bernard} le faisait. L’expérience en physiologie n’était assurément pas une chose absolument neuve. {\persName Descartes}, dans les heures fécondes qu’il consacra à la science de la vie, en eut l’idée la plus claire. {\persName Harvey} avait vérifié la circulation du sang sur les daims des parcs royaux, que lui livrait {\persName Charles I\textsuperscript{er}}. {\persName Haller}, {\persName Réaumur}, {\persName Spallanzani} avaient imaginé les moyens les plus ingénieux pour prendre la nature sur le fait. De graves objections s’élevaient pourtant contre l’application de la méthode expérimentale à la vie. Le grand {\persName Cuvier} s’en fit l’interprète. La vie est une, disait-on ; l’attaquer dans sa simplicité est impossible ; attaquer chaque partie, la séparer de la masse, c’est la reporter dans l’ordre des substances inertes. On opposait trop la nature inorganique à la nature organisée. On se figurait que la vie résulte de forces à part, que les faits qui se passent dans l’être vivant sont assujettis à des lois toutes particulières, qu’un principe secret préside en chaque individu à la naissance, à la maladie, à la mort. {\persName Lavoisier} et {\persName Laplace} rompirent le charme et créèrent la physique animale en prouvant que la respiration est une combustion, source de la chaleur qui nous anime. {\persName Bichat} secoua le joug de l’ancien vitalisme, sans pourtant réussir à s’en dégager complètement. Il restait un principe mystérieux, en vertu duquel les phénomènes vitaux, contrairement aux lois des corps bruts, semblaient n’être pas identiques dans des circonstances identiques. Voilà ce que {\persName Magendie} nia tout à fait ; voilà ce que {\persName Claude Bernard} réfuta par des expériences sans nombre. En s’appliquant à produire les faits mêmes de la vie, en s’ingéniant à les gêner, à les contrarier, il réussit à les soumettre à des lois précises. La physiologie ainsi conçue devint la sœur de la physique et de la chimie. Dans les corps vivants, comme dans les corps bruts, les lois sont immuables. Le mot d’exception est antiscientifique. Ce qu’on appelle exception est un phénomène dont une ou plusieurs conditions sont inconnues.\par
L’expérimentateur chez {\persName Claude Bernard} était admirable, et jamais on ne fit parler la nature avec une si merveilleuse sagacité. Difficile envers lui-même, il était pour ses systèmes le pire des adversaires. Il critiquait ses propres idées aussi âprement que si elles eussent été celles d’un rival ; il s’acharnait à se démolir comme l’eût fait son pire ennemi. Aucune preuve ne lui paraissait solide que quand une contre-épreuve venait la confirmer. \emph{« Le grand principe expérimental, disait-il, est le doute, ce doute philosophique, qui laisse à l’esprit sa liberté et son initiative… Le raisonnement expérimental est précisément l’inverse du raisonnement scolastique. La scolastique veut toujours un point de départ fixe et indubitable, et, ne pouvant le trouver ni dans les choses extérieures ni dans la raison, elle l’emprunte à une source irrationnelle quelconque, telle qu’une révélation, une tradition, une autorité conventionnelle ou arbitraire… Le scolastique ou le systématique, ce qui est la même chose, ne doute jamais de son point de départ, auquel il veut tout ramener ; il a l’esprit orgueilleux et intolérant et n’accepte pas la contradiction… Au contraire, l’expérimentateur, qui doute toujours et qui ne croit posséder la certitude absolue sur rien, arrive à maîtriser les phénomènes qui l’entourent et à étendre sa puissance sur la nature. »}\par
Le courage que {\persName Bernard} montra dans ces luttes terribles contre un {\persName Protée} qui semble vouloir défendre ses secrets fut quelque chose d’admirable. Ses ressources étaient chétives. Ces merveilleuses expériences, qui frappaient d’admiration l’{\orgName Europe} savante, se faisaient dans une sorte de cave humide, malsaine, où notre confrère contracta probablement le germe de la maladie qui l’enleva ; d’autres se faisaient à {\placeName Alfort} ou dans les abattoirs. Ces expériences sur des chevaux furieux, sur des êtres imprégnés de tous les virus, étaient quelquefois effroyables. Le {\persName docteur Rayer} venait de découvrir que la plus terrible maladie du cheval se transmet à l’homme qui le soigne. {\persName Bernard} voulut étudier la nature de ce mal hideux. Dans une convulsion suprême, le cheval lui déchire le dessus de la main, la couvre de sa bave. \emph{« Lavez-vous vite, lui dit {\persName Rayer}, qui était à côté de lui. — Non, ne vous lavez pas, lui dit {\persName Magendie}, vous hâteriez l’absorption du virus. »} Il y eut une seconde d’hésitation. \emph{« Je me lave, dit {\persName Bernard}, en mettant la main sous la fontaine, c’est plus propre. »}\par
C’était un spectacle frappant de le voir dans son laboratoire, pensif, triste, absorbé, ne se permettant pas une distraction, pas un sourire. Il sentait qu’il faisait œuvre de prêtre, qu’il célébrait une sorte de sacrifice. Ses longs doigts plongés dans les plaies semblaient ceux de l’augure antique, poursuivant dans les entrailles des victimes de mystérieux secrets. \emph{« Le physiologiste n’est pas un homme du monde, disait-il ; c’est un savant, c’est un homme absorbé par une idée scientifique qu’il poursuit ; il n’entend plus les cris des animaux, il ne voit plus le sang qui coule, il ne voit que son idée et n’aperçoit que des organismes qui lui cachent des problèmes qu’il veut découvrir. De même le chirurgien n’est pas arrêté par les cris et les sanglots, parce qu’il ne voit que son idée et le but de son opération. De même encore l’anatomiste ne sent pas qu’il est dans un charnier horrible ; sous l’influence d’une idée scientifique, il poursuit avec délices un filet nerveux dans des chairs puantes et livides, qui seraient pour tout autre homme un objet de dégoût et d’horreur. »}\par
La fécondité dans l’invention des moyens de recherche répondait chez notre confrère à la profondeur des intuitions. Ce fut un vrai coup de génie d’avoir su faire du poison son grand agent expérimentateur. Le poison, en effet, va où ni la main ni l’œil ne peuvent aller. Il atteint les éléments mêmes de l’organisme, s’introduit dans la circulation, devient un réactif d’une délicatesse extrême pour disséquer les éléments vitaux, désassocier les nerfs sans les lacérer, pénétrer les derniers mystères du système nerveux. C’est par le poison, ainsi qu’on l’a très-bien dit, que {\persName Bernard} \emph{« installa son laboratoire au sein de l’économie animale ; il eut son réseau de communications instantanées, sa police secrète, si l’on peut s’exprimer ainsi, qui l’avertissait du trouble le plus furtif ».} Miracle ! Il rendit la mort locale et passagère, locale par les empoisonnements partiels, passagère par les anesthésiques ; et de la sorte, au scalpel qui mutile la vie, au microscope qui en fausse les proportions, il substitua ce qu’on a très-bien appelé l’autopsie vivante, sans mutilation ni effusion de sang.\par
Ainsi se produisirent ces étonnants travaux sur la formation du sucre chez les animaux, sur le grand sympathique, sur les mouvements réflexes, sur la respiration des tissus. L’unité de la vie fut, de la part de {\persName Claude Bernard}, l’objet des plus fines observations. À côté du système central il trouva en quelque sorte des autonomies provinciales, des circulations locales. Le cœur ne fut plus le point unique d’émission de vie. À côté de cette principale source de mouvement, {\persName Bernard} trouva des réseaux de circulation capillaire ayant leur vie propre, leurs accidents, leurs maladies, leurs anémies, leurs congestions en dehors du grand courant de la circulation générale.\par
Comme tous les esprits complets, {\persName Claude Bernard} a donné l’exemple et le précepte. En dehors de ses mémoires spéciaux, il a tracé à deux ou trois reprises son \emph{Discours de la méthode}, le secret même de sa pensée philosophique. C’est à {\placeName Saint-Julien}, loin de son laboratoire, pendant ses mois de repos ou de maladie, qu’il écrivit ces belles pages, et notamment cette \emph{Introduction à la médecine expérimentale}, qui le désigna surtout à votre choix. Il faut remonter à nos maîtres de Port-Royal pour trouver une telle sobriété, une telle absence de tout souci de briller, un tel dédain des procédés d’une littérature mesquine, cherchant à relever par de fades agréments l’austérité des sujets. Le style scientifique ne doit faire aucun sacrifice au désir de plaire. On n’égaye ces graves matières qu’en les rapetissant. C’est surtout quand il s’agit du style de la science que le grand principe \emph{évangélique « Qui perd son âme la sauve »}, est aussi un grand principe littéraire. C’est en pareil cas qu’il est vrai de dire : \emph{« Soyez aussi peu littérateur que possible, si vous voulez être bon littérateur. »}\par
La parole de {\persName Claude Bernard} était comme son style, pleine de bonne foi, d’honnêteté\emph{. « Il n’essayait jamais, dit un de ses meilleurs élèves, de produire aucun effet, et, se figurant les autres à son image, il pensait que la recherche de ce qui est devait suffire à les passionner, comme elle le passionnait lui-même. »} À l’exemple de son maître {\persName Magendie}, il faisait de son cours le spectacle vivant de ses recherches, initiant le public à tous ses secrets. On assistait au travail de sa pensée. La science ne veut pas être crue sur parole, et les cours du {\orgName Collège de France} ont pour objet de montrer aux yeux de tous ce qui d’ordinaire se cache dans les laboratoires. {\persName Bernard} pensait en parlant ; il pouvait en résulter par moments un peu de confusion. L’objection lui venait, le troublait. Les pensées se heurtaient dans sa tête ; au milieu d’une exposition, l’idée d’une expérience lui traversait l’esprit, l’arrêtait court, le rendait distrait. Mais tout à coup la lumière éclatait. Dans sa conversation avec ses élèves, dans ces causeries où \emph{« il faisait, selon l’expression de l’un d’eux, l’apprentissage de son génie »}, il était admirable\emph{. « Il y a dans tout ce que j’écris, avouait-il, certaines parties qui ne sauraient être comprises par d’autres que moi. Ce sont des germes d’idées que je dépose en quelque sorte pour les reprendre plus tard. »} Dans la conversation, ces flots de vérités pressées débordaient en toute liberté.\par
La plus haute philosophie, en effet, résultait de cet ensemble de faits constatés avec une inflexible rigueur. Comme loi suprême de l’univers, {\persName Bernard} reconnaît ce qu’il appelle le {\itshape déterminisme}, c’est-à-dire la liaison inflexible des phénomènes, sans que nul agent extra-naturel intervienne jamais pour en modifier la résultante. Il n’y a pas, comme on l’avait dit souvent, deux ordres de sciences : celles-ci d’une précision absolue, celles-là toujours en crainte d’être dérangées par des forces mystérieuses. Cette grande inconnue de la physiologie, que {\persName Bichat} admettait encore, cette puissance capricieuse qui, prétendait-on, résistait aux lois de la matière et faisait de la vie une sorte de miracle, {\persName Bernard} l’exclut absolument. \emph{« L’obscure notion de cause, disait-il, doit être reportée à l’origine des choses ; … elle doit faire place dans la science à la notion du rapport et des conditions. Le déterminisme fixe les conditions des phénomènes ; il permet d’en prévoir l’apparition et de la provoquer… Il ne nous rend pas compte de la nature, il nous en rend maîtres… Que si, après cela, nous laissons notre esprit se bercer au vent de l’inconnu et dans les sublimités de l’ignorance, nous aurons au moins fait la part de ce qui est la science et de ce qui ne l’est pas. »}\par
Être maître de la nature, tel est, en effet, selon {\persName Claude Bernard}, le but de la science de la vie. Il pensait, après {\persName Descartes}, que les espérances les plus hardies sont dans cet ordre permises, et que la science des êtres vivants doit apprendre à subjuguer la nature vivante, comme la physique et la chimie subjuguent la nature morte. \emph{« Dans toute manifestation vitale, écrivait-il, la nature répète une leçon qu’elle a apprise et dont elle se souvient plus ou moins bien. Pourrait-on apprendre à la nature une nouvelle leçon, et sa mémoire la reproduirait-elle dans une série d’êtres nouveaux ? Je le crois ; c’est toujours ma vieille idée de refaire des êtres, non par génération spontanée, comme on l’a rêvé, mais par la répétition de phénomènes organiques dont la nature garderait souvenir. »}\par
Quoiqu’il parlât peu des questions sociales, il avait l’esprit trop grand pour n’y pas appliquer ses principes généraux. Ce caractère conquérant de la science, il l’admettait jusque dans le domaine des sciences de l’humanité. \emph{« Le rôle actif des sciences expérimentales, disait-il, ne s’arrête pas aux sciences physico-chimiques et physiologiques ; il s’étend jusqu’aux sciences historiques et morales. On a compris qu’il ne suffit pas de rester spectateur inerte du bien et du mal, en jouissant de l’un et en se préservant de l’autre. La morale moderne aspire à un rôle plus grand : elle recherche les causes, veut les expliquer et agir sur elles ; elle veut en un mot dominer le bien et le mal, faire naître l’un et le développer, lutter avec l’autre pour l’extirper et le détruire. »}\par
Les récompenses vinrent lentement à cette grande carrière, qui, à vrai dire, pouvait s’en passer, car elle était à elle-même sa propre récompense. Notre confrère avait eu les rudes commencements de la vie du savant, il en eut les tardives douceurs. L’{\orgName Académie des Sciences}, la {\orgName Sorbonne}, le {\orgName Collège de France,} le {\orgName Muséum} tinrent à honneur de le posséder. Votre {\orgName Compagnie} mit le comble à ces faveurs en lui conférant le premier des titres auquel puisse aspirer l’homme voué aux travaux de l’esprit. Une volonté personnelle de l’{\persName empereur Napoléon III} l’appela au {\orgName Sénat}. D’illustres et douces amitiés le consolèrent, des mains affectueuses furent de tous côtés attentives à lui diminuer les difficultés de la vie ; des élèves tels que {\persName Paul Bert}, {\persName Armand Moreau}, ses amis de la {\orgName Société de biologie}, recueillaient toutes ses paroles et l’assuraient que sa pensée était garantie contre la mort. Sa tête magistrale, toujours méditative, était devenue extrêmement belle à soixante ans. Il travaillait sans cesse et pourtant il ne savait pas ce que c’était que la fatigue, car il ne poursuivait jamais l’impossible ; il laissait la pensée venir, sans la solliciter. Sa sérénité était absolue ; il savait bien que l’emploi qu’il faisait de sa vie était le meilleur. Sa fête de tous les ans, les vendanges de {\placeName Saint-Julien}, suffisait pour réparer ses forces. \emph{« J’ai dans l’esprit des choses que je veux absolument finir »}, écrivait-il en 1876. Une maladie grave, qu’il avait traversée victorieusement, semblait n’avoir fait que redoubler l’activité de son esprit. Entouré de sa famille scientifique, il s’avançait vers la vieillesse sans paraître en ressentir les atteintes. Les projets qu’il roulait dans son esprit étaient plus grands que ceux qu’il avait jusque-là réalisés.\par
Dans sa marche hardie vers les derniers secrets de la nature animée, il arrivait, en effet, aux confins de la vie, aux sources obscures de l’organisme. Peu à peu la différence entre la physiologie animale et la physiologie végétale s’évanouissait à ses yeux. Le germe de la vie, des deux côtés, lui paraissait le même. La plante, comme l’animal, est susceptible d’être anesthésiée. Même certains ferments peuvent être atteints par les agents insensibilisateurs, et, pour une moitié au moins de leur être, ils semblent s’endormir. {\persName Claude Bernard} touchait ainsi au problème par excellence, au problème de la fermentation, impliquant la question même des origines de la cellule. Il y consacra toutes ses réflexions de l’été de 1877 ; il annonçait à ses disciples qu’il croyait avoir trouvé la voie pour arriver à ce sanctuaire impénétrable. Ô fragilité de la vie humaine ! Ô jeu cruel d’une nature marâtre qui se plaît à briser stupidement une tête formée par quarante ans de méditations et où va éclore la plus belle combinaison du génie ! La terrible maladie à laquelle il avait échappé dix ans auparavant n’avait pardonné qu’en apparence. Elle revint plus implacable que jamais. Il mourut sans avoir pu réaliser son rêve ; il mourut triste, pensant à l’idée destinée à périr avec lui, et disant : \emph{« C’eût été pourtant bien beau de finir par là ! »}\par
Il a fait assez pour sa gloire, et sa trace sera éternelle. Sa religion était la vérité ; il n’eut jamais ni mécompte ni faiblesse ; car il ne douta pas un moment de la science ; or la science donne le bonheur, quand on se contente d’elle et qu’on ne lui demande que ce qu’elle peut donner. Si elle ne répond pas à toutes les questions que lui adressent les avides ou les empressés, au moins ce qu’elle apprend est sûr. Pour être acquis par des oscillations successives, les résultats de la science moderne n’en sont pas moins précieux. Ces délicates approximations, cet affinage successif qui nous amène à des manières de voir de plus en plus rapprochées de la vérité, sont la condition même de l’esprit humain. La science donnait ainsi à notre confrère tout le calme que procure la certitude d’avoir raison. Il ne portait envie à personne ; il croyait avoir la meilleure part.\par
{\persName Claude Bernard} n’ignorait pas que les problèmes qu’il soulevait touchaient aux plus graves questions de l’ordre philosophique. Il n’en fut jamais ému. Il ne croyait pas qu’il fût permis au savant de s’occuper des conséquences qui peuvent sortir de ses recherches. Il était, à cet égard, d’une impassibilité absolue. Peu lui importait qu’on l’appelât de tel ou tel nom de secte. Il n’était d’aucune secte. Il cherchait la vérité, et voilà tout. Les héros de l’esprit humain sont ceux qui savent ainsi ignorer pour que l’avenir sache. Tous n’ont pas ce courage. Il est difficile de s’abstenir dans des questions où c’est éminemment de nous qu’il s’agit. Ignorer si l’univers a un but idéal, ou si, fils du hasard, il va au hasard, sans qu’une conscience aimante le suive dans son évolution ; ignorer si, à l’origine, quelque chose de divin fut mis en lui, et si, à la fin, un sort plus consolant lui est réservé ; ignorer si nos instincts profonds de justice sont un leurre ou la dictée impérieuse d’une vérité qui s’impose, on est excusable de ne pas s’y résigner. Il est des sujets où l’on aime mieux déraisonner que de se taire. Vérité ou chimère, le rêve de l’infini nous attirera toujours, et, comme ce héros d’un conte celtique qui, ayant vu en songe une beauté ravissante, court le monde toute sa vie pour la trouver, l’homme qui un moment s’est assis pour réfléchir sur sa destinée porte au cœur une flèche qu’il ne s’arrache plus. En pareille matière, la puérilité même des efforts est touchante. Il ne faut pas demander de logique aux solutions que l’homme imagine pour se rendre quelque raison du sort étrange qui lui est échu. Invinciblement porté à croire à la justice et jeté dans un monde qui est et sera toujours l’injustice même, ayant besoin de l’éternité pour ses revendications et brusquement arrêté par le fossé de la mort, que voulez-vous qu’il fasse ? Il se révolte contre le cercueil, il rend la chair à l’os décharné, la vie au cerveau plein de pourriture, la lumière à l’œil éteint ; il imagine des sophismes dont il rirait chez un enfant, pour ne pas avouer que la nature a pu pousser l’ironie jusqu’à lui imposer le fardeau du devoir sans compensation.\par
Si parfois, à ces confins extrêmes où toutes nos pensées tournent à l’éblouissement, la philosophie de notre illustre confrère parut un peu contradictoire, ce n’est pas moi qui l’en blâmerai. J’estime qu’il est des sujets sur lesquels il est bon de se contredire ; car aucune vue partielle n’en saurait épuiser les intimes replis. Les vérités de la conscience sont des phares à feux changeants. À certaines heures, ces vérités paraissent évidentes ; puis, on s’étonne qu’on ait pu y croire. Ce sont choses que l’on aperçoit furtivement, et qu’on ne peut plus revoir telles qu’on les a entrevues. Vingt fois l’humanité les a niées et affirmées ; vingt fois l’humanité les niera et les affirmera encore. La vraie religion de l’âme est-elle ébranlée par ces alternatives ? Non, Messieurs. Elle réside dans un empyrée où le mouvement de tous les autres cercles ne sauraient l’atteindre. Le monde roulera durant l’éternité sans que la sphère du réel et la sphère de l’idéal se touchent. La plus grande faute que puissent commettre la philosophie et la religion est de faire dépendre leurs vérités de telle ou telle théorie scientifique et historique ; car les théories passent, et les vérités nécessaires doivent rester. L’objet de la religion n’est pas de nous donner des leçons de physiologie, de géologie, de chronologie ; qu’elle n’affirme rien en ces matières, et elle ne sera pas blessée. Qu’elle n’attache pas son sort à ce qui peut périr. La réalité dépasse toujours les idées qu’on s’en fait ; toutes nos imaginations sont basses auprès de ce qui est. De même que la science, en détruisant un monde matériel enfantin, nous a rendu un monde mille fois plus beau, de même la disparition de quelques rêves ne fera que donner au monde idéal plus de sublimité. Pour moi, j’ai une confiance invincible en la bonté de la pensée qui a fait l’univers. \emph{« Enfants ! disons-nous des hommes antiques, enfants ! qui n’avaient point d’yeux pour voir ce que nous voyons ! » — « Enfants ! dira de nous l’avenir, qui pleuraient sur la ruine d’un millenium chimérique et ne voyaient pas le soleil de la vérité nouvelle blanchir déjà derrière eux les sommets de l’horizon ! »}\par
Vous résolvez ces graves problèmes, Messieurs, par la tolérance, par votre bonne confraternité, en vous aimant, en vous estimant. Vous ne vous effrayez pas de luttes qui sont aussi vieilles que le monde, de contradictions qui dureront autant que l’esprit humain, d’erreurs même qui sont la condition de la vérité. Votre philosophie est indulgente et optimiste, parce qu’elle est fondée sur une connaissance étendue de l’esprit humain. Ce désintéressement qu’un observateur superficiel se croit en droit de nier dans les choses humaines, vous savez le voir, vous à qui l’étude de la société apprend la justice et la modération. Ne trouvez-vous pas, Messieurs, que les hommes sont trop sévères les uns pour les autres ? On s’anathématise, on se traite de haut en bas, quand souvent, de part et d’autre, c’est l’honnêteté qui insulte l’honnêteté, la vérité qui injurie la vérité. Oh ! le bon être que l’homme ! Comme il a travaillé ! Quelle somme de dévouement il a dépensée pour le vrai, pour le bien ! Et quand on pense que, ces sacrifices à un Dieu inconnu, il les a faits, pauvre, souffrant, jeté sur la terre comme un orphelin, à peine sûr du lendemain, ah ! je ne peux souffrir qu’on l’insulte, cet être de douleur, qui, entre le gémissement de la naissance et celui de l’agonie, trouve moyen de créer l’art, la science, la vertu. Qu’importent les malentendus aux yeux de la vérité éternelle ? Le culte le plus pur de la {\persName Divinité} se cache parfois derrière d’apparentes négations ; le plus parfait idéaliste est souvent celui qui croit devoir à une certaine franchise de se dire matérialiste. Combien de saints sous l’apparence de l’irréligion ! Combien, parmi ceux qui nient l’immortalité, mériteraient une belle déception ! La raison triomphe de la mort, et travailler pour elle, c’est travailler pour l’éternité. Toute perdue qu’elle est dans le chœur des millions d’êtres qui chantent l’hymne éternel, chaque voix a compté et comptera toujours. La joie, la gaieté que donnent ces pensées est un signe qu’elles ne sont pas vaines. Elles ont l’éclat ; elles rajeunissent ; elles prêtent au talent, le créent et l’appellent. Vous qui jugez des choses par l’étincelle qui en jaillit, par le talent qu’elles provoquent, vous avez après tout un bon moyen de discernement. Le talent qu’inspire une doctrine est, à beaucoup d’égards, la mesure de sa vérité. Ce n’est pas sans raison qu’on ne peut être grand poète qu’avec l’idéalisme, grand artiste qu’avec la foi et l’amour, bon écrivain qu’avec la logique, éloquent orateur qu’avec la passion du bien et de la liberté.
\section[{Lettre à un ami d’Allemagne à propos du discours précédent}]{Lettre à un ami d’{\placeName Allemagne} à propos du discours précédent}\renewcommand{\leftmark}{Lettre à un ami d’{\placeName Allemagne} à propos du discours précédent}


\dateline{Journal des débats, 16 avril 1879}

\salute{Mon cher ami,}
\noindent Vous m’apprenez qu’un passage de mon discours de réception a été accueilli parmi vous comme la voix d’un ennemi. Relisez ce que j’ai dit, et vous verrez combien ce jugement est superficiel. J’ai défendu notre vieil esprit français contre d’injustes reproches, qui viennent presque aussi souvent de chez nous que de chez vous. J’ai soutenu contre des novateurs, qui sont loin d’être tous {\orgName Allemands}, que notre tradition intellectuelle est grande et bonne, qu’il faut l’appliquer à des ordres de connaissances sans cesse élargis, mais non pas la changer. J’ai exprimé des doutes sur la possibilité pour une dynastie de jouer dans le monde un rôle universel, sans bienveillance, sans générosité, sans éclat. J’ai pu aller à l’encontre de certaines opinions des militaires et des hommes d’État de Berlin ; je n’ai pas dit un mot contre l’{\orgName Allemagne} et son génie. Plus que jamais je pense que, si nous avons besoin de vous, vous aussi, à quelques égards, avez besoin de nous. La collaboration de la {\orgName France} et de l’{\orgName Allemagne}, ma plus vieille illusion de jeunesse, redevient la conviction de mon âge mûr, et mon espérance est que, si nous arrivons à la vieillesse, si nous survivons à cette génération d’hommes de fer, dédaigneux de tout ce qui n’est pas la force, auxquels vous avez confié vos destinées, nous verrons ce que nous avons rêvé autrefois, la réconciliation des deux moitiés de l’esprit humain. Oui, sans nous, vous serez solitaires et vous aurez les défauts de l’homme solitaire ; le monde n’appréciera parfaitement de vous que ce que nous lui aurons fait comprendre. Je me hâte d’ajouter que sans vous notre œuvre serait maigre, insuffisante. Voilà ce que j’ai toujours dit. Je n’ai nullement changé ; ce sont les événements qui ont si complètement interverti les rôles, que nous avons peine à nous reconnaître dans nos affections et dans nos souvenirs.\par
Personne n’a aimé ni admiré plus que moi votre grande {\orgName Allemagne}, {\orgName Allemagne} d’il y a cinquante et soixante ans, personnifiée dans le génie de {\persName Gœthe}, représentée aux yeux du monde par cette merveilleuse réunion de poètes, de philosophes, d’historiens, de critiques, de penseurs, qui a vraiment ajouté un domaine nouveau aux richesses de l’esprit humain. Tous tant que nous sommes, nous lui devons beaucoup, à cette {\orgName Allemagne} large, intelligente et profonde, qui nous enseignait l’idéalisme par {\persName Fichte}, la foi dans l’humanité par Herder, la poésie du sens moral par {\persName Schiller}, le devoir abstrait par {\persName Kant}. Loin que ces acquisitions nouvelles nous parussent la contradiction de l’ancienne discipline française, elles nous en semblaient la continuation. Nous prenions au sérieux vos grands esprits quand ils reconnaissaient ce qu’ils devaient à notre dix-huitième siècle ; nous admettions avec {\persName Gœthe} que la {\orgName France}, que {\orgName Paris} étaient des organes essentiels du génie moderne et de la conscience européenne. Nous travaillions de toutes nos forces à bannir de la science et de la philosophie ces mesquines idées de rivalité nationale qui sont le pire obstacle aux progrès de l’esprit humain.\par
Depuis 1848, époque où les questions commencèrent à se poser avec netteté, nous avons toujours admis que l’unité politique de l’{\orgName Allemagne} se ferait, que c’était là une révolution juste et nécessaire. Nous concevions l’{\orgName Allemagne} devenue nation comme un élément capital de l’harmonie du monde. Voyez notre naïveté ! Cette {\orgName nation allemande} que nous désirions voir entrer comme une individualité nouvelle dans le concert des peuples, nous l’imaginions sur le modèle de ce que nous avions lu, d’après les principes tracés par {\persName Fichte} ou {\persName Kant}. Nous formions les plus belles espérances pour le jour où prendrait place dans la grande confédération européenne un peuple philosophe, rationnel, ami de toutes les libertés, ennemi des vieilles superstitions, ayant pour symbole la justice et l’idéal. Que de rêves nous faisions ! Un protestantisme rationaliste s’épurant toujours entre vos mains et s’absorbant en la philosophie, — un haut sentiment d’humanité s’introduisant avec vous dans la conduite du monde, — un élément de raison plus mûre se mêlant au mouvement général de l’{\orgName Europe} et préparant des bandages à plusieurs des plaies que notre grande mais terrible révolution avait laissées saignantes ! Vos admirables aptitudes scientifiques sortaient d’une obscurité imméritée, devenaient un rouage essentiel de la civilisation, et ainsi, grâce à vous et un peu grâce à nous, un pas considérable s’accomplissait dans l’histoire du progrès.\par
Les choses humaines ne se passent jamais comme le veulent les sages. Aussi les esprits éclairés parmi nous ne furent-ils pas trop surpris de voir proclamer à {\placeName Versailles}, sur les ruines de la {\orgName France} vaincue, cette unité allemande qu’ils s’étaient représentée comme une œuvre sympathique à la {\orgName France}. Grande fut leur douleur en voyant l’apparition nationale qu’ils avaient appelée de leurs vœux indissolublement liée aux désastres de leur pays. Ils se consolaient au moins par la pensée que l’{\orgName Allemagne}, devenue toute-puissante en {\placeName Europe}, allait planter haut et ferme le drapeau d’une civilisation qu’elle nous avait appris à concevoir d’une façon si élevée.\par
La grandeur oblige, en effet. Une nation a d’ordinaire le droit de se renfermer dans le soin de ses intérêts particuliers et de récuser la gloire périlleuse des rôles humanitaires. Mais la modestie n’est pas permise à tous. Vos publicistes, interprètes d’un instinct profond, ont pu être moins discrets à cet égard que vos hommes d’État et proclamer tout haut que l’ère de l’{\orgName Allemagne} commençait dans l’histoire. La fatalité vous traînait. Il n’est pas permis, quand on est tout-puissant, de ne rien faire. La victoire défère au victorieux, qu’il le veuille ou non, l’hégémonie du monde.\par
Tour à tour la fortune élève sur le pavois une nation, une dynastie. Jusqu’à ce que l’humanité soit devenue bien différente de ce qu’elle est, toutes les fois qu’elle verra passer un char de triomphe, elle saluera, et, les yeux fixés sur le héros du jour, elle lui dira : « Parle, tu es notre chef, sois notre prophète. » La solution des grandes questions pendantes a un moment donné (et {\persName Dieu} sait si le moment présent se voit obsédé de problèmes impérieux !) est dévolue à celui que les destins désignent. {\persName Alexandre}, {\persName Auguste}, {\persName Charles-Quint}, {\persName Napoléon} n’avaient pas le droit de se désintéresser des choses humaines ; sur aucune question, ils ne pouvaient dire : « Cela ne me regarde pas ! » Chaque âge a son président responsable, chargé de frapper, d’étonner, d’éblouir, de consoler l’humanité. Autant le rôle du vaincu, obligé de s’abstenir en tout, est facile, autant la victoire impose de devoirs. Il ne sert de rien de prétendre qu’on a le droit d’abdiquer une mission qu’on n’a pas voulue. Le devoir devant lequel on recule vous prend à la gorge, vous tue ; la grandeur est un sort implacable auquel on ne peut se soustraire. Celui qui manque à sa vocation providentielle est puni par ce qu’il n’a pas fait, par les exigences qu’il n’a pas contentées, par les espérances qu’il n’a pas remplies, et surtout par épuisement qui résulte d’une force non employée, d’une tension sans résultat.\par
Faire de grandes choses dans le sens marqué par le génie de l’{\orgName Allemagne}, tel était donc le devoir de la {\orgName Prusse} quand le sort des armes eut mis les destinées de l’{\orgName Allemagne} entre ses mains. Elle pouvait tout pour le bien ; car la condition pour réaliser le bien, c’est d’être fort. Qu’y avait-il à faire ? Qu’a-t-elle fait ? Huit ans, plus de la moitié de ce que {\persName Tacite} appelle \emph{grande mortalis œvi spatium}, se sont écoulés depuis qu’elle jouit en {\placeName Europe} d’une supériorité incontestée. Par quels progrès en {\placeName Allemagne} et dans le monde cette période historique aura-t-elle été marquée ?\par
Et d’abord, après la victoire, la nation victorieuse a bien le droit de trouver chez elle les récompenses de ses héroïques efforts, le bien-être, la richesse, le contentement, l’estime réciproque des classes, la joie d’une patrie glorieuse et pacifiée. En politique, elle a droit surtout au premier des biens, à la première des récompenses, je veux dire à ces libertés fondamentales de la parole, de la pensée, de la presse, de la tribune, toutes choses dangereuses dans un État faible ou vaincu, possibles seulement dans un État fort. Ces grandes questions sociales qui agitent notre siècle ne peuvent être résolues que par un victorieux, se servant du prestige de la gloire pour imposer des concessions, des sacrifices, l’amnistie, à tous les partis. Donner la paix, autant que la paix est de ce monde, et la liberté aussi large qu’il est possible, à cette Europe continentale qui n’a pas encore trouvé son équilibre, fonder définitivement le gouvernement représentatif, aborder franchement les problèmes sociaux, élever les classes abaissées sans leur inspirer la jalousie des supériorités nécessaires, diminuer la somme des souffrances, supprimer la misère imméritée, résoudre la délicate question de la situation économique de la femme, montrer par un grand exemple la possibilité défaire face en même temps aux nécessités politiques opposées que l’{\orgName Angleterre} a conciliées, parce que le problème se posait pour elle d’une manière relativement facile : voilà ce qui eût justifié la victoire, voilà ce qui l’eût maintenue. La victoire, en effet, a toujours besoin d’être légitimée par des bienfaits. La force qu’on a déchaînée devient impérieuse à son tour. Dès qu’il a reçu la première salutation impériale, le César appartient à la fatalité jusqu’à sa mort. De ce programme que la force des choses semblait vous imposer, qu’avez-vous réalisé ? Votre peuple est-il devenu plus heureux, plus moral, plus satisfait de son sort ? Il est clair que non ; des symptômes comme on n’en a jamais vu après la victoire se sont manifestés parmi vous. La gloire est le foin avec lequel on nourrit la bête humaine, votre peuple en a été saturé, et il regimbe !… {\persName Napoléon I\textsuperscript{er}}, en 1805, 1806, avait imposé silence par l’admiration à toute voix opposante ; une centaine de personnes tout au plus murmuraient ; l’idée d’un attentat contre sa personne eût paru un non-sens. Comment se fait-il qu’au lendemain de triomphes comme on n’en avait pas vu depuis soixante ans, votre gouvernement se soit trouvé en présence d’un mécontentement profond ? Pourquoi est-il toujours préoccupé de mesures restrictives de la liberté ? D’ordinaire, on n’a pas à réprimer après la victoire ; la répression est le propre des faibles. Ce qui se passe chez vous, n’importe comment on l’explique, renferme un blâme contre vos hommes d’État. Si votre peuple est aussi mauvais qu’ils le disent, c’est leur condamnation. Âpres et durs, comprenant l’État comme une chaîne et non comme quelque chose de bienveillant, ils croient connaître la nature allemande et ne connaissent pas la nature humaine. Ils ont trop compté sur la vertu germanique, ils en verront le bout. On a fait de vous une nation organisée pour la guerre ; comme ces chevaliers du XVI\textsuperscript{e} siècle, bardés de fer, vous êtes écrasés par votre armement. S’imaginer qu’en continuant de subir un pareil fardeau sans en retirer aucun avantage, votre peuple aura la souplesse nécessaire pour les arts de la paix, c’est trop espérer. Ces sacrifices militaires vous mettent dans la nécessité ou de faire la guerre indéfiniment, — et vous avez trop de bon sens pour ne pas voir que ces parties à la {\persName Napoléon I\textsuperscript{er}} mènent aux abîmes, — ou d’avoir une place désavantageuse dans la lutte pacifique de la civilisation. Les agitations socialistes sont, comme la fièvre, à la fois une maladie et un symptôme ; on doit en tenir compte ; il ne suffit pas de les étouffer, il faut en voir la cause et à quelques égards y donner satisfaction. Les erreurs populaires s’affaiblissent par la publicité ; on les fortifie en essayant de ramener le peuple à des croyances devenues sans efficacité. Vos maîtres d’école auront beau revenir au pur catéchisme, cela n’y fera rien. Les lois répressives n’y peuvent pas davantage ; on ne tue par des mouches à coups de canon.\par
Et dans l’ordre politique, dans la réalisation de cet idéal du gouvernement constitutionnel qui nous est si cher à tous et où l’{\orgName Europe continentale} n’a pas encore réussi, quel progrès avez-vous accompli ? En quoi votre vie parlementaire a-t-elle été plus brillante, plus libre, plus féconde que celle des autres peuples ? Je n’arrive pas à le voir, et ici encore, au lieu de cette largeur libérale qui est le propre des forts, je trouve vos hommes d’État surtout préoccupés de restrictions, de répressions, de règlements coercitifs. Non, je le répète, ce n’est pas par ces moyens-là que vous séduirez le monde. La répression est chose toute négative. Et si, pendant que vos hommes d’État sont plongés dans cette ingrate besogne, le paysan français, avec son gros bon sens, sa politique peu raffinée, son travail et ses économies, réussissait à fonder une République régulière et durable ! Ce serait plaisant. L’entreprise est trop difficile et trop périlleuse pour qu’il soit permis d’en escompter le succès ; mais ce qui est incroyable est souvent ce qui arrive. Les soldats écervelés du général Custine, les grenadiers héroïques et burlesques qui semèrent à tous les vents les idées de la Révolution, ont réussi à leur manière.\par
La gloire nationale est une grande excitation pour le génie national. Vous avez eu quatre-vingts ans d’un admirable mouvement littéraire, durant lesquels on a vu fleurir chez vous des écrivains comparables aux plus grands des autres nations. Comment se fait-il que cette veine soit comme tarie ? Après notre âge littéraire classique du XVII\textsuperscript{e} siècle, nous avons eu le XVIII\textsuperscript{e} siècle, {\persName Montesquieu}, {\persName Voltaire}, {\persName Rousseau}, {\persName d’Alembert}, {\persName Diderot}, {\persName Turgot}, {\persName Condorcet}. Où est votre continuation de {\persName Gœthe}, de {\persName Schiller}, de {\persName Heine} ? Le talent ne vous manque certes pas ; mais il y a, selon moi, deux causes qui nuisent à votre production littéraire : d’abord, vos charges militaires exagérées, et en second lieu, votre état social. Supposez {\persName Gœthe} obligé de faire son apprentissage militaire, exposé aux gros mots de vos sergents instructeurs, croyez-vous qu’il ne perdrait pas à cet exercice sa fleur d’élégance et de liberté ? L’homme qui a obéi est à jamais perdu pour certaines délicatesses de la vie ; il est diminué intellectuellement. Votre service militaire est une école de respect exagéré. Si {\persName Molière} et {\persName Voltaire} eussent traversé cette éducation-là, ils y auraient perdu leur fin sourire, leur malignité parfois irrévérencieuse. L’état de conscrit est funeste au génie. Vous me direz que ce régime, nous l’avons adopté de notre côté. Ce n’est peut-être pas ce que nous avons fait de mieux ; en tous cas, on ne voit guère encore venir le jour où nous serons malades par exagération du respect.\par
Votre état social me paraît aussi très peu favorable à la grande littérature. La littérature suppose une société gaie, brillante, facile, disposée à rire d’elle-même, où l’inégalité peut être aussi forte que l’on voudra, mais où les classes se mêlent, où tous vivent de la même vie. On me dit que vous avez fait depuis dix ans de grands progrès vers cette unité de la vie sociale ; cependant je n’en vois pas encore le principal fruit, qui est une littérature commune, exprimant avec talent ou avec génie toutes les faces de l’esprit national, une littérature aimée, admirée, acceptée, discutée par tous. Je n’ignore pas les noms très honorables que vous allez me citer ; je ne peux trouver néanmoins que votre nouvel empire ait réalisé ce qu’on devait attendre d’un gouvernement concentrant en lui toutes les forces du génie allemand. C’était à vous de faire sonner bien haut le clairon de la pensée ; ces accents nouveaux qui devaient faire battre tous les cœurs, nous les attendons, et nous ne voyons pas bien comment, de l’état moral que certains faits récents nous ont révélé, sortirait un mouvement de libre expansion et de chaude générosité.\par
Vous étiez forts, et vous n’avez pas fait la liberté ! Votre campagne contre l’ultramontanisme, légitime quand elle s’est bornée à réprimer l’intolérance catholique, n’a pas fait avancer d’un pas la grande question de la séparation de l’{\orgName Église} et de l’{\orgName État}. Vos ministres sont toujours restés dans le vieux système où l’{\orgName État} confère à l’{\orgName Église} des privilèges et a pour elle des exigences, sans voir que ces exigences, qui ont une apparence tyrannique, sont loin d’égaler les privilèges qu’on lui accorde d’une autre main. Certes, vous n’irez pas à {\placeName Canossa}. {\persName Léon XIII} n’est pas {\persName Grégoire VII} ; c’est lui qui viendra où vous voudrez. Mais ici encore nous attendions du grand et du neuf, et nous ne le voyons pas venir.\par
Je ferais sourire vos hommes d’État si je disais que votre empire, dans ces premières années qui sont toujours les plus fécondes, n’a pas non plus rempli ses devoirs envers l’humanité, et que l’avenir lui demandera compte de beaucoup de questions auxquelles il a tourné le dos comme à des rêves d’idéologues. Nos habitudes d’esprit et notre histoire nous donnent peut-être des idées fausses en ce qui concerne l’idéal d’une grande hégémonie nationale et dynastique. Nous pensons toujours à {\persName Auguste}, à {\persName Louis XIV} ; nous ne comprenons pas qu’on règne sur le monde sans grandeur, sans éclat, sans rechercher l’amour du monde et forcer sa reconnaissance. Une nation ou une dynastie dirigeante nous apparaît comme quelque chose de noble, de sympathique, comme une force chargée de patronner tout ce qui est beau, de favoriser le progrès de la civilisation sous toutes ses formes. Éclat, générosité, bienveillance nous semblent des conditions nécessaires de ces grands règnes momentanés qui sont tour à tour dévolus à chaque nation. {\persName Louis XIV} n’entendait pas parler d’un homme de mérite, de quelque pays qu’il fût, sans se demander : \emph{« Ne pourrais-je pas lui faire une pension ? »} Il se croyait le dieu bienfaisant du monde ; l’{\orgName Europe} a vécu pendant cent ans de son soleil en cuivre doré. Vanité des vanités ! L’humanité est quelque chose d’assez frivole ; il faut le savoir si l’on aspire à la gagner ou à la gouverner.\par
Pour la gagner, il faut lui plaire ; pour lui plaire, il faut être aimable. Or vos hommes d’État prussiens ont tous les dons, excepté celui-là. Force de volonté, application, génie contenu et obstiné, ils ne se sont montrés inférieurs pour les qualités solides à aucun des grands génies politiques du passé. Mais ils se sont trompés en se figurant qu’avec cela on peut se dispenser de plaire au monde, de le gagner par des bienfaits. Erreur ! On ne s’impose à l’humanité que par l’amour de l’humanité, par un sentiment large, libéral, sympathique, dont vos nouveaux maîtres se raillent hautement, qu’ils traitent de chimère sentimentale et prétentieuse. On ne discute pas contre des poses, contre des modes passagères ; mais il est bien permis de dire qu’une ostentation d’égoïsme et de froid calcul n’a jamais été le ton des grands hommes qui méritent de figurer éternellement au {\orgName Panthéon} de l’humanité.\par
Traitez-moi d’arriéré, mais je ne reconnaîtrai jamais comme ayant réalisé l’ancien idéal allemand ces hommes durs, étroits, détracteurs de la gloire, affectant un terre-à-terre vulgaire et positif, prétextant un dédain de la postérité qu’au fond ils n’ont pas. Dans le passage de mon discours de réception qui vous a blessés, je n’ai pas voulu dire autre chose. Le génie de l’{\orgName Allemagne} est grand et puissant ; il reste un des organes principaux de l’esprit humain ; mais vous l’avez mis dans un étau où il souffre. Vous êtes égarés par une école sèche et froide, qui écrase plus qu’elle ne développe. Nous sommes sûrs que vous vous retrouverez vous-mêmes, et qu’un jour nous serons de nouveau collaborateurs dans la recherche de tout ce qui peut donner de la grâce, de la gaieté, du bonheur à la vie.
\section[{Réponse au discours de M. Louis Pasteur}]{Réponse au discours de {\persName M. Louis Pasteur}}\renewcommand{\leftmark}{Réponse au discours de {\persName M. Louis Pasteur}}


\dateline{Jeudi 27 avril 1882, {\placeName Palais de l’Institut}.}

\salute{Monsieur,}
\noindent Nous sommes bien incompétents pour louer ce qui fait votre gloire véritable, ces admirables expériences par lesquelles vous atteignez jusqu’aux confins de la vie, cette ingénieuse façon d’interroger la nature qui tant de fois vous a valu de sa part les plus claires réponses, ces précieuses découvertes qui se transforment chaque jour en conquêtes de premier ordre pour l’humanité. Vous répudieriez nos éloges, habitué que vous êtes à n’estimer que les jugements de vos pairs, et, dans les débats scientifiques que soulèvent tant d’idées neuves, vous ne voudriez pas voir des appréciations littéraires venir se mêler au suffrage des savants que rapproche de vous la confraternité de la gloire et du travail. Entre vous et vos savants émules nous n’avons point à intervenir. Mais, en dehors du fond de la doctrine, qui n’est point de notre ressort, il est une maîtrise, Monsieur, où notre pratique de l’esprit humain nous donne le droit d’émettre un avis. Il y a quelque chose que nous savons reconnaître dans les applications les plus diverses ; quelque chose qui appartint ou même degré à {\persName Galilée}, à {\persName Pascal}, à {\persName Michel-Ange}, à {\persName Molière} ; quelque chose qui fait la sublimité du poète, la profondeur du philosophe, la fascination de l’orateur, la divination du savant. Cette base commune de toutes les œuvres belles et vraies, cette flamme divine, ce souffle indéfinissable qui inspire la science, la littérature et l’art, nous l’avons trouvé en vous, Monsieur ; c’est le génie. Nul n’a parcouru d’une marche aussi sûre les cercles de la nature élémentaire ; votre vie scientifique est comme une traînée lumineuse dans la grande nuit de l’infiniment petit, dans ces derniers abîmes de l’être où naît la vie.\par
Vous avez commencé, Monsieur, par le vrai commencement de la nature. Avec {\persName Haüy} et {\persName Malus}, vous demandiez d’abord au cristal le secret de ses caprices apparents. Vous étiez encore à l’{\orgName École normale}. Une note de {\persName Mitscherlich} vous troubla dans votre foi chimique. Deux substances identiques par la nature, le nombre, l’arrangement et la distance des atomes agissaient d’une manière essentiellement différente sur la lumière. Vous reprîtes avec passion l’étude de la forme cristalline des deux sels de {\persName M. Mitscherlich}, et vous arrivâtes à votre belle théorie de la dissymétrie moléculaire. Oui, deux groupes atomiques qui se montrent identiques au travers de toutes les épreuves de la chimie peuvent être, l’un à l’égard de l’autre, dans la même relation qu’un objet à l’égard de son image vue dans un miroir. Ils ont une droite et une gauche ; on peut les opposer, non les superposer, comme les deux mains. L’illustre {\persName M. Biot}, chargé de rendre compte de ces faits nouveaux à l’{\orgName Académie des sciences}, eut d’abord quelques doutes. Quand vous allâtes le voir au {\placeName Collège de France}, il s’était déjà procuré lui-même les matières de l’expérience. Il vous les fit préparer sous ses yeux, sur le fourneau de sa cuisine. Vous placiez à sa droite les cristaux qui devaient dévier la lumière à droite, à sa gauche, les cristaux qui devaient dévier la lumière à gauche. Il fit lui-même l’épreuve de la polarisation ; mais il n’alla pas jusqu’au bout ; quelques indices lui suffirent. \emph{« Mon cher enfant, vous dit-il, en serrant votre bras, j’ai tant aimé les sciences dans ma vie que cela me fait battre le cœur. »}\par
Toutes vos découvertes ultérieures sont sorties de celle-là par une sorte de développement naturel. Bientôt, en effet, vous arriverez à voir que tous les produits artificiels des laboratoires et toutes les espèces minérales sont à image superposable, tandis que les produits essentiels de la vie sont dissymétriques. La vie vous conduit à la fermentation ; l’élément dissymétrique fait fermenter ; l’élément symétrique ne fait pas fermenter. La fermentation est toujours d’origine vitale ; elle vient d’êtres microscopiques qui trouvent dans la matière organique leur nourriture, non leur raison de naître ; le groupe droit et le groupe gauche ne satisfont pas également à la nutrition des microbes. Vos études sur les corpuscules organisés qui existent dans l’atmosphère servent de point de départ à tout un ordre de recherches, où vos disciples sont des maîtres qui s’appellent {\persName Lister}, {\persName Tyndall}.\par
La fermentation vous mène aux maladies, qui sont en quelque sorte la fermentation de l’être vivant ; de la cristallographie vous êtes conduit à la médecine ; vous arrivez à voir que les maladies transmissibles tiennent le plus souvent à des développements irréguliers d’êtres étrangers à l’organisme, qui le troublent ou le détruisent. De là vos savantes recherches sur les maladies du vin, de la bière, des vers à soie, puis sur ces terribles accidents de la machine humaine, le charbon, la septicémie, la rage, qui peuvent amener la mort à l’organisme par lui-même le plus sain et le plus robuste. La claire vue de la nature du mal vous indique le remède ; on guérit bientôt la maladie dont on connaît la cause. Votre théorie des germes de putréfaction ouvre une voie qui sera un jour et qui est déjà féconde pour le bien de notre pauvre espèce. La vaccination, qui n’avait été jusqu’ici qu’une application très particulière d’une théorie à peine ébauchée, devient entre vos mains un principe général, susceptible des usages les plus variés. C’est la rage, Monsieur, qui est en ce moment l’objet de vos études ; vous en cherchez l’organisme microscopique, vous le trouverez ; l’humanité vous devra la suppression d’un mal horrible, et aussi d’une triste anomalie, je veux parler de la défiance qui se mêle toujours un peu pour nous aux caresses de l’animal dans lequel la nature nous montre le mieux son sourire bienveillant.\par
Que vous êtes heureux, Monsieur, de toucher ainsi, par votre art, aux sources mêmes de la vie ! Admirables sciences que les vôtres ! Rien ne s’y perd. Vous aurez inséré une pierre de prix dans les assises de l’édifice éternel de la vérité. Parmi ceux qui s’adonnent aux autres parties du travail de l’esprit, qui peut avoir la même assurance ? {\persName M. de Maistre} peint quelque part la science moderne « sous l’habit étriqué du {\placeName Nord}…, les bras chargés de livres et d’instruments, pâle de veilles et de travaux, se traînant souillée d’encre et toute pantelante sur la route de la vérité, baissant toujours vers la terre son front sillonné d’algèbre ». Comme vous avez bien fait, Monsieur, de ne pas vous arrêter à ce souci de gentilhomme ! La nature est roturière ; elle veut qu’on travaille ; elle aime les mains calleuses et ne se révèle qu’aux fronts soucieux.\par
Votre vie austère, toute consacrée à la recherche désintéressée, est la meilleure réponse à ceux qui regardent notre siècle comme déshérité des grands dons de l’âme. Votre laborieuse assiduité n’a voulu connaître ni distractions ni repos. Recevez-en la récompense dans le respect qui vous entoure, dans cette sympathie dont les marques se produisent aujourd’hui si nombreuses autour de vous, et surtout dans la joie d’avoir bien accompli votre tâche, d’avoir pris place au premier rang dans la compagnie d’élite qui s’assure contre le néant par un moyen bien simple, en faisant des œuvres qui restent.\par
Vous ayez placé à sa juste hauteur l’homme illustre que vous venez remplacer parmi nous. Vous avez dit ses commencements, ses viriles origines, cette nature pleine d’énergie, tenant, par son père, aux races sérieuses et obstinées de l’{\placeName Ouest}, par sa mère, à l’ardente et forte complexion des populations protestantes des {\placeName Cévennes}. Canonnier de la première République, {\persName M. Littré} père garda, sous l’Empire et la royauté constitutionnelle, le culte de la Révolution. Les républicains étaient rares alors ; c’était, comme aux siècles de la primitive {\orgName Église}, le temps des convictions personnelles, passionnées. Les conversions en masse et sans grand discernement devaient venir plus tard. Les républicains que forma {\persName M. Littré} père avaient au moins quelque mérite à l’être ; car ils étaient deux (deux qui valaient, certes, à eux seuls tous ceux qu’on a plus tard vus éclore), son fils d’abord, puis l’intime ami de son fils, celui à qui je dois ces détails, notre respecté confrère {\persName M. Barthélemy Saint Hilaire}. En philosophie et en religion, {\persName M. Littré} père professait sans réserve les principes de l’{\orgName école française} du XVIII\textsuperscript{e} siècle. Devenu père de famille, il eut un scrupule touchant. Craignant que les railleries de {\persName Voltaire} n’eussent une part dans ses opinions religieuses, et se regardant comme responsable de sa théologie à l’égard de ses enfants, il reprit avec le plus grand sérieux la question des croyances. Ce nouvel examen confirma ses premiers jugements, et, dès lors, il enseigna en toute sécurité à ses fils ce qu’une double épreuve lui faisait regarder comme certain. Quelle honnêteté !\par
Cette impression de l’éducation première ne s’effaça jamais chez {\persName M. Littré}. Sa nature héroïque le porta toujours à ce qu’il y eut de plus âpre et de plus fort. Fils de la révolution française, il crut qu’en elle était contenue toute justice. D’autres, plus raffinés, distinguèrent, acceptèrent des moyens termes, des conciliations. Lui, entier dans sa foi, ne voulut aucune atténuation à ce qu’il tenait pour la vérité. La foi démocratique, comme tous les genres de foi, est exposée à des tentations ; il y a quelquefois du mérite à y persévérer. {\persName M. Littré} nous a raconté qu’un jour, sa mère, une petite vieille débile, avec de beaux yeux, cheminant à côté de lui dans une rue de {\placeName Paris}, fut brutalement poussée par un ouvrier qui ne voulait pas se déranger. Comme {\persName M. Littré} la relevait : \emph{« Mon fils, lui dit-elle, il faut bien aimer le peuple pour demeurer de son parti. »} La croyance de {\persName M. Littré} était de celles que rien n’ébranle. D’ordinaire les effervescences révolutionnaires viennent du tempérament ; la raison intervient pour les régler. Chez {\persName M. Littré}, le tempérament était tout à fait calme ; c’était l’esprit qui était révolutionnaire ; aussi ne recula-t-il jamais. On le trouve toujours au front de bataille des combattants. En juillet 1830, il était de la première ligne de ceux qui pénétrèrent sur la {\placeName place du Carrousel} par l’ouverture du {\placeName pavillon de Rohan}. {\persName Georges Farcy} fut percé d’une balle à côté de lui.\par
C’est la conviction qui crée la vertu. La sélection des nobles âmes se fait sans acception de croyances. Comme vous l’avez parfaitement dit, Monsieur, aucune foi n’a de privilège à cet égard ; on peut être un chrétien des premiers jours avec les idées en apparence les plus négatives ; on peut voir soudés dans le même homme un ascète et un jacobin. La {\placeName bibliothèque Sainte-Geneviève} possède un catalogue de ses incunables, écrit tout entier de la main de {\persName M. Daunou} durant les années les plus terribles de la Révolution. Chaque matin, avant d’aller présider la {\orgName Convention} ou le {\orgName conseil des Cinq-Cents}, il en rédigeait un certain nombre de pages, toujours le même, à des dates qui s’appelaient 13 vendémiaire, 18 fructidor. {\persName Littré} associait de même à la vie militante les habitudes d’un bénédictin. Révolutionnaire d’une espèce bien rare ! Le soir des jours d’émeute, comme le soir des jours où il avait combattu de sa plume au \emph{National} à côté de {\persName Carrel}, il se reposait dans sa mansarde en préparant une édition d’{\persName Hippocrate}, ou en traduisant les œuvres les plus importantes de la critique moderne, ou en rassemblant les matériaux de cet admirable \emph{Dictionnaire historique de la langue française} qui sera, sans doute, un jour surpassé, si nous finissons le nôtre… Grandes et fortes natures de l’âge héroïque de notre race ! Rien ne leur restait étranger. Ils avaient changé les bases de la vie ; mais leur confiance dans l’esprit humain était absolue. C’étaient des croisés, à leur manière ; ils héritaient, sans le savoir, de dix siècles de vertu ; ils dépensaient, en un jour, le capital accumulé par vingt générations de silencieuse obscurité.\par
Leur scepticisme n’était qu’une apparence ; ils étaient, en réalité, de fougueux croyants. Ils pratiquaient le désintéressement absolu ; ils aimaient la glorieuse pauvreté. À toutes les propositions de fonctions rémunérées qui lui furent faites dans l’esprit le plus libéral, {\persName Littré} répondit par un refus. Un jour qu’on le pressait : \emph{« Je ne peux rien accepter, dit-il ; en ce moment, ce sont mes idées qui triomphent. »} Sa vie fut longtemps celle d’un artisan modeste. Si plus tard le travail amena pour lui la fortune, ce fut à son insu, sans qu’il l’eût voulu et presque malgré lui. Il alla jusqu’à ces paradoxes qui caractérisent parfois les héroïsmes vertueux. Il eût tenu pour déplacé tout souci de plaire ; les séductions les plus légitimes du talent, il se les interdisait ; à dessein, il laissait son style un peu négligé. Rien chez lui de l’homme de lettres. Sa modestie certainement fut exagérée, puisqu’elle lui fit croire qu’il était disciple quand, en réalité, il était maître, et qu’on le vit se subordonner à des personnes auxquelles il était fort supérieur. Tel était son amour de la vérité que, seul peut-être en notre siècle, il put se rétracter sans s’amoindrir. La vérité le menait comme un enfant ; il se soumit à elle quand il pensa l’avoir trouvée ; il s’arrêta quand il craignit de n’être plus avec elle ; il recula quand il crut l’avoir dépassée.\par
Et voyez, Monsieur, combien notre sort est étrange et quelle ironie supérieure semble s’attacher à nos pauvres efforts ! Même dans l’ordre de la vérité, nos qualités nous servent souvent moins que nos défauts. Il ne faut pas être trop parfait. Moins sincère, {\persName Littré} eût peut-être évité quelques erreurs. Les défauts de sa philosophie furent ceux d’une âme trop timorée. Ses apparentes négations n’étaient que la réserve extrême d’un esprit qui redoute les affirmations hasardées. Il avait tant de peur d’aller au-delà de ce qu’il voyait clairement, qu’il restait souvent en deçà. Vertueuse abstention ; doute fécond, que {\persName Descartes} eût compris ; respect exagéré peut-être de la vérité ! Il craignait de sembler escompter ce qu’il désirait et de prendre trop vite pour une réalité ce qui vraiment n’eût été que juste. Hésitation qui implique un culte mille fois plus délicat de l’éternel idéal que les téméraires solutions qui satisfont tout d’abord les esprits superficiels ! La vérité est une grande coquette, Monsieur ! Elle ne veut pas être cherchée avec trop de passion. L’indifférence réussit souvent mieux avec elle. Quand on croit la tenir, elle vous échappe ; elle se livre quand on sait l’attendre. C’est aux heures où on croyait lui avoir dit adieu qu’elle se révèle ; elle vous tient, au contraire, rigueur quand on l’affirme, c’est-à-dire quand on l’aime trop.\par
Vous avez fait des réserves, Monsieur, sur les doctrines philosophiques auxquelles {\persName M. Littré} s’était attaché et auxquelles il déclarait devoir le bonheur de sa vie. C’était votre droit. Je n’userai pas du droit semblable que j’aurais. Le résumé ou, comme on disait autrefois, le « bouquet spirituel » de cette séance doit être que l’ardeur pour le bien ne tient à aucune opinion spéculative. Je vous ferai, d’ailleurs, ma confession ; en politique et en philosophie, quand je me trouve en présence d’idées arrêtées, je suis toujours de l’avis de mon interlocuteur. En ces délicates matières, chacun a raison par quelque côté. Il y a déférence et justice à ne chercher dans l’opinion qu’on vous propose que la part de vérité qu’elle contient. Il s’agit ici, en effet, de ces questions sur lesquelles la providence (j’entends par ces mots l’ensemble des conditions fondamentales de la marche de l’univers) a voulu qu’il planât un absolu mystère. En cet ordre d’idées, il faut se garder d’un parti pris ; il est bon de varier ses points de vue et d’écouter les bruits qui viennent de tous les côtés de l’horizon.\par
C’est ce que fit {\persName M. Littré} toute sa vie. Je regrette cependant, comme vous, que ce grand et fidèle ami de la vérité se soit renfermé dans une école portant un nom déterminé, et ait salué comme son maître un homme qui, bien que considérable à beaucoup d’égards, ne méritait pas un tel hommage. Si je m’abandonnais à mon goût personnel, je serais peut-être aussi peu favorable que vous à {\persName M. Auguste Comte}, qui me semble, le plus souvent, répéter en mauvais style ce qu’ont pensé et dit avant lui, en très bon style, {\persName Descartes}, {\persName d’Alembert}, {\persName Condorcet}, {\persName Laplace}. Mais je me défie de mon avis, car je suis un peu, à l’égard de ce penseur distingué, dans la situation d’un jaloux. {\persName M. Littré} avait pour moi une bonté dont je garde un profond souvenir ; je sentais cependant qu’il m’aurait aimé beaucoup plus si j’avais voulu être comtiste. J’ai fait ce que j’ai pu ; je n’ai pas réussi. Je sentais chez lui un reproche secret. Quand nous nous trouvions tous les deux seuls à nos séances de \emph{l’Histoire littéraire de la France} de l’{\orgName Académie des inscriptions et belles-lettres}, je me croyais en face d’un confesseur, mécontent de moi pour quelque motif secret qu’il ne me disait pas. Cela me troublait. Pas plus que vous, Monsieur, je ne suis donc en situation de rendre pleine justice à {\persName M. Comte}. Je ne puis cependant m’empêcher d’être ému quand je vois tant d’hommes de valeur, en {\placeName France}, en {\placeName Angleterre}, en {\placeName Amérique}, accepter ce nom comme un drapeau. Avec l’habitude que je peux avoir des choses de l’esprit humain, je suis amené à croire que {\persName M. Comte} sera une étiquette dans l’avenir, et qu’il occupera une place importante dans les futures histoires de la philosophie. Ce sera une erreur, j’en conviens ; mais l’avenir, commettra tant d’autres erreurs ! L’humanité veut des noms qui lui servent de types et de chefs de file ; elle ne met pas dans son choix beaucoup de discernement.\par
Le positivisme, dites-vous, dans ses applications à la politique, n’a pas vu ses prophéties réalisées. Cela est très vrai. La condition du prophète est devenue de nos jours singulièrement difficile. La politique et la philosophie n’ont plus grand-chose à faire ensemble. Connaissez-vous une école qui ait mieux deviné ces jeux de la force, de la passion et du hasard, qu’on a bien tort assurément de vouloir assujettir à des lois ? Pour moi, je ne vois pas une théorie politique au nom de laquelle on ait le droit de jeter la première pierre aux théories vaincues. Je ne vois qu’une différence, c’est que le principal représentant du positivisme a confessé son erreur, tandis que nous attendons encore l’aveu de ceux qui n’ont pas été plus infaillibles que lui.\par
À la philosophie de {\persName M. Littré} vous en préférez une autre, qui, vous le supposez, aurait ici « un dernier refuge ». Ah ! ne vous y fiez pas trop, Monsieur. La zone de notre protection littéraire est bien large ; elle s’étend depuis {\persName Bossuet} jusqu’à {\persName Voltaire}. Souvent, nous aimons à être l’asile des vaincus ; la cause qui aurait chez nous son dernier refuge pourrait donc être assez malade. Nous ne patronnons pas les doctrines ; nous discernons le talent. Voilà comment nous n’avons jamais de déconvenues ni de démentis. Tout passe, et nous ne passons pas ; car nous ne nous attachons qu’à deux choses qui, nous l’espérons, seront éternelles en {\placeName France} : l’esprit et le génie. Nous respectons toutes les formes dont on peut revêtir une croyance élevée. Vous vous servez de deux mots, par exemple, dont, pour ma part, je ne me sers jamais, spiritualisme et matérialisme. Le but du monde, c’est l’idée ; mais je ne connais pas un cas où l’idée se soit produite sans matière ; je ne connais pas d’esprit pur ni d’œuvre d’esprit pur. L’œuvre divine s’accomplit par la tendance intime au bien et au vrai qui est dans l’univers ; je ne sais pas bien si je suis spiritualiste ou matérialiste.\par
Il est prudent de n’associer le sort des croyances morales à aucun système. Le mot de l’énigme qui nous tourmente et nous charme ne nous sera jamais livré. Pour moi, quand on nie ces dogmes fondamentaux, j’ai envie d’y croire ; quand on les affirme autrement qu’en beaux vers, je suis pris d’un, doute invincible. J’ai peur qu’on n’en soit trop sûr, et, comme la mystique dont parle Joinville, je voudrais par moments brûler le paradis par amour de Dieu. C’est le doute, en pareil cas, qui fait le mérite. La grandeur des vérités de cet ordre est de se présenter à nous avec le double caractère d’impossibilités physiques et d’absolues nécessités morales. Si je vois la vertu songer trop à ses placements sur une vie éternelle, je suis tenté de lui insinuer discrètement la possibilité d’un mécompte. L’humanité doit sûrement être écoutée en ses instincts ; l’humanité, au fond, a raison ; mais dans la forme, dans le détail, oh ! la chère et touchante rêveuse, comme sa piété peut l’égarer ! Et cela est tout simple ; il est des questions insolubles sur lesquelles le sentiment moral veut une réponse. On prend à cet égard les plus belles résolutions de sobriété intellectuelle, et on ne les tient pas. Notre grand {\persName Littré} passa toute sa vie à s’interdire de penser aux problèmes supérieurs et à y penser toujours. Pauvre bonne conscience humaine ! que d’efforts elle fait pour saisir l’insaisissable ! Comme on aime à la voir se gourmander, se reprendre, se critiquer, se maudire, s’irriter contre elle-même, se remettre à l’œuvre après chaque découragement, pour renfermer dans une formule ce qu’il lui est interdit de savoir et ce qu’elle ne peut se résigner à ignorer !\par
Vous avez mille fois raison, Monsieur, quand vous mettez au-dessus de tout pour le progrès de l’esprit humain le savant, qui fait des expériences et crée des résultats nouveaux. {\persName M. Comte} n’en a pas fait ; mais je vois dans votre {\orgName Académie} d’habiles inventeurs qui déclarent cependant lui devoir beaucoup. {\persName Littré} non plus n’a pas fait d’expériences ; mais vraiment il n’en pouvait pas faire ; son champ, c’était l’esprit humain, on ne fait pas d’expériences sur l’esprit humain, sur l’histoire. La méthode scientifique, en cet ordre, est ce qu’on appelle la critique. Ah ! sa critique, je vous assure, était excellente. Il ne s’agit pas seulement, en ces obscures matières, de savoir ce qui est possible, il s’agit de savoir ce qui est arrivé. Ici la discussion historique retrouve tous ses droits. Ce que {\persName Pascal} a dit de l’esprit de finesse et de l’esprit géométrique reste la loi suprême de ces discussions, où le malentendu est si facile. Les problèmes moraux exigent ce qu’on peut appeler la critique générale. Ils ne se laissent point attaquer par la méthode scolastique. Pour être apte à jouir de ces vérités, qu’on aperçoit, non de face, mais de côté et comme du coin de l’œil, il faut la culture variée de l’esprit, la connaissance de l’humanité, de ses états, divers, de ses faiblesses, de ses illusions, de ses préjugés, à tant d’égards fondés, en raison de ses respectables absurdités ; — il faut l’histoire de la philosophie, qui parfois rend religieux, l’histoire de la religion, qui souvent rend philosophe, l’histoire de la science, qui devrait toujours rendre modeste ; — il faut la connaissance d’une foule de choses qu’on apprend uniquement pour voir que ce sont des vanités ; — il faut, par-dessus tout, l’esprit, la gaieté, la bonne santé intellectuelle d’un {\persName Lucien}, d’un {\persName Montaigne}, d’un {\persName Voltaire}. Et le résultat final, c’est encore que le plus grand des sages a été l’{\persName Ecclésiaste}, quand il représente le monde livré aux disputes des hommes, pour qu’ils n’y comprennent rien depuis un bout jusqu’à l’autre. Qu’importe, après tout, puisque le coin imperceptible de la réalité que nous entrevoyons est plein de ravissantes harmonies, et que la vie, telle qu’elle nous a été octroyée, est un don excellent et pour chacun de nous la révélation d’une bonté infinie ?\par
\emph{« Celui qui proclame, dites-vous, l’existence de l’infini accumulé dans cette affirmation plus de surnaturel qu’il n’y en a dans tous les miracles de toutes les religions. »} Vous allez, je crois, un peu loin, Monsieur ; vous donnez là un certificat de crédibilité à des choses étranges. Permettez-moi une distinction. Dans le champ de l’idéal, oh ! vous avez raison ; là on peut évoluer durant toute l’éternité sans se rencontrer jamais. Mais l’idéal n’est pas le surnaturel particulier, qui est censé avoir fait son apparition à un point du temps et de l’espace. Celui-ci tombe sous le coup de la critique. L’ordre du possible, qui touche de près à celui du rêve, n’est pas l’ordre des faits. Les religions se donnent comme des faits et doivent être discutées comme des faits, c’est-à-dire par la critique historique. Or, les faits surnaturels, du genre de ceux qui remplissent l’histoire religieuse, {\persName M. Littré} excelle à montrer qu’ils n’arrivent pas ; et, s’ils n’arrivent pas, n’est-ce point le cas de se poser la question de {\persName Cicéron} : \emph{« Pourquoi ces forces secrètes ont-elles disparu ? Ne serait-ce pas parce que les hommes sont devenus moins crédules ? »}\par
La méthode de {\persName M. Littré} reste donc excellente dans l’ordre des faits auxquels il l’applique d’ordinaire. Les faits où l’on croit voir des interventions de volontés particulières, supérieures à l’homme et à la nature, disparaissent à mesure qu’on les serre de plus près. Aucun fait historique de ce genre n’est prouvé ni dans le présent, ni dans le passé, — j’entends prouvé sérieusement, d’une de ces preuves qui excluent toute chance d’erreur, — d’une de ces preuves comme celles que {\persName M. Biot} vous demandait et que vous lui avez fournies, — d’une de ces preuves telles que vous les exigez de vos contradicteurs et que rarement ils peuvent vous fournir. Or il n’est pas conforme à l’esprit scientifique d’admettre un ordre de faits qui n’est appuyé sur aucune induction, sur aucune analogie. \emph{Quod gratis asseritur gratis negatur.} Croyez-moi, Monsieur, la critique historique a ses bonnes parties. L’esprit humain ne serait pas ce qu’il est sans elle, et j’ose dire que vos sciences, dont j’admire si hautement les résultats n’existeraient pas s’il n’y avait, à côté d’elles, une gardienne vigilante pour empêcher le monde d’être dévoré par la superstition et livré sans défense à toutes les assertions de la crédulité.\par
Soyez donc indulgent, Monsieur, pour des études où l’on n’a pas, il est vrai, l’instrument de l’expérience, si merveilleux entre vos mains, mais qui, néanmoins, peuvent créer la certitude et amener des résultats importants. Permettez-moi de vous rappeler votre belle découverte de l’acide droit et de l’acide gauche. Il y a aussi dans l’ordre intellectuel des sens divers, des oppositions apparentes qui n’excluent pas au fond la similitude. Il y a des esprits qu’il est aussi impossible de ramener l’un à l’autre qu’il est impossible, selon la comparaison dont vous aimez à vous servir, de faire rentrer deux gants l’un dans l’autre. Et pourtant les deux gants sont également nécessaires ; tous deux se complètent. Nos deux mains ne se superposent pas ; mais elles peuvent se joindre. Dans le vaste sein de la nature, les efforts les plus divers s’ajoutent, se combinent et aboutissent à une résultante de la plus majestueuse unité.\par
Par sa science colossale, puisée aux sources les plus diverses, par la sagacité de son esprit et son ardent besoin de vérité, {\persName Littré} a été à son jour une des consciences les plus complètes de l’univers. Le moment où il est venu au monde est un âge particulier, comme tous les autres âges, dans l’histoire de notre globe et de l’humanité. Mais sa haute vie l’a mis en rapport avec l’esprit éternel qui agit et se continue à travers les siècles ; il est immortel. Il a compris son heure mieux que personne ; il a vécu et senti avec l’humanité de son temps ; il a partagé ses espérances, si l’on veut ses erreurs ; il n’a reculé devant aucune responsabilité. Penseur, il ne vécut que pour le vrai. En politique, il suivit la règle que doit s’imposer le patriote consciencieux : il ne sollicita aucun mandat ; il n’en refusa aucun. Son honnêteté supérieure couvrit tout, en l’élevant à ces hauteurs où ce que les uns blâment, ce que les autres approuvent, n’est plus que raison impersonnelle, dévouement et devoir.\par
Dans ses dernières années, il vit la forme de gouvernement pour laquelle il avait toujours combattu devenir une réalité. Vous croyez peut-être qu’il va triompher. Triompher ! oh ! sentiment dénué de sens pour une âme philosophe ! Le lendemain de sa victoire, {\persName Littré} est plus modeste que jamais. Il a l’air de redouter son succès ; il se repent presque ; je dis mal ; non, il ne se repent pas ; mais il devient le sage accompli ; il se fait le conseiller, le modérateur de ses compagnons de lutte, si bien que les esprits superficiels cessèrent de le comprendre, et peu s’en fallut qu’il ne fût aussi appelé traître à son jour. Il vit juste ; car il vit la solution suprême des problèmes de la politique contemporaine dans la liberté, non dans cette collision puérile où chacun invoque à son profit un principe dont il est bien décidé à ne pas faire profiter les autres, mais dans la vraie liberté, égale pour tous, fondée sur la notion de la neutralité de l’État en fait de choses spéculatives. La mesure qu’il voulait pour lui, il la réclamait pour les autres, même quand il savait que ceux-ci ne lui rendraient pas la pareille s’ils étaient les maîtres. Il ne se faisait à cet égard aucune illusion ; un an avant sa mort, il appelle encore le catholicisme « l’adversaire naturel de toutes les libertés » ; mais, tolérant pour les intolérants, il réclamait l’application abstraite des principes. Il était persuadé que les tolérants posséderont la terre et que le libéralisme qui n’a pas peur de la liberté des autres est le signe de la vérité. En 1872, visitant un phare sur les côtes de {\placeName Bretagne}, il tomba de la hauteur d’un premier étage ; il en fut quitte pour quelques contusions ; un journaliste des environs regretta qu’il ne se fût pas tout à fait rompu le cou\emph{. « Nous ne pensions pas de même sur les croyances théologiques, ajoute {\persName M. Littré} en racontant cette histoire, et telle est la forme que prenait son dissentiment. »}\par
S’il fut quelquefois faible, ce fut toujours par bonté. Nous vivons dans un temps où il y a des inconvénients à être poli ; on vous prend à la lettre. {\persName M. Littré} avait pour principe de ne rien faire pour éviter les malentendus. Il votait souvent pour ses adversaires, afin de s’assurer à lui-même qu’il était bien impartial. Quel homme, Monsieur, et que vous avez eu raison de le comparer à un saint ! On ne trouve à reprendre en lui que des excès de vertu.\par
Lui manqua-t-il, en effet, quelque chose ? il ne lui manqua que des défauts. Parfois peut être on regrettait qu’il ne sût pas sourire. L’ironie lui échappait ; il ne la comprenait pas en philosophie ; elle lui déplaisait en politique. Or, le monde prêtant à la fois au rire et à la pitié, la gaieté a bien aussi sa raison d’être ; une foule de choses ne peuvent s’exprimer que par là. {\persName Socrate} trouvait son profit aux soupers d’{\persName Aspasie} ; {\persName Littré} n’aima que la bonté. Il prit la meilleure part ; c’est la bonté qui fait vivre. Il se plaisait avec le peuple ; il était compris et apprécié de lui. Heureux celui qui est assez grand pour que les petits l’admirent ! La vraie grandeur c’est d’être vu grand par l’œil des humbles. Le chef-d’œuvre de {\persName Spinoza} fut d’avoir été estimé de son logeur. Ce brave homme ne savait pas un mot des systèmes de son hôte ; il n’avait vu en lui qu’un homme bien tranquille, un parfait locataire. Ce furent ses renseignements qui fournirent à {\persName Colerus} les traits de cette \emph{Vie admirable} qui, bien plus que \emph{l’Éthique démontrée géométriquement}, a fait de {\persName Spinoza} un des saints de l’âge moderne. {\persName Littré}, de même, avait le goût des simples ; les simples le lui rendirent. Quand il allait en {\placeName Bretagne} il remplissait de respect ces bonnes gens de {\placeName Plouha} et de {\placeName Roscoff}, qui le prenaient pour un ecclésiastique. Il nous a raconté comment, étant à {\placeName Lion-sur-Mer}, sur la plage, deux messieurs vinrent à passer : \emph{« Voilà {\persName Littré}, dit l’un deux. — {\persName Littré} ! dit l’autre, il a l’air d’un vieux prêtre. »}\par
C’était là sa vraie définition. Grâce à lui et à quelques autres comme lui, la libre philosophie de notre âge a possédé dans son sein des vertus susceptibles d’être comparées à celles dont les religions sont le plus fières. Nature essentiellement religieuse, il ne douta que par la foi profonde et par respect de la vérité. {\persName Littré} a vraiment été une gloire de notre patrie et de notre race. En lui s’est montré au plus haut degré ce que « le {\orgName peuple gallican} », comme on disait au moyen âge, a de droiture, de sincérité, d’honnêteté, et, sous apparence révolutionnaire, de sage réserve et de prudente raison. Sa foi dans le bien fut absolue ; les mobiles inférieurs de la vie, l’intérêt, les jouissances, le plaisir, furent chez lui entièrement subordonnés à la poursuite que sa conviction lui marquait comme le devoir.\par
La fin d’une si belle vie aurait dû être calme, douce et consolée. Mais cette marâtre nature qui récompense si mal ici-bas ce qu’on fait pour coopérer à ses fins montra, en ce qui le concerne, sa noire ingratitude. Les dernières années de notre éminent confrère furent remplies par de cruelles souffrances. Dans un écrit intitulé : \emph{Pour la dernière fois}, il fit entendre sa plainte doucement résignée : \emph{« Je ne suis pas stoïcien, dit-il, et je n’ai jamais nié que la douleur fût un mal. Or, depuis bien des mois, la douleur m’accable avec une persistance désespérante. {\persName Cornélius Népos} rapporte que {\persName Pomponius Atticus}, étant parvenu à l’âge de soixante-dix-sept ans et se sentant atteint d’une maladie incurable, appela auprès de lui son gendre et sa fille. Il leur exposa son état et leur demanda la permission de sortir d’une vie qui allait finir bientôt, et d’abréger ainsi la durée de ses souffrances… Cette véridique histoire m’est revenue bien souvent en l’esprit, sans que je prémédite rien de semblable à la résolution d’{\persName Atticus}, sachant qu’aucune permission ne me serait donnée !… »}\par
Sa foi ne fut nullement atteinte par l’affaiblissement des organes\emph{. « Dans les temps modernes, dit-il à la fin du morceau que je citais tout à l’heure et qui est en quelque sorte son testament philosophique, est survenu un grave événement d’évolution, qui n’est plus ni une hérésie ni une religion nouvelle. Le ciel théologique a disparu, et à sa place s’est montré le ciel scientifique ; les deux n’ont rien de commun. Sous cette influence, il s’est produit un vaste déchirement dans les esprits. Il est bien vrai qu’une masse considérable est restée attachée à l’antique tradition. Il est bien vrai aussi que, dans la tourmente morale qui s’ensuit, plusieurs, renonçant aux doctrines modernes, retournent au giron théologique. Quoi qu’il en soit de ce va-et-vient qui demeure trop individuel pour fournir une base d’appréciation, deux faits prépondérants continuent à exercer leur action sociale. Le premier, c’est le progrès continu de la laïcité, c’est-à-dire de l’État neutre entre les religions, tolérant pour tous les cultes et forçant l’{\orgName Église} à lui obéir en ce point capital ; le second, c’est la confirmation incessante que le ciel scientifique reçoit de toutes les découvertes, sans que le ciel théologique obtienne rien qui en étaye la structure chancelante. »}\par
\emph{« Je me résigne, ajoute-t-il, aux lois inexorables de la nature… La philosophie positive, qui m’a tant secouru depuis trente ans, et qui, me donnant un idéal, la soif du meilleur, la vue de l’histoire et le souci de l’humanité, m’a préservé d’être un simple négateur, m’accompagne fidèlement en ces dernières épreuves. Les questions qu’elle résout à sa manière, les règles qu’elle prescrit en vertu de son principe, les croyances qu’elle déconseille au nom de notre ignorance de tout absolu, je viens, aux pages qui précèdent, d’en faire un examen que je termine par la parole suprême du début : Pour la dernière fois. »}\par
J’ai toujours eu peine, je l’avoue, devant les cercueils illustres, à partager cette héroïque résignation. \emph{« La mort, selon une pensée qu’admire {\persName M. Littré}, n’est qu’une fonction, la dernière et la plus tranquille de toutes. »} Pour moi, je la trouve odieuse, haïssable, insensée, quand elle étend sa main froidement aveugle sur la vertu et le génie. Une voix est en nous, que seules les bonnes et grandes âmes savent entendre, et cette voix nous crie sans cesse : \emph{« La vérité et le bien sont la fin de ta vie ; sacrifie tout le reste à ce but »} ; et quand, suivant l’appel de cette sirène intérieure, qui dit avoir les promesses de vie, nous sommes arrivés au terme où devrait être la récompense, ah ! la trompeuse consolatrice ! elle nous manque. Cette philosophie, qui nous promettait le secret de la mort, s’excuse en balbutiant, et l’idéal, qui nous avait attirés jusqu’aux limites de l’air respirable, nous fait défaut quand, à l’heure suprême, notre œil le cherche. Le but de la nature a été atteint ; un puissant effort a été tenté ; une vie admirable a été réalisée, et alors, avec cette insouciance qui la caractérise, l’enchanteresse nous abandonne, et nous laisse en proie aux tristes oiseaux de nuit.\par
Mais laissons là ces amères pensées ; car il est quelque chose que nous gardons de lui : ce sont les leçons qu’il nous a données, cet ardent amour du droit et de la vérité, qui ont été l’âme de sa vie. La patrie, qu’il a tant aimée, la science, qu’il a préférée à lui-même, la vertu, dont il fit la règle de sa conduite, sont des choses éternelles. Nous entendrons toujours ces sages paroles qui semblaient, par leur calme gravité, venir du fond d’un tombeau, et nous dirons pour finir par une grande pensée de lui : \emph{« Le temps, qui est beaucoup pour les individus, n’est rien pour ces longues évolutions qui s’accomplissent, dans la destinée de l’humanité. Déjà, du sein de la vie individuelle, il est permis de s’associer à cet avenir, de travailler à le préparer, de devenir ainsi, par la pensée et par le cœur, membre de la société éternelle, et de trouver en cette association profonde, malgré les anarchies contemporaines et les découragements, la foi qui soutient, l’ardeur qui vivifie, et l’intime satisfaction de se confondre sciemment avec cette grande existence, satisfaction qui est le terme de la béatitude humaine. »}\par
Votre dévouement absolu à la science vous donnait le droit, Monsieur, de succéder à un tel homme et de rappeler ici cette grande et sainte mémoire. Vous trouverez à nos séances un délassement pour votre esprit toujours préoccupé de découvertes nouvelles. Cette rencontre en une même compagnie de toutes les opinions et de tous les genres d’esprit vous plaira : ici le rire charmant de la comédie, le roman pur et tendre, la poésie au puissant coup d’aile ou au rythme harmonieux ; là, toute la finesse de l’observation morale, l’analyse la plus exquise des ouvrages de l’esprit, le sens profond de l’histoire. Tout cela n’ébranlera pas votre foi en vos expériences ; l’acide droit restera l’acide droit ; l’acide gauche restera l’acide gauche. Mais vous trouverez que les prudentes abstentions de {\persName M. Littré} avaient du bon. Vous assisterez avec quelque intérêt aux peines que se donne notre philosophie critique pour faire la part de l’erreur, en se défiant de ses procédés, en limitant l’étendue de ses propres affirmations. À la vue de tant de bonnes choses qu’enseignent les lettres, en apparence frivoles, vous arriverez à penser que le doute discret, le sourire, l’esprit de finesse dont parle {\persName Pascal}, ont bien aussi leur prix. Vous n’aurez pas chez nous d’expériences à faire ; mais cette modeste observation que vous maltraitez si fort suffira pour vous procurer de bien douces heures. Nous vous communiquerons nos hésitations ; vous nous communiquerez votre assurance. Vous nous apporterez surtout votre gloire, votre génie, l’éclat de vos découvertes. Soyez le bienvenu, Monsieur.
\section[{Réponse au discours de réception de M. Cherbuliez}]{Réponse au discours de réception de {\persName M. Cherbuliez}}\renewcommand{\leftmark}{Réponse au discours de réception de {\persName M. Cherbuliez}}


\dateline{25 mai 1882}

\salute{Monsieur,}
\noindent Nous savions ce que nous faisions en vous choisissant pour remplacer parmi nous {\persName M. Dufaure}. Nous savions que cette mémoire respectée recevrait de vous l’éloge le plus complet. Votre étude, de tous points accomplie, est l’image même de cette belle vie toute consacrée au bien public, remplie par une seule passion, celle de la justice. Les brusques revirements de la politique ont pu, de notre temps, excuser plus d’une défaillance ; {\persName M. Dufaure}, lui, n’eut jamais besoin de changer. Ce n’était point un libéral de circonstance ; le lendemain de ces révolutions qui déjouent les solutions les mieux concertées, il se retrouvait tel qu’il fut la veille. La {\orgName France} et les principes étaient pour lui deux choses immuables. Vous nous avez montré, en ses derniers jours, la sérénité de sa conscience, sa noble tranquillité. Il avait le droit, en effet, d’être tranquille. Tout ce que peut la droiture appliquée à la direction des choses humaines, il l’avait fait. Les révolutions dont il hérita, il n’y avait point contribué ; les maux qu’il répara ne lui étaient pas imputables.\par
Comme vous nous l’avez très bien dit. Monsieur, les idées de {\persName M. Dufaure} n’eurent point, en quelque sorte, d’origine. Il n’y eut pas pour lui de lutte, de tiraillement entre des principes opposés ; ses opinions étaient nées avec lui ; il les trouva dans son naturel raisonnable, modéré, et dans l’atmosphère où s’écoula sa jeunesse. C’étaient, j’ose le dire, les opinions de la {\orgName France} même. Le temps où il entra dans la vie est un de ceux où les aspirations françaises eurent le tour le plus prononcé. Après les grandeurs de l’ancien régime, après les ivresses tour à tour brillantes et sombres de la Révolution et de l’{\orgName Empire}, presque tous les esprits éclairés conçurent pour la {\orgName France}, sous la pacifique garantie de la royauté constitutionnelle, un nouvel avenir de gloire et de bonheur. La difficulté de faire de l’éclectisme dans une œuvre aussi passionnée que celle de la Révolution ne les arrêtait pas. Le principe que nous avons entendu proclamer de nos jours : \emph{« La Révolution, on l’adore ou on la maudit ; on ne la critique pas »}, ce principe, dis-je, n’était alors dans la pensée de personne. Heureuse génération ! les options tranchées entre les extrêmes n’existèrent pas pour elle. Elle conçut la vie politique comme un tournoi entre des rivaux pleins de courtoisie et qui s’entendent sur les questions fondamentales. Elle n’oubliait qu’une chose, c’est que les fortes fièvres, même disparues, ont toujours une tendance à recommencer. Qu’on le prenne pour un signe d’élection ou pour un signe de réprobation, la {\orgName France} est condamnée à ne dormir jamais du sommeil tranquille de la médiocrité satisfaite. Deux mondes luttent dans son sein. Même en ses heures d’assoupissement, elle a les tremblements convulsifs qui décèlent dans les profondeurs de l’organisme un travail mystérieux.\par
L’humanité s’arrêterait si tous y voyaient trop clair. Aux huit béatitudes de l’Évangile, je suis quelquefois tenté d’en ajouter une neuvième : « Heureux les aveugles, car ils ne doutent de rien ! » La {\orgName France} n’avait pas lu {\persName M. de Maistre} ; un pays n’est jamais très fort en histoire, et d’ailleurs beaucoup de maximes politiques devenues maintenant évidentes n’étaient pas claires alors. La suite constitutionnelle, qui, de {\persName Hugues Capet} au 10 août 1792, ne subit pas d’interruption durable, avait été depuis vingt-cinq ans deux fois renversée ; mais on pouvait croire que la crise était finie ; tous les compromis, toutes les substitutions semblaient possibles. De là une quiétude qui nous étonne. Chaque secousse paraissait la dernière ; on proclamait avec un profond sérieux la naïve prétention de fermer l’ère des révolutions. On ne la fermait pas le moins du monde ; au contraire, on prenait son parti de l’incorrection, on s’habituait à l’instabilité. Les populations des flancs du {\placeName Vésuve} savent bien que le volcan se réveillera ; mais d’ici là que de belles heures ! quelles vendanges ! quelles moissons ! Et puis la coulée de lave marche lentement ! Avant qu’elle arrive, on a le temps de fuir.\par
Que peut faire l’honnête homme en présence d’une situation générale qu’il n’a point amenée, dont il voit les inconvénients, mais qu’il n’a pas le pouvoir de modifier ? {\persName M. Dufaure} nous a donné à ce sujet d’excellents exemples. Après chaque révolution, quelque chose durait pour lui ; c’était la {\orgName France}. Il continuait de servir la {\orgName France} et de chercher pour elle ce que sa raison lui présentait comme le meilleur. La bonne gestion des affaires était à ses yeux un intérêt supérieur à la politique proprement dite. Erreur, si vous voulez ; mais erreur nécessaire ! Que les partisans d’une légitimité absolue regardent comme un devoir de se renfermer chez eux après leur défaite et de tenir rancune au pays qui n’a pas suivi leurs avis, nous respectons leur ardente conviction ; disons cependant que cette abstention un peu orgueilleuse ne sera jamais la règle française. Il obéit aux plus nobles et aux plus vraies dictées de son cœur, ce prince de la {\orgName maison de France} qui siège au milieu de nous, quand il voulut reparaître citoyen et soldat dans la grande patrie que ses ancêtres avaient fondée par dix siècles de prudence et d’habileté. Notre illustre confrère {\persName M. Thiers }eut également pour principe qu’après le fait accompli, il n’est point patriotique de donner tort à son pays, ni de vouloir paraître plus sage que lui.\par
Certes, il n’est pas défendu de porter envie aux âges où le problème de la fidélité était plus simple et où le devoir se bornait à servir de son mieux un pouvoir établi sur des bases indéniables. Dans les temps les plus troublés, néanmoins, le patriote libéral trouve encore moyen de contribuer au bien de la patrie. Il y a toujours une {\orgName France} à aimer. Ils nous approuveraient dans nos apparentes faiblesses, ces créateurs de l’unité française qui mirent avant tout le salut de l’{\orgName État}. Le jour où une bande d’idiots profana le tombeau de {\persName Richelieu} à la {\orgName Sorbonne}, le crâne de notre illustre fondateur tomba sur les dalles, et les enfants du quartier le firent rouler à terre comme un jouet. Vanité des vanités ! dira-t-on ; la voilà finie comme le reste cette pensée altière au succès de laquelle on avait fait servir tant de force de volonté, tant de savantes combinaisons, tant de crimes… — Pas aussi finie qu’il semble. Si cet œil éteint, où rayonna autrefois le génie, s’était rouvert à la lumière, il eût vu se dessiner sur les murailles voisines des lettres fraîchement tracées : République française, une et indivisible. Sauf un mot, c’était là ce que le grand politique avait voulu. Il n’était donc pas vaincu, malgré l’affront que des misérables faisaient à ses os.\par
{\persName M. Dufaure} fut le loyal serviteur de cette légitimité de la nation, qui a survécu chez nous à celle des dynasties. Dans les jours les plus sombres, il eut une étoile. Au milieu des plus écœurantes incertitudes, durant ces années où on vit le sort de la {\orgName France} suspendu presque à une voix, il maintint son ferme équilibre. Le mot de République ne l’avait point séduit, pendant qu’il fut une menace ; ce mot ne l’effraya pas, quand il désigna une chose établie. Homme de légalité absolue, {\persName M. Dufaure} fut surtout homme de cœur. Quand il reprit la direction des affaires au mois de décembre 1877, dans des circonstances qui l’obligeaient à être très net en son programme, il arrêta d’avance les paroles qu’il voulait adresser au {\persName chef de l’État}. Il entre : que trouve-t-il ? Un vieux soldat, qui, en le voyant, se met à verser des larmes. Il pleure de son côté, et le petit discours ne fut pas prononcé.\par
Un calme parfait de conscience était le fruit de cette attention unique donnée au bien du pays. Il était impossible de dire à quel parti appartenait {\persName M. Dufaure}. C’était à peine un homme politique (j’entends le dire à son éloge) ; c’était un homme de réforme et de justice. Ce détachement des questions constitutionnelles peut avoir ses inconvénients, quand il fournit des prétextes aux transactions de l’intérêt personnel ou de l’ambition. L’intérêt, l’ambition n’existaient pas pour {\persName M. Dufaure}. Le ministère ne le grandissait ni ne l’amoindrissait. On porte avec aisance les fardeaux qu’on n’a pas brigués. Voilà pourquoi, en politique, la désignation de la naissance donne tant de force. L’âme la plus tranquille, au milieu des terribles épreuves de la Révolution, dut être celle de {\persName Louis XVI} ; car seul il pouvait dire : \emph{« La responsabilité sous laquelle je succombe, je ne l’ai point cherchée. »}\par
{\persName M. Dufaure} comprit de même que, après 1870, le gouvernement ne pouvait être accepté que par devoir. Persuadé que le pouvoir, dans les circonstances où nous sommes, ne saurait plus donner de gloire, et qu’il peut, au contraire, condamner celui qui l’exerce aux plus tristes nécessités, {\persName M. Dufaure} ne se départit pas un moment, depuis nos malheurs, de cette attitude vertueusement dédaigneuse. Là était le secret de son autorité. On n’est fort sur les hommes qu’en leur faisant sentir qu’on n’a pas besoin d’eux.\par
Excellente leçon donnée aux empressements de la légèreté, tout occupée à rechercher une tâche où les plus habiles ne peuvent manquer d’échouer. Quel défaut de prévoyance et de philosophie ! Comment ne pas voir que, dans la direction des affaires d’un peuple vaincu, il n’y a plus d’autre gloire à recueillir que celle que donnent le sacrifice et le devoir accompli ? Mille fois honneur à celui qui ne refuse pas une œuvre qu’il sait ingrate et peu récompensée ! Décernons, au contraire, les grelots de la parfaite étourderie à celui qui va joyeusement au-devant d’une mission qu’il faudrait accepter avec tristesse, et où l’on est assuré d’avance de succomber. Il semble que les mandats politiques sont d’autant plus désirés qu’ils sont moins enviables. Vous rappelez-vous, Monsieur, à {\placeName Rome}, sur la {\placeName voie Appienne}, la très ancienne petite église consacrée aux saints {\persName Nérée} et {\persName Achillée}, les pieux eunuques de {\persName Flavie Domitille} ? On y lit, gravée sur le dossier du siège presbytéral, la belle homélie que le {\persName pape saint Grégoire le Grand}, au seuil de la plus sombre époque du moyen âge, prononça dans cette église le jour de la station : \emph{« Ces deux saints, dit-il, méprisèrent le monde, quand le monde valait encore la peine d’être aimé. La vie alors était longue, la race féconde, la sécurité parfaite, la richesse garantie, la tranquillité assurée. Le monde, en un mot, se dessécha dans leurs cœurs quand au dehors il verdoyait. Maintenant, Dieu a rendu la résignation sans mérite et le détachement facile. Le monde s’est desséché, et pourtant il vit toujours dans notre cœur. »}\par
Il vivra longtemps encore. Je me figure souvent, Monsieur, qu’à la veille du jugement dernier, quand les signes au ciel seront si évidents que le doute ne sera plus possible, il y aura encore des gens pour briguer l’honneur d’être maire de village ou conseiller municipal. La froideur de {\persName M. Dufaure} le préserva d’une faute que tous n’évitèrent pas. Il était aussi exempt de vanité qu’on peut être ; il ne portait dans l’exercice du pouvoir aucun air de satisfaction. Il ne domina pas les événements (qui, de nos jours, a su les dominer ?) ; il s’y conforma en honnête homme. Il fit comme le navigateur, qui n’est pas dans les secrets du vent, mais qui se sert de tous les vents pour arriver à son but. D’autres, tels que {\persName M. Guizot}, {\persName M. de Lamartine}, eurent des théories politiques plus arrêtées ; cela les conduisit aux chutes éclatantes. {\persName M. Thiers}, {\persName M. Dufaure}, au contraire, eurent des jours d’attente, de retraite ; ils ne tombèrent jamais tout à fait.\par
Nous vous remercions. Monsieur, d’avoir fait revivre devant nous, en traits qui resteront, ce grand et noble caractère. Votre patriotique discours est un morceau digne d’être joint à tant d’excellentes pages qui, depuis longtemps, vous ont fait nôtre. Selon la lettre de la loi, vous n’êtes Français que depuis deux ans. Vous l’avez toujours été par votre talent ; vous l’avez été surtout depuis le jour où, sous le nom de {\persName Valbert}, vous êtes devenu l’éloquent interprète de nos griefs, de nos froissements, de ce que nous avons à dire contre des attaques injustes et passionnées.\par
Que vous avez bien choisi votre heure, Monsieur, pour vous rattacher de nouveau à une patrie dont une funeste erreur de l’ancienne politique vous avait séparé ! Issu d’une de ces familles protestantes qui durent, il y a deux cents ans, choisir entre leur pays et la liberté de leurs croyances, vous aviez toujours eu dans le cœur un sentiment affectueux pour la patrie de vos pères. Aux jours où la {\orgName France} était heureuse, cela vous suffisait. Mais il y eut un moment où il vous fallut davantage ; c’est le moment où la {\orgName France} subit la plus grande épreuve qu’elle ait connue depuis qu’elle existe. Quand cette vieille mère, abandonnée de ceux qui lui devaient le plus, s’entendait dire, comme le {\persName Christ} au {\placeName Calvaire} \emph{: « Toi qui as sauvé les autres, sauve-toi maintenant »} ; quand l’Europe presque entière, après les fautes expiées, raillait notre agonie et ne voyait qu’une bonne place à prendre dans le vide que nous allions laisser ; ce jour où l’ingratitude a été érigée en loi du monde, vous vous êtes pris à aimer plus vivement que jamais votre patrie d’il y a deux cents ans, et vous, descendant d’exilés qui avaient bien quelque chose à oublier, vous avez consacré votre talent à la cause vaincue, et, dès que les devoirs qui vous retenaient à {\placeName Genève} vous l’ont permis, vous avez profité de la loi réparatrice de 1790, qui rend la pleine nationalité française à \emph{« toute personne qui, née en pays étranger, descendrait, en quelque degré que ce soit, d’un Français ou d’une Française expatriés pour cause de religion »}. Vos preuves étaient faciles à réunir. Le {\placeName Dauphiné}, d’où votre nom est originaire, le {\placeName Poitou}, les {\placeName Cévennes} vous ont fourni au complet la série de vos ascendants.\par
Le sérieux des temps modernes dérivant presque tout entier du christianisme, chacun de nous trouve d’ordinaire ses origines en quelque respectable société religieuse, où la gravité des mœurs entretenait la gravité de l’esprit et où la discussion théologique préparait l’aptitude aux longs raisonnements. Ces austères traditions, continuées durant des siècles, ont accumulé les économies intellectuelles et morales que nous dépensons. La vertu ne se développe fortement que dans les milieux un peu sectaires. Il nous est permis, à nous, de sourire et de douter ; car des générations avant nous ont cru sans réserve. Quelles têtes excellentes n’ont pas fournies le jansénisme, le vieux gallicanisme, les sectes protestantes, la synagogue Israélite ! {\orgName Genève} mérite d’être placée au premier rang parmi ces sources glorieuses du libéralisme européen. République fondée sur la théologie, cette cité de protestation et de dispute a été une des plus fortes écoles de culture rationnelle qu’il y ait eu. La contrainte souvent pharisaïque qui pesait sur les mœurs et la nécessité imposée à tout laïque d’être controversiste entretenaient une grande activité et posaient fatalement la question sur laquelle se fait le partage des esprits, la question du rationalisme et de la foi. Les fortes éducations religieuses amènent toujours cette lutte solennelle. Ainsi que vous le rappeliez tout à l’heure, on en sort, au lever de l’aurore, comme Jacob, fortifié, mais souvent avec quelque nerf un peu froissé.\par
Cette épreuve, vous ne l’avez point traversée. Monsieur. Elle se passa en monsieur votre père, qui, après des études faites pour le ministère pastoral, rompit avec la vieille tradition genevoise et entra dans la voie de la philosophie et de la critique allemandes. Ce changement, comme il arrive souvent, ne modifia en rien ses règles morales. Monsieur votre père, quoique rationaliste, resta toujours un homme pieux et de mœurs exemplaires. Pour le bien comprendre, il faut avoir eu comme moi le bonheur de vous entendre parler de lui. Une vie entière était parfumée par le souvenir de ces croyances fécondes dont on pouvait sacrifier la lettre sans abandonner l’esprit. Vous avez bénéficié du combat intérieur de monsieur votre père ; vous avez pu observer en lui cette heure excellente du développement psychologique où l’on garde encore la sève morale de la vieille croyance sans en porter les chaînes scientifiques. À notre insu, c’est souvent à ces formules que nous devons les restes de notre vertu. Nous vivons d’une ombre, Monsieur, du parfum d’un vase vide ; après nous, on vivra de l’ombre d’une ombre ; je crains par moments que ce ne soit un peu léger.\par
Votre éducation supérieure dura plus de douze ans. Cette période, où le talent se forme et où l’essentiel est de pouvoir attendre en toute liberté l’heure de la maturité, se continua pour vous jusqu’à trente ans. {\placeName Paris} et les principales universités d’{\placeName Allemagne} vous virent assidu aux chaires savantes, avide de toutes les études nouvelles. A {\placeName Paris}, votre instinct si sûr vous conduit à la petite salle où enseignait le premier maître de notre siècle en fait de philologie et de critique, {\persName Eugène Burnouf}. Quelle fatalité pour moi. Monsieur ! L’année où vous suiviez ce cours, au {\orgName Collège de France}, j’étais en {\placeName Italie} ; sans cela, nous nous serions connus vingt ans plus tôt. À {\placeName Berlin}, vous avez vu le vieux {\persName Schelling}, qui vous parlait de tout, excepté de philosophie. Oh ! l’habile homme ! Ce qui vous préoccupait surtout à cette époque, c’était le puissant effort intellectuel de Hegel, bien que les élèves fissent déjà tort au maître. Les hégéliens, dont vous suiviez les leçons, vous choquèrent par l’abus de ces formules toutes faites, qui furent le tombeau d’une école créée par le génie, émaciée par la médiocrité. Vous méditiez quelque grande publication hégélienne. Mais une révélation vous fut faite vers cette époque ; vous vîtes l’{\persName Éternel} face à face ; l’idéal du développement humain sur terre vous fut montré ; tout le plan de votre vie en fut profondément modifié.\par
Au mois d’août 1859, un voyage d’{\placeName Orient} vous conduisit à {\placeName Athènes}. Il ne vous fallut pas longtemps pour découvrir qu’il y a un lieu au monde (il n’y en a pas un second) où la parfaite beauté a été réalisée. Les cinq ou six petits monuments d’{\placeName Athènes} vous apparurent comme ce qu’ils sont, c’est-à-dire comme les restes d’un monde de miracles, d’une éclosion divine qui ne se renouvellera plus. Tout le reste, en effet, du développement athénien fut à l’avenant. Un peuple entier admira cet art de l’{\orgName Acropole}, dont la perfection réside en des ténuités infinies ; ce même peuple vit la perfection de l’éloquence dans cette argumentation de {\persName Démosthène}, qui est un vrai tissu de fer ; il applaudit un théâtre qu’on dirait fait pour un groupe de raffinés ; il conversa dans cette langue adorable d’élégance et de simplicité qui est celle des interlocuteurs de {\persName Platon}. Vous comprîtes à fond ; vous étiez dès lors fixé sur la conception idéale de la vie humaine qui doit servir de règle pour juger tout le reste. Sur le bateau qui vous ramenait à {\placeName Trieste}, vous écriviez ce dialogue exquis où, à propos d’un cheval de {\persName Phidias}, vous exprimez vos idées sur la transformation la plus profonde qui se soit opérée dans l’humanité, puisque le passage du paganisme au christianisme a été avant tout une révolution esthétique. Vous étiez, dès lors, un excellent écrivain, sans avoir jamais été à aucune des écoles où l’on prétend apprendre à le devenir. Vous pensiez bien et vous saviez beaucoup. Ce fut {\persName M. Sainte-Beuve}, Monsieur, qui me fit connaître votre livre. Peu de jours après la première édition genevoise ; \emph{« Lisez {\persName Victor Cherbuliez}, me dit-il ; c’est un des nôtres. »} Voyez comme il était prophète. Si ce maître illustre vivait encore, ce que la mesure ordinaire de la vie humaine permettrait, vous auriez eu un suffrage de plus, et quel suffrage !\par
Un autre jugement, qui valait celui-là, fut celui de madame Sand. Votre livre l’enchanta ; sans vous en prévenir, elle écrivit au directeur de la \emph{Revue des Deux Mondes} ce qu’elle pensait de l’auteur. Telle fut l’origine de vos rapports avec un homme que j’ai aussi beaucoup connu aux débuts de ma vie littéraire, et que nous jugeons exactement de la même manière. Comme vous, j’ai gardé de lui un excellent souvenir. Chacun vaut en proportion de l’œuvre à laquelle il consacre sa vie. {\persName M. Buloz} n’avait en vue que le succès de son recueil ; jamais rien ni personne ne le fit dévier de ce but unique. Il connaissait son public. Libre aux orgueilleux de la littérature de soutenir qu’il n’est pas utile à un écrivain d’avoir devant lui un homme qui représente le public tout entier. Les modestes comme nous n’ont jamais eu d’aussi superbes attitudes ; nous avons vingt fois trouvé commode de pouvoir entendre, avant l’irrévocable tirage, l’avis d’un lecteur qui nous donnait, par anticipation, le sentiment de tous les autres. Pour moi, il m’est arrivé souvent, quand j’avais des doutes sur la mesure ou l’opportunité d’un trait, de le laisser, pour voir ce qu’en dirait {\persName M. Buloz}, bien décidé à le supprimer s’il hésitait le moins du monde. N’en avez-vous pas quelquefois fait autant, Monsieur ? Cela mettait notre conscience en repos. Que {\persName M. Buloz} nous eût tous sacrifiés à l’intérêt de la \emph{Revue}, cela ne faisait doute pour aucun de nous ; mais aussi, quand nous servions au succès de la \emph{Revue}, il nous eût défendus envers et contre tous. Vous m’avez raconté qu’un jour, à {\placeName Ronjoux}, un des hôtes, qui se disait très exercé dans le discernement des champignons, ayant fait servir sa récolte au dîner, il y eut un moment d’hésitation. Vous fûtes le seul à entamer bravement le plat suspect. {\persName M. Buloz} vous regardait : \emph{« {\persName Cherbuliez}, vous dit-il, que faites-vous ? Vous n’avez pas fini votre roman pour la Revue ! »}\par
Le succès du \emph{Comte Kostia} justifia pleinement le jugement de {\persName madame Sand}. Le roman était dès lors la forme que vous aviez adoptée. Ceux qui connaissaient la direction philosophique de vos idées en éprouvèrent d’abord quelque surprise. Je l’avoue, une des erreurs que je professais alors était de croire que l’art du romancier peut difficilement produire des œuvres destinées à durer. Une longue fiction en prose me paraissait une faute littéraire. L’antiquité n’a composé de romans qu’à son âge de décadence, et, en général, n’en a composé que de courts. L’illusion des faiseurs de \emph{Cyrus} et d’\emph{Astrées} est de supposer qu’on a le temps de les lire. Le grand inconvénient du roman moderne, en effet, est d’avoir créé à son usage une catégorie spéciale de lecteurs. D’un côté, ceux qui lisent des romans ne lisent guère autre chose. D’un autre côté, la vie est courte, et l’histoire, la science, les études sociales ont tant d’intérêt ! Pour moi, devant ces attrayants volumes, qui offrent le tableau, souvent vrai, des mœurs contemporaines, je suis partagé entre deux sentiments, l’ardent désir de les lire et le regret qu’on n’ait pas pratiqué, en les imprimant, l’ancien système des manchettes, qui permettait de ne parcourir que les marges. La vulgarité et la prolixité sont le danger d’un genre où le lecteur ne cherche guère qu’une distraction et un amusement.\par
Avec quelques maîtres exquis, dont vous devenez aujourd’hui le confrère, vous avez su éviter ces défauts, Monsieur. Toujours une haute pensée vous guide. Vous ne tombez jamais dans ces interminables histoires bourgeoises, prétendues images d’un monde qui, s’il est tel qu’on le dit, ne vaut pas la peine d’être représenté. Loin de songer à une imitation servile de la réalité (imitation bien inutile, puisque celui qui aime tant la réalité n’a qu’à la regarder), vous cherchez les combinaisons capables de mettre en lumière ce que la situation de l’homme a de tragique et de contradictoire. La nature slave vous frappa tout d’abord par quelque chose de neuf et de jeune ; comme {\persName Tourguéneff}, vous aimez à vous perdre dans cet abîme d’imprévu, et quels étranges récits, quel trésor de vraisemblables folies vous en avez rapportés ! Ce n’était point là, de votre part, le fait d’une invention aux abois, se rabattant sur le bizarre par épuisement ou par incapacité de traiter la saine nature. Qui a su mieux que vous peindre la vertu sans marque d’origine, sans tache de naissance, sans race, sans signe particulier ? Vos charmantes peintures de \emph{Noirs et Rouges}, votre \emph{Roman d’une honnête femme} ont ravi tous ceux qui savent sentir. En {\placeName Allemagne}, votre \emph{Isabelle} est tenue pour une des créations les plus délicates du roman moderne.\par
Dans \emph{Paule Méré} et dans \emph{Meta Holdenis}, vous avez abordé le problème délicat par excellence, celui où se complaisent les grands artistes. Vous avez voulu chercher comment le bien confine au mal, et, en dernière analyse, où est le bien. Nos hésitations à cet égard viennent de cette divine parabole de l’Enfant prodigue. La question a été posée le jour où des pécheresses ont été présentées comme étant plus près du royaume de Dieu que le pharisien méthodique et pédantesque en ses observances. Le plus bel enseignement du christianisme est que la vertu consiste moins dans les œuvres que dans les sentiments du cœur, si bien que l’{\persName Éternel} a des tendresses pour la faute qui vient d’une ardeur généreuse ou d’un égarement d’amour, tandis qu’il n’a que de l’aversion pour l’apparente régularité, laquelle peut fort bien s’associer à la bassesse et à l’égoïsme. En d’autres termes, le don gratuit qui fait la noblesse de l’âme l’emporte infiniment sur nos chétifs efforts pour changer l’arrêt divin. Voilà probablement le seul dogme, Monsieur, sur lequel nous soyons d’accord avec {\persName Calvin} ; l’élection divine est souvent le renversement des jugements des hommes ; la grâce est la source de tout bien ; notre seul mérite est de ne pas la contrarier.\par
C’est le sentiment profond de cette vérité qui nous entraîne parfois à scandaliser les pharisiens, et même (ici peut-être. Monsieur, nous avons sur la conscience quelques péchés véniels) à y trouver un certain plaisir. Nous aimons à troubler dans leur quiétude des gens assurés de leur salut, qui damnent si facilement les autres ; à ces vertus \emph{« confites dans l’orgueil et dans le fiel »}, ainsi que vous dites si bien, nous opposons volontiers des caractères qui n’ont pas la prétention d’être impeccables. Au fond, cependant, nous sommes justes ; nous réclamons le droit des exceptions ; mais nous respectons la règle ; nous savons que les préjugés sociaux sont utiles au monde et que l’étroitesse d’esprit a maintenu durant des siècles une tradition précieuse dont nous touchons les arrérages. Les traits du roman moral, comme ceux de la comédie philosophique, n’ont pas de pointe acérée ; les faiblesses ou les ridicules qu’ils atteignent ne sont ceux de personne, étant ceux de tous.\par
Bien mal inspirés, par conséquent, ont été les esprits chagrins qui se sont formalisés de vos libertés, Monsieur. Quand on écrit avec sincérité comme vous faites, il faut être résigné d’avance à ne pas satisfaire tout le monde. Le patriotisme, en particulier, a des exigences qu’on ne réussit jamais à contenter. Il veut de l’encens, rien que de l’encens. Pour moi, je n’ai jamais parlé d’un des groupes nationaux ou religieux qui se partagent le monde sans avoir été traité de calomniateur. On accepte de bien bon cœur vos éloges comme des concessions arrachées par la vérité ; on met vos critiques sur le compte de l’injustice et de la malignité. L’hypocrisie est de toutes les races et de toutes les sectes. Les railleries de l’Évangile contre les {\orgName pharisiens} de {\placeName Jérusalem} frappent un travers éternel de l’espèce humaine. {\persName Molière} n’a pas injurié la {\orgName France} en faisant de {\persName Tartufe} un Français ; en peignant dans \emph{Meta Holdenis} l’hypocrisie protestante et sentimentale, vous n’avez critiqué aucun pays. Vous avez seulement présenté avec force les dangers d’une éducation systématique donnée aux femmes ; vous avez montré aussi combien sont injustes beaucoup de reproches qu’on nous adresse. Je lisais dernièrement une critique allemande de votre chef-d’œuvre, où l’on cherchait à excuser {\persName Meta} de ses trahisons par la raison que le milieu français qui l’entoure lui est tout à fait inférieur. Mais c’est qu’il n’en est rien vraiment. Vos Français sont d’honnêtes gens sans philosophie transcendante ; votre créole est tout simplement belle à ravir. Cela vaut mieux que de débiter des tirades sur l’idéal, et de savoir déguiser sous une sentimentalité prétentieuse le sophisme de l’esprit et la fausseté du cœur. Ce qui est vrai, c’est que les âmes très religieuses n’aiment pas beaucoup qu’on admette la possibilité de l’hypocrisie. {\persName Tartufe} ne leur plaît pas, bien qu’elles ne se sentent rien de commun avec le héros de la pièce. Une enfant très innocente qui venait de lire votre roman m’avoua qu’elle était choquée qu’une jeune fille aussi accomplie que {\persName Meta} pût être jugée capable de telles perfidies. Un Berlinois, plus expérimenté que cette enfant, vous écrivait, au moment de l’achèvement de la publication, un billet anonyme contenant ces trois mots : \emph{« C’est bien cela ! »}\par
\emph{Meta Holdenis} reste certainement votre création la plus achevée, Monsieur. Un art savant maintient d’un bout à l’autre l’équivoque qui fait l’essence même du livre. À mesure que le récit approchait de sa conclusion dans la \emph{Revue}, des lettres arrivaient de toutes parts qui vous suppliaient de ne pas faire trop mal finir la jeune personne sur laquelle vous aviez concentré tant d’intérêt. Je ne vous cacherai pas que je suis de ceux qui ont pour {\persName Meta} le plus de prédilection. Je me dis quelquefois qu’il faudrait peu de chose pour faire d’elle une bonne fille, et je me vois, composant une suite à votre beau livre, où j’essayerais de convertir votre paradoxale diaconesse à notre bonne morale gauloise. {\persName M. Buloz} fut aussi quelque temps ensorcelé ; il ne parlait que de {\persName Meta}. Il faillit se brouiller avec vous à cause du dénouement, et ce qu’il y a de plus grave, c’est que, pendant quelques mois, il rêva une {\persName Meta Holdenis} pour secrétaire de la \emph{Revue} ; oui, une jeune Allemande, instruite, active, qui eût mené la \emph{Revue} comme {\persName Meta Holdenis} mène la maison de {\persName M. de Mauserre}. Nous l’avons échappé belle, Monsieur. Voyez la conséquence de trop bien réussir. Vous aviez voulu que votre héroïne fût à la fois attachante et perverse, laide et jolie. Ne vous plaignez pas si quelques-uns, oubliant ce qu’elle a de haïssable, se sont mis à l’aimer.\par
Aimable et haïssable…, que de choses méritent cette double épithète, et que les sentiments simples ont peu de place en un siècle comme le nôtre, où la vie humaine se développe dans les sens les plus divers sans souci de la contradiction ! \emph{« Aime comme devant haïr un jour »}, disait ce prétendu sage de la {\placeName Grèce}. « Mais comme devant aimer un jour », suis-je parfois tenté de dire. En politique, du moins, tout est sujet à d’étranges retours. Nous autres, nous ne changeons pas ; mais le monde change, et alors il se trouve que ce que nous avions aimé vient parfois nous percer le cœur. Voilà ce que c’est que d’avoir eu le goût du bien, du juste, du progrès et de la liberté dans un siècle qui ne comprend plus que l’égoïsme national. Nous sommes vieux maintenant, Monsieur : nous n’aimerons plus rien ; tel est le seul parti qui, en politique, ne mène pas aux déceptions.\par
Dans notre éducation intellectuelle, nous avions été l’un et l’autre à cette grande école allemande de science et de critique qui, à la fin du dernier siècle et au commencement de celui-ci, a renouvelé tant de branches de l’esprit humain. Nous avions admiré l’application, la sagacité, la force d’esprit qui ont été déployées dans cette œuvre. Nous n’avons rien à dédire de ce que nous avons dit ; nos éloges sont sans repentance. Ce que nous avions aimé était vraiment aimable ; ce que nous avions admiré était admirable. Nous n’avons pas changé nos jugements sur {\persName Gœthe}, sur {\persName Herder}. Est-ce notre faute si, en restant fidèles à nos anciens jugements, nous nous trouvons un peu dépaysés en présence de ce qu’on proclame maintenant comme un nouvel idéal ? Ceux qui se sont plaints de la sévérité des critiques de {\persName Valbert} n’oublient qu’une chose, c’est que {\persName Valbert} ne dit rien sur l’{\orgName Allemagne} qu’il n’ait appris à son école. Oui, je ne crains pas de le dire, Monsieur, c’est votre ancienne éducation allemande qui vous a fait Français en 1870. C’est ce haut idéalisme de {\persName Kant} et de {\persName Fichte} qui vous a donné la force de regarder en face le succès, de le critiquer et de vous constituer par libre choix l’avocat des vaincus. C’est là, de notre temps. Monsieur, de toutes les tâches la plus ingrate. Le monde maintenant aime les forts ; il est toujours porté à croire que les forts ont raison. Le faible, au contraire, aux yeux d’une opinion basse, est d’avance condamné ; on s’expose à paraître singulièrement attardé en prenant sa défense. Et voilà pourtant, Monsieur, ce que votre courageux instinct de vieux Français vous a conseillé de faire. Vous êtes allé à l’encontre des applaudissements du monde entier. Il est si facile de se faire applaudir quand on a la toute-puissance ! Le victorieux est sûr de l’enthousiasme du grand nombre. Ses fautes deviennent bien vite des traits de génie ; ses maladresses, de profonds calculs ; ses reculades, des prodiges d’habileté. Ah ! qu’il est commode d’être le publiciste des causes triomphantes ! Qu’il est aisé de se passer de talent quand on a derrière soi la force ! Vous avez préféré la tâche difficile, Monsieur, celle où l’on ne réussit qu’à force de tact et d’esprit. Vous avez choisi exprès d’être démodé, je veux dire libéral, juste, humain. De notre temps, la dureté est une qualité si prisée ! Que n’avez-vous fait comme tout le monde, raillé la chevalerie, fait des gorges chaudes de la générosité ? Plaisant propos que le nôtre, de vouloir être de l’âge d’or en un siècle de fer ! Que nous serions forts, Monsieur, si nous pouvions nous compléter d’un peu de méchanceté ! Il est vrai que nous n’y réussirions pas ; n’est pas {\persName Méphistophélès} qui veut ; la méchanceté est ce qu’on réussit le moins à se donner.\par
Et ce qui ajoute à votre mérite, c’est que ce dévouement à la justice devait nécessairement rester sans récompense. Le vaincu ne distribue pas de couronnes. Ses mandats sont des charges, et, à part notre compagnie, qui aime à réparer les torts du public, il n’y aura guère que votre conscience qui vous saura gré de ce que vous avez fait. Le client que vous défendez ne vous soumettra pas un seul de ses actes. Cette France dont vous plaidez si chaleureusement la cause n’aura pas l’idée de vous consulter sur telle ou telle des questions que vous savez si bien.\par
Est-ce vous que je plains de cet oubli ? Non certes, Monsieur. Rien n’a manqué à votre bonheur ; une vie honorable, tous les plaisirs de l’esprit, toutes les joies de l’intérieur, que vous faut-il de plus ? C’est le pays que je plains. La foule ne voit pas les difficultés de la politique ; elle ne peut comprendre à quel point, dans certaines situations, il faut être habile pour ne pas commettre de faute mortelle. La foule veut gouverner ; le mandat impératif, plus ou moins déguisé, est au fond de toutes ses erreurs. Et voilà, Monsieur, pourquoi nous lui inspirons si peu de confiance. Il y a en nous quelque chose qui lui échappe, qui nous échappe à nous-mêmes, quelque chose dont nous ne sommes pas les maîtres : c’est l’esprit qui souffle où il veut. Avec une sagacité instinctive, l’homme imbu des préjugés démocratiques voit que sans cesse nous nous déroberions à ce qu’il tient pour des dogmes. Il sent que nous avons une maîtresse, sur le moindre signe de laquelle nous serions infidèles à tout le reste : c’est l’idéal, la raison, le mandat impératif de notre conscience, lequel rend impossible tous les autres. Le suffrage universel n’a donc pas tout à fait tort quand il se défie de nous. Nous ne saurions servir deux maîtres. Nous sommes les hommes liges d’un souverain qui nous traîne où il lui plaît ; selon le langage reçu, nous serions vite des traîtres…, traîtres à tout, en effet, excepté à notre devoir.\par
Telle est la condition toute nouvelle que notre siècle a faite au patriotisme. Loin de nous les paroles d’amertume ! Elles seraient plus qu’inopportunes, elles seraient l’injustice même, à un moment où nous voyons tant d’hommes de cœur dépenser ce qu’ils ont de raison et de chaleur d’âme pour le bien public. Nos craintes portent sur l’avenir. Des pressentiments, venant sans doute d’une sollicitude inquiète par excès d’amour, nous font entrevoir un temps où l’homme cultivé devra aimer une patrie aux conseils de laquelle il aura peu de part, comme {\persName Fénelon}, {\persName Beauvilliers} aimèrent une monarchie qui ne les écoutait pas et dont ils connaissaient les fautes. L’idéalisme est habitué à ces injustices. L’âge brillant où la politique fut conçue comme un genre de littérature n’a été peut-être qu’une erreur passagère. Le monde, par moments, laisse entendre aux gens d’esprit qu’il n’a pas besoin d’eux, que même ses affaires ne vont jamais mieux que quand ils ne s’en occupent pas. Tout cela serait parfait si les choses humaines n’exigeaient ni prévoyance ni sagesse. Mais jusqu’ici on n’a pas trouvé moyen de faire voguer un navire à pleines voiles par les mers les plus dangereuses sans pilote ni commandement.\par
On raconte que, quand la ville d’{\placeName Antioche} fut prise par les {\orgName Perses} sous {\placeName Valérien}, toute la population se trouvait rassemblée au théâtre. Les gradins de ce théâtre étaient taillés dans le pied de la montagne escarpée que couronnaient les remparts. Tous les yeux, toutes les oreilles étaient tendus vers l’acteur, quand tout à coup celui-ci se met à balbutier ; ses mains se crispent, ses bras se paralysent, ses yeux deviennent fixes. De la scène où il était, il voyait les Perses, déjà maîtres du rempart, descendre la montagne au pas de course. En même temps, les flèches commencèrent à pleuvoir dans l’enceinte du théâtre et rappelèrent les spectateurs à la réalité.\par
Notre situation est un peu celle de l’acteur d’{\placeName Antioche}, Monsieur. Nous voyons ce que la foule ne voit pas. Cette patrie française, construite au prix de mille ans d’héroïsme et de patience, par la bravoure des uns, par l’esprit des autres, par les souffrances de tous, nous la voyons guidée par une conscience insuffisante, qui ne sait rien d’hier et ne se doute pas de demain. Comme il arrive dans les passages difficiles de montagnes, nous voyons ce que nous avons de plus cher vaciller au bord du précipice, se balancer sur le vide, confié au pas irresponsable d’un être instinctif ! Ah ! chère patrie française ! Ceux qui tremblent sont ceux qui aiment. Ses vrais ennemis sont les présomptueux qui flattent ses défauts, enchérissent sur ses erreurs, et qui, sûrs d’avance de l’amnistie des imprévoyants, se montreraient, le lendemain des désastres, frais, légers, alertes, prêts à recommencer. Une nation ne peut durer si elle ne tire de son sein la quantité de raison suffisante pour prévenir les causes de ruine extérieure ou de relâchement intérieur qui la menacent. Les anciens organismes y pourvoyaient d’une manière qui ne suffisait pas toujours pour faire éviter de grandes fautes, mais qui suffisait pour assurer la continuité de l’existence. La question est de savoir si les formes nouvelles où l’on a renfermé la vie nationale n’amèneront pas pour le cerveau de la {\orgName France} de funestes moments d’étourdissement, de passagères anémies.\par
Je dis passagères ; car il n’est pas possible qu’un pays qui possède dans son sein tant d’esprit, tant de cœur, tant de force de travail, une telle somme de conscience et d’honnêteté, ne surmonte pas les germes de maladie qu’il porte en lui. Les dix justes qui auraient pu sauver Sodome eussent pesé d’un poids bien léger, les jours d’élection, dans les scrutins de cette ville coupable, et pourtant, au jour solennel où l’Éternel compte les siens, ils auraient suffi pour faire absoudre la cité entière. Finissons donc par l’espérance, Monsieur. Oui, nous la reverrons encore avant de mourir (vous surtout qui êtes plus jeune que moi), cette vieille {\orgName France} rétablie dans des conditions de vie séculaire, avec ses haines pacifiées, ses horizons rouverts, les ombres de ses victimes apaisées, ses gloires réconciliées. Nous la verrons telle qu’elle fut en ses beaux jours, forte, modérée, raisonnable, relevant le drapeau abandonné du progrès libéral, nullement corrigée de son amour désintéressé pour le bien, instruite cependant par l’expérience et attentive à éviter certaines erreurs où l’indulgence trompeuse du monde, au moins autant que ses défauts, l’avaient engagée.\par
Heureux les patriotes sincères qui, ce jour-là, pourront se rendre le même témoignage que {\persName M. Dufaure} à son lit de mort : \emph{« Nous avons l’ait ce que nous devions faire ! »} {\persName Valbert} au moins sera du nombre. Il s’est plu, comme le {\persName Samaritain}, à soigner le blessé que les voleurs avaient laissé pour mort sur la route. Il a versé dans ses plaies l’huile et le vin. Aujourd’hui, quelques-unes des plaies du blessé sont cicatrisées. Vous nous aiderez. Monsieur, par votre jugement viril, droit et ferme, à continuer l’œuvre de guérison commencée. La tradition d’une discipline nationale ne saurait plus être l’attachement exclusif à certaines idées. Les idées sont maintenant l’élaboration commune de toutes les nations civilisées ; mais chaque pays se les approprie selon son goût et son génie. Nous sommes heureux, Monsieur, que, vous qui avez pu comparer toutes les formes de l’esprit humain et qui avez procédé par choix libre en ce qui d’ordinaire est réglé par le fait de la naissance, vous ayez jugé que la forme française, dont on dit aujourd’hui tant de mal, a aussi de réels avantages. Nous sommes fiers surtout que vous ayez prouvé par votre exemple qu’on peut s’en servir pour exprimer des pensées vraies, fines, généreuses et sensées.
\section[{Réponse au discours de réception de M. de Lesseps}]{Réponse au discours de réception de {\persName M. de Lesseps}}\renewcommand{\leftmark}{Réponse au discours de réception de {\persName M. de Lesseps}}


\dateline{23 avril 1885}

\salute{Monsieur,}
\noindent Votre discours est charmant ; car il est bien vous-même. Savez-vous qu’elle était par moments notre inquiétude pendant que vous le composiez ? C’est que, pour cette circonstance, assez exceptionnelle en votre vie, vous ne vous crussiez obligé de faire une composition littéraire. Votre tact exquis vous a préservé de cette faute. J’ai retrouvé, dans le ton de vos paroles, la bonhomie, la chaleur communicative, qui font l’agrément de vos conversations. J’ai regretté l’absence de quelques traits qui vous sont familiers, de certains détails, par exemple, que vous savez sur {\persName Abraham} et {\persName Sara}, de renseignements inédits que vous possédez sur {\persName Joseph} et la {\persName reine de Saba}. Une foule de choses que vous connaissez mieux que personne manquent en votre discours ; mais rien de vous n’y manque. Vous avez la première des qualités littéraires et la plus rare de notre temps, le naturel ; jamais vous n’avez déclamé. Votre éloquence est cette mâle et piquante manière de se mettre en rapport avec le public, que l’{\orgName Angleterre} et l’{\orgName Amérique} ont créée. Personne assurément, en notre siècle, n’a plus persuadé que vous ; personne n’a été par conséquent plus éloquent ; et cependant personne n’est plus étranger aux artifices du langage, à ces vaines curiosités de la forme que ne connaît pas une ardente conviction.\par
\emph{« J’approuve, dites-vous quelque part, qu’on enseigne le grec et le latin à nos enfants ; mais ce qu’il ne faut pas négliger, c’est de leur apprendre à penser sagement et à parler bravement. »} Ah ! voilà ce que j’aime. Vous avez horreur de la rhétorique, et vous avez bien raison. C’est, avec la {\itshape poétique}, la seule erreur des {\orgName Grecs}. Après avoir fait des chefs-d’œuvre, ils crurent pouvoir donner des règles pour en faire : erreur profonde ! Il n’y a pas d’art de parler, pas plus qu’il n’y a d’art d’écrire. Bien parler, c’est bien penser tout haut. Le succès oratoire ou littéraire n’a jamais qu’une cause, l’absolue sincérité. Quand vous enthousiasmez une réunion et que vous réussissez à séduire la chose du monde la plus sourde aux métaphores, la plus réfractaire aux artifices de l’art prétendu de bien dire, le capital, ce n’est pas votre parole, c’est votre personne qui plaît ; ou plutôt vous parlez tout entier ; vous charmez ; vous avez ce don suprême qui fait les miracles, comme la foi, et qui est, à vrai dire, du même ordre. Le charme a ses motifs secrets, mais non ses raisons définies. C’est une action toute de l’âme. Vous avez les mêmes succès à {\placeName Chicago}, une ville qui n’a pas le tiers de votre âge, que dans nos vieilles villes d’{\placeName Europe}. Vous entraînez le Turc, l’Arabe, l’Abyssin, le spéculateur de {\placeName Paris}, le négociant de {\placeName Liverpool}, par des raisons qui ne sont différentes qu’en apparence. La vraie raison de votre ascendant, c’est qu’on devine en vous un cœur sympathique à tout ce qui est humain, une passion véritable pour l’amélioration du sort des êtres. On trouve en vous ce \emph{Misereor super turbas}, « J’ai pitié des masses », qui est le sentiment de tous les grands organisateurs. On vous aime, on veut vous voir, et, avant que vous ayez ouvert la bouche, on vous applaudit. Vos ennemis appellent cela votre habileté. Nous autres, nous appelons cela votre magie. Les âmes ordinaires ne comprennent pas la séduction des grandes âmes. La fascination du magicien échappe aux pesées vulgaires ; les qualités enchanteresses sont un don gratuit, une grâce octroyée, et, parce qu’elles sont impondérables, la médiocrité les nie. Or, c’est l’impondérable qui existe vraiment. L’humanité sera toujours menée par de secrets philtres d’amour, dont la foule ne voit que les effets superficiels ; comme la raison dernière du monde physique est dans des fluides invisibles que l’œil ordinaire ne sait pas discerner.\par
Votre éloquence a gagné le monde ; comment ne vous eût-elle pas mérité une place parmi nous ? Le programme de notre {\orgName Compagnie} n’est pas une simple culture littéraire, poursuivie pour elle-même, et n’aboutissant qu’à de frivoles jeux, à peine supérieurs à ces difficiles enfantillages où se sont perdues les littératures de l’{\placeName Orient}. Ce sont les choses qui sont belles ; les mots n’ont pas de beauté en dehors de la cause noble ou vraie qu’ils servent. Qu’importe que {\persName Tyrtée} ait eu ou n’ait pas eu de talent ? Il a réussi ; il valut une armée. La \emph{Marseillaise}, quoi qu’en disent les musiciens et les puristes, est le premier chant des temps modernes, puisqu’à son jour elle entraîna les hommes et les fit vaincre. Le mérite personnel, à cette hauteur, est peu de chose ; tout dépend de la prédestination, ou si l’on veut, du succès. Il ne sert de rien de dire qu’un général aurait dû gagner une bataille, s’il la perd. Le grand général (et on en peut dire presque autant du grand politique) est celui qui réussit, et non celui qui aurait dû réussir.\par
Les personnes qui, un moment, ont été surprises de votre élection connaissaient donc bien peu l’esprit de notre {\orgName Compagnie}. Vous avez cultivé le plus difficile des genres, un genre depuis longtemps abandonné parmi nous, la grande action ; vous avez été du petit nombre de ceux qui ont gardé la vieille tradition française de la vie brillante, glorieuse, utile à tous. La politique et la guerre sont de trop hautes applications de l’esprit pour que nous les ayons jamais négligées. Le {\persName maréchal de Villars}, le {\persName maréchal de Belle-Isle}, le {\persName maréchal de Richelieu}, le {\persName maréchal de Beauvau} n’avaient pas plus de titres littéraires que vous. Ils avaient remporté des victoires. A défaut de ce titre, devenu rare, nous avons pris le maître par excellence en fait de difficulté vaincue, le joueur hardi qui a toujours gagné son pari dans la poursuite du probable, le virtuose qui a pratiqué avec un tact consommé le grand art perdu de la vie. Si {\persName Christophe Colomb} existait chez nous de nos jours, nous le ferions membre de l’{\orgName Académie}. Quelqu’un qui est bien sûr d’en être, c’est le général qui nous ramènera un jour la victoire. En voilà un que nous ne chicanerons pas sur sa prose, et qui nous paraîtra tout d’abord un sujet fort académique. Comme nous le nommerons par acclamation, sans nous inquiéter de ses écrits ! Oh ! la belle séance que celle où on le recevra ! Comme les places y seront recherchées ! Heureux celui qui la présidera !…\par
Vous avez été de ces collaborateurs de la fortune, qui semblent avoir reçu la confidence de ce que veut, à une heure donnée, le génie de la civilisation. Le premier des devoirs que l’homme a dû s’imposer pour devenir vraiment maître de la planète qu’il habite, a été de redresser, en vue de ses besoins, les combinaisons souvent malheureuses que les révolutions du globe, dans leur parfaite insouciance des intérêts de l’humanité, n’ont pu manquer de produire. Les événements les plus importants de l’histoire se sont passés avant l’histoire. Quel eût été le sort de notre planète, si les parties émergentes eussent été infiniment plus petites qu’elles ne sont, si le champ d’évolution de la vie terrestre n’eût pas été plus grand que l’{\placeName île de Pâques} ou {\placeName Tahiti} ? Quel fait historique a jamais égalé en conséquences le coup de mer qui opposa un jour le {\placeName cap Gris-Nez} aux {\placeName falaises de Douvres}, et créa la {\placeName France} et l’{\placeName Angleterre} en les séparant ? Parfois bienfaisants, ces hasards d’une nature imprévoyante sont aussi bien souvent funestes, et alors le devoir de l’homme est, par des retouches habiles, de corriger les mauvais services que les forces aveugles lui ont rendus.\par
On a dit, non sans quelque raison, que, si l’astronomie physique disposait de moyens assez puissants, on pourrait juger du degré plus ou moins avancé de la civilisation des mondes habités, à ce critérium que leurs isthmes seraient coupés ou ne le seraient pas. Une planète n’est, en effet, mûre pour le progrès que quand toutes ses parties habitées sont arrivées à d’intimes rapports qui constituent en organisme vivant ; si bien qu’aucune partie ne peut jouir, souffrir, agir, sans que les autres ne sentent et ne réagissent. Nous assistons à cette heure solennelle pour la {\placeName Terre}. Autrefois, la {\orgName Chine}, le {\orgName Japon}, l’{\orgName Inde}, l’{\orgName Amérique} pouvaient traverser les révolutions les plus graves, sans que l’{\orgName Europe} en fût même informée. L’{\placeName Atlantique}, pendant des siècles, divisa la terre habitable en deux moitiés aussi étrangères l’une à l’autre que le sont deux globes différents. Aujourd’hui les {\orgName Bourses} de {\placeName Paris} et de {\placeName Londres} sont émues de ce qui se passe à {\placeName Pékin}, au {\placeName Congo}, au {\placeName Kordofan}, en {\placeName Californie} ; il n’y a presque plus de parties mortes dans le corps de l’humanité. Le télégraphe électrique et la téléphonie ont supprimé la distance en ce qui concerne la communication des esprits ; les chemins de fer et la navigation à vapeur ont décuplé les facilités pour le transport des corps. N’était-il pas inévitable que le siècle regardât comme une partie essentielle de sa tâche de faire disparaître les obstacles qui gênaient ses communications rapides ? Était-il possible que la génération qui devait percer le {\placeName Cenis}, le {\placeName Gothard}, s’arrêtât devant quelques bancs de sable ou de rocher à {\placeName Suez}, à {\placeName Corinthe}, à {\placeName Panama} ?\par
C’est vous, Monsieur, qui avez été l’artisan élu pour cette grande œuvre. L’{\placeName isthme de Suez} était depuis longtemps désigné comme celui dont la section était la plus urgente. L’antiquité l’avait voulue et tentée par des moyens insuffisants. {\persName Leibniz} désignait cette entreprise à {\persName Louis XIV} comme digne de sa puissance. Mais il fallait pour une telle œuvre une croyance à l’instinct que le XVII\textsuperscript{e} siècle n’avait pas. Ce fut la Révolution française qui, en ramenant l’âge des expéditions fabuleuses et un état d’enfance héroïque où l’homme, dans ses aventures, s’inspire du vol des oiseaux et des signes au ciel, posa le problème de telle manière qu’il ne fut plus possible de le laisser dormir. Le percement de l’isthme figurait au programme que le {\orgName Directoire} donna à l’expédition d’{\placeName Égypte}. Comme au temps d’{\persName Alexandre}, la conquête des armes fut une conquête de la science. Le 24 décembre 1798, notre illustre confrère, le {\persName général Bonaparte}, partait du {\placeName Caire}, accompagné de {\persName Berthier}, de {\persName Monge}, de {\persName Berthollet}, de quelques autres membres de l’{\orgName Institut}, et de négociants qui avaient obtenu de marcher dans son escorte. Le 30, il retrouvait, au nord de {\placeName Suez}, les vestiges de l’ancien canal, et il les suivait pendant cinq lieues ; le 3 janvier 1799, il voyait, près de {\placeName Belbeys}, l’autre extrémité du {\placeName canal des Pharaons}. Les recherches de la {\orgName commission d’Égypte} ont été la base de tous les travaux postérieurs. Une seule erreur, celle de l’inégalité de niveau des deux mers, toujours combattue par {\persName Laplace} et {\persName Fourier}, se mêla à des recherches précieuses et retarda d’un demi-siècle l’exécution de l’œuvre rêvée par les ingénieurs héroïques de 1798.\par
Cette grande école saint-simonienne, qui eut un si haut sentiment du travail commun de l’humanité releva l’idée, se l’appropria par le martyre. Plus de douze ingénieurs saint-simoniens moururent de la peste en 1833, au barrage du {\placeName Nil}. À travers plusieurs chimères, une vérité était entrevue, je veux dire la place exceptionnelle de l’{\orgName Égypte} dans l’histoire du monde. Clef de l’{\orgName Afrique intérieure}, par le {\placeName Nil} ; par son isthme, gardienne du point le plus important de l’empire des mers, l’{\orgName Égypte} n’est pas une nation ; c’est un enjeu, tantôt récompense d’une domination maritime légitimement conquise, tantôt châtiment d’une ambition qui n’a pas mesuré ses forces. Quand on a un rôle touchant aux intérêts généraux de l’humanité, on y est toujours sacrifié. Une terre qui importe à ce point au reste du monde ne saurait s’appartenir à elle-même ; elle est neutralisée au profit de l’humanité ; le principe national y est tué. Nous nous étonnons de voir apparaître, parmi les folles pensées qui se croisent dans la tête de {\persName Néron}, durant les heures qui séparent sa chute de sa mort, l’idée d’aller se présenter au peuple en habits de deuil et de lui demander, en échange de l’empire, la préfecture de l’{\placeName Égypte}. C’est que la préfecture de l’{\placeName Égypte} sera toujours un lot à part ; le souverain de l’{\placeName Égypte} ne s’appellera jamais du même nom que les autres souverains. L’{\orgName Égypte} sera toujours gouvernée par l’ensemble des nations civilisées. L’exploitation rationnelle et scientifique du inonde tournera sans cesse vers cette étrange vallée ses regards curieux, avides ou attentifs.\par
La {\orgName France}, pendant trois quarts de siècle, a eu pour ce difficile problème une solution qu’on admirera quand l’expérience aura montré combien les autres solutions coûteront au monde de larmes et de sang. Elle imagina, par une dynastie musulmane en apparence, mais au fond sans fanatisme et prompte à reconnaître la supériorité de l’{\orgName Occident}, de faire régner l’esprit moderne sur cette terre exceptionnelle, qui ne saurait, sans un détriment extrême du bien général, appartenir à la barbarie. Par l’{\placeName Égypte}, ainsi organisée et encadrée, la civilisation avait la main sur tout le {\placeName Soudan oriental}. Les dangereux cyclones que produira périodiquement l’{\orgName Afrique centrale}, depuis qu’on a eu l’Imprudence de la laisser se faire musulmane, étaient réprimés. La science européenne avait ses libres allures en un pays qui lui est en quelque sorte dévolu comme champ d’étude et d’expérience. Mais on aurait dû porter dans ce plan excellent quelque conséquence. Il fallait ne pas affaiblir une dynastie par laquelle la pointe de l’épée de l’{\orgName Europe} pénétrait presque jusqu’à l’{\placeName équateur}. Il fallait surtout surveiller la {\placeName mosquée Et-Azliar}, centre d’où la propagande musulmane s’étendait sur toute l’{\placeName Afrique}. Isolées et livrées au fétichisme, les races soudaniennes sont peu de chose, mais, converties à l’islam, elles deviennent des foyers de fanatisme intense. Faute de prévoyance, on a laissé se former à l’ouest du {\placeName Nil} une {\placeName Arabie} bien plus dangereuse que l’{\placeName Arabie} véritable. N’êtes-vous pas frappé, Monsieur, qu’il n’y ait encore aucun sens commun des grands intérêts du monde ? C’est à croire vraiment qu’il y a un ange gardien pour l’humanité, qui l’empêche de tomber dans tous les fossés du chemin ? S’il n’y avait que les diplomates, j’aimerais autant voir notre pauvre espèce confiée à la prudence d’une bande d’écoliers ayant pris la clef des champs.\par
L’origine de votre entreprise se rattache aux débuts de cette dynastie de {\persName Méhémet-Ali}, née sous les auspices de la {\placeName France}, et que, par contre-coup, un abaissement passager de la fortune de la {\orgName France} a dû faire chanceler. Monsieur votre père fut le premier agent français qui résida en {\placeName Égypte} après le départ de notre armée. Il était chargé par le premier consul et par {\persName M. de Talleyrand} de contre-balancer la tyrannie des {\orgName mamelouks}, appuyée par l’{\orgName Angleterre}. Le chef des janissaires de monsieur votre père lui amena un jour, comme capable de s’opposer à l’anarchie, un jeune Macédonien qui commandait alors mille Albanais, et sur qui l’expédition française avait fait la plus vive impression. Ce compatriote d’{\persName Alexandre} ne savait encore ni lire ni écrire. Sa fortune grandit rapidement, et, comme il n’oubliait rien de ce qu’on avait fait pour lui ou contre lui, quand vous arrivâtes en Égypte, dans les premiers mois de 1832, en qualité d’élève consul, le puissant vice-roi vous distingua tout d’abord. {\persName Mohammed-Saïd}, un de ses fils, fut votre ami de jeunesse. Vous prîtes sur lui un empire étrange, et quand il monta sur le trône, vous régnâtes avec lui. Il touchait par vous quelque chose de supérieur, qu’il ne comprenait qu’à demi, tout un idéal de lumière et de justice dont son âme ardente avait soif, mais que de sombres nuages, sortant d’un abîme séculaire de barbarie, voilaient passagèrement à ses yeux.\par
Vous avez raconté, avec le parfait naturel qui n’appartient qu’à vous, cette liaison qui a eu pour le monde des conséquences si graves, ces alternatives bizarres d’emportement et de raison, cet enthousiasme de la science au sortir de l’ignorance la plus absolue, ces torrents de larmes succédant à des crises de fureur ; puis des éclats de rire, des mouvements de folle vanité ; dans le même cerveau, enfin, la lutte d’un {\persName Tamerlan} et d’un {\persName Marc Aurèle}. Votre récit du féerique voyage que vous fîtes avec {\persName Saïd} dans le {\placeName Soudan} est un document sans égal pour la psychologie de l’Oriental. Tantôt vous surpreniez votre compagnon de voyage plongé dans une morne tristesse, par suite de son impuissance à soulever un monde de bassesse et d’abus ; tantôt vous le trouviez en proie à des accès de frénésie. Il se levait alors, tirait son sabre, le jetait le plus loin possible, par crainte du sauvage qu’il sentait en lui. La nuit le calmait, et, quand vous reveniez le voir le matin, vous le trouviez dans la désolation, se reprochant de n’avoir pas eu le premier les excellentes idées de progrès et de réforme que vous lui aviez suggérées. Vous inspiriez à cet impétueux despote un respect singulier. Vous contenter était son but suprême ; il ne voyait que vous au monde. Le barbare est toujours un enfant, et cette amitié était un verre que la moindre jalousie pouvait fêler. Vous le sentiez ; votre esprit riche et souple parait à tout. Il n’y a que les fortes natures qui sachent traiter avec les barbares. {\persName Saïd} avait emporté pour lui un service de {\placeName Sèvres}, et vous en avait donné un autre pour votre usage personnel. Le service du vice-roi, faute de soins, fut bientôt en morceaux, pendant que le vôtre était intact. Il n’était pas bon que cela durât. Un jour le chameau bien dressé qui portait votre vaisselle se trouva remplacé par un chameau très vif et presque sauvage. Vous n’eûtes garde de réclamer. Au bout de quelques minutes, votre service de {\placeName Sèvres} volait en pièces. Le vice-roi éclata de rire, et l’œuvre de l’isthme fut sauvée.\par
Dès cette époque, en effet, le percement de l’{\placeName isthme de Suez} était votre constante préoccupation, et vous aviez à peu près réussi à en faire adopter l’idée à votre tout-puissant ami. Vos vues à cet égard dataient d’un incident de votre arrivée en {\placeName Égypte}. Vous veniez d’un pays parfaitement sain ; vous entriez dans un pays plein de maladies ; et, selon une logique qui n’a pas changé jusqu’à nos jours, on vous fît faire une longue quarantaine à {\placeName Alexandrie}. Le consul, {\persName M. Mimault}, pour diminuer l’ennui de votre séquestration, vous apporta quelques volumes du grand ouvrage de la {\orgName commission d’Égypte}, en vous recommandant surtout le mémoire de {\persName Lepère} sur la jonction des deux mers. C’est ainsi que vous fîtes connaissance avec l’isthme et son histoire. Dès lors l’ambition de réaliser ce que d’autres avaient rêvé s’empara de vous. Vous dûtes attendre vingt-trois ans. Mais rien ne vous rebuta ; vous étiez né perceur d’isthme ; l’antiquité eût fait un mythe à votre sujet ; vous êtes l’homme de notre siècle sur le front duquel se lit le plus clairement le signe d’une vocation absolue.\par
Le principe de la grande action, c’est de prendre la force vive où elle est, de l’acheter au prix qu’elle coûte et de savoir s’en servir. Dans l’état présent du monde, la barbarie est encore un dépôt énorme de forces vives. Votre intelligence si ouverte comprit qu’il y a une puissance immense entre des mains incapables de s’en servir et que cette puissance appartient à qui sait la prendre. Vous acceptez bravement les choses humaines comme elles sont. Le contact de la sottise et de la folie ne vous déplaît pas. Libre à celui qui ne touche pas les réalités de la vie de faire le difficile et de rester immaculé. L’humanité se compose de deux milliards de pauvres créatures, ignorantes, bornées, avec lesquelles une élite marquée d’un signe est chargée de faire de la raison, de la justice, de la gloire. Arrière les timides et les délicats, arrière les dégoûtés, qui ont la prétention de sortir sans une tache de boue de la bataille engagée contre la sottise et la méchanceté ! Ils ne sont pas propres à une œuvre pour laquelle il faut plus de pitié que de dégoût, un cœur haut et fier, la grande bonté, souvent assez différente de la philantrophie superficielle, quelque chose enfin du sentiment large de {\persName Scipion l’Africain}, répondant à je ne sais quelle chicane : \emph{« À pareil jour, j’ai gagné la bataille de Zama : montons au {\placeName Capitole} et rendons grâce aux dieux. »} Vous devez, en grande partie, à l’{\placeName Orient}, ces allures de cheval arabe, qui ont parfois surpris vos amis plus timides que vous. L’Orient inspire le goût des grandes aventures ; car, en {\placeName Orient}, l’ère des grandes aventures fécondes dure encore. La vue d’un troupeau sans pasteur inspire l’idée de se mettre à sa tête. Que de fois, en {\placeName Syrie}, j’ai porté envie au sous-lieutenant qui m’accompagnait ! Celui qui fondera l’ordre et la civilisation en {\placeName Orient} grandit peut-être maintenant dans quelque école de cadets. Vous évitez, dans votre appréciation des hommes, les étroits jugements des idéologues à outrance, qui croient que toutes les races se valent, et des théoriciens cruels, qui ne voient pas la nécessité des humbles dans la création. Ces gens du {\placeName lac de Menzaleh}, qui ont construit les berges de votre canal en recueillant la vase dans leurs larges mains, et en la pressant pour l’égoutter contre leur poitrine, auront leur place dans le royaume de Dieu. Inférieures, oui certes, elles le sont, ces pauvres familles humaines si cruellement trahies par le sort ; mais elles ne sont pas pour cela exclues de l’œuvre commune. Elles peuvent produire des grands hommes ; parfois d’un bond elles nous dépassent ; elles sont capables de prodiges d’abnégation et de dévouement. Telles qu’elles sont, vous les aimez. Vous êtes optimiste, Monsieur, et vous avez raison. L’art suprême est de savoir faire du bien avec du mal, du grand avec du médiocre. On l’emporte à ce jeu transcendant par la sympathie, par l’amour qu’on a pour les hommes et par celui qu’on leur inspire, par l’audace avec laquelle on s’affirme à soi-même que la cause du progrès est gagnée et qu’on y sert.\par
L’Oriental veut avant tout être charmé. Vous y réussissiez à merveille. Votre franchise, votre aisance inspiraient une confiance sans bornes. {\persName Saïd} ne pouvait vivre sans vous. Votre étonnante habileté à monter à cheval vous gagnait l’amitié de la vieille école de {\persName Méhémet-Ali}, plus rompue à ces sortes d’exercices qu’à ceux de l’esprit. Le 30 novembre 1854, vous étiez avec {\persName Saïd} en plein désert. La tente du vice-roi était placée sur une éminence en pierres sèches. Vous aviez remarqué qu’il y avait un endroit où l’on pouvait sauter à cheval par-dessus le parapet. Ce fut là le chemin que vous choisîtes. Franchement, Monsieur, vous auriez dû vous rompre le cou ; mais les folies en {\placeName Orient} servent autant que la sagesse. Votre hardiesse excita l’universelle admiration, et, ce jour-là même, la charte de concession vous fut octroyée. {\persName Saïd} considéra, désormais, le percement de l’isthme comme son œuvre propre ; il y mit sa ténacité d’enthousiaste, sa vanité de barbare. Un mois après, vous partiez pour la première exploration du désert, sur lequel vous alliez remporter en quinze ans une victoire si décisive.\par
Ces quinze années furent comme un rêve, digne d’être raconté dans les \emph{Prairies d’or de Massoudi} ou dans les \emph{Mille et une Nuits}. Votre ascendant sur ce monde étrange de grandeur et d’énergie inculte fut vraiment inouï. Vous étonniez {\persName M. Barthélémy Saint-Hilaire} ; il ne vous suivait plus. En fait, vous avez été roi ; vous avez eu les avantages de la souveraineté ; vous avez appris ce qu’elle apprend, l’indulgence, la pitié, le pardon, le dédain. J’ai vu votre royauté dans le désert. Pour traverser le {\placeName Ouadi de Zagazig} à {\placeName Ismaïlia}, vous m’aviez donné un de vos sujets. C’était, je crois, un ancien brigand, que vous aviez rattaché momentanément à la cause de l’ordre. En m’expliquant le maniement d’un vieux tromblon du XVI\textsuperscript{e} siècle, qui faisait partie de son arsenal, il m’exposait ses sentiments les plus intimes, qui se résumaient en une admiration sans bornes pour vous. Vous aviez vos fidèles, je dirai presque vos fanatiques, quelquefois dans le camp de ceux qu’on devait supposer vos ennemis. A {\placeName Ismaïlia}, nous rencontrâmes une dame anglaise, qui suivait d’un œil avide le travail de vos ouvriers, pour voir si les prophéties de la Bible n’en recevaient pas de confirmation. Elle nous mena voir quelques touffes d’herbes et de fleurs que les infiltrations du canal d’eau douce avaient fait pousser dans le sable. Cela lui paraissait démonstratif ; n’est-il pas écrit, en effet, que, à la veille du grand avènement de l’âge messianique, « le désert fleurira » ? (\emph{Isaïe}, ch. 35.) Vous aviez pour tous une chimère toute prête ; vous fournissiez à tous un rêve, à chacun le rêve selon son cœur.\par
Le mot de religion n’est pas trop fort pour exprimer l’enthousiasme que vous excitiez. Votre œuvre fut durant quelques années une sorte de bonne nouvelle et de rédemption, une ère de grâce et de pardon. Les idées de réhabilitation, d’amnistie morale tiennent toujours une grande place dans les origines religieuses. Le bandit est reconnaissant à quiconque vient prêcher un jubilé, qui remet les choses en leur place. Vous étiez bon pour tous ceux qui s’offraient ; vous leur faisiez sentir que leur passé serait effacé, qu’on était absous, reclassé dans la vie morale, pourvu qu’on aimât le percement de l’isthme. Tant de gens sont prêts à s’améliorer, pourvu qu’on veuille bien leur oublier quelque chose ! Un jour, toute une bande de galériens, qui avait réussi à s’échapper de je ne sais quel bagne autrichien des bords de l’{\placeName Adriatique}, vint s’abattre sur l’isthme comme sur une terre de promission. Le {\persName consul d’Autriche} les réclama. Vous fîtes traîner l’affaire en longueur. Au bout de quelques semaines, le {\orgName consulat d’Autriche }n’avait d’autre occupation que d’expédier l’argent que ces braves gens envoyaient à leurs parents pauvres, peut-être à leurs victimes. Le consul alors vous fit prier instamment de les garder, puisque vous saviez tirer d’eux un parti si excellent.\par
Je lis, dans le compte rendu d’une de vos conférences, ce qui suit \emph{: « {\persName M. de Lesseps} a constaté que les hommes sont fidèles, nullement méchants, lorsqu’ils ont de quoi vivre. L’homme ne devient méchant que par peur ou lorsqu’il a faim. »} Il faudrait peut-être ajouter : quand il est jaloux. \emph{« Jamais, disiez-vous, je n’ai eu à me plaindre de mes travailleurs, et j’ai pourtant employé des pirates et des forçats. Tous, par le travail, redevenaient honnêtes ; on ne m’a jamais rien volé, pas même un mouchoir. »} Le fait est qu’on tire tout de nos hommes en leur témoignant de l’estime et en leur persuadant qu’ils travaillent, à une œuvre d’intérêt universel.\par
Vous avez ainsi fait reverdir de nos jours une chose qui semblait flétrie à jamais. Vous avez donné, en un siècle sceptique, une preuve éclatante de l’efficacité de la foi, et vérifié au sens positiviste ces hautes paroles : \emph{« Je vous dis que si vous aviez de la foi pas plus gros qu’un grain de sénevé, vous diriez à cette montagne : Va, et jette-toi dans la mer ; et elle irait. »} Le dévouement que vous obteniez de votre personnel était immense. Je passai une nuit à {\placeName Schalloufet-errabah}, sur le canal d’eau douce, dans une baraque absolument isolée, occupée par un seul de vos employés. Cet homme me frappa d’admiration. Il était sûr de remplir une mission ; il s’envisageait comme une sentinelle perdue en un poste avancé, comme un missionnaire de la {\placeName France}, comme un agent de civilisation. Tous vos fonctionnaires croyaient que le monde les contemplait et avait intérêt à ce qu’ils fissent bien leur devoir.\par
Voilà, Monsieur, ce que notre suffrage a voulu récompenser. Nous sommes incompétents pour apprécier le travail de l’ingénieur ; les mérites de l’administrateur, du financier, du diplomate n’ont pas ici leur juge ; mais nous avons été frappés de l’œuvre morale, de cette résurrection de la foi, non de la foi à un dogme particulier, mais de la foi à l’humanité, à ses brillantes destinées. Ce n’est pas pour l’œuvre matérielle que nous vous couronnons, pour ce ruban bleu, qui nous vaudrait, à ce qu’il paraît, l’estime des habitants de la Lune, s’il y en avait. Non ; là n’est pas votre gloire. Votre gloire, c’est d’avoir provoqué le dernier mouvement d’enthousiasme, la dernière floraison de dévouement. Vous avez renouvelé de nos jours les miracles des jours antiques. Le secret des grandes choses, l’art de se faire aimer, vous l’avez eu au plus haut degré. Vous avez su, avec des masses incohérentes, former une petite armée compacte, où les meilleures qualités de la race française sont apparues dans tout leur jour. Des milliers d’hommes ont eu en vous leur conscience, leur raison de vivre, leur principe de noblesse ou de relèvement.\par
Ce que vous avez dépensé, en cette lutte, de vaillance, de bravoure, de ressources de toute sorte, tient du prodige. Quel trésor de bonne humeur, en particulier, ne vous a-t-il pas fallu pour répondre patiemment à tant d’objections puériles que l’on vous opposait : sables mouvants du désert, vases sans fonds du {\placeName lac de Menzaleh}, menaces d’un déluge universel, amené par l’inégalité de niveau des deux mers ! Pendant les quatre premières années, votre activité n’a pas d’égale ; vous faites par an dix mille lieues, plus que le tour du monde. Il fallait persuader l’{\orgName Europe} ; il fallait surtout convertir l’{\orgName Angleterre}, notre grande et chère rivale. Vous prîtes les mœurs du pays. Vous alliez de ville en ville, avec un seul compagnon de voyage, portant avec vous des cartes colossales, des chargements de brochures et de prospectus. En arrivant dans une ville, vous vous rendiez chez le lord-maire ou chez le personnage le plus important de la localité pour lui offrir la présidence du meeting, puis vous choisissiez le secrétaire ; vous alliez voir les rédacteurs de tous les journaux. En quarante-cinq jours, vous fîtes ainsi trente-deux meetings dans les principales villes des trois royaumes. La nuit se passait à corriger les épreuves du discours de la veille ; vous en emportiez mille exemplaires, que vous distribuiez le lendemain. Vous ne repoussez aucun des moyens dont notre siècle a fait les conditions du succès. Vous ne dédaignez pas la presse, et vous avez raison ; car, à n’envisager que l’effet sur le monde, la manière dont un fait se raconte est plus importante que le fait en lui-même. La presse a remplacé de nos jours tout ce qui autrefois mettait les hommes en rapport les uns avec les autres, la correspondance, la parole publique, le livre, presque la conversation. Renoncer à ce puissant moyen, c’est renoncer à sa part légitime d’action dans les choses humaines. Il y a, je le sais, des personnes puritaines qui se contentent d’avoir raison pour elles-mêmes et qui regardent comme humiliante l’obligation d’avoir raison devant tous. Je respecte infiniment cette manière de voir ; je crains seulement qu’il ne s’y mêle une légère erreur historique. Autrefois, on gagnait le roi et la cour par des procédés de peu supérieurs à ceux par lesquels, de notre temps, on gagne tout le monde. Le gros public voit par son journal ; {\persName Louis XIV} et {\persName Louis XV} voyaient par les étroites idées de leur entourage. Pour arriver à être ministre, {\persName Turgot}, le plus modeste des hommes, n’eut besoin de convaincre de son mérite que quatre personnes : d’abord l’{\persName abbé de Véry}, son condisciple en {\orgName Sorbonne}, homme d’un esprit très éclairé, qui parla de lui avec admiration à une femme très intelligente, {\persName madame de Maurepas}. {\persName Madame de Maurepas} le signala à {\persName M. de Maurepas}, qui le présenta à {\persName Louis XVI}. Certes, il faut plus de candidatures que cela pour arriver par le suffrage universel. Mais voyez le revers. Pour faire tomber le ministre qui seul pouvait sauver la monarchie, il suffit de quelques épigrammes de courtisans et d’un revirement dans les idées de {\persName Maurepas}. Ah ! qu’on ferait un long chapitre des erreurs du suffrage restreint !\par
Notre temps n’est pas plus frivole que les autres. On parle du règne de la médiocrité. Mon Dieu ! Monsieur, qu’il y a longtemps que ce règne-là est commencé ! La somme de raison qui émerge d’une société pour la gouverner a toujours été très faible. Toujours l’homme supérieur qui veut le bien a dû se prêter aux faiblesses de la foule. Pauvre humanité ! Pour la servir, il faut se mettre à sa taille, parler sa langue, adopter ses préjugés, entrer avec elle à l’atelier, au bouge, à l’hôtel garni, au cabaret.\par
Vous avez donc bien fait de ne pas vous arrêter à ces mesquines susceptibilités, qui, si l’on en tenait trop de compte, feraient de l’inaction la suprême sagesse. Les temps sont obscurs ; nous travaillons dans la nuit ; travaillons tout de même. L’{\persName Ecclésiaste} avait mille fois raison de dire que nul ne sait si l’héritier de la fortune qu’il a créée sera sage ou fou. Ce philosophe accompli en conclut-il qu’il ne faut rien faire ? Nullement. Une voix secrète nous pousse à l’action. L’homme fait les grandes choses par instinct, comme l’oiseau entreprend ses voyages guidé par une mystérieuse carte de vieille géographie qu’il porte en son petit cerveau.\par
Vous ne vous êtes pas dissimulé que le percement de l’isthme servirait tour à tour des intérêts fort divers. Le grand mot : \emph{« Je suis venu apporter non la paix, mais la guerre »}, a dû se présenter fréquemment à votre esprit. L’isthme coupé devient un détroit, c’est-à-dire un champ de bataille. Un seul {\placeName Bosphore} avait suffi jusqu’ici aux embarras du monde ; vous en avez créé un second, bien plus important que l’autre, car il ne met pas seulement en communication deux parties de mer intérieure ; il sert de couloir de communication à toutes les grandes mers du globe. En cas de guerre maritime, il serait le suprême intérêt, le point pour l’occupation duquel tout le monde lutterait de vitesse. Vous aurez ainsi marqué la place des grandes batailles de l’avenir. Que pouvons-nous, si ce n’est cerner le champ clos où se choquent les masses aveugles, favoriser de notre sympathie, dans leur effort vers l’existence, toutes ces choses obscures qui gémissent, pleurent et souffrent avant d’être ? Aucune déception ne nous arrêtera ; nous serons incorrigibles ; même au milieu de nos désastres, les œuvres universelles continueront de nous tenter. Le {\persName roi d’Abyssinie} a dit de vous : \emph{« {\persName Lesseps}, qui est de la tribu de la lumière… »} Ce roi, vraiment, parle d’or. Nous sommes tous de cette tribu-là. C’est une règle militaire de marcher au canon, de quelque côté qu’on l’entende. Nous autres, nous avons pour loi de marcher à la lumière, souvent sans bien savoir où elle nous conduit.\par
Vous avez si parfaitement rendu justice à l’homme illustre à qui vous succédez parmi nous, que je n’ai pas à y revenir. C’était un excellent citoyen. Il pensait en tout comme la {\orgName France}. Quand le pays faisait un pas dans le sens de ce qui paraît être sa politique préférée, il le suivait, quelquefois même il le devançait ; mais toujours il était sincère. Le mot d’ordre qu’il semblait recevoir du dehors venait en réalité de lui ; car il était en parfait accord avec le milieu pu il vivait. Il professait tous les préjugés dont se compose l’opinion commune si honnêtement, qu’il arrivait à les prendre pour des vérités primitives et incréées. Mais, comme c’était un vrai libéral, il n’éprouvait aucun regret à voir ses convictions les plus arrêtées faire un stage. Il voulait que le progrès s’accomplît par l’amélioration des esprits et par la persuasion. Il put avoir, comme tout le monde, ses illusions ; jamais il ne s’aveugla que quand le doute lui eût paru un manque de générosité, un péché contre la foi.\par
La meilleure preuve de son ardent patriotisme fut ce grand ouvrage historique qui a marqué sa place parmi nous. La {\orgName France} avait besoin d’une histoire étendue qui, sans remplacer l’étude des sources, présentât à l’homme instruit l’ensemble complet des résultats obtenus par la critique moderne. Pour rédiger une telle histoire, il fallait de l’abnégation. Comme l’a fort bien dit {\persName M. Villemain}, il n’y a pas de chef-d’œuvre en vingt volumes. Un tel livre, en effet, ne pouvait être un livre de style, {\persName M. Augustin Thierry} ne l’aurait jamais fait. Ce ne pouvait non plus être un livre de science ; {\persName M. Léopold Delisle} ne le fera jamais. Il fallait pourtant que le livre existât. Les exquises ou étincelantes fantaisies de {\persName M. Michelet} étaient à la fois plus et moins que l’ouvrage de conscience et de bonne foi réclamé par l’intérêt public. {\persName M. Henri Martin} se dévoua. Il n’ignorait pas que la {\orgName France} et, en général, les pays très littéraires sont injustes pour les œuvres qui se distinguent par la modération et le jugement plutôt que par l’éclat du talent. Il s’assujettit courageusement à une œuvre condamnée d’avance à une foule de défauts. Il renonça aux jouissances de l’écrivain, aux plaisirs intimes du savant. Pour moi, je pense que rien ne vaut un honnête homme, et je trouve qu’il fit bien. Le livre de {\persName M. Henri Martin} est un des plus estimables que notre siècle ait produits. Il est beau de l’avoir écrit ; il est honorable pour un pays d’avoir inspiré le courage de l’écrire.\par
Telle est d’ailleurs l’unité grandiose du sujet que les proportions en éclatent, bien qu’il soit difficile d’embrasser toutes les parties d’un seul coup d’œil. La formation de la {\orgName France} par l’action de la dynastie capétienne est le plus bel exemple de création vivante que présente l’histoire d’aucun pays. Ce n’est pas une concrétion grossière dont les éléments souffrent d’être séparés les uns des autres. Le roi de France est comme le cœur ou, si l’on veut, la tête d’un organisme puissant, où chaque partie vit en solidarité avec le tout. Merveilleuse unité, dont le défaut, si j’ose le dire, fut d’être trop parfaite, puisqu’elle induisit de vrais patriotes à croire, imprudemment peut-être, qu’elle devait nécessairement survivre à la cause qui l’avait formée ! Problème étrange, devant lequel d’autres patriotes non moins sincères gardent un silence douloureux, se demandant avec angoisse si l’unité d’un être vivant, fortement centralisé, peut continuer après l’ablation de la tête ! {\persName Henri Martin} fut de ceux qui osèrent résoudre, d’après les aspirations de leur cœur et avec leur confiance absolue dans l’étoile de la {\orgName France}, une question sur laquelle le temps seul permettra de se prononcer avec certitude. Ce fut un révolutionnaire, mais un révolutionnaire juste pour le passé. Il comprenait qu’il n’y a pas de nation sans histoire, et qu’une patrie se compose des morts qui l’ont fondée aussi bien que des vivants qui la continuent.\par
Le pays récompensa comme il devait ce large et haut sentiment d’amour filial. Hâtons-nous de le dire : il y a un patriotisme supérieur à celui que le pays récompense, c’est le patriotisme de l’homme qui ne craint pas l’impopularité, qui applique tout ce qu’il a d’intelligence au bien public, qui dit son avis avec réserve, puis attend, sans chercher à tirer profit de l’accomplissement de ses prophéties. {\persName Henri Martin}, à qui la direction de la politique française, depuis la Révolution, paraissait légitime dans son ensemble, ne devait pas avoir ces rigueurs de critique. C’étaient, après tout, ses idées qui triomphaient, et, même au cas où il serait prouvé qu’il eut quelquefois, pour les faits contemporains, l’applaudissement un peu facile, pensez-vous, Monsieur, que nous aurions le droit d’être envers lui bien sévères ? Au fond, notre façon d’aimer la France ne diffère pas beaucoup de la sienne. Nous aimons trop cette vieille mère, dont nous avons sucé tous les instincts, si l’on veut toutes les erreurs, pour oser prendre avec elle le ton rogue de l’homme sûr d’avoir raison. L’amour nous rend inconséquents. Nous voyons les imprudences, et nous suivons tout de même. Il est si triste d’être plus sage que son pays. Par moments, on prend la résolution d’être ferme : on se promet, quand viendront les jours sombres, de se laveries mains des fautes qu’on a déconseillées. Eh bien, non ! quand viennent les jours sombres, on est aussi triste que ceux qui n’avaient rien prévu, et le fait d’avoir eu raison quand presque tout le monde avait tort devient une faible consolation. On ne tient pas rancune à sa patrie. Mieux vaut se tromper avec elle que d’avoir trop raison avec ceux qui lui disent de dures vérités. Que vous avez bien fait, Monsieur, de placer le centre de gravité de votre vie au-dessus de ces navrantes incertitudes de la politique, qui ne laissent si souvent le choix qu’entre deux fautes ! Votre gloire ne souffrira pas d’intermittences. Déjà vous jouissez presque des jugements de la postérité. Votre vieillesse heureuse, puissante, honorée, rappelle celle que l’on prêle à Salomon, l’ennui sans doute excepté. L’ennui, vous n’avez jamais su ce que c’est ; et, quoique très bien placé pour voir que tout est vanité, vous ne vous êtes jamais, je crois, arrêté à cette pensée. Vous devez être très heureux, Monsieur ; content de votre vie, indifférent à la mort, car vous êtes brave. Vous éprouvez, disiez-vous dans une de vos conférences, quelque inquiétude en songeant qu’au jour du jugement dernier, l’{\persName Éternel} pourra vous reprocher d’avoir modifié sa création. C’est là une crainte bien éloignée ; rassurez-vous. S’il y a quelqu’un dont l’attitude dans la vallée de Josaphat ne me cause aucune appréhension, c’est bien vous. Vous y continuerez votre rôle de charmeur, et quant au grand juge, vous saurez facilement le gagner. Vous avez amélioré son œuvre ; il sera sûrement content de vous.\par
En attendant, vous viendrez parmi nous vous reposer de cette vie d’infatigable activité que vous vous êtes imposée. Dans l’intervalle de vos voyages de {\placeName Suez} à {\placeName Panama} et de {\placeName Panama} à {\placeName Suez}, vous nous direz vos observations nouvelles sur le monde, s’il s’améliore ou s’abaisse, s’il rajeunit ou vieillit, si, à mesure que les isthmes se coupent, le nombre des hautes et bonnes âmes augmente ou diminue. Notre vie, le plus souvent passée à l’ombre, se complétera par la vôtre, toute passée au grand air. Pour moi, je ne vous vois jamais sans rêver à ce que nous aurions pu faire tous les deux, si nous nous étions associés pour fonder quelque chose. Tenez, si je n’étais pas déjà vieux, je ne sais pas quelle œuvre de bienfaisante séduction je ne vous proposerais pas. Mais il faudrait pour cela donner ma démission de l’{\orgName Académie des inscriptions et belles-lettres}, amie absolue de la vérité ; ce que je ne ferai jamais, je m’y amuse trop. Et puis le monde est bizarre ; en général, il n’accorde à un homme qu’une seule maîtrise. Il vous écoute quand il s’agit de percer des isthmes ; il me prête attention sur certaines questions où il lui plaît de m’entendre. Pour le reste, on ne nous consulte pas. Nous aurions cependant, peut-être, quelques bons conseils à donner. La volonté de {\persName Dieu} soit faite ; ne nous plaignons pas trop de la part qui nous est échue.\par
La vôtre, assurément, a été de premier choix. Après Lamartine, vous avez, je crois, été l’homme le plus aimé de notre siècle, celui sur la tête duquel se sont formés le plus de légendes et de rêves. Nous vous remercions, nous remercions le haut poète qui siège à côté de vous et vous introduit dans cette compagnie, d’avoir, en un temps dont le défaut est le dénigrement et la jalousie, donné à notre peuple attristé l’occasion d’exercer ce que le cœur humain a de meilleur, la faculté d’admirer et d’aimer. La nation qui sait aimer et admirer n’est pas près de mourir. À ceux qui disent que, sous la poitrine de ce peuple, rien ne bat plus, qu’il ne sait plus adorer, que le spectacle de tant d’avortements et de déceptions a éteint en lui toute confiance dans le bien, toute foi en la grandeur, nous vous citons, chers et glorieux confrères ; nous rappelons le culte dont vous êtes l’objet, ces couronnes, ces fêtes qui n’ont coutume d’être décernées qu’à la mort, ces tressaillements de cœur, enfin, que nos foules éprouvent au nom de Victor Hugo, de Ferdinand de Lesseps. Voilà ce qui nous console, ce qui nous soutient, ce qui nous fait dire avec assurance : Pauvre et chère {\orgName France}, non, tu ne périras pas ; car tu aimes encore, et tu es encore aimée.
\section[{Rapport sur les prix de vertu lu dans la séance publique annuelle de l’Académie française}]{Rapport sur les prix de vertu lu dans la séance publique annuelle de l’{\orgName Académie française}}\renewcommand{\leftmark}{Rapport sur les prix de vertu lu dans la séance publique annuelle de l’{\orgName Académie française}}


\dateline{4 août 1881}
\noindent Il y a un jour dans l’année, Messieurs, où la vertu est récompensée. Par suite des fondations de {\persName M. de Montyon} et de quelques autres philanthropes éclairés, il est dérogé ici une fois par an à cette loi profonde de la nature qui a voulu que la récompense du devoir accompli fût obscure et insaisissable. La vertu a justement pour trait de haute noblesse de ne correspondre à aucun salaire. Mille expériences désastreuses prouveraient à l’homme qu’en faisant le bien il obéit à une duperie, que l’homme n’en persévérerait pas moins dans cette voie ingrate, improductive, folie selon le bon sens vulgaire, sagesse selon l’esprit supérieur. Dans les légendes du moyen-âge, souvent si philosophiques, on voit percer à cet égard un sentiment dont la naïveté fait sourire. Selon ces beaux récits, qui ont charmé des siècles, l’homme ne trouve sur la terre que l’épreuve ; cela est tout simple, il aura un jour la vie éternelle ; mais l’animal, qui n’a point de place dans l’éternité, est toujours récompensé ici-bas de ce qu’il fait pour le bien ; car enfin il faut que Dieu soit juste. Quand les deux lions qui accourent du désert, sur l’appel de {\persName saint Antoine}, pour creuser la fosse de l’ermite {\persName Paul}, ont gaillardement accompli leur besogne, {\persName saint Antoine} leur donne sa bénédiction dans l’ordre des choses temporelles. L’effet de cette bénédiction fut probablement qu’ils trouvèrent à quelques pas de là une brebis ou un chevreau égaré, qu’ils mangèrent. Ce fut là leur paradis. La récompense temporelle passait ainsi, dans ces âges de foi, pour quelque chose de grossier ; elle était considérée comme une diminution des titres supérieurs qu’on acquiert par la pratique du bien.\par
Il n’y a pas du tout à craindre. Messieurs, que les prix que vous décernez prêtent à de si fortes objections et que la vertu y perde quelque chose de son mérite. Et d’abord, vous êtes seuls au monde à la récompenser ; puis, vous ne récompensez que les plus humbles vertus ; puis, vous les récompensez si modestement que, si quelqu’un pouvait avoir l’idée de concourir en vue de vos médailles, oh ! vraiment ce serait de sa part le plus misérable des calculs. Les vertus éclatantes qui donnent la gloire, les épreuves de l’homme de génie, tout ce qui attire les applaudissements de la foule, les grands désespoirs aristocratiques comme les efforts sublimes dont parle l’histoire, ne sont point de votre programme. Même celui qui est soutenu dans l’accomplissement du devoir par sa situation sociale, le bourgeois vertueux, s’il est permis de s’exprimer ainsi, vous ne le couronnez pas. Vous réservez vos prix pour la femme dévouée, pour l’homme du peuple courageux, qui, sans se douter de l’existence de vos fondations, ont suivi l’inspiration spontanée de leur cœur. Il n’y a donc aucun danger. Messieurs, que vos récompenses, comme on l’a dit, gâtent la vertu dans sa source et renversent les fondements de l’ordre moral. Malgré tout ce que vous faites et ce que vous ferez, le métier de la vertu restera toujours le plus pauvre des métiers. Nul ne sera tenté de l’embrasser par l’espoir des profits qu’on y trouve. Parmi les quarante ou cinquante vies vertueuses dont les actes authentiques ont passé sous nos yeux, il n’y en a pas une qui, à n’envisager que les rémunérations mondaines, n’eût gagné à suivre une autre direction. Le monde est plein de gens singulièrement habiles à deviner ce qui mène à la fortune ; or jamais on n’a vu personne prendre la vertu comme une carrière avantageuse, comme un moyen de réussir. La concurrence sur ce champ-là est tout à fait nulle ; les gens avisés vont ailleurs.\par
Vous avez donné, par exemple, votre première récompense, deux mille cinq cents francs, à une personne admirable, qui a pris pour tâche d’aller chercher le mal sous ses formes les plus répugnantes et de faire renaître la conscience dans les pauvres êtres où elle est le plus effacée. {\persName Madame Gros}, institutrice libre à {\placeName Lyon}, est peut-être la personne de notre temps qui possède le mieux l’art exquis de faire vibrer, par une sorte de savant coup d’archet, le sentiment moral non encore éveillé. L’amour de l’éducation du peuple est inné chez {\persName madame Gros}. À {\placeName Condrieu}, le souvenir de ses écoles du dimanche et surtout des promenades où elle menait ses élèves est resté comme une légende. Ce n’était point assez pour elle. En 1870, elle revint à {\placeName Lyon}, rêvant d’une œuvre qui eût certainement fait reculer un esprit moins décidé et une âme moins vigoureusement trempée. Elle voulait porter son apostolat jusqu’aux derniers confins du mal et voir si là encore la voix du bien peut être entendue. Un sentiment particulier, comme il en existe presque toujours chez les grands fondateurs, entraîna sa conviction et fixa son choix. Elle crut trouver chez les jeunes garçons pervertis plus de droiture, de franchise, d’aptitude au relèvement que chez les jeunes filles, prises au même étiage de démoralisation. Nous ne donnons cette impression que comme un jugement tout personnel ; l’expérience eût peut-être tourné tout autrement avec un éducateur d’un autre sexe. Quoi qu’il en soit, la véritable vocation de {\persName madame Gros} fut dès lors trouvée. Elle s’établit dans la sentine de {\placeName Lyon}, près des {\placeName Brotteaux}, au milieu des vagabonds que la cristallerie et les verreries de la {\placeName Guillotière} attirent de ce côté. Le tableau, énergiquement tracé par elle et par les témoins de son œuvre, de l’ignorance et de la méchanceté contre lesquelles elle eut à combattre fait véritablement frémir. Elle débuta dans la charité en achetant une petite fille que son père vendait pour boire. Ce misérable lui demanda cinquante francs ; {\persName madame Gros} les donna. Ce qu’elle vit ensuite dans ce monde de précoces débauches dépasse toute créance. Trois fois des messieurs dévoués entreprirent de la seconder dans son œuvre ; trois fois ils reculèrent, révoltés par ce contact odieux. Au début, deux jeunes scélérats se risquèrent à adresser à {\persName madame Gros} des paroles inconvenantes ; sa froideur absolue et sa fermeté leur imposèrent silence ; jamais depuis il n’est arrivé qu’on ait osé prononcer devant elle un mot déplacé. Elle s’est fait une famille de ces enfants sauvages et abandonnés. Elle ne doit se garder que de leurs démonstrations amicales, parfois trop vives, toujours respectueuses. Elle prétend que ces natures brutes ont un grand fond de poésie naïve et qu’on s’empare aisément d’elles. Des figures laides, bestiales, grimaçantes, s’éclaircissent, s’embellissent peu à peu ; des êtres sinistres deviennent gais, expansifs, polis même ; \emph{« enfin, dit {\persName madame Gros}, ils ont un charme original et un cachet qui n’appartient qu’à eux »}.\par
{\persName Madame Gros} a rassemblé, dans un travail qui nous a été communiqué, les souvenirs les plus originaux de ses chers petits sauvages, comme elle les appelle, leurs bons mots, leurs hauts faits et surtout leurs progrès dans le bien. Les confidences de ces jeunes pervertis sont faciles à obtenir ; car, ainsi que {\persName madame Gros} le remarque, le premier sentiment qu’elle trouve toujours chez eux est la fierté de leurs crimes. Ils s’en vantent, et sont glorieux de la crainte qu’ils inspirent. Un nouveau venu lui avoua un jour qu’il avait noyé trois de ses camarades dans le {\placeName Rhône}. \emph{« Ils m’avaient ennuyé, dit-il, je les ai poussés et je les ai regardés se débattre. »} Un an après, ce petit misérable sauvait trois personnes en danger ; c’est maintenant un excellent soldat.\par
« L’enfant de feu », comme l’appelle {\persName madame Gros}, était dans l’école un véritable fléau, par l’abus qu’il faisait de sa force sur ses camarades. {\persName Madame Gros} lui fit promettre de ne se battre qu’une fois par jour, pour commencer. Trois semaines après, il ne se battait plus ; à tel point, qu’ayant un jour reçu un soufflet, il sauta sur un bureau, et, trépignant, furibond, les yeux étincelants, il dit à celui qui l’avait frappé : \emph{« Tu as du bonheur que j’aie promis à la dame de ne plus me battre ; sans cela je t’aurais étranglé. »}\par
Il y avait à {\placeName La Mouche} (quartier des verriers) un nid de petits vauriens nommé {\persName Bonhomme}. Leur spécialité était de jeter des pierres aux passants pour le plaisir de les blesser. Les plus âgés, après une année de résistance, se décidèrent enfin à ne jeter qu’un nombre de cailloux fixé, avec promesse de n’atteindre personne. Ils ont tous fini par se corriger, et ils y ont mis tant de zèle que maintenant ils pourchassent avec acharnement tous ceux qui jettent des pierres. {\persName Madame Gros} fait à ce sujet une réflexion que nous recommandons à ceux qui s’occupent, dans la philosophie de l’histoire, du chapitre important : « Comment le brigand devient gendarme. » \emph{« En général, dit {\persName madame Gros}, ils se communiquent leurs qualités nouvelles, au besoin par des voies de fait, en faveur du bon ordre. »}\par
{\persName Walch} est évidemment un des naufragés dont le sauvetage a laissé le plus profond souvenir dans le cœur de {\persName madame Gros}, \emph{« Il avait quinze ans ; carrure, tournure, visage, crinière, regard, caractère, le tout représentant à merveille le lion du désert dans sa force sauvage. »} Quatre années l’avaient à peine apprivoisé, lorsqu’un jour une dame vient à l’école avec une rose rouge jetée coquettement sur un chapeau de velours noir. — Voyez, Mesdames, comme il faut peu de chose pour ramener l’homme à la vertu ! — A la vue de cette rose, les regards du lion s’éclairent pour la première fois ; il sourit à cette fleur. {\persName Madame Gros} profite de ce moment pour faire pénétrer dans cette âme inculte un germe d’amour-propre et un peu de honte sur sa tenue plus que négligée. Le dimanche suivant, pour obtenir la faveur d’être placé à côté de la rose, il vint à l’école en costume propre : lui-même avait lavé sa jaquette dans le {\placeName Rhône} de grand malin. \emph{« Elle n’a pas pu séquer, dit-il ; mais elle séquera sur mon dos. »} \emph{« Depuis ce jour, dit {\persName madame Gros}, il s’est peu à peu civilisé : ses manières brusques ont disparu, il n’a gardé du fauve qu’il représente que l’extérieur avantageux et les qualités qui en sont l’apanage. »} {\persName Madame Gros} ayant été malade, le brave lion faisait chaque dimanche quatre heures de route pour venir s’informer de sa santé. {\persName Madame Gros} lui parlant un jour de sa mère : \emph{« Oh ! j’ai deux mères, dit-il, celle qui m’a né et puis vous. »}\par
Les batailles rangées dans les graviers du {\placeName Rhône}, et surtout les atroces cruautés qu’exerçaient les uns sur les autres les enfants de la cristallerie ont été supprimées par {\persName madame Gros}. On ne se souvient pas qu’un seul de ses élèves, et elle en a eu par centaines, soit revenu au mal. Ceux qui se marient envoient leurs frères à {\persName madame Gros} et se font les recruteurs de l’école. Le naturel, l’élan de cœur, la vivacité, l’entraînement, un esprit prodigieusement inventif, joint à une fermeté à toute épreuve, font de madame Gros un exemple unique peut-être de l’art d’exprimer au peuple dans son langage les plus hauts sentiments. Ce qu’elle a surtout, c’est le don d’amuser. Sa force est dans les histoires qu’elle raconte avec une connaissance achevée des moyens de toucher la fibre populaire. Ce fut l’art de tous les grands initiateurs. La parabole a toujours entraîné l’humanité. L’humanité, en effet, aime l’idéal ; mais il faut que l’idéal soit une personne, un fait, un récit ; elle n’aime pas une abstraction. Il paraît que, pendant que madame Gros raconte ses histoires à ceux qu’elle appelle ses « brigands du dimanche », son auditoire est tout oreilles. Ah ! si nous avions les récits de {\persName madame Gros}, sténographiés sans qu’elle le sût ! Comme cela vaudrait mieux que les fadaises de notre littérature usée ! Je porte envie aux gamins qui entendent ces chefs-d’œuvre, destinés sans doute, comme les vrais chefs-d’œuvre, à rester toujours inédits. Ils ont, du reste, le genre de succès qu’ils méritent : ils entraînent, ils convertissent. Après une histoire racontée par {\persName madame Gros} sur l’assistance que l’on doit à ses parents, {\persName Michel} renonce à l’ivrognerie pour construire une cabane à sa mère qui couchait sous une charrette. Aujourd’hui {\persName Michel} est marié et presque dans l’aisance. \emph{« Je me livrais à la boisson, disait-il dernièrement à {\persName madame Gros}, quand votre histoire m’a sauvé. Maintenant la bénédiction de Dieu est sur moi. »}\par
Dans la clientèle de {\persName madame Gros}, il y une catégorie que {\persName madame Gros} appelle, on ne voit pas bien pourquoi, la « série des {\orgName Mongols} ». Deux frères de cette bande se relayaient pour venir à l’école à tour de rôle. Cela parut singulier à {\persName madame Gros}, qui en fit un jour l’observation à l’un deux. \emph{« Mon frère ne peut pas venir, lui répondit celui-ci ; il est sur l’arbre. — Et que fait-il sur l’arbre ? — Il attend queje lui porte mes souliers ; je les lui porterai quand la leçon sera finie, et il entendra l’histoire. Dimanche ce sera son tour d’avoir la leçon, et moi j’aurai l’histoire. — Alors vous n’avez qu’une paire de souliers pour vous deux ? — Eh oui ! c’est pour cela que, quand il fait mouillé, nous nous tenons sur l’arbre, en attendant notre tour de venir à l’école. »}\par
Ce spectacle d’une terre avide de boire la rosée du bien, et qui s’ouvre au premier doux rayon de soleil, cette charmante inoculation du sens moral, par un mot, par un regard, en de pauvres êtres qui n’ont pas eu de mère, qui n’ont jamais vu un œil bienveillant leur sourire, rappellent les miracles qui remplissent la vie de tous les grands maîtres de la vertu. Remercions {\persName madame Gros} d’avoir fait revivre dans notre âge, devenu étranger aux grands secrets de l’âme, les merveilles de conversion qui semblaient réservées aux temps où la grâce vivante se promenait sur la terre avec ses trésors d’indulgence et de pardon.\par
À {\persName madame Gros}, vous avez voulu associer dans vos récompenses {\persName mademoiselle Paula Gagny}, qui a déployé sur le même théâtre, à {\placeName Lyon}, les ressources de l’esprit le plus fertile pour le bien. Née d’une famille honorable de Schelestadt, elle recueille chez elle, élève et entretient gratuitement deux, quatre, huit et jusqu’à vingt petites filles, de trois ans et au-dessus. Dans la fatale année 1871, elle part pour {\placeName Schelestadt} et revient à {\placeName Lyon}, à travers les lignes prussiennes, ramenant une douzaine d’enfants, de deux à trois ans, inconnus ou abandonnés. L’ennemi, frappé de son courage, lui avait donné un sauf-conduit. Peu de temps après, elle part de nouveau pour l’{\placeName Alsace}, d’où elle ramène encore quelques enfants ; puis ce sont les autorités mêmes de l’{\placeName Alsace} et de la {\placeName Lorraine} qui lui envoient à {\placeName Lyon} les orphelins sans asile. L’espace manquait dans son modeste appartement pour ces hôtes nouveaux ; les plus petits enfants furent pendant quelque temps couchés dans son propre lit ; puis, par des prodiges d’intelligence et d’activité, elle réussit à constituer cet étonnant établissement qui renferme aujourd’hui soixante Alsaciennes ou Lorraines âgées de dix-huit mois à dix-huit ans. {\persName Mademoiselle Gagny} place en ville dans des maisons recommandables les plus âgées de ces filles, les aidant de ses conseils et les rappelant à elle quand elles ne sont pas heureuses. Toujours vêtue de deuil, le visage pâle et amaigri par la tristesse, {\persName mademoiselle Gagny} représente admirablement parmi nous la dignité, la résignation qui ont porté si haut devant leurs sœurs de {\placeName France} le caractère des femmes d’{\placeName Alsace-Lorraine}. La médaille de deux mille francs que vous avez décernée à {\persName mademoiselle Gagny} sera d’un précieux secours pour l’œuvre à laquelle elle a consacré sa vie.\par
Une somme pareille, que vous avez décernée à {\persName M. l’abbé Carton}, servira également à une excellente œuvre de charité. Connaissez-vous rien de plus triste que cette plaine de mesquine misère et de désolation sans poésie, que l’on traverse en sortant de Paris pour se rendre à {\placeName Versailles} par la rive gauche, cet amas sans ordre apparent de constructions qui ne sont plus urbaines et ne sont pas encore rustiques, ces chaumières (quelles chaumières ! oh ciel !) bâties de pièces incongrues, arrachées aux démolitions de la grande ville ; ce qui n’empêche pas qu’au milieu de ces tristes cabarets de barrière, de ces maisons qu’on dirait abandonnées ou hantées par le mal, éclatent tout à coup, par endroits, un champ de verdure qui vous sourit, de fraîches cultures que n’atteint pas la vulgarité environnante ? C’est le {\orgName Petit-Montrouge}, dont {\persName M. l’abbé Carton} est curé depuis treize ans. {\persName M. l’abbé Carton} a trouvé moyen, dans cette triste zone de la banlieue parisienne, de créer un véritable paradis, un asile propre, bien bâti, presque gai, où cinquante vieillards des deux sexes sont logés, chauffés, blanchis, habillés et nourris. Comme tous les fondateurs charitables, {\persName M. l’abbé Carton} dépasse souvent la mesure de ce que semblerait commander la prudence humaine. Il a foi dans son œuvre, et jamais sa confiance n’a été déçue. Plus de cent vieillards attendent leur tour d’admission dans l’asile ; vos deux mille francs vont faire des heureux et prouver à {\persName M. l’abbé Carton} l’intérêt que vous prenez à ses nobles efforts. Vous avez également accordé un prix de la valeur de deux mille francs à deux frères jumeaux, {\orgName Edouard et Calixte Chaix}, qui ont su faire du lien étroit que la nature a établi entre eux une touchante association de vertu. Les actes de la {\orgName Société des Sauveteurs de la Méditerranée} sont pleins des traits de courage de ces deux rivaux en dévouement et en amitié. On se souvient surtout de l’incendie des soûles à charbon du paquebot le {\placeName Caire} dans le port de {\placeName Marseille}, le 6 décembre 1856. L’incendie du navire pouvait devenir l’incendie du port lui-même. On désespérait d’arrêter le feu, car la pompe du bord, quoique donnant avec force, ne pouvait être bien dirigée. Les deux intrépides enfants se font attacher par la ceinture et descendent résolument dans le foyer de l’incendie. À eux deux, ils saisissent le tuyau de la pompe, visent le foyer ardent, le maîtrisent. On les retire évanouis, presque asphyxiés ; {\persName Édouard} était couvert d’affreuses brûlures, dont il porte encore la trace profonde. Sauver est pour ces deux frères une vocation, un besoin. Prodigues de leur vie, qui pourtant est bien nécessaire au soutien de leur famille, ils ont arraché plus de vingt personnes à la mort. Dans un de ces sauvetages, Édouard tombe sur une chaîne, s’enfonce deux côtes, s’évanouit presque ; un heureux hasard lui permet de prendre pied. La vue du malheureux qui allait disparaître lui rend des forces ; un instant après il le dépose sur le quai, dont il rougit les dalles de son propre sang.\par
La {\orgName Société des Sauveteurs de la Méditerranée}, ne croyant pas pouvoir présenter deux candidats à la fois, demandait la récompense pour {\persName Édouard}, ajoutant qu’une telle récompense serait considérée par son frère comme s’appliquant à lui-même. Vous avez eu une pensée délicate. Messieurs ; vous n’avez pas voulu séparer deux personnes si intimement unies par le sang et par le cœur. Vous les avez considérées comme une seule et même personne, et vous avez décidé que les noms d’{\persName Édouard} et de {\persName Calixte Chaix} figureraient indivis dans la liste des principales récompenses que vous décernez.\par
La vertu, Messieurs, s’est présentée à vous cette année, aussi diverse que sublime. Vous avez pu en couronner toutes les variétés. Vous venez d’applaudir ce que faisaient deux jeunes héros à quinze ans ; je vous présente maintenant la vertu centenaire, en la personne de {\persName Marie Coustot}, de {\placeName Condom}. Oui, elle a cent deux ans, et elle continue toujours de faire le bien. Servante depuis l’âge de seize ans dans une famille d’abord riche, elle a donné ses économies à ses maîtres ruinés ; elle continue sans gages son œuvre de fidélité. Aujourd’hui, elle sert les petits-enfants de ses premiers maîtres, et, quoique devenue presque aveugle, elle travaille, elle se prive de nourriture pour ceux à qui elle a consacré sa vie. Elle a cent deux ans et elle est vertueuse ! Vous avez vu là un mérite de plus. Le vieillard, en perdant ses illusions, ne perd-il pas ses meilleures raisons d’être vertueux ? Illusion divine, illusion providentielle assurément, la vertu n’en est pas moins comme l’amour le résultat d’un charme en dehors de la raison, qui nous entraîne, nous séduit. Il ne faut pas, pour s’y livrer, qu’on ait trop bien vu que tout est vanité. La bonne {\persName Marie Coustot} ne s’arrête pas à cette philosophie désespérée ; elle mourra dans sa simplicité, toujours obstinée à s’oublier et à se sacrifier.\par
Les vertus qui précèdent vous sont attestées par des préfets, des sous-préfets, des gendarmes, des autorités constituées. Le bon {\persName Simian}, dont je vais maintenant vous parler, vous est surtout présenté par {\persName Mistral} ; oui. {\persName Mistral}, votre lauréat, qui vous a écrit une lettre charmante pour vous recommander un de ses compatriotes de {\placeName Maillane}, dont les vertus ont quelque chose d’archaïque et de touchant. Le bon {\persName Simian}, ou, comme on l’appelle dans le pays, {\persName Cadet Simian}, est un petit propriétaire cultivateur qui s’est consacré depuis trente ans, avec un désintéressement absolu, à toutes les besognes tristes, à la garde des agonisants, au soin des moribonds, à l’assistance des chirurgiens, et enfin à l’œuvre du vieux {\persName Tobie}, à l’ensevelissement des morts. Avec une conscience, une modestie et une discrétion au-dessus de tout éloge, le brave {\persName Cadet Simian} met son dévouement au service de tout le monde, et les maladies les plus dangereuses comme les offices les plus rebutants ne l’ont jamais fait reculer. Dans les épidémies, il veille jusqu’au dernier soupir les malades abandonnés par leurs proches ; il a assisté les chirurgiens dans toutes les opérations qui ont été pratiquées à {\placeName Maillane} depuis trente ans. {\persName Cadet Simian} est la providence des jours sombres ; on vient frapper à sa porte toutes les fois que la vie se montre à {\placeName Maillane} par ses côtés austères. Il a cinq ou six cents francs de rente, qui lui viennent de quelques coins de terre, et cela lui suffit, car il ne va jamais au café, ne fait pas usage de tabac et ne sort de chez lui que pour ses bonnes œuvres. Il est profondément religieux et n’a d’autres délassements que la lecture et le travail des champs.\par
La lettre de {\persName Mistral} est contresignée par le maire, le curé et le médecin. \emph{« Quant à faire intervenir le sous-préfet ou le préfet en cette affaire, ajoute {\persName Mistral}, c’est complètement inutile, attendu que ces messieurs sont trop souvent renouvelés et trop étrangers à notre vie pour qu’ils puissent se douter de ce qui se passe d’intime parmi nous. »} Ce jour discret jeté sur ce qui se passe d’intime à {\placeName Maillane} vous a vivement touchés. {\persName Mistral} a obéi là à un sentiment très juste ; il a craint peut-être que les vertus un peu démodées du bon {\persName Simian} n’eussent pas quelque chose d’assez civique pour mériter de grosses approbations officielles. Il s’est défié des sceaux de l’État, et il a pensé qu’il ne fallait mettre en mouvement l’autorité préfectorale que pour des vertus qui ne supposent pas un petit cercle d’initiés.\par
{\persName Francilie Laquinte} est peut-être le premier exemple d’une personne née dans la condition de l’esclavage à laquelle ait été décerné le prix Montyon. Elle et sa mère servirent durant des années une vieille dame de la {\placeName Guadeloupe}, qui récompensa leurs soins par l’affranchissement. Cette faveur n’eut d’autre résultat que de resserrer de plus en plus les liens d’affection qui les unissaient à leur maîtresse ; elles restèrent comme servantes dans la maison où elles avaient été esclaves. Après la mort de leur bienfaitrice, {\persName Francilie} nourrit sa mère de ses petits travaux de couture. Malgré sa pauvreté, elle trouva encore le moyen d’être charitable. Dans les désordres entraînés par le décret d’émancipation de 1848, elle fut la raison, la prévoyance d’un monde entièrement désorganisé. Elle adopta les orphelins ; de ses ressources précaires elle consola une mère que son mari avait délaissée. C’est toute la commune de {\placeName Saint-François}, à la {\placeName Guadeloupe}, qui vous demande de couronner {\persName Francilie}. L’esclavage heureusement n’est plus à supprimer. Messieurs ; s’il l’était, c’est par des exemples comme celui de {\persName Francilie Laquinte} que l’émancipation serait accomplie. L’esclavage cesse le jour où l’esclave, que l’antiquité concevait comme sans moralité et sans religion, devient moralement l’égal de son maître.\par
L’esclavage antique fut aboli virtuellement quand une pauvre esclave de {\placeName Lyon} se fut montrée dans l’amphithéâtre aussi héroïque que sa maîtresse. L’esclavage moderne a sans doute été condamné avant tout par nos principes de philosophie ; mais quelques vertus d’esclaves ont aussi concouru à la même fin. Le hasard a voulu que nous ayons encore une vertueuse mulâtresse à joindre à {\persName Francilie}. Peut-être même le dévouement dont je vais vous parler a-t-il encore quelque chose déplus touchant. {\persName Paula Yvor} demeure à {\placeName Paris}, et nous avons probablement quelquefois rencontrée dans le dédale des petites rues qui entourent le chevet de l’{\placeName église Saint-Germain des Prés}. À l’âge de onze ans, elle s’est attachée à une famille qu’elle a toujours servie avec amour. Le malheur étant venu frapper cette famille, {\persName Paula Yvor}, sans espoir de récompense, fit vivre celle qui avait été sa maîtresse des gains modiques d’un petit commerce de produits coloniaux, péniblement exploité du haut de sa mansarde. Sa maîtresse, à son lit de mort, lui lègue ses deux filles en bas âge : la sollicitude de {\persName Paula} ne se dément pas un instant. Quand, en marchant les pieds dans la neige, la pauvre créole a réussi à placer quelques-uns des ananas qu’elle colporte et à ramasser quelques sous, c’est pour se rendre à la maison de la {\orgName Légion d’honneur de Saint-Denis} et pour porter à ses filles d’adoption un vêtement chaud, de petites douceurs qui prouveront aux orphelines qu’elle ne sont pas déshéritées de toute tendresse. Avec une persistance sans égale, le malheur continue à frapper les deux jeunes filles à leur entrée dans le monde ; l’une d’elles, au moins, tombe dans une misère navrante. La vieille mulâtresse est toujours là ; elle a soixante-douze ans ; un cancer lui a rongé la moitié de la figure, et pourtant elle court encore les rues avec son panier d’ananas, cherchant à récolter la petite somme nécessaire au repas des délaissées dont elle est le seul soutien. Songez quel accueil sera fait à vos cinq cents francs dans ce réduit d’où est bannie depuis longtemps toute joie !\par
Je ne finirais pas, Messieurs, si je voulais énumérer tant de vertus humbles et en particulier les sacrifices discrets accomplis dans cette classe si intéressante des domestiques fidèles que vous aimez à récompenser. {\persName Marie Arot} ({\placeName Saint-Servan}, {\placeName Ille-et-Vilaine}) sert depuis cinquante-quatre ans les mêmes maîtres. Elle a élevé et soigné neuf enfants ; la famille à laquelle elle est attachée ayant perdu toute sa fortune, elle refuse de la quitter ; elle sert gratuitement avec un courage que de pénibles circonstances mettent à de rudes épreuves (médaille de mille francs). {\persName Céline Landrin} à {\placeName Saint-Denis}, {\placeName île de la Réunion}, a soutenu, de son travail, pendant plus de quinze ans, une vieille demoiselle délaissée par sa famille ; elle a soigné un noir atteint de la lèpre ; sa vie est un perpétuel exercice de suave pitié. Ce serait presque les mêmes actes de dévouement domestique que j’aurais à relever dans le dossier de {\persName Rosalie-Yictorine Sauray}, à {\placeName Sassy} ({\placeName Calvados}) ; de {\persName Marguerite Daumin}, à {\placeName Moulins} ({\placeName Allier}) ; de {\persName Segondine Fernet}, à {\placeName Anthuille} ({\placeName Somme}) ; de {\persName Françoise Paon}, à {\placeName Morlaix} ({\placeName Finistère}) ; de {\persName Marguerite Lanusse}, à {\placeName Caudéran} ({\placeName Gironde}). La vertu. Messieurs, est plus monotone que le vice ; mais elle peut, sans inconvénient, se répéter. Remercions-la de se répéter ; cette monotonie, qui peut être en littérature un gros défaut, est la cause par laquelle le monde moral subsiste.\par
Oui, la charité est exercée chez nous avec une persévérance qui doit rassurer ceux qu’alarment tant de symptômes de refroidissement. Je ne puis que citer {\persName Désirée Chardon}, à {\placeName Segré} ({\placeName Maine-et-Loire}), simple ouvrière modiste, vrai modèle d’abnégation ; {\persName Eucharis Michel}, directrice d’asile à {\placeName Aix} ; {\persName Hélène Perron}, à {\placeName Saint-Martin-des-Prés} ({\placeName Côtes-du-Nord}) ; {\persName Alexandrine Nétrelle}, à {\placeName Cormontreuil} ({\placeName Marne}) ; {\persName Désiré Guillot-Envrard}, à la {\placeName Chapelle-Saint-Sauveur} ({\placeName Saône-et-Loire}) ; les {\orgName époux Joyaux}, à la {\placeName Frette} ({\placeName Seine-et-Oise}) ; {\persName Sophie Tufféry}, à {\placeName Lajo} ({\placeName Lozère}) ; la {\persName femme Bertrand Guilhaume}, à {\placeName Clermont-l’Hérault} ({\placeName Hérault}) ; {\persName Françoise Boulestreau}, à {\placeName Bourgneuf} ({\placeName Maine-et-Loire}) ; {\persName Anne-Marie Gesnouin}, à {\placeName Saint-James} ({\placeName Manche}) ; {\persName Olympe Gay}, à {\placeName Thueyts} ({\placeName Ardèche}) ; {\persName Jenny Marchandeau}, à {\placeName Chaudenay-sur-Dheune} ({\placeName Saône}), paralytique des deux jambes, qui n’a que ses mains pour vivre et trouve encore moyen d’être bienfaisante ; enfin, {\persName madame veuve Lamoute}, la providence de {\placeName Bergerac}, qui emploie tout son bien à secourir les jeunes filles abandonnées.\par
Le temps me presse, Messieurs. L’exemple de la philanthropie de {\persName M. de Montyon} a trouvé, en effet, des imitateurs. Des personnes bienfaisantes ont voulu, comme lui, consacrer leur fortune à l’encouragement du bien. Les sommes que vous ont léguées {\persName MM. Souriau}, {\persName Gémond}, {\persName Laussat} et la personne charitable qui a voulu rester anonyme, vous ont aidés à récompenser des vertus non moins touchantes que celles que nous avons déjà énumérées.\par
{\persName M. Joachim Fontaine}, maître porion aux mines de {\placeName Liévin} ({\placeName Pas-de-Calais}), a sauvé la vie à seize personnes, surprises par des éboulements ou atteintes par le feu grisou. Une note, que notre savant confrère {\persName M. Daubrée} a jointe au dossier de {\persName Joachim Fontaine}, constate que, malgré les précautions, chaque jour plus minutieuses, que prend l’administration, soixante mille tonnes de charbon coûtent en moyenne une existence d’homme. La classe des mineurs est riche en actes de dévouement, qui commandent d’autant plus d’admiration qu’ils ont été accomplis sans témoin. Des faits de ce genre ont valu à Fontaine une médaille d’argent de première classe. Vous y avez ajouté les mille francs du prix Souriau.\par
{\persName Félix Rieu}, d’{\placeName Avignon}, qui a opéré des miracles de sauvetage, aura les mille francs du prix Gémond ; vous attribuez le prix Laussat à la {\persName veuve Malécot}, {\placeName Saint-Martin-de-Mâcon} ({\placeName Deux-Sèvres}) ; vous donnez les mille francs de la personne charitable à {\persName Rose Mélanie}, de {\placeName Pontorson} ({\placeName Manche}), enfant abandonnée, dont tous les actes sont empreints d’une dignité et d’une délicatesse qui feraient envie aux personnes les mieux nées. Enfin, un reliquat vous permet de donner mille francs à {\persName Jeanne Pécusseau}, de {\placeName Nantes}, également enfant d’hospice, dont le dossier est un document inappréciable de ce qu’il peut y avoir de joie et d’affection dans un petit cercle de pauvres et d’humbles qui se connaissent et s’aiment entre eux. Qui croirait qu’il y a un monde des enfants trouvés ? Ce monde existe, et l’on y est très heureux.\par
{\persName Jeanne Pécusseau} fut élevée par une nommée Albert, elle-même pupille des hospices, qui a consacré sa vie tout entière à l’éducation d’enfants abandonnés comme elle. \emph{« Tous les enfants élevés par cette bonne {\persName fille Albert}, nous dit l’inspecteur de l’{\orgName Assistance publique} de la {\placeName Loire-Inférieure}, ont bien tourné. Ils ont été et sont encore la joie du service… Mais {\persName Jeanne Pécusseau}, au milieu de cette famille de hasard, dont elle est l’aînée, et dont tous les membres furent bons, devait donner l’exemple de toutes les vertus. Elle conserva, en particulier, pour sa vieille nourrice une piété filiale sans bornes. Dès qu’elle put gagner quelques sous, ce fut pour les rapporter, tout heureuse et toute fière, à sa mère adoptive, afin qu’ils fussent employés à soulager les petites sœurs qui étaient venues prendre place comme elle au foyer de la bonne nourrice. Sa conduite exemplaire, sa bonne tenue, sa modestie, son caractère enjoué et sérieux à la fois la firent chérir. »} Elle arriva à une position qu’elle envisagea comme de l’aisance. Dans son petit budget, il y avait tous les ans une réserve pour être adressée (c’était sa joie) à la bonne {\persName tante Albert}, comme elle l’appelait, depuis qu’elle savait que la vieille nourrice n’était pas sa mère… Pauvre fille ! À force de recherches, elle est parvenue à découvrir sa vraie famille. Ce n’a pas été pour elle la source de beaucoup de joie. {\persName Jeanne Pécusseau} a consacré ses économies à l’achat d’un terrain au cimetière, — pour y déposer sa chère nourrice. \emph{« Tous nos enfants, dit l’inspecteur, ont pleuré avec elle sur cette tombe, où il est entendu qu’elle viendra dormir à son tour. Cet exemple, ajoute-t-il, a vivement frappé notre famille assistée, et quand il m’arrive d’en parler, tous les yeux se remplissent de larmes ! »}\par
Ah ! que l’homme est bon. Messieurs, et qu’on en peut tirer de belles choses, quand un artiste habile se trouve à côté de lui, pour faire jaillir en son cœur la source des larmes, de la prière intime et de l’amour !\par
La {\orgName fondation Marie Lasne} vous donne six médailles de trois cents francs. Vous avez décerné la première à {\persName Emmeline Nadaud}, à {\placeName Chancelade} ({\placeName Dordogne}). L’impression que produit {\persName Emmeline Nadaud} sur tous ceux qui la voient est des plus vives. Nous possédons un excellent crayon de cette physionomie modeste, franche, ouverte, chagrine, mais résignée, en une petite biographie, chef-d’œuvre de simplicité et de vertueuse bonne grâce, écrite par {\persName M. le curé de Château-l’Évêque}. La pauvre fille a été jetée comme une perle au milieu d’un triste monde d’infirmes et d’incapables. Dans son enfance, elle voit l’intempérance du père ruiner la petite industrie qui fait vivre la famille. Le {\placeName moulin Nadaud}, mis en détresse par la concurrence des voisins plus sobres, chôme la plupart du temps. Dès l’âge de douze ans, {\persName Emmeline} est ménagère, ouvrière, institutrice, infirmière. Elle fait marcher le moulin, charge les sacs, soigne les bêtes de somme, fait le ménage à elle seule. Tous admirent qu’elle puisse suffire à tant de soins dans une maison aussi désemparée. Ses vertus et ses charmes extérieurs lui font trouver des mariages très avantageux ; elle les refuse tous. Son frère, perclus, qui n’a pas un mouvement, reçoit d’elle une instruction et des sentiments religieux qui le consolent ; un vieux grand-père, dans la misère, est adopté ; la mère, devenue paralytique, une jeune sœur, victime d’un accident, sont soignées, remplacées ; l’intempérance du père est limitée ; grâce à Emmeline, tout va pour le moins mal possible dans la plus triste des maisons.\par
Recueille-t-elle beaucoup de reconnaissance pour tant de bienfaits ? Hélas ! non. \emph{« Les larmes les plus amères que cette enfant verse secrètement dans le sein de Dieu, dit {\persName M. le curé de Château-l’Évêque}, ne viennent pas de ce que nous avons dit mais de ce que nous ne pouvons dire sans blesser l’amour-propre, la discrétion, le mutisme de notre protégée… Malgré l’espèce de violation du domicile de l’amitié que nous avons dû commettre pour apprendre ce que nous vous écrivons, il restera beaucoup de choses dans l’oubli et dans le secret de la conscience. »} Emmeline ne se plaint jamais et, si elle ouvre son cœur ulcéré, c’est seulement à la sœur de {\orgName Saint-Vincent-de-Paul} de {\placeName Château-l’Évêque}. Les scènes déplorables, les traitements indignes, les paroles offensantes, les injustices les plus criantes sont les conséquences de l’ivrognerie du père. Ce qu’il y a d’admirable, c’est la patience, la résignation, la douceur avec lesquelles cette jeune fille supporte tout ; lors même que son père la rudoie, elle est caressante et dévouée. Souvent on la voit assise sur une chaise, dans la salle du cabaret, attendant que son père veuille la suivre ; elle espère abréger ainsi la séance et diminuer des dépenses funestes à la famille. Le public, qui est juste quelquefois, se prononce hautement pour la touchante victime ; elle, toujours réservée, ne consent pas à se laisser trop plaindre. ; Le dimanche suffit à sa consolation. Ce jour-là, elle se donne des délassements de son choix ; elle préfère à tous les autres la compagnie de la fille de charité et le soin des malades. Un groupe de jeunes filles que ses vertus ont spécialement captivées, et qui cherchent l’estime publique en s’approchant d’elle, ne la quitte pas. Dans le village, chacun a part à ses attentions ; sans distinction et sans prétention, avec une simplicité admirable, elle soutient l’un, console l’autre, et verse sur ceux qui s’approchent d’elle une partie de cette grande résignation qui la caractérise. Sa tenue modeste et sans apprêt frappe tout le monde. Elle n’a pas, comme les autres jeunes filles de la campagne, suivi le changement des modes ; elle a gardé son costume et sa coiffure de villageoise ; elle le porte avec une rare distinction ; car voici la silhouette exquise que {\persName M. le curé de Château-l’Évêque} nous a envoyée d’elle : \emph{« Un trait vous fera comprendre l’impression profonde que l’on ressent en voyant {\persName Emmeline Nadaud}. Un jour (il y a de cela quelques années), {\persName Emmeline} revenait de porter la farine de ses clients ; elle était assise sur sa mule, tricotant comme elle le fait d’ordinaire dans ses courses, pour ne pas perdre le temps. Elle est rencontrée sur la route par un monsieur qui la remarque. À son arrivée à {\placeName Château-l’Évêque}, ce monsieur, qui est médecin, demande immédiatement des renseignements sur cette jeune fille qui l’a frappée, et, après qu’on lui a dit ce qu’elle est, ce qu’elle fait : — Mais cette jeune fille, dit-il, mérite le prix Montyon ; je la signalerai à l’{\orgName Académie}. »} Je ne sais si la signature de cet admirateur d’{\persName Emmeline} figure parmi les innombrables attestations qui montrent l’estimé que l’on professe pour elle à {\placeName Chancelade} et à Château-l’Évêque ; mais ce qui est bien honorable pour cette jeune fille, c’est la notice qu’a faite sur elle M. le curé de {\placeName Château-l’Évêque}, notice composée avec un sentiment des plus justes, un tact parfait, et une pleine inconscience littéraire. Votre récompense fera mieux que de justifier la prophétie du médecin qui la rencontre tricotant sur sa mule ; elle confirmera le suffrage de l’opinion publique qui, dans le pays, entoure {\persName Emmeline} d’une véritable auréole de respect.\par
Vous avez trouvé, Messieurs, une sœur dans le bien, digne d’être associée à {\persName Emmeline}, en la personne d’{\persName Euphrosine Almiès}, au {\placeName Pompidou} ({\placeName Lozère}), née infirme et, comme {\persName Emmeline}, unique soutien d’une famille qui ne lui rend en retour que l’ingratitude et les mauvais traitements. {\persName Marcelline-Lucie Michaut}, de {\placeName Provins}, est de la même famille de saintes résignées. {\persName Émilie Montel}, de la {\placeName Suze} ({\placeName Sarthe}) reste aussi infirmière toute sa vie, par choix, par le goût désintéressé de bien faire. {\persName Sylvain-Clément Détourné}, du {\placeName Vieil-Hesdin} ({\placeName Pas-de-Calais}) et {\persName Germain Barbe}, de la {\placeName Basse-Pointe} ({\placeName Martinique}) sont des modèles de piété filiale. Vous les avez également récompensés sur la {\orgName fondation Marie Lasne}.\par
Que de vertus, Messieurs, ont passé devant vous, et que serait-ce si nous avions à parler des vertus qu’on ne récompense pas, de ces héroïsmes de tous les jours, qui se traduisent non par un acte, mais par une habitude constante de dévouement : l’héroïsme calme et scientifique du médecin, l’héroïsme maternel de la sœur de charité, l’héroïsme voulu du soldat ! Songez à {\placeName Sfax}, à cette poignée de braves jetés sur une plage de boue et de feu ; partout les ruses cachées du désespoir, les embûches du fanatisme, et, au milieu de cet enfer, un nombre imperceptible de soldats, de marins, courant où les mène la voix de leurs chefs, car le chef, est pour eux la patrie, le devoir. Bonne et solide race française, vertueuse depuis deux et trois mille ans, comme on la calomnie en la croyant livrée aux calculs étroits de l’égoïsme ! Oui, certes, elle a de graves défauts : c’est de s’éprendre trop vite pour l’utopie généreuse, c’est de trop croire au bien et de se laisser surprendre par le mal, c’est de rêver le bonheur du monde et d’obliger des ingrats ! Mais, croyez-moi, aucune autre race n’a dans ses entrailles autant de cette force qui fait vivre une nation, la rend immortelle malgré ses fautes, et lui fait trouver en elle-même, au travers de tous ses désastres et de toutes ses décadences, un principe éternel de renaissance et de résurrection.\par
Oui, Messieurs, chez nous la vertu surabonde ; elle est dans nos instincts, dans notre sang. Nous abusons même de notre richesse, car je vous avoue que, à part l’{\orgName Académie}, qui l’encourage, je trouve que souvent nous faisons trop de choses pour la décourager. Nous lui demandons trop de certificats ; nous voulons trop savoir ses origines. Les origines de la vertu !… Mais, Messieurs, personne n’en sait rien, ou plutôt nous ne savons qu’une seule chose, c’est que chacun la trouve dans les inspirations de son cœur. Parmi les dix ou vingt théories philosophiques sur les fondements du devoir, il n’y en a pas une qui supporte l’examen. La signification transcendante de l’acte vertueux est précisément qu’en le faisant, on ne pourrait pas bien dire pourquoi on le fait. Il n’y a pas d’acte vertueux qui puisse raisonnablement se déduire. Le héros, quand il se met à réfléchir, trouve qu’il a agi comme un être absurde, et c’est justement pour cela qu’il a été un héros. Il a obéi à un ordre supérieur, à un oracle infaillible, à une voix qui commande de la façon la plus claire, sans donner ses raisons.\par
Prenons donc la vertu de quelque côté qu’elle vienne et sous quelque costume qu’elle se présente. Il y a, vous disais-je, beaucoup de vertu dans notre monde ; il n’y en a pas tant cependant que l’on puisse impunément se montrer difficile et faire passer à chacun un examen sur les motifs pour lesquels il est vertueux. Ne nous privons d’aucun auxiliaire utile. Vertu laïque, vertu congréganiste, vertu philosophique, vertu chrétienne ; vertu d’ancien régime, vertu de régime nouveau ; vertu civique, vertu cléricale ; prenons tout, croyez-moi ; il y en aura assez, il n’y en aura pas trop pour les rudes moments que la conscience humaine peut avoir à traverser. Plus j’y réfléchis, messieurs, plus je trouve que le {\persName baron Montyon}, à qui l’on reproche souvent d’être parti des principes d’une philosophie un peu superficielle, a obéi au contraire à une pensée très profonde. Il a vu le lien étroit qu’il y a entre la vertu et le talent ; il a vu que la vertu est un genre charmant de littérature. Selon votre vieille et bonne manière d’entendre les choses, la littérature n’est pas seulement ce qui s’écrit ; le grand politique qui résout avec éclat les problèmes de son temps, l’homme du monde qui représente bien l’idéal d’une société brillante et polie, n’eussent-ils pas écrit une ligne, sont de votre ordre. Qui fait le bien en est aussi. Dans ce genre, il est vrai, vous ne prenez pas vos lauréats pour confrères ; mais la confiance que le public vous témoigne est quelque chose de touchant. On vous regarde comme des connaisseurs en fait de vertu, on suppose que vous en avez des réserves, si bien que, quand on en veut, c’est à vous qu’on s’adresse. Permettez-moi de vous rappeler un souvenir de ces derniers mois. Une pauvre jeune fille très vertueuse meurt, laissant deux couverts et un petit sucrier d’argent qu’elle avait achetés de ses économies. Elle aimait beaucoup ce petit sucrier, qui représentait pour elle des privations, et, se voyant mourir, elle souffrait à l’idée qu’il passerait en des mains peut-être moins pures que les siennes. Elle stipule donc, dans son testament, que les deux couverts et le sucrier seraient légués à une jeune fille vertueuse et pratiquant la piété catholique. Le digne exécuteur testamentaire, ne sachant trop où chercher une personne qui remplît ces conditions, eut l’idée de s’adressera vous. Messieurs. Il vint à vous comme à un bureau de vertu. Je n’étais pas à la séance quand l’affaire est revenue ; je crois que les règles établies ne vous ont point permis d’accepter. Je l’ai regretté ; peut-être, en nous entendant avec {\persName M. le curé de Saint-Germain-des-Prés} pour la condition du catholicisme, aurions-nous pu mettre en repos l’âme de la pauvre fille et l’assurer que son petit ménage, auquel elle tenait tant, passerait entre les mains d’une personne partageant toutes ses idées et toutes ses vertus.\par
On dirait, en lisant les œuvres d’imagination de nos jours, qu’il n’y a que le mal et le laid qui soient des réalités. Quand donc nous fera-t-on aussi le roman réaliste du bien ? Le bien est tout aussi réel que le mal ; les dossiers que vous m’avez chargé de lire renferment autant de vérité que les abominables peintures dont malheureusement nous ne pouvons contester l’exactitude. {\persName Emmeline Nadaud} existe aussi bien que telle héroïne pervertie de tel roman pris surnature. Qui nous fera un jour le tableau du bien à Paris ? Qui nous dira la lutte de tant de vertus pauvres, de tant de mères admirables, de sœurs dévouées ? Avons-nous donc tant d’intérêt à prouver que le monde où nous vivons est entièrement pervers ? Non, grâce à la vertu, la Providence se justifie ; le pessimisme ne peut citer que quelques cas bien rares d’êtres pour lesquels l’existence n’ait pas été un bien. Un dessein d’amour éclate dans l’univers ; malgré ses immenses défauts, ce monde reste après tout une œuvre de bonté infinie.
\section[{Discours lors de la distribution des prix du lycée Louis-le-Grand}]{Discours lors de la distribution des prix du lycée {\orgName Louis-le-Grand}}\renewcommand{\leftmark}{Discours lors de la distribution des prix du lycée {\orgName Louis-le-Grand}}


\dateline{7 août 1883}

\salute{Jeunes Élèves,}
\noindent C’est sans doute le voisinage de notre vieux {\orgName Collège de France} et de votre maison, la première de toutes en noblesse universitaire, qui m’a valu l’honneur d’être désigné par {\persName M. le ministre de l’instruction publique} pour présider à cette cérémonie. Vos triomphes d’hier me rendent fier d’un tel choix. Chaque année, c’est dans une plus forte proportion que votre supériorité s’accentue. Que de couronnes jeunes élèves ! Quelle ardeur, quel goût du travail elles supposent ! Quel prix elles donnent aux récompenses qui vont vous être décernées aujourd’hui ! Je ne veux pas retarder, par un long discours, le moment que vous attendez avec un si légitime empressement. La parole aujourd’hui doit être à vos succès. Le sens de cette fête annuelle des bonnes études vient d’ailleurs de vous être indiqué d’une façon si judicieuse, qu’à peine est-il nécessaire d’ajouter quelques mots.\par
On vous l’a dit avec une parfaite raison : la culture rationnelle de l’esprit, le perfectionnement de l’être intellectuel et moral ne s’improvisent pas. Il y faut une gymnastique, des exercices longtemps continués sous des maîtres expérimentés. Le progrès dans les sociétés modernes se fait par la raison réfléchie. Autrefois, une sorte de génie spontané, aidé par la rudesse des mœurs et l’inconscience des masses, créait ces grands développements politiques et religieux dont les conséquences nous régissent encore à beaucoup d’égards. La barbarie fondait autrefois ; elle fondait, avec une solidité qui ne saurait plus être égalée, des édifices sombres, majestueux, incommodes, durables ; trop durables même, car ils devenaient bientôt gênants pour ceux qui ne les avaient pas bâtis, et souvent ils s’imposaient trop à l’avenir. La raison cultivée fondera seule désormais. Elle élèvera des constructions plus légères, mais plus faciles à modifier, moins massives, mais aussi moins tyranniques pour ceux qui en hériteront.\par
Le problème du gouvernement des sociétés devient de plus en plus un problème scientifique, dont la solution suppose l’exercice des plus rares facultés de l’esprit. La guerre, l’industrie, l’administration économique sont maintenant des sciences compliquées. Ces fonctions sociales, auxquelles on suffisait autrefois avec du courage, de l’élégance et de l’honnêteté, supposent aujourd’hui des têtes puissantes, capables d’embrasser à la fois beaucoup d’idées et de les tenir toutes en même temps fixées sous le regard. On se plaint souvent que la force devienne l’unique reine du monde. Il faudrait ajouter que la grande force de nos jours, c’est la culture de l’esprit à tous ses degrés. La barbarie est vaincue sans retour, parce que tout aspire à devenir scientifique. La barbarie n’aura jamais d’artillerie, et, si elle en avait, elle ne saurait pas la manier. La barbarie n’aura jamais d’industrie savante, de forte organisation politique ; car tout cela suppose une grande application intellectuelle. Or la barbarie n’est pas capable d’application intellectuelle. L’habitude de l’application s’acquiert par les fortes disciplines, dont l’éducation scientifique et littéraire possède le secret.\par
Ce n’est pas de nos jours, assurément, que ce privilège de la culture intellectuelle a commencé. Sans parler de l’antiquité, le XVI\textsuperscript{e}, le XVII\textsuperscript{e} et le XVII\textsuperscript{e} siècles virent se constituer une {\orgName Europe} maîtresse du monde, au nom d’une civilisation supérieure. Depuis cent ans, le mouvement s’est accéléré, bien que l’organisation intérieure des nations civilisées ait été profondément modifiée. Les sociétés actuelles ne peuvent plus compter uniquement, comme celles d’autrefois, sur les qualités héréditaires de quelques familles choisies, sur des institutions tutélaires, sur des organismes politiques où la valeur du cadre était souvent fort supérieure à celle des individus. La culture de l’individu est devenue, chez nous, une nécessité de premier ordre. Ce que faisaient autrefois l’hérédité du sang, les usages séculaires, les traditions de famille et de corporations, il faut le faire de nos jours par l’éducation.\par
L’importance de l’instruction publique se trouve ainsi en quelque sorte décuplée. La lutte pour la vie s’est transportée sur le terrain de l’école. La race la moins cultivée sera infailliblement supprimée, ou, ce qui à la longue revient au même, rejetée au second plan par la race la plus cultivée. Le soin de l’instruction publique dans un État deviendra ainsi une préoccupation au moins égale à celle de l’armement et de la production de la richesse. Une nation, en effet, combat et produit par les individus qui la composent. Or l’individu, c’est l’instruction qui le crée, au moins pour une moitié. Il y a sans doute le don inné, que rien ne remplace ; mais le don inné, sans l’instruction, reste stérile, improductif, comme un bloc aurifère non exploité.\par
Tenez donc pour décisives, jeunes élèves, les années où vous êtes, et que trop souvent on considère comme des années sacrifiées. Des devoirs austères vous attendent, et nous manquerions de sincérité si nous ne vous faisions voir dans les récentes modifications de la société humaine qu’une diminution des obstacles à vaincre et, en quelque sorte, un dégrèvement des charges de la vie. La liberté est en apparence un allégement ; en réalité c’est un fardeau. Voilà justement sa noblesse. La liberté engage et oblige ; elle augmente la somme des efforts imposés à chacun.\par
Considérez la vie qui vous est réservée comme une chose grave et pleine de responsabilités. Est-ce là une raison pour vous envisager comme moins favorisés par le sort que ceux qui vous ont précédés ? Tout au contraire, jeunes élèves. Ne dites jamais, comme les mécontents dont parle le {\persName prophète d’Israël} : \emph{« Nos pères ont mangé le raisin vert, et les dents de leurs fils sont agacées. »} Votre part est la bonne, et je vois mille raisons de vous porter envie, non seulement parce que vous êtes jeunes et que la jeunesse est la découverte d’une chose excellente, qui est la vie, mais parce que vous verrez ce que nous ne pourrons voir, vous saurez ce que nous cherchons avec inquiétude, vous posséderez la solution de plusieurs des problèmes politiques sur lesquels nous hésitons parce que les faits n’ont point encore parlé assez clairement. Préparez-vous à porter dans ces grandes luttes la part virile de votre raison, cultivée par la science, et de votre courage, mûri par une saine philosophie.\par
Votre âge ne vous permet pas l’hésitation. Nul n’a tremblé en entrant dans la vie. Une sorte d’aveuglement, habilement ménagé par la nature, vous présente l’existence comme une proie désirable, que vous aspirez à saisir. De plus sages que moi vous prémuniront contre la part d’illusion que suppose votre jeune ardeur. Ils vous annonceront des déconvenues ; ils vous diront que la vie ne tient pas ce qu’elle promet, et que, si on la connaissait quand on s’y engage, on n’aurait pas pour y entrer le naïf empressement de votre âge. Pour moi, je vous l’avoue, tel n’est pas mon sentiment. La vie, qui est là devant vous comme un pays inconnu et sans limites, je l’ai parcourue ; je n’en attends plus grand chose d’imprévu ; ce terme, que vous croyez à l’infini, je le vois très près de moi. Eh bien ! la main sur la conscience, cette vie, dont il est devenu à la mode de médire, je l’ai trouvée bonne et digne du goût que les jeunes ont pour elles.\par
La seule illusion que vous vous fassiez, c’est que vous la supposez longue. Non ; elle est très courte ; mais à cela près, je vous l’assure, il est bon d’avoir vécu, et le premier devoir de l’homme envers l’infini d’où il sort, c’est la reconnaissance. La généreuse imprudence qui vous fait entrer sans une ombre d’arrière-pensée dans la carrière au bout de laquelle tant de désabusés déclarent n’avoir trouvé que le dégoût, est donc très philosophique à sa manière. C’est vous qui avez raison. Allez de l’avant avec courage ; ne supprimez rien de votre ardeur ; ce feu qui brûle en vous, c’est l’esprit même qui, répandu providentiellement au sein de l’humanité, est comme le principe de sa force motrice. Allez, allez, ne perdez jamais le goût de la vie. Ne blasphémez jamais la bonté infinie d’où émane votre être, et, dans l’ordre plus spécial des faveurs individuelles, bénissez le sort heureux qui vous a donné une patrie bienfaisante, des maîtres dévoués, des parents excellents, des conditions de développement où vous n’avez plus à lutter contre l’antique barbarie.\par
La joyeuse ivresse du vin nouveau de la vie, qui vous rend sourds aux plaintes pusillanimes des découragés, est donc légitime, jeunes élèves. Ne vous reprochez pas devons y abandonner. Vous trouverez l’existence savoureuse, si vous n’attendez pas d’elle ce qu’elle ne saurait donner. Quand on se plaint de la vie, c’est presque toujours parce qu’on lui a demandé l’impossible. Ici, croyez tout à fait l’expérience des sages. Il n’y a qu’une base à la vie heureuse, c’est la recherche du bien et du vrai. Vous serez contents de la vie si vous en faites bon usage, si vous êtes contents de vous-mêmes. Une sentence excellente est celle-ci : \emph{« Cherchez d’abord le royaume du ciel ; tout le reste vous sera donné par surcroît. »}\par
Dans une circonstance analogue à celle d’aujourd’hui, il y a quarante-trois ans, l’illustre {\persName M. Jouffroy} adressait aux élèves du {\orgName lycée Charlemagne} ces sévères paroles : \emph{« C’est notre rôle à nous, à qui l’expérience a révélé la vraie vérité sur les choses de ce monde, de vous la dire. Le sommet de la vie vous en dérobe le déclin ; de ses deux pentes vous n’en connaissez qu’une, celle que vous montez ; elle est riante, elle est belle, elle est parfumée comme le printemps. Il ne vous est pas donné, comme à nous, de contempler l’autre, avec ses aspects mélancoliques, le pâle soleil qui l’éclairé et le rivage glacé qui la termine. »}\par
Non, jeunes élèves ! C’est trop triste. Le soleil n’est jamais pâle ; quelquefois seulement il est voilé. Parce qu’on vieillit, a-t-on le droit de dire que les fleurs sont moins belles et les printemps moins radieux ? Est-ce que, par hasard, on voudrait se plaindre de ce qu’on n’est pas immortel ici-bas ? Quel non-sens, juste ciel ! Entre toutes les fleurs, et Dieu sait s’il en est de belles (quel monde admirable que celui de la fleur !), il n’y en a qu’une seule qui soit à peu près sans beauté : c’est une fleur jaune, sèche, raide, étiolée, d’un luisant désagréable, qu’on appelle bien à tort immortelle. Ce n’est vraiment pas une fleur. J’aime mieux la rose, quoiqu’elle ait un défaut, c’est de se faner un peu vite.\par
Et puis, hâtons-nous de le dire, cette vie de quatre jours produit des fruits qui durent : la vertu, la bonté, le dévouement, l’amour de la patrie, la stricte observation du devoir. Voilà, si vous savez donner une règle supérieure à votre vie, ce qui ne vous manquera jamais. Croyez à une loi suprême de raison et d’amour qui embrasse ce monde et l’explique. Soyez assurés que la meilleure part est celle de l’honnête homme, et que c’est lui, après tout, qui est le vrai sage. Évitez le grand mal de notre temps, ce pessimisme qui empêche de croire au désintéressement, à la vertu. Croyez au bien ; le bien est aussi réel que le mal, et seul il fonde quelque chose ; le mal est stérile. Ceux d’entre vous qui ont une prière, dont ils feront aujourd’hui le bonheur en lui apportant leurs couronnes, sauront me comprendre. Que toujours votre mère soit au centre de votre vie. Ne faites jamais rien sans qu’elle vous approuve. Exposez-lui vos raisons ; si elles sont bonnes, vous l’amènerez facilement à être de votre avis. On est toujours bien éloquent auprès d’une mère qu’on aime.\par
Vous verrez le XX\textsuperscript{e} siècle, jeunes élèves. Ah ! voilà, je l’avoue, un privilège que je vous envie ; vous verrez de l’imprévu. Vous entendrez ce qu’on dira de nous, vous saurez ce qu’il y aura eu de fragile ou de solide dans nos rêves. Croyez-moi, soyez alors indulgents. Ce pauvre XIX\textsuperscript{e} siècle dont on dira tant de mal, aura eu ses bonnes parties, des esprits sincères, des cœurs chauds, des héros du devoir. Les générations qui se succèdent sont en général injustes les unes pour les autres. Vous êtes la pépinière du talent de l’avenir. Je me figure voir assis là, parmi vous, le critique qui, vers 1910 ou 1920, fera le procès du XIX\textsuperscript{e} siècle. Je vois d’ici son article (permettez-moi un peu de fantaisie) ; \emph{« Quel signe du temps, par exemple ! Quel complet renversement de toutes les saines notions des choses ! Quoi ! n’eut-on pas ridée, en 1883, de désigner pour présider à notre distribution des prix, au {\orgName lycée Louis-le-Grand}, un homme, inoffensif assurément, mais le dernier qu’il aurait fallu choisir à un moment où il s’agissait avant tout de relever l’autorité, de se montrer ferme et de faire chaleureusement le convicium seculi ? Il nous donna de bons conseils ; mais quelle mollesse ! quelle absence de colère contre son temps ! »} Voilà ce que dira le critique conservateur du XX\textsuperscript{e} siècle. Mon Dieu ! il n’aura peut-être pas tout à fait tort. Je voudrais seulement qu’il n’oublie pas d’ajouter quel plaisir j’eus à me trouver parmi vous, combien vos marques de sympathie m’allèrent au cœur, combien le contact de votre jeunesse me raviva et me réjouit.\par
Ce qu’on appelle indulgence n’est, le plus souvent, que justice. On reproche à l’opinion sa mobilité ; hélas ! jeunes élèves, ce sont les choses humaines qui sont mobiles. La largeur d’esprit n’exclut pas de fortes règles de conduite. Tenez toujours invinciblement pour la légalité. Défendez jalousement votre liberté, et respectez celle des autres. Gardez l’indépendance de votre jugement ; mais n’émigrez jamais de votre patrie, ni de fait, ni de cœur. Consolez-vous en tenant ferme à quelque chose d’éternel. Tout se transformera autour de vous. Vous serez peut-être les témoins des changements les plus considérables qu’ait présentés jusqu’ici l’histoire de l’humanité. Mais il y a une chose sûre, c’est que, dans tous les états sociaux que vous pourrez traverser, il y aura du bien à faire, du vrai à chercher, une patrie à servir et à aimer.\par
Venez maintenant, jeunes élèves, recevoir les récompenses que vous avez si bien méritées.
\section[{Discours prononcé à Tréguier}]{Discours prononcé à {\placeName Tréguier}}\renewcommand{\leftmark}{Discours prononcé à {\placeName Tréguier}}


\dateline{2 août 1884}

\salute{Messieurs et amis,}
\noindent Que je vous remercie de m’avoir enlevé, moi déjà si peu enlevable, à cet éternel fauteuil où je m’ankylose, à ces douleurs par lesquelles je me laisse envahir, à ces hésitations d’où j’ai besoin d’être tiré de force ! Je vous devrai la joie d’avoir revu une fois encore ma vieille ville de {\placeName Tréguier}, à laquelle m’attachent de bien chers souvenirs. Si courtes et si rares ont été les apparitions que j’y ai faites depuis que le vaste monde m’a pris, que je peux dire qu’il y a quarante ans que je l’ai quittée. Quarante ans, quel long espace dans la vie humaine ! Que de choses changent en quarante ans ! Mais, nous autres, {\orgName Bretons}, nous sommes tenaces, et, hier, en faisant le tour du cloître et de la cathédrale, en visitant ma vieille maison, je me disais que rien absolument, n’était changé ni en moi, ni en ce qui m’entourait.\par
Ah ! sûrement, dans les hommes, le changement est énorme. Presque tous ceux qui ont entouré mon enfance ont disparu : ma mère, à qui je dois le fond de ma nature, qui est la gaieté ; ma sœur, si pure, si dévouée, ne sont plus aux lieux où je les ai vues autrefois vivre et m’aimer. Ma bonne, Marie-Jeanne, est morte, il y a quelques années. Et mes excellents maîtres, à qui je dois ce qu’il y a de meilleur en moi…, un seul d’entre eux, je crois, et des plus méritants, vit encore. {\persName M. Pothier} et {\persName M. Duchesne}, qui m’apprirent les deux choses qui m’ont été les plus utiles, le latin et les mathématiques ; {\persName M. Pasco}, si plein de bonté ; {\persName M. Auffret}, le principal, qui me fit comprendre ce que peut avoir de charme austère une vie grave, consacrée à la raison et au devoir, tous ces hommes excellents ne sont plus. Ils ont disparu, après avoir fait le bien et compté dans une tradition de sérieux et de vertu. Mais le cadre où ils vécurent dure toujours. Hier, je retrouvais presque pierre pour pierre le {\placeName Tréguier} d’autrefois ; j’aurais pu mettre un nom sur chaque maison. La cathédrale a toujours son adorable légèreté. L’herbe qui pousse sur les vieilles tombes du cloître est toujours aussi touffue ; j’aurais pu croire que c’est la même vache qui la broute depuis quarante ans.\par
Et alors, je me suis demandé si moi-même j’avais changé, et je me suis répondu bien fermement : Non. De corps, oui sans doute ; et pourtant, même sur ce point, j’aurais beaucoup à dire. Enfant, j’étais difficile à remuer, ne jouant jamais, porté à m’asseoir et m’acoquiner. La distance de la maison au collège, je la parcourais deux fois par jour, sans me détourner d’un pas à droite ni à gauche. Les rhumatismes, qui me rendent maintenant la marche si difficile, je les avais déjà. — Quant à l’âme, oh ! ç’a toujours bien été la même. Ce petit écolier consciencieux, laborieux, désireux de plaire à ses maîtres, c’est bien moi tout entier ; j’étais doué dès lors ; j’avais tout ce que j’ai maintenant ; je n’ai rien acquis depuis, si ce n’est l’art douteux de le faire valoir. Il vaudrait sûrement mieux vivre et mourir solitaire ; mais on n’est pas le maître ; le monde vous prend par les cheveux, fait un peu de vous ce qu’il veut.\par
Ce que j’ai toujours eu, c’est l’amour de la vérité. Je veux qu’on mette sur ma tombe (ah ! si elle pouvait être au milieu du cloître ! mais le cloître, c’est l’église, et l’église, bien à tort, ne veut pas de moi), je veux, dis-je, qu’on mette sur ma tombe : \foreign{{\itshape Veritatem dilexi}}. Oui, j’ai aimé la vérité ; je l’ai cherchée ; je l’ai suivie où elle m’a appelé, sans regarder aux durs sacrifices qu’elle m’imposait. J’ai déchiré les liens les plus chers pour lui obéir. Je suis sûr d’avoir bien fait. Je m’explique. Nul n’est certain de posséder le mot de l’énigme de l’univers, et l’infini qui nous enserre échappe à tous les cadres, à toutes les formules que nous voudrions lui imposer. Mais il y a une chose qu’on peut affirmer ; c’est la sincérité du cœur, c’est le dévouement au vrai et le sentiment des sacrifices qu’on a faits pour lui. Ce témoignage, je le porterai haut et ferme sur ma tête au jugement dernier.\par
En cela, j’ai été vraiment Breton. Nous sommes une race naïve, qui a la simplicité de croire au vrai et au bien. Avec le nécessaire et une petite part d’idéal, nous sommes heureux comme des rois. Cela sert médiocrement à réussir ; cela sert à quelque chose de mieux ; cela rend heureux. Oui, notre gaieté vient de notre honnêteté. Dans un temps où le mal général est le dégoût de la vie, nous continuons à croire que la vie vaut la peine qu’on en poursuive le but idéal. Nous sommes les vrais fils de {\persName Pelage}, qui niait le péché originel. Une critique que m’adressent toujours les protestants est celle-ci : « Qu’est-ce que {\persName M. Renan} fait du péché ? » Mon Dieu ! je crois que je le supprime. Je ne comprends rien à ces dogmes tristes. Je vous l’avoue, plus j’y réfléchis, plus je trouve que toute la philosophie se résume dans la bonne humeur. Nous autres, Celtes, nous ne serons jamais pessimistes, nihilistes. Sur le bord de ces abîmes, un sourire de la nature ou d’une femme nous sauverait… Ma mère, mourant à quatre-vingt-sept ans, après une maladie longue et terrible, plaisantait encore une heure avant de mourir.\par
Croyez-moi ; ne changez pas. Vos qualités sont de celles qui reprendront de la valeur. Le monde est en train de se laisser envahir par des races tristes, qui n’ont jamais su ce que c’est que jouir, races dures, sans sympathie, qui n’ont ni l’amour ni l’estime des hommes. Votre santé morale sera le sel de la terre. Vous aurez du talent, quand il n’y en aura plus ; de la gaieté, quand on médira d’elle ; vous aimerez la gloire, l’honneur, le bien, le beau, quand il sera convenu que ce sont là de pures vanités. Sachons être à notre jour des arriérés ; les rôles changent si vite en ce monde ! Ce sont presque toujours ces prétendus arriérés qui fondent ce que les empressés compromettent. Je songe souvent que c’est votre adhésion, en apparence tardive, qui donnera l’existence définitive à ces délicates choses que l’on perd par trop de zèle : un état légal, où l’ordre soit aussi assuré que la liberté ; un état social, où la justice ne soit pas trop violée ; un état religieux, qui donne à l’âme humaine son aliment idéal, sans contrainte officielle ni chimères superstitieuses.\par
Votre place est marquée dans l’exécution de ces grandes œuvres modernes ; car, en même temps que vous êtes du présent, vous tenez fortement au passé. Ce n’est pas le moment de changer ; restez ce que vous êtes. Entrez hardiment, avec votre génie propre, dans le concert de l’œuvre française ; portez-y votre raison, votre maturité. Osez valoir ce que vous valez. Notre cote en ce monde est au-dessous de notre valeur réelle. Nous avons des défauts, cela est hors de doute. Le principal est peut-être de trop douter de nous-mêmes.\par
Croyez-en l’expérience d’un compatriote, qui vous a quittés jeune et qui vous revient vieux, après avoir vu des mondes assez divers. Je ne vous enseignerai pas l’art de faire fortune, ni, comme on dit vulgairement, l’art de faire son chemin ; cette spécialité-là m’est assez étrangère. Mais, touchant au terme de ma vie, je peux vous dire un mot d’un art où j’ai pleinement réussi, c’est l’art d’être heureux. Eh bien ! pour cela les recettes ne sont pas nombreuses ; il n’y en a qu’une, à vrai dire : c’est de ne pas chercher le bonheur ; c’est de poursuivre un objet désintéressé, la science, l’art, le bien de nos semblables, le service de la patrie. À part un très petit nombre d’êtres, dont il sera possible de diminuer indéfiniment le nombre, il n’y a pas de déshérités du bonheur ; notre bonheur, sauf de rares exceptions, est entre nos mains.\par
Voilà le résultat de mon expérience. Je vous le livre pour ce qu’il vaut. J’ai toujours eu le goût de la vie, j’en verrai la fin sans tristesse ; car je l’ai pleinement goûtée. Et je mourrai en félicitant les jeunes ; car la vie est devant eux, et la vie est une chose excellente.\par

\salute{Merci encore, chers amis, de m’avoir procuré cette réunion, qui m’a rajeuni.}
\section[{Discours prononcé à Quimper}]{Discours prononcé à {\placeName Quimper}}\renewcommand{\leftmark}{Discours prononcé à {\placeName Quimper}}


\dateline{17 août 1885}
\noindent Que je suis touché, Messieurs, de vos bonnes paroles, et que je sais gré à nos jeunes amis qui, me rendant breton une fois par année, m’ont fait faire connaissance avec cette ville antique et charmante, que je désirais voir depuis si longtemps. Ainsi va la vie. Né à {\placeName Tréguier}, j’ai poussé mes voyages, du côté de l’orient, jusqu’à {\placeName Antioche}, du côté du nord, jusqu’à {\placeName Tromsoë}, du côté du sud, jusqu’à {\placeName Philé} ; et hier matin encore, je n’avais pas dépassé, du côté de l’ouest, la plage de {\placeName Saint-Michel} en {\placeName Grève}. Notre race est coutumière de courir ainsi le monde quand le devoir l’appelle. Elle a raison. Le monde nous écoute volontiers, quand nous lui parlons de ses intérêts généraux ; car nous avons le don de la sympathie, cette intuition, cette illusion si l’on veut, qui, dans tout homme, je dirai presque dans tout être conscient, nous fait toucher une vie sœur de la nôtre, dans toute fleur nous montre un sourire, dans l’univers entier nous fait voir un grand acte d’amour. Ainsi, tout petits que nous sommes, nous avons notre place dans l’agitation générale, et nous y servons.\par
Laissez-moi même dire que le monde ferait peut-être bien de nous écouter davantage et de tenir plus de compte de nos timides observations. Le mal de notre temps, c’est l’âpreté dans les jugements, quelque chose de rogue et de dur, un ton âpre que l’on aurait tout au plus droit de prendre, si l’on était en possession de la vérité absolue. Et encore !… je crois que celui qui aurait ce privilège serait justement fort modeste. Nous autres, que beaucoup de circonstances ont tenus jusqu’ici en dehors de la grande arène du monde, nous avons des nerfs moins excités, un sens plus rassis. On oublie qu’à la Révolution, la {\placeName Bretagne}, avant la chouannerie, avait été girondine. Nous sommes en tout des modérés. Voilà une attitude que nous ferons bien de garder ; car le siècle, à force d’intransigeance, comme on dit m’a l’air de dégénérer en pugilat. Chacun croit avoir trop raison ; heureux celui qui se résignerait à n’avoir raison que modérément.\par
La démocratie, par exemple, est certainement un des besoins, et des besoins légitimes de notre temps. Eh bien ! je trouve que nous sommes de très bons démocrates. Je ne connais pas de pays qui ait plus que le nôtre le sentiment de l’égalité. Je passe l’été près de {\placeName Perros}, au milieu d’un hameau de très pauvres gens ; notre petite aisance doit leur paraître de la richesse ; mais, comme dit {\persName Dante}, \emph{« cela ne leur abaisse pas le cil »}. Dès que je leur ai parlé breton, ils m’ont tenu absolument pour un des leurs. Comme, dans « les hauts pays » (\foreign{{\itshape er broïo huel}}) où j’ai été, il n’y a qu’à se baisser pour récolter l’or, ils trouvent tout simple que je sois un peu plus riche qu’eux. L’idée ne leur vient pas plus qu’à moi qu’il y ait par ailleurs entre nous quelque différence.\par
Nous passons dans le monde pour d’affreux réactionnaires ; nous sommes, je vous l’assure, de très bons libéraux. Nous, voulons la liberté pour nous et pour les autres. Ce terrible problème religieux qui pèse comme un spectre sur la conscience du XIX\textsuperscript{e} siècle, nous le résoudrions si nous étions seuls au monde. Nous sommes très religieux ; jamais nous n’admettrons qu’il n’y ait pas une loi de l’honnêteté, que la destinée de l’homme soit sans rapport avec l’idéal. Mais nous admettons parfaitement que chacun se taille à sa guise son roman de l’infini. Nous allons plus loin ; nous admettons, pour ceux qui croient bien à tort qu’eux seuls ont raison, le droit à l’intolérance, pourvu que cette intolérance ne se traduise pas en actes contraires à la liberté des autres. La Bretagne a pu à quelques égards paraître superstitieuse ; elle n’a jamais été fanatique. Pour moi, j’aime mieux la superstition que le fanatisme. Toutes nos vieilles races de l’{\placeName Occident} et du {\placeName Nord} ont été ou sont encore superstitieuses ; c’est l’{\placeName Orient}, le mauvais {\placeName Orient} qui est fanatique. Oui, si nous étions maîtres, nous résoudrions le problème de la liberté religieuse, que ne résoudront jamais ni les sectaires ni les irréligieux. Nous ne croyons nullement manquer de respect à nos pères en agissant autrement qu’ils n’ont agi. Nos pères ont fait ce qu’ils ont cru le meilleur en leur temps ; nous faisons de même. Que feraient de notre temps {\persName saint Corentin} et {\persName saint Tugdual} ? Je n’en sais rien vraiment. Tâchons de bien saisir ce que veut l’heure présente. C’est de cette façon que nous avons le plus de chance de nous rencontrer avec eux.\par
Je crois donc qu’en restant fidèles à notre esprit, nous pouvons rendre au monde de réels services. Notre vieux fonds d’honnêteté, du train dont vont les choses, pourra être plus que jamais utile. C’est là une qualité qui jusqu’ici n’a pas fait grande fortune dans le monde. Mon opinion est que sa valeur montera par suite de la rareté de la denrée. Gardons, gardons ce petit capital. Le monde se vide de dévouement, d’esprit de sacrifice. Nous en aurons longtemps à revendre. On aura besoin de nous. Ce n’est pas le moment de changer.\par
J’en dirai autant du courage ; nous sommes tous fils de marins et de soldats ; nos pères ont combattu, ont monté à l’abordage. J’ai voulu m’enquérir de ce qui reste de {\orgName Renans}, dans le {\placeName Goëlo}, le pays d’origine de ma famille. Il y en a encore tout un clan. Ils n’ont pas oublié que leurs aïeux, depuis des siècles, avaient pour profession de casser des têtes d’Anglais ou de se faire casser la leur ; c’était honorable, car c’était réciproque. Sur le bateau-torpille qui est venu, il y a quelques mois, s’amarrer au {\placeName pont de Solferino}, à {\placeName Paris}, il y avait un torpilleur du nom de Renan. Ce doit être un bien honnête homme, et qui ne sait pas ce que c’est que la réclame ; car il n’est pas venu me voir. Je vous demanderai tout à l’heure de boire un verre de cet excellent cidre à la santé de {\persName Renan le torpilleur}. Ce brave homme a eu vraiment une idée de génie. Quel état merveilleusement approprié à nos aptitudes ! Cette invention a vraiment l’air d’avoir été faite pour nous ! Elle attribue sa pleine valeur aux deux grandes choses de ce monde, l’intelligence (c’est-à-dire la science) et le courage. Je voudrais que l’état de torpilleur devînt la profession noble par excellence, celle des grands idéalistes, à qui l’on donnerait le moyen de rêver tranquillement en ce monde, sauf à les engager, aux heures héroïques, avec quatre ou cinq chances contre une de n’en pas revenir.\par
Voici encore une autre découverte que j’ai faite, au pays de Goëlo. On me parla d’un Renan qui était mort en laissant un avoir d’environ cinquante mille francs. Cela me parut surprenant ; ils sont tous pauvres comme {\persName Job}. On m’ajouta qu’il avait tout donné à l’Église, ce qui ne m’étonna pas ; mais je voulus savoir comment il avait gagné ce capital énorme. Eh bien ! je vous assure que voici ce qu’on m’a répondu : il était taupier ; il se faisait payer cinq sous par taupe qu’il prenait. Cela m’a fait faire des retours sur moi-même. Moi aussi, j’ai été bon taupier ; j’ai détruit quelques bêtes souterraines assez malfaisantes. J’ai été un torpilleur à ma manière ; j’ai donné quelques secousses électriques à des gens qui auraient mieux aimé dormir. Je n’ai pas manqué à la tradition des bonnes gens de {\placeName Goëlo}.\par
Voilà pourquoi, bien que brisé de corps avant l’âge, j’ai gardé jusqu’à la vieillesse une gaieté d’enfant, comme les marins, une facilité étrange à me contenter. Un critique de beaucoup de talent me soutenait dernièrement que ma philosophie m’obligeait à être toujours éploré. Il me reprochait comme une hypocrisie une façon allègre de prendre la vie, dont il ne voyait pas les vraies causes. Eh bien ! je vais vous les dire. Je suis très gai ; d’abord sans doute, parce que, m’étant très peu amusé quand j’étais jeune, j’ai gardé à cet égard toute ma fraîcheur d’illusions ; puis, voici qui est plus sérieux, je suis sûr d’avoir fait en ma vie une bonne action ; j’en suis sûr. Je ne demanderais pour récompense que de recommencer ; je ne me plains que d’une seule chose, c’est d’être vieux dix ans trop tôt. Je ne suis pas un homme de lettres ; je suis un homme du peuple ; je suis l’aboutissant de longues files obscures de paysans et de marins. Je jouis de leurs économies de pensée ; je suis reconnaissant à ces pauvres gens qui m’ont procuré, par leur sobriété intellectuelle, de si vives jouissances.\par
Là est le secret de notre jeunesse. Nous sommes prêts à vivre, quand tant de gens ne parlent que de mourir. Le groupe humain auquel nous ressemblons le plus, et qui nous comprend le mieux, ce sont les {\orgName Slaves} ; car ils sont dans une position analogue à la nôtre, neufs dans la vie et antiques à la fois. Nous croyons à la race, car nous la sentons en nous.\par
C’est ce que je me disais, ces jours-ci à Perros, en retrouvant toute sorte de vieilles petites connaissances, des oiseaux, des fleurs poussant sur les vieux murs, dont j’avais oublié le nom, et, en particulier, ce rocher du groupe des Sept-Îles qui est, au printemps, rempli d’innombrables oiseaux de mer. J’ai demandé à un de mes confrères du {\orgName Muséum} la vérité sur ce point. Ce sont les oiseaux des {\placeName îles Shetland}, qui viennent déposer leurs œufs en cette terre attiédie par le gulf-stream ; là ils éclosent ; puis les oisillons, tout d’une volée, regagnent leurs rochers des {\placeName mers du Nord}. Ah ! voyez, je vous prie, comme ces petits êtres sortent de l’œuf maternel avec une profonde sagesse ! On ne comprend rien à l’humanité, si l’on s’en tient aux vues d’un individualisme étroit. Ce qu’il y a de meilleur en nous vient d’avant nous.\par
Une race donne sa fleur, quand elle émerge de l’oubli. Les brillantes éclosions intellectuelles sortent d’un vaste fonds d’inconscience, j’ai presque envie de dire de vastes réservoirs d’ignorance. Ne craignez pas que je vienne vous engager à cultiver une herbe qui pousse fort bien d’elle-même ; malgré l’instruction intégrale et obligatoire, il y aura toujours assez d’ignorance. Mais je redouterais pour l’humanité le jour où la clarté aurait pénétré toutes ses couches. D’où viendrait le génie, qui est presque toujours le résultat d’un long sommeil antérieur ? D’où viendraient les sentiments instinctifs, la bravoure, qui est si essentiellement héréditaire, l’amour noble, qui n’a rien à faire avec la réflexion, toutes ces pensées, qui ne se rendent pas compte d’elles-mêmes, qui sont en nous sans nous, et forment la meilleure partie de l’apanage d’une race et d’une nation ?\par
Merci donc, chers amis, d’avoir ramené pour moi une si précieuse occasion, de me réjouir avec vous et de me retremper au vieil esprit. Votre jeunesse m’enchante, me soutient. Merci, dignes représentants d’une ville qui me sera désormais chère, de cet accueil si aimable. Merci, cher {\persName Ilémon} ; merci, cher {\persName Luzel}, de cette fête qui m’a touché profondément au cœur. Je ne sais si j’en verrai d’autres de ce genre. Comme mon âge me conseille de penser à la vie future, je me surprends parfois occupé à garnir ma mémoire des pensées qui devront l’occuper durant toute l’éternité. Eh bien ! je vous assure que cette journée sera des meilleures entre celles dont je veux me souvenir. Votre cordialité, vos marques d’estime comptent entre les récompenses de ma vie, et, quoi qu’en disent les critiques qui voudraient me confiner dans un {\itshape De profundis} perpétuel, je continuerai d’être gai, puisque votre accueil m’assure que, depuis quarante-sept ans que j’ai quitté la {\placeName Bretagne}, je n’ai pas en somme démérité.
\section[{Discours à la conférence Scientia : Banquet en l’honneur de M. Berthelot}]{Discours à la conférence Scientia : \\
Banquet en l’honneur de {\persName M. Berthelot}}\renewcommand{\leftmark}{Discours à la conférence Scientia : \\
Banquet en l’honneur de {\persName M. Berthelot}}


\dateline{26 novembre 1885}
\noindent Quelle joie vous m’avez préparée, Messieurs, en pensant à moi pour être l’interprète de vos sentiments envers l’homme illustre que vous fêtez aujourd’hui, vous vous êtes souvenus d’une vieille amitié qui, ces jours-ci justement, atteint à sa quarantième année. Oui, c’était au mois de novembre 1845. Je venais d’accomplir de pénibles sacrifices. En sortant du séminaire {\placeName Saint-Sulpice}, le monde s’offrait à moi comme un vaste désert d’hommes ; ma récompense fut de vous trouver, cher ami, dans cette petite pension de la {\placeName rue de l’Abbé-de-l’Épée} (alors {\placeName rue des Deux-Églises}), où y exerçais au pair les fonctions de répétiteur. Vous faisiez votre classe de philosophie au {\orgName collège Henri IV} ; vous eûtes, je crois, le prix d’honneur au grand concours à la fin de l’année. J’avais quatre ans de plus que vous. Deux ou trois mots que nous échangeâmes discrètement nous eurent bientôt prouvé que nous avions ce qui crée le principal lien entre les hommes, je veux dire la même religion.\par
Cette religion, c’était le culte de la vérité. Dès cette époque, nous étions des {\itshape nazirs}, des gens qui ont fait un vœu, les hommes-liges de la vérité. Notre part d’héritage était choisie, et cette part était la meilleure. Ce que nous entendions par la vérité, en effet, c’était bien la science. Les premiers jugements de l’homme sur l’univers furent un tissu d’erreurs. C’est la science rationnelle qui a rectifié les aperceptions erronées de l’humanité. La science est donc l’unique maîtresse de la vérité. Au bout de quarante ans, je trouve encore que nous eûmes pleinement raison de nous attacher à elle. Il y a trois belles choses, disait {\persName saint Paul} : la foi, l’espérance, la charité ; la plus grande des trois, c’est la charité. Il y a trois grandes choses, pouvons-nous dire à notre tour, le bien, la beauté, la vérité ; la plus grande des trois, c’est la vérité. Et pourquoi ? Parce qu’elle est vraie. La vertu et l’art n’excluent pas de fortes illusions. La vérité est ce qui est. En ce monde, la science est encore ce qu’il y a de plus sérieux. La philosophie du doute subjectif élève ici ses objections contre la légitimité même des facultés rationnelles de l’esprit. Cela ne m’a jamais beaucoup touché, je l’avoue. Oh ! si je n’avais d’autre doute que celui-là, comme je me sentirais léger ! La science est un ensemble dont toutes les parties se contrôlent. Je crois absolument vrai ce qui est prouvé scientifiquement, c’est-à-dire par l’expérience rigoureusement pratiquée.\par
Que la science rigoureuse ne réponde pas à toutes les questions que lui pose notre légitime curiosité, cela est sûr. Mais qu’y faire ? Mieux vaut savoir peu de choses, mais les savoir effectivement, que de s’imaginer savoir beaucoup de choses et se repaître de chimères. Que de bases, d’ailleurs, établies et solidement établies ! La terre est un globe d’environ trois mille lieues de diamètre, et dont la densité approche de celle du fer. Voilà qui est incontestable ! Eh bien ! cela fixe singulièrement mes idées. Je préfère cette vérité à une série de propositions métaphysiques plus ou moins dénuées de sens. Il ne pouvait pas y avoir d’exercice normal de l’esprit avant qu’on ne fût fixé sur des points comme celui-là. Quand on croyait que la terre était une plaine, recouverte par une voûte en berceau, où les étoiles filaient, à quelques lieues de nous, dans des rainures, il était vraiment bien superflu de raisonner sur l’homme et sa destinée. Nous devons plus à l’astronomie qu’à aucune théologie du monde. Supposons une planète dont l’atmosphère fût laiteuse, si bien que les habitants de cette planète ne pussent constater l’existence d’aucun corps déterminé dans l’espace. Les habitants de cette planète seraient les plus bornés des êtres. Ils seraient emprisonnés fatalement dans l’hypothèse géocentrique, dans les idées, familières à la vieille théologie, d’un développement divin se déroulant à leur profit exclusif.\par
J’estime donc très peu fondée l’éternelle jérémiade de certains esprits sur les prétendus paradis dont nous prive la science. Nous savons plus que le passé ; l’avenir saura plus que nous. Vive l’avenir ! Vous aurez largement contribué, cher ami, à ce progrès de l’esprit, où la part de notre siècle, quoi qu’on dise, sera belle. Dans la plus philosophique peut-être des sciences, la chimie, vous avez porté les limites de ce que l’on sait au-delà du point où s’étaient arrêtés vos devanciers. Dilater le {\itshape pomœrium}, c’est-à-dire reculer l’enceinte de la ville, était, à {\placeName Rome}, l’acte de mémoire le plus envié. Vous avez dilaté, cher ami, au secteur où vous travaillez, le {\itshape pomœrium} de l’esprit humain. Vivez longtemps pour la science, pour ceux qui vous aiment ; vivez pour notre chère patrie, qui se console de bien des défaillances en montrant au monde quelques enfants tels que vous.
\section[{Discours à l’Association des étudiants}]{Discours à l’{\orgName Association des étudiants}}\renewcommand{\leftmark}{Discours à l’{\orgName Association des étudiants}}


\dateline{15 mai 1886}

\salute{Messieurs,}
\noindent Je vous remercie de m’avoir invité à venir me réjouir aujourd’hui avec vous ! Notre fête serait plus complète, si vos maîtres attristés par de récents désordres\footnote{ Les troubles de l’{\orgName École de pharmacie}.} avec lesquels vous n’avez aucune solidarité, n’avaient cru devoir manquer à cette réunion. Nous le regrettons vivement ; mais, en un sens, tant mieux pour moi ; je vous aurai tout entiers. Votre jeunesse me réchauffe et me ravive. Il est si doux, quand les fenêtres se ferment d’un côté, de les voir s’ouvrir de l’autre ! J’ai coutume de dire : « Heureux les jeunes ! car la vie est devant eux. » Des deux parties du programme de la vie scolaire, travailler beaucoup, s’amuser beaucoup, je n’ai connu, à vrai dire, que la première. Le temps où les autres s’amusent fut pour moi un temps d’ardent travail intérieur. J’eus tort peut-être ; il en est résulté que, sur mes vieux jours, au lieu d’être, selon l’usage, un conservateur rigide, un moraliste austère, je n’ai pas su me défendre de certaines indulgences que les puritains ont qualifiées de relâchement moral. J’aurais mieux fait peut-être de me réjouir quand j’étais jeune et de chantera ma guise le Gaudeamus des clercs du moyen âge :\par


\begin{verse}
Gaudeamus igitur, dum juvenes sumus ;\\
Post jocundam juventutem,\\
Post moleslam senectutem,\\
Nos habebit humus.\\
\end{verse}

\noindent Ce qu’il y a de sûr, c’est qu’une des moitiés de l’activité de votre âge n’empêche pas l’autre. La joie et le travail sont deux choses saines et qui s’appellent réciproquement.\par
Oui, travaillez, travaillez sans cesse, et, pourtant, amusez-vous ; ne vous fatiguez jamais. Ce qui fatigue, c’est la contention, c’est l’effort pénible. Laissez la pensée venir à vous, avec son vêtement naturel, qui est la parole ; ne l’appelez pas, ne la pressez pas. Je vais vous donner à cet égard quelques-unes de mes recettes. Reposez-vous d’un travail par un autre ; ayez des objets d’étude assez divers. Les cases du cerveau occupées par un travail laissent des vides, qui sont avantageusement remplis par un autre travail. Il y a un beau mot d’un vieux rabbin du premier siècle. On lui reprochait de faire déborder le vase de la Loi en y mettant trop de préceptes : \emph{« Dans un tonneau plein de noix, répondit-il, on peut encore verser plusieurs mesures d’huile de sésame. »} Que c’est bien dit ! Oui, on peut faire à la fois des choses très diverses, à condition de les caser dans les interstices les unes des autres. Le temps qu’on donne au travail n’est pas seulement celui qu’on passe devant sa table et son écritoire. Il faut savoir travailler toujours, ou, pour mieux dire, il faut s’arranger pour que le temps du travail et celui du repos ne soient pas distincts. Pendant que vous causez, si la conversation ne vous passionne pas beaucoup, suivez vos idées. De même, pendant vos promenades, pendant vos repas, pendant tous les actes de la vie.\par
Ne mettez pas de bornes à votre curiosité ; aspirez à tout savoir ; les limites viendront d’elles-mêmes. C’est ici surtout que je vous porte envie.\par
Dans l’humanité, les derniers venus sont les privilégiés. Que de choses vous saurez que nous ne saurons jamais ! Que de problèmes dont j’achèterais la solution par des années de vie, si j’en avais à ma disposition, seront clairs pour vous ! Les sociétés modernes sortiront-elles de la crise où elles sont engagées ? Les questions sociales sont-elles des impasses comme les essais manqués du XIV\textsuperscript{e} et du XV\textsuperscript{e} siècle, ou bien trouveront-elles des solutions applicables ? Que sera le monde en 1920 ou 1930 ? Et, dans l’ordre purement scientifique, à quelles vues arrivera-t-on sur la race, l’embryon, l’espèce, l’individu, la vie, la conscience ? En histoire, de quelles admirables découvertes vous jouirez, si ces belles recherches se continuent. Dans cinquante ans, la littérature babylonienne comptera des vingtaines de volumes, et on la lira. A l’heure qu’il est, nous avons deux inscriptions hébraïques anciennes, qui sont pour le pauvre historien comme des phares lumineux dans cette obscure antiquité. Vous verrez peut-être un temps où l’on en connaîtra des dizaines. Voilà un bonheur dont vous vous ne doutiez pas. Ah ! que je vous porte envie ! Que je voudrais ressusciter dans cinquante ans !\par
Soyez toujours de très honnêtes gens. Vous ne pourriez pas bien travailler sans cela. Il me semble qu’on ne saurait bien travailler, ni même bien s’amuser, que si on est un honnête homme. La gaieté de la conscience suppose une bonne vie. Il y a des sujets délicats ; il est convenu qu’on n’en parle pas. Mais vous me témoignez tant de confiance que je vous dirai tout ce que je pense. Ne profanez jamais l’amour ; c’est la chose la plus sacrée du monde ; la vie de l’humanité, c’est-à-dire de la plus haute réalité qu’il y ait, en dépend. Regardez comme une lâcheté de trahir la femme qui vous a ouvert pour un moment le paradis de l’idéal ; tenez pour le plus grand des crimes de vous exposer aux malédictions futures d’un être qui vous devrait la vie et qui, par votre faute peut-être, serait voué au mal. Vous êtes des hommes d’honneur ; regardez cet acte, qu’on traite avec tant de légèreté, comme un acte abominable. Mon opinion est que la règle morale et légale du mariage sera changée. La vieille loi romaine et chrétienne paraîtra un jour trop exclusive, trop étroite. Mais il y a une vérité qui sera éternelle, c’est que des relations des deux sexes résultent des obligations sacrées, et que le premier des devoirs humains est de s’interdire, dans l’acte le plus gros de conséquence pour l’avenir du monde, une coupable étourderie.\par
N’oubliez jamais que, par votre éducation exceptionnelle, vous avez des devoirs plus stricts que les autres envers la société dont vous faites partie. Pauvre {\orgName France} !… Vous la verrez, j’en suis sûr, vengée, florissante, apaisée. Ayez une règle absolue : c’est de suivre la {\orgName France}, c’est-à-dire la légalité, malgré toutes les objections, toutes les répugnances, toutes les antipathies. Que ce soit là le panache blanc qui vous guide. Ne vous brouillez jamais avec la {\orgName France}. Donnez-lui toujours de bons conseils ; ne vous fâchez pas si elle ne les suit pas. Elle a peut-être eu ses raisons pour cela. Quelque chose de mystérieux agite ce peuple ; suivez-le, même quand il refuse de vous écouter, quand il s’abandonne aux plus indignes. Ne vous croyez pas obligés de prendre des airs consternés, parce que les choses ne vont pas de la façon que vous croyez la meilleure. Que de fois on arrive à se féliciter que l’avis qu’on avait émis n’ait pas été suivi et que les événements vous aient donné tort.\par
En politique, si c’est à une brillante carrière que vous tenez, ne suivez pas trop mes conseils. J’ai visé par-dessus tout, dans ma vie, à conserver le repos de ma conscience, et j’y ai réussi. Je suis, par essence, un légitimiste ; j’étais né pour servir fidèlement, et avec toute l’application dont je suis capable, une dynastie ou une Constitution tenues pour autorité incontestée. Les révolutions m’ont rendu la tâche difficile. Mon vieux principe de fidélité bretonne fait que je ne m’attache pas volontiers aux gouvernements nouveaux. Il me faut une dizaine d’années pour que je m’habitue à regarder un gouvernement comme légitime. Et, de fait, c’est au bout de ce temps que les gouvernements peuvent se mettre à essayer quelque chose de bon. Jusque-là ils ne font que payer leurs dettes de premier établissement. Mais voyez la fatalité ! Ce moment où je me réconcilie, et où les gouvernements commencent de leur côté à devenir assez aimables avec moi, est justement le moment où ils sont sur le point de tomber et où les gens avisés s’en écartent. Je passe ainsi mon temps à cumuler des amitiés fort diverses et à escorter de mes regrets, par tous les chemins de l’{\placeName Europe}, les gouvernements qui ne sont plus. Je leur suis plus fidèle que leurs affidés. Si la République venait jamais à tomber (ce qu’à Dieu ne plaise !), voyez quel serait mon sort. Moi qui ne suis pas un républicain {\itshape a priori}, qui suis un simple libéral, s’accommodant volontiers d’une bonne monarchie constitutionnelle, je serais plus fidèle à la République que des républicains de la veille. Je porterais le deuil du régime que je n’ai pas contribué à fonder. Or, j’ai soixante-trois ans ; vous voyez combien mon cas est étrange ; les légitimistes à ma façon se préparent en notre siècle de cruels embarras, car il faudrait aussi que les gouvernements fussent fidèles à eux-mêmes, et ils ne le sont guère, il faut l’avouer.\par
Ne venez donc pas me demander des conseils d’habileté ; je suis peu qualifié pour cela. Mais si vous voulez que je vous indique des moyens pour être en paix avec vous-mêmes, je puis vous en donner. Mettez-vous toujours en règle avec la patrie. Ne demandez jamais aucun mandat ; n’en refusez aucun ; ne déclinez pas la responsabilité ; mais ne la cherchez pas. De la sorte, on vous laissera bien tranquilles. Vous aurez votre repos, et vous vous rendrez en même temps ce témoignage que vous avez fait ce qui dépendait de vous. Vous pourrez vous dire intérieurement : \foreign{{\itshape Dixi, salvavi animam meam}}. Nous devons à la patrie d’être à sa disposition pour la servir ; mais nous ne sommes pas obligés de sortir de notre caractère pour obtenir ses mandats. Ne croyons jamais être nécessaires à la partie ; il suffit qu’à un jour donné nous puissions lui être utiles.\par
En somme, le temps où vous vivez n’est pas plus mauvais que bien d’autres. Le sol tremble quelquefois ; mais les tremblements de terre n’empêchent pas le pied du {\placeName Vésuve} d’être un lieu fort agréable. Préparez-vous pour la vie une ample provision de bonne humeur. Hors les cas de désastre national, faites une part au sourire et à l’hypothèse où ce monde ne serait pas quelque chose de bien sérieux. Il est sûr, en tout cas, qu’il est charmant tel qu’il est. Soyez contents de vivre, comme nous sommes contents d’avoir vécu. La vieille gaieté gauloise est peut-être la plus profonde des philosophies. Ne vous corrigez pas trop radicalement de ce qu’on appelle les défauts français ; ces défauts sont susceptibles de devenir un jour des qualités.\par
Pardonnez-moi ce long sermon laïque. En vieillissant, on devient donneur de conseils. Quand vous nous succéderez sur la scène de la vie, soyez indulgents pour la génération qui vous précéda. Il y eut dans cette génération beaucoup de goût pour la justice et la vérité. Vous ferez mieux sans doute ; mais souvenez-vous de ceux qui vous ont préparé la voie dans des temps difficiles. Je prie ceux d’entre vous qui ne me verront que cette fois de garder de moi, quand je ne serai plus de ce monde, un souvenir affectueux.
\section[{Adieu à Tourguéneff}]{Adieu à {\persName Tourguéneff}}\renewcommand{\leftmark}{Adieu à {\persName Tourguéneff}}


\dateline{Prononcé à la gare du chemin de fer du Nord, le 1\textsuperscript{er} octobre 1883.}
\noindent Nous ne laisserons point partir sans un adieu le cercueil qui va rendre à sa patrie l’hôte de génie qu’il nous a été donné, pendant de longues années, de connaître et d’aimer. Un maître en l’art de juger les choses de l’esprit vous dira le secret de ces œuvres exquises qui ont charmé notre siècle. {\persName Tourguéneff} fut un écrivain éminent ; ce fut surtout un grand homme. Je ne vous parlerai que de son âme, telle qu’elle m’est apparue dans la douce retraite que lui avait ménagée parmi nous une illustre amitié.\par
{\persName Tourguéneff} reçut du décret mystérieux qui fait les vocations humaines le don noble par excellence : il naquit essentiellement impersonnel. Sa conscience ne fut pas celle d’un individu plus ou moins bien doué par la nature : ce fut, en quelque sorte, la conscience d’un peuple. Avant de naître, il avait vécu des milliers d’années ; des suites infinies de rêves se concentraient au fond de son cœur. Aucun homme n’a été à ce point l’incarnation d’une race entière. Un monde vivait en lui, parlait par sa bouche ; des générations d’ancêtres perdues dans le sommeil des siècles, sans parole, arrivaient par lui à la vie et à la voix.\par
Le génie silencieux des masses collectives est la source de toutes les grandes choses. Mais la masse n’a pas de voix. Elle ne sait que sentir et bégayer. Il lui faut un interprète, un prophète qui parle pour elle. Quel sera ce prophète ? Qui dira ces souffrances, niées par ceux qui ont intérêt à ne pas les voir, ces secrètes aspirations, qui dérangent l’optimisme béat des satisfaits ? Le grand homme, Messieurs, quand il est en même temps homme de génie est homme de cœur. Voilà pourquoi le grand homme est le moins libre des hommes. Il ne fait pas, il ne dit pas ce qu’il veut. Un Dieu parle en lui ; dix siècles de douleur et d’espérance l’obsèdent et le commandent. Parfois il lui arrive, comme au {\persName Voyant} des antiques récits de la Bible, qu’appelé pour maudire, il bénit ; selon l’{\persName Esprit} qui souffle, sa langue ne lui obéit pas.\par
C’est l’honneur de cette grande race slave, dont l’apparition sur l’avant-scène du monde est le phénomène le plus inattendu de notre siècle, de s’être tout d’abord exprimée par un maître aussi accompli. Jamais les mystères d’une conscience obscure et encore contradictoire ne furent révélés avec une aussi merveilleuse sagacité. C’est que {\persName Tourguéneff} à la fois sentait et se regardait sentir ; il était peuple et il était d’élite. Il était touché comme une femme et impassible comme un anatomiste, désabusé comme un philosophe et tendre comme un enfant. Heureuse la race qui, à ses débuts dans la vie réfléchie, a pu être représentée par de telles images, naïves autant que savantes, réelles et mystiques en même temps ! Quand l’avenir aura donné la mesure des surprises que nous réserve cet étonnant génie slave, avec sa foi fougueuse, sa profondeur d’intuition, sa notion particulière de la vie et de la mort, son besoin de martyre, sa soif d’idéal, les peintures de Tourguéneff seront des documents sans prix, quelque chose comme serait, si on pouvait l’avoir, le portrait de tel homme de génie dans son enfance. Ce rôle d’interprète d’une des grandes familles de l’humanité, Tourguéneff en voyait la périlleuse gravité. Il sentait qu’il avait charge d’âmes, et, comme il était honnête homme, il pesait chacune de ses paroles, il tremblait pour ce qu’il disait et ce qu’il ne disait pas.\par
Sa mission fut ainsi toute pacificatrice. Il était comme le {\persName Dieu} du \emph{Livre de Job} qui \emph{« fait la paix sur les hauteurs »}. Ce qui ailleurs produisait le déchirement devenait chez lui le principe d’harmonie. Dans sa large poitrine, les contradictoires s’embrassaient ; l’anathème et la haine étaient désarmés par les magiques enchantements de son art.\par
Voilà pourquoi il est la gloire commune d’écoles entre lesquelles existent tant de dissentiments. Une grande race, divisée parce qu’elle est grande, retrouve en lui son unité. Frères ennemis que sépare une diverse façon de concevoir l’idéal, venez tous à sa tombe ; tous vous avez droit de l’aimer, car il vous appartenait à tous, il vous tenait tous dans son sein. Admirable privilège du génie ! Les côtés répulsifs des choses n’existent pas pour lui. En lui tout se réconcilie : les partis les plus opposés se réunissent pour le louer et l’admirer. Dans la région où il nous transporte, les mots dont s’irrite le vulgaire perdent leur venin. Le génie fait en un jour ce que font les siècles. Il crée une atmosphère de paix supérieure où ceux qui furent adversaires se trouvent en réalité avoir été collaborateurs ; il ouvre l’ère de la grande amnistie, où ceux qui se sont combattus dans l’arène du progrès dorment côte à côte en se donnant la main.\par
Au-dessus de la race, en effet, il y a l’humanité ou, si l’on veut, la raison. Tourguéneff fut d’une race par sa manière de sentir et de peindre ; il appartient à l’humanité tout entière par une haute philosophie, envisageant d’un œil ferme les conditions de l’existence humaine et cherchant sans parti pris à savoir la réalité. Cette philosophie aboutissait chez lui à la douceur, à la joie de vivre, à la pitié pour les créatures, pour les victimes surtout. Cette pauvre humanité, souvent aveugle assurément, mais si souvent aussi trahie par ses chefs, il l’aimait ardemment. Il applaudissait à son effort spontané vers le vrai et le bien. Il ne gourmandait pas ses illusions ; il ne lui en voulait pas de se plaindre.\par
La politique de fer qui raille ceux qui souffrent n’était pas la sienne. Aucune déception ne l’arrêtait. Comme l’univers, il eut recommencé mille fois l’œuvre manquée ; il savait que la justice peut attendre ; on finira toujours par y revenir. Il avait vraiment les paroles de la vie éternelle, les paroles de paix, de justice, d’amour et de liberté. Adieu donc, grand et cher ami. Ce qui va s’éloigner de nous n’est que la cendre. Ce qu’il y eut d’immortel en toi, ton image spirituelle nous restera. Puisse ton cercueil être, pour ceux qui viendront le baiser, un gage d’amour en une même foi au progrès libéral. Et, quand tu reposeras dans la terre de ta patrie, puissent tous ceux qui salueront la tombe avoir un souvenir sympathique pour la terre lointaine où tu trouvas tant de cœurs pour te comprendre et t’aimer.
\section[{Discours pour l’inauguration du médaillon de MM. Michelet, Quinet, Mickiewicz}]{Discours pour l’inauguration du médaillon de MM. {\persName Michelet}, {\persName Quinet}, {\persName Mickiewicz}}\renewcommand{\leftmark}{Discours pour l’inauguration du médaillon de MM. {\persName Michelet}, {\persName Quinet}, {\persName Mickiewicz}}


\dateline{Prononcé au {\placeName Collège de France}, 13 avril 1884.}

\salute{Messieurs\footnote{ Ce remerciement s’adresse à un comité d’anciens auditeurs de MM. {\persName Michelet}, {\persName Quinet}, {\persName Mickiewicz}, qui avaient pris l’initiative du médaillon commémoratif.},}
\noindent Le {\orgName Collège de France} est infiniment touché de l’heureuse pensée que vous avez eue de fixer dans cette salle, qui a entendu tant d’éloquentes paroles, le souvenir de trois de nos confrères les plus illustres. Divers autant que possible par les dons supérieurs qui leur furent départis, MM. {\persName Michelet}, {\persName Quinet}, {\persName Mickiewicz} furent associés par l’admiration de leurs auditeurs en une sorte de trinité, qu’un médaillon bien connu a consacrée. Suspendu au-dessus de cette chaire, avec l’expression écrite de votre piété, ce médaillon sera, pour l’avenir, le témoin des sentiments que laissa dans l’âme de toute une génération l’enseignement des maîtres qui nous ont précédés.\par
Cette chaire, qui est là devant nous, en effet, vous devez la reconnaître : c’est bien là que s’assirent les maîtres éminents dont vous avez si précieusement gardé la mémoire. Quels hommes, Messieurs, et quel moment dans notre histoire que celui qui les rassembla dans la même enceinte ! Ils touchèrent à la fois les pôles opposés de la nature humaine. — L’un, véritable enchanteur, né pour montrer les mille secrets de magie que la {\persName fée Mélusine} a cachés dans les entrailles de notre race ; divin par le timbre exquis de sa voix ; divin par son sourire ; capable de saisir d’une oreille également fine l’harmonie des sphères, que nul n’entend plus, et le bruit souterrain de la fourmi qui sort pour son travail du matin ; interprète inspiré du génie de la {\orgName France}, du cœur de la femme, de l’âme du peuple. — L’autre grave et austère, devenu étranger aux réalités vulgaires par la noble obsession de l’idée du devoir, sévère pour ce qu’il aimait ; dévoré d’une soif ardente d’idéal religieux, que l’état présent du monde ne satisfaisait guère. — Le troisième, plein de la sève primitive des grandes races au lendemain de leur éveil, sorte de géant lithuanien, fraîchement né de la terre ou plutôt récemment inspiré du ciel, joignant aux intuitions des prophètes parfois leurs illusions, mais toujours plein d’une imperturbable foi dans l’avenir de l’humanité et de sa race, idéaliste obstiné malgré toutes les déceptions, optimiste vingt fois relaps. Tous trois eurent le don suprême des grands hommes, le charme, l’art de gagner les cœurs. Ceux qui les ont connus les aimèrent comme on aime la femme dont on rattache le souvenir aux premiers enivrements de la jeunesse, à la première découverte de la vie.\par
Votre démarche d’aujourd’hui, Messieurs, n’est-elle pas la meilleure preuve de l’impression que faisaient ces grands initiateurs ? A quarante ans d’intervalle, cette impression s’est réveillée si forte en vous, que vous avez voulu la consacrer, à la manière des anciens, par une image de bronze et une couronne. Au seuil de la vieillesse, vous avez été fidèles au culte de vos jeunes années ; vous vous êtes souvenus de ces voix tantôt profondes, tantôt charmeresses, qui pénétrèrent au plus intime de votre être et s’y enfoncèrent comme des flèches. C’est que les leçons de nos trois glorieux confrères prenaient l’homme tout entier. Ils voulaient, non seulement éclairer les esprits, mais conquérir les âmes. L’amélioration des mœurs et le progrès de la raison étaient pour eux inséparables. Leur cours était une prédication ; la religion de plusieurs en est sortie.\par
La loi constitutive de notre collège, Messieurs, fait une place à ces hautes personnalités et leur assure la liberté qui leur est due. La pensée qui guida le {\persName roi François I\textsuperscript{er}} dans la création du {\orgName Collège de France} restera toujours vraie. À côté de l’enseignement régulier, visant à des cours complets, correspondant à des épreuves, à des diplômes professionnels, il y a un enseignement plus original, plus personnel, plus complètement livré à la volonté du maître ; un enseignement, dont les règlements doivent être assez larges pour offrir à toute nouveauté sérieuse le moyen de s’exprimer, pour accorder à la science qui naît toute la liberté de ses tâtonnements ; un enseignement, dis-je, où le savant, qui cherche le vrai par l’analyse la plus patiente, siège à côté du penseur, qui rapporte des hauts sommets où il demeure le souffle de l’infini. Comme une humanité complète, nous n’excluons rien, si ce n’est l’absurde ; nous sommes ouverts à tout, excepté à l’irrationnel. Notre collection de physique contient, à côté de la table grossière sur laquelle {\persName Ampère} établit le premier télégraphe électrique, les élégants appareils par lesquels {\persName M. Regnault} poussait les lois de la nature à la dernière précision. Dans la plus petite de nos salles, l’illustre {\persName Eugène Burnouf}, devant trois ou quatre élèves, consacrait une heure à fixer le sens d’un passage sanscrit non compris jusque-là, pendant que notre glorieux triumvirat poursuivait le neuf à sa manière ; en morale. Messieurs, le neuf est ce qui va au cœur. Cet amphithéâtre, où retentirent de si chaudes paroles, est aujourd’hui un laboratoire scientifique. {\persName M. Michelet}, qui s’intéressait si vivement au progrès de la science, applaudirait s’il pouvait voir les belles recherches qu’a faites ici notre confrère {\persName M. Marey} sur la marche et sur les phénomènes du mouvement volontaire.\par
Voilà ce qui explique cette variété infinie que présente notre {\orgName Collège}, et qui parfois empêche d’en voir l’unité. Il n’y a qu’une vérité ; mais on y atteint par les procédés les plus divers. Sûrement, l’objet spécial de ce {\orgName Collège}, c’est la création de la science. Le cours le plus fécond, au {\orgName Collège de France}, est celui où les secrets nouveaux de la nature, de l’histoire ou de la vie sont communiqués par le professeur à quelques auditeurs, déjà savants, capables de le comprendre. Mais nous ne nous méprenons pas sur la manière dont se produit et se conserve la conscience du genre humain. Une vérité n’est complète que quand elle vit dans la foule, quand elle a mille voix, quand elle agit, quand elle court dans toute une nation, quand elle circule comme un fluide nerveux dans tous les organes de l’humanité. {\persName Abélard} n’a été ce qu’il est dans l’histoire que parce qu’il a eu pour élève le révolutionnaire {\persName Arnauld de Bresse}. Nous sommes toujours calmes ; mais il est inévitable qu’on se passionne autour de nous. Ce qu’on dit des paisibles études et des temples sereins de la science est un honnête lieu commun. Non, nous sommes posés en signe de guerre, et la paix n’est pas notre sort. Au XV\textsuperscript{e} siècle, qui fut notre siècle héroïque, les colères que nous excitions allaient jusqu’à l’assassinat. Nous avons eu deux victimes de la Saint-Barthélemy. C’est là notre gloire ; chaque fois que l’esprit humain subit une éclipse, nous sommes frappés.\par
La liberté est notre essence. Plutôt que de nous en passer, nous aimons mieux être brisés. Quand le vent de l’esprit mesquin et du dogmatisme intolérant souffle sur le monde, nous attendons… Et on revient toujours à nous ; car nous sommes les découvreurs de la vérité, et la découverte de la vérité, soit dans l’ordre de nature, soit dans l’ordre historique, soit dans l’ordre social, est ce qui importe le plus à l’homme. La société, privée de mainteneurs jurés de la vérité, est livrée sans défense aux monstres que l’humanité a vaincus, mais qui tendent sans cesse à la ressaisir, l’ignorance, la brutalité, la superstition.\par
La base d’une nation civilisée, c’est la science ; or, la science n’existe qu’à la condition d’être progressive, de se renouveler sans cesse par des découvertes. La découverte se fait à voix basse, par le maître courbé sur l’expérience ou le document, expliquant sa trouvaille à quatre ou cinq personnes. Mais la proclamation se fait à voix haute par le talent ou le génie. Ces deux divisions du travail intellectuel sont nécessaires. Toutes les parties de la raison se tiennent ; l’amour même ne s’en sépare pas. C’est en maintenant cet esprit que nous pensons remplir notre mission, et, comme la {\orgName France} ne se distingue pas pour nous de l’idéal et du devoir, c’est en étant d’âpres chercheurs de vérité dans tous les ordres que nous croyons servir la patrie et être à notre manière des citoyens utiles.\par
Maîtres illustres qui fûtes, au jour de votre vie terrestre, des porteurs de vérité, de cette vérité qui est à la fois lumière et chaleur, apprenez-nous à marcher sur vos traces. Vous renouvelâtes, en votre temps, les miracles que vit au moyen âge cette {\placeName montagne Sainte-Geneviève}, quand toutes les nations de l’{\placeName Europe} venaient autour d’{\persName Abélard}, ou bien au {\placeName Clos-Bruneau}, à la {\placeName rue du Fouarre}, chercher les principes de la liberté dans la communauté de l’esprit humain. Ces hommages qui viennent aujourd’hui, de toutes les parties régénérées de l’{\placeName Europe}, se mêler à notre fête, montrent que votre parole eut le grand caractère du vrai ; elle fut universelle, elle remua toutes les races. Nous ne sommes pas changés. D’autres ont pu changer dans le monde ; mais rassurez-vous, nous resterons incorrigibles. Nous ne séparerons jamais l’intérêt de la {\orgName France} de celui de la vérité. Jamais nous n’envisagerons la science, la civilisation, la justice comme l’œuvre d’une seule race ou d’un seul peuple. Nous persisterons à croire que toutes les nations y servent, chacune selon son génie. En cultivant la science, nous ne dirons jamais : « Notre science » ; le vrai, le bien et le beau étant, à nos yeux, l’apanage de tous. Le pédantisme, qui scinde l’esprit humain en compartiments, et introduit dans le domaine de l’âme des espèces de cloisons étanches ; l’hypocrisie, qui accapare la Providence et dit avec affectation : « Notre Dieu » (comme si l’on pouvait dire : Notre absolu ! Notre infini !) n’auront jamais nos sympathies. Votre vieux {\orgName Collège}, chers maîtres, restera ce qu’il fut toujours, l’asile de la recherche et de la pensée indépendantes, la forteresse de l’honnêteté intellectuelle. Comme vous, nous nous laisserions arracher de notre chaire, plutôt que de dire autre chose que ce que nous avons résolu de dire. Votre génie planera sur ces lieux, pleins encore de votre parole. Le souvenir de votre courage et de votre sincérité, ravivé par cette image, nous soutiendra dans l’accomplissement de notre grand devoir, le culte absolu de la vérité.
\section[{Discours prononcé aux funérailles de M. Stanislas Guyard, Professeur au Collège de France}]{Discours prononcé aux funérailles de {\persName M. Stanislas Guyard}, \\
Professeur au {\orgName Collège de France}}\renewcommand{\leftmark}{Discours prononcé aux funérailles de {\persName M. Stanislas Guyard}, \\
Professeur au {\orgName Collège de France}}


\dateline{9 septembre 1884}
\noindent Quelle fatalité, Messieurs, que la mort soit venue prendre parmi nous le plus jeune, le plus désigné pour les grandes œuvres, le plus aimé ! Six mois à peine se sont écoulés depuis que {\persName Stanislas Guyard} remplaçait, dans la chaire d’arabe, au {\orgName Collège de France}, le regretté {\persName Defrémery}, et voilà que le coup le plus imprévu nous l’enlève, au milieu d’une féconde activité ! Il n’avait que trente-huit ans. En peu d’années, il a su remplir le cadre d’une longue vie scientifique ; il en a fait assez pour sa tâche virile ; mais nous qui fondions sur lui tant d’espérances, nous qui nous consolions de vieillir en voyant grandir à côté de nous cette laborieuse et vaillante jeunesse, c’est pour nous qu’est le deuil. Depuis le jour où j’ai serré sa main sur son lit d’agonie, sans qu’elle m’ait répondu, il me semble que nos études ont été atteintes dans quelque organe vivant, près du cœur.\par
Le goût de {\persName Stanislas Guyard} pour les études orientales data de sa première jeunesse. Son esprit ferme et sagace lui révéla tout d’abord qu’en fait de sciences historiques, c’était là qu’il y avait le plus de travail utile à dépenser, le plus de vrai à découvrir. Il fit à sa vocation les plus grands sacrifices, et il fallut la ténacité extrême de sa volonté pour continuer les études de son choix, malgré la situation extérieurement défavorable où sont placées des études capitales, il est vrai, par leurs résultats philosophiques, mais qui n’ont presque point d’application professionnelle. Longtemps il n’eut pour récompense que l’estime des témoins de ses travaux ; cette estime du moins lui fut bien vite acquise. Nous éprouvâmes tous une sensible joie, quand nous vîmes venir à notre {\orgName Société asiatique} ce jeune homme sérieux, ardent, consciencieux, ami passionné du vrai, ennemi de tout charlatanisme et de toute hypocrisie. On sentait, derrière sa modestie, les qualités essentielles du savant, la droiture et l’indépendance du caractère, la sincérité absolue de l’esprit.\par
Bientôt des travaux de haute valeur se succédèdèrent. {\persName Guyard} s’attaqua successivement aux problèmes les plus difficiles des langues et des littératures de l’{\placeName Asie occidentale}. Les questions délicates relatives au khalifat de {\placeName Bagdad}, l’histoire des {\orgName Ismaéliens} et des sectes incrédules au sein de l’islam, la métrique arabe, où tant de choses nous surprennent, les formes bizarres de ce qu’on appelle les pluriels brisés, chapitre si curieux de la théorie comparée des langues sémitiques, furent pour notre savant collègue l’objet de travaux approfondis, toujours fondés sur l’étude directe des sources. Sa lecture de l’arabe était rapide et sûre. Quand une société composée des arabisants les plus éminents de toute l’{\placeName Europe} se partagea le travail immense d’une édition complète du texte des \emph{Annales} de {\persName Tabari}, {\persName M. Guyard} se chargea d’un volume, et c’est grâce à lui que la {\orgName France} a été représentée dans cette entreprise monumentale. L’achèvement de la traduction de la \emph{Géographie} d’{\persName Aboulféda}, commencée par {\persName M. Reinaud}, lui fut confié. Attaché comme auxiliaire au \emph{Recueil des historiens arabes des croisades}, publié par l’{\orgName Académie des inscriptions et belles-lettres}, {\persName M. Guyard} a été, en ce travail, pour {\persName M. Barbier de Meynard}, le plus précieux de ses collaborateurs.\par
Tous les grands problèmes l’attiraient. L’intérêt hors ligne que présente l’assyriologie le frappa ; il est probable que, s’il eût vécu davantage, il eût de plus en plus tourné ses études de ce côté. Il voyait l’immense avenir d’une science qui nous fournira un jour sur la haute antiquité des lumières inattendues. Son nom figurera parmi ceux des vaillants travailleurs qui auront marché au premier rang à la conquête de ce monde nouveau.\par
Comme professeur d’abord à l’{\orgName École des hautes études}, puis parmi nous, {\persName M. Stanislas Guyard} n’a pas rendu de moindres services. Il savait s’attacher ses élèves, leur inspirer le goût du travail qui le remplissait lui-même. Son assiduité était admirable ; il aimait à dépasser à cet égard les obligations qui nous sont imposées. L’amour du bien public, le sentiment abstrait du devoir formaient l’unique mobile de sa vie. Il était, dans les relations privées, d’une douceur charmante ; ses frères, ses sœurs, l’adoraient. Tout ceux qui l’ont approché ont gardé de lui l’impression de quelque chose de supérieur.\par
Hélas ! il était trop parfait, et quand on est arrivé à ce degré extrême de désintéressement, la terre ne vous retient plus assez ; on est trop prêt, au moindre signe, à la quitter. La soif du travail allait chez lui jusqu’à l’obsession ; il avait tué en lui la possibilité du repos. Quand il pensait à tant de belles choses qui seraient à faire, quand il voyait la moisson si belle et les ouvriers si peu nombreux, il était pris d’une sorte de fièvre ; il assumait pour lui la tâche de dix autres. La fatigue amena bientôt l’insomnie, l’incapacité du travail. L’incapacité du travail, c’était pour lui la mort. Vivre sans penser, sans chercher, lui parut un supplice. Imitez-le en tout, jeunes amis, excepté en cette espèce de tension dangereuse qui fait qu’on ne peut plus associer au devoir le sourire, le divertissement honnête, le plaisir de contempler un monde, où, à côté de tant de parties sombres, il y a des touches si lumineuses. \foreign{{\itshape Indulgere genio}} est un art que notre ami ne savait pas, ne voulait pas savoir ; il ne pécha que par excès d’amour pour le bien. La vie était pour lui tellement identifiée avec le travail qu’un ordre de repos lui sembla insupportable. La perspective de vivre sans travailler lui parut un cauchemar plus affreux que la mort.\par
Et puis il y avait en tout cela quelque chose de plus profond encore. L’espèce de providence inconsciente qui veille à la destinée des grandes âmes semble faire en sorte que la récompense ne leur vienne que tard et quand elle a perdu son attrait. Il en fut ainsi pour Guyard. La vie s’était toujours montrée à lui par le côté austère. Quand elle commença à lui sourire, le stoïcien eut des scrupules ; il crut qu’il allait perdre de sa noblesse en acceptant le prix qu’il avait si bien mérité ; il sembla se dérober, se soustraire…\par
On ne se console de ces dures leçons infligées à notre orgueil qu’en songeant que la science est éternelle, qu’elle n’est point assujettie aux lois fatales de notre fragilité. {\persName Guyard} a largement travaillé pour sa part au grand édifice de la science moderne, dont les profondeurs cachent tant d’efforts anonymes. Ces monuments immenses, ou plutôt ces collines bâties, qui couvrent la plaine de {\placeName Babylone}, sont faites en briques de quelques centimètres de long. Courte est une vie scientifique ; mais immense est un capital où rien ne se perd.\par
Pauvre cher ami, entré maintenant dans la sérénité absolue, donne le repos à ce cœur inquiet, à cette conscience timorée, à cette âme toujours craintive de ne pas assez bien faire. Tu as été un bon ouvrier dans l’œuvre excellente qui se construit avec nos efforts. Ta tristesse fut seule parfois un peu injuste, injuste pour la Providence, injuste pour ton siècle et pour toi-même. Sois tranquille ; ta gerbe fleurira ; tu as montré la route ; ce que tu n’as pu faire, d’autres le feront. Ta vie sera pour tous ceux qui t’ont connu une leçon de désintéressement, de patriotisme, de travail et de vertu.
\section[{Discours prononcé au nom de l’Académie des inscriptions et belles-lettres aux funérailles de M. .Villemain}]{Discours prononcé au nom de l’{\orgName Académie des inscriptions et belles-lettres} aux funérailles de {\persName M. .Villemain}}\renewcommand{\leftmark}{Discours prononcé au nom de l’{\orgName Académie des inscriptions et belles-lettres} aux funérailles de {\persName M. .Villemain}}


\dateline{10 mai 1870}

\salute{Messieurs,}
\noindent L’{\orgName Académie des inscriptions et belles-lettres} ne saurait rester muette devant cette tombe, près de se refermer sur l’un des hommes qu’elle est le plus fière d’avoir possédés dans son sein. On vient de vous dire avec éloquence les services de l’homme d’État, les qualités de l’orateur, les rares mérites de l’écrivain. Qu’il soit permis à notre compagnie, gardienne des souvenirs et des œuvres du passé, de montrer les sources où {\persName M. Villemain} puisa cette noble et saine activité, cet amour désintéressé du beau, cette libérale conception de la vie, qui le soutint dans sa longue carrière et l’empêcha de défaillir jamais. L’étude de l’antiquité n’était pas pour lui une simple recherche de curiosité ; il demandait aux anciens des exemples et des leçons, des exemples de l’association intime du bien faire et du bien dire, des leçons de goût, de mesure, d’honnêteté, données avec l’autorité du génie. Expressions parfaites d’un idéal où la raison, la vertu et la beauté sont inséparables, les littératures antiques, soit sous leur forme profane, soit sous leur forme chrétienne, étaient pour lui une révélation lumineuse, où il trouvait à toute heure ce qui nourrit l’esprit et réchauffe le cœur. Son goût littéraire n’était pas l’admiration banale qui s’arrête au beau langage ; c’était un salutaire commerce avec un monde plus pur que le nôtre, un sentiment filial pour ces vieux sages dont nous faisons à bon droit les précepteurs et les consolateurs de notre vie.\par
Aux plus brillants succès de son talent d’écrivain et d’orateur, il eût préféré la bonne fortune de découvrir un de ces chefs-d’œuvre de la poésie grecque, modèles d’un art incomparable, où la suprême élégance elle bon sens, loin de s’exclure, se réunissent en une divine harmonie. Ministre de l’instruction publique à une époque où la culture élevée de l’esprit était tenue par la politique, non pour un amusement frivole d’aristocrates oisifs, mais pour un intérêt public, {\persName M. Villemain} n’eut pas de plus constante pensée que de faire rendre aux couvents grecs les chefs-d’œuvre qu’ils pouvaient receler encore, et, si cette investigation, conçue et dirigée par lui, prouva que les chefs-d’œuvre inédits sont devenus bien rares, elle rendit au moins à la science des textes de première importance pour l’histoire de l’esprit humain. Quand le docte {\persName cardinal Maï} arrache aux palimpsestes les pages à demi dévorées de la \emph{République} de {\persName Cicéron}, {\persName M. Villemain} reçoit les feuilles savantes à mesure qu’elles sont imprimées, et en donne cette traduction où l’on trouve revêtues du plus beau style les plus profondes idées qu’on ait jamais émises sur la constitution de la société civile. Il ne séparait pas l’antiquité chrétienne de l’antiquité profane ; le génie des Pères de l’{\orgName Église} trouva en lui pour la première fois un digne interprète. Cette grande école chrétienne du IV\textsuperscript{e} siècle, tout imprégnée encore d’hellénisme et de philosophie, {\persName M. Villemain} la comprit, l’aima, la dépeignit en traits immortels. Ce noble christianisme des {\persName Basile} et des {\persName Chrysostome} était pour lui une autre {\orgName Grèce}, un monde classique en une certaine manière, où son imagination se complaisait. Rien n’était en dehors de sa large et intelligente critique : il était aussi maître de son sujet quand il retraçait l’histoire de la poésie lyrique des {\orgName Grecs} que quand il expliquait {\persName Dante} et la poésie du moyen âge. L’admirateur passionné de la {\placeName Grèce antique} trouvait de la sympathie et de la pitié, même pour cette Grèce en décadence, dont il s’interdisait respectueusement de voir les faiblesses et les décrépitudes.\par
Ainsi soutenu par tous les enseignements du passé, en communion littéraire et philosophique avec ce que l’humanité a produit de bon et de beau, il tenait tête aux défaillances du présent ; il en dominait les tristesses, et, sans dissimuler ses craintes, il accueillait toute pensée d’avenir. Quoique depuis des années sa santé ne lui permît pas de prendre part à nos discussions, nous le sentions présent parmi nous ; tout absent qu’il était, nous recherchions son suffrage. Comme de tous les grands hommes, on peut dire de lui que la mort ne l’atteindra pas. Pour prendre les expressions d’un des auteurs anciens qui lui furent les plus chers : \emph{Quidquid exeo amavimus, quidquid mirati sumus manet}. Parmi les recherches qui conduisent l’esprit humain à la découverte de la vérité, les unes donnent la gloire et l’éclat de la vie ; les autres restent obscures ; mais toutes sont immortelles, — immortelles, même quand elles ne visent qu’à être utiles et qu’elles se renferment dans le cercle étroit des amis de la vérité ; immortelles surtout quand elles ont été, comme celles de notre confrère, éclairées par le rayon du génie.
\section[{Identité originelle et séparation graduelle du judaïsme et du christianisme}]{Identité originelle et séparation graduelle du judaïsme et du christianisme}\renewcommand{\leftmark}{Identité originelle et séparation graduelle du judaïsme et du christianisme}

\begin{center}\emph{(Reproduction sténographique)}\end{center}

\dateline{Conférence faite à la {\orgName Société des études juives}, le 26 mai 1883.}
\noindent Je suis infiniment heureux, monsieur le baron\footnote{ {\persName M. le baron Alphonse de Rothschild} avait ouvert la séance par une allocution.}, des paroles par lesquelles vous avez eu la bonté de m’introduire auprès de cette assemblée. J’ai été, en effet, du nombre de ceux qui ont applaudi tout d’abord à la formation de cette {\orgName Société des études juives}, à laquelle je crois un si grand avenir. J’ai applaudi, en particulier, à l’article de vos statuts qui permet à des personnes étrangères à la {\orgName communauté Israélite} de faire partie de votre {\orgName Société}.\par
Vous avez eu parfaitement raison, Messieurs, d’introduire cette disposition dans votre règlement. Sans doute, les études juives vous appartiennent de plein droit ; mais, permettez-moi de le dire à votre gloire, elles appartiennent aussi à l’humanité. Les recherches relatives au passé Israélite intéressent tout le monde. Toutes les croyances trouvent dans vos livres le secret de leur formation. Celui qui veut fouiller ses origines religieuses arrive nécessairement à l’hébreu. Ces études, tout en étant votre domaine propre, sont donc en même temps le domaine commun de tous ceux qui croient ou qui cherchent. ({\itshape Applaudissements}.)\par
Quelle merveilleuse destinée, en effet, que celle de votre livre sacré, de cette Bible qui est devenue l’aliment intellectuel et moral de l’humanité civilisée ! S’il est une portion du monde qui rappelle peu la {\placeName Judée}, ce sont assurément nos îles perdues de l’{\placeName Occident} et du {\placeName Nord}. Eh bien ! de quoi s’occupe-t-on dans ces pays lointains, habités par des races si différentes de celles de l’{\placeName Orient} ? De quoi s’occupe-t-on ? De la Bible, Messieurs, de la Bible avant tout.\par
Il y a au nord de l’{\placeName Ecosse}, à trente lieues à peu près de la côte, au milieu d’une mer sauvage, un rocher isolé qui pendant la moitié de l’année est presque plongé dans les ténèbres. Cette petite île s’appelle {\placeName Saint-Kilda}. Je lisais dernièrement des renseignements très curieux sur cet îlot, qui pourrait nous fournir des données intéressantes sur la race celtique à son état pur. Pendant des mois entiers, on y est sans relations avec le reste du monde. On doit s’ennuyer beaucoup à {\placeName Saint-Kilda}, et la société doit y être peu variée. ({\itshape Rires}.) Eh bien ! que fait-on sur cette petite terre oubliée ? On lit la Bible du matin au soir ; on cherche à la comprendre.\par
J’ai un peu visité le nord de la {\placeName Scandinavie} ; j’ai aperçu quelques campements de Lapons. Ces Lapons sont à demi civilisés. Ils savent lire maintenant. Eh bien ! Que lisent-ils ? La Bible, toujours la Bible. Ils l’entendent à leur façon, l’interprètent de la manière la plus originale, avec une sorte de passion sombre et d’intelligence profonde.\par
Vous avez donc ce privilège incomparable que votre livre est devenu le livre du monde entier ; par conséquent ne vous en prenez qu’à vous-mêmes si tout le monde veut se mêler à vos études. Vous partagez ce privilège de l’universalité avec une autre race, qui, elle aussi, a imposé sa littérature à tous les siècles et à tous les pays, c’est la {\placeName Grèce}. Assurément nous nous plaindrions si les {\orgName Grecs modernes} venaient dire : « Nous seuls, avons le droit de nous occuper du grec. — Pardon, répondrions-nous, tout le monde admire votre ancienne littérature, tout le monde a le droit de l’étudier. » La Bible, de même, étant le bien commun de l’humanité, appartient à l’espèce humaine tout entière ; nous avons droit de collaborer avec vous. — Nous vous remercions donc. Messieurs, de ce que vous avez bien voulu nous admettre, comme des Samaritains de bonne volonté, à travailler à côté de vous à l’œuvre qui nous intéresse tous également. ({\itshape Très bien ! très bien ! et vifs applaudissements}.)\par
Il est si vrai que les études hébraïques sont la substruction commune des études religieuses de notre monde, que tous ceux qui cherchent à se rendre compte de leur foi sont amenés à s’occuper de votre passé religieux. Quand on veut approfondir le christianisme, par exemple, c’est le judaïsme qu’il faut étudier. Attaché par une de ces traditions d’enfance qui sont les plus chères et qui durent le plus, attaché, dis-je, au christianisme par une de ces traditions, je crus ne pouvoir mieux prouver mon respect pour la doctrine chrétienne qu’en l’examinant de très près. Je pense qu’un examen sérieux, consciencieux, est la plus grande marque de respect que l’on puisse donner aux croyances religieuses. ({\itshape Marques d’approbation}.)\par
À quoi me trouvai-je mené par cette analyse du christianisme ? Je me trouvai mené à l’étude du judaïsme ; car, je le répète, le chrétien qui veut se rendre compte de sa foi est nécessairement conduit à l’hébreu. Et, assurément, cette étude produisit dans mon esprit la révolution la plus profonde. C’est du jour où j’ai commencé à connaître votre passé que mes idées se sont, je puis le dire, fixées sur l’histoire religieuse de l’humanité.\par
L’étude du christianisme m’inspira la résolution d’écrire l’histoire des origines chrétiennes. Mais l’histoire des origines chrétiennes, qu’est-ce, Messieurs ? C’est essentiellement votre histoire. Je reconnais que, pour être complètement logique, j’aurais dû commencer mon histoire des \emph{Origines du christianisme} par une histoire du peuple juif. Si je me suis jeté, comme on dit, au milieu du sujet, c’est qu’on ne sait pas la durée de la vie et qu’on va d’abord au plus pressé. Aussi, maintenant que j’ai raconté comme je l’ai pu les cent cinquante premières années du christianisme, je voudrais que ce qui me reste de vie et de force fût consacré à l’histoire antérieure, où se trouve, je le reconnais, la véritable explication du christianisme.\par
Les origines du christianisme, en effet, doivent être placées au moins sept cent cinquante ans avant Jésus-Christ, à l’époque où apparaissent les grands prophètes, créateurs d’une idée entièrement nouvelle de la religion.\par
C’est là votre gloire, Messieurs, la gloire d’{\orgName Israël} ; c’est là le grand secret dont vous êtes les dépositaires : c’est dans le sein de votre race, environ sept ou huit cents ans avant Jésus-Christ, — ces années-là ne se supputent pas d’une manière bien rigoureuse, — c’est dans le sein d’{\placeName Israël} que s’est accompli d’une manière définitive le passage de la religion primitive, pleine de superstitions malsaines, à la religion pure et, on peut le dire, définitive de l’humanité.\par
La religion primitive, autant qu’il est permis de l’entrevoir, a dû participer de la grossièreté inhérente aux origines de l’humanité. Ce fut une religion tout égoïste. On se figurait {\persName Dieu} ou les {\orgName dieux} d’une manière plus ou moins analogue à l’homme, on cherchait à prendre la {\persName Divinité} ou les {\orgName divinités} comme on prend les hommes, c’est-à-dire par l’intérêt, par des dons, des cadeaux. On cherchait à s’insinuer dans la faveur des dieux en leur offrant quelque chose d’agréable, des sacrifices surtout, qu’on supposait devoir être bien accueillis d’eux.\par
C’était un culte essentiellement intéressé. L’homme était entouré de terreurs, de causes inconnues, et il s’imaginait arriver à ses fins en captant la faveur de ces causes inconnues, en les attachant au service de son ambition ou de sa passion.\par
Lisez cette inappréciable inscription de {\persName Mésa} que nous avons au {\orgName musée du Louvre}, et qui montre si bien l’état de la conscience d’un roi de Moab près de neuf cents ans avant Jésus-Christ. {\persName Mésa} offre des sacrifices ; il cherche à être agréable de toutes les manières au {\persName dieu Camos}, qui lui rend le prix de sa piété en lui faisant remporter des victoires et en le protégeant dans toutes les occasions. {\persName Mésa}, en un mot, c’est le favori de {\persName Camos}. Et pourquoi cela ? Parce que {\persName Mésa} est un homme d’une grande élévation morale ? Oh ! c’est bien peu probable. Nous n’avons pas beaucoup de renseignements sur cette époque reculée ; mais je crois que nous ne nous avancerions pas beaucoup en disant que {\persName Camos} était attaché à {\persName Mésa} pour de tout autres raisons que parce que celui-ci était un très galant homme. Le {\persName dieu Camos} ne paraît pas avoir été sensible à cette considération-là. ({\itshape On rit}.)\par
Que si nous passons de la religion de {\placeName Moab} à la religion d’{\placeName Israël}, le contraste est frappant. Lisons, par exemple, le psaume \textsc{xv}, qui, comme la plupart des psaumes, n’est pas daté, mais où nous trouvons l’expression d’un sentiment fort ancien. Qu’y lisons-nous ? Le psalmiste se demande ce qu’il faut faire pour être le protégé de {\persName Iahvé}, pour être son {\itshape ger}, « son voisin ». Cette situation du {\itshape ger} à l’égard du dieu qu’il servait est devenue bien claire par les inscriptions phéniciennes, rapprochées de certaines expressions arabes. Le {\itshape ger}, le voisin d’un dieu, c’était celui qui vivait à côté du temple de ce dieu ; c’était son parasite, son commensal, participant à la bombance qui résultait de sacrifices offerts au dieu. ({\itshape On rit}.) Le voisin du dieu était ainsi couvert par la protection du dieu, qui s’étendait comme deux grandes ailes autour du temple. Or cette protection, ces avantages, cette faveur, le {\itshape ger}, chez les {\orgName Phéniciens}, par exemple, cherchait-il à s’en rendre digne en étant un honnête homme, en perfectionnant son être moral ? Non, certes ; les renseignements que nous avons sur ces {\itshape gérim} porteraient à croire tout le contraire. Lisons maintenant notre psaume \textsc{xv}. Nous allons voir quelles doivent être les qualités du protégé, du voisin de {\persName Iahvé}, de celui que le {\persName Dieu d’Israël} couvre de ses ailes.\par


\begin{verse}
{\persName Iahvé}, qui mérite d’être le {\itshape ger} de la tente ?\\
Qui mérite d’habiter sur ta montagne sainte ?\\
\end{verse}

\noindent Écoutez la réponse :\par


\begin{verse}
Celui qui marche immaculé et pratique la justice,\\
Qui parle vrai dans son cœur ;\\
Qui ne calomnie point avec sa langue,\\
Qui ne fait point de mal à son prochain,\\
Qui n’outrage point son semblable,\\
Qui n’accepte point de cadeaux au détriment du faible.\\
\end{verse}

\noindent Voilà donc, Messieurs, les qualités du {\itshape ger}, du voisin, du protégé de {\persName Iahvé}. On est le protégé de {\persName Iahvé} en étant un honnête homme\footnote{ Se rappeler aussi la belle formule : {\itshape Lo iegurka ra}, « un méchant ne saurait être ton {\itshape ger} ». Ps. v, 5.}.\par
Je ne dis pas que le psaume \textsc{xv} ait exprimé cela pour la première fois ; mais c’est bien {\orgName Israël} qui a dit cela pour la première fois. Si le psaume n’est pas daté, voici un texte qui l’est d’une façon incontestable ; c’est le premier chapitre d’Isaïe :\par

\begin{quoteblock}
 \noindent Écoutez la parole de {\persName Iahvé}, juges de {\placeName Sodome} ; prêtez l’oreille à l’enseignement de notre {\persName Dieu}, peuple de {\placeName Gomorrhe}. « Que m’importe la multitude de vos sacrifices ? dit {\persName Iahvé} ; je suis écœuré de la fumée des béliers, de la graisse des veaux ; le sang des taureaux, des agneaux et des boucs, je n’en veux plus. Cessez de m’apporter ces vaines offrandes ; leur odeur me fait mal au cœur… Vos fêtes et vos néoménies, mon âme les hait, elles me sont à charge ; je ne les peux plus supporter. Multipliez vos prières tant que vous voudrez ; je ne les écoute pas, car vos mains sont pleines de sang. Lavez-vous d’abord, purifiez-vous ; ôtez de devant mes yeux vos actions coupables ; cessez défaire le mal ; apprenez à bien faire ; cherchez la justice, soutenez l’opprimé, faites droit à l’orphelin, défendez la veuve.
 \end{quoteblock}

\noindent Ah ! Messieurs, voilà un dieu tout nouveau, un dieu profondément distinct du {\placeName Gamos} de ce {\persName roi Mésa} et de tous les {\orgName dieux de l’antiquité}. La morale est entrée dans la religion ; la religion est devenue la morale. L’essentiel n’est plus le sacrifice matériel. C’est la disposition du cœur, c’est l’honnêteté de l’âme qui est le véritable culte. ({\itshape Applaudissements prolongés}.)\par
Eh bien ! ces paroles sont datées ; elles sont authentiques, elles sont d’environ sept cent vingt-cinq ans avant Jésus-Christ. Elles signalent l’avènement de la religion pure dans l’humanité. Logiquement parlant, un tel mouvement devait aboutir à la suppression des sacrifices ; mais il est rare qu’on atteigne à l’idéal absolu ; il est difficile de faire disparaître des usages chers à un peuple et devenus nationaux. L’esprit du moins resta. L’esprit des prophètes, c’est l’esprit même d’{\orgName Israël}. Après la captivité, nous le retrouvons plus éclatant que jamais dans ces admirables écrivains du VI\textsuperscript{e} siècle avant Jésus-Christ, dont le rêve est une religion qui puisse convenir à l’humanité tout entière.\par
Tant que le culte réside dans des pratiques matérielles, on ne saurait demander à tous les peuples de l’accepter ; chaque nation a ses pratiques ; pourquoi les changer ? Mais un culte qui réside dans l’idéal pur de la morale et du bien, un tel culte, dis-je, est bon pour tout le monde.\par
Et c’est là une idée qui se produit sans cesse chez les anciens prophètes : ce culte épuré d’{\orgName Israël} deviendra la religion du genre humain. Il ne s’agit plus d’un culte particulier ; il s’agit du culte universel, du règne de la justice. ({\itshape Applaudissements}.)\par
Le règne de la justice ! oui, telle est bien la foi de ces anciens prophètes ; c’est l’idéal qui apparaît dans leurs œuvres. Cet idéal ne se réalise pas complètement — jamais l’idéal ne se réalise d’une façon plénière ; — mais la croyance obstinée que, grâce à Israël, la justice régnera sur la terre, devient, dans la pensée du pieux juif, une sorte d’obsession.\par
Voilà où réside la merveilleuse originalité des prophètes ; voilà l’idée qui a été le noyau de la religion pure, et qui a dû être adoptée par l’humanité entière. Cette idée, proclamée avec un accent si populaire et si touchant par les fondateurs du christianisme, et exprimée avec une grandeur admirable par les prophètes du VI\textsuperscript{e} siècle avant Jésus-Christ.\par
C’est dans ce sens que j’ai dit que les origines du christianisme sont dans le judaïsme. Les vrais fondateurs du christianisme ; ce sont les grands prophètes qui ont annoncé la religion pure, détachée des pratiques grossières et résidant dans les dispositions de l’esprit et du cœur, religion par conséquent qui peut et doit être commune à tous, religion idéale, consistant en la proclamation du règne de Dieu sur la terre et en l’espérance d’une ère de justice pour la pauvre humanité.\par
Les poèmes sibyllins, ces œuvres apocryphes autant que l’on voudra, mais si touchantes, de l’{\orgName école d’Alexandrie}, tournent autour du même rêve, qui, par des échos mystérieux, est arrivé jusqu’à Virgile, à savoir un avenir brillant, un avenir de paix, de bonheur et de fraternité, réservé au monde renouvelé. Ce paradis sur terre résultera de l’accession de l’humanité au culte d’Israël.\par
Il nous est très difficile de parler d’une manière précise de ces premiers fondateurs du christianisme, dont la physionomie est recouverte à nos yeux par un triple voile ; mais ce qui est certain, c’est que toute la première génération chrétienne est essentiellement juive. On eût demandé à ces grands fondateurs s’ils croyaient se mettre en dehors de la famille juive : « Oh ! non, auraient-ils répondu ; nous continuons la ligne des inspirés d’Israël ; c’est nous qui sommes les vrais aboutissants des anciens prophètes. » Ils croyaient, en un mot, accomplir la Loi et non la supprimer.\par
Pour avoir des témoignages bien positifs, il faut arriver à {\persName saint Paul}, dont les plus anciennes épîtres conservées sont à peu près de l’an 54 après Jésus-Christ. Ici, le déchirement est en apparence éclatant. {\persName Paul} cependant proteste sans cesse qu’il n’abandonne pas sa foi dans les promesses. Il veut élargir le judaïsme, en faciliter l’accès aux populations qui désirent entrer dans son sein. Il a quelquefois des paroles dures pour son ancien peuple ; mais il a aussi des paroles tendres, pleines de douceur, et jamais {\persName saint Paul} n’a cru se séparer de l’{\orgName Église juive}. Dans la primitive Église, d’ailleurs, Paul est considéré presque comme un hérétique, comme un esprit hardi, comme une sorte de trouble-fête. Il fut en tout cas une exception, et des petites épîtres comme celles qui figurent dans le canon chrétien sous le nom de {\persName saint Jacques}, de {\persName saint Jude}, représentent bien mieux l’esprit de la première Église. Or, de tels écrits sont tout à fait juifs ; ils auraient pu se lire dans la synagogue s’ils avaient été écrits en hébreu.\par
Il en de même de l’Apocalypse dite de {\persName saint Jean}, celle qui est dans le canon chrétien. Ce livre, daté de la fin de l’an 68 ou du commencement de l’an 69, est un livre juif au plus haut degré. L’auteur est passionné pour la nationalité juive. La guerre de {\placeName Judée} a commencé, {\placeName Jérusalem} va être investie ; on sent chez le {\persName Voyant} la sympathie la plus profonde pour les révoltés de {\placeName Judée}. {\placeName Jérusalem} est pour lui « la ville aimée » ; son idéal de l’humanité est une Jérusalem d’or, de perles et de pierreries. On n’est pas plus juif que l’auteur de l’Apocalypse.\par
Au lendemain de la prise de {\placeName Jérusalem}, se place la rédaction des Évangiles dits synoptiques. Ici il y a partage. L’esprit de ces Évangiles est en quelque sorte double. Il y a dans les vieux livres chrétiens un mot qui donne une idée assez juste de l’état moral des évangélistes ; c’est le mot διψνχος, « qui a deux âmes », pour signifier « flottant entre deux esprits ». On lit dans les synoptiques des paroles très sévères, quelquefois injustes contre les pharisiens ; mais ce qui montre bien que le déchirement n’était pas encore fait, c’est que le moins juif de tous les synoptiques, Luc, tient à constater que {\persName Jésus} a pratiqué toutes les cérémonies de la loi, en particulier qu’il a été circoncis. Un fait bien curieux, d’ailleurs, est celui-ci :\par
Vers 75 ou 80 et dans les années qui suivent, il se produit beaucoup délivres inspirés par le patriotisme juif, tels que le livre de {\persName Judith}, l’Apocalypse d’{\persName Esdras}, l’Apocalypse de {\persName Baruch}, et même le livre de {\persName Tobie}, qui n’apparaît qu’à une époque tardive. Il n’y a rien de plus juif que le livre de {\persName Judith}, par exemple. Et pourtant ces livres se perdent chez les {\orgName juifs} et ne sont conservés que par les {\orgName chrétiens} ; tant il est vrai que le lien entre l’{\orgName Église} et la {\orgName synagogue} n’était pas encore rompu quand ils parurent.\par
L’épître de {\persName Clément Romain}, quel qu’en soit l’auteur, exprime très bien les sentiments de l’{\orgName Église romaine} vers l’an 98 après Jésus-Christ. Cet opuscule est d’un judaïsme tout à fait orthodoxe ; {\persName Judith} y est citée pour la première fois comme une héroïne, ce qui prouve que la scission, vers l’an 100, n’était pas le moins du monde accomplie.\par
Si nous passons maintenant aux épîtres et aux Évangiles attribués à {\persName Jean}, le cas est tout autre. Nous pouvons placer la composition de ces écrits vers l’an 125 après Jésus-Christ, c’est-à-dire environ cent ans après la mort de {\persName Jésus}. Le judaïsme y est traité en ennemi. On pressent l’avènement des systèmes qui, sous le nom de gnosticisme, porteront les chrétiens à renier leurs origines juives. Le gnosticisme est tout à fait opposé au judaïsme. Selon les gnostiques, le christianisme est né spontanément et sans antécédent ; ou plutôt il est une réaction contre la loi antérieure. Il est inconcevable qu’une conception historique aussi erronée ait pu se produire en aussi peu de temps (cent ou cent vingt ans après {\persName Jésus} !) Les nouveaux docteurs déclarent que le christianisme n’a rien à faire avec le judaïsme. Marcion, plus exagéré encore, prétend que la religion juive est une religion mauvaise, que {\persName Jésus-Christ} est venu abolir.\par
Il est, je le répète, tout à fait singulier que, dans l’espace d’un siècle, une semblable erreur ait pu se produire ; mais remarquez que le gnosticisme est dans l’{\orgName Église chrétienne} ce qu’un courant latéral est pour un fleuve. L’{\orgName Église orthodoxe} se considéra toujours, au second siècle, comme liée à la {\orgName synagogue} par le lien le plus intime.\par
{\persName Papias} est bien un chrétien juif, renfermé dans les idées des Évangiles synoptiques et de l’Apocalypse. Le \emph{Testament des douze patriarches}, qui paraît vers le même temps, est une œuvre toute juive. Le \emph{Pasteur} d’{\persName Hermas} est encore un livre édifiant dans le sens juif, un véritable {\itshape agada}. Je voudrais qu’on le traduisît ; je suis sûr qu’on le lirait avec charme, aussi bien dans le camp des personnes qui croient que dans le camp de celles qui s’intéressent simplement à l’histoire religieuse.\par
Enfin il y a cet évêque de {\placeName Sardes}, {\persName Méliton}, qui vers l’an 160, passe sa vie à chercher les livres saints parmi les juifs. On ne possédait la liste des livres saints, au fond de l’{\placeName Asie-Mineure}, que d’une manière fort incomplète. {\persName Méliton} fait une enquête, va en {\placeName Syrie}, arrive à connaître exactement le canon des juifs ; pour lui, c’est bien là le canon des livres sacrés.\par
Nous touchons aux temps de {\persName Marc-Aurèle}. La scission, maintenant, se prononce de plus en plus. {\persName Polycarpe} et son entourage sont ennemis des juifs. Les {\orgName Apologistes} sont en général aussi de grands adversaires du judaïsme. Ce sont des avocats ; ils taillent à pans coupés, comme une forteresse, la cause qu’ils défendent. L’écrit anonyme connu sous le nom d’\emph{Épître à Diognète}, est surtout frappant à cet égard. Il fait très bien comprendre l’erreur étrange où étaient arrivées, vers la fin du second siècle, des branches entières de la famille chrétienne : on eût dit que le christianisme avait germé du sol tout seul, indépendamment du judaïsme. L’auteur de l’\emph{Épître à Diognète} traite les rites juifs, d’où le christianisme est sorti, de « superstitions ». On ne vit jamais contradiction plus singulière.\par
La séparation, je le répète, se faisait surtout par l’influence des doctrines gnostiques. Sous {\persName Marc-Aurèle} le divorce était loin encore d’être absolu. Voilà le montanisme qui se produit vers 170 ; le montanisme est une recrudescence de l’ancien esprit millénaire, prophétique, apocalyptique, parmi les populations ardentes et crédules de la {\placeName Phrygie}. Quelle est l’idée constante du montanisme ? C’est que {\orgName Jérusalem} va venir se fixer à {\placeName Pépuze}, en {\placeName Phrygie}. Les sectaires passaient les jours, les yeux tendus vers le ciel, pour voir cette {\orgName Jérusalem} nouvelle éclater dans les nues, puis descendre et venir s’établir dans les cantons brûlés de la {\placeName Phrygie Catacécaumène}. Le lien, pour eux, n’était nullement rompu avec les anciennes espérances d’{\orgName Israël}.\par
Il y a un livre, surtout, qui est un véritable trésor historique : c’est le roman dont Clément Romain est le héros et qui est connu sous le nom de \emph{Reconnaissances}. Si l’on veut bien comprendre les relations du judaïsme avec le christianisme sous {\persName Marc-Aurèle}, c’est ce livre-là qu’il faut lire. La question est traitée en quelque sorte {\itshape ex professa} dans un sermon censé prononcé par {\persName saint Pierre}, à {\placeName Tripoli}, sur la côte de {\placeName Syrie}. Les bases du système de conciliation exposé par {\persName saint Pierre} sont celles-ci : Le judaïsme et le christianisme ne diffèrent pas l’un de l’autre ; {\persName Moïse}, c’est {\persName Jésus} ; {\persName Jésus} c’est {\persName Moïse}. Il n’y a eu, à proprement parler, depuis l’origine, qu’un seul prophète sans cesse renaissant ; un même esprit prophétique a inspiré tous les prophètes. Le judaïsme suffit à celui qui ne connaît pas le christianisme. Le christianisme suffit à celui qui ne connaît pas le judaïsme. On peut faire son salut également dans les deux.\par
Les expressions dont se sert cet auteur si intéressant méritent d’être pesées. Selon la fable du roman, la famille de {\persName Clément Romain} se convertit à la vérité. Ce sont des païens très vertueux et qui, pour prix de leur vertu, arrivent à la vraie religion. « Ils se font juifs », Ιονδαιονς γεγενημενους. Se faire juif, pour l’auteur, c’est adopter la vérité religieuse, laquelle n’est pas coupée en deux. Il n’y a pour lui qu’une révélation, dont le judaïsme et le christianisme sont les deux formes équivalentes et parallèles. Voilà comment, sous {\persName Marc-Aurèle}, on comprenait les relations entre le judaïsme et le christianisme.\par
Plus tard, au III\textsuperscript{e} siècle, la scission devient plus accusée, sous l’influence de l’{\orgName école d’Alexandrie}, héritière d’un gnosticisme mitigé. {\persName Clément d’Alexandrie} et {\persName Origène} n’aiment pas le judaïsme et en parlent avec beaucoup d’injustice. On sent que la séparation est en train de se faire ; cependant elle ne s’opère d’une manière complète que quand le christianisme affecte les allures de religion d’État, sous {\persName Constantin}. Le christianisme alors devient officiel, tandis que le judaïsme garde son caractère libre. La séparation est-elle dès lors tout à fait complète ? Eh bien ! non, pas encore.\par
Je rappelais dernièrement les sermons de {\persName saint Jean Chrysostome} contre les juifs. Il n’y a pas de document historique plus intéressant. L’orateur s’y montre naturellement rude, dogmatique ; il fait toute sorte de raisonnements, dont quelques-uns ne sont pas très forts. ({\itshape Rires}). Mais on voit que ses fidèles étaient encore dans une communauté des plus intimes avec la synagogue. Il leur dit plus de vingt fois (car {\persName saint Jean Chrysostome} se répète beaucoup ; il est un peu prolixe) : ({\itshape On rit.}) \emph{« Qu’allez-vous faire à la synagogue ? Vous voulez célébrer la Pâque ? Eh bien, nous aussi, nous célébrons la Pâque ; venez chez nous ! »} ({\itshape Hilarité générale et applaudissements)}\par
Les {\orgName chrétiens} d’{\placeName Antioche}, en 380, allaient donc encore à la synagogue dans beaucoup de circonstances. Pour donner à un serment plus de force, on se rendait à la synagogue parce qu’on y trouvait les livres saints. C’est ici, à vrai dire, la cause de l’usage que {\persName Jean Chrysostome} combat comme un abus des plus graves. \emph{« Je sais bien, dit {\persName Chrysostome}, ce que vous allez me répondre. Vous me direz que c’est là que se trouvent la {\orgName Loi} et les {\orgName prophètes}. »} Les {\orgName chrétiens} ne pratiquaient pas assez la Bible hébraïque, et ils avaient le sentiment que les {\orgName juifs} en étaient les vrais gardiens.\par
Mais ce ne sont plus là que des traces de la communauté primitive, car la séparation devient de plus en plus profonde. Nous entrons dans le moyen âge, les barbares arrivent, et alors commence cette déplorable ingratitude de l’humanité, devenue chrétienne, contre le judaïsme. C’est toujours ainsi que les choses se passent : quand on travaille pour l’humanité, on est sûr d’être volé d’abord et, par-dessus le marché, d’être battu. ({\itshape On rit. — Applaudissements}.)\par
Le monde avait pris la vérité religieuse au judaïsme, et il traite le judaïsme de la manière la plus cruelle. Ce n’est cependant pas dans la première moitié du moyen âge que se passent les faits les plus déplorables, à cette époque, il y a malveillance ; cela est hors de doute ; mais il n’y a pas encore de persécutions organisées, ou du moins il y en a peu. Les croisades donnent le signal des massacres de juifs. La scolastique aussi contribua beaucoup à envenimer les choses.\par
La théologie chrétienne venait de s’organiser en une espèce de science, où la révélation était en quelque sorte encadrée dans les syllogismes de la dialectique d’{\persName Aristote}. Un des côtés les plus faux de cette scolastique, c’était de chercher et de trouver partout des erreurs. Nous avons de ces énumérations d’erreurs qui remplissent des volumes, et souvent, parmi ces prétendues erreurs condamnées, il y a de très bonnes choses. Dans cette fureur de condamnations théologiques, on songea que le Talmud devait renfermer les erreurs les plus graves. Les renégats s’en mêlèrent et se firent dénonciateurs. Alors on instruit le procès du Talmud (1248) ; on le brûle, et, comme dit mon savant maître, {\persName M. Victor Le Clerc}, dans son \emph{Discours sur l’histoire littéraire de la France au XIV\textsuperscript{e} siècle} : \emph{« On brûlait le Talmud, et quelquefois le juif avec le Talmud. »} ({\itshape Rires et applaudissements}.) C’est le temps des persécutions abominables, des autodafés comme celui de {\placeName Troyes} en 1288.\par
À la fin du XIII\textsuperscript{e} siècle, la fiscalité de {\persName Philippe le Bel} vint tout perdre. On commençait à s’occuper de grandes choses ; mais il fallait de l’argent, et, à cette époque, on se procurait de l’argent par de bien mauvais moyens. La spoliation des juifs se présenta tout d’abord. C’est un des actes les plus fâcheux de l’histoire de {\orgName France}. Jusque-là, la {\orgName France} avait été une terre relativement tolérante pour {\orgName Israël} ; et, si quelque chose résulte du travail que nous avons inséré dans l’{\itshape Histoire littéraire de la France} sur la situation des juifs en {\placeName France} au moyen âge, travail dont je me plais à rapporter tout le mérite à {\persName M. Neubauer}, c’est qu’avant la fin du XIII\textsuperscript{e} siècle les {\orgName juifs} exerçaient exactement les mêmes professions que les autres {\orgName Français}. C’est à la suite des tristes événements dont nous venons de parler que se fait la distinction des professions entre {\orgName israélites} et {\orgName non-israélites}. On force les {\orgName israélites} à mener un genre de vie différent de celui des autres. La vie de l’Israélite devient une vie de séquestration, de proscription. Or c’est une loi historique que la société qui condamne une partie de ses membres à une vie à part est la première victime de ces mesures maladroites ; car une des conséquences de la proscription, c’est, jusqu’à un certain point, de créer un privilège pour le proscrit. On le soustrait aux charges : on le condamne aux professions qui ne sont que lucratives. C’est ainsi qu’on a presque forcé l’israélite à être riche. Dans cette société du moyen âge, au moins à partir de la fin du XIII\textsuperscript{e} siècle, l’israélite n’a plus qu’une profession libre, celle qui consiste à s’enrichir, si bien qu’il y a là un cercle vicieux des plus singuliers. Le moyen âge reproche à l’israélite la profession même à laquelle il l’a condamné. Il lui a enlevé la culture de la terre, il lui interdit l’exercice de toutes les professions onéreuses, et il trouve mauvais que l’israélite profite de ce qu’une telle situation a de lucratif. C’est un sophisme des plus déplorables.\par
Ce fait de la dévolution aux juifs des affaires d’argent et de finances au moyen âge était, du reste, la conséquence de leur situation en dehors du droit canonique. L’{\orgName Église}, au moins en {\placeName France}, professait alors sur l’usure les idéM. Jourdaines les plus exagérées et les plus fausses. Les doctrines des casuistes sur la question de l’intérêt de l’argent rendaient presque toutes les affaires impossibles à la société chrétienne\footnote{ Voir le mémoire de {\persName M. Jourdain} sur les commencements de l’économie politique dans les écoles du moyen âge, dans les \emph{Mém. de l’Acad. des inscr, et belles lettres}, t. XXYIII, 2\textsuperscript{e} partie.}. Pour faire la moindre opération d’argent, il fallait employer des personnes qui ne fussent pas soumises au droit canon. L’usure (et on était usurier par le fait de tirer le moindre profit d’un placement), l’usure, dis-je, était un crime ecclésiastique ; l’usurier ne pouvait tester, n’était pas enterré en terre sainte, sa famille était notée d’infamie, si bien que les {\orgName chrétiens} étaient absolument exclus des opérations de banque et même d’assurance et de commerce. C’est donc le moyen âge qui est lui-même coupable de ce qu’il a reproché aux {\orgName israélites}.\par
N’insistons pas sur ce triste spectacle. Arrivons à une époque plus consolante, à ce XVIII\textsuperscript{e} siècle qui a proclamé enfin les droits de la raison, les droits de l’homme, la vraie théorie de la société humaine, je veux dire l’État sans dogme officiel, l’État neutre au milieu des opinions métaphysiques et théologiques : c’est ce jour-là que l’égalité des droits a commencé pour les juifs. C’est la Révolution qui a proclamé l’égalité des {\orgName juifs} avec les autres citoyens dans l’État. ({\itshape Vifs et unanimes applaudissements.})\par
La Révolution a trouvé ici la solution vraie avec un sentiment d’une justesse absolue, et tout le monde y viendra. ({\itshape Assentiment général}.)\par
Et qui mieux que le {\orgName peuple juif}. Messieurs, pouvait accepter une pareille solution ? C’était le {\orgName peuple juif} lui-même qui l’avait préparée ; il l’avait préparée par tout son passé, par ses {\orgName prophètes}, les grands créateurs religieux d’{\orgName Israël}, qui avaient appelé l’unité future du genre humain dans la foi et dans le droit.\par
Les promoteurs d’un tel mouvement, sont : d’abord l’ancien et authentique {\persName Isaïe} ; puis son continuateur du temps de la captivité, ce génie religieux hors de pair ; puis les esséniens, ces poétiques ascètes qui rêvaient un idéal qu’on n’a pas encore atteint. Le christianisme a aussi puissamment contribué aux progrès de la civilisation ; or le christianisme, si admirable dans sa lutte contre les barbares, quand il cherche à maintenir quelque trace de raison et de droit au milieu des débordements de la brutalité, le christianisme, dis-je, n’était que la continuation de vos prophètes. La gloire du christianisme, c’est la gloire du judaïsme. Oui, le monde s’est fait juif en se convertissant aux lois de douceur et d’humanité prêchées par les disciples de {\persName Jésus}. ({\itshape Applaudissements}.)\par
Et maintenant que ces grandes choses sont accomplies, disons-le avec assurance : le judaïsme, qui a tant servi dans le passé, servira encore dans l’avenir. Il servira la vraie cause, la cause du libéralisme, de l’esprit moderne. Tout juif est un libéral\footnote{Ceci s’applique naturellement aux juifs français, tels que les a faits la Révolution ; mais nous sommes persuadés que tout pays qui voudra renouveler l’expérience, renoncer à la religion d’État, laïciser la vie civile et pratiquer l’égalité de tous les citoyens devant la loi, arrivera au même résultat et trouvera d’excellents patriotes dans le culte israélite comme dans les autres cultes.}. ({\itshape Assentiment général et applaudissements}.) Il l’est par essence. Les ennemis du judaïsme, au contraire, regardez-y de près, vous verrez que ce sont en général des ennemis de l’esprit moderne. ({\itshape Nouvelles marques d’approbation}).\par
Les créateurs du dogme libéral en religion, ce sont, je le répète, vos anciens prophètes, {\persName Isaïe}, les {\orgName Sibyllins}, l’{\orgName école juive d’Alexandrie}, les premiers chrétiens, continuateurs des prophètes. Voilà les véritables fondateurs de l’esprit de justice dans le monde. En servant l’esprit moderne, le juif ne fait, en réalité, que servir l’œuvre à laquelle il a contribué plus que personne dans le passé et, ajoutons-le, pour laquelle il a tant souffert. ({\itshape Mouvement}.)\par
La religion pure, en un mot, que nous entrevoyons comme pouvant rallier l’humanité tout entière, sera la réalisation de la religion d’Isaïe, la religion juive idéale, dégagée des scories qui ont pu y être mêlées.\par
Vous avez donc bien fait, Messieurs, de fonder la {\orgName société des études juives}, qui mettra ces vérités dans une lumière toute particulière. Travaillons tous ensemble, car l’œuvre est commune. Je me suis quelquefois plu à rêver le jour où l’humanité, reconnaissante envers la {\orgName Grèce}, apporterait à l’{\orgName Acropole} d’{\placeName Athènes} les morceaux que tout le monde lui a volés. ({\itshape On rit}.) C’est un rêve qui ne se réalisera jamais. Eh bien, je rêverais au moins quelque chose d’analogue pour votre {\orgName Parthénon}. Votre {\orgName Parthénon}, Messieurs, c’est, je le veux bien, {\placeName Jérusalem}, cette ville unique et si hautement respectable ; mais vous êtes idéalistes avant tout, et votre vrai {\orgName Parthénon}, c’est la Bible. ({\itshape Applaudissements unanimes}.)\par
L’étude, l’éclaircissement, l’explication de la Bible, voilà votre œuvre — à laquelle nous sommes heureux d’avoir été conviés. — Et quel plus bel hommage rendu à l’esprit d’{\orgName Israël}, que ce prodigieux travail de l’exégèse moderne, que ces innombrables recherches critiques, pour élucider, je ne dirai pas chaque phrase, mais chaque mot, mais chaque lettre de vos textes anciens ?\par
Votre livre est une chose tellement unique dans l’humanité, que chacune des syllabes que vous avez écrites est devenue un sujet de bataille sans fin. ({\itshape Rires et bravos}.)\par
Le dictionnaire hébreu décide du sort de l’humanité. Il y a tel dogme qui repose sur une erreur d’interprétation de certain passage de votre Bible, sur une faute de vos copistes. Tel de vos anciens scribes, par une de ses distractions, a décidé de la théologie de l’avenir.\par
Quand j’avais l’honneur d’être attaché au département des manuscrits à la {\orgName Bibliothèque impériale}, — la {\orgName Bibliothèque nationale} aujourd’hui, — je reçus la visite du célèbre {\persName docteur Pusey}, homme respectable s’il en fut, et, comme on sait, très orthodoxe. Lorsque je lui eus remis les manuscrits arabes qu’il désirait consulter, il vit sur ma table le \emph{Thésaurus} de {\persName Gesenius}. Aussitôt sa figure se rembrunit, devint sévère, et il me dit : \emph{« C’est là un livre extrêmement dangereux, plein de rationalisme et d’erreurs. »} ({\itshape On rit}.) Le lendemain, je reçus de lui une lettre de plus de dix pages, — que je conserve précieusement, — pour me démontrer qu’il ne fallait que des yeux pour voir les prédictions les plus claires du {\persName Messie} dans le cinquante-troisième chapitre d’{\persName Isaïe}.\par
Eh bien, c’est là votre gloire. Messieurs ; combien ce cinquante-troisième chapitre a-t-il déjà produit de volumes ? Que n’a-t-on pas écrit sur un certain pronom contenu dans ce cinquante-troisième chapitre ? Que de recherches, que d’efforts pour déterminer si ce pronom {\itshape lamo} doit être pris au singulier ou au pluriel ! La foi d’une foule de gens a reposé sur la syntaxe de ce pronom {\itshape lamo}.\par
Ce sont là des subtilités ; mais, en même temps, ce sont autant d’hommages rendus à la grandeur de votre passé.\par
Travaillez donc, Messieurs, comme vous l’avez fait jusqu’ici, et veuillez bien accepter notre collaboration. ({\itshape Très bien ! Très bien ! — Assentiment}.)\par
Votre Bible, Messieurs, est le livre de l’humanité tout entière, c’est le document fondamental de l’histoire des développements successifs de l’idée religieuse dans l’humanité. ({\itshape Vive adhésion et applaudissements prolongés}.)
\section[{Le judaïsme comme race et comme religion}]{Le judaïsme comme race et comme religion}\renewcommand{\leftmark}{Le judaïsme comme race et comme religion}

\begin{center}\emph{(Reproduction sténographique)}\end{center}

\dateline{Conférence faite au {\placeName Cercle saint-Simon}, le 27 janvier 1883.}

\salute{Messieurs,}
\noindent Votre accueil bienveillant me touche plus que je ne saurais dire ; mais la solennité de cette tribune me trouble un peu. J’avais accepté de parler ce soir devant vous, à la condition que notre entretien ne serait qu’un simple échange de réflexions sans nul artifice oratoire. Cet appareil de sténographie m’intimide ; car, ce que je voulais, c’était simplement de penser en quelque sorte tout haut devant vous sur un des sujets vers lesquels mes recherches se portent le plus souvent depuis quelque temps. Je réclame votre indulgence pour un exposé qui ne devait être, dans ma pensée, qu’une simple conversation et que votre aimable empressement transforme en conférence. Le sujet parle de lui-même et me soutiendra.\par
Je voudrais échanger quelques idées avec vous sur la distinction que, selon moi, il importe de faire entre la question religieuse et la question ethnographique en ce qui concerne le judaïsme. Que le judaïsme soit une religion et une grande religion, cela est clair comme le jour. Mais on va d’ordinaire plus loin. On considère le judaïsme comme un fait de race, on dit : « la race juive » ; on suppose, en un mot, que le peuple juif, qui, à l’origine, créa cette religion, l’a toujours gardée pour lui seul. On voit bien que le christianisme s’en est détaché à une certaine époque ; mais on se laisse aller volontiers à croire que ce petit peuple créateur est resté toujours identique à lui-même, si bien qu’un juif de religion serait toujours un juif de sang. Jusqu’à quel point cela est-il vrai ? Dans quelle mesure ne convient-il pas de modifier une telle conception ? Nous allons l’examiner. Mais auparavant permettez-moi de poser bien nettement la question au moyen d’une comparaison.\par
Il y a dans le monde, à {\placeName Bombay}, une petite religion qui est celle des {\orgName parsis}, l’ancienne religion de la {\placeName Perse}. Dans ce cas, la question est bien claire. Le parsisme est une religion qui a été nationale à l’origine et qui est aujourd’hui gardée par une race à peu près homogène ; je ne crois pas qu’il y ait jamais eu, en effet, beaucoup de conversions au parsisme. Voilà donc un fait religieux exactement connexe à un fait de race.\par
Prenons, au contraire, le protestantisme dans les pays où il est en minorité, comme en {\placeName France}. Ici la situation est inverse, il n’y a pas de fait ethnographique. Pourquoi un homme est-il protestant ? Parce que ses ancêtres l’ont été. Pourquoi ses ancêtres l’ont-ils été ? Parce qu’au XVI\textsuperscript{e} siècle, ils se sont trouvés dans une disposition intellectuelle et morale qui les a amenés à adopter la réforme du christianisme. L’ethnographie n’a que faire en pareil cas, et c’est vainement qu’on viendrait dire que ceux qui se sont faits protestants au XVI\textsuperscript{e} siècle avaient bien pour cela quelque raison de race. Ce serait là une subtilité, ou du moins une considération d’un autre ordre que celles dont nous nous occupons en ce moment.\par
Dans le parsisme, au contraire, il y a certainement un fait ethnographique ; car, je le répète, il y a très peu d’esprit de prosélytisme dans cette petite société religieuse parquée à {\placeName Bombay}.\par
Eh bien, quelle est la situation du {\orgName judaïsme} ? Est-ce quelque chose d’analogue au {\orgName protestantisme}, ou bien est-ce une religion ethnographique comme le {\orgName parsisme} ? Voilà le point sur lequel je voudrais que nous réfléchissions ensemble aujourd’hui.\par
Il y a un principe fondamental qui ne m’arrêtera pas longtemps, Messieurs. Je parle devant des personnes au courant de la science, et le principe dont il s’agit est en quelque sorte l’{\itshape a b c} de la science des religions : c’est la distinction des religions nationales ou locales et des religions universelles.\par
De religions universelles, il n’y en a que trois. C’est d’abord le {\orgName bouddhisme} ou, pour mieux dire, l’{\orgName hindouisme} ; car nous voyons très bien maintenant qu’avant la propagande bouddhiste, il y eut une propagande hindoue. Les anciens monuments de l’{\placeName Indo-Chine} ne sont pas {\orgName bouddhistes}, ils sont {\orgName brahmanistes}, et le {\orgName bouddhisme} n’est venu là que plus tard ; mais c’est surtout sous la forme bouddhiste, nous le reconnaissons, que la religion hindoue a été conquérante. La seconde des religions universelles est le {\orgName christianisme}, et la troisième l’{\orgName islamisme}. Ce sont là trois grands faits qui n’ont rien d’ethnographique ; il y a des {\orgName bouddhistes}, des {\orgName chrétiens} et des {\orgName musulmans} de toutes les races. Nous savons au moins par à peu près la date de l’apparition dans le monde de ces trois religions. Le {\orgName bouddhisme} remonte à quatre ou cinq cents ans avant Jésus-Christ ; ses grandes conquêtes viennent plus tard. Quant au {\orgName christianisme}, à l’{\orgName islamisme}, nul doute sur l’époque de leur formation.\par
Mais, en dehors de ces religions universelles, il y a eu des milliers de religions locales et nationales. Athènes a eu sa religion, Sparte a eu sa religion, toutes les nations de l’antiquité avaient leur religion. Les lieux, dans le monde ancien, avaient aussi leur religion. C’est ici une des idées les plus enracinées de l’antiquité. Au II\textsuperscript{e}, au III\textsuperscript{e} siècle de notre ère, l’éternel raisonnement de Celse et des adversaires du christianisme est que les pays ont des dieux qui les protègent, qui s’intéressent à leurs destinées.\par
Cette vieille idée est exprimée de la manière la plus naïve dans un récit du second livre des {\itshape Rois}, relatif à la situation où se trouvèrent les {\orgName Cuthéens} qui avaient été amenés par les {\orgName Assyriens} en {\placeName Samarie}. Il leur arrive des mésaventures. Ils sont attaqués par des lions, qu’ils regardent comme des émissaires du dieu du pays, mécontent de n’être pas adoré à sa manière, et ils envoient au {\orgName gouvernement assyrien} une pétition se résumant à peu près en ceci : « Le dieu du pays nous en veut de ce qu’il n’est pas servi comme il voudrait l’être ; envoyez-nous des prêtres qui sachent comment nous pourrions le satisfaire. » Voilà donc une idée tout autre assurément que celle du christianisme et que celle du bouddhisme. Le dieu, en ce cas, est essentiellement local et provincial.\par
Toutes les religions locales ou nationales ont péri. L’humanité a voulu de plus en plus des religions universelles, expliquant à l’homme ses devoirs généraux et ayant la prétention d’apprendre à l’humanité le secret de ses destinées. Les religions nationales avaient un programme plus limité : c’était le patriotisme, doublé de cette idée que chaque pays a un génie qui veille sur lui et qui demande à être servi d’une certaine manière. Cette théologie étroite a complètement disparu. Elle a disparu devant l’idée chrétienne, l’idée bouddhique et l’idée musulmane. Cela a été un immense progrès. Je ne vois guère, dans l’histoire des nations civilisées, que deux exemples d’anciennes religions nationales qui aient survécu ; c’est d’abord le parsisme (et encore il faut dire que, pour ses sectateurs, le parsisme présente, à beaucoup d’égards, une physionomie universelle), — puis le judaïsme, qui, d’après une certaine conception, serait la religion d’un pays, le {\orgName pays d’Israël} ou le {\orgName pays de Juda}, conservée par les descendants des habitants de ce pays.\par
Eh bien, je le répète, cela demande à être examiné d’excessivement près. Que la religion israélite, que le judaïsme ait été à l’origine une religion nationale, cela est absolument hors de doute. C’est la religion des {\orgName Beni-Israël}, laquelle, pendant des siècles, n’a pas été essentiellement différente de celle des peuples voisins, des {\orgName Moabites}, par exemple, {\persName Iahvé}, le dieu israélite, protège {\placeName Israël}, comme {\persName Camos}, le dieu moabite, protège {\placeName Moab}. Nous savons maintenant fort bien quelle était la manière de sentir en religion d’un Moabite, depuis la découverte de cette inscription du {\persName roi Mésa} qui est au {\orgName Louvre}, et dans laquelle ce roi du IX\textsuperscript{e} siècle avant Jésus-Christ nous fait en quelque sorte ses confidences religieuses. Je crois bien que les idées de David étaient à peu près les mêmes que celles de {\persName Mésa}. Il y a une association intime entre {\persName Mésa} et son {\persName dieu Camos} : {\persName Camos} intervient dans toutes les circonstances de la vie du roi, lui donne des ordres, des conseils ; toutes les victoires, c’est {\persName Camos} qui les remporte ; le roi lui fait de beaux sacrifices et traîne devant lui la vaisselle sacrée des dieux vaincus. Il rémunère le dieu en proportion de ce que le dieu lui a donné ; c’est la religion du prêté-rendu. La religion d’{\orgName Israël}, elle aussi, a sans doute été bien longtemps une religion égoïste, intéressée, la religion d’un dieu particulier, {\persName Iahvé}.\par
Qu’est-ce qui a fait que ce culte de {\persName Iahvé} est devenu la religion universelle du monde civilisé ? Ce sont les prophètes, vers le VII\textsuperscript{e} siècle avant Jésus-Christ. Voilà la gloire propre d’{\orgName Israël}. Nous n’avons pas la preuve que, chez les peuples voisins et plus ou moins congénères des {\orgName Israélites}, chez les {\orgName Phéniciens} par exemple, il y ait eu des prophètes. Il y avait sans doute des nabis, que l’on consultait lorsqu’on avait perdu son âne ou que l’on voulait savoir un secret. C’étaient des sorciers. Mais les nabis d’{\placeName Israël} sont tout autre chose. Ils ont été les créateurs de la religion pure. Nous voyons, vers le VIII\textsuperscript{e} siècle avant Jésus-Christ, apparaître ces hommes, dont {\persName Isaïe} est le plus illustre, qui ne sont pas du tout des prêtres et qui viennent dire : \emph{« Les sacrifices sont inutiles ; {\persName Dieu} n’y prend aucun plaisir. Comment pouvez-vous avoir une idée assez basse de la {\persName Divinité} pour ne pas comprendre que ces mauvaises odeurs de graisse brûlée lui font mal au cœur ? Soyez justes ; adorez {\persName Dieu} avec des mains pures ; voilà le culte qu’il réclame de vous. »} Je ne crois pas que, du temps du {\persName roi Mésa} ou du {\persName roi David}, on ait beaucoup fait ce raisonnement. Dans ce temps-là, la religion n’est qu’un échange de bons services et d’hommages entre le dieu et son serviteur ; au contraire, les {\orgName prophètes} du VII\textsuperscript{e} siècle proclament que le vrai serviteur de Iahvé, c’est celui qui fait le bien. La religion devient de la sorte quelque chose de moral, d’universel ; elle se pénètre de l’idée de justice, et c’est pour cela que ces {\orgName prophètes d’Israël} sont les tribuns les plus exaltés qu’il y ait jamais eu, tribuns d’autant plus âpres qu’ils n’ont pas la conception d’une vie future pour se consoler, et que c’est ici-bas, d’après eux, que la justice doit régner.\par
Voilà une apparition unique dans le monde, celle de la religion pure. Vous voyez, en effet, qu’une pareille religion n’a rien de national. Quand on adore un {\persName Dieu} qui a fait le ciel et la terre, qui aime le bien et punit le mal (ceci était assez difficile à prouver sans les idées d’outre-tombe ; mais enfin on s’en tirait comme on pouvait) ; quand on proclame une telle religion, on n’est plus dans les limites d’une nationalité, on est en pleine conscience humaine, au sens le plus large. Aussi ces grands créateurs tirent-ils parfaitement les conséquences de leur doctrine, conséquences dont la dernière aurait été certainement de supprimer les sacrifices et le temple. Ils y seraient arrivés ; que dis-je ! Ils y sont arrivés ; les fondateurs du christianisme sont les derniers représentants de l’esprit prophétique ; or le christianisme proclame que les sacrifices sont un fait asbolument archaïque et qui ne doit plus exister dans la religion selon l’esprit.\par
Quant au temple, on accusa le fondateur du christianisme d’avoir parlé contre lui ; l’a-t-il fait réellement ? Nous ne le saurons jamais. Mais, en tout cas, un événement est survenu qui a tranché la question : c’est la destruction du temple par les Romains. Cette destruction a été un immense bonheur, parce qu’il est douteux que le christianisme eût réussi à se détacher complètement du temple, si le temple eût subsisté.\par
Je le répète, le premier fondateur du {\orgName christianisme}, c’est {\persName Isaïe}, vers l’an 725 avant Jésus-Christ. En introduisant dans le {\orgName monde Israélite} l’idée d’une religion morale, l’idée de la justice et de la valeur secondaire des sacrifices, {\persName Isaïe} a précédé {\persName Jésus} de sept siècles. A l’idée de la religion pure se joint, chez les prophètes du même temps, la conception d’une espèce d’âge d’or, qui apparaît déjà dans l’avenir. Le trait caractéristique d’{\orgName Israël}, c’est l’annonce obstinée d’un avenir brillant pour l’humanité, d’un état où la justice régnera sur la terre, où les cultes inférieurs, grossiers, idolâtriques, disparaîtront. Cela se trouve dans les parties authentiques d’{\persName Isaïe}. Vous savez qu’il y a une analyse délicate à faire dans les œuvres de ce prophète. La dernière partie du livre qu’on lui attribue est postérieure à la captivité ; mais les chapitres que j’ai en vue, les chapitres \textsc{xi, xix, xxiii, xxxii}, par exemple, sont indubitablement d’{\persName Isaïe} lui-même ; or c’est là qu’on insiste le plus sur la conversion des païens de l’{\placeName Égypte}, de {\placeName Tyr}, de l’{\placeName Assyrie}.\par
Ainsi l’idolâtrie disparaîtra du monde, elle disparaîtra par le fait du {\orgName peuple juif} ; le {\orgName peuple juif} sera alors comme « une bannière » que les peuples verront à l’horizon et autour de laquelle ils viendront se rallier. L’idéal messianique ou sibyllin est donc arrêté dans ses lignes essentielles bien avant la captivité de {\placeName Babylone}. {\orgName Israël} rêve un avenir de bonheur pour l’humanité, un royaume parfait, dont la capitale sera {\placeName Jérusalem}, où tous les peuples viendront rendre hommage à l’{\persName Éternel}. Il est clair qu’une pareille religion n’est pas nationale. Il y a au fond de tout cela une part d’orgueil national, sans contredit : quelle est l’œuvre historique où un tel fond ne se retrouve pas ? Mais l’idée, vous le voyez, est universelle au premier chef, et, de là à la propagande, à la prédication, il n’y avait qu’un pas. Le monde, à cette époque, ne se prêtait pas à une grande propagande comme fut plus tard l’apostolat chrétien. Les missions de {\persName saint Paul}, les relations des Églises entre elles n’étaient possibles qu’avec l’{\orgName empire romain}. Mais la conception d’une religion universelle n’en est pas moins née dans le sein du vieil {\orgName Israël}. Elle se manifeste bien plus énergiquement encore dans les écrits de la captivité. Le siècle qui suivit la destruction de {\placeName Jérusalem} fut pour le génie juif une époque de merveilleux épanouissement. Rappelez-vous les beaux chapitres qu’on a mis à la suite du livre d’{\persName Isaïe} : \emph{« Lève-toi, resplendis, {\orgName Jérusalem} ; car la lumière de l’{\persName Éternel} va se lever sur toi ! »} Rappelez-vous encore l’image de {\persName Zacharie}. \emph{« Il arrivera un jour où dix hommes de toutes les langues s’attacheront aux pans de la robe d’un juif et lui diront : Mène-nous à {\placeName Jérusalem} ; c’est là qu’on fait les vrais sacrifices, les seuls qu’agrée l’{\persName Éternel}. »} La lumière émanera donc du peuple juif, et cette lumière remplira le monde entier. Une telle idée n’a rien d’ethnographique ; elle est universelle au plus haut degré, et le peuple qui la proclame est évidemment appelé à une destinée qui dépassera de beaucoup les bornes d’un rôle national déterminé.\par
Qu’arriva-t-il, au point de vue de la race, pendant la captivité et surtout pendant cette longue période de la domination perse, depuis l’an 530 environ avant Jésus-Christ jusqu’à {\persName Alexandre} ? Nous ne le savons pas. Y eut-il, à cette époque, en {\placeName Israël}, beaucoup de mélanges ethniques ? Il serait téméraire de l’affirmer ; mais, d’un autre côté, on ne peut s’empêcher d’en reconnaître la possibilité. La haie qui entourait {\placeName Israël} dut, pendant ce temps de désorganisation, subir plus d’une brèche. Je ne vois guère qu’un fait qu’on puisse rattacher au sujet qui nous occupe : c’est la profonde aversion que les réformateurs {\persName Néhémie} et {\persName Esdras} manifestent pour les mariages mixtes. C’est chez eux une idée fixe. Il est probable que, dans les bandes de juifs qui revenaient de l’Orient, il y avait plus d’hommes que de femmes ; ce qui obligea les émigrants à prendre des femmes dans les tribus voisines. Ces unions sont prohibées au point de vue religieux ; mais c’est précisément parce qu’elles sont sévèrement interdites qu’il est probable qu’elles avaient lieu sur une très grande échelle.\par
Un fait qui a aussi son importance, est ce que l’on raconte du {\placeName royaume de Samarie}, lequel, depuis sa destruction par les {\orgName Assyriens}, aurait été, nous dit-on, peuplé par des étrangers. Il y a là probablement quelque exagération. Le pays, d’après les récits des livres des {\itshape Rois}, aurait été un désert, ce qui n’est pas probable. Il n’est guère douteux cependant que les colons amenés par les {\orgName Assyriens} n’aient introduit dans la masse israélite beaucoup d’éléments qui n’avaient rien de commun avec elle.\par
Arrivons à l’époque grecque et romaine. C’est le moment où le prosélytisme juif arrive à la plus complète expansion ; c’est le moment aussi où l’ethnographie du {\orgName peuple juif}, jusque-là renfermée dans des limites assez resserrées, s’élargit tout à fait et admet une foule d’éléments étrangers. Je parle à des personnes trop instruites pour qu’il me soit nécessaire d’insister sur les détails. Tout le monde sait combien fut active cette propagande juive, durant l’époque grecque, à {\placeName Antioche} et à {\placeName Alexandrie}.\par
En ce qui concerne {\placeName Antioche}, je voudrais appeler votre attention sur un passage de {\persName Josèphe} qui m’a toujours paru fort curieux. C’est dans la \emph{Guerre des Juifs}, livre VII\textsuperscript{e}, chapitre \textsc{iii}, paragraphe 3. {\persName Josèphe} parle de la prospérité extraordinaire de la juiverie d’{\placeName Antioche}, et il dit (je vous traduis littéralement ses paroles) :\par

\begin{quoteblock}
 \noindent Ayant amené à leur culte un grand nombre d’Hellènes, ils en firent une partie de leur communauté.
 \end{quoteblock}

\noindent Il ne s’agit donc pas ici seulement d’hommes menant la vie juive, comme cela eut lieu à {\placeName Rome} plus tard, de prosélytes incirconcis ; non, ce sont des {\orgName Hellènes} en grand nombre (πολυ πληθος) qui se convertissent au judaïsme et qui font partie de la synagogue. Ce ne sont pas ici des demi-juifs, comme seront les judaïsants de la maison des {\orgName Flavius} ; ce sont des gens qui se font juifs et qui acceptent l’acte capital de l’initiation au judaïsme, la circoncision.\par
À {\placeName Alexandrie}, ce fut bien autre chose. Certainement l’{\orgName Église juive d’Alexandrie} était recrutée en très grande partie dans la population égypto-hellénique ; l’hébreu y fut vite oublié. C’est là qu’eut lieu cette production énorme de livres de propagande qui a devancé le christianisme ; c’est là que virent le jour ces livres sibyllins, ces faux auteurs classiques destinés à prêcher le monothéisme. On voulait à tout prix convertir les païens ; les propagandistes, dans leur zèle, ne trouvaient rien de mieux que de prêter à des écrivains anciens, ayant de l’autorité, des ouvrages ou les bonnes doctrines étaient enseignées. C’est ainsi qu’ont été fabriqués le {\itshape Pseudo-Phocylide}, le {\itshape Pseudo-Héraclite}, destinés à prêcher un judaïsme mitigé, réduit aune sorte de religion naturelle.\par
Le fait de cette propagande extraordinaire du judaïsme, de 150 ans environ avant Jésus-Christ jusqu’à 200 ans environ après notre ère, est incontestable. Mais, me direz-vous, qui prouve trop ne prouve rien. Le résultat de ce prosélytisme a été, pour le judaïsme, religieux bien plus qu’ethnographique. Les gens convertis de la sorte se faisaient très rarement circoncire. Ce qu’on appelait à {\placeName Rome} \foreign{{\itshape vitam judaïcam agere}}, c’était simplement pratiquer le sabbat et la morale juive. Les gens « craignant Dieu », les {\itshape metuentes}, les σεβομενοτ, {\itshape judæi improfessi}, ne sont pas restés juifs ; ils n’ont fait que traverser le judaïsme pour devenir chrétiens.\par
Sans doute, la plus grande partie de ces {\orgName Hellènes} qui avaient adopté la vie juive sans la circoncision sont devenus ensuite chrétiens. C’est chez eux que le christianisme a trouvé son terrain primitif. Mais il est certain également qu’un très grand nombre d’entre eux devenaient de véritables juifs.\par
Vous venez d’en avoir la preuve par le passage de {\persName Josèphe} que je vous lisais tout à l’heure. Je pourrais vous citer bien d’autres faits ; ce fait, par exemple, des femmes de {\placeName Damas} qui, selon {\persName Josèphe}, à un moment se trouvèrent toutes juives. La {\placeName Syrie} était le théâtre d’une propagande immense. Mon savant confrère, {\persName M. Joseph Derenbourg}, l’a parfaitement établi. Nous en avons la preuve directe pour {\placeName Palmyre}, pour l’{\placeName Iturée}, pour le {\placeName Hauran}. Rien de plus connu que l’histoire d’{\persName Hélène}, {\persName reine de l’Adiabène}, qui se fit juive avec toute sa famille ; et il est bien probable qu’une grande partie de la population suivit l’exemple de la dynastie. Dans tous ces cas, il ne s’agit point de simples θεσεβεις, de gens « aimant les juifs » ; il s’agit de juifs parfaits, de juifs circoncis.\par
Quand on nierait l’importance des conversions au judaïsme pour les pays grecs et latins, on ne saurait la nier pour l’{\placeName Orient}, pour la {\placeName Syrie} surtout. A {\placeName Palmyre}, par exemple, les inscriptions ont un caractère juif très prononcé.\par
La dynastie des {\orgName Asmonéens} et celle des {\orgName Hérodeg} contribuèrent beaucoup à ce grand courant religieux, qui entraîna dans le judaïsme une masse d’éléments syriens. Les {\orgName Asmonéens} furent conquérants ; ils reconstituèrent à peu près l’ancien domaine d’{\placeName Israël} par la force. Il y avait là des populations qui n’étaient plus juives ; il y en avait beaucoup de païennes. Elles furent conquises par {\persName Jean Hyrcan}, par {\persName Alexandre Jannée}, et forcées d’accepter la circoncision. Le {\itshape compelle intrare} fut assez violent. Sous les {\orgName Hérodes}, l’entraînement se fît par d’autres motifs. Les {\orgName Hérodes} étaient une famille extrêmement riche, et l’appât de beaux mariages amena beaucoup de petits princes de l’{\placeName Orient}, d’{\placeName Émèse}, de {\placeName Cilicie}, de {\placeName Comagène}, à se faire juifs. Il y eut ainsi un nombre considérable de conversions ; si bien qu’on ne saurait exagérer le degré auquel la {\orgName Syrie} a été réellement judaïsée.\par
Permettez-moi de vous lire à ce propos un passage de {\persName Josèphe}, dans son traité {\itshape Contre Apion}, IV, 39.\par

\begin{quoteblock}
 \noindent De là le désir qui s’empara de grandes multitudes d’adopter notre culte, si bien qu’il n’y a pas une ville grecque ou barbare, qu’il n’y a pas une nation où ne se pratique l’usage du sabbat, des jeûnes, des lampes, des distinctions de nourritures que nous observons. Ils cherchent aussi à imiter notre concorde, nos aumônes, notre goût pour le travail (τό φιλεργον εν ταις τεχναις), notre courage à tout souffrir pour la Loi. Car, ce qu’il y a de plus surprenant, c’est que, sans aucun attrait de volupté, la Loi par elle-même a fait ces miracles, et, de même que {\persName Dieu} pénètre l’univers, ainsi la Loi s’est infiltrée parmi tous les hommes. Si quelqu’un doute de ma parole, je l’engage à jeter les yeux sur sa patrie, sur sa famille.
 \end{quoteblock}

\noindent Remarquez ce τό φιλεργον εν ταις τεχναις, « le goût que nous portons dans nos métiers ». En effet, les {\orgName juifs} et les {\orgName chrétiens} pratiquaient en général de petits métiers. C’étaient de bons ouvriers. Là est un des secrets de la grande révolution sociale du christianisme. Ce fut la réhabilitation du travail libre.\par
Il y a dans le passage de {\persName Josèphe} un peu d’exagération ; {\persName Josèphe} est très porté à ce défaut ; mais le fait général qu’il signale a certainement son côté de vérité.\par
Voici maintenant un passage de {\persName Dion Cassius}, qui écrivait vers l’an 225. C’était un homme d’État, un sénateur, qui connaissait son temps. Il va parler d’une des guerres de {\placeName Judée} :\par

\begin{quoteblock}
 \noindent Ce pays, dit-il (livre XXXVII, chap. \textsc{xvii}), se nomme {\placeName Judée}, et les habitants s’appellent {\orgName Juifs}. Je ne connais pas l’origine de ce second nom ; mais il s’applique à d’autres hommes qui ont adopté les institutions de ce peuple, quoique étant d’une autre race. Et il y a parmi les Romains beaucoup de gens de cette sorte, et ce qu’on a fait pour les arrêter n’a fait que les multiplier ; si bien qu’il a fallu leur accorder la liberté de vivre selon leurs lois.
 \end{quoteblock}

\noindent Ce passage est clair : {\persName Dion Cassius} sait qu’il y a des juifs de race, continuateurs de l’ancienne tradition, mais qu’à côté d’eux, il y a des juifs qui ne sont pas juifs de sang, qui néanmoins sont absolument semblables aux juifs pour les observances religieuses.\par
Incontestablement, beaucoup de gens attirés vers le monothéisme restaient dans cette espèce de déisme dont nous trouvons la parfaite expression dans les livres sibyllins ou dans le {\itshape Pseudo-Phocylide}, curieux petit livre, sorte de traité de morale fait pour les païens, dont nous avons, du reste, comme une édition chrétienne dans les prescriptions de ce qu’on appelle le {\orgName concile de Jérusalem}. Ce judaïsme mitigé, fait à l’usage des gentils, supprimait le grand obstacle aux conversions, la circoncision. Il fit, grâce à la prédication chrétienne, une fortune extraordinaire. Mais ce qu’il faut absolument maintenir, c’est que, d’un autre côté, un grand nombre de convertis se faisaient circoncire et devenaient des juifs selon toutes les conditions imposées aux descendants supposés d’{\persName Abraham}.\par
Laissez-moi vous lire un passage de {\persName Juvénal} (\emph{Sat}. XIV, vers 95 et suiv.) qui mérite qu’on en pèse tous les mots :\par


\begin{verse}
Quidam sortiti metuentem sabbata patrem\\
Nil præter nubes et cœli numen adorant,\\
Nec distare putant humana carne suillam,\\
Qua pater abstinuit, mox et præpiitia ponant ;\\
Romanas autem solili contemncre leges,\\
Judaïcum cdiscunt et servant ac metuunt jus,\\
Tradidit arcano quodcumque volumine Moses :\\
Non monstrare vias eadem nisi sacra colenti,\\
Quæsitum ad fontem solos deducere verpos.\\
Sed pater in causa est cui septima quæque fuit lux\\
Ignava et partem vitæ non afligit ullam.\\
\end{verse}

\noindent Ainsi cela commence par un père qui est un simple « craignant {\persName Dieu} » et se borne à pratiquer le sabbat ; mais le fils de ce {\itshape metuens} devient un juif selon toute la force du terme et même un juif fanatique, un contempteur des choses romaines.\par
Ce qu’ajoute {\persName Juvénal} est probablement une calomnie. Je ne crois pas que beaucoup de juifs, à cette époque, aient porté le fanatisme jusqu’à ne pas montrer le chemin à ceux qui n’étaient pas de leur religion. Qu’importe, du reste ? Il n’y a pas d’histoire immaculée. L’histoire du peuple juif est une des plus belles qu’il y ait, et je ne regrette pas d’y avoir consacré ma vie. Mais, que ce soit une histoire absolument sans tache, je suis loin de le prétendre ; ce serait alors une histoire en dehors de l’humanité. Si je pouvais mener une seconde vie, certainement je la consacrerais à l’histoire grecque, qui est encore plus belle, à certains égards, que l’histoire juive. Ce sont là, en quelque sorte, les deux histoires maîtresses du monde. Or, si j’écrivais l’histoire des {\orgName peuples grecs}, cette histoire la plus merveilleuse de toutes, je ne me refuserais pas à y signaler de fâcheuses parties. On peut admirer la {\orgName Grèce} sans se croire obligé d’admirer {\persName Cléon} et les mauvaises pages des annales de la démagogie athénienne. De même, parce qu’on trouve que le {\orgName peuple juif} a été l’apparition peut-être la plus extraordinaire de l’histoire, on n’est pas obligé pour cela de nier qu’il ne se trouve dans sa longue vie de peuple des faits regrettables.\par
Prenons donc les allégations de {\persName Juvénal} pour ce qu’elles valent ; mais suivons son raisonnement. Le mal, selon lui, est l’entraînement de la société romaine vers le judaïsme. Pourquoi y a-t-il tant de gens qui renoncent à la tradition romaine pour adopter la tradition des juifs ? C’est la faute de ceux qui ont d’abord embrassé les pratiques juives, sans s’astreindre à la circoncision. Les pères se sont mis à observer le sabbat ; ils ont été tout simplement des {\itshape metuentes}, des hommes craignant {\persName Dieu} ; les fils se font circoncire et deviennent des juifs ardents.\par
Vous voyez que la grande propagande qui s’exerce depuis {\persName Alexandre} jusque vers le III\textsuperscript{e} siècle de notre ère s’est faite surtout (ceci est hors de doute) au profit du christianisme, mais s’est faite aussi au profit du judaïsme étroit, impliquant les pratiques rigoureuses de la vieille religion d’{\orgName Israël}. Oui, le monde, à une certaine époque, dégoûté des anciennes religions nationales, s’est converti du paganisme au monothéisme. Je vous ai cité quelques textes ; je pourrais vous en citer d’autres. « \emph{Tramgressi in morem eorum, dit {\persName Tacite}, idem usurpant} » (\emph{Hist}., V, 5.). Il s’agit là de la circoncision. Selon {\persName Tacite}, ceux qui passaient au judaïsme se faisaient circoncire. Il y avait donc, parmi les convertis, des gens qui menaient la vie juive sans être circoncis, et d’autres qui étaient de véritables juifs.\par
Une distinction profondément significative est celle qui est établie par une loi d’{\persName Antonin le Pieux}, commentée par {\persName Modestin}. {\persName Antonin} permet aux juifs de circoncire leurs fils, mais leurs fils seulement. Je le répète, quand l’autorité est amenée à défendre une pratique, c’est que cette pratique est répandue et a pris une extension considérable.\par
Je crois. Messieurs, que ces faits suffisent pour établir qu’à l’époque grecque et à l’époque romaine il y a eu une foule de conversions directes au judaïsme. Il en résulte qu’à partir de cette époque le mot judaïsme n’a plus guère de signification ethnographique. Conformément à la prédiction des prophètes, le judaïsme était devenu quelque chose d’universel. Tout le monde y entrait. Le mouvement qui éloigna du paganisme, aux premiers siècles de notre ère, les personnes animées de sentiments religieux délicats, amena une foule de conversions. Le plus grand nombre de ces conversions se fit certainement au christianisme, mais un très grand nombre aussi se fit au judaïsme. La plupart des juifs de {\placeName Gaule} et d’{\placeName Italie}, par exemple, durent provenir de telles conversions, et la synagogue resta à côté de l’{\orgName Église}, comme une minorité dissidente. Il est vrai qu’après cela se produit la grande réaction talmudique, à la suite de la guerre de {\persName Bar-Goziba}. Il en est presque toujours ainsi dans l’histoire : quand un grand et large courant d’idées se produit dans le monde, ceux qui ont été les premiers à le provoquer en sont les premières victimes ; alors ils se repentent presque de ce qu’ils ont fait, et, d’excessivement libéraux qu’ils étaient, ils deviennent étonnamment réactionnaires. ({\itshape On rit}.) Le Talmud, c’est la réaction. Le judaïsme sent qu’il a été trop loin, qu’il va se fondre, se dissoudre dans le christianisme. Alors il se resserre. À partir de ce moment-là, le prosélytisme disparaît ; les prosélytes sont traités de fléau, de « lèpre d’{\placeName Israël} ».\par
Mais, avant cela, je le répète, les portes avaient été largement ouvertes.\par
Le talmudisme même les a-t-il complètement fermées ? Non, certes ; le prosélytisme, condamné par les docteurs, n’en continua pas moins d’être pratiqué par des laïques pieux, plus fidèles à l’ancien esprit que les puritains de la Thora. Seulement, désormais, il faut faire une distinction. Les {\orgName juifs orthodoxes}, observateurs rigoureux de la Loi, se serrent les uns contre les autres, et, comme la Loi ne se peut très bien observer que dans une société religieuse étroitement fermée, ils se séquestrent systématiquement du reste du monde pendant des siècles. Mais, en dehors des talmudistes scrupuleux, il y a des juifs à idées relativement larges.\par
Je ne connais rien de plus curieux à cet égard que les sermons de {\persName saint Jean Chrysostome} contre les {\orgName juifs}. Le fond de la discussion, dans ces sermons, n’a pas un grand intérêt ; mais l’orateur, alors prêtre d’{\placeName Antioche}, se montre constamment obsédé d’une idée fixe : c’est d’empêcher ses fidèles d’aller à la synagogue pour y prêter serment, pour y célébrer la fête de Pâques. Il est évident que la distinction des deux sectes, dans cette grande ville d’{\placeName Antioche}, était, à cette époque, encore à peine faite.\par
{\persName Grégoire de Tours} nous a conservé, sur le judaïsme dans les {\placeName Gaules}, des renseignements inappréciables. Il y avait beaucoup de juifs à {\placeName Paris}, à {\placeName Orléans}, à {\placeName Clermont}. {\persName Grégoire de Tours} les combat comme des hérétiques. Il ne se doute pas que ce soient des gens d’une autre race. Vous me direz que l’ethnographie n’était pas très familière à un esprit aussi simple. Cela est vrai ; mais d’où venaient ces juifs d’{\placeName Orléans} et de {\placeName Paris} ? Pouvons-nous supposer que tous fussent les descendants d’Orientaux venus de {\placeName Palestine} à une certaine époque, et qui auraient fondé des espèces de colonies dans certaines villes ? Je ne le crois pas. Il y eut sans doute, en {\placeName Gaule}, des émigrés juifs, qui remontèrent le {\placeName Rhône} et la {\placeName Saône}, et servirent en quelque sorte de levain ; mais il y eut aussi une foule de gens qui se rattachèrent au judaïsme par conversion et qui n’avaient pas un seul ancêtre en {\placeName Palestine}. Et quand on pense que les juiveries d’{\placeName Allemagne} et d’{\placeName Angleterre} sont venues de {\placeName France}, on se prend à regretter de n’avoir pas plus de données sur les origines du judaïsme dans notre pays. On verrait probablement que le juif des {\placeName Gaules} du temps de {\persName Gontran} et de {\persName Chilpéric} n’était, le plus souvent, qu’un Gaulois professant la religion Israélite.\par
Laissons de côté ces faits obscurs ; il y en a beaucoup de parfaitement clairs : d’abord la conversion de l’{\orgName Arabie} et de l’{\orgName Abyssinie}, qui n’est niée par personne. Le judaïsme avait accompli en {\placeName Arabie}, avant {\persName Mahomet}, d’immenses conquêtes ; une foule d’{\orgName Arabes} s’y étaient rattachés. Il n’a tenu qu’à un fil que l’{\orgName Arabie} ne soit devenue juive. {\persName Mahomet} a été juif à une certaine époque de sa vie, et on peut dire, jusqu’à un certain point, qu’il l’est resté toujours. Les {\orgName Falaschas}, ou {\orgName juifs d’Abyssinie}, sont des {\orgName Africains}, parlant une langue africaine et lisant la Bible traduite en cet idiome africain.\par
Mais il y a un événement historique plus important, plus rapproché de nous, et qui semble avoir eu des suites très graves : c’est la conversion des {\orgName Khozars}, sur laquelle nous avons des renseignements précis. Ce royaume des {\orgName Khozars}, qui occupait presque toute la {\placeName Russie méridionale}, adopta le judaïsme vers le temps de {\persName Charlemagne}. A ce fait historique, se rattachent les {\orgName karaïtes} de la {\placeName Russie méridionale} et ces inscriptions hébraïques de la {\placeName Crimée} où, dès le VIII\textsuperscript{e} siècle, on trouve des noms tatars et turcs, tels que Toktamisch. Est-ce qu’un juif d’origine palestinienne se serait jamais appelé Toktamisch, au lieu de s’appeler Abraham, Lévy ou Jacob ? Évidemment non ; ce Toktamisch était un Tatar, un Nogaï converti ou fils de converti.\par
Cette conversion du royaume des {\orgName Khozars} a une importance considérable dans la question de l’origine des juifs qui habitent les pays danubiens et le midi de la {\placeName Russie}. Ces régions renferment de grandes masses de populations juives qui n’ont probablement rien ou presque rien d’ethnographiquement juif. Une circonstance particulière a dû amener dans le sein du judaïsme beaucoup de gens non juifs de race. C’est l’esclavage ou la domesticité. Nous voyons que, dans tous les pays chrétiens, surtout dans les pays slaves, la grande préoccupation des évêques, des conciles, est de défendre aux juifs d’avoir des serviteurs chrétiens. La domesticité favorisait le prosélytisme, et les esclaves des juifs étaient entraînés plus ou moins à la profession du judaïsme.\par
Il est donc hors de doute que le judaïsme représenta d’abord la tradition d’une race particulière. Il est hors de doute aussi qu’il y a eu dans le phénomène de la formation de la race Israélite actuelle un apport de sang palestinien primitif ; mais, en même temps, j’ai la conviction qu’il y a dans l’ensemble de la population juive, telle qu’elle existe de nos jours, une part considérable de sang non sémitique ; si bien que cette race, que l’on considère comme l’idéal de l’ethnos pur, se conservant à travers les siècles par l’interdiction des mariage mixtes, a été fortement pénétrée d’infusions étrangères, un peu comme cela a lieu pour toutes les autres races. En d’autres termes, le judaïsme à l’origine fut une religion nationale ; il est redevenu de nos jours une religion fermée ; mais, dans l’intervalle, pendant de longs siècles, le judaïsme a été ouvert ; des masses très considérables de populations non Israélites de sang ont embrassé le judaïsme ; en sorte que la signification de ce mot, au point de vue de l’ethnographie, est devenue fort douteuse.\par
On m’objectera ce qu’on appelle le type juif. Il y en aurait long à dire sur ce point. Mon opinion est qu’il n’y a pas un type juif, mais qu’il y a des types juifs. J’ai acquis à cet égard une assez grande expérience, ayant été pendant dix ans à la {\orgName Bibliothèque nationale}, attaché à la collection des manuscrits hébreux, en sorte que les savants israélites du monde entier s’adressaient à moi pour consulter notre précieuse collection. Je reconnaissais très vite mes clients, et, d’un bout à l’autre de la salle, je devinais ceux qui allaient venir à mon bureau. Eh bien, le résultat de mon expérience est qu’il n’y a pas un type juif unique, mais qu’il y en a plusieurs, lesquels sont absolument irréductibles les uns aux autres. Comment la race s’est-elle ainsi cantonnée en quelque sorte dans un certain nombre de types ?\par
Par suite de ce que nous disions tout à l’heure, par la séquestration, le ghetto, par l’interdiction des mariages mixtes.\par
L’ethnographie est une science fort obscure ; car on ne peut pas y faire d’expérience, et il n’y a de certain que ce qu’on peut expérimenter. Ce que je vais dire n’est pas pour prouver, c’est seulement pour expliquer ma pensée. Je crois que, si l’on prenait au hasard des milliers de personnes, celles, par exemple, qui se promènent en ce moment d’un bout à l’autre du {\placeName boulevard Saint-Germain}, qu’on les suppose déportées dans une île déserte et libres de s’y multiplier ; je crois, dis-je, qu’au bout d’un temps donné, les types seraient réduits, massés en quelque sorte, concentrés en un certain nombre de types vainqueurs des autres, qui auraient persisté et qui se seraient constitués d’une façon irréductible. La concentration des types résulte du fait des mariages s’effectuant, pendant des siècles, dans un cercle resserré.\par
On allègue aussi en faveur de l’unité ethnique des {\orgName juifs} la similitude des mœurs, des habitudes. Toutes les fois que vous mettrez ensemble des personnes de n’importe quelle race et que vous les astreindrez à une vie de ghetto, vous aurez les mêmes résultats. Il y a, si l’on peut s’exprimer ainsi, une psychologie des minorités religieuses, et cette psychologie est indépendante de la race. La position des {\orgName protestants}, dans un pays où, comme en {\placeName France}, le protestantisme est en minorité, a beaucoup d’analogie avec celle des juifs, parce que les protestants, pendant fort longtemps, ont été obligés de vivre entre eux et qu’une foule de choses leur ont été interdites, comme aux juifs. Il se crée ainsi des similitudes qui ne viennent pas de la race, mais qui sont le résultat de certaines analogies de situation. Les habitudes d’une vie concentrée, gênée, pleine d’interdictions, séquestrée en quelque sorte, se retrouvent partout les mêmes, quelle que soit la race. Les calomnies répandues dans les parties peu éclairées de la population contre les {\orgName protestants} et contre les {\orgName juifs} sont les mêmes. Les professions vers lesquelles une secte exclue de la vie commune est obligée de se porter sont les mêmes. Comme les {\orgName juifs}, les {\orgName protestants} n’ont ni peuple ni paysans ; on les a empêchés d’en avoir\footnote{Le travail sur les {\orgName juifs de France} dans la première moitié du moyen âge, inséré dans le tome \textsc{xxvii}\textsuperscript{e} de \emph{L’Histoire littéraire de la France}, montre que, jusqu’aux ordonnances de {\persName Phillippe le Bel}, les {\orgName juifs de France} exerçaient les mêmes métiers et professions que les autres Français.}. — Quant à la similitude d’esprit dans le sein d’une même secte, elle s’explique suffisamment par la similitude d’éducation, de lectures, de pratiques religieuses.\par
On observe en {\placeName Syrie} un fait qui vient à l’appui de ma thèse. Il existe, à une douzaine de lieues au nord de {\placeName Damas}, des villages où l’on parle encore le syriaque, qui a presque disparu partout ailleurs, et qu’on ne retrouve plus que là et à une grande distance au nord, du côté de {\placeName Van} et d’{\placeName Ourmia}. Les gens de ces villages sont musulmans et ressemblent à tous les musulmans de {\placeName Syrie} sous le rapport des mœurs. Rien de plus dissemblable, à première vue, que le chrétien et le musulman en {\placeName Syrie} : le chrétien, qui est la créature la plus timide du monde ; le musulman, qui a l’habitude de porter les armes et de dominer. On dirait, au premier coup d’œil, qu’il y a là une différence ethnographique bien caractérisée. À propos de l’émotion qui eut lieu à {\placeName Beyrouth} il y a quelques mois, mon excellent ami, le docteur S…, m’écrivait que son domestique rentra en lui disant : \emph{« S’il y avait eu là un enfant musulman avec un sabre, il aurait pu tuer mille chrétiens. »} Eh bien, c’est ici que le fait des villages aux environs de {\placeName Damas} prend un vif intérêt. S’il y a au monde des Syriens authentiques, ce sont ces gens-là, puisqu’ils parlent encore leur vieille langue ; et pourtant ils sont musulmans et ressemblent pour les habitudes et les mœurs à tous les autres musulmans. La différence qui existe entre eux et les {\orgName Syriens chrétiens} résulte donc de la différence du genre de vie et d’une situation sociale prolongée durant des siècles ; elle n’a absolument rien d’ethnographique.\par
De même, chez les {\orgName juifs}, la physionomie particulière et les habitudes de vie sont bien plus le résultat de nécessités sociales qui ont pesé sur eux pendant des siècles, qu’elles ne sont un phénomène de race.\par
Réjouissons-nous, Messieurs, que ces questions, si intéressantes pour l’histoire et l’ethnographie, n’aient en {\placeName France} aucune importance pratique. Nous avons, en effet, résolu la difficulté politique qui s’y rattache de la bonne manière. Quand il s’agit de nationalité, nous faisons de la question de race une question tout à fait secondaire, et nous avons raison. Le fait ethnographique, capital aux origines de l’histoire, va toujours perdant de son importance à mesure qu’on avance en civilisation. Quand l’{\orgName Assemblée nationale}, en 1791, décréta l’émancipation des {\orgName juifs}, elle s’occupa extrêmement peu de la race. Elle estima que les hommes devaient être jugés non par le sang qui coule dans leurs veines, mais par leur valeur morale et intellectuelle. C’est la gloire de la France de prendre les questions par le côté humain. L’œuvre du XIX\textsuperscript{e} siècle est d’abattre tous les ghettos, et je ne fais pas mon compliment à ceux qui ailleurs cherchent à les relever. La race israélite a rendu au monde les plus grands services. Assimilée aux différentes nations, en harmonie avec les diverses unités nationales, elle continuera à faire dans l’avenir ce qu’elle a fait dans le passé. Par sa collaboration avec toutes les forces libérales de l’{\placeName Europe}, elle contribuera éminemment au progrès social de l’humanité. ({\itshape Applaudissements prolongés}.)
\section[{L’islamisme et la science}]{L’islamisme et la science}\renewcommand{\leftmark}{L’islamisme et la science}


\dateline{Conférence faite à la {\placeName Sorbonne}, le 29 mars 1883.}

\salute{Mesdames et Messieurs,}
\noindent J’ai déjà tant de fois fait l’épreuve de l’attention bienveillante de cet auditoire, que j’ai osé choisir, pour le traiter aujourd’hui devant vous, un sujet des plus subtils, rempli de ces distinctions délicates où il faut entrer résolument quand on veut faire sortir l’histoire du domaine des à peu près. Ce qui cause presque toujours les malentendus en histoire, c’est le manque de précision dans l’emploi des mots qui désignent les nations et les races. On parle des {\orgName Grecs}, des {\orgName Romains}, des {\orgName Arabes} comme si ces mots désignaient des groupes humains toujours identiques à eux-mêmes, sans tenir compte des changements produits par les conquêtes militaires, religieuses, linguistiques, par la mode et les grands courants de toute sorte qui traversent l’histoire de l’humanité. La réalité ne se gouverne pas selon des catégories aussi simples. Nous autres {\orgName Français}, par exemple, nous sommes romains par la langue, grecs par la civilisation, juifs par la religion. Le fait de la race, capital à l’origine, va toujours perdant de son importance à mesure que les grands faits universels qui s’appellent civilisation grecque, conquête romaine, conquête germanique, christianisme, islamisme, renaissance, philosophie, révolution, passent comme des rouleaux broyeurs sur les primitives variétés de la famille humaine et les forcent à se confondre en masses plus ou moins homogènes. Je voudrais essayer de débrouiller avec vous une des plus fortes confusions d’idées que l’on commette dans cet ordre, je veux parler de l’équivoque contenue dans ces mots : science arabe, philosophie arabe, art arabe, science musulmane, civilisation musulmane. Des idées vagues qu’on se fait sur ce point résultent beaucoup de faux jugements et même des erreurs pratiques quelquefois assez graves.\par
Toute personne un peu instruite des choses de notre temps voit clairement l’infériorité actuelle des pays musulmans, la décadence des États gouvernés par l’islam, la nullité intellectuelle des races qui tiennent uniquement de cette religion leur culture et leur éducation. Tous ceux qui ont été en {\placeName Orient} ou en {\placeName Afrique} sont frappés de ce qu’a de fatalement borné l’esprit d’un vrai croyant, de cette espèce de cercle de fer qui entoure sa tête, la rend absolument fermée à la science, incapable de rien apprendre ni de s’ouvrir à aucune idée nouvelle. À partir de son initiation religieuse, vers l’âge de dix ou douze ans, l’enfant musulman, jusque-là quelquefois assez éveillé, devient tout à coup fanatique, plein d’une sotte fierté de posséder ce qu’il croit la vérité absolue, heureux comme d’un privilège de ce qui fait son infériorité. Ce fol orgueil est le vice radical du musulman. L’apparente simplicité de son culte lui inspire un mépris peu justifié pour les autres religions. Persuadé que {\persName Dieu} donne la fortune et le pouvoir à qui bon lui semble, sans tenir compte de l’instruction ni du mérite personnel, le musulman a le plus profond mépris pour l’instruction, pour la science, pour tout ce qui constitue l’esprit européen. Ce pli inculqué par la foi musulmane est si fort que toutes les différences de race et de nationalité disparaissent par le fait de la conversion à l’islam. Le Berber, le Soudanien, le Circassien, le Malais, l’Égyptien, le Nubien, devenus musulmans, ne sont plus des Berbers, des Soudaniens, des Égyptiens, etc. ; ce sont des musulmans. La {\orgName Perse} seule fait ici exception ; elle a su garder son génie propre ; car la {\orgName Perse} a su prendre dans l’islam une place à part ; elle est au fond bien plus chiite que musulmane.\par
Pour atténuer les fâcheuses inductions qu’on est porté à tirer de ce fait si général, contre l’islam, beaucoup de personnes font remarquer que cette décadence, après tout, peut n’être qu’un fait transitoire. Pour se rassurer sur l’avenir elles font appel au passé. Cette civilisation musulmane, maintenant si abaissée, a été autrefois très brillante. Elle a eu des savants, des philosophes. Elle a été, pendant des siècles, la maîtresse de l’{\placeName Occident chrétien}. Pourquoi ce qui a été ne serait-il pas encore ? Voilà le point précis sur lequel je voudrais faire porter le débat. Y a-t-il eu réellement une science musulmane, ou du moins une science admise par l’islam, tolérée par l’islam ?\par
Il y a dans les faits qu’on allègue une très réelle part de vérité. Oui ; de l’an 775 à peu près, jusque vers le milieu du XIII\textsuperscript{e} siècle, c’est-à-dire pendant cinq cents ans environ, il y a eu dans les pays musulmans des savants, des penseurs très distingués. On peut même dire que, pendant ce temps, le monde musulman a été supérieur, pour la culture intellectuelle, au monde chrétien. Mais il importe de bien analyser ce fait pour n’en pas tirer des conséquences erronées. Il importe de suivre siècle par siècle l’histoire de la civilisation en {\placeName Orient} pour faire la part des éléments divers qui ont amené cette supériorité momentanée, laquelle s’est bientôt changée en une infériorité tout à fait caractérisée.\par
Rien de plus étranger à tout ce qui peut s’appeler philosophie ou science que le premier siècle de l’islam. Résultat d’une lutte religieuse qui durait depuis plusieurs siècles et tenait la conscience de l’{\placeName Arabie} en suspens entre les diverses formes de monothéisme sémitique, l’islam est à mille lieues de tout ce qui peut s’appeler rationalisme ou science. Les cavaliers arabes qui s’y rattachèrent comme à un prétexte pour conquérir et piller furent, à leur heure, les premiers guerriers du monde ; mais c’étaient assurément les moins philosophes des hommes. Un écrivain oriental du XIII\textsuperscript{e} siècle, {\persName Aboul-Faradj}, traçant le caractère du peuple arabe, s’exprime ainsi : \emph{« La science de ce peuple, celle dont il se faisait gloire, était la science de la langue, la connaissance de ses idiotismes, la texture des vers, l’habile composition de la prose… Quant à la philosophie, {\persName Dieu} ne lui en avait rien appris, et ne l’y avait pas rendu propre. »} Rien de plus vrai. L’Arabe nomade, le plus littéraire des hommes, est de tous les hommes le moins mystique, le moins porté à la méditation. L’Arabe religieux se contente, pour l’explication des choses, d’un Dieu créateur, gouvernant le monde directement et se révélant à l’homme par des prophètes successifs. Aussi, tant que l’islam fut entre les mains de la race arabe, c’est-à-dire-sous les quatre premiers califes et sous les {\orgName Omeyyades}, ne se produisit-il dans son sein aucun mouvement intellectuel d’un caractère profane. Omar n’a pas brûlé, comme on le répète souvent, la {\orgName bibliothèque d’Alexandrie} ; cette bibliothèque, de son temps, avait à peu près disparu ; mais le principe qu’il a fait triompher dans le monde était bien en réalité destructeur de la recherche savante et du travail varié de l’esprit.\par
Tout fut changé, quand, vers l’an 750, la {\orgName Perse} prit le dessus et fit triompher la dynastie des {\orgName enfants d’Abbas} sur celle des {\orgName Beni-Omeyya}. Le centre de l’islam se trouva transporté dans la région du {\placeName Tigre} et de l’{\placeName Euphrate}. Or ce pays était plein encore des traces d’une des plus brillantes civilisations que l’{\placeName Orient} ait connues, celles des {\orgName Perses Sassanides}, qui avait été portée à son comble sous le règne de {\persName Chosroès Nouschirvan}. L’art et l’industrie florissaient en ces pays depuis des siècles. {\persName Chosroès} y ajouta l’activité intellectuelle. La philosophie, chassée de {\placeName Constantinople}, vint se réfugier en {\placeName Perse} ; {\persName Chosroès} fit traduire les livres de l’{\placeName Inde}. Les {\orgName chrétiens nestoriens}, qui formaient l’élément le plus considérable de la population, étaient versés dans la science et la philosophie grecques ; la médecine était tout entière entre leurs mains ; leurs évoques étaient des logiciens, des géomètres. Dans les épopées persanes, dont la couleur locale est empruntée aux temps sassanides, quand {\persName Roustem} veut construire un pont, il fait venir un {\itshape djathalik} ({\itshape catholicos}, nom des patriarches ou évêques nestoriens) en guise d’ingénieur.\par
Le terrible coup de vent de l’islam arrêta net, pendant une centaine d’années, tout ce beau développement iranien. Mais l’avènement des {\orgName Abbasides} sembla une résurrection de l’éclat des {\orgName Chosroès}. La révolution qui porta cette dynastie au trône fut faite par des troupes persanes, ayant des chefs persans. Ses fondateurs, {\persName Aboul-Abbas} et surtout {\persName Mansour}, sont toujours entourés de Persans. Ce sont, en quelque sorte, des Sassanides ressuscites ; les conseillers intimes, les précepteurs des princes, les premiers ministres sont les Barmékides, famille de l’ancienne Perse, très éclairée, restée fidèle au culte national, au parsisme, et qui ne se convertit à l’islam que tard et sans conviction. Les nestoriens entourèrent bientôt ces califes peu croyants et devinrent, par une sorte de privilège exclusif, leurs premiers médecins. Une ville qui a eu dans l’histoire de l’esprit humain un rôle tout à fait à part, la ville de {\placeName Harran}, était restée païenne et avait gardé toute la tradition scientifique de l’antiquité grecque ; elle fournit à la nouvelle école un contingent considérable de savants étrangers aux religions révélées, surtout d’habiles astronomes.\par
{\placeName Bagdad} s’éleva comme la capitale de cette {\placeName Perse} renaissante. La langue de la conquête, l’arabe, ne put être supplantée, pas plus que la religion tout à fait reniée ; mais l’esprit de cette nouvelle civilisation fut essentiellement mixte. Les parsis, les chrétiens, l’emportèrent ; l’administration, la police en particulier, fut entre les mains des chrétiens. Tous ces brillants califes, contemporains de nos {\orgName Carlovingiens}, {\persName Mansour}, {\persName Haroun al-Raschid}, {\persName Mamoun} sont à peine musulmans. Ils pratiquent extérieurement la religion dont ils sont les chefs, les papes, si l’on peut s’exprimer ainsi ; mais leur esprit est ailleurs. Ils sont curieux de toute chose, principalement des choses exotiques et païennes ; ils interrogent l’{\orgName Inde}, la vieille {\orgName Perse}, la {\orgName Grèce} surtout. Parfois, il est vrai, les piétistes musulmans amènent à la cour d’étranges réactions ; le calife, à certains moments, se fait dévot et sacrifie ses amis infidèles ou libres penseurs ; puis le souffle de l’indépendance reprend le dessus ; alors le calife rappelle ses savants et ses compagnons de plaisir, et la libre vie recommence, au grand scandale des musulmans puritains.\par
Telle est l’explication de cette curieuse et attachante civilisation de {\placeName Bagdad}, dont les fables des \emph{Mille et une Nuits} ont fixé les traits dans toutes les imaginations, mélange bizarre de rigorisme officiel et de secret relâchement, âge de jeunesse et d’inconséquence, où les arts sérieux et les arts de la vie joyeuse fleurissent grâce à la protection des chefs mal pensants d’une religion fanatique ; où le libertin, bien que toujours sous la menace des plus cruels châtiments, est flatté, recherché à la cour. Sous le règne de ces califes, parfois tolérants, parfois persécuteurs à regret, la libre pensée se développa ; les {\itshape motecallemîn} ou « disputeurs » tenaient des séances où toutes les religions étaient examinées d’après la raison. Nous avons en quelque sorte le compte rendu d’une de ces séances fait par un dévot. Permettez-moi de vous le lire, tel que {\persName M. Dozy} l’a traduit.\par
Un docteur de {\placeName Kairoan} demande à un pieux théologien espagnol, qui avait fait le voyage de {\placeName Bagdad}, si, pendant son séjour dans cette ville, il a jamais assisté aux séances des {\itshape motecallemîn} . \emph{« J’y ai assisté deux fois, répond l’Espagnol, mais je me suis bien gardé d’y retourner. — Et pourquoi ? lui demanda son interlocuteur. — Vous allez en juger, répondit le voyageur. A la première séance à laquelle j’assistai, se trouvaient non seulement des musulmans de toute sorte, orthodoxes et hétérodoxes, mais aussi des mécréants, des guèbres, des matérialistes, des athées, des juifs, des chrétiens ; bref, il y avait des incrédules de toute espèce. Chaque secte avait son chef, chargé de défendre les opinions qu’elle professait, et, chaque fois qu’un de ces chefs entrait dans la salle, tous se levaient en signe de respect, et personne ne reprenait sa place avant que le chef se fût assis. La salle fut bientôt comble, et, lorsqu’on se vit au complet, un des incrédules prit la parole : “Nous sommes réunis pour raisonner, dit-il. Vous connaissez tous les conditions. Vous autres, musulmans, vous ne vous alléguerez pas des raisons tirées de votre livre ou fondées sur l’autorité de votre prophète ; car nous ne croyons ni à l’un ni à l’autre. Chacun doit se borner à des arguments tirés de la raison.” Tous applaudirent à ces paroles. — Vous comprenez, ajoute l’Espagnol, qu’après avoir entendu de telles choses, je ne retournai plus dans cette assemblée. On me proposa d’en visiter une autre ; mais c’était le même scandale. »}\par
Un véritable mouvement philosophique et scientifique fut la conséquence de ce ralentissement momentané de la rigueur orthodoxe. Les médecins syriens chrétiens, continuateurs des dernières écoles grecques, étaient fort versés dans la philosophie péripatéticienne, dans les mathématiques, dans la médecine, l’astronomie. Les califes les employèrent à traduire en arabe l’encyclopédie d’{\persName Aristote}, {\persName Euclide}, {\persName Galien}, {\persName Ptolémée}, en un mot tout l’ensemble de la science grecque tel qu’on le possédait alors. Des esprits actifs, tels qu’{\persName Alkindi}, commencèrent à spéculer sur les problèmes éternels que l’homme se pose sans pouvoir les résoudre. On les appela {\itshape filsouf} ({\itshape philosophos}), et dès lors ce mot exotique fut pris en mauvaise part comme désignant quelque chose d’étranger à l’islam. {\itshape Filsouf} devint chez les musulmans une appellation redoutable, entraînant souvent la mort ou la persécution, comme {\itshape zendik} et plus tard {\itshape farmaçoun} (franc-maçon). C’était, il faut l’avouer, le rationalisme le plus complet qui se produisait au sein de l’islam. Une sorte de société philosophique, qui s’appelait les {\itshape Ikhwan es-safa}, « les frères de la sincérité », se mit à publier une encyclopédie philosophique, remarquable par la sagesse et l’élévation des idées. Deux très grands hommes, {\persName Alfarabi} et {\persName Avicenne}, se placent bientôt au rang des penseurs les plus complets qui aient existé. L’astronomie et l’algèbre prennent, en Perse surtout, de remarquables développements. La chimie poursuit son long travail souterrain, qui se révèle au dehors par d’étonnants résultats, tels que la distillation, peut-être la poudre. L’{\orgName Espagne musulmane} se met à ces études à la suite de l’{\orgName Orient} ; les juifs y apportent une collaboration active. {\persName Ibn-Badja}, {\persName Ibn-Tofaïl}, {\persName Averroès} élèvent la pensée philosophique, au XII\textsuperscript{e} siècle, à des hauteurs où, depuis l’antiquité, on ne l’avait point vue portée.\par
Tel est ce grand ensemble philosophique, que l’on a coutume d’appeler arabe, parce qu’il est écrit en arabe, mais qui est en réalité gréco-sassanide. Il serait plus exact de dire grec ; car l’élément vraiment fécond de tout cela venait de la Grèce. On valait, dans ces temps d’abaissement, en proportion de ce qu’on savait de la vieille {\orgName Grèce}. La {\orgName Grèce} était la source unique du savoir et de la droite pensée. La supériorité de la {\orgName Syrie} et de {\orgName Bagdad} sur l’{\orgName Occident latin} venait uniquement de ce qu’on y touchait de bien plus près la tradition grecque. Il était plus facile d’avoir un {\persName Euclide}, un {\persName Ptolémée}, un {\persName Aristote} à {\placeName Harran}, à {\placeName Bagdad} qu’à {\placeName Paris}. Ah ! si les {\orgName Byzantins} avaient voulu être gardiens moins jaloux des trésors qu’à ce moment ils ne lisaient guère ; si, dès le VIII\textsuperscript{e} ou le IX\textsuperscript{e} siècle, il y avait eu des {\orgName Bessarion} et des {\orgName Lascaris} ! On n’aurait pas eu besoin de ce détour étrange qui fit que la science grecque nous arriva au XII\textsuperscript{e} siècle, en passant par la {\placeName Syrie}, par {\placeName Bagdad}, par {\placeName Cordoue}, par {\placeName Tolède}. Mais cette espèce de providence secrète qui fait que, quand le flambeau de l’esprit humain va s’éteindre entre les mains d’un peuple, un autre se trouve là pour le relever et le rallumer, donna une valeur de premier ordre à l’œuvre, sans cela obscure, de ces pauvres {\orgName Syriens}, de ces {\itshape filsouf} persécutés, de ces {\orgName Harraniens} que leur incrédulité mettait au ban de l’humanité d’alors. Ce fut par ces traductions arabes des ouvrages de science et de philosophie grecque que l’{\orgName Europe} reçut le ferment de tradition antique nécessaire à l’éclosion de son génie.\par
En effet, pendant qu’{\persName Averroès}, le dernier philosophe arabe, mourait à {\placeName Maroc}, dans la tristesse et l’abandon, notre {\orgName Occident} était en plein éveil. {\persName Abélard} a déjà poussé le cri du rationalisme renaissant. L’{\orgName Europe} a trouvé son génie et commence cette évolution extraordinaire, dont le dernier terme sera la complète émancipation de l’esprit humain. Ici, sur la {\placeName montagne Sainte-Geneviève}, se créait un sensorium nouveau pour le travail de l’esprit. Ce qui manquait, c’étaient les livres, les sources pures de l’antiquité. Il semble au premier coup d’œil qu’il eût été plus naturel d’aller les demander aux bibliothèques de {\placeName Constantinople}, où se trouvaient les originaux, qu’à des traductions souvent médiocres en une langue qui se prêtait peu à rendre la pensée grecque. Mais les discussions religieuses avaient créé entre le monde latin et le monde grec une déplorable antipathie ; la funeste croisade de 1204 ne fit que l’exaspérer. Et puis, nous n’avions pas d’hellénistes ; il fallait encore attendre trois cents ans pour que nous eussions un {\persName Lefèvre d’Étaples}, un {\persName Budé}.\par
A défaut de la vraie philosophie grecque authentique, qui était dans les bibliothèques byzantines, on alla donc chercher en {\placeName Espagne} une science grecque mal traduite et frelatée. Je ne parlerai pas de {\persName Gerbert}, dont les voyages parmi les musulmans sont chose fort douteuse ; mais, dès le XI\textsuperscript{e} siècle, {\persName Constantin l’Africain} est supérieur en connaissances à son temps et à son pays, parce qu’il a reçu une éducation musulmane. De 1130 à 1150, un collège actif de traducteurs, établi à {\placeName Tolède} sous le patronage de l’{\persName archevêque Raymond}, fait passer en latin les ouvrages les plus importants de la science arabe. Dès les premières années du XIII\textsuperscript{e} siècle, l’{\persName Aristote} arabe fait dans l’{\orgName Université de Paris} son entrée triomphante. L’{\orgName Occident} a secoué son infériorité de quatre ou cinq cents ans. Jusqu’ici l’{\orgName Europe} a été scientifiquement tributaire des musulmans. Vers le milieu du XIII\textsuperscript{e} siècle, la balance est incertaine encore. À partir de 1275 à peu près, deux mouvements apparaissent avec évidence : d’une part, les pays musulmans s’abîment dans la plus triste décadence intellectuelle ; de l’autre, l’{\orgName Europe occidentale} entre résolument pour son compte dans cette grande voie de la recherche scientifique de la vérité, courbe immense dont l’amplitude ne peut pas encore être mesurée.\par
Malheur à qui devient inutile au progrès humain ! Il est supprimé presque aussitôt. Quand la science dite arabe a inoculé son germe de vie à l’{\orgName Occident latin}, elle disparaît. Pendant qu’{\persName Averroès} arrive dans les écoles latines à une célébrité presque égale à celle d’{\persName Aristote}, il est oublié chez ses coreligionnaires. Passé l’an 1200 à peu près, il n’y a plus un seul philosophe arabe de renom. La philosophie avait toujours été persécutée au sein de l’islam, mais d’une façon qui n’avait pas réussi à la supprimer. À partir de 1200, la réaction théologique l’emporte tout à fait. La philosophie est abolie dans les pays musulmans. Les historiens et les polygraphes n’en parlent que comme d’un souvenir, et d’un mauvais souvenir. Les manuscrits philosophiques sont détruits et deviennent rares. L’astronomie n’est tolérée que pour la partie qui sert à déterminer la direction de la prière. Bientôt la race turque prendra l’hégémonie de l’islam, et fera prévaloir partout son manque total d’esprit philosophique et scientifique. À partir de ce moment, à quelques rares exceptions près comme {\persName Ibn-Khaldoun}, l’islam ne comptera plus aucun esprit large ; il a tué la science et la philosophie dans son sein. Je n’ai point cherché. Messieurs, à diminuer le rôle de cette grande science dite arabe qui marque une étape si importante dans l’histoire de l’esprit humain. On en a exagéré l’originalité sur quelques points, notamment en ce qui touche l’astronomie ; il ne faut pas verser dans l’autre excès en la dépréciant outre mesure. Entre la disparition de la civilisation antique, au VI\textsuperscript{e} siècle, et la naissance du génie européen au XI\textsuperscript{e} et au XII\textsuperscript{e}, il y a eu ce qu’on peut appeler la période arabe, durant laquelle la tradition de l’esprit humain s’est faite par les régions conquises à l’islam. Cette science dite arabe, qu’a-t-elle d’arabe en réalité ? La langue, rien que la langue. La conquête musulmane avait porté la langue de l’Hedjaz jusqu’au bout du monde. Il arriva pour l’arabe ce qui est arrivé pour le latin, lequel est devenu, en {\placeName Occident}, l’expression de sentiments et de pensées qui n’avaient rien à faire avec le vieux Latium. {\persName Averroès}, {\persName Avicenne}, {\persName Albaténi} sont des Arabes, comme {\persName Albert le Grand}, {\persName Roger Bacon}, {\persName François Bacon}, {\persName Spinoza} sont des Latins. Il y a un aussi grand malentendu à mettre la science et la philosophie arabes au compte de l’{\placeName Arabie} qu’à mettre toute la littérature chrétienne latine, tous les scolastiques, toute la Renaissance, toute la science du XVI\textsuperscript{e} et en partie du XVII\textsuperscript{e} siècle au compte de la ville de {\placeName Rome}, parce que tout cela est écrit en latin. Ce qu’il y a de bien remarquable, en effet, c’est que, parmi les philosophes et les savants dits arabes, il n’y en a guère qu’un seul, {\persName Alkindi}, qui soit d’origine arabe ; tous les autres sont des Persans, des Transoxiens, des Espagnols, des gens de {\placeName Bokhara}, de {\placeName Samarkande}, de {\placeName Cordoue}, de {\placeName Séville}. Non seulement, ce ne sont pas des Arabes de sang ; mais ils n’ont rien d’arabe d’esprit. Ils se servent de l’arabe ; mais ils en sont gênés, comme les penseurs du moyen âge sont gênés par le latin et le brisent à leur usage. L’arabe, qui se prête si bien à la poésie et à une certaine éloquence, est un instrument fort incommode pour la métaphysique. Les philosophes et les savants arabes sont en général d’assez mauvais écrivains.\par
Cette science n’est pas arabe. Est-elle du moins musulmane ? L’islamisme a-t-il offert à ces recherches rationnelles quelque secours tutélaire ? Oh ! en aucune façon ! Ce beau mouvement d’études est tout entier l’œuvre de parsis, de chrétiens, de juifs, de harraniens, d’ismaéliens, de musulmans intérieurement révoltés contre leur propre religion. Il n’a recueilli des musulmans orthodoxes que des malédictions. {\persName Mamoun}, celui des califes qui montra le plus de zèle pour l’introduction de la philosophie grecque, fut damné sans pitié par les théologiens ; les malheurs qui affligèrent son règne furent présentés comme des punitions de sa tolérance pour des doctrines étrangères à l’islam. Il n’était pas rare que, pour plaire à la multitude ameutée par les imams, on brûlât sur les places publiques, on jetât dans les puits et les citernes les livres de philosophie, d’astronomie. Ceux qui cultivaient ces études étaient appelés {\itshape zendiks} (mécréants) ; on les frappait dans les rues, on brûlait leurs maisons, et souvent le pouvoir, quand il voulait se donner de la popularité, les faisait mettre à mort.\par
L’islamisme, en réalité, a donc toujours persécuté la science et la philosophie. Il a fini par les étouffer. Seulement il faut distinguer à cet égard deux périodes dans l’histoire de l’islam ; l’une, depuis ses commencements jusqu’au XII\textsuperscript{e} siècle, l’autre, depuis le XIII\textsuperscript{e} siècle jusqu’à nos jours. Dans la première période, l’islam, miné par les sectes et tempéré par une espèce de protestantisme (ce qu’on appelle le motazélisme), est bien moins organisé et moins fanatique qu’il ne l’a été dans le second âge, quand il est tombé entre les mains des races tartares et berbères, races lourdes, brutales et sans esprit. L’islamisme offre cette particularité qu’il a obtenu de ses adeptes une foi toujours de plus en plus forte. Les premiers Arabes qui s’engagèrent dans le mouvement croyaient à peine en la mission du {\persName Prophète}. Pendant deux ou trois siècles, l’incrédulité est à peine dissimulée. Puis vient le règne absolu du dogme, sans aucune séparation possible du spirituel et du temporel ; le règne avec coercition et châtiments corporels pour celui qui ne pratique pas ; un système, enfin, qui n’a guère été dépassé, en fait de vexations, que par l’Inquisition espagnole. La liberté n’est jamais plus profondément blessée que par une organisation sociale où la religion domine absolument la vie civile. Dans les temps modernes, nous n’avons vu que deux exemples d’un tel régime : d’une part, les États musulmans ; de l’autre, l’ancien État pontifical du temps du pouvoir temporel. Et il faut dire que la papauté temporelle n’a pesé que sur un bien petit pays, tandis que l’islamisme opprime de vastes portions de notre globe et y maintient l’idée la plus opposée au progrès : l’État fondé sur une prétendue révélation, la théologie gouvernant la société.\par
Les libéraux qui défendent l’islam ne le connaissent pas. L’islam, c’est l’union indiscernable du spirituel et du temporel, c’est le règne d’un dogme, c’est la chaîne la plus lourde que l’humanité ait jamais portée. Dans la première moitié du moyen âge, je le répète, l’islam a supporté la philosophie, parce qu’il n’a pas pu l’empêcher ; il n’a pas pu l’empêcher, car il était sans cohésion, peu outillé pour la terreur. La police, comme je l’ai dit, était entre les mains de chrétiens et occupée principalement à poursuivre les tentatives des {\orgName Alides}. Une foule de choses passaient à travers les mailles de ce filet assez lâche. Mais, quand l’islam a disposé de masses ardemment croyantes, il a tout détruit. La terreur religieuse et l’hypocrisie ont été à l’ordre du jour. L’islam a été libéral quand il a été faible, et violent quand il a été fort. Ne lui faisons donc pas honneur de ce qu’il n’a pas pu supprimer. Faire honneur à l’islam de la philosophie et de la science qu’il n’a pas tout d’abord anéanties, c’est comme si l’on faisait honneur aux théologiens des découvertes de la science moderne. Ces découvertes se sont faites malgré les théologiens. La théologie occidentale n’a pas été moins persécutrice que celle de l’islamisme. Seulement elle n’a pas réussi, elle n’a pas écrasé l’esprit moderne, comme l’islamisme a écrasé l’esprit des pays qu’il a conquis. Dans notre {\placeName Occident}, la persécution théologique n’a réussi qu’en un seul pays : c’est en {\placeName Espagne}. Là, un terrible système d’oppression a étouffé l’esprit scientifique. Hâtons-nous de le dire, ce noble pays prendra sa revanche. Dans les pays musulmans, il s’est passé ce qui serait arrivé en {\placeName Europe} si l’Inquisition, {\persName Philippe II} et {\persName Pie V}, avaient réussi dans leur plan d’arrêter l’esprit humain. Franchement, j’ai beaucoup de peine à savoir gré aux gens du mal qu’ils n’ont pas pu faire. Non ; les religions ont leurs grandes et belles heures, quand elles consolent et relèvent les parties faibles de notre pauvre humanité ; mais il ne faut pas leur faire compliment de ce qui est né malgré elles, de ce qu’elles ont cherché à suffoquer au berceau. On n’hérite pas des gens qu’on assassine ; on ne doit point faire bénéficier les persécuteurs des choses qu’ils ont persécutées.\par
C’est pourtant là l’erreur que l’on commet, par excès de générosité, quand on attribue à l’influence de l’islam un mouvement qui s’est produit malgré l’islam, contre l’islam, et que l’islam, heureusement, n’a pas pu empêcher. Faire honneur à l’islam d’{\persName Avicenne}, d’{\persName Avenzoar}, d’{\persName Averroès}, c’est comme si l’on faisait honneur au catholicisme de {\persName Galilée}. La théologie a gêné {\persName Galilée} ; elle n’a pas été assez forte pour l’entraver tout à fait ; ce n’est pas une raison pour qu’il faille lui avoir une grande reconnaissance. Loin de moi des paroles d’amertume contre aucun des symboles dans lesquels la conscience humaine a cherché le repos au milieu des insolubles problèmes que lui présentent l’univers et sa destinée ! L’islamisme a de belles parties comme religion ; je ne suis jamais entré dans une mosquée sans une vive émotion, le dirai-je ? sans un certain regret de n’être pas musulman. Mais, pour la raison humaine, l’islamisme n’a été que nuisible. Les esprits qu’il a fermés à la lumière y étaient déjà sans doute fermés par leurs propres bornes intérieures ; mais il a persécuté la libre pensée, je ne dirai pas plus violemment que d’autres systèmes religieux, mais plus efficacement. Il a fait des pays qu’il a conquis un champ fermé à la culture rationnelle de l’esprit.\par
Ce qui distingue, en effet, essentiellement le musulman, c’est la haine de la science, c’est la persuasion que la recherche est inutile, frivole, presque impie : la science de la nature, parce qu’elle est une concurrence faite à {\persName Dieu} ; la science historique, parce que, s’appliquant à des temps antérieurs à l’islam, elle pourrait raviver d’anciennes erreurs. Un des témoignages les plus curieux à cet égard est celui du {\persName cheik Rifaa}, qui avait résidé plusieurs années à {\placeName Paris} comme aumônier de l’{\orgName École égyptienne}, et qui, après son retour en {\placeName Égypte}, fit un ouvrage plein des observations les plus curieuses sur la société française. Son idée fixe est que la science européenne, surtout par son principe de la permanence des lois de la nature, est d’un bout à l’autre une hérésie ; et, il faut le dire, au point de vue de l’islam, il n’a pas tout à fait tort. Un dogme révélé est toujours opposé à la recherche libre, qui peut le contredire. Le résultat de la science est non pas d’expulser, mais d’éloigner toujours le divin, de l’éloigner, dis-je, du monde des faits particuliers où l’on croyait le voir. L’expérience fait reculer le surnaturel et restreint son domaine. Or le surnaturel est la base de toute théologie. L’islam, en traitant la science comme son ennemie, n’est que conséquent ; mais il est dangereux d’être trop conséquent. L’islam a réussi pour son malheur. En tuant la science, il s’est tué lui-même, et s’est condamné dans le monde à une complète infériorité.\par
Quand on part de cette idée que la recherche est une chose attentatoire aux droits de {\persName Dieu}, on arrive inévitablement à la paresse d’esprit, au manque de précision, à l’incapacité d’être exact. {\itshape Allah aalam}, « Dieu sait mieux ce qui en est », est le dernier mot de toute discussion musulmane. Il est bon de croire en {\persName Dieu}, mais pas tant que cela. Dans les premiers temps de son séjour à {\placeName Mossoul}, {\persName M. Layard} désira, en esprit clair qu’il était, avoir quelques données sur la population de la ville, sur son commerce, ses traditions historiques. Il s’adressa au’ cadi, qui lui fit la réponse suivante, dont je dois la traduction à une communication affectueuse :\par

\begin{quoteblock}
 \noindent Ô mon illustre ami, ô joie des vivants !\par
 Ce que tu me demandes est à la fois inutile et nuisible. Bien que tous mes jours se soient écoulés dans ce pays, je n’ai jamais songé à en compter les maisons, ni à m’informer du nombre de leurs habitants. Et, quant à ce que celui-ci met de marchandises sur ses mulets, celui-là au fond de sa barque, en vérité, c’est là une chose qui ne me regarde nullement. Pour l’histoire antérieure de cette cité, {\persName Dieu} seul la sait, et seul il pourrait dire de combien d’erreurs ses habitants se sont abreuvés avant la conquête de l’islamisme. Il serait dangereux à nous de vouloir les connaître.\par
 Ô mon ami, ô ma brebis, ne cherche pas à connaître ce qui ne te concerne pas. Tu es venu parmi nous et nous t’avons donné le salut de bienvenue ; va-t’en en paix ! A la vérité, toutes les paroles que tu m’as dites ne m’ont fait aucun mal ; car celui qui parle est un, et celui qui écoule est un autre. Selon la coutume des hommes de ta nation, tu as parcouru beaucoup de contrées jusqu’à ce que tu n’aies plus trouvé le bonheur nulle part. Nous ({\persName Dieu} en soit béni !), nous sommes nés ici, et nous ne désirons point en partir.\par
 Écoute, ô mon fils, il n’y a point de sagesse égale à celle de croire en Dieu. Il a créé le monde ; devons-nous tenter de l’égaler en cherchant à pénétrer les mystères de sa création ? Vois cette étoile qui tourne là-haut autour de cette étoile ; regarde cette autre étoile qui traîne une queue et qui met tant d’années à venir et tant d’années à s’éloigner ; laisse-la, mon fils ; celui dont les mains la formèrent saura bien la conduire et la diriger.\par
 Mais tu me diras peut-être : « Ô homme ! retire-toi, car je suis plus savant que toi, et j’ai vu des choses que tu ignores ! » Si tu penses que ces choses t’ont rendu meilleur que je ne le suis, sois doublement le bienvenu ; mais, moi, je bénis {\persName Dieu} de ne pas chercher ce dont je n’ai pas besoin. Tu es instruit dans des choses qui ne m’intéressent pas, et ce que tu as vu, je le dédaigne. Une science plus vaste te créera-t-elle un second estomac, et tes yeux, qui vont furetant partout, te feront-ils trouver un paradis ?\par
 Ô mon ami, si tu veux être heureux, écrie-toi : « {\persName Dieu} seul est {\persName Dieu} ! » Ne fais point de mal, et alors tu ne craindras ni les hommes ni la mort, car ton heure viendra.
 \end{quoteblock}

\noindent Ce cadi est très philosophe à sa manière ; mais voici la différence. Nous trouvons charmante la lettre du cadi, et lui, il trouverait ce que nous disons ici abominable. C’est pour une société, d’ailleurs, que les suites d’un pareil esprit sont funestes. Des deux conséquences qu’entraîne le manque d’esprit scientifique, la superstition ou le dogmatisme, la seconde est peut-être pire que la première. L’{\orgName Orient} n’est pas superstitieux ; son grand mal, c’est le dogmatisme étroit, qui s’impose par la force de la société tout entière. Le but de l’humanité, ce n’est pas le repos dans une ignorance résignée ; c’est la guerre implacable contre le faux, la lutte contre le mal. . La science est l’âme d’une société ; car la science, c’est la raison. Elle crée la supériorité militaire et la supériorité industrielle. Elle créera un jour la supériorité sociale, je veux dire un état de société où la quantité de justice qui est compatible avec l’essence de l’univers sera procurée. La science met la force au service de la raison. Il y a en {\placeName Asie} des éléments de barbarie analogues à ceux qui ont formé les premières armées musulmanes et ces grands cyclones d’{\persName Attila}, de {\persName Gengiskhan}. Mais la science leur barre le chemin. Si {\persName Omar}, si {\persName Gengiskhan} avaient rencontré devant eux une bonne artillerie, ils n’eussent pas dépassé les limites de leur désert. Il ne faut pas s’arrêter à des aberrations momentanées. Que n’a-t-on pas dit, à l’origine, contre les armes à feu, lesquelles pourtant ont bien contribué à la victoire de la civilisation ? Pour moi, j’ai la conviction que la science est bonne, qu’elle seule fournit des armes contre le mal qu’on peut faire avec elle, qu’en définitive elle ne servira que le progrès, j’entends le vrai progrès, celui qui est inséparable du respect de l’homme et de la liberté.
\section[{Appendice à la précédente conférence}]{Appendice à la précédente conférence}\renewcommand{\leftmark}{Appendice à la précédente conférence}

\noindent {\itshape Un cheik afghan, remarquablement intelligent, de passage à {\placeName Paris}, ayant présenté dans le numéro du 18 mai 1883 du \emph{Journal des Débats} des observations sur la conférence précédente, j’y répondis le lendemain, dans le même journal, ainsi qu’il suit.}\par
On a lu hier avec l’intérêt qu’elles méritent les très judicieuses réflexions que ma dernière conférence à la Sorbonne a suggérées au {\persName cheik Gemmal-Eddin}. Rien de plus instructif que d’étudier ainsi, dans ses manifestations originales et sincères, la conscience de l’Asiatique éclairé. C’est en écoutant les voix les plus diverses, venant des quatre coins de l’horizon en faveur du rationalisme, qu’on arrive à se convaincre que, si les religions divisent les hommes, la raison les rapproche, et qu’au fond, il n’y a qu’une seule raison. L’unité de l’esprit humain est le grand et consolant résultat qui sort du choc pacifique des idées, quand on met de côté les prétentions opposées des révélations dites surnaturelles. La ligue des bons esprits de la terre entière contre le fanatisme et la superstition est en apparence le fait d’une imperceptible minorité ; au fond, c’est la seule ligue durable, car elle repose sur la vérité, et elle finira par l’emporter, après que les fables rivales se seront épuisées en des séries séculaires d’impuissantes convulsions.\par
Il y a deux mois à peu près, je fis la connaissance du {\persName cheik Gemmal-Eddin}, grâce à notre cher collaborateur {\persName M. Ganem}. Peu de personnes ont produit sur moi une plus vive impression. C’est en grande partie la conversation que j’eus avec lui qui me décida à choisir pour sujet de ma conférence à la {\placeName Sorbonne} les rapports de l’esprit scientifique et de l’islamisme. Le {\persName cheik Gemmal-Eddin} est un Afghan entièrement dégagé des préjugés de l’islam ; il appartient à ces races énergiques du {\placeName haut Iran}, voisin de l’{\placeName Inde}, où l’esprit aryen vit encore si énergique sous la couche superficielle de l’islamisme officiel. Il est la meilleure preuve de ce grand axiome que nous avons souvent proclamé, savoir que les religions valent ce que valent les races qui les professent. La liberté de sa pensée, son noble et loyal caractère me faisaient croire, pendant que je m’entretenais avec lui, que j’avais devant moi, à l’état de ressuscité, quelqu’une de mes anciennes connaissances, {\persName Avicenne}, {\persName Averroès}, ou tel autre de ces grands infidèles qui ont représenté pendant cinq siècles la tradition de l’esprit humain. Le contraste était surtout sensible pour moi quand je comparais cette frappante apparition au spectacle que présentent les pays musulmans en-deçà de la {\placeName Perse}, pays où la curiosité scientifique et philosophique est si rare. Le {\persName cheik Gemmal-Eddin} est le plus beau cas de protestation ethnique contre la conquête religieuse, que l’on puisse citer. Il confirme ce que les orientalistes intelligents de l’Europe ont souvent dit : c’est que l’{\orgName Afghanistan} est de toute l’{\placeName Asie}, le {\placeName Japon} excepté, le pays qui présente le plus d’éléments constitutifs de ce que nous appelons une nation.\par
Je ne vois guère dans le savant écrit du cheik qu’un point sur lequel nous soyons réellement en désaccord. Le cheik n’admet pas les distinctions que la critique historique nous conduit à faire dans ces grands faits complexes qui s’appellent empires et conquêtes. L’empire romain, avec lequel la conquête arabe a tant de rapports, a fait de la langue latine l’organe de l’esprit humain dans tout l’{\placeName Occident}, jusqu’au XVI\textsuperscript{e} siècle. {\persName Albert le Grand}, {\persName Roger Bacon}, {\persName Spinoza} ont écrit en latin. Ce ne sont pas néanmoins pour nous des Latins. Dans une histoire de la littérature anglaise, on donne une place à {\persName Bède} et {\persName Alcuin} ; dans une histoire de la littérature française, nous mettons Grégoire de Tours et Abélard. Ce n’est pas certes que nous méconnaissions l’action de {\orgName Rome} dans l’histoire de la civilisation, pas plus que nous ne méconnaissons l’action arabe. Mais ces grands courants humanitaires demandent à être analysés. Tout ce qui s’est écrit en latin n’est pas la gloire de {\orgName Rome} ; tout ce qui s’est écrit en grec n’est pas œuvre hellénique ; tout ce qui s’est écrit en arabe n’est pas un produit arabe ; tout ce qui s’est fait en pays chrétien n’est pas l’effet du christianisme ; tout ce qui s’est fait en pays musulman n’est pas un fruit de l’islam. C’est le principe que le profond historien de l’{\orgName Espagne musulmane}, {\persName M. Reinhard Dozy}, dont l’{\orgName Europe} savante déplore en ce moment la perte, appliquait avec une rare sagacité. Ces sortes de distinctions sont nécessaires, si l’on ne veut pas que l’histoire soit un tissu d’à peu près et de malentendus.\par
Un côté par lequel j’ai pu paraître injuste au cheik, c’est que je n’ai pas assez développé cette idée que toute religion révélée est amenée à se montrer hostile à la science positive, et que le christianisme n’a sous ce rapport rien à envier à l’islam. Cela est hors de doute. {\persName Galilée} n’a pas été mieux traité par le catholicisme qu’{\persName Averroès} n’a été traité par l’islamisme. {\persName Galilée} a trouvé la vérité en pays catholique, malgré le catholicisme, comme {\persName Averroès} a philosophé noblement en pays musulman, malgré l’islam. Si je n’ai pas insisté davantage sur ce point, c’est que, à vrai dire, mes opinions à cet égard sont assez connues pour que je n’eusse pas à y revenir devant un public au courant de mes travaux. J’ai dit assez souvent, pour que je n’aie pas à le répéter à tout propos, que l’esprit humain doit être dégagé de toute croyance surnaturelle, s’il veut travailler à son œuvre essentielle, qui est la construction de la science positive. Cela n’implique pas de destruction violente, ni de rupture brusque. Il ne s’agit pas pour le chrétien d’abandonner le christianisme, ni pour le musulman d’abandonner l’islam. Il s’agit pour les parties éclairées du christianisme et de l’islam, d’arriver à cet état d’indifférence bienveillante où les croyances religieuses deviennent inoffensives. Cela est fait dans une moitié à peu près des pays chrétiens ; espérons que cela se fera pour l’islam. Naturellement, ce jour-là, le cheik et moi nous serons d’accord pour applaudir des deux mains.\par
Je n’ai pas dit que tous les musulmans, sans distinction de race, sont et seront toujours des ignorants ; j’ai dit que l’islamisme crée de grandes difficultés à la science, et malheureusement, a réussi, depuis cinq ou six cents ans, à la supprimer presque dans les pays qu’il détient ; ce qui est pour ces pays une cause d’extrême faiblesse. Je crois, en effets que la régénération des pays musulmans ne se fera pas par l’islam : elle se fera par l’affaiblissement de l’islam, comme du reste le grand élan des pays dits chrétiens a commencé par la destruction de l’Église tyrannique du moyen âge. Quelques personnes ont vu, dans ma conférence, une pensée malveillante contre les individus professant la religion musulmane. Il n’en est rien ; les musulmans sont les premières victimes de l’islam. Plusieurs fois, j’ai pu observer, dans mes voyages en {\placeName Orient}, que le fanatisme vient d’un petit nombre d’hommes dangereux qui maintiennent les autres dans la pratique religieuse par la terreur. Émanciper le musulman de sa religion est le meilleur service qu’on puisse lui rendre. En souhaitant à ces populations, chez lesquelles il y a tant de bons éléments, la délivrance du joug qui pèse sur elles, je ne crois pas leur faire un mauvais souhait. Et, puisque le {\persName cheik Gemmal-Eddin} veut que je tienne la balance égale entre les cultes divers, je ne croirais pas non plus faire un mauvais souhait à certains pays européens en désirant que le christianisme ait chez eux un caractère moins dominateur.\par
Le désaccord entre les libéraux sur ces différents points n’est pas très profond, puisque, favorables ou non à l’islam, tous arrivent à la même conclusion pratique : répandre l’instruction chez les musulmans. Voilà qui est parfait, pourvu qu’il s’agisse de l’instruction sérieuse, de celle qui cultive la raison. Que les chefs religieux de l’islamisme contribuent à cette œuvre excellente, j’en serai ravi. Pour parler franchement, je doute un peu qu’ils le fassent. Il se formera des individualités distinguées (il y en aura peu d’aussi distinguées que le {\persName cheik Gemmal-Eddin}) qui se sépareront de l’islam, comme nous nous séparons du catholicisme. Certains pays, avec le temps, rompront à peu près avec la religion du Coran ; mais je doute que le mouvement de renaissance se fasse avec l’appui de l’islam officiel. La renaissance scientifique de l’{\orgName Europe} ne s’est pas faite non plus avec le catholicisme, et, à l’heure qu’il est, sans qu’il faille beaucoup s’en étonner, le catholicisme lutte encore pour empêcher la pleine réalisation de ce qui résume le code rationnel de l’humanité, l’État neutre, en dehors des dogmes censés révélés.\par
Au-dessus de tout, comme règle suprême, mettons la liberté et le respect des hommes. Ne pas détruire les religions, les traiter même avec bienveillance, comme des manifestations libres de la nature humaine, mais ne pas les garantir, surtout ne pas les défendre contre leurs propres fidèles qui tendent à se séparer d’elles, voilà le devoir de la société civile. Réduites ainsi à la condition de choses libres et individuelles, comme la littérature, le goût, les religions se transformeront entièrement. Privées du lien officiel ou concordataire, elles se désagrégeront, et perdront la plus grande partie de leurs inconvénients. Tout cela est utopie à l’heure présente ; tout cela sera réalité dans l’avenir. Comment chaque religion se comportera-t-elle avec le régime de la liberté, qui s’imposera, après bien des actions et réactions, aux sociétés humaines ? Ce n’est pas en quelques lignes qu’on peut examiner un pareil problème. Dans ma conférence, j’ai voulu seulement traiter une question historique. Le {\persName cheik Gemmal-Eddin} me paraît avoir apporté des arguments considérables à mes deux thèses fondamentales : — Pendant la première moitié de son existence, l’islamisme n’empêcha pas le mouvement scientifique de se produire en terre musulmane ; — pendant la seconde moitié de son existence, il étouffa dans son sein le mouvement scientifique, et cela pour son malheur !
 


% at least one empty page at end (for booklet couv)
\ifbooklet
  \pagestyle{empty}
  \clearpage
  % 2 empty pages maybe needed for 4e cover
  \ifnum\modulo{\value{page}}{4}=0 \hbox{}\newpage\hbox{}\newpage\fi
  \ifnum\modulo{\value{page}}{4}=1 \hbox{}\newpage\hbox{}\newpage\fi


  \hbox{}\newpage
  \ifodd\value{page}\hbox{}\newpage\fi
  {\centering\color{rubric}\bfseries\noindent\large
    Hurlus ? Qu’est-ce.\par
    \bigskip
  }
  \noindent Des bouquinistes électroniques, pour du texte libre à participation libre,
  téléchargeable gratuitement sur \href{https://hurlus.fr}{\dotuline{hurlus.fr}}.\par
  \bigskip
  \noindent Cette brochure a été produite par des éditeurs bénévoles.
  Elle n’est pas faîte pour être possédée, mais pour être lue, et puis donnée.
  Que circule le texte !
  En page de garde, on peut ajouter une date, un lieu, un nom ; pour suivre le voyage des idées.
  \par

  Ce texte a été choisi parce qu’une personne l’a aimé,
  ou haï, elle a en tous cas pensé qu’il partipait à la formation de notre présent ;
  sans le souci de plaire, vendre, ou militer pour une cause.
  \par

  L’édition électronique est soigneuse, tant sur la technique
  que sur l’établissement du texte ; mais sans aucune prétention scolaire, au contraire.
  Le but est de s’adresser à tous, sans distinction de science ou de diplôme.
  Au plus direct ! (possible)
  \par

  Cet exemplaire en papier a été tiré sur une imprimante personnelle
   ou une photocopieuse. Tout le monde peut le faire.
  Il suffit de
  télécharger un fichier sur \href{https://hurlus.fr}{\dotuline{hurlus.fr}},
  d’imprimer, et agrafer ; puis de lire et donner.\par

  \bigskip

  \noindent PS : Les hurlus furent aussi des rebelles protestants qui cassaient les statues dans les églises catholiques. En 1566 démarra la révolte des gueux dans le pays de Lille. L’insurrection enflamma la région jusqu’à Anvers où les gueux de mer bloquèrent les bateaux espagnols.
  Ce fut une rare guerre de libération dont naquit un pays toujours libre : les Pays-Bas.
  En plat pays francophone, par contre, restèrent des bandes de huguenots, les hurlus, progressivement réprimés par la très catholique Espagne.
  Cette mémoire d’une défaite est éteinte, rallumons-la. Sortons les livres du culte universitaire, cherchons les idoles de l’époque, pour les briser.
\fi

\ifdev % autotext in dev mode
\fontname\font — \textsc{Les règles du jeu}\par
(\hyperref[utopie]{\underline{Lien}})\par
\noindent \initialiv{A}{lors là}\blindtext\par
\noindent \initialiv{À}{ la bonheur des dames}\blindtext\par
\noindent \initialiv{É}{tonnez-le}\blindtext\par
\noindent \initialiv{Q}{ualitativement}\blindtext\par
\noindent \initialiv{V}{aloriser}\blindtext\par
\Blindtext
\phantomsection
\label{utopie}
\Blinddocument
\fi
\end{document}
