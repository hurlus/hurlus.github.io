%%%%%%%%%%%%%%%%%%%%%%%%%%%%%%%%%
% LaTeX model https://hurlus.fr %
%%%%%%%%%%%%%%%%%%%%%%%%%%%%%%%%%

% Needed before document class
\RequirePackage{pdftexcmds} % needed for tests expressions
\RequirePackage{fix-cm} % correct units

% Define mode
\def\mode{a4}

\newif\ifaiv % a4
\newif\ifav % a5
\newif\ifbooklet % booklet
\newif\ifcover % cover for booklet

\ifnum \strcmp{\mode}{cover}=0
  \covertrue
\else\ifnum \strcmp{\mode}{booklet}=0
  \booklettrue
\else\ifnum \strcmp{\mode}{a5}=0
  \avtrue
\else
  \aivtrue
\fi\fi\fi

\ifbooklet % do not enclose with {}
  \documentclass[french,twoside]{book} % ,notitlepage
  \usepackage[%
    papersize={105mm, 297mm},
    inner=12mm,
    outer=12mm,
    top=20mm,
    bottom=15mm,
    marginparsep=3pt,
    marginpar=7mm,
  ]{geometry}
  \usepackage[fontsize=9.5pt]{scrextend} % for Roboto
\else\ifav % A5
  \documentclass[french,twoside]{book} % ,notitlepage
  \usepackage[%
    a5paper
  ]{geometry}
  \usepackage[fontsize=12pt]{scrextend}
\else% A4 2 cols
  \documentclass[twocolumn]{report}
  \usepackage[%
    a4paper,
    inner=15mm,
    outer=10mm,
    top=25mm,
    bottom=18mm,
    marginparsep=0pt,
  ]{geometry}
  \setlength{\columnsep}{20mm}
  \usepackage[fontsize=9.5pt]{scrextend}
\fi\fi

%%%%%%%%%%%%%%
% Alignments %
%%%%%%%%%%%%%%
% before teinte macros

\setlength{\arrayrulewidth}{0.2pt}
\setlength{\columnseprule}{\arrayrulewidth} % twocol
\setlength{\parskip}{0pt} % 1pt allow better vertical justification
\setlength{\parindent}{1.5em}

%%%%%%%%%%
% Colors %
%%%%%%%%%%
% before Teinte macros

\usepackage[dvipsnames]{xcolor}
\definecolor{rubric}{HTML}{800000} % the tonic 0c71c3
\def\columnseprulecolor{\color{rubric}}
\colorlet{borderline}{rubric!30!} % definecolor need exact code
\definecolor{shadecolor}{gray}{0.95}
\definecolor{bghi}{gray}{0.5}

%%%%%%%%%%%%%%%%%
% Teinte macros %
%%%%%%%%%%%%%%%%%
%%%%%%%%%%%%%%%%%%%%%%%%%%%%%%%%%%%%%%%%%%%%%%%%%%%
% <TEI> generic (LaTeX names generated by Teinte) %
%%%%%%%%%%%%%%%%%%%%%%%%%%%%%%%%%%%%%%%%%%%%%%%%%%%
% This template is inserted in a specific design
% It is XeLaTeX and otf fonts

\makeatletter % <@@@

\usepackage{alphalph} % for alph couter z, aa, ab…
\usepackage{blindtext} % generate text for testing
\usepackage{booktabs} % for tables: \toprule, \midrule…
\usepackage[strict]{changepage} % for modulo 4
\usepackage{contour} % rounding words
\usepackage[nodayofweek]{datetime}
\usepackage{enumitem} % <list>
\usepackage{etoolbox} % patch commands
\usepackage{fancyvrb}
\usepackage{fancyhdr}
\usepackage{float}
\usepackage{fontspec} % XeLaTeX mandatory for fonts
\usepackage{footnote} % used to capture notes in minipage (ex: quote)
\usepackage{framed} % bordering correct with footnote hack
\usepackage{graphicx}
\usepackage{lettrine} % drop caps
\usepackage{lipsum} % generate text for testing
\usepackage{manyfoot} % for parallel footnote numerotation
\usepackage[framemethod=tikz,]{mdframed} % maybe used for frame with footnotes inside
\usepackage[defaultlines=2,all]{nowidow} % at least 2 lines by par (works well!)
\usepackage{pdftexcmds} % needed for tests expressions
\usepackage{poetry} % <l>, bad for theater
\usepackage{polyglossia} % bug Warning: "Failed to patch part"
\usepackage[%
  indentfirst=false,
  vskip=1em,
  noorphanfirst=true,
  noorphanafter=true,
  leftmargin=\parindent,
  rightmargin=0pt,
]{quoting}
\usepackage{ragged2e}
\usepackage{setspace} % \setstretch for <quote>
\usepackage{scrextend} % KOMA-common, used for addmargin
\usepackage{tabularx} % <table>
\usepackage[explicit]{titlesec} % wear titles, !NO implicit
\usepackage{tikz} % ornaments
\usepackage{tocloft} % styling tocs
\usepackage[fit]{truncate} % used im runing titles
\usepackage{unicode-math}
\usepackage[normalem]{ulem} % breakable \uline, normalem is absolutely necessary to keep \emph
\usepackage{xcolor} % named colors
\usepackage{xparse} % @ifundefined
\XeTeXdefaultencoding "iso-8859-1" % bad encoding of xstring
\usepackage{xstring} % string tests
\XeTeXdefaultencoding "utf-8"

\defaultfontfeatures{
  % Mapping=tex-text, % no effect seen
  Scale=MatchLowercase,
  Ligatures={TeX,Common},
}
\newfontfamily\zhfont{Noto Sans CJK SC}

% Metadata inserted by a program, from the TEI source, for title page and runing heads
\title{\textbf{ L’Ecclésiaste, c’est-à-dire : le Prêcheur. }\par
\medskip
\textit{ traduction de 1555par Sébastien Castellion \emph{dîte} « la Bible pour les idiots » }\par
}
\date{-250}
\author{Qohélet}
\def\elbibl{Qohélet. -250. \emph{L’Ecclésiaste, c’est-à-dire : le Prêcheur.}}
\def\elabstract{%
 
\labelblock{« je n’entends pas ces quatre »\footnote{Sébastien Castellion, en note sur \hyperref[Eccl.12.6]{\dotuline{\emph{Eccl}.12.6}}}}

 \noindent L’\emph{Ecclésiaste} (ou \emph{Qohélet} selon l’hébreu : « qui parle à la foule »), est un livre très lu de \emph{la Bible}, bien qu’il puisse sembler nihiliste et épicurien. Il a été écrit après −250, en période hellénistique où l’orient baignait dans la culture grecque, donc en contact avec les philosophies stoïciennes ou épicuriennes. Les morales juives ou chrétiennes y trouvent une inspiration, c’est une référence à connaître, il est par exemple pastiché par un athée, le communiste Paul Lafargue, dans \href{https://hurlus.fr/lafargue1887\_religion-capital/}{\dotuline{La religion du capital}}\footnote{\href{https://hurlus.fr/lafargue1887\_religion-capital/}{\url{https://hurlus.fr/lafargue1887\_religion-capital/}}} (1887). Le texte est donné ici dans une traduction méconnue de 1555, dite « la Bible des idiots », par Sébastien Castellion (1515–1563), un humaniste et réformé de la Renaissance, compagnon un moment de Calvin, puis persécuté par lui.\par
 
\asterism

 
\begin{quoteblock}
 \noindent Conclusion, quand tout est dit, \\
crains Dieu, et garde ses commandements : \\
car c’est le devoir de tous hommes.\par
 Car de toute œuvre, tant soit secrète, Dieu en fera rendre compte, \\
soit bonne soit mauvaise. (\hyperref[Eccl.12.13]{\dotuline{\emph{Eccl}.12.13-14}})
 \end{quoteblock}

 
\labelblock{Pessimisme}

 \noindent Ce texte propose une philosophie pessimiste, répétant selon la traduction traditionnelle : \emph{tout est vain, tout n’est que vanité} (Castellion utilise d’autres mots). Le plaisir aussi est donc vain, puisqu’il faut mourir. Épicure (–341…–270), à la recherche du bonheur et donc d’un plaisir raisonnable, considérait qu’il s’obtient en se libérant de la crainte des dieux. Si les dieux sont immortels et plus parfaits que nous, ils sont bons, pourquoi les craindre ?\par
 Le grec est confirmé par Qohélet, par contradiction. Le juif dit par exemple qu’il vaut mieux un enfant mort-né, qui n’a rien à regretter, qu’un homme chargé d’ans, d’enfants, et de biens, car il finira pareil et la mort lui sera plus dure (\hyperref[Eccl.6.3]{\dotuline{6.3-6}}). Par anticipation, Épicure lui répond dans la \emph{lettre à Ménécée} : \emph{« Bien pire encore celui qui dit qu’il est beau de “n’être pas né”. Car, s’il est convaincu de ce qu’il dit, comment se fait-il qu’il ne quitte pas la vie ? Cela est tout à fait en son pouvoir, s’il y est fermement décidé. »\footnote{Épicure, \emph{Lettre à Ménécée}, traduction Marcel Conche, 1977}}\par
 D’abord, on a le droit d’être lâche et incohérent ; ensuite, on peut aimer se plaindre ou désespérer les autres, c’est pour certains une raison de vivre, quand bien même elle soit peu enviable ; et enfin, si l’on craint Dieu, tuer soi ou d’autres, c’est pêcher contre la Vie. Ainsi même pour des chrétiens, à qui Dieu est amour mais pas à craindre, il est un soutien dans l’adversité pour aller jusqu’au bout de la tâche que l’on se croit porter. Se libérer des dieux autorise le plaisir, mais en fait perdre les consolations et secours moraux.\par
 Qohélet pose comme vérité première qu’un dieu unique existe, et que sa perfection nous juge, comme une personne supérieure peut nous faire honte. Si cette proposition est prise au sérieux, pas seulement sur son lit de mort comme les catholiques, alors même le \emph{« rire ce n’est autre chose qu’être hors du sens (\hyperref[Eccl.2.2]{\dotuline{2.2}}) »}, et il faudra payer pour nos errements ici-bas.\par
 
\begin{quoteblock}
 \noindent Jouis de ta jeunesse, jouvenceau, \\
et te donne de bon temps tandis que tu es jeune, \\
et mène un tel train que requiert le souhait de ton cœur, \\
ou le regard de tes yeux : \\
mais sache que de tout cela Dieu t’en fera rendre compte. (\hyperref[Eccl.11.9]{\dotuline{\emph{Eccl}.11.9)}}
 \end{quoteblock}

 En nous rappelant à nos devoirs, Dieu rend la vie morne.\par
 
\begin{quoteblock}
 Mieux vaut ouïr tancer un sage, \\
que chanter un fol. (\hyperref[Eccl.7.5]{\dotuline{\emph{Eccl}.7.5}})
 \end{quoteblock}

 
\labelblock{Épicurisme}

 \noindent Il arrive cependant ailleurs que Qohélet ne soit pas aussi négatif sur les plaisirs de la vie. Il se peut que l’on cherche ici une cohérence qui n’est pas aussi clairement articulée dans le texte, Qohélet se dit lui-même \emph{éplucheur} et compilateur de pensées des sages (\hyperref[Eccl.12.9]{\dotuline{12.9-11}}), la composition ne semble pas suivre un développement. Mais parce que cette cohérence a été cherchée pendant des siècles, elle est maintenant dans le texte. Quand bien même Qohélet ne serait qu’une chimère de compilateur, désormais il existe comme une pensée personnelle. Ainsi, les incohérences apparentes du texte nous invitent à réfléchir.\par
 
\begin{quoteblock}
 \noindent Il n’y a autre bien en l’homme que de manger et boire, \\
et se donner du bon temps en son travail, \\
laquelle chose je vois bien qu’elle vient aussi de Dieu (\hyperref[Eccl.2.24]{\dotuline{\emph{Eccl}.2,24)}}
 \end{quoteblock}

 \noindent Qohélet n’interdit pas de boire le vin, il ne promeut pas une prohibition puritaine dans l’espoir de nous faire devenir des anges, contrairement à Calvin, ou aux intégristes musulmans qui reconnaissent pourtant \emph{la Bible} comme leur livre. Le sage juif insiste sur une nuance qui vaut bien \emph{l’Avare} de Molière (\emph{« il faut manger pour vivre et non vivre pour manger »}) : \emph{« manger à l’heure qu’il faut, pour reprendre ses forces, et non pour boire (\hyperref[Eccl.10.17]{\dotuline{10.17}}) »}. Le prêcheur a une morale du plaisir mérité. Boire au matin, et manger pour boire plus, n’est pas un commandement de Dieu, et d’ailleurs nuit à la santé ; mais c’est une récompense justifiée après le travail. Ainsi le riche qui ne travaille pas mange trop et dort mal (\hyperref[Eccl.5.1]{\dotuline{5.1}}). Si Dieu a tout créé, pourquoi aurait-il fait le vin si ce n’est pas pour nous faire plaisir ? Le mal n’est pas dans la chose mais son usage.\par
 \emph{« D’avantage si deux couchent ensemble, ils s’échauffent : mais un comment s’échauffera-t-il ? (\hyperref[Eccl.4.11]{\dotuline{4.11}}) »}, \emph{« Passe le temps avec ta bien aimée (\hyperref[Eccl.9.9]{\dotuline{9.9}}) »} conseille même Qohélet. Sa société était alors conjugale et monogame. Le plaisir sexuel entre époux librement choisis plaît à son Dieu. \par
 Enfin \emph{« se donner du bon temps en son travail »}, car \emph{« Dieu prend plaisir en tes œuvres (\hyperref[Eccl.9.7]{\dotuline{9.7}}) »}, c’est une morale de travailleur\footnote{Mais quand Qohélet se prend pour Salomon fils de David, il se donne plus d’esclaves que tout le monde à Jérusalem \hyperref[Eccl.2.7]{\dotuline{\emph{Eccl}.2.7}}, mais cela semble une exagération fictive.}, un Grec n’y aurait pas pensé.\par
 Épicure avait des esclaves et ne propose sa morale du bonheur que pour des oisifs qui cherchent un \emph{régime}, une règle, un équilibre propice à la santé du corps et au calme de l’âme : \emph{« du pain d’orge et de l’eau donnent le plaisir extrême, lorsqu’on les porte à sa bouche dans le besoin »\footnote{Épicure, \emph{Lettre à Ménécée}, traduction Marcel Conche, 1977}}. Il conseillera la diète aux riches, mais il ne trouve pas cette joie plus grande du plaisir après le travail.\par
 
\labelblock{Fortune et Providence}

 
\begin{quoteblock}
 \noindent Qui prend garde au vent ne sème point : \\
et qui regarde les nuées, ne moissonne point. (\hyperref[Eccl.11.4]{\dotuline{\emph{Eccl}.11.4}})
 \end{quoteblock}

 \noindent La morale de Qohélet s’appuie sur une métaphysique, en tous cas une conception du temps, que l’on comprendra plus finement par contraste avec Épicure. Le grec écrit : \emph{« Il faut encore se rappeler que l’avenir n’est ni tout à fait nôtre ni tout à fait non-nôtre, afin que nous ne l’attendions pas à coup sûr comme devant être, ni n’en désespérions comme devant absolument ne pas être »}. On retrouve la distinction stoïcienne entre ce qui dépend de nous (l’avenir qui est mien) et ce sur quoi nous ne pouvons rien et dont il ne faudrait pas s’affecter. Épicure poursuit contre ceux qui croient au hasard, ou au destin, alors que l’on ne peut ps savoir.\par
 Qohélet prend aussi la mesure de notre ignorance de l’avenir et le formule dans une image concrète et paysanne : sème sans craindre le vent, moissonne même si les nuages menacent (on verra). Mais contrairement à Épicure, Qohélet ne suppose pas que nous ayons une fortune personnelle, un particulier sur lequel se replier pendant les revers de fortune. Ce qui lui est cher ne dépend pas de lui seul.\par
 De cette ignorance du futur, Épicure tire une conclusion virile et aristocratique. Il est inutile de craindre la mort, puisque nous ne savons pas son heure ; et qu’avant, nous sommes vivants ; et qu’après, nous n’y sommes plus. Qohélet partage les prémisses, « les hommes ne savent point leur temps (\hyperref[Eccl.9.12]{\dotuline{9.12}}) » ; mais pas la conclusion.\par
 Épicure a une conception très abstraite du temps, comme s’il était une suite d’instants qui ne durent pas, qui ne pèsent pas plus qu’ils ne s’accumulent. S’il est si difficile de se convaincre de ses raisons, c’est que dans notre chair est inscrite cette peur panique de sa disparition, sa fièvre de se reproduire, et de se survivre.\par
 Qohélet, comme le reste de la Bible d’ailleurs, aime les enfants, il désespère de voir s’épuiser des personnes dont les héritiers ne sont pas venus, \emph{« je vois sous le soleil une chose qui rien ne vaut […] tel qui est tout seul sans hoir [héritier …] qui néanmoins ne cesse jamais de travailler (\hyperref[Eccl.4.8]{\dotuline{4.8}}) »} ; le philosophe grec n’en parle pas. Le temps de la moisson, des naissances, attache Qohélet à un monde qui lui est cher, dont il se sent une partie, il n’est pas un homme seul cherchant seulement l’indépendance, il aime le monde de toute sa chair, la mort est nécessairement un arrachement. Ce qu’il pleure au fond, est la beauté de la vie, dont il voudrait qu’elle soit vaine si elle est sans lui, mais les générations passent et la terre dure (\hyperref[Eccl.1.4]{\dotuline{1.4}}).\par
 
\labelblock{La Conspiration}

 \noindent Parmi les formules plus répétées, comme \emph{rien ne vaut rien}, il en est une plus équivoque, traduite par la \emph{TOB} (2010) en \emph{« tout est vanité et poursuite de vent »}, et par Sacy (1667) en \emph{« tout est vanité et affliction d’esprit »}. \emph{Vent} ou \emph{esprit} ? Qui se trompe ? Le vent résulte pour nous d’une différence de pression dans l’atmosphère, mais que pouvait-il signifier alors ? Il n’était pas irrationnel de penser que le souffle du ciel participait du même mystère que celui de la vie. L’hésitation des traducteurs entre sens abstrait ou concret concerne un même mot hébreu, \emph{rū·aḥ}, que Chouraki (1974) traduit de manière conséquente partout par \emph{souffle}, ainsi dans 12.7 : \emph{« La poussière retourne à la terre comme elle était, et le souffle retourne vers Elohîms qui l’a donné… »} Le monde de Qohélet conspire, les corps et la terre reçoivent leur souffle de Dieu.\par
 La théologie païenne est par principe, multiple. Les dieux jouent et s’affrontent sur terre et dans l’humanité. Les gens sont inégaux, le peuple a peur de la colère des dieux, il soudoie des prêtres pour essayer d’être en paix avec eux, Épicure propose une solution rationnelle qui est la nôtre aujourd’hui. Le monde n’est qu’un ensemble d’atomes qui s’entrechoquent selon les lois de la matière. Prier n’amène pas la pluie. C’est à cette condition de séparation radicale de l’âme d’avec les choses que le sage trouve une paix entre ses éléments, comme un bouchon trouve une flottaison sur une mer agitée.\par
 Si notre science donne maintenant raison à Épicure, Kant écrivait encore en 1790 dans la \emph{Critique de la Faculté de Juger} qu’il est \emph{« absurde d’espérer qu’il surgira un jour quelque Newton qui pourrait faire comprendre ne serait-ce que la production d’un brin d’herbe »}. Nous l’avons maintenant, avec Darwin et la génétique, mais Épicure n’expliquait déjà pas l’inertie du mouvement, c’était un acte de foi assez fou de croire que le souffle de la vie et la pensée ne tenait qu’à des atomes qui se choquent. Qohélet a désormais tort, mais il était plus raisonnable qu’Épicure. La théologie d’un Dieu unique qui ne se laisse pas acheter par les prières lui évitait les superstitions que combattaient Épicure, sans perdre l’âme du monde dans une soupe mécanique d’atomes.\par
 Selon cette perspective, l’Église a eu raison de condamner Galilée, moins à cause du soleil comme centre du monde que du principe d’inertie qui assure l’éternité du mouvement des planètes sans l’action de Dieu, si bien que le monde peut très bien ne pas avoir été créé, comme celui d’Épicure. Si Galilée est vrai, alors \emph{La Bible} est fausse ; les philosophes antiques ne sont plus les égarés du premier cercle des enfers selon Dante, c’est Qohélet et les pères de l’Église qu’il faut mettre aux enfers des erreurs de la science. Il est de coutume aujourd’hui de se moquer des inquisiteurs qui n’auraient rien compris à Galilée. Ils avaient au contraire mesuré les conséquences sociales de ses vérités, comme ils l’avaient fait par le passé à l’égard des différentes hérésies élitistes et sectaires. Le paysan pouvait faire corps avec Qohélet, vibrer des mêmes joies et désespoirs, du souffle qui descend dans l’enfant au ventre de la mère (\hyperref[Eccl.11.5]{\dotuline{11.5}}) ou qui ramène les nuages qui menacent la moisson (\hyperref[Eccl.11.4]{\dotuline{11.4}}). Dans le monde froid de la science, seuls les meilleurs savent, et les autres qui essaient de suivre ne peuvent qu’obéir ou s’inventer des théories plus sentimentales mais fausses.\par
 
\labelblock{Pour conclure}

 \noindent Lire Qohélet, c’est se convaincre que l’on ne peut absolument plus croire à son Dieu, nous ne sentons plus son monde pneumatique de souffles ; mais nous aimerions trouver une sagesse aussi humaine, et compatible avec notre physique de mécanique et d’information. Mais de même, nous ne voulons pas de la société d’Épicure, de sages calmes par-dessus les passions et l’histoire, ne cherchant au fond qu’un petit confort égoïste assez vain, qui apporte peu au monde ; nous voudrions cette paix mais dans l’action. Nos vies ressemblent plus à celle de Qohélet, avec si possible une famille, des enfants, un travail utile aux autres et où on se réalise, et des congés pour se réunir et banqueter ensemble. Peu importe que ce soit ou pas un don de Dieu, c’est en tous cas une joie de la vie. Plus on sera heureux dans la vie, plus on sera malheureux de la quitter, c’est peut-être dans l’ordre des choses, tant pis, il faudra pleurer, mais pour qui se lamenter est un problème ? Sans doute pour un Grec élitiste et viriliste.\par
 
\asterism

 
\labelblock{Castellion}

 \noindent Ce n’est pas seulement par goût pour sa langue que Qohélet est donné ici dans la traduction de Castellion (1515, Savoie – 1563, Bâle). Pour mieux connaître cet érudit, on peut lire Stefan Zweig, \href{https://hurlus.fr/zweig1936\_conscience-violence/}{\dotuline{une Conscience contre la violence : Castellion contre Calvin}}\footnote{\href{https://hurlus.fr/zweig1936\_conscience-violence/}{\url{https://hurlus.fr/zweig1936\_conscience-violence/}}}. Après avoir traduit \emph{la Bible} en latin pour les doctes, Castellion a voulu donner un texte en français « \emph{pour les idiots} » (selon ses propres mots) ; pour être le livre de toutes les familles. Mais cette traduction de 1555 est maudite.\par
 Castellion est issu de paysans pauvres et arrive à faire des études à Lyon où il découvre l’humanisme classique. Il sait le grec et lira Épicure. En 1534, Luther traduit \emph{la Bible} pour la première fois dans une langue vulgaire, son allemand, depuis l’hébreu original et non le latin de \emph{la Vulgate} officielle de l’Église. Castellion est alors pris de la fièvre de son époque, il rejoint Calvin en 1540, il est excommunié de Genève en 1544, parce qu’il défend que le \emph{Cantique des Cantiques} n’est pas une pure allégorie de Dieu et son Église, c’est une véritable histoire d’amour charnel qui déplaît fort au puritain Calvin.\par
 Castellion ne veut rien perdre des leçons de l’humanisme païen, comme par exemple Rabelais, tout en cherchant un humanisme chrétien authentique, tel qu’il est dans les textes. Il a réussi à faire paraître sa Bible en 1555 à Bâle, mais attaqué par l’Église catholique, ainsi que les calvinistes, et n’ayant pas écrit en néerlandais pour intéresser la Hollande (qui a relu et parfois imprimé son latin contre les puritains), cette traduction n’a été rééditée au complet qu’en 2005\footnote{\emph{La Bible nouvellement translatée par Sébastien Castellion (1555)}, Paris, Bayard, 2005 (édition : Gomez-Géraud, Marie-Christine \& Mistral, Laure),}, 450 ans après. Cette édition est d’ailleurs introuvable en 2020, cette \emph{Bible} est maudite. Remercions heureusement ce siècle numérique, une version de 1555 a été numérisée par \href{https://books.google.fr/books?id=aShJAAAAcAAJ\&printsec=frontcover\#v=onepage\&q\&f=false}{\dotuline{Google}}\footnote{\href{https://books.google.fr/books?id=aShJAAAAcAAJ\&printsec=frontcover\#v=onepage\&q\&f=false}{\url{https://books.google.fr/books?id=aShJAAAAcAAJ\&printsec=frontcover\#v=onepage\&q\&f=false}}}.\par
 
\labelblock{Cette édition}

 \noindent Le texte donné ici a été établi sur l’imprimé original, et modernisé pour nous rendre la langue plus familière. À l’époque, l’orthographe n’était pas fixée, chaque imprimeur ou auteur pouvait chercher la sienne. Celle de Castellion était très proche de la nôtre, et même un peu plus régulière. Par exemple il écrivait notre \emph{il est} avec un accent circonflexe, « \emph{il êt »}, comme dans \emph{forêt} ; ou bien la conjonction \emph{et} comme en espagnol, « \emph{e} », puisque le \emph{t} ne se prononce jamais, même en liaison.\par
 Dans chaque détail, ce traducteur met de l’intelligence. Il ajoute très peu de notes, dont l’appel se fait avant ce qu’il veut éclairer, comme pour préparer le lecteur par une information ; par contre il évite ces surcharges marginales des catholiques ou des calvinistes qui ensevelissent la parole de Dieu sous leurs commentaires pour que le lecteur lisent bien leur dogme et pas le texte. Vouloir absolument que les prophètes annoncent déjà le Christ ou l’Église, ou combattent l’épicurisme et le nihilisme, c’est pour le moins construire un Qohélet parallèle.\par
 Castellion pense que la parole de Dieu parle toute seule, ou alors elle ne vaut pas d’être crue. Ses remarques très brèves sont parfois lumineuses, tranchant dans des siècles de contradictions rabbiniques ou de reformulations charitables des chrétiens, par exemple en \hyperref[Eccl.12.2]{\dotuline{12.2-5}}.\par
 \bigbreak
 \noindent Ce qui s’adressait aux idiots d’hier garde un sel populaire mais demande tout de même un petit effort aujourd’hui. L’éditeur a ajouté quelques notes de vocabulaire (appels en a, b, c…). Les versets sont numérotés comme de nos jours, les passages à la ligne sont repris de \emph{la TOB}, assurant que Castellion a scrupuleusement suivi le découpage à l’intérieur des versets de l’hébreu, à la réserve de quelques rares inversions pouvant s’expliquer par le texte qu’il avait à l’époque.\par
 Une note de bas de page en 12.6 nous rend son scrupule définitivement attachant : \emph{« je n’entends pas ces quatre »} (divisions qui font un verset). L’Église ou Calvin auraient-ils osé dire qu’ils ne comprenaient pas \emph{la Bible}, alors qu’ils s’instituaient les interprètes de la « \emph{parole de Dieu} » sur terre ? Castellion est par ailleurs lumineux en bien des endroits, donnant des leçons encore retenues, il a quelques faiblesses, elles sont rares. Sa traduction est une école de probité intellectuelle. Il montre ce qu’est comprendre simplement sans se payer de mots obscurs. Ce n’est pas facile de de s’adresser aux « idiots », il faut beaucoup travailler pour être simple. C’est une joie de notre siècle numérique de pouvoir venger Castellion de Calvin\footnote{Voir note de l’éditeur en 2.3}, et le donner gratuitement.\par
 
\dateline{Neuch, le 20 août 2021}
 \newpage

}
\def\elsource{\href{}{\dotuline{}}\footnote{\href{}{\url{}}}}
\def\eltitlepage{%
{\centering\parindent0pt
  {\LARGE\addfontfeature{LetterSpace=25}\bfseries Qohélet\par}\bigskip
  {\Large -250\par}\bigskip
  {\LARGE
\bigskip\textbf{L’Ecclésiaste, c’est-à-dire : \\
le Prêcheur.}\par
\bigskip\emph{traduction de 1555\\
par Sébastien Castellion \\
\emph{dîte} \\
« la Bible pour les idiots »}\par

  }
}

}

% Default metas
\newcommand{\colorprovide}[2]{\@ifundefinedcolor{#1}{\colorlet{#1}{#2}}{}}
\colorprovide{rubric}{red}
\colorprovide{silver}{lightgray}
\@ifundefined{syms}{\newfontfamily\syms{DejaVu Sans}}{}
\newif\ifdev
\@ifundefined{elbibl}{% No meta defined, maybe dev mode
  \newcommand{\elbibl}{Titre court ?}
  \newcommand{\elbook}{Titre du livre source ?}
  \newcommand{\elabstract}{Résumé\par}
  \newcommand{\elurl}{http://oeuvres.github.io/elbook/2}
  \author{Éric Lœchien}
  \title{Un titre de test assez long pour vérifier le comportement d’une maquette}
  \date{1566}
  \devtrue
}{}
\let\eltitle\@title
\let\elauthor\@author
\let\eldate\@date




% generic typo commands
\newcommand{\astermono}{\medskip\centerline{\color{rubric}\large\selectfont{\syms ✻}}\medskip\par}%
\newcommand{\astertri}{\medskip\par\centerline{\color{rubric}\large\selectfont{\syms ✻\,✻\,✻}}\medskip\par}%
\newcommand{\asterism}{\bigskip\par\noindent\parbox{\linewidth}{\centering\color{rubric}\large{\syms ✻}\\{\syms ✻}\hskip 0.75em{\syms ✻}}\bigskip\par}%

% lists
\newlength{\listmod}
\setlength{\listmod}{\parindent}
\setlist{
  itemindent=!,
  listparindent=\listmod,
  labelsep=0.2\listmod,
  parsep=0pt,
  % topsep=0.2em, % default topsep is best
}
\setlist[itemize]{
  label=—,
  leftmargin=0pt,
  labelindent=1.2em,
  labelwidth=0pt,
}
\setlist[enumerate]{
  label={\arabic*°},
  labelindent=0.8\listmod,
  leftmargin=\listmod,
  labelwidth=0pt,
}
% list for big items
\newlist{decbig}{enumerate}{1}
\setlist[decbig]{
  label={\bf\color{rubric}\arabic*.},
  labelindent=0.8\listmod,
  leftmargin=\listmod,
  labelwidth=0pt,
}
\newlist{listalpha}{enumerate}{1}
\setlist[listalpha]{
  label={\bf\color{rubric}\alph*.},
  leftmargin=0pt,
  labelindent=0.8\listmod,
  labelwidth=0pt,
}
\newcommand{\listhead}[1]{\hspace{-1\listmod}\emph{#1}}

\renewcommand{\hrulefill}{%
  \leavevmode\leaders\hrule height 0.2pt\hfill\kern\z@}

% General typo
\DeclareTextFontCommand{\textlarge}{\large}
\DeclareTextFontCommand{\textsmall}{\small}

% commands, inlines
\newcommand{\anchor}[1]{\Hy@raisedlink{\hypertarget{#1}{}}} % link to top of an anchor (not baseline)
\newcommand\abbr[1]{#1}
\newcommand{\autour}[1]{\tikz[baseline=(X.base)]\node [draw=rubric,thin,rectangle,inner sep=1.5pt, rounded corners=3pt] (X) {\color{rubric}#1};}
\newcommand\corr[1]{#1}
\newcommand{\ed}[1]{ {\color{silver}\sffamily\footnotesize (#1)} } % <milestone ed="1688"/>
\newcommand\expan[1]{#1}
\newcommand\foreign[1]{\emph{#1}}
\newcommand\gap[1]{#1}
\renewcommand{\LettrineFontHook}{\color{rubric}}
\newcommand{\initial}[2]{\lettrine[lines=2, loversize=0.3, lhang=0.3]{#1}{#2}}
\newcommand{\initialiv}[2]{%
  \let\oldLFH\LettrineFontHook
  % \renewcommand{\LettrineFontHook}{\color{rubric}\ttfamily}
  \IfSubStr{QJ’}{#1}{
    \lettrine[lines=4, lhang=0.2, loversize=-0.1, lraise=0.2]{\smash{#1}}{#2}
  }{\IfSubStr{É}{#1}{
    \lettrine[lines=4, lhang=0.2, loversize=-0, lraise=0]{\smash{#1}}{#2}
  }{\IfSubStr{ÀÂ}{#1}{
    \lettrine[lines=4, lhang=0.2, loversize=-0, lraise=0, slope=0.6em]{\smash{#1}}{#2}
  }{\IfSubStr{A}{#1}{
    \lettrine[lines=4, lhang=0.2, loversize=0.2, slope=0.6em]{\smash{#1}}{#2}
  }{\IfSubStr{V}{#1}{
    \lettrine[lines=4, lhang=0.2, loversize=0.2, slope=-0.5em]{\smash{#1}}{#2}
  }{
    \lettrine[lines=4, lhang=0.2, loversize=0.2]{\smash{#1}}{#2}
  }}}}}
  \let\LettrineFontHook\oldLFH
}
\newcommand{\labelchar}[1]{\textbf{\color{rubric} #1}}
\newcommand{\lnatt}[1]{\reversemarginpar\marginpar[\sffamily\scriptsize #1]{}}
\newcommand{\milestone}[1]{\autour{\footnotesize\color{rubric} #1}} % <milestone n="4"/>
\newcommand\name[1]{#1}
\newcommand\orig[1]{#1}
\newcommand\orgName[1]{#1}
\newcommand\persName[1]{#1}
\newcommand\placeName[1]{#1}
\newcommand{\pn}[1]{\IfSubStr{-—–¶}{#1}% <p n="3"/>
  {\noindent{\bfseries\color{rubric}   ¶  }}
  {{\footnotesize\autour{#1}}}}
\newcommand\reg{}
% \newcommand\ref{} % already defined
\newcommand\sic[1]{#1}
\newcommand\surname[1]{\textsc{#1}}
\newcommand\term[1]{\textbf{#1}}
\newcommand\zh[1]{{\zhfont #1}}


\def\mednobreak{\ifdim\lastskip<\medskipamount
  \removelastskip\nopagebreak\medskip\fi}
\def\bignobreak{\ifdim\lastskip<\bigskipamount
  \removelastskip\nopagebreak\bigskip\fi}

% commands, blocks

\newcommand{\byline}[1]{\bigskip{\RaggedLeft{#1}\par}\bigskip}
% \setlength{\RaggedLeftLeftskip}{2em plus \leftskip}
\newcommand{\bibl}[1]{{\smallskip\RaggedLeft\normalsize\normalfont #1\par\medskip}}
\newcommand{\biblitem}[1]{{\noindent\hangindent=\parindent   #1\par}}
\newcommand{\castItem}[1]{{\noindent\hangindent=\parindent #1\par}}
\newcommand{\dateline}[1]{\medskip{\RaggedLeft{#1}\par}\bigskip}
\newcommand{\docAuthor}[1]{{\large\bigskip #1 \par\bigskip}}
\newcommand{\docDate}[1]{#1 \ifvmode\par\fi }
\newcommand{\docImprint}[1]{\ifvmode\medskip\fi #1 \ifvmode\par\fi }
\newcommand{\labelblock}[1]{\medbreak{\noindent\color{rubric}\bfseries #1}\par\mednobreak}
\newcommand{\salute}[1]{\bigbreak{#1}\par\medbreak}
\newcommand{\signed}[1]{\medskip{\RaggedLeft #1\par}\bigbreak} % supposed bottom
\newcommand{\speaker}[1]{\medskip{\Centering\sffamily #1 \par\nopagebreak}} % supposed bottom
\newcommand{\stagescene}[1]{{\Centering\sffamily\textsf{#1}\par}\bigskip}
\newcommand{\stageblock}[1]{\begingroup\leftskip\parindent\noindent\it\sffamily\footnotesize #1\par\endgroup} % left margin, better than list envs
\newcommand{\spl}[1]{\noindent\hangindent=2\parindent  #1\par} % sp/l
\newcommand{\trailer}[1]{{\Centering\bigskip #1\par}} % sp/l

% environments for blocks (some may become commands)
\newenvironment{borderbox}{}{} % framing content
\newenvironment{citbibl}{\ifvmode\hfill\fi}{\ifvmode\par\fi }
\newenvironment{msHead}{\vskip6pt}{\par}
\newenvironment{msItem}{\vskip6pt}{\par}


% environments for block containers
\newenvironment{argument}{\itshape\parindent0pt}{\bigskip}
\newenvironment{biblfree}{}{\ifvmode\par\fi }
\newenvironment{bibitemlist}[1]{%
  \list{\@biblabel{\@arabic\c@enumiv}}%
  {%
    \settowidth\labelwidth{\@biblabel{#1}}%
    \leftmargin\labelwidth
    \advance\leftmargin\labelsep
    \@openbib@code
    \usecounter{enumiv}%
    \let\p@enumiv\@empty
    \renewcommand\theenumiv{\@arabic\c@enumiv}%
  }
  \sloppy
  \clubpenalty4000
  \@clubpenalty \clubpenalty
  \widowpenalty4000%
  \sfcode`\.\@m
}%
{\def\@noitemerr
  {\@latex@warning{Empty `bibitemlist' environment}}%
\endlist}
\newenvironment{docTitle}{\LARGE\bigskip\bfseries\onehalfspacing}{\bigskip}
% leftskip makes big bugs in Lexmark printing \sffamily
\newenvironment{epigraph}{\begin{addmargin}[2\parindent]{0em}\sffamily\large\setstretch{0.95}}{\end{addmargin}\bigskip}
\newenvironment{quoteblock}% may be used for ornaments
  {\begin{quoting}}
  {\end{quoting}}
\newenvironment{titlePage}
  {\Centering}
  {}






% table () is preceded and finished by custom command
\renewcommand\tabularxcolumn[1]{m{#1}}% for vertical centering text in X column
\newcommand{\tableopen}[1]{%
  \ifnum\strcmp{#1}{wide}=0{%
    \begin{center}
  }
  \else\ifnum\strcmp{#1}{long}=0{%
    \begin{center}
  }
  \else{%
    \begin{center}
  }
  \fi\fi
}
\newcommand{\tableclose}[1]{%
  \ifnum\strcmp{#1}{wide}=0{%
    \end{center}
  }
  \else\ifnum\strcmp{#1}{long}=0{%
    \end{center}
  }
  \else{%
    \end{center}
  }
  \fi\fi
}


% text structure
\newcommand\chapteropen{} % before chapter title
\newcommand\chaptercont{} % after title, argument, epigraph…
\newcommand\chapterclose{} % maybe useful for multicol settings
\setcounter{secnumdepth}{-2} % no counters for hierarchy titles
\setcounter{tocdepth}{5} % deep toc
\renewcommand\tableofcontents{\@starttoc{toc}}
% toclof format
% \renewcommand{\@tocrmarg}{0.1em} % Useless command?
% \renewcommand{\@pnumwidth}{0.5em} % {1.75em}
\renewcommand{\@cftmaketoctitle}{}
\setlength{\cftbeforesecskip}{\z@ \@plus.2\p@}
\renewcommand{\cftchapfont}{}
\renewcommand{\cftchapdotsep}{\cftdotsep}
\renewcommand{\cftchapleader}{\normalfont\cftdotfill{\cftchapdotsep}}
\renewcommand{\cftchappagefont}{\bfseries}
\setlength{\cftbeforechapskip}{0em \@plus\p@}
% \renewcommand{\cftsecfont}{\small\relax}
\renewcommand{\cftsecpagefont}{\normalfont}
% \renewcommand{\cftsubsecfont}{\small\relax}
\renewcommand{\cftsecdotsep}{\cftdotsep}
\renewcommand{\cftsecpagefont}{\normalfont}
\renewcommand{\cftsecleader}{\normalfont\cftdotfill{\cftsecdotsep}}
\setlength{\cftsecindent}{1em}
\setlength{\cftsubsecindent}{2em}
\setlength{\cftsubsubsecindent}{3em}
\setlength{\cftchapnumwidth}{1em}
\setlength{\cftsecnumwidth}{1em}
\setlength{\cftsubsecnumwidth}{1em}
\setlength{\cftsubsubsecnumwidth}{1em}

% footnotes
\newif\ifheading
\newcommand*{\fnmarkscale}{\ifheading 0.70 \else 1 \fi}
\renewcommand\footnoterule{\vspace*{0.3cm}\hrule height \arrayrulewidth width 3cm \vspace*{0.3cm}}
\setlength\footnotesep{1.5\footnotesep} % footnote separator
\renewcommand\@makefntext[1]{\parindent 1.5em \noindent \hb@xt@1.8em{\hss{\normalfont\@thefnmark . }}#1} % no superscipt in foot
\patchcmd{\@footnotetext}{\footnotesize}{\footnotesize\sffamily}{}{} % before scrextend, hyperref
\DeclareNewFootnote{A}[alph] % for editor notes
\renewcommand*{\thefootnoteA}{\alphalph{\value{footnoteA}}} % z, aa, ab…

% poem
\setlength{\poembotskip}{0pt}
\setlength{\poemtopskip}{0pt}
\setlength{\poemindent}{0pt}
\poemlinenumsfalse

%   see https://tex.stackexchange.com/a/34449/5049
\def\truncdiv#1#2{((#1-(#2-1)/2)/#2)}
\def\moduloop#1#2{(#1-\truncdiv{#1}{#2}*#2)}
\def\modulo#1#2{\number\numexpr\moduloop{#1}{#2}\relax}

% orphans and widows, nowidow package in test
% from memoir package
\clubpenalty=9996
\widowpenalty=9999
\brokenpenalty=4991
\predisplaypenalty=10000
\postdisplaypenalty=1549
\displaywidowpenalty=1602
\hyphenpenalty=400
% report h or v overfull ?
\hbadness=4000
\vbadness=4000
% good to avoid lines too wide
\emergencystretch 3em
\pretolerance=750
\tolerance=2000
\def\Gin@extensions{.pdf,.png,.jpg,.mps,.tif}

\PassOptionsToPackage{hyphens}{url} % before hyperref and biblatex, which load url package
\usepackage{hyperref} % supposed to be the last one, :o) except for the ones to follow
\hypersetup{
  % pdftex, % no effect
  pdftitle={\elbibl},
  % pdfauthor={Your name here},
  % pdfsubject={Your subject here},
  % pdfkeywords={keyword1, keyword2},
  bookmarksnumbered=true,
  bookmarksopen=true,
  bookmarksopenlevel=1,
  pdfstartview=Fit,
  breaklinks=true, % avoid long links, overrided by url package
  pdfpagemode=UseOutlines,    % pdf toc
  hyperfootnotes=true,
  colorlinks=false,
  pdfborder=0 0 0,
  % pdfpagelayout=TwoPageRight,
  % linktocpage=true, % NO, toc, link only on page no
}
\urlstyle{same} % after hyperref



\makeatother % /@@@>
%%%%%%%%%%%%%%
% </TEI> end %
%%%%%%%%%%%%%%


%%%%%%%%%%%%%
% footnotes %
%%%%%%%%%%%%%
\renewcommand{\thefootnote}{\bfseries\textcolor{rubric}{\arabic{footnote}}} % color for footnote marks

%%%%%%%%%
% Fonts %
%%%%%%%%%
% \linespread{0.90} % too compact, keep font natural
\ifav % A5
  \usepackage{DejaVuSans} % correct
  \setsansfont{DejaVuSans} % seen, if not set, problem with printer
\else\ifbooklet
  \usepackage[]{roboto} % SmallCaps, Regular is a bit bold
  \setmainfont[
    ItalicFont={Roboto Light Italic},
  ]{Roboto}
  \setsansfont{Roboto Light} % seen, if not set, problem with printer
  \newfontfamily\fontrun[]{Roboto Condensed Light} % condensed runing heads
\else
  \usepackage[]{roboto} % SmallCaps, Regular is a bit bold
  \setmainfont[
    ItalicFont={Roboto Italic},
  ]{Roboto Light}
  \setsansfont{Roboto Light} % seen, if not set, problem with printer
  \newfontfamily\fontrun[]{Roboto Condensed Light} % condensed runing heads
\fi\fi
\renewcommand{\LettrineFontHook}{\bfseries\color{rubric}}
% \renewenvironment{labelblock}{\begin{center}\bfseries\color{rubric}}{\end{center}}

%%%%%%%%
% MISC %
%%%%%%%%

\setdefaultlanguage[frenchpart=false]{french} % bug on part


\newenvironment{quotebar}{%
    \def\FrameCommand{{\color{rubric!10!}\vrule width 0.5em} \hspace{0.9em}}%
    \def\OuterFrameSep{0pt} % séparateur vertical
    \MakeFramed {\advance\hsize-\width \FrameRestore}
  }%
  {%
    \endMakeFramed
  }
\renewenvironment{quoteblock}% may be used for ornaments
  {%
    \savenotes
    \setstretch{0.9}
    \begin{quotebar}
    \smallskip
  }
  {%
    \smallskip
    \end{quotebar}
    \spewnotes
  }


\renewcommand{\headrulewidth}{\arrayrulewidth}
\renewcommand{\headrule}{{\color{rubric}\hrule}}
\renewcommand{\lnatt}[1]{\marginpar{\sffamily\scriptsize #1}}

% delicate tuning, image has produce line-height problems in title on 2 lines
\titleformat{name=\chapter} % command
  [display] % shape
  {\vspace{1.5em}\centering} % format
  {} % label
  {0pt} % separator between n
  {}
[{\color{rubric}\huge\textbf{#1}}\bigskip] % after code
% \titlespacing{command}{left spacing}{before spacing}{after spacing}[right]
\titlespacing*{\chapter}{0pt}{-2em}{0pt}[0pt]

\titleformat{name=\section}
  [display]{}{}{}{}
  [\vbox{\color{rubric}\large\centering\textbf{#1}}]
\titlespacing{\section}{0pt}{0pt plus 4pt minus 2pt}{\baselineskip}

\titleformat{name=\subsection}
  [block]
  {}
  {} % \thesection
  {} % separator \arrayrulewidth
  {}
[\vbox{\large\textbf{#1}}]
% \titlespacing{\subsection}{0pt}{0pt plus 4pt minus 2pt}{\baselineskip}

\ifaiv
  \fancypagestyle{main}{%
    \fancyhf{}
    \setlength{\headheight}{1.5em}
    \fancyhead{} % reset head
    \fancyfoot{} % reset foot
    \fancyhead[L]{\truncate{0.45\headwidth}{\fontrun\elbibl}} % book ref
    \fancyhead[R]{\truncate{0.45\headwidth}{ \fontrun\nouppercase\leftmark}} % Chapter title
    \fancyhead[C]{\thepage}
  }
  \fancypagestyle{plain}{% apply to chapter
    \fancyhf{}% clear all header and footer fields
    \setlength{\headheight}{1.5em}
    \fancyhead[L]{\truncate{0.9\headwidth}{\fontrun\elbibl}}
    \fancyhead[R]{\thepage}
  }
\else
  \fancypagestyle{main}{%
    \fancyhf{}
    \setlength{\headheight}{1.5em}
    \fancyhead{} % reset head
    \fancyfoot{} % reset foot
    \fancyhead[RE]{\truncate{0.9\headwidth}{\fontrun\elbibl}} % book ref
    \fancyhead[LO]{\truncate{0.9\headwidth}{\fontrun\nouppercase\leftmark}} % Chapter title, \nouppercase needed
    \fancyhead[RO,LE]{\thepage}
  }
  \fancypagestyle{plain}{% apply to chapter
    \fancyhf{}% clear all header and footer fields
    \setlength{\headheight}{1.5em}
    \fancyhead[L]{\truncate{0.9\headwidth}{\fontrun\elbibl}}
    \fancyhead[R]{\thepage}
  }
\fi

\ifav % a5 only
  \titleclass{\section}{top}
\fi

\newcommand\chapo{{%
  \vspace*{-3em}
  \centering\parindent0pt % no vskip ()
  \eltitlepage
  \bigskip
  {\color{rubric}\hline}
  \bigskip
  {\Large TEXTE LIBRE À PARTICIPATIONS LIBRES\par}
  \centerline{\small\color{rubric} {\href{https://hurlus.fr}{\dotuline{hurlus.fr}}}, tiré le \today}\par
  \bigskip
}}

\newcommand\cover{{%
  \thispagestyle{empty}
  \centering\parindent0pt
  \eltitlepage
  \vfill\null
  {\color{rubric}\setlength{\arrayrulewidth}{2pt}\hline}
  \vfill\null
  {\Large TEXTE LIBRE À PARTICIPATIONS LIBRES\par}
  \centerline{\href{https://hurlus.fr}{\dotuline{hurlus.fr}}, tiré le \today}\par
}}

\begin{document}
\pagestyle{empty}
\ifbooklet{
  \cover\newpage
  \thispagestyle{empty}\hbox{}\newpage
  \cover\newpage\noindent Les voyages de la brochure\par
  \bigskip
  \begin{tabularx}{\textwidth}{l|X|X}
    \textbf{Date} & \textbf{Lieu}& \textbf{Nom/pseudo} \\ \hline
    \rule{0pt}{25cm} &  &   \\
  \end{tabularx}
  \newpage
  \addtocounter{page}{-4}
}\fi

\thispagestyle{empty}
\ifaiv
  \twocolumn[\chapo]
\else
  \chapo
\fi
{\it\elabstract}
\bigskip
\makeatletter\@starttoc{toc}\makeatother % toc without new page
\bigskip

\pagestyle{main} % after style
\setcounter{footnote}{0}
\setcounter{footnoteA}{0}
  
\section[{Eccl.1}]{\emph{Eccl}.1}
\renewcommand{\leftmark}{\emph{Eccl}.1}

\phantomsection
\label{v1.1}\noindent\hangindent=2\parindent\pn{1} \emph{Les paroles du prêcheur fils de David roi de Jérusalem.}\par
\bigbreak
\phantomsection
\label{v1.2}\noindent\hangindent=2\parindent\pn{2} Tout ne vaut rien, dit le prêcheur, \par
\noindent\hangindent=2\parindent tout ne vaut rien, tout ne vaut du tout rien.\par

\astermono

\phantomsection
\label{v1.3}\noindent\hangindent=2\parindent\pn{3} Que gagne l’homme \par
\noindent\hangindent=2\parindent par toute la peine qu’il prend sous le soleil ?\par
\phantomsection
\label{v1.4}\noindent\hangindent=2\parindent\pn{4} L’âge s’en va, et l’âge vient\footnoteA{Castellion n’a pas été ici le plus inspiré. \href{https://www.e-rara.ch/gep\_g/ch16/content/titleinfo/976561}{\dotuline{Calvin}} [\url{https://www.e-rara.ch/gep\_g/ch16/content/titleinfo/976561}] reprend la leçon de la \emph{Vulgate} ou des \emph{Septantes} : « \emph{Une génération passe, \& l’autre génération vient} ».}, \par
\noindent\hangindent=2\parindent et la terre demeure toujours.\par
\phantomsection
\label{v1.5}\noindent\hangindent=2\parindent\pn{5} Le soleil lève, et le soleil couche \par
\noindent\hangindent=2\parindent  et ahane pour aller au lieu même où il est levé.\par
\phantomsection
\label{v1.6}\noindent\hangindent=2\parindent\pn{6} Il s’en va contre le midi, et retourne contre la bise : \par
\noindent\hangindent=2\parindent le vent s’en va tout alentour, \par
\noindent\hangindent=2\parindent et retourne le même vent à son tour.\par
\phantomsection
\label{v1.7}\noindent\hangindent=2\parindent\pn{7} Toutes rivières vont en la mer, \par
\noindent\hangindent=2\parindent et si n’est pas la mer pleine : \par
\noindent\hangindent=2\parindent au même lieu que vont les rivières, \par
\noindent\hangindent=2\parindent elles y re-vont derechef.\par
\phantomsection
\label{v1.8}\noindent\hangindent=2\parindent\pn{8} Toutes choses sont si difficiles, qu’homme ne les saurait déchiffrer. \par
\noindent\hangindent=2\parindent L’œil n’est jamais soul de voir, \par
\noindent\hangindent=2\parindent ni l’oreille pleine d’ouïr.\par
\phantomsection
\label{v1.9}\noindent\hangindent=2\parindent\pn{9} Ce qui a été, sera : \par
\noindent\hangindent=2\parindent et ce qui a été fait, sera fait, \par
\noindent\hangindent=2\parindent et n’y a rien de nouveau sous le soleil.\par
\bigbreak
\phantomsection
\label{v1.10}\noindent\hangindent=2\parindent\pn{10} Il y a telle chose \par
\noindent\hangindent=2\parindent qu’on montre comme nouvelle, \par
\noindent\hangindent=2\parindent laquelle toutefois a déjà été au temps passé, qui a été devant nous.\par
\phantomsection
\label{v1.11}\noindent\hangindent=2\parindent\pn{11} Il n’est mémoire des passés : \par
\noindent\hangindent=2\parindent et même de ceux qui sont à venir, \par
\noindent\hangindent=2\parindent il n’en sera mémoire \par
\noindent\hangindent=2\parindent vers ceux qui seront après.\par

\astermono

\phantomsection
\label{v1.12}\noindent\hangindent=2\parindent\pn{12} Moi prêcheur, qui suis roi d’Israël en Jérusalem,\par
\phantomsection
\label{v1.13}\noindent\hangindent=2\parindent\pn{13} ai appliqué mon entendement à examiner et éplucher\footnoteA{« examiner »} par sagesse \par
\noindent\hangindent=2\parindent tout ce qui se fait sous le ciel \par
\noindent\hangindent=2\parindent (voila une mauvaise fâcherie, que Dieu a donnée \par
\noindent\hangindent=2\parindent à la race des hommes pour les tourmenter)\par
\phantomsection
\label{v1.14}\noindent\hangindent=2\parindent\pn{14} et en considérant toutes les choses qui se font sous le soleil, \par
\noindent\hangindent=2\parindent j’ai trouvé que tout ne vaut rien, et n’est qu’un tourment d’esprit\footnoteA{Se traduit de nos jours de manière plus littérale par « \emph{poursuivre ou ruminer du vent} », peut-être une locution perdue. Dès les \emph{Septantes} grecs (-270), ou pour les catholiques dans la Vulgate (405), le vent est interprété comme \emph{esprit, afflictio spiritus} ; l’humeur n’est que vanité.},\par
\phantomsection
\label{v1.15}\noindent\hangindent=2\parindent\pn{15} vu qu’il y a tant de choses gâtées, que c’est chose inamendable : \par
\noindent\hangindent=2\parindent et tant de fautes, que c’est chose infinie.\par
\bigbreak
\phantomsection
\label{v1.16}\noindent\hangindent=2\parindent\pn{16} J’ai quelquefois pensé en ma fantaisie : \par
\noindent\hangindent=2\parindent Or-ça, je suis un grand personnage, et ai plus acquis de sagesse, \par
\noindent\hangindent=2\parindent que tous ceux qui ont été devant moi en Jérusalem, \par
\noindent\hangindent=2\parindent et ai en mon cœur la connaissance de beaucoup de sagesse et science.\par
\phantomsection
\label{v1.17}\noindent\hangindent=2\parindent\pn{17} Mais quand j’appliquai mon entendement à connaître tant la sagesse \par
\noindent\hangindent=2\parindent que la folie et sottise, \par
\noindent\hangindent=2\parindent j’ai entendu que ce n’était encore qu’une fâcherie d’esprit\footnoteA{\emph{afflictio spiritus}}.\par
\phantomsection
\label{v1.18}\noindent\hangindent=2\parindent\pn{18} Car tant de sagesse, tant de chagrin : \par
\noindent\hangindent=2\parindent et qui plus apprend, plus se tourmente.

\section[{Eccl.2}]{\emph{Eccl}.2}
\renewcommand{\leftmark}{\emph{Eccl}.2}

\phantomsection
\label{Eccl.2.1}\noindent\hangindent=2\parindent\pn{1} Je vins une fois à penser ainsi : \par
\noindent\hangindent=2\parindent Or-ça, il me faut prendre mes plaisirs, et me donner de bon temps : \par
\noindent\hangindent=2\parindent mais je trouvai que cela ne vaut encore rien,\par
\phantomsection
\label{Eccl.2.2}\noindent\hangindent=2\parindent\pn{2} tellement que j’étais contraint de dire, que de rire ce n’est autre chose qu’être hors du sens : \par
\noindent\hangindent=2\parindent et que plaisir ne sert de rien.\par
\phantomsection
\label{Eccl.2.3}\noindent\hangindent=2\parindent\pn{3} Je délibérais en ma fantaisie \par
\noindent\hangindent=2\parindent d’abandonner mon corps à boire\footnoteA{Cela ne se disait pas dans la \emph{Vulgate} « \emph{Cogitavi in corde meo abstrahere a vino carnem meam} ; j’ai pensé en cœur mien d’abstraire du vin chair mienne » ; que l’on retrouve mieux dit chez le catholique Sacy (1667) : « \emph{j’ai pensé en moi-même de retirer ma chair du vin} ». La \emph{Bible de Genève} (\href{https://www.e-rara.ch/gep\_g/doi/10.3931/e-rara-3320}{\dotuline{1588, révision de Bèze}} [\url{https://www.e-rara.ch/gep\_g/doi/10.3931/e-rara-3320}]) n’a pas retenu la leçon de Castellion, préférant le puritanisme à la vérité du texte, proposant un étrange : \emph{« J’ai recherché en mon cœur le moyen de me traiter délicatement »}. Les \emph{Septantes} grecs (-270), peut-être plus proche de la source, proposent une solution très astucieuse pour concilier la morale et l’ivrognerie « \emph{Et j’ai examiné si mon cœur enivrerait ma chair comme du vin ; et mon cœur m’a conduit à la sagesse et au désir de posséder le bonheur} » (\href{https://archive.org/details/LaBibleDesSeptanteEnFrancaisVol3/page/n388/mode/1up}{\dotuline{traduction 1865}} [\url{https://archive.org/details/LaBibleDesSeptanteEnFrancaisVol3/page/n388/mode/1up}], Pierre Giguet). Les traductions actuelles s’accordent désormais à considérer que le sage essaie bien ici de faire l’expérience de l’ivresse et de la folie, pour savoir ce qui convient le mieux à l’humanité.} \par
\noindent\hangindent=2\parindent (sans toutefois laisser de pratiquer sagesse en mon cœur) \par
\noindent\hangindent=2\parindent et m’appliquer à folie, \par
\noindent\hangindent=2\parindent jusqu’à tant que je verrais où gît le bien de la race des hommes, \par
\noindent\hangindent=2\parindent lequel ils doivent pourchasser sous le ciel \par
\noindent\hangindent=2\parindent tout le temps de leur vie.\par
\bigbreak
\phantomsection
\label{Eccl.2.4}\noindent\hangindent=2\parindent\pn{4} Je fis des œuvres magnifiques : \par
\noindent\hangindent=2\parindent me bâtis des maisons : plantais vignes :\par
\phantomsection
\label{Eccl.2.5}\noindent\hangindent=2\parindent\pn{5} fis jardins et vergers, \par
\noindent\hangindent=2\parindent et y plantai toutes sortes d’arbres fruitiers.\par
\phantomsection
\label{Eccl.2.6}\noindent\hangindent=2\parindent\pn{6} Je fis des étangs, \par
\noindent\hangindent=2\parindent pour abreuver un bocage planté d’arbres.\par
\phantomsection
\label{Eccl.2.7}\noindent\hangindent=2\parindent\pn{7} J’achetai serviteurs et servantes, et non seulement eu ménage, \par
\noindent\hangindent=2\parindent mais même eu plus de bestiaux, tant gros que menu, \par
\noindent\hangindent=2\parindent que tous ceux qui devant moi avaient été en Jérusalem.\par
\phantomsection
\label{Eccl.2.8}\noindent\hangindent=2\parindent\pn{8} J’amassai aussi argent et or, \par
\noindent\hangindent=2\parindent et chevance\footnoteA{« biens, patrimoine »} de rois et provinces. \par
\noindent\hangindent=2\parindent Je fis provision de chantres et chanteresses, \par
\noindent\hangindent=2\parindent et des passe-temps de la race des hommes, échansons et tasses :\par
\phantomsection
\label{Eccl.2.9}\noindent\hangindent=2\parindent\pn{9} et devins si grand, \par
\noindent\hangindent=2\parindent que j’avais plus que personne de ceux qui furent devant moi en Jérusalem, retenant néanmoins ma sagesse.\par
\bigbreak
\phantomsection
\label{Eccl.2.10}\noindent\hangindent=2\parindent\pn{10} Item de tout ce que mes yeux souhaitaient, je ne leur refusais rien, \par
\noindent\hangindent=2\parindent et n’épargnais à mon cœur plaisir quelconque, \par
\noindent\hangindent=2\parindent mais le laissait jouir de tout mon travail, \par
\noindent\hangindent=2\parindent et  voila que me valait tout mon travail.\par
\phantomsection
\label{Eccl.2.11}\noindent\hangindent=2\parindent\pn{11} Mais en contemplant toutes les œuvres \par
\noindent\hangindent=2\parindent que j’avais maniées, \par
\noindent\hangindent=2\parindent et la peine que j’avais prise à les faire, \par
\noindent\hangindent=2\parindent je trouvais que tout n’est rien qu’une fâcherie d’esprit\footnoteA{\emph{afflictio spiritus}}, \par
\noindent\hangindent=2\parindent et que sous le soleil n’y a rien qui vaille.\par

\astermono

\phantomsection
\label{Eccl.2.12}\noindent\hangindent=2\parindent\pn{12} Donc quand je me mis à considérer \par
\noindent\hangindent=2\parindent tant sagesse que folie et sottise \par
\noindent\hangindent=2\parindent (car \footnote{il n’y a homme pareil à moi en sagesse}y a-t-il homme qui puisse seconder le roi, \par
\noindent\hangindent=2\parindent depuis qu’il a été fait roi ?)\par
\bigbreak
\phantomsection
\label{Eccl.2.13}\noindent\hangindent=2\parindent\pn{13} j’aperçus bien \par
\noindent\hangindent=2\parindent que sagesse est d’autant plus excellente que folie, \par
\noindent\hangindent=2\parindent que la lumière est plus excellente que les ténèbres.\par
\phantomsection
\label{Eccl.2.14}\noindent\hangindent=2\parindent\pn{14} Le sage a des yeux en la tête \par
\noindent\hangindent=2\parindent et le fol chemine en ténèbres. \par
\noindent\hangindent=2\parindent Mais aussi sais-je bien qu’il en prendra à tous à l’un comme à l’autre :\par
\phantomsection
\label{Eccl.2.15}\noindent\hangindent=2\parindent\pn{15} et pourtant je faisais ainsi mon compte : \par
\noindent\hangindent=2\parindent Puis qu’il m’adviendra tout ainsi qu’à un fol, \par
\noindent\hangindent=2\parindent à quoi me sert d’être de tant plus sage ? \par
\noindent\hangindent=2\parindent Si concluait en mon courage, que cela ne vaut encore rien.\par
\bigbreak
\phantomsection
\label{Eccl.2.16}\noindent\hangindent=2\parindent\pn{16} Car non plus d’un sage que d’un fol, \par
\noindent\hangindent=2\parindent la mémoire n’est perdurable, \par
\noindent\hangindent=2\parindent attendu que toutes choses, tant passées qu’à venir, viennent en oubli, \par
\noindent\hangindent=2\parindent et qu’aussi bien meurt sage que fol.\par
\bigbreak
\phantomsection
\label{Eccl.2.17}\noindent\hangindent=2\parindent\pn{17} Et pourtant hais-je la vie  \par
\noindent\hangindent=2\parindent tant me déplaisent les choses qui se font sous le soleil, \par
\noindent\hangindent=2\parindent pour-autant qu’elles ne valent toutes rien, et ne sont qu’une fâcherie d’esprit\footnoteA{\emph{afflictio spiritus}}.\par
\bigbreak
\phantomsection
\label{Eccl.2.18}\noindent\hangindent=2\parindent\pn{18} Aussi hais-je tout ce que par mon travail j’ai fait sous le soleil \par
\noindent\hangindent=2\parindent ce que je laisserai à celui qui viendra après moi.\par
\phantomsection
\label{Eccl.2.19}\noindent\hangindent=2\parindent\pn{19} Et qui sait si celui sera sage ou fol, \par
\noindent\hangindent=2\parindent qui sera maître de tout ce que j’ai, \par
\noindent\hangindent=2\parindent avec tant de peine et sagesse, acquis sous le soleil ? \par
\noindent\hangindent=2\parindent Dont voyant que cela ne valait encore rien,\par
\bigbreak
\phantomsection
\label{Eccl.2.20}\noindent\hangindent=2\parindent\pn{20} je suis venu à avoir en dédain \par
\noindent\hangindent=2\parindent tout ce que j’avais acquis sous le soleil par travail et sagesse.\par
\phantomsection
\label{Eccl.2.21}\noindent\hangindent=2\parindent\pn{21} Car il y en a tel qui travaille \par
\noindent\hangindent=2\parindent avec sagesse et science et devoir, \par
\noindent\hangindent=2\parindent qui laisse son avoir à tel qui n’y a point pris peine : \par
\noindent\hangindent=2\parindent qui est une chose fort mauvaise et qui rien ne vaut.\par
\bigbreak
\phantomsection
\label{Eccl.2.22}\noindent\hangindent=2\parindent\pn{22} Car que sert à un homme \par
\noindent\hangindent=2\parindent tout le travail et fâcherie d’esprit\footnoteA{\emph{afflictio spiritus}} \par
\noindent\hangindent=2\parindent qu’il endure sous le soleil,\par
\phantomsection
\label{Eccl.2.23}\noindent\hangindent=2\parindent\pn{23} vu qu’il ne fait toute sa vie que souffrir peine et tourment \par
\noindent\hangindent=2\parindent chagrigneux, \par
\noindent\hangindent=2\parindent  tellement que son cœur ne repose pas même la nuit ? \par
\noindent\hangindent=2\parindent ce qui ne vaut encore rien.\par
\bigbreak
\phantomsection
\label{Eccl.2.24}\noindent\hangindent=2\parindent\pn{24} Il n’y a autre bien en l’homme que de manger et boire, \par
\noindent\hangindent=2\parindent et se donner du bon temps en son travail, \par
\noindent\hangindent=2\parindent laquelle chose je vois bien qu’elle vient aussi de Dieu\par
\bigbreak
\phantomsection
\label{Eccl.2.25}\noindent\hangindent=2\parindent\pn{25} (car \footnote{je le peux bien savoir vu que j’ai tant de biens acquis par la sagesse que Dieu m’a donnée.}qui est celui qui puisse plus manger ou gourmander que moi ?)\par
\phantomsection
\label{Eccl.2.26}\noindent\hangindent=2\parindent\pn{26} vu qu’aux hommes, qui lui plaisent, il donne sagesse, science et plaisir, \par
\noindent\hangindent=2\parindent et aux mal-vivants donne le tourment d’assembler et amasser pour donner à ceux qui plaisent à Dieu. \par
\noindent\hangindent=2\parindent Ceci n’est encore rien qui vaille, et n’est qu’une fâcherie d’esprit\footnoteA{\emph{afflictio spiritus}}.

\section[{Eccl.3}]{\emph{Eccl}.3}
\renewcommand{\leftmark}{\emph{Eccl}.3}

\phantomsection
\label{Eccl.3.1}\noindent\hangindent=2\parindent\pn{1} Toutes choses ont leur saison, \par
\noindent\hangindent=2\parindent et tout ce qui plaît sous le ciel, à son temps.\par
\phantomsection
\label{Eccl.3.2}\noindent\hangindent=2\parindent\pn{2} Il y a temps de naître, et temps de mourir : \par
\noindent\hangindent=2\parindent temps de planter, et temps d’arracher ce qui est planté :\par
\phantomsection
\label{Eccl.3.3}\noindent\hangindent=2\parindent\pn{3} temps de tuer, et temps de guérir : \par
\noindent\hangindent=2\parindent temps de débâtir, et temps de bâtir :\par
\bigbreak
\phantomsection
\label{Eccl.3.4}\noindent\hangindent=2\parindent\pn{4} temps de pleurer, et temps de rire : \par
\noindent\hangindent=2\parindent temps de mener deuil, et temps de danser :\par
\phantomsection
\label{Eccl.3.5}\noindent\hangindent=2\parindent\pn{5} temps de jeter pierres, et temps d’amasser pierres : \par
\noindent\hangindent=2\parindent temps d’embrasser, et temps de s’en garder :\par
\phantomsection
\label{Eccl.3.6}\noindent\hangindent=2\parindent\pn{6} temps d’acquérir, et temps de perdre : \par
\noindent\hangindent=2\parindent temps de garder, et temps de jeter en voie :\par
\phantomsection
\label{Eccl.3.7}\noindent\hangindent=2\parindent\pn{7} temps de coudre, et temps de découdre : \par
\noindent\hangindent=2\parindent temps de se taire, et temps de parler :\par
\phantomsection
\label{Eccl.3.8}\noindent\hangindent=2\parindent\pn{8} temps d’aimer, et temps de haïr : \par
\noindent\hangindent=2\parindent temps de guerre, et temps de paix.\par
\phantomsection
\label{Eccl.3.9}\noindent\hangindent=2\parindent\pn{9} Que vaut le travail à un qui fait quelque chose ?\par
\bigbreak
\phantomsection
\label{Eccl.3.10}\noindent\hangindent=2\parindent\pn{10} Je vois le méchef\footnoteA{« malheur »} que Dieu a donné \par
\noindent\hangindent=2\parindent à la race des hommes pour les tourmenter.\par
\phantomsection
\label{Eccl.3.11}\noindent\hangindent=2\parindent\pn{11} Il fait bien tout en son temps, \par
\noindent\hangindent=2\parindent et leur a tellement mis la vie\footnoteA{Ailleurs, Castellion traduit le même mot hébreu par « à tout jamais » (Ps.28.9), « toujours » (Ps.41.13), ou « de tous temps en tous temps » (ps.106.48) ; comme c’est entendu désormais, et déjà dans \emph{les Septantes} (qui met \foreign{aion}, \emph{l’éternité}, dans le cœur des hommes). Il n’a en tous cas pas repris l’interprétation de \emph{la Vulgate} \emph{« et \emph{mundum} tradidit disputationi eorum »}, qui dans les mots de Sacy donne cet étrange : \emph{« et il a livré le \emph{monde} à leurs disputes »}, ou bien dans Calvin-Bèze 1588 : « aussi a-t-il mis le \emph{monde} dans leur cœur ». Ce verset équivoque concentre tout le paradoxe d’un Dieu bon, qui fait tout au mieux et à temps, sans que l’on puisse en pénétrer la raison (\hyperref[Eccl.8.17]{\dotuline{8.17}}), car notre désir est infini, cet infini que Dieu a inscrit en nous et qui nous rend à jamais insatisfait. En choisissant de mettre \emph{la vie} au cœur des humains, Castellion évite les spéculations abstraites et ose une interprétation profondément humaniste de l’éternité. C’est grâce à Dieu que nous voulons vivre éternellement, et c’est donc à cause de Dieu que nous ne pouvons pas accepter que notre fin est dans l’ordre des choses.} au cœur, \par
\noindent\hangindent=2\parindent que \footnote{de toute leur vie.}depuis le commencement jusqu’à la fin \par
\noindent\hangindent=2\parindent les hommes ne peuvent trouver \footnote{par quelle cause et raison il fait ce qu’il fait.}que c’est que fait Dieu.\par
\bigbreak
\phantomsection
\label{Eccl.3.12}\noindent\hangindent=2\parindent\pn{12} Je sais qu’il n’y a point de bien en eux, \par
\noindent\hangindent=2\parindent sinon qu’ils s’éjouissent et fassent bien en leur vie.\par
\phantomsection
\label{Eccl.3.13}\noindent\hangindent=2\parindent\pn{13} Voire ce que tout homme banquette, \par
\noindent\hangindent=2\parindent et parmi tout son travail jouit du bien, \par
\noindent\hangindent=2\parindent c’est un don de Dieu.\par
\phantomsection
\label{Eccl.3.14}\noindent\hangindent=2\parindent\pn{14} Je sais que tout ce que fait Dieu, est perdurable\footnoteA{de \emph{perdurer} « éternel »}, \par
\noindent\hangindent=2\parindent et n’y faut ajouter ni ôter : \par
\noindent\hangindent=2\parindent or Dieu se fait craindre.\par
\bigbreak
\phantomsection
\label{Eccl.3.15}\noindent\hangindent=2\parindent\pn{15} Ce qui a été, est déjà : et ce qui sera  a déjà été, \par
\noindent\hangindent=2\parindent et Dieu rappelle ce qui a été chassé.\par

\astermono

\phantomsection
\label{Eccl.3.16}\noindent\hangindent=2\parindent\pn{16} D’avantage voyant que sous le soleil \par
\noindent\hangindent=2\parindent en lieu de droit et justice, il y a méchanceté et injustice,\par
\phantomsection
\label{Eccl.3.17}\noindent\hangindent=2\parindent\pn{17} je pense en moi-même \par
\noindent\hangindent=2\parindent que Dieu jugera et les justes et les injustes : \par
\noindent\hangindent=2\parindent car tout bandon\footnoteA{de \emph{ban} « gouvernement, justice »} et œuvres \par
\noindent\hangindent=2\parindent auront une fois leur temps.\par
\bigbreak
\phantomsection
\label{Eccl.3.18}\noindent\hangindent=2\parindent\pn{18} Je pense en moi-même, \par
\noindent\hangindent=2\parindent touchant les hommes, \par
\noindent\hangindent=2\parindent que Dieu les a tellement destinés, \par
\noindent\hangindent=2\parindent qu’il semble qu’ils soient bêtes.\par
\phantomsection
\label{Eccl.3.19}\noindent\hangindent=2\parindent\pn{19} Car il en prend tout ainsi d’un homme que d’une bête : \par
\noindent\hangindent=2\parindent comme elle meurt, aussi fait-il, \par
\noindent\hangindent=2\parindent et ont tous deux un même esprit\footnoteA{« souffle de vie »}, \par
\noindent\hangindent=2\parindent et n’y a rien en quoi l’homme surmonte la bête, \par
\noindent\hangindent=2\parindent vu que tous deux ne valent rien.\par
\phantomsection
\label{Eccl.3.20}\noindent\hangindent=2\parindent\pn{20} Tous deux s’en vont en un même lieu : \par
\noindent\hangindent=2\parindent tous deux sont venus de poudre\footnoteA{« poussière, cendre »}, \par
\noindent\hangindent=2\parindent et tous deux re-vont en poudre.\par
\phantomsection
\label{Eccl.3.21}\noindent\hangindent=2\parindent\pn{21} Qui sait si l’esprit de la race des hommes \par
\noindent\hangindent=2\parindent monte en haut ? \par
\noindent\hangindent=2\parindent ou si l’esprit d’une bête \par
\noindent\hangindent=2\parindent descend dessous terre ?\par
\bigbreak
\phantomsection
\label{Eccl.3.22}\noindent\hangindent=2\parindent\pn{22} Je vois bien qu’il n’y a point de bien, \par
\noindent\hangindent=2\parindent sinon que l’homme se réjouisse en ses œuvres : car c’est-ce qu’il en a\footnoteA{« sa part »}. \par
\noindent\hangindent=2\parindent Car qui l’amènera à-savoir ce qui sera après lui ?

\section[{Eccl.4}]{\emph{Eccl}.4}
\renewcommand{\leftmark}{\emph{Eccl}.4}

\phantomsection
\label{Eccl.4.1}\noindent\hangindent=2\parindent\pn{1} Derechef voyant tant de torts \par
\noindent\hangindent=2\parindent qui se font sous le soleil, \par
\noindent\hangindent=2\parindent et les larmes de ceux auxquels on fait tort, \par
\noindent\hangindent=2\parindent lesquels nul ne console : \par
\noindent\hangindent=2\parindent on leur fait tort par force, \par
\noindent\hangindent=2\parindent et nul ne les console :\par
\phantomsection
\label{Eccl.4.2}\noindent\hangindent=2\parindent\pn{2} je prise plus les morts qui sont déjà morts, \par
\noindent\hangindent=2\parindent que les vifs qui sont encore en vie :\par
\phantomsection
\label{Eccl.4.3}\noindent\hangindent=2\parindent\pn{3} et si estime encore plus que les uns ni les autres, \par
\noindent\hangindent=2\parindent ceux qui ne sont pas encore, \par
\noindent\hangindent=2\parindent lesquels ne voient pas les mauvaises choses \par
\noindent\hangindent=2\parindent qui se font sous le soleil.\par

\astermono

\phantomsection
\label{Eccl.4.4}\noindent\hangindent=2\parindent\pn{4} Item je vois que tout le travail \par
\noindent\hangindent=2\parindent et devoir de ce qu’on fait, \par
\noindent\hangindent=2\parindent n’est autre chose qu’envie des uns contre les autres : \par
\noindent\hangindent=2\parindent ce qui ne vaut encore rien, et n’est qu’une fâcherie d’esprit\footnoteA{\emph{afflictio spiritus}}.\par
\bigbreak
\phantomsection
\label{Eccl.4.5}\noindent\hangindent=2\parindent\pn{5} Un fol \footnote{est paresseux et meurt de faim.}plie ses mains, \par
\noindent\hangindent=2\parindent et mange sa propre chair.\par
\phantomsection
\label{Eccl.4.6}\noindent\hangindent=2\parindent\pn{6} Mieux vaut une pognée\footnoteA{« pogne, main »} en repos, \par
\noindent\hangindent=2\parindent qu’une havée\footnoteA{« une poignée, redevance due au marché »} avec peine et fâcherie d’esprit\footnoteA{\emph{afflictio spiritus}}.\par

\astermono

\phantomsection
\label{Eccl.4.7}\noindent\hangindent=2\parindent\pn{7} Derechef je vois sous le soleil une chose qui rien ne vaut,\par
\phantomsection
\label{Eccl.4.8}\noindent\hangindent=2\parindent\pn{8} qu’il y en a tel qui est tout seul sans hoir\footnoteA{« héritier »}, \par
\noindent\hangindent=2\parindent voire sans fils ni frère, \par
\noindent\hangindent=2\parindent  qui néanmoins ne cesse jamais de travailler, \par
\noindent\hangindent=2\parindent et n’a jamais \footnote{convoitise}l’œil soul de richesses : \par
\noindent\hangindent=2\parindent et \footnote{il devrait penser pour qui, etc.}pour qui travaille-je, \par
\noindent\hangindent=2\parindent et ne mange pas demi mon soul ? \par
\noindent\hangindent=2\parindent ce qui ne vaut encore rien, et est un mauvais tourment.\par
\bigbreak
\phantomsection
\label{Eccl.4.9}\noindent\hangindent=2\parindent\pn{9} Mieux valent deux qu’un, \par
\noindent\hangindent=2\parindent et sont bien récompensés de leur peine.\par
\phantomsection
\label{Eccl.4.10}\noindent\hangindent=2\parindent\pn{10} Car s’ils tombent, ils se lèveront l’un l’autre. \par
\noindent\hangindent=2\parindent Mais il fait mal être seul : \par
\noindent\hangindent=2\parindent car s’il tombe, il n’a personne pour le lever.\par
\phantomsection
\label{Eccl.4.11}\noindent\hangindent=2\parindent\pn{11} D’avantage si deux couchent ensemble, ils s’échauffent : \par
\noindent\hangindent=2\parindent mais un comment s’échauffera-t-il ?\par
\phantomsection
\label{Eccl.4.12}\noindent\hangindent=2\parindent\pn{12} et si l’un est vaincu, \par
\noindent\hangindent=2\parindent les deux tiendront bon, \par
\noindent\hangindent=2\parindent et ne se rompt pas tôt une corde à trois cordons.\par

\astermono

\phantomsection
\label{Eccl.4.13}\noindent\hangindent=2\parindent\pn{13} Mieux vaut un enfant bien appris et sage, \par
\noindent\hangindent=2\parindent que ne fait un roi vieux et fol, lequel ne saurait plus être endoctriné\footnoteA{« instruire »}.\par
\phantomsection
\label{Eccl.4.14}\noindent\hangindent=2\parindent\pn{14} Car tel sort de prison, qui devient roi : \par
\noindent\hangindent=2\parindent et tel est né roi, qui devient pauvre.\par
\phantomsection
\label{Eccl.4.15}\noindent\hangindent=2\parindent\pn{15} J’ai autrefois vu tous les vivants qui se tiennent sous le soleil, accompagner un enfant second, \par
\noindent\hangindent=2\parindent qui devait être hoir de son père,\par
\phantomsection
\label{Eccl.4.16}\noindent\hangindent=2\parindent\pn{16} tellement que tant de gens allaient devant et après lui, que c’était une chose infinie, \par
\noindent\hangindent=2\parindent et si \footnote{il ne venait pas à être \emph{hoir} de son père.}ne venaient point à s’en réjouir : \par
\noindent\hangindent=2\parindent ce qui ne vaut encore rien, et n’est qu’une fâcherie d’esprit\footnoteA{\emph{afflictio spiritus}}.\par

\astermono

\phantomsection
\label{Eccl.4.17}\noindent\hangindent=2\parindent\pn{17} \footnote{porte-toi sagement.}Garde tes pieds quand tu t’en vas en la maison Dieu, \par
\noindent\hangindent=2\parindent et t’avance plus pour ouïr que pour offrir sacrifice de fols : \par
\noindent\hangindent=2\parindent car \footnote{leurs sacrifices déplaisent à Dieu.}ils ne savent pas le mal qu’ils font.

\section[{Eccl.5}]{\emph{Eccl}.5}
\renewcommand{\leftmark}{\emph{Eccl}.5}

\phantomsection
\label{Eccl.5.1}\noindent\hangindent=2\parindent\pn{1} Ne te hâte point légèrement de prononcer paroles de bouche, \par
\noindent\hangindent=2\parindent ou les tirer de ton cœur, \par
\noindent\hangindent=2\parindent devant Dieu : \par
\noindent\hangindent=2\parindent car Dieu est au ciel, et tu es en terre : \par
\noindent\hangindent=2\parindent et pourtant parle peu :\par
\phantomsection
\label{Eccl.5.2}\noindent\hangindent=2\parindent\pn{2} car trop grand souci fait songer, \par
\noindent\hangindent=2\parindent et trop parler fait dire quelque sot propos.\par
\bigbreak
\phantomsection
\label{Eccl.5.3}\noindent\hangindent=2\parindent\pn{3} Quand tu auras fait vœu à Dieu, \par
\noindent\hangindent=2\parindent ne faut\footnoteA{faillir} point à le rendre : \par
\noindent\hangindent=2\parindent car \footnote{\emph{Deut}.23.22, ceux qui font vœu sans le rendre.}les fols ne sont point agréables : \par
\noindent\hangindent=2\parindent rend ce que tu auras voué.\par
\bigbreak
\phantomsection
\label{Eccl.5.4}\noindent\hangindent=2\parindent\pn{4} Il vaut mieux que tu ne voues\footnoteA{« faire un vœu »} point, \par
\noindent\hangindent=2\parindent que de vouer sans rendre.\par
\phantomsection
\label{Eccl.5.5}\noindent\hangindent=2\parindent\pn{5} N’emploie pas \footnote{à-savoir en vouant.}ta bouche pour endommager toi-même, \par
\noindent\hangindent=2\parindent et ne dis pas devant l’ange que c’est par mégarde, \par
\noindent\hangindent=2\parindent  de peur que Dieu n’ait dépit de ta parole, \par
\noindent\hangindent=2\parindent et ne renverse tes affaires.\par
\phantomsection
\label{Eccl.5.6}\noindent\hangindent=2\parindent\pn{6} Car en beaucoup de paroles y a beaucoup de \footnote{sottises}songes, \par
\noindent\hangindent=2\parindent et propos qui rien ne valent : et pourtant craint Dieu.\par

\astermono

\phantomsection
\label{Eccl.5.7}\noindent\hangindent=2\parindent\pn{7} Si tu vois qu’en une province on fasse tort aux pauvres, \par
\noindent\hangindent=2\parindent et qu’on force\footnoteA{« soumettre par la violence »} droit et justice, \par
\noindent\hangindent=2\parindent ne t’ébahit pas d’un tel bandon\footnoteA{de \emph{ban} « gouvernement, justice »} : \par
\noindent\hangindent=2\parindent car \footnote{il y a tant d’officiers sur officiers, que le roi ne peut pourvoir a tout.}il y a des officiers qui prennent garde sur les autres officiers, \par
\noindent\hangindent=2\parindent  et eux-mêmes sont encore sujets à des autres,\par
\phantomsection
\label{Eccl.5.8}\noindent\hangindent=2\parindent\pn{8} et le roi de la contrée qui est cultivée, \par
\noindent\hangindent=2\parindent est par-dessus tous ceux du pays.\par

\astermono

\phantomsection
\label{Eccl.5.9}\noindent\hangindent=2\parindent\pn{9} Qui argent aime, jamais d’argent ne soule : \par
\noindent\hangindent=2\parindent et qui aime richesses n’a point de profit, \par
\noindent\hangindent=2\parindent ce qui ne vaut encore rien.\par
\bigbreak
\phantomsection
\label{Eccl.5.10}\noindent\hangindent=2\parindent\pn{10} A force biens, force mangeurs : \par
\noindent\hangindent=2\parindent et n’en a le maître autre profit \par
\noindent\hangindent=2\parindent que la vue.\par
\phantomsection
\label{Eccl.5.11}\noindent\hangindent=2\parindent\pn{11} Un qui travaille, dort à son aise, \par
\noindent\hangindent=2\parindent soit qu’il mange peu, ou \footnote{\emph{Job}.20.22, car le travail lui fait faire digestion.}prou : \par
\noindent\hangindent=2\parindent mais quand un riche mange son soul, \par
\noindent\hangindent=2\parindent cela \footnote{à cause qu’il ne travaille point.}le garde de dormir.\par
\bigbreak
\phantomsection
\label{Eccl.5.12}\noindent\hangindent=2\parindent\pn{12} Il y a un mauvais vice que je vois sous le soleil, \par
\noindent\hangindent=2\parindent c’est des richesses qui sont gardées à leur maitre pour son mal,\par
\phantomsection
\label{Eccl.5.13}\noindent\hangindent=2\parindent\pn{13} lesquelles richesses périssent le plus misérablement du monde, \par
\noindent\hangindent=2\parindent vu qu’il a engendré un fils qui n’aura rien :\par
\phantomsection
\label{Eccl.5.14}\noindent\hangindent=2\parindent\pn{14} \footnote{\emph{Job}.1.21, \emph{Tim}.6.7.}et tout ainsi qu’il est sorti tout nu du ventre de sa mère, \par
\noindent\hangindent=2\parindent il retourne comme il était venu, \par
\noindent\hangindent=2\parindent sans rien emporter de sa peine \par
\noindent\hangindent=2\parindent pour lui tenir compagnie :\par
\phantomsection
\label{Eccl.5.15}\noindent\hangindent=2\parindent\pn{15} ce qui est aussi un mauvais vice, \par
\noindent\hangindent=2\parindent vu qu’il s’en va tout ainsi qu’il était venu, \par
\noindent\hangindent=2\parindent sans avoir rien gagné d’avoir travaillé au vent.\par
\phantomsection
\label{Eccl.5.16}\noindent\hangindent=2\parindent\pn{16} Je me tais que toute sa vie il mange en ténèbres, \par
\noindent\hangindent=2\parindent en maint chagrin, maladie, et dépit.\par
\phantomsection
\label{Eccl.5.17}\noindent\hangindent=2\parindent\pn{17} Et pourtant ce que je vois de bon et beau, \par
\noindent\hangindent=2\parindent c’est qu’il mange et boive, \par
\noindent\hangindent=2\parindent et que toute sa vie, parmi toute la peine \par
\noindent\hangindent=2\parindent qu’il endure sous le soleil, \par
\noindent\hangindent=2\parindent il fasse bonne chère des biens que Dieu lui a donnés : \par
\noindent\hangindent=2\parindent car c’est son parti.\par
\phantomsection
\label{Eccl.5.18}\noindent\hangindent=2\parindent\pn{18} Et de vrai, à tout homme que Dieu donne richesses et chevance, \par
\noindent\hangindent=2\parindent et lui donne puissance d’en banqueter, \par
\noindent\hangindent=2\parindent et emporter sa pièce, et jouir de son travail, \par
\noindent\hangindent=2\parindent c’est un don de Dieu.\par
\phantomsection
\label{Eccl.5.19}\noindent\hangindent=2\parindent\pn{19} Car il ne lui souvient guère \footnote{de ses maux.}du temps de sa vie, \par
\noindent\hangindent=2\parindent puis-que Dieu  lui octroie joie de cœur.

\section[{Eccl.6}]{\emph{Eccl}.6}
\renewcommand{\leftmark}{\emph{Eccl}.6}

\phantomsection
\label{Eccl.6.1}\noindent\hangindent=2\parindent\pn{1} Il y a un mal que je vois sous le soleil, \par
\noindent\hangindent=2\parindent voire qui se trouve coutumièrement entre les hommes,\par
\phantomsection
\label{Eccl.6.2}\noindent\hangindent=2\parindent\pn{2} c’est qu’il y en a tel, à qui Dieu donne tant de richesses, chevance et honneur, \par
\noindent\hangindent=2\parindent qu’il ne saurait souhaiter chose qu’il n’ait, \par
\noindent\hangindent=2\parindent et si ne lui donne pas Dieu puissance d’en manger, \par
\noindent\hangindent=2\parindent mais en mange un qui ne lui est rien : \par
\noindent\hangindent=2\parindent ce qui ne vaut rien, et est une mauvaise faute.\par
\bigbreak
\phantomsection
\label{Eccl.6.3}\noindent\hangindent=2\parindent\pn{3} Si quelqu’un engendre bien cent enfants, \par
\noindent\hangindent=2\parindent et qu’il vive beaucoup d’ans, \par
\noindent\hangindent=2\parindent et que non seulement il ne soûle point son appétit de biens, \par
\noindent\hangindent=2\parindent mais même ne soit point enterré, \par
\noindent\hangindent=2\parindent je dis que son cas se porte plus mal, que d’un avorton\footnoteA{« enfant mort-né »}.\par
\phantomsection
\label{Eccl.6.4}\noindent\hangindent=2\parindent\pn{4} Car un avorton qui est venu pour néant, \par
\noindent\hangindent=2\parindent et s’en va en ténèbres, \par
\noindent\hangindent=2\parindent et est son nom couvert de ténèbres,\par
\phantomsection
\label{Eccl.6.5}\noindent\hangindent=2\parindent\pn{5} et ne vit ni ne connut même le soleil, \par
\noindent\hangindent=2\parindent est plus en repos qu’un tel homme.\par
\phantomsection
\label{Eccl.6.6}\noindent\hangindent=2\parindent\pn{6} Mais un tel homme, quand bien il aurait vécu mille et autres mille ans, \par
\noindent\hangindent=2\parindent s’il n’a joui des biens, \par
\noindent\hangindent=2\parindent ne s’en vont-ils pas tous deux en un même lieu ?\par

\astermono

\phantomsection
\label{Eccl.6.7}\noindent\hangindent=2\parindent\pn{7} Toute la peine que prend l’homme, sert à sa bouche, \par
\noindent\hangindent=2\parindent et si a un appétit qui n’est jamais plein.\par
\phantomsection
\label{Eccl.6.8}\noindent\hangindent=2\parindent\pn{8} Car de combien vaut mieux un sage qu’un fol ? \par
\noindent\hangindent=2\parindent ou un humble qui se sait bien gouverner entre les vivants ?\par
\phantomsection
\label{Eccl.6.9}\noindent\hangindent=2\parindent\pn{9} Mieux vaut \footnote{bien présent qu’espéré.}vue d’œil, qu’attente de cœur : \par
\noindent\hangindent=2\parindent ce qui ne vaut encore rien, et est une fâcherie d’esprit\footnoteA{\emph{afflictio spiritus}}.\par
\bigbreak
\phantomsection
\label{Eccl.6.10}\noindent\hangindent=2\parindent\pn{10} Celui qui a été, est déjà nommé, \par
\noindent\hangindent=2\parindent et sait-on bien qu’il a été homme, \par
\noindent\hangindent=2\parindent et n’a pu combattre \par
\noindent\hangindent=2\parindent \footnote{contre la mort.}plus fort que soi.\par
\phantomsection
\label{Eccl.6.11}\noindent\hangindent=2\parindent\pn{11} Donc puis qu’il y a tant de choses, \par
\noindent\hangindent=2\parindent qui font que tout ne vaut rien, \par
\noindent\hangindent=2\parindent que gagne l’homme ?\par
\bigbreak
\phantomsection
\label{Eccl.6.12}\noindent\hangindent=2\parindent\pn{12} Car qui sait que c’est qui est bon à l’homme, \par
\noindent\hangindent=2\parindent tous les jours de sa vie tant néante, \par
\noindent\hangindent=2\parindent lesquels il passe comme une ombre ? \par
\noindent\hangindent=2\parindent Et qui fera savoir à un homme \par
\noindent\hangindent=2\parindent ce qui sera après lui sous le soleil ?

\section[{Eccl.7}]{\emph{Eccl}.7}
\renewcommand{\leftmark}{\emph{Eccl}.7}

\phantomsection
\label{Eccl.7.1}\noindent\hangindent=2\parindent\pn{1} Mieux vaut bonne renommée, que bonne eau de senteur, \par
\noindent\hangindent=2\parindent et jour de mort, que de naissance.\par
\phantomsection
\label{Eccl.7.2}\noindent\hangindent=2\parindent\pn{2} Mieux  vaut aller en maison de deuil, \par
\noindent\hangindent=2\parindent qu’en maison de banquets : \par
\noindent\hangindent=2\parindent en la maison qui est la fin à tous hommes, \par
\noindent\hangindent=2\parindent qu’en celle qui leur met la vie au cœur.\par
\phantomsection
\label{Eccl.7.3}\noindent\hangindent=2\parindent\pn{3} Mieux vaut chagrin que ris : \par
\noindent\hangindent=2\parindent car de triste visage vient joie de cœur.\par
\phantomsection
\label{Eccl.7.4}\noindent\hangindent=2\parindent\pn{4} cœur de sage est en maison de deuil : \par
\noindent\hangindent=2\parindent et cœur de fol, en maison de joie.\par
\bigbreak
\phantomsection
\label{Eccl.7.5}\noindent\hangindent=2\parindent\pn{5} Mieux vaut ouïr tancer\footnoteA{« disputer »} un sage, \par
\noindent\hangindent=2\parindent que chanter un fol.\par
\phantomsection
\label{Eccl.7.6}\noindent\hangindent=2\parindent\pn{6} Car bruit d’épines sous un pot\footnoteA{L’image ne nous est plus familière, on peut supposer un feu d’épine qui crépite, qui s’épuise vite et ne chauffe pas.}, \par
\noindent\hangindent=2\parindent et ris de fol, c’est tout un. \par
\noindent\hangindent=2\parindent Item ceci ne vaut rien :\par
\phantomsection
\label{Eccl.7.7}\noindent\hangindent=2\parindent\pn{7} c’est que \footnote{un présent qui se donne pour faire tort a quelqu’un.}tort affole un sage, \par
\noindent\hangindent=2\parindent et les dons mettent un homme hors du sens.\par
\bigbreak
\phantomsection
\label{Eccl.7.8}\noindent\hangindent=2\parindent\pn{8} Mieux vaut la fin d’une chose, que son commencement : \par
\noindent\hangindent=2\parindent mieux vaut tardif\footnoteA{« lent »}, que hautain courage.\par
\phantomsection
\label{Eccl.7.9}\noindent\hangindent=2\parindent\pn{9} Ne sois point léger de courage à te dépiter : \par
\noindent\hangindent=2\parindent car en sein de fol, loge dépit.\par
\phantomsection
\label{Eccl.7.10}\noindent\hangindent=2\parindent\pn{10} Ne demande point pourquoi \par
\noindent\hangindent=2\parindent c’est que le temps passé a été meilleur que le présent : \par
\noindent\hangindent=2\parindent car c’est mal sagement \par
\noindent\hangindent=2\parindent demandé à toi.\par
\bigbreak
\phantomsection
\label{Eccl.7.11}\noindent\hangindent=2\parindent\pn{11} Mieux vaut sagesse qu’héritage, \par
\noindent\hangindent=2\parindent et est plus profitable \footnote{aux vivants.}à ceux qui voient le soleil.\par
\phantomsection
\label{Eccl.7.12}\noindent\hangindent=2\parindent\pn{12} Car s’il est question du secours qui gît en sagesse, \par
\noindent\hangindent=2\parindent et de celui qui gît en argent, \par
\noindent\hangindent=2\parindent la science et sagesse est d’autant plus profitable, \par
\noindent\hangindent=2\parindent qu’elle sauve la vie à son maître.\par
\phantomsection
\label{Eccl.7.13}\noindent\hangindent=2\parindent\pn{13} Regarde l’ouvrage de Dieu, \par
\noindent\hangindent=2\parindent qui est tel, que ce qu’il courbe, nul ne peut dresser.\par
\phantomsection
\label{Eccl.7.14}\noindent\hangindent=2\parindent\pn{14} Quand tu as bon temps donne-toi tellement de bon temps, \par
\noindent\hangindent=2\parindent que tu regardes le mauvais temps : \par
\noindent\hangindent=2\parindent car Dieu a fait l’un accompagné de l’autre, \par
\noindent\hangindent=2\parindent à celle fin que l’homme \footnote{sache qu’en ce monde n’y a rien de certain, et pourtant sois appareillé a toutes aventures.}n’y sache rien trouver.\par

\astermono

\phantomsection
\label{Eccl.7.15}\noindent\hangindent=2\parindent\pn{15} Je vois tout en mon âge, pour néant qu’il soit : \par
\noindent\hangindent=2\parindent il y a tel innocent, qui périt en son innocence : \par
\noindent\hangindent=2\parindent et y a tel méchant qui dure en sa mauvaitie\footnoteA{« malice »}.\par
\phantomsection
\label{Eccl.7.16}\noindent\hangindent=2\parindent\pn{16} Ne sois ni trop innocent, \par
\noindent\hangindent=2\parindent ni trop sage, \par
\noindent\hangindent=2\parindent de peur que tu ne sois détruit.\par
\phantomsection
\label{Eccl.7.17}\noindent\hangindent=2\parindent\pn{17} Ne sois ni trop méchant, \par
\noindent\hangindent=2\parindent ni trop fol, \par
\noindent\hangindent=2\parindent de peur que tu ne meures devant ton temps.\par
\phantomsection
\label{Eccl.7.18}\noindent\hangindent=2\parindent\pn{18} Il est bon que tu tiennes ceci, \par
\noindent\hangindent=2\parindent voire sans le lâcher de ta main : \par
\noindent\hangindent=2\parindent car de tout échappe qui craint Dieu.\par
\phantomsection
\label{Eccl.7.19}\noindent\hangindent=2\parindent\pn{19} La sagesse assure dix fois plus un sage, \par
\noindent\hangindent=2\parindent que d’être le principal d’une ville.\par
\bigbreak
\phantomsection
\label{Eccl.7.20}\noindent\hangindent=2\parindent\pn{20} \footnote{\emph{1Rois}.8,46 ; \emph{2Chr.}6.36 ; \emph{Prov}.20.9 ; 1Jehan.1.9-10}Car il n’y a au monde homme si juste, \par
\noindent\hangindent=2\parindent qu’il fasse si bien qu’il ne pêche.\par
\phantomsection
\label{Eccl.7.21}\noindent\hangindent=2\parindent\pn{21} N’applique aussi point ton cœur à tous les propos qu’on tient, \par
\noindent\hangindent=2\parindent de peur que tu ne t’oyes maudire par ton serviteur.\par
\phantomsection
\label{Eccl.7.22}\noindent\hangindent=2\parindent\pn{22} Car tu sais bien que mainte-fois toi-même \par
\noindent\hangindent=2\parindent as bien maudit les autres.\par

\astermono

\phantomsection
\label{Eccl.7.23}\noindent\hangindent=2\parindent\pn{23} J’ai essayé tout ceci par sagesse, \par
\noindent\hangindent=2\parindent tâchant de devenir sage : \par
\noindent\hangindent=2\parindent mais j’en suis bien loin.\par
\phantomsection
\label{Eccl.7.24}\noindent\hangindent=2\parindent\pn{24} C’est une chose si très-loin \par
\noindent\hangindent=2\parindent et si très-profonde, qu’on n’en saurait venir à bout.\par
\phantomsection
\label{Eccl.7.25}\noindent\hangindent=2\parindent\pn{25} Quand je tourne mon cœur \par
\noindent\hangindent=2\parindent pour savoir, examiner, et chercher \par
\noindent\hangindent=2\parindent sagesse et raison, \par
\noindent\hangindent=2\parindent et pour savoir la méchanceté des fols, \par
\noindent\hangindent=2\parindent et la sottise des forcenés,\par
\bigbreak
\phantomsection
\label{Eccl.7.26}\noindent\hangindent=2\parindent\pn{26} je trouve que la femme est plus amère que la mort : \par
\noindent\hangindent=2\parindent de laquelle femme le cœur \par
\noindent\hangindent=2\parindent sont filets et rets, et les mains sont liens, \par
\noindent\hangindent=2\parindent dont qui est en la grâce de Dieu, en échappe : \par
\noindent\hangindent=2\parindent mais qui est méchant, y est pris.\par
\phantomsection
\label{Eccl.7.27}\noindent\hangindent=2\parindent\pn{27} Voila que j’ai trouvé (dit le prêcheur) \par
\noindent\hangindent=2\parindent en cherchant raison de point en point,\par
\phantomsection
\label{Eccl.7.28}\noindent\hangindent=2\parindent\pn{28} laquelle je cherche encore de mon esprit, et ne l’ai pas trouvée. \par
\noindent\hangindent=2\parindent J’ai trouvé \footnote{un vrai homme est tel qu’il doit être.}un homme entre mille : \par
\noindent\hangindent=2\parindent mais entre toutes les femmes, \par
\noindent\hangindent=2\parindent je n’en ai pas trouvé une.\par
\phantomsection
\label{Eccl.7.29}\noindent\hangindent=2\parindent\pn{29} D’avantage voici que j’ai trouvé : \par
\noindent\hangindent=2\parindent c’est que Dieu fit l’homme droit, \par
\noindent\hangindent=2\parindent mais \footnote{les hommes ont été cause de leur malheur, quand ils ont voulu savoir bien et mal.}on a cherché beaucoup de raisons.

\section[{Eccl.8}]{\emph{Eccl}.8}
\renewcommand{\leftmark}{\emph{Eccl}.8}

\phantomsection
\label{Eccl.8.1}\noindent\hangindent=2\parindent\pn{1} Qui est à comparer à un sage ? \par
\noindent\hangindent=2\parindent et qui sait déchiffrer les matières ? \par
\noindent\hangindent=2\parindent La sagesse d’un homme illumine son visage, \par
\noindent\hangindent=2\parindent et lui ôte sa faroucheté.\par
\phantomsection
\label{Eccl.8.2}\noindent\hangindent=2\parindent\pn{2} \footnote{\emph{Prov}.17.24}Je te conseille de prendre garde à la bouche du roi, \par
\noindent\hangindent=2\parindent et d’avoir égard au serment de dieu.\par
\phantomsection
\label{Eccl.8.3}\noindent\hangindent=2\parindent\pn{3} Ne \footnote{t’étrange}t’en va pas légèrement de devant lui : \par
\noindent\hangindent=2\parindent ne persévère pas en mauvaise chose : \par
\noindent\hangindent=2\parindent car tout ce qu’il lui plaît, il fait.\par
\phantomsection
\label{Eccl.8.4}\noindent\hangindent=2\parindent\pn{4} En parole de roi gît quant-et-quant puissance, \par
\noindent\hangindent=2\parindent tellement qu’il n’y a celui qui lui demande raison de ce qu’il fait.\par

\astermono

\phantomsection
\label{Eccl.8.5}\noindent\hangindent=2\parindent\pn{5} Qui exécute ce qui lui est commandé, \par
\noindent\hangindent=2\parindent se garde de malencontre\footnoteA{« mal-heur »} : \par
\noindent\hangindent=2\parindent et cœur sage connaît temps et raison :\par
\phantomsection
\label{Eccl.8.6}\noindent\hangindent=2\parindent\pn{6} car tout ce qui plaît, à temps et raison, \par
\noindent\hangindent=2\parindent pourtant que l’homme endure beaucoup de maux :\par
\phantomsection
\label{Eccl.8.7}\noindent\hangindent=2\parindent\pn{7} à cause qu’il ne sait ce qui est à venir : \par
\noindent\hangindent=2\parindent car qui lui donnera à connaître l’avenir ?\par
\phantomsection
\label{Eccl.8.8}\noindent\hangindent=2\parindent\pn{8} Ainsi qu’un homme ne  peut être maître du vent \par
\noindent\hangindent=2\parindent et l’atenir\footnoteA{« retenir »}, \par
\noindent\hangindent=2\parindent ni faire à sa guise du jour de la mort, \par
\noindent\hangindent=2\parindent ni jouir de la guerre, \par
\noindent\hangindent=2\parindent ainsi ne peut méchanceté délivrer son maître.\par

\astermono

\phantomsection
\label{Eccl.8.9}\noindent\hangindent=2\parindent\pn{9} Tout ceci ai-je vu, et ai appliqué mon cœur \par
\noindent\hangindent=2\parindent à toutes les choses qui se font sous le soleil, \par
\noindent\hangindent=2\parindent ce-pendant que les hommes sont maîtres \par
\noindent\hangindent=2\parindent les uns des autres à leur dommage,\par
\phantomsection
\label{Eccl.8.10}\noindent\hangindent=2\parindent\pn{10} Aussi ai-je vu des méchants qui étaient enterrés, \par
\noindent\hangindent=2\parindent et s’en étaient allés, et délogés \footnote{de Jérusalem.}du saint lieu, \par
\noindent\hangindent=2\parindent qui avaient bon bruit\footnoteA{« bonne réputation »} en la ville, en laquelle ils avaient ainsi vécu : \par
\noindent\hangindent=2\parindent et cela ne vaut encore rien.\par
\bigbreak
\phantomsection
\label{Eccl.8.11}\noindent\hangindent=2\parindent\pn{11} Pourtant que les malfaisants \par
\noindent\hangindent=2\parindent ne sont pas incontinent justiciés, \par
\noindent\hangindent=2\parindent la race des hommes a le cœur totalement prompt à malfaire.\par
\phantomsection
\label{Eccl.8.12}\noindent\hangindent=2\parindent\pn{12} Mais combien que les mauvais fassent cent fois mal, \par
\noindent\hangindent=2\parindent et néanmoins durent, \par
\noindent\hangindent=2\parindent si sais-je bien \par
\noindent\hangindent=2\parindent que de ceux qui ont la crainte et révérence de Dieu, \par
\noindent\hangindent=2\parindent leur cas se portera bien :\par
\phantomsection
\label{Eccl.8.13}\noindent\hangindent=2\parindent\pn{13} et celui des méchants ne se portera pas bien, \par
\noindent\hangindent=2\parindent et ne vivront pas si long âge, \par
\noindent\hangindent=2\parindent qu’il ne soit comme une ombre, \par
\noindent\hangindent=2\parindent puis-qu’ils ne craignent point dieu.\par
\phantomsection
\label{Eccl.8.14}\noindent\hangindent=2\parindent\pn{14} Il y a une chose qui rien ne vaut, laquelle se fait au monde, \par
\noindent\hangindent=2\parindent c’est qu’il y a des innocents qui sont fortunés\footnoteA{« traités »} comme méchants, \par
\noindent\hangindent=2\parindent et des méchants qui sont fortunés comme innocents : \par
\noindent\hangindent=2\parindent et je dis que cela ne vaut encore rien.\par
\bigbreak
\phantomsection
\label{Eccl.8.15}\noindent\hangindent=2\parindent\pn{15} Et pourtant je prise plaisir, \par
\noindent\hangindent=2\parindent en tant qu’un homme n’a autre bien sous le soleil, \par
\noindent\hangindent=2\parindent que de manger et boire, et faire grand chère, \par
\noindent\hangindent=2\parindent et pour le moins retenir de son travail \par
\noindent\hangindent=2\parindent en sa vie, \par
\noindent\hangindent=2\parindent ce que Dieu lui donne sous le soleil.\par

\astermono

\phantomsection
\label{Eccl.8.16}\noindent\hangindent=2\parindent\pn{16} Comme ainsi fût que j’eusse adonné mon cœur à connaître sagesse, \par
\noindent\hangindent=2\parindent et à considérer le tourment qu’on endure au monde, \par
\noindent\hangindent=2\parindent jusqu’à ne pouvoir dormir jour ni nuit,\par
\phantomsection
\label{Eccl.8.17}\noindent\hangindent=2\parindent\pn{17} j’ai aperçu que toutes les œuvres de Dieu sont telles, \par
\noindent\hangindent=2\parindent que l’homme ne peut trouver la raison de ce qui se fait sous le soleil : \par
\noindent\hangindent=2\parindent et quelque peine qu’il prenne à la chercher, si ne la peut-il trouver : \par
\noindent\hangindent=2\parindent et combien que le sage se délibère de l’apprendre, \par
\noindent\hangindent=2\parindent si ne la peut-il trouver.

\section[{Eccl.9}]{\emph{Eccl}.9}
\renewcommand{\leftmark}{\emph{Eccl}.9}

\phantomsection
\label{Eccl.9.1}\noindent\hangindent=2\parindent\pn{1} Car j’ai cherché et épluché\footnoteA{« examiner »} \par
\noindent\hangindent=2\parindent en mon  esprit toute cette matière, \par
\noindent\hangindent=2\parindent c’est que les justes et sages, et leurs faits, \par
\noindent\hangindent=2\parindent sont en la main de Dieu : \par
\noindent\hangindent=2\parindent tellement que les hommes ne savent si l’on est \footnote{à-savoir de Dieu, par ce qui advient, vu que souvent les bons sont mal a leur aise, et les mauvais sont a leur aise, et qu’aussi bien meurt bon que mauvais.}aimé ou haï, \par
\noindent\hangindent=2\parindent vu qu’ils voient évidemment qu’autant en est des uns que des autres.\par
\phantomsection
\label{Eccl.9.2}\noindent\hangindent=2\parindent\pn{2} Autant en prend du juste que de l’injuste, \par
\noindent\hangindent=2\parindent du bon et net que du souillé, \par
\noindent\hangindent=2\parindent de celui qui sacrifie que de celui qui ne sacrifie, \par
\noindent\hangindent=2\parindent du bon que du mauvais, \par
\noindent\hangindent=2\parindent du parjure que de celui qui craint de se parjurer.\par
\bigbreak
\phantomsection
\label{Eccl.9.3}\noindent\hangindent=2\parindent\pn{3} C’est un mauvais cas en tout ce qui se fait sous le soleil, \par
\noindent\hangindent=2\parindent que comme la fortune de tous est tout une, \par
\noindent\hangindent=2\parindent ainsi ont les hommes le cœur plein de mauvaitie \par
\noindent\hangindent=2\parindent et forcennerie\footnoteA{de \emph{forcené}, « folie, fureur »}, durant leur vie, \par
\noindent\hangindent=2\parindent puis s’en vont trouver les morts.\par
\bigbreak
\phantomsection
\label{Eccl.9.4}\noindent\hangindent=2\parindent\pn{4} Car en tous vifs (qui est chose désirable) \par
\noindent\hangindent=2\parindent il y a espérance, \par
\noindent\hangindent=2\parindent car un chien vif vaut mieux qu’un lion mort,\par
\phantomsection
\label{Eccl.9.5}\noindent\hangindent=2\parindent\pn{5} vu que les vifs savent bien qu’ils mourront : \par
\noindent\hangindent=2\parindent mais les morts ne savent rien, \par
\noindent\hangindent=2\parindent et ne leur reste plus nulle récompense, \par
\noindent\hangindent=2\parindent attendu que la mémoire en est effacée,\par
\phantomsection
\label{Eccl.9.6}\noindent\hangindent=2\parindent\pn{6} et leur amour et leur haine et leur envie \par
\noindent\hangindent=2\parindent est déjà périe, et n’ont jamais plus rien à faire \par
\noindent\hangindent=2\parindent avec chose qui se fasse sous le soleil.\par

\astermono

\phantomsection
\label{Eccl.9.7}\noindent\hangindent=2\parindent\pn{7} Va, mange ton pain joyeusement, \par
\noindent\hangindent=2\parindent et bois ton vin d’un cœur gai, \par
\noindent\hangindent=2\parindent puis-que Dieu prend plaisir en tes œuvres.\par
\phantomsection
\label{Eccl.9.8}\noindent\hangindent=2\parindent\pn{8} Porte tous-jours\footnoteA{Tous les jours habillé dimanche.} des habillements blancs, \par
\noindent\hangindent=2\parindent et la tête mouillée de baume, sans y faillir.\par
\bigbreak
\phantomsection
\label{Eccl.9.9}\noindent\hangindent=2\parindent\pn{9} Passe le temps avec ta bien aimée, \par
\noindent\hangindent=2\parindent tant que durera ta néante vie, \par
\noindent\hangindent=2\parindent qui t’est octroyée sous le soleil, tant que durera ton néant. \par
\noindent\hangindent=2\parindent Car c’est-ce que tu gagnes en la vie, \par
\noindent\hangindent=2\parindent par la peine que tu prends sous le soleil.\par
\bigbreak
\phantomsection
\label{Eccl.9.10}\noindent\hangindent=2\parindent\pn{10} Tout ce que tu auras puissance de faire, \par
\noindent\hangindent=2\parindent fais-le de tout ton pouvoir : \par
\noindent\hangindent=2\parindent car en l’autre monde où tu t’en vas, \par
\noindent\hangindent=2\parindent il n’y a ni œuvre, ni raison, ni science ou sagesse.\par

\astermono

\phantomsection
\label{Eccl.9.11}\noindent\hangindent=2\parindent\pn{11} Derechef je vois que sous le soleil \par
\noindent\hangindent=2\parindent il n’y a ni vitesse qui serve pour courir, \par
\noindent\hangindent=2\parindent ni force pour guerroyer, \par
\noindent\hangindent=2\parindent ni sagesse pour acquérir de quoi vivre, \par
\noindent\hangindent=2\parindent ni entendement pour richesses, \par
\noindent\hangindent=2\parindent ni savoir pour entrer en grâce, \par
\noindent\hangindent=2\parindent mais n’y a que temps et fortune qui gouverne tout.\par
\bigbreak
\phantomsection
\label{Eccl.9.12}\noindent\hangindent=2\parindent\pn{12} Car les hommes ne savent point leur temps : \par
\noindent\hangindent=2\parindent et comme les poissons se prennent  au cauteleux filé, \par
\noindent\hangindent=2\parindent et les oiseaux aux lacs\footnoteA{« lacets »}, \par
\noindent\hangindent=2\parindent ainsi les hommes sont enfilés au temps d’adversité, \par
\noindent\hangindent=2\parindent et accablés au dépourvu.\par

\astermono

\phantomsection
\label{Eccl.9.13}\noindent\hangindent=2\parindent\pn{13} Item je vois une sagesse sous le soleil, \par
\noindent\hangindent=2\parindent laquelle j’estime beaucoup.\par
\phantomsection
\label{Eccl.9.14}\noindent\hangindent=2\parindent\pn{14} Il y a une petite ville, et peu de gens dedans, \par
\noindent\hangindent=2\parindent laquelle est assaillie et assiégée d’un grand roi, \par
\noindent\hangindent=2\parindent qui dresse contre elle des gros engins.\par
\phantomsection
\label{Eccl.9.15}\noindent\hangindent=2\parindent\pn{15} Et se trouve en elle un homme roturier, qui est si sage, \par
\noindent\hangindent=2\parindent que par sa sagesse il délivre la ville : \par
\noindent\hangindent=2\parindent et toutefois homme n’avait souvenance dudit homme roturier.\par
\phantomsection
\label{Eccl.9.16}\noindent\hangindent=2\parindent\pn{16} Et pourtant je dis \par
\noindent\hangindent=2\parindent que sagesse vaut mieux que force, \par
\noindent\hangindent=2\parindent d’jà soit que la sagesse d’un homme de basse condition soit méprisée, \par
\noindent\hangindent=2\parindent et qu’on n’obéisse pas à ses paroles.\par
\phantomsection
\label{Eccl.9.17}\noindent\hangindent=2\parindent\pn{17} On écoute mieux les paisibles paroles d’un sage, \par
\noindent\hangindent=2\parindent que la crierie d’un maître des fols.\par
\phantomsection
\label{Eccl.9.18}\noindent\hangindent=2\parindent\pn{18} \footnote{\emph{Sus}.6.8}Mieux vaut sagesse, que bâtons de guerre, \par
\noindent\hangindent=2\parindent et un mauvais gâte beaucoup de bien.

\section[{Eccl.10}]{\emph{Eccl}.10}
\renewcommand{\leftmark}{\emph{Eccl}.10}

\phantomsection
\label{Eccl.10.1}\noindent\hangindent=2\parindent\pn{1} Comme les mouches venimeuses font puer et gâtent \par
\noindent\hangindent=2\parindent le baume, \par
\noindent\hangindent=2\parindent ainsi un peu de folie gâte \par
\noindent\hangindent=2\parindent une excellente sagesse et honneur.\par
\bigbreak
\phantomsection
\label{Eccl.10.2}\noindent\hangindent=2\parindent\pn{2} Un sage a le cœur à la droite, \par
\noindent\hangindent=2\parindent et un fol à la gauche\footnoteA{Traduction littérale de l’hébreu. Le cœur est ici organe du sentiment et de la mémoire, de la connaissance, c’est aussi le siège des intentions. Avoir le cœur à droite est assez improbable, comprendre plutôt : sagesse du cœur droit et juste ; sottise du cœur gauche et tort.}.\par
\phantomsection
\label{Eccl.10.3}\noindent\hangindent=2\parindent\pn{3} Un fol même en allant par le chemin \par
\noindent\hangindent=2\parindent est hors du sens, \par
\noindent\hangindent=2\parindent et montre à chacun qu’il est fol.\par

\astermono

\phantomsection
\label{Eccl.10.4}\noindent\hangindent=2\parindent\pn{4} Si \footnote{le prince,}celui qui est maître se courrouce contre toi, \par
\noindent\hangindent=2\parindent \footnote{tiens-toi tout coi.}n’abandonne point ta place : \par
\noindent\hangindent=2\parindent car se tenir coi \footnote{fait pardonner.}est le remède de maintes fautes.\par
\bigbreak
\phantomsection
\label{Eccl.10.5}\noindent\hangindent=2\parindent\pn{5} Un mal y a que je vois sous le soleil, \par
\noindent\hangindent=2\parindent comme partant du més-entendement de celui qui gouverne :\par
\phantomsection
\label{Eccl.10.6}\noindent\hangindent=2\parindent\pn{6} c’est que le fol est mis en haut degré de dignité, \par
\noindent\hangindent=2\parindent et les riches sont assis tout bas :\par
\phantomsection
\label{Eccl.10.7}\noindent\hangindent=2\parindent\pn{7} j’ai vu des serviteurs sur des chevaux, \par
\noindent\hangindent=2\parindent et les princes aller à pied comme serviteurs.\par

\astermono

\phantomsection
\label{Eccl.10.8}\noindent\hangindent=2\parindent\pn{8} Qui fosse cave\footnoteA{\emph{caver} « creuser »}, en fosse trébuche : \par
\noindent\hangindent=2\parindent et qui haie défait, sera mort d’un serpent.\par
\phantomsection
\label{Eccl.10.9}\noindent\hangindent=2\parindent\pn{9} \footnote{à gros courage, grosse peine.}Qui pierres porte, il y travaille : \par
\noindent\hangindent=2\parindent et qui bois fend, il y ahanne.\par
\phantomsection
\label{Eccl.10.10}\noindent\hangindent=2\parindent\pn{10} Comme quand un outil est rebouché\footnoteA{« émoussé, [d’une personne] obtus, stupide »} et mal émoulu, \par
\noindent\hangindent=2\parindent il n’y a si fort qu’il ne lasse\footnoteA{Mauvais outil épuise bonne force.}, \par
\noindent\hangindent=2\parindent ainsi \footnote{engin fait valoir force. NdE \emph{engin} « ingéniosité, intelligence »}sagesse fait valoir excellence.\par

\astermono

\phantomsection
\label{Eccl.10.11}\noindent\hangindent=2\parindent\pn{11} Un languard ne vaut de rien mieux qu’un \footnote{\emph{Ps} 58.5. aspic NdE Mordu par un serpent sourd, que la musique ne charme pas.}serpent, \par
\noindent\hangindent=2\parindent quand il mord sans être charmé.\par
\phantomsection
\label{Eccl.10.12}\noindent\hangindent=2\parindent\pn{12}  Paroles de sage ont crédit : \par
\noindent\hangindent=2\parindent lèvres de fol gâtent leur maître.\par
\phantomsection
\label{Eccl.10.13}\noindent\hangindent=2\parindent\pn{13} Le commencement de ses propos n’est que folie, \par
\noindent\hangindent=2\parindent et la fin n’est qu’une malheureuse forcennerie\footnoteA{de \emph{forcené}, « folie, fureur »}.\par
\phantomsection
\label{Eccl.10.14}\noindent\hangindent=2\parindent\pn{14} Quelque causer que fassent les fols, \par
\noindent\hangindent=2\parindent l’homme ne sait ce qui est avenir, \par
\noindent\hangindent=2\parindent et n’y a nul qui lui donne à connaître ce qui sera après lui.\par
\phantomsection
\label{Eccl.10.15}\noindent\hangindent=2\parindent\pn{15} \footnote{pour néant travaille qui n’a moyen.}Un fol qui ne sait aller en la ville, \par
\noindent\hangindent=2\parindent travaille tant qu’il se lasse\footnoteA{« fatiguer »}.\par

\astermono

\phantomsection
\label{Eccl.10.16}\noindent\hangindent=2\parindent\pn{16} Ha pauvre pays qui as un roi enfant, \par
\noindent\hangindent=2\parindent et des princes qui \footnote{ivrognes et gourmands.}mangent de matin.\par
\phantomsection
\label{Eccl.10.17}\noindent\hangindent=2\parindent\pn{17} Heureux pays qui as un roi chenu, \par
\noindent\hangindent=2\parindent et des princes qui mangent à l’heure qu’ils doivent, \par
\noindent\hangindent=2\parindent pour reprendre leur force, et non pour boire.\par
\phantomsection
\label{Eccl.10.18}\noindent\hangindent=2\parindent\pn{18} Par paresse dé-cale\footnoteA{« déséquilibrer »} le plancher\footnoteA{« étage »}, \par
\noindent\hangindent=2\parindent et mains lâches font pleuvoir en la maison.\par
\phantomsection
\label{Eccl.10.19}\noindent\hangindent=2\parindent\pn{19} De la panse vient la danse, \par
\noindent\hangindent=2\parindent et du vin joyeuse vie, \par
\noindent\hangindent=2\parindent et argent dompte tout.\par
\phantomsection
\label{Eccl.10.20}\noindent\hangindent=2\parindent\pn{20} Ne maudis point le roi, même en ta pensée, \par
\noindent\hangindent=2\parindent et ne maudis point un riche, même en l’arrière-chambre où tu couches : \par
\noindent\hangindent=2\parindent car les oiseaux mêmes de l’air emporteront le propos, \par
\noindent\hangindent=2\parindent et y aura quelque chose volante qui en fera le rapport.

\section[{Eccl.11}]{\emph{Eccl}.11}
\renewcommand{\leftmark}{\emph{Eccl}.11}

\phantomsection
\label{Eccl.11.1}\noindent\hangindent=2\parindent\pn{1} Jette \footnote{fais aumône.}ton blé en lieu humide : \par
\noindent\hangindent=2\parindent car par succession de temps tu le trouveras.\par
\phantomsection
\label{Eccl.11.2}\noindent\hangindent=2\parindent\pn{2} Départ-en\footnoteA{« départir, couper, partager »} \footnote{à plusieurs.}à sept, voire à huit : \par
\noindent\hangindent=2\parindent car tu \footnote{tu pourras bien être en disette.}ne sais quel mal il adviendra au monde.\par
\phantomsection
\label{Eccl.11.3}\noindent\hangindent=2\parindent\pn{3} \footnote{donne tandis que tu as de quoi, quand tu seras mort, tu ne pourras donner, non plus qu’un arbre ne peut bouger quand il est tombé.}Quand les nuées sont pleines, \par
\noindent\hangindent=2\parindent elles épandent de la pluie sur terre : \par
\noindent\hangindent=2\parindent et soit qu’un arbre tombe contre le midi, soit contre la bise, \par
\noindent\hangindent=2\parindent là même où il tombe, il demeure.\par
\phantomsection
\label{Eccl.11.4}\noindent\hangindent=2\parindent\pn{4} \footnote{qui ne fait aumône pourtant qu’il ne sait l’avenir, fait comme celui qui laisse de semer ou moissonner, de peur du vent ou de la pluie.}Qui prend garde au vent ne sème point : \par
\noindent\hangindent=2\parindent et qui regarde les nuées, ne moissonne point.\par
\phantomsection
\label{Eccl.11.5}\noindent\hangindent=2\parindent\pn{5} Comme tu ne saurais connaître la trace du vent\footnoteA{« empreinte du vent, trace de l’esprit » (donnée à l’enfant au ventre). Leçon déjà dans la \emph{Vulgate}, par ex chez \emph{Sacy 1667} : « \emph{Comme vous ignorez par où l’âme vient} ».}, \par
\noindent\hangindent=2\parindent ni les os qui sont au ventre d’une femme grosse, \par
\noindent\hangindent=2\parindent ainsi ne saurais-tu connaître l’ouvrage de Dieu, \par
\noindent\hangindent=2\parindent qui fait tout.\par
\phantomsection
\label{Eccl.11.6}\noindent\hangindent=2\parindent\pn{6} Au matin sème ta semence, \par
\noindent\hangindent=2\parindent et au soir n’y aie point la main lâche : \par
\noindent\hangindent=2\parindent car tu ne sais lequel des deux vaut mieux, \par
\noindent\hangindent=2\parindent ou s’ils sont tous deux aussi bons l’un que l’autre.\par

\astermono

\phantomsection
\label{Eccl.11.7}\noindent\hangindent=2\parindent\pn{7} Et la lumière est chose ami-able\footnoteA{De \emph{ami} « aimable »}, \par
\noindent\hangindent=2\parindent et voir le soleil est  chose plaisante aux yeux :\par
\phantomsection
\label{Eccl.11.8}\noindent\hangindent=2\parindent\pn{8} toutefois combien qu’un homme vive plusieurs ans, \par
\noindent\hangindent=2\parindent voire toujours à son aise, \par
\noindent\hangindent=2\parindent s’il lui souvient combien long sera le temps de ténèbres, \par
\noindent\hangindent=2\parindent tout ce qui vient n’est rien.\par
\bigbreak
\phantomsection
\label{Eccl.11.9}\noindent\hangindent=2\parindent\pn{9} Jouis de ta jeunesse, jouvenceau, \par
\noindent\hangindent=2\parindent et te donne de bon temps tandis que tu es jeune, \par
\noindent\hangindent=2\parindent et mène un tel train que requiert le souhait de ton cœur, \par
\noindent\hangindent=2\parindent ou le regard de tes yeux : \par
\noindent\hangindent=2\parindent mais sache que de tout cela Dieu t’en fera rendre compte.\par
\phantomsection
\label{Eccl.11.10}\noindent\hangindent=2\parindent\pn{10} Ôte donc fierté de ton courage, \par
\noindent\hangindent=2\parindent et chasse méchanceté de ton corps : \par
\noindent\hangindent=2\parindent car jeunesse et peu savoir, ne vaut rien.

\section[{Eccl.12}]{\emph{Eccl}.12}
\renewcommand{\leftmark}{\emph{Eccl}.12}

\phantomsection
\label{Eccl.12.1}\noindent\hangindent=2\parindent\pn{1} Et te souvienne de ton créateur, \par
\noindent\hangindent=2\parindent tandis que tu es jeune, \par
\noindent\hangindent=2\parindent devant que vienne le mal-temps, \par
\noindent\hangindent=2\parindent et que les ans arrivent, desquels tu diras \par
\noindent\hangindent=2\parindent que tu n’y prends pas plaisir :\par
\phantomsection
\label{Eccl.12.2}\noindent\hangindent=2\parindent\pn{2} devant que \footnote{tu aies courte vue par vieillesse.}le soleil, et la lumière, \par
\noindent\hangindent=2\parindent et la lune, et les étoiles perdent leur clarté, \par
\noindent\hangindent=2\parindent et que les \footnote{yeux te pleurent et soient troublés.}nuées retournent après la pluie,\par
\phantomsection
\label{Eccl.12.3}\noindent\hangindent=2\parindent\pn{3} lors-que les \footnote{mains.}gardes de la maison trembleront, \par
\noindent\hangindent=2\parindent et les \footnote{jambes.}soudards chancelleront, \par
\noindent\hangindent=2\parindent et les \footnote{dents.}meules cesseront, tant seront amoindries : \par
\noindent\hangindent=2\parindent et \footnote{la vue.}les regardant par les trous n’y pourront plus voir,\par
\phantomsection
\label{Eccl.12.4}\noindent\hangindent=2\parindent\pn{4} et les \footnote{lèvres.}huis seront fermés par dehors, \par
\noindent\hangindent=2\parindent \footnote{les dents ne pourront plus mâcher.}avec un bas son de la meule : \par
\noindent\hangindent=2\parindent et qu’on se lèvera au chant \footnote{du coq, c’est qu’on ne pourra dormir.}d’oiseau, \par
\noindent\hangindent=2\parindent et que toutes les \footnote{instruments de la voix.}chanteresses seront cassées.\par
\phantomsection
\label{Eccl.12.5}\noindent\hangindent=2\parindent\pn{5} Item lors-qu’on aura peur des  lieux hauts, \par
\noindent\hangindent=2\parindent et de chopper\footnoteA{« trébucher »} en la voie, \par
\noindent\hangindent=2\parindent et que \footnote{le poil gris. [NdE] La fleur de l’amandier est blanche.}l’amandier fleurira, \par
\noindent\hangindent=2\parindent et que \footnote{on ne sera que plaindre et gémir}les cigales s’assembleront, \par
\noindent\hangindent=2\parindent et se perdra l’appétit, \par
\noindent\hangindent=2\parindent quand l’homme s’en ira en son logis éternel, \par
\noindent\hangindent=2\parindent et que les portants-deuil tourneront par la rue.\par
\phantomsection
\label{Eccl.12.6}\noindent\hangindent=2\parindent\pn{6} Devant que \footnote{je n’entends pas ces quatre}la chaîne d’argent soit rompue, \par
\noindent\hangindent=2\parindent et la fiole d’or cassée, \par
\noindent\hangindent=2\parindent et la bouteille brisée sur la source, \par
\noindent\hangindent=2\parindent et le chariot froissé\footnoteA{« briser, fracasser »} vers la fosse,\par
\phantomsection
\label{Eccl.12.7}\noindent\hangindent=2\parindent\pn{7} et que la poudre\footnoteA{« poussière, cendre »} retourne en terre, comme elle avait été, \par
\noindent\hangindent=2\parindent et que l’esprit retourne à Dieu qui l’a donné.\par

\astermono

\phantomsection
\label{Eccl.12.8}\noindent\hangindent=2\parindent\pn{8} Tout ne vaut rien, dit le prêcheur, tout ne vaut rien.\par
\bigbreak
\phantomsection
\label{Eccl.12.9}\noindent\hangindent=2\parindent\pn{9} D’avantage par l’excellente sagesse qu’avait le prêcheur, \par
\noindent\hangindent=2\parindent il enseigna aux gens autre savoir, \par
\noindent\hangindent=2\parindent et proposa ce qu’il avait épluché\footnoteA{« examiner »}, composant maintes sentences.\par
\phantomsection
\label{Eccl.12.10}\noindent\hangindent=2\parindent\pn{10} Ledit prêcheur tâcha de trouver paroles plaisantes, \par
\noindent\hangindent=2\parindent et droite écriture \par
\noindent\hangindent=2\parindent de vrais propos.\par
\phantomsection
\label{Eccl.12.11}\noindent\hangindent=2\parindent\pn{11} Paroles de sages sont comme aiguillons, \par
\noindent\hangindent=2\parindent et sont ramasseurs donnés d’un pasteur, comme pointes fichées\footnoteA{Passage obscur. La distinction en hébreu vient peut-être de pratiques pastorales et pourrait opposer l’\emph{aiguillon} pour faire avancer le troupeau, et les \emph{pointes fichées} qui dessineraient comme un chemin. Nous éviterons d’ajouter de la confusion et laissons l’obscurité.}.\par
\phantomsection
\label{Eccl.12.12}\noindent\hangindent=2\parindent\pn{12} Au reste, mon fils, soi bien avisé : \par
\noindent\hangindent=2\parindent de faire tant de livres, il n’y a point de fin : \par
\noindent\hangindent=2\parindent et trop grand souci, lasse le corps.\par
\phantomsection
\label{Eccl.12.13}\noindent\hangindent=2\parindent\pn{13} Conclusion, quand tout est dit, \par
\noindent\hangindent=2\parindent crains Dieu, et garde ses commandements : \par
\noindent\hangindent=2\parindent car c’est le devoir de tous hommes.\par
\bigbreak
\phantomsection
\label{Eccl.12.14}\noindent\hangindent=2\parindent\pn{14} Car de toute œuvre, tant soit secrète, Dieu en fera rendre compte, \par
\noindent\hangindent=2\parindent soit bonne soit mauvaise.\par
{\itshape La fin de l’Ecclésiaste}
 


% at least one empty page at end (for booklet couv)
\ifbooklet
  \pagestyle{empty}
  \clearpage
  % 2 empty pages maybe needed for 4e cover
  \ifnum\modulo{\value{page}}{4}=0 \hbox{}\newpage\hbox{}\newpage\fi
  \ifnum\modulo{\value{page}}{4}=1 \hbox{}\newpage\hbox{}\newpage\fi


  \hbox{}\newpage
  \ifodd\value{page}\hbox{}\newpage\fi
  {\centering\color{rubric}\bfseries\noindent\large
    Hurlus ? Qu’est-ce.\par
    \bigskip
  }
  \noindent Des bouquinistes électroniques, pour du texte libre à participations libres,
  téléchargeable gratuitement sur \href{https://hurlus.fr}{\dotuline{hurlus.fr}}.\par
  \bigskip
  \noindent Cette brochure a été produite par des éditeurs bénévoles.
  Elle n’est pas faite pour être possédée, mais pour être lue, et puis donnée.
  En page de garde, on peut ajouter une date, un lieu, un nom ;
  comme une fiche de bibliothèque en papier,
  pour suivre le voyage du texte. Qui sait, un jour, il vous reviendra ?
  \par

  Ce texte a été choisi parce qu’une personne l’a aimé,
  ou haï, elle a pensé qu’il partipait à la formation de notre présent ;
  sans le souci de plaire, vendre, ou militer pour une cause.
  \par

  L’édition électronique est soigneuse, tant sur la technique
  que sur l’établissement du texte ; mais sans aucune prétention scolaire, au contraire.
  Le but est de s’adresser à tous, sans distinction de science ou de diplôme.
  \par

  Cet exemplaire en papier a été tiré sur une imprimante personnelle
   ou une photocopieuse. Tout le monde peut le faire.
  Il suffit de
  télécharger un fichier sur \href{https://hurlus.fr}{\dotuline{hurlus.fr}},
  d’imprimer, et agrafer (puis lire et donner).\par

  \bigskip

  \noindent PS : Les hurlus furent aussi des rebelles protestants qui cassaient les statues dans les églises catholiques. En 1566 démarra la révolte des gueux dans le pays de Lille. L’insurrection enflamma la région jusqu’à Anvers où les gueux de mer bloquèrent les bateaux espagnols.
  Ce fut une rare guerre de libération dont naquit un pays toujours libre : les Pays-Bas.
  En plat pays francophone, par contre, restèrent des bandes de huguenots, les hurlus, progressivement réprimés par la très catholique Espagne.
  Cette mémoire d’une défaite est éteinte, rallumons-la. Sortons les livres du culte universitaire, débusquons les idoles de l’époque, pour les démonter.
\fi

\end{document}
