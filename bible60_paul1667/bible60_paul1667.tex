%%%%%%%%%%%%%%%%%%%%%%%%%%%%%%%%%
% LaTeX model https://hurlus.fr %
%%%%%%%%%%%%%%%%%%%%%%%%%%%%%%%%%

% Needed before document class
\RequirePackage{pdftexcmds} % needed for tests expressions
\RequirePackage{fix-cm} % correct units

% Define mode
\def\mode{a4}

\newif\ifaiv % a4
\newif\ifav % a5
\newif\ifbooklet % booklet
\newif\ifcover % cover for booklet

\ifnum \strcmp{\mode}{cover}=0
  \covertrue
\else\ifnum \strcmp{\mode}{booklet}=0
  \booklettrue
\else\ifnum \strcmp{\mode}{a5}=0
  \avtrue
\else
  \aivtrue
\fi\fi\fi

\ifbooklet % do not enclose with {}
  \documentclass[twoside]{book} % ,notitlepage
  \usepackage[%
    papersize={105mm, 297mm},
    inner=12mm,
    outer=12mm,
    top=20mm,
    bottom=15mm,
    marginparsep=3pt,
    marginpar=7mm,
  ]{geometry}
  \usepackage[fontsize=9.5pt]{scrextend} % for Roboto
\else\ifav % A5
  \documentclass[twoside]{book} % ,notitlepage
  \usepackage[%
    a5paper
  ]{geometry}
  \usepackage[fontsize=12pt]{scrextend}
\else% A4 2 cols
  \documentclass[twocolumn]{report}
  \usepackage[%
    a4paper,
    inner=15mm,
    outer=10mm,
    top=25mm,
    bottom=18mm,
    marginparsep=0pt,
  ]{geometry}
  \setlength{\columnsep}{20mm}
  \usepackage[fontsize=9.5pt]{scrextend}
\fi\fi

%%%%%%%%%%%%%%
% Alignments %
%%%%%%%%%%%%%%
% before teinte macros

\setlength{\arrayrulewidth}{0.2pt}
\setlength{\columnseprule}{\arrayrulewidth} % twocol

%%%%%%%%%%
% Colors %
%%%%%%%%%%
% before Teinte macros

\usepackage[dvipsnames]{xcolor}
\definecolor{rubric}{HTML}{0c71c3} % the tonic
\def\columnseprulecolor{\color{rubric}}
\colorlet{borderline}{rubric!30!} % definecolor need exact code
\definecolor{shadecolor}{gray}{0.95}
\definecolor{bghi}{gray}{0.5}

%%%%%%%%%%%%%%%%%
% Teinte macros %
%%%%%%%%%%%%%%%%%
%%%%%%%%%%%%%%%%%%%%%%%%%%%%%%%%%%%%%%%%%%%%%%%%%%%
% <TEI> generic (LaTeX names generated by Teinte) %
%%%%%%%%%%%%%%%%%%%%%%%%%%%%%%%%%%%%%%%%%%%%%%%%%%%
% This template is inserted in a specific design
% It is XeLaTeX and otf fonts

\makeatletter % <@@@

\setlength{\parskip}{0pt} % 1pt allow better vertical justification
\setlength{\parindent}{1.5em}

\usepackage{alphalph} % for alph couter z, aa, ab…
\usepackage{blindtext} % generate text for testing
\usepackage{booktabs} % for tables: \toprule, \midrule…
\usepackage[strict]{changepage} % for modulo 4
\usepackage{contour} % rounding words
\usepackage[nodayofweek]{datetime}
\usepackage{enumitem} % <list>
\usepackage{etoolbox} % patch commands
\usepackage{fancyvrb}
\usepackage{fancyhdr}
\usepackage{float}
\usepackage{fontspec} % XeLaTeX mandatory for fonts
\usepackage{footnote} % used to capture notes in minipage (ex: quote)
\usepackage{graphicx}
\usepackage{lettrine} % drop caps
\usepackage{lipsum} % generate text for testing
\usepackage{relsize} % \smaller \larger (ex: quotes in body and footnotes)
\usepackage{manyfoot} % for parallel footnote numerotation
\usepackage[framemethod=tikz,]{mdframed} % maybe used for frame with footnotes inside
\usepackage[defaultlines=2,all]{nowidow} % at least 2 lines by par (works well!)
\usepackage{pdftexcmds} % needed for tests expressions
\usepackage{poetry} % <l>, bad for theater
\usepackage{polyglossia} % bug Warning: "Failed to patch part"
\usepackage[%
  indentfirst=false,
  vskip=1em,
  noorphanfirst=true,
  noorphanafter=true,
  leftmargin=\parindent,
  rightmargin=0pt,
]{quoting}
\usepackage{ragged2e}
\usepackage{setspace} % \setstretch for <quote>
\usepackage{scrextend} % KOMA-common, used for addmargin
\usepackage{tabularx} % <table>
\usepackage[explicit]{titlesec} % wear titles, !NO implicit
\usepackage{tikz} % ornaments
\usepackage{tocloft} % styling tocs
\usepackage[fit]{truncate} % used im runing titles
\usepackage{unicode-math}
\usepackage[normalem]{ulem} % breakable \uline, normalem is absolutely necessary to keep \emph
\usepackage{xcolor} % named colors
\usepackage{xparse} % @ifundefined
\XeTeXdefaultencoding "iso-8859-1" % bad encoding of xstring
\usepackage{xstring} % string tests
\XeTeXdefaultencoding "utf-8"

\defaultfontfeatures{
  % Mapping=tex-text, % no effect seen
  Scale=MatchLowercase,
  Ligatures={TeX,Common},
}
\newfontfamily\zhfont{Noto Sans CJK SC}

% Metadata inserted by a program, from the TEI source, for title page and runing heads
\title{Lettres de l’apôtre\par
\medskip
\emph{traduction Sacy, 1667}}
\date{50–60}
\author{Paul de Tarse}
\def\elbibl{Paul de Tarse. 50–60. \emph{Lettres de l’apôtre}}
\def\elabstract{%
 
\labelblock{Préface hurlue}

 \noindent  Les lettres de Paul de Tarse (ou \emph{Épîtres de saint Paul apôtre}) sont parmi les textes les plus lus depuis des siècles. Elles continuent d’influencer des chrétiens, spécialement les évangélistes, qui y trouvent une inspiration pour leurs cérémonies, leur vie sexuelle, le statut des femmes, ou la collecte de fonds. Il n’est pas nécessaire de croire à un texte pour s’y intéresser. Ce sont les premiers écrits chrétiens, avant les évangiles, un témoignage de l’inspiration initiale de cette religion. C’est aussi un document très vivant sur la société gréco-romaine. \par
  Ne sont données ici que les lettres attestées de Paul de Tarse, en ordre chronologique, les autres épîtres reconnues par le canon biblique apportent peu à la doctrine. Paul écrivait en grec (du très mauvais selon Nietzsche). Il est ici dans un français somptueux, la traduction de Sacy (1613–1684), dite aussi \emph{Bible de Port Royal} (1667) — un haut lieu parisien de réflexion linguistique et religieuse (Racine, Pascal, Arnault…). Pourquoi celle-là et pas une autre ? Parce qu’elle témoigne d’un moment rare en français, où le style servait la religion. Depuis les Lumières et la Révolution, le talent littéraire s’est moins souvent occupé de la Bible. \par
  Ce français est parfois beaucoup plus intelligent que le grec initial, liant par exemple des énumérations brutes et parfois énigmatiques, dans la progression d’une phrase qui lui donne du sens. Si quelqu’un veut absolument goûter l’effet du grec de Paul dans une langue actuelle, il peut essayer une traduction de Darby (1800–1882), un prédicateur anglais qui a rendu la \emph{Bible} le plus littéralement possible en anglais, mais aussi en allemand, et en français. Le résultat demande une tournure d’esprit un peu masochiste pour se plaire au texte. Sacy, plus fidèle à Paul – « la lettre tue, et l’esprit donne la vie » (2 Co 3.6) — s’explique ainsi  : \par
 \quoteskip\begin{quoteblock}
 \noindent On ne voyait pas qu’on pût garder l’exactitude littérale dans la traduction de Saint Paul, sans la rendre si obscure en plusieurs endroits que l’on n’aurait pu y rien comprendre.\par
 En faisant réflexion sur saint Jérôme, qui est comme le modèle des traducteurs de l’Écriture, puisque l’Église a si solennellement approuvé sa version, on reconnaissait qu’il n’avait point cru être obligé de s’attacher servilement à la lettre, puisque l’on voit par la manière dont il a traduit Job \& les Prophètes, que bien loin de n’en faire qu’une glose, il a tellement éclairci ces Écrits divins qui étaient beaucoup plus obscurs dans les Septante [\emph{en grec}], \& il leur a donné dans sa traduction tant de force \& tant de vigueur, que saint Augustin en cite les paroles, lors même qu’elle n’était pas encore en usage dans l’Église, pour faire voir l’éclat \& la majesté de l’Écriture.\par
 
\bibl{Sacy, 1667, \emph{Préface à la première édition du Nouveau Testament}.}
 \end{quoteblock}\quoteskip
 \noindent  Les \emph{Lettres de Paul} sont un monument collectif et millénaire, comme une cathédrale. En rester à la crypte initiale et souterraine des premiers chrétiens est une illusion à jamais perdue, et ne comprendrait pas l’ampleur que l’édification a prise depuis. La traduction idéale garderait l’inspiration originale et chercherait à retrouver l’entousiasme que produisait Paul, et qu’il ne peut plus produire parmi nos esprits bien plus critiques. \par
  Sacy « \emph{a eu dessein de représenter les paroles de l’Ecriture} », comme Racine représentait les romains sur scène. Re-présenter, rendre présent, mais avec les siècles qui nous séparent de Sacy, Paul est habillé en majesté \emph{Grand Siècle}, alors qu’on le savait pauvre, laid, et sale. \par
  Les lettres racontent un militant qui essaie de conserver une cohérence de pratiques et de doctrines entre différents groupes éparpillés sur la Méditerrannée d’orient. Comment entretenir l’espoir que la fin de l’Histoire est pour bientôt, alors que l’inertie du monde s’obstine à durer et à démentir les prophéties ? \par
  Il y a un congrès (Jérusalem, 50), une scission (Antioche, 50), trois campagnes missionnaires (45\textasciitilde49, 50\textasciitilde52, 53\textasciitilde58), une arrestation (Jérusalem, \textasciitilde60), et un procès (Rome, \textasciitilde60). L’exactitude des péripéties importe moins que la manière dont elles sont vécues. Paul hésite souvent entre le cosmique et le trivial. \par
 \quoteskip\begin{quoteblock}
 \noindent Je connais un homme en Jésus-Christ \emph{[lui-même]}, qui fut ravi il y a quatorze ans (si ce fut avec son corps, ou sans son corps, je ne sais, Dieu le sait), qui fut ravi, dis-je, jusqu’au troisième ciel \emph{[…]} il y entendit des paroles ineffables, qu’il n’est pas permis à un homme de rapporter. Je pourrais me glorifier en parlant d’un tel homme ; mais pour moi, je ne veux me glorifier que dans mes faiblesses et dans mes afflictions. (2 Co 12.2-5)\par
 
\bibl{}
 \end{quoteblock}\quoteskip
 \noindent Il se vexe quand on le prend pour un fou.\par
 \quoteskip\begin{quoteblock}
 \noindent Je vous le dis encore une fois : « que personne ne me juge fou ; ou au moins souffrez-moi comme fou, et permettez-moi de me glorifier un peu » (2 Co 11.16)
 \end{quoteblock}\quoteskip
 \noindent Il se vante.\par
 \quoteskip\begin{quoteblock}
 \noindent J’ai été battu de verges par trois fois, j’ai été lapidé une fois, j’ai fait naufrage trois fois, j’ai passé un jour et une nuit au fond de la mer […] (2 Co 11.25)
 \end{quoteblock}\quoteskip
 \noindent Et il insiste régulièrement pour se justifier de ne rien coûter, il ne prêche par par intérêt matériel.\par
 \quoteskip\begin{quoteblock}
 \noindent Et lorsque je demeurais parmi vous, et que j’étais dans la nécessité, je n’ai été a charge à personne (2 Cor 11.9)
 \end{quoteblock}\quoteskip
 \noindent Il n’est pas exactement mythomane, une part de ce qu’il rapporte est corroboré, mais il a dû voir Dieu d’un peu trop près, son bon sens a mal supporté. Il soigne sa vanité en entraînant les autres dans ses délires, notamment la glossolalie — la prière en langues — oubliée des catholiques mais très pratiquée par les évangélistes.\par
 \quoteskip\begin{quoteblock}
 \noindent Si toute une Église étant assemblée en un lieu, tous parlent diverses langues, et que des infidèles, ou des hommes qui ne savent que leur propre langue, entrent dans cette assemblée, ne diront-ils pas que vous êtes des insensés ? (1 Co 14.23)
 \end{quoteblock}\quoteskip
 \noindent  Il est étrange de lire dans ce texte comment les idées de peu de monde, et même les lubies d’un seul personnage, ont pu prendre corps et constituer une religion universelle. La force de l’impulsion initiale de ce texte est toujours présente dans les sociétés qui s’en sont réclamées, même chez les incroyants. Le trait le plus marquant du complexe chrétien est sans doute politique. \par
  Par la peur de la mort, la religion humilie les grands sous Dieu et devant les humbles ; et elle console des humbles par une grandeur imaginaire, remise au jour où Dieu reviendra. Nous ne craignons plus Dieu ou autres fictions, mais nous mourrons encore, et nos grands sont devenus fous, ils pensent trouver le secret de leur immortalité (et il n’y en aura pas pour tout le monde). Qu’ils ne croient pas à des dieux bien anciens, pourquoi pas, mais qu’ils croient leur science plus intelligente est subtile que les milliards d’années de la vie, qu’ils ne comprennent pas leur misère, qu’ils ne s’inclinent pas devant une planète qui les dépasse et qui est indispensable aux autres, et qu’ils entraînent le monde dans leur chute ; nous n’avons pas la joie mauvaise de l’apocalypse des aztèques ou des maïas, nous ne pouvant pas l’accepter. On ne se console pas en leur souhaitant une réincarnation humiliante ; nous sommes juste révoltés qu’ils se moquent aussi ouvertement de leur propre conscience. Nous ne pouvons pas nous empêcher de penser qu’ils ont une âme pareille à la nôtre, et cela nous a été appris par le christianisme (même si bien d’autres cultures le savent aussi bien). Nul n’est bon par nature, le bien est un effort perpétuel devant lequel nous sommes tous égaux. \par
 \quoteskip\begin{quoteblock}
\noindent  Car encore que ma conscience ne me reproche rien, je ne suis pas justifié pour cela (1 Co 4.4) \end{quoteblock}
 \newpage

}
\def\eltitlepage{%
{\centering\parindent0pt
  {\LARGE\addfontfeature{LetterSpace=25}\bfseries Paul de Tarse\par}\bigskip
  {\Large 50–60\par}\bigskip
  {\LARGE
\bigskip\textbf{Lettres de l’apôtre}\par
\bigskip\emph{traduction Sacy, 1667}\par

  }
}

}

% Default metas
\newcommand{\colorprovide}[2]{\@ifundefinedcolor{#1}{\colorlet{#1}{#2}}{}}
\colorprovide{rubric}{red}
\colorprovide{silver}{lightgray}
\@ifundefined{syms}{\newfontfamily\syms{DejaVu Sans}}{}
\newif\ifdev
\@ifundefined{elbibl}{% No meta defined, maybe dev mode
  \newcommand{\elbibl}{Titre court ?}
  \newcommand{\elbook}{Titre du livre source ?}
  \newcommand{\elabstract}{Résumé\par}
  \newcommand{\elurl}{http://oeuvres.github.io/elbook/2}
  \author{Éric Lœchien}
  \title{Un titre de test assez long pour vérifier le comportement d’une maquette}
  \date{1566}
  \devtrue
}{}
\let\eltitle\@title
\let\elauthor\@author
\let\eldate\@date




% generic typo commands
\newcommand{\astermono}{\medskip\centerline{\color{rubric}\large\selectfont{\syms ✻}}\medskip\par}%
\newcommand{\astertri}{\medskip\par\centerline{\color{rubric}\large\selectfont{\syms ✻\,✻\,✻}}\medskip\par}%
\newcommand{\asterism}{\bigskip\par\noindent\parbox{\linewidth}{\centering\color{rubric}\large{\syms ✻}\\{\syms ✻}\hskip 0.75em{\syms ✻}}\bigskip\par}%

% lists
\newlength{\listmod}
\setlength{\listmod}{\parindent}
\setlist{
  itemindent=!,
  listparindent=\listmod,
  labelsep=0.2\listmod,
  parsep=0pt,
  % topsep=0.2em, % default topsep is best
}
\setlist[itemize]{
  label=—,
  leftmargin=0pt,
  labelindent=1.2em,
  labelwidth=0pt,
}
\setlist[enumerate]{
  label={\arabic*°},
  labelindent=0.8\listmod,
  leftmargin=\listmod,
  labelwidth=0pt,
}
% list for big items
\newlist{decbig}{enumerate}{1}
\setlist[decbig]{
  label={\bf\color{rubric}\arabic*.},
  labelindent=0.8\listmod,
  leftmargin=\listmod,
  labelwidth=0pt,
}
\newlist{listalpha}{enumerate}{1}
\setlist[listalpha]{
  label={\bf\color{rubric}\alph*.},
  leftmargin=0pt,
  labelindent=0.8\listmod,
  labelwidth=0pt,
}
\newcommand{\listhead}[1]{\hspace{-1\listmod}\emph{#1}}

\renewcommand{\hrulefill}{%
  \leavevmode\leaders\hrule height 0.2pt\hfill\kern\z@}

% General typo
\DeclareTextFontCommand{\textlarge}{\large}
\DeclareTextFontCommand{\textsmall}{\small}

% commands, inlines
\newcommand{\anchor}[1]{\Hy@raisedlink{\hypertarget{#1}{}}} % link to top of an anchor (not baseline)
\newcommand\abbr[1]{#1}
\newcommand{\autour}[1]{\tikz[baseline=(X.base)]\node [draw=rubric,thin,rectangle,inner sep=1.5pt, rounded corners=3pt] (X) {\color{rubric}#1};}
\newcommand\corr[1]{#1}
\newcommand{\ed}[1]{ {\color{silver}\sffamily\footnotesize (#1)} } % <milestone ed="1688"/>
\newcommand\expan[1]{#1}
\newcommand\foreign[1]{\emph{#1}}
\newcommand\gap[1]{#1}
\renewcommand{\LettrineFontHook}{\color{rubric}}
\newcommand{\initial}[2]{\lettrine[lines=2, loversize=0.3, lhang=0.3]{#1}{#2}}
\newcommand{\initialiv}[2]{%
  \let\oldLFH\LettrineFontHook
  % \renewcommand{\LettrineFontHook}{\color{rubric}\ttfamily}
  \IfSubStr{QJ’}{#1}{
    \lettrine[lines=4, lhang=0.2, loversize=-0.1, lraise=0.2]{\smash{#1}}{#2}
  }{\IfSubStr{É}{#1}{
    \lettrine[lines=4, lhang=0.2, loversize=-0, lraise=0]{\smash{#1}}{#2}
  }{\IfSubStr{ÀÂ}{#1}{
    \lettrine[lines=4, lhang=0.2, loversize=-0, lraise=0, slope=0.6em]{\smash{#1}}{#2}
  }{\IfSubStr{A}{#1}{
    \lettrine[lines=4, lhang=0.2, loversize=0.2, slope=0.6em]{\smash{#1}}{#2}
  }{\IfSubStr{V}{#1}{
    \lettrine[lines=4, lhang=0.2, loversize=0.2, slope=-0.5em]{\smash{#1}}{#2}
  }{
    \lettrine[lines=4, lhang=0.2, loversize=0.2]{\smash{#1}}{#2}
  }}}}}
  \let\LettrineFontHook\oldLFH
}
\newcommand{\labelchar}[1]{\textbf{\color{rubric} #1}}
\newcommand{\lnatt}[1]{\reversemarginpar\marginpar[\sffamily\scriptsize #1]{}}
\newcommand{\milestone}[1]{\autour{\footnotesize\color{rubric} #1}} % <milestone n="4"/>
\newcommand\name[1]{#1}
\newcommand\orig[1]{#1}
\newcommand\orgName[1]{#1}
\newcommand\persName[1]{#1}
\newcommand\placeName[1]{#1}
\newcommand{\pn}[1]{\IfSubStr{-—–¶}{#1}% <p n="3"/>
  {\noindent{\bfseries\color{rubric}   ¶  }}
  {{\footnotesize\autour{#1}}}}
\newcommand\reg{}
% \newcommand\ref{} % already defined
\newcommand\sic[1]{#1}
\newcommand\surname[1]{\textsc{#1}}
\newcommand\term[1]{\textbf{#1}}
\newcommand\zh[1]{{\zhfont #1}}


\def\mednobreak{\ifdim\lastskip<\medskipamount
  \removelastskip\nopagebreak\medskip\fi}
\def\bignobreak{\ifdim\lastskip<\bigskipamount
  \removelastskip\nopagebreak\bigskip\fi}

% commands, blocks

\newcommand{\byline}[1]{\bigskip{\RaggedLeft{#1}\par}\bigskip}
% \setlength{\RaggedLeftLeftskip}{2em plus \leftskip}
\newcommand{\bibl}[1]{{\RaggedLeft\normalfont #1\par}}
\newcommand{\biblitem}[1]{{\noindent\hangindent=\parindent   #1\par}}
\newcommand{\castItem}[1]{{\noindent\hangindent=\parindent #1\par}}
\newcommand{\dateline}[1]{\medskip{\RaggedLeft{#1}\par}\bigskip}
\newcommand{\docAuthor}[1]{{\large\bigskip #1 \par\bigskip}}
\newcommand{\docDate}[1]{#1 \ifvmode\par\fi }
\newcommand{\docImprint}[1]{\ifvmode\medskip\fi #1 \ifvmode\par\fi }
\newcommand{\labelblock}[1]{\medbreak{\noindent\color{rubric}\bfseries #1}\par\mednobreak}
\newcommand{\question}[1]{\bigbreak{\RaggedRight\noindent\emph{#1}\par}\mednobreak}
\newcommand{\salute}[1]{\bigbreak{#1}\par\medbreak}
\newcommand{\signed}[1]{\medskip{\RaggedLeft #1\par}\bigbreak} % supposed bottom
\newcommand{\speaker}[1]{\medskip{\Centering\sffamily #1 \par\nopagebreak}} % supposed bottom
\newcommand{\stagescene}[1]{{\Centering\sffamily\textsf{#1}\par}\bigskip}
\newcommand{\stageblock}[1]{\begingroup\leftskip\parindent\noindent\it\sffamily\footnotesize #1\par\endgroup} % left margin, better than list envs
\newcommand{\lpar}[1]{\noindent\hangindent=2\parindent  #1\par} % sp/l
\newcommand{\trailer}[1]{{\Centering\bigskip #1\par}} % sp/l

% environments for blocks (some may become commands)
\newenvironment{borderbox}{}{} % framing content
\newenvironment{citbibl}{\ifvmode\hfill\fi}{\ifvmode\par\fi }
\newenvironment{msHead}{\vskip6pt}{\par}
\newenvironment{msItem}{\vskip6pt}{\par}


% environments for block containers
\newenvironment{argument}{\itshape\parindent0pt}{\bigskip}
\newenvironment{biblfree}{}{\ifvmode\par\fi }
\newenvironment{bibitemlist}[1]{%
  \list{\@biblabel{\@arabic\c@enumiv}}%
  {%
    \settowidth\labelwidth{\@biblabel{#1}}%
    \leftmargin\labelwidth
    \advance\leftmargin\labelsep
    \@openbib@code
    \usecounter{enumiv}%
    \let\p@enumiv\@empty
    \renewcommand\theenumiv{\@arabic\c@enumiv}%
  }
  \sloppy
  \clubpenalty4000
  \@clubpenalty \clubpenalty
  \widowpenalty4000%
  \sfcode`\.\@m
}%
{\def\@noitemerr
  {\@latex@warning{Empty `bibitemlist' environment}}%
\endlist}
\newenvironment{docTitle}{\LARGE\bigskip\bfseries\onehalfspacing}{\bigskip}
% leftskip makes big bugs in Lexmark printing \sffamily
\newenvironment{epigraph}{\begin{addmargin}[2\parindent]{0em}\sffamily\large\setstretch{0.95}}{\end{addmargin}\bigskip}
\newenvironment{quoteblock}
  {\begin{quoting}\setstretch{0.9}} %
  {\end{quoting}}
\newenvironment{frametext}
  {\begin{mdframed}[default]} %
  {\end{mdframed}}

\quotingsetup{vskip=0pt}
\newcommand{\quoteskip}{\medskip}
\newenvironment{titlePage}
  {\Centering}
  {}






% table () is preceded and finished by custom command
\renewcommand\tabularxcolumn[1]{m{#1}}% for vertical centering text in X column
\newcommand{\tableopen}[1]{%
  \ifnum\strcmp{#1}{wide}=0{%
    \begin{center}
  }
  \else\ifnum\strcmp{#1}{long}=0{%
    \begin{center}
  }
  \else{%
    \begin{center}
  }
  \fi\fi
}
\newcommand{\tableclose}[1]{%
  \ifnum\strcmp{#1}{wide}=0{%
    \end{center}
  }
  \else\ifnum\strcmp{#1}{long}=0{%
    \end{center}
  }
  \else{%
    \end{center}
  }
  \fi\fi
}


% text structure
\newcommand\chapteropen{} % before chapter title
\newcommand\chaptercont{} % after title, argument, epigraph…
\newcommand\chapterclose{} % maybe useful for multicol settings
\setcounter{secnumdepth}{-2} % no counters for hierarchy titles
\setcounter{tocdepth}{5} % deep toc
\renewcommand\tableofcontents{\@starttoc{toc}}
% toclof format
% \renewcommand{\@tocrmarg}{0.1em} % Useless command?
% \renewcommand{\@pnumwidth}{0.5em} % {1.75em}
\renewcommand{\@cftmaketoctitle}{}
\setlength{\cftbeforesecskip}{\z@ \@plus.2\p@}

\@ifclassloaded{article}{%
  \typeout{class: article}%
}{%
  \renewcommand{\cftchapfont}{}
  \renewcommand{\cftchapdotsep}{\cftdotsep}
  \renewcommand{\cftchapleader}{\normalfont\cftdotfill{\cftchapdotsep}}
  \renewcommand{\cftchappagefont}{\bfseries}
  \setlength{\cftbeforechapskip}{0pt}
  \setlength{\cftchapnumwidth}{1em}
}
\renewcommand{\cftsecfont}{\normalfont}
\renewcommand{\cftsecpagefont}{\normalfont}
% \renewcommand{\cftsubsecfont}{\small\relax}
\renewcommand{\cftsecdotsep}{\cftdotsep}
\renewcommand{\cftsecpagefont}{\normalfont}
\renewcommand{\cftsecleader}{\normalfont\cftdotfill{\cftsecdotsep}}
\setlength{\cftsecindent}{1em}
\setlength{\cftsubsecindent}{2em}
\setlength{\cftsubsubsecindent}{3em}
\setlength{\cftsecnumwidth}{1em}
\setlength{\cftsubsecnumwidth}{1em}
\setlength{\cftsubsubsecnumwidth}{1em}

% footnotes
\newif\ifheading
\newcommand*{\fnmarkscale}{\ifheading 0.70 \else 1 \fi}
\renewcommand\footnoterule{\vspace*{0.3cm}\hrule height \arrayrulewidth width 3cm \vspace*{0.3cm}}
\setlength\footnotesep{1.5\footnotesep} % footnote separator
\renewcommand\@makefntext[1]{\parindent 1.5em \noindent \hb@xt@1.8em{\hss{\normalfont\@thefnmark . }}#1} % no superscipt in foot
\patchcmd{\@footnotetext}{\footnotesize}{\footnotesize\sffamily}{}{} % before scrextend, hyperref
\DeclareNewFootnote{A}[alph] % for editor notes
\renewcommand*{\thefootnoteA}{\alphalph{\value{footnoteA}}} % z, aa, ab…

% poem
\setlength{\poembotskip}{0pt}
\setlength{\poemtopskip}{0pt}
\setlength{\poemindent}{0pt}
\setlength{\poemmaxlinewd}{\linewidth}
\poemlinenumsfalse

%   see https://tex.stackexchange.com/a/34449/5049
\def\truncdiv#1#2{((#1-(#2-1)/2)/#2)}
\def\moduloop#1#2{(#1-\truncdiv{#1}{#2}*#2)}
\def\modulo#1#2{\number\numexpr\moduloop{#1}{#2}\relax}

% orphans and widows, nowidow package in test
% from memoir package
\clubpenalty=9996
\widowpenalty=9999
\brokenpenalty=4991
\predisplaypenalty=10000
\postdisplaypenalty=1549
\displaywidowpenalty=1602
\hyphenpenalty=400
% report h or v overfull ?
\hbadness=4000
\vbadness=4000
% good to avoid lines too wide
\emergencystretch 3em
\pretolerance=750
\tolerance=2000
\def\Gin@extensions{.pdf,.png,.jpg,.mps,.tif}

\PassOptionsToPackage{hyphens}{url} % before hyperref and biblatex, which load url package
\usepackage{hyperref} % supposed to be the last one, :o) except for the ones to follow
\hypersetup{
  % pdftex, % no effect
  pdftitle={\elbibl},
  % pdfauthor={Your name here},
  % pdfsubject={Your subject here},
  % pdfkeywords={keyword1, keyword2},
  bookmarksnumbered=true,
  bookmarksopen=true,
  bookmarksopenlevel=1,
  pdfstartview=Fit,
  breaklinks=true, % avoid long links, overrided by url package
  pdfpagemode=UseOutlines,    % pdf toc
  hyperfootnotes=true,
  colorlinks=false,
  pdfborder=0 0 0,
  % pdfpagelayout=TwoPageRight,
  % linktocpage=true, % NO, toc, link only on page no
}
\urlstyle{same} % after hyperref



\makeatother % /@@@>
%%%%%%%%%%%%%%
% </TEI> end %
%%%%%%%%%%%%%%

\setmainlanguage{french}
%%%%%%%%%%%%%
% footnotes %
%%%%%%%%%%%%%
\renewcommand{\thefootnote}{\bfseries\textcolor{rubric}{\arabic{footnote}}} % color for footnote marks

%%%%%%%%%
% Fonts %
%%%%%%%%%
% \linespread{0.90} % too compact, keep font natural
\ifav % A5
  \usepackage{DejaVuSans} % correct
  \setsansfont{DejaVuSans} % seen, if not set, problem with printer
\else\ifbooklet
  \usepackage[]{roboto} % SmallCaps, Regular is a bit bold
  \setmainfont[
    ItalicFont={Roboto Light Italic},
  ]{Roboto}
  \setsansfont{Roboto Light} % seen, if not set, problem with printer
  \newfontfamily\fontrun[]{Roboto Condensed Light} % condensed runing heads
\else
  \usepackage[]{roboto} % SmallCaps, Regular is a bit bold
  \setmainfont[
    ItalicFont={Roboto Italic},
  ]{Roboto Light}
  \setsansfont{Roboto Light} % seen, if not set, problem with printer
  \newfontfamily\fontrun[]{Roboto Condensed Light} % condensed runing heads
\fi\fi
\renewcommand{\LettrineFontHook}{\bfseries\color{rubric}}
% \renewenvironment{labelblock}{\begin{center}\bfseries\color{rubric}}{\end{center}}

%%%%%%%%
% MISC %
%%%%%%%%

\setdefaultlanguage[frenchpart=false]{french} % bug on part


\newenvironment{quotebar}{%
    \def\FrameCommand{{\color{rubric!10!}\vrule width 0.5em} \hspace{0.9em}}%
    \def\OuterFrameSep{0pt} % séparateur vertical
    \MakeFramed {\advance\hsize-\width \FrameRestore}
  }%
  {%
    \endMakeFramed
  }
\renewenvironment{quoteblock}% may be used for ornaments
  {%
    \savenotes
    \setstretch{0.9}
    \begin{quotebar}
    \smallskip
  }
  {%
    \smallskip
    \end{quotebar}
    \spewnotes
  }


\renewcommand{\headrulewidth}{\arrayrulewidth}
\renewcommand{\headrule}{{\color{rubric}\hrule}}
\renewcommand{\lnatt}[1]{\marginpar{\sffamily\scriptsize #1}}

\titleformat{name=\chapter} % command
  [display] % shape
  {\vspace{1.5em}\centering} % format
  {} % label
  {0pt} % separator between n
  {}
[{\color{rubric}\huge\textbf{#1}}\bigskip] % after code
% \titlespacing{command}{left spacing}{before spacing}{after spacing}[right]
\titlespacing*{\chapter}{0pt}{-2em}{0pt}[0pt]

\titleformat{name=\section}
  [display]{}{}{}{}
  [\vbox{\color{rubric}\large\centering\textbf{#1}}]
\titlespacing{\section}{0pt}{0pt plus 4pt minus 2pt}{\baselineskip}

\titleformat{name=\subsection}
  [block]
  {}
  {} % \thesection
  {} % separator \arrayrulewidth
  {}
[\vbox{\large\textbf{#1}}]
% \titlespacing{\subsection}{0pt}{0pt plus 4pt minus 2pt}{\baselineskip}

\ifaiv
  \fancypagestyle{main}{%
    \fancyhf{}
    \setlength{\headheight}{1.5em}
    \fancyhead{} % reset head
    \fancyfoot{} % reset foot
    \fancyhead[L]{\truncate{0.45\headwidth}{\fontrun\elbibl}} % book ref
    \fancyhead[R]{\truncate{0.45\headwidth}{ \fontrun\nouppercase\leftmark}} % Chapter title
    \fancyhead[C]{\thepage}
  }
  \fancypagestyle{plain}{% apply to chapter
    \fancyhf{}% clear all header and footer fields
    \setlength{\headheight}{1.5em}
    \fancyhead[L]{\truncate{0.9\headwidth}{\fontrun\elbibl}}
    \fancyhead[R]{\thepage}
  }
\else
  \fancypagestyle{main}{%
    \fancyhf{}
    \setlength{\headheight}{1.5em}
    \fancyhead{} % reset head
    \fancyfoot{} % reset foot
    \fancyhead[RE]{\truncate{0.9\headwidth}{\fontrun\elbibl}} % book ref
    \fancyhead[LO]{\truncate{0.9\headwidth}{\fontrun\nouppercase\leftmark}} % Chapter title, \nouppercase needed
    \fancyhead[RO,LE]{\thepage}
  }
  \fancypagestyle{plain}{% apply to chapter
    \fancyhf{}% clear all header and footer fields
    \setlength{\headheight}{1.5em}
    \fancyhead[L]{\truncate{0.9\headwidth}{\fontrun\elbibl}}
    \fancyhead[R]{\thepage}
  }
\fi

\ifav % a5 only
  \titleclass{\section}{top}
\fi

\newcommand\chapo{{%
  \vspace*{-3em}
  \centering\parindent0pt % no vskip ()
  \eltitlepage
  \bigskip
  {\color{rubric}\hline}
  \bigskip
  {\Large TEXTE LIBRE À PARTICIPATIONS LIBRES\par}
  \centerline{\small\color{rubric} {\href{https://hurlus.fr}{\dotuline{hurlus.fr}}}, tiré le \today}\par
  \bigskip
}}

\newcommand\cover{{%
  \thispagestyle{empty}
  \centering\parindent0pt
  \eltitlepage
  \vfill\null
  {\color{rubric}\setlength{\arrayrulewidth}{2pt}\hline}
  \vfill\null
  {\Large TEXTE LIBRE À PARTICIPATIONS LIBRES\par}
  \centerline{\href{https://hurlus.fr}{\dotuline{hurlus.fr}}, tiré le \today}\par
}}

\begin{document}
\pagestyle{empty}
\ifbooklet{
  \cover\newpage
  \thispagestyle{empty}\hbox{}\newpage
  \cover\newpage\noindent Les voyages de la brochure\par
  \bigskip
  \begin{tabularx}{\textwidth}{l|X|X}
    \textbf{Date} & \textbf{Lieu}& \textbf{Nom/pseudo} \\ \hline
    \rule{0pt}{25cm} &  &   \\
  \end{tabularx}
  \newpage
  \addtocounter{page}{-4}
}\fi

\thispagestyle{empty}
\ifaiv
  \twocolumn[\chapo]
\else
  \chapo
\fi
{\it\elabstract}
\bigskip
\makeatletter\@starttoc{toc}\makeatother % toc without new page
\bigskip

\pagestyle{main} % after style
\setcounter{footnote}{0}
\setcounter{footnoteA}{0}
  
\chapteropen

\chapter[{Première épître de Paul aux Thessaloniciens (50\textasciitilde51)}]{Première épître de Paul aux Thessaloniciens (50\textasciitilde51)}
\renewcommand{\leftmark}{Première épître de Paul aux Thessaloniciens (50\textasciitilde51)}


\chaptercont

\section[{1 Thess 1}]{1 Thess 1}

\noindent \initialiv{P}{AUL}, Silvain, et Timothée : à l’Église de Thessalonique, qui est en Dieu le Père, et en Jésus-Christ notre Seigneur. Que la grâce et la paix vous soient données !\par
\bigbreak
\noindent   \milestone{2}  Nous rendons sans cesse grâces à Dieu pour vous tous, nous souvenant continuellement de vous dans nos prières ;  \milestone{3}  et nous représentant devant Dieu, qui est notre Père, les œuvres de votre foi, les travaux de votre charité, et la fermeté de l’espérance que vous avez en notre Seigneur Jésus-Christ.  \milestone{4}  Car nous savons, mes frères chéris de Dieu, quelle a été votre élection ;  \milestone{5}  la prédication que nous vous avons faite de l’Évangile, n’ayant pas été seulement en paroles, mais ayant été accompagnée de miracles, de la vertu du Saint-Esprit, d’une pleine abondance de ses dons. Et vous savez aussi de quelle manière j’ai agi parmi vous pour votre salut.  \milestone{6}  Ainsi vous êtes devenus nos imitateurs, et les imitateurs du Seigneur, ayant reçu la parole parmi de grandes afflictions avec la joie du Saint-Esprit ;  \milestone{7}  de sorte que vous avez servi de modèle à tous ceux qui ont embrassé la foi dans la Macédoine et dans l’Achaïe.  \milestone{8}  Car non-seulement vous êtes cause que la parole du Seigneur s’est répandue avec éclat dans la Macédoine et dans l’Achaïe ; mais même la foi que vous avez en Dieu est devenue si célèbre partout, qu’il n’est point nécessaire que nous en parlions :  \milestone{9}  puisque tout le monde nous raconte à nous-mêmes quel a été le succès de notre arrivée parmi vous, et comme ayant quitté les idoles, vous vous êtes convertis à Dieu, pour servir le Dieu vivant et véritable,  \milestone{10}  et pour attendre du ciel son Fils Jésus, qu’il a ressuscité d’entre les morts, et qui nous a délivrés de la colère à venir.

\section[{1 Thess 2}]{1 Thess 2}

\noindent \initial{C}{AR} vous savez vous-mêmes, mes frères, que notre arrivée vers vous n’a pas été vaine et sans fruit ;  \milestone{2}  mais après avoir beaucoup souffert auparavant, comme vous savez, et avoir été traités avec outrage dans Philippes, nous ne laissâmes pas, en nous confiant en notre Dieu, de vous prêcher hardiment l’Évangile de Dieu, parmi beaucoup de peines et de combats.  \milestone{3}  Car nous ne vous avons point prêché une doctrine d’erreur ou d’impureté ; et nous n’avons point eu dessein de vous tromper.  \milestone{4}  Mais comme Dieu nous a choisis pour nous confier son Évangile, nous parlons aussi, non pour plaire aux hommes, mais à Dieu, qui voit le fond de nos cœurs.  \milestone{5}  Car nous n’avons usé d’aucune parole de flatterie, comme vous le savez ; et notre ministère n’a point servi de prétexte à notre avarice, Dieu en est témoin ;  \milestone{6}  et nous n’avons point non plus recherché aucune gloire de la part des hommes, ni de vous, ni d’aucun autre ;  \milestone{7}  quoique nous eussions pu, comme apôtres de Jésus-Christ, vous charger de notre subsistance ; mais nous nous sommes conduits parmi vous avec une douceur d’enfant, comme une nourrice qui a soin de ses enfants.  \milestone{8}  Ainsi dans l’affection que nous ressentions pour vous, nous aurions souhaité de vous donner, non-seulement la connaissance de l’Évangile de Dieu, mais aussi notre propre vie, tant était grand l’amour que nous vous portions.  \milestone{9}  Car vous n’avez pas oublié, mes frères, quelle peine et quelle fatigue nous avons soufferte, et comme nous vous avons prêché l’Évangile de Dieu en travaillant jour et nuit, pour n’être à charge à aucun de vous.  \milestone{10}  Vous êtes témoins vous-mêmes, et Dieu l’est aussi, combien la manière dont je me suis conduit envers vous qui avez embrassé la foi, a été sainte, juste et irréprochable.  \milestone{11}  Et vous savez que j’ai agi envers chacun de vous comme un père envers ses enfants ;  \milestone{12}  vous exhortant, vous consolant, et vous conjurant de vous conduire d’une manière digne de Dieu, qui vous a appelés à son royaume et à sa gloire.\par
\bigbreak
\noindent   \milestone{13}  C’est pourquoi aussi, nous rendons à Dieu de continuelles actions de grâces, de ce qu’ayant entendu la parole de Dieu que nous vous prêchions, vous l’avez reçue, non comme la parole des hommes, mais comme étant, ainsi qu’elle l’est véritablement, la parole de Dieu, qui agit efficacement en vous qui êtes fidèles.  \milestone{14}  Car, mes frères, vous êtes devenus les imitateurs des Églises de Dieu qui ont embrassé la foi de Jésus-Christ dans la Judée, ayant souffert les mêmes persécutions de la part de vos concitoyens, que ces Églises ont souffertes de la part des Juifs ;  \milestone{15}  qui ont tué même le Seigneur Jésus, et leurs prophètes ; qui nous ont persécutés ; qui ne plaisent point à Dieu, et qui sont ennemis de tous les hommes ;  \milestone{16}  qui nous empêchent d’annoncer aux gentils la parole qui doit les sauver, pour combler ainsi la mesure de leurs péchés. Car la colère de Dieu est tombée sur eux, et y demeurera jusqu’à la fin.\par
\bigbreak
\noindent   \milestone{17}  Aussi, mes frères, ayant été pour un peu de temps séparés de vous, de corps, non de cœur, nous avons désiré avec d’autant plus d’ardeur et d’empressement de vous revoir.  \milestone{18}  C’est pourquoi nous avons voulu vous aller trouver ; et moi, Paul, j’en ai eu le dessein une et deux fois ; mais Satan nous en a empêché.  \milestone{19}  Et certes quelle est notre espérance, notre joie, et la couronne de notre gloire ? N’est-ce pas vous qui l’êtes devant le Seigneur Jésus-Christ, pour le jour de son avènement ?  \milestone{20}  Car vous êtes notre gloire et notre joie.

\section[{1 Thess 3}]{1 Thess 3}

\noindent \initial{A}{INSI} ne pouvant souffrir plus longtemps de n’avoir point de vos nouvelles, j’aimai mieux demeurer tout seul à Athènes ;  \milestone{2}  et je vous envoyai Timothée, notre frère, et ministre de Dieu dans la prédication de l’Évangile de Jésus-Christ, afin qu’il vous fortifiât, et qu’il vous exhortât à demeurer fermes dans votre foi ;  \milestone{3}  et que personne ne, fût ébranlé pour les persécutions qui nous arrivent : car vous savez que c’est à quoi nous sommes destines.  \milestone{4}  Dès lors même que nous étions parmi vous, nous vous prédisions que nous aurions des afflictions à souffrir ; et nous en avons eu en effet, comme vous le savez.  \milestone{5}  Ne pouvant donc attendre plus longtemps, je vous l’ai envoyé pour reconnaître l’état de votre foi, ayant appréhendé que le tentateur ne vous eût tentés, et que notre travail ne devînt inutile.\par
  \milestone{6}  Mais Timothée étant revenu vers nous après vous avoir vus, et nous ayant rendu un si bon témoignage de votre foi et de votre charité, et du souvenir plein d’affection que vous avez sans cesse de nous, qui vous porte à désirer de nous voir, comme nous avons aussi le même désir pour vous ;  \milestone{7}  il est vrai, mes frères, que dans toutes les afflictions et dans tous les maux qui nous arrivent, votre foi nous fait trouver notre consolation en vous.  \milestone{8}  Car nous vivons maintenant, si vous demeurez fermes dans le Seigneur.  \milestone{9}  Et certes quelles assez dignes actions de grâces pouvons-nous rendre à Dieu pour la joie dont nous nous sentons comblés devant lui à cause de vous ?  \milestone{10}  Ce qui nous porte à le conjurer jour et nuit avec une ardeur extrême de nous permettre d’aller vous voir, afin d’ajouter ce qui peut manquer encore à votre foi.\par
  \milestone{11}  Que Dieu lui-même, notre Père, et Jésus-Christ notre Seigneur, nous conduise vers vous !  \milestone{12}  Que le Seigneur vous fasse croître de plus en plus dans la charité que vous avez les uns pour les autres, et envers tous, et qu’il la rende telle que la nôtre est envers vous !  \milestone{13}  Qu’il affermisse vos cœurs, en vous rendant irréprochables par la sainteté devant Dieu, notre Père, au jour que Jésus-Christ notre Seigneur paraîtra avec tous ses saints ! Amen !

\section[{1 Thess 4}]{1 Thess 4}

\noindent \initial{A}{U} reste, mes frères, nous vous supplions et vous conjurons par le Seigneur Jésus, qu’ayant appris de nous comment vous devez marcher dans la voie de Dieu pour lui plaire, vous y marchiez en effet de telle sorte, que vous vous y avanciez de plus en plus.\par
  \milestone{2}  En effet, vous savez quels préceptes nous vous avons donnés de la part du Seigneur Jésus.  \milestone{3}  Car la volonté de Dieu est que vous soyez saints et purs ; que vous vous absteniez de la fornication ;  \milestone{4}  que chacun de vous sache conserver le vase de son corps saintement et honnêtement,  \milestone{5}  et non point en suivant les mouvements de la concupiscence, comme les païens qui ne connaissent point Dieu ;  \milestone{6}  et que surtout à cet égard nul ne passe les bornes, ni ne fasse tort à son frère : parce que le Seigneur est le vengeur de tous ces péchés, comme nous vous l’avons déjà déclaré et assuré de sa part.  \milestone{7}  Car Dieu ne nous a pas appelés pour être impurs, mais pour être saints.  \milestone{8}  Celui donc qui méprise ces règles, méprise non un homme, mais Dieu, qui nous a même donné son Saint-Esprit.\par
  \milestone{9}  Quant à ce qui regarde la charité fraternelle, vous n’avez pas besoin que je vous en écrive, puisque Dieu vous a appris lui-même à vous aimer les uns les autres.  \milestone{10}  Et vraiment vous le faites à l’égard de tous nos frères qui sont dans toute la Macédoine ; mais je vous exhorte, mes frères, de vous avancer de plus en plus dans cet amour ;  \milestone{11}  de vous étudier à vivre en repos ; de vous appliquer chacun à ce que vous avez à faire ; de travailler de vos propres mains, ainsi que nous vous l’avons ordonné :  \milestone{12}  afin que vous vous conduisiez honnêtement envers ceux qui sont hors de l’Église, et que vous vous mettiez en état de n’avoir besoin de personne.\par
\bigbreak
\noindent   \milestone{13}  Or nous ne voulons pas, mes frères, que vous ignoriez ce que vous devez savoir, touchant ceux qui dorment du sommeil de la mort, afin que vous ne vous attristiez pas, comme font les autres hommes qui n’ont point d’espérance.  \milestone{14}  Car si nous croyons que Jésus est mort et ressuscité, nous devons croire aussi que Dieu amènera avec Jésus ceux qui se seront endormis en lui.  \milestone{15}  Ainsi nous vous déclarons, comme, l’ayant appris du Seigneur, que nous qui serons vivants et qui aurons été réservés pour son avènement, nous ne préviendrons point ceux qui seront dans le sommeil de la mort.  \milestone{16}  Car aussitôt que le signal aura été donné par la voix de l’archange, et par le son de la trompette de Dieu, le Seigneur lui-même descendra du ciel, et ceux qui seront morts en Jésus-Christ, ressusciteront d’abord.  \milestone{17}  Puis nous autres qui serons vivants, et qui aurons été réservés jusqu’alors, nous serons emportés avec eux dans les nuées, pour aller au-devant du Seigneur au milieu de l’air ; et ainsi nous serons pour jamais avec le Seigneur.  \milestone{18}  Consolez-vous donc les uns les autres par ces vérités.

\section[{1 Thess 5}]{1 Thess 5}

\noindent \initial{O}{R} pour ce qui regarde le temps et les moments, il n’est pas besoin, mes frères, de vous en écrire ;  \milestone{2}  parce que vous savez bien vous-mêmes, que le jour du Seigneur doit venir comme un voleur de nuit.  \milestone{3}  Car lorsqu’ils diront, Nous voici en paix et en sûreté, ils se trouveront surpris tout d’un coup par une ruine imprévue, comme l’est une femme grosse par les douleurs de l’enfantement, sans qu’il leur reste aucun moyen de se sauver.  \milestone{4}  Mais quant à vous, mes frères, vous n’êtes pas dans les ténèbres pour être surpris de ce jour comme d’un voleur.  \milestone{5}  Vous êtes tous des enfants de lumière et des enfants du jour ; nous ne sommes point enfants de la nuit, ni des ténèbres.  \milestone{6}  Ne dormons donc point comme les autres ; mais veillons, et gardons-nous de l’enivrement de l’âme.  \milestone{7}  Car ceux qui dorment, dorment durant la nuit ; et ceux qui s’enivrent, s’enivrent durant la nuit.  \milestone{8}  Mais nous qui sommes enfants du jour, gardons-nous de cette ivresse ; et armons-nous en prenant pour cuirasse la foi et la charité, et pour casque l’espérance du salut.\par
  \milestone{9}  Car Dieu ne nous a pas destinés à être les objets de sa colère, mais à acquérir le salut par notre Seigneur Jésus-Christ ;  \milestone{10}  qui est mort pour nous, afin que, soit que nous veillions, ou que nous dormions, nous vivions toujours avec lui.  \milestone{11}  C’est pourquoi consolez-vous mutuellement, et édifiez-vous les uns les autres, ainsi que vous le faites.\par
\bigbreak
\noindent   \milestone{12}  Or nous vous supplions, mes frères, de considérer beaucoup ceux qui travaillent parmi vous, qui vous gouvernent selon le Seigneur, et qui vous avertissent de votre devoir ;  \milestone{13}  et d’avoir pour eux une particulière vénération par un sentiment de charité, à cause qu’ils travaillent pour votre salut. Conservez toujours la paix avec eux.\par
  \milestone{14}  Je vous prie encore, mes frères, reprenez ceux qui sont déréglés ; consolez ceux qui ont l’esprit abattu ; supportez les faibles ; soyez patients envers tous.  \milestone{15}  Prenez garde que nul ne rende à un autre le mal pour le mal ; mais cherchez toujours à faire du bien, et à vos frères, et à tout le monde.\par
  \milestone{16}  Soyez toujours dans la joie.  \milestone{17}  Priez sans cesse.  \milestone{18}  Rendez grâces à Dieu en toutes choses : car c’est la ce que Dieu veut que vous fassiez tous en Jésus-Christ.\par
  \milestone{19}  N’éteignez pas l’esprit.  \milestone{20}  Ne méprisez pas les prophéties.  \milestone{21}  Éprouvez tout, et approuvez ce qui est bon.  \milestone{22}  Abstenez-vous de tout ce qui a quelque apparence de mal.\par
  \milestone{23}  Que le Dieu de paix vous sanctifie lui-même en toute manière : afin que tout ce qui est en vous, l’esprit, l’âme et le corps, se conservent sans tache pour l’avènement de notre Seigneur Jésus-Christ.  \milestone{24}  Celui qui vous a appelés est fidèle, et c’est lui qui fera cela en vous.\par
  \milestone{25}  Mes frères, priez pour nous.  \milestone{26}  Saluez tous nos frères en leur donnant le saint baiser.\par
  \milestone{27}  Je vous conjure par le Seigneur, de faire lire cette lettre devant tous les saints frères.\par
  \milestone{28}  La grâce de notre Seigneur Jésus-Christ soit avec vous ! Amen !
\chapterclose


\chapteropen

\chapter[{Seconde épître de Paul aux Thessaloniciens (51\textasciitilde52)}]{Seconde épître de Paul aux Thessaloniciens (51\textasciitilde52)}
\renewcommand{\leftmark}{Seconde épître de Paul aux Thessaloniciens (51\textasciitilde52)}


\chaptercont

\section[{2 Thess 1}]{2 Thess 1}

\noindent \initialiv{P}{AUL}, Silvain, et Timothée : à l’Église de Thessalonique, qui est en Dieu, notre Père, et en Jésus-Christ notre Seigneur.  \milestone{2}  Que Dieu, notre Père, et le Seigneur Jésus-Christ, vous donnent la grâce et la paix !\par
\bigbreak
\noindent   \milestone{3}  Nous devons, mes frères, rendre pour vous à Dieu de continuelles actions de grâces ; et il est bien juste que nous le fassions, puisque votre foi s’augmente de plus en plus, et que la charité que vous avez les uns pour les autres, prend tous les jours un nouvel accroissement :  \milestone{4}  de sorte que nous nous glorifions en vous dans les Églises de Dieu, à cause de la patience et de la foi avec laquelle vous demeurez fermes dans toutes les persécutions et les afflictions qui vous arrivent ;  \milestone{5}  qui sont les marques du juste jugement de Dieu, et qui servent à vous rendre dignes de son royaume, pour lequel aussi vous souffrez.\par
  \milestone{6}  Car il est bien juste devant Dieu, qu’il afflige à leur tour ceux qui vous affligent maintenant ;  \milestone{7}  et qu’il vous console avec nous, vous qui êtes dans l’affliction, lorsque le Seigneur Jésus descendra du ciel, et paraîtra avec les anges qui sont les ministres de sa puissance ;  \milestone{8}  lorsqu’il viendra au milieu des flammes se venger de ceux qui ne connaissent point Dieu, et qui n’obéissent point à l’Évangile de notre Seigneur Jésus-Christ ;  \milestone{9}  qui souffriront la peine d’une éternelle damnation, étant confondus par la face du Seigneur, et par la gloire de sa puissance ;  \milestone{10}  lorsqu’il viendra pour être glorifié dans ses saints, et pour se faire admirer dans tous ceux qui auront cru en lui ; puisque le témoignage que nous avons rendu à sa parole, a été reçu de vous dans l’attente de ce jour-là.  \milestone{11}  C’est pourquoi nous prions sans cesse pour vous, et nous demandons à notre Dieu, qu’il vous rende dignes de sa vocation, et qu’il accomplisse par sa puissance tous les desseins favorables de sa bonté sur vous, et l’œuvre de votre foi :  \milestone{12}  afin que le nom de notre Seigneur Jésus-Christ soit glorifié en vous, et que vous soyez glorifiés en lui, par la grâce de notre Dieu et du Seigneur Jésus-Christ.

\section[{2 Thess 2}]{2 Thess 2}

\noindent \initial{O}{R} nous vous conjurons, mes frères, par l’avènement de notre Seigneur Jésus-Christ, et par notre réunion avec lui,  \milestone{2}  que vous ne vous laissiez pas légèrement ébranler dans votre premier sentiment, et que vous ne vous troubliez pas en croyant sur la foi de quelque prophétie, sur quelque discours, ou sur quelque lettre qu’on supposerait venir de nous, que le jour du Seigneur soit près d’arriver.  \milestone{3}  Que personne ne vous séduise en quelque manière que ce soit : car ce jour ne viendra point que l’apostasie ne soit arrivée auparavant, et qu’on n’ait vu paraître l’homme de péché, cet enfant de perdition,  \milestone{4}  cet ennemi de Dieu, qui s’élèvera au-dessus de tout ce qui est appelé Dieu, ou qui est adoré, jusqu’à s’asseoir dans le temple de Dieu, voulant lui-même passer pour Dieu.\par
  \milestone{5}  Ne vous souvient-il pas que je vous ai dit ces choses, lorsque j’étais encore avec vous ?  \milestone{6}  Et vous savez bien ce qui empêche qu’il ne vienne, afin qu’il paraisse en son temps.  \milestone{7}  Car le mystère d’iniquité se forme dès à présent ; et il reste seulement, que celui qui tient maintenant, tienne encore, jusqu’à ce qu’il soit ôté du monde.  \milestone{8}  Et alors se découvrira l’impie, que le Seigneur Jésus détruira par le souffle de sa bouche, et qu’il perdra par l’éclat de sa présence :  \milestone{9}  cet impie qui doit venir, accompagné de la puissance de Satan, avec toutes sortes de miracles, de signes et de prodiges trompeurs,  \milestone{10}  et avec toutes les illusions qui peuvent porter à l’iniquité ceux qui périssent, parce qu’ils n’ont pas reçu et aimé la vérité pour être sauvés.  \milestone{11}  C’est pourquoi Dieu leur enverra des illusions si efficaces, qu’ils croiront au mensonge ;  \milestone{12}  afin que tous ceux qui n’ont point cru la vérité, mais qui ont consenti à l’iniquité, soient condamnés.\par
\bigbreak
\noindent   \milestone{13}  Quant à nous, mes frères chéris du Seigneur, nous nous sentons obligés de rendre pour vous à Dieu de continuelles actions de grâces, de ce qu’il vous a choisis comme des prémices, pour vous sauver par la sanctification de l’Esprit, et par la foi de la vérité ;  \milestone{14}  vous appelant à cet état par notre Évangile, pour vous faire acquérir la gloire de notre Seigneur Jésus-Christ.  \milestone{15}  C’est pourquoi, mes frères, demeurez fermes, et conservez les traditions que vous avez apprises, soit par nos paroles, soit par notre lettre.  \milestone{16}  Que notre Seigneur Jésus-Christ, et Dieu, notre Père, qui nous a aimés, et qui nous a donné par sa grâce une consolation éternelle et une si heureuse espérance,  \milestone{17}  console lui-même vos cœurs, et vous affermisse dans toutes sortes de bonnes œuvres, et dans la bonne doctrine !

\section[{2 Thess 3}]{2 Thess 3}

\noindent \initial{A}{U} reste, mes frères, priez pour nous, afin que la parole de Dieu se répande de plus en plus, et qu’elle soit honorée partout comme elle l’est parmi vous ;  \milestone{2}  et afin que nous soyons délivrés des hommes intraitables et méchants ; car la foi n’est pas commune à tous.  \milestone{3}  Mais Dieu est fidèle, et il vous affermira, et vous préservera du malin esprit.  \milestone{4}  Pour ce qui vous regarde, nous avons cette confiance en la bonté du Seigneur, que vous accomplissez, et que vous accomplirez à l’avenir, ce que nous vous ordonnons.  \milestone{5}  Que le Seigneur vous donne un cœur droit, dans l’amour de Dieu et dans la patience de Jésus-Christ !\par
  \milestone{6}  Nous vous ordonnons, mes frères, au nom de notre Seigneur Jésus-Christ, de vous retirer de tous ceux d’entre vos frères qui se conduisent d’une manière déréglée, et non selon la tradition et la forme de vie qu’ils ont reçue de nous.  \milestone{7}  Car vous savez vous-mêmes ce qu’il faut faire pour nous imiter, puisqu’il n’y a rien eu de déréglé dans la manière dont nous avons vécu parmi vous.  \milestone{8}  Et nous n’avons mangé gratuitement le pain de personne ; mais nous avons travaillé jour et nuit avec peine et avec fatigue, pour n’être à charge à aucun de vous.\par
  \milestone{9}  Ce n’est pas que nous n’en eussions le pouvoir ; mais c’est que nous avons voulu nous donner nous-mêmes pour modèle, afin que vous nous imitassiez.  \milestone{10}  Aussi lorsque nous étions avec vous, nous vous déclarions, que celui qui ne veut point travailler, ne doit point manger.  \milestone{11}  Car nous apprenons qu’il y en a parmi vous qui se conduisent d’une manière déréglée, qui ne travaillent point, et qui se mêlent de ce qui ne les regarde pas.  \milestone{12}  Or nous ordonnons à ces personnes, et nous les conjurons par notre Seigneur Jésus-Christ, de manger leur pain en travaillant en silence.\par
  \milestone{13}  Et pour vous, mes frères, ne vous lassez point de faire du bien.  \milestone{14}  Si quelqu’un n’obéit pas à ce que nous ordonnons par notre lettre, notez-le, et n’ayez point de commerce avec lui, afin qu’il en ait de la confusion et de la honte.  \milestone{15}  Ne le considérez pas néanmoins comme un ennemi, mais reprenez-le comme votre frère.\par
\bigbreak
\noindent   \milestone{16}  Que le Seigneur de paix vous donne sa paix en tout temps et en tout lieu ! Que le Seigneur soit avec vous tous !\par
  \milestone{17}  Je vous salue ici de ma propre main, moi, Paul. C’est là mon seing dans toutes mes lettres ; j’écris ainsi.  \milestone{18}  La grâce de notre Seigneur Jésus-Christ soit avec vous tous ! Amen !
\chapterclose


\chapteropen

\chapter[{Épitre de Paul aux Galates (54\textasciitilde55)}]{Épitre de Paul aux Galates (54\textasciitilde55)}
\renewcommand{\leftmark}{Épitre de Paul aux Galates (54\textasciitilde55)}


\chaptercont

\section[{Gal 1}]{Gal 1}

\noindent \initialiv{P}{AUL}, apôtre, non de la part des hommes, ni par un homme, mais par Jésus-Christ, et Dieu, son Père, qui l’a ressuscité d’entre les morts ;  \milestone{2}  et tous les frères qui sont avec moi : aux Églises de Galatie. \\
  \milestone{3}  Que la grâce et la paix vous soient données par la bonté de Dieu le Père, et par notre Seigneur Jésus-Christ ;  \milestone{4}  qui s’est livré lui-même pour nos péchés, et pour nous retirer de la corruption du siècle présent, selon la volonté de Dieu, notre Père ;  \milestone{5}  à qui soit gloire dans tous les siècles des siècles ! Amen !  \milestone{6}  Je m’étonne qu’abandonnant celui qui vous a appelés à la grâce de Jésus-Christ, vous passiez sitôt a un autre évangile.\par
\bigbreak
\noindent   \milestone{7}  Ce n’est pas qu’il y en ait d’autre ; mais c’est qu’il y a des gens qui vous troublent, et qui veulent renverser l’Évangile de Jésus-Christ.  \milestone{8}  Mais quand nous vous annoncerions nous-mêmes, ou quand un ange du ciel vous annoncerait un évangile différent de celui que nous vous avons annoncé, qu’il soit anathème !  \milestone{9}  Je vous l’ai dit, et je vous le dis encore une fois : Si quelqu’un vous annonce un évangile différent de celui que vous avez reçu, qu’il soit anathème !  \milestone{10}  Car enfin est-ce des hommes, ou de Dieu, que je recherche maintenant d’être approuvé ? ou ai-je pour but de plaire aux hommes ? Si je voulais encore plaire aux hommes, je ne serais pas serviteur de Jésus-Christ.\par
\bigbreak
\noindent   \milestone{11}  Je vous déclare donc, mes frères, que l’Évangile que je vous ai prêché, n’a rien de l’homme ;  \milestone{12}  parce que je ne l’ai point reçu, ni appris d’aucun homme, mais par la révélation de Jésus-Christ.\par
  \milestone{13}  Car vous savez de quelle manière j’ai vécu autrefois dans le judaïsme, avec quel excès de fureur je persécutais l’Église de Dieu, et la ravageais ;  \milestone{14}  me signalant dans le judaïsme au-dessus de plusieurs de ma nation et de mon âge, et ayant un zèle démesuré pour les traditions de mes pères.  \milestone{15}  Mais lorsqu’il a plu à Dieu, qui m’a choisi particulièrement dès le ventre de ma mère, et qui m’a appelé par sa grâce,  \milestone{16}  de me révéler son Fils, afin que je le prêchasse parmi les nations, je l’ai fait aussitôt, sans prendre conseil de la chair et du sang ;  \milestone{17}  et je ne suis point retourné à Jérusalem, pour voir ceux qui étaient apôtres avant moi ; mais je m’en suis allé en Arabie, et puis je suis revenu encore à Damas.  \milestone{18}  Ainsi trois ans s’étant écoulés, je retournai à Jérusalem pour visiter Pierre ; et je demeurai quinze jours avec lui ;  \milestone{19}  et je ne vis aucun des autres apôtres, sinon Jacques, frère du Seigneur.  \milestone{20}  Je prends Dieu à témoin, que je ne vous mens point en tout ce que je vous écris.  \milestone{21}  J’allai ensuite dans la Syrie et dans la Cilicie.  \milestone{22}  Or les Églises de Judée qui croyaient en Jésus-Christ, ne me connaissaient pas de visage.  \milestone{23}  Ils avaient seulement entendu dire : Celui qui autrefois nous persécutait, annonce maintenant la foi qu’il s’efforçait alors de détruire.  \milestone{24}  Et ils rendaient gloire à Dieu de ce qu’il avait fait en moi.

\section[{Gal 2}]{Gal 2}

\noindent \initial{Q}{UATORZE} ans après, j’allai de nouveau à Jérusalem avec Barnabé, et je pris aussi Tite avec moi.  \milestone{2}  Or j’y allai suivant une révélation que j’en avais eue, et j’exposai aux fidèles, et en particulier à ceux qui paraissaient les plus considérables, l’Évangile que je prêche parmi les gentils ; afin de ne perdre pas le fruit de ce que j’avais déjà fait, ou de ce que je devais faire dans le cours de mon ministère.  \milestone{3}  Mais on n’obligea point Tite, que j’avais amené avec moi, et qui était gentil, de se faire circoncire.  \milestone{4}  Et la considération des faux frères, qui s’étaient introduits par surprise dans l’Église, et qui s’étaient glissés parmi nous, pour observer la liberté que nous avons en Jésus-Christ, et nous réduire en servitude,  \milestone{5}  ne nous porta pas à leur céder, même pour un moment ; et nous refusâmes de nous assujettir à ce qu’ils voulaient, afin que la vérité de l’Évangile demeurât parmi vous.  \milestone{6}  Aussi ceux qui paraissaient les plus considérables (je ne m’arrête pas à ce qu’ils ont été autrefois, Dieu n’a point d’égard à la qualité des personnes) ; ceux, dis-je, qui paraissaient les plus considérables, ne m’ont rien appris de nouveau.  \milestone{7}  Mais au contraire, ayant reconnu que la charge de prêcher l’Évangile aux incirconcis m’avait été donnée, comme à Pierre celle de le prêcher aux circoncis ;  \milestone{8}  (car celui qui a agi efficacement dans Pierre pour le rendre apôtre des circoncis, a aussi agi efficacement en moi pour me rendre apôtre des gentils ;)  \milestone{9}  ceux, dis-je, qui paraissaient comme les colonnes de l’Église, Jacques, Céphas et Jean, ayant reconnu la grâce que j’avais recue, nous donnèrent la main, à Barnabé et à moi, pour marque de la société et de l’union qui était entre eux et nous, afin que nous prêchassions l’Évangile aux gentils, et eux aux circoncis.  \milestone{10}  Ils nous recommandèrent seulement de nous ressouvenir des pauvres : ce que j’ai eu aussi grand soin de faire.\par
\bigbreak
\noindent   \milestone{11}  Or Céphas étant venu à Antioche, je lui résistai en face, parce qu’il était répréhensible.  \milestone{12}  Car avant que quelques-uns qui venaient de la part de Jacques fussent arrivés, il mangeait avec les gentils ; mais après leur arrivée, il se retira, et se sépara d’avec les gentils, craignant de blesser les circoncis.  \milestone{13}  Les autres Juifs usèrent comme lui de cette dissimulation, et Barnabé même s’y laissa aussi emporter.  \milestone{14}  Mais quand je vis qu’ils ne marchaient pas droit selon la vérité de l’Évangile, je dis à Céphas devant tout le monde : Si vous qui êtes Juif, vivez comme les gentils, et non pas comme les Juifs, pourquoi contraignez-vous les gentils de judaïser ?  \milestone{15}  Nous sommes Juifs par notre naissance, et non du nombre des gentils, qui sont des pécheurs.  \milestone{16}  Et cependant, sachant que l’homme n’est point justifié par les œuvres de la loi, mais par la foi en Jésus-Christ, nous avons nous-mêmes cru en Jésus-Christ, pour être justifiés par la foi que nous aurions en lui, et non par les œuvres de la loi ; parce que nul homme ne sera justifié par les œuvres de la loi.  \milestone{17}  Mais si, cherchant à être justifiés par Jésus-Christ, nous sommes aussi nous-mêmes trouvés pécheurs, Jésus-Christ sera-t-il donc ministre du péché ? Dieu nous garde de le penser !  \milestone{18}  Car si je rétablissais de nouveau ce que j’ai détruit, je me ferais voir moi-même prévaricateur.  \milestone{19}  Mais je suis mort à la loi par la loi même, afin de ne vivre plus que pour Dieu. J’ai été crucifié avec Jésus-Christ ;  \milestone{20}  et je vis, ou plutôt ce n’est plus moi qui vis, mais c’est Jésus-Christ qui vit en moi ; et si je vis maintenant dans ce corps mortel, j’y vis en la foi du Fils de Dieu, qui m’a aimé, et qui s’est livré lui-même à la mort pour moi.  \milestone{21}  Je ne veux point rendre la grâce de Dieu inutile. Car si la justice s’acquiert par la loi, Jésus-Christ sera donc mort en vain.

\section[{Gal 3}]{Gal 3}

\noindent \initial{Ô}{} GALATES insensés ! qui vous a ensorcelés, pour vous rendre ainsi rebelles à la vérité, vous, aux yeux de qui Jésus-Christ a été représenté, ayant été lui-même crucifié en vous ?  \milestone{2}  Je ne veux savoir de vous qu’une seule chose : Est-ce par les œuvres de la loi, que vous avez reçu le Saint-Esprit, ou par la foi que vous avez entendu prêcher ?  \milestone{3}  Êtes-vous si insensés qu’après avoir commencé par l’esprit, vous finissiez maintenant par la chair ?  \milestone{4}  Sera-ce donc en vain que vous avez tant souffert ? si toutefois ce n’est qu’en vain.  \milestone{5}  Celui donc qui vous communique son Esprit, et qui fait des miracles parmi vous, le fait-il par les œuvres de la loi, ou par la foi que vous avez entendu prêcher ?\par
\bigbreak
\noindent   \milestone{6}  selon qu’il est écrit d’Abraham, qu’il crut ce que Dieu lui avait dit, et que sa foi lui fut imputée à justice.  \milestone{7}  Sachez donc, que ceux qui s’appuient sur la foi, sont les vrais enfants d’Abraham.  \milestone{8}  Aussi Dieu, dans l’Écriture, prévoyant qu’il justifierait les nations par la foi, l’a annoncé par avance à Abraham, en lui disant : Toutes les nations de la terre seront bénies en vous. !  \milestone{9}  Ceux qui s’appuient sur la foi, sont donc bénis avec le fidèle Abraham.  \milestone{10}  Au lieu que tous ceux qui s’appuient sur les œuvres de la loi, sont dans la malédiction, puisqu’il est écrit : Malédiction sur tous ceux qui n’observent pas tout ce qui est prescrit dans le livre de la loi.  \milestone{11}  Et il est clair que nul par la loi n’est justifié devant Dieu ; puisque, selon l’Écriture : Le juste vit de la foi.  \milestone{12}  Or la loi ne s’appuie point sur la foi ; au contraire, elle dit : Celui qui observera ces préceptes, y trouvera la vie.  \milestone{13}  Mais Jésus-Christ nous a rachetés de la malédiction de la loi, s’étant rendu lui-même malédiction pour nous, selon qu’il est écrit, Maudit est celui qui est pendu au bois ;  \milestone{14}  afin que la bénédiction donnée à Abraham fût communiquée aux gentils en Jésus-Christ, et qu’ainsi nous reçussions par la foi le Saint-Esprit qui avait été promis.\par
\bigbreak
\noindent   \milestone{15}  Mes frères, je me servirai de l’exemple d’une chose humaine et ordinaire : Lorsqu’un homme a fait un testament en bonne forme, nul ne peut le casser, ni y ajouter.  \milestone{16}  Or les promesses de Dieu ont été faites à Abraham, et à celui qui devait naître de lui. Il ne dit pas, À ceux qui naîtront de vous ; comme s’il eût parlé de plusieurs ; mais comme parlant d’un seul, À celui qui naîtra de vous ; qui est Jésus-Christ.  \milestone{17}  Ce que je veux donc dire est, que Dieu ayant fait un testament en bonne forme en faveur de Jésus-Christ, la loi qui n’a été donnée que quatre cent trente ans après, n’a pu le rendre nul, ni en abroger la promesse.  \milestone{18}  Car si c’est par la loi que l’héritage nous est donné, ce n’est donc plus par la promesse. Or c’est par la promesse que Dieu l’a donné à Abraham.  \milestone{19}  À quoi servait donc la loi ? Elle a été établie pour faire connaître les prévarications que l’on commettait en la violant, jusqu’à l’avènement de celui qui devait naître d’Abraham et que la promesse regardait. Et cette loi a été donnée par le ministère des anges, et par l’entremise d’un médiateur.  \milestone{20}  Mais il n’y a point de médiateur quand un seul s’engage : or Dieu traitant avec Abraham est le seul qui s’engage.  \milestone{21}  La loi est-elle donc opposée aux promesses de Dieu ? Nullement. Car si la loi qui a été donnée avait pu donner la vie, on pourrait dire alors avec vérité, que la justice s’obtiendrait par la loi.  \milestone{22}  Mais la loi écrite a comme renfermé tous les hommes sous le péché ; afin que ce que Dieu avait promis, fût donné par la foi en Jésus-Christ par ceux qui croiraient en lui.\par
  \milestone{23}  Or, avant que la foi fût venue, nous étions sous la garde de la loi, qui nous tenait renfermés, pour nous disposer à cette foi qui devait être révélée un jour.  \milestone{24}  Ainsi la loi nous a servi de conducteur, pour nous mener comme des enfants a Jésus-Christ ; afin que nous fussions justifiés par la foi.  \milestone{25}  Mais la foi étant venue, nous ne sommes plus sous un conducteur, comme des enfants ;  \milestone{26}  puisque vous êtes tous enfants de Dieu par la foi en Jésus-Christ.  \milestone{27}  Car vous tous qui avez été baptisés en Jésus-Christ, vous avez été revêtus de Jésus-Christ.  \milestone{28}  Il n’y a plus maintenant ni de Juif, ni de gentil : ni d’esclave, ni de libre ; ni d’homme, ni de femme ; mais vous n’êtes tous qu’un en Jésus-Christ.  \milestone{29}  Si vous êtes à Jésus-Christ, vous êtes donc la race d’Abraham, et les héritiers selon la promesse.

\section[{Gal 4}]{Gal 4}

\noindent \initial{J}{E} dis de plus : Tant que l’héritier est encore enfant, il n’est point différent d’un serviteur, quoiqu’il soit le maître de tout ;  \milestone{2}  mais il est sous la puissance des tuteurs et des curateurs, jusqu’au temps marqué par son père.  \milestone{3}  Ainsi, lorsque nous étions encore enfants, nous étions assujettis aux premières et plus grossières instructions que Dieu a données au monde.  \milestone{4}  Mais lorsque les temps ont été accomplis, Dieu a envoyé son Fils, formé d’une femme, et assujetti a la loi,  \milestone{5}  pour racheter ceux qui étaient sous la loi, et pour nous rendre enfants adoptifs.  \milestone{6}  Et parce que vous êtes enfants, Dieu a envoyé dans vos cœurs l’Esprit de son Fils, qui crie : Mon Père ! mon Père !  \milestone{7}  Aucun de vous n’est donc plus maintenant serviteur, mais enfant. S’il est enfant, il est aussi héritier de Dieu par Jésus-Christ.\par
\bigbreak
\noindent   \milestone{8}  Autrefois, lorsque vous ne connaissiez point Dieu, vous étiez assujettis à ceux qui par leur nature ne sont point véritablement dieux.  \milestone{9}  Mais maintenant, après que vous avez connu Dieu, ou plutôt que vous avez été connus de lui, comment vous tournez vous vers ces observations légales, défectueuses et impuissantes, auxquelles vous voulez vous assujettir par une nouvelle servitude ?  \milestone{10}  Vous observez les jours et les mois, les saisons et les années.  \milestone{11}  J’appréhende pour vous, que je n’aie peut-être travaillé en vain parmi vous.\par
  \milestone{12}  Soyez envers moi comme je suis envers vous ; je vous en prie, mes frères. Vous ne m’avez jamais offensé dans aucune chose.  \milestone{13}  Vous savez que lorsque je vous ai annoncé premièrement l’Évangile, ç’a été parmi les persécutions et les afflictions de la chair ;  \milestone{14}  et que vous ne m’avez point méprisé, ni rejeté, à cause de ces épreuves que je souffrais en ma chair ; mais vous m’avez reçu comme un ange de Dieu, comme Jésus-Christ même.  \milestone{15}  Où est donc le temps où vous vous estimiez si heureux ? Car je puis vous rendre ce témoignage, que vous étiez prêts alors, s’il eût été possible, à vous arracher les yeux mêmes pour me les donner.  \milestone{16}  Suis-je donc devenu votre ennemi, parce que je vous ai dit la vérité ?\par
  \milestone{17}  Ils s’attachent fortement à vous ; mais ce n’est pas d’une bonne affection, puisqu’ils veulent vous séparer de nous, afin que vous vous attachiez fortement à eux.  \milestone{18}  Je veux que vous soyez zélés pour les gens de bien dans le bien, en tout temps, et non pas seulement quand je suis parmi vous.  \milestone{19}  Mes petits enfants, pour qui je sens de nouveau les douleurs de l’enfantement, jusqu’à ce que Jésus-Christ soit formé en vous,  \milestone{20}  je voudrais maintenant être avec vous pour diversifier mes paroles selon vos besoins : car je suis en peine comment je dois vous parler.\par
\bigbreak
\noindent   \milestone{21}  Dites-moi, je vous prie, vous qui voulez être sous la loi, n’avez-vous point lu ce que dit la loi ?  \milestone{22}  Car il est écrit qu’Abraham eut deux fils, l’un de la servante, et l’autre de la femme libre.  \milestone{23}  Mais celui qui naquit de la servante, naquit selon la chair ; et celui qui naquit de la femme libre, naquit en vertu de la promesse de Dieu.  \milestone{24}  Tout ceci est une allégorie. Car ces deux femmes sont les deux alliances, dont la première, qui a été établie sur le mont Sina, et qui n’engendre que des esclaves, est figurée par Agar.  \milestone{25}  Car Sina est une montagne d’Arabie, qui représente la Jérusalem d’ici-bas, qui est esclave avec ses enfants ;  \milestone{26}  au lieu que la Jérusalem d’en haut est vraiment libre ; et c’est elle qui est notre mère.\par
\bigbreak
\noindent   \milestone{27}  Car il est écrit : Réjouissez-vous, stérile, qui n’enfantiez point ; poussez des cris de joie, vous qui ne deveniez point mère : parce que celle qui était délaissée, a plus d’enfants que celle qui a un mari.\par
\bigbreak
\noindent   \milestone{28}  Nous sommes donc, mes frères, les enfants de la promesse, figurés dans Isaac.  \milestone{29}  Et comme alors celui qui était né selon la chair, persécutait celui qui était né selon l’esprit, il en arrive de même encore aujourd’hui.  \milestone{30}  Mais que dit l’Écriture ? Chassez la servante et son fils : car le fils de la servante ne sera point héritier avec le fils de la femme libre.  \milestone{31}  Or, mes frères, nous ne sommes point les enfants de la servante, mais de la femme libre ; et c’est Jésus-Christ qui nous a acquis cette liberté.

\section[{Gal 5}]{Gal 5}

\noindent \initial{T}{ENEZ-VOUS-EN} là, et ne vous mettez point sous le joug d’une nouvelle servitude.  \milestone{2}  Car je vous dis, moi, Paul, que si vous vous faites circoncire, Jésus-Christ ne vous servira de rien.  \milestone{3}  Et de plus, je déclare à tout homme qui se fera circoncire, qu’il est obligé de garder toute la loi.  \milestone{4}  Vous qui voulez être justifiés par la loi, vous n’avez plus de part à Jésus-Christ ; vous êtes déchus de la grâce.  \milestone{5}  Mais pour nous, selon l’impression de l’Esprit de Dieu, c’est en vertu de la foi que nous espérons recevoir la justice.  \milestone{6}  Car en Jésus-Christ, ni la circoncision, ni l’incirconcision, ne servent de rien, mais la foi qui est animée de la charité.\par
  \milestone{7}  Vous couriez si bien ; qui vous a arrêtés pour vous empêcher d’obéir à la vérité ?  \milestone{8}  Ce sentiment dont vous vous êtes laissés persuader, ne vient pas de celui qui vous a appelés.  \milestone{9}  Un peu de levain aigrit toute la pâte.  \milestone{10}  J’espère de la bonté du Seigneur, que vous n’aurez point à l’avenir d’autres sentiments que les miens ; mais celui qui vous trouble, en portera la peine, quel qu’il soit.  \milestone{11}  Et pour moi, mes frères, si je prêche encore la circoncision, pourquoi est-ce que je souffre tant de persécutions ? Le scandale de la croix est donc anéanti ?  \milestone{12}  Plût à Dieu que ceux qui vous troublent, fussent même retranchés du milieu de vous !\par
\bigbreak
\noindent   \milestone{13}  Car vous êtes appelés, mes frères, à un état de liberté ; ayez soin seulement que cette liberté ne vous serve pas d’occasion pour vivre selon la chair ; mais assujettissez-vous les uns aux autres par une charité spirituelle.  \milestone{14}  Car toute la loi est renfermée dans ce seul précepte : Vous aimerez votre prochain comme vous-même.  \milestone{15}  Si vous vous mordez et vous dévorez les uns les autres, prenez garde que vous ne vous consumiez les uns les autres.  \milestone{16}  Je vous le dis donc : conduisez-vous selon l’esprit, et vous n’accomplirez point les désirs de la chair.  \milestone{17}  Car la chair a des désirs contraires à ceux de l’esprit, et l’esprit en a de contraires à ceux de la chair, et ils sont opposés l’un à l’autre ; de sorte que vous ne faites pas les choses que vous voudriez.  \milestone{18}  Si vous êtes poussés par l’esprit, vous n’êtes point sous la loi.\par
  \milestone{19}  Or il est aisé de connaître les œuvres de la chair, qui sont la fornication, l’impureté, l’impudicité, la dissolution,  \milestone{20}  l’idolâtrie, les empoisonnements, les inimitiés, les dissensions, les jalousies, les animosités, les querelles, les divisions, les hérésies,  \milestone{21}  les envies, les meurtres, les ivrogneries, les débauches, et autres crimes semblables, dont je vous déclare, comme je vous l’ai déjà dit, que ceux qui commettent ces crimes, ne seront point héritiers du royaume de Dieu.\par
  \milestone{22}  Les fruits de l’esprit, au contraire, sont la charité, la joie, la paix, la patience, l’humanité, la bonté, la longanimité,  \milestone{23}  la douceur, la foi, la modestie, la continence, la chasteté. Il n’y a point de loi contre ceux qui vivent de la sorte.  \milestone{24}  Or ceux qui sont à Jésus-Christ, ont crucifié leur chair avec ses passions et ses désirs déréglés.  \milestone{25}  Si nous vivons par l’Esprit, conduisons-nous aussi par l’Esprit.\par
\bigbreak
\noindent   \milestone{26}  Ne nous laissons point aller à la vaine gloire, nous piquant les uns les autres, et étant envieux les uns des autres.

\section[{Gal 6}]{Gal 6}

\noindent \initial{M}{ES} frères, si quelqu’un est tombé par surprise en quelque péché, vous autres qui êtes spirituels, ayez soin de le relever dans un esprit de douceur ; chacun de vous faisant réflexion sur soi-même, et craignant d’être tenté aussi bien que lui.  \milestone{2}  Portez les fardeaux les uns des autres ; et vous accomplirez ainsi la loi de Jésus-Christ.  \milestone{3}  Car si quelqu’un s’estime être quelque chose, il se trompe lui-même, parce qu’il n’est rien.  \milestone{4}  Or que chacun examine bien ses propres actions ; et alors il trouvera sa gloire en ce qu’il verra de bon dans lui-même, et non point en se comparant avec les autres.  \milestone{5}  Car chacun portera son propre fardeau.  \milestone{6}  Que celui que l’on instruit dans les choses de la foi, assiste de ses biens en toutes manières celui qui l’instruit.  \milestone{7}  Ne vous trompez pas, on ne se moque point de Dieu.  \milestone{8}  L’homme ne recueillera que ce qu’il aura semé : car celui qui sème dans sa chair, recueillera de la chair la corruption et la mort ; et celui qui sème dans l’esprit, recueillera de l’esprit la vie éternelle.  \milestone{9}  Ne nous lassons donc point de faire le bien, puisque si nous ne perdons point courage, nous en recueillerons le fruit en son temps.  \milestone{10}  C’est pourquoi, pendant que nous en avons le temps, faisons du bien à tous, mais principalement à ceux qu’une même foi a rendus comme nous domestiques du Seigneur.\par
\bigbreak
\noindent   \milestone{11}  Voyez quelle lettre je vous ai écrite de ma propre main.  \milestone{12}  Tous ceux qui mettent leur gloire en des cérémonies charnelles, ne vous obligent à vous faire circoncire, qu’afin de n’être point eux-mêmes persécutés pour la croix de Jésus-Christ.  \milestone{13}  Car eux-mêmes qui sont circoncis, ne gardent point la loi ; mais ils veulent que vous receviez la circoncision, afin qu’ils se glorifient en votre chair.  \milestone{14}  Pour moi, à Dieu ne plaise que je me glorifie en autre chose qu’en la croix de notre Seigneur Jésus-Christ, par qui le monde est mort et crucifié pour moi, comme je suis mort et crucifié pour le monde !  \milestone{15}  Car en Jésus-Christ la circoncision ne sert de rien, ni l’incirconcision, mais l’être nouveau que Dieu crée en nous.  \milestone{16}  Je souhaite la paix et la miséricorde à tous ceux qui se conduiront selon cette règle, et à l’Israël de Dieu.\par
  \milestone{17}  Au reste, que personne ne me cause de nouvelles peines : car je porte imprimées sur mon corps les marques du Seigneur Jésus.  \milestone{18}  Que la grâce de notre Seigneur Jésus-Christ, mes frères, demeure avec votre esprit ! Amen !
\chapterclose


\chapteropen

\chapter[{Épître de Paul aux Philippiens (56\textasciitilde63)}]{Épître de Paul aux Philippiens (56\textasciitilde63)}
\renewcommand{\leftmark}{Épître de Paul aux Philippiens (56\textasciitilde63)}


\chaptercont

\section[{Phil 1}]{Phil 1}

\noindent \initialiv{P}{AUL} et Timothée, serviteurs de Jésus-Christ : à tous les saints en Jésus-Christ qui sont à Philippes, aux évêques et aux diacres.  \milestone{2}  Que Dieu, notre Père, et Jésus-Christ notre Seigneur, vous donnent la grâce et la paix !\par
\bigbreak
\noindent   \milestone{3}  Je rends grâces à mon Dieu toutes les fois que je me souviens de vous ;  \milestone{4}  et je ne fais jamais de prières, que je ne prie aussi pour vous tous, ressentant une grande joie  \milestone{5}  de ce que vous avez reçu l’Évangile, et y avez persévéré depuis le premier jour jusqu’à présent.  \milestone{6}  Car j’ai une ferme confiance, que celui qui a commencé le bien en vous, ne cessera de le perfectionner jusqu’au jour de Jésus-Christ.  \milestone{7}  Et il est juste que j’aie ce sentiment de vous tous, parce que je vous ai dans le cœur, comme ayant tous part à ma joie, par celle que vous avez prise à mes liens, à ma défense, et à l’affermissement de l’Évangile.  \milestone{8}  Car Dieu m’est témoin avec quelle tendresse je vous aime tous dans les entrailles de Jésus-Christ.\par
  \milestone{9}  Et ce que je lui demande est, que votre charité croisse de plus en plus en lumière et en toute intelligence :  \milestone{10}  afin que vous sachiez discerner ce qui est meilleur et plus utile ; que vous soyez purs et sincères ; que vous marchiez jusqu’au jour de Jésus-Christ, sans que votre course soit interrompue par aucune chute ;  \milestone{11}  et que pour la gloire et la louange de Dieu, vous soyez remplis des fruits de justice par Jésus-Christ.\par
\bigbreak
\noindent   \milestone{12}  Or je veux bien que vous sachiez, mes frères, que ce qui m’est arrivé, loin de nuire, a plutôt servi au progrès de l’Évangile ;  \milestone{13}  en sorte que mes liens sont devenus célèbres dans toute la cour de l’empereur, et parmi tous les habitants de Rome, à la gloire de Jésus-Christ ;  \milestone{14}  et que plusieurs de nos frères en notre Seigneur, se rassurant par mes liens, ont conçu une hardiesse nouvelle pour annoncer la parole de Dieu sans aucune crainte.  \milestone{15}  Il est vrai que quelques-uns prêchent Jésus-Christ par un esprit d’envie et de contention, et que les autres le font par une bonne volonté :  \milestone{16}  les uns prêchent Jésus-Christ par charité, sachant que j’ai été établi pour la défense de l’Évangile ;  \milestone{17}  et les autres le prêchent par un esprit de pique et de jalousie, avec une intention qui n’est pas pure, croyant me causer de l’affliction dans mes liens.  \milestone{18}  Mais qu’importe ? pourvu que Jésus-Christ soit annoncé en quelque manière que ce soit, soit par occasion, soit par un vrai zèle ; je m’en réjouis, et m’en réjouirai toujours.  \milestone{19}  Car je sais que l’événement m’en sera salutaire par vos prières, et par l’infusion de l’Esprit de Jésus-Christ ;  \milestone{20}  selon la ferme espérance où je suis, que je ne recevrai point la confusion d’être trompé en rien de ce que j’attends ; mais que parlant avec toute sorte de liberté, Jésus-Christ sera encore maintenant, comme toujours, glorifié dans mon corps, soit par ma vie, soit par ma mort.  \milestone{21}  Car Jésus-Christ est ma vie, et la mort m’est un gain.  \milestone{22}  Si je demeure plus longtemps dans ce corps mortel, je tirerai du fruit de mon travail ; et ainsi je ne sais que choisir.  \milestone{23}  Je me trouve pressé des deux côtés : car d’une part je désire d’être dégagé des liens du corps, et d’être avec Jésus-Christ, ce qui est sans comparaison le meilleur ;  \milestone{24}  et de l’autre, il est plus utile pour votre bien que je demeure encore en cette vie.  \milestone{25}  C’est pourquoi j’ai une certaine confiance, qui me persuade que je demeurerai encore avec vous tous, et que j’y demeurerai même assez longtemps pour votre avancement, et pour la joie de votre foi ;  \milestone{26}  afin que lorsque je serai de nouveau présent parmi vous, je trouve en vous un sujet de me glorifier de plus en plus en Jésus-Christ.\par
\bigbreak
\noindent   \milestone{27}  Ayez soin seulement de vous conduire d’une manière digne de l’Évangile de Jésus-Christ : afin que je voie moi-même étant présent parmi vous, ou que j’entende dire en étant absent, que vous demeurez fermes dans un même esprit, combattant tous d’un même cœur pour la foi de l’Évangile ;  \milestone{28}  et que vous demeuriez intrépides parmi tous les efforts de vos adversaires, ce qui est pour eux le sujet de leur perte, comme pour vous celui de votre salut ; et cet avantage vous vient de Dieu.  \milestone{29}  Car c’est une grâce qu’il vous a faite, non-seulement de ce que vous croyez en Jésus-Christ, mais encore de ce que vous souffrez pour lui ;  \milestone{30}  vous trouvant dans les mêmes combats où vous m’avez vu, et où vous entendez dire que je suis encore maintenant.

\section[{Phil 2}]{Phil 2}

\noindent \initial{S}{I} donc il y a quelque consolation en Jésus-Christ ; s’il y a quelque douceur et quelque soulagement dans la charité ; s’il y a quelque union dans la participation d’un même esprit ; s’il y a quelque tendresse et quelque compassion parmi nous,  \milestone{2}  rendez ma joie parfaite, vous tenant tous unis ensemble, n’ayant tous qu’un même amour, une même âme, et les mêmes sentiments ;  \milestone{3}  en sorte que vous ne fassiez rien par un esprit de contention ou de vaine gloire ; mais que chacun par humilité croie les autres au-dessus de soi.  \milestone{4}  Que chacun ait égard, non à ses propres intérêts, mais à ceux des autres.\par
  \milestone{5}  Soyez dans la même disposition et dans le même sentiment où a été Jésus-Christ ;\par
  \milestone{6}  qui, ayant la forme et la nature de Dieu, n’a point cru que ce fût pour lui une usurpation d’être égal à Dieu ;\par
  \milestone{7}  mais il s’est anéanti lui-même en prenant la forme et la nature de serviteur, en se rendant semblable aux hommes, et étant reconnu pour homme par tout ce qui a paru de lui au-dehors.\par
  \milestone{8}  Il s’est rabaissé lui-même, se rendant obéissant jusqu’à la mort, et jusqu’à la mort de la croix.\par
  \milestone{9}  C’est pourquoi Dieu l’a élevé par-dessus toutes choses, et lui a donné un nom qui est au-dessus de tout nom :\par
  \milestone{10}  afin qu’au nom de Jésus tout genou fléchisse dans le ciel, sur la terre, et dans les enfers ;\par
  \milestone{11}  et que toute langue confesse que le Seigneur Jésus-Christ est dans la gloire de Dieu, son Père.\par
\bigbreak
\noindent   \milestone{12}  Ainsi, mes chers frères, comme vous avez toujours été obéissants, ayez soin, non-seulement lorsque je suis présent parmi vous, mais encore plus maintenant que je suis absent, d’opérer votre salut avec crainte et tremblement.  \milestone{13}  Car c’est Dieu qui opère en vous et le vouloir et le faire, selon qu’il lui plaît.  \milestone{14}  Faites donc toutes choses sans murmures et sans disputes ;  \milestone{15}  afin que vous soyez irrépréhensibles et sincères, et qu’étant enfants de Dieu, vous soyez sans tache au milieu d’une nation dépravée et corrompue, parmi laquelle vous brillez comme des astres dans le monde ;  \milestone{16}  portant en vous la parole de vie, pour m’être un sujet de gloire au jour de Jésus-Christ, comme n’ayant pas couru en vain, ni travaillé en vain.  \milestone{17}  Mais quand même je devrais répandre mon sang sur la victime et le sacrifice de votre foi, je m’en réjouirais en moi-même, et je m’en conjouirais avec vous tous ;  \milestone{18}  et vous devriez aussi vous en réjouir, et vous en conjouir avec moi.\par
\bigbreak
\noindent   \milestone{19}  J’espère qu’avec la grâce du Seigneur Jésus, je vous enverrai bientôt Timothée, afin que je sois aussi consolé apprenant de vos nouvelles ;  \milestone{20}  n’ayant personne qui soit autant que lui uni avec moi d’esprit et de cœur, ni qui se porte plus sincèrement à prendre soin de ce qui vous touche :  \milestone{21}  car tous cherchent leurs propres intérêts, et non ceux de Jésus-Christ.  \milestone{22}  Or vous savez déjà l’épreuve que j’ai faite de lui, puisqu’il a servi avec moi dans la prédication de l’Évangile, comme un fils sert à son père.  \milestone{23}  J’espère donc vous l’envoyer aussitôt que j’aurai mis ordre à ce qui me regarde ;  \milestone{24}  et je me promets aussi de la bonté du Seigneur, que j’irai moi-même vous voir bientôt.\par
  \milestone{25}  Cependant j’ai cru qu’il était nécessaire de vous renvoyer mon frère Épaphrodite, qui est mon aide dans mon ministère, et mon compagnon dans mes combats, qui est votre apôtre, et qui m’a servi dans mes besoins :  \milestone{26}  parce qu’il désirait de vous voir tous ; et il était fort en peine de ce que vous aviez su sa maladie.  \milestone{27}  Car il a été en effet malade jusqu’à la mort : mais Dieu a eu pitié de lui ; et non-seulement de lui, mais aussi de moi, afin que je n’eusse pas affliction sur affliction.  \milestone{28}  C’est pourquoi je me suis hâté de vous le renvoyer, pour vous donner la joie de le revoir, et pour me tirer moi-même de peine.  \milestone{29}  Recevez-le donc avec toute sorte de joie en notre Seigneur, et honorez de telles personnes.  \milestone{30}  Car il s’est vu tout proche de la mort pour avoir voulu servir à l’œuvre de Jésus-Christ, exposant sa vie afin de suppléer par son assistance à celle que vous ne pouviez me rendre vous-mêmes.

\section[{Phil 3}]{Phil 3}

\noindent \initial{A}{U} reste, mes frères, réjouissez-vous en notre Seigneur. Il ne m’est pas pénible, et il vous est avantageux que je vous écrive les mêmes choses.  \milestone{2}  Gardez-vous des chiens, gardez-vous des mauvais ouvriers, gardez-vous des faux circoncis.  \milestone{3}  Car c’est nous qui sommes les vrais circoncis, puisque nous servons Dieu en esprit, et que nous nous glorifions en Jésus-Christ, sans nous flatter d’aucun avantage charnel.\par
  \milestone{4}  Ce n’est pas que je ne puisse prendre moi-même avantage de ce qui n’est que charnel ; et si quelqu’un croit pouvoir le faire, je le puis encore plus que lui :  \milestone{5}  ayant été circoncis au huitième jour, étant de la race d’Israël, de la tribu de Benjamin, né Hébreu, de pères hébreux ; pour ce qui est de la manière d’observer la loi, ayant été pharisien ;  \milestone{6}  pour ce qui est du zèle du judaïsme, en ayant eu jusqu’à persécuter l’Église ; et pour ce qui est de la justice de la loi, ayant mené une vie irréprochable.\par
  \milestone{7}  Mais ce que je considérais alors comme un gain et un avantage, m’a paru depuis, en regardant Jésus-Christ, un désavantage et une perte.  \milestone{8}  Je dis plus : Tout me semble une perte au prix de cette haute connaissance de Jésus-Christ mon Seigneur, pour l’amour duquel je me suis privé de toutes choses, les regardant comme des ordures, afin que je gagne Jésus-Christ ;  \milestone{9}  que je sois trouvé en lui, n’ayant point une justice qui me soit propre, et qui me soit venue de la loi ; mais ayant celle qui naît de la foi en Jésus-Christ, cette justice qui vient de Dieu par la foi ;  \milestone{10}  et que je connaisse Jésus-Christ, avec la vertu de sa résurrection, et la participation de ses souffrances, étant rendu conforme à sa mort,  \milestone{11}  pour tâcher enfin de parvenir à la bienheureuse résurrection des morts.  \milestone{12}  Ce n’est pas que j’aie déjà reçu ce que j’espère, ou que je sois déjà parfait ; mais je poursuis ma course, pour tâcher d’atteindre où Jésus-Christ m’a destiné en me prenant.  \milestone{13}  Mes frères, je ne pense point avoir encore atteint ou je tends ; mais tout ce que je fais maintenant, c’est qu’oubliant ce qui est derrière moi, et m’avançant vers ce qui est devant moi,  \milestone{14}  je cours incessamment vers le bout de la carrière, pour remporter le prix de la félicité du ciel, à laquelle Dieu nous a appelés par Jésus-Christ.  \milestone{15}  Tout ce que nous sommes donc de parfaits, soyons dans ce sentiment ; et si en quelque point vous pensez autrement, Dieu vous découvrira aussi ce que vous devez en croire.  \milestone{16}  Cependant pour ce qui regarde les points à l’égard desquels nous sommes parvenus à être dans les mêmes sentiments, demeurons tous dans la même règle.\par
  \milestone{17}  Mes frères, rendez-vous mes imitateurs, et proposez-vous l’exemple de ceux qui se conduisent selon le modèle que vous avez vu en nous.  \milestone{18}  Car il y en a plusieurs dont je vous ai souvent parlé, et dont je vous parle encore avec larmes, qui se conduisent en ennemis de la croix de Jésus-Christ ;  \milestone{19}  qui auront pour fin la damnation, qui font leur Dieu de leur ventre, qui mettent leur gloire dans leur propre honte, et qui n’ont de pensées et d’affections que pour la terre.  \milestone{20}  Mais pour nous, nous vivons déjà dans le ciel, comme en étant citoyens ; et c’est de là aussi que nous attendons le Sauveur, notre Seigneur Jésus-Christ ;  \milestone{21}  qui transformera notre corps tout vil et abject qu’il est, afin de le rendre conforme à son corps glorieux, par cette vertu efficace par laquelle il peut s’assujettir toutes choses.

\section[{Phil 4}]{Phil 4}

\noindent \initial{C\kern-0.08em{’}}{EST} pourquoi, mes très-chers et très-aimés frères, qui êtes ma joie et ma couronne, continuez, mes bien-aimés, et demeurez fermes dans le Seigneur.\par
\bigbreak
\noindent   \milestone{2}  Je conjure Évodie, et je conjure Syntyche, de s’unir dans les mêmes sentiments en notre Seigneur.  \milestone{3}  Je vous prie aussi, vous qui avez été le fidèle compagnon de mes travaux, de les assister, elles qui ont travaillé avec moi dans l’établissement de l’Évangile, avec Clément et les autres qui m’ont aidé dans mon ministère, dont les noms sont écrits dans le livre de vie.\par
  \milestone{4}  Soyez toujours dans la joie en notre Seigneur ; je le dis encore une fois, soyez dans la joie.  \milestone{5}  Que votre modestie soit connue de tous les hommes. Le Seigneur est proche.  \milestone{6}  Ne vous inquiétez de rien ; mais en quelque état que vous soyez, présentez à Dieu vos demandes par des supplications et des prières, accompagnées d’actions de grâces.  \milestone{7}  Et que la paix de Dieu, qui surpasse toutes pensées, garde vos cœurs et vos esprits en Jésus-Christ !\par
  \milestone{8}  Enfin, mes frères, que tout ce qui est véritable et sincère, tout ce qui est honnête, tout ce qui est juste, tout ce qui est saint, tout ce qui peut vous rendre aimable, tout ce qui est d’édification et de bonne odeur, tout ce qui est vertueux, et tout ce qui est louable dans le règlement des mœurs, soit l’entretien de vos pensées.  \milestone{9}  Pratiquez ce que vous avez appris et reçu de moi, ce que vous avez entendu dire de moi, et ce que vous avez vu en moi ; et le Dieu de paix sera avec vous.\par
\bigbreak
\noindent   \milestone{10}  Au reste, j’ai reçu une grande joie en notre Seigneur, de ce qu’enfin vous avez renouvelé les sentiments que vous aviez pour moi ; non que vous ne les eussiez toujours dans le cœur, mais vous n’aviez pas d’occasion de les faire paraître.  \milestone{11}  Ce n’est pas la vue de mon besoin qui me fait parler de la sorte : car j’ai appris à me contenter de l’état où je me trouve.  \milestone{12}  Je sais vivre pauvrement ; je sais vivre dans l’abondance : ayant éprouvé de tout, je suis fait à tout, au bon traitement et à la faim, à l’abondance et à l’indigence.  \milestone{13}  Je puis tout en celui qui me fortifie.  \milestone{14}  Vous avez bien fait néanmoins de prendre part à l’affliction où je suis.  \milestone{15}  Or vous savez, mes frères de Philippes, qu’après avoir commencé à vous prêcher l’Évangile, ayant depuis quitté la Macédoine, nulle autre Église ne m’a fait part de ses biens, et que je n’ai rien reçu que de vous seuls,  \milestone{16}  qui m’avez envoyé deux fois à Thessalonique de quoi satisfaire à mes besoins.\par
  \milestone{17}  Ce n’est pas que je désire vos dons ; mais je désire le fruit que vous en tirez, qui augmentera le compte que Dieu tient de vos bonnes œuvres.  \milestone{18}  Or j’ai maintenant tout ce que vous m’avez envoyé, et je suis dans l’abondance : je suis rempli de vos biens que j’ai recus d’Épaphrodite, comme une oblation d’excellente odeur, comme une hostie que Dieu accepte volontiers, et qui lui est agréable.  \milestone{19}  Je souhaite que mon Dieu, selon les richesses de sa bonté, remplisse tous vos besoins, et vous donne encore sa gloire par Jésus-Christ.  \milestone{20}  Gloire soit à Dieu, notre Père, dans tous les siècles des siècles ! Amen !\par
\bigbreak
\noindent   \milestone{21}  Saluez de ma part tous les saints en Jésus-Christ.  \milestone{22}  Les frères qui sont avec moi vous saluent. Tous les saints vous saluent, mais principalement ceux qui sont de la maison de César.  \milestone{23}  La grâce de notre Seigneur Jésus-Christ soit avec votre esprit ! Amen !
\chapterclose


\chapteropen

\chapter[{Épître de Paul à Philémon (55\textasciitilde63)}]{Épître de Paul à Philémon (55\textasciitilde63)}
\renewcommand{\leftmark}{Épître de Paul à Philémon (55\textasciitilde63)}


\chaptercont

\section[{Phm 1}]{Phm 1}

\noindent \initialiv{P}{AUL}, prisonnier de Jésus-Christ ; et Timothée, son frère : à notre cher Philémon, notre coopérateur ;  \milestone{2}  à notre très-chère sœur Appie ; à Archippe, le compagnon de nos combats, et a l’Église qui est en votre maison.  \milestone{3}  Que Dieu, notre Père, et Jésus-Christ notre Seigneur, vous donnent la grâce et la paix !\par
  \milestone{4}  Me souvenant sans cesse de vous dans mes prières, je rends grâces à mon Dieu,  \milestone{5}  apprenant quelle est votre foi envers le Seigneur Jésus, et votre charité envers tous les saints ;  \milestone{6}  et de quelle sorte la libéralité qui naît de votre foi, éclate aux yeux de tout le monde, se faisant connaître par tant de bonnes œuvres qui se pratiquent dans votre maison pour l’amour de Jésus-Christ.  \milestone{7}  Car votre charité, mon cher frère, nous a comblés de joie et de consolation, voyant que les cœurs des saints ont reçu tant de soulagement de votre bonté.\par
  \milestone{8}  C’est pourquoi, encore que je puisse prendre en Jésus-Christ une entière liberté de vous ordonner une chose qui est de votre devoir ;  \milestone{9}  néanmoins l’amour que j’ai pour vous, fait que j’aime mieux vous supplier, quoique je sois tel que je suis à votre égard, c’est-à-dire, quoique je sois Paul, et déjà vieux, et de plus maintenant prisonnier de Jésus-Christ.  \milestone{10}  Or la prière que je vous fais est pour mon fils Onésime, que j’ai engendré dans mes liens ;  \milestone{11}  qui vous a été autrefois inutile, mais qui vous sera maintenant très - utile, aussi bien qu’à moi.  \milestone{12}  Je vous le renvoie, et je vous prie de le recevoir comme mes entrailles.  \milestone{13}  J’avais pensé de le retenir auprès de moi, afin qu’il me rendît quelque service en votre place, dans les chaînes que je porte pour l’Évangile ;  \milestone{14}  mais je n’ai rien voulu faire sans votre avis, désirant que le bien que je vous propose n’ait rien de forcé, mais soit entièrement volontaire.  \milestone{15}  Car peut-être qu’il n’a été séparé de vous pour un temps, qu’afin que vous le recouvriez pour jamais,  \milestone{16}  non plus comme un simple esclave, mais comme celui qui d’esclave est devenu l’un de nos frères bien-aimés, qui m’est très-cher à moi en particulier, et qui doit vous l’être encore beaucoup plus, étant à vous et selon le monde, et selon le Seigneur.\par
  \milestone{17}  Si donc vous me considérez comme étroitement uni à vous, recevez-le comme moi-même.  \milestone{18}  S’il vous a fait tort, ou s’il vous est redevable de quelque chose, mettez cela sur mon compte.  \milestone{19}  C’est moi, Paul, qui vous écris de ma main ; c’est moi qui vous le rendrai, pour ne pas vous dire que vous vous devez vous-même à moi.  \milestone{20}  Oui, mon frère, que je reçoive de vous cet avantage dans le Seigneur. Donnez-moi, au nom du Seigneur, cette sensible consolation.  \milestone{21}  Je vous écris ceci dans la confiance que votre soumission me donne, sachant que vous en ferez encore plus que je ne dis.\par
  \milestone{22}  Je vous prie aussi de me préparer un logement. Car j’espère que Dieu me redonnera à vous encore une fois, par le mérite de vos prières.  \milestone{23}  Épaphras, qui est comme moi prisonnier pour Jésus-Christ, vous salue,  \milestone{24}  avec Marc, Aristarque, Démas et Luc, qui sont mes coopérateurs.  \milestone{25}  Que la grâce de notre Seigneur Jésus-Christ soit avec votre esprit ! Amen !
\chapterclose


\chapteropen

\chapter[{Première épître de Paul aux Corinthiens (56\textasciitilde57)}]{Première épître de Paul aux Corinthiens (56\textasciitilde57)}
\renewcommand{\leftmark}{Première épître de Paul aux Corinthiens (56\textasciitilde57)}


\chaptercont

\section[{1 Cor 1}]{1 Cor 1}

\noindent \initialiv{P}{AUL}, apôtre de Jésus-Christ, par la vocation et la volonté de Dieu ; et Sosthène, son frère :  \milestone{2}  à l’Église de Dieu qui est à Corinthe, aux fidèles que Jésus-Christ a sanctifiés, et qu’il a appelés pour être saints, et à tous ceux qui, en quelque lieu que ce soit, invoquent le nom de notre Seigneur Jésus-Christ, qui est leur Seigneur comme le nôtre.  \milestone{3}  Dieu, notre Père, et Jésus-Christ notre Seigneur, vous donnent la grâce et la paix !\par
  \milestone{4}  Je rends pour vous à mon Dieu des actions de grâces continuelles, à cause de la grâce de Dieu, qui vous a été donnée en Jésus-Christ ;  \milestone{5}  et de toutes les richesses dont vous avez été comblés en lui, dans tout ce qui regarde le don de la parole et de la science :  \milestone{6}  le témoignage qu’on vous a rendu de Jésus-Christ, ayant été ainsi confirmé parmi vous ;  \milestone{7}  de sorte qu’il ne vous manque aucun don, dans l’attente où vous êtes de la manifestation de notre Seigneur Jésus-Christ.  \milestone{8}  Et Dieu vous affermira encore jusqu’à la fin, afin que vous soyez trouvés irrépréhensibles au jour de l’avènement de Jésus-Christ notre Seigneur.  \milestone{9}  Dieu, par lequel vous avez été appelés à la société de son Fils Jésus-Christ notre Seigneur, est fidèle et véritable.\par
\bigbreak
\noindent   \milestone{10}  Or je vous conjure, mes frères, par le nom de Jésus-Christ notre Seigneur, d’avoir tous un même langage, et de ne point souffrir parmi vous de divisions, ni de schismes, mais d’être tous unis ensemble dans un même esprit et dans un même sentiment.  \milestone{11}  Car j’ai été averti, mes frères, par ceux de la maison de Chloé, qu’il y a des contestations parmi vous.  \milestone{12}  Ce que je veux dire est, que chacun de vous prend parti en disant : Pour moi, je suis à Paul ; Et moi, je suis à Apollon ; Et moi, je suis à Céphas ; Et moi, je suis à Jésus-Christ.  \milestone{13}  Jésus-Christ est-il donc divisé ? Est-ce Paul qui a été crucifié pour vous ? ou avez-vous été baptisés au nom de Paul ?  \milestone{14}  Je rends grâces à Dieu de ce que je n’ai baptisé aucun de vous, sinon Crispe et Caïus ;  \milestone{15}  afin que personne ne dise que vous avez été baptisés en mon nom.  \milestone{16}  J’ai encore baptisé ceux de la famille de Stéphanas ; et je ne sache point en avoir baptisé d’autres :  \milestone{17}  parce que Jésus-Christ ne m’a pas envoyé pour baptiser, mais pour prêcher l’Evangile, et le prêcher sans y employer la sagesse de la parole, pour ne pas anéantir la vertu de la croix de Jésus-Christ.\par
\bigbreak
\noindent   \milestone{18}  Car la parole de la croix est une folie pour ceux qui se perdent ; mais pour ceux qui se sauvent, c’est-à-dire, pour nous, elle est l’instrument de la puissance de Dieu.  \milestone{19}  C’est pourquoi il est écrit : Je détruirai la sagesse des sages, et je rejetterai la science des savants.  \milestone{20}  Que sont devenus les sages ? Que sont devenus les docteurs de la loi ? Que sont devenus ces esprits curieux des sciences de ce siècle ? Dieu n’a-t-il pas convaincu de folie la sagesse de ce monde ?  \milestone{21}  Car Dieu voyant que le monde avec la sagesse humaine, ne l’avait point connu dans les ouvrages de sa sagesse divine, il lui a plu de sauver par la folie de la prédication ceux qui croiraient en lui.  \milestone{22}  Les Juifs demandent des miracles, et les gentils cherchent la sagesse.  \milestone{23}  Et pour nous, nous prêchons Jésus-Christ crucifié, qui est un scandale aux Juifs, et une folie aux gentils ;  \milestone{24}  mais qui est la force de Dieu et la sagesse de Dieu pour ceux qui sont appelés, soit Juifs ou gentils :  \milestone{25}  parce que ce qui paraît en Dieu une folie, est plus sage que la sagesse de tous les hommes ; et que ce qui paraît en Dieu une faiblesse, est plus fort que la force de tous les hommes.\par
\bigbreak
\noindent   \milestone{26}  Considérez, mes frères, qui sont ceux d’entre vous qui ont été appelés à la foi : il y en a peu de sages selon la chair, peu de puissants, et peu de nobles.  \milestone{27}  Mais Dieu a choisi les moins sages selon le monde, pour confondre les sages ; il a choisi les faibles selon le monde, pour confondre les puissants ;  \milestone{28}  il a choisi les plus vils et les plus méprisables selon le monde, et ce qui n’était rien, pour détruire ce qui était de plus grand ;  \milestone{29}  afin que nul homme ne se glorifie devant lui.  \milestone{30}  C’est par lui que vous êtes établis en Jésus-Christ, qui nous a été donné de Dieu pour être notre sagesse, notre justice, notre sanctification et notre rédemption ;  \milestone{31}  afin que, selon qu’il est écrit, celui qui se glorifie, ne se glorifie que dans le Seigneur.

\section[{1 Cor 2}]{1 Cor 2}

\noindent \initial{P}{OUR} moi, mes frères, lorsque je suis venu vers vous pour vous annoncer l’Evangile de Jésus-Christ, je n’y suis point venu avec les discours élevés d’une éloquence et d’une sagesse humaine.  \milestone{2}  Car je n’ai point fait profession de savoir autre chose parmi vous que Jésus-Christ, et Jésus-Christ crucifié.  \milestone{3}  Et tant que j’ai été parmi vous, j’y ai toujours été dans un état de faiblesse, de crainte et de tremblement.  \milestone{4}  Je n’ai point employé en vous parlant et en vous prêchant, les discours persuasifs de la sagesse humaine, mais les effets sensibles de l’Esprit et de la vertu de Dieu ;  \milestone{5}  afin que votre foi ne soit pas établie sur la sagesse des hommes, mais sur la puissance de Dieu.\par
\bigbreak
\noindent   \milestone{6}  Nous prêchons néanmoins la sagesse aux parfaits, non la sagesse de ce monde, ni des princes de ce monde, qui se détruisent :  \milestone{7}  mais nous prêchons la sagesse de Dieu, renfermée dans son mystère, cette sagesse cachée qu’il avait prédestinée et préparée avant tous les siècles pour notre gloire ;  \milestone{8}  que nul des princes de ce monde n’a connue, puisque s’ils l’eussent connue, ils n’eussent jamais crucifié le Seigneur de la gloire ;  \milestone{9}  et de laquelle il est écrit : Que l’œil n’a point vu, que l’oreille n’a point entendu, et que le cœur de l’homme n’a jamais conçu ce que Dieu a préparé pour ceux qui l’aiment.  \milestone{10}  Mais pour nous, Dieu nous l’a révélé par son Esprit, parce que l’Esprit de Dieu pénètre tout, et même ce qu’il y a de plus caché dans la profondeur de Dieu.  \milestone{11}  Car qui des hommes connaît ce qui est en l’homme, sinon l’esprit de l’homme, qui est en lui ? Ainsi nul ne connaît ce qui est en Dieu, que l’Esprit de Dieu.  \milestone{12}  Or nous n’avons point reçu l’esprit du monde, mais l’Esprit de Dieu, afin que nous connaissions les dons que Dieu nous a faits ;  \milestone{13}  et nous les annonçons, non avec les discours qu’enseigne la sagesse humaine, mais avec ceux que le Saint-Esprit enseigne, traitant spirituellement les choses spirituelles.  \milestone{14}  Or l’homme animal n’est point capable des choses qui sont de l’Esprit de Dieu : elles lui paraissent une folie, et il ne peut les comprendre ; parce que c’est par une lumière spirituelle qu’on doit en juger.  \milestone{15}  Mais l’homme spirituel juge de tout, et n’est jugé de personne.  \milestone{16}  Car qui connaît l’Esprit du Seigneur et qui peut l’instruire et le conseiller ? Mais pour nous, nous avons l’Esprit de Jésus-Christ.

\section[{1 Cor 3}]{1 Cor 3}

\noindent \initial{A}{USSI}, mes frères, je n’ai pu vous parler comme à des hommes spirituels, mais comme à des personnes encore charnelles, comme à des enfants en Jésus-Christ.  \milestone{2}  Je ne vous ai nourris que de lait, et non de viandes solides, parce que vous n’en étiez pas capables ; et à présent même vous ne l’êtes pas encore, parce que vous êtes encore charnels.  \milestone{3}  Car puisqu’il y a parmi vous des jalousies et des disputes, n’est-il pas visible que vous êtes charnels, et que votre conduite est bien humaine ?  \milestone{4}  En effet, puisque l’un dit, Je suis à Paul ; et l’autre, Je suis à Apollon ; n’êtes-vous pas encore charnels ? Qu’est donc Paul ? et qu’est Apollon ?\par
  \milestone{5}  Ce sont des ministres de celui en qui vous avez cru, et chacun selon le don qu’il a reçu du Seigneur.  \milestone{6}  C’est moi qui ai planté, c’est Apollon qui a arrosé ; mais c’est Dieu qui a donné l’accroissement.  \milestone{7}  Ainsi celui qui plante, n’est rien ; celui qui arrose, n’est rien : mais tout vient de Dieu, qui donne l’accroissement.  \milestone{8}  Celui donc qui plante, et celui qui arrose, ne sont qu’une même chose ; mais chacun recevra sa récompense particulière, selon son travail.  \milestone{9}  Car nous sommes les coopérateurs de Dieu ; et vous, vous êtes le champ que Dieu cultive, et l’édifice que Dieu bâtit.  \milestone{10}  Pour moi, selon la grâce que Dieu m’a donnée, j’ai jeté le fondement comme fait un sage architecte : un autre bâtit dessus ; mais que chacun prenne garde comment il bâtit sur ce fondement.  \milestone{11}  Car personne ne peut poser d’autre fondement que celui qui a été posé ; et ce fondement, c’est Jésus-Christ.  \milestone{12}  Si l’on élève sur ce fondement un édifice d’or, d’argent, de pierres précieuses, de bois, de foin, de paille ;  \milestone{13}  l’ouvrage de chacun paraîtra enfin, et le jour du Seigneur fera voir quel il est : parce que ce jour sera manifesté par le feu, et que le feu mettra à l’épreuve l’ouvrage de chacun.  \milestone{14}  Si l’ouvrage que quelqu’un aura bâti sur ce fondement, demeure sans être brûlé, il en recevra la récompense.  \milestone{15}  Si au contraire l’ouvrage de quelqu’un est brûlé, il en souffrira la perte : il ne laissera pas néanmoins d’être sauvé, mais comme en passant par le feu.\par
  \milestone{16}  Ne savez-vous pas que vous êtes le temple de Dieu, et que l’Esprit de Dieu habite en vous ?  \milestone{17}  Si quelqu’un profane le temple de Dieu, Dieu le perdra : car le temple de Dieu est saint ; et c’est vous qui êtes ce temple.\par
  \milestone{18}  Que nul ne se trompe soi-même. Si quelqu’un d’entre vous pense être sage selon le monde, qu’il devienne fou pour devenir sage :  \milestone{19}  car la sagesse de ce monde est une folie devant Dieu ; selon qu’il est écrit : Je surprendrai les sages par leur fausse prudence.  \milestone{20}  Et ailleurs : Le Seigneur connaît les pensées des sages, et il sait combien elles sont vaines.  \milestone{21}  Que personne donc ne mette sa gloire dans les hommes.  \milestone{22}  Car tout est à vous : soit Paul, soit Apollon, soit Céphas, soit le monde, soit la vie, soit la mort, soit les choses présentes, soit les choses futures : tout est à vous.  \milestone{23}  Et vous, vous êtes à Jésus-Christ ; et Jésus-Christ est à Dieu.

\section[{1 Cor 4}]{1 Cor 4}

\noindent \initial{Q}{UE} les hommes nous considèrent comme les ministres de Jésus-Christ, et les dispensateurs des mystères de Dieu.  \milestone{2}  Or ce qui est à désirer dans les dispensateurs, est qu’ils soient trouvés fidèles.  \milestone{3}  Pour moi, je me mets fort peu en peine d’être jugé par vous, ou par quelque homme que ce soit ; je n’ose pas même me juger moi-même.  \milestone{4}  Car encore que ma conscience ne me reproche rien, je ne suis pas justifié pour cela ; mais c’est le Seigneur qui est mon juge.  \milestone{5}  C’est pourquoi ne jugez point avant le temps, jusqu’à ce que le Seigneur vienne : c’est lui qui portera la lumière dans les ténèbres les plus profondes, et qui découvrira les plus secrètes pensées des cœurs ; et alors chacun recevra de Dieu la louange qui lui sera due.\par
  \milestone{6}  Au reste, mes frères, j’ai proposé ces choses sous mon nom, et sous celui d’Apollon, à cause de vous : afin que vous appreniez, par notre exemple, à n’avoir pas de vous d’autres sentiments que ceux que je viens de marquer ; et que nul, pour s’attacher à quelqu’un, ne s’enfle de vanité contre un autre.  \milestone{7}  Car qui est-ce qui vous discerne d’entre les autres ? Qu’avez-vous que vous n’ayez reçu ? Et si vous l’avez reçu, pourquoi vous en glorifiez-vous, comme si vous ne l’aviez point reçu ?  \milestone{8}  Vous êtes déjà rassasiés, vous êtes déjà riches ; vous régnez sans nous ! et plût à Dieu que vous régnassiez, afin que nous régnassions aussi avec vous !  \milestone{9}  Car il semble que Dieu nous traite, nous autres apôtres, comme les derniers des hommes, comme ceux qui sont condamnés à la mort, nous faisant servir de spectacle au monde, c’est-à-dire, aux anges et aux hommes.  \milestone{10}  Nous sommes fous pour l’amour de Jésus-Christ ; mais vous autres, vous êtes sages en Jésus-Christ : nous sommes faibles, et vous êtes forts ; vous êtes honorés, et nous sommes méprisés.  \milestone{11}  Jusqu’à cette heure nous souffrons la faim et la soif, la nudité et les mauvais traitements ; nous n’avons point de demeure stable ;  \milestone{12}  nous travaillons avec beaucoup de peine de nos propres mains ; on nous maudit, et nous bénissons ; on nous persécute, et nous le souffrons ;  \milestone{13}  on nous dit des injures, et nous répondons par des prières ; nous sommes jusqu’à présent regardés comme les ordures du monde, comme les balayures qui sont rejetées de tous.\par
\bigbreak
\noindent   \milestone{14}  Je ne vous écris pas ceci pour vous causer de la honte ; mais je vous avertis de votre devoir, comme mes très-chers enfants.  \milestone{15}  Car quand vous auriez dix mille maîtres en Jésus-Christ, vous n’avez pas néanmoins plusieurs pères ; puisque c’est moi qui vous ai engendres en Jésus-Christ par l’Evangile.  \milestone{16}  Soyez donc mes imitateurs, je vous en conjure, comme je le suis moi-même de Jésus-Christ.  \milestone{17}  C’est pour cette raison que je vous ai envoyé Timothée, qui est mon fils très-cher et très-fidèle en notre Seigneur, afin qu’il vous fasse ressouvenir de la manière dont je vis moi-même en Jésus-Christ, selon ce que j’enseigne partout dans toutes les Églises.  \milestone{18}  Il y en a parmi vous qui s’enflent de présomption, comme si je ne devais plus vous aller voir.  \milestone{19}  J’irai néanmoins vous voir dans peu de temps, s’il plaît au Seigneur ; et alors je reconnaîtrai, non quelles sont les paroles, mais quels sont les effets de ceux qui sont enflés de vanité.  \milestone{20}  Car le royaume de Dieu ne consiste pas dans les paroles, mais dans les effets.  \milestone{21}  Que voulez-vous que je fasse ? Aimez-vous mieux que j’aille vous voir la verge à la main ; ou avec charité, et dans un esprit de douceur ?

\section[{1 Cor 5}]{1 Cor 5}

\noindent \initial{C\kern-0.08em{’}}{EST} un bruit constant qu’il y a de l’impureté parmi vous, et une telle impureté qu’on n’entend point dire qu’il s’en commette de semblable parmi les païens, jusque-là qu’un d’entre vous abuse de la femme de son père.  \milestone{2}  Et après cela vous êtes encore enflés d’orgueil ; et vous n’avez pas au contraire été dans les pleurs, pour faire retrancher du milieu de vous celui qui a commis une action si honteuse !  \milestone{3}  Pour moi, étant absent de corps, mais présent en esprit, j’ai déjà porté, comme présent, ce jugement contre celui qui a fait une telle action ;  \milestone{4}  qui est, que vous et mon esprit étant assemblés au nom de notre Seigneur Jésus-Christ, cet homme-là soit, par la puissance de notre Seigneur Jésus,  \milestone{5}  livré à Satan, pour mortifier sa chair, afin que son âme soit sauvée au jour de notre Seigneur Jésus-Christ.\par
  \milestone{6}  Vous n’avez donc point sujet de vous tant glorifier. Ne savez-vous pas qu’un peu de levain aigrit toute la pâte ?  \milestone{7}  Purifiez-vous du vieux levain, afin que vous soyez une pâte toute nouvelle, comme vous êtes vraiment les pains purs et sans levain. Car Jésus-Christ a été immolé, lui qui est notre Agneau pascal.  \milestone{8}  C’est pourquoi célébrons cette fête, ! non avec le vieux levain, ni avec le levain de la malice et de la corruption, mais avec les pains sans levain de la sincérité et de la vérité.\par
  \milestone{9}  Je vous ai écrit dans une lettre, que vous n’eussiez point de commerce avec les fornicateurs :  \milestone{10}  ce que je n’entends pas des fornicateurs de ce monde, non plus que des avares, des ravisseurs du bien d’autrui, ou des idolâtres : autrement il faudrait que vous sortissiez de ce monde.  \milestone{11}  Mais quand je vous ai écrit que vous n’eussiez point de commerce avec ces sortes de personnes, j’ai entendu que si celui qui est du nombre de vos frères, est fornicateur, ou avare, ou idolâtre, ou médisant, ou ivrogne, ou ravisseur du bien d’autrui, vous ne mangiez pas même avec lui.  \milestone{12}  Car pourquoi entreprendrais-je de juger ceux qui sont hors de l’Église ? N’est-ce pas de ceux qui sont dans l’Église que vous avez droit de juger ?  \milestone{13}  Dieu jugera ceux qui sont dehors ; mais pour vous, retranchez ce méchant du milieu de vous.

\section[{1 Cor 6}]{1 Cor 6}

\noindent \initial{C}{OMMENT} se trouve-t-il quelqu’un parmi vous qui ayant un différend avec son frère, ose l’appeler en jugement devant les méchants et les infidèles, et non pas devant les saints ?  \milestone{2}  Ne savez-vous pas que les saints doivent un jour juger le monde ? Si vous devez juger le monde, êtes-vous indignes de juger des moindres choses ?  \milestone{3}  Ne savez-vous pas que nous serons les juges des anges mêmes ? Combien plus devons-nous l’être de ce qui ne regarde que la vie présente ?  \milestone{4}  Si donc vous avez des différends entre vous touchant les choses de cette vie, prenez pour juges dans ces matières les moindres personnes de l’Église.  \milestone{5}  Je vous le dis pour vous faire confusion : Est-il possible qu’il ne se trouve point parmi vous un seul homme sage qui puisse être juge entre ses frères ;  \milestone{6}  et qu’il faille qu’on voie un frère plaider contre son frère, et encore devant des infidèles ?  \milestone{7}  C’est déjà un péché parmi vous, de ce que vous avez des procès les uns contre les autres. Pourquoi ne souffrez-vous pas plutôt qu’on vous fasse tort ? Pourquoi ne souffrez-vous pas plutôt qu’on vous trompe ?  \milestone{8}  Mais c’est vous-mêmes qui faites tort aux autres, qui les trompez, et qui traitez ainsi vos propres frères.  \milestone{9}  Ne savez-vous pas que les injustes ne seront point héritiers du royaume de Dieu ? Ne vous y trompez pas : ni les fornicateurs, ni les idolâtres, ni les adultères,  \milestone{10}  ni les impudiques [éfféminés], ni les abominables [pédérastes], ni les voleurs, ni les avares, ni les ivrognes, ni les médisants, ni les ravisseurs du bien d’autrui, ne seront point héritiers du royaume de Dieu.  \milestone{11}  C’est ce que quelques-uns de vous ont été autrefois ; mais vous avez été lavés, vous avez été sanctifiés, vous avez été justifiés au nom de notre Seigneur Jésus-Christ, et par l’Esprit de notre Dieu.\par
\bigbreak
\noindent   \milestone{12}  Tout m’est permis, mais tout n’est pas avantageux ; tout m’est permis, mais je ne me rendrai esclave de quoi que ce soit.  \milestone{13}  Les aliments sont pour le ventre, et le ventre est pour les aliments ; et Dieu détruira un jour cette destination de l’un et de l’autre. Mais le corps n’est point pour la fornication ; il est pour le Seigneur, et le Seigneur est pour le corps.  \milestone{14}  Car comme Dieu a ressuscité le Seigneur, il nous ressuscitera de même par sa puissance.  \milestone{15}  Ne savez-vous pas que vos corps sont les membres de Jésus-Christ ? Arracherai-je donc à Jésus-Christ ses propres membres, pour les faire devenir les membres d’une prostituée ? À Dieu ne plaise !  \milestone{16}  Ne savez-vous pas que celui qui se joint à une prostituée, est un même corps avec elle ? Car ceux qui étaient deux, ne seront plus qu’une chair, dit l’Écriture.  \milestone{17}  Mais celui qui demeure attaché au Seigneur, est un même, esprit avec lui.  \milestone{18}  Fuyez la fornication. Quelque autre péché que l’homme commette, il est hors du corps ; mais celui qui commet une fornication, pèche contre son propre corps.  \milestone{19}  Ne savez-vous pas que votre corps est le temple du Saint-Esprit qui réside en vous, et qui vous a été donné de Dieu, et que vous n’êtes plus à vous-mêmes ?  \milestone{20}  Car vous avez été achetés d’un grand prix. Glorifiez donc, et portez Dieu dans votre corps.

\section[{1 Cor 7}]{1 Cor 7}

\noindent \initial{P}{OUR} ce qui regarde les choses dont vous m’avez écrit, je vous dirai qu’il est avantageux à l’homme de ne toucher aucune femme.  \milestone{2}  Néanmoins, pour éviter la fornication, que chaque homme vive avec sa femme, et chaque femme avec son mari.  \milestone{3}  Que le mari rende à sa femme ce qu’il lui doit, et la femme ce qu’elle doit à son mari.  \milestone{4}  Le corps de la femme n’est point en sa puissance, mais en celle du mari ; de même, le corps du mari n’est point en sa puissance, mais en celle de sa femme.  \milestone{5}  Ne vous refusez point l’un à l’autre ce devoir, si ce n’est du consentement de l’un et de l’autre pour un temps, afin de vous exercer à la prière ; et ensuite vivez ensemble comme auparavant, de peur que le démon ne prenne sujet de votre incontinence pour vous tenter.  \milestone{6}  Ce que je vous dis comme une chose qu’on vous pardonne, et non pas qu’on vous commande.  \milestone{7}  Car je voudrais que vous fussiez tous comme moi ; mais chacun a son don particulier, selon qu’il le reçoit de Dieu, l’un d’une manière, et l’autre d’une autre.\par
  \milestone{8}  Quant aux personnes qui ne sont point mariées, ou qui sont veuves, je leur déclare qu’il leur est bon de demeurer en cet état, comme j’y demeure moi-même.  \milestone{9}  S’ils sont trop faibles pour garder la continence, qu’ils se marient : car il vaut mieux se marier que de brûler.  \milestone{10}  Quant à ceux qui sont mariés, ce n’est pas moi, mais le Seigneur qui leur fait ce commandement, qui est, que la femme ne se sépare point d’avec son mari.  \milestone{11}  Si elle s’en est séparée, qu’elle demeure sans se marier, ou qu’elle se réconcilie avec son mari ; et que le mari de même ne quitte point sa femme.  \milestone{12}  Pour ce qui est des autres, ce n’est pas le Seigneur, mais c’est moi qui leur dis : Si un fidèle a une femme qui soit infidèle, et qu’elle consente à demeurer avec lui, qu’il ne se sépare point d’avec elle ;  \milestone{13}  et si une femme fidèle a un mari qui soit infidèle, et qu’il consente à demeurer avec elle, qu’elle ne se sépare point d’avec lui.  \milestone{14}  Car le mari infidèle est sanctifié par la femme fidèle, et la femme infidèle est sanctifiée par le mari fidèle : autrement vos enfants seraient impurs, au lieu que maintenant ils sont saints.  \milestone{15}  Mais si la partie infidèle se sépare, qu’elle se sépare ; car un frère ou une sœur ne sont plus assujettis en cette rencontre : mais Dieu nous a appelés pour vivre en paix.  \milestone{16}  Car que savez-vous, ô femme, si vous sauveriez votre mari ? Et que savez-vous aussi, ô mari, si vous sauveriez votre femme ?\par
\bigbreak
\noindent   \milestone{17}  Mais que chacun se conduise selon le don particulier qu’il a reçu du Seigneur, et selon l’état dans lequel Dieu l’a appelé ; et c’est ce que j’ordonne dans toutes les Églises.  \milestone{18}  Un homme est-il appelé à la foi étant circoncis, qu’il n’affecte point de paraître incirconcis. Y est-il appelé n’étant point circoncis, qu’il ne se fasse point circoncire.  \milestone{19}  Ce n’est rien d’être circoncis, et ce n’est rien d’être incirconcis ; mais le tout est d’observer les commandements de Dieu.  \milestone{20}  Que chacun demeure dans l’état où il était quand Dieu l’a appelé.  \milestone{21}  Avez-vous été appelé à la foi étant esclave, ne portez point cet état avec peine ; mais plutôt faites-en un bon usage, quand même vous pourriez devenir libre.  \milestone{22}  Car celui qui étant esclave est appelé au service du Seigneur, devient affranchi du Seigneur ; et de même, celui qui est appelé étant libre, devient esclave de Jésus-Christ.  \milestone{23}  Vous avez été achetés d’un grand prix ; ne vous rendez pas esclaves des hommes.  \milestone{24}  Que chacun, mes frères, demeure donc dans l’état où il était lorsqu’il a été appelé, et qu’il s’y tienne devant Dieu.\par
\bigbreak
\noindent   \milestone{25}  Quant aux vierges, je n’ai point reçu de commandement du Seigneur ; mais voici le conseil que je leur donne, comme étant fidèle ministre du Seigneur, par la miséricorde qu’il m’en a faite.  \milestone{26}  Je crois donc qu’il est avantageux, à cause des fâcheuses nécessités de la vie présente ; qu’il est, dis-je, avantageux à l’homme de ne se point marier.  \milestone{27}  Êtes-vous lié avec une femme, ne cherchez point à vous délier. N’êtes-vous point lié avec une femme, ne cherchez point de femme.  \milestone{28}  Si vous épousez une femme, vous ne péchez pas ; et si une fille se marie, elle ne pèche pas. Mais ces personnes souffriront dans leur chair des afflictions et des peines : or je voudrais vous les épargner.\par
  \milestone{29}  Voici donc, mes frères, ce que j’ai à vous dire : Le temps est court ; et ainsi, que ceux mêmes qui ont des femmes, soient comme n’en ayant point ;  \milestone{30}  et ceux qui pleurent, comme ne pleurant point ; ceux qui se réjouissent, comme ne se réjouissant point ; ceux qui achètent, comme ne possédant point ;  \milestone{31}  enfin ceux qui usent de ce monde, comme n’en usant point : car la figure de ce monde passe.  \milestone{32}  Pour moi, je désire de vous voir dégagés de soins et d’inquiétudes. Celui qui n’est point marié, s’occupe du soin des choses du Seigneur, et de ce qu’il doit faire pour plaire à Dieu.  \milestone{33}  Mais celui qui est marié, s’occupe du soin des choses du monde, et de ce qu’il doit faire pour plaire à sa femme ; et ainsi il se trouve partagé.  \milestone{34}  De même, une femme qui n’est point mariée, et une vierge, s’occupe du soin des choses du Seigneur, afin d’être sainte de corps et d’esprit ; mais celle qui est mariée, s’occupe du soin des choses du monde, et de ce qu’elle doit faire pour plaire à son mari.  \milestone{35}  Or je vous dis ceci pour votre avantage, non pour vous tendre un piège ; mais pour vous porter à ce qui est de plus saint, et qui vous donne un moyen plus facile de prier Dieu sans empêchement.\par
  \milestone{36}  Si quelqu’un croit que ce lui soit un déshonneur que sa fille passe la fleur de son âge sans être mariée, et qu’il juge devoir la marier, qu’il fasse ce qu’il voudra ; il ne péchera point si elle se marie.  \milestone{37}  Mais celui qui n’étant engagé par aucune nécessité, et qui se trouvant dans un plein pouvoir de faire ce qu’il voudra, prend une ferme résolution dans son cœur, et juge en lui-même qu’il doit conserver sa fille vierge, fait une bonne œuvre.  \milestone{38}  Ainsi celui qui marie sa fille, fait bien ; et celui qui ne la marie point, fait encore mieux.\par
  \milestone{39}  La femme est liée à la loi du mariage, tant que son mari est vivant : mais si son mari meurt, elle est libre ; qu’elle se marie à qui elle voudra, pourvu que ce soit selon le Seigneur.  \milestone{40}  Cependant elle sera plus heureuse si elle demeure veuve, comme je le lui conseille ; et je crois que j’ai aussi l’Esprit de Dieu.

\section[{1 Cor 8}]{1 Cor 8}

\noindent \initial{Q}{UANT} aux viandes qui ont été immolées aux idoles, nous n’ignorons pas que nous avons tous sur ce sujet assez de science ; mais la science enfle, et la charité édifie.  \milestone{2}  Si quelqu’un se flatte de savoir quelque, chose, il ne sait encore rien comme on doit le savoir.  \milestone{3}  Mais si quelqu’un aime Dieu, il est connu et aimé de Dieu.\par
  \milestone{4}  Quant à ce qui est donc de manger des viandes immolées aux idoles, nous savons que les idoles ne sont rien dans le monde, et qu’il n’y a nul autre Dieu que le seul Dieu.  \milestone{5}  Car encore qu’il y en ait qui soient appelés dieux, soit dans le ciel, ou dans la terre, et qu’ainsi il y ait plusieurs dieux et plusieurs seigneurs ;  \milestone{6}  il n’y a néanmoins pour nous qu’un seul Dieu, qui est le Père, de qui toutes choses tirent leur être, et qui nous a faits pour lui ; et il n’y a qu’un seul Seigneur, qui est Jésus-Christ, par qui toutes choses ont été faites, comme c’est aussi par lui que nous sommes tout ce que nous sommes.\par
  \milestone{7}  Mais tous n’ont pas la science. Car il y en a qui mangent des viandes offertes aux idoles, croyant encore que l’idole est quelque chose ; et ainsi leur conscience qui est faible, en est souillée.  \milestone{8}  Les aliments par eux-mêmes ne nous rendront pas agréables à Dieu : si nous mangeons, nous n’en aurons rien davantage devant lui ; ni rien de moins, si nous ne mangeons pas.  \milestone{9}  Mais prenez garde que cette liberté que vous avez, ne soit aux faibles une occasion de chute.  \milestone{10}  Car si l’un d’eux en voit un de ceux qui sont plus instruits, assis à table dans un lieu consacré aux idoles, ne sera-t-il pas porté, lui dont la conscience est encore faible, à manger aussi de ces viandes sacrifiées aux idoles ?  \milestone{11}  Et ainsi par votre science vous perdrez votre frère encore faible, pour qui Jésus-Christ est mort.  \milestone{12}  Or péchant de la sorte contre vos frères, et blessant leur conscience qui est faible, vous péchez contre Jésus-Christ.  \milestone{13}  Si donc ce que je mange scandalise mon frère, je ne mangerai plutôt jamais de chair toute ma vie, pour ne pas scandaliser mon frère.

\section[{1 Cor 9}]{1 Cor 9}

\noindent \initial{N}{E} suis-je pas libre ? Ne suis-je pas apôtre ? N’ai-je pas vu Jésus-Christ notre Seigneur ? N’êtes-vous pas vous-mêmes mon ouvrage en Notre-Seigneur ?  \milestone{2}  Quand je ne serais pas apôtre à l’égard des autres, je le suis au moins à votre égard : car vous êtes le sceau de mon apostolat en Notre-Seigneur.  \milestone{3}  Voici ma défense contre ceux qui me reprennent :  \milestone{4}  N’avons-nous pas droit d’être nourris à vos dépens ? \\
  \milestone{5}  N’avons-nous pas le pouvoir de mener partout avec nous une femme qui soit notre sœur en Jésus-Christ, comme font les autres apôtres, et les frères de Notre-Seigneur, et Céphas ?  \milestone{6}  Serions-nous donc seuls, Barnabé et moi, qui n’aurions pas le pouvoir d’en user de la sorte ?  \milestone{7}  Qui est-ce qui va jamais à la guerre à ses dépens ? Qui est-ce qui plante une vigne, et n’en mange point du fruit ? ou qui est celui qui mène paître un troupeau, et n’en mange point du lait ?  \milestone{8}  Ce que je dis ici, n’est-il qu’un raisonnement humain ? La loi même ne le dit-elle pas aussi ?  \milestone{9}  Car il est écrit dans la loi de Moïse : Vous ne tiendrez point la bouche liée au bœuf qui foule les grains. Dieu se met-il en peine de ce qui regarde les bœufs ?  \milestone{10}  Et n’est-ce pas plutôt pour nous-mêmes qu’il a fait cette ordonnance ? Oui, sans doute, c’est pour nous que cela a été écrit. En effet, celui qui laboure, doit labourer avec espérance de participer aux fruits de la terre ; et aussi celui qui bat le grain, doit le faire avec espérance d’y avoir part.  \milestone{11}  Si donc nous avons semé parmi vous des biens spirituels, est-ce une grande chose que nous recueillions un peu de vos biens temporels ?  \milestone{12}  Si d’autres usent de ce pouvoir à votre égard, pourquoi ne pourrons-nous pas en user plutôt qu’eux ? Mais nous n’avons point usé de ce pouvoir ; et nous souffrons au contraire toutes sortes d’incommodités, pour n’apporter aucun obstacle à l’Évangile de Jésus-Christ.\par
  \milestone{13}  Ne savez-vous pas que les ministres du temple mangent de ce qui est offert dans le temple, et que ceux qui servent à l’autel, ont part aux oblations de l’autel ?  \milestone{14}  Ainsi le Seigneur a aussi ordonné à ceux qui annoncent l’Évangile, de vivre de l’Évangile.  \milestone{15}  Mais pour moi, je n’ai usé d’aucun de ces droits ; et encore maintenant je ne vous écris point ceci afin qu’on en use ainsi envers moi, puisque j’aimerais mieux mourir que de souffrir que quelqu’un me fît perdre cette gloire.  \milestone{16}  Car si je prêche l’Évangile, ce ne m’est point un sujet de gloire, puisque je suis obligé nécessairement à ce ministère ; et malheur à moi, si je ne prêche pas l’Évangile !  \milestone{17}  Si je le prêche de bon cœur, j’en aurai la récompense ; mais si je ne le fais qu’à regret, je dispense seulement ce qui m’a été confié.  \milestone{18}  En quoi trouverai-je donc un sujet de récompense ? En prêchant de telle sorte l’Évangile, que je le prêche gratuitement, sans abuser du pouvoir que j’ai dans la prédication de l’Évangile.\par
  \milestone{19}  Car étant libre à l’égard de tous, je me suis rendu le serviteur de tous, pour gagner à Dieu plus de personnes.  \milestone{20}  J’ai vécu avec les Juifs comme Juif, pour gagner les Juifs ;  \milestone{21}  avec ceux qui sont sous la loi, comme si j’eusse encore été sous la loi, quoique je n’y fusse plus assujetti, pour gagner ceux qui sont sous la loi ; avec ceux qui n’avaient point de loi, comme si je n’en eusse point eu moi-même (quoique j’en eusse une à l’égard de Dieu, ayant celle de Jésus-Christ), pour gagner ceux qui étaient sans loi.  \milestone{22}  Je me suis rendu faible avec les faibles, pour gagner les faibles. Enfin je me suis fait tout à tous, pour les sauver tous.  \milestone{23}  Or je fais toutes ces choses pour l’Évangile, afin d’avoir part à ce qu’il promet.\par
\bigbreak
\noindent   \milestone{24}  Ne savez-vous pas, que quand on court dans la carrière, tous courent, mais un seul remporte le prix ? Courez donc de telle sorte que vous remportiez le prix.  \milestone{25}  Or tous les athlètes gardent en toutes choses une exacte tempérance ; et cependant ce n’est que pour gagner une couronne corruptible, au lieu que nous en attendons une incorruptible.  \milestone{26}  Pour moi, je cours, et je ne cours pas au hasard. Je combats, et je ne donne pas des coups en l’air ;  \milestone{27}  mais je traite rudement mon corps, et je le réduis en servitude ; de peur qu’ayant prêché aux autres, je ne sois réprouvé moi-même.

\section[{1 Cor 10}]{1 Cor 10}

\noindent \initial{C}{AR} je ne veux pas que vous ignoriez, mes frères, que nos pères ont tous été sous la nuée ; qu’ils ont tous passé la mer Rouge ;  \milestone{2}  qu’ils ont tous été baptisés sous la conduite de Moïse, dans la nuée et dans la mer ;  \milestone{3}  qu’ils ont tous mangé d’un même aliment spirituel ;  \milestone{4}  et qu’ils ont tous bu d’un même breuvage spirituel : car ils buvaient de l’eau de la pierre spirituelle qui les suivait ; et Jésus-Christ était cette pierre.  \milestone{5}  Mais il y en eut peu d’un si grand nombre qui fussent agréables à Dieu, étant presque tous péris dans le désert.  \milestone{6}  Or toutes ces choses ont été des figures de ce qui nous regarde, afin que nous ne nous abandonnions pas aux mauvais désirs, comme ils s’y abandonnèrent.  \milestone{7}  Ne devenez point aussi idolâtres, comme quelques-uns d’eux, dont il est écrit : Le peuple s’assit pour manger et pour boire, et ils se levèrent pour se divertir.  \milestone{8}  Ne commettons point de fornication, comme quelques-uns d’eux commirent ce crime, pour lequel il y en eut vingttrois mille qui furent frappés de mort en un seul jour.  \milestone{9}  Ne tentons point Jésus-Christ, comme le tentèrent quelques-uns d’eux, qui furent tués par les serpents.  \milestone{10}  Ne murmurez point, comme murmurèrent quelques-uns d’eux, qui furent frappés de mort par l’ange exterminateur.  \milestone{11}  Or toutes ces choses qui leur arrivaient, étaient des figures ; et elles ont été écrites pour nous servir d’instruction, à nous autres qui nous trouvons à la fin des temps.\par
  \milestone{12}  Que celui donc qui croit être ferme, prenne bien garde à ne pas tomber.  \milestone{13}  Vous n’avez eu encore que des tentations humaines et ordinaires. Dieu est fidèle, et il ne souffrira pas que vous soyez tentés au delà de vos forces ; mais il vous fera tirer avantage de la tentation même, afin que vous puissiez persévérer.\par
\bigbreak
\noindent   \milestone{14}  C’est pourquoi, mes très-chers frères, fuyez l’idolâtrie.  \milestone{15}  Je vous parle comme à des personnes sages ; jugez vous-mêmes de ce que je dis.  \milestone{16}  N’est-il pas vrai que le calice de bénédiction que nous bénissons, est la communion du sang de Jésus-Christ ; et que le pain que nous rompons, est la communion du corps du Seigneur ?  \milestone{17}  Car nous ne sommes tous ensemble qu’un seul pain et un seul corps ; parce que nous participons tous à un même pain.  \milestone{18}  Considérez les Israélites selon la chair : ceux d’entre eux qui mangent de la victime immolée, ne prennent-ils pas ainsi part à l’autel ?  \milestone{19}  Est-ce donc que je veuille dire que ce qui a été immolé aux idoles ait quelque vertu, ou que l’idole soit quelque chose ?  \milestone{20}  Non ; mais je dis que ce que les païens immolent, ils l’immolent aux démons, et non pas à Dieu. Or je désire que vous n’ayez aucune société avec les démons. Vous ne pouvez pas boire le calice du Seigneur, et le calice des démons.  \milestone{21}  Vous ne pouvez pas participer à la table du Seigneur, et à la table des démons.  \milestone{22}  Est-ce que nous voulons irriter Dieu, en le piquant de jalousie ? Sommes-nous plus forts que lui ? Tout m’est permis ; mais tout n’est pas avantageux.\par
\bigbreak
\noindent   \milestone{23}  Tout m’est permis ; mais tout n’édifie pas.  \milestone{24}  Que personne ne cherche sa propre satisfaction, mais le bien des autres.  \milestone{25}  Mangez de tout ce qui se vend à la boucherie, sans vous enquérir d’où il vient, par un scrupule de conscience :  \milestone{26}  car la terre est au Seigneur, avec tout ce qu’elle contient.  \milestone{27}  Si un infidèle vous prie à manger chez lui, et que vous vouliez y aller, mangez de tout ce qu’on vous servira, sans vous enquérir d’où il vient, par un scrupule de conscience.  \milestone{28}  Si quelqu’un vous dit, Ceci a été immolé aux idoles ; n’en mangez pas à cause de celui qui vous a donné cet avis, et aussi de peur de blesser, je ne dis pas, votre conscience, mais celle d’un autre.  \milestone{29}  Car pourquoi m’exposerais-je à faire condamner, par la conscience d’un autre, cette liberté que j’ai de manger de tout ?  \milestone{30}  Si je prends avec action de grâces ce que je mange, pourquoi donnerai-je sujet à un autre de parler mal de moi, pour une chose dont je rends grâces à Dieu !  \milestone{31}  Soit donc que vous mangiez, ou que vous buviez, et quelque chose que vous fassiez, faites tout pour la gloire de Dieu.  \milestone{32}  Ne donnez pas occasion de scandale ni aux Juifs ni aux gentils, ni à l’Église de Dieu :  \milestone{33}  comme je tâche moi-même de plaire à tous en toutes choses, ne cherchant point ce qui m’est avantageux en particulier, mais ce qui est avantageux a plusieurs pour être sauvés.

\section[{1 Cor 11}]{1 Cor 11}

\noindent \initial{S}{OYEZ} mes imitateurs, comme je le suis moi-même de Jésus-Christ.\par
\bigbreak
\noindent   \milestone{2}  Je vous loue, mes frères, de ce que vous vous souvenez de moi en toutes choses, et que vous gardez les traditions et les règles que je vous ai données.  \milestone{3}  Mais je désire que vous sachiez que Jésus-Christ est le chef et la tête de tout homme, que l’homme est le chef de la femme, et que Dieu est le chef de Jésus-Christ.  \milestone{4}  Tout homme qui prie, ou qui prophétise, ayant la tête couverte, déshonore sa tête.  \milestone{5}  Mais toute femme qui prie, ou qui prophétise, n’ayant point la tête couverte d’un voile, déshonore sa tête : car c’est comme si elle était rasée.  \milestone{6}  Si une femme ne se voile point la tête, elle devrait donc avoir aussi les cheveux coupés. Mais s’il est honteux à une femme d’avoir les cheveux coupés, ou d’être rasée, qu’elle se voile la tête.  \milestone{7}  Pour ce qui est de l’homme, il ne doit point se couvrir la tête, parce qu’il est l’image et la gloire de Dieu ; au lieu que la femme est la gloire de l’homme.  \milestone{8}  Car l’homme n’a point été tiré de la femme, mais la femme a été tirée de l’homme ;  \milestone{9}  et l’homme n’a point été créé pour la femme, mais la femme pour l’homme.  \milestone{10}  C’est pourquoi la femme doit porter sur sa tête, à cause des anges, la marque de la puissance que l’homme a sur elle.\par
  \milestone{11}  Toutefois ni l’homme n’est point sans la femme, ni la femme sans l’homme, en notre Seigneur.  \milestone{12}  Car comme la femme au commencement a été tirée de l’homme, aussi l’homme maintenant naît de la femme, et tout vient de Dieu.  \milestone{13}  Jugez vous-mêmes s’il est bienséant à une femme de prier Dieu sans avoir un voile sur la tête.  \milestone{14}  La nature même ne vous enseigne-t-elle pas, qu’il serait honteux à un homme de laisser toujours croître ses cheveux ;  \milestone{15}  et qu’il est au contraire honorable à une femme de les laisser toujours croître, parce qu’ils lui ont été donnés comme un voile qui doit la couvrir ?  \milestone{16}  Si après cela quelqu’un aime à contester, il nous suffit de répondre, que ce n’est point là notre coutume, ni celle de l’Église de Dieu.\par
\bigbreak
\noindent   \milestone{17}  Mais je ne puis vous louer en ce que je vais vous dire, qui est, que vous vous conduisez de telle sorte dans vos assemblées, qu’elles vous nuisent, au lieu de vous servir.  \milestone{18}  Premièrement, j’apprends que, lorsque vous vous assemblez dans l’Église, il y a des partialités parmi vous ; et je le crois en partie :  \milestone{19}  car il faut qu’il y ait même des hérésies, afin qu’on découvre par là ceux d’entre vous qui ont une vertu éprouvée.  \milestone{20}  Lors donc que vous vous assemblez comme vous faites, ce n’est plus manger la Cène du Seigneur :  \milestone{21}  car chacun y mange son souper particulier, sans attendre les autres ; et ainsi les uns n’ont rien à manger, pendant que les autres le font avec excès.  \milestone{22}  N’avez-vous pas vos maisons pour y boire et pour y manger ? Ou méprisez-vous l’Église de Dieu ? et voulez-vous faire honte à ceux qui sont pauvres ? Que vous dirai-je sur cela ? Vous en louerai-je ? Non, certes, je ne vous en loue point.\par
  \milestone{23}  Car c’est du Seigneur que j’ai appris ce que je vous ai aussi enseigné, qui est : Que le Seigneur Jésus, la nuit même en laquelle il devait être livré à la mort, prit du pain,  \milestone{24}  et ayant rendu grâces, le rompit, et dit à ses disciples : Prenez et mangez : ceci est mon corps, qui sera livré pour vous : faites ceci en mémoire de moi.  \milestone{25}  Il prit de même le calice, après avoir soupé, en disant : Ce calice est la nouvelle alliance en mon sang : faites ceci en mémoire de moi, toutes les fois que vous le boirez.  \milestone{26}  Car toutes les fois que vous mangerez ce pain, et que vous boirez ce calice, vous annoncerez la mort du Seigneur, jusqu’à ce qu’il vienne.  \milestone{27}  C’est pourquoi, quiconque mangera ce pain ou boira le calice du Seigneur indignement, il sera coupable du corps et du sang du Seigneur.  \milestone{28}  Que l’homme donc s’éprouve soi-même, et qu’il mange ainsi de ce pain, et boive de ce calice.  \milestone{29}  Car quiconque en mange et en boit indignement, mange et boit sa propre condamnation, ne faisant point le discernement qu’il doit du corps du Seigneur.  \milestone{30}  C’est pour cette raison qu’il y a parmi vous beaucoup de malades et de languissants, et que plusieurs dorment du sommeil de la mort.  \milestone{31}  Si nous nous jugions nous-mêmes, nous ne serions pas jugés de Dieu.  \milestone{32}  Mais lorsque nous sommes jugés de la sorte, c’est le Seigneur qui nous châtie, afin que nous ne soyons pas condamnés avec le monde.  \milestone{33}  C’est pourquoi, mes frères, lorsque vous vous assemblez pour ces repas, attendez-vous les uns les autres.  \milestone{34}  Si quelqu’un est pressé de manger, qu’il mange chez lui ; afin que vous ne vous assembliez point à votre condamnation. Je réglerai les autres choses, lorsque je serai venu.

\section[{1 Cor 12}]{1 Cor 12}

\noindent \initial{P}{OUR} ce qui est des dons spirituels, mes frères, je ne veux pas que vous ignoriez ce que vous devez savoir.  \milestone{2}  Vous vous souvenez bien, qu’étant païens vous vous laissiez entraîner, selon qu’on vous menait, vers les idoles muettes.  \milestone{3}  Je vous déclare donc, que nul homme parlant par l’Esprit de Dieu, ne dit anathème à Jésus ; et que nul ne peut confesser que Jésus est le Seigneur, sinon par le Saint-Esprit.\par
  \milestone{4}  Or il y a diversité de dons spirituels ; mais il n’y a qu’un même Esprit.  \milestone{5}  Il y a diversité de ministères ; mais il n’y a qu’un même Seigneur.  \milestone{6}  Et il y a diversité d’opérations surnaturelles ; mais il n’y a qu’un même Dieu, qui opère tout en tous.  \milestone{7}  Or les dons du Saint-Esprit, qui se font connaître au dehors, sont donnés à chacun pour l’utilité de l’Église.  \milestone{8}  L’un reçoit du Saint-Esprit le don de parler dans une haute sagesse : un autre reçoit du même Esprit le don de parler avec science ;  \milestone{9}  un autre reçoit le don de la foi par le même Esprit ; un autre reçoit du même Esprit la grâce de guérir les maladies ;  \milestone{10}  un autre, le don de faire des miracles ; un autre, le don de prophétie ; un autre, le don du discernement des esprits ; un autre, le don de parler diverses langues ; un autre, le don de l’interprétation des langues.  \milestone{11}  Or c’est un seul et même Esprit qui opère toutes ces choses, distribuant à chacun ses dons, selon qu’il lui plaît.\par
\bigbreak
\noindent   \milestone{12}  Et comme notre corps n’étant qu’un, est composé de plusieurs membres ; et qu’encore qu’il y ait plusieurs membres, ils ne sont tous néanmoins qu’un même corps ; il en est de même du Christ.  \milestone{13}  Car nous avons tous été baptisés dans le même Esprit, pour n’être tous ensemble qu’un même corps, soit Juifs ou gentils, soit esclaves ou libres ; et nous avons tous reçu un divin breuvage, pour n’être tous aussi qu’un même esprit.  \milestone{14}  Aussi le corps n’est pas un seul membre, mais plusieurs.  \milestone{15}  Si le pied disait, Puisque je ne suis pas la main, je ne suis pas du corps ; ne serait-il point pour cela du corps ?  \milestone{16}  Et si l’oreille disait, Puisque je ne suis pas œil, je ne suis pas du corps ; ne serait-elle point pour cela du corps ?  \milestone{17}  Si tout le corps était œil, ou serait l’ouïe ? et s’il était tout ouïe, ou serait l’odorat ?  \milestone{18}  Mais Dieu a mis dans le corps plusieurs membres, et il les y a placés comme il lui a plu.  \milestone{19}  Si tous les membres n’étaient qu’un seul membre, où serait le corps ?  \milestone{20}  Mais il y a plusieurs membres, et tous ne font qu’un seul corps.  \milestone{21}  Or l’œil ne peut pas dire à la main, Je n’ai pas besoin de votre secours ; non plus que la tête ne peut pas dire aux pieds, Vous ne m’êtes point nécessaires.  \milestone{22}  Mais au contraire, les membres du corps qui paraissent les plus faibles, sont les plus nécessaires.  \milestone{23}  Nous honorons même davantage par nos vêtements les parties du corps qui paraissent les moins honorables ; et nous couvrons avec plus de soin et d’honnêteté celles qui sont moins honnêtes.  \milestone{24}  Car pour celles qui sont honnêtes, elles n’en ont pas besoin : mais Dieu a mis un tel ordre dans tout le corps, qu’on honore davantage ce qui est moins honorable de soi-même ;  \milestone{25}  afin qu’il n’y ait point de schisme, ni de division dans le corps, mais que tous les membres conspirent mutuellement à s’entr’aider les uns les autres.  \milestone{26}  Et si l’un des membres souffre, tous les autres souffrent avec lui ; ou si l’un des membres reçoit de l’honneur, tous les autres s’en réjouissent avec lui.  \milestone{27}  Or vous êtes le corps de Jésus-Christ, et membres les uns des autres.  \milestone{28}  Ainsi Dieu a établi dans son Église, premièrement, des apôtres ; secondement, des prophètes ; troisièmement, des docteurs ; ensuite, ceux qui ont la vertu de faire des miracles ; puis, ceux qui ont la grâce de guérir les maladies ; ceux qui ont le don d’assister les frères ; ceux qui ont le don de gouverner ; ceux qui ont le don de parler diverses langues ; ceux qui ont le don de les interpréter.  \milestone{29}  Tous sont-ils apôtres ? Tous sont-ils prophètes ? Tous sont-ils docteurs ?  \milestone{30}  Tous font-ils des miracles ? Tous ont-ils la grâce de guérir les maladies ? Tous parlent-ils plusieurs langues ? Tous ont-ils le don de les interpréter ?  \milestone{31}  Entre ces dons, ayez plus d’empressement pour les meilleurs. Mais je vais vous montrer encore une voie beaucoup plus excellente.

\section[{1 Cor 13}]{1 Cor 13}

\noindent \initial{Q}{UAND} je parlerais toutes les langues des hommes, et le langage des anges mêmes, \\
si je n’ai point la charité, je ne suis que comme un airain sonnant, ou une cymbale retentissante.\par
  \milestone{2}  Et quand j’aurais le don de prophétie, que je pénétrerais tous les mystères, et que j’aurais une parfaite science de toutes choses ; quand j’aurais encore toute la foi possible, jusqu’à transporter les montagnes, si je n’ai point la charité, je ne suis rien.\par
  \milestone{3}  Et quand j’aurais distribué tout mon bien pour nourrir les pauvres, et que j’aurais livré mon corps pour être brûlé, si je n’ai point la charité, tout cela ne me sert de rien.\par
  \milestone{4}  La charité est patiente ; elle est douce et bienfaisante ; la charité n’est point envieuse ; elle n’est point téméraire et précipitée ; elle ne s’enfle point d’orgueil ;\par
  \milestone{5}  elle n’est point dédaigneuse, elle ne cherche point ses propres intérêts, elle ne se pique et ne s’aigrit de rien, elle n’a point de mauvais soupçons ;\par
  \milestone{6}  elle ne se réjouit point de l’injustice ; mais elle se réjouit de la vérité ;\par
  \milestone{7}  elle supporte tout, elle croit tout, elle espère tout, elle souffre tout.\par
  \milestone{8}  La charité ne finira jamais. Les prophéties n’auront plus de lieu ; les langues cesseront ; et la science sera abolie :\par
  \milestone{9}  car ce que nous avons maintenant de science et de prophétie, est très-imparfait ;\par
  \milestone{10}  mais lorsque nous serons dans l’état parfait, tout ce qui est imparfait sera aboli.\par
  \milestone{11}  Quand j’étais enfant, je parlais en enfant, je jugeais en enfant, je raisonnais en enfant ; mais lorsque je suis devenu homme, je me suis défait de tout ce qui tenait de l’enfant.\par
  \milestone{12}  Nous ne voyons maintenant que comme en un miroir, et en des énigmes ; mais alors nous verrons Dieu face à face. Je ne connais maintenant Dieu qu’imparfaitement ; mais alors je le connaîtrai, comme je suis moi-même connu de lui.\par
  \milestone{13}  Maintenant ces trois vertus, la foi, l’espérance, et la charité, demeurent ; mais entre elles la plus excellente est la charité.

\section[{1 Cor 14}]{1 Cor 14}

\noindent \initial{R}{ECHERCHEZ} avec ardeur la charité ; désirez les dons spirituels, et surtout de prophétiser.  \milestone{2}  Car celui qui parle une langue inconnue, ne parle pas aux hommes, mais à Dieu ; puisque personne ne l’entend, et qu’il parle en esprit des choses cachées.  \milestone{3}  Mais celui qui prophétise, parle aux hommes pour les édifier, les exhorter et les consoler.  \milestone{4}  Celui qui parle une langue inconnue, s’édifie lui-même ; au lieu que celui qui prophétise, édifie l’Église de Dieu.  \milestone{5}  Je souhaite que vous ayez tous le don des langues, mais encore plus celui de prophétiser ; parce que celui qui prophétise, est préférable à celui qui parle une langue inconnue, si ce n’est qu’il interprète ce qu’il dit, afin que l’Église en soit édifiée.  \milestone{6}  Aussi, mes frères, quand je voudrais vous parler en des langues inconnues, quelle utilité vous apporterais-je, si ce n’est que je vous parle en vous instruisant, ou par la révélation, ou par la science, ou par la prophétie, ou par la doctrine ?  \milestone{7}  Et dans les choses mêmes inanimées qui rendent des sons, comme les flûtes et les harpes, si elles ne forment des tons différents, comment pourra-t-on distinguer ce que l’on joue sur ces instruments ?  \milestone{8}  Si la trompette ne rend qu’un son confus, qui se préparera au combat ?  \milestone{9}  De même, si la langue que vous parlez n’est intelligible, comment pourra-t-on savoir ce que vous dites ? Vous ne parlerez qu’en l’air.  \milestone{10}  En effet, il y a tant de diverses langues dans le monde, et il n’y a point de peuple qui n’ait la sienne.  \milestone{11}  Si donc je n’entends pas ce que signifient les paroles, je serai barbare à celui à qui je parle, et celui qui me parle, me sera barbare.  \milestone{12}  Ainsi, mes frères, puisque vous avez tant d’ardeur pour les dons spirituels, désirez d’en être enrichis pour l’édification de l’Église.  \milestone{13}  C’est pourquoi, que celui qui parle une langue inconnue, demande à Dieu le don de l’interpréter.  \milestone{14}  Car si je prie en une langue que je n’entends pas, mon cœur prie, mais mon intelligence est sans fruit.\par
  \milestone{15}  Que ferai-je donc ? Je prierai de cœur ; mais je prierai aussi avec intelligence : je chanterai de cœur des cantiques ; mais je les chanterai aussi avec intelligence.  \milestone{16}  Si vous ne louez Dieu que du cœur, comment un homme du nombre de ceux qui n’entendent que leur propre langue, répondra-t-il, Amen, à la fin de votre action de grâces, puisqu’il n’entend pas ce que vous dites ?  \milestone{17}  Ce n’est pas que votre action de grâces ne soit bonne ; mais les autres n’en sont pas édifiés.  \milestone{18}  Je remercie mon Dieu de ce que je parle toutes les langues que vous parlez ;  \milestone{19}  mais j’aimerais mieux ne dire dans l’Église que cinq paroles dont j’aurais l’intelligence, pour en instruire aussi les autres, que d’en dire dix mille en une langue inconnue.\par
  \milestone{20}  Mes frères, ne soyez point enfants pour n’avoir point de sagesse ; mais soyez enfants pour être sans malice, et soyez sages comme des hommes parfaits.  \milestone{21}  Il est dit dans l’Écriture : Je parlerai à ce peuple en des langues étrangères et inconnues ; et après cela même, ils ne m’entendront point, dit le Seigneur.\par
  \milestone{22}  Ainsi la diversité des langues est un signe, non pour les fidèles, mais pour les infidèles ; et le don de prophétie au contraire n’est pas pour les infidèles, mais pour les fidèles.  \milestone{23}  Si toute une Église étant assemblée en un lieu, tous parlent diverses langues, et que des infidèles, ou des hommes qui ne savent que leur propre langue, entrent dans cette assemblée, ne diront-ils pas que vous êtes des insensés ?  \milestone{24}  Mais si tous prophétisent, et qu’un infidèle, ou un homme qui ne sait que sa langue, entre dans votre assemblée, tous le convainquent, tous le jugent.  \milestone{25}  Le secret de son cœur est découvert : de sorte que se prosternant le visage contre terre, il adorera Dieu, rendant témoignage que Dieu est véritablement parmi vous.\par
\bigbreak
\noindent   \milestone{26}  Que faut-il donc, mes frères, que vous fassiez ? Si lorsque vous êtes assemblés, l’un est inspiré de Dieu pour composer un cantique, l’autre pour instruire, un autre pour révéler les secrets de Dieu, un autre pour parler une langue inconnue, un autre pour l’interpréter, que tout se fasse pour l’édification.  \milestone{27}  S’il y en a qui aient le don des langues, qu’il n’y en ait point plus de deux ou trois qui parlent en une langue inconnue ; qu’ils parlent l’un après l’autre ; et qu’il y ait quelqu’un qui interprété ce qu’ils auront dit.  \milestone{28}  S’il n’y a point d’interprète, que celui qui a ce don se taise dans l’Église, qu’il ne parle qu’à soi-même et à Dieu.  \milestone{29}  Pour ce qui est des prophètes, qu’il n’y en ait point plus de deux ou trois qui parlent, et que les autres en jugent.  \milestone{30}  S’il se fait quelque révélation à un autre de ceux qui sont assis dans l’assemblée, que le premier se taise.  \milestone{31}  Car vous pouvez tous prophétiser l’un après l’autre, afin que tous apprennent, et que tous soient consolés.  \milestone{32}  Et les esprits des prophètes sont soumis aux prophètes.  \milestone{33}  Car Dieu est un Dieu de paix, et non de désordre ; et c’est ce que j’enseigne dans toutes les Églises des saints.\par
  \milestone{34}  Que les femmes parmi vous se taisent dans les Églises, parce qu’il ne leur est pas permis d’y parler ; mais elles doivent être soumises, selon que la loi l’ordonne.  \milestone{35}  Si elles veulent s’instruire de quelque chose, qu’elles le demandent à leurs maris, lorsqu’elles seront dans leurs maisons : car il est honteux aux femmes de parler dans l’Église.  \milestone{36}  Est-ce de vous que la parole de Dieu est premièrement sortie ? ou n’est-elle venue qu’à vous seuls ?  \milestone{37}  Si quelqu’un croit être prophète ou spirituel, qu’il reconnaisse que les choses que je vous écris sont des ordonnances du Seigneur.  \milestone{38}  Si quelqu’un veut l’ignorer, il sera lui-même ignoré.\par
  \milestone{39}  Enfin, mes frères, désirez surtout le don de prophétie, et n’empêchez pas l’usage du don des langues ;  \milestone{40}  mais que tout se fasse dans la bienséance, et avec ordre.

\section[{1 Cor 15}]{1 Cor 15}

\noindent \initial{J}{E} vais maintenant, mes frères, vous remettre devant les yeux l’Évangile que je vous ai prêché, que vous avez reçu, dans lequel vous demeurez fermes,  \milestone{2}  et par lequel vous serez sauvés ; si toutefois vous l’avez retenu comme je vous l’ai annoncé, et si ce n’est pas en vain que vous avez embrassé la foi.  \milestone{3}  Car, premièrement, je vous ai enseigné, et comme donné en dépôt ce que j’avais moi-même reçu ; savoir : Que Jésus-Christ est mort pour nos péchés, selon les Écritures ;  \milestone{4}  qu’il a été enseveli, et qu’il est ressuscité le troisième jour, selon les mêmes Écritures ;  \milestone{5}  qu’il s’est fait voir à Céphas, puis aux onze apôtres ;  \milestone{6}  qu’après il a été vu en une seule fois de plus de cinq cents frères, dont il y en a plusieurs qui vivent encore aujourd’hui, et quelques-uns sont déjà morts ;  \milestone{7}  qu’ensuite il s’est fait voir à Jacques, puis à tous les apôtres,  \milestone{8}  et qu’enfin après tous les autres, il s’est fait voir à moi-même, qui ne suis qu’un avorton.  \milestone{9}  Car je suis le moindre des apôtres ; et je ne suis pas digne d’être appelé apôtre, parce que j’ai persécuté l’Église de Dieu.  \milestone{10}  Mais c’est par la grâce de Dieu que je suis ce que je suis, et sa grâce n’a point été stérile en moi : mais j’ai travaillé plus que tous les autres ; non pas moi toutefois, mais la grâce de Dieu qui est avec moi.  \milestone{11}  Ainsi, soit que ce soit moi, soit que ce soit eux qui vous prêchent, voilà ce que nous prêchons, et voilà ce que vous avez cru.\par
\bigbreak
\noindent   \milestone{12}  Puis donc qu’on vous a prêché que Jésus-Christ est ressuscité d’entre les morts, comment se trouve-t-il parmi vous des personnes qui osent dire que les morts ne ressuscitent point ?  \milestone{13}  Si les morts ne ressuscitent point, Jésus-Christ n’est donc point ressuscité :  \milestone{14}  et si Jésus-Christ n’est point ressuscité, notre prédication est vaine, et votre foi est vaine aussi.  \milestone{15}  Nous sommes même convaincus d’être de faux témoins à l’égard de Dieu ; comme ayant rendu ce témoignage contre Dieu même, qu’il a ressuscité Jésus-Christ, tandis que néanmoins il ne l’a pas ressuscité, si les morts ne ressuscitent point.  \milestone{16}  Car si les morts ne ressuscitent point, Jésus-Christ n’est point non plus ressuscité.  \milestone{17}  Si Jésus-Christ n’est point ressuscité, votre foi est donc vaine : vous êtes encore engagés dans vos péchés :  \milestone{18}  ceux qui sont morts en Jésus-Christ, sont donc péris sans ressource !  \milestone{19}  Si nous n’avions d’espérance en Jésus-Christ que pour cette vie, nous serions les plus misérables de tous les hommes.\par
  \milestone{20}  Mais maintenant Jésus-Christ est ressuscité d’entre les morts, et il est devenu les prémices de ceux qui dorment.  \milestone{21}  Ainsi, parce que la mort est venue par un homme, la résurrection des morts doit venir aussi par un homme.  \milestone{22}  Car comme tous meurent en Adam, tous revivront aussi en Jésus-Christ ;  \milestone{23}  et chacun en son rang : Jésus-Christ le premier, comme les prémices de tous ; puis ceux qui sont à lui, qui ont cru en son avènement.  \milestone{24}  Ensuite viendra la consommation de toutes choses, lorsqu’il aura remis son royaume à Dieu, son Père, et qu’il aura détruit tout empire, toute domination et toute puissance.  \milestone{25}  Car Jésus-Christ doit régner jusqu’à ce que son Père lui ait mis tous ses ennemis sous les pieds.  \milestone{26}  Or la mort sera le dernier ennemi qui sera détruit. Car l’Écriture dit que Dieu lui a mis tout sous les pieds, et lui a tout assujetti. Et quand elle dit  \milestone{27}  que tout lui est assujetti, il est indubitable qu’il faut en excepter celui qui lui a assujetti toutes choses.\par
  \milestone{28}  Lors donc que toutes choses auront été assujetties au Fils, alors le Fils sera lui-même assujetti à celui qui lui aura assujetti toutes choses, afin que Dieu soit tout en tous.\par
  \milestone{29}  Autrement que feront ceux qui se font baptiser pour les morts, s’il est vrai que les morts ne ressuscitent point ? Pourquoi se font-ils baptiser pour les morts ?  \milestone{30}  Et pourquoi nous-mêmes nous exposons-nous à toute heure à tant de périls ?  \milestone{31}  Il n’y a point de jour que je ne meure, je vous en assure, mes frères, par la gloire que je reçois de vous en Jésus-Christ notre Seigneur.  \milestone{32}  Si, pour parler à la manière des hommes, j’ai combattu à Éphèse contre des bêtes farouches, quel avantage en tirerai-je, si les morts ne ressuscitent point ? Ne pensons qu’à boire et à manger, puisque nous mourrons demain !  \milestone{33}  Ne vous laissez pas séduire : les mauvais entretiens gâtent les bonnes mœurs.  \milestone{34}  Justes, tenez-vous dans la vigilance, et gardez-vous du péché : car il y en a quelques-uns parmi vous qui ne connaissent point Dieu : je vous le dis pour vous faire honte.\par
\bigbreak
\noindent   \milestone{35}  Mais quelqu’un me dira : En quelle manière les morts ressusciteront-ils, et quel sera le corps dans lequel ils reviendront ?  \milestone{36}  Insensé que vous êtes ! ne voyez-vous pas que ce que vous semez ne reprend point vie s’il ne meurt auparavant ?\par
  \milestone{37}  Et quand vous semez, vous ne semez pas le corps de la plante qui doit naître, mais la graine seulement, comme du blé, ou de quelque autre chose.  \milestone{38}  Après quoi Dieu lui donne un corps tel qu’il lui plaît ; et il donne à chaque semence le corps qui est propre à chaque plante.  \milestone{39}  Toute chair n’est pas la même chair ; mais autre est la chair des hommes, autre la chair des bêtes, autre celle des oiseaux, autre celle des poissons.  \milestone{40}  Il y a aussi des corps célestes et des corps terrestres ; mais les corps célestes ont un autre éclat que les corps terrestres.  \milestone{41}  Le soleil a son éclat, qui diffère de l’éclat de la lune, comme l’éclat de la lune diffère de l’éclat des étoiles ; et entre les étoiles, l’une est plus éclatante que l’autre.\par
  \milestone{42}  Il en arrivera de même dans la résurrection des morts. Le corps comme une semence est maintenant mis en terre plein de corruption, et il ressuscitera incorruptible ;  \milestone{43}  il est mis en terre tout difforme, et il ressuscitera tout glorieux ; il est mis en terre privé de mouvement, et il ressuscitera plein de vigueur ;  \milestone{44}  il est mis en terre comme un corps animal, et il ressuscitera comme un corps spirituel. Comme il y a un corps animal, il y a aussi un corps spirituel ; selon qu’il est écrit :  \milestone{45}  Adam, le premier homme, a été créé avec une âme vivante ; et le second Adam a été rempli d’un esprit vivifiant.  \milestone{46}  Mais ce n’est pas le corps spirituel qui a été formé le premier ; c’est le corps animal, et ensuite le spirituel.  \milestone{47}  Le premier homme est le terrestre, formé de la terre ; et le second homme est le céleste, qui est venu du ciel.  \milestone{48}  Comme le premier homme a été terrestre, ses enfants aussi sont terrestres ; et comme le second homme est céleste, ses enfants aussi sont célestes.\par
  \milestone{49}  Comme donc nous avons porté l’image de l’homme terrestre, portons aussi l’image de l’homme céleste.  \milestone{50}  Je veux dire, mes frères, que la chair et le sang ne peuvent posséder le royaume de Dieu, et que la corruption ne possédera point cet héritage incorruptible.\par
  \milestone{51}  Voici un mystère que je vais vous dire : Nous ressusciterons tous, mais nous ne serons pas tous changés.  \milestone{52}  En un moment, en un clin d’œil, au son de la dernière trompette (car la trompette sonnera), les morts alors ressusciteront en un état incorruptible ; et nous serons changés.  \milestone{53}  Car il faut que ce corps corruptible soit revêtu de l’incorruptibilité, et que ce corps mortel soit revêtu de l’immortalité.\par
  \milestone{54}  Et quand ce corps mortel aura été revêtu de l’immortalité, alors cette parole de l’Écriture sera accomplie : La mort est absorbée par la victoire.  \milestone{55}  Ô mort ! où est ta victoire ? Ô mort ! où est ton aiguillon ?  \milestone{56}  Or le péché est l’aiguillon de la mort ; et la loi est la force du péché.\par
  \milestone{57}  C’est pourquoi rendons grâces à Dieu, qui nous donne la victoire par notre Seigneur Jésus-Christ.  \milestone{58}  Ainsi, mes chers frères, demeurez fermes et inébranlables, et travaillez sans cesse de plus en plus à l’œuvre de Dieu, sachant que votre travail ne sera pas sans récompense en notre Seigneur.

\section[{1 Cor 16}]{1 Cor 16}

\noindent \initial{Q}{UANT} aux aumônes qu’on recueille pour les saints, faites la même chose que j’ai ordonnée aux Églises de Galatie.  \milestone{2}  Que chacun de vous mette à part chez soi le premier jour de la semaine ce qu’il voudra, l’amassant peu à peu selon sa bonne volonté ; afin qu’on n’attende pas à mon arrivée à recueillir les aumônes.  \milestone{3}  Et lorsque je serai arrivé, j’enverrai avec des lettres de recommandation ceux que vous aurez jugés propres pour porter vos charités à Jérusalem.  \milestone{4}  Si la chose mérite que j’y aille moi-même, ils viendront avec moi.\par
\bigbreak
\noindent   \milestone{5}  Or je vous irai voir quand j’aurai passé par la Macédoine : car je passerai par cette province ;  \milestone{6}  et peut-être que je m’arrêterai chez vous, et que même j’y passerai l’hiver, afin que vous me conduisiez ensuite au lieu ou je pourrai aller.  \milestone{7}  Car je ne veux pas cette fois vous voir seulement en passant, et j’espère que je demeurerai assez longtemps chez vous, si le Seigneur le permet,  \milestone{8}  Je demeurerai à Éphèse jusqu’à la Pentecôte.  \milestone{9}  Car Dieu m’y ouvre visiblement une grande porte, et il s’y élève contre moi plusieurs ennemis.  \milestone{10}  Si Timothée va vous trouver, ayez soin qu’il soit en sûreté parmi vous, parce qu’il travaille à l’œuvre du Seigneur aussi bien que moi.  \milestone{11}  Que personne donc ne le méprise ; mais conduisez-le en paix, afin qu’il vienne me trouver ; parce que je l’attends avec nos frères.  \milestone{12}  Pour ce qui est de mon frère Apollon, je vous assure que je l’ai fort prié d’aller vous voir avec quelques-uns de nos frères : mais enfin il n’a pas cru devoir le faire présentement : il ira vous voir lorsqu’il en aura la commodité.\par
\bigbreak
\noindent   \milestone{13}  Soyez vigilants, demeurez fermes dans la foi, agissez courageusement, soyez pleins de force ;  \milestone{14}  faites avec amour tout ce que vous faites.  \milestone{15}  Vous connaissez, mes frères, la famille de Stéphanas, de Fortunat et d’Achaïque : vous savez qu’ils ont été les prémices de l’Achaïe, et qu’ils se sont consacrés au service des saints :  \milestone{16}  c’est pourquoi je vous supplie d’avoir pour eux la déférence due à des personnes de cette sorte, et pour tous ceux qui contribuent par leur peine et par leur travail à l’œuvre de Dieu.\par
  \milestone{17}  Je me réjouis de l’arrivée de Stéphanas, de Fortunat et d’Achaïque ; parce qu’ils ont suppléé ce que vous n’étiez pas à portée de faire par vous-mêmes :  \milestone{18}  car ils ont consolé mon esprit aussi bien que le vôtre. Honorez donc de telles personnes.\par
  \milestone{19}  Les Églises d’Asie vous saluent. Aquilas et Priscille, chez qui je demeure, et l’Église qui est dans leur maison, vous saluent avec beaucoup d’affection en notre Seigneur.  \milestone{20}  Tous nos frères vous saluent, saluez-vous les uns les autres par le saint baiser.\par
  \milestone{21}  Moi, Paul, j’ai écrit de ma main cette salutation.  \milestone{22}  Si quelqu’un n’aime point notre Seigneur Jésus-Christ, qu’il soit anathème ! Maran-Atha (c’est-à-dire, le Seigneur vient).\par
  \milestone{23}  Que la grâce de notre Seigneur Jésus-Christ soit avec vous !\par
  \milestone{24}  J’ai pour vous tous une charité sincère en Jésus-Christ. Amen !
\chapterclose


\chapteropen

\chapter[{Seconde épître de Paul aux Corinthiens (\textasciitilde57)}]{Seconde épître de Paul aux Corinthiens (\textasciitilde57)}
\renewcommand{\leftmark}{Seconde épître de Paul aux Corinthiens (\textasciitilde57)}


\chaptercont

\section[{2 Cor 1}]{2 Cor 1}

\noindent \initialiv{P}{AUL}, apôtre de Jésus-Christ, par la volonté de Dieu ; et Timothée, son frère : à l’Église de Dieu qui est à Corinthe, et à tous les saints qui sont dans toute l’Achaïe.  \milestone{2}  Que Dieu, notre Père, et Jésus-Christ notre Seigneur, vous donnent la grâce et la paix !\par
\bigbreak
\noindent   \milestone{3}  Béni soit Dieu, qui est le Père de notre Seigneur Jésus-Christ, le Père des miséricordes, et le Dieu de toute consolation ;  \milestone{4}  qui nous console dans tous nos maux, afin que nous puissions aussi consoler les autres dans tous leurs maux, par la même consolation dont nous sommes nous-mêmes consolés de Dieu.  \milestone{5}  Car à mesure que les souffrances de Jésus-Christ augmentent en nous, nos consolations aussi s’augmentent par Jésus-Christ.  \milestone{6}  Or, soit que nous soyons affligés, c’est pour votre instruction et pour votre salut ; soit que nous soyons consolés, c’est aussi pour votre consolation ; soit que nous soyons encouragés, c’est encore pour votre instruction et pour votre salut, qui s’accomplit dans la souffrance des mêmes maux que nous souffrons :  \milestone{7}  ce qui nous donne une ferme confiance pour vous, sachant qu’ainsi que vous avez part aux souffrances, vous aurez part aussi à la consolation.  \milestone{8}  Car je suis bien aise, mes frères, que vous sachiez l’affliction qui nous est survenue en Asie, qui a été telle que les maux dont nous nous sommes trouvés accablés, ont été excessifs et au-dessus de nos forces, jusqu’à nous rendre même la vie ennuyeuse.\par
  \milestone{9}  Mais nous avons comme entendu prononcer en nous-mêmes l’arrêt de notre mort, afin que nous ne mettions point notre confiance en nous, mais en Dieu, qui ressuscite les morts ;  \milestone{10}  qui nous a délivrés d’un si grand péril, qui nous en délivre encore, et nous en délivrera à l’avenir, comme nous l’espérons de sa bonté,  \milestone{11}  avec le secours des prières que vous faites pour nous ; afin que la grâce que nous avons reçue en considération de plusieurs personnes, soit aussi reconnue par les actions de grâces que plusieurs en rendront pour nous.\par
\bigbreak
\noindent   \milestone{12}  Car le sujet de notre gloire est le témoignage que nous rend notre conscience, de nous être conduits dans ce monde, et surtout à votre égard, dans la simplicité de cœur et dans la sincérité de Dieu, non avec la sagesse de la chair, mais dans la grâce de Dieu.  \milestone{13}  Je ne vous écris que des choses dont vous reconnaissez la vérité en les lisant ; et j’espère qu’à l’avenir vous connaîtrez entièrement,  \milestone{14}  ainsi que vous l’avez déjà reconnu en partie, que nous sommes votre gloire, comme vous serez la nôtre au jour de notre Seigneur Jésus-Christ.  \milestone{15}  C’est dans cette confiance que j’avais résolu auparavant d’aller vous voir ; afin que vous reçussiez une seconde grâce.  \milestone{16}  Je voulais passer par chez vous en allant en Macédoine, revenir ensuite de Macédoine chez vous, et de là me faire conduire par vous en Judée.  \milestone{17}  Ayant donc pour lors ce dessein, est-ce par inconstance que je ne l’ai point exécuté ? ou, quand je prends une résolution, cette résolution n’est-elle qu’humaine ? et trouve-t-on ainsi en moi le oui et le non ?  \milestone{18}  Mais Dieu qui est véritable, m’est témoin qu’il n’y a point eu de oui et de, non dans la parole que je vous ai annoncée.  \milestone{19}  Car Jésus-Christ, Fils de Dieu, qui vous a été prêché par nous, c’est-à-dire, par moi, par Silvain et par Timothée, n’est pas tel que le oui et le non se trouvent en lui ; mais tout ce qui est en lui, est très-ferme.  \milestone{20}  Car c’est en lui que toutes les promesses de Dieu ont leur vérité, et c’est par lui aussi qu’elles s’accomplissent à l’honneur de Dieu : ce qui fait la gloire de notre ministère.  \milestone{21}  Or celui qui nous confirme et nous affermit avec vous en Jésus-Christ, et qui nous a oints de son onction, c’est Dieu même.  \milestone{22}  Et c’est lui aussi qui nous a marqués de son sceau, et qui pour arrhes nous a donné le Saint-Esprit dans nos cœurs.  \milestone{23}  Pour moi, je prends Dieu à témoin, et je veux bien qu’il me punisse si je ne dis la vérité, que ç’a été pour vous épargner que je n’ai point encore voulu aller à Corinthe. Ce n’est pas que nous dominions sur votre foi ; mais nous tâchons au contraire de contribuer à votre joie, puisque vous demeurez fermes dans la foi.

\section[{2 Cor 2}]{2 Cor 2}

\noindent \initial{J}{E} résolus donc en moi-même, de ne point aller vous voir de nouveau, de peur de vous causer de la tristesse.  \milestone{2}  Car si je vous avais attristés, qui pourrait me réjouir ? puisque vous qui devriez le faire, seriez vous-mêmes dans la tristesse que je vous aurais causée.  \milestone{3}  C’est aussi ce que je vous avais écrit, afin que venant vers vous, je ne reçusse point tristesse sur tristesse de la part même de ceux qui devaient me donner de la joie ; ayant cette confiance en vous tous, que chacun de vous trouvera sa joie dans la mienne.  \milestone{4}  Et il est vrai que je vous écrivis alors dans une extrême affliction, dans un serrement de cœur, et avec une grande abondance de larmes, non dans le dessein de vous attrister, mais pour vous faire connaître la charité toute particulière que j’ai pour vous.\par
\bigbreak
\noindent   \milestone{5}  Si l’un de vous m’a attristé, il ne m’a pas attristé moi seul, mais vous tous aussi, au moins en quelque sorte : ce que je dis pour ne le point surcharger dans son affliction.  \milestone{6}  Il suffit pour cet homme, qu’il ait subi la correction et la peine qui lui a été imposée par votre assemblée ;  \milestone{7}  et vous devez plutôt le traiter maintenant avec indulgence et le consoler, de peur qu’il ne soit accablé par un excès de tristesse.\par
  \milestone{8}  C’est pourquoi je vous prie de lui donner des preuves effectives de votre charité.  \milestone{9}  Et c’est pour cela même que je vous en écris, afin de vous éprouver, et de reconnaître si vous êtes obéissants en toutes choses.  \milestone{10}  Ce que vous accordez à quelqu’un par indulgence, je l’accorde aussi. Car si j’use moi-même d’indulgence, j’en use à cause de vous, au nom et en la personne de Jésus-Christ ;  \milestone{11}  afin que Satan n’emporte rien sur nous : car nous n’ignorons pas ses desseins.\par
\bigbreak
\noindent   \milestone{12}  Or étant venu à Troade pour prêcher l’Evangile de Jésus-Christ, quoique le Seigneur m’y eût ouvert une entrée favorable,  \milestone{13}  je n’ai point eu l’esprit en repos, parce que je n’y avais point trouvé mon frère Tite. Mais ayant pris congé d’eux, je m’en suis allé en Macédoine.  \milestone{14}  Je rends grâces à Dieu, qui nous fait toujours triompher en Jésus-Christ, et qui répand par nous en tous lieux l’odeur de la connaissance de son nom.  \milestone{15}  Car nous sommes devant Dieu la bonne odeur de Jésus-Christ, soit à l’égard de ceux qui se sauvent, soit à l’égard de ceux qui se perdent :  \milestone{16}  aux uns une odeur de mort qui les fait mourir, et aux autres une odeur de vie qui les fait vivre. Et qui est capable d’un tel ministère ?\par
  \milestone{17}  Car nous ne sommes pas comme plusieurs, qui corrompent la parole de Dieu ; mais nous la prêchons avec une entière sincérité, comme de la part de Dieu, en la présence de Dieu, et dans la personne de Jésus-Christ.

\section[{2 Cor 3}]{2 Cor 3}

\noindent \initial{C}{OMMENCERONS}-nous de nouveau à nous relever nous-mêmes ? et avons-nous besoin, comme quelques-uns, que d’autres nous donnent des lettres de recommandation envers vous, ou que vous nous en donniez envers les autres ?  \milestone{2}  Vous êtes vous-mêmes notre lettre de recommandation, qui est écrite dans notre cœur, qui est reconnue et lue de tous les hommes ;  \milestone{3}  vos actions faisant voir que vous êtes la lettre de Jésus-Christ, dont nous avons été les secrétaires ; et qui est écrite, non avec de l’encre, mais avec l’Esprit du Dieu vivant ; non sur des tables de pierre, mais sur des tables de chair, qui sont vos cœurs.  \milestone{4}  C’est par Jésus-Christ que nous avons une si grande confiance en Dieu ;  \milestone{5}  non que nous soyons capables de former de nous-mêmes aucune bonne pensée comme de nous-mêmes ; mais c’est Dieu qui nous en rend capables.\par
  \milestone{6}  Et c’est lui aussi qui nous a rendus capables d’être les ministres de la nouvelle alliance, non pas de la lettre, mais de l’esprit : car la lettre tue, et l’esprit donne la vie.  \milestone{7}  Si le ministère de la lettre gravée sur des pierres, qui était un ministère de mort, a été accompagné d’une telle gloire, que les enfants d’Israël ne pouvaient regarder le visage de Moïse, à cause de la gloire dont il éclatait, laquelle devait néanmoins finir ;  \milestone{8}  combien le ministère de l’esprit doit-il être plus glorieux !  \milestone{9}  Car si le ministère de la condamnation a été accompagné de gloire, le ministère de la justice en aura incomparablement davantage.  \milestone{10}  Et cette gloire même de la loi n’est point une véritable gloire, si on la compare avec la sublimité de celle de l’Évangile.  \milestone{11}  Car si le ministère qui devait finir a été glorieux, celui qui durera toujours doit l’être beaucoup davantage.  \milestone{12}  Ayant donc une telle espérance, nous vous parlons avec toute sorte de liberté ;  \milestone{13}  et nous ne faisons pas comme Moïse, qui se mettait un voile sur le visage, afin que les enfants d’Israël ne vissent pas cette lumière passagère qui éclatait sur son visage.  \milestone{14}  Mais leurs esprits sont demeurés endurcis et aveuglés : car aujourd’hui même, lorsqu’ils lisent le Vieux Testament, ce voile demeure toujours sur leur cœur, sans être levé, parce qu’il ne s’ôte que par Jésus-Christ.  \milestone{15}  Ainsi jusqu’à cette heure, lorsqu’on leur lit Moïse, ils ont un voile sur le cœur.  \milestone{16}  Mais quand leur cœur se tournera vers le Seigneur, alors le voile en sera ôté.  \milestone{17}  Or le Seigneur est esprit ; et où est l’Esprit du Seigneur, là est aussi la liberté.  \milestone{18}  Ainsi nous tous, n’ayant point de voile qui nous couvre le visage, et contemplant la gloire du Seigneur, nous sommes transformés en la même image, nous avançant de clarté en clarté comme par l’illumination de l’Esprit du Seigneur.

\section[{2 Cor 4}]{2 Cor 4}

\noindent \initial{C\kern-0.08em{’}}{EST} pourquoi ayant reçu un tel ministère selon la miséricorde qui nous a été faite, nous ne nous laissons point abattre ;  \milestone{2}  mais nous rejetons loin de nous les passions qui se cachent comme étant honteuses, ne nous conduisant point avec artifice, et n’altérant point la parole de Dieu ; mais n’employant pour notre recommandation envers tous les hommes qui jugeront de nous selon le sentiment de leur conscience, que la sincérité avec laquelle nous prêchons devant Dieu la vérité de son Évangile.  \milestone{3}  Si l’Évangile que nous prêchons, est encore voilé, c’est pour ceux qui périssent qu’il est voilé ;  \milestone{4}  pour ces infidèles dont le dieu de ce siècle a aveuglé les esprits, afin qu’ils ne soient point éclairés par la lumière de l’Évangile de la gloire de Jésus-Christ, qui est l’image de Dieu.  \milestone{5}  Car nous ne nous prêchons pas nous-mêmes ; mais nous prêchons Jésus-Christ notre Seigneur ; et quant à nous, nous nous regardons comme vos serviteurs pour Jésus :  \milestone{6}  parce que le même Dieu qui a commandé que la lumière sortît des ténèbres, est celui qui a fait luire sa clarté dans nos cœurs, afin que nous puissions éclairer les autres, en leur donnant la connaissance de la gloire de Dieu, selon qu’elle paraît en Jésus-Christ.  \milestone{7}  Or nous portons ce trésor dans des vases de terre, afin qu’on reconnaisse que la grandeur de la puissance qui est en nous, est de Dieu, et non pas de nous. \\
  \milestone{8}  Nous sommes pressés de toutes sortes d’afflictions, mais nous n’en sommes pas accablés ; nous nous trouvons dans des difficultés insurmontables, mais nous n’y succombons pas ;  \milestone{9}  nous sommes persécutés, mais non pas abandonnés ; nous sommes abattus, mais non pas entièrement perdus ;  \milestone{10}  portant toujours en notre corps la mort de Jésus, afin que la vie de Jésus paraisse aussi dans notre corps.  \milestone{11}  Car nous qui vivons, nous sommes à toute heure livrés à la mort pour Jésus, afin que la vie de Jésus paraisse aussi dans notre chair mortelle.  \milestone{12}  Ainsi sa mort imprime ses effets en nous, et sa vie en vous.  \milestone{13}  Et parce que nous avons un même esprit de foi, selon qu’il est écrit, J’ai cru, c’est pourquoi j’ai parlé ; nous croyons aussi nous autres, et c’est aussi pourquoi nous parlons ;  \milestone{14}  sachant que celui qui a ressuscité Jésus, nous ressuscitera aussi avec Jésus, et nous fera comparaître avec vous en sa présence.  \milestone{15}  Car toutes choses sont pour vous, afin que plus la grâce se répand avec abondance, il en revienne aussi à Dieu plus de gloire par les témoignages de reconnaissance qui lui en seront rendus par plusieurs.\par
\bigbreak
\noindent   \milestone{16}  C’est pourquoi nous ne perdons point courage ; mais encore que dans nous l’homme extérieur se détruise, néanmoins l’homme intérieur se renouvelle de jour en jour :  \milestone{17}  car le moment si court et si léger des afflictions que nous souffrons en cette vie, produit en nous le poids éternel d’une souveraine et incomparable gloire :  \milestone{18}  ainsi nous ne considérons point les choses visibles, mais les invisibles ; parce que les choses visibles sont temporelles, mais les invisibles sont éternelles.

\section[{2 Cor 5}]{2 Cor 5}

\noindent \initial{A}{USSI} nous savons que si cette maison de terre où nous habitons vient à se dissoudre, Dieu nous donnera dans le ciel une autre maison, une maison qui ne sera point faite de main d’homme, et qui durera éternellement.  \milestone{2}  C’est ce qui nous fait soupirer dans le désir que nous avons d’être revêtus de la gloire de cette maison céleste qui nous est destinée, comme d’un second vêtement ;  \milestone{3}  si toutefois nous sommes trouvés vêtus, et non pas nus.  \milestone{4}  Car pendant que nous sommes dans ce corps comme dans une tente, nous soupirons sous sa pesanteur ; parce que nous ne désirons pas d’en être dépouillés, mais d’être revêtus par-dessus, en sorte que ce qu’il y a de mortel en nous, soit absorbé par la vie.  \milestone{5}  Or c’est Dieu qui nous a formés pour cet état d’immortalité, et qui nous a donné pour arrhes son Esprit.  \milestone{6}  Nous sommes donc toujours pleins de confiance ; et comme nous savons que pendant que nous habitons dans ce corps, nous sommes éloignés du Seigneur et hors de notre patrie ;  \milestone{7}  (parce que nous marchons vers lui par la foi, et que nous n’en jouissons pas encore par la claire vue ; )  \milestone{8}  dans cette confiance que nous avons, nous aimons mieux sortir de la maison de ce corps, pour aller habiter avec le Seigneur.  \milestone{9}  C’est pourquoi toute notre ambition est d’être agréables à Dieu, soit que nous habitions dans le corps, ou que nous en sortions pour aller à lui.  \milestone{10}  Car nous devons tous comparaître devant le tribunal de Jésus-Christ, afin que chacun reçoive ce qui est dû aux bonnes ou aux mauvaises actions qu’il aura faites pendant qu’il était revêtu de son corps.\par
\bigbreak
\noindent   \milestone{11}  Sachant donc combien le Seigneur est redoutable, nous tâchons de persuader les hommes de notre innocence : mais Dieu connaît qui nous sommes ; et je veux croire que nous sommes aussi connus de vous dans le secret de votre conscience.  \milestone{12}  Nous ne prétendons point nous relever encore ici nous-mêmes à votre égard, mais seulement vous donner occasion de vous glorifier à notre sujet ; afin que vous puissiez répondre à ceux qui mettent leur gloire dans ce qui paraît, et non dans ce qui est au fond du cœur.  \milestone{13}  Car soit que nous soyons emportés comme hors de nous-mêmes, c’est pour Dieu ; soit que nous nous tempérions, c’est pour vous :  \milestone{14}  parce que l’amour de Jésus-Christ nous presse ; considérant que si un seul est mort pour tous, donc tous sont morts ;  \milestone{15}  et en effet Jésus-Christ est mort pour tous, afin que ceux qui vivent, ne vivent plus pour eux-mêmes, mais pour celui qui est mort et qui est ressuscité pour eux.  \milestone{16}  C’est pourquoi nous ne connaissons plus désormais personne selon la chair ; et si nous avons connu Jésus-Christ selon la chair, maintenant nous ne le connaissons plus de cette sorte.  \milestone{17}  Si donc quelqu’un est en Jésus-Christ, il est devenu une nouvelle créature ; ce qui était devenu vieux est passé, et tout est devenu nouveau :  \milestone{18}  et le tout vient de Dieu, qui nous a réconciliés avec lui-même par Jésus-Christ, et qui nous a confié le ministère de la réconciliation.  \milestone{19}  Car c’est Dieu qui a réconcilié le monde avec soi en Jésus-Christ, ne leur imputant point leurs péchés ; et c’est lui qui a mis en nous la parole de réconciliation.  \milestone{20}  Nous faisons donc la fonction d’ambassadeurs pour Jésus-Christ, et c’est Dieu même qui vous exhorte par notre bouche. Ainsi nous vous conjurons, au nom de Jésus-Christ, de vous réconcilier avec Dieu ;  \milestone{21}  puisque pour l’amour de nous il a rendu victime pour le péché celui qui ne connaissait point le péché, afin qu’en lui nous devinssions justes de la justice de Dieu.

\section[{2 Cor 6}]{2 Cor 6}

\noindent \initial{É}{TANT} donc les coopérateurs de Dieu, nous vous exhortons de ne pas recevoir en vain la grâce de Dieu.  \milestone{2}  Car il dit lui-même :\par
\lpar{Je vous ai exaucé au temps favorable,}
\lpar{et je vous ai aidé au jour du salut.}
\lpar{Voici maintenant le temps favorable ;}
\lpar{voici maintenant le jour du salut.}
\noindent   \milestone{3}  Et nous prenons garde aussi nous-mêmes de ne donner à personne aucun sujet de scandale, afin que notre ministère ne soit point déshonoré.  \milestone{4}  Mais agissant en toutes choses comme des ministres de Dieu, nous nous rendons recommandables par une grande patience\par
dans les maux, dans les nécessités pressantes, et dans les extrêmes afflictions ;\par
  \milestone{5}  dans les plaies, dans les prisons, dans les séditions, dans les travaux, dans les veilles, dans les jeûnes ;\par
  \milestone{6}  par la pureté, par la science, par une douceur persévérante, par la bonté, par les fruits du Saint-Esprit, par une charité sincère ;\par
  \milestone{7}  par la parole de vérité, par la force de Dieu, par les armes de la justice, pour combattre à droite et à gauche ;\par
  \milestone{8}  parmi l’honneur et l’ignominie, parmi la mauvaise et la bonne réputation ; comme des séducteurs, quoique sincères et véritables ; comme inconnus, quoique très-connus ;\par
  \milestone{9}  comme toujours mourants, et vivants néanmoins ; comme châtiés, mais non jusqu’à être tués ;\par
  \milestone{10}  comme tristes, et toujours dans la joie ; comme pauvres, et enrichissant plusieurs ; comme n’ayant rien, et possédant tout.\par
\bigbreak
\noindent   \milestone{11}  Ô Corinthiens ! ma bouche s’ouvre, et mon cœur s’étend par l’affection que je vous porte.  \milestone{12}  Mes entrailles ne sont point resserrées pour vous, mais les vôtres le sont pour moi.  \milestone{13}  Rendez-moi donc amour pour amour. Je vous parle comme à mes enfants ; étendez aussi pour moi votre cœur.\par
\bigbreak
\noindent   \milestone{14}  Ne vous attachez point à un même joug avec les infidèles : car quelle union peut-il y avoir entre la justice et l’iniquité ? quel commerce entre la lumière et les ténèbres ?  \milestone{15}  Quel accord entre Jésus-Christ et Bélial ? quelle société entre le fidèle et l’infidèle ?  \milestone{16}  Quel rapport entre le temple de Dieu et les idoles ? Car vous êtes le temple du Dieu vivant, comme Dieu dit lui-même :\par
J’habiterai en eux, et je marcherai au milieu d’eux ;\par
je serai leur Dieu, et ils seront mon peuple.\par
  \milestone{17}  C’est pourquoi, sortez du milieu de ces personnes,\par
dit le Seigneur ;\par
séparez vous d’eux, et ne touchez point à ce qui est impur ;\par
  \milestone{18}  et je vous recevrai :\par
je serai votre Père, et vous serez mes fils et mes filles, dit le Seigneur tout-puissant.

\section[{2 Cor 7}]{2 Cor 7}

\noindent \initial{A}{YANT} donc reçu de Dieu de telles promesses, mes chers frères, purifions-nous de tout ce qui souille le corps ou l’esprit, achevant l’œuvre de notre sanctification dans la crainte de Dieu.\par
\bigbreak
\noindent   \milestone{2}  Donnez-nous place dans votre cœur. Nous n’avons fait tort à personne ; nous n’avons corrompu l’esprit de personne ; nous n’avons pris le bien de personne.  \milestone{3}  Je ne vous dis pas ceci pour vous condamner ; puisque je vous ai déjà dit que vous êtes dans mon cœur à la mort et à la vie.\par
  \milestone{4}  Je vous parle avec grande liberté ; j’ai grand sujet de me glorifier de vous ; je suis rempli de consolation, je suis comblé de joie parmi toutes mes souffrances.  \milestone{5}  Car étant venus en Macédoine, nous n’avons eu aucun relâche selon la chair ; mais nous avons toujours eu à souffrir. Ce n’a été que combats au dehors, et que frayeurs au dedans.  \milestone{6}  Mais Dieu, qui console les humbles et les affligés, nous a consolés par l’arrivée de Tite ;  \milestone{7}  et non-seulement par son arrivée, mais encore par la consolation qu’il a lui-même reçue de vous ; m’ayant rapporté l’extrême désir que vous avez de me revoir, la douleur que vous avez ressentie, et l’ardente affection que vous me portez : ce qui m’a été un plus grand sujet de joie.\par
  \milestone{8}  Car encore que je vous aie attristés par ma lettre, je n’en suis plus fâché néanmoins, quoique je l’aie été auparavant, en voyant qu’elle vous avait attristés pour un peu de temps.  \milestone{9}  Mais maintenant j’ai de la joie, non de ce que vous avez eu de la tristesse, mais de ce que votre tristesse vous a portés à la pénitence. La tristesse que vous avez eue a été selon Dieu ; et ainsi la peine que nous vous avons causée, ne vous a été nullement désavantageuse.  \milestone{10}  Car la tristesse qui est selon Dieu, produit pour le salut une pénitence stable ; mais la tristesse de ce monde produit la mort.  \milestone{11}  Considérez combien cette tristesse selon Dieu, que vous avez ressentie, a produit en vous non-seulement de soin et de vigilance, mais de satisfaction envers nous, d’indignation contre cet incestueux, de crainte de la colère de Dieu, de désir de nous revoir, de zèle pour nous défendre, d’ardeur à venger ce crime. Vous avez fait voir par toute votre conduite, que vous étiez purs et irréprochables dans cette affaire.  \milestone{12}  Aussi lorsque nous vous avons écrit, ce n’a été ni à cause de celui qui avait fait l’injure, ni à cause de celui qui l’avait soufferte, mais pour vous faire connaître le soin que nous avons de vous devant Dieu.\par
  \milestone{13}  C’est pourquoi ce que vous avez fait pour nous consoler, nous a en effet consolés ; et notre joie s’est encore beaucoup augmentée par celle de Tite, voyant que vous avez tous contribué au repos de son esprit ;  \milestone{14}  et que si je me suis loué de vous en lui parlant, je n’ai point eu sujet d’en rougir ; mais qu’ainsi que nous ne vous avions rien dit que dans la vérité, aussi le témoignage avantageux que nous avions rendu de vous à Tite, s’est trouvé conforme à la vérité.  \milestone{15}  C’est pourquoi il ressent dans ses entrailles un redoublement d’affection envers vous, lorsqu’il se souvient de l’obéissance que vous lui avez tous rendue, et comment vous l’avez reçu avec crainte et tremblement.  \milestone{16}  Je me réjouis donc de ce que je puis me promettre tout de vous.

\section[{2 Cor 8}]{2 Cor 8}

\noindent \initial{M}{AIS} il faut, mes frères, que je vous fasse savoir la grâce que Dieu a faite aux Églises de Macédoine :  \milestone{2}  c’est que leur joie est d’autant plus redoublée, qu’ils ont été éprouvés par de plus grandes afflictions ; et que leur profonde pauvreté a répandu avec abondance les richesses de leur charité sincère.  \milestone{3}  Car il est vrai, et il faut que je leur rende ce témoignage, qu’ils se sont portés d’eux-mêmes à donner autant qu’ils pouvaient, et même au delà de ce qu’ils pouvaient ;  \milestone{4}  nous conjurant avec beaucoup de prières de recevoir l’aumône qu’ils offraient pour prendre part à l’assistance destinée aux saints.  \milestone{5}  Et ils n’ont pas fait seulement en cela ce que nous avions espéré d’eux ; mais ils se sont donnés eux-mêmes premièrement au Seigneur, et puis à nous, par la volonté de Dieu.  \milestone{6}  C’est ce qui nous a portés à supplier Tite, que comme il a déjà commencé, il achève aussi de vous rendre parfaits en cette grâce ;  \milestone{7}  et que comme vous êtes riches en toutes choses, en foi, en paroles, en science, en toute sorte de soins, et en l’affection que vous nous portez, vous le soyez aussi en cette sorte de grâce.  \milestone{8}  Ce que je ne vous dis pas néanmoins pour vous imposer une loi, mais seulement pour vous porter, par l’exemple de l’ardeur des autres, à donner des preuves de votre charité sincère.  \milestone{9}  Car vous savez quelle a été la bonté de notre Seigneur Jésus-Christ, qui étant riche s’est rendu pauvre pour l’amour de vous, afin que vous devinssiez riches par sa pauvreté.\par
  \milestone{10}  C’est ici un conseil que je vous donne, parce que cela vous est utile, et que vous n’avez pas seulement commencé les premiers à faire cette charité, mais que vous en avez de vous-mêmes formé le dessein dès l’année passée.  \milestone{11}  Achevez donc maintenant ce que vous avez commencé de faire dès lors ; afin que comme vous avez une si prompte volonté d’assister vos frères, vous les assistiez aussi effectivement de ce que vous avez.  \milestone{12}  Car lorsqu’un homme a une grande volonté de donner, Dieu la reçoit, ne demandant de lui que ce qu’il peut, et non ce qu’il ne peut pas.  \milestone{13}  Ainsi je n’entends pas que les autres soient soulagés, et que vous soyez surchargés ;  \milestone{14}  mais que pour ôter l’inégalité, votre abondance supplée maintenant à leur pauvreté, afin que votre pauvreté soit soulagée un jour par leur abondance, et qu’ainsi tout soit réduit à l’égalité ;  \milestone{15}  selon ce qui est écrit de la manne : Celui qui en recueillit beaucoup, n’en eut pas plus que les autres ; et celui qui en recueillit peu, n’en eut pas moins.  \milestone{16}  Or je rends grâces à Dieu de ce qu’il a donné au cœur de Tite la même sollicitude que j’ai pour vous.  \milestone{17}  Car non-seulement il a bien reçu la prière que je lui ai faite, mais s’y étant porté avec encore plus d’affection par lui-même, il est parti de son propre mouvement pour vous aller voir.  \milestone{18}  Nous avons envoyé aussi avec lui notre frère qui est devenu célèbre par l’Évangile dans toutes les Églises ;  \milestone{19}  et qui de plus a été choisi par les Églises pour nous accompagner dans nos voyages, et prendre part au soin que nous avons de procurer cette assistance à nos frères, pour la gloire du Seigneur, et pour seconder notre bonne volonté.  \milestone{20}  Et notre dessein en cela a été d’éviter que personne ne puisse nous rien reprocher sur cette aumône abondante, dont nous sommes les dispensateurs :  \milestone{21}  car nous tâchons de faire le bien avec tant de circonspection, qu’il soit approuvé non-seulement de Dieu, mais aussi des hommes.  \milestone{22}  Nous avons envoyé encore avec eux notre frère, que nous avons reconnu zélé et vigilant en plusieurs rencontres, et qui l’est encore beaucoup plus en celle-ci ; et nous avons grande confiance que vous le recevrez bien ;  \milestone{23}  et que vous traiterez de même Tite, qui est uni avec moi, et qui travaille comme moi pour votre salut ; et nos autres frères qui sont les apôtres des Églises, et la gloire de Jésus-Christ.  \milestone{24}  Donnez-leur donc devant les Églises les preuves de votre charité, et faites voir que c’est avec sujet que nous nous sommes loués de vous.

\section[{2 Cor 9}]{2 Cor 9}

\noindent \initial{I}{L} serait superflu de vous écrire davantage touchant cette assistance qui se prépare pour les saints de Jérusalem.  \milestone{2}  Car je sais avec quelle affection vous vous y portez ; et c’est aussi ce qui me donne lieu de me glorifier de vous devant les Macédoniens, leur disant que la province d’Achaïe était disposée à faire cette charité dès l’année passée ; et votre exemple a excité le même zèle dans l’esprit de plusieurs.  \milestone{3}  C’est pourquoi j’ai envoyé nos frères vers vous, afin que ce ne soit pas en vain que je me sois loué de vous en ce point, et qu’on vous trouve tout prêts, selon l’assurance que j’en ai donnée ;  \milestone{4}  de peur que si ceux de Macédoine qui viendront avec moi, trouvaient que vous n’eussiez rien préparé, ce ne fût à nous, pour ne pas dire à vous-mêmes, un sujet de confusion dans cette conjoncture, de nous être loués de vous.  \milestone{5}  C’est ce qui m’a fait juger nécessaire de prier nos frères d’aller vous trouver avant moi, afin qu’ils aient soin que la charité que vous avez promis de faire, soit toute prête avant notre arrivée ; mais de telle sorte que ce soit un don offert par la charité, et non arraché à l’avarice.\par
  \milestone{6}  Or je vous avertis, mes frères, que celui qui sème peu, moissonnera peu ; et que celui qui sème avec abondance, moissonnera aussi avec abondance.\par
\bigbreak
\noindent   \milestone{7}  Ainsi que chacun donne ce qu’il aura résolu en lui-même de donner, non avec tristesse, ni comme par force : car Dieu aime celui qui donne avec joie.  \milestone{8}  Et Dieu est tout-puissant pour vous combler de toute grâce ; afin qu’ayant en tout temps et en toutes choses tout ce qui suffit pour votre subsistance, vous ayez abondamment de quoi exercer toutes sortes de bonnes œuvres ;  \milestone{9}  selon ce qui est écrit : Le juste distribue son bien ; il donne aux pauvres ; sa justice demeure éternellement.\par
\bigbreak
\noindent   \milestone{10}  Dieu, qui donne la semence à celui qui sème, vous donnera le pain dont vous avez besoin pour vivre, et multipliera ce que vous aurez semé, et fera croître de plus en plus les fruits de votre justice ;  \milestone{11}  afin que vous soyez riches en tout, pour exercer avec un cœur simple toute sorte de charités : ce qui nous donne sujet de rendre à Dieu de grandes actions de grâces.  \milestone{12}  Car cette oblation dont nous sommes les ministres, ne supplée pas seulement aux besoins des saints ; mais elle est riche et abondante par le grand nombre d’actions de grâces qu’elle fait rendre à Dieu ;  \milestone{13}  parce que ces saints recevant ces preuves de votre libéralité par notre ministère, se portent à glorifier Dieu de la soumission que vous témoignez à l’Évangile de Jésus-Christ, et de la bonté avec laquelle vous faites part de vos biens, soit à eux, soit à tous les autres ;  \milestone{14}  et de plus elle est riche et abondante par les prières qu’ils font pour vous, dans l’affection qu’ils vous portent à cause de l’excellente grâce que vous avez reçue de Dieu.  \milestone{15}  Dieu soit loué de son ineffable don !

\section[{2 Cor 10}]{2 Cor 10}

\noindent \initial{M}{AIS} moi, Paul, moi-même qui vous parle, je vous conjure par la douceur et la modestie de Jésus-Christ : moi qui, selon quelques-uns, étant présent parais bas et méprisable parmi vous ; au lieu qu’étant absent j’agis envers vous avec hardiesse :  \milestone{2}  je vous prie, que quand je serai présent je ne sois point obligé d’user avec confiance de cette hardiesse qu’on m’attribue ; d’en user, dis-je, envers quelques-uns qui s’imaginent que nous nous conduisons selon la chair.  \milestone{3}  Car encore que nous vivions dans la chair, nous ne combattons pas selon la chair.  \milestone{4}  Les armes de notre milice ne sont point charnelles, mais puissantes en Dieu, pour renverser les remparts qu’on leur oppose ; et c’est par ces armes que nous détruisons les raisonnements humains,  \milestone{5}  et tout ce qui s’élève avec hauteur contre la science de Dieu ; et que nous réduisons en servitude tous les esprits, pour les soumettre à l’obéissance de Jésus-Christ ;  \milestone{6}  ayant en notre main le pouvoir de punir tous les désobéissants, lorsque vous aurez satisfait à tout ce que l’obéissance demande de vous.  \milestone{7}  Jugez au moins des choses selon l’apparence. Si quelqu’un se persuade en lui-même qu’il est à Jésus-Christ, il doit aussi considérer en lui-même que comme il est à Jésus-Christ, nous sommes aussi à Jésus-Christ.  \milestone{8}  Car quand je me glorifierais un peu davantage de la puissance que le Seigneur m’a donnée pour votre édification, et non pour votre destruction, je n’aurais pas sujet d’en rougir. \\
  \milestone{9}  Mais afin qu’il ne semble pas que nous voulions vous étonner par des lettres :  \milestone{10}  (parce que les lettres de Paul, disent-ils, sont graves et fortes ; mais lorsqu’il est présent il paraît bas en sa personne, et méprisable en son discours : )  \milestone{11}  que celui qui est dans ce sentiment, considère qu’étant présents, nous nous conduisons dans nos actions de la même manière que nous parlons dans nos lettres étant absents.  \milestone{12}  Car nous n’osons pas nous mettre au rang de quelques-uns qui se relèvent eux-mêmes, ni nous comparer à eux ; mais nous nous mesurons sur ce que nous sommes véritablement en nous, et nous ne nous comparons qu’avec nous-mêmes.  \milestone{13}  Et ainsi quant à nous, nous ne nous glorifierons point démesurément ; mais nous renfermant dans les bornes du partage que Dieu nous a donné, nous nous glorifierons d’être parvenus jusqu’à vous.  \milestone{14}  Car nous ne nous étendons pas au delà de ce que nous devons, comme si nous n’étions pas parvenus jusqu’à vous, puisque nous sommes arrivés jusqu’à vous en prêchant l’Évangile de Jésus-Christ.  \milestone{15}  Nous ne nous relevons donc point démesurément, en nous attribuant les travaux des autres ; mais nous espérons que votre foi croissant toujours de plus en plus, nous étendrons beaucoup en vous notre partage,  \milestone{16}  et que nous prêcherons l’Évangile aux nations mêmes qui sont au delà de vous, sans entreprendre sur le partage d’un autre, en nous glorifiant d’avoir bâti sur ce qu’il aurait déjà préparé.  \milestone{17}  Que celui donc qui se glorifie, ne se glorifie que dans le Seigneur.  \milestone{18}  Car ce n’est pas celui qui se rend témoignage à soi-même qui est vraiment estimable ; mais c’est celui à qui Dieu rend témoignage.

\section[{2 Cor 11}]{2 Cor 11}

\noindent \initial{P}{LUT} à Dieu que vous voulussiez un peu supporter mon imprudence ! et supportez-la, je vous prie.  \milestone{2}  Car j’ai pour vous un amour de jalousie, et d’une jalousie de Dieu ; parce que je vous ai fiancés à cet unique époux, qui est Jésus-Christ, pour vous présenter à lui comme une vierge toute pure.  \milestone{3}  Mais j’appréhende qu’ainsi que le serpent séduisit Ève par ses artifices, vos esprits aussi ne se corrompent, et ne dégénèrent de la simplicité chrétienne.  \milestone{4}  Car si celui qui vient vous prêcher, vous annonçait un autre Jésus-Christ que celui que nous vous avons annoncé ; ou s’il vous faisait recevoir un autre esprit que celui que vous avez reçu ; ou s’il vous prêchait un autre évangile que celui que vous avez embrassé, vous auriez raison de le souffrir !  \milestone{5}  Mais je ne pense pas avoir été inférieur en rien aux plus grands d’entre les apôtres.  \milestone{6}  Si je suis grossier et peu instruit pour la parole, il n’en est pas de même pour la science ; mais nous nous sommes fait assez connaître parmi vous en toutes choses.  \milestone{7}  Est-ce que j’ai fait une faute, lorsque afin de vous élever je me suis rabaissé moi-même en vous prêchant gratuitement l’Évangile de Dieu ?  \milestone{8}  J’ai dépouillé les autres Églises, en recevant d’elles l’assistance dont j’avais besoin pour vous servir.  \milestone{9}  Et lorsque je demeurais parmi vous, et que j’étais dans la nécessité, je n’ai été a charge à personne ; mais nos frères qui étaient venus de Macédoine, ont suppléé aux besoins que je pouvais avoir ; et j’ai pris garde à ne vous être à charge en quoi que ce soit, comme je ferai encore à l’avenir.  \milestone{10}  Je vous assure par la vérité de Jésus-Christ qui est en moi, qu’on ne me ravira point cette gloire dans toute l’Achaïe.  \milestone{11}  Et pourquoi ? Est-ce que je ne vous aime pas ? Dieu le sait.  \milestone{12}  Mais je fais cela, et je le ferai encore, afin de retrancher une occasion de se glorifier à ceux qui la cherchent, en voulant paraître tout à fait semblables à nous, pour trouver en cela un sujet de gloire.  \milestone{13}  Car ces personnes sont de faux apôtres, des ouvriers trompeurs, qui se transforment en apôtres de Jésus-Christ.  \milestone{14}  Et l’on ne doit pas s’en étonner, puisque Satan même se transforme en ange de lumière.  \milestone{15}  Il n’est donc pas étrange que ses ministres aussi se transforment en ministres de la justice ; mais leur fin sera conforme à leurs œuvres.\par
\bigbreak
\noindent   \milestone{16}  Je vous le dis encore une fois (que personne ne me juge imprudent ; ou au moins souffrez-moi comme imprudent, et permettez-moi de me glorifier un peu) :  \milestone{17}  Croyez, si vous voulez, que ce que je dis, je ne le dis pas selon Dieu, mais que je fais paraître de l’imprudence dans ce que je prends pour un sujet de me glorifier.  \milestone{18}  Puisque plusieurs se glorifient selon la chair, je puis bien aussi me glorifier comme eux.  \milestone{19}  Car étant sages comme vous êtes, vous souffrez sans peine les imprudents.  \milestone{20}  Vous souffrez même qu’on vous asservisse, qu’on vous mange, qu’on prenne votre bien, qu’on vous traite avec hauteur, qu’on vous frappe au visage.  \milestone{21}  C’est à ma confusion que je le dis, puisque nous passons pour avoir été trop faibles en ce point. Mais puisqu’il y en a qui sont si hardis à parler d’eux-mêmes, je veux bien faire une imprudence en me rendant aussi hardi qu’eux.  \milestone{22}  Sont-ils Hébreux ? Je le suis aussi. Sont-ils Israélites ? Je le suis aussi. Sont-ils de la race d’Abraham ? J’en suis aussi.  \milestone{23}  Sont-ils ministres de Jésus-Christ ? Quand je devrais passer pour imprudent, j’ose dire que je le suis encore plus qu’eux.\par
\bigbreak
\noindent J’ai plus souffert de travaux, plus reçu de coups, plus enduré de prisons ; je me suis souvent vu tout près de la mort.\par
  \milestone{24}  J’ai reçu des Juifs cinq différentes fois, trente-neuf coups de fouet.\par
  \milestone{25}  J’ai été battu de verges par trois fois, j’ai été lapidé une fois, j’ai fait naufrage trois fois, j’ai passé un jour et une nuit au fond de la mer.\par
  \milestone{26}  J’ai été souvent dans les voyages, dans les périls sur les fleuves, dans les périls des voleurs, dans les périls de la part de ceux de ma nation, dans les périls de la part des païens, dans les périls au milieu des villes, dans les périls au milieu des déserts, dans les périls sur mer, dans les périls entre les faux frères.\par
  \milestone{27}  J’ai souffert toutes sortes de travaux et de fatigues, de fréquentes veilles, la faim, la soif, beaucoup de jeûnes, le froid et la nudité.\par
  \milestone{28}  Outre ces maux qui ne sont qu’extérieurs, le soin que j’ai de toutes les Églises m’attire une foule d’affaires dont je suis assiégé tous les jours.\par
  \milestone{29}  Qui est faible sans que je m’affaiblisse avec lui ? Qui est scandalisé sans que je brûle ?  \milestone{30}  S’il faut se glorifier de quelque chose, je me glorifierai des souffrances qui me font paraître faible.  \milestone{31}  Dieu, qui est le Père de notre Seigneur Jésus-Christ, et qui est béni dans tous les siècles, sait que je ne mens point.  \milestone{32}  Étant à Damas, celui qui était gouverneur de la province pour le roi Arétas, faisait faire garde dans la ville pour m’arrêter prisonnier ;  \milestone{33}  mais on me descendit dans une corbeille, par une fenêtre, le long de la muraille ; et je me sauvai ainsi de ses mains.

\section[{2 Cor 12}]{2 Cor 12}

\noindent \initial{S\kern-0.08em{’}}{IL} faut se glorifier (quoiqu’il ne soit pas avantageux de le faire), je viendrai maintenant aux visions et aux révélations du Seigneur.  \milestone{2}  Je connais un homme en Jésus-Christ, qui fut ravi il y a quatorze ans (si ce fut avec son corps, ou sans son corps, je ne sais, Dieu le sait), qui fut ravi, dis-je, jusqu’au troisième ciel ;\par
  \milestone{3}  et je sais que cet homme (si ce fut avec son corps, ou sans son corps, je n’en sais rien, Dieu le sait),  \milestone{4}  que cet homme, dis-je, fut ravi dans le paradis, et qu’il y entendit des paroles ineffables, qu’il n’est pas permis à un homme de rapporter.  \milestone{5}  Je pourrais me glorifier en parlant d’un tel homme ; mais pour moi, je ne veux me glorifier que dans mes faiblesses et dans mes afflictions.  \milestone{6}  Si je voulais me glorifier, je pourrais le faire sans être imprudent ; car je dirais la vérité : mais je me retiens, de peur que quelqu’un ne m’estime au-dessus de ce qu’il voit en moi, ou de ce qu’il entend dire de moi.  \milestone{7}  Aussi, de peur que la grandeur de mes révélations ne me causât de l’élèvement, Dieu a permis que je ressentisse dans ma chair un aiguillon, qui est l’ange et le ministre de Satan, pour me donner des soufflets.  \milestone{8}  C’est pourquoi j’ai prié trois fois le Seigneur, afin que cet ange de Satan se retirât de moi ;  \milestone{9}  et il m’a répondu : Ma grâce vous suffit : car ma puissance éclate davantage dans la faiblesse. Je prendrai donc plaisir à me glorifier dans mes faiblesses, afin que la puissance de Jésus-Christ habite en moi.\par
\bigbreak
\noindent   \milestone{10}  Et ainsi je sens de la satisfaction et de la joie dans les faiblesses, dans les outrages, dans les nécessités où je me trouve réduit, dans les persécutions, dans les afflictions pressantes que je souffre pour Jésus-Christ : car lorsque je suis faible, c’est alors que je suis fort.\par
\bigbreak
\noindent   \milestone{11}  J’ai été imprudent ; c’est vous qui m’y avez contraint. Car c’était à vous de parler avantageusement de moi, puisque je n’ai été en rien inférieur aux plus éminents d’entre les apôtres, encore que je ne sois rien.  \milestone{12}  Aussi les marques de mon apostolat ont paru parmi vous dans toute sorte de tolérance et de patience, dans les miracles, dans les prodiges, et dans les effets extraordinaires de la puissance divine.  \milestone{13}  Car en quoi avez-vous été inférieurs aux autres Églises, si ce n’est en ce que je n’ai point voulu vous être à charge ? Pardonnez-moi ce tort que je vous ai fait.\par
  \milestone{14}  Voici la troisième fois que je me prépare pour vous aller voir, et ce sera encore sans vous être à charge. Car c’est vous que je cherche, et non vos biens ; puisque ce n’est pas aux enfants à amasser des trésors pour leurs pères, mais aux pères à en amasser pour leurs enfants.  \milestone{15}  Aussi, pour ce qui est de moi, je donnerai très-volontiers tout ce que j’ai, et je me donnerai encore moi-même, pour le salut de vos âmes ; quoique ayant tant d’affection pour vous, vous en ayez peu pour moi.  \milestone{16}  On dira peut-être, qu’il est vrai que je ne vous ai point été à charge, mais qu’étant artificieux, j’ai usé d’adresse pour vous surprendre.  \milestone{17}  Mais me suis-je servi de quelqu’un de ceux que je vous ai envoyés, pour tirer quelque chose de vous ?  \milestone{18}  J’ai prié Tite de vous aller trouver, et j’ai envoyé encore avec lui un de nos frères. Tite a-t-il tiré quelque chose de vous ? N’avons-nous pas suivi le même esprit ? N’avons-nous pas marché sur les mêmes traces ?  \milestone{19}  Pensez-vous que ce soit encore ici notre dessein de nous justifier devant vous ? Nous vous parlons devant Dieu en Jésus-Christ, et tout ce que nous vous disons, mes chers frères, est pour votre édification.  \milestone{20}  Car j’appréhende qu’arrivant vers vous, je ne vous trouve pas tels que je voudrais, et que vous ne me trouviez aussi tel que vous ne voudriez pas. Je crains de rencontrer parmi vous des dissensions, des jalousies, des animosités, des querelles, des médisances, de faux rapports, des élèvements d’orgueil, des troubles et des tumultes ;  \milestone{21}  et qu’ainsi Dieu ne m’humilie, lorsque je serai revenu chez vous, et que je ne sois obligé d’en pleurer plusieurs, qui étant déjà tombés en des impuretés, des fornications et des dérèglements infâmes, n’en ont point fait pénitence.

\section[{2 Cor 13}]{2 Cor 13}

\noindent \initial{V}{OICI} donc la troisième fois que je me dispose à vous aller voir. Tout se jugera sur le témoignage de deux ou trois témoins.  \milestone{2}  Je vous l’ai déjà dit, et je vous le dis encore maintenant, quoique absent, mais comme devant être bientôt parmi vous, que si j’y viens encore une fois, je ne pardonnerai ni à ceux qui avaient péché auparavant, ni à tous les autres.  \milestone{3}  Est-ce que vous voulez éprouver la puissance de Jésus-Christ qui parle par ma bouche, qui n’a point paru faible, mais très-puissant parmi vous ?  \milestone{4}  Car encore qu’il ait été crucifié selon la faiblesse de la chair, il vit néanmoins maintenant par la vertu de Dieu. Nous sommes faibles aussi avec lui ; mais nous vivrons avec lui par la vertu de Dieu qui éclate parmi vous.  \milestone{5}  Examinez-vous vous-mêmes, pour reconnaître si vous êtes dans la foi ; éprouvez-vous vous-mêmes. Ne connaissez-vous pas vous-mêmes que Jésus-Christ est en vous ? si ce n’est, peut-être, que vous fussiez déchus de ce que vous étiez.  \milestone{6}  Mais j’espère que vous connaîtrez que pour nous, nous ne sommes point déchus de ce que nous étions.  \milestone{7}  Ce que nous demandons à Dieu, est que vous ne commettiez aucun mal, et non pas que nous paraissions n’être point déchus de ce que nous étions ; mais que vous fassiez ce qui est de votre devoir, quand même nous devrions paraître déchus de ce que nous étions.  \milestone{8}  Car nous ne pouvons rien contre la vérité, mais seulement pour la vérité.  \milestone{9}  Et nous nous réjouissons de ce que nous paraissons faibles pendant que vous êtes forts ; et nous demandons aussi à Dieu, qu’il vous rende parfaits.  \milestone{10}  Je vous écris ceci étant absent, afin de n’avoir pas lieu, lorsque je serai présent, d’user avec sévérité de la puissance que le Seigneur m’a donnée pour édifier, et non pour détruire.  \milestone{11}  Enfin, mes frères, soyez dans la joie, travaillez à être parfaits, consolez-vous, soyez unis d’esprit et de cœur, vivez dans la paix ; et le Dieu d’amour et de paix sera avec vous.  \milestone{12}  Saluez-vous les uns les autres par un saint baiser. Tous les saints vous saluent.  \milestone{13}  Que la grâce de notre Seigneur Jésus-Christ, l’amour de Dieu, et la communication du Saint-Esprit, demeure avec vous tous ! Amen !
\chapterclose


\chapteropen

\chapter[{Épître de Paul aux Romains (57\textasciitilde58)}]{Épître de Paul aux Romains (57\textasciitilde58)}
\renewcommand{\leftmark}{Épître de Paul aux Romains (57\textasciitilde58)}


\chaptercont

\section[{Rom 1}]{Rom 1}

\noindent \initialiv{P}{AUL}, serviteur de Jésus-Christ, apôtre par la vocation divine, choisi et destiné pour annoncer l’Évangile de Dieu,\par
  \milestone{2}  qu’il avait promis auparavant par ses prophètes, dans les Écritures saintes,  \milestone{3}  touchant son Fils, qui lui est né, selon la chair, du sang et de la race de David ;  \milestone{4}  qui a été prédestiné pour être Fils de Dieu dans une souveraine puissance, selon l’Esprit de sainteté, par sa résurrection d’entre les morts ; touchant, dis-je, Jésus-Christ notre Seigneur ;  \milestone{5}  par qui nous avons reçu la grâce et l’apostolat, pour faire obéir à la foi toutes les nations, par la vertu de son nom ;  \milestone{6}  au rang desquelles vous êtes aussi, comme ayant été appelés par Jésus-Christ :\par
  \milestone{7}  à vous tous qui êtes à Rome, qui êtes chéris de Dieu, et appelés pour être saints. Que Dieu, notre Père, et Jésus-Christ notre Seigneur, vous donnent la grâce et la paix !\par
\bigbreak
\noindent   \milestone{8}  Premièrement, je rends grâces à mon Dieu pour vous tous, par Jésus-Christ, de ce qu’on parle de votre foi dans tout le monde.  \milestone{9}  Car le Dieu que je sers par le culte intérieur de mon esprit dans l’Évangile de son Fils, m’est témoin que je me souviens sans cesse de vous ;  \milestone{10}  lui demandant continuellement dans mes prières, que, si c’est sa volonté, il m’ouvre enfin quelque voie favorable pour aller vers vous :  \milestone{11}  car j’ai grand désir de vous voir, pour vous faire part de quelque grâce spirituelle, afin de vous fortifier ;  \milestone{12}  c’est-à-dire, afin qu’étant parmi vous, nous recevions une mutuelle consolation dans la foi qui nous est commune.\par
\bigbreak
\noindent   \milestone{13}  Aussi, mes frères, je ne veux pas que vous ignoriez que j’avais souvent proposé de vous aller voir, pour faire quelque fruit parmi vous, comme parmi les autres nations ; mais j’en ai été empêché jusqu’à cette heure.  \milestone{14}  Je suis redevable aux Grecs et aux barbares, aux savants et aux simples.  \milestone{15}  Ainsi, pour ce qui est de moi, je suis prêt à vous annoncer aussi l’Évangile, à vous qui êtes à Rome :\par
  \milestone{16}  car je ne rougis point de l’Évangile, parce qu’il est la vertu de Dieu, pour sauver tous ceux qui croient, premièrement les Juifs, et puis les gentils.  \milestone{17}  Car la justice de Dieu nous y est révélée, la justice qui vient de la foi, et se perfectionne dans la foi, selon qu’il est écrit : Le juste vit de la foi.\par
\bigbreak
\noindent   \milestone{18}  On y découvre aussi la colère de Dieu, qui éclatera du ciel contre toute l’impiété et l’injustice des hommes, qui retiennent la vérité de Dieu dans l’injustice ;  \milestone{19}  parce qu’ils ont connu ce qui peut se découvrir de Dieu, Dieu même le leur ayant fait connaître.  \milestone{20}  Car les perfections invisibles de Dieu, sa puissance éternelle et sa divinité, sont devenues visibles depuis la création du monde, par la connaissance que ses créatures nous en donnent ; et ainsi ces personnes sont inexcusables ;  \milestone{21}  parce qu’ayant connu Dieu, ils ne l’ont point glorifié comme Dieu, et ne lui ont point rendu grâces ; mais ils se sont égarés dans leurs vains raisonnements, et leur cœur insensé a été rempli de ténèbres.  \milestone{22}  Ils sont devenus fous, en s’attribuant le nom de sages ;  \milestone{23}  et ils ont transféré l’honneur qui n’est dû qu’au Dieu incorruptible, à l’image d’un homme corruptible, et à des figures d’oiseaux, de bêtes à quatre pieds, et de reptiles.\par
  \milestone{24}  C’est pourquoi Dieu les a livrés aux désirs de leur cœur, aux vices de l’impureté ; en sorte qu’en s’y plongeant, ils ont déshonoré eux-mêmes leurs propres corps,  \milestone{25}  eux qui avaient mis le mensonge en la place de la vérité de Dieu, et rendu à la créature l’adoration et le culte souverain, au lieu de le rendre au Créateur, qui est béni dans tous les siècles. Amen !\par
  \milestone{26}  C’est pourquoi Dieu les a livrés à des passions honteuses ; car les femmes parmi eux ont changé l’usage qui est selon la nature, en un autre qui est contre la nature.  \milestone{27}  Les hommes de même, rejetant l’alliance des deux sexes, qui est selon la nature, ont été embrasés d’un désir brutal les uns envers les autres, l’homme commettant avec l’homme une infamie détestable, et recevant ainsi en eux-mêmes la juste peine qui était due à leur égarement.\par
  \milestone{28}  Et comme ils n’ont pas voulu reconnaître Dieu, Dieu aussi les a livrés à un sens dépravé ; en sorte qu’ils ont fait des actions indignes de l’homme ;  \milestone{29}  ils ont été remplis de toute sorte d’injustice, de méchanceté, de fornication, d’avarice, de malignité ; ils ont été envieux, meurtriers, querelleurs, trompeurs ; ils ont été corrompus dans leurs mœurs, semeurs de faux rapports,  \milestone{30}  calomniateurs, et ennemis de Dieu; ils ont été outrageux, superbes, altiers, inventeurs de nouveaux moyens de faire le mal, désobéissants à leurs pères et à leurs mères ;  \milestone{31}  sans prudence, sans modestie, sans affection, sans fidélité, sans miséricorde.  \milestone{32}  Et après avoir connu la justice de Dieu, ils n’ont pas compris que ceux qui font ces choses sont dignes de mort, et non-seulement ceux qui les font, mais aussi ceux qui approuvent ceux qui les font.

\section[{Rom 2}]{Rom 2}

\noindent \initial{C\kern-0.08em{’}}{EST} pourquoi, vous, ô homme ! qui que vous soyez, qui condamnez les autres, vous êtes inexcusable ; parce qu’en les condamnant, vous vous condamnez vous-même, puisque vous faites les mêmes choses que vous condamnez.  \milestone{2}  Car nous savons que Dieu condamne, selon sa vérité, ceux qui commettent ces actions.  \milestone{3}  Vous donc, qui condamnez ceux qui les commettent, et qui les commettez vous-même, pensez-vous pouvoir éviter la condamnation de Dieu ?  \milestone{4}  Est-ce que vous méprisez les richesses de sa bonté, de sa patience, et de sa longue tolérance ? Ignorez-vous que la bonté de Dieu vous invite à la pénitence ?  \milestone{5}  Et cependant, par votre dureté et par l’impénitence de votre cœur, vous vous amassez un trésor de colère pour le jour de la colère et de la manifestation du juste jugement de Dieu ;  \milestone{6}  qui rendra à chacun selon ses œuvres,  \milestone{7}  en donnant la vie éternelle à ceux qui par leur persévérance dans les bonnes œuvres, cherchent la gloire, l’honneur et l’immortalité ;  \milestone{8}  et répandant sa fureur et sa colère sur ceux qui ont l’esprit contentieux, et qui ne se rendent point à la vérité, mais qui embrassent l’iniquité.  \milestone{9}  L’affliction et le désespoir accablera l’âme de tout homme qui fait le mal, du Juif premièrement, et puis du gentil ;  \milestone{10}  mais la gloire, l’honneur et la paix seront le partage de tout homme qui fait le bien, du Juif premièrement, et puis du gentil.  \milestone{11}  Car Dieu ne fait point acception de personnes.\par
  \milestone{12}  Et ainsi tous ceux qui ont péché sans avoir reçu la loi, périront aussi sans être jugés par la loi ; et tous ceux qui ont péché étant sous la loi, seront jugés par la loi.  \milestone{13}  (Car ce ne sont point ceux qui écoutent la loi, qui sont justes devant Dieu ; mais ce sont ceux qui gardent la loi, qui seront justifiés.  \milestone{14}  Lors donc que les gentils qui n’ont point la loi, font naturellement les choses que la loi commande, n’ayant point la loi, ils se tiennent à eux-mêmes lieu de loi ;  \milestone{15}  faisant voir que ce qui est prescrit par la loi, est écrit dans leur cœur, comme leur conscience en rend témoignage par la diversité des réflexions et des pensées qui les accusent, ou qui les défendent.)  \milestone{16}  Tous ceux, dis-je, qui ont péché, périront et seront condamnés au jour où Dieu jugera par Jésus-Christ, selon l’Évangile que je prêche, tout ce qui est caché dans le cœur des hommes.\par
\bigbreak
\noindent   \milestone{17}  Mais vous qui portez le nom de Juif, qui vous reposez sur la loi, qui vous glorifiez des faveurs de Dieu ;  \milestone{18}  qui connaissez sa volonté, et qui étant instruit par la loi, savez discerner ce qui est de plus utile ;  \milestone{19}  vous vous flattez d’être le conducteur des aveugles, la lumière de ceux qui sont dans les ténèbres,  \milestone{20}  le docteur des ignorants, le maître des simples et des enfants, comme ayant dans la loi la règle de la science et de la vérité.  \milestone{21}  Et cependant, vous qui instruisez les autres, vous ne vous instruisez pas vous-même ; vous qui publiez qu’on ne doit point voler, vous volez ;  \milestone{22}  vous qui dites qu’on ne doit point commettre d’adultère, vous commettez des adultères ; vous qui avez en horreur les idoles, vous faites des sacrilèges ;  \milestone{23}  vous qui vous glorifiez dans la loi, vous déshonorez Dieu par le violement de la loi.  \milestone{24}  Car vous êtes cause, comme dit l’Écriture, que le nom de Dieu est blasphémé parmi les nations.\par
  \milestone{25}  Ce n’est pas que la circoncision ne soit utile, si vous accomplissez la loi ; mais si vous la violez, tout circoncis que vous êtes, vous devenez comme un homme incirconcis.  \milestone{26}  Si donc un homme incirconcis garde les ordonnances de la loi, n’est-il pas vrai que tout incirconcis qu’il est, il sera considéré comme circoncis ?  \milestone{27}  Et ainsi celui qui étant naturellement incirconcis accomplit la loi, vous condamnera, vous qui ayant reçu la lettre de la loi, et étant circoncis, êtes un violateur de la loi.  \milestone{28}  Car le vrai Juif n’est pas celui qui l’est au dehors ; et la véritable circoncision n’est pas celle qui se fait dans la chair, et qui n’est qu’extérieure.  \milestone{29}  Mais le vrai Juif est celui qui l’est intérieurement ; et la circoncision véritable est celle du cœur, qui se fait par l’esprit, et non selon la lettre ; et ce vrai Juif tire sa louange, non des hommes, mais de Dieu.

\section[{Rom 3}]{Rom 3}

\noindent \initial{Q}{UEL} est donc l’avantage des Juifs, et quelle est l’utilité de la circoncision ?\par
  \milestone{2}  Leur avantage est grand en toutes manières, principalement en ce que les oracles de Dieu leur ont été confiés.\par
  \milestone{3}  Car enfin, si quelques-uns d’entre eux n’ont pas cru, leur infidélité anéantira-t-elle la fidélité de Dieu ? Non, certes.\par
  \milestone{4}  Dieu est véritable, et tout homme est menteur, selon ce que David dit à Dieu : Afin que vous soyez reconnu fidèle en vos paroles, et victorieux dans les jugements que les hommes feront de vous.\par
  \milestone{5}  Si notre injustice fait paraître davantage la justice de Dieu, que dirons-nous ? Dieu (pour parler selon l’homme) est-il injuste de nous punir ?\par
  \milestone{6}  Non, certes ; car si cela était, comment Dieu serait-il le juge du monde ?\par
  \milestone{7}  Mais, dira-t-on, si par mon infidélité la fidélité de Dieu a éclaté davantage pour sa gloire, pourquoi me condamne-t-on encore comme pécheur ?\par
  \milestone{8}  Et pourquoi ne ferons-nous pas le mal, afin qu’il en arrive du bien (selon que quelques-uns, pour nous noircir, nous accusent de dire) ? Ces personnes seront justement condamnées.\par
\bigbreak
\noindent   \milestone{9}  Dirons-nous donc que nous sommes préférables aux gentils ? Nullement ; car nous avons déjà convaincu et les Juifs et les gentils, d’être tous dans le péché ;  \milestone{10}  selon qu’il est écrit ; Il n’y a point de juste, il n’y en a pas un seul.\par
  \milestone{11}  Il n’y a point d’homme qui ait de l’intelligence ; il n’y en a point qui cherche Dieu.\par
  \milestone{12}  Ils se sont tous détournés du droit chemin ; ils sont tous devenus inutiles ; il n’y en a point qui fasse le bien, il n’y en a pas un seul.\par
  \milestone{13}  Leur gosier est un sépulcre ouvert ; ils se sont servis de leurs langues pour tromper avec adresse ; ils ont sous leurs lèvres un venin d’aspic.\par
  \milestone{14}  Leur bouche est remplie de malédiction et d’amertume ;\par
  \milestone{15}  leurs pieds sont vites pour répandre le sang ;\par
  \milestone{16}  leur conduite ne tend qu’à opprimer les autres, et à les rendre malheureux.\par
  \milestone{17}  Ils ne connaissent point la voie de la paix ;\par
  \milestone{18}  ils n’ont point la crainte de Dieu devant les yeux.\par
  \milestone{19}  Or nous savons que toutes les paroles de la loi s’adressent à ceux qui sont sous la loi, afin que toute bouche soit fermée, et que tout le monde se reconnaisse condamnable devant Dieu ;  \milestone{20}  parce que nul homme ne sera justifié devant Dieu par les œuvres de la loi ; car la loi ne donne que la connaissance du péché.\par
\bigbreak
\noindent   \milestone{21}  Mais maintenant, sans la loi, la justice qui vient de Dieu, à laquelle la loi et les prophètes rendent témoignage, a été manifestée ;  \milestone{22}  cette justice qui vient de Dieu par la foi en Jésus-Christ, et qui est répandue en tous ceux et sur tous ceux qui croient en lui ; car il n’y a point de distinction ;  \milestone{23}  parce que tous ont péché, et ont besoin de rendre gloire à Dieu ;  \milestone{24}  étant justifiés gratuitement par sa grâce, par la rédemption qui est en Jésus-Christ,  \milestone{25}  que Dieu a proposé pour être la victime de propitiation, par la foi qu’on aurait en son sang, pour faire paraître la justice qu’il donne lui-même, en pardonnant les péchés passés,  \milestone{26}  qu’il a soufferts avec tant de patience ; pour faire, dis-je, paraître en ce temps la justice qui vient de lui ; montrant tout ensemble qu’il est juste, et qu’il justifie celui qui a la foi en Jésus-Christ.\par
  \milestone{27}  Où est donc le sujet de votre gloire ? Il est exclu. Et par quelle loi ? Est-ce par la loi des œuvres ? Non ; mais par la loi de la foi.  \milestone{28}  Car nous devons reconnaître que l’homme est justifié par la foi, sans les œuvres de la loi.\par
  \milestone{29}  Dieu n’est-il le Dieu que des Juifs ? Ne l’est-il pas aussi des gentils ? Oui, certes, il l’est aussi des gentils.  \milestone{30}  Car il n’y a qu’un seul Dieu, qui justifie par la foi les circoncis, et qui par la foi justifie aussi les incirconcis.\par
  \milestone{31}  Détruisons-nous donc la loi par la foi ? À Dieu ne plaise ! mais au contraire nous l’établissons.

\section[{Rom 4}]{Rom 4}

\noindent \initial{Q}{UEL} avantage dirons-nous donc qu’Abraham, notre père, a eu selon la chair ?  \milestone{2}  Certes, si Abraham a été justifié par ses œuvres, il a de quoi se glorifier, mais non devant Dieu.  \milestone{3}  Et cependant que dit l’Écriture ? Abraham crut à la parole de Dieu, et sa foi lui fut imputée a justice.  \milestone{4}  Or la récompense qui se donne à quelqu’un pour ses œuvres, ne lui est pas imputée comme une grâce, mais comme une dette.  \milestone{5}  Et au contraire, lorsqu’un homme, sans faire des œuvres, croit en celui qui justifie le pécheur, sa foi lui est imputée à justice, selon le décret de la grâce de Dieu.  \milestone{6}  C’est ainsi que David dit, qu’un homme est heureux à qui Dieu impute la justice sans les œuvres.\par
  \milestone{7}  Heureux ceux à qui les iniquités sont pardonnées, et dont les péchés sont couverts !\par
  \milestone{8}  Heureux celui à qui Dieu n’a point imputé de péché !\par
  \milestone{9}  Or ce bonheur n’est-il que pour les circoncis ? N’est-il point aussi pour les incirconcis ? Car nous venons de dire que la foi d’Abraham lui fut imputée à justice.  \milestone{10}  Quand donc lui a-t-elle été imputée ? Est-ce après qu’il a été circoncis, ou lorsqu’il était incirconcis ? Ce n’a point été après qu’il eut reçu la circoncision, mais avant qu’il l’eût reçue.  \milestone{11}  Et ainsi il reçut la marque de la circoncision, comme le sceau de la justice qu’il avait eue par la foi, lorsqu’il était encore incirconcis : pour être et le père de tous ceux qui croient n’étant point circoncis, afin que leur foi leur soit aussi imputée à justice ;  \milestone{12}  et le père des circoncis, qui non-seulement ont reçu la circoncision, mais qui suivent aussi les traces de la foi qu’eut notre père Abraham, lorsqu’il était encore incirconcis.\par
  \milestone{13}  Aussi n’est-ce point par la loi que la promesse a été faite à Abraham, ou à sa postérité, d’avoir tout le monde pour héritage, mais par la justice de la foi.  \milestone{14}  Car si ceux qui appartiennent à la loi sont les héritiers, la foi devient inutile, et la promesse de Dieu sans effet ;  \milestone{15}  parce que la loi produit la colère et le châtiment ; puisque lorsqu’il n’y a point de loi, il n’y a point de violemment de la loi.\par
  \milestone{16}  Ainsi c’est par la foi que nous sommes héritiers, afin que nous le soyons par grâce, et que la promesse faite à Abraham demeure ferme pour tous les enfants d’Abraham, non-seulement pour ceux qui ont reçu la loi, mais encore pour ceux qui suivent la foi d’Abraham, qui est le père de nous tous,  \milestone{17}  (selon qu’il est écrit ; Je vous ai établi le père de plusieurs nations ;) et qui l’est devant Dieu, auquel il a cru comme à celui qui ranime les morts, et qui appelle ce qui n’est point, comme ce qui est.  \milestone{18}  Aussi ayant espéré contre toute espérance, il a cru qu’il deviendrait le père de plusieurs nations, selon qu’il lui avait été dit ; Votre postérité sera sans nombre.  \milestone{19}  Il ne s’affaiblit point dans sa foi, et il ne considéra point qu’étant âgé de cent ans, son corps était déjà comme mort, et que la vertu de concevoir était éteinte dans celui de Sara.  \milestone{20}  Il n’hésita point, et il n’eut pas la moindre défiance de la promesse de Dieu ; mais il se fortifia par la foi, rendant gloire à Dieu,  \milestone{21}  pleinement persuadé qu’il est tout-puissant pour faire tout ce qu’il a promis.  \milestone{22}  C’est pour cette raison que sa foi lui a été imputée à justice.  \milestone{23}  Or ce n’est pas pour lui seul qu’il est écrit, que sa foi lui a été imputée à justice ;  \milestone{24}  mais aussi pour nous, à qui elle sera imputée de même, si nous croyons en celui qui a ressuscité d’entre les morts Jésus-Christ notre Seigneur,  \milestone{25}  qui a été livré à la mort pour nos péchés, et qui est ressuscité pour notre justification.

\section[{Rom 5}]{Rom 5}

\noindent \initial{A}{INSI} étant justifiés par la foi, ayons la paix avec Dieu, par Jésus-Christ notre Seigneur ;  \milestone{2}  qui nous a donné aussi entrée par la foi à cette grâce en laquelle nous demeurons fermes, et nous nous glorifions dans l’espérance de la gloire des enfants de Dieu ;  \milestone{3}  et non-seulement dans cette espérance, mais nous nous glorifions encore dans les afflictions : sachant que l’affliction produit la patience ;  \milestone{4}  la patience, l’épreuve ; et l’épreuve, l’espérance.  \milestone{5}  Or cette espérance n’est point trompeuse, parce que l’amour de Dieu a été répandu dans nos cœurs par le Saint-Esprit qui nous a été donné.\par
  \milestone{6}  Car pourquoi, lorsque nous étions encore dans les langueurs du péché, Jésus-Christ est-il mort pour des impies comme nous, dans le temps destiné de Dieu ?  \milestone{7}  Et certes, à peine quelqu’un voudrait-il mourir pour un juste ; peut-être néanmoins que quelqu’un aurait le courage de donner sa vie pour un homme de bien.  \milestone{8}  Mais ce qui fait éclater davantage l’amour de Dieu envers nous, c’est que lors même que nous étions encore pécheurs,  \milestone{9}  Jésus-Christ n’a pas laissé de mourir pour nous dans le temps destiné de Dieu. Ainsi étant maintenant justifiés par son sang, nous serons à plus forte raison délivrés par lui de la colère de Dieu.  \milestone{10}  Car si lorsque nous étions ennemis de Dieu, nous avons été réconciliés avec lui par la mort de son Fils ; à plus forte raison, étant maintenant réconciliés avec lui, nous serons sauvés par la vie de ce même Fils.  \milestone{11}  Et non-seulement nous avons été réconciliés, mais nous nous glorifions même en Dieu, par Jésus-Christ notre Seigneur, par qui nous avons obtenu maintenant cette réconciliation.\par
\bigbreak
\noindent   \milestone{12}  C’est pourquoi, comme le péché est entré dans le monde par un seul homme, et la mort par le péché ; et qu’ainsi la mort est passée dans tous les hommes, tous ayant péché dans un seul ;  \milestone{13}  (car le péché a toujours été dans le monde jusqu’à la loi ; mais la loi n’étant point encore, le péché n’était pas imputé.  \milestone{14}  cependant la mort a exercé son règne depuis Adam jusqu’à Moïse, à l’égard de ceux mêmes qui n’ont pas péché par une transgression semblable à celle d’Adam, qui est la figure d’un autre chef qui devait venir ;  \milestone{15}  mais il n’en est pas de la grâce comme du péché ; car si par le péché d’un seul plusieurs sont morts, la miséricorde et le don de Dieu s’est répandu à plus forte raison abondamment sur plusieurs par la grâce d’un seul homme, qui est Jésus-Christ ;  \milestone{16}  et il n’en est pas de ce don comme de ce seul péché ; car nous avons été condamnés par le jugement de Dieu pour un seul péché, au lieu que nous sommes justifiés par la grâce après plusieurs péchés ;  \milestone{17}  si donc, à cause du péché d’un seul, la mort a régné par un seul homme ; à plus forte raison ceux qui reçoivent l’abondance de la grâce et du don de la justice, régneront dans la vie par un seul homme, qui est Jésus-Christ :)\par
  \milestone{18}  comme donc c’est par le péché d’un seul, que tous les hommes sont tombés dans la condamnation ; ainsi c’est par la justice d’un seul, que tous les hommes reçoivent la justification qui donne la vie.  \milestone{19}  Car comme plusieurs sont devenus pécheurs par la désobéissance d’un seul, ainsi plusieurs seront rendus justes par l’obéissance d’un seul.\par
  \milestone{20}  Or la loi est survenue pour donner lieu à l’abondance du péché ; mais où il y a eu une abondance de péché, il y a eu ensuite une surabondance de grâce ;  \milestone{21}  afin que comme le péché avait régné en donnant la mort, la grâce de même règne par la justice en donnant la vie éternelle, par Jésus-Christ notre Seigneur.

\section[{Rom 6}]{Rom 6}

\noindent \initial{Q}{UE} dirons-nous donc ? Demeurerons-nous dans le péché, pour donner lieu à cette surabondance de grâce ?  \milestone{2}  À Dieu ne plaise ! Car étant une fois morts au péché, comment vivrons-nous encore dans le péché ?  \milestone{3}  Ne savez-vous pas que nous tous qui avons été baptisés en Jésus-Christ, nous avons été baptisés en sa mort ?  \milestone{4}  Car nous avons été ensevelis avec lui par le baptême pour mourir au péché ; afin que comme Jésus-Christ est ressuscité d’entre les morts par la gloire de son Père, nous marchions aussi dans une nouvelle vie.\par
  \milestone{5}  Car si nous avons été entés [\emph{unis}] en lui par la ressemblance de sa mort, nous y serons aussi entés [\emph{unis}] par la ressemblance de sa résurrection ;  \milestone{6}  sachant que notre vieil homme a été crucifié avec lui, afin que le corps du péché soit détruit, et que désormais nous ne soyons plus asservis au péché.  \milestone{7}  Car celui qui est mort, est délivré du péché.  \milestone{8}  Si donc nous sommes morts avec Jésus-Christ, nous croyons que nous vivrons aussi avec Jésus-Christ ;  \milestone{9}  parce que nous savons que Jésus-Christ étant ressuscité d’entre les morts ne mourra plus, et que la mort n’aura plus d’empire sur lui.  \milestone{10}  Car quant à ce qu’il est mort, il est mort seulement une fois pour le péché ; mais quant à la vie qu’il a maintenant, il vit pour Dieu.  \milestone{11}  Considérez-vous de même comme étant morts au péché, et comme ne vivant plus que pour Dieu, en Jésus-Christ notre Seigneur.\par
  \milestone{12}  Que le péché donc ne règne point dans votre corps mortel, en sorte que vous obéissiez a ses désirs déréglés.  \milestone{13}  Et n’abandonnez point au péché les membres de votre corps, pour lui servir d’armes d’iniquité ; mais donnez-vous à Dieu, comme devenus vivants de morts que vous étiez, et consacrez-lui les membres de votre corps, pour lui servir d’armes de justice.  \milestone{14}  Car le péché ne vous dominera plus, parce que vous n’êtes plus sous la loi, mais sous la grâce.\par
\bigbreak
\noindent   \milestone{15}  Quoi donc ! pécherons-nous parce que nous ne sommes plus sous la loi, mais sous la grâce ? Dieu nous en garde !  \milestone{16}  Ne savez-vous pas que de qui que ce soit que vous vous soyez rendus esclaves pour lui obéir, vous demeurez esclaves de celui à qui vous obéissez, soit du péché pour y trouver la mort, ou de l’obéissance à la foi pour y trouver la justice ?  \milestone{17}  Mais Dieu soit loué de ce qu’ayant été auparavant esclaves du péché, vous avez obéi du fond du cœur à la doctrine de l’Évangile, sur le modèle de laquelle vous avez été formés.  \milestone{18}  Ainsi ayant été affranchis du péché, vous êtes devenus esclaves de la justice.  \milestone{19}  Je vous parle humainement, à cause de la faiblesse de votre chair. Comme vous avez fait servir les membres de votre corps à l’impureté et à l’injustice, pour commettre l’iniquité, faites-les servir maintenant à la justice, pour votre sanctification.\par
  \milestone{20}  Car lorsque vous étiez esclaves du péché, vous étiez libres à l’égard de la justice.  \milestone{21}  Quel fruit tiriez-vous donc alors de ces désordres, dont vous rougissez maintenant ? puisqu’ils n’ont pour fin que la mort.  \milestone{22}  Mais à présent, étant affranchis du péché, et devenus esclaves de Dieu, votre sanctification est le fruit que vous en tirez, et la vie éternelle en sera la fin.  \milestone{23}  Car la mort est la solde et le payement du péché ; mais la vie éternelle est une grâce et un don de Dieu, en Jésus-Christ notre Seigneur.

\section[{Rom 7}]{Rom 7}

\noindent \initial{I}{GNOREZ-VOUS}, mes frères (car je parle à des hommes instruits de la loi), que la loi ne domine sur l’homme que pour autant de temps qu’elle vit ?  \milestone{2}  Ainsi une femme mariée est liée à son mari par la loi du mariage, tant qu’il est vivant ; mais lorsqu’il est mort, elle est dégagée de la loi qui la liait à son mari.  \milestone{3}  Si donc elle épouse un autre homme pendant la vie de son mari, elle sera tenue pour adultère ; mais si son mari vient a mourir, elle est affranchie de cette loi, et elle peut en épouser un autre sans être adultère.  \milestone{4}  Ainsi, mes frères, vous êtes vous-mêmes morts à la loi par le corps de Jésus-Christ, pour être a un autre qui est ressuscité d’entre les morts, afin que nous produisions des fruits pour Dieu.  \milestone{5}  Car lorsque nous étions dans la chair, les passions criminelles étant excitées par la loi, agissaient dans les membres de notre corps, et leur faisaient produire des fruits pour la mort.  \milestone{6}  Mais maintenant nous sommes affranchis de la loi de mort, dans laquelle nous étions retenus ; de sorte que nous servons Dieu dans la nouveauté de l’esprit, et non dans la vieillesse de la lettre.\par
\bigbreak
\noindent   \milestone{7}  Que dirons-nous donc ? La loi est-elle péché ? Dieu nous garde d’une telle pensée ! Mais je n’ai connu le péché que par la loi ; car je n’aurais point connu la concupiscence, si la loi n’avait dit ; Vous n’aurez point de mauvais désirs.  \milestone{8}  Mais le péché ayant pris occasion de s’irriter du commandement, a produit en moi toutes sortes de mauvais désirs ; car sans la loi le péché était comme mort.  \milestone{9}  Et pour moi, je vivais autrefois sans loi ; mais le commandement étant survenu, le péché est ressuscité,  \milestone{10}  et moi, je suis mort ; et il s’est trouvé que le commandement qui devait servir à me donner la vie, a servi à me donner la mort.  \milestone{11}  Car le péché ayant pris occasion du commandement, m’a trompé, et m’a tué par le commandement même.\par
  \milestone{12}  Ainsi la loi est sainte à la vérité, et le commandement est saint, juste et bon.  \milestone{13}  Ce qui était bon en soi, m’a-t-il donc causé la mort ? Nullement ; mais c’est le péché et la concupiscence, qui en se manifestant m’a causé la mort par une chose qui était bonne ; le péché, c’est-à-dire, la concupiscence, devenant ainsi par le commandement même une source plus abondante de péché.\par
  \milestone{14}  Car nous savons que la loi est spirituelle ; mais pour moi, je suis charnel, étant vendu pour être assujetti au péché.\par
\bigbreak
\noindent   \milestone{15}  Je n’approuve pas ce que je fais, parce que je ne fais pas le bien que je veux ; mais je fais le mal que je hais.  \milestone{16}  Si je fais ce que je ne veux pas, je consens à la loi, et je reconnais qu’elle est bonne.  \milestone{17}  Ainsi ce n’est plus moi qui fais cela ; mais c’est le péché qui habite en moi.\par
  \milestone{18}  Car je sais qu’il n’y a rien de bon en moi, c’est-à-dire, dans ma chair ; parce que je trouve en moi la volonté de faire le bien ; mais je ne trouve point le moyen de l’accomplir.  \milestone{19}  Car je ne fais pas le bien que je veux ; mais je fais le mal que je ne veux pas.  \milestone{20}  Si je fais ce que je ne veux pas, ce n’est plus moi qui le fais ; mais c’est le péché qui habite en moi.\par
  \milestone{21}  Lors donc que je veux faire le bien, je trouve en moi une loi qui s’y oppose, parce que le mal réside en moi.  \milestone{22}  Car je me plais dans la loi de Dieu selon l’homme intérieur ;  \milestone{23}  mais je sens dans les membres de mon corps une autre loi qui combat contre la loi de mon esprit, et qui me rend captif sous la loi du péché, qui est dans les membres de mon corps.  \milestone{24}  Malheureux homme que je suis ! qui me délivrera de ce corps de mort ?  \milestone{25}  Ce sera la grâce de Dieu, par Jésus-Christ notre Seigneur. Et ainsi je suis moi-même soumis et à la loi de Dieu selon l’esprit, et à la loi du péché selon la chair.

\section[{Rom 8}]{Rom 8}

\noindent \initial{I}{L} n’y a donc point maintenant de condamnation pour ceux qui sont en Jesus-Christ, et qui ne marchent point selon la chair ;  \milestone{2}  parce que la loi de l’Esprit de vie, qui est en Jésus-Christ, m’a délivré de la loi de péché et de mort.  \milestone{3}  Car ce qu’il était impossible que la loi fît, la chair la rendant faible et impuissante, Dieu l’a fait, ayant envoyé son propre Fils revêtu d’une chair semblable à la chair de péché ; et à cause du péché il a condamné le péché dans la chair ;  \milestone{4}  afin que la justice de la loi soit accomplie en nous, qui ne marchons pas selon la chair, mais selon l’esprit.  \milestone{5}  Car ceux qui sont charnels, aiment et goûtent les choses de la chair ; et ceux qui sont spirituels, aiment et goûtent les choses de l’esprit.  \milestone{6}  Or cet amour des choses de la chair est une mort, au lieu que l’amour des choses de l’esprit est la vie et la paix.  \milestone{7}  Car cet amour des choses de la chair est ennemi de Dieu, parce qu’il n’est point soumis à la loi de Dieu, et ne le peut être.  \milestone{8}  Ceux donc qui vivent selon la chair, ne peuvent plaire à Dieu.  \milestone{9}  Mais pour vous, vous ne vivez pas selon la chair, mais selon l’esprit ; si toutefois l’Esprit de Dieu habite en vous ; car si quelqu’un n’a point l’Esprit de Jésus-Christ, il n’est point à lui.  \milestone{10}  Mais si Jésus-Christ est en vous, quoique le corps soit mort en vous à cause du péché, l’esprit est vivant à cause de la justice.  \milestone{11}  Si donc l’Esprit de celui qui a ressuscité Jésus d’entre les morts habite en vous, celui qui a ressuscité Jésus-Christ d’entre les morts, donnera aussi la vie à vos corps mortels par son Esprit qui habite en vous.\par
  \milestone{12}  Ainsi, mes frères, nous ne sommes point redevables à la chair, pour vivre selon la chair.  \milestone{13}  Si vous vivez selon la chair, vous mourrez ; mais si vous faites mourir par l’esprit les œuvres de la chair, vous vivrez.  \milestone{14}  Car tous ceux qui sont poussés par l’Esprit de Dieu, sont enfants de Dieu.  \milestone{15}  Aussi vous n’avez point reçu l’esprit de servitude, pour vous conduire encore par la crainte ; mais vous avez reçu l’Esprit de l’adoption des enfants par lequel nous crions ; Mon Père ! mon Père !  \milestone{16}  Et c’est cet Esprit qui rend lui-même témoignage à notre esprit, que nous sommes enfants de Dieu.  \milestone{17}  Si nous sommes enfants, nous sommes aussi héritiers ; héritiers de Dieu, et cohéritiers de Jésus-Christ ; pourvu toutefois que nous souffrions avec lui, afin que nous soyons glorifiés avec lui.\par
\bigbreak
\noindent   \milestone{18}  Car je suis persuadé que les souffrances de la vie présente n’ont point de proportion avec cette gloire qui sera un jour découverte en nous.  \milestone{19}  Aussi les créatures attendent avec grand désir la manifestation des enfants de Dieu ;  \milestone{20}  parce qu’elles sont assujetties à la vanité, et elles ne le sont pas volontairement, mais à cause de celui qui les y a assujetties ;  \milestone{21}  avec espérance d’être délivrées aussi elles-mêmes de cet asservissement à la corruption, pour participer à la glorieuse liberté des enfants de Dieu.\par
  \milestone{22}  Car nous savons que jusqu’à maintenant toutes les créatures soupirent, et sont comme dans le travail de l’enfantement ;  \milestone{23}  et non-seulement elles, mais nous encore qui possédons les prémices de l’Esprit, nous soupirons et nous gémissons en nous-mêmes, attendant l’effet de l’adoption divine, la rédemption et la délivrance de nos corps.  \milestone{24}  Car ce n’est encore qu’en espérance que nous sommes sauvés. Or quand on voit ce qu’on a espéré, ce n’est plus espérance, puisque nul n’espère ce qu’il voit déjà.  \milestone{25}  Mais si nous espérons ce que nous ne voyons pas encore, nous l’attendons avec patience.  \milestone{26}  De plus, l’Esprit de Dieu nous aide dans notre faiblesse. Car nous ne savons ce que nous devons demander à Dieu dans nos prières, pour le prier comme il faut ; mais le Saint-Esprit même prie pour nous par des gémissements ineffables.  \milestone{27}  Et celui qui pénètre le fond des cœurs, entend bien quel est le désir de l’esprit, parce qu’il ne demande pour les saints que ce qui est conforme à la volonté de Dieu.\par
  \milestone{28}  Or nous savons que tout contribue au bien de ceux qui aiment Dieu, de ceux qu’il a appelés selon son décret pour être saints.  \milestone{29}  Car ceux qu’il a connus dans sa prescience, il les a aussi prédestinés pour être conformes à l’image de son Fils, afin qu’il fût l’aîné entre plusieurs frères.  \milestone{30}  Et ceux qu’il a prédestinés, il les a aussi appelés ; et ceux qu’il a appelés, il les a aussi justifiés ; et ceux qu’il a justifiés, il les a aussi glorifiés.\par
\bigbreak
\noindent   \milestone{31}  Après cela que devons-nous dire ? Si Dieu est pour nous, qui sera contre nous ?  \milestone{32}  Lui qui n’a pas épargné son propre Fils, mais qui l’a livré à la mort pour nous tous, que ne nous donnera-t-il point, après nous l’avoir donné ?  \milestone{33}  Qui accusera les élus de Dieu ? Sera-ce Dieu, lui qui les justifie.  \milestone{34}  Qui osera les condamner ? Sera-ce Jésus-Christ, lui qui est mort pour nous, qui de plus est ressuscité ; qui est à la droite de Dieu, et qui intercède pour nous ?  \milestone{35}  Qui donc nous séparera de l’amour de Jésus-Christ ? Sera-ce l’affliction, ou les déplaisirs, ou la persécution, ou la faim, ou la nudité, ou les périls, ou le fer et la violence ?  \milestone{36}  selon qu’il est écrit ; On nous égorge tous les jours pour l’amour de vous, Seigneur ! on nous regarde comme des brebis destinées à la boucherie.  \milestone{37}  Mais parmi tous ces maux, nous demeurons victorieux par celui qui nous a aimés.  \milestone{38}  Car je suis assuré que ni la mort, ni la vie, ni les anges, ni les principautés, ni les puissances, ni les choses présentes, ni les futures, ni la puissance des hommes,  \milestone{39}  ni tout ce qu’il y a de plus haut, ou de plus profond, ni toute autre créature, ne pourra jamais nous séparer de l’amour de Dieu, en Jésus-Christ notre Seigneur.

\section[{Rom 9}]{Rom 9}

\noindent \initial{J}{ÉSUS-CHRIST} m’est témoin que je dis la vérité ; je ne mens point, ma conscience me rendant ce témoignage par le Saint-Esprit,  \milestone{2}  que je suis saisi d’une tristesse profonde, et que mon cœur est pressé sans cesse d’une vive douleur ;  \milestone{3}  jusque-là que j’eusse désiré que Jésus-Christ m’eût fait servir moi-même de victime soumise à l’anathème pour mes frères, qui sont d’un même sang que moi selon la chair ;  \milestone{4}  qui sont les Israélites, à qui appartient l’adoption des enfants de Dieu, sa gloire, son alliance, sa loi, son culte et ses promesses ;  \milestone{5}  de qui les patriarches sont les pères, et desquels est sorti selon la chair Jésus-Christ même, qui est Dieu au-dessus de tout, et béni dans tous les siècles. Amen !\par
  \milestone{6}  Ce n’est pas néanmoins que la parole de Dieu soit demeurée sans effet. Car tous ceux qui descendent d’Israël, ne sont pas pour cela Israélites ;  \milestone{7}  et tous ceux qui sont de la race d’Abraham, ne sont pas pour cela ses enfants ; mais Dieu lui dit : C’est d’Isaac que sortira la race qui doit porter votre nom.  \milestone{8}  C’est-à-dire, que ceux qui sont enfants d’Abraham selon la chair, ne sont pas pour cela enfants de Dieu ; mais que ce sont les enfants de la promesse, qui sont réputés être les enfants d’Abraham.  \milestone{9}  Car voici les termes de la promesse : Je viendrai dans un an en ce même temps, et Sara aura un fils.  \milestone{10}  Et cela ne se voit pas seulement dans Sara, mais aussi dans Rebecca, qui conçut en même temps deux enfants d’Isaac, notre père.  \milestone{11}  Car avant qu’ils fussent nés, et avant qu’ils eussent fait aucun bien ni aucun mal, afin que le décret de Dieu demeurât ferme selon son élection,  \milestone{12}  non à cause de leurs œuvres, mais à cause de l’appel et du choix de Dieu, il lui fut dit,  \milestone{13}  L’aîné sera assujetti au plus jeune ; selon qu’il est écrit : J’ai aimé Jacob, et j’ai haï Esaü.\par
  \milestone{14}  Que dirons-nous donc ? Est-ce qu’il y a en Dieu de l’injustice ? Dieu nous garde de cette pensée !  \milestone{15}  Car il dit à Moïse : Je ferai miséricorde à qui il me plaira de faire miséricorde ; et j’aurai pitié de qui il me plaira d’avoir pitié.  \milestone{16}  Cela ne dépend donc ni de celui qui veut, ni de celui qui court ; mais de Dieu qui fait miséricorde.  \milestone{17}  Car dans l’Écriture, il dit à Pharaon : C’est pour cela même que je vous ai établi, pour faire éclater en vous ma puissance, et pour rendre mon nom célèbre dans toute la terre.  \milestone{18}  Il est donc vrai qu’il fait miséricorde à qui il lui plaît, et qu’il endurcit qui il lui plaît.\par
\bigbreak
\noindent   \milestone{19}  Vous me direz peut-être : Après cela pourquoi Dieu se plaint-il ? Car qui est-ce qui résiste à sa volonté ?  \milestone{20}  Mais, ô homme ! qui êtes-vous pour contester avec Dieu ? Un vase d’argile dit-il à celui qui l’a fait ? Pourquoi m’avez-vous fait ainsi ?  \milestone{21}  Le potier n’a-t-il pas le pouvoir de faire de la même masse d’argile un vase destiné a des usages honorables, et un autre destiné à des usages vils et honteux ?  \milestone{22}  Que dirons-nous donc, si Dieu voulant montrer sa juste colère, et faire connaître sa puissance, souffre avec une patience extrême les vases de colère préparés pour la perdition ;  \milestone{23}  afin de faire paraître les richesses de sa gloire sur les vases de miséricorde qu’il a préparés pour la gloire,  \milestone{24}  sur nous, qu’il a appelés non-seulement d’entre les Juifs, mais aussi d’entre les gentils ?  \milestone{25}  selon ce qu’il dit dans Osée : J’appellerai mon peuple, ceux qui n’étaient point mon peuple ; ma bien-aimée, celle que je n’avais point aimée ; et l’objet de ma miséricorde, celle à qui je n’avais point fait miséricorde ;  \milestone{26}  et il arrivera que dans le même lieu où je leur avais dit autrefois, Vous n’êtes point mon peuple ; ils seront appelés les enfants du Dieu vivant.  \milestone{27}  Et pour ce qui est d’Israël, Isaïe s’écrie : Quand le nombre des enfants d’Israël serait égal à celui du sable de la mer, il n’y en aura qu’un petit reste de sauvés.  \milestone{28}  Car Dieu dans sa justice consumera et retranchera son peuple ; le Seigneur fera un grand retranchement sur la terre.  \milestone{29}  Et comme le même Isaïe avait dit auparavant : Si le Seigneur des armées ne nous avait réservé quelques-uns de notre race, nous serions devenus semblables à Sodome et à Gomorrhe.\par
  \milestone{30}  Que dirons-nous donc à cela ? sinon que les gentils qui ne cherchaient point la justice, ont embrassé la justice, et la justice qui vient de la foi ;  \milestone{31}  et que les Israélites au contraire, qui recherchaient la loi de la justice, ne sont point parvenus à la loi de la justice.  \milestone{32}  Et pourquoi ? Parce qu’ils ne l’ont point recherchée par la foi, mais comme par les œuvres de la loi. Car ils se sont heurtés contre la pierre d’achoppement,  \milestone{33}  selon qu’il est écrit : Je vais mettre dans Sion celui qui est une pierre d’achoppement, une pierre de scandale ; et tous ceux qui croiront en lui, ne seront point confondus.

\section[{Rom 10}]{Rom 10}

\noindent \initial{I}{L} est vrai, mes frères, que je sens dans mon cœur une grande affection pour le salut d’Israël, et je le demande à Dieu par mes prières.  \milestone{2}  Car je puis leur rendre ce témoignage, qu’ils ont du zèle pour Dieu ; mais leur zèle n’est point selon la science ;  \milestone{3}  parce que ne connaissant point la justice qui vient de Dieu, et s’efforçant d établir leur propre justice, ils né se sont point soumis à Dieu, pour recevoir cette justice qui vient de lui.  \milestone{4}  Car Jésus-Christ est la fin de la loi, pour justifier tous ceux qui croient en lui.\par
  \milestone{5}  Or Moïse dit touchant la justice qui vient de la loi, que celui qui en observera les ordonnances, y trouvera la vie.  \milestone{6}  Mais pour ce qui est de la justice qui vient de la foi, voici comme il en parle : Ne dites point en votre cœur ; Qui pourra monter au ciel ? c’est-à-dire, pour en faire descendre Jésus-Christ ;  \milestone{7}  ou, Qui pourra descendre au fond de la terre ? c’est-à-dire, pour appeler Jésus-Christ d’entre les morts.  \milestone{8}  Mais que dit l’Écriture ? La parole qui vous est annoncée, n’est point éloignée de vous ; elle est dans votre bouche et dans votre cœur. Telle est la parole de la foi que nous vous prêchons ;  \milestone{9}  parce que si vous confessez de bouche que Jésus est le Seigneur, et si vous croyez de cœur que Dieu l’a ressuscité d’entre les morts, vous serez sauvé.  \milestone{10}  Car il faut croire de cœur pour être justifié, et confesser sa foi par ses paroles pour être sauvé.  \milestone{11}  C’est pourquoi l’Écriture dit : Tous ceux qui croient en lui, ne seront point confondus.  \milestone{12}  Il n’y a point en cela de distinction entre les Juifs et les gentils ; parce qu’ils n’ont tous qu’un même Seigneur, qui répand ses richesses sur tous ceux qui l’invoquent.  \milestone{13}  Car tous ceux qui invoqueront le nom du Seigneur, seront sauvés.\par
  \milestone{14}  Mais comment l’invoqueront-ils, s’ils ne croient point en lui ? Et comment croiront-ils en lui, s’ils n’en ont point entendu parler ? Et comment en entendront-ils parler, si personne ne le leur prêche ?  \milestone{15}  Et comment les prédicateurs leur prêcheront-ils, s’ils ne sont envoyés ? selon ce qui est écrit : Combien sont beaux les pieds de ceux qui annoncent l’Évangile de paix, de ceux qui annoncent les vrais biens !  \milestone{16}  Mais tous n’obéissent pas à l’Évangile. C’est ce qui a fait dire à Isaïe : Seigneur ! qui a cru ce qu’il nous a entendu prêcher ?  \milestone{17}  La foi donc vient de ce qu’on a entendu ; et on a entendu, parce que la parole de Jésus-Christ a été prêchée.  \milestone{18}  Mais je demande ; Ne l’ont-ils pas déjà entendue ? Oui, certes ; leur voix a retenti par toute la terre, et leur parole s’est fait entendre jusqu’aux extrémités du monde.  \milestone{19}  Et Israël n’en a-t-il point eu aussi connaissance ? C’est Moïse qui le premier a dit : Je vous rendrai jaloux d’un peuple qui n’est pas un peuple, et je ferai qu’une nation insensée deviendra l’objet de votre indignation et de votre envie.  \milestone{20}  Mais Isaïe dit hautement ; J’ai été trouvé par ceux qui ne me cherchaient pas ; et je me suis fait voir à ceux qui ne demandaient point à me connaître.  \milestone{21}  Et il dit contre Israël : J’ai tendu les bras durant tout le jour à ce peuple incrédule et rebelle à mes paroles.

\section[{Rom 11}]{Rom 11}

\noindent \initial{Q}{UE} dirai-je donc ? Est-ce que Dieu a rejeté son peuple ? Non, certes. Car je suis moi-même Israélite, de la race d’Abraham, et de la tribu de Benjamin.  \milestone{2}  Dieu n’a point rejeté son peuple qu’il a connu dans sa prescience. Ne savez-vous pas ce qui est rapporté d’Élie dans l’Écriture ? de quelle sorte il demande justice à Dieu contre Israël, en disant :  \milestone{3}  Seigneur ! ils ont tué vos prophètes, ils ont renversé vos autels ; je suis demeuré tout seul, et ils me cherchent pour m’ôter la vie.  \milestone{4}  Mais qu’est-ce que Dieu lui répond ? Je me suis réservé sept mille hommes qui n’ont point fléchi le genou devant Baal.  \milestone{5}  Ainsi Dieu a sauvé en ce temps, selon l’élection de sa grâce, un petit nombre qu’il s’est réservé.  \milestone{6}  Si c’est par grâce, ce n’est donc point par les œuvres ; autrement la grâce ne serait plus grâce.  \milestone{7}  Après cela que dirons-nous ? Israël n’a-t-il donc point trouvé ce qu’il cherchait ? Ceux qui ont été choisis de Dieu, l’ont trouvé ; mais les autres ont été aveuglés ;  \milestone{8}  selon qu’il est écrit ; Dieu leur a donné un esprit d’assoupissement et d’insensibilité, des yeux qui ne voient point, et des oreilles qui n’entendent point ; tel est leur état jusqu’à ce jour.  \milestone{9}  David dit encore d’eux ; Que leur table leur soit un filet où ils se trouvent enveloppés ; qu’elle leur devienne une pierre de scandale, et qu’elle soit leur juste punition ;  \milestone{10}  que leurs yeux soient tellement obscurcis qu’ils ne voient point ; et faites qu’ils soient toujours courbés contre terre.\par
  \milestone{11}  Je demande donc : Ne se sont-ils heurtés que pour tomber et périr sans ressource ? À Dieu ne plaise ! Mais leur chute est devenue une occasion de salut aux gentils, afin que l’exemple des gentils leur donnât de l’émulation pour les suivre.  \milestone{12}  Si leur chute a été la richesse du monde, et si le petit nombre auquel ils ont été réduits a été la richesse des gentils, combien leur plénitude enrichira-t-elle le monde encore davantage ?\par
  \milestone{13}  Car je vous le dis, à vous qui êtes gentils : tant que je serai l’apôtre des gentils, je travaillerai à rendre illustre mon ministère,  \milestone{14}  pour tâcher d’exciter de l’émulation dans l’esprit des Juifs qui me sont unis selon la chair, et d’en sauver quelques-uns.  \milestone{15}  Car si leur réprobation est devenue la réconciliation du monde, que sera leur rappel, sinon un retour de la mort à la vie ?\par
  \milestone{16}  Si les prémices des Juifs sont saintes, la masse l’est aussi ; et si la racine est sainte, les rameaux le sont aussi.  \milestone{17}  Si donc quelques-unes des branches ont été rompues ; et si vous, qui n’étiez qu’un olivier sauvage, avez été enté parmi celles qui sont demeurées sur l’olivier franc, et avez été rendu participant de la sève et du suc qui sort de la racine de l’olivier ;  \milestone{18}  ne vous élevez point de présomption contre les branches naturelles. Si vous pensez vous élever au-dessus d’elles, considérez que ce n’est pas vous qui portez la racine, mais que c’est la racine qui vous porte.  \milestone{19}  Mais, direz-vous, ces branches naturelles ont été rompues, afin que je fusse enté en leur place.  \milestone{20}  Il est vrai : elles ont été rompues à cause de leur incrédulité ; et pour vous, vous demeurez ferme par votre foi ; mais prenez garde de ne pas vous élever, et tenez-vous dans la crainte.  \milestone{21}  Car si Dieu n’a point épargné les branches naturelles, vous devez craindre qu’il ne vous épargne pas non plus.  \milestone{22}  Considérez donc la bonté et la sévérité de Dieu : sa sévérité envers ceux qui sont tombés ; et sa bonté envers vous, si toutefois vous demeurez ferme dans l’état où sa bonté vous a mis ; autrement vous serez aussi vous-même retranché comme eux.  \milestone{23}  Eux, au contraire, s’ils ne demeurent pas dans leur incrédulité, ils seront de nouveau entés sur leur tige, puisque Dieu est tout-puissant pour les enter encore.  \milestone{24}  Car si vous avez été coupé de l’olivier sauvage, qui était votre tige naturelle, pour être enté contre votre nature sur l’olivier franc ; à combien plus forte raison les branches naturelles de l’olivier même, seront-elles entées sur leur propre tronc ?\par
\bigbreak
\noindent   \milestone{25}  Car je ne veux pas, mes frères, que vous ignoriez ce mystère, afin que vous ne soyez point sages à vos propres yeux ; qui est, qu’une partie des Juifs est tombée dans l’aveuglement, jusqu’à ce que la multitude des nations soit entrée dans l’Église ;  \milestone{26}  et qu’ainsi tout Israël soit sauvé, selon qu’il est écrit ; Il sortira de Sion un libérateur qui bannira l’impiété de Jacob.  \milestone{27}  Et c’est là l’alliance que je ferai avec eux, lorsque j’effacerai leurs péchés.  \milestone{28}  Ainsi quant à l’Évangile, ils sont maintenant ennemis à cause de vous ; mais quant à l’élection, ils sont aimés à cause de leurs pères.  \milestone{29}  Car les dons et la vocation de Dieu sont immuables, et il ne s’en repent point.  \milestone{30}  Comme donc autrefois vous étiez incrédules à l’égard de Dieu, et que vous avez maintenant obtenu miséricorde, à l’occasion de l’incrédulité des Juifs ;  \milestone{31}  ainsi les Juifs sont maintenant tombés dans une incrédulité qui a donné lieu à la miséricorde que vous avez reçue, afin qu’un jour ils obtiennent eux-mêmes miséricorde.  \milestone{32}  Car Dieu a permis que tous fussent enveloppés dans l’incrédulité, pour exercer sa miséricorde envers tous.\par
  \milestone{33}  Ô profondeur des trésors de la sagesse et de la science de Dieu ! Que ses jugements sont incompréhensibles, et ses voies impénétrables !  \milestone{34}  Car qui a connu les desseins de Dieu, ou qui est entré dans le secret de ses conseils ?  \milestone{35}  ou qui lui a donné quelque chose le premier, pour en prétendre récompense ?  \milestone{36}  Car tout est de lui, tout est par lui, et tout est en lui ; à lui soit gloire dans tous les siècles ! Amen !

\section[{Rom 12}]{Rom 12}

\noindent \initial{J}{E} vous conjure donc, mes frères, par la miséricorde de Dieu, de lui offrir vos corps comme une hostie vivante, sainte et agréable à ses yeux, pour lui rendre un culte raisonnable et spirituel.  \milestone{2}  Ne vous conformez point au siècle présent ; mais qu’il se fasse en vous une transformation par le renouvellement de votre esprit, afin que vous reconnaissiez quelle est la volonté de Dieu, ce qui est bon, ce qui est agréable à ses yeux, et ce qui est parfait.\par
  \milestone{3}  Je vous exhorte donc, vous tous, selon le ministère qui m’a été donné par grâce, de ne vous point élever au delà de ce que vous devez, dans les sentiments que vous avez de vous-mêmes ; mais de vous tenir dans les bornes de la modération, selon la mesure du don de la foi que Dieu a départie à chacun de vous.  \milestone{4}  Car comme dans un seul corps nous avons plusieurs membres, et que tous ces membres n’ont pas la même fonction ;  \milestone{5}  de même en Jésus-Christ, quoique nous soyons plusieurs, nous ne sommes néanmoins qu’un seul corps, étant tous réciproquement membres les uns des autres.  \milestone{6}  C’est pourquoi, comme nous avons tous des dons différents selon la grâce qui nous a été donnée ; que celui qui a reçu le don de prophétie, en use selon l’analogie et la règle de la foi ;  \milestone{7}  que celui qui est appelé au ministère de l’Église, s’attache à son ministère ; que celui qui a reçu le don d’enseigner, s’applique à enseigner ;  \milestone{8}  et que celui qui a reçu le don d’exhorter, exhorte les autres ; que celui qui fait l’aumône, la fasse avec simplicité ; que celui qui a la conduite de ses frères, s’en acquitte avec vigilance ; et que celui qui exerce les œuvres de miséricorde, le fasse avec joie.  \milestone{9}  Que votre charité soit sincère, et sans déguisement. Ayez le mal en horreur, et attachez-vous fortement au bien.  \milestone{10}  Que chacun ait pour son prochain une affection et une tendresse vraiment fraternelle ; prévenez-vous les uns les autres par des témoignages d’honneur et de déférence.  \milestone{11}  Ne soyez point lâches dans votre devoir ; conservez-vous dans la ferveur de l’esprit ; souvenez-vous que c’est le Seigneur que vous servez.  \milestone{12}  Réjouissez-vous dans l’espérance ; soyez patients dans les maux, persévérants dans la prière,  \milestone{13}  charitables pour soulager les nécessités des saints, prompts à exercer l’hospitalité.  \milestone{14}  Bénissez ceux qui vous persécutent ; bénissez-les, et ne faites point d’imprécation contre eux.  \milestone{15}  Soyez dans la joie avec ceux qui sont dans la joie, et pleurez avec ceux qui pleurent.  \milestone{16}  Tenez-vous toujours unis dans les mêmes sentiments et les mêmes affections ; n’aspirez point à ce qui est élevé ; mais accommodez-vous à ce qui est de plus bas et de plus humble ; ne soyez point sages à vos propres yeux.  \milestone{17}  Ne rendez à personne le mal pour le mal ; ayez soin de faire le bien, non-seulement devant Dieu, mais aussi devant tous les hommes.  \milestone{18}  Vivez en paix, si cela se peut, et autant qu’il est en vous, avec toutes sortes de personnes.  \milestone{19}  Ne vous vengez point vous-mêmes, mes chers frères ; mais donnez lieu à la colère ; car il est écrit : C’est à moi que la vengeance est réservée ; et c’est moi qui la ferai, dit le Seigneur.  \milestone{20}  Au contraire, si votre ennemi a faim, donnez-lui à manger ; s’il a soif, donnez-lui à boire ; car agissant de la sorte, vous amasserez des charbons de feu sur sa tête.  \milestone{21}  Ne vous laissez point vaincre par le mal ; mais travaillez à vaincre le mal par le bien.

\section[{Rom 13}]{Rom 13}

\noindent \initial{Q}{UE} toute personne soit soumise aux puissances supérieures ; car il n’y a point de puissance qui ne vienne de Dieu, et c’est lui qui a établi toutes celles qui sont sur la terre.  \milestone{2}  Celui donc qui résiste aux puissances, résiste à l’ordre de Dieu ; et ceux qui y résistent, attirent la condamnation sur eux-mêmes.  \milestone{3}  Car les princes ne sont point à craindre, lorsqu’on ne fait que de bonnes actions, mais lorsqu’on en fait de mauvaises. Voulez-vous ne point craindre les puissances, faites bien, et elles vous en loueront.  \milestone{4}  Car le prince est le ministre de Dieu pour vous favoriser dans le bien. Si vous faites mal, vous avez raison de craindre, parce que ce n’est pas en vain qu’il porte l’épée. Car il est le ministre de Dieu pour exécuter sa vengeance, en punissant celui qui fait de mauvaises actions.  \milestone{5}  Il est donc nécessaire de vous y soumettre, non-seulement par la crainte du châtiment, mais aussi par un devoir de conscience.  \milestone{6}  C’est pour cette même raison que vous payez le tribut aux princes : parce qu’ils sont les ministres de Dieu, toujours appliqués aux fonctions de leur ministère.  \milestone{7}  Rendez donc à chacun ce qui lui est dû : le tribut, à qui vous devez le tribut ; les impôts, à qui vous devez les impôts ; la crainte, à qui vous devez de la crainte ; l’honneur, à qui vous devez de l’honneur.\par
\bigbreak
\noindent   \milestone{8}  Acquittez-vous envers tous de tout ce que vous leur devez, ne demeurant redevables que de l’amour qu’on se doit les uns aux autres. Car celui qui aime le prochain, accomplit la loi ;  \milestone{9}  parce que ces commandements de Dieu, Vous ne commettrez point d’adultère ; Vous ne tuerez point ; Vous ne déroberez point ; Vous ne porterez point de faux témoignage ; Vous ne désirerez rien des biens de votre prochain ; et s’il y en a quelque autre semblable ; tous ces commandements, dis-je, sont compris en abrégé dans cette parole ; Vous aimerez le prochain comme vous-même.  \milestone{10}  L’amour qu’on a pour le prochain, ne souffre point qu’on lui fasse du mal ; ainsi l’amour est l’accomplissement de la loi.\par
  \milestone{11}  Acquittons-nous donc de cet amour, et d’autant plus que nous savons que le temps presse, et que l’heure est déjà venue de nous réveiller de notre assoupissement ; puisque nous sommes plus proches de notre salut que lorsque nous avons reçu la foi.  \milestone{12}  La nuit est déjà fort avancée, et le jour s’approche ; quittons donc les œuvres de ténèbres, et revêtons-nous des armes de lumière.  \milestone{13}  Marchons avec bienséance et avec honnêteté, comme on marche durant le jour. Ne vous laissez point aller aux débauches, ni aux ivrogneries ; aux impudicités, ni aux dissolutions ; aux querelles, ni aux envies ;  \milestone{14}  mais revêtez-vous de notre Seigneur Jésus-Christ ; et ne cherchez pas à contenter votre sensualité, en satisfaisant à ses désirs.

\section[{Rom 14}]{Rom 14}

\noindent \initial{R}{ECEVEZ} avec charité celui qui est encore faible dans la foi, sans vous amuser à contester avec lui.  \milestone{2}  Car l’un croit qu’il lui est permis de manger de toutes choses ; et l’autre, au contraire, qui est faible dans la foi, ne mange que des légumes.  \milestone{3}  Que celui qui mange de tout, ne méprise point celui qui n’ose manger de tout ; et que celui qui ne mange pas de tout, ne condamne point celui qui mange de tout, puisque Dieu l’a pris à son service.  \milestone{4}  Qui êtes-vous, pour oser ainsi condamner le serviteur d’autrui ? S’il tombe, ou s’il demeure ferme, cela regarde son maître. Mais il demeurera ferme, parce que Dieu est tout-puissant pour l’affermir.  \milestone{5}  De même, l’un met de la différence entre les jours ; l’autre considère tous les jours comme égaux. Que chacun agisse selon qu’il est pleinement persuadé dans son esprit.  \milestone{6}  Celui qui distingue les jours, les distingue pour plaire au Seigneur. Celui qui mange de tout, le fait pour plaire au Seigneur, car il en rend grâces à Dieu ; et celui qui ne mange pas de tout, le fait aussi pour plaire au Seigneur, et il en rend aussi grâces à Dieu.  \milestone{7}  Car aucun de nous ne vit pour soi-même, et aucun de nous ne meurt pour soi-même.  \milestone{8}  Soit que nous vivions, c’est pour le Seigneur que nous vivons ; soit que nous mourions, c’est pour le Seigneur que nous mourons ; soit donc que nous vivions, soit que nous mourions, nous sommes toujours au Seigneur.  \milestone{9}  Car c’est pour cela même que Jésus-Christ est mort et qu’il est ressuscité, afin d’avoir un empire souverain sur les morts et sur les vivants.  \milestone{10}  Vous donc, pourquoi condamnez-vous votre frère ? Et vous, pourquoi méprisez-vous le vôtre ? Car nous paraîtrons tous devant le tribunal de Jésus-Christ ;  \milestone{11}  selon cette parole de l’Écriture : Je jure par moi-même, dit le Seigneur, que tout genou fléchira devant moi, et que toute langue confessera que c’est moi qui suis Dieu.  \milestone{12}  Ainsi chacun de nous rendra compte à Dieu de soi-même.\par
  \milestone{13}  Ne nous jugeons donc plus les uns les autres ; mais jugez plutôt que vous ne devez pas donner à votre frère une occasion de chute et de scandale.  \milestone{14}  Je sais et je suis persuadé, selon la doctrine du Seigneur Jésus, que rien n’est impur de soi-même, et qu’il n’est impur qu’à celui qui le croit impur.  \milestone{15}  Mais si en mangeant de quelque chose, vous attristez votre frère, dès là vous ne vous conduisez plus par la charité. Ne faites pas périr par votre manger celui pour qui Jésus-Christ est mort.  \milestone{16}  Prenez donc garde de ne pas exposer aux médisances des hommes le bien dont nous jouissons.  \milestone{17}  Car le royaume de Dieu ne consiste pas dans le boire ni dans le manger, mais dans la justice, dans la paix et dans la joie que donne le Saint-Esprit.  \milestone{18}  Et celui qui sert Jésus-Christ en cette manière, est agréable à Dieu, et approuvé des hommes.  \milestone{19}  Appliquons-nous donc à rechercher ce qui peut entretenir la paix parmi nous, et observons tout ce qui peut nous édifier les uns les autres.  \milestone{20}  Que le manger ne soit pas cause que vous détruisiez l’ouvrage de Dieu. Ce n’est pas que toutes les viandes ne soient pures ; mais un homme fait mal d’en manger, lorsqu’en le faisant il scandalise les autres.  \milestone{21}  Et il vaut mieux ne point manger de chair, et ne point boire de vin, ni rien faire de ce qui est à votre frère une occasion de chute et de scandale, ou qui le blesse, parce qu’il est faible.  \milestone{22}  Avez-vous une foi éclairée, contentez-vous de l’avoir dans le cœur aux yeux de Dieu. Heureux celui que sa conscience ne condamne point en ce qu’il veut faire !  \milestone{23}  Mais celui qui étant en doute s’il peut manger d’une viande, ne laisse point d’en manger, il est condamné ; parce qu’il n’agit pas selon la foi. Or tout ce qui ne se fait point selon la foi, est péché.

\section[{Rom 15}]{Rom 15}

\noindent \initial{N}{OUS} devons donc, nous qui sommes plus forts, supporter les faiblesses des infirmes, et non pas chercher notre propre satisfaction.  \milestone{2}  Que chacun de vous tâche de satisfaire son prochain dans ce qui est bon, et qui peut l’édifier ;  \milestone{3}  puisque Jésus-Christ n’a pas cherché à se satisfaire lui-même, mais qu’il dit à son Père, dans l’Ecriture : Les injures qu’on vous a faites, sont retombées sur moi.  \milestone{4}  Car tout ce qui est écrit, a été écrit pour notre instruction ; afin que nous concevions une espérance ferme par la patience, et par la consolation que les Écritures nous donnent.  \milestone{5}  Que le Dieu de patience et de consolation vous fasse la grâce d’être toujours unis de sentiment et d’affection les uns avec les autres, selon l’Esprit de Jésus-Christ ;  \milestone{6}  afin que vous puissiez, d’un même cœur et d’une même bouche, glorifier Dieu, le Père de notre Seigneur Jésus-Christ.\par
\bigbreak
\noindent   \milestone{7}  C’est pourquoi unissez-vous les uns aux autres pour vous soutenir mutuellement, comme Jésus-Christ vous a unis avec lui pour la gloire de Dieu.  \milestone{8}  Car je vous déclare que Jésus-Christ a été le dispensateur et le ministre de l’Évangile à l’égard des Juifs circoncis, afin que Dieu fût reconnu pour véritable par l’accomplissement des promesses qu’il avait faites à leurs pères.  \milestone{9}  Et quant aux gentils, ils doivent glorifier Dieu de sa miséricorde, selon qu’il est écrit : C’est pour cette raison, Seigneur ! que je publierai vos louanges parmi les nations, et que je chanterai des cantiques à la gloire de votre nom.  \milestone{10}  Et l’Écriture dit encore : Réjouissez-vous, nations, avec son peuple.  \milestone{11}  Et ailleurs ; Nations, louez toutes le Seigneur ; peuples, glorifiez-le tous.  \milestone{12}  Isaïe dit aussi : Il sortira de la tige de Jessé un rejeton qui s’élèvera pour régner sur les nations, et les nations espéreront en lui.  \milestone{13}  Que le Dieu d’espérance vous comble de joie et de paix dans votre foi, afin que votre espérance croisse toujours de plus en plus par la vertu et la puissance du Saint-Esprit.\par
\bigbreak
\noindent   \milestone{14}  Pour moi, mes frères, je suis persuadé que vous êtes pleins de charité, que vous êtes remplis de toutes sortes de connaissances, et qu’ainsi vous pouvez vous instruire les uns les autres.  \milestone{15}  Néanmoins je vous ai écrit ceci, mes frères, et peut-être avec un peu de liberté, voulant seulement vous faire ressouvenir de ce que vous savez déjà, selon la grâce que Dieu m’a donnée,  \milestone{16}  d’être le ministre de Jésus-Christ parmi les nations, en exerçant la sacrificature de l’Évangile de Dieu, afin que l’oblation des gentils lui soit agréable, étant sanctifiée par le Saint-Esprit.  \milestone{17}  J’ai donc sujet de me glorifier en Jésus-Christ du succès de l’œuvre de Dieu.  \milestone{18}  Car je n’oserais vous parler de ce que Jésus-Christ a fait par moi, pour amener les nations à l’obéissance de la foi par la parole et par les œuvres,  \milestone{19}  par la vertu des miracles et des prodiges, et par la puissance du Saint-Esprit ; de sorte que j’ai porté l’Évangile de Jésus-Christ dans cette grande étendue de pays qui est depuis Jérusalem jusqu’à l’Illyrie.  \milestone{20}  Et je me suis tellement acquitté de ce ministère, que j’ai eu soin de ne point prêcher l’Évangile dans les lieux où Jésus-Christ avait déjà été prêché, pour ne point bâtir sur le fondement d’autrui, vérifiant ainsi cette parole de l’Écriture :  \milestone{21}  Ceux à qui il n’avait point été annoncé, verront sa lumière ; et ceux qui n’avaient point encore entendu parler de lui, auront l’intelligence de sa doctrine.\par
\bigbreak
\noindent   \milestone{22}  C’est ce qui m’a souvent empêché d’aller vers vous, et je ne l’ai pu faire jusqu’à cette heure.  \milestone{23}  Mais n’ayant plus maintenant aucun sujet de demeurer davantage dans ce pays-ci, et désirant depuis plusieurs années de vous aller voir,  \milestone{24}  lorsque je ferai le voyage d’Espagne, j’espère vous voir en passant ; afin qu’après avoir un peu joui de votre présence, vous me conduisiez en ce pays-là.  \milestone{25}  Maintenant je m’en vais à Jérusalem, porter aux saints quelques aumônes.  \milestone{26}  Car les Églises de Macédoine et d’Achaïe ont résolu avec beaucoup d’affection, de faire quelque part de leurs biens à ceux d’entre les saints de Jérusalem qui sont pauvres.  \milestone{27}  Ils l’ont résolu, dis-je, avec beaucoup d’affection, et en effet ils leur sont redevables Car si les gentils ont participé aux richesses spirituelles des Juifs, ils doivent aussi leur faire part de leurs biens temporels. \\
  \milestone{28}  Lors donc que je me serai acquitté de ce devoir, et que je leur aurai rendu ce dépôt qui est le fruit de la piété des fidèles, je passerai par vos quartiers, en allant en Espagne.  \milestone{29}  Or je sais que quand j’irai vous voir, ma venue sera accompagnée d’une abondante bénédiction de l’Évangile de Jésus-Christ.\par
  \milestone{30}  Je vous conjure donc, mes frères, par Jésus-Christ notre Seigneur, et par la charité du Saint-Esprit, de combattre avec moi par les prières que vous ferez à Dieu pour moi ;  \milestone{31}  afin qu’il me délivre des Juifs incrédules qui sont en Judée, et que les saints de Jérusalem reçoivent favorablement le service que je vais leur rendre ;  \milestone{32}  et qu’ainsi, étant plein de joie, je puisse aller vous voir, si c’est la volonté de Dieu, et jouir avec vous d’une consolation mutuelle.  \milestone{33}  Je prie le Dieu de paix de demeurer avec vous tous. Amen !

\section[{Rom 16}]{Rom 16}

\noindent \initial{J}{E} vous recommande notre sœur Phébé, diaconesse de l’Église qui est au port de Cenchrée ;  \milestone{2}  afin que vous la receviez au nom du Seigneur, comme on doit recevoir les saints, et que vous l’assistiez dans toutes les choses où elle pourrait avoir besoin de vous : car elle en a assisté elle-même plusieurs, et moi en particulier.\par
  \milestone{3}  Saluez de ma part Prisque et Aquilas, qui ont travaillé avec moi pour le service de Jésus-Christ ;  \milestone{4}  qui ont exposé leur tête pour me sauver la vie, et à qui je ne suis pas le seul qui soit obligé, mais encore toutes les Églises des gentils.  \milestone{5}  Saluez aussi de ma part l’Église qui est dans leur maison. Saluez mon cher Épénète, qui a été les prémices de l’Asie par la foi en Jésus-Christ.  \milestone{6}  Saluez Marie, qui a beaucoup travaillé pour vous.  \milestone{7}  Saluez Andronique et Junie, mes parents, qui ont été compagnons de mes liens, qui sont considérables entre les apôtres, et qui ont embrassé la foi de Jésus-Christ avant moi.  \milestone{8}  Saluez Amplias, que j’aime particulièrement en notre Seigneur.  \milestone{9}  Saluez Urbain, qui a travaillé avec nous pour le service de Jésus-Christ ; et mon cher Stachys.  \milestone{10}  Saluez Apelle, qui est un fidèle serviteur de Jésus-Christ.  \milestone{11}  Saluez ceux qui sont de la famille d’Aristobule. Saluez Hérodion, mon cousin. Saluez ceux de la maison de Narcisse, qui sont nos frères dans le Seigneur.  \milestone{12}  Saluez Tryphène et Tryphose, lesquelles travaillent pour le service du Seigneur. Saluez notre chère Perside, qui a aussi beaucoup travaillé pour le service du Seigneur.  \milestone{13}  Saluez Rufus, qui est un élu du Seigneur ; et sa mère, que je regarde comme la mienne.  \milestone{14}  Saluez Asyncrite, Phlégon, Hermas, Patrobe, Hermès, et nos frères qui sont avec eux.  \milestone{15}  Saluez Philologue et Julie, Nérée et sa sœur, et Olympiade, et tous les saints qui sont avec eux.  \milestone{16}  Saluez-vous les uns les autres par un saint baiser. Toutes les Églises de Jésus-Christ vous saluent.\par
  \milestone{17}  Mais je vous exhorte, mes frères, de prendre garde à ceux qui causent parmi vous des divisions et des scandales contre la doctrine que vous avez apprise, et d’éviter leur compagnie.  \milestone{18}  Car ces sortes de gens ne servent point Jésus-Christ notre Seigneur, mais sont esclaves de leur sensualité ; et par des paroles douces et flatteuses, ils séduisent les âmes simples.  \milestone{19}  L’obéissance que vous avez rendue à la foi, est venue à la connaissance de tout le monde, et je m’en réjouis pour vous ; mais je désire que vous soyez sages dans le bien, et simples dans le mal.  \milestone{20}  Que le Dieu de paix brise bientôt Satan sous vos pieds ! Que la grâce de notre Seigneur Jésus-Christ soit avec vous !\par
  \milestone{21}  Timothée, qui est le compagnon de mes travaux, vous salue ; comme aussi Lucius, Jason et Sosipatre, qui sont mes parents.  \milestone{22}  Je vous salue au nom du Seigneur, moi Tertius, qui ai écrit cette lettre.  \milestone{23}  Caïus, qui est mon hôte, et toute l’Église, vous saluent. Éraste, trésorier de la ville, vous salue, et notre frère Quartus.  \milestone{24}  Que la grâce de notre Seigneur Jésus-Christ soit avec vous tous ! Amen !\par
\bigbreak
\noindent   \milestone{25}  À celui qui est tout-puissant pour vous affermir dans la foi de l’Évangile et de la doctrine de Jésus-Christ, que je prêche suivant la révélation du mystère qui, étant demeuré caché dans tous les siècles passés,  \milestone{26}  a été découvert maintenant par les oracles des prophètes, selon l’ordre du Dieu éternel, pour amener les hommes à l’obéissance de la foi, et est venu à la connaissance de toutes les nations ;  \milestone{27}  à Dieu, dis-je, qui est le seul sage, honneur et gloire par Jésus-Christ dans les siècles des siècles ! Amen !
\chapterclose

 


% at least one empty page at end (for booklet couv)
\ifbooklet
  \pagestyle{empty}
  \clearpage
  % 2 empty pages maybe needed for 4e cover
  \ifnum\modulo{\value{page}}{4}=0 \hbox{}\newpage\hbox{}\newpage\fi
  \ifnum\modulo{\value{page}}{4}=1 \hbox{}\newpage\hbox{}\newpage\fi


  \hbox{}\newpage
  \ifodd\value{page}\hbox{}\newpage\fi
  {\centering\color{rubric}\bfseries\noindent\large
    Hurlus ? Qu’est-ce.\par
    \bigskip
  }
  \noindent Des bouquinistes électroniques, pour du texte libre à participations libres,
  téléchargeable gratuitement sur \href{https://hurlus.fr}{\dotuline{hurlus.fr}}.\par
  \bigskip
  \noindent Cette brochure a été produite par des éditeurs bénévoles.
  Elle n’est pas faite pour être possédée, mais pour être lue, et puis donnée, ou déposée dans une boîte à livres.
  En page de garde, on peut ajouter une date, un lieu, un nom ;
  comme une fiche de bibliothèque en papier qui enregistre \emph{les voyages de la brochure}.
  \par

  Ce texte a été choisi parce qu’une personne l’a aimé,
  ou haï, elle a pensé qu’il partipait à la formation de notre présent ;
  sans le souci de plaire, vendre, ou militer pour une cause.
  \par

  L’édition électronique est soigneuse, tant sur la technique
  que sur l’établissement du texte ; mais sans aucune prétention scolaire, au contraire.
  Le but est de s’adresser à tous, sans distinction de science ou de diplôme.
  \par

  Cet exemplaire en papier a été tiré sur une imprimante personnelle
   ou une photocopieuse. Tout le monde peut le faire.
  Il suffit de
  télécharger un fichier sur \href{https://hurlus.fr}{\dotuline{hurlus.fr}},
  d’imprimer, et agrafer (puis lire et donner).\par

  \bigskip

  \noindent PS : Les hurlus furent aussi des rebelles protestants qui cassaient les statues dans les églises catholiques. En 1566 démarra la révolte des gueux dans le pays de Lille. L’insurrection enflamma la région jusqu’à Anvers où les gueux de mer bloquèrent les bateaux espagnols.
  Ce fut une rare guerre de libération dont naquit un pays toujours libre : les Pays-Bas.
  En plat pays francophone, par contre, restèrent des bandes de huguenots, les hurlus, progressivement réprimés par la très catholique Espagne.
  Cette mémoire d’une défaite est éteinte, rallumons-la. Sortons les livres du culte universitaire, débusquons les idoles de l’époque, pour les démonter.
\fi

\end{document}
