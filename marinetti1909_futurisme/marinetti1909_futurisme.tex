%%%%%%%%%%%%%%%%%%%%%%%%%%%%%%%%%
% LaTeX model https://hurlus.fr %
%%%%%%%%%%%%%%%%%%%%%%%%%%%%%%%%%

% Needed before document class
\RequirePackage{pdftexcmds} % needed for tests expressions
\RequirePackage{fix-cm} % correct units

% Define mode
\def\mode{a4}

\newif\ifaiv % a4
\newif\ifav % a5
\newif\ifbooklet % booklet
\newif\ifcover % cover for booklet

\ifnum \strcmp{\mode}{cover}=0
  \covertrue
\else\ifnum \strcmp{\mode}{booklet}=0
  \booklettrue
\else\ifnum \strcmp{\mode}{a5}=0
  \avtrue
\else
  \aivtrue
\fi\fi\fi

\ifbooklet % do not enclose with {}
  \documentclass[french,twoside]{book} % ,notitlepage
  \usepackage[%
    papersize={105mm, 297mm},
    inner=12mm,
    outer=12mm,
    top=20mm,
    bottom=15mm,
    marginparsep=0pt,
  ]{geometry}
  \usepackage[fontsize=9.5pt]{scrextend} % for Roboto
\else\ifav
  \documentclass[french,twoside]{book} % ,notitlepage
  \usepackage[%
    a5paper,
    inner=25mm,
    outer=15mm,
    top=15mm,
    bottom=15mm,
    marginparsep=0pt,
  ]{geometry}
  \usepackage[fontsize=12pt]{scrextend}
\else% A4 2 cols
  \documentclass[twocolumn]{report}
  \usepackage[%
    a4paper,
    inner=15mm,
    outer=10mm,
    top=25mm,
    bottom=18mm,
    marginparsep=0pt,
  ]{geometry}
  \setlength{\columnsep}{20mm}
  \usepackage[fontsize=9.5pt]{scrextend}
\fi\fi

%%%%%%%%%%%%%%
% Alignments %
%%%%%%%%%%%%%%
% before teinte macros

\setlength{\arrayrulewidth}{0.2pt}
\setlength{\columnseprule}{\arrayrulewidth} % twocol
\setlength{\parskip}{0pt} % classical para with no margin
\setlength{\parindent}{1.5em}

%%%%%%%%%%
% Colors %
%%%%%%%%%%
% before Teinte macros

\usepackage[dvipsnames]{xcolor}
\definecolor{rubric}{HTML}{800000} % the tonic 0c71c3
\def\columnseprulecolor{\color{rubric}}
\colorlet{borderline}{rubric!30!} % definecolor need exact code
\definecolor{shadecolor}{gray}{0.95}
\definecolor{bghi}{gray}{0.5}

%%%%%%%%%%%%%%%%%
% Teinte macros %
%%%%%%%%%%%%%%%%%
%%%%%%%%%%%%%%%%%%%%%%%%%%%%%%%%%%%%%%%%%%%%%%%%%%%
% <TEI> generic (LaTeX names generated by Teinte) %
%%%%%%%%%%%%%%%%%%%%%%%%%%%%%%%%%%%%%%%%%%%%%%%%%%%
% This template is inserted in a specific design
% It is XeLaTeX and otf fonts

\makeatletter % <@@@


\usepackage{blindtext} % generate text for testing
\usepackage[strict]{changepage} % for modulo 4
\usepackage{contour} % rounding words
\usepackage[nodayofweek]{datetime}
% \usepackage{DejaVuSans} % seems buggy for sffont font for symbols
\usepackage{enumitem} % <list>
\usepackage{etoolbox} % patch commands
\usepackage{fancyvrb}
\usepackage{fancyhdr}
\usepackage{float}
\usepackage{fontspec} % XeLaTeX mandatory for fonts
\usepackage{footnote} % used to capture notes in minipage (ex: quote)
\usepackage{framed} % bordering correct with footnote hack
\usepackage{graphicx}
\usepackage{lettrine} % drop caps
\usepackage{lipsum} % generate text for testing
\usepackage[framemethod=tikz,]{mdframed} % maybe used for frame with footnotes inside
\usepackage{pdftexcmds} % needed for tests expressions
\usepackage{polyglossia} % non-break space french punct, bug Warning: "Failed to patch part"
\usepackage[%
  indentfirst=false,
  vskip=1em,
  noorphanfirst=true,
  noorphanafter=true,
  leftmargin=\parindent,
  rightmargin=0pt,
]{quoting}
\usepackage{ragged2e}
\usepackage{setspace} % \setstretch for <quote>
\usepackage{tabularx} % <table>
\usepackage[explicit]{titlesec} % wear titles, !NO implicit
\usepackage{tikz} % ornaments
\usepackage{tocloft} % styling tocs
\usepackage[fit]{truncate} % used im runing titles
\usepackage{unicode-math}
\usepackage[normalem]{ulem} % breakable \uline, normalem is absolutely necessary to keep \emph
\usepackage{verse} % <l>
\usepackage{xcolor} % named colors
\usepackage{xparse} % @ifundefined
\XeTeXdefaultencoding "iso-8859-1" % bad encoding of xstring
\usepackage{xstring} % string tests
\XeTeXdefaultencoding "utf-8"
\PassOptionsToPackage{hyphens}{url} % before hyperref, which load url package

% TOTEST
% \usepackage{hypcap} % links in caption ?
% \usepackage{marginnote}
% TESTED
% \usepackage{background} % doesn’t work with xetek
% \usepackage{bookmark} % prefers the hyperref hack \phantomsection
% \usepackage[color, leftbars]{changebar} % 2 cols doc, impossible to keep bar left
% \usepackage[utf8x]{inputenc} % inputenc package ignored with utf8 based engines
% \usepackage[sfdefault,medium]{inter} % no small caps
% \usepackage{firamath} % choose firasans instead, firamath unavailable in Ubuntu 21-04
% \usepackage{flushend} % bad for last notes, supposed flush end of columns
% \usepackage[stable]{footmisc} % BAD for complex notes https://texfaq.org/FAQ-ftnsect
% \usepackage{helvet} % not for XeLaTeX
% \usepackage{multicol} % not compatible with too much packages (longtable, framed, memoir…)
% \usepackage[default,oldstyle,scale=0.95]{opensans} % no small caps
% \usepackage{sectsty} % \chapterfont OBSOLETE
% \usepackage{soul} % \ul for underline, OBSOLETE with XeTeX
% \usepackage[breakable]{tcolorbox} % text styling gone, footnote hack not kept with breakable


% Metadata inserted by a program, from the TEI source, for title page and runing heads
\title{\textbf{ Manifeste du Futurisme }}
\date{1909}
\author{Marinetti, Filippo Tommaso}
\def\elbibl{Marinetti, Filippo Tommaso. 1909. \emph{Manifeste du Futurisme}}
\def\elsource{ \href{http://gallica.bnf.fr/ark:/12148/bpt6k2883730}{\dotuline{http://gallica.bnf.fr/ark:/12148/bpt6k2883730}}\footnote{\href{http://gallica.bnf.fr/ark:/12148/bpt6k2883730}{\url{http://gallica.bnf.fr/ark:/12148/bpt6k2883730}}}  \href{http://efele.net/ebooks/livres/000400}{\dotuline{http://efele.net/ebooks/livres/000400}}\footnote{\href{http://efele.net/ebooks/livres/000400}{\url{http://efele.net/ebooks/livres/000400}}} }

% Default metas
\newcommand{\colorprovide}[2]{\@ifundefinedcolor{#1}{\colorlet{#1}{#2}}{}}
\colorprovide{rubric}{red}
\colorprovide{silver}{lightgray}
\@ifundefined{syms}{\newfontfamily\syms{DejaVu Sans}}{}
\newif\ifdev
\@ifundefined{elbibl}{% No meta defined, maybe dev mode
  \newcommand{\elbibl}{Titre court ?}
  \newcommand{\elbook}{Titre du livre source ?}
  \newcommand{\elabstract}{Résumé\par}
  \newcommand{\elurl}{http://oeuvres.github.io/elbook/2}
  \author{Éric Lœchien}
  \title{Un titre de test assez long pour vérifier le comportement d’une maquette}
  \date{1566}
  \devtrue
}{}
\let\eltitle\@title
\let\elauthor\@author
\let\eldate\@date


\defaultfontfeatures{
  % Mapping=tex-text, % no effect seen
  Scale=MatchLowercase,
  Ligatures={TeX,Common},
}


% generic typo commands
\newcommand{\astermono}{\medskip\centerline{\color{rubric}\large\selectfont{\syms ✻}}\medskip\par}%
\newcommand{\astertri}{\medskip\par\centerline{\color{rubric}\large\selectfont{\syms ✻\,✻\,✻}}\medskip\par}%
\newcommand{\asterism}{\bigskip\par\noindent\parbox{\linewidth}{\centering\color{rubric}\large{\syms ✻}\\{\syms ✻}\hskip 0.75em{\syms ✻}}\bigskip\par}%

% lists
\newlength{\listmod}
\setlength{\listmod}{\parindent}
\setlist{
  itemindent=!,
  listparindent=\listmod,
  labelsep=0.2\listmod,
  parsep=0pt,
  % topsep=0.2em, % default topsep is best
}
\setlist[itemize]{
  label=—,
  leftmargin=0pt,
  labelindent=1.2em,
  labelwidth=0pt,
}
\setlist[enumerate]{
  label={\bf\color{rubric}\arabic*.},
  labelindent=0.8\listmod,
  leftmargin=\listmod,
  labelwidth=0pt,
}
\newlist{listalpha}{enumerate}{1}
\setlist[listalpha]{
  label={\bf\color{rubric}\alph*.},
  leftmargin=0pt,
  labelindent=0.8\listmod,
  labelwidth=0pt,
}
\newcommand{\listhead}[1]{\hspace{-1\listmod}\emph{#1}}

\renewcommand{\hrulefill}{%
  \leavevmode\leaders\hrule height 0.2pt\hfill\kern\z@}

% General typo
\DeclareTextFontCommand{\textlarge}{\large}
\DeclareTextFontCommand{\textsmall}{\small}

% commands, inlines
\newcommand{\anchor}[1]{\Hy@raisedlink{\hypertarget{#1}{}}} % link to top of an anchor (not baseline)
\newcommand\abbr[1]{#1}
\newcommand{\autour}[1]{\tikz[baseline=(X.base)]\node [draw=rubric,thin,rectangle,inner sep=1.5pt, rounded corners=3pt] (X) {\color{rubric}#1};}
\newcommand\corr[1]{#1}
\newcommand{\ed}[1]{ {\color{silver}\sffamily\footnotesize (#1)} } % <milestone ed="1688"/>
\newcommand\expan[1]{#1}
\newcommand\foreign[1]{\emph{#1}}
\newcommand\gap[1]{#1}
\renewcommand{\LettrineFontHook}{\color{rubric}}
\newcommand{\initial}[2]{\lettrine[lines=2, loversize=0.3, lhang=0.3]{#1}{#2}}
\newcommand{\initialiv}[2]{%
  \let\oldLFH\LettrineFontHook
  % \renewcommand{\LettrineFontHook}{\color{rubric}\ttfamily}
  \IfSubStr{QJ’}{#1}{
    \lettrine[lines=4, lhang=0.2, loversize=-0.1, lraise=0.2]{\smash{#1}}{#2}
  }{\IfSubStr{É}{#1}{
    \lettrine[lines=4, lhang=0.2, loversize=-0, lraise=0]{\smash{#1}}{#2}
  }{\IfSubStr{ÀÂ}{#1}{
    \lettrine[lines=4, lhang=0.2, loversize=-0, lraise=0, slope=0.6em]{\smash{#1}}{#2}
  }{\IfSubStr{A}{#1}{
    \lettrine[lines=4, lhang=0.2, loversize=0.2, slope=0.6em]{\smash{#1}}{#2}
  }{\IfSubStr{V}{#1}{
    \lettrine[lines=4, lhang=0.2, loversize=0.2, slope=-0.5em]{\smash{#1}}{#2}
  }{
    \lettrine[lines=4, lhang=0.2, loversize=0.2]{\smash{#1}}{#2}
  }}}}}
  \let\LettrineFontHook\oldLFH
}
\newcommand{\labelchar}[1]{\textbf{\color{rubric} #1}}
\newcommand{\milestone}[1]{\autour{\footnotesize\color{rubric} #1}} % <milestone n="4"/>
\newcommand\name[1]{#1}
\newcommand\orig[1]{#1}
\newcommand\orgName[1]{#1}
\newcommand\persName[1]{#1}
\newcommand\placeName[1]{#1}
\newcommand{\pn}[1]{\IfSubStr{-—–¶}{#1}% <p n="3"/>
  {\noindent{\bfseries\color{rubric}   ¶  }}
  {{\footnotesize\autour{ #1}  }}}
\newcommand\reg{}
% \newcommand\ref{} % already defined
\newcommand\sic[1]{#1}
\newcommand\surname[1]{\textsc{#1}}
\newcommand\term[1]{\textbf{#1}}

\def\mednobreak{\ifdim\lastskip<\medskipamount
  \removelastskip\nopagebreak\medskip\fi}
\def\bignobreak{\ifdim\lastskip<\bigskipamount
  \removelastskip\nopagebreak\bigskip\fi}

% commands, blocks
\newcommand{\byline}[1]{\bigskip{\RaggedLeft{#1}\par}\bigskip}
\newcommand{\bibl}[1]{{\RaggedLeft{#1}\par\bigskip}}
\newcommand{\biblitem}[1]{{\noindent\hangindent=\parindent   #1\par}}
\newcommand{\dateline}[1]{\medskip{\RaggedLeft{#1}\par}\bigskip}
\newcommand{\labelblock}[1]{\medbreak{\noindent\color{rubric}\bfseries #1}\par\mednobreak}
\newcommand{\salute}[1]{\bigbreak{#1}\par\medbreak}
\newcommand{\signed}[1]{\bigbreak\filbreak{\raggedleft #1\par}\medskip}

% environments for blocks (some may become commands)
\newenvironment{borderbox}{}{} % framing content
\newenvironment{citbibl}{\ifvmode\hfill\fi}{\ifvmode\par\fi }
\newenvironment{docAuthor}{\ifvmode\vskip4pt\fontsize{16pt}{18pt}\selectfont\fi\itshape}{\ifvmode\par\fi }
\newenvironment{docDate}{}{\ifvmode\par\fi }
\newenvironment{docImprint}{\vskip6pt}{\ifvmode\par\fi }
\newenvironment{docTitle}{\vskip6pt\bfseries\fontsize{18pt}{22pt}\selectfont}{\par }
\newenvironment{msHead}{\vskip6pt}{\par}
\newenvironment{msItem}{\vskip6pt}{\par}
\newenvironment{titlePart}{}{\par }


% environments for block containers
\newenvironment{argument}{\itshape\parindent0pt}{\vskip1.5em}
\newenvironment{biblfree}{}{\ifvmode\par\fi }
\newenvironment{bibitemlist}[1]{%
  \list{\@biblabel{\@arabic\c@enumiv}}%
  {%
    \settowidth\labelwidth{\@biblabel{#1}}%
    \leftmargin\labelwidth
    \advance\leftmargin\labelsep
    \@openbib@code
    \usecounter{enumiv}%
    \let\p@enumiv\@empty
    \renewcommand\theenumiv{\@arabic\c@enumiv}%
  }
  \sloppy
  \clubpenalty4000
  \@clubpenalty \clubpenalty
  \widowpenalty4000%
  \sfcode`\.\@m
}%
{\def\@noitemerr
  {\@latex@warning{Empty `bibitemlist' environment}}%
\endlist}
\newenvironment{quoteblock}% may be used for ornaments
  {\begin{quoting}}
  {\end{quoting}}

% table () is preceded and finished by custom command
\newcommand{\tableopen}[1]{%
  \ifnum\strcmp{#1}{wide}=0{%
    \begin{center}
  }
  \else\ifnum\strcmp{#1}{long}=0{%
    \begin{center}
  }
  \else{%
    \begin{center}
  }
  \fi\fi
}
\newcommand{\tableclose}[1]{%
  \ifnum\strcmp{#1}{wide}=0{%
    \end{center}
  }
  \else\ifnum\strcmp{#1}{long}=0{%
    \end{center}
  }
  \else{%
    \end{center}
  }
  \fi\fi
}


% text structure
\newcommand\chapteropen{} % before chapter title
\newcommand\chaptercont{} % after title, argument, epigraph…
\newcommand\chapterclose{} % maybe useful for multicol settings
\setcounter{secnumdepth}{-2} % no counters for hierarchy titles
\setcounter{tocdepth}{5} % deep toc
\markright{\@title} % ???
\markboth{\@title}{\@author} % ???
\renewcommand\tableofcontents{\@starttoc{toc}}
% toclof format
% \renewcommand{\@tocrmarg}{0.1em} % Useless command?
% \renewcommand{\@pnumwidth}{0.5em} % {1.75em}
\renewcommand{\@cftmaketoctitle}{}
\setlength{\cftbeforesecskip}{\z@ \@plus.2\p@}
\renewcommand{\cftchapfont}{}
\renewcommand{\cftchapdotsep}{\cftdotsep}
\renewcommand{\cftchapleader}{\normalfont\cftdotfill{\cftchapdotsep}}
\renewcommand{\cftchappagefont}{\bfseries}
\setlength{\cftbeforechapskip}{0em \@plus\p@}
% \renewcommand{\cftsecfont}{\small\relax}
\renewcommand{\cftsecpagefont}{\normalfont}
% \renewcommand{\cftsubsecfont}{\small\relax}
\renewcommand{\cftsecdotsep}{\cftdotsep}
\renewcommand{\cftsecpagefont}{\normalfont}
\renewcommand{\cftsecleader}{\normalfont\cftdotfill{\cftsecdotsep}}
\setlength{\cftsecindent}{1em}
\setlength{\cftsubsecindent}{2em}
\setlength{\cftsubsubsecindent}{3em}
\setlength{\cftchapnumwidth}{1em}
\setlength{\cftsecnumwidth}{1em}
\setlength{\cftsubsecnumwidth}{1em}
\setlength{\cftsubsubsecnumwidth}{1em}

% footnotes
\newif\ifheading
\newcommand*{\fnmarkscale}{\ifheading 0.70 \else 1 \fi}
\renewcommand\footnoterule{\vspace*{0.3cm}\hrule height \arrayrulewidth width 3cm \vspace*{0.3cm}}
\setlength\footnotesep{1.5\footnotesep} % footnote separator
\renewcommand\@makefntext[1]{\parindent 1.5em \noindent \hb@xt@1.8em{\hss{\normalfont\@thefnmark . }}#1} % no superscipt in foot
\patchcmd{\@footnotetext}{\footnotesize}{\footnotesize\sffamily}{}{} % before scrextend, hyperref


%   see https://tex.stackexchange.com/a/34449/5049
\def\truncdiv#1#2{((#1-(#2-1)/2)/#2)}
\def\moduloop#1#2{(#1-\truncdiv{#1}{#2}*#2)}
\def\modulo#1#2{\number\numexpr\moduloop{#1}{#2}\relax}

% orphans and widows
\clubpenalty=9996
\widowpenalty=9999
\brokenpenalty=4991
\predisplaypenalty=10000
\postdisplaypenalty=1549
\displaywidowpenalty=1602
\hyphenpenalty=400
% Copied from Rahtz but not understood
\def\@pnumwidth{1.55em}
\def\@tocrmarg {2.55em}
\def\@dotsep{4.5}
\emergencystretch 3em
\hbadness=4000
\pretolerance=750
\tolerance=2000
\vbadness=4000
\def\Gin@extensions{.pdf,.png,.jpg,.mps,.tif}
% \renewcommand{\@cite}[1]{#1} % biblio

\usepackage{hyperref} % supposed to be the last one, :o) except for the ones to follow
\urlstyle{same} % after hyperref
\hypersetup{
  % pdftex, % no effect
  pdftitle={\elbibl},
  % pdfauthor={Your name here},
  % pdfsubject={Your subject here},
  % pdfkeywords={keyword1, keyword2},
  bookmarksnumbered=true,
  bookmarksopen=true,
  bookmarksopenlevel=1,
  pdfstartview=Fit,
  breaklinks=true, % avoid long links
  pdfpagemode=UseOutlines,    % pdf toc
  hyperfootnotes=true,
  colorlinks=false,
  pdfborder=0 0 0,
  % pdfpagelayout=TwoPageRight,
  % linktocpage=true, % NO, toc, link only on page no
}

\makeatother % /@@@>
%%%%%%%%%%%%%%
% </TEI> end %
%%%%%%%%%%%%%%


%%%%%%%%%%%%%
% footnotes %
%%%%%%%%%%%%%
\renewcommand{\thefootnote}{\bfseries\textcolor{rubric}{\arabic{footnote}}} % color for footnote marks

%%%%%%%%%
% Fonts %
%%%%%%%%%
\usepackage[]{roboto} % SmallCaps, Regular is a bit bold
% \linespread{0.90} % too compact, keep font natural
\newfontfamily\fontrun[]{Roboto Condensed Light} % condensed runing heads
\ifav
  \setmainfont[
    ItalicFont={Roboto Light Italic},
  ]{Roboto}
\else\ifbooklet
  \setmainfont[
    ItalicFont={Roboto Light Italic},
  ]{Roboto}
\else
\setmainfont[
  ItalicFont={Roboto Italic},
]{Roboto Light}
\fi\fi
\renewcommand{\LettrineFontHook}{\bfseries\color{rubric}}
% \renewenvironment{labelblock}{\begin{center}\bfseries\color{rubric}}{\end{center}}

%%%%%%%%
% MISC %
%%%%%%%%

\setdefaultlanguage[frenchpart=false]{french} % bug on part


\newenvironment{quotebar}{%
    \def\FrameCommand{{\color{rubric!10!}\vrule width 0.5em} \hspace{0.9em}}%
    \def\OuterFrameSep{\itemsep} % séparateur vertical
    \MakeFramed {\advance\hsize-\width \FrameRestore}
  }%
  {%
    \endMakeFramed
  }
\renewenvironment{quoteblock}% may be used for ornaments
  {%
    \savenotes
    \setstretch{0.9}
    \normalfont
    \begin{quotebar}
  }
  {%
    \end{quotebar}
    \spewnotes
  }


\renewcommand{\headrulewidth}{\arrayrulewidth}
\renewcommand{\headrule}{{\color{rubric}\hrule}}

% delicate tuning, image has produce line-height problems in title on 2 lines
\titleformat{name=\chapter} % command
  [display] % shape
  {\vspace{1.5em}\centering} % format
  {} % label
  {0pt} % separator between n
  {}
[{\color{rubric}\huge\textbf{#1}}\bigskip] % after code
% \titlespacing{command}{left spacing}{before spacing}{after spacing}[right]
\titlespacing*{\chapter}{0pt}{-2em}{0pt}[0pt]

\titleformat{name=\section}
  [block]{}{}{}{}
  [\vbox{\color{rubric}\large\raggedleft\textbf{#1}}]
\titlespacing{\section}{0pt}{0pt plus 4pt minus 2pt}{\baselineskip}

\titleformat{name=\subsection}
  [block]
  {}
  {} % \thesection
  {} % separator \arrayrulewidth
  {}
[\vbox{\large\textbf{#1}}]
% \titlespacing{\subsection}{0pt}{0pt plus 4pt minus 2pt}{\baselineskip}

\ifaiv
  \fancypagestyle{main}{%
    \fancyhf{}
    \setlength{\headheight}{1.5em}
    \fancyhead{} % reset head
    \fancyfoot{} % reset foot
    \fancyhead[L]{\truncate{0.45\headwidth}{\fontrun\elbibl}} % book ref
    \fancyhead[R]{\truncate{0.45\headwidth}{ \fontrun\nouppercase\leftmark}} % Chapter title
    \fancyhead[C]{\thepage}
  }
  \fancypagestyle{plain}{% apply to chapter
    \fancyhf{}% clear all header and footer fields
    \setlength{\headheight}{1.5em}
    \fancyhead[L]{\truncate{0.9\headwidth}{\fontrun\elbibl}}
    \fancyhead[R]{\thepage}
  }
\else
  \fancypagestyle{main}{%
    \fancyhf{}
    \setlength{\headheight}{1.5em}
    \fancyhead{} % reset head
    \fancyfoot{} % reset foot
    \fancyhead[RE]{\truncate{0.9\headwidth}{\fontrun\elbibl}} % book ref
    \fancyhead[LO]{\truncate{0.9\headwidth}{\fontrun\nouppercase\leftmark}} % Chapter title, \nouppercase needed
    \fancyhead[RO,LE]{\thepage}
  }
  \fancypagestyle{plain}{% apply to chapter
    \fancyhf{}% clear all header and footer fields
    \setlength{\headheight}{1.5em}
    \fancyhead[L]{\truncate{0.9\headwidth}{\fontrun\elbibl}}
    \fancyhead[R]{\thepage}
  }
\fi

\ifav % a5 only
  \titleclass{\section}{top}
\fi

\newcommand\chapo{{%
  \vspace*{-3em}
  \centering % no vskip ()
  {\Large\addfontfeature{LetterSpace=25}\bfseries{\elauthor}}\par
  \smallskip
  {\large\eldate}\par
  \bigskip
  {\Large\selectfont{\eltitle}}\par
  \bigskip
  {\color{rubric}\hline\par}
  \bigskip
  {\Large TEXTE LIBRE À PARTICPATION LIBRE\par}
  \centerline{\small\color{rubric} {hurlus.fr, tiré le \today}}\par
  \bigskip
}}

\newcommand\cover{{%
  \thispagestyle{empty}
  \centering
  {\LARGE\bfseries{\elauthor}}\par
  \bigskip
  {\Large\eldate}\par
  \bigskip
  \bigskip
  {\LARGE\selectfont{\eltitle}}\par
  \vfill\null
  {\color{rubric}\setlength{\arrayrulewidth}{2pt}\hline\par}
  \vfill\null
  {\Large TEXTE LIBRE À PARTICPATION LIBRE\par}
  \centerline{{\href{https://hurlus.fr}{\dotuline{hurlus.fr}}, tiré le \today}}\par
}}

\begin{document}
\pagestyle{empty}
\ifbooklet{
  \cover\newpage
  \thispagestyle{empty}\hbox{}\newpage
  \cover\newpage\noindent Les voyages de la brochure\par
  \bigskip
  \begin{tabularx}{\textwidth}{l|X|X}
    \textbf{Date} & \textbf{Lieu}& \textbf{Nom/pseudo} \\ \hline
    \rule{0pt}{25cm} &  &   \\
  \end{tabularx}
  \newpage
  \addtocounter{page}{-4}
}\fi

\thispagestyle{empty}
\ifaiv
  \twocolumn[\chapo]
\else
  \chapo
\fi
{\it\elabstract}
\bigskip
\makeatletter\@starttoc{toc}\makeatother % toc without new page
\bigskip

\pagestyle{main} % after style

   \section[{Le Futurisme}]{Le Futurisme}\renewcommand{\leftmark}{Le Futurisme}

\noindent M. Marinetti, le jeune poète italien et français, au talent remarquable et fougueux, que de retentissantes manifestations ont fait connaître dans tous les pays latins, suivi d’une pléiade d’enthousiastes disciples, vient de fonder l’Ecole du « Futurisme » dont les théories dépassent en hardiesse toutes celles des écoles antérieures ou contemporaines. Le Figaro qui a déjà servi de tribune à plusieurs d’entre elles, et non des moindres, offre aujourd’hui à ses lecteurs le manifeste des « Futuristes ». Est-il besoin de dire que nous laissons au signataire toute la responsabilité de ses idées singulièrement audacieuses et d’une outrance souvent injuste pour des choses éminemment respectables et, heureusement, partout respectées? Mais il était interessant de réserver à nos lecteurs la primeur de cette manifestation, quel que soit le jugement qu’on porte sur elle.\par

\astertri

\noindent Nous avions veillé toute la nuit, mes amis et moi, sous des lampes de mosquée dont les coupoles de cuivre aussi ajourées que notre âme avaient pourtant des cœurs électriques. Et tout en piétinant notre native paresse sur d’opulents tapis persans, nous avions discuté aux frontières extrèmes de la logique et griffé le papier de démentes écritures.\par
Un immense orgueil gonflait nos poitrines à nous sentir debout tous seuls, comme des phares ou comme des sentinelles avancées, face à l’armée des étoiles ennemies, qui campent dans leurs bivouacs célestes. Seuls avec les mécaniciens dans les infernales chaufferies des grands navires, seuls avec les noirs fantômes qui fourragent dans le ventre rouge des locomotives affolées, seuls avec les ivrognes battant des ailes contre les murs !\par
Et nous voilà brusquement distraits par le roulement des énormes tramways à double étage, qui passent sursautants, bariolés de lumières, tels les hameaux en fête que le Pô débordé ébranle tout à coup et déracine, pour les entrainer, sur les cascades et les remous d’un déluge, jusqu’à la mer.\par
Puis le silence s’aggrava. Comme nous écoutions la prière exténuée du vieux canal et crisser les os des palais moribonds dans leur barbe de verdure, soudain rugirent sous nos fenêtres les automobiles affamées.\par
– Allons, dis-je, mes amis ! Partons ! Enfin, la Mythologie et l’Idéal mystique sont surpassés. Nous allons assister à la naissance du Centaure et nous verrons bientôt voler les premiers anges ! – Il faudra ébranler les portes de la vie pour en essayer les gonds et les verrous ! Partons ! Voilà bien le premier soleil levant sur la terre !… Rien n’égale la splendeur de son épée rouge qui s’escrime pour la première fois dans nos ténèbres millénaires.\par
Nous nous approchâmes des trois machines renâclantes pour flatter leur poitrail. Je m’allongeai sur la mienne…\par
Le grand balai de la folie nous arracha à nous-mêmes et nous poussa à travers les rues escarpées et profondes comme des torrents desséchés. Çà et là, des lampes malheureuses, aux fenêtres, nous enseignaient à mépriser nos yeux mathématiques.\par
– Le flair, criai-je, le flair suffit aux fauves !…\par
Sortons de la Sagesse comme d’une gangue hideuse et entrons, comme des fruits pimentés d’orgueil, dans la bouche immense et torse du vent !… Donnons-nous à manger à l’Inconnu, non par désespoir, mais simplement pour enrichir les insondables réservoirs de l’Absurde !\par
Comme j’avais dit ces mots, je virai brusquement sur moi-même avec l’ivresse folle des caniches qui se mordent la queue, et voilà tout à coup que deux cyclistes me désapprouvèrent, titubant devant moi ainsi que deux raisonnements persuasifs et pourtant contradictoires. Leur ondoiement stupide discutait sur mon terrain… Quel ennui ! Pouah ! … Je coupai court et, par dégoût, je me flanquai dans un fossé…\par
Oh ! maternel fossé, à moitié plein d’une eau vaseuse ! Fossé d’usine ! J’ai savouré à pleine bouche la boue fortifiante !\par
Le visage masqué de la bonne boue des usines, pleine de scories de métal, de sueurs inutiles et de suie céleste, portant nos bras foulés en écharpe, parmi la complainte des sages pêcheurs à la ligne et des naturalistes navrés, nous dictâmes nos premières volontés à tous les hommes vivants de la terre :\par

\labelblock{Manifeste du Futurisme}

\noindent 1. Nous voulons chanter l’amour du danger, l’habitude de l’énergie et de la témérité.\par
2. Les éléments essentiels de notre poésie seront le courage, l’audace et la révolte.\par
3. La littérature ayant jusqu’ici magnifié l’immobilité pensive, l’extase et le sommeil, nous voulons exalter le mouvement agressif, l’insomnie fiévreuse, le pas gymnastique, le saut périlleux, la gifle et le coup de poing.\par
4. Nous déclarons que la splendeur du monde s’est enrichie d’une beauté nouvelle : la beauté de la vitesse. Une automobile de course avec son coffre orné de gros tuyaux, tels des serpents à l’haleine explosive… une automobile rugissante, qui a l’air de courir sur de la mitraille, est plus belle que la \emph{Victoire de Samothrace}.\par
5. Nous voulons chanter l’homme qui tient le volant, dont la tige idéale traverse la terre, lancée elle-même sur le circuit de son orbite.\par
6. Il faut que le poète se dépense avec chaleur, éclat et prodigalité, pour augmenter la ferveur enthousiaste des éléments primordiaux.\par
7. Il n’y a plus de beauté que dans la lutte. Pas de chef-d’œuvre sans un caractère agressif. La poésie doit être un assaut violent contre les forces inconnues, pour les sommer de se coucher devant l’homme.\par
8. Nous sommes sur le promontoire extrême des siècles !… A quoi bon regarder derrrière nous, du moment qu’il nous faut défoncer les vantaux mystérieux de l’impossible ? Le Temps et l’Espace sont morts hier. Nous vivons déjà dans l’absolu, puisque nous avons déjà créé l’éternelle vitesse omniprésente.\par
9. Nous voulons glorifier la guerre, — seule hygiène du monde, — le militarisme, le patriotisme, le geste destructeur des anarchistes, les belles Idées qui tuent et le mépris de la femme.\par
10. Nous voulons démolir les musées, les bibliothèques, combattre le moralisme, le féminisme et toutes les lâchetés opportunistes et utilitaires.\par
11. Nous chanterons les grandes foules agitées par le travail, le plaisir ou la révolte ; les ressacs multicolores et polyphoniques des révolutions dans les capitales modernes ; la vibration nocturne des arsenaux et des chantiers sous leurs violentes lunes électriques ; les gares gloutonnes avaleuses de serpents qui fument ; les usines suspendues aux nuages par les ficelles de leurs fumées ; les ponts aux bonds de gymnastes lancés sur la coutellerie diabolique des fleuves ensoleillés ; les paquebots aventureux flairant l’horizon ; les locomotives au grand poitrail qui piaffent sur les rails, tels d’énormes chevaux d’acier bridés de longs tuyaux et le vol glissant des aéroplanes, dont l’hélice a des claquements de drapeaux et des applaudissements de foule enthousiaste.\par
C’est en Italie que nous lançons ce manifeste de violence culbutante et incendiaire, par lequel nous fondons aujourd’hui le \emph{Futurisme}, parce que nous voulons délivrer l’Italie de sa gangrène de professeurs, d’archéologues, de cicérones et d’antiquaires.\par
L’Italie a été trop longtemps le marché des brocanteurs qui fournissaient au monde le mobilier de nos ancêtres, sans cesse renouvelé et soigneusement mitraillé pour simuler le travail des tarets vénérables. Nous voulons débarasser l’Italie des musées innombrables qui la couvrent d’innombrables cimetières.\par
Musées, cimetières !… Identiques vraiment dans leur sinistre coudoiement de corps qui ne se connaissent pas. Dortoirs publics où l’on dort à jamais côte à côte avec des êtres haïs ou inconnus. Férocité réciproque des peintres et des sculpteurs s’entre-tuant à coups de lignes et de couleurs dans le même musée.\par
Qu’on y fasse une visite chaque année comme on va voir ses morts une fois par an !… Nous pouvons bien l’admettre !… Qu’on dépose même des fleurs une fois par an aux pieds de la \emph{Joconde}, nous le concevons !… Mais que l’on aille promener quotidiennement dans les musées nos tristesses, nos courages fragiles et notre inquiétude, nous ne l’admettons pas !…\par
Admirer un vieux tableau, c’est verser notre sensibilité dans une urne funéraire au lieu de la lancer en avant par jets violents de création et d’action. Voulez-vous donc gâcher ainsi vos meilleures forces dans une admiration inutile du passé, dont vous sortez forcément épuisés, amoindris, piétinés ?\par
En vérité, la fréquentation quotidienne des musées, des bibliothèques et des académies (ces cimetières d’efforts perdus, ces calvaires de rêves crucifiés, ces registres d’élans brisés !…) est pour les artistes ce qu’est la tutelle prolongée des parents pour des jeunes gens intelligents, ivres de leur talent et de leur volonté ambitieuse.\par
Pour des moribonds, des invalides et des prisonniers, passe encore. C’est peut-être un baume à leurs blessures, que l’admirable passé, du moment que l’avenir leur est interdit … Mais nous n’en voulons pas, nous, les jeunes, les forts et les vivants \emph{futuristes !}\par
Viennent donc les bons incendiaires aux doigts carbonisés !… Les voici ! Les voici !… Et boutez donc le feu aux rayons des bibliothèques ! Détournez le cours des canaux pour inonder les caveaux des musées !… Oh ! qu’elles nagent à la dérive, les toiles glorieuses ! A vous les pioches et les marteaux !… sapez les fondements des villes vénérables.\par
Les plus âgés d’entre nous ont trente ans : nous avons donc au moins dix ans pour accomplir notre tâche. Quand nous aurons quarante ans, que de plus jeunes et plus vaillants que nous veuillent bien nous jeter au panier comme des manuscrits inutiles !… Ils viendront contre nous de très loin, de partout, en bondissant sur la cadence légère de leurs premiers poèmes, griffant l’air de leurs doigts crochus, et humant, aux portes des académies, la bonne odeur de nos esprits pourrissants déjà promis aux catacombes des bibliothèques.\par
Mais nous ne serons pas là. Ils nous trouveront enfin, par une nuit d’hiver, en pleine campagne, sous un triste hangar pianoté par la pluie monotone, accroupis près de nos aéroplanes trépidants, en train de chauffer nos mains sur le misérable feu que feront nos livres d’aujourd’hui flambant gaiement sous le vol étincelant de leurs images.\par
Ils s’ameuteront autour de nous, haletants d’angoisse et de dépit, et, tous, exaspérés par notre fier courage infatigable, s’élanceront pour nous tuer, avec d’autant plus de haine que leur cœur sera ivre d’amour et d’admiration pour nous. Et la forte et la saine Injustice éclatera radieusement dans leurs yeux. Car l’art ne peut être que violence, cruauté et injustice.\par
Les plus âgés d’entre nous n’ont pas encore trente ans, et pourtant nous avons déjà gaspillé des trésors, des trésors de force, d’amour, de courage et d’âpre volonté, à la hâte, en délire, sans compter, à tour de bras, à perdre haleine.\par
Regardez-nous ! Nous ne sommes pas essoufflés… Notre cœur n’a pas la moindre fatigue ! Car il s’est nourri de feu, de haine et de vitesse ! Cela vous étonne ? C’est que vous ne vous souvenez même pas d’avoir vécu ! — Debout sur la cime du monde, nous lançons encore une fois le défi aux étoiles !\par
Vos objections? Assez ! assez ! Je les connais ! C’est entendu ! Nous savons bien ce que notre belle et fausse intelligence nous affirme. – Nous ne sommes, dit-elle, que le résumé et le prolongement de nos ancêtres. — Peut-être ! soit !… Qu’importe ?… Mais nous ne voulons pas entendre ! Gardez-vous de répéter ces mots infâmes ! Levez plutôt la tête !\par
Debout sur la cime du monde, nous lançons encore une fois le défi insolent aux étoiles !\par


\signed{F.-T. Marinetti.}
 


% at least one empty page at end (for booklet couv)
\ifbooklet
  \pagestyle{empty}
  \clearpage
  % 2 empty pages maybe needed for 4e cover
  \ifnum\modulo{\value{page}}{4}=0 \hbox{}\newpage\hbox{}\newpage\fi
  \ifnum\modulo{\value{page}}{4}=1 \hbox{}\newpage\hbox{}\newpage\fi


  \hbox{}\newpage
  \ifodd\value{page}\hbox{}\newpage\fi
  {\centering\color{rubric}\bfseries\noindent\large
    Hurlus ? Qu’est-ce.\par
    \bigskip
  }
  \noindent Des bouquinistes électroniques, pour du texte libre à participation libre,
  téléchargeable gratuitement sur \href{https://hurlus.fr}{\dotuline{hurlus.fr}}.\par
  \bigskip
  \noindent Cette brochure a été produite par des éditeurs bénévoles.
  Elle n’est pas faîte pour être possédée, mais pour être lue, et puis donnée.
  Que circule le texte !
  En page de garde, on peut ajouter une date, un lieu, un nom ; pour suivre le voyage des idées.
  \par

  Ce texte a été choisi parce qu’une personne l’a aimé,
  ou haï, elle a en tous cas pensé qu’il partipait à la formation de notre présent ;
  sans le souci de plaire, vendre, ou militer pour une cause.
  \par

  L’édition électronique est soigneuse, tant sur la technique
  que sur l’établissement du texte ; mais sans aucune prétention scolaire, au contraire.
  Le but est de s’adresser à tous, sans distinction de science ou de diplôme.
  Au plus direct ! (possible)
  \par

  Cet exemplaire en papier a été tiré sur une imprimante personnelle
   ou une photocopieuse. Tout le monde peut le faire.
  Il suffit de
  télécharger un fichier sur \href{https://hurlus.fr}{\dotuline{hurlus.fr}},
  d’imprimer, et agrafer ; puis de lire et donner.\par

  \bigskip

  \noindent PS : Les hurlus furent aussi des rebelles protestants qui cassaient les statues dans les églises catholiques. En 1566 démarra la révolte des gueux dans le pays de Lille. L’insurrection enflamma la région jusqu’à Anvers où les gueux de mer bloquèrent les bateaux espagnols.
  Ce fut une rare guerre de libération dont naquit un pays toujours libre : les Pays-Bas.
  En plat pays francophone, par contre, restèrent des bandes de huguenots, les hurlus, progressivement réprimés par la très catholique Espagne.
  Cette mémoire d’une défaite est éteinte, rallumons-la. Sortons les livres du culte universitaire, cherchons les idoles de l’époque, pour les briser.
\fi

\ifdev % autotext in dev mode
\fontname\font — \textsc{Les règles du jeu}\par
(\hyperref[utopie]{\underline{Lien}})\par
\noindent \initialiv{A}{lors là}\blindtext\par
\noindent \initialiv{À}{ la bonheur des dames}\blindtext\par
\noindent \initialiv{É}{tonnez-le}\blindtext\par
\noindent \initialiv{Q}{ualitativement}\blindtext\par
\noindent \initialiv{V}{aloriser}\blindtext\par
\Blindtext
\phantomsection
\label{utopie}
\Blinddocument
\fi
\end{document}
