%%%%%%%%%%%%%%%%%%%%%%%%%%%%%%%%%
% LaTeX model https://hurlus.fr %
%%%%%%%%%%%%%%%%%%%%%%%%%%%%%%%%%

% Needed before document class
\RequirePackage{pdftexcmds} % needed for tests expressions
\RequirePackage{fix-cm} % correct units

% Define mode
\def\mode{a4}

\newif\ifaiv % a4
\newif\ifav % a5
\newif\ifbooklet % booklet
\newif\ifcover % cover for booklet

\ifnum \strcmp{\mode}{cover}=0
  \covertrue
\else\ifnum \strcmp{\mode}{booklet}=0
  \booklettrue
\else\ifnum \strcmp{\mode}{a5}=0
  \avtrue
\else
  \aivtrue
\fi\fi\fi

\ifbooklet % do not enclose with {}
  \documentclass[french,twoside]{book} % ,notitlepage
  \usepackage[%
    papersize={105mm, 297mm},
    inner=12mm,
    outer=12mm,
    top=20mm,
    bottom=15mm,
    marginparsep=0pt,
  ]{geometry}
  \usepackage[fontsize=9.5pt]{scrextend} % for Roboto
\else\ifav
  \documentclass[french,twoside]{book} % ,notitlepage
  \usepackage[%
    a5paper,
    inner=25mm,
    outer=15mm,
    top=15mm,
    bottom=15mm,
    marginparsep=0pt,
  ]{geometry}
  \usepackage[fontsize=12pt]{scrextend}
\else% A4 2 cols
  \documentclass[twocolumn]{report}
  \usepackage[%
    a4paper,
    inner=15mm,
    outer=10mm,
    top=25mm,
    bottom=18mm,
    marginparsep=0pt,
  ]{geometry}
  \setlength{\columnsep}{20mm}
  \usepackage[fontsize=9.5pt]{scrextend}
\fi\fi

%%%%%%%%%%%%%%
% Alignments %
%%%%%%%%%%%%%%
% before teinte macros

\setlength{\arrayrulewidth}{0.2pt}
\setlength{\columnseprule}{\arrayrulewidth} % twocol
\setlength{\parskip}{0pt} % classical para with no margin
\setlength{\parindent}{1.5em}

%%%%%%%%%%
% Colors %
%%%%%%%%%%
% before Teinte macros

\usepackage[dvipsnames]{xcolor}
\definecolor{rubric}{HTML}{800000} % the tonic 0c71c3
\def\columnseprulecolor{\color{rubric}}
\colorlet{borderline}{rubric!30!} % definecolor need exact code
\definecolor{shadecolor}{gray}{0.95}
\definecolor{bghi}{gray}{0.5}

%%%%%%%%%%%%%%%%%
% Teinte macros %
%%%%%%%%%%%%%%%%%
%%%%%%%%%%%%%%%%%%%%%%%%%%%%%%%%%%%%%%%%%%%%%%%%%%%
% <TEI> generic (LaTeX names generated by Teinte) %
%%%%%%%%%%%%%%%%%%%%%%%%%%%%%%%%%%%%%%%%%%%%%%%%%%%
% This template is inserted in a specific design
% It is XeLaTeX and otf fonts

\makeatletter % <@@@


\usepackage{blindtext} % generate text for testing
\usepackage[strict]{changepage} % for modulo 4
\usepackage{contour} % rounding words
\usepackage[nodayofweek]{datetime}
% \usepackage{DejaVuSans} % seems buggy for sffont font for symbols
\usepackage{enumitem} % <list>
\usepackage{etoolbox} % patch commands
\usepackage{fancyvrb}
\usepackage{fancyhdr}
\usepackage{float}
\usepackage{fontspec} % XeLaTeX mandatory for fonts
\usepackage{footnote} % used to capture notes in minipage (ex: quote)
\usepackage{framed} % bordering correct with footnote hack
\usepackage{graphicx}
\usepackage{lettrine} % drop caps
\usepackage{lipsum} % generate text for testing
\usepackage[framemethod=tikz,]{mdframed} % maybe used for frame with footnotes inside
\usepackage{pdftexcmds} % needed for tests expressions
\usepackage{polyglossia} % non-break space french punct, bug Warning: "Failed to patch part"
\usepackage[%
  indentfirst=false,
  vskip=1em,
  noorphanfirst=true,
  noorphanafter=true,
  leftmargin=\parindent,
  rightmargin=0pt,
]{quoting}
\usepackage{ragged2e}
\usepackage{setspace} % \setstretch for <quote>
\usepackage{tabularx} % <table>
\usepackage[explicit]{titlesec} % wear titles, !NO implicit
\usepackage{tikz} % ornaments
\usepackage{tocloft} % styling tocs
\usepackage[fit]{truncate} % used im runing titles
\usepackage{unicode-math}
\usepackage[normalem]{ulem} % breakable \uline, normalem is absolutely necessary to keep \emph
\usepackage{verse} % <l>
\usepackage{xcolor} % named colors
\usepackage{xparse} % @ifundefined
\XeTeXdefaultencoding "iso-8859-1" % bad encoding of xstring
\usepackage{xstring} % string tests
\XeTeXdefaultencoding "utf-8"
\PassOptionsToPackage{hyphens}{url} % before hyperref, which load url package

% TOTEST
% \usepackage{hypcap} % links in caption ?
% \usepackage{marginnote}
% TESTED
% \usepackage{background} % doesn’t work with xetek
% \usepackage{bookmark} % prefers the hyperref hack \phantomsection
% \usepackage[color, leftbars]{changebar} % 2 cols doc, impossible to keep bar left
% \usepackage[utf8x]{inputenc} % inputenc package ignored with utf8 based engines
% \usepackage[sfdefault,medium]{inter} % no small caps
% \usepackage{firamath} % choose firasans instead, firamath unavailable in Ubuntu 21-04
% \usepackage{flushend} % bad for last notes, supposed flush end of columns
% \usepackage[stable]{footmisc} % BAD for complex notes https://texfaq.org/FAQ-ftnsect
% \usepackage{helvet} % not for XeLaTeX
% \usepackage{multicol} % not compatible with too much packages (longtable, framed, memoir…)
% \usepackage[default,oldstyle,scale=0.95]{opensans} % no small caps
% \usepackage{sectsty} % \chapterfont OBSOLETE
% \usepackage{soul} % \ul for underline, OBSOLETE with XeTeX
% \usepackage[breakable]{tcolorbox} % text styling gone, footnote hack not kept with breakable


% Metadata inserted by a program, from the TEI source, for title page and runing heads
\title{\textbf{ La guerre des Gaules }}
\date{1865}
\author{César, Jules}
\def\elbibl{César, Jules. 1865. \emph{La guerre des Gaules}}
\def\elsource{ \href{http://fr.wikisource.org/wiki/La_Guerre_des_Gaules}{\dotuline{http://fr.wikisource.org/wiki/La\_Guerre\_des\_Gaules}}\footnote{\href{http://fr.wikisource.org/wiki/La_Guerre_des_Gaules}{\url{http://fr.wikisource.org/wiki/La_Guerre_des_Gaules}}} }

% Default metas
\newcommand{\colorprovide}[2]{\@ifundefinedcolor{#1}{\colorlet{#1}{#2}}{}}
\colorprovide{rubric}{red}
\colorprovide{silver}{lightgray}
\@ifundefined{syms}{\newfontfamily\syms{DejaVu Sans}}{}
\newif\ifdev
\@ifundefined{elbibl}{% No meta defined, maybe dev mode
  \newcommand{\elbibl}{Titre court ?}
  \newcommand{\elbook}{Titre du livre source ?}
  \newcommand{\elabstract}{Résumé\par}
  \newcommand{\elurl}{http://oeuvres.github.io/elbook/2}
  \author{Éric Lœchien}
  \title{Un titre de test assez long pour vérifier le comportement d’une maquette}
  \date{1566}
  \devtrue
}{}
\let\eltitle\@title
\let\elauthor\@author
\let\eldate\@date


\defaultfontfeatures{
  % Mapping=tex-text, % no effect seen
  Scale=MatchLowercase,
  Ligatures={TeX,Common},
}


% generic typo commands
\newcommand{\astermono}{\medskip\centerline{\color{rubric}\large\selectfont{\syms ✻}}\medskip\par}%
\newcommand{\astertri}{\medskip\par\centerline{\color{rubric}\large\selectfont{\syms ✻\,✻\,✻}}\medskip\par}%
\newcommand{\asterism}{\bigskip\par\noindent\parbox{\linewidth}{\centering\color{rubric}\large{\syms ✻}\\{\syms ✻}\hskip 0.75em{\syms ✻}}\bigskip\par}%

% lists
\newlength{\listmod}
\setlength{\listmod}{\parindent}
\setlist{
  itemindent=!,
  listparindent=\listmod,
  labelsep=0.2\listmod,
  parsep=0pt,
  % topsep=0.2em, % default topsep is best
}
\setlist[itemize]{
  label=—,
  leftmargin=0pt,
  labelindent=1.2em,
  labelwidth=0pt,
}
\setlist[enumerate]{
  label={\bf\color{rubric}\arabic*.},
  labelindent=0.8\listmod,
  leftmargin=\listmod,
  labelwidth=0pt,
}
\newlist{listalpha}{enumerate}{1}
\setlist[listalpha]{
  label={\bf\color{rubric}\alph*.},
  leftmargin=0pt,
  labelindent=0.8\listmod,
  labelwidth=0pt,
}
\newcommand{\listhead}[1]{\hspace{-1\listmod}\emph{#1}}

\renewcommand{\hrulefill}{%
  \leavevmode\leaders\hrule height 0.2pt\hfill\kern\z@}

% General typo
\DeclareTextFontCommand{\textlarge}{\large}
\DeclareTextFontCommand{\textsmall}{\small}

% commands, inlines
\newcommand{\anchor}[1]{\Hy@raisedlink{\hypertarget{#1}{}}} % link to top of an anchor (not baseline)
\newcommand\abbr[1]{#1}
\newcommand{\autour}[1]{\tikz[baseline=(X.base)]\node [draw=rubric,thin,rectangle,inner sep=1.5pt, rounded corners=3pt] (X) {\color{rubric}#1};}
\newcommand\corr[1]{#1}
\newcommand{\ed}[1]{ {\color{silver}\sffamily\footnotesize (#1)} } % <milestone ed="1688"/>
\newcommand\expan[1]{#1}
\newcommand\foreign[1]{\emph{#1}}
\newcommand\gap[1]{#1}
\renewcommand{\LettrineFontHook}{\color{rubric}}
\newcommand{\initial}[2]{\lettrine[lines=2, loversize=0.3, lhang=0.3]{#1}{#2}}
\newcommand{\initialiv}[2]{%
  \let\oldLFH\LettrineFontHook
  % \renewcommand{\LettrineFontHook}{\color{rubric}\ttfamily}
  \IfSubStr{QJ’}{#1}{
    \lettrine[lines=4, lhang=0.2, loversize=-0.1, lraise=0.2]{\smash{#1}}{#2}
  }{\IfSubStr{É}{#1}{
    \lettrine[lines=4, lhang=0.2, loversize=-0, lraise=0]{\smash{#1}}{#2}
  }{\IfSubStr{ÀÂ}{#1}{
    \lettrine[lines=4, lhang=0.2, loversize=-0, lraise=0, slope=0.6em]{\smash{#1}}{#2}
  }{\IfSubStr{A}{#1}{
    \lettrine[lines=4, lhang=0.2, loversize=0.2, slope=0.6em]{\smash{#1}}{#2}
  }{\IfSubStr{V}{#1}{
    \lettrine[lines=4, lhang=0.2, loversize=0.2, slope=-0.5em]{\smash{#1}}{#2}
  }{
    \lettrine[lines=4, lhang=0.2, loversize=0.2]{\smash{#1}}{#2}
  }}}}}
  \let\LettrineFontHook\oldLFH
}
\newcommand{\labelchar}[1]{\textbf{\color{rubric} #1}}
\newcommand{\milestone}[1]{\autour{\footnotesize\color{rubric} #1}} % <milestone n="4"/>
\newcommand\name[1]{#1}
\newcommand\orig[1]{#1}
\newcommand\orgName[1]{#1}
\newcommand\persName[1]{#1}
\newcommand\placeName[1]{#1}
\newcommand{\pn}[1]{\IfSubStr{-—–¶}{#1}% <p n="3"/>
  {\noindent{\bfseries\color{rubric}   ¶  }}
  {{\footnotesize\autour{ #1}  }}}
\newcommand\reg{}
% \newcommand\ref{} % already defined
\newcommand\sic[1]{#1}
\newcommand\surname[1]{\textsc{#1}}
\newcommand\term[1]{\textbf{#1}}

\def\mednobreak{\ifdim\lastskip<\medskipamount
  \removelastskip\nopagebreak\medskip\fi}
\def\bignobreak{\ifdim\lastskip<\bigskipamount
  \removelastskip\nopagebreak\bigskip\fi}

% commands, blocks
\newcommand{\byline}[1]{\bigskip{\RaggedLeft{#1}\par}\bigskip}
\newcommand{\bibl}[1]{{\RaggedLeft{#1}\par\bigskip}}
\newcommand{\biblitem}[1]{{\noindent\hangindent=\parindent   #1\par}}
\newcommand{\dateline}[1]{\medskip{\RaggedLeft{#1}\par}\bigskip}
\newcommand{\labelblock}[1]{\medbreak{\noindent\color{rubric}\bfseries #1}\par\mednobreak}
\newcommand{\salute}[1]{\bigbreak{#1}\par\medbreak}
\newcommand{\signed}[1]{\bigbreak\filbreak{\raggedleft #1\par}\medskip}

% environments for blocks (some may become commands)
\newenvironment{borderbox}{}{} % framing content
\newenvironment{citbibl}{\ifvmode\hfill\fi}{\ifvmode\par\fi }
\newenvironment{docAuthor}{\ifvmode\vskip4pt\fontsize{16pt}{18pt}\selectfont\fi\itshape}{\ifvmode\par\fi }
\newenvironment{docDate}{}{\ifvmode\par\fi }
\newenvironment{docImprint}{\vskip6pt}{\ifvmode\par\fi }
\newenvironment{docTitle}{\vskip6pt\bfseries\fontsize{18pt}{22pt}\selectfont}{\par }
\newenvironment{msHead}{\vskip6pt}{\par}
\newenvironment{msItem}{\vskip6pt}{\par}
\newenvironment{titlePart}{}{\par }


% environments for block containers
\newenvironment{argument}{\itshape\parindent0pt}{\vskip1.5em}
\newenvironment{biblfree}{}{\ifvmode\par\fi }
\newenvironment{bibitemlist}[1]{%
  \list{\@biblabel{\@arabic\c@enumiv}}%
  {%
    \settowidth\labelwidth{\@biblabel{#1}}%
    \leftmargin\labelwidth
    \advance\leftmargin\labelsep
    \@openbib@code
    \usecounter{enumiv}%
    \let\p@enumiv\@empty
    \renewcommand\theenumiv{\@arabic\c@enumiv}%
  }
  \sloppy
  \clubpenalty4000
  \@clubpenalty \clubpenalty
  \widowpenalty4000%
  \sfcode`\.\@m
}%
{\def\@noitemerr
  {\@latex@warning{Empty `bibitemlist' environment}}%
\endlist}
\newenvironment{quoteblock}% may be used for ornaments
  {\begin{quoting}}
  {\end{quoting}}

% table () is preceded and finished by custom command
\newcommand{\tableopen}[1]{%
  \ifnum\strcmp{#1}{wide}=0{%
    \begin{center}
  }
  \else\ifnum\strcmp{#1}{long}=0{%
    \begin{center}
  }
  \else{%
    \begin{center}
  }
  \fi\fi
}
\newcommand{\tableclose}[1]{%
  \ifnum\strcmp{#1}{wide}=0{%
    \end{center}
  }
  \else\ifnum\strcmp{#1}{long}=0{%
    \end{center}
  }
  \else{%
    \end{center}
  }
  \fi\fi
}


% text structure
\newcommand\chapteropen{} % before chapter title
\newcommand\chaptercont{} % after title, argument, epigraph…
\newcommand\chapterclose{} % maybe useful for multicol settings
\setcounter{secnumdepth}{-2} % no counters for hierarchy titles
\setcounter{tocdepth}{5} % deep toc
\markright{\@title} % ???
\markboth{\@title}{\@author} % ???
\renewcommand\tableofcontents{\@starttoc{toc}}
% toclof format
% \renewcommand{\@tocrmarg}{0.1em} % Useless command?
% \renewcommand{\@pnumwidth}{0.5em} % {1.75em}
\renewcommand{\@cftmaketoctitle}{}
\setlength{\cftbeforesecskip}{\z@ \@plus.2\p@}
\renewcommand{\cftchapfont}{}
\renewcommand{\cftchapdotsep}{\cftdotsep}
\renewcommand{\cftchapleader}{\normalfont\cftdotfill{\cftchapdotsep}}
\renewcommand{\cftchappagefont}{\bfseries}
\setlength{\cftbeforechapskip}{0em \@plus\p@}
% \renewcommand{\cftsecfont}{\small\relax}
\renewcommand{\cftsecpagefont}{\normalfont}
% \renewcommand{\cftsubsecfont}{\small\relax}
\renewcommand{\cftsecdotsep}{\cftdotsep}
\renewcommand{\cftsecpagefont}{\normalfont}
\renewcommand{\cftsecleader}{\normalfont\cftdotfill{\cftsecdotsep}}
\setlength{\cftsecindent}{1em}
\setlength{\cftsubsecindent}{2em}
\setlength{\cftsubsubsecindent}{3em}
\setlength{\cftchapnumwidth}{1em}
\setlength{\cftsecnumwidth}{1em}
\setlength{\cftsubsecnumwidth}{1em}
\setlength{\cftsubsubsecnumwidth}{1em}

% footnotes
\newif\ifheading
\newcommand*{\fnmarkscale}{\ifheading 0.70 \else 1 \fi}
\renewcommand\footnoterule{\vspace*{0.3cm}\hrule height \arrayrulewidth width 3cm \vspace*{0.3cm}}
\setlength\footnotesep{1.5\footnotesep} % footnote separator
\renewcommand\@makefntext[1]{\parindent 1.5em \noindent \hb@xt@1.8em{\hss{\normalfont\@thefnmark . }}#1} % no superscipt in foot
\patchcmd{\@footnotetext}{\footnotesize}{\footnotesize\sffamily}{}{} % before scrextend, hyperref


%   see https://tex.stackexchange.com/a/34449/5049
\def\truncdiv#1#2{((#1-(#2-1)/2)/#2)}
\def\moduloop#1#2{(#1-\truncdiv{#1}{#2}*#2)}
\def\modulo#1#2{\number\numexpr\moduloop{#1}{#2}\relax}

% orphans and widows
\clubpenalty=9996
\widowpenalty=9999
\brokenpenalty=4991
\predisplaypenalty=10000
\postdisplaypenalty=1549
\displaywidowpenalty=1602
\hyphenpenalty=400
% Copied from Rahtz but not understood
\def\@pnumwidth{1.55em}
\def\@tocrmarg {2.55em}
\def\@dotsep{4.5}
\emergencystretch 3em
\hbadness=4000
\pretolerance=750
\tolerance=2000
\vbadness=4000
\def\Gin@extensions{.pdf,.png,.jpg,.mps,.tif}
% \renewcommand{\@cite}[1]{#1} % biblio

\usepackage{hyperref} % supposed to be the last one, :o) except for the ones to follow
\urlstyle{same} % after hyperref
\hypersetup{
  % pdftex, % no effect
  pdftitle={\elbibl},
  % pdfauthor={Your name here},
  % pdfsubject={Your subject here},
  % pdfkeywords={keyword1, keyword2},
  bookmarksnumbered=true,
  bookmarksopen=true,
  bookmarksopenlevel=1,
  pdfstartview=Fit,
  breaklinks=true, % avoid long links
  pdfpagemode=UseOutlines,    % pdf toc
  hyperfootnotes=true,
  colorlinks=false,
  pdfborder=0 0 0,
  % pdfpagelayout=TwoPageRight,
  % linktocpage=true, % NO, toc, link only on page no
}

\makeatother % /@@@>
%%%%%%%%%%%%%%
% </TEI> end %
%%%%%%%%%%%%%%


%%%%%%%%%%%%%
% footnotes %
%%%%%%%%%%%%%
\renewcommand{\thefootnote}{\bfseries\textcolor{rubric}{\arabic{footnote}}} % color for footnote marks

%%%%%%%%%
% Fonts %
%%%%%%%%%
\usepackage[]{roboto} % SmallCaps, Regular is a bit bold
% \linespread{0.90} % too compact, keep font natural
\newfontfamily\fontrun[]{Roboto Condensed Light} % condensed runing heads
\ifav
  \setmainfont[
    ItalicFont={Roboto Light Italic},
  ]{Roboto}
\else\ifbooklet
  \setmainfont[
    ItalicFont={Roboto Light Italic},
  ]{Roboto}
\else
\setmainfont[
  ItalicFont={Roboto Italic},
]{Roboto Light}
\fi\fi
\renewcommand{\LettrineFontHook}{\bfseries\color{rubric}}
% \renewenvironment{labelblock}{\begin{center}\bfseries\color{rubric}}{\end{center}}

%%%%%%%%
% MISC %
%%%%%%%%

\setdefaultlanguage[frenchpart=false]{french} % bug on part


\newenvironment{quotebar}{%
    \def\FrameCommand{{\color{rubric!10!}\vrule width 0.5em} \hspace{0.9em}}%
    \def\OuterFrameSep{\itemsep} % séparateur vertical
    \MakeFramed {\advance\hsize-\width \FrameRestore}
  }%
  {%
    \endMakeFramed
  }
\renewenvironment{quoteblock}% may be used for ornaments
  {%
    \savenotes
    \setstretch{0.9}
    \normalfont
    \begin{quotebar}
  }
  {%
    \end{quotebar}
    \spewnotes
  }


\renewcommand{\headrulewidth}{\arrayrulewidth}
\renewcommand{\headrule}{{\color{rubric}\hrule}}

% delicate tuning, image has produce line-height problems in title on 2 lines
\titleformat{name=\chapter} % command
  [display] % shape
  {\vspace{1.5em}\centering} % format
  {} % label
  {0pt} % separator between n
  {}
[{\color{rubric}\huge\textbf{#1}}\bigskip] % after code
% \titlespacing{command}{left spacing}{before spacing}{after spacing}[right]
\titlespacing*{\chapter}{0pt}{-2em}{0pt}[0pt]

\titleformat{name=\section}
  [block]{}{}{}{}
  [\vbox{\color{rubric}\large\raggedleft\textbf{#1}}]
\titlespacing{\section}{0pt}{0pt plus 4pt minus 2pt}{\baselineskip}

\titleformat{name=\subsection}
  [block]
  {}
  {} % \thesection
  {} % separator \arrayrulewidth
  {}
[\vbox{\large\textbf{#1}}]
% \titlespacing{\subsection}{0pt}{0pt plus 4pt minus 2pt}{\baselineskip}

\ifaiv
  \fancypagestyle{main}{%
    \fancyhf{}
    \setlength{\headheight}{1.5em}
    \fancyhead{} % reset head
    \fancyfoot{} % reset foot
    \fancyhead[L]{\truncate{0.45\headwidth}{\fontrun\elbibl}} % book ref
    \fancyhead[R]{\truncate{0.45\headwidth}{ \fontrun\nouppercase\leftmark}} % Chapter title
    \fancyhead[C]{\thepage}
  }
  \fancypagestyle{plain}{% apply to chapter
    \fancyhf{}% clear all header and footer fields
    \setlength{\headheight}{1.5em}
    \fancyhead[L]{\truncate{0.9\headwidth}{\fontrun\elbibl}}
    \fancyhead[R]{\thepage}
  }
\else
  \fancypagestyle{main}{%
    \fancyhf{}
    \setlength{\headheight}{1.5em}
    \fancyhead{} % reset head
    \fancyfoot{} % reset foot
    \fancyhead[RE]{\truncate{0.9\headwidth}{\fontrun\elbibl}} % book ref
    \fancyhead[LO]{\truncate{0.9\headwidth}{\fontrun\nouppercase\leftmark}} % Chapter title, \nouppercase needed
    \fancyhead[RO,LE]{\thepage}
  }
  \fancypagestyle{plain}{% apply to chapter
    \fancyhf{}% clear all header and footer fields
    \setlength{\headheight}{1.5em}
    \fancyhead[L]{\truncate{0.9\headwidth}{\fontrun\elbibl}}
    \fancyhead[R]{\thepage}
  }
\fi

\ifav % a5 only
  \titleclass{\section}{top}
\fi

\newcommand\chapo{{%
  \vspace*{-3em}
  \centering % no vskip ()
  {\Large\addfontfeature{LetterSpace=25}\bfseries{\elauthor}}\par
  \smallskip
  {\large\eldate}\par
  \bigskip
  {\Large\selectfont{\eltitle}}\par
  \bigskip
  {\color{rubric}\hline\par}
  \bigskip
  {\Large TEXTE LIBRE À PARTICPATION LIBRE\par}
  \centerline{\small\color{rubric} {hurlus.fr, tiré le \today}}\par
  \bigskip
}}

\newcommand\cover{{%
  \thispagestyle{empty}
  \centering
  {\LARGE\bfseries{\elauthor}}\par
  \bigskip
  {\Large\eldate}\par
  \bigskip
  \bigskip
  {\LARGE\selectfont{\eltitle}}\par
  \vfill\null
  {\color{rubric}\setlength{\arrayrulewidth}{2pt}\hline\par}
  \vfill\null
  {\Large TEXTE LIBRE À PARTICPATION LIBRE\par}
  \centerline{{\href{https://hurlus.fr}{\dotuline{hurlus.fr}}, tiré le \today}}\par
}}

\begin{document}
\pagestyle{empty}
\ifbooklet{
  \cover\newpage
  \thispagestyle{empty}\hbox{}\newpage
  \cover\newpage\noindent Les voyages de la brochure\par
  \bigskip
  \begin{tabularx}{\textwidth}{l|X|X}
    \textbf{Date} & \textbf{Lieu}& \textbf{Nom/pseudo} \\ \hline
    \rule{0pt}{25cm} &  &   \\
  \end{tabularx}
  \newpage
  \addtocounter{page}{-4}
}\fi

\thispagestyle{empty}
\ifaiv
  \twocolumn[\chapo]
\else
  \chapo
\fi
{\it\elabstract}
\bigskip
\makeatletter\@starttoc{toc}\makeatother % toc without new page
\bigskip

\pagestyle{main} % after style

  \chapter[{Guerre des Gaules (trad. Nisard 1865)}]{\emph{Guerre des Gaules} (trad. Nisard 1865)}
\section[{Livre I}]{Livre I}\renewcommand{\leftmark}{Livre I}

\subsection[{§ 1.}]{ \textsc{§ 1.} }
\noindent (1) Toute la Gaule est divisée en trois parties, dont l’une est habitée par les Belges, l’autre par les Aquitains, la troisième par ceux qui, dans leur langue, se nomment Celtes, et dans la nôtre, Gaulois. (2) Ces nations diffèrent entre elles par le langage, les institutions et les lois. Les Gaulois sont séparés des Aquitains par la Garonne, des Belges par la Marne et la Seine. (3) Les Belges sont les plus braves de tous ces peuples, parce qu’ils restent tout à fait étrangers à la politesse et à la civilisation de la province romaine, et que les marchands, allant rarement chez eux, ne leur portent point ce qui contribue à énerver le courage : d’ailleurs, voisins des Germains qui habitent au-delà du Rhin, ils sont continuellement en guerre avec eux. (4) Par la même raison, les Helvètes surpassent aussi en valeur les autres Gaulois ; car ils engagent contre les Germains des luttes presque journalières, soit qu’ils les repoussent de leur propre territoire, soit qu’ils envahissent celui de leurs ennemis. (5) Le pays habité, comme nous l’avons dit, par les Gaulois, commence au Rhône, et est borné par la Garonne, l’Océan et les frontières des Belges ; du côté des Séquanes et des Helvètes, il va jusqu’au Rhin ; il est situé au nord. (6) Celui des Belges commence à l’extrême frontière de la Gaule, et est borné par la partie inférieure du Rhin ; il regarde le nord et l’orient. L'Aquitaine s’étend de la Garonne aux Pyrénées, et à cette partie de l’Océan qui baigne les côtes d’Espagne ; elle est entre le couchant et le nord.
\subsection[{§ 2.}]{ \textsc{§ 2.} }
\noindent (1) Orgétorix était, chez les Helvètes, le premier par sa naissance et par ses richesses. Sous le consulat de M. Messala et de M. Pison, cet homme, poussé par l’ambition, conjura avec la noblesse et engagea les habitants à sortir du pays avec toutes leurs forces ; (2) il leur dit que, l’emportant par le courage sur tous les peuples de la Gaule, ils la soumettraient aisément tout entière à leur empire. (3) Il eut d’autant moins de peine à les persuader que les Helvètes sont de toutes parts resserrés par la nature des lieux ; d’un côté par le Rhin, fleuve très large et très profond, qui sépare leur territoire de la Germanie, d’un autre par le Jura, haute montagne qui s’élève entre la Séquanie et l’Helvétie ; d’un troisième côté, par le lac Léman et le Rhône qui sépare cette dernière de notre Province. (4) Il résultait de cette position qu’ils ne pouvaient ni s’étendre au loin, ni porter facilement la guerre chez leurs voisins ; et c’était une cause de vive affliction pour des hommes belliqueux. (5) Leur population nombreuse, et la gloire qu’ils acquéraient dans la guerre par leur courage, leur faisaient regarder comme étroites des limites qui avaient deux cent quarante milles de long sur cent quatre-vingts milles de large.
\subsection[{§ 3.}]{ \textsc{§ 3.} }
\noindent (1) Poussés par ces motifs et entraînés par l’ascendant d’Orgétorix, ils commencent à tout disposer pour le départ, rassemblent un grand nombre de bêtes de somme et de chariots, ensemencent toutes leurs terres, afin de s’assurer des vivres dans leur marche et renouvellent avec leurs voisins les traités de paix et d’alliance. (2) Ils pensèrent que deux ans leur suffiraient pour ces préparatifs ; et une loi fixa le départ à la troisième année. (3) Orgétorix est choisi pour présider à l’entreprise. Envoyé en qualité de député vers les cités voisines, (4) sur sa route, il engage le Séquanais Casticos, fils de Catamantaloédis, et dont le père avait longtemps régné en Séquanie et avait reçu du peuple romain le titre d’ami, à reprendre sur ses concitoyens l’autorité suprême, précédemment exercée par son père. (5) Il inspire le même dessein à l’Héduen Dumnorix, frère de Diviciacos, qui tenait alors le premier rang dans la cité et était très aimé du peuple ; il lui donne sa fille en mariage. (6) Il leur démontre la facilité du succès de leurs efforts ; devant lui-même s’emparer du pouvoir chez les Helvètes, (7) et ce peuple étant le plus considérable de toute la Gaule, il les aidera de ses forces et de son armée pour leur assurer l’autorité souveraine. (8) Persuadés par ces discours, ils se lient sous la foi du serment : ils espéraient qu’une fois maîtres du pouvoir, au moyen de cette ligue des trois peuples les plus puissants et les plus braves, ils soumettraient la Gaule entière.
\subsection[{§ 4.}]{ \textsc{§ 4.} }
\noindent (1) Ce projet fut dénoncé aux Helvètes ; et, selon leurs coutumes, Orgétorix fut mis dans les fers pour répondre à l’accusation. Le supplice du condamné devait être celui du feu. (2) Au jour fixé pour le procès, Orgétorix fit paraître au tribunal tous ceux qui lui étaient attachés, au nombre de dix mille hommes ; il y réunit aussi tous ses clients et ses débiteurs dont la foule était grande : secondé par eux, il put se soustraire au jugement. (3) Les citoyens, indignés de cette conduite, voulaient maintenir leur droit par les armes, et les magistrats rassemblaient la population des campagnes, lorsque Orgétorix mourut. (4) Il y a lieu de penser, selon l’opinion des Helvètes, qu’il se donna lui-même la mort.
\subsection[{§ 5.}]{ \textsc{§ 5.} }
\noindent (1) Cet événement ne ralentit pas l’ardeur des Helvètes pour l’exécution de leur projet d’invasion. (2) Lorsqu’ils se croient suffisamment préparés, ils incendient toutes leurs villes au nombre de douze, leurs bourgs au nombre de quatre cents et toutes les habitations particulières ; (3) ils brûlent tout le blé qu’ils ne peuvent emporter, afin que, ne conservant aucun espoir de retour, ils s’offrent plus hardiment aux périls. Chacun reçoit l’ordre de se pourvoir de vivres pour trois mois. (4) Ils persuadent aux Rauraques, aux Tulinges et aux Latobices, leurs voisins, de livrer aux flammes leurs villes et leurs bourgs, et de partir avec eux. Ils associent à leur projet et s’adjoignent les Boïens qui s’étaient établis au-delà du Rhin, dans le Norique, après avoir pris Noréia.
\subsection[{§ 6.}]{ \textsc{§ 6.} }
\noindent (1) Il n’y avait absolument que deux chemins par lesquels ils pussent sortir de leur pays : l’un par la Séquanie, étroit et difficile, entre le Jura et le Rhône, où pouvait à peine passer un chariot ; il était dominé par une haute montagne, et une faible troupe suffisait pour en défendre l’entrée ; (2) l’autre, à travers notre Province, plus aisé et plus court, en ce que le Rhône, qui sépare les terres des Helvètes de celles des Allobroges, nouvellement soumis, est guéable en plusieurs endroits, (3) et que la dernière ville des Allobroges, Genève, est la plus rapprochée de l’Helvétie, avec laquelle elle communique par un pont. Ils crurent qu’ils persuaderaient facilement aux Allobroges, qui ne paraissaient pas encore bien fermement attachés au peuple romain, de leur permettre de traverser leur territoire, ou qu’ils les y contraindraient par la force. (4) Tout étant prêt pour le départ, ils fixent le jour où l’on doit se réunir sur la rive du Rhône. Ce jour était le 5 avant les calendes d’avril, sous le consulat de L. Pison et de A. Gabinius.
\subsection[{§ 7.}]{ \textsc{§ 7.} }
\noindent (1) César, apprenant qu’ils se disposent à passer par notre Province, part aussitôt de Rome, se rend à grandes journées dans la Gaule ultérieure et arrive à Genève. (2) Il ordonne de lever dans toute la province le plus de soldats qu’elle peut fournir (il n’y avait qu’une légion dans la Gaule ultérieure), et fait rompre le pont de Genève. (3) Les Helvètes, avertis de son arrivée, députent vers lui les plus nobles de leur cité, à la tête desquels étaient Namméios et Verucloétios, pour dire qu’ils avaient l’intention de traverser la province, sans y commettre le moindre dommage, n’y ayant pour eux aucun autre chemin, qu’ils le priaient d’y donner son consentement. (4) César, se rappelant que les Helvètes avaient tué le consul L. Cassius et repoussé son armée qu’ils avaient fait passer sous le joug, ne crut pas devoir leur accorder cette demande. (5) Il ne pensait pas que des hommes pleins d’inimitié pussent, s’ils obtenaient la permission de traverser la province, s’abstenir de violences et de désordres. (6) Cependant, pour laisser aux troupes qu’il avait commandées le temps de se réunir, il répondit aux députés qu’il y réfléchirait, et que, s’ils voulaient connaître sa résolution, ils eussent à revenir aux ides d’avril.
\subsection[{§ 8.}]{ \textsc{§ 8.} }
\noindent (1) Dans cet intervalle, César, avec la légion qu’il avait avec lui et les troupes qui arrivaient de la Province, éleva, depuis le lac Léman, que traverse le Rhône, jusqu’au mont Jura, qui sépare la Séquanie de l’Helvétie, un rempart de dix-neuf mille pas de longueur et de seize pieds de haut : un fossé y fut joint. (2) Ce travail achevé, il établit des postes, fortifie des positions, pour repousser plus facilement les Helvètes, s’ils voulaient passer contre son gré. (3) Dès que le jour qu’il avait assigné à leurs députés fut arrivé, ceux-ci revinrent auprès de lui. Il leur déclara que les usages et l’exemple du peuple romain lui défendaient d’accorder le passage à travers la Province, et que, s’ils tentaient de le forcer, il s’y opposerait. (4) Les Helvètes, déçus dans cette espérance, essaient de passer le Rhône, les uns sur des barques jointes ensemble et sur des radeaux faits dans ce dessein, les autres à gué, à l’endroit où le fleuve a le moins de profondeur, quelquefois le jour, plus souvent la nuit. Arrêtés par le rempart, par le nombre et par les armes de nos soldats, ils renoncent à cette tentative.
\subsection[{§ 9.}]{ \textsc{§ 9.} }
\noindent (1) Il leur restait un chemin par la Séquanie, mais si étroit qu’ils ne pouvaient le traverser malgré les habitants. (2) N'espérant pas en obtenir la permission par eux-mêmes, ils envoient des députés à l’Héduen Dumnorix, pour le prier de la demander aux Séquanes. (3) Dumnorix, puissant chez eux par son crédit et par ses largesses, était en outre l’ami des Helvètes, à cause de son mariage avec la fille de leur concitoyen Orgétorix. Excité d’ailleurs par le désir de régner, il aimait les innovations, et voulait s’attacher par des services un grand nombre de cités. (4) Il consentit donc à ce qu’on lui demandait, et obtint des Séquanes que les Helvètes traverseraient leur territoire : on se donna mutuellement des otages ; les Séquanes s’engagèrent à ne point s’opposer au passage des Helvètes, et ceux-ci à l’effectuer sans violences ni dégâts.
\subsection[{§ 10.}]{ \textsc{§ 10.} }
\noindent (1) On rapporte à César que les Helvètes ont le projet de traverser les terres des Séquanes et des Héduens, pour se diriger vers celles des Santons, peu distantes de Toulouse, ville située dans la province romaine. (2) Il comprit que, si cela arrivait, cette province serait exposée à un grand péril, ayant pour voisins, dans un pays fertile et découvert, des hommes belliqueux, ennemis du peuple romain. (3) Il confie donc à son lieutenant T. Labiénus la garde du retranchement qu’il avait é levé. Pour lui, il va en Italie à grandes journées, y lève deux légions, en tire trois de leurs quartiers d’hiver, aux environs d’Aquilée, et prend par les Alpes le plus court chemin de la Gaule ultérieure, à la tête de ces cinq légions. (4) Là, les Ceutrons, les Graïocèles et les Caturiges, qui s’étaient emparés des hauteurs, veulent arrêter la marche de son armée. (5) Il les repousse dans plusieurs combats, et se rend, en sept journées, d’Océlum, dernière place de la province citérieure, au territoire des Voconces, dans la province ultérieure ; de là il conduit ses troupes dans le pays des Allobroges, puis chez les Ségusiaves. C'est le premier peuple hors de la province, au-delà du Rhône.
\subsection[{§ 11.}]{ \textsc{§ 11.} }
\noindent (1) Déjà les Helvètes avaient franchi les défilés et le pays des Séquanes ; et, arrivés dans celui des Héduens, ils en ravageaient les terres. (2) Ceux-ci, trop faibles pour défendre contre eux leurs personnes et leurs biens, députent vers César, pour lui demander du secours : (3) "Dans toutes les circonstances, ils avaient trop bien mérité du peuple romain pour qu’on laissât, presque à la vue de notre armée, dévaster leurs champs, emmener leurs enfants en servitude, prendre leurs villes. (4) Dans le même temps, les Ambarres, amis et alliés des Héduens, informent également César que leur territoire est ravagé et qu’ils peuvent à peine garantir leurs villes de la fureur de leurs ennemis. (5) Enfin les Allobroges, qui avaient des bourgs et des terres au-delà du Rhône, viennent se réfugier auprès de lui, et lui déclarent qu’il ne leur reste rien que le sol de leurs champs. (6) César, déterminé par ce concours de plaintes, crut ne devoir pas attendre que tous les pays des alliés fussent ruinés, et les Helvètes arrivés jusque dans celui des Santons.
\subsection[{§ 12.}]{ \textsc{§ 12.} }
\noindent (1) La Saône est une rivière dont le cours, entre les terres des Héduens et celles des Séquanes et jusqu’au Rhône, est si paisible que l’oeil ne peut en distinguer la direction. Les Helvètes la passaient sur des radeaux et des barques jointes ensemble. (2) César, averti par ses éclaireurs que les trois quarts de l’armée helvète avaient déjà traversé la Saône, et que le reste était sur l’autre rive, part de son camp, à la troisième veille, avec trois légions, et atteint ceux qui n’avaient pas encore effectué leur passage. (3) Il les surprend en désordre, les attaque à l’improviste et en tue un grand nombre. Les autres prennent la fuite, et vont se cacher dans les forêts voisines. (4) Ils appartenaient au canton des Tigurins ; car tout le territoire de l’Helvétie est divisé en quatre cantons. (5) C'étaient ceux de ce canton qui, dans une excursion du temps de nos pères, avaient tué le consul L. Cassius et fait passer son armée sous le joug. (6) Ainsi, soit effet du hasard, soit par la volonté des dieux immortels, cette partie des citoyens de l’Helvétie, qui avait fait éprouver une si grande perte au peuple romain, fut la première à en porter la peine. (7) César trouva aussi dans cette vengeance publique l’occasion d’une vengeance personnelle ; car l’aïeul de son beau- père, L. Pison, lieutenant de Cassius, avait été tué avec lui par les Tigurins, dans la même bataille.
\subsection[{§ 13.}]{ \textsc{§ 13.} }
\noindent (1) Après ce combat, César, afin de poursuivre le reste des Helvètes, fait jeter un pont sur la Saône et la traverse avec son armée. (2) Ceux-ci, effrayés de son arrivée soudaine, et voyant qu’il lui avait suffi d’un seul jour pour ce passage qu’ils avaient eu beaucoup de peine à effectuer en vingt jours, lui envoient des députés ; à la tête de cette députation était Divico, qui commandait les Helvètes à la défaite de Cassius. (3) Il dit à César que, "si le peuple romain faisait la paix avec eux, ils se rendraient et s’établiraient dans les lieux que leur aurait assignés sa volonté ; (4) mais que, s’il persistait à leur faire la guerre, il eût à se rappeler l’échec passé de l’armée romaine et l’antique valeur des Helvètes ; (5) que pour s’être jeté à l’improviste sur un seul canton, lorsque leurs compagnons, qui avaient passé la rivière, ne pouvaient lui porter secours, il ne devait nullement attribuer cet avantage à son courage, ni concevoir du mépris pour eux ; (7) qu’ils avaient appris de leurs pères et de leurs ancêtres à se fier à leur valeur plutôt qu’à la ruse et que d’avoir recours aux embuscades ; qu’il prît donc garde que ce lieu où ils se trouvaient, marqué par le désastre des Romains et la destruction de leur armée, n’en tirât son nom et n’en transmit le souvenir à la postérité."
\subsection[{§ 14.}]{ \textsc{§ 14.} }
\noindent (1) À ce discours César répondit "qu’il était loin d’avoir oublié les choses que lui rappelaient les députés helvètes, et que son ressentiment en était d’autant plus vif que les Romains avaient moins mérité leur malheur ; (2) que s’ils eussent pu se douter de quelque injure, il leur était facile de se tenir sur leurs gardes ; mais qu’ils avaient été surpris parce que, n’ayant rien fait qui dût leur inspirer des craintes, ils ne pouvaient en concevoir sans motif. (3) Quand même César voudrait bien oublier cette ancienne injure, pourrait-il aussi effacer de son souvenir celles qui étaient récentes ; les efforts qu’ils avaient faits pour traverser malgré lui la province romaine, et leurs ravages chez les Héduens, chez les Ambarres, chez les Allobroges ? (4) L'insolente vanité qu’ils tiraient de leur victoire, et leur étonnement de voir leurs outrages si longtemps impunis, lui démontraient (5) que les dieux immortels, afin de rendre, par un revers subit, un châtiment plus terrible, accordent souvent à ceux-là même qu’ils veulent punir des succès passagers et une plus longue impunité. (6) Quoi qu’il en soit, s’ils lui livrent des otages comme garants de leurs promesses, et s’ils donnent aux Héduens, à leurs alliés et aux Allobroges, satisfaction du tort qu’ils leur ont fait, il consent à conclure avec eux la paix." (7) Divico répondit "qu’ils tenaient de leurs pères la coutume de recevoir des otages, et de n’en point donner ; que le peuple romain devait le savoir."
\subsection[{§ 15.}]{ \textsc{§ 15.} }
\noindent (1) Après cette réponse, il se retira. Le lendemain, ils lèvent leur camp ; César en fait autant, et envoie en avant toute sa cavalerie, au nombre de quatre mille hommes, qu’il avait levés dans la province entière, chez les Héduens et chez leurs alliés. Elle devait observer la direction que prendraient les ennemis. (2) Cette cavalerie, ayant poursuivi leur arrière-garde avec trop d’ardeur, en vint aux mains avec la cavalerie helvète dans un lieu désavantageux et éprouva quelque perte. (3) Les Helvètes, fiers d’avoir dans cette rencontre repoussé avec cinq cents chevaux un si grand nombre de cavaliers, nous attendirent plus hardiment, et nous inquiétèrent quelquefois avec leur arrière-garde. (4) César retenait l’ardeur de ses soldats, et se contentait pour le moment de s’opposer aux rapines, au pillage et aux dévastations de l’ennemi. (5) On fit route ainsi durant quinze jours, sans que l’arrière-garde des Helvètes fût séparée de notre avant-garde de plus de cinq ou six mille pas.
\subsection[{§ 16.}]{ \textsc{§ 16.} }
\noindent (1) Cependant César pressait chaque jour les Héduens de lui livrer le blé qu’ils lui avaient promis ; (2) car le climat froid de la Gaule, située au nord, comme il a été dit précédemment, faisait non seulement que la moisson n’était pas parvenue, dans les campagnes, à sa maturité, mais que le fourrage même y était insuffisant ; (3) quant au blé qu’il avait fait charger sur la Saône, il pouvait d’autant moins lui servir, que les Helvètes s’étaient éloignés de cette rivière, et il ne voulait pas les perdre de vue. (4) Les Héduens différaient de jour en jour, disant qu’on le rassemblait, qu’on le transportait, qu’il était arrivé. (5) Voyant que ces divers discours se prolongeaient trop, et touchant au jour où il fallait faire aux soldats la distribution des vivres, César convoqua les principaux Héduens, qui étaient en grand nombre dans le camp, entre autres Diviciacos et Liscos. Ce dernier occupait la magistrature suprême que les Héduens appellent vergobret, fonctions annuelles et qui confèrent le droit de vie et de mort. (6) César se plaint vivement à eux de ce que, ne pouvant acheter des vivres ni en prendre dans les campagnes, il ne trouve, dans un besoin si pressant et presque en présence de l’ennemi, aucun secours dans des alliés ; l’abandon où ils le laissaient était d’autant plus coupable, que c’était en grande partie à leur prière qu’il avait entrepris la guerre.
\subsection[{§ 17.}]{ \textsc{§ 17.} }
\noindent (1) Enfin Liscos, ému par les paroles de César, déclare ce qu’il avait vu jusque-là : "qu’il y avait quelques hommes dans le plus grand crédit auprès du peuple et dont l’influence privée l’emportait sur celle des magistrats ; (2) qu’au moyen de discours séditieux et pervers, ils détournaient la multitude de fournir le blé qu’on s’était engagé à livrer, (3) disant que s’ils ne pouvaient obtenir la suprématie sur la Gaule, ils devaient du moins préférer la domination des Gaulois à celle des Romains ; (4) qu’on devait être certain que ceux-ci, une fois vainqueurs des Helvètes, dépouilleraient de leur liberté les Héduens et les autres peuples de la Gaule ; (5) que ces mêmes hommes informaient l’ennemi de nos projets et de tout ce qui se passait dans le camp ; qu’il n’avait pas le pouvoir de les réprimer ; (6) qu’il savait bien à quel péril l’exposait la déclaration que la nécessité l’avait contraint à faire à César, et que telle avait été la cause du long silence qu’il avait gardé."
\subsection[{§ 18.}]{ \textsc{§ 18.} }
\noindent (1) César sentit bien que ce discours désignait Dumnorix, frère de Diviciacos ; mais, ne voulant pas traiter cette affaire en présence d’un grand nombre de témoins, il rompt précipitamment l’assemblée, et ne retient que Liscos. (2) Demeuré seul avec lui, il le presse de reprendre ce qu’il avait dit dans le conseil. Liscos parle avec plus de liberté et de hardiesse. D'autres informations secrètes prouvent la vérité des siennes. (3) "Dumnorix, homme plein d’audace, avait acquis par ses largesses une grande influence sur le peuple, et était avide de changement. II avait, depuis plusieurs années, obtenu à bas prix la perception des péages et autres impôts des Héduens, parce que personne n’avait osé enchérir sur lui. (4) Sa fortune, accrue encore de cette sorte, lui donnait les moyens de prodiguer ses libéralités. (5) On le voyait entouré d’une cavalerie nombreuse, entretenue à ses frais. (6) Son crédit n’était pas restreint à sa cité, mais s’étendait jusque chez les peuples voisins ; c’est dans cette vue qu’il avait fait épouser à sa mère l’un des personnages les plus nobles et les plus puissants parmi les Bituriges, (7) que lui-même avait pris une femme chez les Helvètes, et qu’il avait marié dans d’autres cités sa soeur et ses parentes. (8) Son mariage le rendait le partisan et l’ami des Helvètes ; il haïssait en outre personnellement César et les Romains, dont l’arrivée avait affaibli son pouvoir et rendu à son frère Diviciacos son ancienne autorité et ses honneurs. (9) Si les Romains éprouvaient quelque échec, il espérait bien, à l’aide des Helvètes, parvenir à la puissance souveraine ; sous leur empire, il perdait un trône et même son crédit actuel." (10) Les informations prises par César lui apprirent aussi que dans le combat de cavalerie livré peu de jours auparavant, l’exemple de la fuite avait été donné par Dumnorix et par sa cavalerie, car c’était lui qui commandait celle que les Héduens avaient envoyée au secours de César : cette fuite avait effrayé le reste.
\subsection[{§ 19.}]{ \textsc{§ 19.} }
\noindent (1) Outre ces rapports, les indices les plus certains venaient confirmer les soupçons de César : c’était Dumnorix qui avait fait passer les Helvètes par le territoire des Séquanes, qui les avait engagés à se donner mutuellement des otages ; il avait tout fait non seulement sans l’ordre de César et des Héduens, mais encore à leur insu ; il était accusé par le magistrat de sa nation. César croyait avoir assez de motifs, soit pour sévir lui- même contre Dumnorix, soit pour exiger que ses concitoyens le punissent. (2) Une seule considération arrêtait ses résolutions, le grand attachement de Diviciacos, son frère, au peuple romain, son dévouement sans bornes, sa fidélité à toute épreuve, sa justice, sa modération ; et il craignait de s’aliéner son esprit par le supplice de son frère. (3) Aussi, avant de rien entreprendre, il fait appeler Diviciacos, et, renvoyant les interprètes habituels, c’est par l’organe de C. Valérius Troucillus, le premier personnage de la province romaine, son ami et son confident le plus intime, qu’il s’entretient avec lui : (4) en même temps qu’il lui rappelle ce qui a été dit de Dumnorix en sa présence dans l’assemblée des Gaulois ; il lui apprend ce dont chacun l’a informé en particulier ; (5) il l’engage et l’exhorte à ne point s’offenser si lui-même, après l’avoir entendu, décide de son sort, ou s’il ordonne à ses concitoyens d’instruire son procès.
\subsection[{§ 20.}]{ \textsc{§ 20.} }
\noindent (1) Diviciacos, tout en larmes, embrasse César et le supplie de ne prendre contre son frère aucune résolution sévère : (2) il convient de la vérité de ces accusations, et personne n’en est plus affligé que lui ; il avait lui-même, par son crédit parmi ses concitoyens et dans le reste de la Gaule, contribué à l’élévation d’un frère qui n’en avait aucun à cause de sa jeunesse ; (3) et celui-ci s’était depuis servi de son influence et de sa supériorité, non seulement pour affaiblir son pouvoir, mais encore pour essayer de le perdre. Cependant l’amour fraternel et l’opinion publique le retiennent. (4) Si César faisait tomber sur son frère quelque châtiment rigoureux, tout le monde, connaissant l’amitié qui les unit, l’en regarderait comme l’auteur, et cette persuasion éloignerait de lui les coeurs de tous les Gaulois. (5) Ses paroles étaient entrecoupées de sanglots ; César lui prend la main, le rassure, le prie de mettre fin à ses demandes, et lui dit qu’il fait assez de cas de lui pour sacrifier à ses désirs et à ses prières les injures de la république et son propre ressentiment. (6) Il fait venir Dumnorix en présence de son frère, lui expose les griefs qu’il a contre lui, lui déclare ses soupçons personnels et les plaintes de ses concitoyens ; il l’engage à éviter de se rendre suspect à l’avenir et lui dit qu’il veut bien oublier le passé en considération de son frère Diviciacos. Il le fait surveiller par des gardes, pour être instruit de ses actions et de ses discours.
\subsection[{§ 21.}]{ \textsc{§ 21.} }
\noindent (1) Le même jour, César apprenant par ses éclaireurs que l’ennemi avait posé son camp au pied d’une montagne, à huit mille pas du sien, envoya reconnaître la nature de cette montagne et les circuits par lesquels on pouvait la gravir. On lui rapporta que l’accès en était facile. (2) À la troisième veille, il ordonne à T. Labiénus, son lieutenant, de partir avec deux légions et les mêmes guides qui avaient reconnu la route, et d’occuper la hauteur, et il lui fait part de son dessein. (3) Pour lui, à la quatrième veille, il marche aux ennemis par le même chemin qu’ils avaient pris, et envoie toute la cavalerie en avant. (4) P. Considius, qui passait pour très expérimenté dans l’art militaire, et avait servi dans l’armée de L. Sylla, et ensuite dans celle de M. Crassus, est détaché à la tête des éclaireurs.
\subsection[{§ 22.}]{ \textsc{§ 22.} }
\noindent (1) Au point du jour, T. Labiénus occupait le sommet de la montagne, et César n’était qu’à quinze cents pas du camp des ennemis, sans qu’ils eussent, ainsi qu’on le sut depuis par des prisonniers, connaissance de son arrivée ni de celle de Labiénus ; (2) lorsque Considius accourt à toute bride ; il annonce que la montagne dont Labiénus avait ordre de s’emparer est au pouvoir de l’ennemi, qu’il a reconnu les armes et les enseignes gauloises. (3) César se retire avec ses troupes sur la plus proche colline, et les range en bataille. Labiénus, à qui il était prescrit de ne point engager le combat avant de voir l’armée de César près du camp ennemi, afin que l’attaque eût lieu en même temps sur tous les points, restait sur la hauteur dont il était maître, attendant nos troupes, et sans engager l’action. (4) Il était enfin tout à fait jour lorsque César apprit par ses éclaireurs que Labiénus occupait la montagne, et que les Helvètes avaient levé leur camp ; Considius, troublé par la peur, avait déclaré avoir vu ce qu’il n’avait pu voir. (5) Ce même jour, César suivit les ennemis à quelque distance selon sa coutume, et campa à trois mille pas de leur armée.
\subsection[{§ 23.}]{ \textsc{§ 23.} }
\noindent (1) Le lendemain, comme il ne restait plus que deux jours jusqu’à la distribution du blé à l’armée, et que Bibracte, la plus grande sans contredit et la plus riche des villes des Héduens, n’était plus qu’à dix-huit mille pas, César crut devoir s’occuper des vivres, s’éloigna des Helvètes et se dirigea vers Bibracte. (2) Quelques transfuges de L. Émilius, décurion de la cavalerie gauloise, en donnèrent avis aux ennemis. (3) Les Helvètes, ou attribuant à la peur la retraite des Romains, d’autant plus que la veille, quoique maîtres des hauteurs, ils n’avaient pas engagé le combat ; ou bien se flattant de pouvoir leur couper les vivres, changèrent de projets, rebroussèrent chemin, et se mirent à suivre et à harceler notre arrière-garde.
\subsection[{§ 24.}]{ \textsc{§ 24.} }
\noindent (1) Voyant ce mouvement, César conduit ses troupes sur une hauteur voisine, et détache sa cavalerie pour soutenir l’attaque de l’ennemi. (2) En même temps il range en bataille sur trois lignes, au milieu de la colline, quatre légions de vieilles troupes, et place au sommet les deux légions qu’il avait nouvellement levées dans la Gaule citérieure, ainsi que tous les auxiliaires ; il fait aussi garnir de soldats toute la montagne, (3) rassembler les bagages en un seul endroit, que fortifient les troupes qui ont pris position sur la hauteur. (4) Les Helvètes, qui suivaient avec tous leurs chariots, réunirent leur bagage dans un même lieu ; leur front serré repousse notre cavalerie ; ils se forment en phalange, et attaquent notre première ligne.
\subsection[{§ 25.}]{ \textsc{§ 25.} }
\noindent (1) César renvoie tous les chevaux, à commencer par le sien, afin de rendre le péril égal pour tous et la fuite impossible, exhorte ses troupes et marche au combat. (2) Nos soldats, lançant leurs traits d’en haut, rompent aisément la phalange des ennemis. L'ayant mise en désordre, ils fondent sur elle, le glaive à la main. (3) Les Gaulois éprouvaient une grande gêne pour combattre, en ce que plusieurs de leurs boucliers se trouvaient, du même coup des javelots, percés et comme cloués ensemble, et que le fer s’étant recourbé, ils ne pouvaient ni l’arracher, ni se servir dans la mêlée de leur bras gauche ainsi embarrassé. Un grand nombre d’entre eux, après de longs efforts de bras, préfèrent jeter leurs boucliers et combattre découverts. (5) Enfin, accablés de blessures, ils commencent à lâcher pied et à faire leur retraite vers une montagne, à mille pas à peu près. (6) Ils l’occupent bientôt, et les nôtres les suivent, lorsque les Boïens et les Tulinges qui, au nombre de quinze mille environ, fermaient la marche de l’ennemi, et en soutenaient l’arrière-garde, nous attaquent sur notre flanc, que la marche avait laissé à découvert, et nous enveloppent. À la vue de cette manoeuvre, les Helvètes, qui s’étaient retirés sur la montagne, se hâtent de revenir et de recommencer le combat. (7) Les Romains tournent leurs enseignes et s’avancent des deux côtés ; ils opposent leur première et leur seconde ligne à ceux qu’ils ont déjà vaincus et repoussés, et leur troisième aux nouveaux assaillants.
\subsection[{§ 26.}]{ \textsc{§ 26.} }
\noindent (1) Aussi ce double combat fut-il long et opiniâtre. Les ennemis, ne pouvant soutenir plus longtemps l’effort de nos armes, se retirèrent, comme ils avaient fait d’abord, les uns sur la montagne, les autres vers leurs bagages et leurs chariots. (2) Durant tout ce combat, qui se prolongea depuis la septième heure jusqu’au soir, personne ne put voir un ennemi tourner le dos. (3) Près des bagages on combattit encore bien avant dans la nuit ; car ils s’étaient fait un rempart de leurs chariots, et lançaient d’en haut une grêle de traits sur les assaillants, tandis que d’autres, entre ces chariots et les roues, nous blessaient de leurs javelots et de leurs flèches. (4) Ce ne fut qu’après de longs efforts que nous nous rendîmes maîtres des bagages et du camp. La fille d’Orgétorix et un de ses fils y tombèrent en notre pouvoir. (5) Après cette bataille, il leur restait environ cent trente mille hommes ; ils marchèrent toute la nuit sans s’arrêter. Continuant leur route sans faire halte nulle part, même pendant les nuits, ils arrivèrent le quatrième jour sur les terres des Lingons. Les blessures des soldats et la sépulture des morts nous ayant retenus trois jours, nous n’avions pu les poursuivre. (6) César envoya aux Lingons des lettres et des courriers pour leur défendre d’accorder aux ennemis ni vivres ni autres secours, sous peine, s’ils le faisaient, d’être traités comme les Helvètes. Lui-même, après ces trois jours, se mit avec toutes ses troupes à leur poursuite.
\subsection[{§ 27.}]{ \textsc{§ 27.} }
\noindent (1) Les Helvètes, réduits à la dernière extrémité, lui envoyèrent des députés pour traiter de leur soumission. (2) L'ayant rencontré en marche, ils se jetèrent à ses pieds, lui parlèrent en suppliants, et implorèrent la paix en pleurant. Il ordonna aux Helvètes de l’attendre dans le lieu même où ils étaient alors ; ils obéirent. (3) César, quand il y fut arrivé, leur demanda des otages, leurs armes, les esclaves qui s’étaient enfuis vers eux. (4) Pendant qu’on cherche et qu’on rassemble ce qu’il avait exigé, profitant de la nuit, six mille hommes environ du canton appelé Verbigénus, soit dans la crainte qu’on ne les mette à mort après leur avoir enlevé leurs armes, soit dans l’espoir que, parmi un si grand nombre de captifs, ils parviendront à cacher et à laisser entièrement ignorer leur fuite, sortent à la première veille du camp des Helvètes, et se dirigent vers le Rhin et les frontières des Germains.
\subsection[{§ 28.}]{ \textsc{§ 28.} }
\noindent (1) Dès que César en fut instruit, il ordonna aux peuples sur les terres desquels ils pouvaient passer de les poursuivre et de les ramener, s’ils voulaient rester innocents à ses yeux. (2) Ils furent livrés et traité s en ennemis. Tous les autres, après avoir donné otages, armes et transfuges, reçurent leur pardon. (3) Il ordonna aux Helvètes, aux Tulinges, aux Latobices de retourner dans le pays d’où ils étaient partis. Comme il ne leur restait plus de vivres et qu’ils ne devaient trouver chez eux aucune subsistance pour apaiser leur faim, il ordonna aux Allobroges de leur fournir du blé ; il enjoignit aux Helvètes de reconstruire les villes et les bourgs qu’ils avaient incendiés. (4) La principale raison qui lui fit exiger ces choses fut qu’il ne voulait pas que le pays d’où les Helvètes s’étaient éloignés restât désert, dans la crainte qu’attirés par la fertilité du sol, les Germains d’outre-Rhin ne quittassent leur pays pour celui des premiers, et ne devinssent les voisins de notre province et des Allobroges. (5) À la demande des Héduens, les Boïens reçurent, à cause de leur grande réputation de valeur, la permission de s’établir sur leur propre territoire ; on leur donna des terres, et ils partagèrent plus tard les droits et la liberté des Héduens eux-mêmes.
\subsection[{§ 29.}]{ \textsc{§ 29.} }
\noindent (1) On trouva dans le camp des Helvètes des registres écrits en lettres grecques et qui furent apportés à César. Sur ces registres étaient nominativement inscrits ceux qui étaient sortis de leur pays, le nombre des hommes capables de porter les armes, et séparément celui des enfants, des vieillards et des femmes. (2) On y comptait en tout 263, 000 Helvètes, 36, 000 Tulinges, 14, 000 Latobices, 23, 000 Rauraques, 32, 000 Boïens. II y avait parmi eux 92, 000 combattants ; le total s’élevait à 368, 000 Gaulois. Le nombre de ceux qui rentrèrent dans leur pays fut, d’après le recensement ordonné par César, de cent dix mille.
\subsection[{§ 30.}]{ \textsc{§ 30.} }
\noindent (1) La guerre des Helvètes étant terminée, des députés de presque toute la Gaule et les principaux habitants des cités vinrent féliciter César ; (2) ils savaient bien, disaient-ils, que sa guerre contre les Helvètes était la vengeance des injures faites au peuple romain ; mais la Gaule n’en tirait pas un moindre profit que la république, (3) puisque les Helvètes n’avaient quitté leurs villes, dont l’état était si florissant, que dans le but de porter leurs armes sur tout le territoire des Gaulois, de s’en rendre maîtres, de choisir parmi tant de contrées, afin de s’y établir, la plus riche et la plus fertile, et d’imposer des tributs au reste des cités. (4) Ils demandèrent à César la permission de convoquer l’assemblée générale de toute la Gaule ; ils avaient une prière à lui faire en commun. (5) Cette permission accordée, ils fixèrent le jour de leur réunion, et s’engagèrent par serment à n’en rien révéler que du consentement de tous.
\subsection[{§ 31.}]{ \textsc{§ 31.} }
\noindent (1) Quand cette assemblée fut close, les mêmes citoyens qui s’étaient déjà présentés devant César revinrent vers lui et demandèrent qu’il leur fût permis de l’entretenir en particulier, touchant leur sûreté et celle de tous les Gaulois. (2) Ayant obtenu audience, ils se jetèrent à ses pieds en versant des larmes, et le prièrent aussi instamment de leur garder le secret sur leurs révélations que de leur accorder l’objet de leur demande : car si leur démarche était connue, ils se verraient exposés aux derniers supplices. (3) L'Héduen Diviciacos prit pour eux la parole, et dit "que deux partis divisaient la Gaule. L'un avait les Héduens pour chef, l’autre les Arvernes. (4) Après une lutte de plusieurs années pour la prééminence, les Arvernes, unis aux Séquanes, attirèrent les Germains en leur promettant des avantages. (5) Quinze mille de ces derniers passèrent d’abord le Rhin ; la fertilité du sol, la civilisation, les richesses des Gaulois, ayant charmé ces hommes grossiers et barbares, il s’en présenta un plus grand nombre, et il y en a maintenant cent vingt mille dans la Gaule. (6) Les Héduens et leurs alliés leur ont livré deux combats, et ont eu, outre leur défaite, de grands malheurs à déplorer, la perte de toute leur noblesse, de tout leur sénat, de toute leur cavalerie. (7) Épuisé par ces combats et par ces revers, ce peuple, que son propre courage ainsi que l’appui et l’amitié des Romains, avaient précédemment rendu si puissant dans la Gaule, s’était vu forcé de donner en otage aux Séquanes ses plus nobles citoyens, et de s’obliger par serment à ne jamais réclamer pour sa liberté ni pour celle des otages, à ne point implorer le secours du peuple romain, à ne pas tenter de se soustraire au joug perpétuel de ses vainqueurs. (8) Il est le seul de tous ses concitoyens qu’on n’ait pu contraindre à prêter serment ni à donner ses enfants en otage. Il n’a fui de son pays et n’est venu à Rome demander du secours au sénat que parce qu’il n’était retenu par aucun de ces deux liens. (10) Mais les Séquanes vainqueurs ont éprouvé un sort plus intolérable que les Héduens vaincus : en effet, Arioviste, roi des Germains, s’est établi dans leur pays, s’est emparé du tiers de leur territoire, qui est le meilleur de toute la Gaule, et leur ordonne maintenant d’en abandonner un autre tiers à vingt-quatre mille Harudes qui, depuis peu de mois, sont venus le joindre, et auxquels il faut préparer un établissement. (11) Il arrivera dans peu d’années que tous les Gaulois seront chassés de leur pays, et que tous les Germains auront passé le Rhin ; car le sol de la Germanie ne peut pas entrer en comparaison avec celui de la Gaule, non plus que la manière de vivre des deux nations. (12) Arioviste, une fois vainqueur de l’armée gauloise dans la bataille qui fut livrée à Admagétobrige, commanda en despote superbe et cruel, exigea pour otage les enfants de tous les nobles, et exerce contre eux tous les genres de cruauté, si l’on n’obéit aussitôt à ses caprices ou à sa volonté : (13) c’est un homme barbare, emporté, féroce ; on ne peut supporter plus longtemps sa tyrannie. (14) Si César et le peuple romain ne viennent pas à leur secours, tous les Gaulois n’ont plus qu’une chose à faire : à l’exemple des Helvètes, ils émigreront de leur pays, chercheront d’autres terres et d’autres demeures éloignées des Germains et tenteront la fortune, quel que soit le sort qui les attende. (15) Si Arioviste venait à connaître leurs révélations, nul doute qu’il ne livrât tous les otages en son pouvoir aux plus affreux supplices. (16) César, par son autorité, par ses forces, par l’éclat de sa victoire récente, et avec le nom du peuple romain, peut empêcher qu’un plus grand nombre de Germains ne passent le Rhin, pour défendre la Gaule entière contre les violences d’Arioviste."
\subsection[{§ 32.}]{ \textsc{§ 32.} }
\noindent (1) Diviciacos cessa de parler, et tous ceux qui étaient présents, fondant en larmes, implorèrent le secours de César. (2) Remarquant que les Séquanes seuls s’abstenaient de faire comme les autres ; que, tristes et la tête baissée, ils regardaient la terre, César s’étonne de cet abattement et leur en demande la cause. (3) Ils ne répondent rien et restent plongés dans cette tristesse muette. Il les presse à plusieurs reprises sans pouvoir tirer d’eux aucune réponse. Alors l’Héduen Diviciacos reprend la parole : (4) " Tel est, dit-il, le sort des Séquanes, plus malheureux encore et plus intolérable que celui des autres Gaulois ; seuls, ils n’osent se plaindre, même en secret, ni réclamer des secours, et la cruauté d’Arioviste absent leur inspire autant d’effroi que s’il était devant eux. (5) Les autres ont du moins la liberté de fuir ; mais les Séquanes, qui ont reçu Arioviste sur leurs terres, et dont toutes les villes sont en son pouvoir, se voient forcés d’endurer tous les tourments."
\subsection[{§ 33.}]{ \textsc{§ 33.} }
\noindent (1) Instruit de tous ces faits, César relève par quelques mots le courage des Gaulois et leur promet de veiller sur eux dans ces conjonctures. Il a tout lieu d’espérer que, par reconnaissance et par respect pour lui, Arioviste mettra un terme à ses violences. (2) Après ces paroles, il congédia l’assemblée. Ces plaintes et beaucoup d’autres motifs l’engageaient à s’occuper sérieusement de cette affaire. D'abord il voyait les Héduens, que le sénat avait souvent appelés du titre de frères et d’alliés, asservis comme des esclaves à la domination des Germains ; il les voyait livrant des otages entre les mains d’Arioviste et des Séquanes, ce qui était honteux pour lui-même et pour la toute-puissance du peuple romain ; (3) il voyait en outre le péril qu’il y avait pour la république à laisser les Germains s’habituer à passer le Rhin et à venir en grand nombre dans la Gaule. (4) Ces peuples grossiers et barbares, une fois en possession de la Gaule entière, ne manqueraient pas sans doute, à l’exemple des Cimbres et des Teutons, de se jeter sur la province romaine et de là sur l’Italie, d’autant plus que la Séquanie n’était séparée de notre province que par le Rhône. César pensa donc qu’il fallait se hâter de prévenir ces dangers. (5) Arioviste, d’ailleurs, en était venu à un degré d’orgueil et d’arrogance qu’il n’était plus possible de souffrir.
\subsection[{§ 34.}]{ \textsc{§ 34.} }
\noindent (1) Il résolut donc d’envoyer à Arioviste des députés chargés de l’inviter à désigner, pour un entretien, quelque lieu intermédiaire. Il voulait conférer avec lui des intérêts de la république et d’affaires importantes pour tous deux. (2) Arioviste répondit à cette députation que s’il avait besoin de César, il irait vers lui ; que si César voulait de lui quelque chose, il eût à venir le trouver ; (3) que, d’ailleurs, il n’osait se rendre sans armée dans la partie de la Gaule que possédait César, et qu’une armée ne pouvait être rassemblée sans beaucoup de frais et de peine ; (4) enfin, qu’il lui semblait étonnant que, dans la Gaule, sa propriété par le droit de la guerre et de la victoire, il eût quelque chose à démêler avec César ou avec le peuple romain.
\subsection[{§ 35.}]{ \textsc{§ 35.} }
\noindent (1) Cette réponse étant rapportée à César, il envoie de nouveaux députés vers Arioviste, avec les instructions suivantes : (2) "Puisqu’après avoir été comblé de bienfaits par le peuple romain et par César, sous le consulat de qui il avait reçu du sénat le titre de roi et d’ami, pour toute reconnaissance de cette faveur, il refuse de se rendre à l’entrevue à laquelle il est invité, et qu’il ne juge pas à propos de traiter avec lui de leurs intérêts communs, voici ce qu’il lui demande (3) premièrement, de ne plus attirer dans la Gaule cette multitude d’hommes venant d’au-delà du Rhin ; en second lieu, de restituer aux Héduens les otages qu’il tient d’eux, et de permettre aux Séquanes de rendre ceux qu’ils ont reçus de leur côté ; de mettre fin à ses violences envers les Héduens et de ne faire la guerre ni à eux ni à leurs alliés. (4) S'il se soumet à ces demandes, il peut compter sur l’éternelle bienveillance et sur l’amitié de César et du peuple romain ; s’il s’y refuse, attendu le décret du sénat rendu sous le consulat de M. Messala et de M. Pison, qui charge le gouverneur de la Gaule de faire ce qui est avantageux pour la république, et de défendre les Héduens et les autres alliés de Rome, il ne négligera pas de venger leur injure."
\subsection[{§ 36.}]{ \textsc{§ 36.} }
\noindent (1) À cela Arioviste répondit que, de par le droit de la guerre, le vainqueur pouvait disposer à son gré du vaincu, et que Rome avait coutume de traiter les peuples conquis à sa guise et non à celle d’autrui ; (2) s’il ne prescrit pas aux Romains comment ils doivent user de leur droit, il ne faut pas qu’ils le gênent dans l’exercice du sien. (3) Les Héduens ont voulu tenter le sort des armes et combattre ; ils ont succombé et sont devenus ses tributaires. (4) Il a lui-même un grave sujet de plainte contre César, dont l’arrivée diminue ses revenus. (5) Il ne rendra point aux Héduens leurs otages ; il ne fera la guerre ni à eux ni à leurs alliés, s’ils restent fidèles à leurs conventions et paient le tribut chaque année ; sinon, le titre de frères du peuple romain sera loin de leur servir. (6) Quant à la déclaration de César "qu’il ne négligerait pas de venger les injures faites aux Héduens", personne ne s’était encore, sans s’en repentir, attaqué à Arioviste ; (7) ils se mesureraient quand il voudrait ; César apprendrait ce que peut la valeur des Germains, nation invincible et aguerrie, qui, depuis quatorze ans, n’avait pas reposé sous un toit.
\subsection[{§ 37.}]{ \textsc{§ 37.} }
\noindent (1) Dans le même temps que César recevait cette réponse, il lui venait des députés des Héduens et des Trévires. (2) Les Héduens se plaignaient que les Harudes, nouvellement arrivés dans la Gaule, dévastaient leur pays ; ils n’avaient pu, même en donnant des otages, acheter la paix d’Arioviste. (3) Les Trévires, de leur côté, l’informaient que cent cantons des Suèves étaient campés sur les rives du Rhin et tentaient de passer ce fleuve ; ils étaient commandés par deux frères, Nasua et Cimbérios. (4) César, vivement ému de ces nouvelles, vit qu’il n’avait pas un instant à perdre ; il craignit, si de nouvelles bandes de Suèves se joignaient aux anciennes troupes d’Arioviste, qu’il ne devînt moins facile de leur résister. (5) II fit donc rassembler des vivres en toute hâte, et marcha à grandes journées contre Arioviste.
\subsection[{§ 38.}]{ \textsc{§ 38.} }
\noindent (1) Il était en marche depuis trois jours, lorsqu’on lui annonça que celui-ci, avec toutes ses forces, se dirigeait contre Besançon, la plus forte place des Séquanes, et que, depuis autant de jours, il avait passé la frontière. (2) César crut devoir faire tous ses efforts pour le prévenir, (3) car cette ville était abondamment pourvue de munitions de toute espèce, (4) et sa position naturelle la défendait de manière à en faire un point très avantageux pour soutenir la guerre. La rivière du Doubs décrit un cercle à l’entour et l’environne presque entièrement ; (5) la partie que l’eau ne baigne pas, et qui n’a pas plus de six cents pieds, est protégée par une haute montagne dont la base touche de chaque côté aux rives du Doubs. (6) Une enceinte de murs fait de cette montagne une citadelle et la joint à la ville. (7) César s’avance à grandes journées, et le jour et la nuit, s’en rend maître et y met garnison.
\subsection[{§ 39.}]{ \textsc{§ 39.} }
\noindent (1) Pendant le peu de jours qu’il passa à Besançon, afin de pourvoir aux subsistances et aux vivres, les réponses que faisaient aux questions de nos soldats les Gaulois et les marchands qui leur parlaient de la taille gigantesque des Germains, de leur incroyable valeur, de leur grande habitude de la guerre, de leur aspect terrible et du feu de leurs regards qu’ils avaient à peine pu soutenir dans de nombreux combats, jetèrent tout à coup une vive terreur dans toute l’armée ; un trouble universel et profond s’empara des esprits. (2) Cette frayeur commença par les tribuns militaires, par les préfets et par ceux qui, ayant suivi César par amitié, n’avaient que peu d’expérience de la guerre ; (3) les uns, alléguant diverses nécessités, lui demandaient qu’il leur permît de partir ; d’autres, retenus par la honte, ne restaient que pour ne pas encourir le reproche de lâcheté ; (4) ils ne pouvaient ni composer leurs visages ni retenir leurs larmes qui s’échappaient quelquefois. Cachés dans leurs tentes, ils se plaignaient de leur sort ou déploraient avec leurs amis le danger commun. (5) Dans tout le camp chacun faisait son testament. Ces plaintes et cette terreur ébranlèrent peu à peu ceux mêmes qui avaient vieilli dans les camps, les soldats, les centurions, les commandants de la cavalerie. (6) Ceux qui voulaient passer pour les moins effrayés disaient que ce n’était pas l’ennemi qu’ils craignaient, mais la difficulté des chemins, la profondeur des forêts qui les séparaient d’Arioviste, et les embarras du transport des vivres. (7) On rapporta même à César que, quand il ordonnerait de lever le camp et de porter les enseignes en avant, les soldats effrayés resteraient sourds à sa voix et laisseraient les enseignes immobiles.
\subsection[{§ 40.}]{ \textsc{§ 40.} }
\noindent (1) Ayant réfléchi sur ces rapports, il convoque une assemblée, y appelle les centurions de tous les rangs et leur reproche vivement et d’abord, de vouloir s’informer du pays où il les mène et juger ses desseins. (2) Pendant son consulat, Arioviste a recherché avec le plus grand empressement l’amitié du peuple romain. Pourquoi le supposerait-on assez téméraire pour s’écarter de son devoir ? (3) Quant à lui, il est persuadé que, dès qu’Arioviste connaîtra ses demandes, et qu’il en aura apprécié l’équité, il ne voudra renoncer ni à ses bonnes grâces ni à celles des Romains. (4) Si, poussé par une démence furieuse, il se décide à la guerre, qu’y a-t-il donc à craindre ? et pourquoi désespérer de leur courage et de son activité ? (5) Le péril dont les menaçait cet ennemi, leurs pères l’avaient bravé, lorsque, sous C. Marius, l’armée, repoussant les Cimbres et les Teutons, s’acquit autant de gloire que le général lui-même ; ils l’avaient eux-mêmes bravé tout récemment en Italie, dans la guerre des esclaves ; et cet ennemi avait cependant le secours de l’expérience et de la discipline qu’il tenait des Romains. (6) On pouvait juger par là des avantages de la fermeté, puisque ceux qu’on avait, sans motif, redoutés quelque temps, bien qu’ils fussent sans armes, on les avait soumis ensuite armés et victorieux ; (7) ce peuple enfin était le même qu’avaient souvent combattu les Helvètes, et qu’ils avaient presque aussi souvent vaincu, non seulement dans leur pays, mais dans le sien même et les Helvètes n’avaient cependant pu résister aux forces romaines. (8) Que s’il en est qu’effraient la défaite et la fuite des Gaulois, ceux-là pourront se convaincre, s’ils en cherchent les causes, que les Gaulois étaient fatigués de la longueur de la guerre ; qu’Arioviste, après s’être tenu plusieurs mois dans son camp et dans ses marais, sans accepter la bataille, les avait soudainement attaqués, désespérant déjà de combattre et dispersés, et les avait vaincus plutôt par adresse et habileté que par le courage. (9) Si de tels moyens ont pu être bons contre des barbares et des ennemis sans expérience, il n’espérait pas sans doute les employer avec le même succès contre des armées romaines. (10) Ceux qui cachent leurs craintes sous le prétexte des subsistances et de la difficulté des chemins sont bien arrogants de croire que le général puisse manquer à son devoir, ou de le lui prescrire. (11) Ce soin lui appartient ; le blé sera fourni par les Séquanes, les Leuques, les Lingons ; déjà même il est mûr dans les campagnes. Quant au chemin ils en jugeront eux-mêmes dans peu de temps. (12) Les soldats, dit-on, n’obéiront pas à ses ordres et ne lèveront pas les enseignes ; ces menaces ne l’inquiètent pas ; car il sait qu’une armée ne se montre rebelle à la voix de son chef que quand, par sa faute, la fortune lui a manqué, ou qu’il est convaincu de quelque crime, comme de cupidité. (13) Sa vie entière prouve son intégrité, et la guerre d’Helvétie le bonheur de ses armes. (14) Aussi le départ qu’il voulait remettre à un jour plus éloigné, il l’avance ; et la nuit suivante, à la quatrième veille, il lèvera le camp, afin de savoir avant tout ce qui prévaut sur eux, ou l’honneur et le devoir, ou la peur. (15) Si cependant personne ne le suit, il partira avec la dixième légion seule, dont il ne doute pas, et elle sera sa cohorte prétorienne. César avait toujours particulièrement favorisé cette légion et se fiait entièrement à sa valeur.
\subsection[{§ 41.}]{ \textsc{§ 41.} }
\noindent (1) Cette harangue, produisant dans tous les esprits un changement extraordinaire, fit naître la plus vive ardeur et le désir de combattre. (2) La dixième légion, par l’organe des tribuns militaires, remercia aussitôt César d’avoir aussi bien présumé d’elle, et déclara qu’elle était prête à marcher au combat. (3) Ensuite les autres légions lui députèrent leurs tribuns et les centurions des premiers rangs, pour lui adresser leurs excuses ; elles n’avaient jamais hésité, ni tremblé, ni prétendu porter sur la guerre un jugement qui n’appartient qu’au général. (4) César reçut leurs excuses, et, après s’être enquis du chemin à prendre auprès de Diviciacos, celui des Gaulois dans lequel il avait le plus de confiance, il résolut de faire un détour de cinquante milles, afin de conduire son armée par un pays ouvert, et partit à la quatrième veille comme il l’avait dit. (5) Le septième jour, il marchait encore quand il apprit, par ses éclaireurs, que les troupes d’Arioviste étaient à vingt mille pas des nôtres.
\subsection[{§ 42.}]{ \textsc{§ 42.} }
\noindent (1) Instruit de l’arrivée de César, Arioviste lui envoie des députés "Il acceptait la demande qui lui avait été faite d’une entrevue, maintenant que César s’était approché davantage, et qu’il pensait pouvoir le faire sans danger." (2) César ne rejeta point sa proposition. Il crut qu’Arioviste était revenu à des idées plus saines, puisque cette conférence qu’il lui avait d’abord refusée, il l’offrait de son propre mouvement. (3) II espérait que, dès qu’il connaîtrait ses demandes, le souvenir des insignes bienfaits de César et du peuple romain triompherait de son opiniâtreté. L'entrevue fut fixée au cinquième jour à partir de celui-là. (4) Dans cet intervalle, on s’envoya de part et d’autre de fréquents messages ; Arioviste demanda que César n’amenât aucun fantassin ; il craignait des embûches et une surprise ; tous deux seraient accompagnés par de la cavalerie ; s’il en était autrement, il ne viendrait point. (5) César, ne voulant pas que la conférence manquât par aucun prétexte, et n’osant commettre sa sûreté à la cavalerie gauloise, trouva un expédient plus commode ; il prit tous leurs chevaux aux cavaliers gaulois, et les fit monter par des soldats de la dixième légion, qui avait toute sa confiance, afin d’avoir, s’il en était besoin, une garde dévouée. (6) Cela fit dire assez plaisamment à un des soldats de cette légion "Que César les favorisait au-delà de ses promesses, puisqu’ayant promis aux soldats de la dixième légion d’en faire sa cohorte prétorienne, il les faisait chevaliers."
\subsection[{§ 43.}]{ \textsc{§ 43.} }
\noindent (1) Dans une vaste plaine était un tertre assez é levé, à une distance à peu près égale des deux camps. Ce fut là que, selon les conventions, eut lieu l’entrevue. (2) César plaça à deux cents pas de ce tertre la légion qu’il avait amenée sur les chevaux des Gaulois. Les cavaliers d’Arioviste s’arrêtèrent à la même distance ; (3) celui-ci demanda que l’on s’entretînt à cheval, et que dix hommes fussent leur seule escorte à cette conférence. (4) Lorsqu’on fut en présence, César commença son discours par lui rappeler ses bienfaits et ceux du sénat : "Il avait reçu du sénat le nom de roi, le titre d’ami ; on lui avait envoyé les plus grands présents, faveur accordée à peu d’étrangers, et qui n’était d’ordinaire que la récompense d’éminents services. (5) Il avait, lui, sans y avoir aucun droit, sans titre suffisant pour y prétendre, obtenu ces honneurs de la bienveillance et de la libéralité de César et du sénat. (6) Il lui rappela aussi les liens aussi anciens que légitimes qui unissaient les Héduens à la république ; (7) les nombreux et honorables sénatus-consultes rendus en leur faveur ; la suprématie dont ils avaient joui de tout temps dans la Gaule entière, avant même de rechercher notre amitié ; (8) l’usage du peuple romain étant de vouloir que ses alliés et ses amis non seulement ne perdissent rien de leur puissance, mais encore gagnassent en crédit, en dignité, en honneur. Comment souffrir que ce qu’ils avaient apporté dans l’alliance romaine leur fût ravi ? " (9) Il finit par lui réitérer les demandes déjà faites par ses députés, qu’il ne fît la guerre ni aux Héduens ni à leurs alliés ; qu’il rendît les otages ; et, s’il ne pouvait renvoyer chez eux aucune partie des Germains, qu’au moins il ne permît pas à d’autres de passer le Rhin.
\subsection[{§ 44.}]{ \textsc{§ 44.} }
\noindent (1) Arioviste répondit peu de choses aux demandes de César, et parla beaucoup de son mérite personnel. (2) "Il n’avait point passé le Rhin de son propre mouvement, mais à la prière et à la sollicitation des Gaulois ; il n’aurait pas quitté son pays et ses proches sans la certitude d’une riche récompense. Les établissements qu’il possédait dans la Gaule lui avaient été concédés par les Gaulois eux-mêmes ; ils avaient donné volontairement des otages ; il levait par le droit de la guerre les contributions que les vainqueurs ont coutume d’imposer aux vaincus ; (3) les Gaulois avaient commencé les hostilités bien loin que ce fût lui ; les peuples de la Gaule étaient venus l’attaquer en masse et poser leur camp en face du sien ; il avait, dans un seul combat, vaincu et dispersé toutes ces forces ; s’ils veulent de nouveau tenter le sort des armes, il est de nouveau prêt à combattre ; s’ils préfèrent la paix, il est injuste de lui refuser le tribut qu’ils avaient jusque-là payé de leur plein gré ; (5) l’amitié du peuple romain devait lui apporter honneur et profit et non pas tourner à son détriment ; il l’avait recherchée dans cet espoir. Si Rome intervient pour lui enlever ses subsides et ses tributaires, il renoncera à son amitié avec autant d’empressement qu’il l’avait désirée. (6) S'il faisait passer dans la Gaule un grand nombre de Germains, c’était pour sa propre sûreté et non pour attaquer les Gaulois ; la preuve c’est qu’il n’était venu que parce qu’on l’avait appelé ; que loin d’être l’agresseur, il n’avait fait que se défendre. (7) Il était entré en Gaule avant les Romains ; jamais, avant ce temps, une armée romaine n’avait dépassé les limites de la province. (8) Que lui voulait-on ? Pourquoi venait-on sur ses terres ? Cette partie de la Gaule était sa province, comme celle-là était la nôtre. De même qu’on ne lui permettait pas d’envahir nos frontières, de même aussi c’était de notre part une iniquité que de l’interpeller dans l’exercice de son droit. Quant au titre de frères que le sénat avait donné aux Héduens, il n’était pas assez barbare, ni assez mal informé de ce qui s’était passé, pour ignorer que dans la dernière guerre des Allobroges, les Héduens n’avaient pas envoyé de secours aux Romains, et qu’ils n’en avaient pas reçu d’eux dans leurs démêlés avec lui et les Séquanes. (10) Il avait lieu de soupçonner que, sous des semblants d’amitié, César destinait à sa ruine l’armée qu’il avait dans la Gaule. (11) S'il ne s’éloignait pas et ne faisait pas retirer ses troupes, il le tiendrait non pour ami mais pour ennemi. (12) En le faisant périr, il remplirait les voeux de beaucoup de nobles et des principaux de Rome ; il le savait par leurs propres messagers ; et sa mort lui vaudrait leur reconnaissance et leur amitié. (13) S'il se retirait et lui laissait la libre possession de la Gaule, il l’en récompenserait amplement, et ferait toutes les guerres que César voudrait entreprendre, sans fatigue ni danger de sa part."
\subsection[{§ 45.}]{ \textsc{§ 45.} }
\noindent (1) César prouva par beaucoup de raisons qu’il ne pouvait pas se désister de son dessein. "Il n’était ni dans ses habitudes ni dans celles du peuple romain d’abandonner des alliés qui avaient bien mérité de la république, et il ne pensait pas que la Gaule appartînt plutôt à Arioviste qu’aux Romains. (2) Q. Fabius Maximus avait vaincu les Arvernes et les Rutènes, et Rome, leur pardonnant, ne les avait par réduits en province, et ne leur avait pas imposé de tribut. (3) S'il fallait s’en rapporter à la priorité de temps, elle serait pour le peuple romain un juste titre à l’empire de la Gaule ; s’il fallait s’en tenir au décret du sénat, elle devait être libre, puisqu’il avait voulu que, vaincue, elle conservât ses lois."
\subsection[{§ 46.}]{ \textsc{§ 46.} }
\noindent (1) Pendant ce colloque, on vint annoncer à César que les cavaliers d’Arioviste s’approchaient du tertre et s’avançaient vers les nôtres, sur lesquels ils lançaient déjà des pierres et des traits. (2) César mit fin à l’entretien, se retira vers les siens et leur défendit de renvoyer un seul trait aux ennemis. (3) Car, bien qu’il jugeât pouvoir avec les cavaliers de sa légion d’élite soutenir le combat sans danger, il ne voulait cependant pas donner aux ennemis qu’il devait repousser sujet de dire qu’on les avait surpris à la faveur d’une conférence perfide. (4) Quand on connut dans le camp l’arrogance des paroles d’Arioviste, la défense par lui faite aux Romains d’entrer dans la Gaule la brusque attaque de ses cavaliers contre les nôtres, laquelle avait rompu l’entrevue, l’armée en ressentit une impatience et un désir plus vifs de combattre.
\subsection[{§ 47.}]{ \textsc{§ 47.} }
\noindent (1) Deux jours après, Arioviste fit dire à César qu’il désirait reprendre les négociations entamées et restées inachevées, le priant de fixer un jour pour un nouvel entretien, ou au moins de lui envoyer un de ses lieutenants. (2) César ne jugea pas à propos d’accepter cette entrevue, d’autant plus que la veille on n’avait pu empêcher les Germains de lancer des traits sur nos troupes. (3) Il sentait aussi qu’il était très dangereux d’envoyer un de ses lieutenants et de l’exposer à la cruauté de ces barbares. (4) Il crut plus convenable de députer vers Arioviste C. Valérius Procillus, jeune homme plein de courage et de mérite, dont le père, C. Valérius Caburus, avait été fait citoyen romain par C. Valérius Flaccus. Sa fidélité était connue et il savait la langue gauloise, qu’une longue habitude avait rendue familière à Arioviste, et les Germains n’avaient aucune raison pour le maltraiter. César lui adjoignit M. Métius qui avait été hôte d’Arioviste. (5) II les chargea de prendre connaissance des propositions de ce dernier et de les lui rapporter. (6) Lorsqu’Arioviste les vit venir à lui dans son camp, il s’écria en présence de ses soldats : "Que venez-vous faire ici ? Est-ce pour espionner ? " Et, sans leur donner le temps de s’expliquer, il les jeta dans les fers.
\subsection[{§ 48.}]{ \textsc{§ 48.} }
\noindent (1) Le même jour il leva son camp, et vint prendre position au pied d’une montagne, à six mille pas de celui de César. (2) Le lendemain, il fit marcher ses troupes à la vue de l’armée romaine et alla camper à deux mille pas de là, dans la vue d’intercepter le grain et les vivres qu’expédiaient les Séquanes et les Héduens. (3) Pendant les cinq jours qui suivirent, César fit avancer ses troupes à la tête du camp, et les rangea en bataille, pour laisser à Arioviste toute liberté d’engager le combat. (4) Arioviste, durant tout ce temps, retint son armée dans son camp, et fit chaque jour des escarmouches de cavalerie. Les Germains étaient particulièrement exercés à ce genre de combat. (5) Ils avaient un corps de six mille cavaliers et d’un pareil nombre de fantassins des plus agiles et des plus courageux ; chaque cavalier avait choisi le sien sur toute l’armée pour lui confier son salut ; ils combattaient ensemble. (6) La cavalerie se repliait sur eux ; ceux-ci, dans les moments difficiles, venaient à son secours ; si un cavalier, grièvement blessé, tombait de cheval, ils l’environnaient ; (7) s’il fallait se porter en avant ou faire une retraite précipitée, l’exercice les avait rendus si agiles qu’en se tenant à la crinière des chevaux, ils les égalaient à la course.
\subsection[{§ 49.}]{ \textsc{§ 49.} }
\noindent (1) Voyant qu’Arioviste se tenait renfermé dans son camp, César, afin de n’être pas plus longtemps séparé des subsistances, choisit une position avantageuse à environ six cents pas au-delà de celle que les Germains occupaient, et ayant formé son armée sur trois lignes, il vint occuper cette position. (2) II fit tenir la première et la seconde sous les armes et travailler la troisième aux retranchements. (3) Ce lieu était, comme nous l’avons dit, à six cents pas à peu près de l’ennemi. Arioviste détacha seize mille hommes de troupes légè res avec toute sa cavalerie pour effrayer nos soldats et interrompre les travaux. (4) Néanmoins, César, selon qu’il l’avait arrêté d’avance, ordonna aux deux premières lignes de repousser l’attaque, à la troisième de continuer le retranchement. (5) Le camp une fois fortifié, César y laissa deux légions et une partie des auxiliaires, et ramena les quatre autres au camp principal.
\subsection[{§ 50.}]{ \textsc{§ 50.} }
\noindent (1) Le lendemain, selon son usage, il fit sortir ses troupes des deux camps, et, s’étant avancé à quelque distance du grand, il les mit en bataille et présenta le combat aux ennemis. (2) Voyant qu’ils ne faisaient aucun mouvement, il fit rentrer l’armée vers le milieu du jour. Alors seulement Arioviste détacha une grande partie de ses forces pour l’attaque du petit camp. Un combat opiniâtre se prolongea jusqu’au soir. (3) Au coucher du soleil, Arioviste retira ses troupes ; il y eut beaucoup de blessés de part et d’autre. (4) Comme César s’enquérait des prisonniers pourquoi Arioviste refusait de combattre, il apprit que c’était la coutume chez les Germains de faire décider par les femmes, d’après les sorts et les règles de la divination, s’il fallait ou non livrer bataille, (5) et qu’elles avaient déclaré toute victoire impossible pour eux, s’ils combattaient avant la nouvelle lune.
\subsection[{§ 51.}]{ \textsc{§ 51.} }
\noindent (1) Le jour suivant, César laissa dans les deux camps une garde qui lui parut suffisante, et plaça en présence des ennemis toutes les troupes auxiliaires, en avant du petit. Comme le nombre des légionnaires était inférieur à celui des Germains, les alliés lui servirent à étendre son front. Il rangea l’armée sur trois lignes et s’avança contre le camp ennemi. (2) Alors, les Germains, forcés enfin de combattre, sortirent de leur camp et se placèrent, par ordre de nations à des intervalles égaux, Harudes, Marcomans, Triboques, Vangions, Némètes, Sédusiens, Suèves ; ils formèrent autour de leur armée une enceinte d’équipages et de chariots, afin de s’interdire tout espoir de fuite. (3) Placées sur ces bagages, les femmes tendaient les bras aux soldats qui marchaient au combat, et les conjuraient en pleurant de ne les point livrer en esclavage aux Romains.
\subsection[{§ 52.}]{ \textsc{§ 52.} }
\noindent (1) César mit à la tête de chaque légion un de ses lieutenants et un questeur, pour que chacun eût en eux des témoins de sa valeur. (2) Il engagea le combat par son aile droite, du côté où il avait remarqué que l’ennemi était le plus faible. (3) Au signal donné, les soldats se précipitèrent avec une telle impétuosité et l’ennemi accourut si vite qu’on n’eut pas le temps de lancer les javelots ; (4) on ne s’en servit point, et l’on combattit de près avec le glaive. Mais les Germains, ayant promptement formé leur phalange accoutumée, soutinrent le choc de nos armes. (5) On vit alors plusieurs de nos soldats s’élancer sur cette phalange, arracher avec la main les boucliers de l’ennemi, et le blesser en le frappant d’en haut. (6) Tandis que l’aile gauche des Germains était rompue et mise en déroute, à l’aile droite les masses ennemies nous pressaient vivement. (7) Le jeune P. Crassus, qui commandait la cavalerie, s’en aperçut, et plus libre que ceux qui étaient engagés dans la mêlée, il envoya la troisième ligne au secours de nos légions ébranlées.
\subsection[{§ 53.}]{ \textsc{§ 53.} }
\noindent (1) Le combat fut ainsi rétabli ; tous les ennemis prirent la fuite, et ne s’arrêtèrent qu’après être parvenus au Rhin à cinquante mille pas environ du champ de bataille ; (2) quelques-uns, se fiant à leurs forces, essayèrent de le passer à la nage, d’autres se sauvèrent sur des barques ; (3) de ce nombre fut Arioviste qui, trouvant une nacelle attachée au rivage, s’échappa ainsi. Tous les autres furent taillés en pièces par notre cavalerie qui s’était mise à leur poursuite. (4) Arioviste avait deux femmes, la première, Suève de nation, qu’il avait amenée avec lui de sa patrie ; la seconde, native du Norique, soeur du roi Voccion, et qu’il avait épousée dans la Gaule, quand son frère la lui eut envoyée ; toutes deux périrent dans la déroute. De leurs filles, l’une fut tuée et l’autre prise. (5) C. Valérius Procillus était entraîné, chargé d’une triple chaîne, par ses gardiens fugitifs. Il fut retrouvé par César lui-même qui poursuivait l’ennemi, à la tête de la cavalerie. (6) Cette rencontre ne lui causa pas moins de plaisir que la victoire même ; l’homme le plus considéré de la province, son ami et son hôte, était arraché des mains des ennemis et lui était rendu ; la fortune n’avait pas voulu troubler par une telle perte sa joie et son triomphe. (7) Procillus lui dit qu’il avait vu trois fois consulter le sort pour savoir s’il serait immédiatement brûlé ou si on renverrait son supplice à un autre temps ; et que le sort favorable l’avait sauvé. (8) M. Métius fut aussi rejoint et ramené à César.
\subsection[{§ 54.}]{ \textsc{§ 54.} }
\noindent (1) Le bruit de cette victoire étant parvenu au-delà du Rhin, les Suèves, qui étaient déjà arrivés sur les bords de ce fleuve, regagnèrent leur pays. Les habitants de la rive, les voyant épouvantés, les poursuivirent et en tuèrent un grand nombre. (2) César, après avoir ainsi terminé deux grandes guerres en une seule campagne, conduisit l’armée en quartier d’hiver chez les Séquanes, un peu plus tôt que la saison ne l’exigeait. Il en confia le commandement à Labiénus et partit pour aller tenir les assemblées dans la Gaule citérieure.
\section[{Livre II}]{Livre II}\renewcommand{\leftmark}{Livre II}

\subsection[{§ 1.}]{ \textsc{§ 1.} }
\noindent (1) Pendant que César était, comme nous l’avons dit, en quartier d’hiver dans la Gaule citérieure, les bruits publics lui apprirent et les lettres de Labiénus lui confirmèrent que les Belges, formant, comme on a vu, la troisième partie de la Gaule, se liguaient contre le peuple romain et se donnaient mutuellement des otages. (2) Cette coalition avait diverses causes ; d’abord, ils craignaient qu’après avoir pacifié toute la Gaule, notre armée ne se portât sur leur territoire ; (3) en second lieu, ils étaient sollicités par un grand nombre de Gaulois : ceux qui n’avaient pas voulu supporter le séjour des Germains en Gaule, voyaient aussi avec peine l’armée des Romains hiverner dans le pays et y rester à demeure : d’autres, par inconstance et légèreté d’esprit, désiraient un changement de domination ; (4) quelques-uns enfin, les plus puissants et ceux qui, à l’aide de leurs richesses, pouvaient soudoyer des hommes et s’emparaient ordinairement du pouvoir, prévoyaient que ces usurpations seraient moins faciles sous notre gouvernement.
\subsection[{§ 2.}]{ \textsc{§ 2.} }
\noindent (1) Inquiet de tous ces rapports, César leva deux nouvelles légions dans la Gaule citérieure, et les envoya, au commencement de l’été, dans la Gaule intérieure, sous le commandement de Q. Pédius, son lieutenant. (2) Lui-même rejoignit l’armée, dès que les fourrages commencèrent à être abondants ; (3) il chargea les Sénons et les autres Gaulois, voisins des Belges, d’observer ce qui se passait chez eux, et de l’en instruire. (4) Ils lui annoncèrent unanimement que ce peuple levait des troupes et qu’une armée se rassemblait. (5) César alors n’hésite plus, et fixe son départ au douzième jour. (6) Après avoir pourvu aux vivres, il lève son camp et arrive en quinze jours à peu près aux frontières de la Belgique.
\subsection[{§ 3.}]{ \textsc{§ 3.} }
\noindent (1) Son arrivée fut imprévue et personne ne s’attendait à tant de célérité ; les Rèmes, voisins immédiats des Belges, lui députèrent Iccios et Andocumborios, les premiers de leur cité, (2) chargés de lui dire qu’ils se mettaient eux et tout ce qu’ils possédaient sous la foi et pouvoir du peuple romain, qu’ils n’avaient point voulu se liguer avec les autres Belges, ni prendre part à cette conjuration contre les Romains ; (3) qu’ils étaient prêts à donner des otages, à faire ce qui leur serait ordonné, à le recevoir dans leurs places, à lui fournir des vivres et tous autres secours ; (4) que tout le reste de la Belgique était en armes ; que les Germains, habitant en deçà du Rhin, s’étaient joints aux Belges (5) et que telle était la fureur de cette multitude, qu’eux-mêmes, frères et alliés des Suessions, obéissant aux mêmes lois, ayant le même gouvernement et les mêmes magistrats, n’avaient pu les détourner d’entrer dans la confédération.
\subsection[{§ 4.}]{ \textsc{§ 4.} }
\noindent (1) César leur demanda quels étaient les peuples en armes, leur nombre et leurs forces militaires. Il apprit que la plupart des Belges étaient originaires de Germanie ; qu’ayant anciennement passé le Rhin, ils s’étaient fixés en Belgique, à cause de la fertilité du sol, et en avaient chassé les Gaulois qui l’habitaient avant eux ; (2) que seuls, du temps de nos pères, quand les Teutons et les Cimbres eurent ravagé toute la Gaule, ils les avaient empêchés d’entrer sur leurs terres. (3) Ce souvenir leur inspirait une haute opinion d’eux-mêmes et leur donnait de hautes prétentions militaires. (4) Quant à leur nombre, les Rèmes avaient à ce sujet les données les plus certaines, en ce que, unis avec eux par le voisinage et les alliances, ils connaissaient le contingent que, dans l’assemblée générale des Belges, chaque peuple avait promis pour cette guerre. (5) Les Bellovaques tenaient le premier rang parmi eux par leur courage, leur influence et leur population : ils pouvaient mettre cent mille hommes sous les armes : ils en avaient promis soixante mille d’élite et demandaient la direction de toute la guerre. (6) Les Suessions, leurs voisins, possédaient un territoire très étendu et très fertile ; (7) ils avaient eu pour roi, de notre temps encore, Diviciacos, le plus puissant chef de la Gaule, qui à une grande partie de ces régions joignait aussi l’empire de la Bretagne. Galba était maintenant leur roi, et le commandement lui avait été déféré d’un commun accord, à cause de son équité et de sa sagesse. Ils possédaient douze villes, et avaient promis cinquante mille hommes. (8) Autant en donnaient les Nerviens, réputés les plus barbares d’entre ces peuples, et placés à l’extrémité de la Belgique ; (9) les Atrébates en fournissaient quinze mille ; les Ambiens, dix mille ; les Morins, vingt-cinq mille ; les Ménapes, neuf mille ; les Calétes, dix mille ; les Véliocasses et les Viromandues le même nombre ; les Atuatuques, dix-neuf mille ; (10) les Condruses, les Éburons, les Caeroesi et les Pémanes, compris sous la dénomination commune de Germains, devaient en envoyer quarante mille.
\subsection[{§ 5.}]{ \textsc{§ 5.} }
\noindent (1) César encouragea les Rèmes par des paroles bienveillantes, et exigea que leur sénat se rendît auprès de lui, et que les enfants des familles les plus distinguées lui fussent amenés en otages ; ce qui fut ponctuellement fait au jour indiqué. (2) Il anime par de vives exhortations le zèle de l’Héduen Diviciacos ; et lui représente combien il importe à la république et au salut commun de diviser les forces de l’ennemi, afin de n’avoir pas une si grande multitude à combattre à la fois. (3) Il suffit pour cela que les Héduens fassent entrer leurs troupes sur le territoire des Bellovaques et se mettent à le ravager. César fait partir Diviciacos avec cette commission. (4) Dès qu’il apprit par ses éclaireurs et par les Rèmes, que les Belges marchaient sur lui avec toutes leurs forces réunies et n’étaient déjà plus qu’à peu de distance, il se hâta de faire passer à son armée la rivière d’Aisne, qui est à l’extrême frontière des Rèmes, et assit son camp sur la rive. (5) De cette manière, la rivière défendait un des côtés du camp ; ce qui était à la suite de l’armée se trouvait à l’abri des atteintes de l’ennemi ; et le transport des vivres qu’envoyaient les Rèmes et les autres peuples pouvait s’effectuer sans péril. (6) Sur cette rivière était un pont. Il y plaça une garde, et laissa sur l’autre rive Q. Titurius Sabinus, son lieutenant, avec six cohortes : il fit fortifier le camp d’un retranchement de douze pieds de haut et d’un fossé de dix-huit pieds de profondeur.
\subsection[{§ 6.}]{ \textsc{§ 6.} }
\noindent (1) À huit mille pas de ce camp était une ville des Rèmes, appelée Bibrax. Les Belges dans leur marche l’attaquèrent vivement. Elle se défendit tout le jour avec peine. (2) Leur manière de faire les sièges est semblable à celle des Gaulois. Lorsqu’ils ont entièrement entouré la place avec leurs troupes, ils lancent de tous côtés des pierres sur le rempart ; quand ils en ont écarté ceux qui le défendent, ils forment la tortue, s’approchent des portes et sapent la muraille. (3) Cela était alors aisé ; car cette grêle de pierres et de traits rendait toute résistance impossible du haut des remparts. (4) Lorsque la nuit eut mis fin à l’attaque, le Rème Iccios, homme d’une haute naissance et d’un grand crédit, qui commandait alors dans la place, et un de ceux qui avaient été députés vers César pour traiter de la paix, lui dépêcha des courriers pour l’informer que s’il n’était promptement secouru, il ne pouvait tenir plus longtemps.
\subsection[{§ 7.}]{ \textsc{§ 7.} }
\noindent (1) Vers le milieu de la nuit, César fit partir, sous la conduite des mêmes hommes que lui avait envoyés Iccios, des Numides, des archers crétois et des frondeurs baléares. (2) Leur arrivée ranima l’espoir des assiégés, leur inspira l’ardeur de se défendre, et enleva en même temps aux ennemis l’espérance de prendre la place. (3) Ils restèrent quelque temps à l’entour, dévastèrent la campagne, brûlèrent les bourgs et les maisons qui se trouvaient sur leur route, se dirigèrent avec toutes leurs troupes vers le camp de César, et placèrent le leur à moins de deux mille pas. (4) On pouvait conjecturer, d’après les feux et la fumée, qu’il avait une étendue de plus de huit mille pas.
\subsection[{§ 8.}]{ \textsc{§ 8.} }
\noindent (1) César résolut d’abord, à cause du grand nombre des ennemis et de la haute idée qu’il avait de leur courage, de différer la bataille. (2) Chaque jour cependant, par des combats de cavalerie, il éprouvait la valeur de l’ennemi et l’audace des siens. (3) Quand il se fut assuré que les nôtres ne lui étaient point inférieurs, il marqua le champ de bataille, en avant du camp, dans une position naturellement avantageuse ; la colline sur laquelle était placé le camp s’élevait insensiblement au-dessus de la plaine, et offrait autant d’étendue qu’il en fallait pour y déployer les troupes ; elle s’abaissait à gauche et à droite, et se relevait vers le centre par une légère éminence qui redescendait en pente douce vers la plaine. À l’un et l’autre côté de cette colline, César fit creuser un fossé transversal d’environ quatre cents pas ; (4) aux deux extrémités, il éleva des forts et y plaça des machines de guerre, afin d’empêcher que des ennemis si supérieurs en nombre ne vinssent le prendre en flanc et l’envelopper pendant le combat. (5) Cela fait, il laissa dans le camp les deux légions qu’il avait levées récemment, pour servir au besoin de réserve, et rangea les six autres en bataille devant le camp. L'ennemi avait aussi fait sortir ses troupes et formé ses lignes.
\subsection[{§ 9.}]{ \textsc{§ 9.} }
\noindent (1) Il y avait un marais peu étendu entre notre armée et celle des ennemis. Ils attendaient que les nôtres le traversassent ; nos troupes de leur côté, sous les armes, se tenaient prêtes à attaquer les Belges, s’ils s’engageaient les premiers dans le passage. (2) Cependant la cavalerie engageait le combat de part et d’autre. Aucun des deux partis ne voulant passer le premier, César, après le succès d’une charge de cavalerie, fit rentrer ses légions dans le camp. (3) Aussitôt les ennemis se dirigèrent vers la rivière d’Aisne, qui était, comme nous l’avons dit, derrière nous. (4) Ayant trouvé des endroits guéables ils essayèrent d’y faire passer une partie de leurs troupes, dans le dessein, soit de prendre, s’ils le pouvaient, le fort commandé par le lieutenant Q. Titurius et de rompre le pont, (5) soit, s’ils n’y réussissaient pas, de ravager le territoire des Rèmes, qui nous étaient d’une grande ressource dans cette guerre, et d’intercepter nos convois.
\subsection[{§ 10.}]{ \textsc{§ 10.} }
\noindent (1) César, averti par Titurius, passa le pont avec toute sa cavalerie, ses Numides armés à la légère, ses frondeurs, ses archers, et marcha à l’ennemi. (2) Alors s’engagea un combat opiniâtre. Les nôtres ayant attaqué les Belges dans les embarras du passage, en tuèrent un grand nombre. (3) Les autres, pleins d’audace, s’efforçaient de passer sur le corps de leurs compagnons ; une grêle de traits les repoussa. Ceux qui avaient les premiers traversé l’Aisne furent enveloppés et taillés en pièces par la cavalerie. (4) Les ennemis, se voyant déchus de l’espoir d’emporter le fort et de traverser la rivière, ne pouvant nous attirer pour combattre sur un terrain désavantageux, et les vivres commençant à leur manquer, tinrent conseil et arrêtèrent que ce qu’il y avait de mieux était de retourner chacun dans son pays, et de se tenir prêts à marcher tous à la défense du premier que l’armée romaine envahirait, ils combattraient avec plus d’avantage sur leur propre territoire que sur des terres étrangères, et les vivres chez eux leur seraient assurés. (5) Celui de leurs motifs qui eut le plus de poids pour cette détermination, ce fut la nouvelle que Diviciacos et les Héduens approchaient des frontières des Bellovaques. On ne put persuader à ces derniers de rester plus longtemps, ni les empêcher d’aller défendre leurs biens.
\subsection[{§ 11.}]{ \textsc{§ 11.} }
\noindent (1) Le départ étant résolu, dès la seconde veille, ils sortirent de leur camp à grand bruit, en tumulte, sans ordre fixe, sans être commandés par personne, prenant chacun le premier chemin qui s’offrait, et se hâtant de gagner leur pays, ce qui faisait ressembler ce départ à une fuite. (2) César aussitôt averti par ses vedettes, mais craignant une embuscade, dans l’ignorance où il était encore de la cause de cette retraite, retint son armée dans le camp même de sa cavalerie. (3) Au point du jour, ce départ lui ayant été confirmé par ses éclaireurs, il détacha toute sa cavalerie, pour arrêter l’arrière-garde. Il en confia le commandement à Q. Pédius et à Aurunculéius Cotta, ses lieutenants. T. Labiénus, un autre de ses lieutenants, eut ordre de les suivre avec trois légions. (4) Ils atteignirent l’arrière-garde ennemie, la poursuivirent pendant plusieurs milles, et on avait tué un grand nombre de ces fuyards, lorsque les derniers rangs, auxquels nous étions arrivés, firent halte et soutinrent notre choc avec beaucoup de vigueur ; (5) mais ceux qui étaient en avant, se voyant éloignés du péril, et n’étant retenus ni par la nécessité de se défendre, ni par les ordres d’aucun chef, eurent à peine entendu les cris des combattants, qu’ils rompirent leurs rangs, et cherchèrent tous leur salut dans la fuite. (6) Ainsi, sans courir aucun danger, les nôtres tuèrent à l’ennemi autant d’hommes que le permit la durée du jour : au coucher du soleil, ils cessèrent la poursuite et rentrèrent au camp, comme il leur avait été ordonné.
\subsection[{§ 12.}]{ \textsc{§ 12.} }
\noindent (1) Le lendemain, César, avant que l’ennemi se fû t rallié et remis de sa terreur, dirigea son armée vers le pays des Suessions, contigu à celui des Rèmes, et, après une longue marche, arriva devant la ville de Noviodunum. (2) Il essaya de l’emporter d’assaut, sur ce qu’il avait appris qu’elle manquait de garnison ; mais la largeur des fossés, la hauteur de ses murs défendus par un petit nombre d’hommes, l’empêchèrent de s’en rendre maître. (3) Il retrancha son camp, et se mit à faire des mantelets et à disposer tout ce qui était nécessaire pour le siège. (4) Pendant ces préparatifs, tous ceux des Suessions qui avaient échappé à la défaite entrèrent la nuit suivante dans la place. (5) On pousse aussitôt les mantelets contre les murs, on élève la terrasse, on établit les tours. Les Gaulois, effrayés de la grandeur de ces travaux qu’ils n’avaient jamais vus, dont ils n’avaient jamais entendu parler, et de la promptitude des Romains à les exécuter, envoient des députés à César pour traiter de leur reddition ; et, sur la prière des Rèmes, ils obtiennent la vie sauve.
\subsection[{§ 13.}]{ \textsc{§ 13.} }
\noindent (1) César reçut pour otages les principaux de la ville, les deux fils du roi Galba lui-même, se fit livrer toutes les armes de la place, accepta la soumission des Suessions, et marcha avec son armée contre les Bellovaques. (2) Ils s’étaient renfermés avec tous leurs biens dans la place de Bratuspantium. Lorsque César et son armée en furent à cinq milles environ, tous les vieillards, sortant de la ville, vinrent lui tendre les mains et lui annoncer qu’ils se mettaient sous sa protection et sous sa puissance ; et qu’ils ne voulaient point prendre les armes contre le peuple romain. (3) Comme il s’était approché de la place et s’occupait à établir son camp, les enfants et les femmes tendaient aussi les mains du haut des murs, selon leur manière de supplier et nous demandaient la paix.
\subsection[{§ 14.}]{ \textsc{§ 14.} }
\noindent (1) Diviciacos intercéda pour eux (car depuis la retraite des Belges, il avait renvoyé les troupes héduennes, et était revenu auprès de César). (2) "De tout temps, dit-il, les Bellovaques ont joui de la confiance et de l’amitié des Héduens ; (3) entraînés par des chefs qui leur disaient que les Héduens, réduits par César à la condition d’esclaves, avaient à souffrir toutes sortes d’indignités et d’outrages, ils se sont détachés de ce peuple, et ont pris les armes contre les Romains. (4) Les auteurs de ces conseils, voyant quelles calamités ils avaient attirées sur leur pays, viennent de s’enfuir en Bretagne. (5) Ce ne sont pas seulement les Bellovaques qui le prient ; les Héduens eux-mêmes réclament pour eux sa clémence et sa douceur. (6) S'il se rend à leurs prières, il augmentera le crédit des Héduens auprès de tous les Belges, qui leur prêtent ordinairement des secours et leur appui quand ils ont quelque guerre à soutenir.
\subsection[{§ 15.}]{ \textsc{§ 15.} }
\noindent (1) César répondit qu’en considération de Diviciacos et des Héduens, il acceptait la soumission des Bellovaques et leur accordait la vie ; mais, comme cette cité était une des premières de la Belgique par son importance et sa population, il demanda six cents otages. (2) Quand ils eurent été livrés ainsi que toutes les armes trouvées dans la ville, il marcha contre les Ambiens, qui mirent aussitôt leurs personnes et leurs biens à sa discrétion. (3) Au territoire de ces derniers touchait celui des Nerviens. César s’informa du caractère et des moeurs de ce peuple, et apprit que (4) chez eux tout accès était interdit aux marchands étrangers ; qu’ils proscrivaient l’usage du vin et des autres superfluités, les regardant comme propres à énerver leurs âmes et à amollir le courage ; (5) que c’étaient des hommes barbares et intrépides ; qu’ils accusaient amèrement les autres Belges de s’être donnés au peuple romain et d’avoir dégénéré de la valeur de leurs pères ; qu’ils avaient résolu de n’envoyer aucun député, et de n’accepter aucune proposition de paix.
\subsection[{§ 16.}]{ \textsc{§ 16.} }
\noindent (1) Après trois jours de marche sur leur territoire, César apprit de ses prisonniers que la Sambre n’était pas à plus de dix milles de son camp, (2) que les Nerviens étaient postés de l’autre côté de cette rivière, et y attendaient l’arrivée des Romains ; ils étaient réunis aux Atrébates et aux Viromandues, leurs voisins, (3) auxquels ils avaient persuadé de partager les chances de cette guerre ; (4) ils attendaient encore des Atuatuques, déjà en route, un renfort de troupes ; (5) les femmes et tous ceux que leur âge rendait inutiles pour le combat avaient été rassemblés dans un lieu dont les marais défendaient l’accès à une armée.
\subsection[{§ 17.}]{ \textsc{§ 17.} }
\noindent (1) Sur cet avis, César envoya des éclaireurs et des centurions pour choisir un emplacement propre à un camp. (2) Un certain nombre de Belges et d’autres Gaulois récemment soumis le suivaient et faisaient route avec lui : quelques-uns d’entre eux, comme on le sut depuis par les prisonniers, ayant observé attentivement, dans ces derniers jours, la marche habituelle de notre armée, se rendirent de nuit auprès des Nerviens, et les informèrent qu’entre chacune des légions il y avait une grande quantité de bagages, qu’il serait aisé d’attaquer la première, au moment où elle entrerait dans le camp, séparée des autres par un grand espace et embarrassée dans ses équipages ; (3) que cette légion une fois repoussée et ses bagages pillés, les autres n’oseraient faire résistance. (4) Un tel avis donné aux Nerviens pouvait leur servir beaucoup, en ce que de tout temps, très faibles en cavalerie (car aujourd’hui même, ils négligent cette partie, et toute leur force ne consiste que dans l’infanterie), ils ont eu l’habitude, pour arrêter plus facilement la cavalerie des peuples voisins, dans le cas où le désir du pillage l’attirerait sur leur territoire, de tailler et de courber de jeunes arbres, dont les branches, horizontalement dirigées et entrelacées de ronces et d’épines, forment des haies semblables à un mur, et qui leur servent de retranchement, à travers lesquels on ne peut ni pénétrer ni même voir. (5) Comme ces dispositions entravaient la marche de notre armée, les Nerviens crurent devoir profiter de l’avis qu’on leur donnait.
\subsection[{§ 18.}]{ \textsc{§ 18.} }
\noindent (1) Voici la nature de l’emplacement que les nôtres avait choisi pour le camp : c’était une colline qui depuis son sommet s’abaissait insensiblement vers la Sambre, rivière que nous avons nommée plus haut ; (2) il s’en élevait une autre d’une pente également douce, vis-à-vis de celle-là et sur le bord opposé, à deux cents pas environ. La partie inférieure en était découverte et la cime assez boisée pour que la vue ne pût y pénétrer. (3) L'ennemi se tenait caché dans ce bois : dans la partie découverte, le long de la rivière, se voyaient quelques postes de cavalerie. Cette rivière avait une profondeur d’à peu près trois pieds.
\subsection[{§ 19.}]{ \textsc{§ 19.} }
\noindent (1) César avait envoyé sa cavalerie en avant et suivait avec toutes ses troupes ; mais l’ordre de marche différait de ce que les Belges avaient rapporté aux Nerviens ; (2) car, en approchant de l’ennemi, César, selon son usage, s’avançait avec six légions sans équipages ; (3) venaient ensuite les bagages de toute l’armée, sous la garde de deux légions nouvellement levées, qui fermaient la marche. (4) Nos cavaliers passèrent la Sambre avec les frondeurs et les archers, et engagèrent le combat avec la cavalerie des ennemis. (5) Ceux-ci tour à tour se repliaient dans le bois vers les leurs et en sortaient de nouveau pour fondre sur nous ; mais les nôtres n’osaient les poursuivre au-delà de l’espace découvert. Cependant les six légions qui étaient arrivées les premières, s’étant partagé le travail, se mirent à fortifier le camp. (6) Dès que les ennemis cachés sur la hauteur aperçurent la tête de nos équipages (c’était le moment qu’ils avaient fixé pour l’attaque), ils sortirent dans le même ordre de bataille qu’ils avaient formé dans le bois, s’élancèrent tout à coup avec toutes leurs forces et tombèrent sur notre cavalerie. (7) Ils la culbutèrent sans peine, la mirent en désordre, et coururent vers la rivière avec une si incroyable vitesse qu’ils semblaient être presque au même instant dans le bois, et au milieu de la rivière, et sur nos bras. (8) Ce fut avec la même promptitude qu’ils attaquèrent notre colline, notre camp et les travailleurs occupés à le retrancher.
\subsection[{§ 20.}]{ \textsc{§ 20.} }
\noindent (1) César avait tout à faire à la fois : il fallait planter l’étendard qui donnait le signal de courir aux armes, faire sonner les trompettes, rappeler les travailleurs, rassembler ceux qui s’étaient écartés pour chercher les matériaux des retranchements, ranger l’armée en bataille, haranguer les soldats et donner le mot d’ordre. (2) De tant de choses à faire, la brièveté du temps et le choc victorieux de l’ennemi en rendaient une grande partie impossible. (3) À côté de ces difficultés, s’offraient pourtant deux ressources, l’expérience et l’habileté des soldats qui, instruits par les combats antérieurs, pouvaient se tracer à eux-mêmes leur conduite aussi bien que l’eussent fait des chefs, et ensuite, près de chaque légion, la présence des lieutenants à qui César avait défendu de s’éloigner avant que le camp fût fortifié. (4) Ces lieutenants, pressés par de si agiles assaillants, n’attendaient plus les ordres de César, et faisaient de leur propre autorité ce qu’ils jugeaient le plus convenable.
\subsection[{§ 21.}]{ \textsc{§ 21.} }
\noindent (1) César, après avoir pourvu au plus nécessaire, courut exhorter les soldats, selon que le hasard les lui offrait, et arriva à la dixième légion. (2) Pour toute harangue, il lui dit de se souvenir de son ancienne valeur, de ne point se troubler, et de soutenir vigoureusement le choc des ennemis ; (3) et, comme ceux-ci n’étaient plus qu’à la portée du trait, il donna le signal du combat. (4) Il partit pour faire ailleurs la même exhortation ; on était déjà aux prises. (5) L'engagement avait été si rapide, et l’ennemi si impatient de combattre, que l’on n’avait eu le temps ni de revêtir les insignes du commandement, ni même de mettre les casques et d’ôter l’enveloppe des boucliers. (6) Chaque soldat en revenant des travaux se plaça au hasard près du premier drapeau qu’il aperçut, afin de ne pas perdre, à chercher le sien, le temps de la bataille.
\subsection[{§ 22.}]{ \textsc{§ 22.} }
\noindent (1) L'armée s’était rangée plutôt comme l’avaient permis la nature du terrain, la pente de la colline et le peu de temps, que comme le demandaient les règles de l’art militaire. Comme les légions soutenaient l’attaque de l’ennemi, chacune de son côté, séparées les unes des autres par ces haies épaisses qui, comme nous l’avons dit précédemment, interceptaient la vue, on ne pouvait ni placer des réserves où il en fallait, ni pourvoir à ce qui était nécessaire sur chaque point, ni faire émaner tous les ordres d’un centre unique. (2) De cette confusion générale, s’ensuivaient des accidents et des fortunes diverses.
\subsection[{§ 23.}]{ \textsc{§ 23.} }
\noindent (1) Les soldats de la neuvième et de la dixième légion, placés à l’aile gauche de l’armée, après avoir lancé leurs traits, tombèrent sur les Atrébates, fatigués de leur course, hors d’haleine, percés de coups, et qui leur faisaient face. Ils les repoussèrent promptement de la hauteur jusqu’à la rivière, qu’ils essayèrent de passer ; mais on les poursuivit l’épée à la main, et on en tua un grand nombre au milieu des difficultés de ce passage. (2) Les nôtres n’hésitèrent pas de leur côté à traverser la rivière ; mais, s’étant engagés dans une position désavantageuse, l’ennemi revint sur ses pas, se défendit, et recommença le combat ; il fut mis en fuite. (3) Sur un autre point, deux de nos légions, la onzième et la huitième, avaient battu les Viromandues, avec lesquels elles en étaient venues aux mains, et les menaient battant depuis la hauteur jusque sur les rives mêmes de la Sambre. (4) Mais ces mouvements du centre et de l’aile gauche avaient laissé le camp presque entièrement à découvert ; l’aile droite se composait de la douzième légion et de la septième, placées à peu de distance l’une de l’autre : ce fut sur ce point que se portèrent, en masses très serrées, tous les Nerviens conduits par Boduognatos, leur général en chef. (5) Les uns enveloppèrent nos légions par le flanc découvert, les autres gagnèrent le haut du camp.
\subsection[{§ 24.}]{ \textsc{§ 24.} }
\noindent (1) En ce moment, nos cavaliers et nos fantassins armés à la légère, qui avaient été, comme je l’ai dit, repoussés ensemble par le premier choc des ennemis, et qui revenaient au camp, les rencontrèrent de front et s’enfuirent de nouveau dans une autre direction. (2) Les valets de l’armée qui, de la porte décumane et du sommet de la colline, avaient vu les nôtres traverser la rivière en vainqueurs, et étaient sortis pour piller, s’étant aperçus, en se retournant, que l’ennemi occupait notre camp, prirent précipitamment la fuite. (3) On entendait en même temps les cris d’épouvante des conducteurs de bagages, que la frayeur entraînait de côté et d’autre. (4) À l’aspect d’un tel désordre, les cavaliers trévires, très estimés chez les Gaulois pour leur valeur, et que leur cité avait envoyés à César comme auxiliaires, voyant notre camp rempli d’une multitude d’ennemis, les légions pressées et presque enveloppées, les valets, les cavaliers, les frondeurs, les Numides, dispersés et fuyant sur tous les points, désespérèrent de nos affaires, et, prenant la route de leur pays, (5) allèrent annoncer chez eux que les Romains avaient été repoussés et vaincus, et que leur camp, ainsi que leurs bagages, étaient au pouvoir des Nerviens.
\subsection[{§ 25.}]{ \textsc{§ 25.} }
\noindent (1) César, après avoir exhorté la dixième légion, s’était porté à l’aile droite, et y avait trouvé les troupes vivement pressées, les enseignes réunies en une seule place, les soldats de la douzième légion entassés et s’embarrassant l’un l’autre pour combattre, tous les centurions de la quatrième cohorte tués, le porte-enseigne mort, le drapeau perdu, presque tous les centurions des autres cohortes blessés ou tués, et, de ce nombre, le primipile P. Sextius Baculus, d’un courage remarquable, couvert de si nombreuses et si profondes blessures, qu’il ne pouvait plus se soutenir. Le reste était découragé ; des soldats des derniers rangs, se voyant sans chefs, quittaient le champ de bataille et se mettaient à l’abri des traits ; l’ennemi ne cessait d’arriver du bas de la colline, de presser le centre et de tourner les deux flancs ; nos affaires enfin étaient dans le plus mauvais état, et tout secours manquait pour les rétablir. (2) César arrache alors à un soldat de l’arrière-garde son bouclier (car il n’avait pas le sien), et s’avance à la première ligne ; il appelle les centurions par leurs noms, exhorte les autres soldats, fait porter en avant les enseignes et desserrer les rangs, pour qu’on puisse plus facilement se servir de l’épée. (3) Son arrivée rend l’espoir aux soldats et relève leur courage. Chacun veut, sous les yeux du général, faire preuve de zèle dans cette extrémité, et l’on parvient à ralentir un peu l’impétuosité de l’ennemi.
\subsection[{§ 26.}]{ \textsc{§ 26.} }
\noindent (1) César, remarquant que la septième légion placée près de là était aussi vivement pressée par l’ennemi, avertit les tribuns militaires de rapprocher peu à peu les deux légions, afin que, réunies, elles pussent marcher contre lui. (2) Comme par cette manoeuvre on se prêtait un mutuel secours, et qu’on ne craignait plus d’être pris à dos et enveloppé, on commença à résister avec plus d’audace et à combattre avec plus de courage. (3) Pendant ce temps, les deux légions qui, comme arrière-garde, portaient les bagages, arrivent au pas de course à la nouvelle du combat, et se montrent aux ennemis sur le haut de la colline. (4) De son côté, T. Labiénus, qui avait forcé leur camp, et qui, de cette position élevée, découvrait ce qui se passait dans le nôtre, envoie la dixième légion à notre secours. (5) Celle-ci, comprenant, par la fuite des cavaliers et des valets, dans quel état se trouvaient nos affaires, et de quel danger étaient menacés à la fois le camp, les légions et le général, fit la plus grande diligence.
\subsection[{§ 27.}]{ \textsc{§ 27.} }
\noindent (1) Leur arrivée changea tellement la face des choses, que ceux même des nôtres dont les blessures avaient é puisé les forces, s’appuyant sur leurs boucliers, recommençaient le combat ; que les valets, voyant l’ennemi frappé de terreur, se jetaient sans armes sur des hommes armés, (2) et que les cavaliers, pour effacer la honte de leur fuite par des actes de courage, devançaient partout les légionnaires dans la mêlée. (3) Mais les ennemis, dans leur dernier espoir de salut, déployèrent un tel courage, que, dès qu’il tombait des soldats aux premiers rangs, les plus proches prenaient leur place et combattaient de dessus leurs corps ; (4) que, de ces cadavres amoncelés, ceux qui survivaient lançaient, comme d’une éminence, leurs traits sur les nôtres, et nous renvoyaient nos propres javelots. (5) II n’y avait plus à s’étonner que des hommes si intrépides eussent osé traverser une large rivière, gravir des bords escarpés et combattre dans une position désavantageuse, difficultés qu’avait aplanies la grandeur de leur courage.
\subsection[{§ 28.}]{ \textsc{§ 28.} }
\noindent (1) Après cette bataille, où la race et le nom des Nerviens furent presque entièrement anéantis, les vieillards, que nous avons dit s’être retirés au milieu des marais avec les enfants et les femmes, instruits de ce désastre, ne voyant plus d’obstacles pour les vainqueurs ni de sûreté pour les vaincus, (2) sur l’avis unanime de ceux qui survivaient à la bataille, envoyèrent des députés à César et se rendirent à lui. Rappelant le malheur de leur pays, ils dirent que le nombre de leurs sénateurs se trouvait réduit de six cents à trois seulement, et que de soixante mille hommes en état de porter les armes, il en restait à peine cinq cents. (3) César voulut user de clémence envers ces infortunés suppliants, pourvut soigneusement à leur conservation, leur rendit leur territoire et leurs villes, et enjoignit aux peuples voisins de ne se permettre envers eux et de ne souffrir qu’il leur fût fait aucun outrage ni aucun mal.
\subsection[{§ 29.}]{ \textsc{§ 29.} }
\noindent (1) Les Atuatuques, dont il a été parlé plus haut, venaient avec toutes leurs troupes au secours des Nerviens ; dès qu’ils apprirent l’issue de la bataille, ils rebroussèrent chemin et retournèrent chez eux. (2) Ayant abandonné leurs villes et leurs forts, ils se retirèrent avec tout ce qu’ils possédaient dans une seule place, admirablement fortifiée par la nature.(3) Environnée sur tous les points de son enceinte par des rochers à pic et de profonds précipices, elle n’était accessible que d’un côté, par une pente douce, large d’environ deux cents pieds, et ils avaient pourvu à la défense de cet endroit au moyen d’une double muraille très élevée, en partie formée d’énormes quartiers de rocs et de poutres aiguisées. (4) C'étaient des descendants de ces Cimbres et de ces Teutons, qui, marchant contre notre province et contre l’Italie, avaient laissé en deçà du Rhin les bagages qu’ils ne pouvaient transporter avec eux, en confiant la garde et la défense à six mille des leurs. (5) Ceux-ci, après la défaite de leurs compagnons, avaient eu de longs démêlés avec les peuples voisins, attaquant et se défendant tour à tour ; et, après avoir fait la paix, ils s’étaient, d’un commun accord, fixés dans ces lieux.
\subsection[{§ 30.}]{ \textsc{§ 30.} }
\noindent (1) À l’arrivée de notre armée, ils firent d’abord de fréquentes sorties et engagèrent de petits combats contre nous ; (2) mais, quand nous eûmes établi une circonvallation de douze pieds de haut, dans un circuit de quinze milles, qu’elle fut garnie de forts nombreux, ils se tinrent renfermés dans la place. (3) Lorsqu’ils virent de loin qu’après avoir posé les mantelets et élevé la terrasse, nous construisions une tour, ils se mirent à en rire du haut de leurs murailles, et à nous demander à grands cris ce que nous prétendions faire, à une si grande distance, d’une si énorme machine ; (4) avec quelles mains, avec quelles forces des nains comme nous (car la plupart des Gaulois, à cause de l’élévation de leur taille, méprisent la petitesse de la nôtre) espéraient approcher de leurs murs une tour d’un si grand poids.
\subsection[{§ 31.}]{ \textsc{§ 31.} }
\noindent (1) Mais, dès qu’ils la virent se mouvoir et s’approcher de leurs murailles, frappés de ce spectacle nouveau et inconnu, ils envoyèrent à César, pour traiter de la paix, des députés qui lui dirent : (2) "Nous ne doutons plus que les Romains ne fassent la guerre avec l’assistance des dieux, puisqu’ils peuvent ébranler avec tant de promptitude de si hautes machines pour combattre de près ; (3) nous remettons entre leurs mains nos personnes et nos biens. (4) Nous ne demandons, nous n’implorons qu’une grâce. Si la clémence et la douceur de César, que nous avons entendu vanter, le portent à nous laisser la vie, qu’il ne nous dépouille pas de nos armes ; (5) tous nos voisins sont des ennemis jaloux de notre courage ; comment, si nous livrons nos armes, pourrons-nous nous défendre contre eux ? (6) Nous préférons, si tel doit être notre sort, souffrir tout du peuple romain que dépérir au milieu des supplices, par les mains de ceux dont nous avons été longtemps les maîtres.
\subsection[{§ 32.}]{ \textsc{§ 32.} }
\noindent (1) À cette demande César répondit "que, plutôt par habitude que par égard pour eux, il conserverait leur nation, pourvu qu’ils se rendissent avant que le bélier touchât leurs murailles ; (2) mais qu’il ne traiterait de la capitulation qu’après la remise de leurs armes : il fera pour eux ce qu’il a fait pour les Nerviens, et défendra à leurs voisins d’exercer aucun mauvais traitement contre un peuple qui s’est rendu aux Romains." (3) Quand on leur eut rapporté cette réponse, ils dirent qu’ils allaient obéir. (4) Du haut de leurs murailles, ils jetèrent dans le fossé qui était devant la place une si grande quantité d’armes que le monceau s’élevait presque à la hauteur du rempart et de notre terrasse ; et cependant, comme on le sut par la suite, ils en avaient caché et gardé un tiers dans la ville. Ils ouvrirent leurs portes et restèrent paisibles le reste du jour.
\subsection[{§ 33.}]{ \textsc{§ 33.} }
\noindent (1) Sur le soir, César fit fermer les portes et sortir ses soldats de la ville, dans la crainte qu’ils ne commissent la nuit des violences contre les habitants. (2) Ceux-ci, comme on le vit bientôt, s’étaient concertés d’avance, pensant qu’après leur soumission nos postes seraient dégarnis ou au moins négligemment gardés : une partie d’entre eux, avec les armes qu’ils avaient retenues et cachées, une autre avec des boucliers d’écorce en d’osier tressé, qu’ils avaient recouverts de peaux à la hâte, vu la brièveté du temps, sortent tout à coup de la place, à la troisième veille, avec toutes leurs troupes, et fondent sur l’endroit des retranchements où l’accès leur parut le moins difficile. (3) L'alarme fut aussitôt donnée par de grands feux, signal prescrit par César, et on accourut de tous les forts voisins sur le point attaqué. (4) Les ennemis combattirent avec acharnement, comme devaient le faire des hommes désespérés, n’attendant plus leur salut que de leur courage, luttant, malgré le désavantage de leur position, contre nos soldats qui lançaient leurs traits sur eux du haut du retranchement et des tours. (5) On en tua quatre mille ; le reste fut repoussé dans la place. (6) Le lendemain, César fit rompre les portes laissées sans défenseurs, entra dans la ville avec ses troupes, et fit vendre à l’encan tout ce qu’elle renfermait. (7) Il apprit des acheteurs que le nombre des têtes était de cinquante-trois mille.
\subsection[{§ 34.}]{ \textsc{§ 34.} }
\noindent (1) Dans le même temps, César fut informé par P. Crassus, envoyé par lui, avec une seule légion, contre les Vénètes, les Unelles, les Osismes, les Curiosolites, les Esuvii, les Aulerques, les Redons, peuples maritimes sur les côtes de l’Océan, qu’ils s’étaient tous soumis au pouvoir du peuple romain.
\subsection[{§ 35.}]{ \textsc{§ 35.} }
\noindent (1) Ces succès, l’entière pacification de la Gaule, toute cette guerre enfin, firent sur les barbares une telle impression, que plusieurs des peuples situés de l’autre côté du Rhin envoyèrent des députés à César, pour lui offrir des otages et leur soumission. (2) César, pressé de se rendre en Italie et en Illyrie, leur dit de revenir au commencement de l’été suivant. (3) Il mit ses légions en quartier d’hiver chez les Carnutes, les Andes et les Turons, pays voisins de ceux où il avait fait la guerre, et partit pour l’Italie. (4) Tous ces événements, annoncés à Rome par les lettres de César, firent décréter quinze jours d’actions de grâces aux dieux, ce qui, avant ce temps, n’avait eu lieu pour aucun général.
\section[{Livre III}]{Livre III}\renewcommand{\leftmark}{Livre III}

\subsection[{§ 1.}]{ \textsc{§ 1.} }
\noindent (1) En partant pour l’Italie, César avait envoyé Servius Galba, avec la douzième légion et une partie de la cavalerie, chez les Nantuates, les Véragres et les Sédunes, dont le territoire s’étend depuis le pays des Allobroges, le lac Léman et le fleuve du Rhône jusqu’aux Hautes-Alpes. (2) L'objet de la mission de Galba était d’ouvrir un chemin à travers ces montagnes, où les marchands ne pouvaient passer sans courir de grands dangers et payer des droits onéreux. (3) César lui permit, s’il le jugeait nécessaire, de mettre sa légion en quartier d’hiver dans ce pays. (4) Après quelques combats heureux pour lui, et la prise de plusieurs forteresses, Galba reçut de toutes parts des députés et des otages, fit la paix, plaça deux cohortes en cantonnement chez les Nantuates, et lui-même, avec les autres cohortes de la légion, prit son quartier d’hiver dans un bourg des Véragres, nommé Octoduros. (5) Ce bourg, situé dans un vallon peu ouvert, est de tous côtés environné de très hautes montagnes. (6) Une rivière le traverse et le divise en deux parties. Galba laissa l’une aux Gaulois, et l’autre, demeurée vide par leur retraite, dut servir de quartier d’hiver aux cohortes romaines. Il s’y fortifia d’un retranchement et d’un fossé.
\subsection[{§ 2.}]{ \textsc{§ 2.} }
\noindent (1) Après plusieurs jours passés dans ce bourg, et employés par Galba à faire venir des vivres, il apprit tout à coup de ses éclaireurs que tous les Gaulois avaient, pendant la nuit, évacué la partie du bourg qui leur avait été laissée, et que les montagnes qui dominent Octoduros étaient occupées par une multitude immense de Sédunes et de Véragres. (2) Plusieurs motifs avaient suggéré aux Gaulois ce projet subit de renouveler la guerre et d’accabler notre légion. (3) Ils savaient que cette légion n’était plus au complet, qu’on en avait retiré deux cohortes, que des détachements partiels, servant d’escorte aux convois, tenaient beaucoup de soldats absents, et ce corps ainsi réduit leur paraissait méprisable  ; (4) ils croyaient de plus que le désavantage de notre position, lorsqu’ils se précipiteraient des montagnes dans le vallon, en lançant leurs traits, ne permettrait pas à nos troupes de soutenir leur premier choc. (5) À ces causes se joignaient la douleur d’être séparés de leurs enfants enlevés à titre d’otages, et la persuasion que les Romains cherchaient à s’emparer des Alpes, moins pour avoir un passage que pour s’y établir à jamais, et les réunir à leur province qui en est voisine.
\subsection[{§ 3.}]{ \textsc{§ 3.} }
\noindent (1) En recevant ces nouvelles, Galba, qui n’avait ni achevé ses retranchements pour l’hivernage, ni suffisamment pourvu aux subsistances, et que la soumission des Gaulois, suivie de la remise de leurs otages, faisait douter de la possibilité d’une attaque, se hâte d’assembler un conseil et de recueillir les avis. (2) Dans ce danger, aussi grand que subit et inattendu, lorsque l’on voyait déjà presque toutes les hauteurs couvertes d’une multitude d’ennemis en armes, qu’on n’avait aucun secours à attendre, aucun moyen de s’assurer des vivres, puisque les chemins étaient interceptés, (3) envisageant cette position presque désespérée, plusieurs, dans ce conseil, étaient d’avis d’abandonner les bagages et de se faire jour à travers les ennemis afin de se sauver par où l’on était venu. (4) Cependant le plus grand nombre, réservant ce parti pour la dernière extrémité, résolut de tenter le sort des armes et de défendre le camp.
\subsection[{§ 4.}]{ \textsc{§ 4.} }
\noindent (1) Peu d’instants s’étaient écoulés depuis cette résolution, et on avait à peine eu le temps de faire les dispositions qu’elle exigeait, lorsque les ennemis accourent de toutes parts à un signal donné, et lancent sur notre camp des pierres et des pieux. (2) Les nôtres, dont les forces étaient encore entières, opposèrent une courageuse résistance : lancés du haut des retranchements, tous leurs traits portaient coup : apercevaient-ils quelque point du camp trop vivement pressé faute de défenseurs, ils couraient y porter secours  ; (3) mais les Gaulois avaient cet avantage, qu’ils pouvaient remplacer par des troupes fraîches celles qui se retiraient fatiguées par un long combat, (4) manoeuvre que le petit nombre des nôtres leur interdisait. Ceux dont les forces étaient épuisées, et les blessés eux-mêmes, ne pouvaient quitter la place où ils se trouvaient, pour reprendre haleine.
\subsection[{§ 5.}]{ \textsc{§ 5.} }
\noindent (1) II y avait déjà plus de six heures que le combat durait sans interruption ; et non seulement les forces, mais les traits même commençaient à manquer ; l’attaque devenait plus pressante et la résistance plus faible. L'ennemi forçait déjà le retranchement et comblait le fossé ; nos affaires enfin étaient dans le plus grand péril, (2) lorsque P. Sextius Baculus, centurion du premier rang, le même que nous avons vu couvert de blessures à la bataille contre les Nerviens, et C. Volusénus, tribun militaire, homme également ferme dans le conseil et dans l’action, accourent auprès de Galba, et lui représentent qu’il n’y a plus de salut à attendre que d’une vigoureuse sortie, qu’il faut tenter cette dernière ressource. (3) Les centurions sont convoqués, et on ordonne aussitôt aux soldats de suspendre un moment le combat, de parer seulement les traits qu’on leur lance, et de reprendre haleine ; puis, au signal donné, de se précipiter hors du camp et de n’espérer leur salut que de leur courage.
\subsection[{§ 6.}]{ \textsc{§ 6.} }
\noindent (1) L'ordre s’exécute, et nos soldats, s’élançant tout à coup hors du camp par toutes les portes, ne laissent pas aux ennemis le temps de juger de ce qui se passe ni de se rallier. (2) Le combat change ainsi de face ; ceux qui se croyaient déjà maîtres du camp sont de tous côtés enveloppés et massacrés ; et, de plus de trente mille hommes dont il était constant que se composait l’armée des barbares, plus du tiers fut tué ; le reste, épouvanté, prit la fuite, et ne put même rester sur les hauteurs. (3) Toutes les forces des ennemis ainsi dispersées et les armes enlevées, on rentra dans le camp et dans les retranchements. (4) Après cette victoire, Galba ne voulut plus tenter le sort des combats ; mais, se rappelant qu’il avait pris ses quartiers d’hiver dans un tout autre dessein, qu’avaient traversé des circonstances imprévues, pressé d’ailleurs par le manque de grains et de vivres, il fit brûler le lendemain toutes les habitations du bourg et prit la route de la province. (5) Aucun ennemi n’arrêtant ni ne retardant sa marche, il ramena la légion sans perte chez les Nantuates, et de là chez les Allobroges, où il hiverna.
\subsection[{§ 7.}]{ \textsc{§ 7.} }
\noindent (1) Après ces événements, César avait tout lieu de croire la Gaule pacifiée ; les Belges avaient été défaits, les Germains repoussés, les Sédunes vaincus dans les Alpes. Il partit donc au commencement de l’hiver pour l’Illyrie, dont il voulait visiter les nations et connaître le territoire, lorsque tout à coup la guerre se ralluma dans la Gaule. (2) Voici quelle en fut la cause. Le jeune P. Crassus hivernait avec la septième légion, près de l’Océan, chez les Andes. (3) Comme il manquait de blé dans ce pays, il envoya des préfets et plusieurs tribuns militaires chez les peuples voisins, pour demander des subsistances ; (4) T. Terrasidius, entre autres, fut délégué chez les Esuvii ; M. Trébius Gallus chez les Coriosolites ; Q. Vélanius avec T. Sillius chez les Vénètes.
\subsection[{§ 8.}]{ \textsc{§ 8.} }
\noindent (1) Cette dernière nation est de beaucoup la plus puissante de toute cette côte maritime. Les Vénètes, en effet, ont un grand nombre de vaisseaux qui leur servent à communiquer avec la Bretagne ; ils surpassent les autres peuples dans l’art et dans la pratique de la navigation, et, maîtres du peu de ports qui se trouvent sur cette orageuse et vaste mer, ils prélèvent des droits sur presque tous ceux qui naviguent dans ces parages. (2) Les premiers, ils retinrent Sillius et Vélanius, espérant, par ce moyen, forcer Crassus à leur rendre les otages qu’ils lui avaient donnés. (3) Entraînés par la force d’un tel exemple, leurs voisins, avec cette prompte et soudaine résolution qui caractérise les Gaulois, retiennent, dans les mêmes vues, Trébius et Terrasidius ; s’étant envoyé des députés, ils conviennent entre eux, par l’organe de leurs principaux habitants, de ne rien faire que de concert, et de courir le même sort. (4) Ils sollicitent les autres états à se maintenir dans la liberté qu’ils ont reçue de leurs pères, plutôt que de subir le joug des Romains. (5) Ces sentiments sont bientôt partagés par toute la côte maritime ; ils envoient alors en commun des députés à Crassus, pour lui signifier qu’il eût à leur remettre leurs otages, s’il voulait que ses envoyés lui fussent rendus.
\subsection[{§ 9.}]{ \textsc{§ 9.} }
\noindent (1) César, instruit de ces faits par Crassus, et se trouvant alors très éloigné, ordonne de construire des galères sur la Loire, qui se jette dans l’Océan, de lever des rameurs dans la province, de rassembler des matelots et des pilotes. (2) Ces ordres ayant été promptement exécutés, lui-même, dès que la saison le permet, se rend à l’armée. (3) Les Vénètes et les autres états coalisés, apprenant l’arrivée de César, et sentant de quel crime ils s’étaient rendus coupables pour avoir retenu et jeté dans les fers des députés dont le nom chez toutes les nations fut toujours sacré et inviolable, se hâtèrent de faire des préparatifs proportionnés à la grandeur du péril, et surtout d’équiper leurs vaisseaux. Ce qui leur inspirait le plus de confiance, c’était l’avantage des lieux. (4) Ils savaient que les chemins de pied étaient interceptés par les marées, et que la navigation serait difficile pour nous sur une mer inconnue et presque sans ports. (5) Ils espéraient en outre que, faute de vivres, notre armée ne pourrait séjourner longtemps chez eux ; (6) dans le cas où leur attente serait trompée, ils comptaient toujours sur la supériorité de leurs forces navales. Les Romains manquaient de marine et ignoraient les rades, les ports et les îles des parages où ils feraient la guerre ; (7) la navigation était tout autre sur une mer fermée que sur une mer aussi vaste et aussi ouverte que l’est l’Océan. (8) Leurs résolutions étant prises, ils fortifient leurs places et transportent les grains de la campagne dans les villes. (9) Ils réunissent en Vénétie le plus de vaisseaux possible, persuadés que César y porterait d’abord la guerre. (10) Ils s’associent pour la faire les Osismes, les Lexovii, les Namnètes, les Ambiliates, les Morins, les Diablintes et les Ménapes ; ils demandent des secours à la Bretagne, située vis-à-vis de leurs côtes.
\subsection[{§ 10.}]{ \textsc{§ 10.} }
\noindent (1) Les difficultés de cette guerre étaient telles que nous venons de les exposer, et cependant plusieurs motifs commandaient à César de l’entreprendre : (2) l’arrestation injurieuse de chevaliers romains, la révolte après la soumission, la défection après les otages livrés, la coalition de tant d’états, la crainte surtout que d’autres peuples, si les premiers rebelles demeuraient impunis, se remissent à suivre leur exemple. (3) Sachant donc que presque tous les Gaulois aspiraient à un changement ; que leur mobilité naturelle les poussait facilement à la guerre, et que, d’ailleurs, il est dans la nature de tous les hommes d’aimer la liberté et de haïr l’esclavage, il crut devoir, avant que d’autres états fussent entrés dans cette ligue, partager son armée et la distribuer sur plus de points.
\subsection[{§ 11.}]{ \textsc{§ 11.} }
\noindent (1) Il envoie son lieutenant T. Labiénus avec de la cavalerie chez les Trévires, peuple voisin du Rhin. (2) Il le charge de visiter les Rèmes et autres Belges, de les maintenir dans le devoir et de s’opposer aux tentatives que pourraient faire, pour passer le fleuve, les vaisseaux des Germains que l’on disait appelés par les Belges. (3) Il ordonne à P. Crassus de se rendre en Aquitaine, avec douze cohortes légionnaires et un grand nombre de cavaliers, pour empêcher ce pays d’envoyer des secours dans la Gaule, et de si grandes nations de se réunir. (4) Il fait partir son lieutenant Q. Titurius Sabinus, avec trois légions, chez les Unelles, les Coriosolites et les Lexovii, pour tenir ces peuples en respect. (5) II donne au jeune D. Brutus le commandement de la flotte et des vaisseaux gaulois, qu’il avait fait venir de chez les Pictons, les Santons et autres pays pacifiés, et il lui enjoint de se rendre au plus tôt chez les Vénètes, lui-même en prend le chemin avec les troupes de terre.
\subsection[{§ 12.}]{ \textsc{§ 12.} }
\noindent (1) Telle était la disposition de la plupart des places de l’ennemi, que, situées à l’extrémité de langues de terre et sur des promontoires, elles n’offraient d’accès ni aux gens de pied quand la mer était haute, ce qui arrive constamment deux fois dans l’espace de vingt-quatre heures, ni aux vaisseaux que la mer, en se retirant, laisserait à sec sur le sable. (2) Ce double obstacle rendait très difficile le siège de ces villes. (3) Si, après de pénibles travaux, on parvenait à contenir la mer par une digue et des môles, et à s’élever jusqu’à la hauteur des murs, les assiégés, commençant à désespérer de leur fortune, rassemblaient leurs nombreux navires, dernière et facile ressource, y transportaient tous leurs biens, et se retiraient dans des villes voisines. (4) Là ils se défendaient de nouveau par les mêmes avantages de position. (5) Cette manoeuvre leur fut d’autant plus facile durant une grande partie de l’été, que nos vaisseaux étaient retenus par les vents contraires et éprouvaient de grandes difficultés à naviguer sur une mer vaste, ouverte, sujette à de hautes marées et presque entièrement dépourvue de ports.
\subsection[{§ 13.}]{ \textsc{§ 13.} }
\noindent (1) Les vaisseaux des ennemis étaient construits et armés de la manière suivante : la carène en est un peu plus plate que celle des nôtres, ce qui leur rend moins dangereux les bas-fonds et le reflux ; (2) les proues sont très élevées, les poupes peuvent résister aux plus grandes vagues et aux tempêtes ; (3) les navires sont tout entiers de chêne et peuvent supporter les chocs les plus violents. (4) Les bancs, faits de poutres d’un pied d’épaisseur, sont attachés par des clous en fer de la grosseur d’un pouce ; (5) les ancres sont retenues par des chaînes de fer au lieu de cordages ; (6) des peaux molles et très amincies leur servent de voiles, soit qu’ils manquent de lin ou qu’ils ne sachent pas l’employer, soit encore qu’ils regardent, ce qui est plus vraisemblable, nos voiles comme insuffisantes pour affronter les tempêtes violentes et les vents impétueux de l’Océan, et pour diriger des vaisseaux aussi pesants. (7) Dans l’abordage de ces navires avec les nôtres, ceux-ci ne pouvaient l’emporter que par l’agilité et la vive action des rames ; du reste, les vaisseaux des ennemis étaient bien plus en état de lutter, sur ces mers orageuses, contre la force des tempêtes. (8) Les nôtres ne pouvaient les entamer avec leurs éperons, tant ils étaient solides ; leur hauteur les mettait à l’abri des traits, et, par la même cause, ils redoutaient moins les écueils. (9) Ajoutons que, lorsqu’ils sont surpris par un vent violent, ils soutiennent sans peine la tourmente et s’arrêtent sans crainte sur les bas-fonds, et, qu’au moment du reflux, ils ne redoutent ni les rochers ni les brisants ; circonstances qui étaient toutes à craindre pour nos vaisseaux.
\subsection[{§ 14.}]{ \textsc{§ 14.} }
\noindent (1) Après avoir enlevé plusieurs places, César, sentant que toute la peine qu’il prenait était inutile, et qu’il ne pouvait ni empêcher la retraite des ennemis en prenant leurs villes, ni leur faire le moindre mal, résolut d’attendre sa flotte. (2) Dès qu’elle parut et qu’elle fut aperçue de l’ennemi deux cent vingt de leurs vaisseaux environ, parfaitement équipés et armés, sortirent du port et vinrent se placer devant les nôtres. (3) Brutus, le chef de la flotte, les tribuns militaires et les centurions qui commandaient chaque vaisseau, n’étaient pas fixés sur ce qu’ils avaient à faire et sur la manière d’engager le combat. (4) Ils savaient que l’éperon de nos galères était sans effet ; que nos tours, à quelque hauteur qu’elles fussent portées, ne pouvaient atteindre même la poupe des vaisseaux des barbares, et qu’ainsi nos traits lancés d’en bas seraient une faible ressource, tandis que ceux des Gaulois nous accableraient. (5) Une seule invention nous fut d’un grand secours : c’étaient des faux extrêmement tranchantes, emmanchées de longues perches, peu différentes de celles employées dans les sièges. (6) Quand, au moyen de ces faux, les câbles qui attachent les vergues aux mâts étaient accrochés et tirés vers nous ; on les rompait en faisant force de rames ; (7) les câbles une fois brisés, les vergues tombaient nécessairement, et cette chute réduisait aussitôt à l’impuissance les vaisseaux gaulois, dont toute la force était dans les voiles et les agrès. (8) L'issue du combat ne dépendait plus que du courage, et en cela nos soldats avaient aisément l’avantage, surtout dans une action qui se passait sous les yeux de César et de toute l’armée ; aucun trait de courage ne pouvait rester inaperçu ; (9) car toutes les collines et les hauteurs, d’où l’on voyait la mer à peu de distance, étaient occupées par l’armée.
\subsection[{§ 15.}]{ \textsc{§ 15.} }
\noindent (1) Dès qu’un vaisseau était ainsi privé de ses vergues, deux ou trois des nôtres l’entouraient, et nos soldats, pleins d’ardeur, tentaient l’abordage. (2) Les barbares ayant, par cette manoeuvre, perdu une partie de leurs navires, et ne voyant nulle ressource contre ce genre d’attaque, cherchèrent leur salut dans la fuite : (3) déjà ils avaient tourné leurs navires de manière à recevoir le vent, lorsque tout à coup eut lieu un calme plat qui leur rendit tout mouvement impossible. (4) Cette heureuse circonstance compléta le succès ; (5) car les nôtres les attaquèrent et les prirent l’un après l’autre, et un bien petit nombre put regagner la terre à la faveur de la nuit, après un combat qui avait duré depuis environ la quatrième heure du jour jusqu’au coucher du soleil.
\subsection[{§ 16.}]{ \textsc{§ 16.} }
\noindent (1) Cette bataille mit fin à la guerre des Vénètes et de tous les états maritimes de cette côte ; (2) car toute la jeunesse et même tous les hommes d’un âge mûr, distingués par leur caractère ou par leur rang, s’étaient rendus à cette guerre, pour laquelle tout ce qu’ils avaient de vaisseaux en divers lieux avait été rassemblé en un seul. (3) La perte qu’ils venaient d’éprouver ne laissait au reste des habitants aucune ressource pour la retraite, aucun moyen de défendre leurs villes. Ils se rendirent donc à César avec tout ce qu’ils possédaient. (4) César crut devoir tirer d’eux une vengeance éclatante, qui apprît aux barbares à respecter désormais le droit des ambassadeurs. II fit mettre à mort tout le sénat, et vendit à l’encan le reste des habitants.
\subsection[{§ 17.}]{ \textsc{§ 17.} }
\noindent (1) Tandis que ces événements se passaient chez les Vénètes, Q. Titurius Sabinus arrivait sur les terres des Unelles avec les troupes qu’il avait reçues de César. (2) Viridovix était à la tête de cette nation et avait le commandement en chef de tous les états révoltés, dont il avait tiré une armée et des forces redoutables. (3) Depuis peu de jours les Aulerques Éburovices et les Lexovii, après avoir égorgé leur sénat qui s’opposait à la guerre, avaient fermé leurs portes et s’étaient joints à Viridovix. (4) Enfin de tous les points de la Gaule était venue une multitude d’hommes perdus et de brigands que l’espoir du pillage et la passion de la guerre avaient arrachés à l’agriculture et à leurs travaux journaliers. (5) Sabinus se tenait dans son camp situé sur le terrain le plus favorable, pendant que Viridovix, campé en face de lui à une distance de deux milles, déployait tous les jours ses troupes, et lui offrait la bataille, de sorte que Sabinus s’attirait non seulement le mépris des ennemis, mais encore les sarcasmes de nos soldats. (6) L'opinion qu’il donna de sa frayeur était telle que déjà l’ennemi osait s’avancer jusqu’aux retranchements du camp. (7) Le motif de Sabinus pour agir ainsi était qu’il ne croyait pas qu’un lieutenant dût, surtout en l’absence du général en chef, combattre une si grande multitude, à moins d’avoir pour lui l’avantage du lieu ou quelque autre circonstance favorable.
\subsection[{§ 18.}]{ \textsc{§ 18.} }
\noindent (1) L'opinion de cette frayeur s’étant affermie, Sabinus choisit parmi les Gaulois qu’il avait près de lui comme auxiliaires, un homme habile et fin. (2) Il lui persuade, à force de récompenses et de promesses, de passer aux ennemis, et l’instruit de ce qu’il doit faire. (3) Dès que cet homme est arrivé parmi eux comme transfuge, il parle de la terreur des Romains, il annonce que César lui-même est enveloppé par les Vénètes, (4) et que, pas plus tard que la nuit suivante, Sabinus doit sortir secrètement de son camp avec son armée, et partir au secours de César. (5) Les Gaulois n’ont pas plus tôt entendu ce rapport qu’ils s’écrient tous qu’il ne faut pas perdre une occasion si belle, et qu’on doit marcher au camp des Romains. (6) Plusieurs motifs excitaient les Gaulois : l’hésitation de Sabinus pendant les jours précédents, le rapport du transfuge, le manque de vivres, chose à laquelle on avait pourvu avec peu de diligence, l’espérance fondée sur la guerre des Vénètes, enfin cette facilité des hommes à croire ce qu’ils désirent. (7) Décidés par tous ces motifs, ils ne laissent point sortir du conseil Viridovix et les autres chefs, qu’ils n’aient obtenu d’eux de prendre les armes, et de marcher contre nous. (8) Joyeux de cette promesse, et comme assurés de la victoire, ils se chargent de sarments et de broussailles pour combler les fossés des Romains, et se dirigent vers leur camp.
\subsection[{§ 19.}]{ \textsc{§ 19.} }
\noindent (1) Le camp était sur une hauteur à laquelle on arrivait par une pente douce d’environ mille pas. Ils s’y portèrent d’une course rapide, afin de laisser aux Romains le moins de temps possible pour se rassembler et s’armer, et arrivèrent hors d’haleine. Sabinus exhorte les siens et donne le signal désiré. (2) II ordonne de sortir par deux portes et de tomber sur l’ennemi embarrassé du fardeau qu’il portait. (3) L'avantage de notre position, l’imprévoyance et la fatigue des ennemis, le courage des soldats, l’expérience acquise dans les précédents combats, firent que les barbares ne soutinrent pas même notre premier choc, et qu’ils prirent aussitôt la fuite. (4) Nos soldats, dont les forces étaient entières, les atteignirent dans ce désordre et en tuèrent un grand nombre. La cavalerie acheva de les poursuivre et ne laissa échapper que peu de ces fuyards. (5) C'est ainsi que, dans le même temps, Sabinus apprit l’issue du combat naval, et César la victoire de Sabinus. Toutes les villes de cette contrée se rendirent sur-le-champ à Titurius ; (6) car, si le Gaulois est prompt et ardent à prendre les armes, il manque de fermeté et de constance pour supporter les revers.
\subsection[{§ 20.}]{ \textsc{§ 20.} }
\noindent (1) Presque à la même époque, P. Crassus était arrivé dans l’Aquitaine, pays qui, à raison de son étendue et de sa population, peut être estimé, comme nous l’avons dit, le tiers de la Gaule. Songeant qu’il aurait à faire la guerre dans les mêmes lieux où, peu d’années auparavant, le lieutenant L. Valérius Préconinus avait été vaincu et tué, et d’où le proconsul Manlius avait été chassé après avoir perdu ses bagages, il crut qu’il ne pouvait déployer trop d’activité. (2) Ayant donc pourvu aux vivres, rassemblé des auxiliaires et de la cavalerie, et fait venir en outre de Toulouse, de Carcasonne et de Narbonne, pays dépendants de la province romaine et voisins de l’Aquitaine, bon nombre d’hommes intrépides qu’il désigna, il mena son armée sur les terres des Sotiates . (3) À la nouvelle de son arrivée, les Sotiates rassemblèrent des troupes considérables et de la cavalerie, qui faisait leur principale force, attaquèrent notre armée dans sa marche, et engagèrent avec elle un combat de cavalerie, (4) dans lequel ayant été repoussés et poursuivis par la nôtre, ils firent tout à coup paraître leur infanterie, placée en embuscade dans un vallon. Ils assaillirent nos soldats épars et recommencèrent le combat.
\subsection[{§ 21.}]{ \textsc{§ 21.} }
\noindent (1) Il fut long et opiniâtre : Les Sotiates, fiers de leurs anciennes victoires, regardaient le salut de toute l’Aquitanie comme attaché à leur valeur ; nos soldats voulaient montrer ce qu’ils pouvaient faire, en l’absence du général, sans l’aide des autres légions, sous la conduite d’un jeune chef. Couverts de blessures, les ennemis enfin tournèrent le dos ; (2) on en tua un grand nombre, et Crassus, sans s’arrêter, mit le siège devant la capitale des Sotiates. Leur résistance courageuse l’obligea d’employer les mantelets et les tours. (3) Tantôt ils faisaient des sorties, tantôt ils pratiquaient des mines jusque sous nos tranchées (sorte d’ouvrage où ils sont très habiles, leur pays étant plein de mines d’airain qu’ils exploitent) ; mais voyant tous leurs efforts échouer devant l’activité de nos soldats, ils députèrent à Crassus, pour lui demander de recevoir leur capitulation. Crassus y consentit, à la condition qu’ils livreraient leurs armes, ce qu’ils firent.
\subsection[{§ 22.}]{ \textsc{§ 22.} }
\noindent (1) Tandis que tous les nôtres s’occupaient de l’exécution de ce traité, d’un autre côté de la ville se présenta le général en chef Adiatuanos, avec six cents hommes dévoués, de ceux que ces peuples appellent soldures. (2) Telle est la condition de ces hommes, qu’ils jouissent de tous les biens de la vie avec ceux auxquels ils se sont consacrés par un pacte d’amitié ; si leur chef périt de mort violente, ils partagent son sort et se tuent de leur propre main ; (3) et il n’est pas encore arrivé, de mémoire d’homme, qu’un de ceux qui s’étaient dévoués à un chef par un pacte semblable, ait refusé, celui-ci mort, de mourir aussitôt. (4) C'est avec cette escorte qu’Adiatuanos tenta une sortie : les cris qui s’élevèrent sur cette partie du rempart firent courir aux armes ; et à la suite d’un combat sanglant, Adiatuanos, repoussé dans la ville, obtint cependant de Crassus d’être compris dans la capitulation générale.
\subsection[{§ 23.}]{ \textsc{§ 23.} }
\noindent (1) Après avoir reçu les armes et les otages, Crassus marcha sur les terres des Vocates et des Tarusates. (2) Les Barbares, vivement effrayés en apprenant qu’une place également défendue par la nature et par la main de l’homme était, peu de jours après l’arrivée de Crassus, tombée en son pouvoir, s’envoient de toutes parts des députés, se liguent ensemble, se donnent mutuellement des otages, rassemblent des troupes. (3) Ils députent aussi vers les états de l’Espagne citérieure, voisins de l’Aquitaine, pour qu’on leur envoie de là des secours et des chefs. (4) À leur arrivée, pleins de confiance dans leur nombre, ils disposent tout pour la guerre. (5) Ils mettent à leur tête ceux qui avaient longtemps servi sous Q. Sertorius et qui passaient pour très habiles dans l’art militaire. (6) Ils commencent, à l’exemple du peuple romain, par prendre leurs positions, par fortifier leur camp, par nous intercepter les vivres. (7) Crassus s’en aperçut, et, sentant bien que ses troupes étaient trop peu nombreuses pour les diviser, tandis que l’ennemi pouvait faire des courses, occuper les chemins, et cependant ne pas dégarnir son camp, ce qui devait rendre difficile l’arrivée des vivres, le nombre des ennemis croissant d’ailleurs de jour en jour, il pensa qu’il fallait se hâter de combattre. (8) Il fit part de cet avis dans un conseil, et le voyant partagé par tout le monde, il fixa le jour suivant pour celui du combat.
\subsection[{§ 24.}]{ \textsc{§ 24.} }
\noindent (1) Au point du jour, il fit sortir toutes les troupes, en forma deux lignes, plaça au milieu les auxiliaires, et attendit ce que feraient les ennemis. (2) Ceux-ci, quoique, à raison de leur nombre et de leur ancienne gloire militaire, ils se crussent assurés de vaincre une poignée de Romains, tenaient cependant pour plus sûr encore, étant maîtres des passages et interceptant les vivres, d’obtenir une victoire qui ne leur coûtât pas de sang. (3) Si la faim nous forçait à la retraite, ils profiteraient de notre découragement pour nous attaquer au milieu des embarras de notre marche et de nos bagages. (4) Ce dessein fut approuvé de leurs chefs, et, tandis que l’armée romaine était en bataille, ils se tinrent dans leur camp. (5) Ayant pénétré le but de cette inaction, dont l’effet fut d’inspirer à nos soldats d’autant plus d’ardeur à combattre que l’hésitation des ennemis passait pour de la crainte, et cédant au cri général qui s’éleva pour qu’on marchât sans délai contre eux, Crassus harangue ses troupes, et, selon leur voeu, il marche contre le camp.
\subsection[{§ 25.}]{ \textsc{§ 25.} }
\noindent (1) Là, tandis que les uns comblent le fossé, que les autres, en lançant une grêle de traits, chassent du rempart ceux qui le défendent, les auxiliaires, sur qui Crassus comptait peu pour le combat, employés soit à passer les pierres et les traits, soit à apporter les fascines, pouvaient cependant figurer comme combattants. De son côté l’ennemi déployait un courage persévérant, et ses traits, lancés d’en haut, ne se perdaient point. (2) Sur ces entrefaites, des cavaliers qui venaient de faire le tour du camp, rapportèrent à Crassus qu’il était faiblement fortifié du côté de la porte décumane, et qu’il offrait sur ce point un accès facile.
\subsection[{§ 26.}]{ \textsc{§ 26.} }
\noindent (1) Crassus recommande aux préfets de la cavalerie d’encourager leurs soldats par la promesse de grandes récompenses, et leur explique ses intentions. (2) Ceux-ci, d’après l’ordre qu’ils ont reçu, prennent avec eux quatre cohortes toutes fraîches, restées à la garde du camp, et, leur faisant faire un long détour, pour dérober leur marche aux yeux de l’ennemi, occupé tout entier à combattre, ils arrivent promptement à cette partie du retranchement dont nous parlions, (3) en forcent l’entrée et pénètrent dans le camp des ennemis avant que ceux-ci aient pu les apercevoir ou apprendre ce qui se passe. (4) Avertis par les cris qui se font entendre de ce côté, les nôtres sentent renaître leurs forces, comme il arrive d’ordinaire quand on a l’espoir de vaincre, et ils pressent l’attaque avec plus de vigueur. (5) Les ennemis, enveloppés de toutes parts, perdent courage, se précipitent du haut de leurs remparts et cherchent leur salut dans la fuite. (6) La cavalerie les atteignit en rase campagne ; de cinquante mille hommes fournis par l’Aquitaine et le pays des Cantabres, elle laissa à peine échapper le quart, et ne rentra au camp que bien avant dans la nuit.
\subsection[{§ 27.}]{ \textsc{§ 27.} }
\noindent (1) Au bruit de cette victoire la plus grande partie de l’Aquitanie se rendit à Crassus, et envoya d’elle-même des otages. De ce nombre furent les Tarbelles, les Bigerrions, les Ptianii, les Vocates, les Tarusates, les Elusates, les Gates, les Ausques, les Garunni, les Sibuzates, et les Cocosates. (2) Quelques états éloignés se fiant sur la saison avancée, négligèrent d’en faire autant.
\subsection[{§ 28.}]{ \textsc{§ 28.} }
\noindent (1) Presque dans le même temps, bien que l’été fût déjà près de sa fin, César, voyant toute la Gaule pacifiée, à l’exception des Morins et des Ménapes qui restaient en armes, et ne lui avaient jamais envoyé de députés pour demander la paix, fit marcher son armée contre eux, espérant que cette guerre serait bientôt terminée. Ces peuples arrêtèrent, pour la soutenir, un plan tout autre que le reste des Gaulois ; (2) car, voyant tant de grandes nations repoussées et vaincues en livrant des batailles, ils se retirèrent avec tous leurs biens dans les bois et les marais, dont leur pays était couvert. (3) César, arrivé à l’entrée de ces forêts, commençait à y retrancher son camp, sans qu’un seul ennemi se fût montré, lorsque tout à coup, et pendant que nos soldats étaient çà et là occupés aux travaux, ils accourent de tous les côtés de la forêt, et fondent sur nous. Les Romains saisissent promptement leurs armes, les repoussent dans le bois et en tuent un grand nombre, mais, les ayant poursuivis trop loin dans des lieux couverts, ils essuyèrent eux-mêmes quelques pertes.
\subsection[{§ 29.}]{ \textsc{§ 29.} }
\noindent (1) Les jours suivants, César fit travailler à abattre la forêt, et, pour empêcher qu’on ne prît en flanc et par surprise les soldats désarmés, il fit entasser en face de l’ennemi tout le bois que l’on coupait, pour s’en faire un rempart de chaque côté. (2) Ce travail avait été, en peu de jours et avec une incroyable vitesse, exécuté sur un grand espace de terrain : nous étions déjà maîtres du bétail et des derniers rangs des bagages de l’ennemi, qui s’enfonçait dans l’épaisseur des forêts, lorsque le temps fut tel, qu’il força d’interrompre les travaux ; il ne fut plus même possible, à cause de la continuité des pluies, de tenir le soldat sous les tentes. (3) Après avoir ravagé tout le pays et brûlé les bourgs et les habitations, César ramena l’armée, et la mit en quartier d’hiver chez les Aulerques, les Lexovii, et les autres peuples qui s’étaient récemment soulevés.
\section[{Livre IV}]{Livre IV}\renewcommand{\leftmark}{Livre IV}

\subsection[{§ 1.}]{ \textsc{§ 1.} }
\noindent (1) Dans l’hiver qui suivit, sous le consulat de Cn. Pompée et de M. Crassus, les Usipètes et les Tencthères, peuples germains, passèrent le Rhin en grand nombre, non loin de l’endroit où il se jette dans la mer. (2) La cause de cette émigration était que les Suèves depuis plusieurs années les tourmentaient, leur faisaient une guerre acharnée, et les empêchaient de cultiver leurs champs. (3) La nation des Suèves est de beaucoup la plus puissante et la plus belliqueuse de toute la Germanie. (4) On dit qu’ils forment cent cantons, de chacun desquels ils font sortir chaque année mille hommes armés qui portent la guerre au dehors. Ceux qui restent dans le pays le cultivent pour eux-mêmes et pour les absents, (5) et, à leur tour, ils s’arment l’année suivante, tandis que les premiers séjournent dans leurs demeures. (6) Ainsi, ni l’agriculture ni la science, ou l’habitude de la guerre ne sont interrompues. (7) Mais nul d’entre eux ne possède de terre séparément et en propre, et ne peut demeurer ni s’établir plus d’un an dans le même lieu. (8) Ils consomment peu de blé, vivent en grande partie de laitage et de la chair de leurs troupeaux, et s’adonnent particulièrement à la chasse. (9) Ce genre de vie et de nourriture, leurs exercices journaliers et la liberté dont ils jouissent (car n’étant dès leur enfance habitués à aucun devoir, à aucune discipline, ils ne suivent absolument que leur volonté), en font des hommes robustes et remarquables par une taille gigantesque. (10) Ils se sont aussi accoutumés, sous un climat très froid, et à n’avoir d’autre vêtement que des peaux, dont l’exiguïté laisse une grande partie de leur corps à découvert, et à se baigner dans les fleuves.
\subsection[{§ 2.}]{ \textsc{§ 2.} }
\noindent (1) Ils donnent accès chez eux aux marchands, plutôt pour leur vendre ce qu’ils ont pris à la guerre que pour leur acheter quoi que ce soit. (2) Bien plus, ces chevaux étrangers qui plaisent tant dans la Gaule, et qu’on y paie à si haut prix, les Germains ne s’en servent pas. Les leurs sont mauvais et difformes, mais en les exerçant tous les jours, ils les rendent infatigables. (3) Dans les engagements de cavalerie, souvent ils sautent à bas de leurs chevaux et combattent à pied ; ils les ont dressés à rester à la même place, et les rejoignent promptement, si le cas le requiert. (4) Rien dans leurs moeurs ne passe pour plus honteux ni pour plus lâche que de se servir de selle. (5) Aussi, si peu nombreux qu’ils soient, osent-ils attaquer de gros corps de cavaliers ainsi montés. (6) L'importation du vin est entièrement interdite chez eux, parce qu’ils pensent que cette liqueur amollit et énerve le courage des hommes.
\subsection[{§ 3.}]{ \textsc{§ 3.} }
\noindent (1) Ils regardent comme leur plus grande gloire nationale d’avoir pour frontières des champs vastes et incultes ; ce qui signifie qu’un grand nombre de nations n’ont pu soutenir leurs efforts. (2) Aussi dit-on que, d’un côté, à six cent mille pas de leur territoire, les campagnes sont désertes. (3) Les Ubiens les avoisinent de l’autre côté. Ce peuple, autrefois considérable et florissant autant qu’on peut le dire des Germains, avec lesquels il a une origine commune, est cependant plus civilisé que le reste de cette nation, parce que, touchant au Rhin, il a de nombreux rapports avec des marchands ; le voisinage des Gaulois l’a en outre façonné à leurs moeurs. (4) Les Suèves lui ont fait des guerres fréquentes sans pouvoir, à cause de sa population et de sa puissance, le chasser de son territoire ; ils sont parvenus cependant à le rendre tributaire et à le réduire à un état d’abaissement et de faiblesse.
\subsection[{§ 4.}]{ \textsc{§ 4.} }
\noindent (1) Il en a été de même des Usipètes et des Tencthères, que nous avons nommés plus haut ; ils résistèrent pendant nombre d’années aux attaques des Suèves : à la fin cependant, chassés de leurs terres, et après avoir erré trois ans à travers plusieurs cantons de la Germanie, ils arrivèrent près du Rhin, dans des contrées habitées par les Ménapes, lesquels possédaient, sur l’une et l’autre rive du fleuve, des champs, des maisons et des bourgs. (3) Effrayés à l’arrivée d’une telle multitude, les Ménapes abandonnèrent les habitations qu’ils possédaient au-delà du fleuve, et, s’étant fortifiés en deçà, ils s’opposèrent au passage des Germains. (4) Ceux-ci, après avoir tout essayé, ne pouvant passer ni de vive force, faute de bateaux, ni à la dérobée, à cause des gardes posées par les Ménapes, feignirent de retourner dans leur pays et dans leurs demeures ; (5) mais, après trois jours de marche, ils revinrent sur leurs pas, et, refaisant en une nuit, avec leurs chevaux, le même chemin, ils tombèrent à l’improviste sur les Ménapes, (6) qui, informés par leurs éclaireurs de la retraite de leurs ennemis, étaient rentrés sans crainte dans leurs bourgs au-delà du Rhin. (7) Après les avoir taillés en pièces et s’être emparés de leurs bateaux, ils traversèrent le fleuve avant que la partie des Ménapes qui était restée tranquille sur l’autre rive eût appris leur retour ; ils se rendirent maîtres de toutes leurs habitations, et se nourrirent, le reste de l’hiver, des vivres qu’ils y trouvèrent.
\subsection[{§ 5.}]{ \textsc{§ 5.} }
\noindent (1) Instruit de ces événements et redoutant la faiblesse des Gaulois, qu’il connaissait si mobiles dans leurs résolutions et avides de nouveautés, César ne crut pas devoir se fier à eux. (2) C'est en Gaule un usage de forcer les voyageurs à s’arrêter malgré eux ; et de les interroger sur ce que chacun d’eux sait ou a entendu dire. Dans les villes, le peuple entoure les marchands, et les oblige de déclarer de quel pays ils viennent, et les choses qu’ils y ont apprises. (3) C'est d’après l’impression produite par ces bruits et ces rapports qu’ils décident souvent les affaires les plus importantes ; et un prompt repentir suit nécessairement des résolutions prises sur des nouvelles incertaines, et le plus souvent inventées pour leur plaire.
\subsection[{§ 6.}]{ \textsc{§ 6.} }
\noindent (1) Connaissant cette habitude des Gaulois, César, pour prévenir une guerre plus sérieuse, rejoignit l’armée plus tôt que de coutume. (2) En y arrivant, il apprit ce qu’il avait soupçonné : (3) que plusieurs peuples de la Gaule avaient envoyé des députations aux Germains, et les avaient invités à quitter les rives du Rhin, les assurant qu’on tiendrait prêt tout ce qu’ils demanderaient. (4) Séduits par cet espoir, les Germains commençaient déjà à s’étendre et étaient parvenus au territoire des Éburons et des Condruses, qui sont dans la clientèle des Trévires. (5) César, ayant fait venir les principaux de la Gaule, crut devoir dissimuler ce qu’il connaissait ; il les flatta, les encouragea, leur prescrivit des levées de cavalerie, et résolut de marcher contre les Germains.
\subsection[{§ 7.}]{ \textsc{§ 7.} }
\noindent (1) Après avoir pourvu aux vivres, et fait un choix de cavalerie, il se dirigea où l’on disait qu’étaient les Germains. (2) II n’en était plus qu’à peu de journées, lorsque des députés vinrent le trouver de leur part ; leur discours fut celui-ci : (3) "Les Germains ne feront point les premiers la guerre au peuple romain ; ils ne refuseront cependant pas de combattre si on les attaque ; car, c’est une coutume que leur ont transmise leurs ancêtres, de résister à quiconque les provoque, et non de recourir à des prières. Au reste, ils déclarent qu’ils sont venus contre leur gré, et parce qu’on les a chassés de leur pays ; (4) si les Romains veulent acquérir leur amitié, elle pourra leur être utile ; qu’on leur assigne des terres ou qu’on leur laisse la possession de celles qu’ils ont conquises par les armes. (5) Ils ne le cèdent qu’aux Suèves auxquels les dieux même ne sauraient se comparer ; sauf ceux-ci, il n’est, sur la terre, aucun autre peuple dont ils ne puissent triompher."
\subsection[{§ 8.}]{ \textsc{§ 8.} }
\noindent (1) César répondit à ce discours ce qu’il jugea convenable, mais sa conclusion fut : "Qu'ils ne pouvaient prétendre à son amitié s’ils restaient dans la Gaule ; (2) qu’il n’était pas juste que ceux qui n’avaient pas su défendre leur territoire occupassent celui d’autrui ; qu’il n’y avait point dans la Gaule de terrain vacant que l’on pût donner sans injustice, surtout à une si grande multitude. (3) Il leur est loisible, s’ils le veulent, de se fixer chez les Ubiens, dont les députés sont venus près de lui se plaindre des outrages des Suèves et réclamer son secours ; il obtiendra des Ubiens cette permission."
\subsection[{§ 9.}]{ \textsc{§ 9.} }
\noindent (1) Les députés dirent qu’ils reporteraient cette réponse à leur nation, et qu’après en avoir délibéré ils reviendraient dans trois jours auprès de César. Cependant ils le priaient de ne pas avancer davantage. (2) César déclara ne pouvoir leur accorder cette demande : (3) car il savait que plusieurs jours auparavant ils avaient envoyé une grande partie de leur cavalerie au-delà de la Meuse, chez les Ambivarites, pour piller et s’approvisionner de vivres. Il était persuadé que l’attente de ces cavaliers était le motif du délai demandé.
\subsection[{§ 10.}]{ \textsc{§ 10.} }
\noindent (1) La Meuse sort des montagnes des Vosges, sur les frontières des Lingons. Après avoir reçu un bras du Rhin que l’on nomme le Wahal ; elle forme l’île des Bataves et, à quatre-vingt millesenviron, va se jeter dans le Rhin. (2) Quant au Rhin, il prend sa source chez les Lépontes, habitants des Alpes, et traverse rapidement dans un long espace les terres des Nantuates, des Helvètes, des Séquanes, des Médiomatrices, des Triboques, des Trévires : (4) lorsqu’il approche de l’Océan, il se divise en plusieurs branches, formant beaucoup de grandes îles, dont la plupart sont habitées par des nations féroces et barbares, (5) parmi lesquelles il en est qui passent pour vivre de poissons et d’oeufs d’oiseaux ; enfin, il se jette dans l’Océan par beaucoup d’embouchures.
\subsection[{§ 11.}]{ \textsc{§ 11.} }
\noindent (1) César n’était plus qu’à douze milles de l’ennemi, quand les députés revinrent, comme il avait été convenu ; l’ayant rencontré en marche, ils le supplièrent de ne pas aller plus avant. (2) Ne l’ayant pas obtenu, ils le prièrent d’envoyer à la cavalerie qui formait l’avant-garde, l’ordre de ne pas commencer le combat, et de leur laisser le temps de députer vers les Ubiens ; (3) en protestant que, si le sénat et les principaux de cette nation s’engageaient à les recevoir sous la foi du serment, ils accepteraient toute condition que César leur imposerait ; ils demandaient trois jours pour consommer cet arrangement. (4) César pensait bien qu’ils sollicitaient ce délai de trois jours pour donner à leurs cavaliers absents le temps de revenir ; cependant il leur dit qu’il ne s’avancerait pas au-delà de quatre milles pour trouver de l’eau, (5) et il leur recommanda de venir le lendemain en grand nombre, pour qu’il prît connaissance de leurs demandes. (6) En même temps il envoya dire aux préfets qui marchaient en avant avec toute la cavalerie de ne point attaquer les ennemis, et, s’ils étaient eux-mêmes attaqués, de tenir seulement jusqu’à ce qu’il se fût rapproché d’eux avec l’armée.
\subsection[{§ 12.}]{ \textsc{§ 12.} }
\noindent (1) Mais, dès que les ennemis aperçurent notre cavalerie forte de cinq mille hommes, ils tombèrent sur elle, quoique la leur n’en eût pas plus de huit cents ; car ceux de leurs cavaliers qui étaient allés fourrager au-delà de la Meuse n’étaient pas encore revenus. Les nôtres étaient sans défiance, vu que les députés germains avaient quitté César peu auparavant et demandé une trêve pour ce jour-là. Cette attaque avait promptement mis le désordre parmi nous. (2) Quand nous nous fûmes ralliés, les ennemis, selon leur coutume, mirent pied à terre, tuèrent plusieurs de nos chevaux, renversèrent quelques cavaliers, défirent le reste et les frappèrent d’une telle frayeur qu’ils ne s’arrêtèrent qu’à la vue de notre armée. (3) II périt dans ce combat soixante-quatorze de nos cavaliers. (4) De ce nombre, fut Pison l’Aquitain, homme d’un grand courage et d’une naissance illustre, dont l’aïeul avait exercé le souverain pouvoir dans sa cité et reçu de notre sénat le titre d’ami. (5) Accouru au secours de son frère, enveloppé par les ennemis, il l’avait arraché à ce danger ; renversé lui-même de son cheval, qui avait été blessé, il se défendit courageusement et aussi longtemps qu’il put. (6) Lorsque entouré de toutes parts, et percé de coups, il eut succombé, son frère, déjà retiré de la mêlée, l’aperçut de loin, poussa son cheval vers les ennemis, s’offrit à eux et se fit tuer.
\subsection[{§ 13.}]{ \textsc{§ 13.} }
\noindent (1) Après cette action, César jugea qu’il ne devait plus écouter les députés ni recevoir les propositions d’un ennemi qui, usant de dol et d’embûches, nous avait attaqués, tout en demandant la paix. (2) Attendre en outre que leurs troupes s’augmentassent par le retour de leur cavalerie, eût été, pensait-il, de la dernière folie ; (3) connaissant d’ailleurs la légèreté des Gaulois, sentant que l’issue d’un seul combat les portait à s’exagérer la puissance de l’ennemi, il estima ne pas devoir leur laisser le temps de prendre un parti. (4) Quand il eut arrêté et communiqué à ses lieutenants et à son questeur sa résolution de ne pas différer de livrer bataille, il arriva fort à propos que le lendemain matin, les Germains, conduits par le même esprit de perfidie et de dissimulation, se réunirent en grand nombre avec tous leurs chefs et les vieillards, et vinrent au camp de César. (5) Ils voulaient, disaient-ils, se justifier de l’attaque faite la veille, contrairement à ce qui avait été réglé et à ce qu’ils avaient eux-mêmes demandé ; leur but était, s’ils le pouvaient, d’obtenir par une ruse la prolongation de la trêve. (6) César, charmé de ce qu’ils s’offraient ainsi à lui, donna ordre de les arrêter ; puis il fit sortir toutes les troupes du camp, et mit à l’arrière-garde la cavalerie qu’il supposait effrayée du dernier combat.
\subsection[{§ 14.}]{ \textsc{§ 14.} }
\noindent (1) Après avoir rangé l’armée sur trois lignes et fait une marche rapide de huit milles, il arriva au camp des Germains avant qu’ils pussent savoir ce qui s’était passé. (2) Frappés tout à la fois d’une terreur subite et par la promptitude de notre arrivée et par l’absence de leurs chefs ; n’ayant le temps ni de délibérer ni de prendre les armes, ils ne savaient, dans leur trouble, s’ils devaient marcher contre nous, défendre le camp ou chercher leur salut dans la fuite. (3) Leur terreur se manifesta par des cris et un grand désordre : nos soldats, animés par la perfidie de la veille, fondirent sur le camp. (4) Là, ceux qui purent prendre promptement les armes firent quelque résistance et combattirent entre les chars et les bagages ; (5) mais la multitude des enfants et des femmes (car les Germains étaient sortis de leur pays et avaient passé le Rhin avec tout ce qu’ils possédaient), se mit à fuir de toutes parts ; César envoya la cavalerie à leur poursuite.
\subsection[{§ 15.}]{ \textsc{§ 15.} }
\noindent (1) Les Germains, entendant des cris derrière eux et voyant le carnage qu’on faisait des leurs, jettent leurs armes, abandonnent leurs enseignes, et s’échappent du camp. (2) Lorsqu’ils furent parvenus au confluent de la Meuse et du Rhin, que l’espoir de prolonger leur fuite leur fut ravi, et qu’un grand nombre d’entre eux eut été tué, ce qui en restait se précipita dans le fleuve, et y périt, accablé par la peur, la fatigue et la force du courant. (3) Les nôtres, sans avoir perdu un seul homme et ne comptant que quelques blessés, délivrés des inquiétudes d’une guerre si redoutable, dans laquelle ils avaient eu en tête quatre cent trente mille ennemis, rentrèrent dans leur camp. (4) César rendit à ceux qu’il avait retenus la faculté de se retirer ; (5) mais, ceux-ci craignant les supplices et la vengeance des Gaulois dont ils avaient ravagé les terres, exprimèrent le désir de rester près de lui. César leur en accorda la permission.
\subsection[{§ 16.}]{ \textsc{§ 16.} }
\noindent (1) Après avoir terminé la guerre contre les Germains, César se détermina, par beaucoup de motifs, à passer le Rhin. Il voulait principalement, voyant les Germains toujours prêts à venir dans la Gaule, leur inspirer des craintes pour leur propre pays, en leur montrant qu’une armée romaine pouvait et osait traverser le Rhin. (2) De plus, cette partie de la cavalerie des Usipètes et des Tencthères que j’ai dit plus haut avoir passé la Meuse pour piller et fourrager, et qui n’avait point assisté au combat, s’était, après la défaite des Germains, retirée au-delà du Rhin, chez les Sugambres, et s’était jointe à eux. (3) César envoya vers ce peuple et fit demander qu’il lui livrât ceux qui avaient porté les armes contre lui et contre les Gaulois. Ils répondirent : (4) "que la domination du peuple romain finissait au Rhin : s’il ne trouvait pas juste que les Germains passassent en Gaule malgré lui, pourquoi prétendait-il exercer quelque domination et quelque pouvoir au-delà du Rhin ? (5) Les Ubiens, qui, seuls des peuples d’outre-Rhin, avaient député vers César, contracté une alliance, livré des otages, le priaient instamment de les secourir contre les Suèves qui les pressaient vivement ; (6) ou, si les affaires de la république l’empêchaient de le faire, de transporter seulement l’armée au-delà du Rhin ; ce serait un secours suffisant et une sécurité pour l’avenir : (7) la défaite d’Arioviste et ce dernier combat avaient tellement établi la réputation de l’armée romaine chez les nations germaines même les plus reculées, que l’autorité et l’amitié du peuple romain devaient les mettre en sûreté. (8) Ils promettaient une grande quantité de navires pour le transport de l’armée.
\subsection[{§ 17.}]{ \textsc{§ 17.} }
\noindent (1) César, déterminé par les motifs dont j’ai parlé, avait résolu de passer le Rhin ; mais le traverser sur des bateaux ne lui semblait ni assez sûr ni assez convenable à sa dignité et à celle du peuple romain. (2) Aussi, malgré l’extrême difficulté qu’offrait la construction d’un pont, à cause de la largeur, de la rapidité et de la profondeur du fleuve, il crut cependant devoir le tenter ; autrement il fallait renoncer à faire passer l’armée. (3) Voici donc sur quel plan il fit construire le pont : on joignait ensemble, à deux pieds d’intervalle, deux poutres d’un pied et demi d’équarrissage, un peu aiguisées par le bas, d’une hauteur proportionnée à celle du fleuve. (4) Introduites dans l’eau à l’aide des machines, elles y étaient fichées et enfoncées à coups de masse, non dans une direction verticale, mais en suivant une ligne oblique et inclinée selon le fil de l’eau. (5) En face et en descendant, à la distance de quarante pieds, on en plaçait deux autres, assemblées de la même manière, et tournées contre la violence et l’effort du courant. (6) Sur ces quatre poutres on en posait une de deux pieds d’équarrissage, qui s’enclavait dans leur intervalle, et était fixée à chaque bout par deux chevilles. (7) Ces quatre pilotis, réunis par une traverse ; offraient un ouvrage si solide, que plus la rapidité du courant était grande, plus elle consolidait cette construction. (8) On jeta ensuite des solives d’une traverse à l’autre, et on couvrit le tout de fascines et de claies. (9) De plus, des pieux obliquement fichés vers la partie inférieure du fleuve s’appuyaient contre les pilotis en forme de contreforts et servaient à briser le courant. (10) Enfin d’autres pieux étaient placés en avant du pont, à peu de distance, afin que, si les barbares lançaient des troncs d’arbres ou des bateaux pour abattre ces constructions, elles fussent ainsi protégées contre ces tentatives inutiles, et que le pont n’en eût point à souffrir.
\subsection[{§ 18.}]{ \textsc{§ 18.} }
\noindent (1) Tout l’ouvrage fut achevé en dix jours, à compter de celui où les matériaux furent apportés sur place. (2) César fit passer l’armée, et, laissant une forte garde à chaque tête de pont, il marcha vers le pays des Sugambres. (3) Ayant, pendant sa marche, reçu des députés de diverses nations, qui venaient réclamer la paix et son amitié, il leur fit une réponse bienveillante, et exigea qu’on lui amenât des otages. (4) De leur côté, les Sugambres qui, du moment où l’on commençait à construire le pont, et sur l’avis des Usipètes et des Tencthères réfugiés chez eux, avaient tout préparé pour fuir, venaient d’abandonner leur pays, emportant tous leurs biens, et s’étaient retirés dans les déserts et dans les forêts.
\subsection[{§ 19.}]{ \textsc{§ 19.} }
\noindent (1) César, après un très court séjour dans ce pays, dont il brûla les bourgs et les habitations et détruisit les récoltes, se rendit chez les Ubiens, et leur promit son secours s’ils étaient attaqués par les Suèves. Il apprit d’eux (2) que ces derniers, informés par leurs éclaireurs que l’on jetait un pont sur le Rhin, avaient, selon leur coutume, tenu conseil et envoyé partout l’ordre de sortir des villes ; de déposer dans les bois les femmes, les enfants et tous les biens, enjoignant à tous les hommes en état de porter les armes de se réunir dans un même lieu : (3) ce rendez-vous était à peu près au centre des régions occupées par les Suèves. C’était là qu’ils avaient décidé d’attendre l’arrivée des Romains pour les combattre. (4) Instruit de ce dessein, et ayant obtenu tous les résultats qu’il s’était proposés en faisant passer le Rhin à l’armée, à savoir : d’intimider les Germains, de se venger des Sugambres, et de délivrer les Ubiens pressés par les Suèves, César, après dix-huit jours en tout passés au-delà du Rhin, crut avoir assez fait pour la gloire et l’intérêt de Rome, revint dans la Gaule et fit rompre le pont.
\subsection[{§ 20.}]{ \textsc{§ 20.} }
\noindent (1) Quoique l’été fût fort avancé et que dans la Gaule, à cause de sa position vers le nord, les hivers soient hâtifs, César résolut néanmoins de passer dans la Bretagne, pays qu’il savait avoir fourni des secours à nos ennemis dans presque toutes les guerres contre les Gaulois. (2) Si la saison ne lui permettait pas de terminer cette expédition, il lui serait cependant, à ce qu’il lui semblait, très utile de visiter seulement cette île, d’en reconnaître les habitants, les localités, les ports, les abords, toutes choses presque inconnues aux Gaulois. (3) Nul en effet, si ce n’est les marchands, ne se hasarde à y aborder, et ceux-ci même n’en connaissent que les côtes et les pays situés vis-à-vis de la Gaule. (4) Ayant donc fait venir de tous côtés des marchands, César n’en put rien apprendre ni sur l’étendue de l’île, ni sur la nature et le nombre des nations qui l’habitaient, ni sur leur manière de faire la guerre, ni sur ceux des ports qui étaient les plus propres à recevoir beaucoup de grands vaisseaux.
\subsection[{§ 21.}]{ \textsc{§ 21.} }
\noindent (1) Pour acquérir ces connaissances avant de s’engager dans l’expédition, il envoie, avec une galère, C. Volusénus, qu’il jugeait propre à cette mission. (2) II lui recommande de revenir au plus tôt dès qu’il aurait tout vu. (3) Lui-même, avec toutes les troupes, part pour le pays des Morins, d’où le trajet en Bretagne est très court. (4) Il y rassemble tous les vaisseaux qu’il peut tirer des régions voisines et la flotte qu’il avait construite l’été précédent pour la guerre des Vénètes. (5) Cependant, instruits de son projet par les rapports des marchands, les Bretons envoient à César des députés de plusieurs cités, qui promettent de livrer des otages et de se soumettre à l’empire du peuple romain. (6) Après les avoir entendus, César leur fait de bienveillantes promesses, les exhorte à persévérer dans ces sentiments, (7) et enfin les renvoie, accompagnés de Commios qu’il avait lui-même, après ses victoires sur les Atrébates, fait roi de cette nation, homme dont le courage et la prudence lui étaient connus, qu’il pensait lui être dévoué, et qui avait un grand crédit en Bretagne. (8) Il lui ordonne de visiter le plus grand nombre possible de nations, de les exhorter à se remettre sous la foi du peuple romain, et de leur annoncer sa prochaine arrivée chez elles. (9) Volusénus ayant inspecté la contrée, autant que pouvait le faire un homme qui n’osait sortir de son vaisseau ni se fier aux barbares, revient le cinquième jour auprès de César, et lui rend un compte détaillé de ses observations.
\subsection[{§ 22.}]{ \textsc{§ 22.} }
\noindent (1) Tandis que César était retenu dans ces lieux pour y rassembler la flotte, les députés d’une grande partie des peuples Morins vinrent le trouver, pour s’excuser de leur conduite passée, rejetant sur leur qualité d’étrangers et sur leur ignorance de nos coutumes, le tort d’avoir fait la guerre au peuple romain, et promettant de faire ce qu’il leur commanderait. (2) César trouva que ces soumissions survenaient assez à propos : il ne voulait point laisser d’ennemi derrière lui ; la saison était trop avancée pour qu’il pût entreprendre cette guerre, et il ne croyait pas d’ailleurs que ces petits intérêts dussent être préférés à son entreprise contre la Bretagne. Il exigea donc un grand nombre d’otages. On les lui amena, et il reçut la soumission de ce peuple. (3) Ayant fait venir et rassemblé quatre-vingts vaisseaux de charge, nombre qu’il jugea suffisant pour le transport de deux légions, il distribua tout ce qu’il avait de galères à son questeur, à ses lieutenants et aux préfets. (4) Il avait de plus dix-huit vaisseaux de charge, que le vent retenait à huit milles de cet endroit, et empêchait d’aborder au même port. Il les destina à sa cavalerie (5) et envoya le reste de l’armée, sous le commandement de Q. Titurius Sabinus et L. Aurunculéius Cotta, ses lieutenants, chez les Ménapes et sur les points du territoire des Morins, dont il n’avait pas encore reçu de députés. (6) II préposa à la garde du port son lieutenant P. Sulpicius Rufus, avec la garnison qui fut jugée nécessaire.
\subsection[{§ 23.}]{ \textsc{§ 23.} }
\noindent (1) Ces dispositions faites, César, profitant d’un vent favorable à sa navigation, leva l’ancre vers la troisième veille, et ordonna à sa cavalerie d’aller s’embarquer au port voisin et de le suivre. (2) Celle-ci fit peu de diligence, et il n’avait que ses premiers vaisseaux lorsqu’il toucha à la Bretagne, environ à la quatrième heure du jour. Là il vit les troupes ennemies occupant, sous les armes, toutes les collines. (3) Telle était la nature des lieux la mer était si resserrée par des montagnes que le trait lancé de ces hauteurs pouvait atteindre le rivage. (4) Jugeant l’endroit tout à fait défavorable pour un débarquement, il resta à l’ancre jusqu’à la neuvième heure, et attendit l’arrivée du reste de la flotte. (5) Cependant il assemble ses lieutenants et les tribuns des soldats, leur fait part des renseignements de Volusénus et de ses desseins ; il les avertit d’agir d’eux-mêmes en tout, selon l’opportunité et le temps, comme le demande la guerre, surtout une guerre maritime, où un seul instant peut aussitôt changer l’état des choses. (6) Quand il les eut renvoyés et que le vent et la marée furent devenus en même temps favorables, il donna lé signal, fit lever l’ancre et s’arrêta à sept milles de là environ, devant une plage ouverte et unie.
\subsection[{§ 24.}]{ \textsc{§ 24.} }
\noindent (1) Mais les Barbares, s’apercevant du dessein des Romains, envoyèrent en avant leur cavalerie et les chariots de guerre dont ils ont coutume de se servir dans les combats, les suivirent avec le reste de leurs troupes, et s’opposèrent à notre débarquement. (2) Plusieurs circonstances le rendaient extrêmement difficile : la grandeur de nos vaisseaux forcés de s’arrêter en pleine mer, l’ignorance où étaient nos soldats de la nature des lieux ; les mains embarrassées, accablés du poids énorme de leurs armes, ils devaient à la fois s’élancer du navire, résister à l’effort des vagues et lutter avec l’ennemi ; (3) tandis que celui-ci combattant à pied sec, ou s’avançant très peu dans la mer, libre de tous ses membres, connaissant parfaitement les lieux, lançait ses traits avec assurance et poussait ses chevaux faits à cette manoeuvre. (4) Frappés d’un tel concours de circonstances, et tout à fait inexpérimentés dans ce genre de combat, nos soldats ne s’y portaient pas avec cette ardeur et avec ce zèle qui leur étaient ordinaires dans les combats de pied ferme.
\subsection[{§ 25.}]{ \textsc{§ 25.} }
\noindent (1) Dès que César s’en aperçut, il ordonna d’éloigner un peu des vaisseaux de charge, les galères dont la forme était moins connue des Barbares et la manoeuvre plus facile et plus prompte, de les diriger à force de rames, de les tenir devant le flanc découvert de l’ennemi, et de là, à l’aide des frondes, des traits et des machines, de le repousser et de le chasser de sa position. Ce mouvement nous fut d’une grande utilité. (2) Car étonnés de la forme de nos navires, de leur manoeuvre et du genre inconnu de nos machines, les Barbares s’arrêtèrent et firent même quelques pas en arrière. (3) Nos soldats hésitaient encore, surtout à cause de la profondeur de la mer : le porte-aigle de la dixième légion, après avoir invoqué les dieux pour que sa légion eût l’honneur du succès : "Compagnons, dit-il, sautez à la mer, si vous ne voulez livrer l’aigle aux ennemis ; pour moi certes j’aurai fait mon devoir envers la république et le général." (4) À ces mots, prononcés d’une voix forte, il s’élance du navire et porte l’aigle vers l’ennemi. (5) Alors les nôtres s’exhortant mutuellement à ne pas souffrir une telle honte, se jettent tous hors du vaisseau. A cette vue, ceux des navires voisins les suivent et marchent à l’ennemi.
\subsection[{§ 26.}]{ \textsc{§ 26.} }
\noindent (1) On combattit de part et d’autre avec acharnement : nos soldats cependant ne pouvant ni garder leurs rangs, ni lutter de pied ferme, ni suivre leurs enseignes, et forcés de se ranger sous le premier drapeau qui s’offrait à eux, à quelque vaisseau qu’il appartint, étaient dans une grande confusion. (2) Les ennemis au contraire, connaissant tous les bas-fonds, avaient à peine vu du rivage quelques-uns des nôtres débarquer, qu’ils poussaient contre eux leurs chevaux et les attaquaient au milieu de leur embarras, (3) un grand nombre en enveloppait un petit ; les autres prenant en flanc le gros de notre armée l’accablaient de leurs traits. (4) Témoin de ce désavantage, César fit remplir de soldats les chaloupes des galères et les esquifs d’observation, et les envoya au secours de ceux qu’il voyait dans une situation critique. (5) Dès que nos soldats eurent pris terre et que tous les autres les eurent suivis, ils fondirent sur les ennemis et les mirent en fuite, mais sans pouvoir les poursuivre bien loin, la cavalerie n’ayant pu suivre sa route ni aborder dans l’île. Cette seule chose manqua à la fortune accoutumée de César.
\subsection[{§ 27.}]{ \textsc{§ 27.} }
\noindent (1) Les ennemis, après leur défaite et dès qu’ils se furent ralliés, envoyèrent des députés à César, pour demander la paix. Ils promirent de donner des otages et de faire tout ce qu’il ordonnerait. (2) Avec ces députés vint Commios, le roi des Atrébates que César, ainsi que je l’ai dit plus haut, avait envoyé avant lui en Bretagne. (3) Ils l’avaient saisi à sa sortie du vaisseau, comme il leur apportait, en qualité d’orateur, les ordres du général, et ils l’avaient jeté dans les fers. (4) Ils le relâchèrent après le combat, et en sollicitant la paix, ils rejetèrent sur la multitude le tort de cet acte, dont ils demandèrent le pardon, motivé par leur ignorance. (5) César se plaignit de ce qu’après lui avoir, de leur propre mouvement, envoyé demander la paix sur le continent, ils lui eussent fait la guerre sans motif : Il leur dit qu’il pardonnait à leur égarement et exigea des otages. (6) Ils en donnèrent sur-le-champ une partie, et promirent que le reste, qui devait venir de contrées éloignées, serait livré sous peu de jours. (7) En même temps ils renvoyèrent leurs soldats dans leurs foyers ; et leurs chefs vinrent de tous côtés recommander à César leurs personnes et leurs cités.
\subsection[{§ 28.}]{ \textsc{§ 28.} }
\noindent (1) La paix était ainsi assurée, et il y avait quatre jours qu’on était arrivé en Bretagne, lorsque les dix-huit navires, dont il a été parlé plus haut, et qui portaient la cavalerie, sortirent par un bon vent, du port des Morins. (2) Comme ils approchaient de la Bretagne et étaient en vue du camp, il s’éleva tout à coup une si violente tempête, qu’aucun d’eux ne put suivre sa route, et qu’ils furent, les uns rejetés dans le port d’où ils étaient partis, les autres poussés vers la partie inférieure de l’île, qui est à l’occident, où ils coururent de grands dangers. (3) Ils y jetèrent l’ancre, mais inondés par les vagues, ils furent forcés, au milieu d’une nuit orageuse, de reprendre la haute mer, et de regagner le continent.
\subsection[{§ 29.}]{ \textsc{§ 29.} }
\noindent (1) Il se trouva que cette nuit-là même la lune était en son plein, époque ordinaire des plus hautes marées de l’Océan. Nos soldats l’ignoraient. (2) L'eau eut donc bientôt rempli les galères dont César s’était servi pour le transport de l’armée et qu’il avait mises à sec. Les vaisseaux de charge, restés à l’ancre dans la rade, étaient battus par les flots, sans qu’il y eût aucun moyen de les gouverner ni de les secourir. (3) Plusieurs furent brisés ; les autres, dépouillés de leurs cordages, de leurs ancres et du reste de leurs agrès, se trouvaient hors d’état de servir, ce qui, chose inévitable, répandit la consternation dans toute l’armée. (4) On n’avait pas en effet d’autres vaisseaux pour la reporter sur le continent, et tout manquait de ce qui est nécessaire pour la réparation. Enfin, comme on s’attendait généralement à hiverner dans la Gaule, aucune provision de blé n’avait été faite pour passer l’hiver dans ce pays.
\subsection[{§ 30.}]{ \textsc{§ 30.} }
\noindent (1) À la nouvelle de cette détresse, les chefs de la Bretagne qui, après la bataille, s’étaient réunis pour l’exécution des ordres de César, tinrent conseil entre eux. Voyant les Romains dépourvus de cavalerie, de vaisseaux et de vivres, et jugeant du petit nombre de nos soldats par l’exiguïté de notre camp, d’autant plus resserré que César avait fait embarquer les légions sans bagages, (2) ils crurent l’occasion très favorable pour se révolter, nous couper les vivres et prolonger la campagne jusqu’à l’hiver. Ils tenaient pour assuré qu’en triomphant de notre armée ou en lui fermant le retour, nul ne traverserait désormais la mer pour porter la guerre en Bretagne(3) Une ligue est donc formée de nouveau ils commencent à s’échapper peu à peu de notre camp et à faire revenir en secret les hommes qu’ils avaient licenciés.
\subsection[{§ 31.}]{ \textsc{§ 31.} }
\noindent (1) César, il est vrai, ne connaissait pas encore leurs desseins ; mais d’après le désastre de sa flotte et le retard des Bretons à livrer les otages, il soupçonnait cependant ce qui arriva. (2) Aussi se tenait-il prêt à tout événement ; car il portait chaque jour des vivres dans le camp, employait le bois et le cuivre des navires les plus avariés à la réparation des autres, et faisait venir du continent ce qui était nécessaire pour ce travail. C'est ainsi que, secondé par le zèle extrême des soldats, il mit, à l’exception de douze vaisseaux perdus, toute la flotte en état de naviguer.
\subsection[{§ 32.}]{ \textsc{§ 32.} }
\noindent (1) Pendant qu’on faisait ces travaux, une légion, la septième, avait été, selon la coutume, envoyée au fourrage ; et, jusqu’à ce jour, il y avait d’autant moins lieu de soupçonner des hostilitésqu’ une partie des Bretons restait dans les campagnes, et que d’autres venaient librement dans le camp. Les soldats en faction aux portes du camp annoncent à César qu’une poussière plus épaisse que de coutume s’élevait du côté où la légion s’était dirigée. (2) César, soupçonnant, ce que c’était, quelque nouvelle entreprise de la part des barbares, prend avec lui les cohortes de garde, les fait remplacer dans leur poste par deux autres, et ordonne au reste des troupes de s’armer et de le suivre aussitôt. (3) Quand il se fut avancé à peu de distance du camp, il vit les siens pressés par l’ennemi, résistant avec peine, la légion serrée, en butte à une grêle de traits. (4) Car tout le grain ayant été moissonné dans les autres endroits, et un seul étant resté intact, les ennemis présumant que nous y viendrions, s’étaient cachés la nuit dans les bois. (5) Alors fondant subitement sur nos soldats dispersés, désarmés, occupés à couper le grain, ils en avaient tué quelques-uns, troublé les autres dans leurs rangs mal formés, et les avaient enveloppés à la fois de leur cavalerie et de leurs chariots.
\subsection[{§ 33.}]{ \textsc{§ 33.} }
\noindent (1) Voici leur manière de combattre avec ces chariots. D'abord ils les font courir sur tous les points en lançant des traits ; et, par la seule crainte qu’inspirent les chevaux et le bruit des roues, ils parviennent souvent à rompre les rangs. Quand ils ont pénétré dans les escadrons, ils sautent à bas de leurs chariots et combattent à pied. (2) Les conducteurs se retirent peu à peu de la mêlée, et placent les chars de telle façon que si les combattants sont pressés par le nombre, ils puissent aisément se replier sur eux. (3) C'est ainsi qu’ils réunissent dans les combats l’agilité du cavalier à la fermeté du fantassin ; et tel est l’effet de l’habitude et de leurs exercices journaliers, que, dans les pentes les plus rapides, ils savent arrêter leurs chevaux au galop, les modérer et les détourner aussitôt, courir sur le timon, se tenir ferme sur le joug, et delà s’élancer précipitamment dans leurs chars.
\subsection[{§ 34.}]{ \textsc{§ 34.} }
\noindent (1) Ce nouveau genre de combat avait jeté le trouble parmi les nôtres, et César leur porta très à propos du secours ; car, à son approche, les ennemis s’arrêtèrent, et les Romains se remirent de leur frayeur. (2) Malgré ce résultat, ne jugeant pas l’occasion favorable pour attaquer l’ennemi et engager un combat, il se tint dans cette position, et peu de temps après, il ramena les légions dans le camp. (3) Cependant, nous voyant occupés ailleurs, ce qui restait de Bretons dans la campagne se retira. (4) Il y eut depuis, pendant plusieurs jours de suite, du mauvais temps qui nous retint dans le camp et empêcha l’ennemi de nous attaquer. (5) Dans cet intervalle, les barbares envoyèrent de tous côtés des messagers pour exposer aux leurs la faiblesse de nos troupes, et la facilité qui s’offrait de faire un riche butin et de recouvrer à jamais leur liberté, s’ils chassaient les Romains de leur camp. Dans cette vue, ils rassemblèrent bientôt une multitude de troupes à pied et à cheval, et marchèrent sur nous.
\subsection[{§ 35.}]{ \textsc{§ 35.} }
\noindent (1) César prévoyait bien qu’il en serait de ce combat comme de ceux des jours précédents, et que, s’il repoussait les ennemis, ils échapperaient aisément aux dangers de leur défaite. Cependant il prit environ trente chevaux que l’Atrébate Commios, dont il a été déjà parlé, avait menés avec lui, et rangea les légions en bataille à la tête du camp. (2) Le combat engagé, les ennemis ne purent soutenir longtemps le choc de nos soldats, et prirent la fuite. (3) On les poursuivit aussi loin qu’on eut de vitesse et de force, et on en tua un grand nombre. Les légions rentrèrent ensuite dans le camp, après avoir tout détruit et brûlé sur une grande étendue de terrain.
\subsection[{§ 36.}]{ \textsc{§ 36.} }
\noindent (1) Le même jour, des députés, envoyés à César par l’ennemi, vinrent demander la paix. (2) César doubla le nombre des otages qu’il leur avait demandés précédemment, et ordonna de les lui amener sur le continent, parce que le temps de l’équinoxe approchait et qu’il ne voulait point exposer à une navigation d’hiver des vaisseaux en mauvais état. (3) Profitant d’un temps favorable, il leva l’ancre peu après minuit ; (4) tous ses navires regagnèrent le continent sans le moindre dommage ; dans le nombre, deux vaisseaux de charge seulement ne purent aborder au même port que les autres, et furent portés un peu plus bas.
\subsection[{§ 37.}]{ \textsc{§ 37.} }
\noindent (1) Ces derniers vaisseaux contenaient environ trois cents soldats qui se dirigèrent vers le camp. Les Morins, que César, avant son départ pour la Bretagne, avait laissés soumis, séduits par l’espoir du butin, les enveloppèrent d’abord en assez petit nombre, et leur ordonnèrent, s’ils ne voulaient pas être tués, de mettre bas les armes.(2) Les nôtres, s’étant formés en cercle, se défendirent ; aux cris de l’ennemi, six mille hommes environ accourent aussitôt : César, à cette nouvelle, envoya du camp toute la cavalerie au secours des siens. (3) Cependant nos soldats avaient soutenu le choc de l’ennemi et vaillamment combattu pendant plus de quatre heures ; ils avaient reçu peu de blessures et tué beaucoup d’ennemis. (4) Mais, lorsque notre cavalerie se montra, les Barbares, jetant leurs armes, tournèrent le dos, et on en fit un grand carnage.
\subsection[{§ 38.}]{ \textsc{§ 38.} }
\noindent (1) Le jour suivant, César envoya son lieutenant T. Labiénus, avec les légions ramenées de la Bretagne, contre les Morins qui venaient de se révolter. (2) Comme les marais se trouvaient à sec, ce qui privait les ennemis d’un refuge qui leur avait servi l’année précédente, ils tombèrent presque tous au pouvoir de Labiénus. (3) D'un autre côté, les lieutenants Q. Titurius et L. Cotta, qui avaient conduit des légions sur le territoire des Ménapes, avaient ravagé leurs champs, coupé leur blés, brûlé leur habitations, tandis qu’ils s’étaient retirés tous dans l’épaisseur des forêts, et étaient revenus vers César. (4) Il établit chez les Belges les quartiers d’hiver de toutes les légions. Deux états seulement de la Bretagne lui envoyèrent en ce lieu des otages ; les autres négligèrent de le faire. (5) Ces guerres ainsi terminées, César écrivit au sénat des lettres qui firent décréter vingt jours d’actions de grâces.
\section[{Livre V}]{Livre V}\renewcommand{\leftmark}{Livre V}

\subsection[{§ 1.}]{ \textsc{§ 1.} }
\noindent (1) Sous le consulat de Lucius Domitius et d’Appius Claudius, César, quittant les quartiers d’hiver pour aller en Italie, comme il avait coutume de le faire chaque année, ordonne aux lieutenants qu’il laissait à la tête des légions de construire, pendant l’hiver, le plus de vaisseaux qu’il serait possible, et de réparer les anciens. (2) Il en détermine la grandeur et la forme. Pour qu’on puisse plus promptement les charger et les mettre à sec, il les fait moins hauts que ceux dont nous nous servons sur notre mer  ; il avait en effet observé que les mouvements fréquents du flux et du reflux rendaient les vagues de l’Océan moins élevées ; il les commande, à cause des bagages et du nombre des chevaux qu’ils devaient transporter, un peu plus larges que ceux dont on fait usage sur les autres mers. (3) Il veut qu’on les fasse tous à voiles et à rames, ce que leur peu de hauteur devait rendre très facile. (4) Tout ce qui est nécessaire pour l’armement de ces vaisseaux, il le fait venir de l’Espagne. (5) Lui-même, après avoir tenu l’assemblée de la Gaule citérieure, part pour l’Illyrie, sur la nouvelle que les Pirustes désolaient, par leurs incursions, la frontière de cette province. (6) À peine arrivé, il ordonne aux cités de lever des troupes, et leur assigne un point de réunion. (7) À cette nouvelle, les Pirustes lui envoient des députés, qui lui exposent que rien de ce qui s’était passé n’était le résultat d’une délibération nationale, et se disent prêts à lui offrir, pour ces torts, toutes les satisfactions. (8) En acceptant leurs excuses, César exige des otages, et qu’ils soient amenés à jour fixe ; à défaut de quoi il leur déclare qu’il portera la guerre dans leur pays. (9) Les otages sont livrés au jour marqué ; il nomme des arbitres pour estimer le dommage et en régler la réparation.
\subsection[{§ 2.}]{ \textsc{§ 2.} }
\noindent (1) Cette affaire terminée, et l’assemblée close, César retourne dans la Gaule citérieure et part de là pour l’armée. (2) Quand il y est arrivé, il en visite tous les quartiers, et trouve que l’activité singulière des soldats était parvenue, malgré l’extrême pénurie de toutes choses, à construire environ six cents navires de la forme décrite plus haut, et vingt-huit galères, le tout prêt à être mis en mer sous peu de jours. (3) Après avoir donné des éloges aux soldats et à ceux qui avaient dirigé l’ouvrage, il les instruit de ses intentions et leur ordonne de se rendre tous au port Itius, d’où il savait que le trajet en Bretagne est très commode, la distance de cette île au continent n’étant que de trente mille pas. Il leur laisse le nombre de soldats qu’il juge suffisant ; (4) pour lui, il marche, avec quatre légions sans bagages et huit cents cavaliers, chez les Trévires, qui ne venaient point aux assemblées, n’obéissaient pas à ses ordres, et qu’on soupçonnait de solliciter les Germains à passer le Rhin.
\subsection[{§ 3.}]{ \textsc{§ 3.} }
\noindent (1) Cette nation est de beaucoup la plus puissante de toute la Gaule par sa cavalerie, et possède de nombreuses troupes de pied ; elle habite, comme nous l’avons dit plus haut, les bords du Rhin. (2) Deux hommes s’y disputaient la souveraineté, Indutiomaros et Cingétorix. (3) Ce dernier, à peine instruit de l’arrivée de César et des légions, se rend près de lui, l’assure que lui et tous les siens resteront dans le devoir, fidèles à l’amitié du peuple romain, et l’instruit de tout ce qui se passait chez les Trévires. (4) Indutiomaros au contraire lève de la cavalerie et de l’infanterie ; tous ceux que leur âge met hors d’état de porter les armes, il les fait cacher dans la forêt des Ardennes, forêt immense, qui traverse le territoire des Trévires, et s’étend depuis le fleuve du Rhin jusqu’au pays des Rèmes ; il se prépare ensuite à la guerre. (5) Mais quand il eut vu plusieurs des principaux de l’état, entraînés par leurs liaisons avec Cingétorix ou effrayés par l’arrivée de notre armée, se rendre auprès de César, et traiter avec lui de leurs intérêts particuliers, ne pouvant rien pour ceux de leur pays, Indutiomaros, craignant d’être abandonné de tous, envoie des députés à César  : (6) il l’assure que, s’il n’a pas quitté les siens et n’est pas venu le trouver, c’était pour retenir plus facilement le pays dans le devoir, et l’empêcher de se porter, en l’absence de toute la noblesse, à d’imprudentes résolutions. (7) Au surplus, il avait tout pouvoir sur la nation ; il se rendrait, si César le permettait, au camp des Romains, et remettrait à sa foi ses intérêts propres et ceux de son pays.
\subsection[{§ 4.}]{ \textsc{§ 4.} }
\noindent (1) Quoique César comprît bien les motifs de ce langage et de ce changement de dessein, comme il ne voulait point être forcé de passer l’été chez les Trévires, tandis que tout était prêt pour la guerre de Bretagne, il ordonna à Indutiomaros de venir avec deux cents otages. (2) Quand celui-ci les eut amenés, et parmi eux son fils et tous ses proches parents, lesquels avaient été spécialement désignés, César le consola et l’exhorta à rester dans le devoir ; (3) toutefois, ayant assemblé les principaux des Trévires, il les rallia personnellement à Cingétorix, tant à cause de son mérite que parce qu’il lui semblait d’un grand intérêt d’augmenter chez les Trévires le crédit d’un homme qui avait fait preuve envers lui d’un zèle si remarquable. (4) Indutiomaros vit avec douleur l’atteinte que l’on portait ainsi à son influence, et, déjà notre ennemi, il devint dès lors irréconciliable.
\subsection[{§ 5.}]{ \textsc{§ 5.} }
\noindent (1) Ces choses terminées, César se rend avec les légions au port Itius. (2) Là, il apprend que quarante navires construits chez les Meldes, repoussés par une tempête, n’avaient pu tenir leur route, et étaient rentrés dans le port d’où ils étaient partis. Il trouva les autres prêts à mettre à la voile et pourvus de tout. (3) La cavalerie de toute la Gaule, au nombre de quatre mille hommes, se réunit en ce lieu, ainsi que les principaux habitants des cités. (4) César avait résolu de ne laisser sur le continent que le petit nombre des hommes influents dont la fidélité lui était bien connue, et d’emmener les autres comme otages avec lui ; car il craignait quelque mouvement dans la Gaule, pendant son absence.
\subsection[{§ 6.}]{ \textsc{§ 6.} }
\noindent (1) Parmi ces chefs était l’Héduen Dumnorix, dont nous avons déjà parlé. C'était celui-là surtout que César voulait avoir avec lui, connaissant son caractère avide de nouveautés, son ambition, son courage, son grand crédit parmi les Gaulois. (2) Il faut ajouter à ces motifs que déjà Dumnorix avait dit dans une assemblée des Héduens que César lui offrait la royauté dans son pays. Ce propos leur avait fortement déplu ; et ils n’osaient adresser à César ni refus ni prières. II n’en fut instruit que par ses hôtes. (3) Dumnorix eut d’abord recours à toutes sortes de supplications pour rester en Gaule, disant, tantôt qu’il craignait la mer à laquelle il n’était pas habitué, tantôt qu’il était retenu par des scrupules de religion. (4) Lorsqu’il vit qu’on lui refusait obstinément sa demande, et que tout espoir de l’obtenir était perdu, il commença à intriguer auprès des chefs de la Gaule, à les prendre à part et à les presser de rester sur le continent ; il cherchait à leur inspirer des craintes ; (5) ce n’était pas sans motif qu’on dégarnissait la Gaule de toute sa noblesse, le dessein de César était de faire périr, après leur passage en Bretagne, ceux qu’il n’osait égorger à la vue des Gaulois ; (6) il leur donna sa foi et sollicita la leur pour faire de concert ce qu’ils croiraient utile à la Gaule. Plusieurs rapports instruisirent César de ces menées.
\subsection[{§ 7.}]{ \textsc{§ 7.} }
\noindent (1) À ces nouvelles, César, qui avait donné tant de considération à la nation héduenne, résolut de réprimer et de prévenir Dumnorix par tous les moyens possibles. (2) Comme il le voyait persévérer dans sa folie, il crut devoir l’empêcher de nuire à ses intérêts et à ceux de la république. (3) Pendant les vingt-cinq jours environ qu’il resta dans le port, retenu par un vent du nord-ouest qui souffle d’ordinaire sur cette côte pendant une grande partie de l’année, il s’appliqua à contenir Dumnorix dans le devoir, et néanmoins à se tenir au fait de toutes ses démarches. (4) Enfin le temps devint favorable, et César ordonna aux soldats et aux cavaliers de s’embarquer. (5) Mais, profitant de la préoccupation générale, Dumnorix sortit du camp avec la cavalerie héduenne, à l’insu de César, pour retourner dans son pays. (6) Sur l’avis qui lui en fut donné, César, suspendant le départ et ajournant toute affaire, envoya à sa poursuite une grande partie de la cavalerie, avec ordre de le ramener, (7) ou, s’il résistait et n’obéissait pas, de le tuer ; bien certain qu’il avait tout à craindre, pendant son absence, d’un homme qui, en sa présence, avait méprisé ses ordres. (8) Dumnorix, lorsqu’on l’eut atteint, fit résistance, mit l’épée à la main, et implora la fidélité des siens, s’écriant à plusieurs reprises qu’il était libre et membre d’une nation libre. (9) Il fut, comme on le leur avait ordonné, entouré et mis à mort. Les cavaliers héduens revinrent tous vers César.
\subsection[{§ 8.}]{ \textsc{§ 8.} }
\noindent (1) Ayant, après cette affaire, laissé sur le continent Labiénus avec trois légions et deux mille cavaliers pour garder le port, pourvoir aux vivres, connaître tout ce qui se passait dans la Gaule, et prendre conseil du temps et des circonstances, (2) César, avec cinq légions et un nombre de cavaliers égal à celui qu’il laissait sur le continent, leva l’ancre au coucher du soleil, par un léger vent du sud-ouest qui, ayant cessé vers le milieu de la nuit, ne lui permit pas de continuer sa route ; entraîné assez loin par la marée, il s’aperçut, au point du jour, qu’il avait laissé la Bretagne sur la gauche. (3) Alors, se laissant aller au reflux, il fit force de rames pour gagner cette partie de l’île, où il avait appris, l’été précédent, que la descente est commode. (4) On ne put trop louer, dans cette circonstance, le zèle des soldats qui, sur des vaisseaux de transport peu maniables, égalèrent, par le travail continu des rames, la vitesse des galères. (5) Toute la flotte prit terre environ vers midi ; aucun ennemi ne se montra dans ces parages ; (6) mais César sut plus tard des captifs que beaucoup de troupes s’y étaient réunies, et que, effrayées à la vue du grand nombre de nos vaisseaux (car y compris les barques légères que chacun destinait à sa commodité particulière, il y en avait plus de huit cents), elles s’étaient éloignées du rivage et réfugiées sur les hauteurs.
\subsection[{§ 9.}]{ \textsc{§ 9.} }
\noindent (1) César, ayant établi l’armée à terre et choisi un terrain propre au campement, dès qu’il eut appris par des prisonniers où s’étaient retirées les troupes ennemies, il laissa près de la mer dix cohortes et trois cents cavaliers pour la garde de la flotte, et, à la troisième veille, marcha contre les Bretons : il craignait d’autant moins pour les vaisseaux qu’il les laissait à l’ancre sur un rivage uni et découvert. II en avait confié le commandement à Q. Atrius. (2) César avait fait dans la nuit environ douze mille pas, lorsqu’il aperçut les troupes des ennemis. (3) Ils s’étaient avancés avec la cavalerie et les chars sur le bord d’une rivière et, placés sur une hauteur ; ils commencèrent à nous disputer le passage et engagèrent le combat. (4) Repoussés par la cavalerie, ils se retirèrent dans les bois, où ils trouvèrent un lieu admirablement fortifié par la nature et par l’art, et qui paraissait avoir été disposé jadis pour une guerre civile ; (5) car toutes les avenues en étaient fermées par d’épais abattis d’arbres. (6) C'était de ces bois qu’ils combattaient disséminés, défendant l’approche de leurs retranchements. (7) Mais les soldats de la septième légion, ayant formé la tortue et élevé une terrasse jusqu’au pied du rempart, s’emparèrent de cette position et les chassèrent du bois, presque sans éprouver de pertes. (8) César défendit toutefois de poursuivre trop loin les fuyards, parce qu’il ne connaissait pas le pays et qu’une grande partie du jour étant écoulée, il voulait employer le reste à la fortification du camp.
\subsection[{§ 10.}]{ \textsc{§ 10.} }
\noindent (1) Le lendemain matin, ayant partagé l’infanterie et la cavalerie en trois corps, il les envoya à la poursuite des fuyards. (2) Elles n’avaient fait que très peu de chemin et les derniers rangs étaient encore à la vue du camp, lorsque des cavaliers, envoyés par Q. Atrius à César, vinrent lui annoncer que, la nuit précédente, une violente tempête avait brisé et jeté sur le rivage presque tous les vaisseaux ; que ni ancres ni cordages n’avaient pu résister ; que les efforts des matelots et des pilotes avaient été impuissants, (3) et que le choc des vaisseaux entre eux leur avait causé de grands dommages.
\subsection[{§ 11.}]{ \textsc{§ 11.} }
\noindent (1) À ces nouvelles, César fait rappeler les légions et la cavalerie, et cesser la poursuite : lui-même il revient à sa flotte. (2) Il reconnut de ses yeux une partie des malheurs que les messagers et des lettres lui avaient annoncés ; environ quarante navires étaient perdus ; le reste lui parut cependant pouvoir se réparer à force de travail. (3) Il choisit donc des travailleurs dans les légions et en fit venir d’autres du continent. (4) Il écrivit à Labiénus de faire construire le plus de vaisseaux qu’il pourrait par les légions qu’il avait avec lui ; (5) lui-même, malgré l’extrême difficulté de l’entreprise, arrêta, comme une chose très importante, que tous les vaisseaux fussent amenés sur la grève et enfermés avec le camp dans des retranchements communs. (6) Il employa environ dix jours à ce travail que le soldat n’interrompait même pas la nuit. (7) Quand les vaisseaux furent à sec et le camp bien fortifié, il y laissa pour garnison les mêmes troupes qu’auparavant, et retourna en personne au même lieu d’où il était parti. (8) Il y trouva de nombreuses troupes de Bretons rassemblées de toutes parts ; ils avaient, d’un avis unanime, confié le commandement général et la conduite de la guerre à Cassivellaunos, dont les états étaient séparés des pays maritimes par un fleuve appelé la Tamise, éloigné de la mer d’environ quatre-vingts mille pas. (9) Dans les temps antérieurs, il y avait eu des guerres continuelles avec les autres peuplades ; mais toutes venaient, dans l’effroi que leur causait notre arrivée de lui déférer le commandement suprême.
\subsection[{§ 12.}]{ \textsc{§ 12.} }
\noindent (1) L'intérieur de la Bretagne est habité par des peuples que la tradition représente comme indigènes. (2) La partie maritime est occupée par des peuplades que l’appât du butin et la guerre ont fait sortir de la Belgique ; elles ont presque toutes conservé les noms des pays dont elles étaient originaires, quand, les armes à la main, elles vinrent s’établir dans la Bretagne, et en cultiver le sol. (3) La population est très forte, les maisons y sont très nombreuses et presque semblables à celles des Gaulois ; le bétail y est abondant. (4) On se sert, pour monnaie, ou de cuivre ou d’anneaux de fer d’un poids déterminé. (5) Dans le centre du pays se trouvent des mines d’étain ; sur les côtes, des mines de fer, mais peu productives ; le cuivre qu’on emploie vient du dehors. II y croît des arbres de toute espèce, comme en Gaule, à l’exception du hêtre et du sapin. (6) Les Bretons regardent comme défendu de manger du lièvre, de la poule ou de l’oie ; ils en élèvent cependant par goût et par plaisir. Le climat est plus tempéré que celui de la Gaule, les froids sont moins rigoureux.
\subsection[{§ 13.}]{ \textsc{§ 13.} }
\noindent (1) Cette île est de forme triangulaire ; l’un des côtés regarde la Gaule. Des deux angles de ce côté, l’un est au levant, vers le pays de Cantium, où abordent presque tous les vaisseaux gaulois ; l’autre, plus bas, est au midi. La longueur de ce côté est d’environ cinq cent mille pas. (2) L'autre côté du triangle regarde l’Espagne et le couchant : dans cette direction est l’Hibernie, qui passe pour moitié moins grande que la Bretagne, et en est séparée par une distance égale à celle de la Bretagne à la Gaule : (3) dans l’espace intermédiaire est l’île de Mona. L'on croit qu’il y en a plusieurs autres de moindre grandeur, dont quelques écrivains ont dit qu’elles étaient, vers la saison de l’hiver, privées de la lumière du soleil pendant trente jours continus. (4) Nos recherches ne nous ont rien appris sur ce point : nous observâmes seulement, au moyen de certaines horloges d’eau, que les nuits étaient plus courtes que sur le continent. (5) La longueur de ce côté de l’île est, selon l’opinion de ces écrivains, de sept cent mille pas. (6) Le troisième côté est au nord et n’a en regard aucune terre, si ce n’est la Germanie à l’un de ses angles. Sa longueur est estimée à huit cent mille pas. (7) Ainsi le circuit de toute l’île est de vingt fois cent mille pas.
\subsection[{§ 14.}]{ \textsc{§ 14.} }
\noindent (1) De tous les peuples bretons, les plus civilisés sont, sans contredit, ceux qui habitent le pays de Cantium, région toute maritime et dont les moeurs diffèrent peu de celles des Gaulois. (2) La plupart des peuples de l’intérieur négligent l’agriculture ; ils vivent de lait et de chair et se couvrent de peaux. Tous les Bretons se teignent avec du pastel, ce qui leur donne une couleur azurée et rend leur aspect horrible dans les combats. (3) Ils portent leurs cheveux longs, et se rasent tout le corps, excepté la tête et la lèvre supérieure. (4) Les femmes y sont en commun entre dix ou douze, surtout entre les frères, les pères et les fils. (5) Quand il naît des enfants, ils appartiennent à celui qui le premier a introduit la mère dans la famille.
\subsection[{§ 15.}]{ \textsc{§ 15.} }
\noindent (1) Les cavaliers ennemis avec leurs chariots de guerre attaquèrent vivement dans sa marche notre cavalerie, qui fut partout victorieuse et les repoussa dans les bois et sur les collines ; (2) mais, après avoir tué un grand nombre d’ennemis, son ardeur à en poursuivre les restes lui coûta quelques pertes. (3) Peu de temps après, comme les nôtres ne s’attendaient à rien et travaillaient au retranchement du camp, les Bretons, s’élançant tout à coup de leurs forêts et fondant sur la garde du camp, l’attaquèrent vigoureusement. (4) César envoie pour la soutenir deux cohortes, qui étaient les premières de leurs légions ; comme elles avaient laissé entre elles un très petit espace, l’ennemi, profitant de leur étonnement à la vue de ce nouveau genre de combat, se précipite avec audace dans l’intervalle et échappe sans perte. (5) Q. Labérius Durus, tribun militaire, fut tué dans cette action. Plusieurs autres cohortes envoyées contre les Barbares les repoussèrent.
\subsection[{§ 16.}]{ \textsc{§ 16.} }
\noindent (1) Ce combat, d’un genre si nouveau, livré sous les yeux de toute l’armée et devant le camp, fit comprendre que la pesanteur des armes de nos soldats, en les empêchant de suivre l’ennemi dans sa retraite et en leur faisant craindre de s’éloigner de leurs drapeaux, les rendait moins propres à une guerre de cette nature. (2) La cavalerie combattait aussi avec désavantage, en ce que les Barbares, feignant souvent de se retirer, l’attiraient loin des légions, et, sautant alors de leurs chars, lui livraient à pied un combat inégal ; (3) or, cette sorte d’engagement était pour nos cavaliers aussi dangereuse dans la retraite que dans l’attaque. (4) En outre, les Bretons ne combattaient jamais en masse mais par troupes séparées et à de grands intervalles, et disposaient des corps de réserve, destinés à les recueillir, et à remplacer par des troupes fraîches celles qui étaient fatiguées.
\subsection[{§ 17.}]{ \textsc{§ 17.} }
\noindent (1) Le jour suivant, les ennemis prirent position loin du camp, sur des collines ; ils ne se montrèrent qu’en petit nombre et escarmouchèrent contre notre cavalerie plus mollement que la veille. (2) Mais, vers le milieu du jour, César ayant envoyé au fourrage trois légions et toute la cavalerie sous les ordres du lieutenant Trébonius, ils fondirent subitement et de toutes parts sur les fourrageurs, peu éloignés de leurs drapeaux et de leurs légions. (3) Les nôtres, tombant vigoureusement sur eux, les repoussèrent ; la cavalerie, comptant sur l’appui des légions qu’elle voyait près d’elle, ne mit point de relâche dans sa poursuite, (4) et en fit un grand carnage, sans leur laisser le temps ni de se rallier, ni de s’arrêter, ni de descendre des chars. (5) Après cette déroute, les secours qui leur étaient venus de tous côtés, se retirèrent ; et depuis ils n’essayèrent plus de nous opposer de grandes forces.
\subsection[{§ 18.}]{ \textsc{§ 18.} }
\noindent (1) César, ayant pénétré leur dessein, se dirigea vers la Tamise sur le territoire de Cassivellaunos. Ce fleuve n’est guéable que dans un seul endroit, encore le passage est-il difficile. (2) Arrivé là, il vit l’ennemi en forces, rangé sur l’autre rive. (3) Cette rive était défendue par une palissade de pieux aigus ; d’autres pieux du même genre étaient enfoncés dans le lit du fleuve et cachés sous l’eau. (4) Instruit de ces dispositions par des prisonniers et des transfuges, César envoya en avant la cavalerie, qu’il fit immédiatement suivre par les légions. (5) Les soldats s’y portèrent avec tant d’ardeur et d’impétuosité, quoique leur tête seule fût hors de l’eau, que les ennemis ne pouvant soutenir le choc des légions et de la cavalerie, abandonnèrent le rivage et prirent la fuite.
\subsection[{§ 19.}]{ \textsc{§ 19.} }
\noindent (1) Cassivellaunos, comme nous l’avons dit plus haut, désespérant de nous vaincre en bataille rangée, renvoya la plus grande partie de ses troupes, ne garda guère que quatre mille hommes montés sur des chars, et se borna à observer notre marche, se tenant à quelque distance de notre route, se cachant dans les lieux de difficile accès et dans les bois, faisant retirer dans les forêts le bétail et les habitants des pays par lesquels il savait que nous devions passer. (2) Puis, lorsque nos cavaliers s’aventuraient dans des campagnes éloignées pour fourrager et butiner, il sortait des bois avec ses chariots armés, par tous les chemins et sentiers qui lui étaient bien connus, et mettait en grand péril notre cavalerie, que la crainte de ces attaques empêchait de se répandre au loin. (3) II ne restait à César d’autre parti à prendre que de ne plus permettre à la cavalerie de trop s’éloigner des légions, et que de porter la dévastation et l’incendie aussi loin que pouvaient le permettre la fatigue et la marche des légionnaires.
\subsection[{§ 20.}]{ \textsc{§ 20.} }
\noindent (1) Cependant des députés sont envoyés à César par les Trinovantes, l’un des plus puissants peuples de ce pays, patrie du jeune Mandubracios, qui s’était mis sous la protection de César, et était venu en Gaule se réfugier près de lui, afin d’éviter par la fuite le sort de son père, qui régnait sur ce peuple et que Cassivellaunos avait tué. (2) Ils offrent de se rendre à lui et de lui obéir, (3) le priant de protéger Mandubracios contre les outrages de Cassivellaunos, et de le renvoyer parmi les siens pour qu’il devienne leur chef et leur roi. (4) César exige d’eux quarante otages, des vivres pour l’armée, et leur envoie Mandubracios. Ils s’empressèrent d’exécuter ces ordres et livrèrent avec les vivres le nombre d’otages exigé.
\subsection[{§ 21.}]{ \textsc{§ 21.} }
\noindent (1) Voyant les Trinovantes protégés, et à l’abri de toute violence de la part des soldats, les Cénimagnes, les Ségontiaques, les Ancalites, les Bibroques, les Casses, députèrent à César pour se soumettre à lui. (2) II apprit d’eux que la place où s’était retiré Cassivellaunos était à peu de distance, qu’elle était défendue par des bois et des marais, et renfermait un assez grand nombre d’hommes et de bestiaux. (3) Les Bretons donnent le nom de place forte à un bois épais qu’ils ont entouré d’un rempart et d’un fossé et qui est leur retraite accoutumée contre les incursions de l’ennemi. (4) César y mène les légions : il trouve le lieu parfaitement défendu par la nature et par l’art. Cependant il essaie de l’attaquer sur deux points. (5) Les ennemis, après quelque résistance, ne purent supporter le choc de nos soldats et s’enfuirent par une autre partie de la place. (6) On y trouva beaucoup de bétail, et un grand nombre de Barbares furent pris ou tués dans leur fuite.
\subsection[{§ 22.}]{ \textsc{§ 22.} }
\noindent (1) Tandis qu’en cet endroit les choses se passaient ainsi, Cassivellaunos avait envoyé des messagers dans le Cantium, situé, comme nous l’avons dit, sur les bords de la mer, aux quatre chefs de cette contrée, à Cingétorix, Carvilios, Taximagulos, Ségovax, leur ordonnant de, rassembler toutes leurs troupes, et d’attaquer à l’improviste le camp qui renfermait nos vaisseaux. (2) À peine y furent-ils arrivés, que les nôtres firent une sortie, en tuèrent un grand nombre, prirent en outre un de leurs principaux chefs, Lugotorix, et rentrèrent sans perte dans le camp. (3) À la nouvelle de cette défaite, Cassivellaunos, découragé par tant de revers, voyant son territoire ravagé, et accablé surtout par la défection de plusieurs peuples, fit offrir sa soumission à César par l’entremise de l’Atrébate Commios. (4) César, qui voulait passer l’hiver sur le continent, à cause des révoltes subites de la Gaule, voyant que l’été touchait à sa fin, et sentant que l’affaire pouvait traîner en longueur, exigea des otages et fixa le tribut que la Bretagne paierait chaque année au peuple romain. (5) Il interdit expressément à Cassivellaunos tout acte d’hostilité contre Mandubracios et les Trinovantes.
\subsection[{§ 23.}]{ \textsc{§ 23.} }
\noindent (1) Après avoir reçu les otages, il ramena l’armée sur la côte, trouva les vaisseaux réparés, (2) et les fit mettre à flot. Comme il avait un grand nombre de prisonniers et que plusieurs vaisseaux avaient péri par la tempête, il résolut de faire repasser en deux transports les troupes sur le continent. (3) Une chose remarquable c’est que de tant de navires qui firent plusieurs fois le trajet cette année et la précédente, aucun de ceux qui portaient des soldats ne périt ; (4) mais de ceux qui revenaient à vide de la Gaule, après y avoir déposé les soldats du premier transport, ainsi que des soixante navires construits par les soins de Labiénus, très peu abordèrent à leur destination ; presque tous furent rejetés sur la côte. (5) César, après les avoir vainement attendus quelque temps, craignant que la saison ne l’empêchât de tenir la mer, à cause de l’approche de l’équinoxe, fut contraint d’entasser ses soldats. (6) Par un grand calme, il leva l’ancre au commencement de la seconde veille, prit terre au point du jour, et vit tous les vaisseaux arriver à bon port.
\subsection[{§ 24.}]{ \textsc{§ 24.} }
\noindent (1) Quand il eut fait mettre les navires à sec et tenu à Samarobriva l’assemblée de la Gaule, comme la récolte de cette année avait été peu abondante à cause de la sécheresse, il fut obligé d’établir les quartiers d’hiver de l’armée autrement que les années précédentes, et de distribuer les légions dans diverses contrées. (2) Il en envoya une chez les Morins, sous les ordres du lieutenant C. Fabius ; une autre chez les Nerviens, sous le commandement de Q. Cicéron ; une troisième chez les Esuvii, sous celui de L. Roscius ; une quatrième, avec T. Labiénus, chez les Rèmes, près des frontières des Trévires ; (3) il en plaça trois chez les Belges, et mit à leur tête M. Crassus, son questeur, L. Munatius Plancus et C. Trébonius, ses lieutenants. (4) La légion qu’il avait récemment levée au-delà du Pô et cinq cohortes furent envoyées chez les Éburons, dont le pays, situé en grande partie entre la Meuse et le Rhin, était gouverné par Ambiorix et Catuvolcos. (5) César ordonna que ces soldats soient placés sous les ordres des légats Q. Titurius Sabinus et L. Aurunculéius Cotta. (6) En distribuant les légions de cette manière, il espérait pouvoir remédier facilement à la disette des vivres ; (7) et cependant tous ces quartiers d’hiver (excepté celui que devait occuper L. Roscius, dans la partie la plus paisible et la plus tranquille de la Gaule), étaient contenus dans un espace de cent mille pas. (8) César résolut de rester dans le pays, jusqu’à ce qu’il eût vu les légions établies et leurs quartiers fortifiés.
\subsection[{§ 25.}]{ \textsc{§ 25.} }
\noindent (1) Il y avait chez les Carnutes un homme de haute naissance, Tasgétios, dont les ancêtres avaient régné sur cette nation. (2) César, en considération de sa valeur, de son zèle et des services importants qu’il lui avait rendus dans toutes les guerres, l’avait rétabli dans le rang de ses aïeux. (3) II régnait depuis trois ans, lorsque ses ennemis, ayant pour complices beaucoup d’hommes de sa nation, le massacrèrent publiquement. On instruisit César de cet événement : (4) craignant ; vu le nombre des coupables, que le pays ne fût entraîné par eux à la révolte, il ordonna à L. Plancus de partir de Belgique avec sa légion, de se rendre en toute hâte chez les Carnutes, d’y établir ses quartiers, de saisir et de lui envoyer ceux qu’il saurait avoir pris part au meurtre de Tasgétios. (5) Dans le même temps, il apprit par les rapports des lieutenants et des questeurs auxquels il avait donné le commandement des légions, qu’elles étaient arrivées à leurs quartiers d’hiver, et y étaient retranchées.
\subsection[{§ 26.}]{ \textsc{§ 26.} }
\noindent (1) On était arrivé dans les quartiers depuis environ quinze jours, lorsqu’un commencement de révolte soudaine et de défection éclata, à l’instigation d’Ambiorix et de Catuvolcos. (2) Après être allés, jusqu’aux limites de leur territoire, au-devant de Sabinus et de Cotta, et leur avoir même porté des vivres dans leurs quartiers, séduits ensuite par des envoyés du Trévire Indutiomaros, ils soulevèrent le pays, tombèrent tout d’un coup sur ceux de nos soldats qui faisaient du bois, et vinrent en grand nombre attaquer le camp. (3) Les nôtres prennent aussitôt les armes et montent sur le rempart ; la cavalerie espagnole est envoyée sur un autre point : nous obtenons l’avantage dans ce combat ; et les ennemis, désespérant du succès, s’éloignent, abandonnant l’attaque. (4) Alors, ils demandent, en poussant de grands cris, selon leur coutume, que quelques-uns des nôtres viennent en pourparlers, voulant les entretenir d’objets d’un intérêt commun qui, selon qu’ils l’espéraient, pourraient terminer les différends.
\subsection[{§ 27.}]{ \textsc{§ 27.} }
\noindent (1) On envoie pour les entendre C. Arpinéius, chevalier romain, ami de Q. Titurius, et un espagnol nommé Q. Junius, qui avait déjà rempli près d’Ambiorix plusieurs missions de la part de César. Ambiorix leur parle ainsi : (2) "Il sait qu’il doit beaucoup à César pour les bienfaits qu’il en a reçus ; c’est par son intervention qu’il a été délivré du tribut qu’il payait jusqu’alors aux Atuatuques, ses voisins ; il lui doit également la liberté de son fils et du fils de son frère lesquels, envoyés comme otages aux Atuatuques, avaient été retenus dans la captivité et dans les fers. (3) Aussi, n’est-ce ni de son avis, ni par sa volonté qu’on est venu assiéger le camp des Romains : la multitude l’y a contraint ; telle est en effet la nature de son autorité que la multitude n’a pas moins de pouvoir sur lui que lui sur elle. (4) Au reste, son pays ne s’est porté à cette guerre que dans l’impuissance de résister au torrent de la conjuration gauloise : sa faiblesse le prouve suffisamment, car il n’est pas si dénué d’expérience qu’il se croie capable de vaincre le peuple romain avec ses seules forces ; (5) mais il s’agit d’un projet commun à toute la Gaule. Ce même jour est fixé pour attaquer à la fois tous les quartiers de César, afin qu’une légion ne puisse venir au secours d’une autre légion ; (6) il était bien difficile à des Gaulois de refuser leur concours à d’autres Gaulois, surtout dans une entreprise où il s’agissait de recouvrer la liberté commune. (7) Après avoir satisfait à ses devoirs envers sa patrie, il a maintenant à remplir envers César ceux de la reconnaissance. II avertit, il supplie Titurius, au nom de l’hospitalité, de pourvoir à son salut et à celui de ses soldats. (8) De nombreuses troupes de Germains ont passé le Rhin ; elles arriveront dans deux jours. (9) C'est à nous, Romains, de décider si nous ne voulons pas, avant que les peuples voisins puissent en être informés, retirer les soldats de leurs quartiers, pour les joindre à ceux ou de Cicéron ou de Labiénus dont l’un est à la distance d’environ cinquante mille pas, et l’autre un peu plus éloigné. (10) II promet, il fait le serment de nous livrer un libre passage sur ses terres ; (11) en le faisant, il croit tout à la fois servir son pays, que notre départ soulagera, et reconnaître les bienfaits de César." Après ce discours, Ambiorix se retira.
\subsection[{§ 28.}]{ \textsc{§ 28.} }
\noindent (1) Arpinéius et Junius rapportent ces paroles aux deux lieutenants. Tout à coup troublés par cette révélation, ils ne crurent pas, quoique l’avis leur vînt d’un ennemi, devoir le négliger. Ce qui les frappa le plus, c’est qu’il était à peine croyable que la nation faible et obscure des Éburons eût osé d’elle-même faire la guerre au peuple romain. (2) Ils soumettent donc l’affaire à un conseil ; elle y excita de vifs débats. (3) L. Aurunculéius, ainsi que plusieurs tribuns militaires et centurions des premiers rangs, furent d’avis "de ne rien faire imprudemment, et de ne point quitter les quartiers (4) sans l’ordre de César. Ils observèrent que, quelque nombreuses que fussent les troupes des Germains, ils pouvaient leur résister dans leurs quartiers retranchés ; le combat de la veille le prouvait assez, puisqu’ils avaient vigoureusement. soutenu le choc des Barbares et leur avaient blessé beaucoup d’hommes. (5) On ne manquait pas de vivres ; pendant la défense, il viendrait du secours et des quartiers les plus proches et de César : (6) enfin était-il rien de plus imprudent ou de plus honteux que de prendre, pour les plus grands intérêts, conseil de l’ennemi  ? "
\subsection[{§ 29.}]{ \textsc{§ 29.} }
\noindent (1) Titurius s’éleva contre cette opinion et répondit avec force "qu’il serait trop tard pour agir, lorsque les troupes ennemies se seraient accrues de l’adjonction des Germains, ou que les quartiers voisins auraient reçu quelque échec : il ne reste qu’un moment, qu’une occasion pour sauver l’armée : (2) César est vraisemblablement parti pour l’Italie : autrement les Carnutes n’auraient pas osé tuer Tasgétios, ni les Éburons, s’il était dans la Gaule, attaquer notre camp avec tant de mépris. (3) Il considère l’avis en lui-même et non l’ennemi qui le donne ; le Rhin est proche, les Germains ont un vif ressentiment de la mort d’Arioviste et de nos précédentes victoires : (4) la Gaule est en feu : elle supporte impatiemment tous les outrages qu’elle a subis sous la domination du peuple romain, et la perte de son ancienne gloire militaire. (5) Enfin, qui pourrait se persuader qu’Ambiorix en vienne, sans être bien instruit, à donner un tel avis  ? (6) Son opinion est, de toute manière, la plus sûre : s’il n’y a rien à craindre, ils joindront sans danger la plus proche légion ; si toute la Gaule s’est unie aux Germains, il n’y a de salut que dans la célérité. (7) Quel serait le résultat de l’avis de Cotta, et des autres opposants ? En le suivant, si le péril n’est pas instantané, on a certainement, après un long siège, la famine à redouter".
\subsection[{§ 30.}]{ \textsc{§ 30.} }
\noindent (1) Après cette dispute de part et d’autre, comme Cotta et les principaux centurions soutenaient vivement leur avis : "Eh bien ! qu’il soit fait comme vous le voulez, " dit Sabinus ; et élevant la voix, pour être entendu d’une grande partie des soldats. (2) "Je ne suis pas, reprit-il, celui de vous que le danger de la mort effraie le plus ; s’il arrive quelque malheur, on saura vous en demander compte ; (3) tandis que, si vous le vouliez, réunis dans deux jours aux quartiers voisins, nous soutiendrions avec eux les chances communes de la guerre, et nous ne serions pas, loin de nos compagnons et tout à fait isolés, exposés à périr par le fer ou par la faim".
\subsection[{§ 31.}]{ \textsc{§ 31.} }
\noindent (1) On se lève pour sortir du conseil ; les soldats entourent les deux lieutenants et les conjurent "de ne pas tout compromettre par leur division et leur opiniâtreté : (2) tout est facile, soit qu’on demeure, soit qu’on parte, si tous partagent et approuvent le même avis ; mais la dissension ne peut laisser aucun espoir." (3) Le débat se prolonge jusqu’au milieu de la nuit. Enfin, Cotta ébranlé se rend ; l’avis de Sabinus l’emporte. (4) On annonce le départ pour le point du jour : le reste de la nuit se passe à veiller ; chaque soldat visite son bagage, pour savoir ce qu’il emportera ou ce qu’il sera contraint de laisser de ses équipements d’hiver. (5) Il semble qu’on ne néglige rien pour ne pouvoir rester sans danger et pour ajouter au péril de la retraite celui de l’opérer avec des soldats affaiblis par la fatigue et la veille. (6) C'est dans cet état qu’on sort du camp à la pointe du jour, avec la persuasion que l’avis donné par Ambiorix l’était non par un ennemi, mais par l’ami le plus sûr ; on se met en marche sur une longue file, avec un nombreux bagage.
\subsection[{§ 32.}]{ \textsc{§ 32.} }
\noindent (1) Mais les ennemis, avertis du départ de nos soldats par le bruit et le mouvement de la nuit, avaient, en se partageant en deux corps, formé une embuscade dans leurs forêts, dans un lieu caché et favorable à leur dessein, à deux mille pas environ ; et ils attendaient l’arrivée des Romains. (2) Quand la plus grande partie de nos troupes se fut engagée dans une vallée profonde, ils se montrèrent tout à coup à l’une et à l’autre de ses issues, attaquèrent l’arrière-garde, arrêtèrent l’avant-garde, et engagèrent le combat dans la position la plus désavantageuse pour nous.
\subsection[{§ 33.}]{ \textsc{§ 33.} }
\noindent (1) Alors Titurius, en homme qui n’avait pourvu à rien, se trouble, court ça et là, et dispose les cohortes ; mais ses mesures sont timides, et tout semble lui manquer à la fois, comme il arrive d’ordinaire à ceux qui, dans le moment même de l’action, sont forcés de prendre un parti. (2) Mais Cotta qui avait pensé au danger de se mettre en marche, et qui, pour cette raison, s’était opposé au départ, n’omettait rien de ce qu’exigeait le salut commun, remplissant à la fois le devoir de général en dirigeant et exhortant les troupes, et celui de soldat en combattant. (3) Comme, à raison de l’étendue de l’armée, les lieutenants pouvaient moins facilement tout faire par eux-mêmes, et pourvoir aux besoins de chaque point, ils firent publier l’ordre d’abandonner les bagages et de se tenir en cercle ; (4) résolution qui, bien qu’elle ne fût pas répréhensible dans cette conjoncture, eut cependant un effet fâcheux ; (5) car elle diminua la confiance de nos soldats et donna au contraire plus d’ardeur aux ennemis, qui prirent cette disposition pour l’indice d’une grande terreur et du désespoir. (6) Il en résulta en outre un inconvénient inévitable ; c’est que partout les soldats s’éloignaient des enseignes pour courir aux bagages, afin d’y prendre et d’en retirer chacun ce qu’il avait de plus cher ; on n’entendait que cris et gémissements.
\subsection[{§ 34.}]{ \textsc{§ 34.} }
\noindent (1) Les Barbares au contraire ne manquèrent pas de prudence : car leurs chefs firent publier dans toute l’armée, "qu’aucun ne quittât son rang ; que tout ce que les Romains auraient abandonné serait la proie réservée au vainqueur ; que tout dépendait donc de la victoire." (2) Les nôtres ne leur étaient inférieurs ni en valeur ni en nombre ; quoique abandonnés de leur chef et de la Fortune, ils plaçaient encore dans leur courage tout l’espoir de leur salut ; et chaque fois qu’une cohorte se portait sur un point elle y faisait un grand carnage. (3) Ambiorix s’en aperçut et fit donner à tous les siens l’ordre de lancer leurs traits de loin, de ne point trop s’approcher et de céder sur les points où les Romains se porteraient vivement : (4) la légèreté de leur armure et l’habitude de ce genre de combat les préserveraient de tout péril : ils ne devaient attaquer l’ennemi que lorsqu’il reviendrait aux drapeaux.
\subsection[{§ 35.}]{ \textsc{§ 35.} }
\noindent (1) Cet ordre fut très fidèlement exécuté ; lorsqu’une cohorte sortait du cercle pour charger l’ennemi, il s’enfuyait avec une extrême vitesse. (2) Cette charge laissait nécessairement notre flanc à découvert, et c’était là que se dirigeaient aussitôt les traits. (3) Puis, quand la cohorte revenait vers le point d’où elle était partie, elle était enveloppée à la fois par ceux qui avaient cédé et par ceux qui s’étaient postés près de nos flancs. (4) Voulait-elle tenir ferme, sa valeur devenait inutile, et ne pouvait la garantir, serrée comme elle l’était, des traits lancés par une si grande multitude. (5) Toutefois, malgré tant de désavantages et tout couverts de blessures, nos soldats résistaient encore ; une grande partie du jour était écoulée, et ils avaient combattu depuis le point du jour jusqu’à la huitième heure, sans avoir rien fait qui fût indigne d’eux. (6) Alors T. Balventio, qui, l’année précédente, avait commandé comme primipile, homme brave et considéré, a les deux cuisses traversées par un javelot. (7) Q. Lucanius, du même grade, est tué en combattant vaillamment pour secourir son fils qui était enveloppé : (8) le lieutenant L. Cotta, allant de rang en rang pour animer les cohortes, est blessé d’un coup de fronde au visage.
\subsection[{§ 36.}]{ \textsc{§ 36.} }
\noindent (1) Effrayé de ce désastre, Q. Titurius, ayant de loin aperçu Ambiorix qui animait ses troupes, lui envoie son interprète Cn. Pompée pour le prier de l’épargner lui et ses soldats. (2) À ce message, Ambiorix répond : "Que si Titurius veut conférer avec lui, il le peut ; qu’il espère obtenir de l’armée gauloise la vie des Romains ; qu’il ne serait fait aucun mal à sa personne et qu’il engage sa foi en garantie." (3) Titurius communique cette réponse à Cotta blessé, et lui propose, s’il y voit de l’avantage, de sortir de la mêlée, et d’aller conférer ensemble avec Ambiorix : il espère en obtenir le salut de l’armée et le leur. (4) Cotta proteste qu’il ne se rendra point auprès d’un ennemi armé, et persiste dans ce refus.
\subsection[{§ 37.}]{ \textsc{§ 37.} }
\noindent (1) Sabinus ordonne aux tribuns des soldats et aux centurions des premiers rangs qu’il avait alors autour de lui, de le suivre. Arrivé près d’Ambiorix, il en reçoit l’ordre de mettre bas les armes ; il obéit, et ordonne aux siens de déposer les leurs. (2) Pendant qu’ils discutent les conditions dans un entretien qu’Ambiorix prolonge à dessein, Sabinus est peu à peu enveloppé, et mis à mort. (3) Alors les Barbares, poussant leurs cris de victoire, se précipitent sur nos troupes et les mettent en désordre. (4) Là fut tué les armes à la main L. Cotta, avec la plus grande partie des soldats romains. Le reste se retira dans le camp d’où l’on était sorti. (5) Un d’entre eux, L. Pétrosidius, porte-aigle, pressé par une multitude d’ennemis, jeta l’aigle dans les retranchements et périt devant le camp, en combattant avec le plus grand courage. (6) Les autres y soutinrent avec peine un siège jusqu’à la nuit, et, cette nuit même, dans leur désespoir, ils se tuèrent tous jusqu’au dernier. (7) Quelques-uns, échappés du combat, gagnèrent, par des chemins détournés à travers les forêts, les quartiers du lieutenant T. Labiénus, et l’instruisirent de ce désastre.
\subsection[{§ 38.}]{ \textsc{§ 38.} }
\noindent (1) Enflé de cette victoire, Ambiorix se rend aussitôt avec sa cavalerie chez les Atuatuques, peuple voisin de ses états, et marche jour et nuit, après avoir ordonné à son infanterie de le suivre. (2) Il leur annonce sa victoire, les excite à se soulever, passe le lendemain chez les Nerviens et les exhorte à ne pas perdre l’occasion de s’affranchir à jamais et de se venger sur les Romains des injures qu’ils en ont reçues ; (3) il leur représente que deux lieutenants ont été tués et qu’une grande partie de l’armée romaine a péri ; (4) qu’il ne sera pas difficile de détruire, en l’attaquant subitement, la légion qui hiverne chez eux avec Cicéron ; il leur offre son aide pour cette entreprise". Les Nerviens sont aisément persuadés par ce discours.
\subsection[{§ 39.}]{ \textsc{§ 39.} }
\noindent (1) Ayant donc sur-le-champ envoyé des courriers aux Ceutrons, aux Grudii, aux Lévaques, aux Pleumoxii, aux Geidumnes, peuples qui sont tous dans leur dépendance, ils rassemblent le plus de troupes qu’ils peuvent ; et volent à l’improviste aux quartiers de Cicéron, avant que le bruit de la mort de Titurius soit parvenu jusqu’à lui. (2) Il arriva, ce qui était inévitable, que quelques soldats occupés à faire du bois pour les fascines, et répandus dans les forêts, furent séparés de leur corps par la soudaine irruption des cavaliers ennemis (3) et enveloppés de toutes parts. Un nombre considérable d’Éburons, de Nerviens, d’Atuatuques ainsi que leurs alliés et auxiliaires, viennent ensuite attaquer la légion. Nos soldats courent sur-le-champ aux armes et bordent le retranchement. (4) Ils eurent ce jour-là beaucoup de peine à résister à des ennemis qui avaient mis tout leur espoir dans la promptitude de leur attaque, et qui se flattaient, en remportant cette victoire, d’être désormais invincibles.
\subsection[{§ 40.}]{ \textsc{§ 40.} }
\noindent (1) Cicéron écrit aussitôt à César, et promet de grandes récompenses à ceux qui lui porteront ses lettres. Tous les chemins étant gardés, les courriers ne peuvent passer. (2) La nuit, on élève jusqu’à cent vingt tours avec le bois destiné à retrancher le camp, ce qui se fait avec une célérité incroyable, et on achève les retranchements. (3) Le lendemain, les ennemis, en bien plus grand nombre, viennent attaquer le camp et comblent le fossé. (4) La résistance de notre côté est aussi vive que la veille. Les jours suivants se passent de même. (5) Le travail se continue sans relâche pendant la nuit : les malades, les blessés ne peuvent prendre aucun repos : (6) on prépare chaque nuit tout ce qui est nécessaire pour la défense du lendemain : on façonne quantité de pieux, et de traits de remparts ; de nouveaux étages sont ajoutés aux tours ; des claies sont tressées, des mantelets construits. (7) Cicéron lui-même, quoique d’une très faible santé, ne se donnait aucun repos, même pendant la nuit, au point que les soldats, par d’unanimes instances, le forçaient à se ménager.
\subsection[{§ 41.}]{ \textsc{§ 41.} }
\noindent Alors les chefs des Nerviens et les principaux de cette nation, qui avaient quelque accès auprès de Cicéron et des rapports d’amitié avec lui, lui font savoir qu’ils désirent l’entretenir. (2) Ils répètent, dans cette entrevue, ce qu’Ambiorix avait dit à Titurius : "Que toute la Gaule était en armes, (3) que les Germains avaient passé le Rhin, que les quartiers de César et de ses lieutenants étaient attaqués". (4) Ils annoncent en outre la mort de Sabinus. Ils montrent Ambiorix pour faire foi de leurs paroles : (5) "Ce serait, disent-ils, une illusion, que d’attendre le moindre secours de légions qui désespèrent de leurs propres affaires. Ils n’ont, au reste, aucune intention hostile à l’égard de Cicéron et du peuple romain, et ne leur demandent que de quitter leurs quartiers d’hiver et de ne pas se faire une habitude de ces campements. (6) Ils peuvent en toute sûreté sortir de leurs quartiers et se retirer sans crainte par tous les chemins qu’ils voudront." (7) Cicéron ne leur répondit qu’un mot : "Le peuple romain n’est point dans l’usage d’accepter aucune condition d’un ennemi armé ; (8) s’ils veulent mettre bas les armes, ils enverront par son entremise des députés à César ; il espère qu’ils obtiendront de sa justice ce qu’ils lui demanderont."
\subsection[{§ 42.}]{ \textsc{§ 42.} }
\noindent (1) Déchus de cet espoir, les Nerviens entourent les quartiers d’un rempart de onze pieds et d’un fossé de quinze. (2) Ils avaient appris de nous cet art dans les campagnes précédentes et se le faisaient enseigner par quelques prisonniers faits sur notre armée ; (3) mais, faute des instruments de fer propres à cet usage, ils étaient réduits à couper le gazon avec leurs épées et à porter la terre dans leurs mains ou dans leurs saies. (4) Du reste, on put juger, par cet ouvrage, de leur nombre prodigieux : car, en moins de trois heures, ils achevèrent un retranchement de quinze mille pas de circuit. (5) Les jours suivants, ils se mirent à élever des tours à la hauteur de notre rempart, à préparer et à faire des faux et des tortues, d’après les instructions des mêmes prisonniers.
\subsection[{§ 43.}]{ \textsc{§ 43.} }
\noindent (1) Le septième jour du siège, un très grand vent s’étant élevé, ils lancèrent avec la fronde des boulets d’argile rougis au feu et des dards enflammés sur les buttes des soldats, couvertes en paille, à la manière gauloise. (2) Elles eurent bientôt pris feu, et la violence du vent porta la flamme sur tout le camp. (3) Les ennemis, poussant alors de grands cris, comme s’ils eussent déjà obtenu et remporté la victoire firent avancer leurs tours et leurs tortues, et montèrent à l’escalade. (4) Mais tels furent le courage et la présence d’esprit des soldats que, de toutes parts brûlés par la flamme, en butte à une multitude innombrable de traits, sachant bien que tous leurs bagages et toute leur fortune étaient la proie de l’incendie, non seulement aucun d’eux ne quitta le rempart pour se sauver, mais en quelque sorte ne tourna même la tête ; ils ne songeaient tous en ce moment qu’à se battre avec la plus grande intrépidité. (5) Ce fut pour nous une bien rude journée ; mais elle eut cependant pour résultat qu’un très grand nombre d’ennemis y furent blessés et tués ; entassés au pied du rempart, les derniers fermaient la retraite aux autres. (6) L'incendie s’étant un peu apaisé, et les barbares ayant roulé et établi une tour près du rempart, les centurions de la troisième cohorte, postés en cet endroit, s’en éloignèrent, emmenèrent toute leur troupe, et, appelant les ennemis, les invitèrent du geste et de la voix, à entrer s’ils voulaient ; aucun d’eux n’osa s’avancer. (7) Alors des pierres lancées de toutes parts mirent le désordre parmi eux, et l’on brûla leur tour.
\subsection[{§ 44.}]{ \textsc{§ 44.} }
\noindent (1) Il y avait dans cette légion deux centurions, hommes du plus grand courage et qui approchaient déjà des premiers grades, T. Pullo et L. Vorénus. (2) Il existait entre eux une continuelle rivalité, et chaque année ils se disputaient le rang avec une ardeur qui dégénérait en haine. (3) Comme on se battait opiniâtrement près des remparts : "Qu'attends-tu, Vorénus ?, " dit Pullo. "Quelle plus belle occasion de prouver ton courage ? Voici, voici le jour qui devra décider entre nous." (4) À ces mots, il sort des retranchements et se précipite vers le plus épais de la mêlée. (5) Vorénus ne peut alors se contenir, et, craignant l’opinion générale, il le suit de près. (6) Arrivé près de l’ennemi, Pullo lance son javelot et perce un de ceux qui s’avançaient en foule sur lui ; il le blesse à mort : aussitôt ils couvrent le cadavre de leurs boucliers, dirigent tous leurs traits contre Pullo, et lui coupent la retraite. (7) Son bouclier est traversé par un dard, qui s’enfonce jusque dans le baudrier. (8) Le même coup détourne le fourreau et arrête sa main droite qui cherche à tirer l’épée : (9) ainsi embarrassé, les ennemis l’enveloppent. (10) Vorénus, son rival, accourt le défendre contre ce danger. Les Barbares se tournent aussitôt contre lui, laissant Pullo qu’ils croient hors de combat. (11) Vorénus, l’épée à la main, se défend au milieu d’eux, en tue un, et commence à faire reculer les autres. (12) Mais emporté par son ardeur, il rencontre un creux et tombe. (13) Pullo vient à son tour pour le dégager ; et tous deux, sans blessure, après avoir tué plusieurs ennemis, rentrent au camp couverts de gloire. (14) Ainsi, dans ce combat où ils luttèrent, la fortune balança leur succès, chacun d’eux défendit et sauva son rival, et l’on ne put décider qui l’avait emporté en courage.
\subsection[{§ 45.}]{ \textsc{§ 45.} }
\noindent (1) Plus le siège devenait rude et difficile à soutenir, surtout avec le peu de défenseurs auxquels nous réduisait chaque jour le grand nombre des blessés, plus Cicéron dépêchait vers César de courriers porteurs de ses lettres ; la plupart étaient arrêtés et cruellement mis à mort à la vue de nos soldats. (2) Dans le camp était un Nervien, nommé Vertico, d’une naissance distinguée, qui, dès le commencement du siège, s’était rendu près de Cicéron et lui avait engagé sa foi. (3) II détermine un de ses esclaves, par l’espoir de la liberté et par de grandes récompenses, à porter une lettre à César. (4) L'esclave la porte, attachée à son javelot, et, Gaulois lui-même, il se mêle aux Gaulois sans inspirer de défiance, et arrive auprès de César, (5) qu’il instruit des dangers de Cicéron et de la légion.
\subsection[{§ 46.}]{ \textsc{§ 46.} }
\noindent (1) César, ayant reçu cette lettre vers la onzième heure du jour, envoie aussitôt un courrier au questeur M. Crassus, dont les quartiers étaient chez les Bellovaques, à vingt-cinq mille pas de distance. (2) Il lui ordonne de partir au milieu de la nuit avec sa légion et de venir le joindre en toute hâte. (3) Crassus part avec le courrier. Un autre est envoyé au lieutenant C. Fabius, pour qu’il conduise sa légion sur les terres des Atrébates, qu’il savait avoir à traverser lui-même. (4) Il écrit à Labiénus de se rendre, s’il le peut sans compromettre les intérêts de la république, dans le pays des Nerviens avec sa légion. Il ne croit pas devoir attendre le reste de l’armée qui était un peu plus éloignée, et rassemble environ quatre cents cavaliers des quartiers voisins.
\subsection[{§ 47.}]{ \textsc{§ 47.} }
\noindent (1) Vers la troisième heure, César fut averti par ses coureurs de l’arrivée de Crassus, et le même jour il avança de vingt mille pas. (2) Il laissa Crassus à Samarobriva, et lui donna une légion pour garder les bagages de l’armée, les otages des cités, les registres et tout le grain qu’on avait rassemblé dans cette ville pour le service de l’hiver. (3) Fabius, selon l’ordre qu’il avait reçu, ne tarda pas à partir avec sa légion, et joignit l’armée sur la route. (4) Labiénus, instruit de la mort de Sabinus et du massacre des cohortes, était entouré de toutes les forces des Trévires ; craignant, s’il effectuait un départ qui ressemblerait à une fuite, de ne pouvoir résister à l’impétuosité d’ennemis qu’une récente victoire devait rendre plus audacieux, (5) il exposa, dans sa réponse à César, le danger de tirer la légion de ses quartiers ; il lui détailla ce qui s’était passé chez les Éburons, et lui apprit que toutes les forces des Trévires, cavalerie et infanterie, étaient réunies à trois mille pas de son camp.
\subsection[{§ 48.}]{ \textsc{§ 48.} }
\noindent (1) César approuva le parti qu’il prenait ; au lieu de trois légions sur lesquelles il comptait, il fut réduit à deux, mais il savait que le salut commun ne dépendait que de sa diligence. (2) Il se rend à marches forcées sur les terres des Nerviens. Là, il apprend des prisonniers ce qui se passe au camp de Cicéron, et son extrême danger. (3) Alors il décide, à force de récompenses, un cavalier gaulois à lui porter une lettre : (4) elle était écrite en caractères grecs, afin que les ennemis, s’ils l’interceptaient, ne pussent connaître nos projets. (5) Dans le cas où il ne pourrait parvenir jusqu’à Cicéron, il lui recommande d’attacher la lettre à la courroie de son javelot et de la lancer dans les retranchements du camp. (6) César écrivait que, parti avec les légions, il allait bientôt arriver, et exhortait Cicéron à conserver tout son courage. (7) Dans la crainte du péril, et selon ses instructions, le Gaulois lance son javelot ; (8) il se fiche par hasard dans une tour, y reste deux jours sans être aperçu, et n’est découvert que le troisième par un soldat, qui prend la lettre et la porte à Cicéron. (9) La lecture qui en est faite en présence des soldats excite parmi eux d’unanimes transports de joie. (10) Déjà on voyait la fumée des incendies, et il ne put rester aucun doute sur l’approche des légions.
\subsection[{§ 49.}]{ \textsc{§ 49.} }
\noindent (1) Les Gaulois, avertis par leurs coureurs, lèvent le siège et marchent contre César avec toutes leurs troupes ; (2) elles se composaient d’environ soixante mille hommes. Cicéron, ainsi dégagé, demande à son tour à ce même Vertico, dont nous avons parlé plus haut, un Gaulois, pour porter une lettre à César, et recommande au porteur la prudence et la célérité. (3) Il annonçait par cette lettre que les ennemis l’avaient quitté, et tournaient toutes leurs forces contre César. (4) Celui-ci, l’ayant reçue vers le milieu de la nuit, en fait part aux siens et les anime pour le combat. (5) Le lendemain, au point du jour, il lève son camp, et il a fait à peine quatre mille pas, qu’il aperçoit une multitude d’ennemis au-delà d’une grande vallée traversée par un ruisseau. (6) Il eût été très dangereux de combattre des troupes si nombreuses dans un lieu défavorable. D'ailleurs il voyait Cicéron délivré des Barbares qui l’assiégeaient, il pouvait ralentir sa marche, (7) et il s’arrêta dans le poste le plus avantageux possible, pour s’y retrancher dans son camp. Quoique ce camp eût peu d’étendue par lui-même, puisqu’il contenait à peine sept mille hommes sans aucuns bagages, il le resserre encore dans le moindre espace possible, afin d’inspirer aux ennemis le plus grand mépris. (8) En même temps, il envoie partout ses éclaireurs reconnaître l’endroit le plus commode pour traverser le vallon.
\subsection[{§ 50.}]{ \textsc{§ 50.} }
\noindent (1) Dans cette journée, qui se passa en escarmouches de cavalerie près du ruisseau, chacun resta dans ses positions : (2) les Gaulois, parce qu’ils attendaient l’arrivée de troupes plus nombreuses ; (3) César, parce qu’en feignant de craindre, il espérait attirer l’ennemi près de ses retranchements et le combattre en deçà du vallon, à la tête de son camp ; dans le cas contraire, il voulait reconnaître assez les chemins pour traverser avec moins de péril le vallon et le ruisseau. (4) Dès le point du jour, la cavalerie ennemie s’approcha du camp et engagea le combat avec la nôtre. (5) Aussitôt César ordonne à ses cavaliers de céder et de rester dans le camp ; il ordonne en même temps de donner partout plus de hauteur aux retranchements, de boucher les portes, et, en exécutant ces travaux, de courir çà et là dans la plus grande confusion, avec tous les signes de l’effroi.
\subsection[{§ 51.}]{ \textsc{§ 51.} }
\noindent (1) Attirées par cette feinte, les troupes ennemies passent le ravin, et se rangent en bataille dans un lieu désavantageux. (2) Voyant même que les nôtres laissaient le rempart dégarni, les Barbares s’en approchent de plus près, y lancent des javelots de toutes parts, (3) et font publier autour de nos retranchements, par la voix des hérauts, que tout Gaulois ou Romain qui voudra passer de leur côté avant la troisième heure, peut le faire sans danger ; qu’après ce temps, il ne le pourra plus. (4) Enfin ils conçurent pour nous un tel mépris que, croyant trouver trop de difficulté à forcer nos portes fermées, pour la forme, par une simple couche de gazon, ils se mirent, les uns à détruire le rempart à l’aide seulement de leurs mains, les autres à combler le fossé. (5) Alors César, faisant une sortie par toutes les portes à la fois, suivi de la cavalerie, met bientôt les ennemis en fuite, sans que personne ose s’arrêter pour combattre. On en tua un grand nombre, et tous jetèrent leurs armes.
\subsection[{§ 52.}]{ \textsc{§ 52.} }
\noindent (1) Craignant de s’engager trop avant dans une poursuite que les bois et les marais rendaient difficile ; et comme, d’ailleurs, c’était pour les ennemis un assez rude échec que d’avoir été chassés de leurs positions, César, sans avoir perdu un seul homme, joignit Cicéron le même jour. (2) Il ne vit pas sans étonnement les tours, les tortues et les retranchements qu’avaient élevés les Barbares. Ayant passé en revue la légion, à peine un dixième des soldats se trouva sans blessure, (3) témoignage certain du courage qu’ils avaient déployé au milieu du péril. Il donna à Cicéron et à la légion les éloges qui leur étaient dus, (4) distinguant par leur nom les centurions et les tribuns des soldats dont l’intrépidité lui avait été signalée par leur chef. Il apprit des prisonniers les détails de la défaite de Sabinus et de Cotta. (5) Le lendemain, il convoque une assemblée, rappelle ce qui s’est passé, console et encourage les soldats, (6) attribue à l’imprudence et à la témérité du lieutenant l’échec qu’il avait reçu, les exhorte à le supporter avec d’autant plus de résignation que, grâce à leur courage et à la protection des dieux immortels, ce revers avait déjà été réparé, et n’avait pas laissé longtemps leur joie aux ennemis, ni aux Romains leur douleur.
\subsection[{§ 53.}]{ \textsc{§ 53.} }
\noindent (1) Cependant le bruit de la victoire de César fut porté à Labiénus, chez les Rèmes, avec une si incroyable vitesse que, bien qu’éloigné de soixante mille pas des quartiers de Cicéron, où César n’était arrivé qu’après la neuvième heure du jour, des acclamations s’élevèrent aux portes du camp avant minuit, et que déjà, par leurs cris de joie, les Rèmes félicitaient Labiénus de cette victoire. (2) Cette nouvelle, parvenue aux Trévires, détermina Indutiomaros, qui comptait attaquer le lendemain le camp de Labiénus, à s’enfuir pendant la nuit et à ramener toutes ses troupes dans son pays. (3) César renvoya Fabius dans ses quartiers avec sa légion, et résolut d’hiverner lui-même aux environs de Samarobriva avec trois légions dont il forma trois quartiers. Les grands mouvements qui avaient eu lieu dans la Gaule le déterminèrent à rester tout l’hiver près de l’armée. (4) En effet, sur le bruit de la mort funeste de Sabinus, presque tous les peuples de la Gaule se disposaient à prendre les armes, envoyaient partout des messagers et des députations, examinaient le parti qui leur restait à prendre, sur quel point commenceraient les hostilités, et tenaient des assemblé es nocturnes dans les lieux écartés. (5) Il ne se passa presque pas un seul jour de cet hiver que César n’eût des motifs d’inquiétude et ne reçût quelques avis des réunions et des mouvements des Gaulois. (6) Ainsi, le lieutenant L. Roscius, qui commandait la treizième légion, lui fit savoir qu’un grand nombre de troupes gauloises des nations que l’on appelle Armoricaines, (7) s’étaient réunies pour l’attaquer, et n’étaient plus qu’à huit mille pas de ses quartiers ; lorsqu’à la nouvelle de la victoire de César, elles s’étaient retirées si précipitamment que leur départ ressembla à une fuite.
\subsection[{§ 54.}]{ \textsc{§ 54.} }
\noindent (1) César, après avoir fait venir près de lui les principaux de chaque cité, effraya les uns en leur déclarant qu’il était instruit de leurs desseins, fit aux autres des exhortations, et contint dans le devoir une grande partie de la Gaule. (2) Cependant les Sénons, nation très puissante et jouissant d’un grand crédit parmi les Gaulois, avaient, en plein conseil, résolu la mort de Cavarinos que César leur avait donné pour roi : il descendait des anciens chefs du pays, et Moritasgos, son frère, y régnait à l’arrivée de César en Gaule. Cavarinos qui, dans le pressentiment de son sort, s’était enfui, avait été poursuivi jusque sur leurs frontières et chassé du trône et de la cité. (3) Ils avaient député vers César pour justifier leur conduite, et en avaient reçu l’ordre de lui envoyer tous leurs sénateurs, ordre auquel ils n’obéirent pas. (4) Les Barbares étaient si fiers d’avoir trouvé parmi eux un peuple qui eût osé le premier faire la guerre aux Romains, et cela avait produit un tel changement dans l’opinion générale, qu’à l’exception des Héduens et des Rèmes, que César considéra toujours singulièrement, les uns pour leur ancienne et constante fidélité au peuple romain, les autres pour leurs services récents dans cette guerre, il n’y eut presque pas une cité qui ne nous fût suspecte. (5) Et je ne sais, sans parler des autres motifs, s’il faut s’étonner qu’une nation qui l’emportait sur toutes les autres par le mérite militaire, et qui se voyait déchue de sa haute renommée, eût pu sans une vive douleur supporter le joug du peuple romain.
\subsection[{§ 55.}]{ \textsc{§ 55.} }
\noindent (1) Les Trévires et Indutiomaros ne cessèrent, durant tout l’hiver, d’envoyer des députés au-delà du Rhin, de solliciter les peuples à prendre les armes, de promettre des subsides, disant qu’une grande partie de notre armée ayant été massacrée, il ne nous en restait que de faibles débris. (2) Cependant ils ne purent déterminer à passer le Rhin aucun des peuples germains, doublement avertis, par la guerre d’Arioviste et le sort des Tencthères, de ne plus tenter la fortune. (3) Déchu de cet espoir, Indutiomaros n’en rassembla pas moins des troupes, les exerça, leva de la cavalerie chez les peuples voisins, et attira à lui de toutes les parties de la Gaule, par l’appât de grandes récompenses, les bannis et les condamnés. (4) Il s’était déjà acquis, par ces moyens, un tel crédit dans la Gaule, que de tous côtés lui venaient des députations des villes et des particuliers, pour solliciter sa protection et son amitié.
\subsection[{§ 56.}]{ \textsc{§ 56.} }
\noindent (1) Dès qu’il vit qu’on se ralliait à lui, que d’un côté les Sénons et les Carnutes étaient engagés par le souvenir de leur crime ; que de l’autre, les Nerviens et les Atuatuques se préparaient à la guerre, et qu’une foule de volontaires se réuniraient à lui sitôt qu’il aurait franchi les limites de son territoire, il convoqua un conseil armé, (2) selon l’usage des Gaulois au commencement d’une guerre. Là, d’après la loi générale, tous les jeunes gens doivent se rendre en armes ; celui d’entre eux qui arrive le dernier est mis à mort, en présence de la multitude, et au milieu des tourments. (3) Dans cette assemblée, il déclara ennemi de la patrie Cingétorix, son gendre, chef du parti opposé, et resté fidèle à César, auquel il s’était attaché, comme nous l’avons dit plus haut. La vente de ses biens fut publiée. (4) Il annonça ensuite dans le conseil qu’appelé par les Sénons, les Carnutes et plusieurs autres peuples de la Gaule ; (5) il se rendrait chez eux par le territoire des Rèmes, qu’il le ravagerait, et, qu’avant tout, il attaquerait le camp de Labiénus ; et il donna ses ordres pour l’exécution.
\subsection[{§ 57.}]{ \textsc{§ 57.} }
\noindent (1) Labiénus, qui occupait une position fortifiée par la nature et par l’art, ne craignait aucun danger ni pour lui ni pour la légion, et songeait à ne pas laisser échapper l’occasion d’un coup d’éclat. (2) Instruit par Cingétorix et ses proches du discours qu’Indutiomaros avait tenu dans l’assemblée, il envoie des messagers aux cités voisines, leur demande à toutes des cavaliers et indique le jour de leur réunion. (3) Cependant, presque chaque jour, Indutiomaros faisait voltiger sa cavalerie autour du camp, soit pour en reconnaître la situation, soit pour entrer en pourparlers ou nous inspirer de l’effroi ; le plus souvent les cavaliers lançaient des traits dans nos retranchements. (4) Labiénus retenait ses troupes dans le camp, et ne négligeait rien pour accroître l’opinion que les ennemis avaient de sa frayeur.
\subsection[{§ 58.}]{ \textsc{§ 58.} }
\noindent (1) Tandis qu’Indutiomaros s’approchait de notre camp, chaque jour avec plus de mépris, Labiénus fit entrer, dans une seule nuit, les cavaliers de toutes les cités voisines, qu’il leur avait demandés, et retint tous les siens au camp, par une garde si vigilante que ce renfort resta entièrement ignoré, et qu’aucun avis ne put être transmis aux Trévires. (2) Cependant Indutiomaros s’approche comme de coutume et passe une grande partie de la journée près de notre camp ; ses cavaliers lancent des traits, et, par des invectives, nous provoquent au combat. (3) Comme on ne leur répondit point, sur le soir ils se retirèrent dispersés et sans ordre. (4) Tout à coup Labiénus fait sortir par les deux portes du camp toute sa cavalerie ; il ordonne expressément, dès que les ennemis effrayés seront en fuite, ce qui arriva comme il l’avait prévu, qu’on s’attache à Indutiomaros seul, et qu’on ne blesse personne, avant que ce chef ne soit mis à mort ; il ne voulait pas que le temps donné à la poursuite des autres lui permît de s’échapper. Il promet de grandes récompenses à ceux qui l’auront tué, (5) et fait soutenir la cavalerie par les cohortes. (6) La Fortune seconde ses desseins. Poursuivi seul par tous, et atteint au gué même de la rivière, Indutiomaros est tué, et sa tête apportée au camp. Les cavaliers, au retour, attaquent et tuent ce qu’ils rencontrent d’ennemis. (7) À la nouvelle de cette déroute, les troupes réunies des Éburons et des Nerviens se retirent ; et César, après cet événement, vit la Gaule un peu plus tranquille.
\section[{Livre VI}]{Livre VI}\renewcommand{\leftmark}{Livre VI}

\subsection[{§ 1.}]{ \textsc{§ 1.} }
\noindent (1) César, qui, par une foule de raisons, s’attendait à de plus grands mouvements dans la Gaule, chargea M. Silanus, C. Antistius Reginus et T. Sextius, ses lieutenants, de faire des levées. (2) En même temps, il demanda à Cn. Pompée, proconsul, qui restait devant Rome avec le commandement, pour la sûreté de la république, d’ordonner aux recrues qu’il avait faites dans la Gaule cisalpine, sous son dernier consulat, de rejoindre leurs enseignes et de se rendre auprès de lui. (3) Il croyait très important, même pour l’avenir, de convaincre les Gaulois que l’Italie avait assez de ressources non seulement pour réparer promptement quelques pertes essuyées à la guerre, mais encore pour déployer plus de forces qu’auparavant. (4) Pompée accorda cette demande au besoin de la république et à l’amitié ; et, grâce à l’activité du recrutement, trois légions furent formées et réunies avant la fin de l’hiver ; le nombre des cohortes perdues sous Q. Titurius se trouva doublé, et l’on montra, par ces levées si promptes et si nombreuses, ce que pouvaient la discipline et les ressources du peuple romain.
\subsection[{§ 2.}]{ \textsc{§ 2.} }
\noindent (1) Après la mort d’Indutiomaros, dont nous avons parlé, les Trévires donnèrent le commandement à ses proches. Ceux-ci ne cessèrent de solliciter les Germains de leur voisinage, et de leur promettre des subsides : (2) ne pouvant rien obtenir des nations voisines, ils s’adressèrent aux peuples plus éloignés. Ils réussirent auprès de quelques cités, se lièrent par des serments et donnèrent des otages pour sûreté des subsides. Ils associèrent Ambiorix à leur pacte. (3) Informé de ces menées, et voyant que la guerre se préparait de toutes parts ; que les Nerviens, les Atuatuques, les Ménapes, ainsi que tous les peuples de la Germanie cisrhénane joints à eux, étaient en armes ; que les Sénons ne se rendaient point à ses ordres et se concertaient avec les Carnutes et les états voisins : que les Trévires sollicitaient les Germains par de nombreuses députations, César crut devoir hâter la guerre.
\subsection[{§ 3.}]{ \textsc{§ 3.} }
\noindent (1) Ayant donc, même avant la fin de l’hiver, réuni les quatre légions les plus proches, il fondit à l’improviste sur les terres des Nerviens ; (2) et, avant qu’ils pussent se rassembler ou fuir, il leur prit un grand nombre d’hommes et de bestiaux, abandonna ce butin aux soldats, dévasta le territoire, et les obligea de se rendre et de lui donner des otages. (3) Après cette expédition rapide, il ramena les légions dans leurs quartiers. (4) Ayant, au commencement du printemps, convoqué, selon son usage, l’assemblée de la Gaule, les différents peuples s’y rendirent, à l’exception des Sénons, des Carnutes et des Trévires. Regardant cette conduite comme un signal de guerre et de révolte, César ajourna toute autre affaire, et transféra l’assemblée à Lutèce, capitale des Parisii. (5) Ces derniers étaient voisins des Sénons, et n’avaient autrefois formé avec eux qu’une seule nation ; mais ils paraissaient étrangers au complot actuel. (6) César ayant, du haut de son siège, prononcé cette translation, partit le même jour à la tête des légions, et se rendit à grandes journées chez les Sénons.
\subsection[{§ 4.}]{ \textsc{§ 4.} }
\noindent (1) À la nouvelle de son approche, Acco, le principal auteur de la révolte, ordonna que la multitude se rassemblât de toutes parts dans les places fortes ; mais avant que cet ordre pût être exécuté, on apprit l’arrivée des Romains. (2) Forcés de renoncer à leur projet, les Sénons députèrent vers César pour demander leur pardon, et eurent recours à la médiation des Héduens, leurs anciens alliés. (3) César, à la prière de ceux-ci, leur fit grâce et reçut leurs excuses, ne voulant pas perdre en discussions un été propre aux expéditions militaires. (4) Il exigea cent otages, qu’il donna en garde aux Héduens. (5) Les Carnutes envoyèrent aussi des députés et des otages, et obtinrent le même traitement, par l’entremise et sur la prière des Rèmes, dont ils étaient les clients. (6) César vint clore l’assemblée de la Gaule, et ordonna aux villes de lui fournir des cavaliers.
\subsection[{§ 5.}]{ \textsc{§ 5.} }
\noindent (1) Après avoir pacifié cette partie de la Gaule, il tourne toutes ses pensées et tous ses projets vers la guerre des Trévires et d’Ambiorix. (2) Il fait partir avec lui Cavarinos et la cavalerie sénonaise, dans la crainte que les ressentiments de ce roi, ou la haine qu’il s’était attirée, ne fissent naître quelque trouble. (3) Cela fait, assuré qu’Ambiorix ne hasarderait point de bataille, il s’appliqua à deviner ses autres desseins. (4) Près du territoire des Éburons étaient les Ménapes, défendus par des marais immenses et de vastes forêts. Seuls entre les Gaulois, ces peuples n’avaient jamais envoyé de députés à César pour demander la paix. II savait qu’Ambiorix était uni avec eux par l’hospitalité, et qu’il s’était allié avec les Germains par l’entremise des Trévires. (5) Il jugea donc à propos de lui enlever cet appui avant de l’attaquer, de peur qu’en se voyant réduit à l’extrémité, il n’allât se cacher chez les Ménapes, ou former une ligue avec les peuples d’outre-Rhin. (6) Cette résolution prise, il envoie à Labiénus, chez les Trévires, tous les bagages de l’armée, et les fait suivre de deux légions. Lui-même, à la tête de cinq légions sans équipages, marche contre les Ménapes. (7) Ils n’avaient rassemblé aucune troupe, se fiant sur leur position ; ils se réfugièrent dans leurs bois et leurs marais, et y emportèrent ce qu’ils possédaient.
\subsection[{§ 6.}]{ \textsc{§ 6.} }
\noindent (1) César, ayant partagé ses troupes avec le lieutenant C. Fabius et le questeur M. Crassus, et fait construire des ponts à la hâte, pénètre dans le pays par trois endroits, incendie les maisons et les bourgs et enlève quantité de bestiaux et d’hommes. (2) Réduits à cet état, les Ménapes lui envoient demander la paix ; (3) César reçoit leurs otages et leur déclare qu’il les traitera en ennemis s’ils donnent asile à Ambiorix ou à ses lieutenants. (4) Cette affaire terminée, il laisse chez les Ménapes l’Atrébate Commios avec de la cavalerie, pour garder ce pays, et marche en personne contre les Trévires.
\subsection[{§ 7.}]{ \textsc{§ 7.} }
\noindent (1) Pendant ces expéditions de César, les Trévires avaient rassemblé des troupes nombreuses d’infanterie et de cavalerie, et se préparaient à attaquer Labiénus et l’unique légion qui hivernait sur leurs terres. (2) Déjà ils n’en étaient plus qu’à deux journées de marche, quand ils apprirent que César venait de lui envoyer deux légions. (3) Ils placèrent leur camp à quinze mille pas, et résolurent d’attendre le secours des Germains. (4) Labiénus, informé de leur dessein, et espérant que leur imprudence lui donnerait une occasion de les combattre, laissa cinq cohortes à la garde des bagages, marcha contre les ennemis avec vingt-cinq autres et beaucoup de cavalerie, et s’établit à mille pas d’eux, dans un camp qu’il fortifia. (5) Entre Labiénus et l’ennemi était une rivière dont le passage était difficile et les rives fort escarpées. Labiénus n’avait point l’intention de la traverser et ne croyait pas que les ennemis voulussent le faire : (6) l’espoir de l’arrivée des Germains croissait de jour en jour. Labiénus déclare hautement dans le conseil que les Germains étant, selon le bruit public, sur le point d’arriver, il ne hasardera pas le sort de l’armée et le sien, et que le lendemain, au point du jour, il lèvera le camp. (7) Ces paroles sont promptement rapportées aux ennemis ; car dans ce grand nombre de cavaliers gaulois, il était naturel qu’il y en eût plusieurs qui s’intéressassent aux succès de la Gaule. (8) Labiénus ayant, pendant la nuit, assemblé les tribuns et les centurions du premier rang, leur expose son dessein ; et, pour mieux inspirer aux ennemis l’opinion de sa frayeur, il ordonne de lever le camp avec plus de bruit et de tumulte que les armées romaines n’ont coutume de le faire. De cette manière il donne à son départ toutes les apparences d’une fuite. (9) La proximité des camps fit que l’ennemi en fut averti avant le jour par ses éclaireurs.
\subsection[{§ 8.}]{ \textsc{§ 8.} }
\noindent (1) À peine notre arrière-garde était-elle sortie du camp, que les Gaulois s’exhortent mutuellement à ne pas laisser échapper de leurs mains cette proie, objet de leurs espérances ; il serait trop long d’attendre le secours des Germains, et l’honneur ne leur permet point, avec tant de forces, de n’oser attaquer une misérable poignée de fuyards embarrassés de leurs bagages. Ils n’hésitent pas à passer la rivière et à engager le combat sur un terrain désavantageux. (2) Labiénus, qui l’avait prévu, et voulait les attirer tous de l’autre côté de la rivière, feignait toujours de se retirer, et s’avançait lentement. (3) Enfin, les bagages ayant été envoyés en avant et placés sur une hauteur : "Soldats, dit-il, le moment que vous désirez est venu ; vous tenez l’ennemi engagé dans une position défavorable ; (4) déployez sous notre conduite cette valeur qui s’est si souvent signalée sous les ordres du général. Croyez qu’il est présent et qu’il vous voit." (5) En même temps il ordonne de tourner les enseignes contre l’ennemi et de marcher sur lui en bataille. Il détache quelques escadrons pour la garde des bagages, et dispose le reste de la cavalerie sur les ailes. (6) Poussant aussitôt un grand cri, les Romains lancent leurs javelots. Les ennemis, voyant, contre leur attente, fondre sur eux les enseignes menaçantes de ceux qu’ils croyaient en fuite, ne purent pas même soutenir notre choc, s’enfuirent à la première attaque et gagnèrent les forêts voisines. (7) Labiénus les poursuivit avec la cavalerie, en tua un grand nombre, fit beaucoup de prisonniers, et reçut peu de jours après la soumission du pays ; car les Germains, qui venaient à leur secours, retournèrent chez eux à la nouvelle de cette défaite. (8) Les parents d’Indutiomaros, qui étaient les auteurs de la révolte, sortirent du territoire et se retirèrent avec eux. (9) Cingétorix qui, comme nous l’avons vu, était toujours resté dans le devoir, reçut le gouvernement suprême de sa nation.
\subsection[{§ 9.}]{ \textsc{§ 9.} }
\noindent (1) César, étant venu du pays des Ménapes dans celui des Trévires, résolut, pour deux motifs, de passer le Rhin : (2) l’un, pour punir les Germains d’avoir envoyé des secours aux Trévires, ses ennemis ; l’autre, pour fermer à Ambiorix toute retraite chez eux. (3) En conséquence, il voulut construire un pont un peu au-dessus de l’endroit où s’était antérieurement effectué le passage de l’armée. (4) Grâce à la connaissance des procédés déjà employés, et à l’ardeur extrême des soldats, l’ouvrage fut achevé en peu de jours. (5) Après avoir laissé une forte garde à la tête du pont, du côté des Trévires, dans la prévision de quelque mouvement subit de la part de ce peuple, il passa le fleuve avec le reste des légions et la cavalerie. (6) Les Ubiens, qui, avant ce temps, lui avaient donné des otages et s’étaient rendus à lui, envoient des députés pour se justifier, et lui exposer "qu’ils n’ont ni prêté des secours aux Trévires, ni violé leur foi." (7) Ils demandent avec prière "qu’on les épargne, et que, dans la haine générale contre les Germains, on ne fasse point supporter aux innocents les châtiments dus aux coupables ; si César exige de nouveaux otages, ils offrent de les donner." (8) César s’informa du fait, et apprit que les secours avaient été envoyés par les Suèves ; il reçut les satisfactions des Ubiens, et s’enquit des chemins et des passages qui conduisaient chez les Suèves.
\subsection[{§ 10.}]{ \textsc{§ 10.} }
\noindent (1) Peu de jours après, il sut des Ubiens que les Suèves rassemblaient toutes leurs troupes en un seul lieu, et qu’ils avaient ordonné aux nations qui étaient dans leur dépendance de leur envoyer des secours tant en infanterie qu’en cavalerie. (2) Sur cet avis, César se pourvoit de vivres, choisit pour le camp une position avantageuse, et enjoint aux Ubiens d’abandonner les campagnes et de faire passer dans les places fortes leur bétail et tous leurs biens, espérant amener, par la famine, ces hommes barbares et ignorants à la nécessité de combattre avec désavantage. (3) Il charge les Ubiens d’envoyer chez les Suèves de nombreux éclaireurs pour connaître tout ce qu’ils font. (4) Ses ordres sont exécutés, et, peu de jours après, on lui rapporte "que les Suèves, instruits par des messagers de l’approche de l’armée romaine, s’étaient, avec toutes leurs troupes et celles de leurs alliés, retirés jusqu’à l’extrémité de leur territoire ; (5) que là est une forêt d’une grandeur immense, appelée Bacenis, qui s’étend fort avant dans l’intérieur du pays, et qui, placée comme un mur naturel entre les Suèves et les Chérusques, met ces deux peuples à l’abri de leurs entreprises et de leurs incursions mutuelles ; c’est à l’entrée de cette forêt que les Suèves avaient résolu d’attendre l’arrivée des Romains.
\subsection[{§ 11.}]{ \textsc{§ 11.} }
\noindent (1) Au point où l’on est arrivé, il n’est pas sans doute hors de propos de parler des moeurs de la Gaule et de la Germanie, et de la différence qui existe entre ces deux nations. (2) Dans la Gaule, ce n’est pas seulement dans chaque ville, dans chaque bourg et dans chaque campagne qu’il existe des factions, mais aussi dans presque chaque famille : (3) ces factions ont pour chefs ceux qu’on estime et qu’on juge les plus puissants ; c’est à leur volonté et à leur jugement que sont soumises la plupart des affaires et des résolutions. (4) La raison de cet antique usage paraît être d’assurer au peuple une protection contre les grands : car personne ne souffre que l’on opprime ou circonvienne ses clients ; si l’on agissait autrement, on perdrait bientôt tout son crédit. (5) Ce même principe régit souverainement toute la Gaule : car toutes les cités sont divisées en deux partis.
\subsection[{§ 12.}]{ \textsc{§ 12.} }
\noindent (1) Lorsque César vint dans la Gaule, les Héduens étaient les chefs d’une de ces factions, les Séquanes, ceux de l’autre. (2) Ces derniers, moins puissants par eux-mêmes, parce que la principale autorité appartenait depuis longtemps aux Héduens, lesquels possédaient de grandes clientèles, s’étaient unis avec Arioviste et les Germains, et les avaient attirés à eux à force de présents et de promesses. (3) Après plusieurs victoires et la destruction de toute la noblesse des Héduens, ils acquirent une telle puissance (4) qu’un grand nombre de peuples, clients des Héduens, passèrent dans leur parti. Ils prirent en otages les fils de leurs principaux citoyens, firent prêter publiquement à cette nation le serment de ne rien entreprendre contre eux, s’attribuèrent la partie du territoire conquise par leurs armes, et obtinrent la suprématie dans toute la Gaule. (5) Réduit à cette extrémité, Diviciacos avait été implorer le secours du sénat romain, et était revenu sans rien obtenir. (6) L'arrivée de César changea la face des choses : les Héduens reprirent leurs otages, recouvrèrent leurs anciens clients, en acquirent de nouveaux par le crédit de César, parce qu’on voyait que ceux qui entraient dans leur amitié jouissaient d’une condition meilleure et d’un gouvernement plus doux ; et ils obtinrent dans tout le reste un crédit et un pouvoir qui firent perdre aux Séquanes leur prépondérance. (7) À ceux-ci avaient succédé les Rèmes ; lorsqu’on remarqua que leur faveur auprès de César égalait celle des Héduens, ceux que d’anciennes inimitiés empêchaient de s’unir à ces derniers se ralliaient à la clientèle des Rèmes, (8) qui les protégeaient avec zèle pour conserver le nouveau crédit qu’ils avaient si rapidement acquis. (9) Tel était alors l’état des choses que les Héduens avaient, sans contredit, le premier rang parmi les Gaulois, et que les Rèmes occupaient le second.
\subsection[{§ 13.}]{ \textsc{§ 13.} }
\noindent (1) Dans toute la Gaule, il n’y a que deux classes d’hommes qui soient comptées pour quelque chose et qui soient honorées ; car la multitude n’a guère que le rang des esclaves, n’osant rien par elle-même, et n’étant admise à aucun conseil. (2) La plupart, accablés de dettes, d’impôts énormes, et de vexations de la part des grands, se livrent eux-mêmes en servitude à des nobles qui exercent sur eux tous les droits des maîtres sur les esclaves. (3) Des deux classes privilégiées, l’une est celle des druides, l’autre celle des chevaliers. (4) Les premiers, ministres des choses divines, sont chargés des sacrifices publics et particuliers, et sont les interprètes des doctrines religieuses. Le désir de l’instruction attire auprès d’eux un grand nombre de jeunes gens qui les ont en grand honneur. (5) Les Druides connaissent de presque toutes les contestations publiques et privées. Si quelque crime a été commis, si un meurtre a eu lieu, s’il s’élève un débat sur un héritage ou sur des limites, ce sont eux qui statuent ; ils dispensent les récompenses et les peines. (6) Si un particulier ou un homme public ne défère point à leur décision, ils lui interdisent les sacrifices ; c’est chez eux la punition la plus grave. (7) Ceux qui encourent cette interdiction sont mis au rang des impies et des criminels, tout le monde s’éloigne d’eux, fuit leur abord et leur entretien, et craint la contagion du mal dont ils sont frappés ; tout accès en justice leur est refusé ; et ils n’ont part à aucun honneur. (8) Tous ces druides n’ont qu’un seul chef dont l’autorité est sans bornes. (9) À sa mort, le plus éminent en dignité lui succède ; ou, si plusieurs ont des titres égaux, l’élection a lieu par le suffrage des druides, et la place est quelquefois disputée par les armes. (10) À une certaine époque de l’année, ils s’assemblent dans un lieu consacré sur la frontière du pays des Carnutes, qui passe pour le point central de toute la Gaule. Là se rendent de toutes parts ceux qui ont des différends, et ils obéissent aux jugements et aux décisions des druides. (11) On croit que leur doctrine a pris naissance dans la Bretagne, et qu’elle fut de là transportée dans la Gaule ; et aujourd’hui ceux qui veulent en avoir une connaissance plus approfondie vont ordinairement dans cette île pour s’y instruire.
\subsection[{§ 14.}]{ \textsc{§ 14.} }
\noindent (1) Les druides ne vont point à la guerre et ne paient aucun des tributs imposés aux autres Gaulois ; ils sont exempts du service militaire et de toute espèce de charges. (2) Séduits par de si grands privilèges, beaucoup de Gaulois viennent auprès d’eux de leur propre mouvement, ou y sont envoyés par leurs parents et leurs proches. (3) Là, dit-on, ils apprennent un grand nombre de vers, et il en est qui passent vingt années dans cet apprentissage. Il n’est pas permis de confier ces vers à l’écriture, tandis que, dans la plupart des autres affaires publiques et privées, ils se servent des lettres grecques. (4) Il y a, ce me semble, deux raisons de cet usage : l’une est d’empêcher que leur science ne se répande dans le vulgaire ; et l’autre, que leurs disciples, se reposant sur l’écriture, ne négligent leur mémoire ; car il arrive presque toujours que le secours des livres fait que l’on s’applique moins à apprendre par coeur et à exercer sa mémoire. (5) Une croyance qu’ils cherchent surtout à établir, c’est que les âmes ne périssent point, et qu’après la mort, elles passent d’un corps dans un autre, croyance qui leur paraît singulièrement propre à inspirer le courage, en éloignant la crainte de la mort. (6) Le mouvement des astres, l’immensité de l’univers, la grandeur de la terre, la nature des choses, la force et le pouvoir des dieux immortels, tels sont en outre les sujets de leurs discussions : ils les transmettent à la jeunesse.
\subsection[{§ 15.}]{ \textsc{§ 15.} }
\noindent (1) La seconde classe est celle des chevaliers. Quand il en est besoin et qu’il survient quelque guerre (ce qui, avant l’arrivée de César, avait lieu presque tous les ans, soit pour faire, soit pour repousser des incursions ), ils prennent tous part à cette guerre, (2) et proportionnent à l’éclat de leur naissance et de leurs richesses le nombre de serviteurs et de clients dont ils s’entourent. C'est pour eux la seule marque du crédit et de la puissance.
\subsection[{§ 16.}]{ \textsc{§ 16.} }
\noindent (1) Toute la nation gauloise est très superstitieuse ; (2) aussi ceux qui sont attaqués de maladies graves, ceux qui vivent au milieu de la guerre et de ses dangers, ou immolent des victimes humaines, ou font voeu d’en immoler, et ont recours pour ces sacrifices au ministère des druides. (3) Ils pensent que la vie d’un homme est nécessaire pour racheter celle d’un homme, et que les dieux immortels ne peuvent être apaisés qu’à ce prix ; ils ont même institué des sacrifices publics de ce genre. (4) Ils ont quelquefois des mannequins d’une grandeur immense et tressés en osier, dont ils remplissent l’intérieur d’hommes vivants ; ils y mettent le feu et font expirer leurs victimes dans les flammes. (5) Ils pensent que le supplice de ceux qui sont convaincus de vol, de brigandage ou de quelque autre délit, est plus agréable aux dieux immortels ; mais quand ces hommes leur manquent, ils se rabattent sur les innocents.
\subsection[{§ 17.}]{ \textsc{§ 17.} }
\noindent (1) Le dieu qu’ils honorent le plus est Mercure. Il a un grand nombre de statues ; ils le regardent comme l’inventeur de tous les arts, comme le guide des voyageurs, et comme présidant à toutes sortes de gains et de commerce. (2) Après lui ils adorent Apollon, Mars, Jupiter et Minerve. Ils ont de ces divinités à peu près la même idée que les autres nations. Apollon guérit les maladies, Minerve enseigne les éléments de l’industrie et des arts ; Jupiter tient l’empire du ciel, Mars celui de la guerre ; (3) c’est à lui, quand ils ont résolu de combattre, qu’ils font voeu d’ordinaire de consacrer les dépouilles de l’ennemi. Ils lui sacrifient ce qui leur reste du bétail qu’ils ont pris, le surplus du butin est placé dans un dépôt public ; (4) et on peut voir, en beaucoup de villes de ces monceaux de dépouilles, entassées en des lieux consacrés. (5) Il n’arrive guère, qu’au mépris de la religion, un Gaulois ose s’approprier clandestinement ce qu’il a pris à la guerre, ou ravir quelque chose de ces dépôts. Le plus cruel supplice et la torture sont réservés pour ce larcin.
\subsection[{§ 18.}]{ \textsc{§ 18.} }
\noindent (1) Les Gaulois se vantent d’être issus de Dis Pater, tradition qu’ils disent tenir des druides. (2) C'est pour cette raison qu’ils mesurent le temps, non par le nombre des jours ; mais par celui des nuits. Ils calculent les jours de naissance, le commencement des mois et celui des années, de manière que le jour suive la nuit dans leur calcul. (3) Dans les autres usages de la vie, ils ne diffèrent guère des autres nations qu’en ce qu’ils ne permettent pas que leurs enfants les abordent en public avant d’être adolescents et en état de porter les armes. Ils regardent comme honteux pour un père d’admettre publiquement en sa présence son fils en bas âge.
\subsection[{§ 19.}]{ \textsc{§ 19.} }
\noindent (1) Autant les maris ont reçu d’argent de leurs épouses à titre de dot, autant ils mettent de leurs propres biens, après estimation faite, en communauté avec cette dot. (2) On dresse conjointement un état de ce capital, et l’on en réserve les intérêts. Quelque époux qui survive, c’est à lui qu’appartient la part de l’un et de l’autre, avec les intérêts des années antérieures. (3) Les hommes ont, sur leurs femmes comme sur leurs enfants, le droit de vie et de mort ; lorsqu’un père de famille d’une haute naissance vient à mourir, ses proches s’assemblent, et s’ils ont quelque soupçon sur sa mort, les femmes sont mises à la question comme des esclaves ; si le crime est prouvé, on les fait périr par le feu et dans les plus horribles tourments. (4) Les funérailles, eu égard à la civilisation des Gaulois, sont magnifiques et somptueuses. Tout ce qu’on croit avoir été cher au défunt pendant sa vie, on le jette dans le bûcher, même les animaux ; et il y a peu de temps encore, on brûlait avec lui les esclaves et les clients qu’on savait qu’il avait aimés, pour complément des honneurs qu’on lui rendait.
\subsection[{§ 20.}]{ \textsc{§ 20.} }
\noindent (1) Dans les cités qui passent pour administrer le mieux les affaires de l’état, c’est une loi sacrée que celui qui apprend, soit de ses voisins, soit par le bruit public, quelque nouvelle intéressant la cité, doit en informer le magistrat, sans la communiquer à nul autre, (2) l’expérience leur ayant fait connaître que souvent des hommes imprudents et sans lumières s’effraient de fausses rumeurs, se portent à des crimes et prennent des partis extrêmes. (3) Les magistrats cachent ce qu’ils jugent convenable, et révèlent à la multitude ce qu’ils croient utile. C'est dans l’assemblée seulement qu’il est permis de s’entretenir des affaires publiques.
\subsection[{§ 21.}]{ \textsc{§ 21.} }
\noindent (1) Les moeurs des Germains sont très différents ; car ils n’ont pas de druides qui président aux choses divines et ne font point de sacrifices. (2) Ils ne mettent au nombre des dieux que ceux qu’ils voient et dont ils reçoivent manifestement les bienfaits, le Soleil, Vulcain, la Lune : ils ne connaissent pas même de nom les autres dieux. (3) Toute leur vie se passe à la chasse et dans les exercices militaires ; ils se livrent dès l’enfance au travail et à la fatigue. (4) Ils estiment singulièrement une puberté tardive ; ils pensent que cela accroît la stature de l’homme, nourrit sa vigueur, et fortifie ses muscles. (5) C'est parmi eux une chose tout à fait honteuse que d’avoir connu les femmes avant l’âge de vingt ans ; ce qu’ils ne peuvent jamais cacher, car ils se baignent ensemble dans les fleuves, et se couvrent de peaux de rennes ou de vêtements courts, laissant à nu la plus grande partie de leur corps.
\subsection[{§ 22.}]{ \textsc{§ 22.} }
\noindent (1) Ils ne s’adonnent pas à l’agriculture, et ne vivent guère que de lait, de fromage et de chair ; (2) nul n’a de champs limités ni de terrain qui soit sa propriété ; mais les magistrats et les chefs assignent tous les ans aux peuplades et aux familles vivant en société commune, des terres en tels lieux et quantité qu’ils jugent à propos ; et l’année suivante ils les obligent de passer ailleurs. (3) Ils donnent beaucoup de raisons de cet usage : la crainte que l’attrait d’une longue habitude ne fasse perdre le goût de la guerre pour celui de l’agriculture ; que chacun, s’occupant d’étendre ses possessions, les plus puissants ne chassent des leurs les plus faibles ; qu’on ne se garantisse du froid et de la chaleur par des habitations trop commodes ; que l’amour des richesses ne s’introduise parmi eux et ne fasse naître les factions et les discordes ; (4) on veut enfin contenir le peuple par un esprit de justice, en lui montrant une parfaite égalité de biens entre les plus humbles et les plus puissants.
\subsection[{§ 23.}]{ \textsc{§ 23.} }
\noindent (1) La plus grande gloire pour un état est d’être entouré de vastes solitudes et de pays ravagés par ses armes. (2) Ils regardent comme le propre de la valeur de forcer leurs voisins à abandonner leur territoire, et de faire que personne n’ose s’établir auprès d’eux. (3) D'ailleurs, ils se croient ainsi plus en sûreté, n’ayant pas à craindre une invasion subite. (4) Lorsqu’un état fait la guerre, soit qu’il se défende, soit qu’il attaque, on choisit, pour y présider, des magistrats qui ont droit de vie et de mort. (5) Pendant la paix, il n’y a point de magistrature générale ; les principaux habitants des cantons et des bourgs rendent la justice à leurs concitoyens et arrangent les procès. (6) Aucune infamie n’est attachée aux larcins qui se commettent hors des limites de l’état ; ils prétendent que c’est un moyen d’exercer la jeunesse et de la préserver de l’oisiveté. (7) Lorsque, dans une assemblée, un des principaux citoyens s’annonce pour chef d’une expédition, et demande qui veut le suivre, ceux qui jugent avantageusement de l’entreprise et de l’homme se lèvent, lui promettent leur assistance, et sont applaudis par la multitude. (8) Ceux d’entre eux qui l’abandonnent sont réputés déserteurs et traîtres, et toute espèce de confiance leur est désormais refusée. (9) Il ne leur est jamais permis de violer l’hospitalité. Ceux qui viennent à eux, pour quelque cause que ce soit, sont garantis de toute injure et regardés comme sacrés : toutes les maisons leur sont ouvertes ; on partage les vivres avec eux.
\subsection[{§ 24.}]{ \textsc{§ 24.} }
\noindent (1) II fut un temps où les Gaulois surpassaient les Germains en valeur, portaient la guerre chez eux, envoyaient des colonies au-delà du Rhin, vu leur nombreuse population et l’insuffisance de leur territoire. (2) C'est ainsi que les terres les plus fertiles de la Germanie, près de la forêt Hercynienne, (qui me paraît avoir été, par la renommée, connue d’Eratosthène et de quelques autres Grecs, sous le nom d’Orcynie), furent envahies par les Volques Tectosages, qui s’y fixèrent. (3) Cette nation s’est jusqu’à ce jour maintenue dans cet établissement et jouit d’une grande réputation de justice et de courage ; (4) et encore aujourd’hui, ils vivent dans la même pauvreté, le même dénuement, la même habitude de privation que les Germains, dont ils ont aussi adopté le genre de vie et l’habillement. (5) Quant aux Gaulois, le voisinage de la province, et l’usage des objets de commerce maritime, leur ont procuré l’abondance et les jouissances du luxe. (6) Accoutumés peu à peu à se laisser surpasser, et, vaincus dans un grand nombre de combats, ils ne se comparent même plus à ces Germains pour la valeur.
\subsection[{§ 25.}]{ \textsc{§ 25.} }
\noindent (1) La largeur de cette forêt d’Hercynie, dont il vient d’être fait mention, est de neuf journées de marche accélérée, et ne peut être autrement déterminée, les mesures itinéraires n’étant point connues des Germains. (2) Elle prend naissance aux frontières des Helvètes, des Némètes et des Rauraques, et s’étend, en suivant le cours du Danube, jusqu’aux pays des Daces et des Anartes : (3) de là elle tourne sur la gauche, en s’éloignant du fleuve ; et, dans son immense étendue, elle borde le territoire d’une foule de nations ; (4) il n’est point d’habitant de ces contrées qui, après soixante jours de marche, puisse dire avoir vu où elle finit, ni savoir où elle commence. (5) On assure qu’il s’y trouve plusieurs espèces d’animaux sauvages qu’on ne voit pas ailleurs. Celles qui diffèrent le plus des autres et qui paraissent mériter une mention spéciale, les voici.
\subsection[{§ 26.}]{ \textsc{§ 26.} }
\noindent (1) On y rencontre un boeuf, ayant la forme d’un cerf et portant au milieu du front, entre les oreilles, une seule corne, plus élevée et plus droite que les cornes qui nous sont connues. (2) À son sommet, elle se partage en rameaux très tendus, semblables à des palmes. (3) La femelle est de même nature que le mâle ; la forme et la grandeur de ses cornes sont les mêmes.
\subsection[{§ 27.}]{ \textsc{§ 27.} }
\noindent (1) Il y a aussi des animaux qu’on appelle élans. Leur forme se rapproche de celle d’une chèvre ; ils ont la peau tachetée, mais la taille un peu plus haute. Ils sont sans cornes, et leurs jambes, sans jointures ni articulations ; (2) ils ne se couchent point pour dormir, et si quelque accident les fait tomber, ils ne peuvent se soulever ni se redresser. (3) Les arbres leur servent de lits ; ils s’y appuient et prennent leur repos, ainsi inclinés légèrement. (4) Lorsqu’à leurs traces les chasseurs découvrent les lieux qu’ils fréquentent, ils y déracinent tous les arbres, ou les coupent à fleur de terre, de manière qu’ils conservent encore toute l’apparence de la solidité. (5) Ces animaux viennent s’y appuyer, selon leur coutume, renversent ce frêle appui par leur poids, et tombent avec l’arbre.
\subsection[{§ 28.}]{ \textsc{§ 28.} }
\noindent (1) Une troisième espèce porte le nom d’urus. La taille de ces animaux est un peu moindre que celle des éléphants ; leur couleur et leur forme les font ressembler au taureau. (2) Leur force et leur vélocité sont également remarquables ; rien de ce qu’ils aperçoivent, hommes ou bêtes, ne leur échappe. (3) On les tue, en les prenant dans des fosses disposées avec soin. Ce genre de chasse est pour les jeunes gens un exercice qui les endurcit à la fatigue ; ceux qui ont tué le plus de ces urus en apportent les cornes en public, comme trophée, et reçoivent de grands éloges. (4) On ne peut les apprivoiser, même dans le jeune âge. (5) La grandeur, la forme et l’espèce de leurs cornes diffèrent beaucoup de celles de nos boeufs. (6) On les recherche avidement, on les garnit d’argent sur les bords, et elles servent de coupes dans les festins solennels.
\subsection[{§ 29.}]{ \textsc{§ 29.} }
\noindent (1) César, informé par les éclaireurs ubiens que les Suèves s’étaient retirés dans leurs forêts, mais craignant de manquer de vivres (car on a vu plus haut que l’agriculture est fort négligée chez les Germains), résolut de ne pas s’engager plus avant. (2) Cependant, pour laisser aux barbares quelque appréhension de son retour et arrêter les renforts envoyés aux Gaulois, il fit couper, après que l’armée eut repassé le Rhin, deux cents pieds du pont du côté de la rive des Ubiens, (3) et élever, à l’extrémité opposée, une tour à quatre étages ; il y laissa pour garde une garnison de douze cohortes, et fortifia ce lieu par de nombreux retranchements. Il en confia le commandement au jeune C. Volcacius Tullus. (4) Comme les blés commençaient à mûrir, il partit lui-même pour la guerre d’Ambiorix, par la forêt des Ardennes, qui est la plus grande de toute la Gaule, et qui, s’étendant depuis les rives du Rhin et le pays des Trévires jusqu’à celui des Nerviens, embrasse dans sa longueur un espace de plus de cinq cents milles ; il envoya en avant Minucius Basilus avec toute la cavalerie, dans l’espoir de profiter au besoin de la célérité des marches et de quelque circonstance favorable. (5) Il lui recommanda d’interdire les feux dans son camp, afin de ne pas révéler de loin son approche ; et lui annonça qu’il le suivrait de près.
\subsection[{§ 30.}]{ \textsc{§ 30.} }
\noindent (1) Basilus suivit exactement ses instructions ; et, après une marche aussi prompte qu’inattendue, il prit au dépourvu un grand nombre d’ennemis répandus dans la campagne : sur leurs renseignements, il se dirigea vers le lieu où l’on disait qu’était Ambiorix avec quelques cavaliers. (2) La fortune peut beaucoup en toute chose, et surtout à la guerre. Car si ce fut un grand hasard de surprendre Ambiorix sans préparatifs de défense, et avant qu’il eût rien appris de l’approche des Romains par le bruit public ou par des courriers, ce fut aussi pour lui un grand bonheur, qu’après s’être vu enlever tout l’attirail de guerre qu’il avait autour de lui, et prendre ses chars et ses chevaux, il pût échapper à la mort. (3) C'est pourtant ce qui arriva, parce que sa maison étant située au milieu des bois (comme le sont généralement celles des Gaulois, qui, pour éviter la chaleur, cherchent le voisinage des forêts et des fleuves), ses compagnons et ses amis purent soutenir quelque temps, dans un défilé, le choc de nos cavaliers. (4) Pendant ce combat, quelqu’un des siens le mit à cheval ; et les bois protégèrent sa fuite. Ainsi la fortune se plut à la fois et à le jeter dans péril et à l’y soustraire.
\subsection[{§ 31.}]{ \textsc{§ 31.} }
\noindent (1) Ambiorix ne rassembla point ses troupes ; était-ce à dessein, et parce qu’il ne jugea pas à propos de combattre, ou faute de temps, et à cause de l’obstacle qu’y put apporter l’arrivée subite de notre cavalerie, qu’il crut suivie du reste de l’armée ; c’est ce qu’on ne peut décider ; (2) il est toutefois certain qu’il envoya secrètement des messagers dans les campagnes pour ordonner à chacun de pourvoir à sa sûreté. Les uns se réfugièrent dans la forêt des Ardennes, les autres dans les marais voisins. (3) Ceux qui étaient le plus près de l’Océan se cachèrent dans ces îles que forment d’ordinaire les marées ; un grand nombre, quittant leur pays, se fixèrent, avec tous leurs biens, dans des contrées tout à fait étrangères. Catuvolcos, roi de la moitié du pays des Éburons, et qui s’était rallié à Ambiorix, vieillard accablé par l’âge, et également incapable de supporter les fatigues de la guerre ou de la fuite, après avoir chargé d’imprécations Ambiorix, auteur de cette entreprise, s’empoisonna avec de l’if, qui croit en abondance dans la Gaule et dans la Germanie.
\subsection[{§ 32.}]{ \textsc{§ 32.} }
\noindent (1) Les Sègnes et les Condruses, peuples d’origine germaine, qui habitent entre les Éburons et les Trévires, envoyèrent des députés à César, pour le prier de ne point les mettre au nombre de ses ennemis, et de ne pas croire que tous les Germains en deçà du Rhin fissent cause commune ; ils n’avaient nullement songé à la guerre, n’avaient donné aucun secours à Ambiorix. (2) César, s’étant informé du fait en questionnant les captifs, ordonna à ces peuples de lui ramener ceux des Eburons qui, après leur déroute, se seraient rassemblés chez eux, et leur promit, s’ils le faisaient, de ne commettre aucun dégât sur leur territoire. (3) Ayant alors distribué ses troupes en trois parties, il envoya les bagages de toutes les légions à Atuatuca. (4) C'est le nom d’un fort. Il est situé presqu’au milieu du pays des Éburons, dans le lieu même où Titurius et Aurunculéius avaient déjà établi leurs quartiers d’hiver. (5) César choisit cette position par divers motifs, et surtout parce que les retranchements de l’année précédente étaient entièrement conservés, ce qui devait épargner beaucoup de travail aux soldats. II laissa pour la garde des bagages la quatorzième légion, l’une des trois qu’il avait récemment levées en Italie, et amenées en Gaule. (6) Il confia à Q. Tullius Cicéron le commandement de cette légion et du camp, et lui donna deux cents cavaliers.
\subsection[{§ 33.}]{ \textsc{§ 33.} }
\noindent (1) Partageant l’armée, il fait partir T. Labiénus, avec trois légions, vers l’Océan, dans le pays qui touche aux Ménapes ; il envoie (2) C. Trébonius, avec le même nombre de légions, vers les contrées voisines des Atuatuques, avec ordre de les ravager. (3) Il arrête de marcher en personne avec les trois autres, vers le fleuve de l’Escaut, qui se jette dans la Meuse, et de gagner l’extrémité des Ardennes, où il entendait dire qu’Ambiorix s’était retiré avec un petit nombre de cavaliers. (4) Il annonce, en partant, qu’il sera de retour dans sept jours ; c’était l’époque où il savait qu’on devait distribuer les vivres à la légion qu’il laissait pour la garde des bagages. (5) Il engage Labiénus et Trébonius à revenir le même jour, si l’état des choses leur permet de le faire, afin de se concerter de nouveau et de diriger la guerre d’après ce qu’on saurait des dispositions des ennemis.
\subsection[{§ 34.}]{ \textsc{§ 34.} }
\noindent (1) Ils n’avaient, comme on l’a dit plus haut, nulle troupe organisée, point de garnison, point de place qui fût en état de défense, c’était une multitude éparse çà et là. (2) Se présentait-il un vallon couvert, un lieu boisé, un marais de difficile accès, qui leur offrit quelque espoir de sûreté ou de salut, ils s’y arrêtaient. (3) Ces asiles étaient connus des habitants voisins, et la chose requérait beaucoup de prudence, non pour protéger le corps de l’armée, car elle n’avait, en masse, aucun danger à craindre de la part d’ennemis effrayés et dispersés, mais pour la conservation de chaque soldat ; et d’ailleurs le salut des individus intéressait celui de l’armée entière. (4) L'appât du butin en entraînait plusieurs au loin, et l’incertitude des chemins, dans ces forêts épaisses, empêchait de marcher en corps de troupes. (5) Si l’on voulait en finir, et exterminer cette race de brigands, il fallait former plusieurs détachements et laisser agir séparément les soldats. (6) Si on voulait retenir ceux-ci près de leurs enseignes, suivant l’ordre et l’usage des armées romaines, la nature même des lieux faisait la sûreté des barbares, et ils ne manquaient pas d’audace pour dresser de secrètes embûches et envelopper nos soldats dispersés. (7) Mais au milieu des difficultés de ce genre, il fallait employer toutes les précautions possibles ; il valait mieux, quelque désir de vengeance qui enflammât nos soldats, faire moins de mal à l’ennemi que d’exposer les troupes à trop de dangers. (8) César envoie des messagers dans les pays voisins ; il les appelle tous à lui par l’espoir du butin, les invite à piller les Éburons, aimant mieux risquer, au milieu de ces forêts, la vie des Gaulois que celle des légionnaires ; il voulait au moyen de cette invasion d’une immense multitude, détruire jusque dans sa race et son nom une nation si criminelle. (9) Une foule nombreuse arriva bientôt.
\subsection[{§ 35.}]{ \textsc{§ 35.} }
\noindent (1) Les choses se passaient ainsi sur tous les points du pays des Éburons, et l’on approchait de ce septième jour, auquel César avait fixé son retour près des bagages et de la légion qui les gardait. (2) On vit alors tout ce que peut le hasard à la guerre et quels événements il produit. (3) L'ennemi était dispersé et frappé d’épouvante : il n’avait, nous l’avons dit, aucune troupe capable d’inspirer la moindre crainte. (4) Le bruit parvint au-delà du Rhin, chez les Germains, que le pays des Éburons était livré au pillage et que l’on conviait tous les peuples à cette proie. (5) Deux mille cavaliers se réunissent chez les Sugambres, voisins du Rhin, et qui avaient, comme nous l’avons déjà vu, recueilli dans leur fuite les Tencthères et les Usipètes ; (6) ils passent le Rhin sur des barques et des radeaux, à trente mille pas au-dessous de l’endroit où César avait établi un pont et laissé une garde. Ils envahissent d’abord les frontières des Eburons, ramassent une foule de fuyards dispersés, et s’emparent d’une grande quantité de bestiaux, dont les barbares sont très avides. (7) L'appât du butin les entraîne plus avant. Il n’est ni marais ni bois capables d’arrêter ces hommes nés au sein de la guerre et du brigandage. Ils s’informent des prisonniers en quels lieux est César ; ils apprennent qu’il est parti au loin et que l’armée s’est retirée. (8) Alors un des captifs leur dit : "Pourquoi poursuivre une paix si misérable et si mince, tandis que la plus haute fortune s’offre à vous ? En trois heures vous pouvez arriver devant Atuatuca : là sont déposées toutes les richesses de l’armée romaine ; (9) la garde est si faible qu’elle ne pourrait même pas border le rempart, et que pas un n’oserait sortir des retranchements." (10) Pleins d’espérances, les Germains mettent à couvert le butin qu’ils avaient fait, et marchent sur Atuatuca, prenant pour guide celui qui leur avait donné cet avis.
\subsection[{§ 36.}]{ \textsc{§ 36.} }
\noindent (1) Cicéron, qui, tous les jours précédents, avait, selon les ordres de César, retenu avec le plus grand soin les soldats dans le camp, et n’avait pas même souffert qu’un seul valet sortît du retranchement, le septième jour, ne comptant plus sur l’exactitude de César à revenir au terme fixé, d’autant plus qu’on annonçait qu’il s’était avancé au loin, et que l’on n’avait aucune nouvelle de son retour, (2) céda aux plaintes des soldats, qui disaient que sa patience équivalait à un siège, puisqu’on ne pouvait sortir du camp. Il croyait d’ailleurs qu’étant couvert par neuf légions et une nombreuse cavalerie, il n’avait rien à craindre, à trois mille pas du camp, d’ennemis dispersés et presque détruits. Il envoya donc cinq cohortes couper du blé dans la campagne la plus voisine, dont il était séparé par une colline seulement. (3) On avait laissé dans le camp des malades de diverses légions ; trois cents environ, qui s’étaient rétablis pendant l’absence de César, sortent ensemble sous une même enseigne ; enfin une multitude de valets obtiennent la permission de suivre avec un grand nombre de chevaux laissés en dépôt dans le camp.
\subsection[{§ 37.}]{ \textsc{§ 37.} }
\noindent (1) Le hasard voulut qu’en ce moment même les cavaliers germains arrivassent : sans faire halte, ils essaient de pénétrer dans le camp par la porte décumane. (2) On ne les avait aperçus, à cause des bois qui couvraient cette partie, que lorsqu’ils étaient déjà près du camp, et les marchands qui avaient leurs tentes sous le rempart n’eurent pas même le temps de rentrer. (3) Nos soldats, surpris, se troublent à cette attaque ; et la cohorte de garde soutint à peine le premier choc. (4) Les ennemis se répandent à l’entour, cherchant un passage. (5) C'est avec peine que les nôtres défendent les portes ; les autres issues étaient garanties par leur position naturelle et par les retranchements. (6) La confusion est dans tout le camp ; chacun se demande la cause du tumulte ; et personne ne sait ni où porter les enseignes ni où doit se rendre chaque soldat. (7) L'un annonce que le camp est déjà pris ; l’autre, que le général a péri avec l’armée, et que les Barbares viennent en vainqueurs ; (8) la plupart se font sur la nature du lieu des idées superstitieuses, et n’ont plus sous les yeux que la catastrophe de Cotta et de Titurius, tués dans le même fort. (9) Une consternation si profonde et si générale confirme les Barbares dans l’opinion que le prisonnier leur avait donnée du dénuement complet de notre garnison. (10) Ils essaient de pénétrer de vive force, et s’exhortent à ne pas laisser échapper de leurs mains une si riche proie.
\subsection[{§ 38.}]{ \textsc{§ 38.} }
\noindent (1) Au nombre des malades laissés dans Atuatuca était P. Sextius Baculus, qui avait servi sous César en qualité de primipile, et dont nous avons fait mention dans le récit des combats précédents. Depuis cinq jours, il n’avait pas pris de nourriture. (2) Inquiet sur le salut de tous et sur le sien, il sort sans armes de sa tente ; il voit combien l’ennemi est proche et le péril pressant, se saisit des premières armes qu’il aperçoit et se place à une porte. (3) Les centurions de la cohorte qui était de garde le suivent ; et tous ensemble ils soutiennent quelque temps le combat. (4) Grièvement blessé, Sextius perd connaissance ; on le passe de mains en mains, et on le sauve avec peine. (5) Dans cet intervalle, les autres reprennent assez de courage pour oser rester sur le rempart et présenter l’apparence d’une défense.
\subsection[{§ 39.}]{ \textsc{§ 39.} }
\noindent (1) Cependant nos soldats, revenant du fourrage, entendent des cris ; la cavalerie prend les devants et reconnaît toute l’imminence du danger. (2) Il n’y a là aucun retranchement où ils puissent se retirer dans leur frayeur ; les soldats nouvellement levés et sans expérience de la guerre se tournent vers le tribun militaire et les centurions, et attendent leurs ordres. (3) Nul n’est si brave que la vue du danger ne l’étonne. (4) Les Barbares, apercevant de loin les enseignes, cessent leur attaque ; ils croient d’abord que c’est le retour des légions que les captifs avaient dit s’être portées au loin ; mais, bientôt, pleins de mépris pour ce petit nombre de troupes, ils fondent sur elles de toutes parts.
\subsection[{§ 40.}]{ \textsc{§ 40.} }
\noindent (1) Les valets courent sur une hauteur voisine. Chassés de ce poste, ils se replient en désordre sur les enseignes et sur les rangs des cohortes, et augmentent la frayeur des soldats. (2) Les uns veulent qu’on forme le coin, afin de percer jusqu’au camp dont ils sont si près, espérant que si une partie d’entre eux est enveloppée et succombe, le reste du moins pourra se sauver ; (3) d’autres veulent qu’on tienne ferme sur la colline, et que tous courent la même fortune. (4) Cet avis n’est pas celui des vieux soldats que nous avons dit s’être mis en marche sous une même enseigne. Après s’être encouragés mutuellement, et sous la conduite de C. Trébonius, chevalier romain, qui les commandait, ils se font jour à travers les ennemis, et tous, jusqu’au dernier, ils rentrent au camp sans perte. (5) Les valets et les cavaliers suivent l’impulsion, et sont sauvés par le courage des soldats. (6) Mais ceux qui s’étaient arrêtés sur la colline, et à qui manquait toute expérience de l’art militaire, ne surent ni persévérer dans le parti qu’ils avaient pris de se défendre sur cette hauteur, ni imiter la vigueur et l’impétuosité qui avaient, sous leurs yeux, sauvé les autres ; en essayant de gagner le camp, ils s’engagèrent dans un lieu désavantageux. (7) Les centurions, dont plusieurs avaient été, à cause de leur valeur, promus, des rangs inférieurs des autres légions, aux premiers grades de celle-ci, voulant conserver leur ancienne gloire militaire, se firent tuer en combattant vaillamment. (8) Une partie des soldats, pendant que l’ennemi reculait devant tant de courage, put rentrer au camp, sauvée contre tout espoir ; le reste fut enveloppé par les Barbares et périt.
\subsection[{§ 41.}]{ \textsc{§ 41.} }
\noindent (1) Les Germains, désespérant de forcer le camp, et nous voyant en état de défense sur les remparts, repassèrent le Rhin avec le butin qu’ ils avaient déposé dans les forêts. (2) Mais tel était encore l’effroi, même après le départ des ennemis, que la nuit suivante, C. Volusénus, envoyé en avant avec la cavalerie, étant venu au camp, ne put faire croire que César approchait avec l’armée intacte. (3) La peur possédait tous les esprits, au point que, dans l’égarement où l’on était, on soutenait que toutes les troupes avaient été détruites, que la cavalerie seule avait échappé par la fuite ; que, si l’armée eût été entière, les Germains n’auraient pas attaqué le camp. (4) L'arrivée de César dissipa cette frayeur.
\subsection[{§ 42.}]{ \textsc{§ 42.} }
\noindent (1) À son retour, celui-ci, qui connaissait les chances de la guerre, se plaignit uniquement qu’on eût fait sortir les cohortes du camp dont elles avaient la garde ; il remontra qu’on n’aurait pas dû donner prise au moindre hasard, et que la fortune avait eu grande part à l’arrivée subite des ennemis ; (2) mais qu’elle avait fait encore plus, en permettant qu’on repoussât les barbares presque maîtres des retranchements et des portes du camp. (3) Ce qui, dans tout cela, lui paraissait le plus étonnant, c’est que les Germains, qui n’avaient passé le Rhin que pour ravager le territoire d’Ambiorix, eussent, en venant attaquer le camp des Romains, rendu à Ambiorix le plus grand service qu’il pût désirer.
\subsection[{§ 43.}]{ \textsc{§ 43.} }
\noindent (1) César partit de nouveau à la poursuite des ennemis, et, rassemblant un grand nombre de troupes des cités voisines, il les lâcha en tous sens. (2) Tous les bourgs et toutes les habitations que chacun rencontrait furent incendiés ; tout fut livré au pillage. (3) Le blé que ne consomma point une si grande multitude de chevaux et d’hommes fut détruit par les pluies et les orages, à tel point que le petit nombre de ceux qui nous échappèrent en se cachant dut, après le départ de l’armée, périr de faim et de misère. (4) Il arrivait souvent que, toute la cavalerie battant le pays, des prisonniers disaient avoir vu Ambiorix dans sa fuite, et assuraient qu’il ne pouvait être loin ; (5) aussi l’espoir de l’atteindre, et le désir de gagner la faveur particulière de César, faisaient supporter des fatigues infinies et triompher presque de la nature à force d’ardeur ; on semblait toujours n’avoir manqué que de peu d’instants une si importante capture, et toujours des cavernes et des bois nous le dérobaient. (6) Il gagna ainsi, à la faveur de la nuit, d’autres régions, d’autres retraites, sans autre escorte que celle de quatre cavaliers, les seuls auxquels il osât confier sa vie.
\subsection[{§ 44.}]{ \textsc{§ 44.} }
\noindent (1) Après la dévastation de ce territoire, César ramena l’armée, diminuée de deux cohortes, à Durocortorum, capitale des Rèmes, et, y ayant convoqué l’assemblée de la Gaule, il résolut de s’occuper de la conjuration des Sénons et des Carnutes. (2) Acco, qui en avait été le chef, reçut sa sentence de mort et subit son supplice selon les anciens usages. Quelques autres prirent la fuite, dans la crainte d’un jugement. (3) Après leur avoir interdit le feu et l’eau, César établit deux légions en quartiers d’hiver chez les Trévires, deux chez les Lingons, et les six autres sur les terres des Sénons, à Agédincum. Lorsqu’il eut pourvu aux subsistances de l’armée, il partit pour l’Italie, selon sa coutume, pour y tenir l’assemblée du pays.
\section[{Livre VII}]{Livre VII}\renewcommand{\leftmark}{Livre VII}

\subsection[{§ 1.}]{ \textsc{§ 1.} }
\noindent (1) Voyant la Gaule tranquille, César, comme il l’avait résolu, va tenir les assemblées en Italie. Il y apprend la mort de P. Clodius, et, d’après le sénatus-consulte qui ordonnait à toute la jeunesse de l’Italie de prêter le serment militaire, il fait des levées dans toute la province. (2) La nouvelle en est bientôt portée dans la Gaule transalpine. Les Gaulois supposent d’eux-mêmes et ajoutent à ces bruits, ce qui semblait assez fondé, "que les mouvements de Rome retiennent César, et qu’au milieu de troubles si grands il ne peut se rendre auprès de l’armée." (3) Excités par ces circonstances favorables, ceux qui déjà se voyaient avec douleur soumis au peuple romain commencent à se livrer plus ouvertement et plus audacieusement à des projets hostiles. (4) Les principaux de la Gaule s’assemblent dans des lieux écartés et dans les bois  ; ils s’y plaignent de la mort d’Acco ; ils se disent qu’il peut leur en arriver autant ; (5) ils déplorent le sort commun de la Gaule ; ils offrent toutes les récompenses à ceux qui commenceront la guerre, et qui rendront la liberté à la Gaule au péril de leur vie. (6) Tous conviennent que la première chose à faire, avant que leurs projets secrets éclatent, est d’empêcher César de rejoindre l’armée ; (7) ce qui sera facile parce que, pendant son absence, les légions n’oseront sortir de leurs quartiers d’hiver, et que lui-même n’y pourra parvenir sans escorte ; (8) qu’enfin il vaut mieux périr dans une bataille que de ne pas recouvrer leur ancienne gloire militaire et la liberté qu’ils ont reçue de leurs ancêtres.
\subsection[{§ 2.}]{ \textsc{§ 2.} }
\noindent (1) A la suite de cette discussion, les Carnutes déclarent "qu’ils s’exposeront à tous les dangers pour la cause commune ; qu’ils prendront les armes les premiers de tous ; (2) et comme, afin de ne rien découvrir, ils ne peuvent se donner des otages, ils demandent que les alliés engagent leur parole, et sur les étendards réunis (cérémonie qui, dans leurs moeurs, est ce qu’il y a de plus sacré), on leur jure de ne pas les abandonner, quand ils se seront déclarés." (3) On comble d’éloges les Carnutes ; tous ceux qui sont présents prêtent le serment exigé ; on fixe le jour pour l’exécution, et l’assemblée se sépare.
\subsection[{§ 3.}]{ \textsc{§ 3.} }
\noindent (1) Ce jour arrivé, les Carnutes, sous les ordres de Cotuatos et de Conconnétodumnos, hommes déterminés à tout, se jettent, à un signal donné, dans Cénabum (Orléans), massacrent les citoyens romains qui s’y trouvaient pour affaires de commerce, entre autres C. Fusius Cita, estimable chevalier romain, que César avait mis à la tête des vivres, et ils pillent tous leurs biens. (2) La nouvelle en parvient bientôt à toutes les cités de la Gaule ; car, dès qu’il arrive quelque chose de remarquable et d’intéressant, les Gaulois l’apprennent par des cris à travers les campagnes et d’un pays à l’autre. Ceux qui les entendent les transmettent aux plus proches comme on fit alors. (3) En effet, la première veille n’était pas encore écoulée que les Arvernes savaient ce qui s’était passé à Cénabum au lever du soleil, c’est-à-dire à cent soixante milles environ de chez eux.
\subsection[{§ 4.}]{ \textsc{§ 4.} }
\noindent (1) Là, dans le même but, un jeune Arverne très puissant, Vercingétorix, fils de Celtillos, qui avait tenu le premier rang dans la Gaule, et que sa cité avait fait mourir parce qu’il visait à la royauté, assemble ses clients et les échauffe sans peine. (2) Dès que l’on connaît son dessein, on court aux armes ; son oncle Gobannitio, et les autres chefs qui ne jugeaient pas à propos de courir une pareille chance, le chassent de la ville de Gergovie. (3) Cependant il ne renonce pas à son projet, et lève dans la campagne un corps de vagabonds et de misérables. Suivi de cette troupe, il amène à ses vues tous ceux de la cité qu’il rencontre ; (4) il les exhorte à prendre les armes pour la liberté commune. Ayant ainsi réuni de grandes forces, il expulse à son tour du pays les adversaires qui, peu de temps auparavant, l’avaient chassé lui-même. On lui donne le titre de roi, (5) et il envoie des députés réclamer partout l’exécution des promesses que l’on a faites. (6) Bientôt il entraîne les Sénons, les Parisii, les Pictons, les Cadurques, les Turons, les Aulerques, les Lémovices, les Andes, et tous les autres peuples qui bordent l’océan : tous s’accordent à lui déférer le commandement. (7) Revêtu de ce pouvoir, il exige des otages de toutes les cités, donne ordre qu’on lui amène promptement un certain nombre de soldats, (8) et règle ce que chaque cité doit fabriquer d’armes, et l’époque où elle les livrera. Surtout il s’occupe de la cavalerie ; (9) à l’activité la plus grande il joint la plus grande sévérité ; il détermine les incertains par l’énormité des châtiments ; (10) un délit grave est puni par le feu et par toute espèce de tortures ; pour les fautes légères il fait couper les oreilles ou crever un oeil, et renvoie chez eux les coupables pour servir d’exemple et pour effrayer les autres par la rigueur du supplice.
\subsection[{§ 5.}]{ \textsc{§ 5.} }
\noindent (1) Après avoir, par ces moyens violents, rassemblé bientôt une armée, il en envoie une partie chez les Rutènes, sous les ordres de Luctérios, du pays des Cadurques, et lui-même va chez les Bituriges. (2) À son approche, ceux-ci députent vers les Héduens dont ils étaient les clients, et leur demandent des secours pour mieux résister aux forces de l’ennemi. (3) Les Héduens, de l’avis des lieutenants que César avait laissés à l’armée, leur envoient de l’infanterie et de la cavalerie. (4) Arrivées à la Loire qui sépare les Bituriges des Héduens, ces troupes s’y arrêtèrent quelques jours et revinrent sans avoir osé la passer. (5) Les chefs dirent à nos lieutenants qu’ils étaient revenus sur leurs pas, craignant une perfidie de la part des Bituriges dont ils avaient appris que le dessein était, s’ils passaient le fleuve, de tomber sur eux d’un côté, tandis que les Arvernes les attaqueraient de l’autre. (6) Est-ce par le motif allégué aux lieutenants ou par trahison que les Héduens en agirent ainsi  ? c’est ce qu’on ne peut décider, n’y ayant rien de positif à cet égard. (7) Après leur départ, les Bituriges se rejoignirent aux Arvernes.
\subsection[{§ 6.}]{ \textsc{§ 6.} }
\noindent (1) Lorsque César apprit ces événements en Italie, il savait déjà que, grâce aux talents de Cn. Pompée, les affaires avaient pris un meilleur aspect à Rome ; il partit donc pour la Gaule transalpine. (2) En arrivant, il se trouva fort embarrassé sur le moyen de rejoindre son armée ; (3) car s’il faisait venir ses légions dans la province, elles auraient dans la marche à combattre sans lui ; (4) que s’il essayait de les aller trouver, il n’était pas prudent de confier sa personne même à un peuple qui à cette époque paraissait soumis.
\subsection[{§ 7.}]{ \textsc{§ 7.} }
\noindent (1) Cependant le Cadurque Luctérios, envoyé chez les Rutènes, les attire au parti des Arvernes, (2) va de là chez les Nitiobroges et les Gabales, qui lui donnent les uns et les autres des otages ; puis, à la tête d’une nombreuse armée, il marche pour envahir la Province du côté de Narbonne. (3) À cette nouvelle, César crut devoir préférablement à tout partir pour cette Province. (4) Il y arrive, rassure les peuples effrayés, établit des postes chez ceux des Rutènes, qui dépendent de la province, chez les Volques Arécomiques, chez les Tolosates et autour de Narbonne, lieux qui tous étaient voisins de l’ennemi. (5) En même temps, il donne ordre à une partie des troupes de la province, et au renfort qu’il avait amené de l’Italie, de se réunir chez les Helviens, qui sont limitrophes des Arvernes.
\subsection[{§ 8.}]{ \textsc{§ 8.} }
\noindent (1) Ces choses ainsi disposées, et Luctérios s’étant arrêté et même retiré parce qu’il crut dangereux de s’engager au milieu de ces différents corps de troupes, César se rendit chez les Helviens, (2) quoique dans cette saison, la plus rigoureuse de l’année, la neige encombrât les chemins des Cévennes, montagnes qui séparent les Helviens des Arvernes. Cependant à force de travail, en faisant écarter par le soldat la neige épaisse de six pieds, César s’y fraie un chemin et parvient sur la frontière des Arvernes. (3) Tombant sur eux au moment où ils ne s’y attendaient pas, parce qu’ils se croyaient défendus par les Cévennes comme par un mur, et que dans cette saison les sentiers n’en avaient jamais été praticables même pour un homme seul, il ordonne à sa cavalerie d’étendre ses courses aussi loin qu’il lui sera possible, afin de causer aux ennemis un plus grand effroi. (4) La renommée et des courriers en informent bientôt Vercingétorix. Tous les Arvernes éperdus l’entourent et le conjurent de penser à leurs intérêts, de ne pas laisser ravager leurs propriétés, maintenant que toute la guerre s’est portée chez eux. (5) Il cède à leurs prières, décampe, quitte le pays des Bituriges, pour se rapprocher de celui des Arvernes.
\subsection[{§ 9.}]{ \textsc{§ 9.} }
\noindent (1) César ne s’arrêta dans le pays que deux jours, prévoyant le parti que prendrait Vercingétorix ; et il quitta l’armée, sous le prétexte de rassembler des renforts et de la cavalerie. (2) Il laisse le commandement des troupes au jeune Brutus, et lui recommande de pousser en tous sens et le plus loin possible des partis de cavalerie ; il aura soin de ne pas être absent du camp plus de trois jours. (3) Les choses ainsi réglées, il arrive en toute diligence à Vienne, sans y être attendu. (4) Il y trouve la nouvelle cavalerie qu’il avait envoyée depuis plusieurs jours, et sans s’arrêter ni de jour ni de nuit, il se rend à travers le pays des Héduens chez les Lingons, où deux légions étaient en quartiers d’hiver : il voulait, si les Héduens avaient eux-mêmes des desseins contre sa personne, en prévenir l’effet par sa célérité. (5) Arrivé chez les Lingons, il dépêche des courriers aux autres légions, et les réunit toutes avant que les Arvernes puissent être instruits de sa marche. (6) À cette nouvelle, Vercingétorix ramène son armée chez les Bituriges, et de là va mettre le siège devant Gergovie, ville des Boïens, que César, après les avoir vaincus dans la même bataille que les Helvètes, avait établie sous la dépendance des Héduens.
\subsection[{§ 10.}]{ \textsc{§ 10.} }
\noindent (1) Cette entreprise mettait César dans un grand embarras. Quel parti prendrait-il ? Si, pendant le reste de l’hiver, il tenait les légions réunies sur un seul point, il craignait que la prise d’une ville tributaire des Héduens ne le fit abandonner de toute la Gaule, parce que l’on verrait que ses amis ne pouvaient compter sur sa protection ; s’il entrait en campagne plus tôt que de coutume, la difficulté des transports pouvait le faire souffrir du côté des vivres. (2) Cependant il crut plus à propos de s’exposer à tous ces inconvénients que d’essuyer un affront propre à aliéner les esprits de tous ses alliés. (3) Ayant donc engagé les Héduens à lui envoyer des vivres, il fait prévenir les Boïens de sa marche, et les exhorte à rester fidèles et à soutenir vigoureusement l’attaque des ennemis. (4) Laissant à Agédincum deux légions avec les bagages de toute l’armée, il se dirige vers les Boïens.
\subsection[{§ 11.}]{ \textsc{§ 11.} }
\noindent (1) Le lendemain, étant arrivé à Vellaunodunum, ville des Sénons, et ne voulant pas laisser d’ennemi derrière lui pour que les vivres circulassent librement, il résolut d’en faire le siège, et en acheva la circonvallation en deux jours. (2) Le troisième jour, la ville envoya des députés pour se rendre ; et il fut ordonné aux assiégés d’apporter leurs armes, de livrer leurs chevaux et de donner six cents otages. (3) César laisse, pour faire exécuter le traité, le lieutenant C. Trébonius ; et, sans perdre de temps, il marche sur Cénabum, ville des Carnutes, (4) qui tout récemment instruits du siège de Vellaunodunum, et croyant qu’il durerait plus longtemps, rassemblaient des troupes qu’ils devaient envoyer au secours de la première ville. (5) César y arrive le second jour, et établit son camp devant la place ; mais l’approche de la nuit le force de remettre l’attaque au lendemain : il ordonne aux soldats de tenir prêt tout ce qu’il faut en pareil cas ; et, (6) comme la ville de Cénabum avait un pont sur la Loire, dans la crainte que les habitants ne s’échappent la nuit, il fait veiller deux légions sous les armes. (7) Un peu avant minuit les assiégés sortent en silence, et commencent à passer le fleuve. (8) César, averti par les éclaireurs, met le feu aux portes, fait entrer les légions qui avaient reçu l’ordre d’être prêtes, et s’empare de la place. Très peu d’ennemis échappèrent ; presque tous furent pris, parce que le peu de largeur du pont et des issues arrêta la multitude dans sa fuite. (9) César pille et brûle la ville, abandonne le butin aux soldats, fait passer la Loire à l’armée, et arrive sur le territoire des Bituriges.
\subsection[{§ 12.}]{ \textsc{§ 12.} }
\noindent (1) Vercingétorix, à la nouvelle de l’approche de César, lève le siège et part au-devant de lui. (2) Celui-ci avait résolu d’assiéger Noviodunum, ville des Bituriges, placée sur sa route. (3) Des députés en étaient sortis pour le prier de leur pardonner et de leur conserver la vie ; César, pour terminer l’expédition avec cette promptitude qui avait fait le succès de ses précédentes, leur ordonne de lui apporter les armes, de lui amener les chevaux, de lui donner des otages. (4) Une partie des otages avait déjà été livrée, et le reste du traité s’exécutait en présence des centurions et de quelques soldats qu’on avait introduits dans la place pour recueillir les armes et les chevaux, lorsqu’on aperçut au loin la cavalerie des ennemis qui précédait l’armée de Vercingétorix. (5) Dès qu’ils l’aperçoivent et qu’ils ont l’espoir d’être secourus, les habitants poussent un cri et commencent à prendre les armes, à fermer les portes et à border le rempart. (6) Les centurions qui étaient dans la ville, comprenant aux mouvements des Gaulois qu’ils trament quelque nouveau dessein s’emparent des portes l’épée à la main et se retirent sans perte ainsi que tous leurs soldats.
\subsection[{§ 13.}]{ \textsc{§ 13.} }
\noindent (1) César fait sortir du camp sa cavalerie et engage le combat avec celle des Gaulois. La nôtre commençant à plier, il la fait soutenir par environ six cents cavaliers germains qu’il s’était attachés depuis le commencement de la guerre. (2) Les Gaulois ne purent soutenir leur choc, prirent la fuite et se replièrent sur leur armée avec beaucoup de pertes. Cette déroute ayant jeté de nouveau la terreur dans la ville, les habitants saisirent ceux qu’ils crurent avoir excité le peuple, les amenèrent à César et se rendirent à lui. (3) Cette affaire terminée, César marcha sur Avaricum (Bourges), la plus grande et la plus forte place des Bituriges, et située sur le territoire le plus fertile ; il espérait que la prise de cette ville le rendrait maître de tout le pays.
\subsection[{§ 14.}]{ \textsc{§ 14.} }
\noindent (1) Vercingétorix, après tant de revers essuyés successivement à Vellaunodunum, à Cénabum, à Noviodunum, convoque un conseil. (2) Il démontre "que cette guerre doit être conduite tout autrement qu’elle ne l’a été jusqu’alors ; qu’il faut employer tous les moyens pour couper aux Romains les vivres et le fourrage ; (3) que cela sera aisé, puisque l’on a beaucoup de cavalerie et qu’on est secondé par la saison ; (4) que, ne trouvant pas d’herbes à couper, les ennemis seront contraints de se disperser pour en chercher dans les maisons, et que la cavalerie pourra chaque jour les détruire ; (5) qu’enfin le salut commun doit faire oublier les intérêts particuliers ; qu’il faut incendier les bourgs et les maisons en tout sens depuis Boïe, aussi loin que l’ennemi peut s’étendre pour fourrager. (6) Pour eux, ils auront tout en abondance, étant secourus par les peuples sur le territoire desquels aura lieu la guerre ; (7) les Romains ne pourront soutenir la disette ou s’exposeront à de grands périls en sortant de leur camp ; (8) il importe peu de les tuer ou de leur enlever leurs bagages, dont la perte leur rend la guerre impossible. (9) Il faut aussi brûler les villes qui par leurs fortifications ou par leur position naturelle ne seraient pas à l’abri de tout danger, afin qu’elles ne servent ni d’asile aux Gaulois qui déserteraient leurs drapeaux, ni de but aux Romains qui voudraient y enlever des vivres et du butin. (10) Si de tels moyens semblent durs et rigoureux, ils doivent trouver plus dur encore de voir leurs enfants, leurs femmes, traînés en esclavage, et de périr eux-mêmes, sort inévitable des vaincus."
\subsection[{§ 15.}]{ \textsc{§ 15.} }
\noindent (1) Cet avis étant unanimement approuvé, on brûle en un jour plus de vingt villes des Bituriges. On fait la même chose dans les autres pays. (2) De toutes parts on ne voit qu’incendies : ce spectacle causait une affliction profonde et universelle, mais on s’en consolait par l’espoir d’une victoire presque certaine, qui indemniserait promptement de tous les sacrifices. (3) On délibère dans l’assemblée générale s’il convient de brûler ou de défendre Avaricum. (4) Les Bituriges se jettent aux pieds des autres Gaulois: "Qu'on ne les force pas à brûler de leurs mains la plus belle ville de presque toute la Gaule, le soutien et l’ornement de leur pays ; (5) ils la défendront facilement, disent-ils, vu sa position naturelle ; car presque de toutes parts entourée d’une rivière et d’un marais, elle n’a qu’une avenue très étroite." (6) Ils obtiennent leur demande ; Vercingétorix, qui l’avait d’abord combattue, cède enfin à leurs prières et à la pitié générale. La défense de la place est confiée à des hommes choisis à cet effet.
\subsection[{§ 16.}]{ \textsc{§ 16.} }
\noindent (1) Vercingétorix suit César à petites journées, et choisit pour son camp un lieu défendu par des marais et des bois, à seize mille pas d’Avaricum. (2) Là, des éclaireurs fidèles l’instruisaient à chaque instant du jour de ce qui se passait dans Avaricum, et y transmettaient ses volontés. (3) Tous nos mouvements pour chercher des grains et des fourrages étaient épiés ; et si nos soldats se dispersaient ou s’éloignaient trop du camp, il les attaquait et leur faisait beaucoup de mal, quoiqu’on prît toutes les précautions possibles pour sortir à des heures incertaines et par des chemins différents.
\subsection[{§ 17.}]{ \textsc{§ 17.} }
\noindent (1) Après avoir assis son camp dans cette partie de la ville qui avait, comme on l’a dit plus haut, une avenue étroite entre la rivière et le marais, César fit commencer une terrasse, pousser des mantelets, et travailler à deux tours ; car la nature du lieu s’opposait à une circonvallation. (2) Il ne cessait d’insister auprès des Boïens et des Héduens pour les vivres ; mais le peu de zèle de ces derniers les lui rendait comme inutiles, et la faible et petite cité des Boïens eut bientôt épuisé ses ressources. (3) L'extrême difficulté d’avoir des vivres, due à la pauvreté des Boïens, à la négligence des Héduens et à l’incendie des habitations, fit souffrir l’armé e au point qu’elle manqua de blé pendant plusieurs jours, et qu’elle n’eut, pour se garantir de la famine, que le bétail enlevé dans les bourgs très éloignés. Cependant on n’entendit pas un mot indigne de la majesté du peuple romain ni des victoires précédentes. (4) Bien plus, comme César, visitant les travaux, s’adressait à chaque légion en particulier, et leur disait que si cette disette leur semblait trop cruelle, il lèverait le siège, (5) tous le conjurèrent de n’en rien faire : "Depuis nombre d’années, disaient-ils, qu’ils servaient sous ses ordres, jamais ils n’avaient reçu d’affront ni renoncé à une entreprise sans l’avoir exécutée ; (6) ils regardaient comme un déshonneur d’abandonner un siège commencé : (7) il valait mieux endurer toutes les extrémités que de ne point venger les citoyens romains égorgés à Cénabum par la perfidie des Gaulois." (8) Ils le répétaient aux centurions et aux tribuns militaires pour qu’ils le rapportassent à César.
\subsection[{§ 18.}]{ \textsc{§ 18.} }
\noindent (1) Déjà les tours approchaient du rempart quand des prisonniers apprirent à César que Vercingétorix, après avoir consommé ses fourrages, avait rapproché son camp d’Avaricum, et qu’avec sa cavalerie et son infanterie légère habituée à combattre entre les chevaux, il était parti lui-même pour dresser une embuscade à l’endroit où il pensait que nos fourrageurs iraient le lendemain. (2) D'après ces renseignements, César partit en silence au milieu de la nuit, et arriva le matin près du camp des ennemis. (3) Ceux-ci, promptement avertis de son approche par leurs éclaireurs, cachèrent leurs chariots et leurs bagages dans l’épaisseur des forêts, et mirent toutes leurs forces en bataille sur un lieu élevé et découvert. (4) César, à cette nouvelle, ordonna de déposer les sacs et de préparer les armes.
\subsection[{§ 19.}]{ \textsc{§ 19.} }
\noindent (1) La colline était en pente douce depuis sa base ; un marais large au plus de cinquante pieds l’entourait presque de tous côtés et en rendait l’accès difficile et dangereux. (2) Les Gaulois, après avoir rompu les ponts, se tenaient sur cette colline, pleins de confiance dans leur position ; et ; rangés par familles et par cités, ils avaient placé des gardes à tous les gués et au détour du marais, et étaient disposés, si les Romains tentaient de le franchir, à profiter de l’élévation de leur poste pour les accabler au passage. (3) À ne voir que la proximité des distances, on aurait cru l’ennemi animé d’une ardeur presque égale à la nôtre ; à considérer l’inégalité des positions, on reconnaissait que ses démonstrations n’étaient qu’une vaine parade. (4) Indignés qu’à si peu de distance il pût soutenir leur aspect, nos soldats demandaient le signal du combat ; César leur représente "par combien de sacrifices, par la mort de combien de braves, il faudrait acheter la victoire ; (5) il serait le plus coupable des hommes si, disposés comme ils le sont à tout braver pour sa gloire, leur vie ne lui était pas plus chère que la sienne." (6) Après les avoir ainsi consolés, il les ramène le même jour au camp, voulant achever tous les préparatifs qui regardaient le siège.
\subsection[{§ 20.}]{ \textsc{§ 20.} }
\noindent (1) Vercingétorix, de retour près des siens, fut accusé de trahison, pour avoir rapproché son camp des Romains, pour s’être éloigné avec toute la cavalerie, pour avoir laissé sans chef des troupes si nombreuses, et parce qu’après son départ les Romains étaient accourus si à propos et avec tant de promptitude. (2) "Toutes ces circonstances ne pouvaient être arrivées par hasard et sans dessein de sa part, il aimait mieux tenir l’empire de la Gaule de l’agrément de César que de la reconnaissance de ses compatriotes." (3) Il répondit à ces accusations "qu’il avait levé le camp faute de fourrage et sur leurs propres instances ; qu’il s’était approché des Romains, déterminé par l’avantage d’une position qui se défendait par elle-même ; (4) qu’on n’avait pas dû sentir le besoin de la cavalerie dans un endroit marécageux, et qu’elle avait été utile là où il l’avait conduite. (5) C'était à dessein qu’en partant il n’avait remis le commandement à personne, de peur qu’un nouveau chef, pour plaire à la multitude, ne consentît à engager une action ; il les y savait tous portés par cette faiblesse qui les rendait incapables de souffrir plus longtemps les fatigues ; (6) si les Romains étaient survenus par hasard, il fallait en remercier la fortune, et, si quelque trahison les avaient appelés, rendre grâce au traître, puisque du haut de la colline on avait pu reconnaître leur petit nombre et apprécier le courage de ces hommes qui s’étaient honteusement retirés dans leur camp, sans oser combattre. (7) Il ne désirait pas obtenir de César par une trahison une autorité qu’il pouvait obtenir par une victoire qui n’était plus douteuse à ses yeux ni à ceux des Gaulois ; mais il est prêt à s’en démettre, s’ils s’imaginent plutôt lui faire honneur que lui devoir leur salut ; (8) "et pour que vous sachiez ;" dit-il, "que je parle sans feinte, écoutez des soldats romains." (9) II produit des esclaves pris quelques jours auparavant parmi les fourrageurs et déjà exténués par les fers et par la faim. (10) Instruits d’avance de ce qu’ils doivent répondre, ils disent qu’ils sont des soldats légionnaires ; que, poussés par la faim et la misère, ils étaient sortis en secret du camp pour tâcher de trouver dans la campagne du blé ou du bétail ; (11) que toute l’armée éprouvait la même disette ; que les soldats étaient sans vigueur et ne pouvaient plus soutenir la fatigue des travaux ; que le général avait en conséquence résolu de se retirer dans trois jours, s’il n’obtenait pas quelque succès dans le siège. (12) "Voilà, " reprend Vercingétorix, "les services que je vous ai rendus, moi que vous accusez de trahison, moi dont les mesures ont, comme vous le voyez, presque détruit par la famine et sans qu’il nous en coûte de sang, une armée nombreuse et triomphante ; moi qui ai pourvu à ce que, dans sa fuite honteuse, aucune cité ne la reçoive sur sont territoire."
\subsection[{§ 21.}]{ \textsc{§ 21.} }
\noindent (1) Un cri général se fait entendre avec un cliquetis d’armes, démonstration ordinaire aux Gaulois quand un discours leur a plu. Vercingétorix est leur chef suprême ; sa fidélité n’est point douteuse ; on ne saurait conduire la guerre avec plus d’habileté. (2) Ils décident qu’on enverra dans la ville dix mille hommes choisis dans toute l’armée : (3) ils ne veulent pas confier le salut commun aux seuls Bituriges, qui, s’ils conservaient la place, ne manqueraient pas de s’attribuer tout l’honneur de la victoire.
\subsection[{§ 22.}]{ \textsc{§ 22.} }
\noindent (1) À la valeur singulière de nos soldats, les Gaulois opposaient des inventions de toute espèce ; car cette nation est très industrieuse et très adroite à imiter et à exécuter tout ce qu’elle voit faire. (2) Ils détournaient nos faux avec des lacets, et lorsqu’ils les avaient saisies, ils les attiraient à eux avec des machines. Ils ruinaient notre terrasse, en la minant avec d’autant plus d’habileté qu’ayant des mines de fer considérables, ils connaissent et pratiquent toutes sortes de galeries souterraines. (3) Ils avaient de tous côtés garni leur muraille de tours recouvertes de cuir. (4) Faisant de jour et de nuit de fréquentes sorties, tantôt ils mettaient le feu aux ouvrages, tantôt ils tombaient sur les travailleurs. L'élévation que gagnaient nos tours par l’accroissement journalier de la terrasse, (5) ils la donnaient aux leurs, en y ajoutant de longues poutres liées ensemble ; ils arrêtaient nos mines avec des pieux aigus, brûlés par le bout, de la poix bouillante, d’énormes quartiers de rochers, et nous empêchaient ainsi de les approcher des remparts.
\subsection[{§ 23.}]{ \textsc{§ 23.} }
\noindent (1) Telle est à peu près la forme des murailles dans toute la Gaule : à la distance régulière de deux pieds, on pose sur leur longueur des poutres d’une seule pièce ; (2) on les assujettit intérieurement entre elles, et on les revêt de terre foulée. Sur le devant, on garnit de grosses pierres les intervalles dont nous avons parlé. (3) Ce rang ainsi disposé et bien lié, on en met un second en conservant le même espace, de manière que les poutres ne se touchent pas, mais que, dans la construction, elles se tiennent à une distance uniforme, un rang de pierres entre chacune. (4) Tout l’ouvrage se continue ainsi, jusqu’à ce que le mur ait atteint la hauteur convenable. (5) Non seulement une telle construction, formée de rangs alternatifs de poutres et de pierres, n’est point, à cause de cette variété même, désagréable à l’oeil ; mais elle est encore d’une grande utilité pour la défense et la sûreté des villes ; car la pierre protége le mur contre l’incendie, et le bois contre le bélier ; et on ne peut renverser ni même entamer un enchaînement de poutres de quarante pieds de long, la plupart liées ensemble dans l’intérieur.
\subsection[{§ 24.}]{ \textsc{§ 24.} }
\noindent (1) Quoique l’on rencontrât tous ces obstacles, et que le froid et les pluies continuelles retardassent constamment les travaux, le soldat, s’y livrant sans relâche ; surmonta tout, et en vingt-cinq jours, il éleva un terrasse large de trois cent trente pieds, et haute de quatre-vingts. (2) Déjà elle touchait presque au mur de la ville, et César qui, suivant sa coutume, passait la nuit dans les ouvrages, exhortait les soldats à ne pas interrompre au seul instant leur travail, quand un peu avant la troisième veille, on vit de la fumée sortir de la terrasse, à laquelle les ennemis avaient mis le feu par une mine. (3) Dans le même instant, aux cris qui s’élevèrent le long du rempart, les Barbares firent une sortie par deux portes, des deux côtés des tours. (4) Du haut des murailles, les uns lançaient sur la terrasse des torches et du bois sec, d’autres y versaient de la poix et des matières propres à rendre le feu plus actif, en sorte qu’on pouvait à peine savoir où se porter et à quoi remédier d’abord. (5) Cependant, comme César avait ordonné que deux légions fussent toujours sous les armes en avant du camp, et que plusieurs autres étaient dans les ouvrages, où elles se relevaient à des heures fixes, on put bientôt, d’une part, faire face aux sorties, de l’autre retirer les tours et couper la terrasse pour arrêter le feu ; enfin toute l’armée accourut du camp pour l’éteindre.
\subsection[{§ 25.}]{ \textsc{§ 25.} }
\noindent (1) Le reste de la nuit s’était écoulé, et l’on combattait encore sur tous les points ; les ennemis étaient sans cesse ranimés par l’espérance de vaincre, avec d’autant plus de sujet, qu’ils voyaient les mantelets de nos tours brûlés, et sentaient toute la difficulté d’y porter secours à découvert ; qu’à tous moments ils remplaçaient par des troupes fraîches celles qui étaient fatiguées, et qu’enfin le salut de toute la Gaule leur semblait dépendre de ce moment unique. Nous fûmes alors témoins d’un trait que nous croyons devoir consigner ici, comme digne de mémoire. (2) Devant la porte de la ville était un Gaulois, à qui l’on passait de main en main des boules de suif et de poix, qu’il lançait dans le feu du haut d’une tour. Un trait de scorpion lui perce le flanc droit ; il tombe mort. (3) Un de ses plus proches voisins passe par-dessus le cadavre et remplit la même tâche ; il est atteint à son tour et tué de la même manière ; un troisième lui succède ; à celui-ci un quatrième ; (4) et le poste n’est abandonné que lorsque le feu de la terrasse est éteint et que la retraite des ennemis partout repoussés a mis fin au combat.
\subsection[{§ 26.}]{ \textsc{§ 26.} }
\noindent (1) Après avoir tout tenté sans réussir en rien, les Gaulois, sur les instances et l’ordre de Vercingétorix, résolurent le lendemain d’évacuer la place. (2) Ils espéraient le faire dans le silence de nuit, sans éprouver de grandes pertes, parce que le camp de Vercingétorix n’était pas éloigné de la ville, et qu’un vaste marais, les séparant des Romains, retarderait ceux-ci dans leur poursuite. (3) Déjà, la nuit venue, ils se préparaient à partir, lorsque tout à coup les mères de famille sortirent de leurs maisons, et se jetèrent, tout éplorées, aux pieds de leurs époux et de leurs fils, les conjurant de ne point les livrer à la cruauté de l’ennemi elles et leurs enfants, que leur âge et leur faiblesse empêchaient de prendre la fuite. (4) Mais comme ils persistaient dans leur dessein, tant la crainte d’un péril extrême étouffe souvent la pitié, ces femmes se mirent à pousser des cris pour avertir les Romains de cette évasion. (5) Les Gaulois effrayés craignant que la cavalerie romaine ne s’emparât des passages, renoncèrent à leur projet.
\subsection[{§ 27.}]{ \textsc{§ 27.} }
\noindent (1) Le lendemain, tandis que César faisait avancer une tour, et dirigeait les ouvrages qu’il avait projetés, il survint une pluie abondante. Il croit que ce temps favoriserait une attaque soudaine, et remarquant que la garde se faisait un peu plus négligemment sur les remparts, il ordonne aux siens de ralentir leur travail, et leur fait connaître ses intentions. (2) II exhorte les légions qu’il tenait toutes prêtes derrière les mantelets à recueillir enfin dans la victoire le prix de tant de fatigues ; il promet des récompenses aux premiers qui escaladeront la muraille, et donne le signal. (3) Ils s’élancent aussitôt de tous les côtés et couvrent bientôt le rempart.
\subsection[{§ 28.}]{ \textsc{§ 28.} }
\noindent (1) Consternés de cette attaque imprévue, renversés des murs et des tours, les ennemis se forment en coin sur la place publique et dans les endroits les plus spacieux, résolus à se défendre en bataille rangée, de quelque côté que l’on vienne à eux. (2) Voyant qu’aucun Romain ne descend, mais que l’ennemi se répand sur toute l’enceinte du rempart, ils craignent qu’on ne leur ôte tout moyen de fuir ; ils jettent leurs armes, et gagnent d’une course les extrémités de la ville. (4) Là, comme ils se nuisaient à eux-mêmes dans l’étroite issue des portes, nos soldats en tuèrent une partie ; une autre déjà sortie fut massacrée par la cavalerie ; (4) personne ne songeait au pillage. Animés par le carnage de Genabum, et par les fatigues du siège, les soldats n’épargnèrent ni les vieillards, ni les femmes, ni les enfants. (5) Enfin de toute cette multitude qui se montait à environ quarante mille individus, à peine en arriva-t-il sans blessures auprès de Vercingétorix, huit cents qui s’étaient, au premier cri, jetés hors de la ville. (6) II les recueillit au milieu de la nuit en silence ; car il craignait, s’ils arrivaient tous ensemble, que la pitié n’excitât quelque sédition dans le camp ; et, à cet effet, il avait eu soin de disposer au loin sur la route ses amis et les principaux chefs des cités pour les séparer et les conduire chacun dans la partie du camp qui, dès le principe, avait été affectée à leur nation.
\subsection[{§ 29.}]{ \textsc{§ 29.} }
\noindent (1) Le lendemain, il convoqua l’armée, la consola, et l’exhorta à ne se laisser ni abattre, ni décourager à l’excès par un revers. (2) "Les Romains n’ont point vaincu par la valeur et en bataille rangée, mais par un art et une habileté dans les sièges, inconnus aux Gaulois ; (3) on se tromperait si on ne s’attendait, à la guerre, qu’à des succès ; (4) il n’avait jamais été d’avis de défendre Avaricum ; ils en sont témoins : cependant cette perte due à la témérité des Bituriges et au trop de complaisance des autres cités, (5) il la réparera bientôt par des avantages plus considérables. (6) Car les peuples qui n’étaient pas du parti du reste de la Gaule, il les y amènera par ses soins ; et la Gaule entière n’aura qu’un but unique, auquel l’univers même s’opposerait en vain. Il a déjà presque réussi. (7) Il était juste néanmoins qu’il obtint d’eux, au nom du salut commun, de prendre la méthode de retrancher leur camp, pour résister plus facilement aux attaques subites de l’ennemi."
\subsection[{§ 30.}]{ \textsc{§ 30.} }
\noindent (1) Ce discours ne déplut pas aux Gaulois, surtout parce qu’un si grand échec n’avait pas abattu son courage, et qu’il ne s’était pas caché pour se dérober aux regards de l’armée. (2) On lui trouvait d’autant plus de prudence et de prévoyance, que quand rien ne périclitait encore, il avait proposé de brûler Avaricum, ensuite de l’évacuer. (3) Ainsi, tandis que les revers ébranlent le crédit des autres généraux, son pouvoir depuis celui qu’il avait éprouvé s’accrut au contraire de jour en jour. (4) En même temps ils se flattaient, sur sa parole, d’être bientôt secondés par les autres cités. Les Gaulois commencèrent alors pour la première fois à retrancher leur camp ; et telle était leur consternation, que ces hommes, inaccoutumés au travail, crurent devoir se soumettre à tout ce qu’on leur commandait.
\subsection[{§ 31.}]{ \textsc{§ 31.} }
\noindent (1) Vercingétorix travailla, suivant sa promesse, à réunir à son alliance les autres cités ; et il en gagna les chefs par des présents et par des promesses. (2) Il choisit pour cette mission des agents adroits et prodigues de belles paroles, aux avances desquels on pouvait aisément se laisser prendre. (3) Il a soin de fournir des vêtements et des armes aux réfugiés d’Avaricum. (4) En même temps, pour compléter ses troupes affaiblies, il commande aux cités l’envoi d’un certain nombre d’hommes, fixe le jour où ils doivent être arrivés, et donne ordre de rechercher et de lui envoyer tous les archers, qui sont très nombreux dans la Gaule. Il a bientôt ainsi remplacé ce qui avait péri dans Avaricum. (5) Dans l’intervalle, Teutomatos, fils d’Ollovico, roi des Nitiobroges, dont le père avait reçu de notre sénat le titre d’ami, était venu le joindre avec un corps considérable de cavalerie levé dans son pays et dans l’Aquitaine.
\subsection[{§ 32.}]{ \textsc{§ 32.} }
\noindent (1) César demeura encore plusieurs jours dans Avaricum, où il trouva beaucoup de blé et d’autres vivres, et où l’armée se remit de ses fatigues. (2) L'hiver étant à sa fin, et la saison même l’appelant en campagne, il avait résolu de marcher à l’ennemi, soit pour l’attirer hors de ses marais et de ses bois, soit pour l’y assiéger, lorsque les principaux des Héduens vinrent en députation auprès de lui, pour le prier d’accorder son secours à leur cité dans une circonstance extrêmement urgente. (3) "L'état, disaient-ils, était dans le plus grand danger ; car, tandis que de tout temps on n’avait créé qu’un magistrat unique qui jouissait de l’autorité suprême pendant une seule année, il y en avait en ce moment deux qui se disaient l’un et l’autre nommés suivant les lois. (4) L'un était Convictolitavis, jeune homme d’une naissance illustre ; l’autre, Cotos, issu d’une très ancienne famille, très puissant par lui-même et par ses grandes alliances ; son frère Valétiacos avait, l’année précédente, rempli la même magistrature. (5) Toute la nation était en armes, le sénat partagé, le peuple divisé, chacun à la tête de ses clients. Si ce différend se prolongeait, la guerre civile était imminente, malheur que préviendraient la promptitude et l’autorité de César.
\subsection[{§ 33.}]{ \textsc{§ 33.} }
\noindent (1) Quoiqu’il crût préjudiciable d’abandonner la guerre et l’ennemi, cependant, comme il savait combien d’inconvénients entraînent de pareilles dissensions, il partit dans la crainte qu’une cité si importante et si étroitement unie au peuple romain, une cité qu’il avait toujours protégée et comblée d’honneurs, n’en vînt aux violences et aux armes, et que le parti qui se croirait le moins fort n’appelât Vercingétorix à son secours ; et il voulait prévenir ce danger. (2) Comme les lois des Héduens ne permettaient pas au souverain magistrat de sortir du territoire, César, ne voulant paraître enfreindre en rien ni leur droit ni leurs lois, prit le parti d’aller lui-même dans leur pays, et cita devant lui, à Décétia (Décize), tout le sénat et les deux prétendants. (3) Presque toute la cité s’y rassembla ; il apprit de quelques personnes appelées en secret, que le frère avait proclamé son frère, dans un temps et dans un lieu contraires aux institutions ; les lois défendaient non seulement de créer magistrats, mais même d’admettre dans le sénat deux personnes de la même famille, du vivant de l’une ou de l’autre. César obligea donc Cotos de se démettre de sa magistrature, (4) et ordonna que le pouvoir fût remis à Convictolitavis que, suivant l’usage de la cité, les prêtres avaient élu avec l’intervention des magistrats.
\subsection[{§ 34.}]{ \textsc{§ 34.} }
\noindent (1) Après cette décision, il engagea les Héduens à oublier leurs querelles et leurs dissensions, pour s’occuper uniquement de la guerre, assurés qu’ils étaient de recevoir, après la soumission de la Gaule, les récompenses qu’ils auraient méritées ; il les chargea de lui envoyer promptement toute leur cavalerie et dix mille fantassins, dont il ferait des détachements pour escorter ses convois. Divisant son armée en deux corps, (2) il donne quatre légions à Labiénus pour aller chez les Sénons et les Parisii ; lui-même, à la tête de six autres légions, il s’avance vers Gergovie, le long de la rivière d’Allier. Il avait donné à Labiénus une partie de la cavalerie, et gardé le reste avec lui. (3) À la nouvelle de la marche de César, Vercingétorix fit aussitôt rompre tous les ponts de la rivière, et remonta l’Allier sur la rive gauche.
\subsection[{§ 35.}]{ \textsc{§ 35.} }
\noindent (1) Les deux armées étaient en présence, les camps presque en face l’un de l’autre ; et les éclaireurs disposés par l’ennemi empêchaient les Romains de construire un pont et de faire passer les troupes. Cette position devenait très embarrassante pour César, qui craignait d’être arrêté une partie de l’été par la rivière, l’Allier n’étant presque jamais guéable avant l’automne. (2) Pour y obvier, il campa dans un lieu couvert de bois, vis-à-vis de l’un des ponts que Vercingétorix avait fait détruire ; et s’y tenant caché le lendemain avec deux légions, (3) il fit partir le reste des troupes avec tous les bagages, dans l’ordre accoutumé, en retenant quelques cohortes ; pour que le nombre des légions parût au complet. (4) II ordonna de faire la plus longue marche possible, et quand il put supposer, d’après le temps écoulé, que l’armée était arrivée au lieu du campement, il se mit à rétablir le pont sur les anciens pilotis, dont la partie inférieure était restée intacte. (5) L'ouvrage fut bientôt achevé : César fit passer les légions, prit une position avantageuse, et rappela les autres troupes. (6) À cette nouvelle, Vercingétorix, pour n’être pas forcé de combattre malgré lui, se porta en avant à grandes journées.
\subsection[{§ 36.}]{ \textsc{§ 36.} }
\noindent (1) De là César parvint en cinq marches à Gergovie ; et le même jour, après une légère escarmouche de cavalerie, il reconnut la position de la ville, qui était assise sur une montagne élevée et d’un accès partout très difficile ; il désespéra de l’enlever de force, et ne voulut s’occuper de ce siège qu’après avoir assuré ses vivres. (2) De son côté, Vercingétorix campait sur une montagne près de la ville, ayant autour de lui, séparément, mais à de faibles distances, les troupes de chaque cité, qui couvrant la chaîne entière des collines, offraient de toutes parts un aspect effrayant. (3) Chaque matin, soit qu’il eût quelque chose à leur communiquer, soit qu’il s’agît de prendre quelque mesure, il faisait, au lever du soleil, venir les chefs dont il avait formé son conseil ; (4) et il ne se passait presque pas de jour que, pour éprouver le courage et l’ardeur de ses troupes, il n’engageât une action avec sa cavalerie entremêlée d’archers. (5) En face de la ville, au pied même de la montagne, était une éminence escarpée de toutes parts et bien fortifiée ; en l’occupant, nous privions probablement l’ennemi d’une grande partie de ses eaux et de la facilité de fourrager ; (6) mais elle avait une garnison, à la vérité un peu faible. (7) César, dans le silence de la nuit, sort de son camp, s’empare du poste, dont il culbute la garde, avant que de la ville on puisse lui envoyer du secours, y met deux légions, et tire du grand au petit camp un double fossé de douze pieds, pour qu’on puisse aller et venir même individuellement, sans crainte d’être surpris par l’ennemi.
\subsection[{§ 37.}]{ \textsc{§ 37.} }
\noindent (1) Tandis que ces choses se passent près de Gergovie, l’Éduen Convictolitavis qui, comme on l’a vu, devait sa magistrature à César, séduit par l’argent des Arvernes, a des entrevues avec plusieurs jeunes gens, à la tête desquels étaient Litaviccos et ses frères, issus d’une illustre famille. (2) Il partage avec eux la somme qu’il a reçue, et les exhorte à se souvenir qu’ils sont nés libres et faits pour commander. (3) "La cité des Héduens retarde seule le triomphe infaillible des Gaulois ; son influence retient les autres nations ; s’ils changent de parti, les Romains ne tiendront point dans la Gaule ; (4) il a quelque obligation à César, qui d’ailleurs n’a été que juste envers lui : mais il doit bien plus à la liberté commune ; (5) car pourquoi les Héduens viendraient-ils discuter leur droit et leurs lois devant César, plutôt que les Romains devant les Héduens ? " (6) Le discours du magistrat et l’appât du gain ont bientôt gagné ces jeunes gens ; ils offrent même de se mettre à la tête de l’entreprise, et on ne songe plus qu’aux moyens de l’exécuter ; car on ne se flattait pas que la nation se laisserait entraîner légèrement à la guerre. (7) On arrêta que Litaviccis prendrait le commandement des dix mille hommes que l’on enverrait à César ; il se chargerait de les conduire, et les frères se rendraient en avant auprès de César. Ils réglèrent ensuite la manière d’agir pour tout le reste.
\subsection[{§ 38.}]{ \textsc{§ 38.} }
\noindent (1) Litaviccos, avec l’armée mise sous ses ordres, n’était plus qu’à trente mille pas environ de Gergovie, quand tout à coup, assemblant les troupes et répandant des larmes : (2) "Où allons-nous, soldats ? leur dit-il ; toute notre cavalerie, toute notre noblesse a péri ; nos principaux citoyens, Éporédorix et Viridomaros, ont été, sous prétexte de trahison, égorgés par les Romains, sans forme de procès. (3) Écoutez ceux qui ont échappé au carnage ; car pour moi, dont les frères et tous les parents ont été massacrés, la douleur m’empêche de vous dire ce qui s’est passé." (4) Il produit alors des soldats qu’il avait instruits à parler selon ses voeux ; ils confirment ce que Litaviccos vient d’avancer ; (5) que tous les cavaliers héduens avaient été tués, pour de prétendues entrevues avec les Arvernes ; qu’eux-mêmes ne s’étaient sauvés du milieu du carnage qu’en se cachant dans la foule des soldats. (6) Les Héduens poussent des cris, et conjurent Litaviccos de pourvoir à leur salut. (7) "Y a-t-il donc à délibérer", reprend-il, "et n’est-ce pas une nécessité pour nous de marcher à Gergovie, et de nous joindre aux Arvernes ? (8) Doutons-nous qu’après ce premier forfait, les Romains n’accourent déjà pour nous égorger ? Si donc il nous reste quelque énergie, vengeons la mort de ceux qui ont été si indignement massacrés, et exterminons ces brigands." (9) Il leur montre les citoyens romains qui étaient là sous sa sauvegarde et sous son escorte, leur enlève aussitôt un convoi de vivres et de blé, et les fait périr dans de cruels tourments. (10) Puis il dépêche des courriers dans tous les cantons de la cité, les soulève par le même mensonge sur le massacre de la cavalerie et de la noblesse, et les exhorte à punir toute perfidie de la même manière que lui.
\subsection[{§ 39.}]{ \textsc{§ 39.} }
\noindent (1) L'Éduen Éporédorix, jeune homme d’une grande famille et très puissant dans son pays, et avec lui Viridomaros, de même âge et de même crédit, mais inférieur en naissance, que César, sur la recommandation de Diviciacos, avait élevé d’une condition obscure aux plus hautes dignités, étaient venus, nominativement appelés par lui, le joindre avec la cavalerie. (2) Ils se disputaient le premier rang, et dans le débat récent pour la suprême magistrature, ils avaient combattu de tous leurs moyens ; l’un pour Convictolitavis, l’autre pour Cotos. (3) Éporédorix, informé du dessein de Litaviccos, en donne avis à César au milieu de la nuit ; il le prie de ne pas souffrir que des jeunes gens, par des manoeuvres perverses, détachent sa cité de l’alliance du peuple romain ; ce qu’il regarde comme inévitable, si tant de milliers d’hommes se joignent à l’ennemi ; car leurs familles ne pourraient manquer de s’intéresser à leur sort, ni la cité d’y attacher une grande importance.
\subsection[{§ 40.}]{ \textsc{§ 40.} }
\noindent (1) Vivement affecté de cette nouvelle, parce qu’il avait toujours porté aux Héduens un intérêt particulier, César, sans balancer un instant, prend quatre légions sans bagage, et toute la cavalerie. (2) On n’eut pas même le temps de replier les tentes, parce que tout, dans ce moment, semblait dépendre de la célérité. (3) Il laissa pour la garde du camp le lieutenant C. Fabius, avec deux légions. Il avait ordonné de saisir les frères de Litaviccos ; mais il apprit qu’ils venaient de s’enfuir vers l’ennemi. (4) Il exhorte les soldats à ne pas se rebuter des fatigues de la marche dans une circonstance aussi urgente. L'ardeur fut générale ; après s’être avancé à la distance de vingt-cinq mille pas, il découvrit les Héduens, et détacha la cavalerie, qui retarda et empêcha leur marche ; elle avait défense expresse de tuer personne. (5) Éporédorix et Viridomaros, que les Héduens croyaient morts, ont ordre de se montrer dans les rangs de la cavalerie et d’appeler leurs compatriotes. (6) On les reconnaît ; et la fourberie de Litaviccos une fois dévoilée, les Héduens tendent les mains, font entendre qu’ils se rendent, jettent leurs armes et demandent la vie. (7) Litaviccos s’enfuit à Gergovie, suivi de ses clients ; car, selon les moeurs gauloises, c’est un crime d’abandonner son patron, même dans un cas désespéré.
\subsection[{§ 41.}]{ \textsc{§ 41.} }
\noindent (1) César dépêcha des courriers pour faire savoir aux Héduens qu’il avait fait grâce à des hommes que le droit de la guerre lui eût permis de tuer ; et après avoir donné trois heures de la nuit à l’armée pour se reposer, il reprit la route de Gergovie. (2) Presque à moitié chemin, des cavaliers, expédiés par Fabius, lui apprirent quel danger avait couru le camp ; il avait été attaqué par de très grandes forces ; des ennemis frais remplaçaient sans cesse ceux qui étaient las, et fatiguaient par leurs efforts continuels les légionnaires forcés, à cause de la grande étendue du camp, de ne pas quitter le rempart ; (3) une grêle de flèches et de traits de toute espèce avait blessé beaucoup de monde ; les machines avaient été fort utiles pour la défense. (4) Après la retraite des assaillants, Fabius, ne conservant que deux portes, avait fait boucher les autres, et ajouter des parapets aux remparts : il s’attendait pour le lendemain à une attaque pareille. (5) Instruit de ces faits, et secondé par le zèle extrême des soldats, César arrive au camp avant le lever du soleil.
\subsection[{§ 42.}]{ \textsc{§ 42.} }
\noindent (1) Tandis que ces événements se passent auprès de Gergovie, les Héduens, aux premières nouvelles qu’ils reçoivent de Litaviccos, ne donnent pas un instant à la réflexion. (2) Les uns sont poussés par la cupidité, les autres par la colère et par cette légèreté qui est si naturelle à ce peuple qu’il prend pour chose avérée ce qui n’est qu’un simple ouï-dire. (3) Ils pillent les citoyens romains, les massacrent, les traînent en prison. (4) Convictolitavis seconde l’impulsion donnée, et précipite la multitude dans les excès les plus coupables, afin que le crime une fois commis, elle ait honte de revenir à la raison. (5) M. Aristius, tribun des soldats, rejoignait sa légion ; on le fait sortir sur parole de la place de Cavillonum ; on force à s’éloigner ceux qui s’y étaient établis pour leur commerce. (6) Harcelés sans relâche sur la route, ils sont dépouillés de tous leurs effets ; ceux qui résistent sont assaillis nuit et jour ; enfin, après beaucoup de pertes de part et d’autres, on excite une plus grande multitude à prendre les armes.
\subsection[{§ 43.}]{ \textsc{§ 43.} }
\noindent (1) Cependant, à la nouvelle que toutes leurs troupes sont au pouvoir de César, les Héduens accourent près d’Aristius ; ils l’assurent que rien ne s’est fait de l’aveu général ; (2) ils ordonnent une enquête sur le pillage des effets, confisquent les biens de Litaviccos et de ses frères, et députent vers César pour se justifier. (3) Leur seul but était de recouvrer leurs troupes ; mais souillés d’un crime, enrichis par le profit du pillage auquel un grand nombre d’entre eux avait eu sa part, et, frappés de la crainte du châtiment, ils ne tardent pas à former secrètement des projets de guerre, et font, par des agents, intriguer auprès des autres cités. (4) César, quoique instruit de ces menées, parla à leurs députés avec toute la douceur possible. "L'aveuglement et l’inconséquence de la populace ne lui feront jamais penser désavantageusement des Héduens, et ne peuvent diminuer sa bienveillance pour eux." (5) S'attendant néanmoins à un mouvement plus général dans la Gaule, et ne voulant pas être investi par toutes les cités, il pensait aux moyens de s’éloigner de Gergovie, pour réunir de nouveau toutes ses forces ; mais il fallait que son départ, qui venait de la crainte d’un soulèvement, n’eût pas l’air d’une fuite.
\subsection[{§ 44.}]{ \textsc{§ 44.} }
\noindent (1) Au milieu de ces pensées, il crut avoir trouvé une occasion favorable. Car, en visitant les travaux du petit camp, il vit qu’il n’y avait plus personne sur la colline qu’occupait l’ennemi les jours précédents, et en si grand nombre qu’à peine en voyait-on le sol. (2) Étonné, il en demande la cause aux transfuges, qui chaque jour venaient en foule se rendre à lui. (3) Tous s’accordent à dire, ce qu’il savait déjà par ses éclaireurs, que le sommet de cette colline étant presque plat, mais boisé et étroit du côté qui conduisait à l’autre partie de la ville, (4) les Gaulois craignaient beaucoup pour ce point, et sentaient que si les Romains, déjà maîtres de l’autre colline, s’emparaient de celle-ci, ils seraient pour ainsi dire enveloppés sans pouvoir ni sortir ni fourrager. (5) Vercingétorix avait donc appelé toutes ses troupes pour fortifier cet endroit.
\subsection[{§ 45.}]{ \textsc{§ 45.} }
\noindent (1) Sur cet avis, César y envoie, au milieu de la nuit, plusieurs escadrons, avec ordre de se répandre dans la campagne d’une manière un peu bruyante. (2) Au point du jour, il fait sortir du camp beaucoup d’équipages et de mulets, qu’on décharge de leurs bagages ; il donne des casques aux muletiers, pour qu’ils aient l’apparence de cavaliers, et leur recommande de faire le tour des collines. (3) Il fait partir avec eux quelques cavaliers qui doivent affecter de se répandre au loin. Il leur assigne à tous un point de réunion qu’ils gagneront par un long circuit. (4) De Gergovie, qui dominait le camp, on voyait tous ces mouvements, mais de trop loin pour pouvoir distinguer ce que c’était au juste. (5) César détache une légion vers la même colline ; quand elle a fait quelque chemin, il l’arrête dans un fond et la cache dans les forêts. (6) Les soupçons des Gaulois redoublent, et toutes leurs troupes passent de ce côté. (7) César, voyant leur camp dégarni, fait couvrir les insignes, cacher les enseignes, et défiler les soldats du grand camp dans le petit, par pelotons pour qu’on ne les remarque pas de la ville ; il donne ses instructions aux lieutenants qui commandent chaque légion, (8) et les avertit surtout de contenir les soldats que l’ardeur de combattre et l’espoir du butin pourraient emporter trop loin ; (9) il leur montre le désavantage que donne l’escarpement du terrain ; la célérité seule peut le compenser ; il s’agit d’une surprise et non d’un combat. (10) Ces mesures prises, il donne le signal, et fait en même temps monter les Héduens sur la droite par un autre chemin.
\subsection[{§ 46.}]{ \textsc{§ 46.} }
\noindent (1) De la plaine et du pied de la colline jusqu’au mur de la ville il y avait douze cents pas en ligne droite, sans compter les sinuosités du terrain. (2) Les détours qu’il fallait faire pour monter moins à pic augmentaient la distance. (3) À mi-côte, les Gaulois avaient tiré en longueur, et suivant la disposition du terrain, un mur de six pieds de haut et formé de grosses pierres, pour arrêter notre attaque ; et laissant vide toute la partie basse, ils avaient entièrement garni de troupes la partie supérieure de la colline jusqu’au mur de la ville. (4) Au signal donné, nos soldats arrivent promptement aux retranchements, les franchissent et se rendent maîtres de trois camps. (5) Le succès de cette attaque avait été si rapide, que Teutomatos, roi des Nitiobroges, surpris dans sa tente, où il reposait au milieu du jour, s’enfuit nu jusqu’à la ceinture, eut son cheval blessé, et n’échappa qu’avec peine aux mains des pillards.
\subsection[{§ 47.}]{ \textsc{§ 47.} }
\noindent (1) César ayant atteint son but, fait sonner la retraite et faire halte à la dixième légion, qui l’accompagnait. (2) Mais les autres n’avaient pas entendu le son de la trompette, parce qu’elles étaient au-delà d’un vallon assez large ; et bien que, pour obéir aux ordres de César, les lieutenants et tribuns s’efforçassent de les retenir, (3) entraînées par l’espérance d’une prompte victoire, par la fuite des ennemis, par leurs anciens succès, et ne voyant rien de si difficile qu’elles n’en pussent triompher par leur courage, elles ne cessèrent leur poursuite qu’aux pieds des murs et jusqu’aux portes de la ville. (4) Un cri s’étant alors élevé de toutes les parties de l’enceinte, ceux qui étaient les plus éloignés, effrayés de cette confusion soudaine, croient les Romains dans la ville et se précipitent des murs. (5) Les mères jettent du haut des murailles des habits et de l’argent, et s’avançant, le sein découvert, les bras étendus, elles supplient les Romains de les épargner et de ne pas agir comme à Avaricum, où l’on n’avait fait grâce, ni aux femmes ni aux enfants. (6) Quelques-unes, s’aidant de main en main à descendre du rempart, se livrèrent aux soldats. (7) L. Fabius, centurion de la huitième légion, qui, ce jour même, avait dit dans les rangs, qu’excité par les récompenses d’Avaricum, il ne laisserait à personne le temps d’escalader le mur ayant lui, ayant pris trois de ses soldats, se fit soulever par eux et monta sur le mur. Il leur tendit la main à son tour, et les fit monter un à un.
\subsection[{§ 48.}]{ \textsc{§ 48.} }
\noindent (1) Cependant, ceux des Gaulois qui, comme nous l’avons dit, s’étaient portés de l’autre côté de la ville pour la fortifier, aux premiers cris qu’ils entendent, et pressés par les nombreux avis qu’on leur donne de l’entrée des Romains dans la ville, détachent en avant leur cavalerie et la suivent en foule. (2) Chacun, à mesure qu’il arrive, se range sous les murs, et augmente le nombre des combattants. (3) Leurs forces s’étant ainsi grossies, les femmes, qui, un instant auparavant, nous tendaient les mains du haut des remparts, s’offrent aux Barbares, échevelées suivant leur usage, et les implorent en leur montrant leurs enfants. (5) Les Romains avaient le désavantage et du lieu et du nombre ; fatigués de leur course et de la durée du combat, ils ne se soutenaient plus qu’avec peine contre des troupes fraîches et sans blessures.
\subsection[{§ 49.}]{ \textsc{§ 49.} }
\noindent (1) César, voyant le désavantage du lieu, et les forces de l’ennemi croître sans cesse, craignit pour les siens, et envoya au lieutenant T. Sextius, qu’il avait chargé de la garde du petit camp, l’ordre d’en faire sortir les cohortes et de les poster au pied de la colline sur la droite des Gaulois, (2) afin que, s’il voyait nos soldats repoussés, il forçât les ennemis à ralentir leur poursuite, en les intimidant ; (3) lui-même s’avançant à la tête de la légion un peu au-delà du lieu où il s’était arrêté, attendit l’issue du combat.
\subsection[{§ 50.}]{ \textsc{§ 50.} }
\noindent (1) Tandis qu’on se battait avec acharnement et corps à corps, les ennemis forts de leur position et de leur nombre, et les nôtres de leur valeur, on vit tout à coup paraître, sur notre flanc découvert, les Héduens que César avait envoyés par un autre chemin, pour faire diversion sur notre droite. (2) La ressemblance de leurs armes avec celles des Barbares alarma vivement nos soldats ; et quoiqu’ils eussent le bras droit nu, signe ordinaire de paix, ceux-ci crurent cependant que c’était un artifice de l’ennemi employé pour les tromper. (3) En même temps, le centurion L. Fabius, et ceux qui étaient montés avec lui sur le rempart, furent enveloppés, et précipités sans vie du haut de la muraille. (4) M. Pétronius, centurion de la même légion, se vit accablé par le nombre comme il s’efforçait de briser les portes ; ayant déjà reçu plusieurs blessures et désespérant de sa vie, il s’adresse aux hommes de sa compagnie qui l’avaient suivi : "Puisque je ne puis me sauver avec vous, dit-il, je veux du moins pourvoir au salut de ceux qu’entraîné par l’amour de la gloire, j’ai conduits dans le péril. Usez du moyen que je vous donnerai de sauver vos jours." (5) Aussitôt il se jette au milieu des ennemis, en tue deux, et écarte un moment les autres de la porte. (6) Comme les siens essayaient de le secourir : "En vain, dit-il, tentez-vous de me conserver la vie ; déjà mon sang et mes forces m’abandonnent. Éloignez-vous donc tandis que vous le pouvez et rejoignez votre légion." Un moment après, il périt en combattant, après avoir ainsi sauvé ses compagnons.
\subsection[{§ 51.}]{ \textsc{§ 51.} }
\noindent (1) Nos soldats, pressés de toutes parts, furent repoussés de leur poste avec une perte de quarante-six centurions ; mais la dixième légion, placée comme corps de réserve dans une position un peu plus avantageuse, arrêta les ennemis trop ardents à nous poursuivre. (2) Elle fut soutenue par les cohortes de la treizième, venue du petit camp et postée un peu plus haut, sous les ordres du lieutenant T. Sextius. (3) Dès que les légions eurent gagné la plaine, elles s’arrêtèrent et firent face à l’ennemi. (4) Vercingétorix ramena ses troupes du pied de la colline dans ses retranchements. Cette journée nous coûta près de sept cents hommes.
\subsection[{§ 52.}]{ \textsc{§ 52.} }
\noindent (1) Le lendemain César assembla les troupes, et reprocha aux soldats leur imprudence et leur cupidité : "Ils avaient eux-mêmes jugé de ce qu’il fallait faire, et jusqu’où l’on devait s’avancer ; ils ne s’étaient point arrêtés au signal de la retraite ; ni les tribuns ni les lieutenants n’avaient pu les retenir. (2) Il leur représenta le danger d’une mauvaise position, et leur rappela sa conduite au siège d’Avaricum, lorsque, surprenant l’ennemi sans chef et sans cavalerie, il avait renoncé à une victoire certaine, plutôt que de l’acheter par la perte même légère qu’aurait entraînée le désavantage du lieu. (3) Autant il admirait leur courage, que n’avaient pu arrêter ni les retranchements de l’ennemi, ni l’élévation de la montagne, ni les murs de la ville, autant il les blâmait d’avoir cru, dans leur insubordination présomptueuse, juger mieux que leur général du succès et de l’issue de l’événement ; (4) il ajouta qu’il n’aimait pas moins dans un soldat la modestie et la retenue que la valeur et la magnanimité."
\subsection[{§ 53.}]{ \textsc{§ 53.} }
\noindent (1) Tel fut le discours de César, à la fin duquel il releva le courage des soldats ; il leur dit de ne pas se laisser abattre par cet événement, et de ne point attribuer au courage de l’ennemi ce qu’il n’avait dû qu’à sa bonne position ; et persistant dans ses projets de départ, il fit sortir les légions du camp et les mit en bataille sur un terrain favorable. (2) Vercingétorix descendit aussi dans la plaine : après une légère escarmouche de cavalerie, où César eut le dessus, il fit rentrer ses troupes. (3) Il en fut de même le lendemain ; jugeant alors l’épreuve suffisante pour rabattre la jactance des Gaulois et raffermir le courage des siens, il décampa pour se rendre chez les Héduens. (4) Les ennemis n’essayèrent même pas de le suivre, et le troisième jour, il arriva sur les bords de l’Allier, rétablit le pont et le passa avec l’armée.
\subsection[{§ 54.}]{ \textsc{§ 54.} }
\noindent (1) C'est là que les Héduens Viridomaros et Éporédorix vinrent le trouver, et lui dirent que Litaviccos était parti avec toute sa cavalerie pour soulever le pays ; eux-mêmes avaient besoin de le devancer pour retenir la nation dans le devoir. (2) Quoique César eût déjà plusieurs preuves de la perfidie des Héduens, et qu’il pensât que le départ de ces deux hommes ne ferait que hâter la révolte, il ne crut cependant pas devoir les retenir, de peur de paraître leur faire violence, ou avoir conçu quelque crainte. (3) À leur départ, il leur rappelle brièvement les services qu’il a rendus aux Héduens, (4) ce qu’ils étaient, et leur abaissement quand il les a pris tous sous sa protection ; rejetés dans leurs villes, dépouillés de leurs champs, ayant perdu toutes leurs troupes, tributaires, réduits ignominieusement à donner des otages, à quelle prospérité, à quelle puissance ne les a-t-il pas élevés ? Non seulement il les a rétablis dans leur ancien état, mais ils jouissent d’une influence, d’une considération bien au-dessus de celle qu’ils avaient jamais eue. Ces recommandations faites, il les congédia.
\subsection[{§ 55.}]{ \textsc{§ 55.} }
\noindent (1) Noviodunum (Nevers), ville des Héduens, était située sur les bords de la Loire, dans une position avantageuse. (2) César y tenait rassemblés tous les otages de la Gaule, les subsistances, les deniers publics, une grande partie de ses équipages et de ceux de l’armée ; (3) il y avait envoyé un grand nombre de chevaux, achetés, pour les besoins de cette guerre, en Italie et en Espagne. (4) En y arrivant, Éporédorix et Viridomaros apprirent où en étaient les choses dans leur pays ; que Litaviccos avait été reçu par les Héduens, dans Bibracte (Autun), ville de la plus grande influence ; que le premier magistrat, Convictolitavis, et une grande partie du sénat s’étaient rendus près de lui ; qu’on avait ouvertement envoyé des ambassadeurs à Vercingétorix, pour faire avec lui un traité de paix et d’alliance. Ils ne voulurent pas laisser échapper une occasion si favorable. (5) Ils massacrèrent la garde laissée à Noviodunum, les marchands et les voyageurs qui s’y trouvaient, partagèrent entre eux l’argent et les chevaux, (6) et firent conduire dans Bibracte, à Convictolitavis, les otages des cités ; (7) puis, jugeant qu’ils étaient hors d’état de garder la ville, ils la brûlèrent, afin qu’elle ne servit pas aux Romains, (8) emportèrent sur des bateaux autant de blé que le moment le permettait, et jetèrent le reste dans la rivière ou dans le feu. (9) Ensuite, levant des troupes dans les pays voisins, ils placèrent des postes et des détachements le long de la Loire, et firent en tout lieu parade de leur cavalerie, pour répandre la terreur et pour essayer de chasser les Romains de la contrée par la disette, en leur coupant les vivres, (10) espoir d’autant mieux fondé que la Loire, alors enflée par la fonte des neiges, ne paraissait guéable en aucun endroit.
\subsection[{§ 56.}]{ \textsc{§ 56.} }
\noindent (1) Instruit de tous ces mouvements, César crut devoir hâter sa marche ; il voulait, au besoin, essayer de jeter des ponts sur la Loire, combattre avant que l’ennemi eût assemblé de plus grandes forces ; (2) car changer de plan et rétrograder sur la province, était un parti que la crainte même ne l’eût pas forcé de prendre, soit parce qu’il sentait la honte et l’indignité de cette mesure, à laquelle s’opposaient d’ailleurs les Cévennes et la difficulté des chemins, soit surtout parce qu’il craignait vivement pour Labiénus, dont il était séparé, et pour les légions parties sous ses ordres. (3) Forçant donc sa marche de jour et de nuit, il arriva, contre l’attente générale, sur les bords de la Loire ; (4) la cavalerie ayant trouvé un gué assez commode eu égard aux circonstances (on y avait seulement hors de l’eau les épaules et les bras pour soutenir les armes), il la disposa de manière à rompre le courant, et l’armée passa sans perte à la vue des ennemis soudainement effrayés. (5) Les troupes s’approvisionnèrent, de bétail et de blé, dont on trouva les champs couverts, et l’on se dirigea vers les Sénons.
\subsection[{§ 57.}]{ \textsc{§ 57.} }
\noindent (1) Pendant ces mouvements de l’armée de César, Labiénus ayant laissé à Agédincum (Sens), pour la garde des bagages, les recrues récemment arrivées d’ltalie, se porte avec quatre légions vers Lutèce (Paris). Cette ville appartient aux Parisii et est située dans une île de la Seine. (2) Au bruit de son arrivée, un grand nombre de troupes ennemies se réunirent des pays voisins. (3) Le commandement en chef fut donné à l’Aulerque Camulogène, vieillard chargé d’années, mais à qui sa profonde expérience dans l’art militaire mérita cet honneur. (4) Ce général ayant remarqué que la ville était entourée d’un marais qui aboutissait à la Seine, et protégeait merveilleusement cette place, y établit ses troupes dans le but de nous disputer le passage.
\subsection[{§ 58.}]{ \textsc{§ 58.} }
\noindent (1) Labiénus travailla d’abord à dresser des mantelets, à combler le marais de claies et de fascines, et à se frayer un chemin sûr. (2) Voyant que les travaux présentaient trop de difficultés, il sortit de son camp en silence à la troisième veille, et arriva à Metlosédum (Melun) par le même chemin qu’il avait pris pour venir. (3) C'est une ville des Sénons, située, comme nous l’avons dit de Lutèce, dans une île de la Seine. (4) S'étant emparé d’environ cinquante bateaux, il les joignit bientôt ensemble, les chargea de soldats, et par l’effet de la peur que cette attaque inopinée causa aux habitants, dont une grande partie d’ailleurs avait été appelée sous les drapeaux de Camulogène, il entra dans la place sans éprouver de résistance. (5) Il rétablit le pont que les ennemis avaient coupé les jours précédents, y fit passer l’armée, et se dirigea vers Lutèce en suivant le cours du fleuve. (6) L'ennemi, averti de cette marche par ceux qui s’étaient enfuis de Metlosédum, fait mettre le feu à Lutèce, couper les ponts de cette ville ; et, protégé par le marais, il vient camper sur les rives de la Seine, vis-à-vis Lutèce et en face du camp de Labiénus.
\subsection[{§ 59.}]{ \textsc{§ 59.} }
\noindent (1) Déjà le bruit courait que César avait quitté le siège de Gergovie ; déjà circulait la nouvelle de la défection des Héduens et des succès obtenus par la Gaule soulevée. Les Gaulois affirmaient, dans leurs entretiens, que César, à qui l’on avait coupé sa route et tout accès à la Loire, avait été, faute de vivres, forcé de se retirer vers la province romaine. (2) De leur côté, les Bellovaques, instruits de la défection des Héduens, et déjà assez disposés à se soulever, se mirent à lever des troupes et à préparer ouvertement la guerre. (3) Labiénus, au milieu de si grands changements, sentit qu’il fallait adopter un tout autre système que celui qu’il avait jusque-là suivi ; (4) il ne songea plus à faire des conquêtes ni à harceler l’ennemi, mais à ramener l’armée sans perte à Agédincum. (5) Car d’un côté, il était menacé par les Bellovaques, peuple jouissant dans la Gaule d’une haute réputation de valeur ; de l’autre, Camulogène, maître du pays, avait une armée toute formée et en état de combattre ; enfin un grand fleuve séparait les légions de leurs bagages et de la garnison qui les gardait. (6) Il ne voyait contre de si grandes et si subites difficultés d’autre ressource que son courage.
\subsection[{§ 60.}]{ \textsc{§ 60.} }
\noindent (1) Ayant donc, sur le soir, convoqué un conseil, il engagea chacun à exécuter avec promptitude et adresse les ordres qu’il donnerait ; il distribua les bateaux, qu’il avait amenés de Metlosédum, à autant de chevaliers romains, et leur prescrivit de descendre la rivière à la fin de la première veille, de s’avancer en silence l’espace de quatre mille pas et de l’attendre là. (2) Il laissa pour la garde du camp les cinq cohortes qu’il jugeait les moins propres à combattre, (3) et commanda à celles qui restaient de la même légion de remonter le fleuve au milieu de la nuit, avec tous les bagages, en faisant beaucoup de bruit. (4) Il rassembla aussi des nacelles et les envoya dans la même direction à grand bruit de rames. Lui-même, peu d’instants après, partit en silence avec trois légions, et se rendit où il avait ordonné de conduire les bateaux.
\subsection[{§ 61.}]{ \textsc{§ 61.} }
\noindent (1) Lorsqu’on y fut arrivé, les éclaireurs de l’ennemi, placés sur toute la rive du fleuve, furent attaqués à l’improviste pendant un grand orage survenu tout à coup ; (2) l’armée et la cavalerie passèrent rapidement le fleuve, avec le secours des chevaliers romains chargés de cette opération. (3) Au point du jour et presque au même instant, on annonce aux ennemis qu’il règne une agitation extraordinaire dans le camp romain, qu’un corps considérable de troupes remonte le fleuve, qu’on entend un grand bruit de rames du même côté, et qu’un peu au-dessous des bateaux transportent des soldats. (4) À ce récit, persuadés que les légions passent le fleuve en trois endroits, et que l’effroi causé par la défection des Héduens précipite notre fuite, ils se partagent aussi en trois corps. (5) Ils en laissent un vis-à-vis de notre camp pour la garde du leur ; le second est envoyé vers Metlosédum, avec ordre de s’avancer aussi loin que les bateaux, et ils marchent contre Labiénus avec le reste de leurs troupes.
\subsection[{§ 62.}]{ \textsc{§ 62.} }
\noindent (1) Au point du jour toutes nos troupes avaient passé, et l’on vit celles de l’ennemi rangées en bataille. (2) Labiénus exhorte les soldats à se rappeler leur ancienne valeur et tant de combats glorieux, et à se croire en présence de César lui-même, sous la conduite duquel ils ont tant de fois défait leurs ennemis, puis il donne le signal du combat. (3) Dès le premier choc, la septième légion, placée à l’aile droite, repousse les ennemis et les met en fuite ; (4) à l’aile gauche qu’occupait la douzième légion, quoique les premiers rangs de l’ennemi fussent tombés percés de nos traits, les autres résistaient vigoureusement, et aucun ne songeait à la fuite. (5) Camulogène, leur général, était lui-même avec eux, et excitait leur courage. (6) Le succès était donc douteux sur ce point, lorsque les tribuns de la septième légion, instruits de ce qui se passait à l’aile gauche, vinrent avec leur légion prendre les ennemis en queue et les chargèrent. (7) Même dans cette position, aucun Gaulois ne quitta sa place ; tous furent enveloppés et tués. (8) Camulogène subit le même sort. D'un autre côté, ceux qu’on avait laissés à la garde du camp opposé à celui de Labiénus, avertis que l’on se battait, marchèrent au secours des leurs, et prirent position sur une colline ; mais ils ne purent soutenir le choc de nos soldats victorieux. (9) Entraînés dans la déroute des autres Gaulois, tous ceux qui ne purent gagner l’abri des bois ou des hauteurs, furent taillés en pièces par notre cavalerie. (10) Après cette expédition, Labiénus retourne vers Agédincum, où avaient été laissés les bagages de toute l’armée. De là il rejoignit César avec toutes les troupes.
\subsection[{§ 63.}]{ \textsc{§ 63.} }
\noindent (1) La nouvelle dé la défection des Héduens propagea la guerre. (2) Des députations sont envoyées sur tous les points ; crédit, autorité, argent, tout est mis en usage pour gagner les différents états. (3) Nantis des otages que César avait déposés chez eux, ils menacent de les faire périr pour effrayer ceux qui hésitent. (4) Les Héduens invitent Vercingétorix à venir conférer avec eux sur les moyens de faire la guerre. (5) II se rend à leur prière ; mais ils prétendent qu’on leur défère le commandement en chef ; et comme il leur est disputé, on convoque une assemblée de toute la Gaule à Bibracte. (6) On s’y rend en foule de toutes parts. La question est soumise aux suffrages de la multitude, et tous confirment le choix de Vercingétorix comme généralissime. (7) On ne vit point à cette assemblée les Rèmes, les Lingons, les Trévires : les deux premiers peuples, parce qu’ils restaient fidèles aux Romains ; les Trévires, parce qu’ils étaient trop éloignés et pressés en outre par les Germains, ce qui fut cause qu’ils ne prirent aucune part à la guerre, et gardèrent la neutralité. (8) Les Héduens souffraient vivement de se voir dépouillés du commandement ; ils déplorèrent le changement de leur fortune, et regrettèrent la bienveillance de César envers eux ; mais, comme la guerre était commencée, ils n’osèrent séparer leur cause de celle des autres Gaulois. (9) Ce ne fut qu’à regret qu’Éporédorix et Viridomaros, jeunes gens de la plus haute espérance, obéirent à Vercingétorix.
\subsection[{§ 64.}]{ \textsc{§ 64.} }
\noindent (1) Il exige des otages des autres nations, fixe le jour où ils lui seront livrés, ordonne la prompte réunion de toute la cavalerie, forte de quinze mille hommes ; et annonce (2) "qu’il se contente de l’infanterie qu’il a déjà ; qu’il ne veut pas tenter le sort des armes en bataille rangée ; qu’avec une cavalerie nombreuse il lui sera très facile de couper les vivres aux Romains et de gêner leurs fourrageurs ; (3) que seulement les Gaulois se résignent à détruire leurs récoltes et à incendier leurs demeures, et ne voient dans ces pertes domestiques que le moyen de recouvrer à jamais leur indépendance et leur liberté." (4) Les choses ainsi réglées, il ordonne aux Héduens et aux Ségusiaves, limitrophes de la province, de lever dix mille fantassins ; il y ajoute huit cents cavaliers. (5) Il confie le commandement de ces troupes au frère d’Éporédorix, et lui dit de porter la guerre chez les Allobroges. (6) D'un autre côté, il envoie les Gabales et les plus proches cantons des Arvernes, ravager le territoire des Helviens, ainsi que les Rutènes et les Cadurques celui des Volques Arécomiques. (7) En même temps, et par des messages secrets, il sollicite les Allobroges, espérant que les ressentiments de la dernière guerre n’y étaient pas encore éteints. (8) Il promet aux chefs de l’argent, et à la nation la souveraineté de toute la province.
\subsection[{§ 65.}]{ \textsc{§ 65.} }
\noindent (1) Pour résister à toutes ces attaques, le lieutenant L. César n’avait à distribuer, comme garnison, sur tout le territoire de la province, que vingt-deux cohortes tirées de cette province même. (2) Les Helviens attaquent spontanément leurs voisins, sont défaits, perdent C. Valérius Domnotaurus, fils de Caburus, chef de leur nation, et sont repoussés dans les murs de leurs villes. (3) Les Allobroges, ayant établi près du Rhône des postes nombreux, mettent beaucoup de zèle et de diligence dans la défense de leur territoire. (4) César, voyant que l’ennemi lui est supérieur en cavalerie, qu’il lui ferme tous les chemins, et qu’il n’y a nul moyen de tirer des secours de l’Italie ni de la province, envoie au-delà du Rhin, en Germanie, vers les peuples qu’il avait soumis les années précédentes, et leur demande des cavaliers et de ces fantassins armés à la légère, accoutumés à se mêler avec la cavalerie dans les combats. (5) À leur arrivée, ne trouvant pas assez bien dressés les chevaux dont ils se servaient, il prit ceux des tribuns, des autres officiers, et même des chevaliers romains et des vétérans, et les distribua aux Germains.
\subsection[{§ 66.}]{ \textsc{§ 66.} }
\noindent (1) Pendant le temps employé à toutes ces choses, les corps ennemis envoyés par les Arvernes et la cavalerie levée dans tous les états de la Gaule, se réunirent. (2) Se voyant à la tête de troupes si nombreuses, et tandis que César se dirigeait vers les Séquanes par l’extrême frontière des Lingons, pour porter plus facilement du secours à la province, Vercingétorix vint, en trois campements, prendre position à environ dix mille pas des Romains ; (3) et ayant convoqué les chefs de la cavalerie, il leur dit : "Que le moment de vaincre est venu ; que les Romains s’enfuient dans leur province et abandonnent la Gaule ; (4) que c’est assez pour la liberté du moment, trop peu pour la paix et le repos de l’avenir ; qu’ils ne manqueront pas de revenir avec de plus grandes forces, et ne mettront point de terme à la guerre. Il faut donc les attaquer dans l’embarras de leur marche ; (5) si les fantassins viennent au secours de leur cavalerie et sont ainsi arrêtés, ils ne pourront achever leur route ; si (ce qui lui paraît plus probable), ils abandonnent leurs bagages pour ne pourvoir qu’à leur sûreté, ils perdront, outre l’honneur, leurs ressources les plus indispensables. (6) Quant à leurs cavaliers, aucun d’eux n’osera seulement s’avancer hors des lignes ; on ne doit pas même en douter. Du reste, pour inspirer encore plus de courage à ses troupes, et plus de crainte à l’ennemi, il rangera toute l’armée en avant du camp." (7) Les cavaliers s’écrient qu’il faut que chacun s’engage par le serment le plus sacré à ne plus entrer dans sa maison, à ne plus revoir ses enfants, sa famille, sa femme, s’il n’a traversé deux fois les rangs de l’ennemi.
\subsection[{§ 67.}]{ \textsc{§ 67.} }
\noindent (1) On approuve la proposition, et tous prêtent ce serment. Le lendemain, la cavalerie est partagée en trois corps, dont deux se montrent sur nos ailes, tandis que le centre se présente de front à notre avant-garde pour lui fermer le passage. (2) Instruit de ces dispositions, César forme également trois divisions de sa cavalerie, et la fait marcher contre l’ennemi. Le combat s’engage de tous les côtés à la fois ; (3) l’armée fait halte ; les bagages sont placés entre les légions. (4) Si nos cavaliers fléchissent sur un point, ou sont trop vivement pressés, César y fait porter les enseignes et marcher les cohortes, ce qui arrête les ennemis dans leur poursuite et ranime nos soldats par l’espoir d’un prompt secours. (5) Enfin, les Germains, sur le flanc droit, gagnent le haut d’une colline, en chassent les ennemis, les poursuivent jusqu’à une rivière où Vercingétorix s’était placé avec son infanterie, et en tuent un grand nombre. (6) Témoins de cette défaite, les autres Gaulois, craignant d’être enveloppés, prennent la fuite. Ce ne fut plus partout que carnage. (7) Trois Héduens de la plus haute distinction sont pris et amenés à César : Cotos, commandant de la cavalerie, qui dans les derniers comices avait été en concurrence avec Convictolitavis ; Cavarillos, qui, après la défection de Litaviccos, avait reçu le commandement de l’infanterie ; et Éporédorix, que les Héduens, avant l’arrivée de César, avaient eu pour chef dans leur guerre contre les Séquanes.
\subsection[{§ 68.}]{ \textsc{§ 68.} }
\noindent (1) Voyant toute sa cavalerie en fuite, Vercingétorix fit rentrer les troupes qu’il avait rangées en avant du camp, et prit aussitôt le chemin d’Alésia, qui est une ville des Mandubiens, après avoir fait, en toute hâte, sortir du camp les bagages, qui le suivirent. (2) César laissa ses équipages sur un coteau voisin, les commit à la garde de deux légions, poursuivit l’ennemi tant que le jour dura, lui tua environ trois mille hommes de l’arrière-garde, et campa le lendemain devant Alésia. (3) Ayant reconnu la situation de la ville, et voyant les ennemis consternés de la défaite de leur cavalerie, qu’ils regardaient comme la principale force de leur armée, il exhorta les siens au travail et fit commencer les lignes de circonvallation.
\subsection[{§ 69.}]{ \textsc{§ 69.} }
\noindent (1) Cette place était située au sommet d’une montagne, dans une position si élevée qu’elle semblait ne pouvoir être prise que par un siège en règle. (2) Au pied de cette montagne coulaient deux rivières de deux côtés différents. (3) Devant la ville s’étendait une plaine d’environ trois mille pas de longueur ; (4) sur tous les autres points, des collines l’entouraient, peu distantes entre elles et d’une égale hauteur. (5) Sous les murailles, le côté qui regardait le soleil levant était garni, dans toute son étendue, de troupes gauloises ayant devant elles un fossé et une muraille sèche de six pieds de haut. (6) La ligne de circonvallation formée par les Romains occupait un circuit de onze mille pas. (7) Notre camp était assis dans une position avantageuse, et l’on y éleva vingt-trois forts, dans lesquels des postes étaient placés pendant le jour pour prévenir toute attaque subite ; on y tenait aussi toute la nuit des sentinelles et de fortes garnisons.
\subsection[{§ 70.}]{ \textsc{§ 70.} }
\noindent (1) Pendant les travaux, il y eut un combat de cavalerie dans cette plaine de trois mille pas qui s’étendait dans l’intervalle laissé libre entre les collines, comme nous l’avons dit plus haut. L'acharnement fut égal de part et d’autre. (2) Les nôtres commençant à souffrir, César envoya les Germains pour les soutenir, et plaça les légions en avant du camp, en cas que l’infanterie ennemie fit subitement quelque tentative. (3) Cet appui des légions releva le courage de nos cavaliers ; les Gaulois mis en fuite s’embarrassent par leur nombre et s’entassent aux portes trop étroites qui leur restent. (4) Alors les Germains les poursuivent vivement jusqu’à leurs retranchements ; (5) on en fait un grand carnage. Plusieurs abandonnant leurs chevaux, essaient de traverser le fossé et de franchir le mur. César fait faire un mouvement en avant aux légions qu’il avait placée à la tête du camp. (6) Ce mouvement porte l’effroi parmi les Gaulois même ceux qui étaient à l’intérieur des retranchements ; croyant qu’on arrive sur eux, ils crient aux armes ; quelques-uns se précipitent tout effrayés dans la ville. (7) Vercingétorix fait fermer les portes, de peur que le camp ne soit tout à fait abandonné. Ce ne fut qu’après avoir tué beaucoup de monde et pris un grand nombre de chevaux, que les Germains se retirèrent.
\subsection[{§ 71.}]{ \textsc{§ 71.} }
\noindent (1) Vercingétorix, avant que les Romains eussent achevé leur circonvallation, prit la résolution de renvoyer de nuit toute sa cavalerie. (2) Avant le départ de ces cavaliers, il leur recommande "d’aller chacun dans leur pays, et d’enrôler tous ceux qui sont en âge de porter les armes ; (3) il leur rappelle ce qu’il a fait pour eux, les conjure de veiller à sa sûreté et de ne pas l’abandonner, lui qui a bien mérité de la liberté commune, à la merci d’ennemis cruels ; leur négligence entraînerait, avec sa perte, celle de quatre-vingt mille hommes d’élite ; (4) il n’a de compte fait, de vivres que pour trente jours au plus ; mais il pourra, en les ménageant, tenir un peu plus longtemps." (5) Après ces recommandations, il fait partir en silence sa cavalerie ; à la seconde veille, par la où l’ouvrage était interrompu. (6) II se fait apporter tout le blé de la ville, et établit la peine de mort contre ceux qui n’obéiront pas ; (7) quant au bétail dont les Mandubiens avaient rassemblé une grande provision, il le distribue par tête ; le grain est mesuré avec épargne et donné en petite quantité ; (8) il fait rentrer dans la ville toutes les troupes qu’il avait disposées devant la place. (9) C'est par ces moyens qu’il se prépare à attendre les secours de la Gaule et à soutenir la guerre.
\subsection[{§ 72.}]{ \textsc{§ 72.} }
\noindent (1) Instruit de ces dispositions par les transfuges et les prisonniers, César arrêta son plan de fortification comme il suit. Il fit creuser un fossé large de vingt pieds, dont les côtés étaient à pic et la profondeur égale à la largeur. (2) Tout le reste des retranchements fut établi à quatre cents pieds en arrière de ce fossé ; il voulait par là (car on avait été obligé d’embrasser un si grand espace, que nos soldats n’auraient pu aisément en garnir tous les points) prévenir les attaques subites ou les irruptions nocturnes, et garantir durant le jour nos travailleurs des traits de l’ennemi. (3) Dans cet espace, César tira deux fossés de quinze pieds de large et d’autant de profondeur ; celui qui était intérieur et creusé dans les parties basses de la plaine, fut rempli d’eau tirée de la rivière. (4) Derrière ces fossés, il éleva une terrasse et un rempart de douze pieds ; il y ajouta un parapet et des créneaux, et fit élever de grosses pièces de bois fourchues à la jonction du parapet et du rempart, pour en rendre l’abord plus difficile aux ennemis. Tout l’ouvrage fut flanqué de tours, placées à quatre-vingts pieds l’une de l’autre.
\subsection[{§ 73.}]{ \textsc{§ 73.} }
\noindent (1) Il fallait dans le même temps aller chercher du bois et des vivres, et employer aux grands travaux des retranchements les troupes, diminuées de celles qu’on employait au loin. Souvent encore les Gaulois essayaient de troubler nos travailleurs, et faisaient par plusieurs portes les sorties les plus vigoureuses. (2) César jugea donc nécessaire d’ajouter quelque chose à ces retranchements, afin qu’un moindre nombre de soldats pût les défendre. À cet effet, on coupa des troncs d’arbres ou de fortes branches, on les dépouilla de leur écorce et on les aiguisa par le sommet ; puis on ouvrit une tranchée de cinq pieds de profondeur, (3) où l’on enfonça ces pieux qui, liés par le pied de manière à ne pouvoir être arrachés, ne montraient que leur partie supérieure. (4) Il y en avait cinq rangs, joints entre eux et entrelacés ; quiconque s’y était engagé s’embarrassait dans leurs pointes aiguës : nos soldats les appelaient des ceps. (5) Au devant, étaient disposés obliquement en quinconce des puits de trois pieds de profondeur, lesquels se rétrécissaient peu à peu jusqu’au bas. (6) On y fit entrer des pieux ronds de la grosseur de la cuisse, durcis au feu et aiguisés à l’extrémité, qui ne sortaient de terre que de quatre doigts ; (7) et pour affermir et consolider l’ouvrage, on foula fortement la terre avec les pieds : le reste était recouvert de ronces et de broussailles, afin de cacher le piège. (8) On avait formé huit rangs de cette espèce, à trois pieds de distance l’un de l’autre : on les nommait des lis à cause de leur ressemblance avec cette fleur. (9) En avant du tout étaient des chausses-trappes d’un pied de long et armées de pointes de fer, qu’on avait fichées en terre ; on en avait mis partout, à de faibles distances les unes des autres ; on les appelait des aiguillons.
\subsection[{§ 74.}]{ \textsc{§ 74.} }
\noindent (1) Ce travail fini, César fit tirer en suivant les zones les plus plates que pût offrir la nature des lieux, et dans un circuit de quatorze mille pas, une contrevallation du même genre, mais du côté opposé, contre l’ennemi du dehors. Il voulait qu’en cas d’attaque, pendant son absence, les retranchements ne pussent être investis par une multitude nombreuse. Enfin, pour prévenir les dangers auxquels les troupes (2) pourraient être exposées en sortant du camp, il ordonna que chacun se pourvût de fourrage et de vivres pour trente jours.
\subsection[{§ 75.}]{ \textsc{§ 75.} }
\noindent (1) Pendant que ces choses se passaient devant Alésia, les principaux de la Gaule, réunis en assemblée, avaient résolu, non d’appeler aux armes tous ceux qui étaient en état de les porter, comme le voulait Vercingétorix, mais d’exiger de chaque peuple un certain nombre d’hommes ; ils craignaient, dans la confusion d’une si grande multitude, de ne pouvoir ni la discipliner, ni se reconnaître, ni se nourrir. Il fut réglé que les divers états fourniraient, savoir :(2) les Édues, avec leurs clients les Ségusiens, les Ambivarètes, les Aulerkes-Brannovikes, les Brannoves, trente-cinq mille hommes ; les Arvernes avec les peuples de leur ressort, tels que les Eleutètes-Cadurkes, les Gabales, et les Vélannes, un pareil nombre ; (3) les Sénons, les Séquanes, les Bituriges, les Santons, les Rutènes, les Carnutes, chacun douze mille ; les Bellovakes, dix mille ; les Lémovikes, autant ; les Pictons, les Turons, les Parises, les Helves, huit mille chacun ; les Suessions, les Ambiens, les Médiomatrikes, les Pétrocores, les Nerves, les Morins, les Nitiobroges, chacun cinq mille ; les Aulerkes-Cénomans, autant ; les Atrébates, quatre mille ; les Bellocasses, les Lexoves, les Aulerkes-Éburovikes, chacun trois mille, les Raurakes avec les Boïes, trente mille ; (4) les pays situés le long de l’Océan, et que les Gaulois ont l’habitude d’appeler Armoriques, au nombre desquels sont les Curiosolites, les Rhédons, les Ambibares, les Calètes, les Osismes, les Lémovikes, les Vénètes, les Unelles, six mille hommes.(5) Les Bellovakes seuls refusèrent leur contingent, alléguant qu’ils voulaient faire la guerre aux Romains en leur nom et à leur gré, sans recevoir d’ordres de personne. Cependant, sur les instances de Commios, leur allié, ils envoyèrent deux mille hommes.
\subsection[{§ 76.}]{ \textsc{§ 76.} }
\noindent (1) C'était ce même Commios dont César, ainsi que nous l’avons dit plus haut, s’était servi comme d’un agent fidèle et utile dans la guerre de Bretagne, quelques années auparavant ; et en reconnaissance de ses services, César avait affranchi sa nation de tout tribut, lui avait rendu ses droits et ses lois et assujetti les Morins. (2) Mais tel fut l’empressement universel des Gaulois pour recouvrer leur liberté et reconquérir leur ancienne gloire militaire, que ni les bienfaits ni les souvenirs de l’amitié ne purent les toucher, et que nul sacrifice ne coûta à leur zèle, (3) puisqu’ils rassemblèrent huit mille cavaliers et environ deux cent quarante mille fantassins. Ces troupes furent passées en revue et le dénombrement en fut fait sur le territoire des Héduens ; on leur choisit des chefs, et le commandement général fut confié à l’Atrébate Commios, aux Héduens Viridomaros et Eporédorix, et à l’Arverne Vercassivellaunos, cousin de Vercingétorix. (4) On leur donna un conseil, formé de membres pris dans chaque cité, pour diriger la guerre. (5) Tous partent vers Alésia, pleins d’ardeur et de confiance ; (6) aucun ne croyait qu’il fût possible de soutenir seulement l’aspect d’une si grande multitude, surtout dans un double combat où les Romains seraient à la fois pressés par les sorties des assiégés, et enveloppés en dehors par tant de cavalerie et d’infanterie.
\subsection[{§ 77.}]{ \textsc{§ 77.} }
\noindent (1) Cependant les Gaulois assiégés dans Alésia, voyant que le jour où ils attendaient du secours était expiré, et que tout leur blé était consommé, ignorant d’ailleurs ce qui se passait citez les Héduens, s’étaient assemblés en conseil et délibéraient sur le parti qu’ils avaient à prendre. (2) Parmi les diverses opinions, dont les unes voulaient qu’on se rendît et les autres qu’on tentât une sortie vigoureuse tandis qu’il leur restait encore assez de forces, l’on ne peut, ce me semble, passer sous silence le discours de Critognatos, à cause de sa singulière et horrible cruauté. (3) C'était un Arverne d’une naissance élevée et qui jouissait d’une haute considération. "Je ne parlerai pas, dit-il, de l’avis de ceux qui appellent du nom de capitulation le plus honteux esclavage ; et je pense qu’on ne doit, ni les compter au nombre des citoyens, ni les admettre dans cette assemblée. (4) Je ne m’adresse qu’à ceux qui proposent une sortie, et dont l’opinion, comme vous le reconnaissez tous, témoigne qu’ils se souviennent encore de notre antique valeur. (5) Mais il y a plutôt de la faiblesse que du courage à ne pouvoir supporter quelques jours de disette. Les hommes qui s’offrent à la mort sans hésitation sont plus faciles à trouver que ceux qui savent endurer la douleur. (6) Et moi aussi je me rangerais à cet avis (tant l’honneur a sur moi d’empire), si je n’y voyais de péril que pour notre vie ; (7) mais, dans le parti que nous avons à prendre, considérons toute la Gaule que nous avons appelée à notre secours. (8) Lorsque quatre-vingt mille hommes auront péri dans cette tentative, quel courage pensez-vous que conservent nos parents et nos proches, s’ils ne peuvent, pour ainsi dire, combattre que sur nos cadavres ? (9) Gardez-vous donc de priver de votre soutien ceux qui n’ont pas craint de s’exposer pour votre salut, et, par précipitation, par imprudence, par pusillanimité, n’allez pas livrer toute la Gaule à l’avilissement d’un perpétuel esclavage. (10) Parce que vos auxiliaires ne sont pas arrivés au jour fixé, douteriez-vous de leur foi et de leur constance ? Eh quoi  ! quand les Romains travaillent tous les jours à des retranchements plus éloignés, pensez-vous que ce soit seulement pour se tenir en haleine ? (11) Si tout chemin vous est fermé par où vous pourriez avoir de leurs nouvelles, les Romains eux-mêmes ne vous assurent-ils pas de leur arrivée prochaine par ces travaux de jour et de nuit qui montrent assez la crainte qu’ils en ont ? (12) Quel est donc mon avis ? De faire ce que firent nos ancêtres dans leurs guerres, bien moins funestes, contre les Cimbres et les Teutons. Forcés, comme nous, de se renfermer dans leurs villes, en proie à la disette, ils soutinrent leur vie en se nourrissant de la chair de ceux que leur âge rendait inutiles à la guerre ; et ils ne se rendirent point. (13) Si nous n’avions pas reçu cet exemple, je dirais que, pour la cause de la liberté, il serait glorieux de le donner à nos descendants. (14) Quelle guerre en effet peut-on comparer à celle-ci ? Les Cimbres, après avoir ravagé la Gaule, et lui avoir fait de grands maux, sortirent enfin de notre territoire, et gagnèrent d’autres contrées ; ils nous laissèrent nos droits, nos lois, nos champs, notre liberté ! (15) Mais que demandent les Romains ? Que veulent-ils ? L'envie les amène contre tous ceux dont la renommée leur a fait connaître la gloire et la puissance dans la guerre ; ils veulent s’établir sur leur territoire, dans leurs villes, et leur imposer le joug d’une éternelle servitude. Car ils n’ont jamais fait la guerre dans d’autres vues. (16) Que si vous ignorez comment ils se conduisent chez les nations éloignées, voyez cette partie de la Gaule qui vous touche ; réduite en province, privée de ses droits et de ses lois, soumise aux haches romaines, elle gémit sous le poids d’un esclavage qui ne doit pas finir.
\subsection[{§ 78.}]{ \textsc{§ 78.} }
\noindent (1) Les avis ayant été recueillis, il fut arrêté que ceux qui, à raison de leur santé ou de leur âge, ne pouvaient rendre de service à la guerre, sortiraient de la place, et qu’on tenterait tout avant d’en venir au parti proposé par Critognatos. (2) On décida toutefois que, si l’on y était contraint et si les secours se faisaient trop attendre, on le suivrait plutôt que de se rendre ou de subir la loi des Romains. (3) Les Mandubiens, qui les avaient reçus dans leur ville, sont forcés d’en sortir avec leurs enfants et leurs femmes. (4) Ils s’approchent des retranchements des Romains, et, fondant en larmes, ils demandent, ils implorent l’esclavage et du pain. (5) Mais César plaça des gardes sur le rempart, et défendit qu’on les reçût.
\subsection[{§ 79.}]{ \textsc{§ 79.} }
\noindent (1) Cependant Commios et les autres chefs, investis du commandement suprême, arrivent avec toutes leurs troupes devant Alésia, après avoir occupé une colline extérieure, prennent position à mille pas au plus de nos retranchements. (2) Ayant le lendemain fait sortir la cavalerie de leur camp, ils couvrent toute cette plaine que nous avons dit avoir trois mille pas d’étendue, et tiennent, non loin de là, leurs troupes de pied cachées sur des hauteurs. (3) On voyait d’Alésia tout ce qui se passait dans la campagne. À la vue de ce secours, on s’empresse, on se félicite mutuellement, et tous les esprits sont dans la joie. (4) On fait sortir toutes les troupes, qui se rangent en avant de la place ; on comble le premier fossé ; on le couvre de claies et de terre, et on se prépare à la sortie et à tous les événements.
\subsection[{§ 80.}]{ \textsc{§ 80.} }
\noindent (1) César, ayant rangé l’armée tout entière sur l’une et l’autre de ses lignes, afin qu’au besoin chacun connût le poste qu’il devait occuper, fit sortir de son camp la cavalerie, à laquelle il ordonna d’engager l’affaire. (2) Du sommet des hauteurs que les camps occupaient, on avait vue sur le champ de bataille, et tous les soldats, attentifs au combat, en attendaient l’issue. (3) Les Gaulois avaient mêlé à leur cavalerie un petit nombre d’archers et de fantassins armés à la légère, tant pour la soutenir si elle pliait, que pour arrêter le choc de la nôtre. Plusieurs de nos cavaliers, surpris par ces fantassins, furent blessés et forcés de quitter la mêlée. (4) Les Gaulois, croyant que les leurs avaient le dessus, et que les nôtres étaient accablés par le nombre, se mirent, assiégés et auxiliaires, à pousser de toutes parts des cris et des hurlements pour encourager ceux de leur nation. (5) Comme l’action se passait sous les yeux des deux partis, nul trait de courage ou de lâcheté ne pouvait échapper aux regards, et l’on était de part et d’autre excité à se bien conduire, par le désir de la gloire et la crainte de la honte. (6) On avait combattu depuis midi jusqu’au coucher du soleil, et la victoire était encore incertaine, lorsque les Germains, réunis sur un seul point, en escadrons serrés, se précipitèrent sur l’ennemi et le repoussèrent. (7) Les archers, abandonnés dans cette déroute, furent enveloppés et taillés en pièces, et les fuyards poursuivis de tous côtés jusqu’à leur camp, sans qu’on leur donnât le temps de se rallier. (8) Alors ceux qui étaient sortis d’Alésia, consternés et désespérant presque de la victoire, rentrèrent dans la place.
\subsection[{§ 81.}]{ \textsc{§ 81.} }
\noindent (1) Après un jour employé par les Gaulois à faire une grande quantité de claies, d’échelles et de harpons, ils sortent silencieusement de leur camp au milieu de la nuit et s’approchent de ceux de nos retranchements qui regardaient la plaine. (2) Tout à coup poussant des cris, signal qui devait avertir de leur approche ceux que nous tenions assiégés, ils jettent leurs claies, attaquent les gardes de nos remparts à coups de frondes, de flèches et de pierres, et font toutes les dispositions pour un assaut. (3) Dans le même temps, Vercingétorix, entendant les cris du dehors, donne le signal avec la trompette et fait sortir les siens de la place. (4) Nos soldats prennent sur le rempart les postes qui avaient été, les jours précédents, assignés à chacun d’eux, et épouvantent les ennemis par la quantité de frondes, de dards, de boulets de plomb, de pierres, qu’ils avaient amassés dans les retranchements, et dont ils les accablent. (5) Comme la nuit empêchait de se voir, il y eut de part et d’autre beaucoup de blessés ; (6) les machines faisaient pleuvoir les traits. Cependant les lieutenants M. Antonius et C. Trébonius, à qui était échue la défense des quartiers attaqué s, tirèrent des forts plus éloignés quelques troupes pour secourir les légionnaires sur les points où ils les savaient pressés par l’ennemi.
\subsection[{§ 82.}]{ \textsc{§ 82.} }
\noindent (1) Tant que les Gaulois combattirent éloignés des retranchements, ils nous incommodèrent beaucoup par la grande quantité de leurs traits ; mais lorsqu’ils se furent avancés davantage, il arriva ou qu’ils se jetèrent sur les aiguillons qu’ils ne voyaient pas, ou qu’ils se percèrent eux-mêmes en tombant dans les fossés garnis de pieux, ou enfin qu’ils périrent sous les traits lancés du rempart et des tours. (2) Après avoir perdu beaucoup de monde, sans être parvenus à entamer les retranchements, voyant le jour approcher, et craignant d’être pris en flanc et enveloppés par les sorties qui se faisaient des camps situés sur les hauteurs, ils se replièrent sur les leurs. (3) Les assiégés, qui mettaient en usage les moyens préparés par Vercingétorix pour combler le premier fossé, (4) après beaucoup de temps employé à ce travail, s’aperçurent de la retraite de leurs compatriotes avant d’avoir pu approcher de nos retranchements. Abandonnant leur entreprise, ils rentrèrent dans la ville.
\subsection[{§ 83.}]{ \textsc{§ 83.} }
\noindent (1) Repoussés deux fois avec de grandes pertes, les Gaulois tiennent conseil sur ce qui leur reste à faire. Ils ont recours à des gens qui connaissent le pays et se font instruire par eux du site de nos forts supérieurs et de la manière dont ils sont fortifiés. (2) Il y avait au nord une colline qu’on n’avait pu comprendre dans l’enceinte de nos retranchements, à cause de son trop grand circuit ; ce qui nous avait obligés d’établir notre camp sur un terrain à mi-côte et dans une position nécessairement peu favorable. (3) Là commandaient les lieutenants C. Antistius Réginus et C. Caninius Rébilus avec deux légions. (4) Ayant fait reconnaître les lieux par leurs éclaireurs, les chefs ennemis forment un corps de soixante mille hommes, choisis dans toute l’armée gauloise et surtout parmi les nations qui avaient la plus haute réputation de courage. (5) Ils arrêtent secrètement entre eux quand et comment ils doivent agir ; ils fixent l’attaque à l’heure de midi, et mettent à la tête de ces troupes l’Arverne Vercasivellaunos, parent de Vercingétorix, et l’un des quatre généraux gaulois. (7) II sort de son camp à la première veille ; et ayant achevé sa route un peu avant le point du jour, il se cache derrière la montagne, et fait reposer ses soldats des fatigues de la nuit. (8) Vers midi, il marche vers cette partie da camp romain dont nous avons parlé plus haut. Dans le même temps la cavalerie ennemie s’approche des retranchements de la plaine, et le reste des troupes : gauloises commence à se déployer en bataille à la tête du camp.
\subsection[{§ 84.}]{ \textsc{§ 84.} }
\noindent (1) Du haut de la citadelle d’Alésia, Vercingétorix les aperçoit, et sort de la place, emportant du camp ses longues perches, ses galeries couvertes, ses faux et ce qu’il avait préparé, pour la sortie. (2) Le combat s’engage à la fois de toutes parts avec acharnement ; partout on fait les plus grands efforts. Un endroit paraît-il faible, on s’empresse d’y courir. (3) La trop grande étendue de leurs fortifications empêche les Romains d’en garder tous les points et de les défendre partout. (4) Les cris qui s’élevaient derrière nos soldats leur imprimaient d’autant plus de terreur, qu’ils songeaient que leur sûreté dépendait du courage d’autrui ; (5) car souvent le danger le plus éloigné est celui qui fait le plus d’impression sur les esprits.
\subsection[{§ 85.}]{ \textsc{§ 85.} }
\noindent (1) César, qui avait choisi un poste d’où il pouvait observer toute l’action, fait porter des secours partout où il en est besoin. (2) De part et d’autre on sent que ce jour est celui où il faut faire les derniers efforts. (3) Les Gaulois désespèrent entièrement de leur salut, s’ils ne forcent nos retranchements ; les Romains ne voient la fin de leurs fatigues que dans la victoire. (4) La plus vive action a lieu surtout aux forts supérieurs où nous avons vu que Vercasivellaunos avait été envoyé. L'étroite sommité qui dominait la pente était d’une grande importance. (5) Les uns nous lancent des traits, les autres, ayant formé la tortue, arrivent au pied du rempart : des troupes fraîches prennent la place de celles qui sont fatiguées. (6) La terre que les Gaulois jettent dans les retranchements les aide à les franchir, et comble les pièges que les Romains avaient cachés ; déjà les armes et les forces commencent à nous manquer.
\subsection[{§ 86.}]{ \textsc{§ 86.} }
\noindent (1) Dès qu’il en a connaissance, César envoie sur ce point Labiénus avec six cohortes ; (2) il lui ordonne, s’il ne peut tenir, de retirer les cohortes et de faire une sortie, mais seulement à la dernière extrémité. (3) II va lui-même exhorter les autres à ne pas céder à la fatigue ; il leur expose que le fruit de tous les combats précédents dépend de ce jour, de cette heure. (4) Les assiégés, désespérant de forcer les retranchements de la plaine, à cause de leur étendue, tentent d’escalader les hauteurs, et y dirigent tous leurs moyens d’attaque ; (4) ils chassent par une grêle de traits ceux qui combattaient du haut des tours ; ils comblent les fossés de terre et de fascines, et se fraient un chemin ; ils coupent avec des faux le rempart et le parapet.
\subsection[{§ 87.}]{ \textsc{§ 87.} }
\noindent (1) César y envoie d’abord le jeune Brutus avec six cohortes, ensuite le lieutenant C. Fabius avec sept autres ; (2) enfin, l’action devenant plus vive, il s’y porte lui-même avec un renfort de troupes fraîches. (3) Le combat rétabli et les ennemis repoussés, il se dirige vers le point où il avait envoyé Labiénus, (4) tire quatre cohortes du fort le plus voisin, ordonne à une partie de la cavalerie de le suivre, et à l’autre de faire le tour des lignes à l’extérieur et de prendre les ennemis à dos. (5) Labiénus, voyant que ni les remparts ni les fossés ne peuvent arrêter leur impétuosité, rassemble trente-neuf cohortes sorties des forts voisins et que le hasard lui présente, et dépêche à César des courriers qui l’informent de son dessein.
\subsection[{§ 88.}]{ \textsc{§ 88.} }
\noindent (1) César hâte sa marche pour assister à l’action. À son arrivée, on le reconnaît à la couleur du vêtement qu’il avait coutume de porter dans les batailles ; les ennemis, qui de la hauteur le voient sur la pente avec les escadrons et les cohortes dont il s’était fait suivre, engagent le combat. (2) Un cri s’élève de part et d’autre, et est répété sur le rempart et dans tous les retranchements. (3) Nos soldats, laissant de côté le javelot, tirent le glaive. Tout à coup, sur les derrières de l’ennemi, paraît notre cavalerie ; d’autres cohortes approchent ; les Gaulois prennent la fuite ; notre cavalerie barre le passage aux fuyards, et en fait un grand carnage. (4) Sédullus, chef et prince des Lémovices, est tué, et l’Arverne Vercasivellaunos pris vivant dans la déroute. Soixante-quatorze enseignes militaires sont rapportées à César ; d’un si grand nombre d’hommes, bien peu rentrent au camp sans blessure. (5) Les assiégés, apercevant du haut de leurs murs la fuite des leurs et le carnage qu’on en fait, désespèrent de leur salut, et retirent leurs troupes de l’attaque de nos retranchements. La nouvelle en arrive au camp des Gaulois, qui l’évacuent à l’instant. (6) Si les soldats n’eussent été harassés par d’aussi nombreux engagements et par les travaux de tout le jour, l’armée ennemie eût pu être détruite tout entière. (7) Au milieu de la nuit, la cavalerie, envoyée à la poursuite, atteint l’arrière-garde ; une grande partie est prise ou tuée ; le reste, échappé par la fuite, se réfugia dans les cités.
\subsection[{§ 89.}]{ \textsc{§ 89.} }
\noindent (1) Le lendemain Vercingétorix convoque l’assemblée, et dit : "Qu'il n’a pas entrepris cette guerre pour ses intérêts personnels, mais pour la défense de la liberté commune ; (2) que, puisqu’il fallait céder à la fortune, il s’offrait à ses compatriotes, leur laissant le choix d’apaiser les Romains par sa mort ou de le livrer vivant." On envoie à ce sujet des députés à César. (3) Il ordonne qu’on lui apporte les armes, qu’on lui amène les chefs. (4) Assis sur son tribunal, à la tête de son camp, il fait paraître devant lui les généraux ennemis. Vercingétorix est mis en son pouvoir ; les armes sont jetées à ses pieds. (5) À l’exception des Héduens et des Arvernes, dont il voulait se servir pour tâcher de regagner ces peuples, le reste des prisonniers fut distribué par tête à chaque soldat, à titre de butin.
\subsection[{§ 90.}]{ \textsc{§ 90.} }
\noindent (1) Ces affaires terminées, il part pour le pays des Héduens, et reçoit leur soumission. (2) Là, des députés envoyés par les Arvernes viennent lui promettre de faire ce qu’il ordonnera. César exige un grand nombre d’otages. (3) Il met ses légions en quartiers d’hiver, et rend environ vingt mille captifs aux Héduens et aux Arvernes. (4) Il fait partir T. Labiénus avec deux légions et la cavalerie pour le pays des Séquanes ; il lui adjoint M. Sempronius Rutilius. (5) Il place C. Fabius et L. Minucius Basilus avec deux légions chez les Rèmes, pour les garantir contre toute attaque des Bellovaques, leurs voisins. (6) Il envoie T. Antistius Réginus chez les Ambivarètes, T. Sextius chez les Bituriges, C. Caninius Rébilus chez les Rutènes, chacun avec une légion. (7) II établit Q. Tullius Cicéron et P. Sulpicius dans les postes de Cabillon (Châlons) et de Matiscon (Mâcon), au pays des Héduens, sur la Saône, pour assurer les vivres. Lui-même résolut de passer l’hiver à Bibracte. (8) Ces événements ayant été annoncés à Rome par les lettres de César, on ordonna vingt jours de prières publiques.
\section[{Livre VIII}]{Livre VIII}\renewcommand{\leftmark}{Livre VIII}

\noindent (1) Cédant à tes instances, Balbus, et puisque mes refus réitérés t’ont semblé moins une excuse tirée de la difficulté de l’entreprise qu’une défaite de la paresse, je me suis imposé une tâche bien difficile. (2) J'ai continué les commentaires de notre César sur ce qu’il a fait dans la Gaule, sans vouloir comparer mon ouvrage aux livres précédents ni à ceux qui le suivent. J'ai aussi achevé son dernier livre, qu’il laissa imparfait, depuis les événements d’Alexandrie jusqu’à la fin, non de nos dissensions civiles dont nous ne voyons pas encore le terme, mais de la vie de César. (3) Puissent ceux qui me liront être persuadés que je n’ai entrepris qu’à regret ce travail, et ne point m’accuser d’une vaine présomption pour m’être ainsi placé au milieu des écrits de César. C'est, en effet, une vérité reconnue de tout le monde, qu’il n’est pas d’ouvrage si laborieusement composé, que ces Commentaires ne surpassent en élégance. (5) Ils n’ont été écrits que pour servir de documents aux historiens ; mais leur supériorité est si généralement sentie qu’ils semblent moins avoir donné que ravi aux écrivains ultérieurs le moyen de traiter le même sujet. (6) Nous avons lieu de les admirer plus que personne : on en connaît la correction et la pureté ; nous seuls savons avec quelle facilité et quelle promptitude ils ont été composés. (7) Au talent d’écrire avec autant d’aisance que d’élégance, César joignait la plus parfaite exactitude dans l’explication de ses desseins. (8) Moi, je n’ai pas même l’avantage d’avoir assisté à la guerre d’Alexandrie ni à celle d’Afrique ; et, bien que je tienne de la bouche de César une partie des détails relatifs à ces guerres, autre chose est d’entendre des faits avec l’étonnement qu’excite la nouveauté, ou d’en avoir été soi-même le témoin. (9) Mais, tandis que je rassemble tous les motifs qui m’excusent de ne pouvoir être comparé avec César, je m’expose par cela même au reproche de vanité, en paraissant croire que l’idée de faire cette comparaison puisse venir à quelqu’un. Adieu.\par
\subsection[{§ 1.}]{ \textsc{§ 1.} }
\noindent (1) Toute la Gaule étant soumise, César, qui avait passé l’été précédent à faire la guerre sans la moindre interruption, désirait que l’armée pût au moins, dans ses quartiers d’hiver, se délasser de si grandes fatigues, lorsqu’on lui annonça que plusieurs nations se concertaient pour reprendre les armes. (2) L'on donnait à ce dessein, pour cause vraisemblable, la conviction où étaient alors tous les Gaulois, que, réunis sur un seul point, ils ne pourraient jamais résister aux Romains ; mais que si la guerre se faisait en diverses contrées à la fois, l’armée romaine n’aurait ni assez d’hommes ni assez de temps pour y faire face ; (3) qu’au reste nulle cité ne refuserait de supporter quelques maux passagers, si, par l’embarras qu’elle causerait ainsi à l’ennemi, elle aidait les autres pays à recouvrer leur liberté.
\subsection[{§ 2.}]{ \textsc{§ 2.} }
\noindent (1) Pour ne point laisser aux Gaulois le temps de s’affermir dans cette opinion, César, après avoir mis le questeur M. Antonius à la tête de ses quartiers d’hiver, partit lui-même de Bibracte avec une escorte de cavalerie, la veille des calendes de janvier, et se rendit près de la treizième légion, qu’il avait placée sur la frontière des Bituriges, à peu de distance de celle des Héduens ; il y ajouta la onzième, qui en était la plus proche. (2) Laissant deux cohortes pour la garde des bagages, il conduisit le reste de l’armée dans le pays fertile des Bituriges, qui, possédant un vaste territoire et beaucoup de places fortes, n’avaient pu être arrêtés par la présence d’une seule légion dans leurs préparatifs de guerre et leurs projets de révolte.
\subsection[{§ 3.}]{ \textsc{§ 3.} }
\noindent (1) La soudaine arrivée de César produisit son effet nécessaire sur des hommes dispersés et qui n’étaient préparés à aucune défense : cultivant leurs champs sans défiance, ils furent écrasés par la cavalerie, avant de pouvoir se réfugier dans leurs villes. (2) César, en effet, avait défendu d’incendier les habitations, signal ordinaire d’une invasion hostile, tant pour ne pas s’exposer à manquer de vivres et de fourrages, s’il voulait s’avancer dans le pays, que pour ne pas jeter la terreur parmi les habitants. (3) On fit plusieurs milliers de captifs. Les Bituriges, qui purent s’échapper à notre première approche, s’enfuirent effrayés chez les nations voisines avec lesquelles ils avaient des alliances ou des liens particuliers d’hospitalité. (4) Ce fut en vain ; César, par des marches forcées, arrivait sur tous les points, et ne laissait à aucune de ces nations le loisir de songer au salut des autres avant le sien. Cette célérité retenait dans le devoir les peuples amis, et ramenait à la soumission par la terreur ceux qui hésitaient encore. (5) En cet état, les Bituriges, voyant que la clémence de César leur offrait un moyen de recouvrer son amitié, et que les états voisins n’avaient eu à subir d’autre peine que de livrer des otages, suivirent cet exemple.
\subsection[{§ 4.}]{ \textsc{§ 4.} }
\noindent (1) César, pour récompenser de tant de fatigues et de patience des soldats dont le zèle extrême n’avait été ralenti, pendant l’hiver, ni par la difficulté des chemins, ni par la rigueur de froids insupportables, promit de leur donner deux cents sesterces, et aux centurions deux mille écus ; puis, ayant renvoyé les légions dans leurs quartiers, il revint lui-même à Bibracte après une absence de quarante jours. (2) Pendant qu’il y rendait la justice, les Bituriges lui envoyèrent des députés pour implorer son secours et se plaindre des Carnutes qui leur avaient déclaré la guerre. (3) À cette nouvelle, et bien qu’il ne se fût pas écoulé plus de dix-huit jours depuis son retour à Bibracte, il tira les quatorzième et sixième légions de leurs quartiers d’hiver, près de la Saône, où il les avait placées pour assurer les vivres, comme il est dit au livre précédent. Il partit avec ces deux légions à la poursuite des Carnutes.
\subsection[{§ 5.}]{ \textsc{§ 5.} }
\noindent (1) À la nouvelle de l’approche de l’armée, les ennemis, craignant le sort des autres peuples, évacuèrent les bourgs et les villes, où la nécessité leur avait fait dresser à la hâte de chétives cabanes pour passer l’hiver (car depuis leurs dernières défaites ils avaient abandonné plusieurs de leurs villes), et ils se dispersèrent de côté et d’autre. (2) Comme César ne voulait point exposer l’armée à toutes les rigueurs de l’âpre saison où l’on était alors, il établit son camp à Cénabum, ville des Carnutes, et logea les soldats, partie sous le toit des habitations gauloises, partie sous des tentes promptement recouvertes d’un peu de chaume. (3) Cependant il envoya la cavalerie et l’infanterie auxiliaire partout où l’on disait que les ennemis s’étaient retirés. Ce ne fut pas en vain ; car la plupart des nôtres revinrent chargés d’un butin considérable. (4) Les Carnutes, accablés par la rigueur de l’hiver et par la crainte du danger, chassés de leurs demeures sans oser s’arrêter longtemps nulle part, ne pouvant même trouver dans leurs forêts un abri contre les plus affreuses tempêtes, se dispersèrent après avoir perdu une grande partie des leurs, et se répandirent chez les nations voisines.
\subsection[{§ 6.}]{ \textsc{§ 6.} }
\noindent (1) Satisfait d’avoir, dans la saison la plus difficile de l’année, dissipé les rassemblements et prévenu la naissance d’une guerre ; persuadé d’ailleurs, autant que la raison pouvait le lui indiquer, qu’aucune guerre importante ne pouvait éclater avant l’été, César mit C. Trébonius en quartiers d’hiver à Cénabum avec les deux légions qui l’avaient suivi. (2) De nombreuses députations des Rèmes l’avertissaient que les Bellovaques, dont la gloire militaire surpassait celle de tous les Gaulois et des Belges, levaient, de concert avec les nations voisines, et rassemblaient, sous les ordres du Bellovaque Corréos et de l’Atrébate Commios, une armée qui devait fondre en masse sur les terres des Suessions. Jugeant alors qu’il n’importait pas moins à sa sûreté qu’à son honneur de préserver de toute injure des alliés qui avaient toujours si bien mérité de la république, (3) il fait de nouveau sortir de ses quartiers la onzième légion, écrit à C. Fabius d’amener sur les frontières des Suessions les deux légions qu’il avait, et demande à T. Labiénus l’une des deux siennes. C'est ainsi que, perpétuellement occupé lui-même, il répartissait le fardeau des expéditions entre les légions, à tour de rôle, et autant que le permettaient la situation des quartiers et le bien du service.
\subsection[{§ 7.}]{ \textsc{§ 7.} }
\noindent (1) Ces troupes réunies, il marche contre les Bellovaques, établit son camp sur leurs frontières, et envoie de tous côtés des détachements de cavalerie pour faire quelques prisonniers qui puissent l’instruire des desseins de l’ennemi. (2) De retour de cette mission, les cavaliers rapportent qu’ils ont trouvé peu d’habitants dans leurs demeures ; que ces gens n’étaient point restés pour cultiver la terre (car on s’était de toute part empressé de fuir), mais qu’ils avaient été laissés pour espionner. (3) César les ayant interrogés sur le lieu où s’était portée la masse des habitants et sur leurs desseins, apprit (4) que tous les Bellovaques en état de porter les armes s’étaient rassemblés sur un seul point avec les Ambiens, les Aulerques, les Calètes, les Véliocasses et les Atrébates ; qu’ils étaient campés sur une hauteur, dans un bois environné d’un marais ; qu’ils avaient porté tous leurs bagages dans des forêts plus reculées. Plusieurs chefs les excitaient à la guerre ; celui d’entre eux qui exerçait le plus d’autorité sur la multitude était Corréos, dont on connaissait la haine implacable pour le nom romain. (5) Peu de jours auparavant, l’Atrébate Commios avait quitté le camp pour se rendre dans les contrées germaines les plus proches, et en ramener des secours considérables. (6) Les Bellovaques avaient arrêté, du consentement de tous les chefs, et selon le voeu de la multitude, que si, comme on le disait, César ne marchait contre eux qu’avec trois légions, ils lui présenteraient la bataille, de peur d’être ensuite obligés de combattre avec plus de désavantage et de perte contre toutes ses troupes ; (7) s’il amenait un plus grand nombre de légions, ils devaient se tenir dans le lieu qu’ils avaient choisi ; et se borner, en tendant des pièges aux Romains, à leur ôter les vivres et les fourrages, qui, vu l’époque où l’on se trouvait, étaient très rares et fort disséminés.
\subsection[{§ 8.}]{ \textsc{§ 8.} }
\noindent (1) S'étant assuré de la vérité de ces faits par l’accord des témoignages, et trouvant ce plan rempli de prudence et bien éloigné de la témérité ordinaire aux Barbares, César jugea qu’il devait tout mettre en oeuvre pour engager les ennemis, par le mépris de ses propres forces, à en venir aux mains avec lui le plus tôt possible. (2) Il avait près de lui de vieilles légions d’un courage éprouvé, la septième, la huitième et la neuvième, et de plus la onzième, composée d’une jeunesse d’élite et de grande espérance, qui comptait déjà huit campagnes, mais n’avait pas encore, comparativement aux autres, la même réputation de valeur et d’ancienneté. (3) Ayant donc convoqué un conseil, il y exposa tout ce qu’il avait appris, échauffa le courage de ses troupes, et régla sa marche de manière à attirer les ennemis au combat en ne leur montrant que trois légions. Les septième, huitième et neuvième devaient marcher en avant, tandis que toute la colonne des bagages (et ils étaient peu nombreux, comme il est d’usage dans de simples expéditions) viendrait à la suite sous l’escorte de la onzième, afin que les ennemis ne pussent voir que le nombre de légions qu’ils désiraient. (4) Dans cet ordre, formant à peu près un bataillon carré, il arriva à la vue des ennemis plus tôt qu’ils ne s’y attendaient.
\subsection[{§ 9.}]{ \textsc{§ 9.} }
\noindent (1) Quand les Gaulois, dont la détermination avait été annoncée à César comme certaine, virent tout à coup les légions marcher à eux en ordre de bataille et d’un pas assuré, soit crainte de combattre, soit simple étonnement de notre arrivée soudaine, ou pour attendre le parti que nous prendrions, ils rangèrent leurs troupes en avant de leur camp et ne quittèrent point la hauteur. (2) Quoiqu’il désirât de combattre, César, considérant cette multitude d’ennemis dont le séparait un vallon plus profond que large, se détermina à asseoir son camp en face du leur. (3) Il ordonne d’élever un rempart de douze pieds avec un parapet proportionné à cette hauteur ; de creuser en avant deux fossés de quinze pieds, dont chaque côté était coupé en ligne droite ; il fait dresser un grand nombre de tours à trois étages, jointes ensemble par des ponts et des galeries, dont le front était garni de mantelets d’osier, (4) de telle sorte que l’ennemi fût arrêté par un double fossé et par un double rang de combattants. Le premier rang sur les galeries, et conséquemment moins exposé, lançait ses traits avec plus d’assurance et de portée ; le second, placé sur le rempart même et plus près de l’ennemi, était protégé par la galerie contre la chute des traits. II plaça des portes et de plus hautes tours aux issues du camp.
\subsection[{§ 10.}]{ \textsc{§ 10.} }
\noindent (1) En se retranchant ainsi, il avait un double motif ; car d’une part il espérait que de si grands travaux, pris pour des marques de frayeur, augmenteraient la confiance des Barbares ; et comme, d’un autre côté, il fallait chercher au loin des fourrages et des vivres, on pouvait, à l’abri de ces retranchements, défendre le camp avec peu de troupes. (2) Cependant il se livrait souvent de petits combats entre les deux camps, séparés par un marais. Quelquefois c’étaient nos auxiliaires gaulois et germains qui passaient ce marais et poursuivaient vivement les ennemis ; quelquefois, à leur tour, c’étaient ceux-ci qui, franchissant le marais, nous repoussaient au loin. (3) II arrivait aussi, vu l’obligation où l’on était tous les jours de se diviser pour aller chercher des vivres dans des habitations éparses, que nos fourrageurs dispersés étaient enveloppés dans des lieux désavantageux ; (4) ce qui, bien que le dommage se réduisît à la perte d’un petit nombre de valets et de chevaux, ne laissait pas d’augmenter la folle présomption des Barbares. Ajoutez que Commios, lequel j’ai dit être parti en Germanie pour y chercher des secours, en était revenu avec des cavaliers. Leur nombre n’excédait pas cinq cents ; toutefois, leur arrivée avait rendu les Barbares plus arrogants.
\subsection[{§ 11.}]{ \textsc{§ 11.} }
\noindent (1) César, voyant que l’ennemi, défendu par un marais et par sa position, se tenait depuis plusieurs jours dans son camp, et jugeant qu’il ne pouvait l’attaquer sans de grandes pertes, ni l’enfermer dans des lignes sans un renfort de troupes, écrivit à Trébonius d’appeler le plus promptement possible la treizième légion qui hivernait chez les Bituriges avec le lieutenant T. Sextius, et de venir le joindre à grandes journées avec trois légions. (2) Il employa tour à tour les cavaliers des Rèmes, des Lingons et des autres états qui lui en avaient fourni un grand nombre, à protéger les fourrages et à soutenir les attaques soudaines des ennemis.
\subsection[{§ 12.}]{ \textsc{§ 12.} }
\noindent (1) Comme cette manoeuvre avait lieu chaque jour et que déjà, par l’habitude même, on était devenu moins diligent (effet ordinaire de la durée), les Bellovaques, connaissant les postes habituels de nos cavaliers, choisirent un corps d’infanterie et le mirent en embuscade dans un bois : (2) le lendemain ils envoyèrent de la cavalerie pour y attirer la nôtre, l’envelopper et l’attaquer. (3) Ce malheureux sort tomba sur les Rèmes qui, ce jour-là, se trouvaient en tour de service. Ils eurent à peine aperçu la cavalerie ennemie à laquelle ils se croyaient supérieurs, que, méprisant son petit nombre, ils la poursuivirent avec ardeur ; ils furent alors entourés de tous côtés par les fantassins. (4) Étonnés de cette attaque, ils se retirèrent plus vite qu’il n’est d’usage dans un combat de cavalerie. Ils avaient perdu dans l’action le chef de leur nation, Vertiscos, commandant de la cavalerie. (5) Il pouvait à peine, à cause de son grand âge, se soutenir à cheval ; mais fidèle aux coutumes gauloises, il n’avait ni fait valoir cette excuse de l’âge pour se dispenser du commandement, ni voulu que l’on combattit sans lui. (6) La fierté des ennemis s’accrut par l’avantage qu’ils venaient de remporter et par la mort du chef et du commandant des Rèmes ; mais cet échec avertit les nôtres de mettre plus de soin à explorer les lieux avant d’y placer des postes, et plus de modération dans la poursuite de l’ennemi lorsqu’il céderait le terrain.
\subsection[{§ 13.}]{ \textsc{§ 13.} }
\noindent (1) Cependant il ne se passait pas un seul jour où il n’y eût, à la vue des deux camps, quelque escarmouche vers les endroits guéables du marais. (2) Dans l’un de ces combats, l’infanterie germaine, que César avait fait venir d’outre-Rhin pour la mêler à la cavalerie, ayant tout entière franchi le marais avec intrépidité, et tué le petit nombre d’ennemis qui résistaient, poursuivit le reste avec une telle vigueur qu’elle frappa d’épouvante non seulement ceux qu’elle serrait de près ou qui étaient encore à la portée du trait, mais même les soldats de la réserve, qui s’enfuirent honteusement. (3) Chassés de hauteurs en hauteurs, ils ne s’arrêtèrent que lorsqu’ils furent arrivés à leur camp ; la peur en emporta même plusieurs au-delà. (4) Tel fut le trouble où le danger avait jeté toutes ces troupes, qu’il était difficile de juger si elles montraient plus d’orgueil au moindre avantage que de timidité au moindre revers.
\subsection[{§ 14.}]{ \textsc{§ 14.} }
\noindent (1) Après plusieurs jours passés dans leur camp, et à la nouvelle de l’approche des légions qu’amenait le lieutenant C. Trébonius, les chefs bellovaques, craignant un siège semblable à celui d’Alésia, firent partir de nuit avec le bagage ceux que l’âge, les infirmités ou le défaut d’armes rendaient inutiles. (2) Tandis qu’ils s’occupaient à mettre en ordre cette multitude remplie de trouble et de confusion (car les Gaulois, dans les moindres expéditions, se font toujours suivre d’un grand nombre de chariots), ils furent surpris par le jour, et rangèrent quelques troupes en bataille à la tête de leur camp, pour donner aux bagages le temps de s’éloigner, avant que les Romains pussent les atteindre. (3) De son côté, César ne jugeant convenable de les attaquer ni de front, ni dans la retraite, à cause de l’escarpement de la colline, résolut toutefois de faire assez avancer les légions pour que les barbares ne pussent se retirer sans péril en leur présence. (4) Mais comme le marais situé entre les deux camps pouvait retarder la poursuite, à cause de la difficulté du passage, et que de l’autre côté de l’eau, la hauteur touchait presque au camp ennemi, dont elle n’était séparée que par un petit vallon, il jeta des ponts de claies sur le marais, fit passer les légions, et gagna rapidement la hauteur dont la pente servait de rempart des deux côtés. (5) Les légions y montèrent en ordre de bataille, et, parvenues au sommet, s’y déployèrent dans une position d’où les traits lancés par nos machines pouvaient porter sur les rangs ennemis.
\subsection[{§ 15.}]{ \textsc{§ 15.} }
\noindent (1) Les Barbares, se fiant à l’avantage de leur position, continuaient de s’y tenir en bataille, prêts à combattre si les Romains venaient les attaquer sur la colline, mais n’osant faire défiler leurs troupes en détail, de peur d’être mis en désordre s’ils se divisaient. (2) César, connaissant leur ferme résolution, laissa vingt cohortes sous les armes, traça le camp en cet endroit et ordonna de le retrancher. (3) Les travaux finis, il rangea les légions en bataille à la tête de ses retranchements, et plaça aux avant-postes les cavaliers avec leurs chevaux tout bridés. (4) Les Bellovaques, voyant les Romains toujours prêts à les suivre, et sentant qu’ils ne pouvaient eux-mêmes, ni veiller toutes les nuits, ni séjourner plus longtemps sans vivres dans leur position, imaginèrent d’effectuer leur retraite par le moyen qui suit. (5) Comme les Gaulois, ainsi qu’il a été dit dans les livres précédents, ont coutume, quand ils restent en ligne, de s’asseoir sur des faisceaux de branches et de paille, ils en avaient une grande quantité qu’ils se passèrent de main en main et qu’ils placèrent à la tête de leur camp ; puis, à la fin du jour, et à un signal donné, ils y mirent le feu en même temps. (6) Cette vaste flamme nous déroba tout à coup la vue de leurs troupes, et les Barbares en profitèrent pour s’enfuir à toutes jambes.
\subsection[{§ 16.}]{ \textsc{§ 16.} }
\noindent (1) Bien que César ne pût apercevoir le départ des ennemis, à cause des flammes qu’il avait en face de lui, il ne laissa pas de soupçonner que cet incendie n’était qu’une ruse propre à cacher leur retraite. II fit alors avancer les légions et envoya des escadrons à la poursuite ; mais, dans la crainte de quelque embuscade, et de peur que l’ennemi, resté peut-être à la même place, ne cherchât à attirer nos soldats dans une mauvaise position, il ne s’avança lui-même que lentement. (2) Nos cavaliers n’osaient pénétrer à travers une flamme et une fumée très épaisses ; et si quelques-uns, plus hardis, essayaient de le faire, à peine voyaient-ils la tête de leurs chevaux. La crainte d’un piège fit qu’on laissa à l’ennemi tout le temps nécessaire pour opérer sa retraite. (3) C'est ainsi que par une ruse où la peur et l’adresse avaient eu également part, les Bellovaques franchirent sans la moindre perte un espace de dix milles, et assirent leur camp dans un lieu très avantageux. (4) De là leurs cavaliers et leurs fantassins incommodaient beaucoup nos fourrageurs par leurs fréquentes embuscades.
\subsection[{§ 17.}]{ \textsc{§ 17.} }
\noindent (1) Ces attaques se renouvelaient souvent, lorsque César apprit d’un prisonnier que Corréos, chef des Bellovaques, avait fait choix de six mille fantassins des plus braves et de mille cavaliers pour les placer en embuscade dans un lieu où l’abondance du blé et des fourrages lui faisait soupçonner que les Romains viendraient s’approvisionner. (2) Instruit de ce dessein, César fit sortir plus de légions que de coutume, et envoya en avant la cavalerie qu’il était dans l’usage de donner pour escorte aux fourrageurs. Il y mêla des fantassins armés à la légère, et lui-même s’avança avec les légions le plus qu’il lui fut possible.
\subsection[{§ 18.}]{ \textsc{§ 18.} }
\noindent (1) Les ennemis avaient fait choix, pour leur embuscade, d’une plaine qui, en tous sens, n’avait pas plus de mille pas d’étendue ; elle était entourée d’épaisses forêts et d’une rivière très profonde ; des pièges nous enveloppaient de tous côtés. (2) Nos cavaliers, instruits du projet de l’ennemi, ayant le coeur non moins préparé que leurs armes au combat, et appuyés d’ailleurs par les légions, auraient accepté tout genre d’engagement ; ils arrivèrent en escadrons. (3) Corréos, jugeant l’occasion favorable, se montra d’abord avec peu de monde, et chargea ceux de nos escadrons qui se trouvèrent le plus près de lui. (4) Les nôtres soutinrent le choc avec fermeté et sans se réunir en masse, manoeuvre ordinaire dans les combats de cavalerie, dans un moment d’alarme, mais qui devient nuisible par la confusion qu’elle produit.
\subsection[{§ 19.}]{ \textsc{§ 19.} }
\noindent (1) Tandis qu’on se bat par escadrons et en petites troupes, et qu’on manoeuvre de manière à ne pas se laisser prendre en flanc, le reste des ennemis, voyant Corréos dans la mêlée, sort tout à coup des forêts. (2) Un vif combat s’engage sur tous les points, et se soutient longtemps à armes égales, lorsque l’infanterie ennemie quitte le bois, s’avance en ordre de bataille, et force nos cavaliers de reculer. À leur secours arrive aussitôt l’infanterie légère que César, comme on l’a dit, avait envoyée en avant des légions ; elle se mêle aux escadrons et combat avec courage. (3) L'affaire resta quelque temps encore indécise ; mais ensuite, comme il devait arriver, ceux qui avaient soutenu le premier choc des ennemis embusqués, obtinrent la supériorité, par cela même qu’ils avaient échappé aux effets de la surprise. (4) Cependant les légions s’approchent de plus en plus, et de nombreux courriers annoncent, tant aux Romains qu’aux ennemis, la prochaine arrivée de César à la tête de ses troupes en bataille. (5) À cette nouvelle, les nôtres, sûrs de l’appui des cohortes, combattent avec plus d’ardeur, de peur de partager avec les légions l’honneur de la victoire. (6) Les ennemis perdent courage et cherchent à s’enfuir par divers chemins ; mais c’est en vain, car ils sont arrêtés par les obstacles même qu’ils avaient disposés pour enfermer les Romains. (7) Vaincus et repoussés, après avoir perdu la plus grande partie des leurs, ils fuient en désordre et au hasard, les uns vers les forêts, d’autres vers le fleuve ; ils sont massacrés par notre cavalerie qui les poursuit à toute bride. (8) Cependant Corréos, que n’avait abattu aucune infortune ; qui n’avait voulu ni quitter le combat, ni gagner les forêts, ni se rendre, malgré nos pressantes invitations, se battit avec le plus grand courage et, par ses coups redoublés, força les vainqueurs irrités à le percer de leurs traits.
\subsection[{§ 20.}]{ \textsc{§ 20.} }
\noindent (1) Après ce succès, César, marchant environné des trophées de sa récente victoire, pensa bien que l’ennemi, abattu par la nouvelle d’un si grand revers, abandonnerait son camp situé à huit mille pas environ du lieu où s’était livrée la bataille. Aussi, et bien qu’il y eût une rivière à traverser, il n’hésita point à la faire passer à l’armée, et marcha en avant. (2) Mais, de leur côté, les Bellovaques et les autres états, instruits de la dernière défaite par le petit nombre de fuyards et de blessés qui avaient pu échapper au carnage à la faveur des bois, voyant que la fortune leur était en tout contraire, que Corréos avait été tué, qu’ils avaient perdu leur cavalerie et l’élite de leur infanterie, qu’enfin les Romains approchaient, convoquèrent aussitôt une assemblée au son de trompe, et s’écrièrent qu’il fallait envoyer à César des députés et des otages.
\subsection[{§ 21.}]{ \textsc{§ 21.} }
\noindent (1) Cet avis étant unanimement adopté, l’Atrébate Commios s’enfuit chez ces mêmes Germains auxquels il avait emprunté des secours pour cette guerre. (2) Les autres envoient sur-le-champ des députés à César et le prient de se contenter d’un châtiment que sa clémence et son humanité ne leur auraient jamais infligé s’il avait eu à les punir avant qu’un combat leur eût fait essuyer tant de désastres : (3) "la dernière bataille a détruit toute leur cavalerie ; plusieurs milliers de fantassins d’élite ont péri ; à peine s’en est-il échappé pour annoncer la défaite. (4) Toutefois, au milieu de tant de calamités, les Bellovaques ont recueilli un grand avantage de la mort de Corréos, auteur de cette guerre, instigateur de la multitude ; de son vivant, le sénat avait moins d’autorité qu’une populace ignorante."
\subsection[{§ 22.}]{ \textsc{§ 22.} }
\noindent (1) César répond à cette harangue et aux prières des députés que déjà l’année précédente les Bellovaques et les autres peuples de la Gaule lui avaient fait la guerre en même temps ; qu’eux seuls avaient persisté dans la révolte, sans se laisser ramener au devoir par l’exemple de la soumission des autres. (2) Il est très facile, il le sait bien, de rejeter sur les morts les fautes que l’on a faites ; mais nul particulier n’est assez puissant par lui-même ou avec le secours d’une misérable poignée de populace, pour exciter et soutenir une guerre malgré les chefs, en dépit du sénat, contre le voeu de tous les gens de bien. Toutefois, il veut bien se contenter du mal qu’ils se sont fait à eux-mêmes.
\subsection[{§ 23.}]{ \textsc{§ 23.} }
\noindent (1) La nuit suivante, les députés rapportent cette réponse à leurs concitoyens, qui préparent aussitôt des otages. Les autres états, qui étaient dans l’attente du résultat, (2) s’empressent également de donner des otages et de se soumettre, à l’exception de Commios, que la crainte empêchait de se confier à la foi de qui que ce fût. (3) En effet, l’année précédente, pendant que César rendait la justice dans la Gaule citérieure, T. Labiénus, instruit que Commios sollicitait les peuples à se soulever contre César, avait cru pouvoir, sans se rendre coupable de perfidie, réprimer cette trahison. (4) Présumant que Commios ne viendrait pas au camp s’il y était appelé, craignant en outre que cette invitation ne l’avertît d’être circonspect, il avait envoyé vers lui C. Volusénus Quadratus qui, sous prétexte d’une entrevue, était chargé de le tuer. Des centurions, propres à l’exécution de ce projet, lui avaient été donnés pour escorte. (5) Lorsqu’on fut en présence, et que, selon le signal convenu, Volusénus eut pris la main de Commios, le centurion, soit qu’il se troublât, soit que les amis de Commios eussent prévenu ce meurtre, ne put achever le Gaulois ; cependant il le blessa grièvement à la tête du premier coup. (6) De part et d’autre on tira l’épée, moins pour se battre que pour s’assurer la retraite ; les nôtres croyaient Commios mortellement blessé ; et les Gaulois, reconnaissant le piège, craignaient de plus grands périls encore. On disait que, depuis cet événement, Commios avait résolu de ne jamais paraître devant un Romain.
\subsection[{§ 24.}]{ \textsc{§ 24.} }
\noindent (1) Vainqueur des nations les plus belliqueuses, César ne voyait plus aucune cité se préparer à la guerre ou en état de lui résister ; mais remarquant qu’un grand nombre d’habitants quittaient les villes et s’enfuyaient des campagnes pour se soustraire à la domination nouvelle, il résolut de distribuer l’armée sur différents points. (2) Il garda près de lui le questeur M. Antonius avec la onzième légion ; il envoya le lieutenant C. Fabius, avec vingt-cinq cohortes, à l’extrémité opposée de la Gaule, où l’on disait qu’il y avait plusieurs peuples en armes ; il ne croyait pas que le lieutenant C. Caninius Rébilus, qui commandait dans ces contrées, fût assez fort avec les deux légions qu’il avait sous ses ordres. (3) Il fit venir près de lui T. Labiénus, et envoya la douzième légion, qui avait hiverné avec ce lieutenant, protéger les colonies romaines dans la Gaule citérieure, et les préserver de calamités semblables à celles qu’avaient essuyées, l’été précédent, les Tergestins, dont le territoire avait été ravagé par suite d’une irruption soudaine de Barbares. (4) Pour lui, il alla dévaster les terres d’Ambiorix. Désespérant de réduire en son pouvoir cet ennemi fugitif et tremblant, il crut, dans l’intérêt de son honneur, devoir détruire si bien, dans les états de ce prince, les citoyens, les édifices, les bestiaux, que désormais en horreur à ceux qui échapperaient par hasard au massacre, Ambiorix ne pût jamais rentrer dans un pays sur lequel il aurait attiré tant de désastres.
\subsection[{§ 25.}]{ \textsc{§ 25.} }
\noindent (1) Lorsque César eut distribué ses légions et ses auxiliaires sur toutes les parties du territoire d’Ambiorix, que tout y eut été détruit par le meurtre, l’incendie, le pillage, et qu’un grand nombre d’hommes eurent été pris ou tués, il envoya Labiénus avec deux légions chez les Trévires, (2) peuple qui, sans cesse en guerre à cause du voisinage de la Germanie, ne différait pas beaucoup des Germains pour les moeurs et la férocité, et n’obéissait jamais aux ordres de César que par la force des armes.
\subsection[{§ 26.}]{ \textsc{§ 26.} }
\noindent (1) Cependant le lieutenant C. Caninius, informé par Duratios qui était toujours resté attaché aux Romains, malgré la défection d’une partie de sa nation, qu’un grand nombre d’ennemis s’étaient rassemblés sur les frontières des Pictons, se dirigea vers la place de Lémonum. (2) Comme il en approchait, des prisonniers l’instruisirent que Duratios s’y trouvait assiégé par plusieurs milliers d’hommes sous la conduite de Dumnacos, chef des Andes. N'osant combattre avec si peu de légions, il campa dans une forte position. (3) Dumnacos, à la nouvelle de l’approche de Caninius, tourna toutes ses forces contre les légions ; et vint attaquer le camp des Romains. (4) Mais ayant perdu plusieurs jours et beaucoup de monde à cette attaque, sans avoir pu faire la moindre brèche à nos retranchements, il revint assiéger Lémonum.
\subsection[{§ 27.}]{ \textsc{§ 27.} }
\noindent (1) Dans le même temps, le lieutenant C. Fabius, occupé à recevoir les soumissions et les otages de diverses nations, apprit, par les lettres de Caninius, ce qui se passait chez les Pictons. À cette nouvelle, il partit au secours de Duratios. (2) Mais Dumnacos fut à peine instruit de son arrivée, que, désespérant de se sauver, s’il lui fallait à la fois résister à l’ennemi du dehors et avoir l’oeil sur les assiégés qui le tenaient en crainte, il se hâta de se retirer avec ses troupes, et ne se crut point en sûreté qu’il ne les eût conduites au-delà de la Loire, qu’il fallait passer sur un pont, à cause de la largeur du fleuve. (3) Quoique Fabius n’eût pas encore paru devant l’ennemi, ni joint Caninius, cependant, sur le rapport de ceux qui connaissaient la nature du pays, il conjectura que les ennemis, frappés de terreur, prendraient la route qui conduisait à ce pont. (4) Il s’y dirigea donc avec ses troupes, et ordonna à la cavalerie de devancer les légions, de manière pourtant à pouvoir, au besoin, se replier sur le camp sans fatiguer les chevaux. (5) Nos cavaliers, conformément à leurs ordres, s’avancent et joignent l’armée de Dumnacos ; ils attaquent, en route et sous leurs bagages, les ennemis qui fuient épouvantés, leur tuent beaucoup de monde, font un riche butin, et rentrent au camp, après ce beau succès.
\subsection[{§ 28.}]{ \textsc{§ 28.} }
\noindent (1) La nuit suivante, Fabius envoie la cavalerie en avant, avec ordre de harceler les ennemis et de retarder leur marche, tandis qu’il suivrait lui-même. (2) Dans ce dessein, Q. Atius Varus, préfet de la cavalerie, homme d’un courage égal à sa prudence, exhorte sa troupe, atteint l’ennemi, fait prendre de bonnes positions à une partie de ses escadrons, et, à la tête des autres, engage le combat. (3) La cavalerie ennemie résiste avec audace, appuyée qu’elle est par le corps entier des fantassins qui avaient fait halte pour lui porter secours. (4) L'action fut très vive ; car nos cavaliers, méprisant des ennemis qu’ils avaient vaincus la veille ; et sachant que les légions étaient à leur suite, se battaient contre les fantassins avec une extrême valeur ; ils étaient animés et par la honte de reculer et par le désir de recueillir seuls toute la gloire de cette affaire. (5) De leur côté, les ennemis, ne croyant pas avoir à combattre plus de troupes que la veille, pensaient avoir trouvé l’occasion de détruire notre cavalerie.
\subsection[{§ 29.}]{ \textsc{§ 29.} }
\noindent (1) Il y avait quelque temps que l’on combattait avec une égale opiniâtreté, lorsque Dumnacos mit son infanterie en bataille pour soutenir sa cavalerie. En ce moment paraissent tout à coup aux yeux des ennemis les légions en rangs serrés. (2) À cette vue, frappés d’une terreur bientôt suivie du plus grand désordre dans les bagages, les Barbares, tant cavaliers que fantassins, s’enfuient çà et là en jetant de grands cris. (3) Notre cavalerie, dont la valeur venait de triompher de la résistance des ennemis, transportée de joie à l’aspect du succès, et faisant partout entendre des cris de victoire, se jette sur les fuyards et en tue autant que les chevaux en peuvent poursuivre et que les bras en peuvent frapper. (4) Ainsi périrent plus de douze mille hommes, soit les armes à la main, soit après les avoir jetées ; tout le bagage tomba en notre pouvoir.
\subsection[{§ 30.}]{ \textsc{§ 30.} }
\noindent (1) Après cette déroute, cinq mille hommes au plus furent recueillis par le Sénon Drappès, le même qui, à la première révolte de la Gaule, avait rassemblé une foule d’hommes perdus, promis la liberté aux esclaves, fait appel aux exilés de tous les pays, et enrôlé des brigands, avec lesquels il interceptait nos bagages et nos convois. Dès qu’on sut qu’il marchait sur la province, de concert avec le Cadurque Luctérios (qui déjà, comme, on l’a vu au livre précédent, avait voulu y faire une invasion, lors du premier soulèvement de la Gaule), (2) le lieutenant Caninius se mit à leur poursuite avec deux légions, pour éviter la honte de voir des hommes souillés de brigandages causer quelque effroi ou quelque dommage à notre province.
\subsection[{§ 31.}]{ \textsc{§ 31.} }
\noindent (1) C. Fabius marcha avec le reste de l’armée contre les Carnutes et les autres nations dont il savait que Dumnacos avait obtenu des secours dans le dernier combat. (2) Il ne doutait point que leur défaite récente ne les rendît plus soumis ; mais que, s’il leur laissait le temps de se remettre de leur effroi, les instances de Dumnacos ne pussent encore les soulever. (3) Dans cette conjoncture, Fabius parvint avec autant de bonheur que de célérité à tout faire rentrer dans le devoir. Les Carnutes qui, souvent battus, n’avaient jamais parlé de paix, donnèrent des otages et se soumirent. (4) Entraînés par leur exemple, les autres peuples, qui habitent à l’extrême limite de la Gaule, près de l’Océan, et qu’on appelle Armoricains, nous obéirent sans délai, à l’arrivée de Fabius et des légions. (5) Dumnacos, chassé de son territoire, errant, réduit à se cacher, fut obligé de gagner seul les régions de la Gaule les plus reculées.
\subsection[{§ 32.}]{ \textsc{§ 32.} }
\noindent (1) Drappès et Luctérios, apprenant l’arrivée des légions et de Caninius, sentirent que, poursuivis par l’armée, ils ne pourraient pénétrer sur le territoire de la province sans une perte certaine, ni continuer en liberté leurs courses et leurs brigandages. Ils s’arrêtèrent sur les terres des Cadurques. (2) Luctérios, qui avant ses revers avait eu beaucoup d’influence sur ses concitoyens, et à qui son audace, toujours prête à de nouvelles entreprises, donnait un grand crédit parmi les Barbares, vint, avec ses troupes, unies à celles de Drappès, occuper la place d’Uxellodunum, anciennement dans sa clientèle, et très forte par sa position. Il en gagna les habitants.
\subsection[{§ 33.}]{ \textsc{§ 33.} }
\noindent (1) C. Caninius, s’y étant aussitôt porté, constata que la place était de tous côtés défendue par des rochers escarpés, qui en eussent rendu, même sans garnison, l’accès difficile à des hommes armés : mais sachant que les bagages des habitants étaient nombreux, et qu’on ne pouvait essayer de les faire sortir en secret, sans qu’ils fussent atteints par la cavalerie et même par les légions, il divisa les cohortes en trois parties, établit trois camps dans des positions très élevées, (2) et de là il commença peu à peu, autant que le permettait le nombre des troupes, à tirer une ligne de circonvallation autour de la place.
\subsection[{§ 34.}]{ \textsc{§ 34.} }
\noindent (1) À cette vue, les habitants, se rappelant tous les malheurs d’Alésia, redoutant un sort semblable, et avertis par Luctérios, qui avait assisté à ce désastre, de pourvoir surtout aux subsistances, arrêtent, d’un consentement unanime, qu’après avoir laissé dans la place une partie des troupes, les autres iront chercher des vivres. (2) Cette résolution prise, Drappès et Luctérios laissent dans la ville deux mille hommes de garnison, et sortent la nuit suivante avec le reste. (3) En peu de jours ils eurent ramassé une grande quantité de blé sur les terres des Cadurques, dont les uns le livrèrent de leur plein gré, et les autres le laissèrent prendre, ne pouvant s’y opposer. Cependant nos forts eurent à essuyer plusieurs fois des attaques nocturnes ; (4) circonstance qui engagea Caninius à suspendre la circonvallation, dans la crainte de ne pouvoir défendre la totalité de ses lignes, ou de n’avoir, sur plusieurs points, que des postes insuffisants.
\subsection[{§ 35.}]{ \textsc{§ 35.} }
\noindent (1) Après avoir fait leurs provisions de blé, Drappès et Luctérios vinrent camper à dix milles de la place, pour y faire entrer peu à peu leurs convois. (2) Ils se partagent les rôles : Drappès reste, avec une partie des troupes, à la garde du camp ; Luctérios escorte les transports. (3) Après avoir disposé des postes, il fait, vers la dixième heure de la nuit, avancer le convoi à travers les forêts et par d’étroits chemins. (4) Les sentinelles du camp ayant entendu du bruit, on dépêche des éclaireurs pour aller voir ce qui se passe. Sur leurs rapports, Caninius tire des forts les plus voisins les cohortes armées, et tombe au point du jour sur les fourrageurs. (5) Ceux-ci, effrayés d’une attaque aussi inopinée, s’enfuient vers leur escorte ; voyant alors qu’ils ont affaire à des ennemis en armes, les nôtres s’irritent, et ne veulent faire, dans cette multitude, aucun prisonnier. Échappé de là avec un petit nombre des siens, Luctérios ne put regagner son camp.
\subsection[{§ 36.}]{ \textsc{§ 36.} }
\noindent (1) Après ce succès, Caninius apprit par des prisonniers qu’une partie des troupes était restée au camp avec Drappès, à une distance qui n’excédait pas douze milles. Cet avis lui ayant été confirmé par plusieurs rapports, il pensa que, l’autre chef étant en lutte, il lui serait aisé d’accabler dans leur effroi le reste des ennemis. Il regardait comme un grand bonheur qu’aucun de ceux qui avaient échappé au carnage n’eût rejoint le camp de Drappès, pour lui porter la nouvelle de cette défaite. (2) Ne trouvant nul danger à faire une tentative, il envoie en avant et fait marcher contre le camp ennemi toute la cavalerie, ainsi que de l’infanterie germaine dont les hommes sont si agiles ; il laisse une légion à la garde des trois camps, et se met en marche à la tête de l’autre sans bagages. (3) Lorsqu’il fut à peu de distance des ennemis, les éclaireurs qu’il avait détachés lui rapportèrent que les Barbares, négligeant les hauteurs, selon leur usage, avaient placé leur camp sur le bord d’une rivière, que les Germains et les cavaliers étaient tombés sur eux tout à fait à l’improviste, et que le combat était engagé. (4) Sur cet avis, il fait avancer la légion sous les armes et en ordre de bataille. Puis il donne partout le signal, et s’empare des hauteurs. Cela fait, les Germains et la cavalerie, à la vue des enseignes de la légion, combattent avec la plus grande vigueur ; (5) aussitôt toutes les cohortes chargent sur tous les points ; tout est tué ou pris ; le butin est immense ; Drappès lui-même est fait prisonnier dans ce combat.
\subsection[{§ 37.}]{ \textsc{§ 37.} }
\noindent (1) Caninius, ayant terminé cette expédition heureusement, et presque sans perte, vint reprendre le siège ; (2) et comme il avait détruit l’ennemi extérieur, dont la présence l’avait jusque-là empêché d’augmenter ses postes et de travailler à ses lignes de circonvallation, il ordonna de continuer les ouvrages sur tous les points. Le jour suivant, C. Fabius arriva avec ses troupes, et se chargea d’assiéger l’un des côtés de la place.
\subsection[{§ 38.}]{ \textsc{§ 38.} }
\noindent (1) Cependant César laisse le questeur M. Antonius chez les Bellovaques avec quinze cohortes, afin d’empêcher les Belges de former de nouveaux projets de révolte. (2) Il visite lui-même les autres états, demande un grand nombre d’otages, rassure tous les esprits par de consolantes paroles. (3) Arrivé chez les Carnutes, dont les conseils, ainsi que César l’a dit dans le livre précédent, avaient été la première cause de la guerre, et voyant que le souvenir de leur conduite leur causait de vives alarmes, il voulut dissiper sur-le-champ leurs craintes, et demanda pour le supplice Gutuater, instigateur de la dernière révolte et principal auteur de cette guerre. (4) Cet homme, bien qu’il n’eût confié à personne le lieu de sa retraite, fut cherché par la multitude avec tant de soin, qu’on l’eut bientôt amené au camp. (5) Ce fut contre son penchant que César se vit contraint d’accorder la mort de ce chef aux instances des soldats, qui lui rappelaient tous les périls, toutes les pertes qu’ils devaient à Gutuater. Celui-ci, après avoir été battu de verges jusqu’à la mort, eut la tête tranchée par la hache.
\subsection[{§ 39.}]{ \textsc{§ 39.} }
\noindent (1) Plusieurs lettres de Caninius apprirent à César le sort de Drappès et de Luctérios et la résolution opiniâtre des habitants. (2) Quoiqu’il méprisât leur petit nombre, il pensa qu’il fallait sévèrement punir leur obstination, afin que la Gaule entière ne crût pas que, pour résister aux Romains, ce n’était point la force qui avait manqué, mais la constance. Il était à craindre en outre, qu’encouragées par cet exemple, les autres villes, profitant de l’avantage de leur position, ne cherchassent à recouvrer leur liberté. (3) César savait d’ailleurs qu’il était connu dans toute la Gaule que son gouvernement ne devait pas se prolonger au-delà d’un été, après lequel, si les villes pouvaient se soutenir jusque-là, elles n’auraient aucun péril à craindre. (4) Il laisse donc deux légions au lieutenant Q. Calénus, avec ordre de le suivre à grandes journées ; lui-même, avec toute la cavalerie, se dirige en toute hâte vers Caninius.
\subsection[{§ 40.}]{ \textsc{§ 40.} }
\noindent (1) Lorsque César fut arrivé à Uxellodunum, où personne ne l’attendait, qu’il y vit la circonvallation achevée, ce qui ne permettait plus d’en abandonner le siège ; et qu’il eut, d’un autre côté, appris par des transfuges que les assiégés étaient abondamment pourvus de vivres, il essaya de les priver d’eau. (2) Une rivière traversait le vallon qui environnait presque en entier le rocher escarpé sur lequel était située la place d’Uxellodunum. (3) La nature du lieu s’opposait à ce qu’on détournât le cours de cette rivière ; car elle coulait au pied même de la montagne, et il était impossible de creuser nulle part des fossés pour en opérer la dérivation. (4) Mais les assiégés n’y descendaient que difficilement et par des chemins escarpés, et, si nos troupes leur coupaient le passage, ils ne pouvaient y arriver ni regagner la hauteur sans s’exposer à nos traits et sans risquer leur vie. (5) César, s’étant aperçu de leur embarras, plaça des archers et des frondeurs, disposa des machines de guerre vers les endroits où la descente était le plus facile, et par là interdit aux assiégés l’accès de la rivière.
\subsection[{§ 41.}]{ \textsc{§ 41.} }
\noindent (1) Toute la population n’eut dès lors plus d’autre ressource que l’eau d’une fontaine abondante, sortant du pied même des murs, dans cet espace, d’environ trois cents pieds, le seul que la rivière n’entourât pas. (2) On désirait pouvoir priver de cette eau les habitants ; César seul en vit le moyen. Il dressa des mantelets et éleva une terrasse vis-à-vis la fontaine, contre la montagne ; mais ce ne fut pas sans de grandes peines et de continuels combats. (3) En effet, les assiégés, accourant des hauteurs, combattaient de loin sans danger, et blessaient beaucoup des nôtres, à mesure qu’ils se présentaient. Rien ne put cependant les empêcher d’avancer à la faveur des mantelets, ni de vaincre par leur persévérance les difficultés du lieu. (4) En même temps, ils conduisaient des galeries souterraines, depuis la terrasse jusqu’à la source de la fontaine, genre de travail qu’ils pouvaient exécuter sans péril, et même sans que les ennemis s’en doutassent. (5) La terrasse s’éleva à la hauteur de neuf pieds ; on y plaça une tour de dix étages, non pour égaler la hauteur des murs, ce qui était absolument impossible, mais de manière à dominer la fontaine. (6) Les avenues se trouvaient ainsi exposées aux traits de nos machines ; et, comme les assiégés ne pouvaient plus y venir prendre de l’eau sans de grands risques, les bestiaux, les chevaux, les hommes même, en grand nombre, mouraient de soif.
\subsection[{§ 42.}]{ \textsc{§ 42.} }
\noindent (1) Effrayés de ce triste sort, les habitants remplissent des tonneaux de suif, de poix et de menu bois, et les roulent tout enflammés sur nos ouvrages. En même temps ils font une vive attaque, afin que les Romains, obligés de combattre pour leur défense, ne puissent porter remède à l’incendie. (2) Dans un instant tous nos ouvrages sont en feu. Ces tonneaux, qui roulaient sur la pente, arrêtés par les mantelets et la terrasse, embrasaient les matières même qui les retenaient. (3) Cependant nos soldats, malgré le péril de ce genre de combat, et la difficulté des lieux, déployaient le plus grand courage ; (4) car l’action se passait sur une hauteur et à la vue de notre armée. De part et d’autre on entendait de grands cris ; chacun cherchait à se signaler, et l’on bravait, pour faire preuve d’une valeur qui avait tant de témoins, les traits des ennemis et la flamme.
\subsection[{§ 43.}]{ \textsc{§ 43.} }
\noindent (1) César, voyant qu’il avait déjà beaucoup de blessés, ordonne aux soldats de gravir de toutes parts la montagne, en jetant de grands cris, comme s’ils eussent voulu escalader les murs. (2) Épouvantés par cette manoeuvre, et ne sachant ce qui se passait sur d’autres points, les habitants rappellent ceux de leurs combattants qui attaquaient nos ouvrages, et les placent sur leurs murailles. (3) Alors nos soldats, n’ayant plus d’adversaires à combattre, se rendent bientôt maîtres de l’incendie, soit en l’étouffant, soit en le coupant. (4) Les assiégés continuaient à se défendre opiniâtrement ; et, après avoir perdu déjà une grande partie des leurs par la soif, ils persévéraient dans leur résistance, lorsqu’enfin nos mines souterraines parvinrent à couper et à détourner les veines de la source. (5) La voyant tout à coup tarie, les assiégés désespérèrent de tout moyen de salut, et ils crurent reconnaître, non l’ouvrage des hommes, mais la volonté des dieux. Vaincus alors par la nécessité, ils se rendirent.
\subsection[{§ 44.}]{ \textsc{§ 44.} }
\noindent (1) César savait sa réputation de clémence trop bien établie, pour craindre qu’un acte de rigueur fût imputé à la cruauté de son caractère ; et comme il sentait que ses efforts n’auraient point de terme si des révoltes de ce genre venaient ainsi à éclater sur plusieurs points, il résolut d’effrayer les autres peuples par un exemple terrible. Il fit donc couper les mains à tous ceux qui avaient porté les armes, et leur laissa la vie, pour mieux témoigner du châtiment réservé aux pervers. (2) Drappès, qui, ainsi que je l’ai dit, avait été fait prisonnier par Caninius, soit honte et douleur de sa captivité, soit crainte d’un supplice plus grand, s’abstint de nourriture pendant plusieurs jours, et mourut de faim. (3) Vers le même temps, Luctérios, qui, comme on l’a vu, s’était échappé du combat, était tombé au pouvoir de l’Arverne Epasnactos ; obligé de changer fréquemment de retraite, et sentant qu’il ne pouvait longtemps demeurer dans le même lieu sans danger, il avait dû se confier à beaucoup de gens ; sa conscience lui disait combien il avait mérité l’inimitié de César. L'Arverne Epasnactos, fidèle au peuple Romain, n’hésita pas à le livrer enchaîné à César.
\subsection[{§ 45.}]{ \textsc{§ 45.} }
\noindent (1) Cependant Labiénus battait les Trévires dans un combat de cavalerie, et leur tuait beaucoup de monde ainsi qu’aux Germains qui ne refusaient jamais leur secours contre les Romains. Il fit leurs chefs prisonniers, (2) et, parmi eux, l’Héduen Suros, également illustre par son courage et par sa naissance, et le seul des Héduens qui n’eût pas encore déposé les armes.
\subsection[{§ 46.}]{ \textsc{§ 46.} }
\noindent (1) Informé de ce succès, et voyant les affaires en bon état sur tous les points de la Gaule, que ses dernières campagnes avaient domptée et soumise, César, qui n’était jamais allé en personne dans l’Aquitaine, et qui n’en avait soumis une partie que par les armes de P. Crassus, s’y rendit avec deux légions, pour y passer le reste de la saison. (2) Cette expédition fut, comme les autres, prompte et heureuse. Car tous les états de l’Aquitaine lui envoyèrent des députés et lui donnèrent des otages. (3) Il partit ensuite pour Narbonne, avec une escorte de cavalerie, et mit l’armée en quartiers d’hiver sous les ordres des lieutenants. (4) Il plaça quatre légions dans la Belgique, avec M. Antonius, C. Trébonius et P. Vatinius ; il en envoya deux chez les Héduens, dont il connaissait le crédit sur toute la Gaule ; il en plaça deux chez les Turons, sur la frontière des Carnutes, pour contenir toutes les contrées qui touchent l’océan ; deux autres chez les Lémovices, non loin des Arvernes, pour ne laisser sans armée aucune partie de la Gaule. (5) Pendant le petit nombre de jours qu’il passa lui-même dans la province, il en parcourut rapidement les assemblées, y prit connaissance des débats publics, distribua des récompenses à ceux qui l’avaient bien servi ; (6) car rien ne lui était plus facile que de discerner de quels sentiments chacun avait été animé envers la république dans cette révolte de toute la Gaule, à laquelle la fidélité et les secours de la province l’avaient mis en état de résister. Ces choses terminées, il alla rejoindre les légions dans la Belgique et passa l’hiver à Némétocenna.
\subsection[{§ 47.}]{ \textsc{§ 47.} }
\noindent (1) Là il apprit que l’Atrébate Commios s’était battu contre notre cavalerie. (2) Antoine avait pris ses quartiers d’hiver dans ce pays ; mais quoique les Atrébates fussent demeurés fidèles, Commios, qui, depuis la blessure dont j’ai parlé plus haut, était toujours prêt à seconder tous les mouvements de ses concitoyens, et à se faire le conseil et le chef de ceux qui voulaient prendre les armes, tandis que sa nation obéissait aux Romains, se nourrissait de brigandages avec sa cavalerie, infestait les chemins et interceptait quantité de convois destinés à nos quartiers.
\subsection[{§ 48.}]{ \textsc{§ 48.} }
\noindent (1) À Antoine était attaché, comme préfet de la cavalerie, C. Volusénus Quadratus, lequel hivernait avec lui. Antoine l’envoya à la poursuite des cavaliers ennemis. (2) Volusénus, qui joignait à un rare courage une grande haine pour Commios, se chargea avec joie de cette expédition. Il disposa des embuscades, attaqua souvent la cavalerie ennemie, et eut toujours l’avantage. (3) Dans un dernier combat, comme on était vivement aux prises, et que Volusénus, emporté par le désir de prendre Commios en personne, le poursuivait vivement avec peu des siens, celui-ci, qui l’avait attiré fort loin par une fuite précipitée, invoque tout- à-coup la foi et le secours de ses compagnons, et les prie de le venger des blessures qu’il avait reçues par trahison ; il tourne bride, se sépare imprudemment de ses cavaliers, et s’élance contre le préfet. (4) Tous les cavaliers l’imitent, font reculer notre faible troupe et la poursuivent. (5) Commios, pressant de l’éperon les flancs de son cheval, joint celui de Quadratus et porte au préfet un coup de lance qui, fortement appliqué, lui perce le milieu de la cuisse. (6) À la vue de leur chef blessé, nos cavaliers n’hésitent pas à faire face aux ennemis, et les repoussent. (7) Dans cette charge ils en blessent un grand nombre, écrasent les autres dans leur fuite et font des prisonniers. Commios ne put échapper à ce sort que grâce à la vitesse de son cheval ; Volusénus, dont la blessure semblait assez grave pour mettre sa vie en danger, fut reporté au camp. (8) Alors Commios, soit qu’il eût satisfait son ressentiment, soit qu’il fût trop affaibli par la perte des siens, députa vers Antoine, promit d’aller où il lui serait prescrit, de faire ce qu’on lui ordonnerait, et scella sa promesse en livrant des otages. (9) Il pria seulement que l’on accordât à sa frayeur de ne paraître jamais devant un Romain. Antoine, jugeant cette demande fondée sur une crainte légitime, y consentit et reçut les otages.\par
(10) Je sais que César a fait un livre particulier pour chacune de ses campagnes. Je n’ai pas cru devoir adopter cette division, parce que l’année suivante, qui fut celle du consulat de L. Paulus et de C. Marcellus, n’offre rien de bien important dans les affaires de la Gaule. (11) Cependant, pour ne pas laisser ignorer où étaient en ce temps César et son armée, j’ai pensé à joindre ici quelques faits au livre qui précède.
\subsection[{§ 49.}]{ \textsc{§ 49.} }
\noindent (1) César, en tenant dans la Belgique ses quartiers d’hiver, n’avait d’autre but que de maintenir dans notre alliance les peuples de la Gaule, et de ne leur donner ni espoir ni motif de guerre. (2) Car, étant près de partir, il ne voulait point se mettre dans la nécessité de recommencer la guerre, au moment où il allait retirer l’armée, ni laisser toute la Gaule disposée à reprendre librement les hostilités pendant son absence. (3) Aussi, par son attention à adresser des éloges aux états, à combler de récompenses leurs principaux habitants, à n’établir aucun nouvel impôt, à rendre l’obéissance plus douce, il contint facilement en paix la Gaule, épuisée par tant de revers.
\subsection[{§ 50.}]{ \textsc{§ 50.} }
\noindent (1) L'hiver fini, César, contre son usage, partit pour l’Italie à grandes journées, afin de visiter les villes municipales et les colonies, auxquelles il voulait recommander son questeur, M. Antonius, qui briguait le sacerdoce. (2) En l’appuyant de son pouvoir, non seulement il suivait son penchant pour un homme qui lui était très attaché et qu’il avait, peu de temps auparavant, envoyé solliciter cette dignité, mais encore il voulait déjouer une faction peu nombreuse qui, en faisant échouer Antoine, désirait d’ébranler le pouvoir de César, dont le gouvernement expirait. (3) Bien qu’il eût appris en route, et avant d’arriver en Italie, qu’Antoine venait d’être nommé augure, il ne crut pas moins nécessaire de parcourir les villes municipales et les colonies, afin de les remercier de leur empressement à servir Antoine, (4) et en même temps de leur recommander sa propre demande du consulat pour l’année suivante ; car ses ennemis se vantaient avec insolence d’avoir fait nommer consuls L. Lentulus et C. Marcellus, qui devaient dépouiller César de toute charge et de toute dignité ; et d’avoir écarté du consulat Servius Galba, quoiqu’il eût plus de crédit et de suffrages, uniquement parce qu’il était lié d’amitié avec César et avait été son lieutenant.
\subsection[{§ 51.}]{ \textsc{§ 51.} }
\noindent (1) César, à son arrivée, fut accueilli par toutes les villes municipales et par les colonies avec des témoignages incroyables de respect et d’affection ; car il y paraissait pour la première fois depuis cette guerre générale de la Gaule. (2) On n’oublia rien de tout ce qui put être imaginé pour l’ornement des portes, des chemins, et de tous les endroits par où il devait passer. (3) Les enfants et toute la population venaient à sa rencontre ; partout on immolait des victimes ; des tables étaient dressées sur les places publiques et dans les temples ; on lui faisait ainsi goûter par avance la joie d’un triomphe vivement désiré, tant les riches montraient de magnificence et les pauvres d’envie de lui plaire.
\subsection[{§ 52.}]{ \textsc{§ 52.} }
\noindent (1) Quand César eut parcouru toutes les contrées de la Gaule citérieure, il rejoignit promptement l’armée à Némétocenna ; et après avoir tiré toutes les légions de leurs quartiers, il les envoya chez les Trévires, se rendit dans ce pays, et y passa l’armée en revue. (2) Il donna à T. Labiénus le commandement de la Gaule citérieure, afin qu’il pût le seconder de son influence dans la poursuite du consulat. Quant à lui, il ne fit marcher l’armée qu’autant qu’il le jugeait nécessaire pour entretenir la santé du soldat par des changements de lieux. (3) Quoiqu’il entendît souvent dire que ses ennemis excitaient Labiénus contre lui, et qu’il fût informé que ces sollicitations étaient l’ouvrage d’un petit nombre d’hommes travaillant à lui faire enlever par le sénat une partie de l’armée, on ne put cependant ni lui rendre Labiénus suspect, ni l’amener à rien entreprendre contre l’autorité du sénat ; (4) car il savait que, si les délibérations étaient libres, il obtiendrait facilement justice de ses membres. Déjà même C. Curion, tribun du peuple, prenant en main la défense des intérêts et de l’honneur de César, avait dit souvent dans le sénat, que si l’on avait quelque ombrage de la puissance militaire de César, celle de Pompée et sa domination ne devaient pas inspirer moins d’inquiétude ; que l’un et l’autre devaient désarmer et licencier leurs troupes ; qu’ainsi Rome recouvrerait sa liberté et ses droits. (5) Non seulement il fit cette déclaration ; mais il commençait à la faire mettre aux voix, quand les consuls et les amis de Pompée s’y opposèrent ; le sénat s’en tira en prenant un parti moyen.
\subsection[{§ 53.}]{ \textsc{§ 53.} }
\noindent (1) C'était là un témoignage positif des sentiments du sénat, et il s’accordait avec un fait plus ancien. En effet, l’année précédente, Marcellus, cherchant à perdre César, avait, contre la loi de Pompée et de Crassus, proposé au sénat de le rappeler de son gouvernement avant le temps. Marcellus, qui voulait établir son crédit sur les ruines de celui de César, s’efforça vainement de faire goûter cet avis ; le sénat tout entier opina contre lui et se rangea à toutes les opinions qui n’étaient par la sienne. (2) Cet échec n’avait point rebuté les ennemis de César, mais les avait avertis de former des liaisons plus étendues, qui pussent forcer le sénat à approuver leurs desseins.
\subsection[{§ 54.}]{ \textsc{§ 54.} }
\noindent (1) Bientôt un sénatus-consulte ordonne à Cn. Pompée et à C. César de fournir chacun une légion pour la guerre des Parthes. Il est évident que ces deux légions étaient enlevées à César seul : (2) Cn. Pompée donna, pour son contingent, la première légion, qu’il avait autrefois envoyée à César, et qui avait été levée tout entière dans la province du dernier. (3) Cependant et bien que les intentions de ses ennemis ne fussent point douteuses, César renvoya à Cn. Pompée cette légion ; et, en exécution du sénatus-consulte, il livra en son nom la quinzième, qu’il avait levée dans la Gaule citérieure. En remplacement de celle-ci, il envoya en Italie la treizième légion pour garder les postes que quittait la quinzième. (4) Il distribua ensuite l’armée dans ses quartiers d’hiver, plaça C. Trébonius dans la Belgique avec quatre légions, et envoya C. Fabius avec le même nombre chez les Héduens. (5) Il pensait qu’il suffisait, pour la tranquillité de la Gaule, que les Belges, le plus courageux de ces peuples, et les Héduens, dont le crédit était immense, fussent contenus par des armées romaines. Il partit lui-même pour l’Italie.
\subsection[{§ 55.}]{ \textsc{§ 55.} }
\noindent (1) Lorsqu’il y fut arrivé, il apprit que les deux légions qu’il avait livrées, et qui, d’après le sénatus-consulte, devaient être menées contre les Parthes, avaient été livrées par le consul C. Marcellus à Cn. Pompée, et qu’elles étaient retenues en Italie. (2) Quoiqu’une telle conduite ne laissât à personne le moindre doute sur les projets tramés contre César, il résolut cependant de tout souffrir, tant qu’il lui resterait quelque espoir de se soutenir par la force de son droit plutôt que par celle des armes. [etc.]
 


% at least one empty page at end (for booklet couv)
\ifbooklet
  \pagestyle{empty}
  \clearpage
  % 2 empty pages maybe needed for 4e cover
  \ifnum\modulo{\value{page}}{4}=0 \hbox{}\newpage\hbox{}\newpage\fi
  \ifnum\modulo{\value{page}}{4}=1 \hbox{}\newpage\hbox{}\newpage\fi


  \hbox{}\newpage
  \ifodd\value{page}\hbox{}\newpage\fi
  {\centering\color{rubric}\bfseries\noindent\large
    Hurlus ? Qu’est-ce.\par
    \bigskip
  }
  \noindent Des bouquinistes électroniques, pour du texte libre à participation libre,
  téléchargeable gratuitement sur \href{https://hurlus.fr}{\dotuline{hurlus.fr}}.\par
  \bigskip
  \noindent Cette brochure a été produite par des éditeurs bénévoles.
  Elle n’est pas faîte pour être possédée, mais pour être lue, et puis donnée.
  Que circule le texte !
  En page de garde, on peut ajouter une date, un lieu, un nom ; pour suivre le voyage des idées.
  \par

  Ce texte a été choisi parce qu’une personne l’a aimé,
  ou haï, elle a en tous cas pensé qu’il partipait à la formation de notre présent ;
  sans le souci de plaire, vendre, ou militer pour une cause.
  \par

  L’édition électronique est soigneuse, tant sur la technique
  que sur l’établissement du texte ; mais sans aucune prétention scolaire, au contraire.
  Le but est de s’adresser à tous, sans distinction de science ou de diplôme.
  Au plus direct ! (possible)
  \par

  Cet exemplaire en papier a été tiré sur une imprimante personnelle
   ou une photocopieuse. Tout le monde peut le faire.
  Il suffit de
  télécharger un fichier sur \href{https://hurlus.fr}{\dotuline{hurlus.fr}},
  d’imprimer, et agrafer ; puis de lire et donner.\par

  \bigskip

  \noindent PS : Les hurlus furent aussi des rebelles protestants qui cassaient les statues dans les églises catholiques. En 1566 démarra la révolte des gueux dans le pays de Lille. L’insurrection enflamma la région jusqu’à Anvers où les gueux de mer bloquèrent les bateaux espagnols.
  Ce fut une rare guerre de libération dont naquit un pays toujours libre : les Pays-Bas.
  En plat pays francophone, par contre, restèrent des bandes de huguenots, les hurlus, progressivement réprimés par la très catholique Espagne.
  Cette mémoire d’une défaite est éteinte, rallumons-la. Sortons les livres du culte universitaire, cherchons les idoles de l’époque, pour les briser.
\fi

\ifdev % autotext in dev mode
\fontname\font — \textsc{Les règles du jeu}\par
(\hyperref[utopie]{\underline{Lien}})\par
\noindent \initialiv{A}{lors là}\blindtext\par
\noindent \initialiv{À}{ la bonheur des dames}\blindtext\par
\noindent \initialiv{É}{tonnez-le}\blindtext\par
\noindent \initialiv{Q}{ualitativement}\blindtext\par
\noindent \initialiv{V}{aloriser}\blindtext\par
\Blindtext
\phantomsection
\label{utopie}
\Blinddocument
\fi
\end{document}
