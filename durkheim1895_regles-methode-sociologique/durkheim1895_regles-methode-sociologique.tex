%%%%%%%%%%%%%%%%%%%%%%%%%%%%%%%%%
% LaTeX model https://hurlus.fr %
%%%%%%%%%%%%%%%%%%%%%%%%%%%%%%%%%

% Needed before document class
\RequirePackage{pdftexcmds} % needed for tests expressions
\RequirePackage{fix-cm} % correct units

% Define mode
\def\mode{a4}

\newif\ifaiv % a4
\newif\ifav % a5
\newif\ifbooklet % booklet
\newif\ifcover % cover for booklet

\ifnum \strcmp{\mode}{cover}=0
  \covertrue
\else\ifnum \strcmp{\mode}{booklet}=0
  \booklettrue
\else\ifnum \strcmp{\mode}{a5}=0
  \avtrue
\else
  \aivtrue
\fi\fi\fi

\ifbooklet % do not enclose with {}
  \documentclass[french,twoside]{book} % ,notitlepage
  \usepackage[%
    papersize={105mm, 297mm},
    inner=12mm,
    outer=12mm,
    top=20mm,
    bottom=15mm,
    marginparsep=0pt,
  ]{geometry}
  \usepackage[fontsize=9.5pt]{scrextend} % for Roboto
\else\ifav
  \documentclass[french,twoside]{book} % ,notitlepage
  \usepackage[%
    a5paper,
    inner=25mm,
    outer=15mm,
    top=15mm,
    bottom=15mm,
    marginparsep=0pt,
  ]{geometry}
  \usepackage[fontsize=12pt]{scrextend}
\else% A4 2 cols
  \documentclass[twocolumn]{report}
  \usepackage[%
    a4paper,
    inner=15mm,
    outer=10mm,
    top=25mm,
    bottom=18mm,
    marginparsep=0pt,
  ]{geometry}
  \setlength{\columnsep}{20mm}
  \usepackage[fontsize=9.5pt]{scrextend}
\fi\fi

%%%%%%%%%%%%%%
% Alignments %
%%%%%%%%%%%%%%
% before teinte macros

\setlength{\arrayrulewidth}{0.2pt}
\setlength{\columnseprule}{\arrayrulewidth} % twocol
\setlength{\parskip}{0pt} % classical para with no margin
\setlength{\parindent}{1.5em}

%%%%%%%%%%
% Colors %
%%%%%%%%%%
% before Teinte macros

\usepackage[dvipsnames]{xcolor}
\definecolor{rubric}{HTML}{800000} % the tonic 0c71c3
\def\columnseprulecolor{\color{rubric}}
\colorlet{borderline}{rubric!30!} % definecolor need exact code
\definecolor{shadecolor}{gray}{0.95}
\definecolor{bghi}{gray}{0.5}

%%%%%%%%%%%%%%%%%
% Teinte macros %
%%%%%%%%%%%%%%%%%
%%%%%%%%%%%%%%%%%%%%%%%%%%%%%%%%%%%%%%%%%%%%%%%%%%%
% <TEI> generic (LaTeX names generated by Teinte) %
%%%%%%%%%%%%%%%%%%%%%%%%%%%%%%%%%%%%%%%%%%%%%%%%%%%
% This template is inserted in a specific design
% It is XeLaTeX and otf fonts

\makeatletter % <@@@


\usepackage{blindtext} % generate text for testing
\usepackage[strict]{changepage} % for modulo 4
\usepackage{contour} % rounding words
\usepackage[nodayofweek]{datetime}
% \usepackage{DejaVuSans} % seems buggy for sffont font for symbols
\usepackage{enumitem} % <list>
\usepackage{etoolbox} % patch commands
\usepackage{fancyvrb}
\usepackage{fancyhdr}
\usepackage{float}
\usepackage{fontspec} % XeLaTeX mandatory for fonts
\usepackage{footnote} % used to capture notes in minipage (ex: quote)
\usepackage{framed} % bordering correct with footnote hack
\usepackage{graphicx}
\usepackage{lettrine} % drop caps
\usepackage{lipsum} % generate text for testing
\usepackage[framemethod=tikz,]{mdframed} % maybe used for frame with footnotes inside
\usepackage{pdftexcmds} % needed for tests expressions
\usepackage{polyglossia} % non-break space french punct, bug Warning: "Failed to patch part"
\usepackage[%
  indentfirst=false,
  vskip=1em,
  noorphanfirst=true,
  noorphanafter=true,
  leftmargin=\parindent,
  rightmargin=0pt,
]{quoting}
\usepackage{ragged2e}
\usepackage{setspace} % \setstretch for <quote>
\usepackage{tabularx} % <table>
\usepackage[explicit]{titlesec} % wear titles, !NO implicit
\usepackage{tikz} % ornaments
\usepackage{tocloft} % styling tocs
\usepackage[fit]{truncate} % used im runing titles
\usepackage{unicode-math}
\usepackage[normalem]{ulem} % breakable \uline, normalem is absolutely necessary to keep \emph
\usepackage{verse} % <l>
\usepackage{xcolor} % named colors
\usepackage{xparse} % @ifundefined
\XeTeXdefaultencoding "iso-8859-1" % bad encoding of xstring
\usepackage{xstring} % string tests
\XeTeXdefaultencoding "utf-8"
\PassOptionsToPackage{hyphens}{url} % before hyperref, which load url package

% TOTEST
% \usepackage{hypcap} % links in caption ?
% \usepackage{marginnote}
% TESTED
% \usepackage{background} % doesn’t work with xetek
% \usepackage{bookmark} % prefers the hyperref hack \phantomsection
% \usepackage[color, leftbars]{changebar} % 2 cols doc, impossible to keep bar left
% \usepackage[utf8x]{inputenc} % inputenc package ignored with utf8 based engines
% \usepackage[sfdefault,medium]{inter} % no small caps
% \usepackage{firamath} % choose firasans instead, firamath unavailable in Ubuntu 21-04
% \usepackage{flushend} % bad for last notes, supposed flush end of columns
% \usepackage[stable]{footmisc} % BAD for complex notes https://texfaq.org/FAQ-ftnsect
% \usepackage{helvet} % not for XeLaTeX
% \usepackage{multicol} % not compatible with too much packages (longtable, framed, memoir…)
% \usepackage[default,oldstyle,scale=0.95]{opensans} % no small caps
% \usepackage{sectsty} % \chapterfont OBSOLETE
% \usepackage{soul} % \ul for underline, OBSOLETE with XeTeX
% \usepackage[breakable]{tcolorbox} % text styling gone, footnote hack not kept with breakable


% Metadata inserted by a program, from the TEI source, for title page and runing heads
\title{\textbf{ Les règles de la méthode sociologique }}
\date{1895}
\author{Émile Durkheim}
\def\elbibl{Émile Durkheim. 1895. \emph{Les règles de la méthode sociologique}}
\def\elsource{Émile Durkheim, \emph{{\itshape Les Règles de la méthode sociologique}}, Paris, Alcan, 1895. \href{http://gallica.bnf.fr/ark:/12148/bpt6k1055050.r=durkheim+r%C3%A8gles+de+la+m%C3%A9thode.langFR}{\dotuline{Source Gallica}}\footnote{\href{http://gallica.bnf.fr/ark:/12148/bpt6k1055050.r=durkheim+r%C3%A8gles+de+la+m%C3%A9thode.langFR}{\url{http://gallica.bnf.fr/ark:/12148/bpt6k1055050.r=durkheim+r%C3%A8gles+de+la+m%C3%A9thode.langFR}}}.}

% Default metas
\newcommand{\colorprovide}[2]{\@ifundefinedcolor{#1}{\colorlet{#1}{#2}}{}}
\colorprovide{rubric}{red}
\colorprovide{silver}{lightgray}
\@ifundefined{syms}{\newfontfamily\syms{DejaVu Sans}}{}
\newif\ifdev
\@ifundefined{elbibl}{% No meta defined, maybe dev mode
  \newcommand{\elbibl}{Titre court ?}
  \newcommand{\elbook}{Titre du livre source ?}
  \newcommand{\elabstract}{Résumé\par}
  \newcommand{\elurl}{http://oeuvres.github.io/elbook/2}
  \author{Éric Lœchien}
  \title{Un titre de test assez long pour vérifier le comportement d’une maquette}
  \date{1566}
  \devtrue
}{}
\let\eltitle\@title
\let\elauthor\@author
\let\eldate\@date


\defaultfontfeatures{
  % Mapping=tex-text, % no effect seen
  Scale=MatchLowercase,
  Ligatures={TeX,Common},
}


% generic typo commands
\newcommand{\astermono}{\medskip\centerline{\color{rubric}\large\selectfont{\syms ✻}}\medskip\par}%
\newcommand{\astertri}{\medskip\par\centerline{\color{rubric}\large\selectfont{\syms ✻\,✻\,✻}}\medskip\par}%
\newcommand{\asterism}{\bigskip\par\noindent\parbox{\linewidth}{\centering\color{rubric}\large{\syms ✻}\\{\syms ✻}\hskip 0.75em{\syms ✻}}\bigskip\par}%

% lists
\newlength{\listmod}
\setlength{\listmod}{\parindent}
\setlist{
  itemindent=!,
  listparindent=\listmod,
  labelsep=0.2\listmod,
  parsep=0pt,
  % topsep=0.2em, % default topsep is best
}
\setlist[itemize]{
  label=—,
  leftmargin=0pt,
  labelindent=1.2em,
  labelwidth=0pt,
}
\setlist[enumerate]{
  label={\bf\color{rubric}\arabic*.},
  labelindent=0.8\listmod,
  leftmargin=\listmod,
  labelwidth=0pt,
}
\newlist{listalpha}{enumerate}{1}
\setlist[listalpha]{
  label={\bf\color{rubric}\alph*.},
  leftmargin=0pt,
  labelindent=0.8\listmod,
  labelwidth=0pt,
}
\newcommand{\listhead}[1]{\hspace{-1\listmod}\emph{#1}}

\renewcommand{\hrulefill}{%
  \leavevmode\leaders\hrule height 0.2pt\hfill\kern\z@}

% General typo
\DeclareTextFontCommand{\textlarge}{\large}
\DeclareTextFontCommand{\textsmall}{\small}

% commands, inlines
\newcommand{\anchor}[1]{\Hy@raisedlink{\hypertarget{#1}{}}} % link to top of an anchor (not baseline)
\newcommand\abbr[1]{#1}
\newcommand{\autour}[1]{\tikz[baseline=(X.base)]\node [draw=rubric,thin,rectangle,inner sep=1.5pt, rounded corners=3pt] (X) {\color{rubric}#1};}
\newcommand\corr[1]{#1}
\newcommand{\ed}[1]{ {\color{silver}\sffamily\footnotesize (#1)} } % <milestone ed="1688"/>
\newcommand\expan[1]{#1}
\newcommand\foreign[1]{\emph{#1}}
\newcommand\gap[1]{#1}
\renewcommand{\LettrineFontHook}{\color{rubric}}
\newcommand{\initial}[2]{\lettrine[lines=2, loversize=0.3, lhang=0.3]{#1}{#2}}
\newcommand{\initialiv}[2]{%
  \let\oldLFH\LettrineFontHook
  % \renewcommand{\LettrineFontHook}{\color{rubric}\ttfamily}
  \IfSubStr{QJ’}{#1}{
    \lettrine[lines=4, lhang=0.2, loversize=-0.1, lraise=0.2]{\smash{#1}}{#2}
  }{\IfSubStr{É}{#1}{
    \lettrine[lines=4, lhang=0.2, loversize=-0, lraise=0]{\smash{#1}}{#2}
  }{\IfSubStr{ÀÂ}{#1}{
    \lettrine[lines=4, lhang=0.2, loversize=-0, lraise=0, slope=0.6em]{\smash{#1}}{#2}
  }{\IfSubStr{A}{#1}{
    \lettrine[lines=4, lhang=0.2, loversize=0.2, slope=0.6em]{\smash{#1}}{#2}
  }{\IfSubStr{V}{#1}{
    \lettrine[lines=4, lhang=0.2, loversize=0.2, slope=-0.5em]{\smash{#1}}{#2}
  }{
    \lettrine[lines=4, lhang=0.2, loversize=0.2]{\smash{#1}}{#2}
  }}}}}
  \let\LettrineFontHook\oldLFH
}
\newcommand{\labelchar}[1]{\textbf{\color{rubric} #1}}
\newcommand{\milestone}[1]{\autour{\footnotesize\color{rubric} #1}} % <milestone n="4"/>
\newcommand\name[1]{#1}
\newcommand\orig[1]{#1}
\newcommand\orgName[1]{#1}
\newcommand\persName[1]{#1}
\newcommand\placeName[1]{#1}
\newcommand{\pn}[1]{\IfSubStr{-—–¶}{#1}% <p n="3"/>
  {\noindent{\bfseries\color{rubric}   ¶  }}
  {{\footnotesize\autour{ #1}  }}}
\newcommand\reg{}
% \newcommand\ref{} % already defined
\newcommand\sic[1]{#1}
\newcommand\surname[1]{\textsc{#1}}
\newcommand\term[1]{\textbf{#1}}

\def\mednobreak{\ifdim\lastskip<\medskipamount
  \removelastskip\nopagebreak\medskip\fi}
\def\bignobreak{\ifdim\lastskip<\bigskipamount
  \removelastskip\nopagebreak\bigskip\fi}

% commands, blocks
\newcommand{\byline}[1]{\bigskip{\RaggedLeft{#1}\par}\bigskip}
\newcommand{\bibl}[1]{{\RaggedLeft{#1}\par\bigskip}}
\newcommand{\biblitem}[1]{{\noindent\hangindent=\parindent   #1\par}}
\newcommand{\dateline}[1]{\medskip{\RaggedLeft{#1}\par}\bigskip}
\newcommand{\labelblock}[1]{\medbreak{\noindent\color{rubric}\bfseries #1}\par\mednobreak}
\newcommand{\salute}[1]{\bigbreak{#1}\par\medbreak}
\newcommand{\signed}[1]{\bigbreak\filbreak{\raggedleft #1\par}\medskip}

% environments for blocks (some may become commands)
\newenvironment{borderbox}{}{} % framing content
\newenvironment{citbibl}{\ifvmode\hfill\fi}{\ifvmode\par\fi }
\newenvironment{docAuthor}{\ifvmode\vskip4pt\fontsize{16pt}{18pt}\selectfont\fi\itshape}{\ifvmode\par\fi }
\newenvironment{docDate}{}{\ifvmode\par\fi }
\newenvironment{docImprint}{\vskip6pt}{\ifvmode\par\fi }
\newenvironment{docTitle}{\vskip6pt\bfseries\fontsize{18pt}{22pt}\selectfont}{\par }
\newenvironment{msHead}{\vskip6pt}{\par}
\newenvironment{msItem}{\vskip6pt}{\par}
\newenvironment{titlePart}{}{\par }


% environments for block containers
\newenvironment{argument}{\itshape\parindent0pt}{\vskip1.5em}
\newenvironment{biblfree}{}{\ifvmode\par\fi }
\newenvironment{bibitemlist}[1]{%
  \list{\@biblabel{\@arabic\c@enumiv}}%
  {%
    \settowidth\labelwidth{\@biblabel{#1}}%
    \leftmargin\labelwidth
    \advance\leftmargin\labelsep
    \@openbib@code
    \usecounter{enumiv}%
    \let\p@enumiv\@empty
    \renewcommand\theenumiv{\@arabic\c@enumiv}%
  }
  \sloppy
  \clubpenalty4000
  \@clubpenalty \clubpenalty
  \widowpenalty4000%
  \sfcode`\.\@m
}%
{\def\@noitemerr
  {\@latex@warning{Empty `bibitemlist' environment}}%
\endlist}
\newenvironment{quoteblock}% may be used for ornaments
  {\begin{quoting}}
  {\end{quoting}}

% table () is preceded and finished by custom command
\newcommand{\tableopen}[1]{%
  \ifnum\strcmp{#1}{wide}=0{%
    \begin{center}
  }
  \else\ifnum\strcmp{#1}{long}=0{%
    \begin{center}
  }
  \else{%
    \begin{center}
  }
  \fi\fi
}
\newcommand{\tableclose}[1]{%
  \ifnum\strcmp{#1}{wide}=0{%
    \end{center}
  }
  \else\ifnum\strcmp{#1}{long}=0{%
    \end{center}
  }
  \else{%
    \end{center}
  }
  \fi\fi
}


% text structure
\newcommand\chapteropen{} % before chapter title
\newcommand\chaptercont{} % after title, argument, epigraph…
\newcommand\chapterclose{} % maybe useful for multicol settings
\setcounter{secnumdepth}{-2} % no counters for hierarchy titles
\setcounter{tocdepth}{5} % deep toc
\markright{\@title} % ???
\markboth{\@title}{\@author} % ???
\renewcommand\tableofcontents{\@starttoc{toc}}
% toclof format
% \renewcommand{\@tocrmarg}{0.1em} % Useless command?
% \renewcommand{\@pnumwidth}{0.5em} % {1.75em}
\renewcommand{\@cftmaketoctitle}{}
\setlength{\cftbeforesecskip}{\z@ \@plus.2\p@}
\renewcommand{\cftchapfont}{}
\renewcommand{\cftchapdotsep}{\cftdotsep}
\renewcommand{\cftchapleader}{\normalfont\cftdotfill{\cftchapdotsep}}
\renewcommand{\cftchappagefont}{\bfseries}
\setlength{\cftbeforechapskip}{0em \@plus\p@}
% \renewcommand{\cftsecfont}{\small\relax}
\renewcommand{\cftsecpagefont}{\normalfont}
% \renewcommand{\cftsubsecfont}{\small\relax}
\renewcommand{\cftsecdotsep}{\cftdotsep}
\renewcommand{\cftsecpagefont}{\normalfont}
\renewcommand{\cftsecleader}{\normalfont\cftdotfill{\cftsecdotsep}}
\setlength{\cftsecindent}{1em}
\setlength{\cftsubsecindent}{2em}
\setlength{\cftsubsubsecindent}{3em}
\setlength{\cftchapnumwidth}{1em}
\setlength{\cftsecnumwidth}{1em}
\setlength{\cftsubsecnumwidth}{1em}
\setlength{\cftsubsubsecnumwidth}{1em}

% footnotes
\newif\ifheading
\newcommand*{\fnmarkscale}{\ifheading 0.70 \else 1 \fi}
\renewcommand\footnoterule{\vspace*{0.3cm}\hrule height \arrayrulewidth width 3cm \vspace*{0.3cm}}
\setlength\footnotesep{1.5\footnotesep} % footnote separator
\renewcommand\@makefntext[1]{\parindent 1.5em \noindent \hb@xt@1.8em{\hss{\normalfont\@thefnmark . }}#1} % no superscipt in foot
\patchcmd{\@footnotetext}{\footnotesize}{\footnotesize\sffamily}{}{} % before scrextend, hyperref


%   see https://tex.stackexchange.com/a/34449/5049
\def\truncdiv#1#2{((#1-(#2-1)/2)/#2)}
\def\moduloop#1#2{(#1-\truncdiv{#1}{#2}*#2)}
\def\modulo#1#2{\number\numexpr\moduloop{#1}{#2}\relax}

% orphans and widows
\clubpenalty=9996
\widowpenalty=9999
\brokenpenalty=4991
\predisplaypenalty=10000
\postdisplaypenalty=1549
\displaywidowpenalty=1602
\hyphenpenalty=400
% Copied from Rahtz but not understood
\def\@pnumwidth{1.55em}
\def\@tocrmarg {2.55em}
\def\@dotsep{4.5}
\emergencystretch 3em
\hbadness=4000
\pretolerance=750
\tolerance=2000
\vbadness=4000
\def\Gin@extensions{.pdf,.png,.jpg,.mps,.tif}
% \renewcommand{\@cite}[1]{#1} % biblio

\usepackage{hyperref} % supposed to be the last one, :o) except for the ones to follow
\urlstyle{same} % after hyperref
\hypersetup{
  % pdftex, % no effect
  pdftitle={\elbibl},
  % pdfauthor={Your name here},
  % pdfsubject={Your subject here},
  % pdfkeywords={keyword1, keyword2},
  bookmarksnumbered=true,
  bookmarksopen=true,
  bookmarksopenlevel=1,
  pdfstartview=Fit,
  breaklinks=true, % avoid long links
  pdfpagemode=UseOutlines,    % pdf toc
  hyperfootnotes=true,
  colorlinks=false,
  pdfborder=0 0 0,
  % pdfpagelayout=TwoPageRight,
  % linktocpage=true, % NO, toc, link only on page no
}

\makeatother % /@@@>
%%%%%%%%%%%%%%
% </TEI> end %
%%%%%%%%%%%%%%


%%%%%%%%%%%%%
% footnotes %
%%%%%%%%%%%%%
\renewcommand{\thefootnote}{\bfseries\textcolor{rubric}{\arabic{footnote}}} % color for footnote marks

%%%%%%%%%
% Fonts %
%%%%%%%%%
\usepackage[]{roboto} % SmallCaps, Regular is a bit bold
% \linespread{0.90} % too compact, keep font natural
\newfontfamily\fontrun[]{Roboto Condensed Light} % condensed runing heads
\ifav
  \setmainfont[
    ItalicFont={Roboto Light Italic},
  ]{Roboto}
\else\ifbooklet
  \setmainfont[
    ItalicFont={Roboto Light Italic},
  ]{Roboto}
\else
\setmainfont[
  ItalicFont={Roboto Italic},
]{Roboto Light}
\fi\fi
\renewcommand{\LettrineFontHook}{\bfseries\color{rubric}}
% \renewenvironment{labelblock}{\begin{center}\bfseries\color{rubric}}{\end{center}}

%%%%%%%%
% MISC %
%%%%%%%%

\setdefaultlanguage[frenchpart=false]{french} % bug on part


\newenvironment{quotebar}{%
    \def\FrameCommand{{\color{rubric!10!}\vrule width 0.5em} \hspace{0.9em}}%
    \def\OuterFrameSep{\itemsep} % séparateur vertical
    \MakeFramed {\advance\hsize-\width \FrameRestore}
  }%
  {%
    \endMakeFramed
  }
\renewenvironment{quoteblock}% may be used for ornaments
  {%
    \savenotes
    \setstretch{0.9}
    \normalfont
    \begin{quotebar}
  }
  {%
    \end{quotebar}
    \spewnotes
  }


\renewcommand{\headrulewidth}{\arrayrulewidth}
\renewcommand{\headrule}{{\color{rubric}\hrule}}

% delicate tuning, image has produce line-height problems in title on 2 lines
\titleformat{name=\chapter} % command
  [display] % shape
  {\vspace{1.5em}\centering} % format
  {} % label
  {0pt} % separator between n
  {}
[{\color{rubric}\huge\textbf{#1}}\bigskip] % after code
% \titlespacing{command}{left spacing}{before spacing}{after spacing}[right]
\titlespacing*{\chapter}{0pt}{-2em}{0pt}[0pt]

\titleformat{name=\section}
  [block]{}{}{}{}
  [\vbox{\color{rubric}\large\raggedleft\textbf{#1}}]
\titlespacing{\section}{0pt}{0pt plus 4pt minus 2pt}{\baselineskip}

\titleformat{name=\subsection}
  [block]
  {}
  {} % \thesection
  {} % separator \arrayrulewidth
  {}
[\vbox{\large\textbf{#1}}]
% \titlespacing{\subsection}{0pt}{0pt plus 4pt minus 2pt}{\baselineskip}

\ifaiv
  \fancypagestyle{main}{%
    \fancyhf{}
    \setlength{\headheight}{1.5em}
    \fancyhead{} % reset head
    \fancyfoot{} % reset foot
    \fancyhead[L]{\truncate{0.45\headwidth}{\fontrun\elbibl}} % book ref
    \fancyhead[R]{\truncate{0.45\headwidth}{ \fontrun\nouppercase\leftmark}} % Chapter title
    \fancyhead[C]{\thepage}
  }
  \fancypagestyle{plain}{% apply to chapter
    \fancyhf{}% clear all header and footer fields
    \setlength{\headheight}{1.5em}
    \fancyhead[L]{\truncate{0.9\headwidth}{\fontrun\elbibl}}
    \fancyhead[R]{\thepage}
  }
\else
  \fancypagestyle{main}{%
    \fancyhf{}
    \setlength{\headheight}{1.5em}
    \fancyhead{} % reset head
    \fancyfoot{} % reset foot
    \fancyhead[RE]{\truncate{0.9\headwidth}{\fontrun\elbibl}} % book ref
    \fancyhead[LO]{\truncate{0.9\headwidth}{\fontrun\nouppercase\leftmark}} % Chapter title, \nouppercase needed
    \fancyhead[RO,LE]{\thepage}
  }
  \fancypagestyle{plain}{% apply to chapter
    \fancyhf{}% clear all header and footer fields
    \setlength{\headheight}{1.5em}
    \fancyhead[L]{\truncate{0.9\headwidth}{\fontrun\elbibl}}
    \fancyhead[R]{\thepage}
  }
\fi

\ifav % a5 only
  \titleclass{\section}{top}
\fi

\newcommand\chapo{{%
  \vspace*{-3em}
  \centering % no vskip ()
  {\Large\addfontfeature{LetterSpace=25}\bfseries{\elauthor}}\par
  \smallskip
  {\large\eldate}\par
  \bigskip
  {\Large\selectfont{\eltitle}}\par
  \bigskip
  {\color{rubric}\hline\par}
  \bigskip
  {\Large TEXTE LIBRE À PARTICIPATIONS LIBRES\par}
  \centerline{\small\color{rubric} {hurlus.fr, tiré le \today}}\par
  \bigskip
}}

\newcommand\cover{{%
  \thispagestyle{empty}
  \centering
  {\LARGE\bfseries{\elauthor}}\par
  \bigskip
  {\Large\eldate}\par
  \bigskip
  \bigskip
  {\LARGE\selectfont{\eltitle}}\par
  \vfill\null
  {\color{rubric}\setlength{\arrayrulewidth}{2pt}\hline\par}
  \vfill\null
  {\Large TEXTE LIBRE À PARTICIPATIONS LIBRES\par}
  \centerline{{\href{https://hurlus.fr}{\dotuline{hurlus.fr}}, tiré le \today}}\par
}}

\begin{document}
\pagestyle{empty}
\ifbooklet{
  \cover\newpage
  \thispagestyle{empty}\hbox{}\newpage
  \cover\newpage\noindent Les voyages de la brochure\par
  \bigskip
  \begin{tabularx}{\textwidth}{l|X|X}
    \textbf{Date} & \textbf{Lieu}& \textbf{Nom/pseudo} \\ \hline
    \rule{0pt}{25cm} &  &   \\
  \end{tabularx}
  \newpage
  \addtocounter{page}{-4}
}\fi

\thispagestyle{empty}
\ifaiv
  \twocolumn[\chapo]
\else
  \chapo
\fi
{\it\elabstract}
\bigskip
\makeatletter\@starttoc{toc}\makeatother % toc without new page
\bigskip

\pagestyle{main} % after style

  \section[{Préface}]{Préface}\renewcommand{\leftmark}{Préface}

\noindent On est si peu habitué à traiter les faits sociaux scientifiquement que certaines des propositions contenues dans cet ouvrage risquent de surprendre le lecteur. Cependant, s’il existe une science des sociétés, il faut bien s’attendre à ce qu’elle ne consiste pas dans une simple paraphrase des préjugés traditionnels, mais nous fasse voir les choses autrement qu’elles n’apparaissent au vulgaire ; car l’objet de toute science est de faire des découvertes et toute découverte déconcerte plus ou moins les opinions reçues. À moins donc qu’on ne prête au sens commun, en sociologie, une autorité qu’il n’a plus depuis longtemps dans les autres sciences ― et on ne voit pas d’où elle pourrait lui venir ― il faut que le savant prenne résolument son parti de ne pas se laisser intimider par les résultats auxquels aboutissent ses recherches, si elles ont été méthodiquement conduites. Si chercher le paradoxe est d’un sophiste, le fuir, quand il est imposé par les faits, est d’un esprit sans courage ou sans foi dans la science.\par
Malheureusement, il est plus aisé d’admettre cette règle en principe et théoriquement que de l’appliquer avec persévérance. Nous sommes encore trop accoutumés à trancher toutes ces questions d’après les suggestions du sens commun pour que nous puissions facilement le tenir à distance des discussions sociologiques. Alors que nous nous en croyons affranchis, il nous impose ses jugements sans que nous y prenions garde. Il n’y a qu’une longue et spéciale pratique qui puisse prévenir de pareilles défaillances. Voilà ce que nous demandons au lecteur de bien vouloir ne pas perdre de vue. Qu’il ait toujours présent à l’esprit que les manières de penser auxquelles il est le plus fait sont plutôt contraires que favorables à l’étude scientifique des phénomènes sociaux et, par conséquent, qu’il se mette en garde contre ses premières impressions. S’il s’y abandonne sans résistance, il risque de nous juger sans nous avoir compris. Ainsi, il pourrait arriver qu’on nous accusât d’avoir voulu absoudre le crime, sous prétexte que nous en faisons un phénomène de sociologie normale. L’objection pourtant serait puérile. Car s’il est normal que, dans toute société, il y ait des crimes, il n’est pas moins normal qu’ils soient punis. L’institution d’un système répressif n’est pas un fait moins universel que l’existence d’une criminalité, ni moins indispensable à la santé collective. Pour qu’il n’y eût pas de crimes, il faudrait un nivellement des consciences individuelles qui, pour des raisons qu’on trouvera plus loin, n’est ni possible ni désirable ; mais pour qu’il n’y eût pas de répression, il faudrait une absence d’homogénéité morale qui est inconciliable avec l’existence d’une société. Seulement, partant de ce fait que le crime est détesté et détestable, le sens commun en conclut à tort qu’il ne saurait disparaître trop complètement. Avec son simplisme ordinaire, il ne conçoit pas qu’une chose qui répugne puisse avoir quelque raison d’être utile, et cependant il n’y a à cela aucune contradiction. N’y a-t-il pas dans l’organisme des fonctions répugnantes dont le jeu régulier est nécessaire à la santé individuelle ? Est-ce que nous ne détestons pas la souffrance ? et cependant un être qui ne la connaîtrait pas serait un monstre. Le caractère normal d’une chose et les sentiments d’éloignement qu’elle inspire peuvent même être solidaires. Si la douleur est un fait normal, c’est à condition de n’être pas aimée ; si le crime est normal, c’est à condition d’être haï\footnote{ Mais, nous objecte-t-on, si la santé contient des éléments haïssables, comment la présenter, ainsi que nous faisons plus loin, comme l’objectif immédiat de la conduite ? ― Il n’y a à cela aucune contradiction. Il arrive sans cesse qu’une chose, tout en étant nuisible par certaines de ses conséquences, soit, par d’autres, utile ou même nécessaire à la vie ; or, si les mauvais effets qu’elle a sont régulièrement neutralisés par une influence contraire, il se trouve en fait qu’elle sert sans nuire, et cependant elle est toujours haïssable, car elle ne laisse pas de constituer par elle-même un danger éventuel qui n’est conjuré que par l’action d’une force antagoniste. C’est le cas du crime ; le tort qu’il fait à la société est annulé par la peine, si elle fonctionne régulièrement. Il reste donc que, sans produire le mal qu’il implique, il soutient avec les conditions fondamentales de la vie sociale les rapports positifs que nous verrons dans la suite. Seulement, comme c’est malgré lui, pour ainsi dire, qu’il est rendu inoffensif, les sentiments d’aversion dont il est l’objet ne laissent pas d’être fondés.}. Notre méthode n’a donc rien de révolutionnaire. Elle est même, en un sens, essentiellement conservatrice, puisqu’elle considère les faits sociaux comme des choses dont la nature, si souple et si malléable qu’elle soit, n’est pourtant pas modifiable à volonté. Combien est plus dangereuse la doctrine qui n’y voit que le produit de combinaisons mentales, qu’un simple artifice dialectique peut, en un instant, bouleverser de fond en comble !\par
De même, parce qu’on est habitué à se représenter la vie sociale comme le développement logique de concepts idéaux, on jugera peut-être grossière une méthode qui fait dépendre l’évolution collective de conditions objectives, définies dans l’espace, et il n’est pas impossible qu’on nous traite de matérialiste. Cependant, nous pourrions plus justement revendiquer la qualification contraire. En effet l’essence du spiritualisme ne tient-elle pas dans cette idée que les phénomènes psychiques ne peuvent pas être immédiatement dérivés des phénomènes organiques ? Or notre méthode n’est en partie qu’une application de ce principe aux faits sociaux. Comme les spiritualistes séparent le règne psychologique du règne biologique, nous séparons le premier du règne social ; comme eux, nous nous refusons à expliquer le plus complexe par le plus simple. À la vérité, pourtant, ni l’une ni l’autre appellation ne nous conviennent exactement ; la seule que nous acceptions est celle de {\itshape rationaliste}. Notre principal objectif, en effet, est d’étendre à la condition humaine le rationalisme scientifique, en faisant voir que, considérée dans le passé, elle est réductible à des rapports de cause à effet qu’une opération non moins rationnelle peut transformer ensuite en règles d’action pour l’avenir. Ce qu’on a appelé notre positivisme n’est qu’une conséquence de ce rationalisme\footnote{ C’est dire qu’il ne doit pas être confondu avec la métaphysique positiviste de Comte et de M. Spencer.}. On ne peut être tenté de dépasser les faits, soit pour en rendre compte soit pour en diriger le cours, que dans la mesure où on les croit irrationnels. S’ils sont intelligibles tout entiers, ils suffisent à la science comme à la pratique : à la science, car il n’y a pas alors de motif pour chercher en dehors d’eux les raisons qu’ils ont d’être ; à la pratique, car leur valeur utile est une de ces raisons. Il nous semble donc que, surtout par ce temps de mysticisme renaissant, une pareille entreprise peut et doit être accueillie sans inquiétude et même avec sympathie par tous ceux qui, tout en se séparant de nous sur certains points, partagent notre foi dans l’avenir de la raison.
\section[{Préface de la seconde édition}]{Préface de la seconde édition}\renewcommand{\leftmark}{Préface de la seconde édition}

\noindent Quand ce livre parut pour la première fois, il souleva d’assez vives controverses. Les idées courantes, comme déconcertées, résistèrent d’abord avec une telle énergie que, pendant un temps, il nous fut presque impossible de nous faire entendre. Sur les points mêmes où nous nous étions exprimé le plus explicitement, on nous prêta gratuitement des vues qui n’avaient rien de commun avec les nôtres, et l’on crut nous réfuter en les réfutant. Alors que nous avions déclaré à maintes reprises que la conscience, tant individuelle que sociale, n’était pour nous rien de substantiel, mais seulement un ensemble, plus ou moins systématisé, de phénomènes {\itshape sui generis}, on nous taxa de réalisme et d’ontologisme. Alors que nous avions dit expressément et répété de toutes les manières que la vie sociale était tout entière faite de représentations, on nous accusa d’éliminer l’élément mental de la sociologie. On alla même jusqu’à restaurer contre nous des procédés de discussion que l’on pouvait croire définitivement disparus. On nous imputa, en effet, certaines opinions que nous n’avions pas soutenues, sous prétexte qu’elles étaient « conformes à nos principes ». L’expérience avait pourtant prouvé tous les dangers de cette méthode qui, en permettant de construire arbitrairement les systèmes que l’on discute, permet aussi d’en triompher sans peine.\par
Nous ne croyons pas nous abuser en disant que, depuis, les résistances ont progressivement faibli. Sans doute, plus d’une proposition nous est encore contestée. Mais nous ne saurions ni nous étonner ni nous plaindre de ces contestations salutaires ; il est bien clair, en effet, que nos formules sont destinées à être réformées dans l’avenir. Résumé d’une pratique personnelle et forcément restreinte, elles devront nécessairement évoluer à mesure que l’on acquerra une expérience plus étendue et plus approfondie de la réalité sociale. En fait de méthode, d’ailleurs, on ne peut jamais faire que du provisoire ; car les méthodes changent à mesure que la science avance. Il n’en reste pas moins que, pendant ces dernières années, en dépit des oppositions, la cause de la sociologie objective, spécifique et méthodique a gagné du terrain sans interruption. La fondation de l’\emph{Année sociologique} a certainement été pour beaucoup dans ce résultat. Parce qu’elle embrasse à la fois tout le domaine de la science, l’\emph{Année} a pu, mieux qu’aucun ouvrage spécial, donner le sentiment de ce que la sociologie doit et peut devenir. On a pu voir ainsi qu’elle n’était pas condamnée à rester une branche de la philosophie générale, et que, d’autre part, elle pouvait entrer en contact avec le détail des faits sans dégénérer en pure érudition. Aussi ne saurions-nous trop rendre hommage à l’ardeur et au dévouement de nos collaborateurs ; c’est grâce à eux que cette démonstration par le fait a pu être tentée et qu’elle peut se poursuivre.\par
Cependant, quelque réels que soient ces progrès, il est incontestable que les méprises et les confusions passées ne sont pas encore tout entières dissipées. C’est pourquoi nous voudrions profiter de cette seconde édition pour ajouter quelques explications à toutes celles que nous avons déjà données, répondre à certaines critiques et apporter sur certains points des précisions nouvelles.\par
\subsection[{I}]{I}
\noindent La proposition d’après laquelle les faits sociaux doivent être traités comme des choses — proposition qui est à la base même de notre méthode — est de celles qui ont provoqué le plus de contradictions. On a trouvé paradoxal et scandaleux que nous assimilions aux réalités du monde extérieur celles du monde social. C’était se méprendre singulièrement sur le sens et la portée de cette assimilation, dont l’objet n’est pas de ravaler les formes supérieures de l’être aux formes inférieures, mais, au contraire, de revendiquer pour les premières un degré de réalité au moins égal à celui que tout le monde reconnaît aux secondes. Nous ne disons pas, en effet, que les faits sociaux sont des choses matérielles, mais sont des choses au même titre que les choses matérielles, quoique d’une autre manière.\par
Qu’est-ce en effet qu’une chose ? La chose s’oppose à l’idée comme ce que l’on connaît du dehors à ce que l’on connaît du dedans. Est chose tout objet de connaissance qui n’est pas naturellement compénétrable à l’intelligence, tout ce dont nous ne pouvons nous faire une notion adéquate par un simple procédé d’analyse mentale, tout ce que l’esprit ne peut arriver à comprendre qu’à condition de sortir de lui-même, par voie d’observations et d’expérimentations, en passant progressivement des caractères les plus extérieurs et les plus immédiatement accessibles aux moins visibles et aux plus profonds. Traiter des faits d’un certain ordre comme des choses, ce n’est donc pas les classer dans telle ou telle catégorie du réel ; c’est observer vis-à-vis d’eux une certaine attitude mentale. C’est en aborder l’étude en prenant pour principe qu’on ignore absolument ce qu’ils sont, et que leurs propriétés caractéristiques, comme les causes inconnues dont elles dépendent, ne peuvent être découvertes par l’introspection même la plus attentive.\par
Les termes ainsi définis, notre proposition, loin d’être un paradoxe, pourrait presque passer pour un truisme si elle n’était encore trop souvent méconnue dans les sciences qui traitent de l’homme, et surtout en sociologie. En effet, on peut dire en ce sens que tout objet de science est une chose, sauf, peut-être, les objets mathématiques ; car, pour ce qui est de ces derniers, comme nous les construisons nous-mêmes depuis les plus simples jusqu’aux plus complexes, il suffit, pour savoir ce qu’ils sont, de regarder au-dedans de nous et d’analyser intérieurement le processus mental d’où ils résultent. Mais dès qu’il s’agit de faits proprement dits, ils sont nécessairement pour nous, au moment où nous entreprenons d’en faire la science, des inconnus, des {\itshape choses} ignorées, car les représentations qu’on a pu s’en faire au cours de la vie, ayant été faites sans méthode et sans critique, sont dénuées de valeur scientifique et doivent être tenues à l’écart. Les faits de la psychologie individuelle eux-mêmes présentent ce caractère et doivent être considérés sous cet aspect. En effet, quoiqu’ils nous soient intérieurs par définition, la conscience que nous en avons ne nous en révèle ni la nature interne ni la genèse. Elle nous les fait bien connaître jusqu’à un certain point, mais seulement comme les sensations nous font connaître la chaleur ou la lumière, le son ou l’électricité ; elle nous en donne des impressions confuses, passagères, subjectives, mais non des notions claires et distinctes, des concepts explicatifs. Et c’est précisément pour cette raison qu’il s’est fondé au cours de ce siècle une psychologie objective dont la règle fondamentale est d’étudier les faits mentaux du dehors, c’est-à-dire comme des choses. À plus forte raison en doit-il être ainsi des faits sociaux ; car la conscience ne saurait être plus compétente pour en connaître que pour connaître de sa vie propre\footnote{ On voit que, pour admettre cette proposition, il n’est pas nécessaire de soutenir que la vie sociale est faite d’autre chose que de représentations ; il suffit de poser que les représentations, individuelles ou collectives, ne peuvent être étudiées scientifiquement qu’à condition d’être étudiées objectivement.}. — On objectera que, comme ils sont notre œuvre, nous n’avons qu’à prendre conscience de nous-mêmes pour savoir ce que nous y avons mis et comment nous les avons formés. Mais, d’abord, la majeure partie des institutions sociales nous sont léguées toutes faites par les générations antérieures ; nous n’avons pris aucune part à leur formation et, par conséquent, ce n’est pas en nous interrogeant que nous pourrons découvrir les causes qui leur ont donné naissance. De plus, alors même que nous avons collaboré à leur genèse, c’est à peine si nous entrevoyons de la manière la plus confuse, et souvent même la plus inexacte, les véritables raisons qui nous ont déterminé à agir et la nature de notre action. Déjà, alors qu’il s’agit simplement de nos démarches privées, nous savons bien mal les mobiles relativement simples qui nous guident ; nous nous croyons désintéressés alors que nous agissons en égoïstes, nous croyons obéir à la haine alors que nous cédons à l’amour, à la raison alors que nous sommes les esclaves de préjugés irraisonnés, etc. Comment donc aurions-nous la faculté de discerner avec plus de clarté les causes, autrement complexes, dont procèdent les démarches de la collectivité ? Car, à tout le moins, chacun de nous n’y prend part que pour une infime partie ; nous avons une multitude de collaborateurs et ce qui se passe dans les autres consciences nous échappe.\par
Notre règle n’implique donc aucune conception métaphysique, aucune spéculation sur le fond des êtres. Ce qu’elle réclame, c’est que le sociologue se mette dans l’état d’esprit où sont physiciens, chimistes, physiologistes, quand ils s’engagent dans une région, encore inexplorée, de leur domaine scientifique. Il faut qu’en pénétrant dans le monde social, il ait conscience qu’il pénètre dans l’inconnu ; il faut qu’il se sente en présence de faits dont les lois sont aussi insoupçonnées que pouvaient l’être celles de la vie, quand la biologie n’était pas constituée ; il faut qu’il se tienne prêt à faire des découvertes qui le surprendront et le déconcerteront. Or il s’en faut que la sociologie en soit arrivée à ce degré de maturité intellectuelle. Tandis que le savant qui étudie la nature physique a le sentiment très vif des résistances qu’elle lui oppose et dont il a tant de peine à triompher, il semble en vérité que le sociologue se meuve au milieu de choses immédiatement transparentes pour l’esprit, tant est grande l’aisance avec laquelle on le voit résoudre les questions les plus obscures. Dans l’état actuel de la science, nous ne savons véritablement pas ce que sont même les principales institutions sociales, comme l’État ou la famille, le droit de propriété ou le contrat, la peine et la responsabilité ; nous ignorons presque complètement les causes dont elles dépendent, les fonctions qu’elles remplissent, les lois de leur évolution ; c’est à peine si, sur certains points, nous commençons à entrevoir quelques lueurs. Et pourtant, il suffit de parcourir les ouvrages de sociologie pour voir combien est rare le sentiment de cette ignorance et de ces difficultés. Non seulement on se considère comme obligé de dogmatiser sur tous les problèmes à la fois, mais on croit pouvoir, en quelques pages ou en quelques phrases, atteindre l’essence même des phénomènes les plus complexes. C’est dire que de semblables théories expriment, non les faits qui ne sauraient être épuisés avec cette rapidité, mais la prénotion qu’en avait l’auteur, antérieurement à la recherche. Et sans doute, l’idée que nous nous faisons des pratiques collectives, de ce qu’elles sont ou de ce qu’elles doivent être, est un facteur de leur développement. Mais cette idée elle-même est un fait qui, pour être convenablement déterminé, doit, lui aussi, être étudié du dehors. Car ce qu’il importe de savoir, ce n’est pas la manière dont tel penseur individuellement se représente telle institution, mais la conception qu’en a le groupe ; seule, en effet, cette conception est socialement efficace. Or elle ne peut être connue par simple observation intérieure puisqu’elle n’est tout entière en aucun de nous ; il faut donc bien trouver quelques signes extérieurs qui la rendent sensible. De plus, elle n’est pas née de rien ; elle est elle-même un effet de causes externes qu’il faut connaître pour pouvoir apprécier son rôle dans l’avenir. Quoiqu’on fasse, c’est donc toujours à la même méthode qu’il en faut revenir.
\subsection[{II}]{II}
\noindent Une autre proposition n’a pas été moins vivement discutée que la précédente : c’est celle qui présente les phénomènes sociaux comme extérieurs aux individus. On nous accorde aujourd’hui assez volontiers que les faits de la vie individuelle et ceux de la vie collective sont hétérogènes à quelque degré ; on peut même dire qu’une entente, sinon unanime, du moins très générale, est en train de se faire sur ce point. Il n’y a plus guère de sociologues qui dénient à la sociologie toute espèce de spécificité. Mais parce que la Société n’est composée que d’individus\footnote{ La proposition n’est, d’ailleurs, que partiellement exacte. ― Outre les individus, il y a les choses qui sont des éléments intégrants de la société. Il est vrai seulement que les individus en sont les seuls éléments actifs.}, il semble au sens commun que la vie sociale ne puisse avoir d’autre substrat que la conscience individuelle ; autrement, elle paraît rester en l’air et planer dans le vide.\par
Pourtant, ce qu’on juge si facilement inadmissible quand il s’agit des faits sociaux, est couramment admis des autres règnes de la nature. Toutes les fois que des éléments quelconques, en se combinant, dégagent, par le fait de leur combinaison, des phénomènes nouveaux, il faut bien concevoir que ces phénomènes sont situés, non dans les éléments, mais dans le tout formé par leur union. La cellule vivante ne contient rien que des particules minérales, comme la société ne contient rien en dehors des individus ; et pourtant il est, de toute évidence, impossible que les phénomènes caractéristiques de la vie résident dans des atomes d’hydrogène, d’oxygène, de carbone et d’azote. Car comment les mouvements vitaux pourraient-ils se produire au sein d’éléments non vivants ? Comment, d’ailleurs, les propriétés biologiques se répartiraient-elles entre ces éléments ? Elles ne sauraient se retrouver également chez tous puisqu’ils ne sont pas de même nature ; le carbone n’est pas l’azote et, par suite, ne peut revêtir les mêmes propriétés ni jouer le même rôle. Il n’est pas moins inadmissible que chaque aspect de la vie, chacun de ses caractères principaux s’incarne dans un groupe différent d’atomes. La vie ne saurait se décomposer ainsi ; elle est une et, par conséquent, elle ne peut avoir pour siège que la substance vivante dans sa totalité. Elle est dans le tout, non dans les parties. Ce ne sont pas les particules non-vivantes de la cellule qui se nourrissent, se reproduisent, en un mot, qui vivent ; c’est la cellule elle-même et elle seule. Et ce que nous disons de la vie pourrait se répéter de toutes les synthèses possibles. La dureté du bronze n’est ni dans le cuivre ni dans l’étain ni dans le plomb qui ont servi à le former et qui sont des corps mous ou flexibles ; elle est dans leur mélange. La fluidité de l’eau, ses propriétés alimentaires et autres ne sont pas dans les deux gaz dont elle est composée, mais dans la substance complexe qu’ils forment par leur association.\par
Appliquons ce principe à la sociologie. Si, comme on nous l’accorde, cette synthèse {\itshape sui generis} qui constitue toute société dégage des phénomènes nouveaux, différents de ceux qui se passent dans les consciences solitaires, il faut bien admettre que ces faits spécifiques résident dans la société même qui les produit, et non dans ses parties, c’est-à-dire dans ses membres. Ils sont donc, en ce sens, extérieurs aux consciences individuelles, considérées comme telles, de même que les caractères distinctifs de la vie sont extérieurs aux substances minérales qui composent l’être vivant. On ne peut les résorber dans les éléments sans se contredire, puisque, par définition, ils supposent autre chose que ce que contiennent ces éléments. Ainsi se trouve justifiée, par une raison nouvelle, la séparation que nous avons établie plus loin entre la psychologie proprement dite, ou science de l’individu mental, et la sociologie. Les faits sociaux ne diffèrent pas seulement en qualité des faits psychiques ; {\itshape ils ont un autre substrat}, ils n’évoluent pas dans le même milieu, ils ne dépendent pas des mêmes conditions. Ce n’est pas à dire qu’ils ne soient, eux aussi, psychiques en quelque manière puisqu’ils consistent tous en des façons de penser ou d’agir. Mais les états de la conscience collective sont d’une autre nature que les états de la conscience individuelle ; ce sont des représentations d’une autre sorte. La mentalité des groupes n’est pas celle des particuliers ; elle a ses lois propres. Les deux sciences sont donc aussi nettement distinctes que deux sciences peuvent l’être, quelques rapports qu’il puisse, par ailleurs, y avoir entre elles.\par
Toutefois, sur ce point, il y a lieu de faire une distinction qui jettera peut-être quelque lumière sur le débat.\par
Que {\itshape la matière} de la vie sociale ne puisse pas s’expliquer par des facteurs purement psychologiques, c’est-à-dire par des états de la conscience individuelle, c’est ce qui nous paraît être l’évidence même. En effet, ce que les représentations collectives traduisent, c’est la façon dont le groupe se pense dans ses rapports avec les objets qui l’affectent. Or le groupe est constitué autrement que l’individu et les choses qui l’affectent sont d’une autre nature. Des représentations qui n’expriment ni les mêmes sujets ni les mêmes objets ne sauraient dépendre des mêmes causes. Pour comprendre la manière dont la société se représente elle-même et le monde qui l’entoure, c’est la nature de la société, et non celle des particuliers, qu’il faut considérer. Les symboles sous lesquels elle se pense changent suivant ce qu’elle est. Si, par exemple, elle se conçoit comme issue d’un animal éponyme, c’est qu’elle forme un de ces groupes spéciaux qu’on appelle des clans. Là où l’animal est remplacé par un ancêtre humain, mais également mythique, c’est que le clan a changé de nature. Si, au-dessus des divinités locales ou familiales, elle en imagine d’autres dont elle croit dépendre, c’est que les groupes locaux et familiaux dont elle est composée tendent à se concentrer et à s’unifier, et le degré d’unité que présente un panthéon religieux correspond au degré d’unité atteint au même moment par la société. Si elle condamne certains modes de conduite, c’est qu’ils froissent certains de ses sentiments fondamentaux ; et ces sentiments tiennent à sa constitution, comme ceux de l’individu à son tempérament physique et à son organisation mentale. Ainsi, alors même que la psychologie individuelle n’aurait plus de secrets pour nous, elle ne saurait nous donner la solution d’aucun de ces problèmes, puisqu’ils se rapportent à des ordres de faits qu’elle ignore.\par
Mais, cette hétérogénéité une fois reconnue, on peut se demander si les représentations individuelles et les représentations collectives ne laissent pas, cependant, de se ressembler en ce que les unes et les autres sont également des représentations ; et si, par suite de ces ressemblances, certaines lois abstraites ne seraient pas communes aux deux règnes. Les mythes, les légendes populaires, les conceptions religieuses de toute sorte, les croyances morales, etc., expriment une autre réalité que la réalité individuelle ; mais il se pourrait que la manière dont elles s’attirent ou se repoussent, s’agrègent ou se désagrègent, soit indépendante de leur contenu et tienne uniquement à leur qualité générale de représentations. Tout en étant faites d’une matière différente, elles se comporteraient dans leurs relations mutuelles comme font les sensations, les images, ou les idées chez l’individu. Ne peut-on croire, par exemple, que la contiguïté et la ressemblance, les contrastes et les antagonismes logiques agissent de la même façon, quelles que soient les choses représentées ? On en vient ainsi à concevoir la possibilité d’une psychologie toute formelle qui serait une sorte de terrain commun à la psychologie individuelle et à la sociologie ; et c’est peut-être ce qui fait le scrupule qu’éprouvent certains esprits à distinguer trop nettement ces deux sciences.\par
À parler rigoureusement, dans l’état actuel de nos connaissances, la question ainsi posée ne saurait recevoir de solution catégorique. En effet, d’une part, tout ce que nous savons sur la manière dont se combinent les idées individuelles se réduit à ces quelques propositions, très générales et très vagues, que l’on appelle communément lois de l’association des idées. Et quant aux lois de l’idéation collective, elles sont encore plus complètement ignorées. La psychologie sociale, qui devrait avoir pour tâche de les déterminer, n’est guère qu’un mot qui désigne toutes sortes de généralités, variées et imprécises, sans objet défini. Ce qu’il faudrait, c’est chercher, par la comparaison des thèmes mythiques, des légendes et des traditions populaires, des langues, de quelle façon les représentations sociales s’appellent et s’excluent, fusionnent les unes dans les autres ou se distinguent, etc. Or si le problème mérite de tenter la curiosité des chercheurs, à peine peut-on dire qu’il soit abordé, et tant que l’on n’aura pas trouvé quelques-unes de ces lois, il sera évidemment impossible de savoir avec certitude si elles répètent ou non celles de la psychologie individuelle.\par
Cependant, à défaut de certitude, il est tout au moins probable que, s’il existe des ressemblances entre ces deux sortes de lois, les différences ne doivent pas être moins marquées. Il paraît, en effet, inadmissible que la matière dont sont faites les représentations n’agisse pas sur leurs modes de combinaisons. Il est vrai que les psychologues parlent parfois des lois de l’association des idées, comme si elles étaient les mêmes pour toutes les espèces de représentations individuelles. Mais rien n’est moins vraisemblable : les images ne se composent pas entre elles comme les sensations, ni les concepts comme les images. Si la psychologie était plus avancée, elle constaterait, sans doute, que chaque catégorie d’états mentaux a ses lois formelles qui lui sont propres. S’il en est ainsi, on doit {\itshape a fortiori} s’attendre à ce que les lois correspondantes de la pensée sociale soient spécifiques comme cette pensée elle-même. En fait, pour peu qu’on ait pratiqué cet ordre de faits, il est difficile de ne pas avoir le sentiment de cette spécificité. N’est-ce pas elle, en effet, qui nous fait paraître si étrange la manière si spéciale dont les conceptions religieuses (qui sont collectives au premier chef) se mêlent ou se séparent, se transforment les unes dans les autres, donnant naissance à des composés contradictoires qui contrastent avec les produits ordinaires de notre pensée privée. Si donc, comme il est présumable, certaines lois de la mentalité sociale rappellent effectivement certaines de celles qu’établissent les psychologues, ce n’est pas que les premières soient un simple cas particulier des secondes ; mais c’est qu’entre les unes et les autres, à côté de différences certainement importantes, il y a des similitudes que l’abstraction pourra dégager, et qui d’ailleurs, sont encore ignorées. C’est dire qu’en aucun cas la sociologie ne saurait emprunter purement et simplement à la psychologie telle ou telle de ses propositions, pour l’appliquer telle quelle aux faits sociaux. Mais la pensée collective tout entière, dans sa forme comme dans sa matière, doit être étudiée en elle-même, pour elle-même, avec le sentiment de ce qu’elle a de spécial, et il faut laisser à l’avenir le soin de rechercher dans quelle mesure elle ressemble à la pensée des particuliers. C’est même là un problème qui ressortit plutôt à la philosophie générale et à la logique abstraite qu’à l’étude scientifique des faits sociaux\footnote{ Il est inutile de montrer comment, de ce point de vue, la nécessité d’étudier les faits du dehors apparaît plus évidente encore, puisqu’ils résultent de synthèses qui ont lieu hors de nous et dont nous n’avons même pas la perception confuse que la conscience peut nous donner des phénomènes intérieurs.}.
\subsection[{III}]{III}
\noindent Il nous reste à dire quelques mots de la définition que nous avons donnée des faits sociaux dans notre premier chapitre. Nous les faisons consister en des manières de faire ou de penser, reconnaissables à cette particularité qu’elles sont susceptibles d’exercer sur les consciences particulières une influence coercitive. — Une confusion s’est produite à ce sujet qui mérite d’être notée.\par
On a tellement l’habitude d’appliquer aux choses sociologiques les formes de la pensée philosophique qu’on a souvent vu dans cette définition préliminaire une sorte de philosophie du fait social. On a dit que nous expliquions les phénomènes sociaux par la contrainte, de même que M. Tarde les explique par l’imitation. Nous n’avions point une telle ambition et il ne nous était même pas venu à l’esprit qu’on pût nous la prêter, tant elle est contraire à toute méthode. Ce que nous nous proposions était, non d’anticiper par une vue philosophique les conclusions de la science, mais simplement d’indiquer à quels signes extérieurs il est possible de reconnaître les faits dont elle doit traiter, afin que le savant sache les apercevoir là où ils sont et ne les confonde pas avec d’autres. Il s’agissait de délimiter le champ de la recherche aussi bien que possible, non de l’embrasser dans une sorte d’intuition exhaustive. Aussi acceptons-nous très volontiers le reproche qu’on a fait à cette définition de ne pas exprimer tous les caractères du fait social et, par suite, de n’être pas la seule possible. Il n’y a, en effet, rien d’inconcevable à ce qu’il puisse être caractérisé de plusieurs manières différentes ; car il n’y a pas de raison pour qu’il n’ait qu’une seule propriété distinctive\footnote{ Le pouvoir coercitif que nous lui attribuons est même si peu le tout du fait social, qu’il peut présenter également le caractère opposé. Car, en même temps que les institutions s’imposent à nous, nous y tenons ; elles nous obligent et nous les aimons ; elles nous contraignent et nous trouvons notre compte à leur fonctionnement et à cette contrainte même. Cette antithèse est celle que les moralistes ont souvent signalée entre les deux notions du bien et du devoir qui expriment deux aspects différents, mais également réels, de la vie morale. Or il n’est peut-être pas de pratiques collectives qui n’exercent sur nous cette double action, qui n’est, d’ailleurs, contradictoire qu’en apparence. Si nous ne les avons pas définies par cet attachement spécial, à la fois intéressé et désintéressé, c’est tout simplement qu’il ne se manifeste pas par des signes extérieurs, facilement perceptibles. Le bien a quelque chose de plus interne, de plus intime que le devoir, partant, de moins saisissable.}. Tout ce qu’il importe, c’est de choisir celle qui paraît la meilleure pour le but qu’on se propose. Même il est très possible d’employer concurremment plusieurs critères, suivant les circonstances. Et c’est ce que nous avons reconnu nous-même être parfois nécessaire en sociologie ; car il y a des cas où le caractère de contrainte n’est pas facilement reconnaissable (voir p. 19). Tout ce qu’il faut, puisqu’il s’agit d’une définition initiale, c’est que les caractéristiques dont on se sert soient immédiatement discernables et puissent être aperçues avant la recherche. Or, c’est cette condition que ne remplissent pas les définitions que l’on a parfois opposées à la nôtre. On a dit, par exemple, que le fait social, c’est « tout ce qui se produit dans et par la société », ou encore « ce qui intéresse et affecte le groupe en quelque façon ». Mais on ne peut savoir si la société est ou non la cause d’un fait ou si ce fait a des effets sociaux que quand la science est déjà avancée. De telles définitions ne sauraient donc servir à déterminer l’objet de l’investigation qui commence. Pour qu’on puisse les utiliser, il faut que l’étude des faits sociaux ait été déjà poussée assez loin et, par suite, qu’on ait découvert quelque autre moyen préalable de les reconnaître là où ils sont.\par
En même temps qu’on a trouvé notre définition trop étroite, on l’a accusée d’être trop large et de comprendre presque tout le réel. En effet, a-t-on dit, tout milieu physique exerce une contrainte sur les êtres qui subissent son action ; car ils sont tenus, dans une certaine mesure, de s’y adapter. — Mais il y a entre ces deux modes de coercition toute la différence qui sépare un milieu physique et un milieu moral. La pression exercée par un ou plusieurs corps sur d’autres corps ou même sur des volontés ne saurait être confondue avec celle qu’exerce la conscience d’un groupe sur la conscience de ses membres. Ce qu’a de tout à fait spécial la contrainte sociale, c’est qu’elle est due, non à la rigidité de certains arrangements moléculaires, mais au prestige dont sont investies certaines représentations. Il est vrai que les habitudes, individuelles ou héréditaires, ont, à certains égards, cette même propriété. Elles nous dominent, nous imposent des croyances ou des pratiques. Seulement, elles nous dominent du dedans ; car elles sont tout entières en chacun de nous. Au contraire, les croyances et les pratiques sociales agissent sur nous du dehors : aussi l’ascendant exercé par les unes et par les autres est-il, au fond, très différent.\par
Il ne faut pas s’étonner, d’ailleurs, que les autres phénomènes de la nature présentent, sous d’autres formes, le caractère même par lequel nous avons défini les phénomènes sociaux. Cette similitude vient simplement de ce que les uns et les autres sont des choses réelles. Car tout ce qui est réel a une nature définie qui s’impose, avec laquelle il faut compter et qui, alors même qu’on parvient à la neutraliser, n’est jamais complètement vaincue. Et, au fond, c’est là ce qu’il y a de plus essentiel dans la notion de la contrainte sociale. Car tout ce qu’elle implique, c’est que les manières collectives d’agir ou de penser ont une réalité en dehors des individus qui, à chaque moment du temps, s’y conforment. Ce sont des choses qui ont leur existence propre. L’individu les trouve toutes formées et il ne peut pas faire qu’elles ne soient pas ou qu’elles soient autrement qu’elles ne sont ; il est donc bien obligé d’en tenir compte et il lui est d’autant plus difficile (nous ne disons pas impossible) de les modifier que, à des degrés divers, elles participent de la suprématie matérielle et morale que la société a sur ses membres. Sans doute, l’individu joue un rôle dans leur genèse. Mais pour qu’il y ait fait social, il faut que plusieurs individus tout au moins aient mêlé leur action et que cette combinaison ait dégagé quelque produit nouveau. Et comme cette synthèse a lieu en dehors de chacun de nous (puisqu’il y entre une pluralité de consciences), elle a nécessairement pour effet de fixer, d’instituer hors de nous de certaines façons d’agir et de certains jugements qui ne dépendent pas de chaque volonté particulière prise à part. Ainsi qu’on l’a fait remarquer\footnote{ Voir Art. « Sociologie » de la \emph{Grande Encyclopédie}, par MM. Fauconnet et Mauss.}, il y a un mot qui, pourvu toutefois qu’on en étende un peu l’acception ordinaire, exprime assez bien cette manière d’être très spéciale : c’est celui d’institution. On peut en effet, sans dénaturer le sens de cette expression, appeler {\itshape institution}, toutes les croyances et tous les modes de conduite institués par la collectivité ; la sociologie peut alors être définie : la science des institutions, de leur genèse et de leur fonctionnement\footnote{ De ce que les croyances et les pratiques sociales nous pénètrent ainsi du dehors, il ne suit pas que nous les recevions passivement et sans leur faire subir de modification. En pensant les institutions collectives, en nous les assimilant, nous les individualisons, nous leur donnons plus ou moins notre marque personnelle ; c’est ainsi qu’en pensant le monde sensible chacun de nous le colore à sa façon et que des sujets différents s’adaptent différemment à un même milieu physique. C’est pourquoi chacun de nous se fait, dans une certaine mesure, sa morale, sa religion, sa technique. Il n’est pas de conformisme social qui ne comporte toute une gamme de nuances individuelles. Il n’en reste pas moins que le champ des variations permises est limité. Il est nul ou très faible dans le cercle des phénomènes religieux et moraux ou la variation devient aisément un crime ; il est plus étendu pour tout ce qui concerne la vie économique. Mais tôt ou tard, même dans ce dernier cas, on rencontre une limite qui ne peut être franchie.}.\par
Sur les autres controverses qu’a suscitées cet ouvrage, il nous paraît inutile de revenir ; car elles ne touchent à rien d’essentiel. L’orientation générale de la méthode ne dépend pas des procédés que l’on préfère employer soit pour classer les types sociaux, soit pour distinguer le normal du pathologique. D’ailleurs, ces contestations sont très souvent venues de ce que l’on se refusait à admettre, ou de ce que l’on n’admettait pas sans réserves, notre principe fondamental : la réalité objective des faits sociaux. C’est donc finalement sur ce principe que tout repose, et tout y ramène. C’est pourquoi il nous a paru utile de le mettre une fois de plus en relief, en le dégageant de toute question secondaire. Et nous sommes assuré qu’en lui attribuant une telle prépondérance nous restons fidèle à la tradition sociologique ; car, au fond, c’est de cette conception que la sociologie tout entière est sortie. Cette science, en effet, ne pouvait naître que le jour où l’on eut pressenti que les phénomènes sociaux, pour n’être pas matériels, ne laissent pas d’être des choses réelles qui comportent l’étude. Pour être arrivé à penser qu’il y avait lieu de rechercher ce qu’ils sont, il fallait avoir compris qu’ils sont d’une façon définie, qu’ils ont une manière d’être constante, une nature qui ne dépend pas de l’arbitraire individuel et d’où dérivent des rapports nécessaires. Aussi l’histoire de la sociologie n’est-elle qu’un long effort en vue de préciser ce sentiment, de l’approfondir, de développer toutes les conséquences qu’il implique. Mais, malgré les grands progrès qui ont été faits en ce sens, on verra par la suite de ce travail qu’il reste encore de nombreuses survivances du postulat anthropocentrique qui, ici comme ailleurs, barre la route à la science. Il déplaît à l’homme de renoncer au pouvoir illimité qu’il s’est si longtemps attribué sur l’ordre social, et, d’autre part, il lui semble que, s’il existe vraiment des forces collectives, il est nécessairement condamné à les subir sans pouvoir les modifier. C’est ce qui l’incline à les nier. En vain des expériences répétées lui ont appris que cette toute-puissance, dans l’illusion de laquelle il s’entretient avec complaisance, a toujours été pour lui une cause de faiblesse ; que son empire sur les choses n’a réellement commencé qu’à partir du moment où il reconnut qu’elles ont une nature propre, et où il se résigna à apprendre d’elles ce qu’elles sont. Chassé de toutes les autres sciences, ce déplorable préjugé se maintient opiniâtrement en sociologie. Il n’y a donc rien de plus urgent que de chercher à en affranchir définitivement notre science ; et c’est le but principal de nos efforts.
\section[{Introduction}]{Introduction}\renewcommand{\leftmark}{Introduction}

\noindent Jusqu’à présent, les sociologues se sont peu préoccupés de caractériser et de définir la méthode qu’ils appliquent à l’étude des faits sociaux. C’est ainsi que, dans toute l’œuvre de M. Spencer, le problème méthodologique n’occupe aucune place ; car l’\emph{Introduction à la science sociale}, dont le titre pourrait faire illusion, est consacrée à démontrer les difficultés et la possibilité de la sociologie, non à exposer les procédés dont elle doit se servir. Mill, il est vrai, s’est assez longuement occupé de la question\footnote{\emph{Système de Logique}, l. VI, ch.VII-XII.} ; mais il n’a fait que passer au crible de sa dialectique ce que Comte en avait dit, sans y rien ajouter de vraiment personnel. Un chapitre du \emph{Cours de philosophie positive}, voilà donc, à peu près, la seule étude originale et importante que nous possédions sur la matière\footnote{ V. 2e éd., p. 294-336.}.\par
Cette insouciance apparente n’a, d’ailleurs, rien qui doive surprendre. En effet, les grands sociologues dont nous venons de rappeler les noms ne sont guère sortis des généralités sur la nature des sociétés, sur les rapports du règne social et du règne biologique, sur la marche générale du progrès ; même la volumineuse sociologie de M. Spencer n’a guère d’autre objet que de montrer comment la loi de l’évolution universelle s’applique aux sociétés. Or, pour traiter ces questions philosophiques, des procédés spéciaux et complexes ne sont pas nécessaires. On se contentait donc de peser les mérites comparés de la déduction et de l’induction et de faire une enquête sommaire sur les ressources les plus générales dont dispose l’investigation sociologique. Mais les précautions à prendre dans l’observation des faits, la manière dont les principaux problèmes doivent être posés, le sens dans lequel les recherches doivent être dirigées, les pratiques spéciales qui peuvent leur permettre d’aboutir, les règles qui doivent présider à l’administration des preuves restaient indéterminées.\par
Un heureux concours de circonstances, au premier rang desquelles il est juste de mettre l’acte d’initiative qui a créé en notre faveur un cours régulier de sociologie à la Faculté des lettres de Bordeaux, nous ayant permis de nous consacrer de bonne heure à l’étude de la science sociale et d’en faire même la matière de nos occupations professionnelles, nous avons pu sortir de ces questions trop générales et aborder un certain nombre de problèmes particuliers. Nous avons donc été amené, par la force même des choses, à nous faire une méthode plus définie, croyons-nous, plus exactement adaptée à la nature particulière des phénomènes sociaux. Ce sont ces résultats de notre pratique que nous voudrions exposer ici dans leur ensemble et soumettre à la discussion. Sans doute, ils sont implicitement contenus dans le livre que nous avons récemment publié sur \emph{La Division du travail social}. Mais il nous paraît qu’il y a quelque intérêt à les en dégager, à les formuler à part, en les accompagnant de leurs preuves et en les illustrant d’exemples empruntés soit à cet ouvrage, soit à des travaux encore inédits. On pourra mieux juger ainsi de l’orientation que nous voudrions essayer de donner aux études de sociologie.

\chapteropen
\chapter[{Chapitre I : Qu’est-ce qu’un fait social ?}]{Chapitre I : \\
Qu’est-ce qu’un fait social ?}\renewcommand{\leftmark}{Chapitre I : \\
Qu’est-ce qu’un fait social ?}


\chaptercont
\noindent Avant de chercher quelle est la méthode qui convient à l’étude des faits sociaux, il importe de savoir quels sont les faits que l’on appelle ainsi.\par
La question est d’autant plus nécessaire que l’on se sert de cette qualification sans beaucoup de précision. On l’emploie couramment pour désigner à peu près tous les phénomènes qui se passent à l’intérieur de la société, pour peu qu’ils présentent, avec une certaine généralité, quelque intérêt social. Mais, à ce compte, il n’y a, pour ainsi dire, pas d’événements humains qui ne puissent être appelés sociaux. Chaque individu boit, dort, mange, raisonne et la société a tout intérêt à ce que ces fonctions s’exercent régulièrement. Si donc ces faits étaient sociaux, la sociologie n’aurait pas d’objet qui lui fût propre, et son domaine se confondrait avec celui de la biologie et de la psychologie.\par
Mais, en réalité, il y a dans toute société un groupe déterminé de phénomènes qui se distinguent par des caractères tranchés de ceux qu’étudient les autres sciences de la nature. Quand je m’acquitte de ma tâche de frère, d’époux ou de citoyen, quand j’exécute les engagements que j’ai contractés, je remplis des devoirs qui sont définis, en dehors de moi et de mes actes, dans le droit et dans les mœurs. Alors même qu’ils sont d’accord avec mes sentiments propres et que j’en sens intérieurement la réalité, celle-ci ne laisse pas d’être objective ; car ce n’est pas moi qui les ai faits, mais je les ai reçus par l’éducation. Que de fois, d’ailleurs, il arrive que nous ignorons le détail des obligations qui nous incombent et que, pour les connaître, il nous faut consulter le Code et ses interprètes autorisés ! De même, les croyances et les pratiques de sa vie religieuse, le fidèle les a trouvées toutes faites en naissant ; si elles existaient avant lui, c’est qu’elles existent en dehors de lui. Le système de signes dont je me sers pour exprimer ma pensée, le système de monnaies que j’emploie pour payer mes dettes, les instruments de crédit que j’utilise dans mes relations commerciales, les pratiques suivies dans ma profession, etc., etc., fonctionnent indépendamment des usages que j’en fais. Qu’on prenne les uns après les autres tous les membres dont est composée la société, ce qui précède pourra être répété à propos de chacun d’eux. Voilà donc des manières d’agir, de penser et de sentir qui présentent cette remarquable propriété qu’elles existent en dehors des consciences individuelles.\par
Non seulement ces types de conduite ou de pensée sont extérieurs à l’individu, mais ils sont doués d’une puissance impérative et coercitive en vertu de laquelle ils s’imposent à lui, qu’il le veuille ou non. Sans doute, quand je m’y conforme de mon plein gré, cette coercition ne se fait pas ou se fait peu sentir, y étant inutile. Mais elle n’en est pas moins un caractère intrinsèque de ces faits, et la preuve, c’est qu’elle s’affirme dès que je tente de résister. Si j’essaye de violer les règles du droit, elles réagissent contre moi de manière à empêcher mon acte s’il en est temps, ou à l’annuler et à le rétablir sous sa forme normale s’il est accompli et réparable, ou à me le faire expier s’il ne peut être réparé autrement. S’agit-il de maximes purement morales ? La conscience publique contient tout acte qui les offense par la surveillance qu’elle exerce sur la conduite des citoyens et les peines spéciales dont elle dispose. Dans d’autres cas, la contrainte est moins violente ; elle ne laisse pas d’exister. Si je ne me soumets pas aux conventions du monde, si, en m’habillant, je ne tiens aucun compte des usages suivis dans mon pays et dans ma classe, le rire que je provoque, l’éloignement ou l’on me tient, produisent, quoique d’une manière plus atténuée, les mêmes effets qu’une peine proprement dite. Ailleurs, la contrainte, pour n’être qu’indirecte, n’en est pas moins efficace. Je ne suis pas obligé de parler français avec mes compatriotes, ni d’employer les monnaies légales ; mais il est impossible que je fasse autrement. Si j’essayais d’échapper à cette nécessité, ma tentative échouerait misérablement. Industriel, rien ne m’interdit de travailler avec des procédés et des méthodes de l’autre siècle ; mais, si je le fais, je me ruinerai à coup sûr. Alors même que, en fait, je puis m’affranchir de ces règles et les violer avec succès, ce n’est jamais sans être obligé de lutter contre elles. Quand même elles sont finalement vaincues, elles font suffisamment sentir leur puissance contraignante par la résistance qu’elles opposent. Il n’y a pas de novateur, même heureux, dont les entreprises ne viennent se heurter à des oppositions de ce genre.\par
Voilà donc un ordre de faits qui présentent des caractères très spéciaux : ils consistent en des manières d’agir, de penser et de sentir, extérieures à l’individu, et qui sont douées d’un pouvoir de coercition en vertu duquel ils s’imposent à lui. Par suite, ils ne sauraient se confondre avec les phénomènes organiques, puisqu’ils consistent en représentations et en actions ; ni avec les phénomènes psychiques, lesquels n’ont d’existence que dans la conscience individuelle et par elle. Ils constituent donc une espèce nouvelle et c’est à eux que doit être donnée et réservée la qualification de {\itshape sociaux}. Elle leur convient ; car il est clair que, n’ayant pas l’individu pour substrat, ils ne peuvent en avoir d’autre que la société, soit la société politique dans son intégralité, soit quelqu’un des groupes partiels qu’elle renferme, confessions religieuses, écoles politiques, littéraires, corporations professionnelles, etc. D’autre part, c’est à eux seuls qu’elle convient ; car le mot de social n’a de sens défini qu’à condition de désigner uniquement des phénomènes qui ne rentrent dans aucune des catégories de faits déjà constituées et dénommées. Ils sont donc le domaine propre de la sociologie. Il est vrai que ce mot de contrainte, par lequel nous les définissons, risque d’effaroucher les zélés partisans d’un individualisme absolu. Comme ils professent que l’individu est parfaitement autonome, il leur semble qu’on le diminue toutes les fois qu’on lui fait sentir qu’il ne dépend pas seulement de lui-même. Mais puisqu’il est aujourd’hui incontestable que la plupart de nos idées et de nos tendances ne sont pas élaborées par nous, mais nous viennent du dehors, elles ne peuvent pénétrer en nous qu’en s’imposant ; c’est tout ce que signifie notre définition. On sait, d’ailleurs, que toute contrainte sociale n’est pas nécessairement exclusive de la personnalité individuelle\footnote{ Ce n’est pas à dire, du reste, que toute contrainte soit normale. Nous reviendrons plus loin sur ce point.}.\par
Cependant, comme les exemples que nous venons le citer (règles juridiques, morales, dogmes religieux, systèmes financiers, etc.), consistent tous en croyances et en pratiques constituées, on pourrait, d’après ce qui précède, croire qu’il n’y a de fait social que là où il y a organisation définie. Mais il est d’autres faits qui, sans présenter ces formes cristallisées, ont et la même objectivité et le même ascendant sur l’individu. C’est ce qu’on appelle les courants sociaux. Ainsi, dans une assemblée, les grands mouvements d’enthousiasme, d’indignation, de pitié qui se produisent, n’ont pour lieu d’origine aucune conscience particulière. Ils viennent à chacun de nous du dehors et sont susceptibles de nous entraîner malgré nous. Sans doute, il peut se faire que, m’y abandonnant sans réserve, je ne sente pas la pression qu’ils exercent sur moi. Mais elle s’accuse dès que j’essaie de lutter contre eux. Qu’un individu tente de s’opposer à l’une de ces manifestations collectives, et les sentiments qu’il nie se retournent contre lui. Or, si cette puissance de coercition externe s’affirme avec cette netteté dans les cas de résistance, c’est qu’elle existe, quoique inconsciente, dans les cas contraires. Nous sommes alors dupes d’une illusion qui nous fait croire que nous avons élaboré nous-même ce qui s’est imposé à nous du dehors. Mais, si la complaisance avec laquelle nous nous y laissons aller masque la poussée subie, elle ne la supprime pas. C’est ainsi que l’air ne laisse pas d’être pesant quoique nous n’en sentions plus le poids. Alors même que nous avons spontanément collaboré, pour notre part, à l’émotion commune, l’impression que nous avons ressentie est tout autre que celle que nous eussions éprouvée si nous avions été seul. Aussi, une fois que l’assemblée s’est séparée, que ces influences sociales ont cessé d’agir sur nous et que nous nous retrouvons seul avec nous-même, les sentiments par lesquels nous avons passé nous font l’effet de quelque chose d’étranger où nous ne nous reconnaissons plus. Nous nous apercevons alors que nous les avions subis beaucoup plus que nous ne les avions faits. Il arrive même qu’ils nous font horreur, tant ils étaient contraires à notre nature. C’est ainsi que des individus, parfaitement inoffensifs pour la plupart, peuvent, réunis en foule, se laisser entraîner à des actes d’atrocité. Or, ce que nous disons de ces explosions passagères s’applique identiquement à ces mouvements d’opinion, plus durables, qui se produisent sans cesse autour de nous, soit dans toute l’étendue de la société, soit dans des cercles plus restreints, sur les matières religieuses, politiques, littéraires, artistiques, etc.\par
On peut, d’ailleurs, confirmer par une expérience caractéristique cette définition du fait social, il suffit d’observer la manière dont sont élevés les enfants. Quand on regarde les faits tels qu’ils sont et tels qu’ils ont toujours été, il saute aux yeux que toute éducation consiste dans un effort continu pour imposer à l’enfant des manières de voir, de sentir et d’agir auxquelles il ne serait pas spontanément arrivé. Dès les premiers temps de sa vie, nous le contraignons à manger, à boire, à dormir à des heures régulières, nous le contraignons à la propreté, au calme, à l’obéissance ; plus tard, nous le contraignons pour qu’il apprenne à tenir compte d’autrui, à respecter les usages, les convenances, nous le contraignons au travail, etc., etc. Si, avec le temps, cette contrainte cesse d’être sentie, c’est qu’elle donne peu à peu naissance à des habitudes, à des tendances internes qui la rendent inutile, mais qui ne la remplacent que parce qu’elles en dérivent. Il est vrai que, d’après M. Spencer, une éducation rationnelle devrait réprouver de tels procédés et laisser faire l’enfant en toute liberté ; mais comme cette théorie pédagogique n’a jamais été pratiquée par aucun peuple connu, elle ne constitue qu’un {\itshape desideratum} personnel, non un fait qui puisse être opposé aux faits qui précèdent. Or, ce qui rend ces derniers particulièrement instructifs, c’est que l’éducation a justement pour objet de faire l’être social ; on y peut donc voir, comme en raccourci, de quelle manière cet être s’est constitué dans l’histoire. Cette pression de tous les instants que subit l’enfant, c’est la pression même du milieu social qui tend à le façonner à son image et dont les parents et les maîtres ne sont que les représentants et les intermédiaires.\par
Ainsi ce n’est pas leur généralité qui peut servir à caractériser les phénomènes sociologiques. Une pensée qui se retrouve dans toutes les consciences particulières, un mouvement que répètent tous les individus ne sont pas pour cela des faits sociaux. Si l’on s’est contenté de ce caractère pour les définir, c’est qu’on les a confondus, à tort, avec ce qu’on pourrait appeler leurs incarnations individuelles. Ce qui les constitue, ce sont les croyances, les tendances, les pratiques du groupe pris collectivement ; quant aux formes que revêtent les états collectifs en se réfractant chez les individus, ce sont choses d’une autre espèce. Ce qui démontre catégoriquement cette dualité de nature, c’est que ces deux ordres de faits se présentent souvent à l’état dissocié. En effet, certaines de ces manières d’agir ou de penser acquièrent, par suite de la répétition, une sorte de consistance qui les précipite, pour ainsi dire, et les isole des événements particuliers qui les reflètent. Elles prennent ainsi un corps, une forme sensible qui leur est propre, et constituent une réalité {\itshape sui generis}, très distincte des faits individuels qui la manifestent. L’habitude collective n’existe pas seulement à l’état d’immanence dans les actes successifs qu’elle détermine, mais, par un privilège dont nous ne trouvons pas d’exemple dans le règne biologique, elle s’exprime une fois pour toutes dans une formule qui se répète de bouche en bouche, qui se transmet par l’éducation, qui se fixe même par écrit. Telle est l’origine et la nature des règles juridiques, morales, des aphorismes et des dictons populaires, des articles de foi où les sectes religieuses ou politiques condensent leurs croyances, des codes de goût que dressent les écoles littéraires, etc. Aucune d’elles ne se retrouve tout entière dans les applications qui en sont faites par les particuliers, puisqu’elles peuvent même être sans être actuellement appliquées.\par
Sans doute, cette dissociation ne se présente pas toujours avec la même netteté. Mais il suffit qu’elle existe d’une manière incontestable dans les cas importants et nombreux que nous venons de rappeler, pour prouver que le fait social est distinct de ses répercussions individuelles. D’ailleurs, alors même qu’elle n’est pas immédiatement donnée à l’observation, on peut souvent la réaliser à l’aide de certains artifices de méthode ; il est même indispensable de procéder à cette opération, si l’on veut dégager le fait social de tout alliage pour l’observer à l’état de pureté. Ainsi, il y a certains courants d’opinion qui nous poussent, avec une intensité inégale, suivant les temps et les pays, l’un au mariage, par exemple, un autre au suicide ou à une natalité plus ou moins forte, etc. Ce sont évidemment des faits sociaux. Au premier abord, ils semblent inséparables des formes qu’ils prennent dans les cas particuliers. Mais la statistique nous fournit le moyen de les isoler. Ils sont, en effet, figurés, non sans exactitude, par le taux de la natalité, de la nuptialité, des suicides, c’est-à-dire par le nombre que l’on obtient en divisant le total moyen annuel des mariages, des naissances, des morts volontaires par celui des hommes en âge de se marier, de procréer, de se suicider\footnote{ On ne se suicide pas à tout âge, ni, à tous les âges, avec la même intensité.}. Car, comme chacun de ces chiffres comprend tous les cas particuliers indistinctement, les circonstances individuelles qui peuvent avoir quelque part dans la production du phénomène s’y neutralisent mutuellement et, par suite, ne contribuent pas à le déterminer. Ce qu’il exprime, c’est un certain état de l’âme collective.\par
Voilà ce que sont les phénomènes sociaux, débarrassés de tout élément étranger. Quant à leurs manifestations privées, elles ont bien quelque chose de social, puisqu’elles reproduisent en partie un modèle collectif ; mais chacune d’elles dépend aussi, et pour une large part, de la constitution organico-psychique de l’individu, des circonstances particulières dans lesquelles il est placé. Elles ne sont donc pas des phénomènes proprement sociologiques. Elles tiennent à la fois aux deux règnes ; on pourrait les appeler socio-psychiques. Elles intéressent le sociologue sans constituer la matière immédiate de la sociologie. On trouve de même à l’intérieur de l’organisme des phénomènes de nature mixte qu’étudient des sciences mixtes, comme la chimie biologique.\par
Mais, dira-t-on, un phénomène ne peut être collectif que s’il est commun à tous les membres de la société ou, tout au moins, à la plupart d’entre eux, partant, s’il est général. Sans doute, mais s’il est général, c’est parce qu’il est collectif (c’est-à-dire plus ou moins obligatoire), bien loin qu’il soit collectif parce qu’il est général. C’est un état du groupe, qui se répète chez les individus parce qu’il s’impose à eux. Il est dans chaque partie parce qu’il est dans le tout, loin qu’il soit dans le tout parce qu’il est dans les parties. C’est ce qui est surtout évident de ces croyances et de ces pratiques qui nous sont transmises toutes faites par les générations antérieures ; nous les recevons et les adoptons parce que, étant à la fois une œuvre collective et une œuvre séculaire, elles sont investies d’une particulière autorité que l’éducation nous a appris à reconnaître et à respecter. Or il est à noter que l’immense majorité des phénomènes sociaux nous vient par cette voie. Mais alors même que le fait social est dû, en partie, à notre collaboration directe, il n’est pas d’une autre nature. Un sentiment collectif, qui éclate dans une assemblée, n’exprime pas simplement ce qu’il y avait de commun entre tous les sentiments individuels. Il est quelque chose de tout autre, comme nous l’avons montré. Il est une résultante de la vie commune, un produit des actions et des réactions qui s’engagent entre les consciences individuelles ; et s’il retentit dans chacune d’elles, c’est en vertu de l’énergie spéciale qu’il doit précisément à son origine collective. Si tous les cœurs vibrent à l’unisson, ce n’est pas par suite d’une concordance spontanée et préétablie ; c’est qu’une même force les meut dans le même sens. Chacun est entraîné par tous.\par
Nous arrivons donc à nous représenter, d’une manière précise, le domaine de la sociologie. Il ne comprend qu’un groupe déterminé de phénomènes. Un fait social se reconnaît au pouvoir de coercition externe qu’il exerce ou est susceptible d’exercer sur les individus ; et la présence de ce pouvoir se reconnaît à son tour soit à l’existence de quelque sanction déterminée, soit à la résistance que le fait oppose à toute entreprise individuelle qui tend à lui faire violence. Cependant, on peut le définir aussi par la diffusion qu’il présente à l’intérieur du groupe, pourvu que, suivant les remarques précédentes, on ait soin d’ajouter comme seconde et essentielle caractéristique qu’il existe indépendamment des formes individuelles qu’il prend en se diffusant. Ce dernier critère est même, dans certains cas, plus facile à appliquer que le précédent. En effet, la contrainte est aisée à constater quand elle se traduit au dehors par quelque réaction directe de la société, comme c’est le cas pour le droit, la morale, les croyances, les usages, les modes même. Mais quand elle n’est qu’indirecte, comme celle qu’exerce une organisation économique, elle ne se laisse pas toujours aussi bien apercevoir. La généralité combinée avec l’objectivité peuvent alors être plus faciles à établir. D’ailleurs, cette seconde définition n’est qu’une autre forme de la première ; car si une manière de se conduire, qui existe extérieurement aux consciences individuelles, se généralise, ce ne peut être qu’en s’imposant\footnote{ On voit combien cette définition du fait social s’éloigne de celle qui sert de base à l’ingénieux système de M. Tarde. D’abord, nous devons déclarer que nos recherches ne nous ont nulle part fait constater cette influence prépondérante que M. Tarde attribue à l’imitation dans la genèse des faits collectifs. De plus, de la définition précédente, qui n’est pas une théorie mais un simple résumé des données immédiates de l’observation, il semble bien résulter que l’imitation, non seulement n’exprime pas toujours, mais même n’exprime jamais ce qu’il y a d’essentiel et de caractéristique dans le fait social. Sans doute, tout fait social est imité, il a, comme nous venons de le montrer, une tendance à se généraliser, mais c’est parce qu’il est social, c’est-à-dire obligatoire. Sa puissance d’expansion est, non la cause, mais la conséquence de son caractère sociologique. Si encore les faits sociaux étaient seuls à produire cette conséquence, l’imitation pourrait servir, sinon à les expliquer, du moins à les définir. Mais un état individuel qui fait ricochet ne laisse pas pour cela d’être individuel. De plus, on peut se demander si le mot d’imitation est bien celui qui convient pour désigner une propagation due à une influence coercitive. Sous cette unique expression, on confond des phénomènes très différents et qui auraient besoin d’être distingués.}.\par
Cependant, on pourrait se demander si cette définition est complète. En effet, les faits qui nous en ont fourni la base sont tous des {\itshape manières de faire} ; ils sont d’ordre physiologique. Or il y a aussi des {\itshape manières d’être} collectives, c’est-à-dire des faits sociaux d’ordre anatomique ou morphologique. La sociologie ne peut se désintéresser de ce qui concerne le substrat de la vie collective. Pourtant, le nombre et la nature des parties élémentaires dont est composée la société, la manière dont elles sont disposées, le degré de coalescence où elles sont parvenues, la distribution de la population sur la surface du territoire, le nombre et la nature des voies de communication, la forme des habitations, etc., ne paraissent pas, à un premier examen, pouvoir se ramener à des façons d’agir ou de sentir ou de penser.\par
Mais, tout d’abord, ces divers phénomènes présentent la même caractéristique qui nous a servi à définir les autres. Ces manières d’être s’imposent à l’individu tout comme les manières de faire dont nous avons parlé. En effet, quand on veut connaître la façon dont une société est divisée politiquement, dont ces divisions sont composées, la fusion plus ou moins complète qui existe entre elles, ce n’est pas à l’aide d’une inspection matérielle et par des observations géographiques qu’on y peut parvenir ; car ces divisions sont morales alors même qu’elles ont quelque base dans la nature physique. C’est seulement à travers le droit public qu’il est possible d’étudier cette organisation, car c’est ce droit qui la détermine, tout comme il détermine nos relations domestiques et civiques. Elle n’est donc pas moins obligatoire. Si la population se presse dans nos villes au lieu de se disperser dans les campagnes, c’est qu’il y a un courant d’opinion, une poussée collective qui impose aux individus cette concentration. Nous ne pouvons pas plus choisir la forme de nos maisons que celle de nos vêtements ; du moins, l’une est obligatoire dans la même mesure que l’autre. Les voies de communication déterminent d’une manière impérieuse le sens dans lequel se font les migrations intérieures et les échanges, et même l’intensité de ces échanges et de ces migrations, etc., etc. Par conséquent, il y aurait, tout au plus, lieu d’ajouter à la liste des phénomènes que nous avons énumérés comme présentant le signe distinctif du fait social une catégorie de plus ; et, comme cette énumération n’avait rien de rigoureusement exhaustif, l’addition ne serait pas indispensable.\par
Mais elle n’est même pas utile ; car ces manières d’être ne sont que des manières de faire consolidées. La structure politique d’une société n’est que la manière dont les différents segments qui la composent ont pris l’habitude de vivre les uns avec les autres. Si leurs rapports sont traditionnellement étroits, les segments tendent à se confondre ; à se distinguer, dans le cas contraire. Le type d’habitation qui s’impose à nous n’est que la manière dont tout le monde autour de nous et, en partie, les générations antérieures se sont accoutumées à construire les maisons. Les voies de communication ne sont que le lit que s’est creusé à lui-même, en coulant dans le même sens, le courant régulier des échanges et des migrations, etc. Sans doute, si les phénomènes d’ordre morphologique étaient les seuls à présenter cette fixité, on pourrait croire qu’ils constituent une espèce à part. Mais une règle juridique est un arrangement non moins permanent qu’un type d’architecture et, pourtant, c’est un fait physiologique. Une simple maxime morale est, assurément, plus malléable ; mais elle a des formes bien plus rigides qu’un simple usage professionnel ou qu’une mode. Il y a ainsi toute une gamme de nuances qui, sans solution de continuité, rattache les faits de structure les plus caractérisés à ces libres courants de la vie sociale qui ne sont encore pris dans aucun moule défini. C’est donc qu’il n’y a entre eux que des différences dans le degré de consolidation qu’ils présentent. Les uns et les autres ne sont que de la vie plus ou moins cristallisée. Sans doute, il peut y avoir intérêt à réserver le nom de morphologiques aux faits sociaux qui concernent le substrat social, mais à condition de ne pas perdre de vue qu’ils sont de même nature que les autres. Notre définition comprendra donc tout le défini si nous disons : {\itshape Est fait social toute manière de faire, fixée ou, non, susceptible d’exercer sur l’individu une contrainte extérieure ; ou bien encore, qui est générale dans l’étendue d’une société donnée tout en ayant une existence propre, indépendante de ses manifestations individuelles}\footnote{ Cette parenté étroite de la vie et de la structure, de l’organe et de la fonction peut être facilement établie en sociologie parce que, entre ces deux termes extrêmes, il existe toute une série d’intermédiaires immédiatement observables et qui montre le lien entre eux. La biologie n’a pas la même ressource. Mais il est permis de croire que les inductions de la première de ces sciences sur ce sujet sont applicables à l’autre et que, dans les organismes comme dans les sociétés, il n’y a entre ces deux ordres de faits que des différences de degré.}.
\chapterclose


\chapteropen
\chapter[{Chapitre II : Règles relatives à l’observation des faits sociaux}]{Chapitre II : \\
Règles relatives à l’observation des faits sociaux}\renewcommand{\leftmark}{Chapitre II : \\
Règles relatives à l’observation des faits sociaux}


\chaptercont
\noindent La première règle et la plus fondamentale est de {\itshape considérer les faits sociaux comme des choses}.\par
\section[{I}]{I}
\noindent Au moment où un ordre nouveau de phénomènes devient objet de science, ils se trouvent déjà représentés dans l’esprit, non seulement par des images sensibles, mais par des sortes de concepts grossièrement formés. Avant les premiers rudiments de la physique et de la chimie, les hommes avaient déjà sur les phénomènes physico-chimiques des notions qui dépassaient la pure perception ; telles sont, par exemple, celles que nous trouvons mêlées à toutes les religions. C’est que, en effet, la réflexion est antérieure à la science qui ne fait que s’en servir avec plus de méthode. L’homme ne peut pas vivre au milieu des choses sans s’en faire des idées d’après lesquelles il règle sa conduite. Seulement, parce que ces notions sont plus près de nous et plus à notre portée que les réalités auxquelles elles correspondent, nous tendons naturellement à les substituer à ces dernières et à en faire la matière même de nos spéculations. Au lieu d’observer les choses, de les décrire, de les comparer, nous nous contentons alors de prendre conscience de nos idées, de les analyser, de les combiner. Au lieu d’une science de réalités, nous ne faisons plus qu’une analyse idéologique. Sans doute, cette analyse n’exclut pas nécessairement toute observation. On peut faire appel aux faits pour confirmer ces notions ou les conclusions qu’on en tire. Mais les faits n’interviennent alors que secondairement, à titre d’exemples ou de preuves confirmatoires ; ils ne sont pas l’objet de la science. Celle-ci va des idées aux choses, non des choses aux idées.\par
Il est clair que cette méthode ne saurait donner de résultats objectifs. Ces notions, en effet, ou concepts, de quelque nom qu’on veuille les appeler, ne sont pas les substituts légitimes des choses. Produits de l’expérience vulgaire, ils ont, avant tout, pour objet de mettre nos actions en harmonie avec le monde qui nous entoure ; ils sont formés par la pratique et pour elle. Or une représentation peut être en état de jouer utilement ce rôle tout en étant théoriquement fausse. Copernic a, depuis plusieurs siècles, dissipé les illusions de nos sens touchant les mouvements des astres ; et pourtant, c’est encore d’après ces illusions que nous réglons couramment la distribution de notre temps. Pour qu’une idée suscite bien les mouvements que réclame la nature d’une chose, il n’est pas nécessaire qu’elle exprime fidèlement cette nature ; mais il suffit qu’elle nous fasse sentir ce que la chose a d’utile ou de désavantageux, par où elle peut nous servir, par où nous contrarier. Encore les notions ainsi formées ne présentent-elles cette justesse pratique que d’une manière approximative et seulement dans la généralité des cas. Que de fois elles sont aussi dangereuses qu’inadéquates ! Ce n’est donc pas en les élaborant, de quelque manière qu’on s’y prenne, que l’on arrivera jamais à découvrir les lois de la réalité. Elles sont, au contraire, comme un voile qui s’interpose entre les choses et nous et qui nous les masque d’autant mieux qu’on le croit plus transparent.\par
Non seulement une telle science ne peut être que tronquée, mais elle manque de matière où elle puisse s’alimenter. À peine existe-t-elle qu’elle disparaît, pour ainsi dire, et se transforme en art. En effet, ces notions sont censées contenir tout ce qu’il y a d’essentiel dans le réel, puisqu’on les confond avec le réel lui-même. Dès lors, elles semblent avoir tout ce qu’il faut pour nous mettre en état non seulement de comprendre ce qui est, mais de prescrire ce qui doit être et les moyens de l’exécuter. Car ce qui est bon, c’est ce qui est conforme à la nature des choses ; ce qui y est contraire est mauvais, et les moyens pour atteindre l’un et fuir l’autre dérivent de cette même nature. Si donc nous la tenons d’emblée, l’étude de la réalité présente n’a plus d’intérêt pratique et, comme c’est cet intérêt qui est la raison d’être de cette étude, celle-ci se trouve désormais sans but. La réflexion est ainsi incitée à se détourner de ce qui est l’objet même de la science, à savoir le présent et le passé, pour s’élancer d’un seul bond vers l’avenir. Au lieu de chercher à comprendre les faits acquis et réalisés, elle entreprend immédiatement d’en réaliser de nouveaux, plus conformes aux fins poursuivies par les hommes. Quand on croit savoir en quoi consiste l’essence de la matière, on se met aussitôt à la recherche de la pierre philosophale. Cet empiétement de l’art sur la science, qui empêche celle-ci de se développer, est d’ailleurs facilité par les circonstances mêmes qui déterminent l’éveil de la réflexion scientifique. Car, comme elle ne prend naissance que pour satisfaire à des nécessités vitales, elle se trouve tout naturellement orientée vers la pratique. Les besoins qu’elle est appelée à soulager sont toujours pressés et, par suite, la pressent d’aboutir ; ils réclament, non des explications, mais des remèdes.\par
Cette manière de procéder est si conforme à la pente naturelle de notre esprit qu’on la retrouve même à l’origine des sciences physiques. C’est elle qui différencie l’alchimie de la chimie, comme l’astrologie de l’astronomie. C’est par elle que Bacon caractérise la méthode que suivaient les savants de son temps et qu’il combat. Les notions dont nous venons de parler, ce sont ces {\itshape notiones vulgares} ou {\itshape praenotiones}\footnote{{\itshape Novum organum}, I, 26.} qu’il signale à la base de toutes les sciences\footnote{{\itshape Ibid.}, I, 11.} où elles prennent la place des faits\footnote{{\itshape Ibid.}, I, 36.}. Ce sont ces {\itshape idola}, sortes de fantômes qui nous défigurent le véritable aspect des choses et que nous prenons pourtant pour les choses mêmes. Et c’est parce que ce milieu imaginaire n’offre à l’esprit aucune résistance que celui-ci, ne se sentant contenu par rien, s’abandonne à des ambitions sans bornes et croit possible de construire ou, plutôt, de reconstruire le monde par ses seules forces et au gré de ses désirs.\par
S’il en a été ainsi des sciences naturelles, à plus forte raison en devait-il être de même pour la sociologie. Les hommes n’ont pas attendu l’avènement de la science sociale pour se faire des idées sur le droit, la morale, la famille, l’État, la société même ; car ils ne pouvaient s’en passer pour vivre. Or, c’est surtout en sociologie que ces prénotions, pour reprendre l’expression de Bacon, sont en état de dominer les esprits et de se substituer aux choses. En effet, les choses sociales ne se réalisent que par les hommes ; elles sont un produit de l’activité humaine. Elles ne paraissent donc pas être autre chose que la mise en œuvre d’idées, innées ou non, que nous portons en nous, que leur application aux diverses circonstances qui accompagnent les relations des hommes entre eux. L’organisation de la famille, du contrat, de la répression, de l’État, de la société apparaissent ainsi comme un simple développement des idées que nous avons sur la société, l’État, la justice, etc. Par conséquent, ces faits et leurs analogues semblent n’avoir de réalité que dans et par les idées qui en sont le germe et qui deviennent, dès lors, la matière propre de la sociologie.\par
Ce qui achève d’accréditer cette manière de voir, c’est que, le détail de la vie sociale débordant de tous les côtés la conscience, celle-ci n’en a pas une perception assez forte pour en sentir la réalité. N’ayant pas en nous d’attaches assez solides ni assez prochaines, tout cela nous fait assez facilement l’effet de ne tenir à rien et de flotter dans le vide, matière à demi irréelle et indéfiniment plastique. Voilà pourquoi tant de penseurs n’ont vu dans les arrangements sociaux que des combinaisons artificielles et plus ou moins arbitraires. Mais si le détail, si les formes concrètes et particulières nous échappent, du moins nous nous représentons les aspects les plus généraux de l’existence collective en gros et par à peu près, et ce sont précisément ces représentations schématiques et sommaires qui constituent ces prénotions dont nous nous servons pour les usages courants de la vie. Nous ne pouvons donc songer à mettre en doute leur existence, puisque nous la percevons en même temps que la nôtre. Non seulement elles sont en nous, mais, comme elles sont un produit d’expériences répétées, elles tiennent de la répétition, et de l’habitude qui en résulte, une sorte d’ascendant et d’autorité. Nous les sentons nous résister quand nous cherchons à nous en affranchir. Or nous ne pouvons pas ne pas regarder comme réel ce qui s’oppose à nous. Tout contribue donc à nous y faire voir la vraie réalité sociale.\par
Et en effet, jusqu’à présent, la sociologie a plus ou moins exclusivement traité non de choses, mais de concepts. Comte, il est vrai, a proclamé que les phénomènes sociaux sont des faits naturels, soumis à des lois naturelles. Par là, il a implicitement reconnu leur caractère de choses ; car il n’y a que des choses dans la nature. Mais quand, sortant de ces généralités philosophiques, il tente d’appliquer son principe et d’en faire sortir la science qui y était contenue, ce sont des idées qu’il prend pour objets d’études. En effet, ce qui fait la matière principale de sa sociologie, c’est le progrès de l’humanité dans le temps. Il part de cette idée qu’il y a une évolution continue du genre humain qui consiste dans une réalisation toujours plus complète de la nature humaine et le problème qu’il traite est de retrouver l’ordre de cette évolution. Or, à supposer que cette évolution existe, la réalité n’en peut être établie que la science une fois faite ; on ne peut donc en faire l’objet même de la recherche que si on la pose comme une conception de l’esprit, non comme une chose. Et en effet, il s’agit si bien d’une représentation toute subjective que, en fait, ce progrès de l’humanité n’existe pas. Ce qui existe, ce qui seul est donné à l’observation, ce sont des sociétés particulières qui naissent, se développent, meurent indépendamment les unes des autres. Si encore les plus récentes continuaient celles qui les ont précédées, chaque type supérieur pourrait être considéré comme la simple répétition du type immédiatement inférieur avec quelque chose en plus ; on pourrait donc les mettre tous bout à bout, pour ainsi dire, en confondant ceux qui se trouvent au même degré de développement, et la série ainsi formée pourrait être regardée comme représentative de l’humanité. Mais les faits ne se présentent pas avec cette extrême simplicité. Un peuple qui en remplace un autre n’est pas simplement un prolongement de ce dernier avec quelques caractères nouveaux ; il est autre, il a des propriétés en plus, d’autres en moins ; il constitue une individualité nouvelle et toutes ces individualités distinctes, étant hétérogènes, ne peuvent pas se fondre en une même série continue, ni surtout en une série unique. Car la suite des sociétés ne saurait être figurée par une ligne géométrique ; elle ressemble plutôt à un arbre dont les rameaux se dirigent dans des sens divergents. En somme, Comte a pris pour le développement historique la notion qu’il en avait et qui ne diffère pas beaucoup de celle que s’en fait le vulgaire. Vue de loin, en effet, l’histoire prend assez bien cet aspect sériaire et simple. On n’aperçoit que des individus qui se succèdent les uns aux autres et marchent tous dans une même direction parce qu’ils ont une même nature. Puisque, d’ailleurs, on ne conçoit pas que l’évolution sociale puisse être autre chose que le développement de quelque idée humaine, il paraît tout naturel de la définir par l’idée que s’en font les hommes. Or, en procédant ainsi, non seulement on reste dans l’idéologie, mais on donne comme objet à la sociologie un concept qui n’a rien de proprement sociologique.\par
Ce concept, M. Spencer l’écarte, mais c’est pour le remplacer par un autre qui n’est pas formé d’une autre façon. Il fait des sociétés, et non de l’humanité, l’objet de la science ; seulement, il donne aussitôt des premières une définition qui fait évanouir la chose dont il parle pour mettre à la place la prénotion qu’il en a. Il pose, en effet, comme une proposition évidente qu’\emph{« une société n’existe que quand, à la juxtaposition, s’ajoute la coopération »}, que c’est par là seulement que l’union des individus devient une société proprement dite\footnote{\emph{Sociol.} Tr. fr., III, 331, 332.}. Puis, partant de ce principe que la coopération est l’essence de la vie sociale, il distingue les sociétés en deux classes suivant la nature de la coopération qui y domine. \emph{« Il y a, dit-il, une coopération spontanée qui s’effectue sans préméditation durant la poursuite de fins d’un caractère privé ; il y a aussi une coopération consciemment instituée qui suppose des fins d’intérêt public nettement reconnues\footnote{\emph{Sociol.}, III, 332.}. »} Aux premières, il donne le nom de sociétés industrielles ; aux secondes, celui de militaires, et on peut dire de cette distinction qu’elle est l’idée mère de sa sociologie.\par
Mais cette définition initiale énonce comme une chose ce qui n’est qu’une vue de l’esprit. Elle se présente, en effet, comme l’expression d’un fait immédiatement visible et que l’observation suffit à constater, puisqu’elle est formulée dès le début de la science comme un axiome. Et cependant, il est impossible de savoir par une simple inspection si réellement la coopération est le tout de la vie sociale. Une telle affirmation n’est scientifiquement légitime que si l’on a commencé par passer en revue toutes les manifestations de l’existence collective et si l’on a fait voir qu’elles sont toutes des formes diverses de la coopération. C’est donc encore une certaine manière de concevoir la réalité sociale qui se substitue à cette réalité\footnote{  Conception, d’ailleurs, controversable. (V. \emph{Division du travail social}, II, 2, § 4.)}. Ce qui est ainsi défini, ce n’est pas la société, mais l’idée que s’en fait M. Spencer. Et s’il n’éprouve aucun scrupule à procéder ainsi, c’est que, pour lui aussi, la société n’est et ne peut être que la réalisation d’une idée, à savoir de cette idée même de coopération par laquelle il la définit\footnote{\emph{« La coopération ne saurait donc exister sans société, et c’est le but pour lequel une société existe. »} (\emph{Principes de Sociol.}, III, 332.)}. Il serait aisé de montrer que, dans chacun des problèmes particuliers qu’il aborde, sa méthode reste la même. Aussi, quoiqu’il affecte de procéder empiriquement, comme les faits accumulés dans sa sociologie sont employés à illustrer des analyses de notions plutôt qu’à décrire et à expliquer des choses, ils semblent bien n’être là que pour faire figure d’arguments. En réalité, tout ce qu’il y a d’essentiel dans sa doctrine peut être immédiatement déduit de sa définition de la société et des différentes formes de coopération. Car si nous n’avons le choix qu’entre une coopération tyranniquement imposée et une coopération libre et spontanée, c’est évidemment cette dernière qui est l’idéal vers lequel l’humanité tend et doit tendre.\par
Ce n’est pas seulement à la base de la science que se rencontrent ces notions vulgaires, mais on les retrouve à chaque instant dans la trame des raisonnements. Dans l’état actuel de nos connaissances, nous ne savons pas avec certitude ce que c’est que l’État, la souveraineté, la liberté politique, la démocratie, le socialisme, le communisme, etc., la méthode voudrait donc que l’on s’interdît tout usage de ces concepts, tant qu’ils ne sont pas scientifiquement constitués. Et cependant les mots qui les expriment reviennent sans cesse dans les discussions des sociologues. On les emploie couramment et avec assurance comme s’ils correspondaient à des choses bien connues et définies, alors qu’ils ne réveillent en nous que des notions confuses, mélanges indistincts d’impressions vagues, de préjugés et de passions. Nous nous moquons aujourd’hui des singuliers raisonnements que les médecins du moyen âge construisaient avec les notions du chaud, du froid, de l’humide, du sec, etc., et nous ne nous apercevons pas que nous continuons à appliquer cette même méthode à l’ordre de phénomènes qui le comporte moins que tout autre, à cause de son extrême complexité.\par
Dans les branches spéciales de la sociologie, ce caractère idéologique est encore plus accusé.\par
C’est surtout le cas pour la morale. On peut dire, en effet, qu’il n’y a pas un seul système où elle ne soit représentée comme le simple développement d’une idée initiale qui la contiendrait tout entière en puissance. Cette idée, les uns croient que l’homme la trouve toute faite en lui dès sa naissance ; d’autres, au contraire, qu’elle se forme plus ou moins lentement au cours de l’histoire. Mais, pour les uns comme pour les autres, pour les empiristes comme pour les rationalistes, elle est tout ce qu’il y a de vraiment réel en morale. Pour ce qui est du détail des règles juridiques et morales, elles n’auraient, pour ainsi dire, pas d’existence par elles-mêmes, mais ne seraient que cette notion fondamentale appliquée aux circonstances particulières de la vie et diversifiée suivant les cas. Dès lors, l’objet de la morale ne saurait être ce système de préceptes sans réalité, mais l’idée de laquelle ils découlent et dont ils ne sont, que des applications variées. Aussi toutes les questions que se pose d’ordinaire l’éthique se rapportent-elles, non à des choses, mais à des idées ; ce qu’il s’agit de savoir, c’est en quoi consiste l’idée du droit, l’idée de la morale, non quelle est la nature de la morale et du droit pris en eux-mêmes. Les moralistes ne sont pas encore parvenus à cette conception très simple que, comme notre représentation des choses sensibles vient de ces choses mêmes et les exprime plus ou moins exactement, notre représentation de la morale, vient du spectacle même des règles qui fonctionnent sous nos yeux et les figure schématiquement ; que, par conséquent, ce sont ces règles et non la vue sommaire que nous en avons, qui forment la matière de la science, de même que la physique a pour objet les corps tels qu’ils existent, non l’idée que s’en fait le vulgaire. Il en résulte qu’on prend pour base de la morale ce qui n’en est que le sommet, à savoir la manière dont elle se prolonge dans les consciences individuelles et y retentit. Et ce n’est pas seulement dans les problèmes les plus généraux de la science que cette méthode est suivie ; elle reste la même dans les questions spéciales. Des idées essentielles qu’il étudie au début, le moraliste passe aux idées secondaires de famille, de patrie, de responsabilité, de charité, de justice ; mais c’est toujours à des idées que s’applique sa réflexion.\par
Il n’en est pas autrement de l’économie politique. Elle a pour objet, dit Stuart Mill, les faits sociaux qui se produisent principalement ou exclusivement en vue de l’acquisition des richesses\footnote{\emph{Système de Logique}, Ill, p. 496.}. Mais pour que les faits ainsi définis pussent être assignés, en tant que choses, à l’observation du savant, il faudrait tout au moins que l’on pût indiquer à quel signe il est possible de reconnaître ceux qui satisfont à cette condition. Or, au début de la science, on n’est même pas en droit d’affirmer qu’il en existe, bien loin qu’on puisse savoir quels ils sont. Dans tout ordre de recherches, en effet, c’est seulement quand l’explication des faits est assez avancée qu’il est possible d’établir qu’ils ont un but et quel il est. Il n’est pas de problème plus complexe ni moins susceptible d’être tranché d’emblée. Rien donc ne nous assure par avance qu’il y ait une sphère de l’activité sociale où le désir de la richesse joue réellement ce rôle prépondérant. Par conséquent, la matière de l’économie politique, ainsi comprise, est faite non de réalités qui peuvent être montrées du doigt, mais de simples possibles, de pures conceptions de l’esprit ; à savoir, des faits que l’économiste conçoit comme se rapportant à la fin considérée, et tels qu’il les conçoit. Entreprend-il, par exemple, d’étudier ce qu’il appelle la production ? D’emblée, il croit pouvoir énumérer les principaux agents à l’aide desquels elle a lieu et les passer en revue. C’est donc qu’il n’a pas reconnu leur existence en observant de quelles conditions dépendait la chose qu’il étudie ; car alors il eût commencé par exposer les expériences d’où il a tiré cette conclusion. Si, dès le début de la recherche et en quelques mots, il procède à cette classification, c’est qu’il l’a obtenue par une simple analyse logique. Il part de l’idée de production ; en la décomposant, il trouve qu’elle implique logiquement celles de forces naturelles, de travail, d’instrument ou de capital et il traite ensuite de la même manière ces idées dérivées\footnote{ Ce caractère ressort des expressions mêmes employées par les économistes. Il est sans cesse question d’idées, de l’idée d’utile, de l’idée d’épargne, de placement, de dépense. (V. Gide, \emph{Principes d’économie politique}, liv. III, ch. I, § 1 ; ch. II, § 1 ; ch. III, § 1.)}.\par
La plus fondamentale de toutes les théories économiques, celle de la valeur, est manifestement construite d’après cette même méthode. Si la valeur y était étudiée comme une réalité doit l’être, on verrait d’abord l’économiste indiquer à quoi l’on peut reconnaître la chose appelée de ce nom, puis en classer les espèces, chercher par des inductions méthodiques en fonction de quelles causes elles varient, comparer enfin ces divers résultats pour en dégager une formule générale. La théorie ne pourrait donc venir que quand la science a été poussée assez loin. Au lieu de cela, on la rencontre dès le début. C’est que, pour la faire, l’économiste se contente de se recueillir, de prendre conscience de l’idée qu’il se fait de la valeur, c’est-à-dire d’un objet susceptible de s’échanger ; il trouve qu’elle implique l’idée de l’utile, celle du rare, etc., et c’est avec ces produits de son analyse qu’il construit sa définition. Sans doute il la confirme par quelques exemples. Mais quand on songe aux faits innombrables dont une pareille théorie doit rendre compte, comment accorder la moindre valeur démonstrative aux faits, nécessairement très rares, qui sont ainsi cités au hasard de la suggestion ?\par
Aussi, en économie politique comme en morale, la part de l’investigation scientifique est-elle très restreinte ; celle de l’art, prépondérante. En morale, la partie théorique est réduite à quelques discussions sur l’idée du devoir, du bien et du droit. Encore ces spéculations abstraites ne constituent-elles pas une science, à parler exactement, puisqu’elles ont pour objet de déterminer non ce qui est, en fait, la règle suprême de la moralité, mais ce qu’elle doit être. De même, ce qui tient le plus de place dans les recherches des économistes, c’est la question de savoir, par exemple, si la société {\itshape doit être} organisée d’après les conceptions des individualistes ou d’après celles des socialistes ; {\itshape s’il est meilleur} que l’État intervienne dans les rapports industriels et commerciaux ou les abandonne entièrement à l’initiative privée ; si le système monétaire {\itshape doit être} le monométallisme ou le bimétallisme, etc., etc. Les lois proprement dites y sont peu nombreuses ; même celles qu’on a l’habitude d’appeler ainsi ne méritent généralement pas cette qualification, mais ne sont que des maximes d’action, des préceptes pratiques déguisés. Voilà, par exemple la fameuse loi de l’offre et de la demande. Elle n’a jamais été établie inductivement, comme expression de la réalité économique. Jamais aucune expérience, aucune comparaison méthodique n’a été instituée pour établir que, {\itshape en fait}, c’est suivant cette loi que procèdent les relations économiques. Tout ce qu’on a pu faire et tout ce qu’on a fait, c’est de démontrer dialectiquement que les individus doivent procéder ainsi, s’ils entendent bien leurs intérêts ; c’est que toute autre manière de faire leur serait nuisible et impliquerait de la part de ceux qui s’y prêteraient une véritable aberration logique. Il est logique que les industries les plus productives soient les plus recherchées ; que les détenteurs des produits les plus demandés et les plus rares les vendent au plus haut prix. Mais cette nécessité toute logique ne ressemble en rien à celle que présentent les vraies lois de la nature. Celles-ci expriment les rapports suivant lesquels les faits s’enchaînent réellement, non la manière dont il est bon qu’ils s’enchaînent.\par
Ce que nous disons de cette loi peut être répété de toutes celles que l’école économique orthodoxe qualifie de naturelles et qui, d’ailleurs, ne sont guère que des cas particuliers de la précédente. Elles sont naturelles, si l’on veut, en ce sens qu’elles énoncent les moyens qu’il est ou qu’il peut paraître naturel d’employer pour atteindre telle fin supposée ; mais elles ne doivent pas être appelées de ce nom, si, par loi naturelle, on entend toute manière d’être de la nature, inductivement constatée. Elles ne sont en somme que des conseils de sagesse pratique et, si l’on a pu, plus ou moins spécieusement, les présenter comme l’expression même de la réalité, c’est que, à tort ou à raison, on a cru pouvoir supposer que ces conseils étaient effectivement suivis par la généralité des hommes et dans la généralité des cas.\par
\par
Et cependant les phénomènes sociaux sont des choses et doivent être traités comme des choses. Pour démontrer cette proposition, il n’est pas nécessaire de philosopher sur leur nature, de discuter les analogies qu’ils présentent avec les phénomènes des règnes inférieurs. Il suffit de constater qu’ils sont l’unique {\itshape datum} offert au sociologue. Est chose, en effet, tout ce qui est donné, tout ce qui s’offre ou, plutôt, s’impose à l’observation. Traiter des phénomènes comme des choses, c’est les traiter en qualité de {\itshape data} qui constituent le point de départ de la science. Les phénomènes sociaux présentent incontestablement ce caractère. Ce qui nous est donné, ce n’est pas l’idée que les hommes se font de la valeur, car elle est inaccessible : ce sont les valeurs qui s’échangent réellement au cours des relations économiques. Ce n’est pas telle ou telle conception de l’idéal moral ; c’est l’ensemble des règles qui déterminent effectivement la conduite. Ce n’est pas l’idée de l’utile ou de la richesse ; c’est tout le détail de l’organisation économique. Il est possible que la vie sociale ne soit que le développement de certaines notions ; mais, à supposer que cela soit, ces notions ne sont pas données immédiatement. On ne peut donc les atteindre directement, mais seulement à travers la réalité phénoménale qui les exprime. Nous ne savons pas {\itshape a priori} quelles idées sont à l’origine des divers courants entre lesquels se partage la vie sociale ni s’il y en a ; c’est seulement après les avoir remontés jusqu’à leurs sources que nous saurons d’où ils proviennent.\par
Il nous faut donc considérer les phénomènes sociaux en eux-mêmes, détachés des sujets conscients qui se les représentent ; il faut les étudier du dehors comme des choses extérieures ; car c’est en cette qualité qu’ils se présentent à nous. Si cette extériorité n’est qu’apparente, l’illusion se dissipera à mesure que la science avancera et l’on verra, pour ainsi dire, le dehors rentrer dans le dedans. Mais la solution ne peut être préjugée et, alors même que, finalement, ils n’auraient pas tous les caractères intrinsèques de la chose, on doit d’abord les traiter comme s’ils les avaient. Cette règle s’applique donc à la réalité sociale tout entière, sans qu’il y ait lieu de faire aucune exception. Même les phénomènes qui paraissent le plus consister en arrangements artificiels doivent être considérés de ce point de vue. {\itshape Le caractère conventionnel d’une pratique ou d’une institution ne doit jamais être présumé.} Si, d’ailleurs, il nous est permis d’invoquer notre expérience personnelle, nous croyons pouvoir assurer que, en procédant de cette manière, on aura souvent la satisfaction de voir les faits en apparence les plus arbitraires présenter ensuite à une observation plus attentive des caractères de constance et de régularité, symptômes de leur objectivité.\par
Du reste, et d’une manière générale, ce qui a été dit précédemment sur les caractères distinctifs du fait social, suffit à nous rassurer sur la nature de cette objectivité et à prouver qu’elle n’est pas illusoire. En effet, on reconnaît principalement une chose à ce signe qu’elle ne peut pas être modifiée par un simple décret de la volonté. Ce n’est pas qu’elle soit réfractaire à toute modification. Mais, pour y produire un changement, il ne suffit pas de le vouloir, il faut encore un effort plus ou moins laborieux, dû à la résistance qu’elle nous oppose et qui, d’ailleurs, ne peut pas toujours être vaincue. Or nous avons vu que les faits sociaux ont cette propriété. Bien loin qu’ils soient un produit de notre volonté, ils la déterminent du dehors ; ils consistent comme en des moules en lesquels nous sommes nécessités à couler nos actions. Souvent même, cette nécessité est telle que nous ne pouvons pas y échapper. Mais alors même que nous parvenons à en triompher, l’opposition que nous rencontrons suffit à nous avertir que nous sommes en présence de quelque chose qui ne dépend pas de nous. Donc, en considérant les phénomènes sociaux comme des choses, nous ne ferons que nous conformer à leur nature.\par
En définitive, la réforme qu’il s’agit d’introduire en sociologie est de tous points identique à celle qui a transformé la psychologie dans ces trente dernières années. De même que Comte et M. Spencer déclarent que les faits sociaux sont des faits de nature, sans cependant les traiter comme des choses, les différentes écoles empiriques avaient, depuis longtemps, reconnu le caractère naturel des phénomènes psychologiques tout en continuant à leur appliquer une méthode purement idéologique. En effet, les empiristes, non moins que leurs adversaires, procédaient exclusivement par introspection. Or, les faits que l’on n’observe que sur soi-même sont trop rares, trop fuyants, trop malléables pour pouvoir s’imposer aux notions correspondantes que l’habitude a fixées en nous et leur faire la loi. Quand donc ces dernières ne sont pas soumises à un autre contrôle, rien ne leur fait contrepoids ; par suite, elles prennent la place des faits et constituent la matière de la science. Aussi ni Locke, ni Condillac n’ont-ils considéré les phénomènes psychiques objectivement. Ce n’est pas la sensation qu’ils étudient, mais une certaine idée de la sensation. C’est pourquoi, quoique, à de certains égards, ils aient préparé l’avènement de la psychologie scientifique, celle-ci n’a vraiment pris naissance que beaucoup plus tard, quand on fut enfin parvenu à cette conception que les états de conscience peuvent et doivent être considérés du dehors, et non du point de vue de la conscience qui les éprouve. Telle est la grande révolution qui s’est accomplie en ce genre d’études. Tous les procédés particuliers, toutes les méthodes nouvelles dont on a enrichi cette science ne sont que des moyens divers pour réaliser plus complètement cette idée fondamentale. C’est ce même progrès qui reste à faire à la sociologie. Il faut qu’elle passe du stade subjectif, qu’elle n’a encore guère dépassé, à la phase objective.\par
Ce passage y est, d’ailleurs, moins difficile à effectuer qu’en psychologie. En effet, les faits psychiques sont naturellement donnés comme des états du sujet, dont ils ne paraissent même pas séparables. Intérieurs par définition, il semble qu’on ne puisse les traiter comme extérieurs qu’en faisant violence à leur nature Il faut non seulement un effort d’abstraction, mais tout un ensemble de procédés et d’artifices pour arriver à les considérer de ce biais. Au contraire, les faits sociaux ont bien plus naturellement et plus immédiatement tous les caractères de la chose. Le droit existe dans les codes, les mouvements de la vie quotidienne s’inscrivent dans les chiffres de la statistique, dans les monuments de l’histoire, les modes dans les costumes, les goûts dans les œuvres d’art. Ils tendent en vertu de leur nature même à se constituer en dehors des consciences individuelles, puisqu’ils les dominent. Pour les voir sous leur aspect de choses, il n’est donc pas nécessaire de les torturer avec ingéniosité. De ce point de vue, la sociologie a sur la psychologie un sérieux avantage qui n’a pas été aperçu jusqu’ici et qui doit en hâter le développement. Les faits sont peut-être plus difficiles à interpréter parce qu’ils sont plus complexes, mais ils sont plus faciles à atteindre. La psychologie, au contraire, n’a pas seulement du mal à les élaborer, mais aussi à les saisir. Par conséquent, il est permis de croire que, du jour où ce principe de la méthode sociologique sera unanimement reconnu et pratiqué, on verra la sociologie progresser avec une rapidité que la lenteur actuelle de son développement ne ferait guère supposer, et regagner même l’avance que la psychologie doit uniquement à son antériorité historique\footnote{ Il est vrai que la complexité plus grande des faits sociaux en rend la science plus malaisée. Mais, par compensation, précisément parce que la sociologie est la dernière venue, elle est en état de profiter des progrès réalisés par les sciences inférieures et de s’instruire à leur école. Cette utilisation des expériences faites ne peut manquer d’en accélérer le développement.}.
\section[{II}]{II}
\noindent Mais l’expérience de nos devanciers nous a montré que, pour assurer la réalisation pratique de la vérité qui vient d’être établie, il ne suffit pas d’en donner une démonstration théorique ni même de s’en pénétrer. L’esprit est si naturellement enclin à la méconnaître qu’on retombera inévitablement dans les anciens errements si l’on ne se soumet à une discipline rigoureuse, dont nous allons formuler les règles principales, corollaires de la précédente.\par
\par
1° Le premier de ces corollaires est que : {\itshape Il faut écarter systématiquement toutes les prénotions.} Une démonstration spéciale de cette règle n’est pas nécessaire ; elle résulte de tout ce que nous avons dit précédemment. Elle est, d’ailleurs, la base de toute méthode scientifique. Le doute méthodique de Descartes n’en est, au fond, qu’une application. Si, au moment où il va fonder la science, Descartes se fait une loi de mettre en doute toutes les idées qu’il a reçues antérieurement, c’est qu’il ne veut employer que des concepts scientifiquement élaborés, c’est-à-dire construits d’après la méthode qu’il institue ; tous ceux qu’il tient d’une autre origine doivent donc être rejetés, au moins provisoirement. Nous avons déjà vu que la théorie des Idoles, chez Bacon, n’a pas d’autre sens. Les deux grandes doctrines que l’on a si souvent opposées l’une à l’autre concordent sur ce point essentiel. Il faut donc que le sociologue, soit au moment où il détermine l’objet de ses recherches, soit dans le cours de ses démonstrations, s’interdise résolument l’emploi de ces concepts qui se sont formés en dehors de la science et pour des besoins qui n’ont rien de scientifique. Il faut qu’il s’affranchisse de ces fausses évidences qui dominent l’esprit du vulgaire, qu’il secoue, une fois pour toutes, le joug de ces catégories empiriques qu’une longue accoutumance finit souvent par rendre tyranniques. Tout au moins, si, parfois, la nécessité l’oblige à y recourir, qu’il le fasse en ayant conscience de leur peu de valeur, afin de ne pas les appeler à jouer dans la doctrine un rôle dont elles ne sont pas dignes.\par
Ce qui rend cet affranchissement particulièrement difficile en sociologie, c’est que le sentiment se met souvent de la partie. Nous nous passionnons, en effet, pour nos croyances politiques et religieuses, pour nos pratiques morales bien autrement que pour les choses du monde physique ; par suite, ce caractère passionnel se communique à la manière dont nous concevons et dont nous nous expliquons les premières. Les idées que nous nous en faisons nous tiennent à cœur, tout comme leurs objets, et prennent ainsi une telle autorité qu’elles ne supportent pas la contradiction. Toute opinion qui les gêne est traitée en ennemie. Une proposition n’est-elle pas d’accord avec l’idée qu’on se fait du patriotisme, ou de la dignité individuelle, par exemple ? Elle est niée, quelles que soient les preuves sur lesquelles elle repose. On ne peut pas admettre qu’elle soit vraie ; on lui oppose une fin de non-recevoir, et la passion, pour se justifier, n’a pas de peine à suggérer des raisons qu’on trouve facilement décisives. Ces notions peuvent même avoir un tel prestige qu’elles ne tolèrent même pas l’examen scientifique. Le seul fait de les soumettre, ainsi que les phénomènes qu’elles expriment, à une froide et sèche analyse révolte certains esprits. Quiconque entreprend d’étudier la morale du dehors et comme une réalité extérieure, paraît à ces délicats dénué de sens moral, comme le vivisectionniste semble au vulgaire dénué de la sensibilité commune. Bien loin d’admettre que ces sentiments relèvent de la science, c’est à eux que l’on croit devoir s’adresser pour faire la science des choses auxquelles ils se rapportent. \emph{« Malheur, écrit un éloquent historien des religions, malheur au savant qui aborde les choses de Dieu sans avoir au fond de sa conscience, dans l’arrière-couche indestructible de son être, là où dort l’âme des ancêtres, un sanctuaire inconnu d’où s’élève par instants un parfum d’encens, une ligne de psaume, un cri douloureux ou triomphal qu’enfant il a jeté vers le ciel à la suite de ses frères et qui le remet en communion soudaine avec les prophètes d’autrefois\footnote{ J. Darmesteter, \emph{Les prophètes d’Israël}, p. 9.} ! »}\par
On ne saurait s’élever avec trop de force contre cette doctrine mystique qui — comme tout mysticisme, d’ailleurs — n’est, au fond, qu’un empirisme déguisé, négateur de toute science. Les sentiments qui ont pour objets les choses sociales n’ont pas de privilège sur les autres, car ils n’ont pas une autre origine. Ils se sont, eux aussi, formés historiquement ; ils sont un produit de l’expérience humaine, mais d’une expérience confuse et inorganisée. Ils ne sont pas dus à je ne sais quelle anticipation transcendantale de la réalité, mais ils sont la résultante de toute sorte d’impressions et d’émotions accumulées sans ordre, au hasard des circonstances, sans interprétation méthodique. Bien loin qu’ils nous apportent des clartés supérieures aux clartés rationnelles, ils sont faits exclusivement d’états forts, il est vrai, mais troubles. Leur accorder une pareille prépondérance, c’est donner aux facultés inférieures de l’intelligence la suprématie sur les plus élevées, c’est se condamner à une logomachie plus ou moins oratoire. Une science ainsi faite ne peut satisfaire que les esprits qui aiment mieux penser avec leur sensibilité qu’avec leur entendement, qui préfèrent les synthèses immédiates et confuses de la sensation aux analyses patientes et lumineuses de la raison. Le sentiment est objet de science, non le critère de la vérité scientifique. Au reste, il n’est pas de science qui, à ses débuts, n’ait rencontré des résistances analogues. Il fut un temps où les sentiments relatifs aux choses du monde physique, ayant eux-mêmes un caractère religieux ou moral, s’opposaient avec non moins de force à l’établissement des sciences physiques. On peut donc croire que, pourchassé de science en science, ce préjugé finira par disparaître de la sociologie elle-même, sa dernière retraite, pour laisser le terrain libre au savant.\par
\par
2° Mais la règle précédente est toute négative. Elle apprend au sociologue à échapper à l’empire des notions vulgaires, pour tourner son attention vers les faits ; mais elle ne dit pas la manière dont il doit se saisir de ces derniers pour en faire une étude objective.\par
Toute investigation scientifique porte sur un groupe déterminé de phénomènes qui répondent à une même définition. La première démarche du sociologue doit donc être de définir les choses dont il traite, afin que l’on sache et qu’il sache bien de quoi il est question. C’est la première et la plus indispensable condition de toute preuve et de toute vérification ; une théorie, en effet, ne peut être contrôlée que si l’on sait reconnaître les faits dont elle doit rendre compte. De plus, puisque c’est par cette définition initiale qu’est constitué l’objet même de la science, celui-ci sera une chose ou non, suivant la manière dont cette définition sera faite.\par
Pour qu’elle soit objective, il faut évidemment qu’elle exprime les phénomènes en fonction, non d’une idée de l’esprit, mais de propriétés qui leur sont inhérentes. Il faut qu’elle les caractérise par un élément intégrant de leur nature, non par leur conformité à une notion plus ou moins idéale. Or, au moment où la recherche va seulement commencer, alors que les faits n’ont encore été soumis à aucune élaboration, les seuls de leurs caractères qui puissent être atteints sont ceux qui se trouvent assez extérieurs pour être immédiatement visibles. Ceux qui sont situés plus profondément sont, sans doute, plus essentiels ; leur valeur explicative est plus haute, mais ils sont inconnus à cette phase de la science et ne peuvent être anticipés que si l’on substitue à la réalité quelque conception de l’esprit. C’est donc parmi les premiers que doit être cherchée la matière de cette définition fondamentale. D’autre part, il est clair que cette définition devra comprendre, sans exception ni distinction, tous les phénomènes qui présentent également ces mêmes caractères ; car nous n’avons aucune raison ni aucun moyen de choisir entre eux. Ces propriétés sont alors tout ce que nous savons du réel ; par conséquent, elles doivent déterminer souverainement la manière dont les faits doivent être groupés. Nous ne possédons aucun autre critère qui puisse, même partiellement, suspendre les effets du précédent. D’où la règle suivante : {\itshape Ne jamais prendre pour objet de recherches qu’un groupe de phénomènes préalablement définis par certains caractères extérieurs qui leur sont communs et comprendre dans la même recherche tous ceux qui répondent à cette définition.} Par exemple, nous constatons l’existence d’un certain nombre d’actes qui présentent tous ce caractère extérieur que, une fois accomplis, ils déterminent de la part de la société cette réaction particulière qu’on nomme la peine. Nous en faisons un groupe {\itshape sui generis}, auquel nous imposons une rubrique commune ; nous appelons crime tout acte puni et nous faisons du crime ainsi défini l’objet d’une science spéciale, la criminologie. De même, nous observons, à l’intérieur de toutes les sociétés connues, l’existence d’une société partielle, reconnaissable à ce signe extérieur qu’elle est formée d’individus consanguins, pour la plupart, les uns des autres et qui sont unis entre eux par des liens juridiques. Nous faisons des faits qui s’y rapportent un groupe particulier, auquel nous donnons un nom particulier ; ce sont les phénomènes de la vie domestique. Nous appelons famille tout agrégat de ce genre et nous faisons de la famille ainsi définie l’objet d’une investigation spéciale qui n’a pas encore reçu de dénomination déterminée dans la terminologie sociologique. Quand, plus tard, on passera de la famille en général aux différents types familiaux, on appliquera la même règle. Quand on abordera, par exemple, l’étude du clan, ou de la famille maternelle, ou de la famille patriarcale, on commencera par les définir et d’après la même méthode. L’objet de chaque problème, qu’il soit général ou particulier, doit être constitué suivant le même principe.\par
En procédant de cette manière, le sociologue, dès sa première démarche, prend immédiatement pied dans la réalité. En effet, la façon dont les faits sont ainsi classés ne dépend pas de lui, de la tournure particulière de son esprit, mais de la nature des choses. Le signe qui les fait ranger dans telle ou telle catégorie peut être montré à tout le monde, reconnu de tout le monde et les affirmations d’un observateur peuvent être contrôlées par les autres. Il est vrai que la notion ainsi constituée ne cadre pas toujours ou même ne cadre généralement pas avec la notion commune. Par exemple, il est évident que, pour le sens commun, les faits de libre pensée ou les manquements à l’étiquette, si régulièrement et si sévèrement punis dans une multitude de sociétés, ne sont pas regardés comme des crimes même par rapport à ces sociétés. De même, un clan n’est pas une famille, dans l’acception usuelle du mot. Mais il n’importe ; car il ne s’agit pas simplement de découvrir un moyen qui nous permette de retrouver assez sûrement les faits auxquels s’appliquent les mots de la langue courante et les idées qu’ils traduisent. Ce qu’il faut, c’est constituer de toutes pièces des concepts nouveaux, appropriés aux besoins de la science et exprimés à l’aide d’une terminologie spéciale. Ce n’est pas, sans doute, que le concept vulgaire soit inutile au savant ; il sert d’indicateur. Par lui, nous sommes informés qu’il existe quelque part un ensemble de phénomènes qui sont réunis sous une même appellation et qui, par conséquent, doivent vraisemblablement avoir des caractères communs ; même, comme il n’est jamais sans avoir eu quelque contact avec les phénomènes, il nous indique parfois, mais en gros, dans quelle direction ils doivent être recherchés. Mais, comme il est grossièrement formé, il est tout naturel qu’il ne coïncide pas exactement avec le concept scientifique institué à son occasion\footnote{ Dans la pratique, c’est toujours du concept vulgaire et du mot vulgaire que l’on part. On cherche si, parmi les choses que connote confusément ce mot, il en est qui présentent des caractères extérieurs communs. S’il y en a et si le concept formé par le groupement des faits ainsi rapprochés coïncide, sinon totalement (ce qui est rare), du moins en majeure partie, avec le concept vulgaire, on pourra continuer à désigner le premier par le même mot que le second et garder dans la science l’expression usitée dans la langue courante. Mais si l’écart est trop considérable, si la notion commune confond une pluralité de notions distinctes, la création de termes nouveaux et spéciaux s’impose.}.\par
Si évidente et si importante que soit cette règle, elle n’est guère observée en sociologie. Précisément parce qu’il y est traité de choses dont nous parlons sans cesse, comme la famille, la propriété, le crime, etc., il paraît le plus souvent inutile au sociologue d’en donner une définition préalable et rigoureuse. Nous sommes tellement habitués à nous servir de ces mots, qui reviennent à tout instant dans le cours des conversations, qu’il semble inutile de préciser le sens dans lequel nous les prenons. On s’en réfère simplement à la notion commune. Or celle-ci est très souvent ambiguë. Cette ambiguïté fait qu’on réunit sous un même nom et dans une même explication des choses, en réalité, très différentes. De là proviennent d’inextricables confusions. Ainsi, il existe deux sortes d’unions monogamiques : les unes le sont de fait, les autres de droit. Dans les premières, le mari n’a qu’une femme quoique, juridiquement, il puisse en avoir plusieurs ; dans les secondes, il lui est légalement interdit d’être polygame. La monogamie de fait se rencontre chez plusieurs espèces animales et dans certaines sociétés inférieures, non pas à l’état sporadique, mais avec la même généralité que si elle était imposée par la loi. Quand la peuplade est dispersée sur une vaste surface, la trame sociale est très lâche et, par suite, les individus vivent isolés les uns des autres. Dès lors, chaque homme cherche naturellement à se procurer une femme et une seule, parce que, dans cet état d’isolement, il lui est difficile d’en avoir plusieurs. La monogamie obligatoire, au contraire, ne s’observe que dans les sociétés les plus élevées. Ces deux espèces de sociétés conjugales ont donc une signification très différente, et pourtant le même mot sert à les désigner ; car on dit couramment de certains animaux qu’ils sont monogames, quoiqu’il n’y ait chez eux rien qui ressemble à une obligation juridique. Or M. Spencer, abordant l’étude du mariage, emploie le mot de monogamie, sans le définir, avec son sens usuel et équivoque. Il en résulte que l’évolution du mariage lui paraît présenter une incompréhensible anomalie, puisqu’il croit observer la forme supérieure de l’union sexuelle dès les premières phases du développement historique, alors qu’elle semble plutôt disparaître dans la période intermédiaire pour réapparaître ensuite. Il en conclut qu’il n’y a pas de rapport régulier entre le progrès social en général et l’avancement progressif vers un type parfait de vie familiale. Une définition opportune eût prévenu cette erreur\footnote{ C’est la même absence de définition qui a fait dire parfois que la démocratie se rencontrait également au commencement et à la fin de l’histoire. La vérité, c’est que la démocratie primitive et celle d’aujourd’hui sont très différentes l’une de l’autre.}.\par
Dans d’autres cas, on prend bien soin de définir l’objet sur lequel va porter la recherche ; mais, au lieu de comprendre dans la définition et de grouper sous la même rubrique tous les phénomènes qui ont les mêmes propriétés extérieures, on fait entre eux un triage. On en choisit certains, sorte d’élite, que l’on regarde comme ayant seuls le droit d’avoir ces caractères. Quant aux autres, on les considère comme ayant usurpé ces signes distinctifs et on n’en tient pas compte. Mais il est aisé de prévoir que l’on ne peut obtenir de cette manière qu’une notion subjective et tronquée. Cette élimination, en effet, ne peut être faite que d’après une idée préconçue, puisque, au début de la science, aucune recherche n’a pu encore établir la réalité de cette usurpation, à supposer qu’elle soit possible. Les phénomènes choisis ne peuvent avoir été retenus que parce qu’ils étaient, plus que les autres, conformes à la conception idéale que l’on se faisait de cette sorte de réalité. Par exemple, M. Garofalo, au commencement de sa \emph{Criminologie}, démontre fort bien que le point de départ de cette science doit être \emph{« la notion sociologique du crime\footnote{\emph{Criminologie}, p. 2.} »}. Seulement, pour constituer cette notion, il ne compare pas indistinctement tous les actes qui, dans les différents types sociaux, ont été réprimés par des peines régulières, mais seulement certains d’entre eux, à savoir ceux qui offensent la partie moyenne et immuable du sens moral. Quant aux sentiments moraux qui ont disparu dans la suite de l’évolution, ils ne lui paraissent pas fondés dans la nature des choses pour cette raison qu’ils n’ont pas réussi à se maintenir ; par suite, les actes qui ont été réputés criminels parce qu’ils les violaient, lui semblent n’avoir dû cette dénomination qu’à des circonstances accidentelles et plus ou moins pathologiques. Mais c’est en vertu d’une conception toute personnelle de la moralité qu’il procède à cette élimination. Il part de cette idée que l’évolution morale, prise à sa source même ou dans les environs, roule toute sorte de scories et d’impuretés qu’elle élimine ensuite progressivement, et qu’aujourd’hui seulement elle est parvenue à se débarrasser de tous les éléments adventices qui, primitivement, en troublaient le cours. Mais ce principe n’est ni un axiome évident, ni une vérité démontrée ; ce n’est qu’une hypothèse, que rien même ne justifie. Les parties variables du sens moral ne sont pas moins fondées dans la nature des choses que les parties immuables ; les variations par lesquelles ont passé les premières témoignent seulement que les choses elles-mêmes ont varié. En zoologie, les formes spéciales aux espèces inférieures ne sont pas regardées comme moins naturelles que celles qui se répètent à tous les degrés de l’échelle animale. De même, les actes taxés crimes par les sociétés primitives, et qui ont perdu cette qualification, sont réellement criminels par rapport à ces sociétés, tout comme ceux que nous continuons à réprimer aujourd’hui. Les premiers correspondent aux conditions changeantes de la vie sociale, les seconds aux conditions constantes ; mais les uns ne sont pas plus artificiels que les autres.\par
Il y a plus : alors même que ces actes auraient indûment revêtu le caractère criminologique, néanmoins ils ne devraient pas être séparés radicalement des autres ; car les formes morbides d’un phénomène ne sont pas d’une autre nature que les formes normales et, par conséquent, il est nécessaire d’observer les premières comme les secondes pour déterminer cette nature. La maladie ne s’oppose pas à la santé ; ce sont deux variétés du même genre et qui s’éclairent mutuellement. C’est une règle depuis longtemps reconnue et pratiquée en biologie comme en psychologie et que le sociologue n’est pas moins tenu de respecter. À moins d’admettre qu’un même phénomène puisse être dû tantôt à une cause et tantôt à une autre, c’est-à-dire à moins de nier le principe de causalité, les causes qui impriment à un acte, mais d’une manière anormale, le signe distinctif du crime, ne sauraient différer en espèce de celles qui produisent normalement le même effet ; elles s’en distinguent seulement en degré ou parce qu’elles n’agissent pas dans le même ensemble de circonstances. Le crime anormal est donc encore un crime et doit, par suite, entrer dans la définition du crime. Aussi qu’arrive-t-il ? C’est que M. Garofalo prend pour le genre ce qui n’est que l’espèce ou même une simple variété. Les faits auxquels s’applique sa formule de la criminalité ne représentent qu’une infime minorité parmi ceux qu’elle devrait comprendre ; car elle ne convient ni aux crimes religieux, ni aux crimes contre l’étiquette, le cérémonial, la tradition, etc., qui, s’ils ont disparu de nos Codes modernes, remplissent, au contraire, presque tout le droit pénal des sociétés antérieures.\par
C’est la même faute de méthode qui fait que certains observateurs refusent aux sauvages toute espèce de moralité\footnote{ V. Lubbock, \emph{Les Origines de la civilisation}, ch. VIII. — Plus généralement encore, on dit, non moins faussement, que les religions anciennes sont amorales ou immorales. La vérité est qu’elles ont leur morale à elles.}. Ils partent de cette idée que notre morale est la morale ; or il est évident qu’elle est inconnue des peuples primitifs ou qu’elle n’y existe qu’à l’état rudimentaire. Mais cette définition est arbitraire. Appliquons notre règle et tout change. Pour décider si un précepte est moral ou non, nous devons examiner s’il présente ou non le signe extérieur de la moralité ; ce signe consiste dans une sanction répressive diffuse, c’est-à-dire dans un blâme de l’opinion publique qui venge toute violation du précepte. Toutes les fois que nous sommes en présence d’un fait qui présente ce caractère, nous n’avons pas le droit de lui dénier la qualification de moral ; car c’est la preuve qu’il est de même nature que les autres faits moraux. Or, non seulement des règles de ce genre se rencontrent dans les sociétés inférieures, mais elles y sont plus nombreuses que chez les civilisés. Une multitude d’actes qui, actuellement, sont abandonnés à la libre appréciation des individus, sont alors imposés obligatoirement. On voit à quelles erreurs on est entraîné soit quand on ne définit pas, soit quand on définit mal.\par
Mais, dira-t-on, définir les phénomènes par leurs caractères apparents, n’est-ce pas attribuer aux propriétés superficielles une sorte de prépondérance sur les attributs fondamentaux ; n’est-ce pas, par un véritable renversement de l’ordre logique, faire reposer les choses sur leurs sommets, et non sur leurs bases ? C’est ainsi que, quand on définit le crime par la peine, on s’expose presque inévitablement à être accusé de vouloir dériver le crime de la peine ou, suivant une citation bien connue, de voir dans l’échafaud la source de la honte, non dans l’acte expié. Mais le reproche repose sur une confusion. Puisque la définition dont nous venons de donner la règle est placée au commencement de la science, elle ne saurait avoir pour objet d’exprimer l’essence de la réalité ; elle doit seulement nous mettre en état d’y parvenir ultérieurement. Elle a pour unique fonction de nous faire prendre contact avec les choses et, comme celles-ci ne peuvent être atteintes par l’esprit que du dehors, c’est par leurs dehors qu’elle les exprime. Mais elle ne les explique pas pour autant ; elle fournit seulement le premier point d’appui nécessaire à nos explications. Non certes, ce n’est pas la peine qui fait le crime, mais c’est par elle qu’il se révèle extérieurement à nous et c’est d’elle, par conséquent, qu’il faut partir si nous voulons arriver à le comprendre.\par
L’objection ne serait fondée que si ces caractères extérieurs étaient en même temps accidentels, c’est-à-dire s’ils n’étaient pas liés aux propriétés fondamentales. Dans ces conditions, en effet, la science, après les avoir signalés, n’aurait aucun moyen d’aller plus loin ; elle ne pourrait descendre plus bas dans la réalité, puisqu’il n’y aurait aucun rapport entre la surface et le fond. Mais, à moins que le principe de causalité ne soit un vain mot, quand des caractères déterminés se retrouvent identiquement et sans aucune exception dans tous les phénomènes d’un certain ordre, on peut être assuré qu’ils tiennent étroitement à la nature de ces derniers et qu’ils en sont solidaires. Si un groupe donné d’actes présente également cette particularité qu’une sanction pénale y est attachée, c’est qu’il existe un lien intime entre la peine et les attributs constitutifs de ces actes. Par conséquent, si superficielles qu’elles soient, ces propriétés, pourvu qu’elles aient été méthodiquement observées, montrent bien au savant la voie qu’il doit suivre pour pénétrer plus au fond des choses ; elles sont le premier et indispensable anneau de la chaîne que la science déroulera ensuite au cours de ses explications.\par
Puisque c’est par la sensation que l’extérieur des choses nous est donné, on peut donc dire en résumé : la science, pour être objective, doit partir, non de concepts qui se sont formés sans elle, mais de la sensation. C’est aux données sensibles qu’elle doit directement emprunter les éléments de ses définitions initiales. Et en effet, il suffit de se représenter en quoi consiste l’œuvre de la science pour comprendre qu’elle ne peut pas procéder autrement. Elle a besoin de concepts qui expriment adéquatement les choses, telles qu’elles sont, non telles qu’il est utile à la pratique de les concevoir. Or ceux qui se sont constitués en dehors de son action ne répondent pas à cette condition. Il faut donc qu’elle en crée de nouveaux et, pour cela, qu’écartant les notions communes et les mots qui les expriment, elle revienne à la sensation, matière première et nécessaire de tous les concepts. C’est de la sensation que se dégagent toutes les idées générales, vraies ou fausses, scientifiques ou non. Le point de départ de la science ou connaissance spéculative ne saurait donc être autre que celui de la connaissance vulgaire ou pratique. C’est seulement au-delà, dans la manière dont cette matière commune est ensuite élaborée, que les divergences commencent.\par
\par
3° Mais la sensation est facilement subjective. Aussi est-il de règle dans les sciences naturelles d’écarter les données sensibles qui risquent d’être trop personnelles à l’observateur, pour retenir exclusivement celles qui présentent un suffisant degré d’objectivité. C’est ainsi que le physicien substitue aux vagues impressions que produisent la température ou l’électricité la représentation visuelle des oscillations du thermomètre ou de l’électromètre. Le sociologue est tenu aux mêmes précautions. Les caractères extérieurs en fonction desquels il définit l’objet de ses recherches doivent être aussi objectifs que possible.\par
On peut poser en principe que les faits sociaux sont d’autant plus susceptibles d’être objectivement représentés qu’ils sont plus complètement dégagés des faits individuels qui les manifestent.\par
En effet, une sensation est d’autant plus objective que l’objet auquel elle se rapporte a plus de fixité ; car la condition de toute objectivité, c’est l’existence d’un point de repère, constant et identique, auquel la représentation peut être rapportée et qui permet d’éliminer tout ce qu’elle a de variable, partant de subjectif. Si les seuls points de repère qui sont donnés sont eux-mêmes variables, s’ils sont perpétuellement divers par rapport à eux-mêmes, toute commune mesure fait défaut et nous n’avons aucun moyen de distinguer dans nos impressions ce qui dépend du dehors, et ce qui leur vient de nous. Or, la vie sociale tant qu’elle n’est pas arrivée à s’isoler des événements particuliers qui l’incarnent pour se constituer à part, a justement cette propriété, car, comme ces événements n’ont pas la même physionomie d’une fois à l’autre, d’un instant à l’autre et qu’elle en est inséparable, ils lui communiquent leur mobilité. Elle consiste alors en libres courants qui sont perpétuellement en voie de transformation et que le regard de l’observateur ne parvient pas à fixer. C’est dire que ce côté n’est pas celui par où le savant peut aborder l’étude de la réalité sociale. Mais nous savons qu’elle présente cette particularité que, sans cesser d’être elle-même, elle est susceptible de se cristalliser. En dehors des actes individuels qu’elles suscitent, les habitudes collectives s’expriment sous des formes définies, règles juridiques, morales, dictons populaires, faits de structure sociale, etc. Comme ces formes existent d’une manière permanente, qu’elles ne changent pas avec les diverses applications qui en sont faites, elles constituent un objet fixe, un étalon constant qui est toujours à la portée de l’observateur et qui ne laisse pas de place aux impressions subjectives et aux observations personnelles. Une règle du droit est ce qu’elle est et il n’y a pas deux manières de la percevoir. Puisque, d’un autre côté, ces pratiques ne sont que de la vie sociale consolidée, il est légitime, sauf indications contraires\footnote{ Il faudrait, par exemple, avoir des raisons de croire que, à un moment donné, le droit n’exprime plus l’état véritable des relations sociales, pour que cette substitution ne fût pas légitime.} d’étudier celle-ci à travers celles-là.\par
{\itshape Quand, donc, le sociologue entreprend d’explorer un ordre quelconque de faits sociaux, il doit s’efforcer de les considérer par un côté où ils se présentent isolés de leurs manifestations individuelles}. C’est en vertu de ce principe que nous avons étudié la solidarité sociale, ses formes diverses et leur évolution à travers le système des règles juridiques qui les expriment\footnote{ V \emph{Division du travail social}, l. I.}. De même, si l’on essaie de distinguer et de classer les différents types familiaux d’après les descriptions littéraires que nous en donnent les voyageurs et, parfois, les historiens, on s’expose à confondre les espèces les plus différentes, à rapprocher les types les plus éloignés. Si, au contraire, on prend pour base de cette classification la constitution juridique de la famille et, plus spécialement, le droit successoral, on aura un critère objectif qui, sans être infaillible, préviendra cependant bien des erreurs\footnote{ Cf. notre \emph{Introduction à la Sociologie de la famille}, in \emph{Annales de la Faculté des lettres de Bordeaux}, année 1889.}. Veut-on classer les différentes sortes de crimes ? on s’efforcera de reconstituer les manières de vivre, les coutumes professionnelles usitées dans les différents mondes du crime, et on reconnaîtra autant de types criminologiques que cette organisation présente de formes différentes. Pour atteindre les mœurs, les croyances populaires, on s’adressera aux proverbes, aux dictons qui les expriment. Sans doute, en procédant ainsi, on laisse provisoirement en dehors de la science la matière concrète de la vie collective, et cependant, si changeante qu’elle soit, on n’a pas le droit d’en postuler a priori l’inintelligibilité. Mais si l’on veut suivre une voie méthodique, il faut établir les premières assises de la science sur un terrain ferme et non sur un sable mouvant. Il faut aborder le règne social par les endroits où il offre le plus prise à l’investigation scientifique. C’est seulement ensuite qu’il sera possible de pousser plus loin la recherche, et, par des travaux d’approche progressifs, d’enserrer peu à peu cette réalité fuyante dont l’esprit humain ne pourra jamais, peut-être, se saisir complètement.
\chapterclose


\chapteropen
\chapter[{Chapitre III : Règles relatives à la distinction du normal et du pathologique}]{Chapitre III : \\
Règles relatives à la distinction du normal et du pathologique}\renewcommand{\leftmark}{Chapitre III : \\
Règles relatives à la distinction du normal et du pathologique}


\chaptercont
\noindent L’observation, conduite d’après les règles qui précèdent, confond deux ordres de faits, très dissemblables par certains côtés : ceux qui sont tout ce qu’ils doivent être et ceux qui devraient être autrement qu’ils ne sont, les phénomènes normaux et les phénomènes pathologiques. Nous avons même vu qu’il était nécessaire de les comprendre également dans la définition par laquelle doit débuter toute recherche. Mais si, à certains égards, ils sont de même nature, ils ne laissent pas de constituer deux variétés différentes et qu’il importe de distinguer. La science dispose-t-elle de moyens qui permettent de faire cette distinction ?\par
La question est de la plus grande importance ; car de la solution qu’on en donne dépend l’idée qu’on se fait du rôle qui revient à la science, surtout à la science de l’homme. D’après une théorie dont les partisans se recrutent dans les écoles les plus diverses, la science ne nous apprendrait rien sur ce que nous devons vouloir. Elle ne connaît, dit-on, que des faits qui ont tous la même valeur et le même intérêt ; elle les observe, les explique, mais ne les juge pas ; pour elle, il n’y en a point qui soient blâmables. Le bien et le mal n’existent pas à ses yeux. Elle peut bien nous dire comment les causes produisent leurs effets, non quelles fins doivent être poursuivies. Pour savoir, non plus ce qui est, mais ce qui est désirable, c’est aux suggestions de l’inconscient qu’il faut recourir, de quelque nom qu’on l’appelle, sentiment, instinct, poussée vitale, etc. La science, dit un écrivain déjà cité, peut bien éclairer le monde, mais elle laisse la nuit dans les cœurs ; c’est au cœur lui-même à se faire sa propre lumière. La science se trouve ainsi destituée, ou à peu près, de toute efficacité pratique, et, par conséquent, sans grande raison d’être ; car à quoi bon se travailler pour connaître le réel, si la connaissance que nous en acquérons ne peut nous servir dans la vie ? Dira-t-on que, en nous révélant les causes des phénomènes, elle nous fournit les moyens de les produire à notre guise et, par suite, de réaliser les fins que notre volonté poursuit pour des raisons supra-scientifiques ? Mais tout moyen est lui-même une fin, par un côté ; car, pour le mettre en œuvre, il faut le vouloir tout comme la fin dont il prépare la réalisation. Il y a toujours plusieurs voies qui mènent à un but donné ; il faut donc choisir entre elles. Or, si la science ne peut nous aider dans le choix du but le meilleur, comment pourrait-elle nous apprendre quelle est la meilleure voie pour y parvenir ? Pourquoi nous recommanderait-elle la plus rapide de préférence à la plus économique, la plus sûre plutôt que la plus simple, ou inversement ? Si elle ne peut nous guider dans la détermination des fins supérieures, elle n’est pas moins impuissante quand il s’agit de ces fins secondaires et subordonnées que l’on appelle des moyens.\par
La méthode idéologique permet, il est vrai, d’échapper à ce mysticisme et c’est, d’ailleurs, le désir d’y échapper qui a fait, en partie, la persistance de cette méthode. Ceux qui l’ont pratiquée, en effet, étaient trop rationalistes pour admettre que la conduite humaine n’eût pas besoin d’être dirigée par la réflexion ; et pourtant, ils ne voyaient dans les phénomènes, pris en eux-mêmes et indépendamment de toute donnée subjective, rien qui permit de les classer suivant leur valeur pratique. Il semblait donc que le seul moyen de les juger fût de les rapporter à quelque concept qui les dominât ; dès lors, l’emploi de notions qui présidassent à la collation des faits, au lieu d’en dériver, devenait indispensable dans toute sociologie rationnelle. Mais nous savons que si, dans ces conditions, la pratique devient réfléchie, la réflexion, ainsi employée, n’est pas scientifique.\par
Le problème que nous venons de poser va nous permettre de revendiquer les droits de la raison sans retomber dans l’idéologie. En effet, pour les sociétés comme pour les individus, la santé est bonne et désirable, la maladie, au contraire, est la chose mauvaise et qui doit être évitée. Si donc nous trouvons un critère objectif, inhérent aux faits eux-mêmes, qui nous permette de distinguer scientifiquement la santé de la maladie dans les divers ordres de phénomènes sociaux, la science sera en état d’éclairer la pratique tout en restant fidèle à sa propre méthode. Sans doute, comme elle ne parvient pas présentement à atteindre l’individu, elle ne peut nous fournir que des indications générales qui ne peuvent être diversifiées convenablement que si l’on entre directement en contact avec le particulier par la sensation. L’état de santé, tel qu’elle le peut définir, ne saurait convenir exactement à aucun sujet individuel, puisqu’il ne peut être établi que par rapport aux circonstances les plus communes, dont tout le monde s’écarte plus ou moins ; ce n’en est pas moins un point de repère précieux pour orienter la conduite. De ce qu’il y a lieu de l’ajuster ensuite à chaque cas spécial, il ne suit pas qu’il n’y ait aucun intérêt à le connaître. Tout au contraire, il est la norme qui doit servir de base à tous nos raisonnements pratiques. Dans ces conditions, on n’a plus le droit de dire que la pensée est inutile à l’action. Entre la science et l’art il n’y a plus un abîme ; mais on passe de l’une à l’autre sans solution de continuité. La science, il est vrai, ne peut descendre dans les faits que par l’intermédiaire de l’art, mais l’art n’est que le prolongement de la science. Encore peut-on se demander si l’insuffisance pratique de cette dernière ne doit pas aller en diminuant, à mesure que les lois qu’elle établit exprimeront de plus en plus complètement la réalité individuelle.\par
\section[{I}]{I}
\noindent Vulgairement, la souffrance est regardée comme l’indice de la maladie et il est certain que, en général, il existe entre ces deux faits un rapport, mais qui manque de constance et de précision. Il y a de graves diathèses qui sont indolores, alors que des troubles sans importance, comme ceux qui résultent de l’introduction d’un grain de charbon dans l’œil, causent un véritable supplice. Même, dans certains cas, c’est l’absence de douleur ou bien encore le plaisir qui sont les symptômes de la maladie. Il y a une certaine disvulnérabilité qui est pathologique. Dans des circonstances où un homme sain souffrirait, il arrive au neurasthénique d’éprouver une sensation de jouissance dont la nature morbide est incontestable. Inversement, la douleur accompagne bien des états, comme la faim, la fatigue, la parturition qui sont des phénomènes purement physiologiques.\par
Dirons-nous que la santé, consistant dans un heureux développement des forces vitales, se reconnaît à la parfaite adaptation de l’organisme avec son milieu, et appellerons-nous, au contraire, maladie tout ce qui trouble cette adaptation ? Mais d’abord — nous aurons plus loin à revenir sur ce point — il n’est pas du tout démontré que chaque état de l’organisme soit en correspondance avec quelque état externe. De plus, et quand bien même ce critère serait vraiment distinctif de l’état de santé, il aurait lui-même besoin d’un autre critère pour pouvoir être reconnu ; car il faudrait, en tout cas, nous dire d’après quel principe on peut décider que tel mode de s’adapter est plus parfait que tel autre.\par
Est-ce d’après la manière dont l’un et l’autre affectent nos chances de survie ? La santé serait l’état d’un organisme où ces chances sont à leur maximum et la maladie, au contraire, tout ce qui a pour effet de les diminuer. Il n’est pas douteux, en effet, que, en général, la maladie n’ait réellement pour conséquence un affaiblissement de l’organisme. Seulement elle n’est pas seule a produire ce résultat. Les fonctions de reproduction, dans certaines espèces inférieures, entraînent fatalement la mort et, même dans les espèces plus élevées, elles créent des risques. Cependant elles sont normales. La vieillesse et l’enfance ont les mêmes effets ; car le vieillard et l’enfant sont plus accessibles aux causes de destruction. Sont-ils donc des malades et faut-il n’admettre d’autre type sain que celui de l’adulte ? Voilà le domaine de la santé et de la physiologie singulièrement rétréci ! Si, d’ailleurs, la vieillesse est déjà, par elle-même, une maladie, comment distinguer le vieillard sain du vieillard maladif ? Du même point de vue, il faudra classer la menstruation parmi les phénomènes morbides ; car, par les troubles qu’elle détermine, elle accroît la réceptivité de la femme à la maladie. Comment, cependant, qualifier de maladif un état dont l’absence ou la disparition prématurée constituent incontestablement un phénomène pathologique ? On raisonne sur cette question comme si, dans un organisme sain, chaque détail, pour ainsi dire, avait un rôle utile à jouer ; comme si chaque état interne répondait exactement à quelque condition externe et, par suite, contribuait à assurer, pour sa part, l’équilibre vital et à diminuer les chances de mort. Il est, au contraire, légitime de supposer que certains arrangements anatomiques ou fonctionnels ne servent directement à rien, mais sont simplement parce qu’ils sont, parce qu’ils ne peuvent pas ne pas être, étant données les conditions générales de la vie. On ne saurait pourtant les taxer de morbides ; car la maladie est, avant tout, quelque chose d’évitable qui n’est pas impliqué dans la constitution régulière de l’être vivant. Or il peut se faire que, au lieu de fortifier l’organisme, ils diminuent sa force de résistance et, par conséquent, accroissent les risques mortels.\par
D’autre part, il n’est pas sûr que la maladie ait toujours le résultat en fonction duquel on la veut définir. N’y a-t-il pas nombre d’affections trop légères pour que nous puissions leur attribuer une influence sensible sur les bases vitales de l’organisme ? Même parmi les plus graves, il en est dont les suites n’ont rien de fâcheux, si nous savons lutter contre elles avec les armes dont nous disposons. Le gastrique qui suit une bonne hygiène peut vivre tout aussi vieux que l’homme sain. Il est, sans doute, obligé a des soins ; mais n’y sommes-nous pas tous également obligés et la vie peut-elle s’entretenir autrement ? Chacun de nous a son hygiène ; celle du malade ne ressemble pas à celle que pratique la moyenne des hommes de son temps et de son milieu ; mais c’est la seule différence qu’il y ait entre eux à ce point de vue. La maladie ne nous laisse pas toujours désemparés, dans un état de désadaptation irrémédiable ; elle nous contraint seulement à nous adapter autrement que la plupart de nos semblables. Qui nous dit même qu’il n’existe pas de maladies qui, finalement, se trouvent être utiles ? La variole que nous nous inoculons par le vaccin est une véritable maladie que nous nous donnons volontairement, et pourtant elle accroît nos chances de survie. Il y a peut-être bien d’autres cas où le trouble causé par la maladie est insignifiant à côté des immunités qu’elle confère.\par
Enfin et surtout, ce critère est le plus souvent inapplicable. On peut bien établir, à la rigueur, que la mortalité la plus basse que l’on connaisse se rencontre dans tel groupe déterminé d’individus ; mais on ne peut pas démontrer qu’il ne saurait y en avoir de plus basse. Qui nous dit que d’autres arrangements ne sont pas possibles, qui auraient pour effet de la diminuer encore ? Ce {\itshape minimum} de fait n’est donc pas la preuve d’une parfaite adaptation, ni, par suite, l’indice sûr de l’état de santé si l’on s’en rapporte à la définition précédente. De plus, un groupe de cette nature est bien difficile à constituer et à isoler de tous les autres, comme il serait nécessaire, pour que l’on pût observer la constitution organique dont il a le privilège et qui est la cause supposée de cette supériorité. Inversement, si, quand il s’agit d’une maladie dont le dénouement est généralement mortel, il est évident que les probabilités que l’être a de survivre sont diminuées, la preuve est singulièrement malaisée, quand l’affection n’est pas de nature à entraîner directement la mort. Il n’y a, en effet, qu’une manière objective de prouver que des êtres, placés dans des conditions définies, ont moins de chances de survivre que d’autres, c’est de faire voir que, en fait, la plupart d’entre eux vivent moins longtemps. Or, si, dans les cas de maladies purement individuelles, cette démonstration est souvent possible, elle est tout à fait impraticable en sociologie. Car nous n’avons pas ici le point de repère dont dispose le biologiste, à savoir le chiffre de la mortalité moyenne. Nous ne savons même pas distinguer avec une exactitude simplement approchée à quel moment naît une société et à quel moment elle meurt. Tous ces problèmes qui, déjà en biologie, sont loin d’être clairement résolus, restent encore, pour le sociologue, enveloppés de mystère. D’ailleurs, les événements qui se produisent au cours de la vie sociale et qui se répètent à peu près identiquement dans toutes les sociétés du même type, sont beaucoup trop variés pour qu’il soit possible de déterminer dans quelle mesure l’un d’eux peut avoir contribué à hâter le dénouement final. Quand il s’agit d’individus, comme ils sont très nombreux, on peut choisir ceux que l’on compare de manière à ce qu’ils n’aient en commun qu’une seule et même anomalie ; celle-ci se trouve ainsi isolée de tous les phénomènes concomitants et on peut, par suite, étudier la nature de son influence sur l’organisme. Si, par exemple, un millier de rhumatisants, pris au hasard, présente une mortalité sensiblement supérieure à la moyenne, on a de bonnes raisons pour attribuer ce résultat à la diathèse rhumatismale. Mais, en sociologie, comme chaque espèce sociale ne compte qu’un petit nombre d’individus, le champ des comparaisons est trop restreint pour que des groupements de ce genre soient démonstratifs.\par
Or, à défaut de cette preuve de fait, il n’y a plus rien de possible que des raisonnements déductifs dont les conclusions ne peuvent avoir d’autre valeur que celle de présomptions subjectives. On démontrera non que tel évènement affaiblit effectivement l’organisme social, mais qu’il doit avoir cet effet. Pour cela, on fera voir qu’il ne peut manquer d’entraîner à sa suite telle ou telle conséquence que l’on juge fâcheuse pour la société et, à ce titre, on le déclarera morbide. Mais, à supposer même qu’il engendre en effet cette conséquence, il peut se faire que les inconvénients qu’elle présente soient compensés, et au-delà, par des avantages que l’on n’aperçoit pas. De plus, il n’y a qu’une raison qui puisse permettre de la traiter de funeste, c’est qu’elle trouble le jeu normal des fonctions. Mais une telle preuve suppose le problème déjà résolu ; car elle n’est possible que si l’on a déterminé au préalable en quoi consiste l’état normal et, par conséquent, si l’on sait à quel signe il peut être reconnu. Essaiera-t-on de le construire de toutes pièces et {\itshape a priori} ? Il n’est pas nécessaire de montrer ce que peut valoir une telle construction. Voilà comment il se fait que, en sociologie comme en histoire, les mêmes événements sont qualifiés, suivant les sentiments personnels du savant, de salutaires ou de désastreux. Ainsi il arrive sans cesse à un théoricien incrédule de signaler, dans les restes de foi qui survivent au milieu de l’ébranlement général des croyances religieuses, un phénomène morbide, tandis que, pour le croyant, c’est l’incrédulité même qui est aujourd’hui la grande maladie sociale. De même, pour le socialiste, l’organisation économique actuelle est un fait de tératologie sociale, alors que, pour l’économiste orthodoxe, ce sont les tendances socialistes qui sont, par excellence, pathologiques. Et chacun trouve à l’appui de son opinion des syllogismes qu’il juge bien faits.\par
Le défaut commun de ces définitions est de vouloir atteindre prématurément l’essence des phénomènes. Aussi supposent-elles acquises des propositions qui, vraies ou non, ne peuvent être prouvées que si la science est déjà suffisamment avancée. C’est pourtant le cas de nous conformer à la règle que nous avons précédemment établie. Au lieu de prétendre déterminer d’emblée les rapports de l’état normal et de son contraire avec les forces vitales, cherchons simplement quelque signe extérieur, immédiatement perceptible, mais objectif, qui nous permette de reconnaître l’un de l’autre ces deux ordres de faits.\par
Tout phénomène sociologique, comme, du reste, tout phénomène biologique, est susceptible, tout en restant essentiellement lui-même, de revêtir des formes différentes suivant les cas. Or, parmi ces formes, il en est de deux sortes. Les unes sont générales dans toute l’étendue de l’espèce ; elles se retrouvent, sinon chez tous les individus, du moins chez la plupart d’entre eux et, si elles ne se répètent pas identiquement dans tous les cas où elles s’observent, mais varient d’un sujet à l’autre, ces variations sont comprises entre des limites très rapprochées. Il en est d’autres, au contraire, qui sont exceptionnelles ; non seulement elles ne se rencontrent que chez la minorité, mais, là même où elles se produisent, il arrive le plus souvent qu’elles ne durent pas toute la vie de l’individu. Elles sont une exception dans le temps comme dans l’espace\footnote{ On peut distinguer par là la maladie de la monstruosité. La seconde n’est une exception que dans l’espace ; elle ne se rencontre pas dans la moyenne de l’espèce, mais elle dure toute la vie des individus où elle se rencontre. On voit, du reste, que ces deux ordres de faits ne diffèrent qu’en degrés et sont au fond de même nature ; les frontières entre eux sont très indécises, car la maladie n’est pas incapable de toute fixité, ni la monstruosité de tout devenir. On ne peut donc guère les séparer radicalement quand on les définit. La distinction entre eux ne peut être plus catégorique qu’entre le morphologique et le physiologique, puisque, en somme, le morbide est l’anormal dans l’ordre physiologique comme le tératologique est l’anormal dans l’ordre anatomique.}. Nous sommes donc en présence de deux variétés distinctes de phénomènes et qui doivent être désignées par des termes différents. Nous appellerons normaux les faits qui présentent les formes les plus générales et nous donnerons aux autres le nom de morbides ou de pathologiques. Si l’on convient de nommer type moyen l’être schématique que l’on constituerait en rassemblant en un même tout, en une sorte d’individualité abstraite, les caractères les plus fréquents dans l’espèce avec leurs formes les plus fréquentes, on pourra dire que le type normal se confond avec le type moyen, et que tout écart par rapport à cet étalon de la santé est un phénomène morbide. Il est vrai que le type moyen ne saurait être déterminé avec la même netteté qu’un type individuel, puisque ses attributs constitutifs ne sont pas absolument fixés, mais sont susceptibles de varier. Mais qu’il puisse être constitué, c’est ce qu’on ne saurait mettre en doute, puisqu’il est la matière immédiate de la science ; car il se confond avec le type générique. Ce que le physiologiste étudie, ce sont les fonctions de l’organisme moyen et il n’en est pas autrement du sociologue. Une fois qu’on sait reconnaître les espèces sociales les unes des autres — nous traitons plus loin la question — il est toujours possible de trouver quelle est la forme la plus générale que présente un phénomène dans une espèce déterminée.\par
On voit qu’un fait ne peut être qualifié de pathologique que par rapport à une espèce donnée. Les conditions de la santé et de la maladie ne peuvent être définies {\itshape in abstracto} et d’une manière absolue. La règle n’est pas contestée en biologie ; il n’est jamais venu à l’esprit de personne que ce qui est normal pour un mollusque le soit aussi pour un vertébré. Chaque espèce a sa santé, parce qu’elle a son type moyen qui lui est propre, et la santé des espèces les plus basses n’est pas moindre que celle des plus élevées. Le même principe s’applique à la sociologie quoiqu’il y soit souvent méconnu. Il faut renoncer à cette habitude, encore trop répandue, de juger une institution, une pratique, une maxime morale, comme si elles étaient bonnes ou mauvaises en elles-mêmes et par elles-mêmes, pour tous les types sociaux indistinctement.\par
Puisque le point de repère par rapport auquel on peut juger de l’état de santé ou de maladie varie avec les espèces, il peut varier aussi pour une seule et même espèce, si celle-ci vient à changer. C’est ainsi que, au point de vue purement biologique, ce qui est normal pour le sauvage ne l’est pas toujours pour le civilisé et réciproquement\footnote{ Par exemple, le sauvage qui aurait le tube digestif réduit et le système nerveux développé du civilisé sain serait un malade par rapport à son milieu.}. Il y a surtout un ordre de variations dont il importe de tenir compte parce qu’elles se produisent régulièrement dans toutes les espèces, ce sont celles qui tiennent à l’âge. La santé du vieillard n’est pas celle de l’adulte, de même que celle-ci n’est pas celle de l’enfant ; et il en est de même des sociétés\footnote{ Nous abrégeons cette partie de notre développement ; car nous ne pouvons que répéter ici à propos des faits sociaux en général ce que nous avons dit ailleurs à propos de la distinction des faits moraux en normaux et anormaux.(V. \emph{Division du travail social}, p. 33-39.)}. Un fait social ne peut donc être dit normal pour une espèce sociale déterminée, que par rapport à une phase, également déterminée, de son développement ; par conséquent, pour savoir s’il a droit à cette dénomination, il ne suffit pas d’observer sous quelle forme il se présente dans la généralité des sociétés qui appartiennent à cette espèce, il faut encore avoir soin de les considérer à la phase correspondante de leur évolution.\par
Il semble que nous venions de procéder simplement à une définition de mots ; car nous n’avons rien fait que grouper des phénomènes suivant leurs ressemblances et leurs différences et qu’imposer des noms aux groupes ainsi formés. Mais, en réalité, les concepts que nous avons ainsi constitués, tout en ayant le grand avantage d’être reconnaissables à des caractères objectifs et facilement perceptibles, ne s’éloignent pas de la notion qu’on se fait communément de la santé et de la maladie. La maladie, en effet, n’est-elle pas conçue par tout le monde comme un accident, que la nature du vivant comporte sans doute, mais n’engendre pas d’ordinaire ? C’est ce que les anciens philosophes exprimaient en disant qu’elle ne dérive pas de la nature des choses, qu’elle est le produit d’une sorte de contingence immanente aux organismes. Une telle conception est, assurément, la négation de toute science ; car la maladie n’a rien de plus miraculeux que la santé ; elle est également fondée dans la nature des êtres. Seulement elle n’est pas fondée dans leur nature normale ; elle n’est pas impliquée dans leur tempérament ordinaire ni liée aux conditions d’existence dont ils dépendent généralement. Inversement, pour tout le monde, le type de la santé se confond avec celui de l’espèce. On ne peut même pas, sans contradiction, concevoir une espèce qui, par elle-même et en vertu de sa constitution fondamentale, serait irrémédiablement malade. Elle est la norme par excellence et, par suite, ne saurait rien contenir d’anormal.\par
Il est vrai que, couramment, on entend aussi par santé un état généralement préférable à la maladie. Mais cette définition est contenue dans la précédente. Si, en effet, les caractères dont la réunion forme le type normal ont pu se généraliser dans une espèce, ce n’est pas sans raison. Cette généralité est elle-même un fait qui a besoin d’être expliqué et qui, pour cela, réclame une cause. Or elle serait inexplicable si les formes d’organisation les plus répandues n’étaient aussi, {\itshape du moins dans leur ensemble}, les plus avantageuses. Comment auraient-elles pu se maintenir dans une aussi grande variété de circonstances si elles ne mettaient les individus en état de mieux résister aux causes de destruction ? Au contraire, si les autres sont plus rares, c’est évidemment que, dans la moyenne des cas, les sujets qui les présentent ont plus de difficulté à survivre. La plus grande fréquence des premières est donc la preuve de leur supériorité\footnote{ M. Garofalo a essayé, il est vrai, de distinguer le morbide de l’anormal (\emph{Criminologie}, p. 109, 110). Mais les deux seuls arguments sur lesquels il appuie cette distinction sont les suivants : l° Le mot de maladie signifie toujours quelque chose qui tend à la destruction totale ou partielle de l’organisme ; s’il n’y a pas destruction, il y a guérison, jamais stabilité comme dans plusieurs anomalies. Mais nous venons de voir que l’anormal, lui aussi, est une menace pour le vivant dans la moyenne des cas. Il est vrai qu’il n’en est pas toujours ainsi ; mais les dangers qu’implique la maladie n’existent également que dans la généralité des circonstances. Quant à l’absence de stabilité qui distinguerait le morbide, c’est oublier les maladies chroniques et séparer radicalement le tératologique du pathologique. Les monstruosités sont fixes. 2° Le normal et l’anormal varient avec les races, dit-on, tandis que la distinction du physiologique et du pathologique est valable pour tout le {\itshape genus homo}. Nous venons de montrer au contraire que, souvent, ce qui est morbide pour le sauvage ne l’est pas pour le civilisé. Les conditions de la santé physique varient avec les milieux.}.
\section[{II}]{II}
\noindent Cette dernière remarque fournit même un moyen de contrôler les résultats de la précédente méthode.\par
Puisque la généralité, qui caractérise extérieurement les phénomènes normaux, est elle-même un phénomène explicable, il y a lieu, après qu’elle a été directement établie par l’observation, de chercher à l’expliquer. Sans doute, on peut être assuré par avance qu’elle n’est pas sans cause, mais il est mieux de savoir au juste quelle est cette cause. Le caractère normal du phénomène sera, en effet, plus incontestable, si l’on démontre que le signe extérieur qui l’avait d’abord révélé n’est pas purement apparent, mais est fondé dans la nature des choses ; si, en un mot, on peut ériger cette normalité de fait en une normalité de droit. Cette démonstration, du reste, ne consistera pas toujours à faire voir que le phénomène est utile à l’organisme, quoique ce soit le cas le plus fréquent pour les raisons que nous venons de dire ; mais il peut se faire aussi, comme nous l’avons remarqué plus haut, qu’un arrangement soit normal sans servir a rien, simplement parce qu’il est nécessairement impliqué dans la nature de l’être. Ainsi, il serait peut-être utile que l’accouchement ne déterminât pas des troubles aussi violents dans l’organisme féminin ; mais c’est impossible. Par conséquent, la normalité du phénomène sera expliquée par cela seul qu’il sera rattaché aux conditions d’existence de l’espèce considérée, soit comme un effet mécaniquement nécessaire de ces conditions, soit comme un moyen qui permet aux organismes de s’y adapter\footnote{ On peut se demander, il est vrai, si, quand un phénomène dérive nécessairement des conditions générales de la vie, il n’est pas utile par cela même. Nous ne pouvons traiter cette question de philosophie. Nous y touchons pourtant un peu plus loin.}.\par
Cette preuve n’est pas simplement utile à titre de contrôle. Il ne faut pas oublier, en effet, que, s’il y a intérêt à distinguer le normal de l’anormal, c’est surtout en vue d’éclairer la pratique. Or, pour agir en connaissance de cause, il ne suffit pas de savoir ce que nous devons vouloir, mais pourquoi nous le devons. Les propositions scientifiques, relatives à l’état normal, seront plus immédiatement applicables aux cas particuliers quand elles seront accompagnées de leurs raisons ; car, alors, on saura mieux reconnaître dans quels cas il convient de les modifier en les appliquant, et dans quel sens.\par
Il y a même des circonstances ou cette vérification est rigoureusement nécessaire, parce que la première méthode, si elle était employée seule, pourrait induire en erreur. C’est ce qui arrive aux périodes de transition où l’espèce tout entière est en train d’évoluer, sans s’être encore définitivement fixée sous une forme nouvelle. Dans ce cas, le seul type normal qui soit dès à présent réalisé et donné dans les faits est celui du passé, et pourtant il n’est plus en rapport avec les nouvelles conditions d’existence. Un fait peut ainsi persister dans toute l’étendue d’une espèce, tout en ne répondant plus aux exigences de la situation. Il n’a donc plus, alors, que les apparences de la normalité ; car la généralité qu’il présente n’est plus qu’une étiquette menteuse, puisque, ne se maintenant que par la force aveugle de l’habitude, elle n’est plus l’indice que le phénomène observé est étroitement lié aux conditions générales de l’existence collective. Cette difficulté est, d’ailleurs, spéciale à la sociologie. Elle n’existe, pour ainsi dire, pas pour le biologiste. Il est, en effet, bien rare que les espèces animales soient nécessitées à prendre des formes imprévues. Les seules modifications normales par lesquelles elles passent sont celles qui se reproduisent régulièrement chez chaque individu, principalement sous l’influence de l’âge. Elles sont donc connues ou peuvent l’être, puisqu’elles se sont déjà réalisées dans une multitude de cas ; par suite, on peut savoir à chaque moment du développement de l’animal, et même aux périodes de crise, en quoi consiste l’état normal. Il en est encore ainsi en sociologie pour les sociétés qui appartiennent aux espèces inférieures. Car, comme nombre d’entre elles ont déjà accompli toute leur carrière, la loi de leur évolution normale est ou, du moins, peut être établie. Mais quand il s’agit des sociétés les plus élevées et les plus récentes, cette loi est inconnue par définition, puisqu’elles n’ont pas encore parcouru toute leur histoire. Le sociologue peut ainsi se trouver embarrassé de savoir si un phénomène est normal ou non, tout point de repère lui faisant défaut.\par
Il sortira d’embarras en procédant comme nous venons de dire. Après avoir établi par l’observation que le fait est général, il remontera aux conditions qui ont déterminé cette généralité dans le passé et cherchera ensuite si ces conditions sont encore données dans le présent ou si, au contraire, elles ont changé. Dans le premier cas, il aura le droit de traiter, le phénomène de normal et, dans le second, de lui refuser ce caractère. Par exemple, pour savoir si l’état économique actuel des peuples européens, avec l’absence d’organisation\footnote{ V. sur ce point une note que nous avons publiée dans la \emph{Revue philosophique} (n° de novembre 1893) sur \emph{La Définition du socialisme}.} qui en est la caractéristique, est normal ou non, on cherchera ce qui, dans le passé, y a donné naissance. Si ces conditions sont encore celles où sont actuellement placées nos sociétés, c’est que cette situation est normale en dépit des protestations qu’elle soulève. Mais s’il se trouve, au contraire, qu’elle est liée à cette vieille structure sociale que nous avons qualifiée ailleurs de segmentaire\footnote{ Les sociétés segmentaires, et notamment les sociétés segmentaires à base territoriale, sont celles dont les articulations essentielles correspondent aux divisions territoriales. (V. \emph{Division du travail social}, p. 189-210.)} et qui, après avoir été l’ossature essentielle des sociétés, va de plus en plus en s’effaçant, on devra conclure qu’elle constitue présentement un état morbide, quelque universelle qu’elle soit. C’est d’après la même méthode que devront être résolues toutes les questions controversées de ce genre, comme celles de savoir si l’affaiblissement des croyances religieuses, si le développement des pouvoirs de l’État sont des phénomènes normaux ou non.\footnote{ Dans certains cas, on peut procéder un peu différemment et démontrer qu’un fait dont le caractère normal est suspecté, mérite ou non cette suspicion, en faisant voir qu’il se rattache étroitement au développement antérieur du type social considéré, et même à l’ensemble de l’évolution sociale en général, ou bien, au contraire, qu’il contredit l’un et l’autre. C’est de cette manière que nous avons pu démontrer que l’affaiblissement actuel des croyances religieuses, plus généralement, des sentiments collectifs à objets collectifs n’a rien que de normal ; nous avons prouvé que cet affaiblissement devient de plus en plus accusé à mesure que les sociétés se rapprochent de notre type actuel et que celui-ci, à son tour, est plus développé (\emph{Division du travail social}, p. 73-182). Mais, au fond, cette méthode n’est qu’un cas particulier de la précédente. Car si la normalité de ce phénomène a pu être établie de cette façon, c’est que, du même coup, il a été rattaché aux conditions les plus générales de notre existence collective. En effet, d’une part, si cette régression de la conscience religieuse est d’autant plus marquée que la structure de nos sociétés est plus déterminée, c’est qu’elle tient, non à quelque cause accidentelle, mais à la constitution même de notre milieu social, et comme, d’un autre côté, les particularités caractéristiques de cette dernière sont certainement plus développées aujourd’hui que naguère, il n’y a rien que de normal à ce que les phénomènes qui en dépendent soient eux-mêmes amplifiés. Cette méthode diffère seulement de la précédente en ce que les conditions qui expliquent et justifient la généralité du phénomène sont induites et non directement observées. On sait qu’il tient à la nature du milieu social sans savoir en quoi ni comment.}\par
Toutefois, cette méthode ne saurait, en aucun cas, être substituée à la précédente, ni même être employée la première. D’abord, elle soulève des questions dont, nous aurons à parler plus loin et qui ne peuvent être abordées que quand on est déjà assez avancé dans la science ; car elle implique, en somme, une explication presque complète des phénomènes, puisqu’elle suppose déterminées ou leurs causes ou leurs fonctions. Or, il importe que, dès le début de la recherche, on puisse classer les faits en normaux et anormaux, sous la réserve de quelques cas exceptionnels, afin de pouvoir assigner à la physiologie son domaine et à la pathologie le sien. Ensuite, c’est par rapport au type normal qu’un fait doit être trouvé utile ou nécessaire pour pouvoir être lui-même qualifié de normal. Autrement, on pourrait démontrer que la maladie se confond avec la santé, puisqu’elle dérive nécessairement de l’organisme qui en est atteint ; ce n’est qu’avec l’organisme moyen qu’elle ne soutient pas la même relation. De même, l’application d’un remède, étant utile au malade, pourrait passer pour un phénomène normal, alors qu’elle est évidemment anormale, car c’est seulement dans des circonstances anormales qu’elle a cette utilité. On ne peut donc se servir de cette méthode que si le type normal a été antérieurement constitué et il ne peut l’avoir été que par un autre procédé. Enfin et surtout, s’il est vrai que tout ce qui est normal est utile, à moins d’être nécessaire, il est faux que tout ce qui est utile soit normal. Nous pouvons bien être certains que les états qui se sont généralisés dans l’espèce sont plus utiles que ceux qui sont restés exceptionnels ; non qu’ils sont les plus utiles qui existent ou qui puissent exister. Nous n’avons aucune raison de croire que toutes les combinaisons possibles ont été essayées au cours de l’expérience et, parmi celles qui n’ont jamais été réalisées mais sont concevables, il en est peut-être de beaucoup plus avantageuses que celles que nous connaissons. La notion de l’utile déborde celle du normal ; elle est à celle-ci ce que le genre est à l’espèce. Or, il est impossible de déduire le plus du moins, l’espèce du genre. Mais on peut retrouver le genre dans l’espèce puisqu’elle le contient. C’est pourquoi, une fois que la généralité du phénomène a été constatée, on peut, en faisant voir comment il sert, confirmer les résultats de la première méthode\footnote{ Mais alors, dira-t-on, la réalisation du type normal n’est pas l’objectif le plus élevé qu’on puisse se proposer et, pour le dépasser, il faut aussi dépasser la science. Nous n’avons pas à traiter ici cette question {\itshape ex professo}; répondons seulement : 1° qu’elle est toute théorique, car, en fait, le type normal, l’état de santé est déjà assez difficile à réaliser et assez rarement atteint pour que nous ne nous travaillions pas l’imagination à chercher quelque chose de mieux ; 2° que ces améliorations, objectivement plus avantageuses, ne sont pas objectivement désirables pour cela ; car si elles ne répondent à aucune tendance latente ou en acte, elles n’ajouteraient rien au bonheur, et si elles répondent à quelque tendance, c’est que le type normal n’est pas réalisé; 3° enfin que, pour améliorer le type normal, il faut le connaître. On ne peut donc, en tout cas, dépasser la science qu’en s’appuyant sur elle.}. Nous pouvons donc formuler les trois règles suivantes :\par
1° {\itshape Un fait social est normal pour un type social déterminé, considéré à une phase déterminée de son développement, quand il se produit dans la moyenne des sociétés de cette espèce, considérées à la phase correspondante de leur évolution}.\par
2° {\itshape On peut vérifier les résultats de la méthode précédente en faisant voir que la généralité du phénomène tient aux conditions générales de la vie collective dans le type social considéré}.\par
3° {\itshape Cette vérification est nécessaire, quand ce fait se rapporte à une espèce sociale qui n’a pas encore accompli son évolution intégrale}.
\section[{III}]{III}
\noindent On est tellement habitué à trancher d’un mot ces questions difficiles et à décider rapidement, d’après des observations sommaires et à coup de syllogismes, si un fait social est normal ou non, qu’on jugera peut-être cette procédure inutilement compliquée. Il ne semble pas qu’il faille faire tant d’affaires pour distinguer la maladie de la santé. Ne faisons-nous pas tous les jours de ces distinctions ? — Il est vrai ; mais il reste à savoir si nous les faisons à propos. Ce qui nous masque les difficultés de ces problèmes, c’est que nous voyons le biologiste les résoudre avec une aisance relative. Mais nous oublions qu’il lui est beaucoup plus facile qu’au sociologue d’apercevoir la manière dont chaque phénomène affecte la force de résistance de l’organisme et d’en déterminer par là le caractère normal ou anormal avec une exactitude pratiquement suffisante. En sociologie, la complexité et la mobilité plus grandes des faits obligent à bien plus de précautions, comme le prouvent les jugements contradictoires dont le même phénomène est l’objet de la part des partis. Pour bien montrer combien cette circonspection est nécessaire, faisons voir par quelques exemples à quelles erreurs on s’expose quand on ne s’y astreint pas et sous quel jour nouveau les phénomènes les plus essentiels apparaissent, quand on les traite méthodiquement.\par
S’il est un fait dont le caractère pathologique paraît incontestable, c’est le crime. Tous les criminologistes s’entendent sur ce point. S’ils expliquent cette morbidité de manières différentes, ils sont unanimes à la reconnaître. Le problème, cependant, demandait à être traité avec moins de promptitude.\par
Appliquons, en effet, les règles précédentes. Le crime ne s’observe pas seulement dans la plupart des sociétés de telle ou telle espèce, mais dans toutes les sociétés de tous les types. Il n’en est pas ou il n’existe une criminalité. Elle change de forme, les actes qui sont ainsi qualifiés ne sont pas partout les mêmes ; mais, partout et toujours, il y a eu des hommes qui se conduisaient de manière à attirer sur eux la répression pénale. Si, du moins, à mesure que les sociétés passent des types inférieurs aux plus élevés, le taux de la criminalité, c’est-à-dire le rapport entre le chiffre annuel des crimes et celui de la population, tendait à baisser, on pourrait croire que, tout en restant un phénomène normal, le crime, cependant, tend à perdre ce caractère. Mais nous n’avons aucune raison qui nous permette de croire à la réalité de cette régression. Bien des faits sembleraient plutôt démontrer l’existence d’un mouvement en sens inverse. Depuis le commencement du siècle, la statistique nous fournit le moyen de suivre la marche de la criminalité ; or, elle a partout augmenté. En France, l’augmentation est de près de 300 \%. Il n’est donc pas de phénomène qui présente de la manière la plus irrécusée tous les symptômes de la normalité, puisqu’il apparaît comme étroitement lié aux conditions de toute vie collective. Faire du crime une maladie sociale, ce serait admettre que la maladie n’est pas quelque chose d’accidentel, mais, au contraire, dérive, dans certains cas, de la constitution fondamentale de l’être vivant ; ce serait effacer toute distinction entre le physiologique et le pathologique. Sans doute, il peut se faire que le crime lui-même ait des formes anormales ; c’est ce qui arrive quand, par exemple, il atteint un taux exagéré. Il n’est pas douteux, en effet, que cet excès ne soit de nature morbide. Ce qui est normal, c’est simplement qu’il y ait une criminalité, pourvu que celle-ci atteigne et ne dépasse pas, pour chaque type social, un certain niveau qu’il n’est peut-être pas impossible de fixer conformément aux règles précédentes\footnote{ De ce que le crime est un phénomène de sociologie normale, il ne suit pas que le criminel soit un individu normalement constitué au point de vue biologique et psychologique. Les deux questions sont indépendantes l’une de l’autre. On comprendra mieux cette indépendance, quand nous aurons montré plus loin la différence qu’il y a entre les faits psychiques et les faits sociologiques.}.\par
Nous voilà en présence d’une conclusion, en apparence, assez paradoxale. Car il ne faut pas s’y méprendre. Classer le crime parmi les phénomènes de sociologie normale, ce n’est pas seulement dire qu’il est un phénomène inévitable quoique regrettable, dû à l’incorrigible méchanceté des hommes ; c’est affirmer qu’il est un facteur de la santé publique, une partie intégrante de toute société saine. Ce résultat est, au premier abord, assez surprenant pour qu’il nous ait nous-même déconcerté et pendant longtemps. Cependant, une fois que l’on a dominé cette première impression de surprise, il n’est pas difficile de trouver les raisons qui expliquent cette normalité et, du même coup, la confirment.\par
En premier lieu, le crime est normal parce qu’une société qui en serait exempte est tout à fait impossible.\par
Le crime, nous l’avons montré ailleurs, consiste dans un acte qui offense certains sentiments collectifs, doués d’une énergie et d’une netteté particulières. Pour que, dans une société donnée, les actes réputés criminels pussent cesser d’être commis, il faudrait donc que les sentiments qu’ils blessent se retrouvassent dans toutes les consciences individuelles sans exception et avec le degré de force nécessaire pour contenir les sentiments contraires. Or, à supposer que cette condition put être effectivement réalisée, le crime ne disparaîtrait pas pour cela, il changerait seulement de forme ; car la cause même qui tarirait ainsi les sources de la criminalité en ouvrirait immédiatement de nouvelles. En effet, pour que les sentiments collectifs que protège le droit pénal d’un peuple, à un moment déterminé de son histoire, parviennent ainsi à pénétrer dans les consciences qui leur étaient jusqu’alors fermées ou à prendre plus d’empire là où ils n’en avaient pas assez, il faut qu’ils acquièrent une intensité supérieure à celle qu’ils avaient jusqu’alors. Il faut que la communauté dans son ensemble les ressente avec plus de vivacité ; car ils ne peuvent pas puiser à une autre source la force plus grande qui leur permet de s’imposer aux individus qui, naguère, leur étaient le plus réfractaires. Pour que les meurtriers disparaissent, il faut que l’horreur du sang versé devienne plus grande dans ces couches sociales où se recrutent les meurtriers ; mais, pour cela, il faut qu’elle devienne plus grande dans toute l’étendue de la société. D’ailleurs, l’absence même du crime contribuerait directement à produire ce résultat ; car un sentiment apparaît comme beaucoup plus respectable quand il est toujours et uniformément respecté. Mais on ne fait pas attention que ces états forts de la conscience commune ne peuvent être ainsi renforcés sans que les états plus faibles, dont la violation ne donnait précédemment naissance qu’à des fautes purement morales, ne soient renforcés du même coup ; car les seconds ne sont que le prolongement, la forme atténuée des premiers. Ainsi, le vol et la simple indélicatesse ne froissent qu’un seul et même sentiment altruiste, le respect de la propriété d’autrui. Seulement, ce même sentiment est offensé plus faiblement par l’un de ces actes que par l’autre ; et comme, d’autre part, il n’a pas dans la moyenne des consciences une intensité suffisante pour ressentir vivement la plus légère de ces deux offenses, celle-ci est l’objet d’une plus grande tolérance. Voilà pourquoi on blâme simplement l’indélicat tandis que le voleur est puni. Mais si ce même sentiment devient plus fort, au point de faire taire dans toutes les consciences le penchant qui incline l’homme au vol, il deviendra plus sensible aux lésions qui, jusqu’alors, ne le touchaient que légèrement ; il réagira donc contre elles avec plus de vivacité ; elles seront l’objet d’une réprobation plus énergique qui fera passer certaines d’entre elles, de simples fautes morales qu’elles étaient, à l’état de crimes. Par exemple, les contrats indélicats ou indélicatement exécutés, qui n’entraînent qu’un blâme public ou des réparations civiles, deviendront des délits. Imaginez une société de saints, un cloître exemplaire et parfait. Les crimes proprement dits y seront inconnus ; mais les fautes qui paraissent vénielles au vulgaire y soulèveront le même scandale que fait le délit ordinaire auprès des consciences ordinaires. Si donc cette société se trouve armée du pouvoir de juger et de punir, elle qualifiera ces actes de criminels et les traitera comme tels. C’est pour la même raison que le parfait honnête homme juge ses moindres défaillances morales avec une sévérité que la foule réserve aux actes vraiment délictueux. Autrefois, les violences contre les personnes étaient plus fréquentes qu’aujourd’hui parce que le respect pour la dignité individuelle était plus faible. Comme il s’est accru, ces crimes sont devenus plus rares ; mais aussi, bien des actes qui lésaient ce sentiment sont entrés dans le droit pénal dont ils ne relevaient primitivement pas\footnote{ Calomnies, injures, diffamation, dol, etc.}.\par
On se demandera peut-être, pour épuiser toutes les hypothèses logiquement possibles, pourquoi cette unanimité ne s’étendrait pas à tous les sentiments collectifs sans exception ; pourquoi même les plus faibles ne prendraient pas assez d’énergie pour prévenir toute dissidence. La conscience morale de la société se retrouverait tout entière chez tous les individus et avec une vitalité suffisante pour empêcher tout acte qui l’offense, les fautes purement morales aussi bien que les crimes. Mais une uniformité aussi universelle et aussi absolue est radicalement impossible ; car le milieu physique immédiat dans lequel chacun de nous est placé, les antécédents héréditaires, les influences sociales dont nous dépendons varient d’un individu à l’autre et, par suite, diversifient les consciences. Il n’est pas possible que tout le monde se ressemble à ce point, par cela seul que chacun a son organisme propre et que ces organismes occupent des portions différentes de l’espace. C’est pourquoi, même chez les peuples inférieurs, ou l’originalité individuelle est très peu développée, elle n’est cependant pas nulle. Ainsi donc, puisqu’il ne peut pas y avoir de société où les individus ne divergent plus ou moins du type collectif, il est inévitable aussi que, parmi ces divergences, il y en ait qui présentent un caractère criminel. Car ce qui leur confère ce caractère, ce n’est pas leur importance intrinsèque, mais celle que leur prête la conscience commune. Si donc celle-ci est plus forte, si elle a assez d’autorité pour rendre ces divergences très faibles en valeur absolue, elle sera aussi plus sensible, plus exigeante, et, réagissant contre de moindres écarts avec l’énergie qu’elle ne déploie ailleurs que contre des dissidences plus considérables, elle leur attribuera la même gravité, c’est-à-dire qu’elle les marquera comme criminels.\par
Le crime est donc nécessaire ; il est lié aux conditions fondamentales de toute vie sociale, mais, par cela même, il est utile ; car ces conditions dont il est solidaire sont elles-mêmes indispensables à l’évolution normale de la morale et du droit.\par
En effet, il n’est plus possible aujourd’hui de contester que non seulement le droit et la morale varient d’un type social à l’autre, mais encore qu’ils changent pour un même type si les conditions de l’existence collective se modifient. Mais, pour que ces transformations soient possibles, il faut que les sentiments collectifs qui sont à la base de la morale ne soient pas réfractaires au changement, par conséquent, n’aient qu’une énergie modérée. S’ils étaient trop forts, ils ne seraient plus plastiques. Tout arrangement, en effet, est un obstacle au réarrangement, et cela d’autant plus que l’arrangement primitif est plus solide. Plus une structure est fortement accusée, plus elle oppose de résistance à toute modification et il en est des arrangements fonctionnels comme des arrangements anatomiques. Or, s’il n’y avait pas de crimes, cette condition ne serait pas remplie ; car une telle hypothèse suppose que les sentiments collectifs seraient parvenus à un degré d’intensité sans exemple dans l’histoire. Rien n’est bon indéfiniment et sans mesure. Il faut que l’autorité dont jouit la conscience morale ne soit pas excessive ; autrement, nul n’oserait y porter la main et elle se figerait trop facilement sous une forme immuable. Pour qu’elle puisse évoluer, il faut que l’originalité individuelle puisse se faire jour ; or, pour que celle de l’idéaliste qui rêve de dépasser son siècle puisse se manifester, il faut que celle du criminel, qui est au-dessous de son temps, soit possible. L’une ne va pas sans l’autre.\par
Ce n’est pas tout. Outre cette utilité indirecte, il arrive que le crime joue lui-même un rôle utile dans cette évolution. Non seulement il implique que la voie reste ouverte aux changements nécessaires, mais encore, dans certains cas, il prépare directement ces changements. Non seulement, là où il existe, les sentiments collectifs sont dans l’état de malléabilité nécessaire pour prendre une forme nouvelle, mais encore il contribue parfois à prédéterminer la forme qu’ils prendront. Que de fois, en effet, il n’est qu’une anticipation de la morale à venir, un acheminement vers ce qui sera ! D’après le droit athénien, Socrate était un criminel et sa condamnation n’avait rien que de juste. Cependant son crime, à savoir l’indépendance de sa pensée, était utile, non seulement à l’humanité, mais à sa patrie. Car il servait à préparer une morale et une foi nouvelles dont les Athéniens avaient alors besoin parce que les traditions dont ils avaient vécu jusqu’alors n’étaient plus en harmonie avec leurs conditions d’existence. Or le cas de Socrate n’est pas isolé ; il se reproduit périodiquement dans l’histoire. La liberté de penser dont nous jouissons actuellement n’aurait jamais pu être proclamée, si les règles qui la prohibaient n’avaient été violées avant d’être solennellement abrogées. Cependant, à ce moment, cette violation était un crime, puisque c’était une offense à des sentiments encore très vifs dans la généralité des consciences. Et néanmoins ce crime était utile puis qu’il préludait à des transformations qui, de jour en jour, devenaient plus nécessaires. La libre philosophie a eu pour précurseurs les hérétiques de toute sorte que le bras séculier a justement frappés pendant tout le cours du moyen âge et jusqu’à la veille des temps contemporains. De ce point de vue, les faits fondamentaux de la criminologie se présentent à nous sous un aspect entièrement nouveau. Contrairement aux idées courantes, le criminel n’apparaît plus comme un être radicalement insociable, comme une sorte d’élément parasitaire, de corps étranger et inassimilable, introduit au sein de la société\footnote{ Nous avons nous-même commis l’erreur de parler ainsi du criminel, faute d’avoir appliqué notre règle (\emph{Division du travail social}, p. 395. 396).} ; c’est un agent régulier de la vie sociale. Le crime, de son côté, ne doit plus être conçu comme un mal qui ne saurait être contenu dans de trop étroites limites ; mais, bien loin qu’il y ait lieu de se féliciter quand il lui arrive de descendre trop sensiblement au-dessous du niveau ordinaire, on peut être certain que ce progrès apparent est à la fois contemporain et solidaire de quelque perturbation sociale. C’est ainsi que jamais le chiffre des coups et blessures ne tombe aussi bas qu’en temps de disette\footnote{ D’ailleurs, de ce que le crime est un fait de sociologie normale, il ne suit pas qu’il ne faille pas le haïr. La douleur, elle non plus, n’a rien de désirable ; l’individu la hait comme la société hait le crime, et pourtant elle relève de la physiologie normale. Non seulement elle dérive nécessairement de la constitution même de tout être vivant, mais elle joue un rôle utile dans la vie et pour lequel elle ne peut être remplacée. Ce serait donc dénaturer singulièrement notre pensée que de la présenter comme une apologie du crime. Nous ne songerions même pas à protester contre une telle interprétation, si nous ne savions à quelles étranges accusations on s’expose et à quels malentendus, quand on entreprend d’étudier les faits moraux objectivement et d’en parler dans une langue qui n’est pas celle du vulgaire.}. En même temps et par contre-coup, la théorie de la peine se trouve renouvelée ou, plutôt à renouveler. Si, en effet, le crime est une maladie, la peine en est le remède et ne peut être conçue autrement ; aussi toutes les discussions qu’elle soulève portent-elles sur le point de savoir ce qu’elle doit être pour remplir son rôle de remède. Mais si le crime n’a rien de morbide, la peine ne saurait avoir pour objet de le guérir et sa vraie fonction doit être cherchée ailleurs.\par
Il s’en faut donc que les règles précédemment énoncées n’aient d’autre raison d’être que de satisfaire à un formalisme logique sans grande utilité, puisque, au contraire, selon qu’on les applique ou non, les faits sociaux les plus essentiels changent totalement de caractère. Si, d’ailleurs, cet exemple est particulièrement démonstratif — et c’est pourquoi nous avons cru devoir nous y arrêter — il en est bien d’autres qui pourraient être utilement cités. Il n’existe pas de société ou il ne soit de règle que la peine doit être proportionnelle au délit ; cependant, pour l’école italienne, ce principe n’est qu’une invention de juristes, dénuée de toute solidité\footnote{ V. Garofalo, \emph{Criminologie}, p. 299.}. Même, pour ces criminologistes, c’est l’institution pénale tout entière, telle qu’elle a fonctionné jusqu’à présent chez tous les peuples connus, qui est un phénomène contre nature. Nous avons déjà vu que, pour M. Garofalo, la criminalité spéciale aux sociétés inférieures n’a rien de naturel. Pour les socialistes, c’est l’organisation capitaliste, malgré sa généralité, qui constitue une déviation de l’état normal, produite par la violence et l’artifice. Au contraire, pour M. Spencer, c’est notre centralisation administrative, c’est l’extension des pouvoirs gouvernementaux qui est le vice radical de nos sociétés, et cela quoique l’une et l’autre progressent de la manière la plus régulière et la plus universelle à mesure qu’on avance dans l’histoire. Nous ne croyons pas que jamais on se soit systématiquement astreint à décider du caractère normal ou anormal des faits sociaux d’après leur degré de généralité. C’est toujours à grand renfort de dialectique que ces questions sont tranchées.\par
Cependant, ce critère écarté, non seulement on s’expose à des confusions et à des erreurs partielles, comme celles que nous venons de rappeler, mais on rend la science même impossible. En effet, elle a pour objet immédiat l’étude du type normal ; or, si les faits les plus généraux peuvent être morbides, il peut se faire que le type normal n’ait jamais existé dans les faits. Dès lors, que sert de les étudier ? Ils ne peuvent que confirmer nos préjugés et enraciner nos erreurs puisqu’ils en résultent. Si la peine, si la responsabilité, telles qu’elles existent dans l’histoire, ne sont qu’un produit de l’ignorance et de la barbarie, à quoi bon s’attacher à les connaître pour en déterminer les formes normales ? C’est ainsi que l’esprit est amené à se détourner d’une réalité désormais sans intérêt pour se replier sur soi-même et chercher au-dedans de soi les matériaux nécessaires pour la reconstruire. Pour que la sociologie traite les faits comme des choses, il faut que le sociologue sente la nécessité de se mettre à leur école. Or, comme l’objet principal de toute science de la vie, soit individuelle soit sociale, est, en somme, de définir l’état normal, de l’expliquer et de le distinguer de son contraire, si la normalité n’est pas donnée dans les choses mêmes, si elle est, au contraire, un caractère que nous leur imprimons du dehors ou que nous leur refusons pour des raisons quelconques, c’en est fait de cette salutaire dépendance. L’esprit se trouve à l’aise en face du réel qui n’a pas grand’chose à lui apprendre ; il n’est plus contenu par la matière à laquelle il s’applique, puisque c’est lui, en quelque sorte, qui la détermine. Les différentes règles que nous avons établies jusqu’à présent sont donc étroitement solidaires. Pour que la sociologie soit vraiment une science de choses, il faut que la généralité des phénomènes soit prise comme critère de leur normalité.\par
Notre méthode a, d’ailleurs, l’avantage de régler l’action en même temps que la pensée. Si le désirable n’est pas objet d’observation, mais peut et doit être déterminé par une sorte de calcul mental, aucune borne, pour ainsi dire, ne peut être assignée aux libres inventions de l’imagination à la recherche du mieux. Car comment assigner à la perfection un terme qu’elle ne puisse dépasser ? Elle échappe, par définition, à toute limitation. Le but de l’humanité recule donc à l’infini, décourageant les uns par son éloignement même, excitant, au contraire, et enfiévrant les autres, qui, pour s’en rapprocher un peu, pressent le pas et se précipitent dans les révolutions. On échappe à ce dilemme pratique si le désirable, c’est la santé, et si la santé est quelque chose de défini et de donné dans les choses, car le terme de l’effort est donné et défini du même coup. Il ne s’agit plus de poursuivre désespérément une fin qui fuit à mesure qu’on avance, mais de travailler avec une régulière persévérance à maintenir l’état normal, à le rétablir s’il est troublé, à en retrouver les conditions si elles viennent à changer. Le devoir de l’homme d’État n’est plus de pousser violemment les sociétés vers un idéal qui lui paraît séduisant, mais son rôle est celui du médecin : il prévient l’éclosion des maladies par une bonne hygiène et, quand elles sont déclarées, il cherche à les guérir\footnote{ De la théorie développée dans ce chapitre on a quelquefois conclu que, suivant nous, la marche ascendante de la criminalité au cours du \textsc{xix}\textsuperscript{e} siècle était un phénomène normal. Rien n`est plus éloigné de notre pensée. Plusieurs faits que nous avons indiqués à propos du suicide (voir \emph{Le Suicide}, p. 420 et suiv.) tendent, au contraire, à nous faire croire que ce développement est, en général, morbide. Toutefois, il pourrait se faire qu’un certain accroissement de certaines formes de la criminalité fût normal, car chaque état de civilisation a sa criminalité propre. Mais on ne peut faire là-dessus que des hypothèses.}.
\chapterclose


\chapteropen
\chapter[{Chapitre IV : Règles relatives à la constitution des types sociaux}]{Chapitre IV : \\
Règles relatives à la constitution des types sociaux}\renewcommand{\leftmark}{Chapitre IV : \\
Règles relatives à la constitution des types sociaux}


\chaptercont
\noindent Puisqu’un fait social ne peut être qualifié de normal ou d’anormal que par rapport à une espèce sociale déterminée, ce qui précède implique qu’une branche de la sociologie est consacrée à la constitution de ces espèces et à leur classification.\par
Cette notion de l’espèce sociale a, d’ailleurs, le très grand avantage de nous fournir un moyen terme entre les deux conceptions contraires de la vie collective qui se sont, pendant longtemps, partagé les esprits ; je veux dire le nominalisme des historiens\footnote{ Je l’appelle ainsi, parce qu’il a été fréquent chez les historiens, mais je ne veux pas dire qu’il se retrouve chez tous.} et le réalisme extrême des philosophes. Pour l’historien, les sociétés constituent autant d’individualités hétérogènes, incomparables entre elles. Chaque peuple a sa physionomie, sa constitution spéciale, son droit, sa morale, son organisation économique qui ne conviennent qu’à lui, et toute généralisation est à peu près impossible. Pour le philosophe, au contraire, tous ces groupements particuliers, que l’on appelle les tribus, les cités, les nations, ne sont que des combinaisons contingentes et provisoires sans réalité propre. Il n’y a de réel que l’humanité et c’est des attributs généraux de la nature humaine que découle toute l’évolution sociale. Pour les premiers, par conséquent, l’histoire n’est qu’une suite d’événements qui s’enchaînent sans se reproduire ; pour les seconds, ces mêmes événements n’ont de valeur et d’intérêt que comme illustration des lois générales qui sont inscrites dans la constitution de l’homme et qui dominent tout le développement historique. Pour ceux-là, ce qui est bon pour une société ne saurait s’appliquer aux autres. Les conditions de l’état de santé varient d’un peuple à l’autre et ne peuvent être déterminées théoriquement ; c’est affaire de pratique, d’expérience, de tâtonnements. Pour les autres, elles peuvent être calculées une fois pour toutes et pour le genre humain tout entier. Il semblait donc que la réalité sociale ne pouvait être l’objet que d’une philosophie abstraite et vague ou de monographies purement descriptives. Mais on échappe à cette alternative une fois qu’on a reconnu qu’entre la multitude confuse des sociétés historiques et le concept unique, mais idéal, de l’humanité, il y a des intermédiaires : ce sont les espèces sociales. Dans l’idée d’espèce, en effet, se trouvent réunies et l’unité qu’exige toute recherche vraiment scientifique et la diversité qui est donnée dans les faits, puisque l’espèce se retrouve la même chez tous les individus qui en font partie et que, d’autre part, les espèces diffèrent entre elles. Il reste vrai que les institutions morales, juridiques, économiques, etc., sont infiniment variables, mais ces variations ne sont pas de telle nature qu’elles n’offrent aucune prise à la pensée scientifique.\par
C’est pour avoir méconnu l’existence d’espèces sociales que Comte a cru pouvoir représenter le progrès des sociétés humaines comme identique à celui d’un peuple unique \emph{« auquel seraient idéalement rapportées toutes les modifications consécutives observées chez les populations distinctes\footnote{\emph{Cours de philos. pos.}, IV, 263.} »}. C’est qu’en effet, s’il n’existe qu’une seule espèce sociale, les sociétés particulières ne peuvent différer entre elles qu’en degrés, suivant qu’elles présentent plus ou moins complètement les traits constitutifs de cette espèce unique, suivant qu’elles expriment plus ou moins parfaitement l’humanité. Si, au contraire, il existe des types sociaux qualitativement distincts les uns des autres, on aura beau les rapprocher, on ne pourra pas faire qu’ils se rejoignent exactement comme les sections homogènes d’une droite géométrique. Le développement historique perd ainsi l’unité idéale et simpliste qu’on lui attribuait ; il se fragmente, pour ainsi dire, en une multitude de tronçons qui, parce qu’ils diffèrent spécifiquement les uns des autres, ne sauraient se relier d’une manière continue. La fameuse métaphore de Pascal, reprise depuis par Comte, se trouve désormais sans vérité.\par
Mais comment faut-il s’y prendre pour constituer ces espèces ?\par
\section[{I}]{I}
\noindent Il peut sembler, au premier abord, qu’il n’y ait pas d’autre manière de procéder que d’étudier chaque société en particulier, d’en faire une monographie aussi exacte et aussi complète que possible, puis de comparer toutes ces monographies entre elles, de voir par où elles concordent et par où elles divergent, et alors, suivant l’importance relative de ces similitudes et de ces divergences, de classer les peuples dans des groupes semblables ou différents. À l’appui de cette méthode, on fait remarquer qu’elle seule est recevable dans une science d’observation. L’espèce, en effet, n’est que le résumé des individus ; comment donc la constituer, si l’on ne commence pas par décrire chacun d’eux et par le décrire tout entier ? N’est-ce pas une règle de ne s’élever au général qu’après avoir observé le particulier et tout le particulier ? C’est pour cette raison que l’on a voulu parfois ajourner la sociologie jusqu’à l’époque indéfiniment éloignée où l’histoire, dans l’étude qu’elle fait des sociétés particulières, sera parvenue à des résultats assez objectifs et définis pour pouvoir être utilement comparés.\par
Mais, en réalité, cette circonspection n’a de scientifique que l’apparence. Il est inexact, en effet, que la science ne puisse instituer de lois qu’après avoir passé en revue tous les faits qu’elles expriment, ni former de genres qu’après avoir décrit, dans leur intégralité, les individus qu’ils comprennent, La vraie méthode expérimentale tend plutôt à substituer aux faits vulgaires, qui ne sont démonstratifs qu’à condition d’être très nombreux et qui, par suite, ne permettent que des conclusions toujours suspectes, des faits {\itshape décisifs} ou {\itshape cruciaux}, comme disait Bacon\footnote{\emph{Novum Organum}, II, § 36.}, qui, par eux-mêmes et indépendamment de leur nombre, ont une valeur et un intérêt scientifiques. Il est surtout nécessaire de procéder ainsi quand il s’agit de constituer des genres et des espèces. Car faire l’inventaire de tous les caractères qui appartiennent à un individu est un problème insoluble. Tout individu est un infini et l’infini ne peut être épuisé. S’en tiendra-t-on aux propriétés les plus essentielles ? Mais d’après quel principe fera-t-on le triage ? Il faut pour cela un critère qui dépasse l’individu et que les monographies les mieux faites ne sauraient, par conséquent, nous fournir. Sans même pousser les choses à cette rigueur, on peut prévoir que, plus les caractères qui serviront de base à la classification seront nombreux, plus aussi il sera difficile que les diverses manières dont ils se combinent dans les cas particuliers présentent des ressemblances assez franches et des différences assez tranchées pour permettre la constitution de groupes et de sous-groupes définis.\par
Mais quand même une classification serait possible d’après cette méthode, elle aurait le très grand défaut de ne pas rendre les services qui en sont la raison d’être. En effet, elle doit, avant tout, avoir pour objet d’abréger le travail scientifique en substituant à la multiplicité indéfinie des individus un nombre restreint de types. Mais elle perd cet avantage si ces types n’ont été constitués qu’après que tous les individus ont été passés en revue et analysés tout entiers. Elle ne peut guère faciliter la recherche, si elle ne fait que résumer les recherches déjà faites. Elle ne sera vraiment utile que si elle nous permet de classer d’autres caractères que ceux qui lui servent de base, que si elle nous procure des cadres pour les faits à venir. Son rôle est de nous mettre en mains des points de repère auxquels nous puissions rattacher d’autres observations que celles qui nous ont fourni ces points de repère eux-mêmes. Mais, pour cela, il faut qu’elle soit faite, non d’après un inventaire complet de tous les caractères individuels, mais d’après un petit nombre d’entre eux, soigneusement choisis. Dans ces conditions, elle ne servira pas seulement à mettre un peu d’ordre dans des connaissances toutes faites ; elle servira à en faire. Elle épargnera à l’observateur bien des démarches parce qu’elle le guidera. Ainsi, une fois la classification établie sur ce principe, pour savoir si un fait est général dans une espèce, il ne sera pas nécessaire d’avoir observé toutes les sociétés de cette espèce ; quelques-unes suffiront. Même, dans bien des cas, ce sera assez d’une observation bien faite, de même que, souvent, une expérience bien conduite suffit à l’établissement d’une loi.\par
Nous devons donc choisir pour notre classification des caractères particulièrement essentiels. Il est vrai qu’on ne peut les connaître que si l’explication des faits est suffisamment avancée. Ces deux parties de la science sont solidaires et progressent l’une par l’autre. Cependant, sans entrer très avant dans l’étude des faits, il n’est pas difficile de conjecturer de quel côté il faut chercher les propriétés caractéristiques des types sociaux. Nous savons, en effet, que les sociétés sont composées de parties ajoutées les unes aux autres. Puisque la nature de toute résultante dépend nécessairement de la nature, du nombre des éléments composants et de leur mode de combinaison, ces caractères sont évidemment ceux que nous devons prendre pour base, et on verra, en effet, dans la suite, que c’est d’eux que dépendent les faits généraux de la vie sociale. D’autre part, comme ils sont d’ordre morphologique, on pourrait appeler \emph{Morphologie sociale} la partie de la sociologie qui a pour tâche de constituer et de classer les types sociaux.\par
On peut même préciser davantage le principe de cette classification. On sait, en effet, que ces parties constitutives dont est formée toute société sont des sociétés plus simples qu’elle. Un peuple est produit par la réunion de deux ou plusieurs peuples qui l’ont précédé. Si donc nous connaissions la société la plus simple qui ait jamais existé, nous n’aurions, pour faire notre classification, qu’à suivre la manière dont cette société se compose avec elle-même et dont ses composés se composent entre eux.
\section[{II}]{II}
\noindent M. Spencer a fort bien compris que la classification méthodique des types sociaux ne pouvait avoir d’autre fondement.\par
\emph{« Nous avons vu, dit-il, que l’évolution sociale commence par de petits agrégats simples ; qu’elle progresse par l’union de quelques-uns de ces agrégats en agrégats plus grands, et qu’après s’être consolidés, ces groupes s’unissent avec d’autres semblables à eux pour former des agrégats encore plus grands. Notre classification doit donc commencer par des sociétés du premier ordre, c’est-à-dire du plus simple\footnote{\emph{Sociologie}, II, 135.}. »}\par
Malheureusement, pour mettre ce principe en pratique, il faudrait commencer par définir avec précision ce que l’on entend par société simple. Or, cette définition, non seulement M. Spencer ne la donne pas, mais il la juge à peu près impossible\footnote{\emph{ « Nous ne pouvons pas toujours dire avec précision ce qui constitue une société simple. »} ({\itshape Ibid.}, 135, 136.)}. C’est que, en effet, la simplicité, comme il l’entend, consiste essentiellement dans une certaine grossièreté d’organisation. Or il n’est pas facile de dire avec exactitude à quel moment l’organisation sociale est assez rudimentaire pour être qualifiée de simple ; c’est affaire d’appréciation. Aussi la formule qu’il en donne est-elle tellement flottante qu’elle convient à toute sorte de sociétés. \emph{« Nous n’avons rien de mieux à faire, dit-il, que de considérer comme une société simple celle qui forme un tout non assujetti à un autre et dont les parties coopèrent, avec ou sans centre régulateur, en vue de certaines fins d’intérêt public\footnote{{\itshape Ibid.}, 136.}. »} Mais il y a nombre de peuples qui satisfont à cette condition. Il en résulte qu’il confond, un peu au hasard, sous cette même rubrique toutes les sociétés les moins civilisées. On imagine ce que peut être, avec un pareil point de départ, tout le reste de sa classification. On y voit rapprochées, dans la plus étonnante confusion, les sociétés les plus disparates, les Grecs homériques mis à côté des fiefs du \textsc{x}\textsuperscript{e} siècle et au-dessous des Bechuanas, des Zoulous et des Fidjiens, la confédération athénienne à côté des fiefs de la France du \textsc{xiii}\textsuperscript{e} siècle et au-dessous des Iroquois et des Araucaniens.\par
Le mot de simplicité n’a de sens défini que s’il signifie une absence complète de parties. Par société simple, il faut donc entendre toute société qui n’en renferme pas d’autres, plus simples qu’elle ; qui non seulement est actuellement réduite à un segment unique, mais encore qui ne présente aucune trace d’une segmentation antérieure. La {\itshape horde}, telle que nous l’avons définie ailleurs\footnote{\emph{Division du travail social}, p. 189.}, répond exactement à cette définition. C’est un agrégat social qui ne comprend et n’a jamais compris dans son sein aucun autre agrégat plus élémentaire, mais qui se résout immédiatement en individus. Ceux-ci ne forment pas, à l’intérieur du groupe total, des groupes spéciaux et différents du précédent ; ils sont juxtaposés atomiquement. On conçoit qu’il ne puisse pas y avoir de société plus simple ; c’est le protoplasme du règne social et, par conséquent, la base naturelle de toute classification.\par
Il est vrai qu’il n’existe peut-être pas de société historique qui réponde exactement à ce signalement ; mais, ainsi que nous l’avons montré dans le livre déjà cité, nous en connaissons une multitude qui sont formées, immédiatement et sans autre intermédiaire, par une répétition de hordes. Quand la horde devient ainsi un segment social au lieu d’être la société tout entière, elle change de nom, elle s’appelle le clan ; mais elle garde les mêmes traits constitutifs. Le clan est, en effet, un agrégat social qui ne se résout en aucun autre, plus restreint. On fera peut-être remarquer que, généralement, là où nous l’observons aujourd’hui, il renferme une pluralité de familles particulières. Mais, d’abord, pour des raisons que nous ne pouvons développer ici, nous croyons que la formation de ces petits groupes familiaux est postérieure au clan ; puis, elles ne constituent pas, à parler exactement, des segments sociaux parce qu’elles ne sont pas des divisions politiques. Partout où on le rencontre, le clan constitue l’ultime division de ce genre. Par conséquent, quand même nous n’aurions pas d’autres faits pour postuler l’existence de la horde — et il en est que nous aurons un jour l’occasion d’exposer — l’existence du clan, c’est-à-dire de sociétés formées par une réunion de hordes, nous autorise à supposer qu’il y a eu d’abord des sociétés plus simples qui se réduisaient à la horde proprement dite, et à faire de celle-ci la souche d’où sont sorties toutes les espèces sociales.\par
Une fois posée cette notion de la horde ou société à segment unique — qu’elle soit conçue comme une réalité historique ou comme un postulat de la science — on a le point d’appui nécessaire pour construire l’échelle complète des types sociaux. On distinguera autant de types fondamentaux qu’il y a de manières, pour la horde, de se combiner avec elle-même en donnant naissance à des sociétés nouvelles et, pour celles-ci, de se combiner entre elles. On rencontrera d’abord des agrégats formés par une simple répétition de hordes ou de clans (pour leur donner leur nom nouveau), sans que ces clans soient associés entre eux de manière à former des groupes intermédiaires entre le groupe total qui les comprend tous, et chacun d’eux. Ils sont simplement juxtaposés comme les individus de la horde. On trouve des exemples de ces sociétés que l’on pourrait appeler {\itshape polysegmentaires simples} dans certaines tribus iroquoises et australiennes. L’{\itshape arch} ou tribu kabyle, a le même caractère ; c’est une réunion de clans fixés sous forme de villages. Très vraisemblablement, il y eut un moment dans l’histoire ou la {\itshape curie} romaine, la {\itshape phratrie} athénienne était une société de ce genre. Au-dessus, viendraient les sociétés formées par un assemblage de sociétés de l’espèce précédente, c’est-à-dire les {\itshape sociétés polysegmentaires simplement composées}. Tel est le caractère de la confédération iroquoise, de celle formée par la réunion des tribus kabyles ; il en fut de même, à l’origine, de chacune des trois tribus primitives dont l’association donna, plus tard, naissance à la cité romaine. On rencontrerait ensuite les sociétés {\itshape polysegmentaires doublement composées} qui résultent de la juxtaposition ou fusion de plusieurs sociétés polysegmentaires simplement composées. Telles sont la cité, agrégat de tribus, qui sont elles-mêmes des agrégats de curies qui, à leur tour, se résolvent en {\itshape gentes} ou clans, et la tribu germanique, avec ses comtés qui se subdivisent en centaines, lesquelles, à leur tour, ont pour unité dernière le clan devenu village.\par
Nous n’avons pas à développer davantage ni à pousser plus loin ces quelques indications, puisqu’il ne saurait être question d’exécuter ici une classification des sociétés. C’est un problème trop complexe pour pouvoir être traité ainsi, comme en passant ; il suppose, au contraire, tout un ensemble de longues et spéciales recherches. Nous avons seulement voulu, par quelques exemples, préciser les idées et montrer comment doit être appliqué le principe de la méthode. Même il ne faudrait pas considérer ce qui précède comme constituant une classification complète des sociétés inférieures. Nous y avons quelque peu simplifié les choses pour plus de clarté. Nous avons supposé, en effet, que chaque type supérieur était formé par une répétition de sociétés d’un même type, à savoir du type immédiatement inférieur. Or, il n’y a rien d’impossible à ce que des sociétés d’espèces différentes, situées inégalement haut sur l’arbre généalogique des types sociaux, se réunissent de manière à former une espèce nouvelle. On en connaît au moins un cas ; c’est l’Empire romain, qui comprenait dans son sein les peuples les plus divers de nature\footnote{ Toutefois il est vraisemblable que, en général, la distance entre les sociétés composantes ne saurait être très grande ; autrement, il ne pourrait y avoir entre elles aucune communauté morale.}.\par
Mais une fois ces types constitués, il y aura lieu de distinguer dans chacun d’eux des variétés différentes selon que les sociétés segmentaires, qui servent à former la société résultante, gardent une certaine individualité, ou bien, au contraire, sont absorbées dans la masse totale. On comprend en effet que les phénomènes sociaux doivent varier, non pas seulement suivant la nature des éléments composants, mais suivant leur mode de composition ; ils doivent surtout être très différents suivant que chacun des groupes partiels garde sa vie locale ou qu’ils sont tous entraînés dans la vie générale, c’est-à-dire suivant qu’ils sont plus ou moins étroitement concentrés. On devra, par conséquent, rechercher si, à un moment quelconque, il se produit une coalescence complète de ces segments. On reconnaîtra qu’elle existe à ce signe que cette composition originelle de la société n’affecte plus son organisation administrative et politique. À ce point de vue, la cité se distingue nettement des tribus germaniques. Chez ces dernières l’organisation à base de clans s’est maintenue, quoique effacée, jusqu’au terme de leur histoire, tandis que, à Rome, à Athènes, les {\itshape gentes} et les γένη cessèrent très tôt d’être des divisions politiques pour devenir des groupements privés.\par
À l’intérieur des cadres ainsi constitués, on pourra chercher à introduire de nouvelles distinctions d’après des caractères morphologiques secondaires. Cependant, pour des raisons que nous donnerons plus loin, nous ne croyons guère possible de dépasser utilement les divisions générales qui viennent d’être indiquées. Au surplus, nous n’avons pas à entrer dans ces détails, il nous suffit d’avoir posé le principe de la classification qui peut être énoncé ainsi : {\itshape On commencera par classer les sociétés d’après le degré de composition qu’elles présentent, en prenant pour base la société parfaitement simple ou à segment unique ; à l’intérieur de ces classes, on distinguera des variétés différentes suivant qu’il se produit ou non une coalescence complète des segments initiaux}.
\section[{III}]{III}
\noindent Ces règles répondent implicitement à une question que le lecteur s’est peut-être posée en nous voyant parler d’espèces sociales comme s’il y en avait, sans en avoir directement établi l’existence. Cette preuve est contenue dans le principe même de la méthode qui vient d’être exposée.\par
Nous venons de voir, en effet, que les sociétés n’étaient que des combinaisons différentes d’une seule et même société originelle. Or, un même élément ne peut se composer avec lui-même et les composés qui en résultent ne peuvent, à leur tour, se composer entre eux que suivant un nombre de modes limité, surtout quand les éléments composants sont peu nombreux ; ce qui est le cas des segments sociaux. La gamme des combinaisons possibles est donc finie et, par suite, la plupart d’entre elles, tout au moins, doivent se répéter. Il se trouve ainsi qu’il y a des espèces sociales. Il reste, d’ailleurs, possible que certaines de ces combinaisons ne se produisent qu’une seule fois. Cela n’empêche pas qu’il y ait des espèces. On dira seulement dans les cas de ce genre que l’espèce ne compte qu’un individu\footnote{ N’est-ce pas le cas de l’empire romain, qui paraît bien être sans analogue dans l’histoire ?}.\par
Il y a donc des espèces sociales pour la même raison qui fait qu’il y a des espèces en biologie. Celles-ci, en effet, sont dues à ce fait que les organismes ne sont que des combinaisons variées d’une seule et même unité anatomique. Toutefois, il y a, à ce point de vue, une grande différence entre les deux règnes. Chez les animaux, en effet, un facteur spécial vient donner aux caractères spécifiques une force de résistance que n’ont pas les autres ; c’est la génération. Les premiers, parce qu’ils sont communs à toute la lignée des ascendants, sont bien plus fortement enracinés dans l’organisme. Ils ne se laissent donc pas facilement entamer par l’action des milieux individuels, mais se maintiennent, identiques à eux-mêmes, malgré la diversité des circonstances extérieures. Il y a une force interne qui les fixe en dépit des sollicitations à varier qui peuvent venir du dehors ; c’est la force des habitudes héréditaires. C’est pourquoi ils sont nettement définis et peuvent être déterminés avec précision. Dans le règne social, cette cause interne leur fait défaut. Ils ne peuvent être renforcés par la génération parce qu’ils ne durent qu’une génération. Il est de règle, en effet, que les sociétés engendrées soient d’une autre espèce que les sociétés génératrices, parce que ces dernières, en se combinant, donnent naissance à des arrangements tout à fait nouveaux. Seule, la colonisation pourrait être comparée à une génération par germination ; encore, pour que l’assimilation soit exacte, faut-il que le groupe des colons n’aille pas se mêler à quelque société d’une autre espèce ou d’une autre variété. Les attributs distinctifs de l’espèce ne reçoivent donc pas de l’hérédité un surcroît de force qui lui permette de résister aux variations individuelles. Mais ils se modifient et se nuancent à l’infini sous l’action des circonstances ; aussi, quand on veut les atteindre, une fois qu’on a écarté toutes les variantes qui les voilent, n’obtient-on souvent qu’un résidu assez indéterminé. Cette indétermination croît naturellement d’autant plus que la complexité des caractères est plus grande ; car plus une chose est complexe, plus les parties qui la composent peuvent former de combinaisons différentes. Il en résulte que le type spécifique, au-delà des caractères les plus généraux et les plus simples, ne présente pas de contours aussi définis qu’en biologie\footnote{ En rédigeant ce chapitre pour la première édition de cet ouvrage, nous n’avons rien dit de la méthode qui consiste à classer les sociétés d’après leur état de civilisation. À ce moment, en effet, il n’existait pas de classifications de ce genre qui fussent proposées par des sociologues autorisés, sauf peut-être celle, trop évidemment archaïque, de Comte. Depuis, plusieurs essais ont été faits dans ce sens, notamment par Vierkandt (\emph{Die Kulturtypen der Menschheit}, in \emph{Archiv. f. Anthropologie}, 1898), par Sutherland (\emph{The Origin and Growth of the Moral Instinct}), et par Steinmetz (\emph{Classification des types sociaux} in \emph{Année sociologique}, III, p. 43-147). Néanmoins, nous ne nous arrêterons pas à les discuter, car ils ne répondent pas au problème posé dans ce chapitre. On y trouve classées, non des espèces sociales, mais, ce qui est bien différent, des phases historiques. La France, depuis ses origines, a passé par des formes de civilisation très différentes ; elle a commencé par être agricole, pour passer ensuite à l’industrie des métiers et au petit commerce, puis à la manufacture et enfin à la grande industrie. Or il est impossible d’admettre qu’une même individualité collective puisse changer d’espèce trois ou quatre fois. Une espèce doit se définir par des caractères plus constants. L’état économique, technologique, etc., présente des phénomènes trop instables et trop complexes pour fournir la base d’une classification. Il est même très possible qu’une même civilisation industrielle, scientifique, artistique puisse se rencontrer dans des sociétés dont la constitution congénitale est très différente. Le Japon pourra nous emprunter nos arts, notre industrie, même notre organisation politique ; il ne laissera pas d’appartenir à une autre espèce sociale que la France et l’Allemagne. Ajoutons que ces tentatives, quoique conduites par des sociologues de valeur, n’ont donné que des résultats vagues, contestables et de peu d’utilité.}.
\chapterclose


\chapteropen
\chapter[{Chapitre V : Règles relatives à l’explication des faits sociaux}]{Chapitre V : \\
Règles relatives à l’explication des faits sociaux}\renewcommand{\leftmark}{Chapitre V : \\
Règles relatives à l’explication des faits sociaux}


\chaptercont
\noindent Mais la constitution des espèces est avant tout un moyen de grouper les faits pour en faciliter l’interprétation ; la morphologie sociale est un acheminement à la partie vraiment explicative de la science. Quelle est la méthode propre de cette dernière ?\par
\section[{I}]{I}
\noindent La plupart des sociologues croient avoir rendu compte des phénomènes une fois qu’ils ont fait voir à quoi ils servent, quel rôle ils jouent. On raisonne comme s’ils n’existaient qu’en vue de ce rôle et n’avaient d’autre cause déterminante que le sentiment, clair ou confus, des services qu’ils sont appelés à rendre. C’est pourquoi on croit avoir dit tout ce qui est nécessaire pour les rendre intelligibles, quand on a établi la réalité de ces services et montré à quel besoin social ils apportent satisfaction. C’est ainsi que Comte ramène toute la force progressive de l’espèce humaine à cette tendance fondamentale \emph{« qui pousse directement l’homme à améliorer sans cesse sous tous les rapports sa condition quelconque\footnote{\emph{Cours de philos. pos.}, IV, 262.} »}, et M. Spencer, au besoin d’un plus grand bonheur. C’est en vertu de ce principe qu’il explique la formation de la société par les avantages qui résultent de la coopération, l’institution du gouvernement par l’utilité qu’il y a à régulariser la coopération militaire\footnote{\emph{Sociologie}, III, 336.}, les transformations par lesquelles a passé la famille par le besoin de concilier de plus en plus parfaitement les intérêts des parents, des enfants et de la société.\par
Mais cette méthode confond deux questions très différentes. Faire voir à quoi un fait est utile n’est pas expliquer comment il est né ni comment il est ce qu’il est. Car les emplois auxquels il sert supposent les propriétés spécifiques qui le caractérisent, mais ne les créent pas. Le besoin que nous avons des choses ne peut pas faire qu’elles soient telles ou telles et, par conséquent, ce n’est pas ce besoin qui peut les tirer du néant et leur conférer l’être. C’est de causes d’un autre genre qu’elles tiennent leur existence. Le sentiment que nous avons de l’utilité qu’elles présentent peut bien nous inciter à mettre ces causes en œuvre et à en tirer les effets qu’elles impliquent, non à susciter ces effets de rien. Cette proposition est évidente tant qu’il ne s’agit que des phénomènes matériels ou même psychologiques. Elle ne serait pas plus contestée en sociologie si les faits sociaux, à cause de leur extrême immatérialité, ne nous paraissaient, à tort, destitués de toute réalité intrinsèque. Comme on n’y voit que des combinaisons purement mentales, il semble qu’ils doivent se produire d’eux-mêmes dès qu’on en a l’idée, si, du moins, on les trouve utiles. Mais puisque chacun d’eux est une force et qui domine la nôtre, puisqu’il a une nature qui lui est propre, il ne saurait suffire, pour lui donner l’être, d’en avoir le désir ni la volonté. Encore faut-il que des forces capables de produire cette force déterminée, que des natures capables de produire cette nature spéciale, soient données. C’est à cette condition seulement qu’il sera possible. Pour ranimer l’esprit de famille là où il est affaibli, il ne suffit pas que tout le monde en comprenne les avantages ; il faut faire directement agir les causes qui, seules, sont susceptibles de l’engendrer. Pour rendre à un gouvernement l’autorité qui lui est nécessaire, il ne suffit pas d’en sentir le besoin ; il faut s’adresser aux seules sources d’où dérive toute autorité, c’est-à-dire constituer des traditions, un esprit commun, etc., etc. ; pour cela, il faut encore remonter plus haut la chaîne des causes et des effets, jusqu’à ce qu’on trouve un point où l’action de l’homme puisse s’insérer efficacement.\par
Ce qui montre bien la dualité de ces deux ordres de recherches, c’est qu’un fait peut exister sans servir à rien, soit qu’il n’ait jamais été ajusté à aucune fin vitale, soit que, après avoir été utile, il ait perdu toute utilité en continuant à exister par la seule force de l’habitude. Il y a, en effet encore plus de survivances dans la société que dans l’organisme. Il y a même des cas où soit une pratique, soit une institution sociale changent de fonctions sans, pour cela, changer de nature. La règle \emph{is pater est quem justae nuptiae declarant} est matériellement restée dans notre code ce qu’elle était dans le vieux droit romain. Mais, tandis qu’alors elle avait pour objet de sauvegarder les droits de propriété du père sur les enfants issus de la femme légitime, c’est bien plutôt le droit des enfants qu’elle protège aujourd’hui. Le serment a commencé par être une sorte d’épreuve judiciaire pour devenir simplement une forme solennelle et imposante du témoignage. Les dogmes religieux du christianisme n’ont pas changé depuis des siècles ; mais le rôle qu’ils jouent dans nos sociétés modernes n’est plus le même qu’au moyen âge. C’est ainsi encore que les mots servent à exprimer des idées nouvelles sans que leur contexture change. C’est, du reste, une proposition vraie en sociologie comme en biologie que l’organe est indépendant de la fonction, c’est-à-dire que, tout en restant le même, il peut servir à des fins différentes. C’est donc que les causes qui le font être sont indépendantes des fins auxquelles il sert.\par
Nous n’entendons pas dire, d’ailleurs, que les tendances, les besoins, les désirs des hommes n’interviennent jamais, d’une manière active, dans l’évolution sociale. Il est, au contraire, certain qu’il leur est possible, suivant la manière dont ils se portent sur les conditions dont dépend un fait, d’en presser ou d’en contenir le développement. Seulement, outre qu’ils ne peuvent, en aucun cas, faire quelque chose de rien, leur intervention elle-même, quels qu’en soient les effets, ne peut avoir lieu qu’en vertu de causes efficientes. En effet, une tendance ne peut concourir, même dans cette mesure restreinte, à la production d’un phénomène nouveau que si elle est nouvelle elle-même, qu’elle se soit constituée de toutes pièces ou qu’elle soit due à quelque transformation d’une tendance antérieure. Car, à moins de postuler une harmonie préétablie vraiment providentielle, on ne saurait admettre que, dès l’origine, l’homme portât en lui à l’état virtuel, mais toutes prêtes à s’éveiller à l’appel des circonstances, toutes les tendances dont l’opportunité devait se faire sentir dans la suite de l’évolution. Or une tendance est, elle aussi, une chose ; elle ne peut donc ni se constituer ni se modifier par cela seul que nous le jugeons utile. C’est une force qui a sa nature propre ; pour que cette nature soit suscitée ou altérée, il ne suffit pas que nous y trouvions quelque avantage. Pour déterminer de tels changements, il faut que des causes agissent qui les impliquent physiquement.\par
Par exemple, nous avons expliqué les progrès constants de la division du travail social en montrant qu’ils sont nécessaires pour que l’homme puisse se maintenir dans les nouvelles conditions d’existence où il se trouve placé à mesure qu’il avance dans l’histoire ; nous avons donc attribué à cette tendance, qu’on appelle assez improprement l’instinct de conservation, un rôle important dans notre explication. Mais, en premier lieu, elle ne saurait à elle seule rendre compte de la spécialisation même la plus rudimentaire. Car elle ne peut rien si les conditions dont dépend ce phénomène ne sont pas déjà réalisées, c’est-à-dire si les différences individuelles ne se sont pas suffisamment accrues par suite de l’indétermination progressive de la conscience commune et des influences héréditaires\footnote{\emph{Division du travail}, l. II, ch. III et IV.}. Même il fallait que la division du travail eût déjà commencé d’exister pour que l’utilité en fût aperçue et que le besoin s’en fit sentir ; et le seul développement des divergences individuelles, en impliquant une plus grande diversité de goûts et d’aptitudes, devait nécessairement produire ce premier résultat. Mais de plus, ce n’est pas de soi-même et sans cause que l’instinct de conservation est venu féconder ce premier germe de spécialisation. S’il s’est orienté et nous a orientés dans cette voie nouvelle, c’est, d’abord, que la voie qu’il suivait et nous faisait suivre antérieurement s’est trouvée comme barrée, parce que l’intensité plus grande de la lutte, due à la condensation plus grande des sociétés, a rendu de plus en plus difficile la survie des individus qui continuaient à se consacrer à des tâches générales. Il a été ainsi nécessité à changer de direction. D’autre part, s’il s’est tourné et a tourné de préférence notre activité dans le sens d’une division du travail toujours plus développée, c’est que c’était aussi le sens de la moindre résistance. Les autres solutions possibles étaient l’émigration, le suicide, le crime. Or, dans la moyenne des cas, les liens qui nous attachent à notre pays, à la vie, la sympathie que nous avons pour nos semblables sont des sentiments plus forts et plus résistants que les habitudes qui peuvent nous détourner d’une spécialisation plus étroite. C’est donc ces dernières qui devaient inévitablement céder à chaque poussée qui s’est produite. Ainsi on ne revient pas, même partiellement, au finalisme parce qu’on ne se refuse pas à faire une place aux besoins humains dans les explications sociologiques. Car ils ne peuvent avoir d’influence sur l’évolution sociale qu’à condition d’évoluer eux-mêmes, et les changements par lesquels ils passent ne peuvent être expliqués que par des causes qui n’ont rien de final.\par
Mais ce qui est plus convaincant encore que les considérations qui précèdent, c’est la pratique même des faits sociaux. Là où règne le finalisme, règne aussi une plus ou moins large contingence ; car il n’est pas de fins, et moins encore de moyens, qui s’imposent nécessairement à tous les hommes, même quand on les suppose placés dans les mêmes circonstances. Étant donné un même milieu, chaque individu, suivant son humeur, s’y adapte à sa manière qu’il préfère à toute autre. L’un cherchera à le changer pour le mettre en harmonie avec ses besoins ; l’autre aimera mieux se changer soi-même et modérer ses désirs, et, pour arriver à un même but, que de voies différentes peuvent être et sont effectivement suivies ! Si donc il était vrai que le développement historique se fit en vue de fins clairement ou obscurément senties, les faits sociaux devraient présenter la plus infinie diversité et toute comparaison presque devrait se trouver impossible. Or c’est le contraire qui est la vérité. Sans doute, les événements extérieurs dont la trame constitue la partie superficielle de la vie sociale varient d’un peuple à l’autre. Mais c’est ainsi que chaque individu a son histoire, quoique les bases de l’organisation physique et morale soient les mêmes chez tous. En fait, quand on est entré quelque peu en contact avec les phénomènes sociaux, on est, au contraire, surpris de l’étonnante régularité avec laquelle ils se reproduisent dans les mêmes circonstances. Même les pratiques les plus minutieuses et, en apparence, les plus puériles, se répètent avec la plus étonnante uniformité. Telle cérémonie nuptiale, purement symbolique à ce qu’il semble, comme l’enlèvement de la fiancée, se retrouve exactement partout où existe un certain type familial, lié lui-même à toute une organisation politique. Les usages les plus bizarres, comme la couvade, le lévirat, l’exogamie, etc., s’observent chez les peuples les plus divers et sont symptomatiques d’un certain état social. Le droit de tester apparaît à une phase déterminée de l’histoire et, d’après les restrictions plus ou moins importantes qui le limitent, on peut dire à quel moment de l’évolution sociale on se trouve. Il serait facile de multiplier les exemples. Or cette généralité des formes collectives serait inexplicable si les causes finales avaient en sociologie la prépondérance qu’on leur attribue.\par
{\itshape Quand donc on entreprend d’expliquer un phénomène social, il faut rechercher séparément la cause efficiente qui le produit et la fonction qu’il remplit}. Nous nous servons du mot de fonction de préférence à celui de fin ou de but, précisément parce que les phénomènes sociaux n’existent généralement pas en vue des résultats utiles qu’ils produisent. Ce qu’il faut déterminer, c’est s’il y a correspondance entre le fait considéré et les besoins généraux de l’organisme social et en quoi consiste cette correspondance, sans se préoccuper de savoir si elle a été intentionnelle ou non. Toutes ces questions d’intention sont, d’ailleurs, trop subjectives pour pouvoir être traitées scientifiquement.\par
Non seulement ces deux ordres de problèmes doivent être disjoints, mais il convient, en général, de traiter le premier avant le second. Cet ordre, en effet, correspond à celui des faits. Il est naturel de chercher la cause d’un phénomène avant d’essayer d’en déterminer les effets. Cette méthode est d’autant plus logique que la première question, une fois résolue, aidera souvent à résoudre la seconde. En effet, le lien de solidarité qui unit la cause à l’effet a un caractère de réciprocité qui n’a pas été assez reconnu. Sans doute, l’effet ne peut pas exister sans sa cause, mais celle-ci, à son tour, a besoin de son effet. C’est d’elle qu’il tire son énergie, mais aussi il la lui restitue à l’occasion et, par conséquent, ne peut pas disparaître sans qu’elle s’en ressente\footnote{ Nous ne voudrions pas soulever ici des questions de philosophie générale qui ne seraient pas à leur place. Remarquons pourtant que, mieux étudiée, cette réciprocité de la cause et de l’effet pourrait fournir un moyen de réconcilier le mécanisme scientifique avec le finalisme qu’impliquent l’existence et surtout la persistance de la vie.}. Par exemple, la réaction sociale qui constitue la peine est due à l’intensité des sentiments collectifs que le crime offense ; mais, d’un autre côté, elle a pour fonction utile d’entretenir ces sentiments au même degré d’intensité, car ils ne tarderaient pas à s’énerver si les offenses qu’ils subissent n’étaient pas châtiées\footnote{\emph{Division du travail social}, l. II, ch. II, notamment p. 105 et suiv.}. De même, à mesure que le milieu social devient plus complexe et plus mobile, les traditions, les croyances toutes faites s’ébranlent, prennent quelque chose de plus indéterminé et de plus souple et les facultés de réflexion se développent ; mais ces mêmes facultés sont indispensables aux sociétés et aux individus pour s’adapter à un milieu plus mobile et plus complexe\footnote{\emph{Division du travail social}, 52, 53.}. À mesure que les hommes sont obligés de fournir un travail plus intense, les produits de ce travail deviennent plus nombreux et de meilleure qualité ; mais ces produits plus abondants et meilleurs sont nécessaires pour réparer les dépenses qu’entraîne ce travail plus considérable\footnote{{\itshape Ibid.}, 301 et suiv.}. Ainsi, bien loin que la cause des phénomènes sociaux consiste dans une anticipation mentale de la fonction qu’ils sont appelés à remplir, cette fonction consiste, au contraire, au moins dans nombre de cas, à maintenir la cause préexistante d’où ils dérivent ; on trouvera donc plus facilement la première, si la seconde est déjà connue.\par
Mais si l’on ne doit procéder qu’en second lieu à la détermination de la fonction, elle ne laisse pas d’être nécessaire pour que l’explication du phénomène soit complète. En effet, si l’utilité du fait n’est pas ce qui le fait être, il faut généralement qu’il soit utile pour pouvoir se maintenir. Car c’est assez qu’il ne serve à rien pour être nuisible par cela même, puisque, dans ce cas, il coûte sans rien rapporter. Si donc la généralité des phénomènes sociaux avait ce caractère parasitaire, le budget de l’organisme serait en déficit, la vie sociale serait impossible. Par conséquent, pour donner de celle-ci une intelligence satisfaisante, il est nécessaire de montrer comment les phénomènes qui en sont la matière concourent entre eux de manière à mettre la société en harmonie avec elle-même et avec le dehors. Sans doute, la formule courante, qui définit la vie une correspondance entre le milieu interne et le milieu externe, n’est qu’approchée ; cependant elle est vraie en général et, par suite, pour expliquer un fait d’ordre vital, il ne suffit pas de montrer la cause dont il dépend, il faut encore, au moins dans la plupart des cas, trouver la part qui lui revient dans l’établissement de cette harmonie générale.
\section[{II}]{II}
\noindent Ces deux questions distinguées, il nous faut déterminer la méthode d’après laquelle elles doivent être résolues.\par
En même temps qu’elle est finaliste, la méthode d’explication généralement suivie par les sociologues est essentiellement psychologique. Ces deux tendances sont solidaires l’une de l’autre. En effet, si la société n’est qu’un système de moyens institués par les hommes en vue de certaines fins, ces fins ne peuvent être qu’individuelles ; car, avant la société, il ne pouvait exister que des individus. C’est donc de l’individu qu’émanent les idées et les besoins qui ont déterminé la formation des sociétés, et, si c’est de lui que tout vient, c’est nécessairement par lui que tout doit s’expliquer. D’ailleurs, il n’y a rien dans la société que des consciences particulières ; c’est donc dans ces dernières que se trouve la source de toute l’évolution sociale. Par suite, les lois sociologiques ne pourront être qu’un corollaire des lois plus générales de la psychologie ; l’explication suprême de la vie collective consistera à faire voir comment elle découle de la nature humaine en général, soit qu’on l’en déduise directement et sans observation préalable, soit qu’on l’y rattache après l’avoir observée.\par
Ces termes sont à peu près textuellement ceux dont se sert Auguste Comte pour caractériser sa méthode. \emph{« Puisque, dit-il, le phénomène social, conçu en totalité, n’est, au fond, qu’un simple développement de l’humanité, sans aucune création de facultés quelconques, ainsi que je l’ai établi ci-dessus, toutes les dispositions effectives que l’observation sociologique pourra successivement dévoiler devront donc se retrouver au moins en germe dans ce type primordial que la biologie a construit par avance pour la sociologie\footnote{\emph{Cours de philos. pos.}, IV. 333.}. »} C’est que, suivant lui, le fait dominateur de la vie sociale est le progrès et que, d’autre part, le progrès dépend d’un facteur exclusivement psychique, à savoir la tendance qui pousse l’homme à développer de plus en plus sa nature. Les faits sociaux dériveraient même si immédiatement de la nature humaine que, pendant les premières phases de l’histoire, ils en pourraient être directement déduits sans qu’il soit nécessaire de recourir à l’observation\footnote{{\itshape Ibid.}, 345.}. Il est vrai que, de l’aveu de Comte, il est impossible d’appliquer cette méthode déductive aux périodes plus avancées de l’évolution. Seulement cette impossibilité est purement pratique, Elle tient à ce que la distance entre le point de départ et le point d’arrivée devient trop considérable pour que l’esprit humain, s’il entreprenait de le parcourir sans guide, ne risquât pas de s’égarer\footnote{\emph{Cours de philos. pos.}, 346.}. Mais le rapport entre les lois fondamentales de la nature humaine et les résultats ultimes du progrès ne laisse pas d’être analytique. Les formes les plus complexes de la civilisation ne sont que de la vie psychique développée. Aussi, alors même que les théories de la psychologie ne peuvent pas suffire comme prémisses au raisonnement sociologique, elles sont la pierre de touche qui seule permet d’éprouver la validité des propositions inductivement établies. \emph{« Aucune loi de succession sociale, dit Comte, indiquée, même avec toute l’autorité possible, par la méthode historique, ne devra être finalement admise qu’après avoir été rationnellement rattachée, d’une manière d’ailleurs directe ou indirecte, mais toujours incontestable, à la théorie positive de la nature humaine\footnote{{\itshape Ibid.}, 335.} »} C’est donc toujours la psychologie qui aura le dernier mot.\par
Telle est également la méthode suivie par M. Spencer. Suivant lui, en effet, les deux facteurs primaires des phénomènes sociaux sont le milieu cosmique et la constitution physique et morale de l’individu\footnote{\emph{Principes de sociologie}, I, 14, 15.}. Or le premier ne peut avoir d’influence sur la société qu’à travers le second, qui se trouve être ainsi le moteur essentiel de l’évolution sociale. Si la société se forme, c’est pour permettre à l’individu de réaliser sa nature, et toutes les transformations par lesquelles elle a passé n’ont d’autre objet que de rendre cette réalisation plus facile et plus complète. C’est en vertu de ce principe que, avant de procéder à aucune recherche sur l’organisation sociale, M. Spencer a cru devoir consacrer presque tout le premier tome de ses \emph{Principes de sociologie} à l’étude de l’homme primitif physique, émotionnel et intellectuel. \emph{« La science de la sociologie, dit-il, part des unités sociales, soumises aux conditions que nous avons vues, constituées physiquement, émotionnellement et intellectuellement, et en possession de certaines idées acquises de bonne heure et des sentiments correspondants\footnote{{\itshape Op. cit.}, I, 583.}. »} Et c’est dans deux de ces sentiments, la crainte des vivants et la crainte des morts, qu’il trouve l’origine du gouvernement politique et du gouvernement religieux\footnote{{\itshape Ibid.}, 582.}. Il admet, il est vrai, que, une fois formée, la société réagit sur les individus\footnote{{\itshape Ibid.}, 18.}. Mais il ne s’ensuit pas qu’elle ait le pouvoir d’engendrer directement le moindre fait social ; elle n’a d’efficacité causale à ce point de vue que par l’intermédiaire des changements qu’elle détermine chez l’individu. C’est donc toujours de la nature humaine, soit primitive, soit dérivée, que tout découle. D’ailleurs, cette action que le corps social exerce sur ses membres ne peut rien avoir de spécifique, puisque les fins politiques ne sont rien en elles-mêmes, mais une simple expression résumée des fins individuelles\footnote{\emph{ « La société existe pour le profit de ses membres, les membres n’existent pas pour le profit de la société… : les droits du corps politique ne sont rien en eux-mêmes, ils ne deviennent quelque chose qu’à condition d’incarner les droits des individus qui le composent. »} ({\itshape Op. cit.}, II, 20.)}. Elle ne peut donc être qu’une sorte de retour de l’activité privée sur elle-même. Surtout, on ne voit pas en quoi elle peut consister dans les sociétés industrielles, qui ont précisément pour objet de rendre l’individu à lui-même et à ses impulsions naturelles, en le débarrassant de toute contrainte sociale.\par
Ce principe n’est pas seulement à la base de ces grandes doctrines de sociologie générale ; il inspire également un très grand nombre de théories particulières. C’est ainsi qu’on explique couramment l’organisation domestique par les sentiments que les parents ont pour leurs enfants et les seconds pour les premiers ; l’institution du mariage, par les avantages qu’il présente pour les époux et leur descendance ; la peine, par la colère que détermine chez l’individu toute lésion grave de ses intérêts. Toute la vie économique, telle que la conçoivent et l’expliquent les économistes, surtout de l’école orthodoxe, est, en définitive, suspendue à ce facteur purement individuel, le désir de la richesse. S’agit-il de la morale ? On fait des devoirs de l’individu envers lui-même la base de l’éthique. De la religion ? On y voit un produit des impressions que les grandes forces de la nature ou certaines personnalités éminentes éveillent chez l’homme, etc., etc.\par
Mais une telle méthode n’est applicable aux phénomènes sociologiques qu’à condition de les dénaturer. Il suffit, pour en avoir la preuve, de se reporter à la définition que nous en avons donnée. Puisque leur caractéristique essentielle consiste dans le pouvoir qu’ils ont d’exercer, du dehors, une pression sur les consciences individuelles, c’est qu’ils n’en dérivent pas et que, par suite, la sociologie n’est pas un corollaire de la psychologie. Car cette puissance contraignante témoigne qu’ils expriment une nature différente de la nôtre puisqu’ils ne pénètrent en nous que de force ou, tout au moins, en pesant sur nous d’un poids plus ou moins lourd. Si la vie sociale n’était qu’un prolongement de l’être individuel, on ne la verrait pas ainsi remonter vers sa source et l’envahir impétueusement. Puisque l’autorité devant laquelle s’incline l’individu, quand il agit, sent ou pense socialement, le domine à ce point, c’est qu’elle est un produit de forces qui le dépassent et dont il ne saurait, par conséquent, rendre compte. Ce n’est pas de lui que peut venir cette poussée extérieure qu’il subit ; ce n’est donc pas ce qui se passe en lui qui la peut expliquer. Il est vrai que nous ne sommes pas incapables de nous contraindre nous-mêmes ; nous pouvons contenir nos tendances, nos habitudes, nos instincts même et en arrêter le développement par un acte d’inhibition. Mais les mouvements inhibitifs ne sauraient être confondus avec ceux qui constituent la contrainte sociale. Le {\itshape processus} des premiers est centrifuge ; celui des seconds, centripète. Les uns s’élaborent dans la conscience individuelle et tendent ensuite à s’extérioriser ; les autres sont d’abord extérieurs à l’individu, qu’ils tendent ensuite à façonner du dehors à leur image. L’inhibition est bien, si l’on veut, le moyen par lequel la contrainte sociale produit ses effets psychiques ; elle n’est pas cette contrainte.\par
Or, l’individu écarté, il ne reste que la société ; c’est donc dans la nature de la société elle-même qu’il faut aller chercher l’explication de la vie sociale. On conçoit, en effet, que, puisqu’elle dépasse infiniment l’individu dans le temps comme dans l’espace, elle soit en état de lui imposer les manières d’agir et de penser qu’elle a consacrées de son autorité. Cette pression, qui est le signe distinctif des faits sociaux, c’est celle que tous exercent sur chacun.\par
Mais, dira-t-on, puisque les seuls éléments dont est formée la société sont des individus, l’origine première des phénomènes sociologiques ne peut être que psychologique. En raisonnant ainsi, on peut tout aussi facilement établir que les phénomènes biologiques s’expliquent analytiquement par les phénomènes inorganiques. En effet, il est bien certain qu’il n’y a dans la cellule vivante que des molécules de matière brute. Seulement, ils y sont associés et c’est cette association qui est la cause de ces phénomènes nouveaux qui caractérisent la vie et dont il est impossible de retrouver même le germe dans aucun des éléments associés. C’est qu’un tout n’est pas identique à la somme de ses parties, il est quelque chose d’autre et dont les propriétés diffèrent de celles que présentent les parties dont il est composé. L’association n’est pas, comme on l’a cru quelquefois, un phénomène, par soi-même, infécond, qui consiste simplement à mettre en rapports extérieurs des faits acquis et des propriétés constituées. N’est-elle pas, au contraire, la source de toutes les nouveautés qui se sont successivement produites au cours de l’évolution générale des choses ? Quelles différences y a-t-il entre les organismes inférieurs et les autres, entre le vivant organisé et le simple plastide, entre celui-ci et les molécules inorganiques qui le composent, sinon des différences d’association ? Tous ces êtres, en dernière analyse, se résolvent en éléments de même nature ; mais ces éléments sont, ici, juxtaposés, là, associés ; ici, associés d’une manière, là, d’une autre. On est même en droit de se demander si cette loi ne pénètre pas jusque dans le monde minéral et si les différences qui séparent les corps inorganisés n’ont pas la même origine.\par
En vertu de ce principe, la société n’est pas une simple somme d’individus, mais le système formé par leur association représente une réalité spécifique qui a ses caractères propres. Sans doute, il ne peut rien se produire de collectif si des consciences particulières ne sont pas données ; mais cette condition nécessaire n’est pas suffisante. Il faut encore que ces consciences soient associées, combinées, et combinées d’une certaine manière ; c’est de cette combinaison que résulte la vie sociale et, par suite, c’est cette combinaison qui l’explique. En s’agrégeant, en se pénétrant, en se fusionnant, les âmes individuelles donnent naissance à un être, psychique si l’on veut, mais qui constitue une individualité psychique d’un genre nouveau\footnote{ Voilà dans quel sens et pour quelles raisons on peut et on doit parler d’une conscience collective distincte des consciences individuelles. Pour justifier cette distinction, il n’est pas nécessaire d’hypostasier la première ; elle est quelque chose de spécial et doit être désignée par un terme spécial, simplement parce que les états qui la constituent diffèrent spécifiquement de ceux qui constituent les consciences particulières. Cette spécificité leur vient de ce qu’ils ne sont pas formés des mêmes éléments. Les uns, en effet, résultent de la nature de l’être organico-psychique pris isolément, les autres de la combinaison d’une pluralité d’êtres de ce genre. Les résultantes ne peuvent donc pas manquer de différer, puisque les composantes diffèrent à ce point. Notre définition du fait, social ne faisait, d’ailleurs, que marquer d’une autre manière cette ligne de démarcation.}. C’est donc dans la nature de cette individualité, non dans celle des unités composantes qu’il faut aller chercher les causes prochaines et déterminantes des faits qui s’y produisent. Le groupe pense, sent, agit tout autrement que ne feraient ses membres, s’ils étaient isolés. Si donc on part de ces derniers, on ne pourra rien comprendre à ce qui se passe dans le groupe. En un mot, il y a entre la psychologie et la sociologie la même solution de continuité qu’entre la biologie et les sciences physico-chimiques. Par conséquent, toutes les fois qu’un phénomène social est directement expliqué par un phénomène psychique, on peut être assuré que l’explication est fausse.\par
On répondra, peut-être, que, si la société, une fois formée, est, en effet, la cause prochaine des phénomènes sociaux, les causes qui en ont déterminé la formation sont de nature psychologique. On accorde que, quand les individus sont associés, leur association peut donner naissance à une vie nouvelle mais on prétend qu’elle ne peut avoir lieu que pour des raisons individuelles. — Mais, en réalité, aussi loin qu’on remonte dans l’histoire, le fait de l’association est le plus obligatoire de tous ; car il est la source de toutes les autres obligations. Par suite de ma naissance, je suis obligatoirement rattaché à un peuple déterminé. On dit que, dans la suite, une fois adulte, j’acquiesce à cette obligation par cela seul que je continue à vivre dans mon pays. Mais qu’importe ? Cet acquiescement ne lui enlève pas son caractère impératif. Une pression acceptée et subie de bonne grâce ne laisse pas d’être une pression. D’ailleurs, quelle peut être la portée d’une telle adhésion ? D’abord, elle est forcée, car, dans l’immense majorité des cas, il nous est matériellement et moralement impossible de dépouiller notre nationalité ; un tel changement passe même généralement pour une apostasie. Ensuite, elle ne peut concerner le passé qui n’a pu être consenti et qui, pourtant, détermine le présent : je n’ai pas voulu l’éducation que j’ai reçue ; or, c’est elle qui, plus que toute autre cause, me fixe au sol natal. Enfin, elle ne saurait avoir de valeur morale pour l’avenir, dans la mesure où il est inconnu. Je ne connais même pas tous les devoirs qui peuvent m’incomber un jour ou l’autre en ma qualité de citoyen ; comment pourrais-je y acquiescer par avance ? Or tout ce qui est obligatoire, nous l’avons démontré, a sa source en dehors de l’individu. Tant donc qu’on ne sort pas de l’histoire, le fait de l’association présente le même caractère que les autres et, par conséquent, s’explique de la même manière. D’autre part, comme toutes les sociétés sont nées d’autres sociétés sans solution de continuité, on peut être assuré que, dans tout le cours de l’évolution sociale, il n’y a pas eu un moment où les individus aient eu vraiment à délibérer pour savoir s’ils entreraient ou non dans la vie collective, et dans celle-ci plutôt que dans celle-là. Pour que la question pût se poser, il faudrait donc remonter jusqu’aux origines premières de toute société. Mais les solutions, toujours douteuses, que l’on peut apporter à de tels problèmes, ne sauraient en aucun cas affecter la méthode d’après laquelle doivent être traités les faits donnés dans l’histoire. Nous n’avons donc pas à les discuter.\par
Mais on se méprendrait étrangement sur notre pensée, si, de ce qui précède, on tirait cette conclusion que la sociologie, suivant nous, doit ou même peut faire abstraction de l’homme et de ses facultés. Il est clair, au contraire, que les caractères généraux de la nature humaine entrent dans le travail d’élaboration d’où résulte la vie sociale. Seulement ce n’est pas eux qui la suscitent ni qui lui donnent sa forme spéciale ; ils ne font que la rendre possible. Les représentations, les émotions, les tendances collectives n’ont pas pour causes génératrices certains états de la conscience des particuliers, mais les conditions où se trouve le corps social dans son ensemble. Sans doute, elles ne peuvent se réaliser que si les natures individuelles n’y sont pas réfractaires ; mais celles-ci ne sont que la matière indéterminée que le facteur social détermine et transforme. Leur contribution consiste exclusivement en états très généraux, en prédispositions vagues et, par suite, plastiques qui, par elles-mêmes, ne sauraient prendre les formes définies et complexes qui caractérisent les phénomènes sociaux, si d’autres agents n’intervenaient.\par
Quel abîme, par exemple, entre les sentiments que l’homme éprouve en face de forces supérieures à la sienne et l’institution religieuse avec ses croyances, ses pratiques si multipliées et si compliquées, son organisation matérielle et morale ; entre les conditions psychiques de la sympathie que deux êtres de même sang éprouvent l’un pour l’autre\footnote{ Si tant est qu’elle existe avant toute vie sociale. V. sur ce point Espinas, \emph{Sociétés animales}, 474.}, et cet ensemble touffu de règles juridiques et morales qui déterminent la structure de la famille, les rapports des personnes entre elles, des choses avec les personnes, etc. ! Nous avons vu que, même quand la société se réduit à une foule inorganisée, les sentiments collectifs qui s’y forment peuvent, non seulement ne pas ressembler, mais être opposés à la moyenne des sentiments individuels. Combien l’écart doit-il être plus considérable encore quand la pression que subit l’individu est celle d’une société régulière, où, à l’action des contemporains, s’ajoute celle des générations antérieures et de la tradition ! Une explication purement psychologique des faits sociaux ne peut donc manquer de laisser échapper tout ce qu’ils ont de spécifique, c’est-à-dire de social.\par
Ce qui a masqué aux yeux de tant de sociologues l’insuffisance de cette méthode, c’est que, prenant l’effet pour la cause, il leur est arrivé très souvent d’assigner comme conditions déterminantes aux phénomènes sociaux certains états psychiques, relativement définis et spéciaux, mais qui, en fait, en sont la conséquence. C’est ainsi qu’on a considéré comme inné à l’homme un certain sentiment de religiosité, un certain {\itshape minimum} de jalousie sexuelle, de piété filiale, d’amour paternel, etc., et c’est par là que l’on a voulu expliquer la religion, le mariage, la famille. Mais l’histoire montre que ces inclinations, loin d’être inhérentes à la nature humaine, ou bien font totalement défaut dans certaines circonstances sociales, ou, d’une société à l’autre, présentent de telles variations que le résidu que l’on obtient en éliminant toutes ces différences, et qui seul peut être considéré comme d’origine psychologique, se réduit à quelque chose de vague et de schématique qui laisse à une distance infinie les faits qu’il s’agit d’expliquer. C’est donc que ces sentiments résultent de l’organisation collective, loin d’en être la base. Même il n’est pas du tout prouvé que la tendance à la sociabilité ait été, dès l’origine, un instinct congénital du genre humain. Il est beaucoup plus naturel d’y voir un produit de la vie sociale, qui s’est lentement organisé en nous ; car c’est un fait d’observation que les animaux sont sociables ou non suivant que les dispositions de leurs habitats les obligent à la vie commune ou les en détournent. — Et encore faut-il ajouter que, même entre ces inclinations plus déterminées et la réalité sociale, l’écart reste considérable.\par
Il y a d’ailleurs un moyen d’isoler à peu près complètement le facteur psychologique de manière à pouvoir préciser l’étendue de son action, c’est de chercher de quelle façon la race affecte l’évolution sociale. En effet, les caractères ethniques sont d’ordre organico-psychique. La vie sociale doit donc varier quand ils varient, si les phénomènes psychologiques ont sur la société l’efficacité causale qu’on leur attribue. Or nous ne connaissons aucun phénomène social qui soit placé sous la dépendance incontestée de la race. Sans doute, nous ne saurions attribuer à cette proposition la valeur d’une loi ; nous pouvons du moins l’affirmer comme un fait constant de notre pratique. Les formes d’organisation les plus diverses se rencontrent dans des sociétés de même race, tandis que des similitudes frappantes s’observent entre des sociétés de races différentes. La cité a existé chez les Phéniciens, comme chez les Romains et les Grecs ; on la trouve en voie de formation chez les Kabyles. La famille patriarcale était presque aussi développée chez les Juifs que chez les Indous, mais elle ne se retrouve pas chez les Slaves qui sont pourtant de race aryenne. En revanche, le type familial qu’on y rencontre existe aussi chez les Arabes. La famille maternelle et le clan s’observent partout. Le détail des preuves judiciaires, des cérémonies nuptiales est le même chez les peuples les plus dissemblables au point de vue ethnique. S’il en est ainsi, c’est que l’apport psychique est trop général pour prédéterminer le cours des phénomènes sociaux. Puisqu’il n’implique pas une forme sociale plutôt qu’une autre, il ne peut en expliquer aucune. Il y a, il est vrai, un certain nombre de faits qu’il est d’usage d’attribuer à l’influence de la race. C’est ainsi, notamment, qu’on explique comment le développement des lettres et des arts a été si rapide et si intense à Athènes, si lent et si médiocre à Rome. Mais cette interprétation des faits, pour être classique, n’a jamais été méthodiquement démontrée ; elle semble bien tirer à peu près toute son autorité de la seule tradition. On n’a même pas essayé si une explication sociologique des mêmes phénomènes n’était pas possible et nous sommes convaincus qu’elle pourrait être tentée avec succès. En somme, quand on rapporte avec cette rapidité à des facultés esthétiques congénitales le caractère artistique de la civilisation athénienne, on procède à peu près comme faisait le moyen âge quand il expliquait le feu par le phlogistique et les effets de l’opium par sa vertu dormitive.\par
Enfin, si vraiment l’évolution sociale avait son origine dans la constitution psychologique de l’homme, on ne voit pas comment elle aurait pu se produire. Car, alors, il faudrait admettre qu’elle a pour moteur quelque ressort intérieur à la nature humaine. Mais quel pourrait être ce ressort ? Serait-ce cette sorte d’instinct dont parle Comte et qui pousse l’homme à réaliser de plus en plus sa nature ? Mais c’est répondre à la question par la question et expliquer le progrès par une tendance innée au progrès, véritable entité métaphysique dont rien, du reste, ne démontre l’existence ; car les espèces animales, même les plus élevées, ne sont aucunement travaillées par le besoin de progresser et, même parmi les sociétés humaines, il en est beaucoup qui se plaisent à rester indéfiniment stationnaires. Serait-ce, comme semble le croire M. Spencer, le besoin d’un plus grand bonheur que les formes de plus en plus complexes de la civilisation seraient destinées à réaliser de plus en plus complètement ? Il faudrait alors établir que le bonheur croît avec la civilisation et nous avons exposé ailleurs toutes les difficultés que soulève cette hypothèse\footnote{\emph{Division du travail social}, l. II, ch. I.}. Mais il y a plus ; alors même que l’un ou l’autre de ces deux postulats devrait être admis, le développement historique ne serait pas, pour cela, rendu intelligible ; car l’explication qui en résulterait serait purement finaliste et nous avons montré plus haut que les faits sociaux, comme tous les phénomènes naturels, ne sont pas expliqués par cela seul qu’on a fait voir qu’ils servent à quelque fin. Quand on a bien prouvé que les organisations sociales de plus en plus savantes qui se sont succédé au cours de l’histoire ont eu pour effet de satisfaire toujours davantage tel ou tel de nos penchants fondamentaux, on n’a pas fait comprendre pour autant comment elles se sont produites. Le fait qu’elles étaient utiles ne nous apprend pas ce qui les a fait être. Alors même qu’on s’expliquerait comment nous sommes parvenus à les imaginer, à en faire comme le plan par avance de manière à nous représenter les services que nous en pouvions attendre — et le problème est déjà difficile — les vœux dont elles pouvaient être ainsi l’objet n’avaient pas la vertu de les tirer du néant. En un mot, étant admis qu’elles sont les moyens nécessaires pour atteindre le but poursuivi, la question reste tout entière : Comment, c’est-à-dire de quoi et par quoi ces moyens ont-ils été constitués ?\par
Nous arrivons donc à la règle suivante : {\itshape La cause déterminante d’un fait social doit être cherchée parmi les faits sociaux antécédents, et non parmi les états de la conscience individuelle}. D’autre part, on conçoit aisément que tout ce qui précède s’applique à la détermination de la fonction, aussi bien qu’à celle de la cause. La fonction d’un fait social ne peut être que sociale, c’est-à-dire qu’elle consiste dans la production d’effets socialement utiles. Sans doute, il peut se faire, et il arrive en effet, que, par contre-coup, il y serve aussi à l’individu. Mais ce résultat heureux n’est pas sa raison d’être immédiate. Nous pouvons donc compléter la proposition précédente en disant : {\itshape La fonction d’un, fait social doit toujours être recherchée dans le rapport qu’il soutient avec quelque fin sociale}.\par
C’est parce que les sociologues ont souvent méconnu cette règle et considéré les phénomènes sociaux d’un point de vue trop psychologique, que leurs théories paraissent à de nombreux esprits trop vagues, trop flottantes, trop éloignées de la nature spéciale des choses qu’ils croient expliquer. L’historien, notamment, qui vit dans l’intimité de la réalité sociale, ne peut manquer de sentir fortement combien ces interprétations trop générales sont impuissantes à rejoindre les faits ; et c’est, sans doute, ce qui a produit, en partie, la défiance que l’histoire a souvent témoignée à la sociologie. Ce n’est pas à dire, assurément, que l’étude des faits psychiques ne soit pas indispensable au sociologue. Si la vie collective ne dérive pas de la vie individuelle, l’une et l’autre sont étroitement en rapports ; si la seconde ne peut expliquer la première, elle peut, du moins, en faciliter l’explication. D’abord comme nous l’avons montré, il est incontestable que les faits sociaux sont produits par une élaboration {\itshape sui generis} de faits psychiques. Mais, en outre, cette élaboration elle-même n’est pas sans analogies avec celle qui se produit dans chaque conscience individuelle et qui transforme progressivement les éléments primaires (sensations, réflexes, instincts) dont elle est originellement constituée. Ce n’est pas sans raison qu’on a pu dire du moi qu’il était lui-même une société, au même titre que l’organisme, quoique d’une autre manière, et il y a longtemps que les psychologues ont montré toute l’importance du facteur {\itshape association} pour l’explication de la vie de l’esprit. Une culture psychologique, plus encore qu’une culture biologique, constitue donc pour le sociologue une propédeutique nécessaire ; mais elle ne lui sera utile qu’à condition qu’il s’en affranchisse après l’avoir reçue et qu’il la dépasse en la complétant par une culture spécialement sociologique. Il faut qu’il renonce à faire de la psychologie, en quelque sorte, le centre de ses opérations, le point d’où doivent partir et où doivent le ramener les incursions qu’il risque dans le monde social, et qu’il s’établisse au cœur même des faits sociaux, pour les observer de front et sans intermédiaire, en ne demandant à la science de l’individu qu’une préparation générale et, au besoin, d’utiles suggestions\footnote{ Les phénomènes psychiques ne peuvent avoir de conséquences sociales que quand ils sont si intimement unis à des phénomènes sociaux que l’action des uns et des autres est nécessairement confondue. C’est le cas de certains faits socio-psychiques. Ainsi, un fonctionnaire est une force sociale, mais c’est en même temps un individu. Il en résulte qu’il peut se servir de l’énergie sociale qu’il détient, dans un sens déterminé par sa nature individuelle, et, par là, il peut avoir une influence sur la constitution de la société. C’est ce qui arrive aux hommes d’État et, plus généralement, aux hommes de génie. Ceux-ci, alors même qu’ils ne remplissent pas une fonction sociale, tirent des sentiments collectifs dont ils sont l’objet, une autorité qui est, elle aussi, une force sociale, et qu’ils peuvent mettre, dans une certaine mesure, au service d’idées personnelles. Mais on voit que ces cas sont dus à des accidents individuels et, par suite, ne sauraient affecter les traits constitutifs de l’espèce sociale qui, seule, est objet de science. La restriction au principe énoncé plus haut n’est donc pas de grande importance pour le sociologue.}.
\section[{III}]{III}
\noindent Puisque les faits de morphologie sociale sont de même nature que les phénomènes physiologiques, ils doivent s’expliquer d’après cette même règle que nous venons d’énoncer. Toutefois, il résulte de tout ce qui précède qu’ils jouent dans la vie collective et, par suite, dans les explications sociologiques un rôle prépondérant.\par
En effet, si la condition déterminante des phénomènes sociaux consiste, comme nous l’avons montré, dans le fait même de l’association, ils doivent varier avec les formes de cette association, c’est-à-dire suivant les manières dont sont groupées les parties constituantes de la société. Puisque, d’autre part, l’ensemble déterminé que forment, par leur réunion, les éléments de toute nature qui entrent dans la composition d’une société, en constitue le milieu interne, de même que l’ensemble des éléments anatomiques, avec la manière dont ils sont disposés dans l’espace, constitue le milieu interne des organismes, on pourra dire : {\itshape L’origine première de tout processus social de quelque importance doit être recherchée dans la constitution du milieu social interne}.\par
Il est même possible de préciser davantage. En effet, les éléments qui composent ce milieu sont de deux sortes : il y a les choses et les·personnes. Parmi les choses, il faut comprendre, outre les objets matériels qui sont incorporés à la société, les produits de l’activité sociale antérieure, le droit constitué, les mœurs établies, les monuments littéraires, artistiques, etc. Mais il est clair que ce n’est ni des uns ni des autres que peut venir l’impulsion qui détermine les transformations sociales ; car ils ne recèlent aucune puissance motrice. Il y a, assurément, lieu d’en tenir compte dans les explications que l’on tente. Ils pèsent, en effet, d’un certain poids sur l’évolution sociale dont la vitesse et la direction même varient suivant ce qu’ils sont ; mais ils n’ont rien de ce qui est nécessaire pour la mettre en branle. Ils sont la matière à laquelle s’appliquent les forces vives de la société, mais ils ne dégagent par eux-mêmes aucune force vive. Reste donc, comme facteur actif, le milieu proprement humain.\par
L’effort principal du sociologue devra donc tendre à découvrir les différentes propriétés de ce milieu qui sont susceptibles d’exercer une action sur le cours des phénomènes sociaux. Jusqu’à présent, nous avons trouvé deux séries de caractères qui répondent d’une manière éminente à cette condition ; c’est le nombre des unités sociales ou, comme nous avons dit aussi, le volume de la société, et le degré de concentration de la masse, ou ce que nous avons appelé la densité dynamique. Par ce dernier mot, il faut entendre non pas le resserrement purement matériel de l’agrégat qui ne peut avoir d’effet si les individus ou plutôt les groupes d’individus restent séparés par des vides moraux, mais le resserrement moral dont le précédent n’est que l’auxiliaire et, assez généralement, la conséquence. La densité dynamique peut se définir, à volume égal, en fonction du nombre des individus qui sont effectivement en relations non pas seulement commerciales, mais morales ; c’est-à-dire, qui non seulement échangent des services ou se font concurrence, mais vivent d’une vie commune. Car, comme les rapports purement économiques laissent les hommes en dehors les uns des autres, on peut en avoir de très, suivis sans participer pour cela à la même existence collective. Les affaires qui se nouent par-dessus les frontières qui séparent les peuples ne font pas que ces frontières n’existent pas. Or, la vie commune ne peut être affectée que par le nombre de ceux qui y collaborent efficacement. C’est pourquoi ce qui exprime le mieux la densité dynamique d’un peuple, c’est le degré de coalescence des segments sociaux. Car si chaque agrégat partiel forme un tout, une individualité distincte, séparée des autres par une barrière, c’est que l’action de ses membres, en général, y reste localisée ; si, au contraire, ces sociétés partielles sont toutes confondues au sein de la société totale ou tendent à s’y confondre, c’est que, dans la même mesure, le cercle de la vie sociale s’est étendu.\par
Quant à la densité matérielle — si, du moins, on entend par là non pas seulement le nombre des habitants par unité de surface, mais le développement des voies de communication et de transmission — elle marche, {\itshape d’ordinaire}, du même pas que la densité dynamique et, en général, peut servir à la mesurer. Car si les différentes parties de la population tendent à se rapprocher, il est inévitable qu’elles se frayent des voies qui permettent ce rapprochement, et, d’un autre côté, des relations ne peuvent s’établir entre des points distants de la masse sociale que si cette distance n’est pas un obstacle, c’est-à-dire, est, en fait, supprimée. Cependant il y a des exceptions\footnote{ Nous avons eu le tort, dans notre \emph{Division du travail}, de trop présenter la densité matérielle comme l’expression exacte de la densité dynamique. Toutefois, la substitution de la première à la seconde est absolument légitime pour tout ce qui concerne les effets économiques de celle-ci, par exemple la division du travail comme fait purement économique.} et on s’exposerait à de sérieuses erreurs si l’on jugeait toujours de la concentration morale d’une société d’après le degré de concentration matérielle qu’elle présente. Les routes, les lignes ferrées, etc., peuvent servir au mouvement des affaires plus qu’à la fusion des populations, qu’elles n’expriment alors que très imparfaitement. C’est le cas de l’Angleterre dont la densité matérielle est supérieure à celle de la France, et où, pourtant, la coalescence des segments est beaucoup moins avancée, comme le prouve la persistance de l’esprit local et de la vie régionale.\par
Nous avons montré ailleurs comment tout accroissement dans le volume et dans la densité dynamique des sociétés, en rendant la vie sociale plus intense, en étendant l’horizon que chaque individu embrasse par sa pensée et emplit de son action, modifie profondément les conditions fondamentales de l’existence collective. Nous n’avons pas à revenir sur l’application que nous avons faite alors de ce principe. Ajoutons seulement qu’il nous a servi à traiter non pas seulement la question encore très générale qui faisait l’objet de cette étude, mais beaucoup d’autres problèmes plus spéciaux, et que nous avons pu en vérifier ainsi l’exactitude par un nombre déjà respectable d’expériences. Toutefois, il s’en faut que nous croyions avoir trouvé toutes les particularités du milieu social qui sont susceptibles de jouer un rôle dans l’explication des faits sociaux. Tout ce que nous pouvons dire, c’est que ce sont les seules que nous ayons aperçues et que nous n’avons pas été amené à en rechercher d’autres.\par
Mais cette espèce de prépondérance que nous attribuons au milieu social et, plus particulièrement, au milieu humain n’implique pas qu’il faille y voir une sorte de fait ultime et absolu au-delà duquel il n’y ait pas lieu de remonter. Il est évident, au contraire, que l’état où il se trouve à chaque moment de l’histoire dépend lui-même de causes sociales, dont les unes sont inhérentes à la société elle-même, tandis que les autres tiennent aux actions et aux réactions qui s’échangent entre cette société et ses voisines. D’ailleurs, la science ne connaît pas de causes premières, au sens absolu du mot. Pour elle, un fait est primaire simplement quand il est assez général pour expliquer un grand nombre d’autres faits. Or le milieu social est certainement un facteur de ce genre ; car les changements qui s’y produisent, quelles qu’en soient les causes, se répercutent dans toutes les directions de l’organisme social et ne peuvent manquer d’en affecter plus ou moins toutes les fonctions.\par
Ce que nous venons de dire du milieu général de la société peut se répéter des milieux spéciaux à chacun des groupes particuliers qu’elle renferme. Par exemple, selon que la famille sera plus ou moins volumineuse, plus ou moins repliée sur elle-même, la vie domestique sera tout autre. De même, si les corporations professionnelles se reconstituent de manière à ce que chacune d’elles soit ramifiée sur toute l’étendue du territoire au lieu de rester enfermée, comme jadis, dans les limites d’une cité, l’action qu’elles exerceront sera très différente de celle qu’elles exercèrent autrefois. Plus généralement, la vie professionnelle sera tout autre suivant que le milieu propre à chaque profession sera fortement constitué ou que la trame en sera lâche, comme elle est aujourd’hui. Toutefois, l’action de ces milieux particuliers ne saurait avoir l’importance du milieu général ; car ils sont soumis eux-mêmes à l’influence de ce dernier. C’est toujours à celui-ci qu’il en faut revenir. C’est la pression qu’il exerce sur ces groupes partiels qui fait varier leur constitution.\par
Cette conception du milieu social comme facteur déterminant de l’évolution collective est de la plus haute importance. Car, si on la rejette, la sociologie est dans l’impossibilité d’établir aucun rapport de causalité.\par
En effet, cet ordre de causes écarté, il n’y a pas de conditions concomitantes dont puissent dépendre les phénomènes sociaux ; car si le milieu social externe, c’est-à-dire celui qui est formé par les sociétés ambiantes, est susceptible d’avoir quelque action, ce n’est guère que sur les fonctions qui ont pour objet l’attaque et la défense et, de plus, il ne peut faire sentir son influence que par l’intermédiaire du milieu social interne. Les principales causes du développement historique ne se trouveraient donc pas parmi les {\itshape circumfusa} ; elles seraient toutes dans le passé. Elles feraient elles-mêmes partie de ce développement dont elles constitueraient simplement des phases plus anciennes. Les événements actuels de la vie sociale dériveraient non de l’état actuel de la société, mais des événements antérieurs, des précédents historiques, et les explications sociologiques consisteraient exclusivement à rattacher le présent au passé.\par
Il peut sembler, il est vrai, que ce soit suffisant. Ne dit-on pas couramment que l’histoire a précisément pour objet d’enchaîner les événements selon leur ordre de succession ? Mais il est impossible de concevoir comment l’état où la civilisation se trouve parvenue à un moment donné pourrait être la cause déterminante de l’état qui suit. Les étapes que parcourt successivement l’humanité ne s’engendrent pas les unes les autres. On comprend bien que les progrès réalisés à une époque déterminée dans l’ordre juridique, économique, politique, etc., rendent possibles de nouveaux progrès, mais en quoi les prédéterminent-ils ? Ils sont un point de départ qui permet d’aller plus loin ; mais qu’est-ce qui nous incite à aller plus loin ? Il faudrait admettre alors une tendance interne qui pousse l’humanité à dépasser sans cesse les résultats acquis, soit pour se réaliser complètement, soit pour accroître son bonheur, et l’objet de la sociologie serait de retrouver l’ordre selon lequel s’est développée cette tendance. Mais, sans revenir sur les difficultés qu’implique une pareille hypothèse, en tout cas, la loi qui exprime ce développement ne saurait avoir rien de causal. Un rapport de causalité, en effet, ne peut s’établir qu’entre deux faits donnés ; or cette tendance, qui est censée être la cause de ce développement, n’est pas donnée ; elle n’est que postulée et construite par l’esprit d’après les effets qu’on lui attribue. C’est une sorte de faculté motrice que nous imaginons sous le mouvement, pour en rendre compte ; mais la cause efficiente d’un mouvement ne peut être qu’un autre mouvement, non une virtualité de ce genre. Tout ce que nous atteignons donc expérimentalement en l’espèce, c’est une suite de changements entre lesquels il n’existe pas de lien causal. L’état antécédent ne produit pas le conséquent, mais le rapport entre eux est exclusivement chronologique. Aussi, dans ces conditions, toute prévision scientifique est-elle impossible. Nous pouvons bien dire comment les choses se sont succédé jusqu’à présent, non dans quel ordre elles se succéderont désormais, parce que la cause dont elles sont censées dépendre n’est pas scientifiquement déterminée, ni déterminable. D’ordinaire, il est vrai, on admet que l’évolution se poursuivra dans le même sens que par le passé, mais c’est en vertu d’un simple postulat. Rien ne nous assure que les faits réalisés expriment assez complètement la nature de cette tendance pour qu’on puisse préjuger le terme auquel elle aspire d’après ceux par lesquels elle a successivement passé. Pourquoi même la direction qu’elle suit et qu’elle imprime serait-elle rectiligne ?\par
Voilà pourquoi, en fait, le nombre des relations causales, établies par les sociologues, se trouve être si restreint. À quelques exceptions près, dont Montesquieu est le plus illustre exemple, l’ancienne philosophie de l’histoire s’est uniquement attachée à découvrir le sens général dans lequel s’oriente l’humanité, sans chercher à relier les phases de cette évolution à aucune condition concomitante. Quelque grands services que Comte ait rendus à la philosophie sociale, les termes dans lesquels il pose le problème sociologique ne diffèrent pas des précédents. Aussi, sa fameuse loi des trois états n’a-t-elle rien d’un rapport de causalité ; fût-elle exacte, elle n’est et ne peut être qu’empirique. C’est un coup d’œil sommaire sur l’histoire écoulée du genre humain. C’est tout à fait arbitrairement que Comte considère le troisième état comme l’état définitif de l’humanité. Qui nous dit qu’il n’en surgira pas un autre dans l’avenir ? Enfin, la loi qui domine la sociologie de M. Spencer ne paraît pas être d’une autre nature. Fût-il vrai que nous tendons actuellement à chercher notre bonheur dans une civilisation industrielle, rien n’assure que, dans la suite, nous ne le chercherons pas ailleurs. Or, ce qui fait la généralité et la persistance de cette méthode, c’est qu’on a vu le plus souvent dans le milieu social un moyen par lequel le progrès se réalise, non la cause qui le détermine.\par
D’un autre côté, c’est également par rapport à ce même milieu que se doit mesurer la valeur utile ou, comme nous avons dit, la fonction des phénomènes sociaux. Parmi les changements dont il est la cause, ceux-là servent qui sont en rapport avec l’état ou il se trouve, puisqu’il est la condition essentielle de l’existence collective. À ce point de vue encore, la conception que nous venons d’exposer est, croyons-nous, fondamentale ; car, seule, elle permet d’expliquer comment le caractère utile des phénomènes sociaux peut varier sans pourtant dépendre d’arrangements arbitraires. Si, en effet, on se représente l’évolution historique comme mue par une sorte de {\itshape vis a tergo} qui pousse les hommes en avant, puisqu’une tendance motrice ne peut avoir qu’un but et qu’un seul, il ne peut y avoir qu’un point de repère par rapport auquel on calcule l’utilité ou la nocivité des phénomènes sociaux. Il en résulte qu’il n’existe et ne peut exister qu’un seul type d’organisation sociale qui convienne parfaitement à l’humanité, et que les différentes sociétés historiques ne sont que des approximations successives de cet unique modèle. Il n’est pas nécessaire de montrer combien un pareil simplisme est aujourd’hui inconciliable avec la variété et la complexité reconnues des formes sociales. Si, au contraire, la convenance ou la disconvenance des institutions ne peut s’établir que par rapport à un milieu donné, comme ces milieux sont divers, il y a dès lors une diversité de points de repère et, par suite, de types qui, tout en étant qualitativement distincts les uns des autres, sont tous également fondés dans la nature des milieux sociaux.\par
La question que nous venons de traiter est donc étroitement connexe de celle qui a trait à la constitution des types sociaux. S’il y a des espèces sociales, c’est que la vie collective dépend avant tout de conditions concomitantes qui présentent une certaine diversité. Si, au contraire, les principales causes des événements sociaux étaient toutes dans le passé, chaque peuple ne serait plus que le prolongement de celui qui l’a précédé et les différentes sociétés perdraient leur individualité pour ne plus devenir que des moments divers d’un seul et même développement. Puisque, d’autre part, la constitution du milieu social résulte du mode de composition des agrégats sociaux, que même ces deux expressions sont, au fond, synonymes, nous avons maintenant la preuve qu’il n’y a pas de caractères plus essentiels que ceux que nous avons assignés comme base à la classification sociologique.\par
Enfin, on doit comprendre maintenant, mieux que précédemment, combien il serait injuste de s’appuyer sur ces mots de conditions extérieures et de milieu, pour accuser notre méthode de chercher les sources de la vie en dehors du vivant. Tout au contraire, les considérations qu’on vient de lire se ramènent à cette idée que les causes des phénomènes sociaux ont internes à la société. C’est bien plutôt à la théorie qui fait dériver la société de l’individu qu’on pourrait justement reprocher de chercher à tirer le dedans du dehors, puisqu’elle explique l’être social par autre chose que lui-même, et le plus du moins. Puisqu’elle entreprend de déduire le tout de la partie. Les principes qui précèdent méconnaissent si peu le caractère spontané de tout vivant que, si on les applique à la biologie et à la psychologie, on devra admettre que la vie individuelle, elle aussi, s’élabore tout entière à l’intérieur de l’individu.
\section[{IV}]{IV}
\noindent Du groupe de règles qui viennent d’être établies se dégage une certaine conception de la société et de la vie collective.\par
Deux théories contraires se partagent sur ce point les esprits.\par
Pour les uns, comme Hobbes, Rousseau, il y a solution de continuité entre l’individu et la société. L’homme est donc naturellement réfractaire à la vie commune, il ne peut s’y résigner que forcé. Les fins sociales ne sont pas simplement le point de rencontre des fins individuelles ; elles leur sont plutôt contraires. Aussi, pour amener l’individu à les poursuivre, est-il nécessaire d’exercer sur lui une contrainte, et c’est dans l’institution et l’organisation de cette contrainte que consiste, par excellence, l’œuvre sociale. Seulement, parce que l’individu est regardé comme la seule et unique réalité du règne humain, cette organisation, qui a pour objet de le gêner et de le contenir, ne peut être conçue que comme artificielle. Elle n’est pas fondée dans la nature, puisqu’elle est destinée à lui faire violence en l’empêchant de produire ses conséquences anti-sociales. C’est une œuvre d’art, une machine construite tout entière de la main des hommes et qui, comme tous les produits de ce genre, n’est ce qu’elle est que parce que les hommes l’ont voulue telle ; un décret de la volonté l’a créée, un autre décret la peut transformer. Ni Hobbes ni Rousseau ne paraissent avoir aperçu tout ce qu’il y a de contradictoire à admettre que l’individu soit lui-même l’auteur d’une machine qui a pour rôle essentiel de le dominer et de le contraindre, ou du moins, il leur a paru que, pour faire disparaître cette contradiction, il suffisait de la dissimuler aux yeux de ceux qui en sont les victimes par l’habile artifice du pacte social.\par
C’est de l’idée contraire que se sont inspirés et les théoriciens du droit naturel et les économistes et, plus récemment, M. Spencer\footnote{ La position de Comte sur ce sujet est d’un éclectisme assez ambigu.}. Pour eux, la vie sociale est essentiellement spontanée et la société une chose naturelle. Mais, s’ils lui confèrent ce caractère, ce n’est pas qu’ils lui reconnaissent une nature spécifique ; c’est qu’ils lui trouvent une base dans la nature de l’individu. Pas plus que les précédents penseurs, ils n’y voient un système de choses qui existe par soi-même, en vertu de causes qui lui sont spéciales. Mais, tandis que ceux-là ne la concevaient que comme un arrangement conventionnel qu’aucun lien ne rattache à la réalité et qui se tient en l’air, pour ainsi dire, ils lui donnent pour assises les instincts fondamentaux du cœur humain. L’homme est naturellement enclin à la vie politique, domestique, religieuse, aux échanges, etc., et c’est de ces penchants naturels que dérive l’organisation sociale. Par conséquent, partout où elle est normale, elle n’a pas besoin de s’imposer. Quand elle recourt à la contrainte, c’est qu’elle n’est pas ce qu’elle doit être ou que les circonstances sont anormales. En principe, il n’y a qu’à laisser les forces individuelles se développer en liberté pour qu’elles s’organisent socialement.\par
Ni l’une ni l’autre de ces doctrines n’est la nôtre.\par
Sans doute, nous faisons de la contrainte la caractéristique de tout fait social. Seulement, cette contrainte ne résulte pas d’une machinerie plus ou moins savante, destinée à masquer aux hommes les pièges dans lesquels ils se sont pris eux-mêmes. Elle est simplement due à ce que l’individu se trouve en présence d’une force qui le domine et devant laquelle il s’incline ; mais cette force est naturelle. Elle ne dérive pas d’un arrangement conventionnel que la volonté humaine a surajouté de toutes pièces au réel ; elle sort des entrailles mêmes de la réalité ; elle est le produit nécessaire de causes données. Aussi, pour amener l’individu à s’y soumettre de son plein gré, n’est-il nécessaire de recourir à aucun artifice ; il suffit de lui faire prendre conscience de son état de dépendance et d’infériorité naturelles — qu’il s’en fasse par la religion une représentation sensible et symbolique ou qu’il arrive à s’en former par la science une notion adéquate et définie. Comme la supériorité que la société a sur lui n’est pas simplement physique, mais intellectuelle et morale, elle n’a rien à craindre du libre examen, pourvu qu’il en soit fait un juste emploi. La réflexion, en faisant comprendre à l’homme combien l’être social est plus riche, plus complexe et plus durable que l’être individuel, ne peut que lui révéler les raisons intelligibles de la subordination qui est exigée de lui et des sentiments d’attachement et de respect que l’habitude a fixés dans son cœur\footnote{ Voilà pourquoi toute contrainte n’est pas normale. Celle-là seulement mérite ce nom qui correspond à quelque supériorité sociale, c’est-à-dire intellectuelle ou morale. Mais celle qu’un individu exerce sur l’autre parce qu’il est plus fort ou plus riche, surtout si cette richesse n’exprime pas sa valeur sociale, est anormale et ne peut se maintenir que par la violence.}.\par
Il n’y a donc qu’une critique singulièrement superficielle qui pourrait reprocher à notre conception de la contrainte sociale de rééditer les théories de Hobbes et de Machiavel. Mais si, contrairement à ces philosophes, nous disons que la vie sociale est naturelle, ce n’est pas que nous en trouvions la source dans la nature de l’individu ; c’est qu’elle dérive directement de l’être collectif qui est, par lui-même, une nature {\itshape sui generis} ; c’est qu’elle résulte de cette élaboration spéciale à laquelle sont soumises les consciences particulières par le fait de leur association et d’où se dégage une nouvelle forme d’existence\footnote{ Notre théorie est même plus contraire à celle de Hobbes que celle du droit naturel. En effet, pour les partisans de cette dernière doctrine, la vie collective n’est naturelle que dans la mesure où elle peut être déduite de la nature individuelle. Or, seules les formes les plus générales de l’organisation sociale peuvent, à la rigueur, être dérivées de cette origine. Quant au détail, il est trop éloigné de l’extrême généralité des propriétés psychiques pour y pouvoir être rattaché ; il paraît donc aux disciples de cette école tout aussi artificiel qu’à leurs adversaires. Pour nous, au contraire, tout est naturel, même les arrangements les plus spéciaux ; car tout est fondé dans la nature de la société.}. Si donc nous reconnaissons avec les uns qu’elle se présente à l’individu sous l’aspect de la contrainte, nous admettons avec les autres qu’elle est un produit spontané de la réalité ; et ce qui relie logiquement ces deux éléments, contradictoires en apparence, c’est que cette réalité d’où elle émane dépasse l’individu. C’est dire que ces mots de contrainte et de spontanéité n’ont pas dans notre terminologie le sens que Hobbes donne au premier et M. Spencer au second.\par
En résumé, à la plupart des tentatives qui ont été faites pour expliquer rationnellement les faits sociaux, on a pu objecter ou qu’elles faisaient évanouir toute idée de discipline sociale, ou qu’elles ne parvenaient à la maintenir qu’à l’aide de subterfuges mensongers. Les règles que nous venons d’exposer permettraient, au contraire, de faire une sociologie qui verrait dans l’esprit de discipline la condition essentielle de toute vie en commun, tout en le fondant en raison et en vérité.
\chapterclose


\chapteropen
\chapter[{Chapitre VI : Règles relatives à l’administration de la preuve}]{Chapitre VI : \\
Règles relatives à l’administration de la preuve}\renewcommand{\leftmark}{Chapitre VI : \\
Règles relatives à l’administration de la preuve}


\chaptercont
\section[{I}]{I}
\noindent Nous n’avons qu’un moyen de démontrer qu’un phénomène est cause d’un autre, c’est de comparer les cas où ils sont simultanément présents ou absents et de chercher si les variations qu’ils présentent dans ces différentes combinaisons de circonstances témoignent que l’un dépend de l’autre. Quand ils peuvent être artificiellement produits au gré de l’observateur, la méthode est l’expérimentation proprement dite. Quand, au contraire, la production des faits n’est pas à notre disposition et que nous ne pouvons que les rapprocher tels qu’ils se sont spontanément produits, la méthode que l’on emploie est celle de l’expérimentation indirecte ou méthode comparative.\par
Nous avons vu que l’explication sociologique consiste exclusivement à établir des rapports de causalité, qu’il s’agisse de rattacher un phénomène à sa cause, ou, au contraire, une cause à ses effets utiles. Puisque, d’autre part, les phénomènes sociaux échappent évidemment à l’action de l’opérateur, la méthode comparative est la seule qui convienne à la sociologie. Comte, il est vrai, ne l’a pas jugée suffisante ; il a trouvé nécessaire de la compléter par ce qu’il nomme la méthode historique ; mais la cause en est dans sa conception particulière des lois sociologiques. Suivant lui, elles doivent principalement exprimer, non des rapports définis de causalité, mais le sens dans lequel se dirige l’évolution humaine en général ; elles ne peuvent donc être découvertes à l’aide de comparaisons, car pour pouvoir comparer les différentes formes que prend un phénomène social chez différents peuples, il faut l’avoir détaché des séries temporelles auxquelles il appartient. Or, si l’on commence par fragmenter ainsi le développement humain, on se met dans l’impossibilité d’en retrouver la suite. Pour y parvenir, ce n’est pas par analyses, mais par larges synthèses qu’il convient de procéder. Ce qu’il faut c’est rapprocher les uns des autres et réunir dans une même intuition, en quelque sorte, les états successifs de l’humanité de manière à apercevoir \emph{« l’accroissement continu de chaque disposition physique, intellectuelle, morale et politique\footnote{\emph{Cours de philosophie positive}, IV, 328.} ».} Telle est la raison d’être de cette méthode que Comte appelle historique et qui, par suite, est dépourvue de tout objet dès qu’on a rejeté la conception fondamentale de la sociologie comtiste.\par
Il est vrai que Mill déclare l’expérimentation, même indirecte, inapplicable à la sociologie. Mais ce qui suffit déjà à enlever à son argumentation une grande partie de son autorité, c’est qu’il l’appliquait également aux phénomènes biologiques, et même aux faits physico-chimiques les plus complexes\footnote{\emph{Système de Logique}, 11, 478.} ; or il n’y a plus à démontrer aujourd’hui que la chimie et la biologie ne peuvent être que des sciences expérimentales. Il n’y a donc pas de raison pour que ses critiques soient mieux fondées en ce qui concerne la sociologie ; car les phénomènes sociaux ne se distinguent des précédents que par une complexité plus grande. Cette différence peut bien impliquer que l’emploi du raisonnement expérimental en sociologie offre plus de difficultés encore que dans les autres sciences ; mais on ne voit pas pourquoi il y serait radicalement impossible.\par
Du reste, toute cette théorie de Mill repose sur un postulat qui, sans doute, est lié aux principes fondamentaux de sa logique, mais qui est en contradiction avec tous les résultats de la science. Il admet, en effet, qu’un même conséquent ne résulte pas toujours d’un même antécédent, mais peut être dû tantôt à une cause et tantôt à une autre. Cette conception du lien causal, en lui enlevant toute détermination, le rend à peu près inaccessible à l’analyse scientifique ; car il introduit une telle complication dans l’enchevêtrement des causes et des effets que l’esprit s’y perd sans retour. Si un effet peut dériver de causes différentes, pour savoir ce qui le détermine dans un ensemble de circonstances données, il faudrait que l’expérience se fît dans des conditions d’isolement pratiquement irréalisables, surtout en sociologie.\par
Mais ce prétendu axiome de la pluralité des causes est une négation du principe de causalité. Sans doute, si l’on croit avec Mill que la cause et l’effet sont absolument hétérogènes, qu’il n’y a entre eux aucune relation logique, il n’y a rien de contradictoire à admettre qu’un effet puisse suivre tantôt une cause et tantôt une autre. Si le rapport qui unit C à A est purement chronologique, il n’est pas exclusif d’un autre rapport du même genre qui unirait C à B par exemple. Mais si, au contraire, le lien causal a quelque chose d’intelligible, il ne saurait être à ce point indéterminé. S’il consiste en un rapport qui résulte de la nature des choses, un même effet ne peut soutenir ce rapport qu’avec une seule cause, car il ne peut exprimer qu’une seule nature. Or il n’y a que les philosophes qui aient jamais mis en doute l’intelligibilité de la relation causale. Pour le savant, elle ne fait pas question ; elle est supposée par la méthode de la science. Comment expliquer autrement et le rôle si important de la déduction dans le raisonnement expérimental et le principe fondamental de la proportionnalité entre la cause et l’effet ? Quant aux cas que l’on cite et où l’on prétend observer une pluralité de causes, pour qu’ils fussent démonstratifs, il faudrait avoir établi au préalable ou que cette pluralité n’est pas simplement apparente, ou que l’unité extérieure de l’effet ne recouvre pas une réelle pluralité. Que de fois il est arrivé à la science de réduire à l’unité des causes dont la diversité, au premier abord, paraissait irréductible ! Stuart Mill en donne lui-même un exemple en rappelant que, suivant les théories modernes, la production de la chaleur par le frottement, la percussion, l’action chimique, etc., dérive d’une seule et même cause. Inversement, quand il s’agit de l’effet, le savant distingue souvent ce que le vulgaire confond. Pour le sens commun, le mot de fièvre désigne une seule et même entité morbide ; pour la science, il y a une multitude de fièvres spécifiquement différentes et la pluralité des causes se trouve en rapport avec celle des effets ; et si entre toutes ces espèces nosologiques il y a pourtant quelque chose de commun, c’est que ces causes, également, se confondent par certains de leur caractères.\par
Il importe d’autant plus d’exorciser ce principe de la sociologie que nombre de sociologues en subissent encore l’influence, et cela alors même qu’ils n’en font pas une objection contre l’emploi de la méthode comparative. Ainsi, on dit couramment que le crime peut être également produit par les causes les plus différentes ; qu’il en est de même du suicide, de la peine, etc. En pratiquant dans cet esprit le raisonnement expérimental, on aura beau réunir un nombre considérable de faits, on ne pourra jamais obtenir de lois précises, de rapports déterminés de causalités. On ne pourra qu’assigner vaguement un conséquent mal défini à un groupe confus et indéfini d’antécédents. Si donc on veut employer la méthode comparative d’une manière scientifique, c’est-à-dire en se conformant au principe de causalité tel qu’il se dégage de la science elle-même, on devra prendre pour base des comparaisons que l’on institue la proposition suivante : A {\itshape un même effet correspond toujours une même cause.} Ainsi, pour reprendre les exemples cités plus haut, si le suicide dépend de plus d’une cause, c’est que, en réalité, il y a plusieurs espèces de suicides. Il en est de même du crime. Pour la peine, au contraire, si l’on a cru qu’elle s’expliquait également bien par des causes différentes, c’est que l’on n’a pas aperçu l’élément commun qui se retrouve dans tous ces antécédents et en vertu duquel ils produisent leur effet commun \footnote{\emph{Division du travail social}, p. 87.}.
\section[{II}]{II}
\noindent Toutefois, si les divers procédés de la méthode comparative ne sont pas inapplicables à la sociologie, ils n’y ont pas tous une force également démonstrative.\par
La méthode dite des résidus, si tant est d’ailleurs qu’elle constitue une forme du raisonnement expérimental, n’est, pour ainsi dire, d’aucun usage dans l’étude des phénomènes sociaux. Outre qu’elle ne peut servir qu’aux sciences assez avancées, puisqu’elle suppose déjà connues un nombre important de lois, les phénomènes sociaux sont beaucoup trop complexes pour que, dans un cas donné, on puisse exactement retrancher l’effet de toutes les causes moins une.\par
La même raison rend difficilement utilisables et la méthode de concordance et celle de différence. Elles supposent, en effet, que les cas comparés ou concordent en un seul point ou diffèrent Par un seul. Sans doute, il n’est pas de science qui ait jamais pu instituer d’expériences où le caractère rigoureusement unique d’une concordance ou d’une différence fût établi d’une manière irréfutable. On n’est jamais sûr de n’avoir pas laissé échapper quelque antécédent qui concorde ou qui diffère comme le conséquent, en même temps et de la même manière que l’unique antécédent connu. Cependant, quoique l’élimination absolue de tout élément adventice soit une limite idéale qui ne peut être réellement atteinte, en fait, les sciences physico-chimiques et même les sciences biologiques s’en rapprochent assez pour que, dans un grand nombre de cas, la démonstration puisse être regardée comme pratiquement suffisante. Mais il n’en est plus de même en sociologie par suite de la complexité trop grande des phénomènes, jointe à l’impossibilité de toute expérience artificielle. Comme on ne saurait faire un inventaire, même à peu près complet, de tous les faits qui coexistent au sein d’une même société ou qui se sont succédé au cours de son histoire, on ne peut jamais être assuré, même d’une manière approximative, que deux peuples concordent ou diffèrent sous tous les rapports, sauf un. Les chances de laisser un phénomène se dérober sont bien supérieures à celles de n’en négliger aucun. Par conséquent, une pareille méthode de démonstration ne peut donner naissance qu’à des conjectures qui, réduites à elles seules, sont presque dénuées de tout caractère scientifique.\par
Mais il en est tout autrement de la méthode des variations concomitantes. En effet, pour qu’elle soit démonstrative, il n’est pas nécessaire que toutes les variations différentes de celles que l’on compare aient été rigoureusement exclues. Le simple parallélisme des valeurs par lesquelles passent les deux phénomènes, pourvu qu’il ait été établi dans un nombre suffisant de cas suffisamment variés, est la preuve qu’il existe entre eux une relation. Cette méthode doit ce privilège à ce qu’elle atteint le rapport causal, non du dehors comme les précédentes, mais par le dedans. Elle ne nous fait pas simplement voir deux faits qui s’accompagnent ou qui s’excluent extérieurement \footnote{ Dans le cas de la méthode de différence, l’absence de la cause exclut la présence de l’effet.}, de sorte que rien ne prouve directement qu’ils soient unis par un lien interne ; au contraire, elle nous les montre participant l’un de l’autre et d’une manière continue, du moins pour ce qui regarde leur quantité. Or cette participation, à elle seule, suffit à démontrer qu’ils ne sont pas étrangers l’un à l’autre. La manière dont un phénomène se développe en exprime la nature ; pour que deux développements se correspondent, il faut qu’il y ait aussi une correspondance dans les natures qu’ils manifestent. La concomitance constante est donc, par elle-même, une loi, quel que soit l’état des phénomènes restés en dehors de la comparaison. Aussi, pour l’infirmer, ne suffit-il pas de montrer qu’elle est mise en échec par quelques applications particulières de la méthode de concordance ou de différence ; ce serait attribuer à ce genre de preuves une autorité qu’il ne peut avoir en sociologie. Quand deux phénomènes varient régulièrement l’un comme l’autre, il faut maintenir ce rapport alors même que, dans certains cas, l’un de ces phénomènes se présenterait sans l’autre. Car il peut se faire, ou bien que la cause ait été empêchée de produire son effet par l’action de quelque cause contraire, ou bien qu’elle se trouve présente, mais sous une forme différente de celle que l’on a précédemment observée. Sans doute, il y a lieu de voir, comme on dit, d’examiner les faits à nouveau, mais non d’abandonner sur-le-champ les résultats d’une démonstration régulièrement faite.\par
Il est vrai que les lois établies par ce procédé ne se présentent pas toujours d’emblée sous la forme de rapports de causalité. La concomitance peut être due non à ce qu’un des phénomènes est la cause de l’autre, mais à ce qu’ils sont tous deux des effets d’une même cause, ou bien encore à ce qu’il existe entre eux un troisième phénomène, intercalé mais inaperçu, qui est l’effet du premier et la cause du second. Les résultats auxquels conduit cette méthode ont donc besoin d’être interprétés. Mais quelle est la méthode expérimentale qui permet d’obtenir mécaniquement un rapport de causalité sans que les faits qu’elle établit aient besoin d’être élaborés par l’esprit ? Tout ce qui importe, c’est que cette élaboration soit méthodiquement conduite et voici de quelle manière on pourra y procéder. On cherchera d’abord, à l’aide de la déduction, comment l’un des deux termes a pu produire l’autre ; puis on s’efforcera de vérifier le résultat de cette déduction à l’aide d’expériences, c’est-à-dire de comparaisons nouvelles. Si la déduction est possible et si la vérification réussit, on pourra regarder la preuve comme faite. Si, au contraire, l’on n’aperçoit entre ces faits aucun lien direct, surtout si l’hypothèse d’un tel lien contredit des lois déjà démontrées, on se mettra à la recherche d’un troisième phénomène dont les deux autres dépendent également ou qui ait pu servir d’intermédiaire entre eux. Par exemple, on peut établir de la manière la plus certaine que la tendance au suicide varie comme la tendance à l’instruction. Mais il est impossible de comprendre comment l’instruction peut conduire au suicide ; une telle explication est en contradiction avec les lois de la psychologie. L’instruction, surtout réduite aux connaissances élémentaires, n’atteint que les régions les plus superficielles de la conscience ; au contraire, l’instinct de conservation est une de nos tendances fondamentales. Il ne saurait donc être sensiblement affecté par un phénomène aussi éloigné et d’un aussi faible retentissement. On en vient ainsi à se demander si l’un et l’autre fait ne seraient pas la conséquence d’un même état. Cette cause commune, c’est l’affaiblissement du traditionalisme religieux qui renforce à la fois le besoin de savoir et le penchant au suicide.\par
Mais il est une autre raison qui fait de la méthode des variations concomitantes l’instrument par excellence des recherches sociologiques. En effet, même quand les circonstances leur sont le plus favorables, les autres méthodes ne peuvent être employées utilement que si le nombre des faits comparés est très considérable. Si l’on ne peut trouver deux sociétés qui ne diffèrent ou qui ne se ressemblent qu’en un point, du moins, on peut constater que deux faits ou s’accompagnent ou s’excluent très généralement. Mais, pour que cette constatation ait une valeur scientifique, il faut qu’elle ait été faite un très grand nombre de fois ; il faudrait presque être assuré que tous les faits ont été passés en revue. Or, non seulement un inventaire aussi complet n’est pas possible, mais encore les faits qu’on accumule ainsi ne peuvent jamais être établis avec une précision suffisante, justement parce qu’ils sont trop nombreux. Non seulement on risque d’en omettre d’essentiels et qui contredisent ceux qui sont connus, mais encore on n’est pas sûr de bien connaître ces derniers. En fait, ce qui a souvent discrédité les raisonnements des sociologues, c’est que, comme ils ont employé de préférence la méthode de concordance ou celle de différence et surtout la première, ils se sont plus préoccupés d’entasser les documents que de les critiquer et de les choisir. C’est ainsi qu’il leur arrive sans cesse de mettre sur le même plan les observations confuses et rapidement faites des voyageurs et les textes précis de l’histoire. Non seulement, en voyant ces démonstrations, on ne peut s’empêcher de se dire qu’un seul fait pourrait suffire à les infirmer, mais les faits mêmes sur lesquels elles sont établies n’inspirent pas toujours confiance.\par
\par
La méthode des variations concomitantes ne nous oblige ni à de ces énumérations incomplètes, ni à de ces observations superficielles. Pour qu’elle donne des résultats, quelques faits suffisent. Dès qu’on a prouvé que, dans un certain nombre de cas, deux phénomènes varient l’un comme l’autre, on peut être certain qu’on se trouve en présence d’une loi. N’ayant pas besoin d’être nombreux, les documents peuvent être choisis et, de plus, étudiés de près par le sociologue qui les emploie. Il pourra donc et, par suite, il devra prendre pour matière principale de ses inductions les sociétés dont les croyances, les traditions, les mœurs, le droit ont pris corps en des monuments écrits et authentiques. Sans doute, il ne dédaignera pas les renseignements de l’ethnographie (il n’est pas de faits qui puissent être dédaignés par le savant), mais il les mettra à leur vraie place. Au lieu d’en faire le centre de gravité de ses recherches, il ne les utilisera en général que comme complément de ceux qu’il doit à l’histoire ou, tout au moins, il s’efforcera de les confirmer par ces derniers. Non seulement il circonscrira ainsi, avec plus de discernement, l’étendue de ses comparaisons, mais il les conduira avec plus de critique ; car, par cela même qu’il s’attachera à un ordre restreint de faits, il pourra les contrôler avec plus de soin. Sans doute, il n’a pas à refaire l’œuvre des historiens ; mais il ne peut pas non plus recevoir passivement et de toutes mains les informations dont il se sert.\par
Mais il ne faut pas croire que la sociologie soit dans un état de sensible infériorité vis-à-vis des autres sciences parce qu’elle ne peut guère se servir que d’un seul procédé expérimental. Cet inconvénient est, en effet, compensé par la richesse des variations qui s’offrent spontanément aux comparaisons du sociologue et dont on ne trouve aucun exemple dans les autres règnes de la nature. Les changements qui ont lieu dans un organisme au cours d’une existence individuelle sont peu nombreux et très restreints ; ceux qu’on peut provoquer artificiellement sans détruire la vie sont eux-mêmes compris dans d’étroites limites. Il est vrai qu’il s’en est produit de plus importants dans la suite de l’évolution zoologique, mais ils n’ont laissé d’eux-mêmes que de rares et obscurs vestiges, et il est encore plus difficile de retrouver les conditions qui les ont déterminés. Au contraire, la vie sociale est une suite ininterrompue de transformations, parallèles à d’autres transformations dans les conditions de l’existence collective ; et nous n’avons pas seulement à notre disposition celles qui se rapportent à une époque récente, mais un grand nombre de celles par lesquelles ont passé les peuples disparus sont parvenues jusqu’à nous. Malgré ses lacunes, l’histoire de l’humanité est autrement claire et complète que celle des espèces animales. De plus, il existe une multitude de phénomènes sociaux qui se produisent dans toute l’étendue de la société, mais qui prennent des formes diverses selon les régions, les professions, les confessions, etc. Tels sont, par exemple, le crime, le suicide, la natalité, la nuptialité, l’épargne, etc. De la diversité de ces milieux spéciaux résultent, pour chacun de ces ordres de faits, de nouvelles séries de variations, en dehors de celles que produit l’évolution historique. Si donc le sociologue ne peut pas employer avec une égale efficacité tous les procédés de la recherche expérimentale, l’unique méthode, dont il doit presque se servir à l’exclusion des autres, peut, dans ses mains, être très féconde, car il a, pour la mettre en œuvre, d’incomparables ressources.\par
Mais elle ne produit les résultats qu’elle comporte que si elle est pratiquée avec rigueur. On ne prouve rien quand, comme il arrive si souvent, on se contente de faire voir par des exemples plus ou moins nombreux que, dans des cas épars, les faits ont varié comme le veut l’hypothèse. De ces concordances sporadiques et fragmentaires, on ne peut tirer aucune conclusion générale. Illustrer une idée n’est pas la démontrer. Ce qu’il faut, c’est comparer non des variations isolées, mais des séries de variations, régulièrement constituées, dont les termes se relient les uns aux autres par une gradation aussi continue que possible, et qui, de plus, soient d’une suffisante étendue. Car les variations d’un phénomène ne permettent d’en induire la loi que si elles expriment clairement la manière dont il se développe dans des circonstances données. Or, pour cela, il faut qu’il y ait entre elles la même suite qu’entre les moments divers d’une même évolution naturelle, et, en outre, que cette évolution qu’elles figurent soit assez prolongée pour que le sens n’en soit pas douteux.
\section[{III}]{III}
\noindent Mais la manière dont doivent être formées ces séries diffère selon les cas. Elles peuvent comprendre des faits empruntés ou à une seule et unique société — ou à plusieurs sociétés de même espèce — ou à plusieurs espèces sociales distinctes.\par
Le premier procédé peut suffire, à la rigueur, quand il s’agit de faits d’une grande généralité et sur lesquels nous avons des informations statistiques assez étendues et variées.\par
Par exemple, en rapprochant la courbe qui exprime la marche du suicide pendant une période de temps suffisamment longue, des variations que présente le même phénomène suivant les provinces, les classes, les habitats ruraux ou urbains, les sexes, les âges, l’état civil, etc., on peut arriver, même sans étendre ses recherches au-delà d’un seul pays, à établir de véritables lois, quoiqu’il soit toujours préférable de confirmer ces résultats par d’autres observations faites sur d’autres peuples de la même espèce. Mais on ne peut se contenter de comparaisons aussi limitées que quand on étudie quelqu’un de ces courants sociaux qui sont répandus dans toute la société, tout en variant d’un point à l’autre. Quand, au contraire, il s’agit d’une institution, d’une règle juridique ou morale, d’une coutume organisée, qui est la même et fonctionne de la même manière sur toute l’étendue du pays et qui ne change que dans le temps, on ne peut se renfermer dans l’étude d’un seul peuple ; car, alors, on n’aurait pour matière de la preuve qu’un seul couple de courbes parallèles, à savoir celles qui expriment la marche historique du phénomène considéré et de la cause conjecturée, mais dans cette seule et unique société. Sans doute, même ce seul parallélisme, s’il est constant, est déjà un fait considérable, mais il ne saurait, à lui seul, constituer une démonstration.\par
En faisant entrer en ligne de compte plusieurs peuples de même espèce, on dispose déjà d’un champ de comparaison plus étendu. D’abord, on peut confronter l’histoire de l’un par celle des autres et voir si, chez chacun d’eux pris à part, le même phénomène évolue dans le temps en fonction des mêmes conditions. Puis on peut établir des comparaisons entre ces divers développements. Par exemple, on déterminera la forme que le fait étudié prend chez ces différentes sociétés au moment où il parvient à son apogée. Comme, tout en appartenant au même type, elles sont pourtant des individualités distinctes, cette forme n’est pas partout la même ; elle est plus ou moins accusée, suivant les cas. On aura ainsi une nouvelle série de variations qu’on rapprochera de celles que présente, au même moment et dans chacun de ces pays, la condition présumée. Ainsi, après avoir suivi l’évolution de la famille patriarcale à travers l’histoire de Rome, d’Athènes, de Sparte, on classera ces mêmes cités suivant le degré maximum de développement qu’atteint chez chacune d’elles ce type familial et on verra ensuite si, par rapport à l’état du milieu social dont il paraît dépendre d’après la première expérience, elles se classent encore de la même manière.\par
Mais cette méthode elle-même ne peut guère se suffire. Elle ne s’applique, en effet, qu’aux phénomènes qui ont pris naissance pendant la vie des peuples comparés. Or, une société ne crée pas de toutes pièces son organisation ; elle la reçoit, en partie, toute faite de celles qui l’ont précédée. Ce qui lui est ainsi transmis n’est, au cours de son histoire, le produit d’aucun développement, par conséquent ne peut être expliqué si l’on ne sort pas des limites de l’espèce dont elle fait partie. Seules, les additions qui se surajoutent à ce fond primitif et le transforment peuvent être traitées de cette manière. Mais, plus on s’élève dans l’échelle sociale, plus les caractères acquis par chaque peuple sont peu de chose à côté des caractères transmis. C’est, d’ailleurs, la condition de tout progrès. Ainsi, les éléments nouveaux que nous avons introduits dans le droit domestique, le droit de propriété, la morale, depuis le commencement de notre histoire, sont relativement peu nombreux et peu importants, comparés à ceux que le passé nous a légués. Les nouveautés qui se produisent ainsi ne sauraient donc se comprendre si l’on n’a pas étudié d’abord ces phénomènes plus fondamentaux qui en sont les racines et ils ne peuvent être étudiés qu’à l’aide de comparaisons beaucoup plus étendues. Pour pouvoir expliquer l’état actuel de la famille, du mariage, de la propriété, etc., il faudrait connaître quelles en sont les origines, quels sont les éléments simples dont ces institutions sont composées et, sur ces points, l’histoire comparée des grandes sociétés européennes ne saurait nous apporter de grandes lumières. Il faut remonter plus haut.\par
Par conséquent, pour rendre compte d’une institution sociale, appartenant à une espèce déterminée, on comparera les formes différentes qu’elle présente, non seulement chez les peuples de cette espèce, mais dans toutes les espèces antérieures. S’agit-il, par exemple, de l’organisation domestique ? On constituera d’abord le type le plus rudimentaire qui ait jamais existé, pour suivre ensuite pas à pas la manière dont il s’est progressivement compliqué. Cette méthode, que l’on pourrait appeler génétique, donnerait d’un seul coup l’analyse et la synthèse du phénomène. Car, d’une part, elle nous montrerait à l’état dissocié les éléments qui le composent, par cela seul qu’elle nous les ferait voir se surajoutant successivement les uns aux autres et, en même temps, grâce à ce large champ de comparaison, elle serait beaucoup mieux en état de déterminer les conditions dont dépendent leur formation et leur association. {\itshape Par conséquent, on ne peut expliquer un fait social de quelque complexité qu’à condition d’en suivre le développement intégral à travers toutes les espèces sociales.} La sociologie comparée n’est pas une branche particulière de la sociologie ; c’est la sociologie même, en tant qu’elle cesse d’être purement descriptive et aspire à rendre compte des faits.\par
Au cours de ces comparaisons étendues, se commet souvent une erreur qui en fausse les résultats. Parfois, pour juger du sens dans lequel se développent les événements sociaux, il est arrivé qu’on a simplement comparé ce qui se passe au déclin de chaque espèce avec ce qui se produit au début de l’espèce suivante. En procédant ainsi, on a cru pouvoir dire, par exemple, que l’affaiblissement des croyances religieuses et de tout traditionalisme ne pouvait jamais être qu’un phénomène passager de la vie des peuples, parce qu’il n’apparaît que pendant la dernière période de leur existence pour cesser dès qu’une évolution nouvelle recommence. Mais, avec une telle méthode, on est exposé à prendre pour la marche régulière et nécessaire du progrès ce qui est l’effet d’une tout autre cause. En effet, l’état où se trouve une société jeune n’est pas le simple prolongement de l’état où étaient parvenues à la fin de leur carrière, les sociétés qu’elle remplace, mais provient en partie de cette jeunesse même qui empêche les produits des expériences faites par les peuples antérieurs d’être tous immédiatement assimilables et utilisables. C’est ainsi que l’enfant reçoit de ses parents des facultés et des prédispositions qui n’entrent en jeu que tardivement dans sa vie. Il est donc possible, pour reprendre le même exemple, que ce retour du traditionalisme que l’on observe au début de chaque histoire soit dû non à ce fait qu’un recul du même phénomène ne peut jamais être que transitoire, mais aux conditions spéciales où se trouve placée toute société qui commence. La comparaison ne peut être démonstrative que si l’on élimine ce facteur de l’âge qui la trouble ; pour y arriver, il {\itshape suffira de considérer les sociétés que l’on compare à la même période de leur développement.} Ainsi, pour savoir dans quel sens évolue un phénomène social, on comparera ce qu’il est pendant la jeunesse de chaque espèce avec ce qu’il devient pendant la jeunesse de l’espèce suivante, et suivant que, de l’une de ces étapes à l’autre, il présentera plus, moins ou autant d’intensité, on dira qu’il progresse, recule ou se maintient.
\chapterclose


\chapteropen
\chapter[{Conclusion}]{Conclusion}\phantomsection
\label{conclusion}\renewcommand{\leftmark}{Conclusion}


\chaptercont
\noindent En résumé, les caractères de cette méthode sont les suivants.\par
D’abord, elle est indépendante de toute philosophie. Parce que la sociologie est née des grandes doctrines philosophiques, elle a gardé l’habitude de s’appuyer sur quelque système dont elle se trouve ainsi solidaire. C’est ainsi qu’elle a été successivement positiviste, évolutionniste, spiritualiste, alors qu’elle doit se contenter d’être la sociologie tout court. Même nous hésiterions à la qualifier de naturaliste à moins qu’on ne veuille seulement indiquer par là qu’elle considère les faits sociaux comme explicables naturellement, et, dans ce cas, l’épithète est assez inutile, puisqu’elle signifie simplement que le sociologue fait œuvre de science et n’est pas un mystique. Mais nous repoussons le mot, si on lui donne un sens doctrinal sur l’essence des choses sociales, si, par exemple, on entend dire qu’elles sont réductibles aux autres forces cosmiques. La sociologie n’a pas à prendre de parti entre les grandes hypothèses qui divisent les métaphysiciens. Elle n’a pas plus à affirmer la liberté que le déterminisme. Tout ce qu’elle demande qu’on lui accorde, c’est que le principe de causalité s’applique aux phénomènes sociaux. Encore ce principe est-il posé par elle, non comme une nécessité rationnelle, mais seulement comme un postulat empirique, produit d’une induction légitime. Puisque la loi de causalité a été vérifiée dans les autres règnes de la nature, que, progressivement, elle a étendu son empire du monde physico-chimique au monde biologique, de celui-ci au monde psychologique, on est en droit d’admettre qu’elle est également vraie du monde social ; et il est possible d’ajouter aujourd’hui que les recherches entreprises sur la base de ce postulat tendent à le confirmer. Mais la question de savoir si la nature du lien causal exclut toute contingence n’est pas tranchée pour cela.\par
Au reste, la philosophie elle-même a tout intérêt à cette émancipation de la sociologie. Car, tant que le sociologue n’a pas suffisamment dépouillé le philosophe, il ne considère les choses sociales que par leur côté le plus général, celui par où elles ressemblent le plus aux autres choses de l’univers. Or, si la sociologie ainsi conçue peut servir à illustrer de faits curieux une philosophie, elle ne saurait l’enrichir de vues nouvelles, puisqu’elle ne signale rien de nouveau dans l’objet qu’elle étudie. Mais en réalité, si les faits fondamentaux des autres règnes se retrouvent dans le règne social, c’est sous des formes spéciales qui en font mieux comprendre la nature parce qu’elles en sont l’expression la plus haute. Seulement, pour les apercevoir sous cet aspect, il faut sortir des généralités et entrer dans le détail des faits. C’est ainsi que la sociologie, à mesure qu’elle se spécialisera, fournira des matériaux plus originaux à la réflexion philosophique. Déjà ce qui précède a pu faire entrevoir comment des notions essentielles, telles que celles d’espèce, d’organe, de fonction, de santé et de maladie, de cause et de fin s’y présentent sous des jours tout nouveaux. D’ailleurs, n’est-ce pas la sociologie qui est destinée à mettre dans tout son relief une idée qui pourrait bien être la base non pas seulement d’une psychologie, mais de toute une philosophie, l’idée d’association ?\par
Vis-à-vis des doctrines pratiques, notre méthode permet et commande la même indépendance. La sociologie ainsi entendue ne sera ni individualiste, ni communiste, ni socialiste, au sens qu’on donne vulgairement à ces mots. Par principe, elle ignorera ces théories auxquelles elle ne saurait reconnaître de valeur scientifique, puisqu’elles tendent directement, non à exprimer les faits, mais à les réformer. Du moins, si elle s’y intéresse, c’est dans la mesure où elle y voit des faits sociaux qui peuvent l’aider à comprendre la réalité sociale en manifestant les besoins qui travaillent la société. Ce n’est pas, toutefois, qu’elle doive se désintéresser des questions pratiques. On a pu voir, au contraire, que notre préoccupation constante était de l’orienter de manière à ce qu’elle puisse aboutir pratiquement. Elle rencontre nécessairement ces problèmes au terme de ses recherches. Mais, par cela même qu’ils ne se présentent à elle qu’à ce moment, que, par suite, ils se dégagent des faits et non des passions, on peut prévoir qu’ils doivent se poser pour le sociologue dans de tout autres termes que pour la foule, et que les solutions, d’ailleurs partielles, qu’il y peut apporter ne sauraient coïncider exactement avec aucune de celles auxquelles s’arrêtent les partis. Mais le rôle de la sociologie à ce point de vue doit justement consister à nous affranchir de tous les partis, non pas tant en opposant une doctrine aux doctrines, qu’en faisant contracter aux esprits, en face de ces questions, une attitude spéciale que la science peut seule donner par le contact direct des choses. Seule, en effet, elle peut apprendre à traiter avec respect, mais sans fétichisme, les institutions historiques quelles qu’elles soient, en nous faisant sentir ce qu’elles ont, à la fois, de nécessaire et de provisoire, leur force de résistance et leur infinie variabilité.\par
En second lieu, notre méthode est objective. Elle est dominée tout entière par cette idée que les faits sociaux sont des choses et doivent être traités comme telles. Sans doute, ce principe se retrouve, sous une forme un peu différente, à la base des doctrines de Comte et de M. Spencer. Mais ces grands penseurs en ont donné la formule théorique, plus qu’ils ne l’ont mise en pratique. Pour qu’elle ne restât pas lettre morte, il ne suffisait pas de la promulguer ; il fallait en faire la base de toute une discipline qui prît le savant au moment même où il aborde l’objet de ses recherches et qui l’accompagnât pas à pas dans toutes ses démarches. C’est à instituer cette discipline que nous nous sommes attaché.\par
Nous avons montré comment le sociologue devait écarter les notions anticipées qu’il avait des faits pour se mettre en face des faits eux-mêmes ; comment il devait les atteindre par leurs caractères les plus objectifs ; comment il devait leur demander à eux-mêmes le moyen de les classer en sains et en morbides ; comment, enfin, il devait s’inspirer du même principe dans les explications qu’il tentait comme dans la manière dont il prouvait ces explications. Car une fois qu’on a le sentiment qu’on se trouve en présence de choses, on ne songe même plus à les expliquer par des calculs utilitaires ni par des raisonnements d’aucune sorte. On comprend trop bien l’écart qu’il y a entre de telles causes et de tels effets. Une chose est une force qui ne peut être engendrée que par une autre force. On cherche donc, pour rendre compte des faits sociaux, des énergies capables de les produire. Non seulement les explications sont autres, mais elles sont autrement démontrées, ou plutôt c’est alors seulement qu’on éprouve le besoin de les démontrer. Si les phénomènes sociologiques ne sont que des systèmes d’idées objectivées, les expliquer, c’est les repenser dans leur ordre logique et cette explication est à elle-même sa propre preuve ; tout au plus peut-il y avoir lieu de la confirmer par quelques exemples. Au contraire, il n’y a que des expériences méthodiques qui puissent arracher leur secret à des choses.\par
Mais si nous considérons les faits sociaux comme des choses, c’est comme {\itshape des choses sociales.} C’est le troisième trait caractéristique de notre méthode d’être exclusivement sociologique. Il a souvent paru que ces phénomènes, à cause de leur extrême complexité, ou bien étaient réfractaires à la science, ou bien n’y pouvaient entrer que réduits à leurs conditions élémentaires, soit psychiques, soit organiques, c’est-à-dire dépouillés de leur nature propre. Nous avons, au contraire, entrepris d’établir qu’il était possible de les traiter scientifiquement sans rien leur enlever de leurs caractères spécifiques. Même nous avons refusé de ramener cette immatérialité sui {\itshape generis} qui les caractérise à celle, déjà complexe pourtant, des phénomènes psychologiques ; à plus forte raison nous sommes-nous interdit de la résorber, à la suite de l’école italienne, dans les propriétés générales de la matière organisée\footnote{ On est donc mal venu à qualifier notre méthode de matérialiste.}. Nous avons fait voir qu’un fait social ne peut être expliqué que par un autre fait social, et, en même temps, nous avons montré comment cette sorte d’explication est possible en signalant dans le milieu social interne le moteur principal de l’évolution collective. La sociologie n’est donc l’annexe d’aucune autre science ; elle est elle-même une science distincte et autonome, et le sentiment de ce qu’a de spécial la réalité sociale est même tellement nécessaire au sociologue que, seule, une culture spécialement sociologique peut le préparer à l’intelligence des faits sociaux.\par
Nous estimons que ce progrès est le plus important de ceux qui restent à faire à la sociologie. Sans doute, quand une science est en train de naître, on est bien obligé, pour la faire, de se référer aux seuls modèles qui existent, c’est-à-dire aux sciences déjà formées. Il y a là un trésor d’expériences toutes faites qu’il serait insensé de ne pas mettre à profit. Cependant, une science ne peut se regarder comme définitivement constituée que quand elle est parvenue à se faire une personnalité indépendante. Car elle n’a de raison d’être que si elle a pour matière un ordre de faits que n’étudient pas les autres sciences. Or il est impossible que les mêmes notions puissent convenir identiquement à des choses de nature différente.\par
Tels nous paraissent être les principes de la méthode sociologique.\par
Cet ensemble de règles paraîtra peut-être inutilement compliqué, si on le compare aux procédés qui sont couramment mis en usage. Tout cet appareil de précautions peut sembler bien laborieux pour une science qui, jusqu’ici, ne réclamait guère, de ceux qui s’y consacraient, qu’une culture générale et philosophique ; et il est, en effet, certain que la mise en pratique d’une telle méthode ne saurait avoir pour effet de vulgariser la curiosité des choses sociologiques. Quand, comme condition d’initiation préalable, on demande aux gens de se défaire des concepts qu’ils ont l’habitude d’appliquer à un ordre de choses, pour repenser celles-ci à nouveaux frais, on ne peut s’attendre à recruter une nombreuse clientèle. Mais ce n’est pas le but où nous tendons. Nous croyons, au contraire, que le moment est venu pour la sociologie de renoncer aux succès mondains, pour ainsi parler, et de prendre le caractère ésotérique qui convient à toute science. Elle gagnera ainsi en dignité et en autorité ce qu’elle perdra peut-être en popularité, Car tant qu’elle reste mêlée aux luttes des partis, tant qu’elle se contente d’élaborer, avec plus de logique que le vulgaire, les idées communes et que, par suite, elle ne suppose aucune compétence spéciale, elle n’est pas en droit de parler assez haut pour faire taire les passions et les préjugés. Assurément, le temps est encore loin où elle pourra jouer ce rôle efficacement ; pourtant, c’est à la mettre en état de le remplir un jour qu’il nous faut, dès maintenant, travailler.
\chapterclose

 


% at least one empty page at end (for booklet couv)
\ifbooklet
  \pagestyle{empty}
  \clearpage
  % 2 empty pages maybe needed for 4e cover
  \ifnum\modulo{\value{page}}{4}=0 \hbox{}\newpage\hbox{}\newpage\fi
  \ifnum\modulo{\value{page}}{4}=1 \hbox{}\newpage\hbox{}\newpage\fi


  \hbox{}\newpage
  \ifodd\value{page}\hbox{}\newpage\fi
  {\centering\color{rubric}\bfseries\noindent\large
    Hurlus ? Qu’est-ce.\par
    \bigskip
  }
  \noindent Des bouquinistes électroniques, pour du texte libre à participation libre,
  téléchargeable gratuitement sur \href{https://hurlus.fr}{\dotuline{hurlus.fr}}.\par
  \bigskip
  \noindent Cette brochure a été produite par des éditeurs bénévoles.
  Elle n’est pas faîte pour être possédée, mais pour être lue, et puis donnée.
  Que circule le texte !
  En page de garde, on peut ajouter une date, un lieu, un nom ; pour suivre le voyage des idées.
  \par

  Ce texte a été choisi parce qu’une personne l’a aimé,
  ou haï, elle a en tous cas pensé qu’il partipait à la formation de notre présent ;
  sans le souci de plaire, vendre, ou militer pour une cause.
  \par

  L’édition électronique est soigneuse, tant sur la technique
  que sur l’établissement du texte ; mais sans aucune prétention scolaire, au contraire.
  Le but est de s’adresser à tous, sans distinction de science ou de diplôme.
  Au plus direct ! (possible)
  \par

  Cet exemplaire en papier a été tiré sur une imprimante personnelle
   ou une photocopieuse. Tout le monde peut le faire.
  Il suffit de
  télécharger un fichier sur \href{https://hurlus.fr}{\dotuline{hurlus.fr}},
  d’imprimer, et agrafer ; puis de lire et donner.\par

  \bigskip

  \noindent PS : Les hurlus furent aussi des rebelles protestants qui cassaient les statues dans les églises catholiques. En 1566 démarra la révolte des gueux dans le pays de Lille. L’insurrection enflamma la région jusqu’à Anvers où les gueux de mer bloquèrent les bateaux espagnols.
  Ce fut une rare guerre de libération dont naquit un pays toujours libre : les Pays-Bas.
  En plat pays francophone, par contre, restèrent des bandes de huguenots, les hurlus, progressivement réprimés par la très catholique Espagne.
  Cette mémoire d’une défaite est éteinte, rallumons-la. Sortons les livres du culte universitaire, débusquons les idoles de l’époque, pour les démonter.
\fi

\ifdev % autotext in dev mode
\fontname\font — \textsc{Les règles du jeu}\par
(\hyperref[utopie]{\underline{Lien}})\par
\noindent \initialiv{A}{lors là}\blindtext\par
\noindent \initialiv{À}{ la bonheur des dames}\blindtext\par
\noindent \initialiv{É}{tonnez-le}\blindtext\par
\noindent \initialiv{Q}{ualitativement}\blindtext\par
\noindent \initialiv{V}{aloriser}\blindtext\par
\Blindtext
\phantomsection
\label{utopie}
\Blinddocument
\fi
\end{document}
