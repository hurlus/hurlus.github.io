%%%%%%%%%%%%%%%%%%%%%%%%%%%%%%%%%
% LaTeX model https://hurlus.fr %
%%%%%%%%%%%%%%%%%%%%%%%%%%%%%%%%%

% Needed before document class
\RequirePackage{pdftexcmds} % needed for tests expressions
\RequirePackage{fix-cm} % correct units

% Define mode
\def\mode{a4}

\newif\ifaiv % a4
\newif\ifav % a5
\newif\ifbooklet % booklet
\newif\ifcover % cover for booklet

\ifnum \strcmp{\mode}{cover}=0
  \covertrue
\else\ifnum \strcmp{\mode}{booklet}=0
  \booklettrue
\else\ifnum \strcmp{\mode}{a5}=0
  \avtrue
\else
  \aivtrue
\fi\fi\fi

\ifbooklet % do not enclose with {}
  \documentclass[french,twoside]{book} % ,notitlepage
  \usepackage[%
    papersize={105mm, 297mm},
    inner=12mm,
    outer=12mm,
    top=20mm,
    bottom=15mm,
    marginparsep=0pt,
  ]{geometry}
  \usepackage[fontsize=9.5pt]{scrextend} % for Roboto
\else\ifav
  \documentclass[french,twoside]{book} % ,notitlepage
  \usepackage[%
    a5paper,
    inner=25mm,
    outer=15mm,
    top=15mm,
    bottom=15mm,
    marginparsep=0pt,
  ]{geometry}
  \usepackage[fontsize=12pt]{scrextend}
\else% A4 2 cols
  \documentclass[twocolumn]{report}
  \usepackage[%
    a4paper,
    inner=15mm,
    outer=10mm,
    top=25mm,
    bottom=18mm,
    marginparsep=0pt,
  ]{geometry}
  \setlength{\columnsep}{20mm}
  \usepackage[fontsize=9.5pt]{scrextend}
\fi\fi

%%%%%%%%%%%%%%
% Alignments %
%%%%%%%%%%%%%%
% before teinte macros

\setlength{\arrayrulewidth}{0.2pt}
\setlength{\columnseprule}{\arrayrulewidth} % twocol
\setlength{\parskip}{0pt} % classical para with no margin
\setlength{\parindent}{1.5em}

%%%%%%%%%%
% Colors %
%%%%%%%%%%
% before Teinte macros

\usepackage[dvipsnames]{xcolor}
\definecolor{rubric}{HTML}{800000} % the tonic 0c71c3
\def\columnseprulecolor{\color{rubric}}
\colorlet{borderline}{rubric!30!} % definecolor need exact code
\definecolor{shadecolor}{gray}{0.95}
\definecolor{bghi}{gray}{0.5}

%%%%%%%%%%%%%%%%%
% Teinte macros %
%%%%%%%%%%%%%%%%%
%%%%%%%%%%%%%%%%%%%%%%%%%%%%%%%%%%%%%%%%%%%%%%%%%%%
% <TEI> generic (LaTeX names generated by Teinte) %
%%%%%%%%%%%%%%%%%%%%%%%%%%%%%%%%%%%%%%%%%%%%%%%%%%%
% This template is inserted in a specific design
% It is XeLaTeX and otf fonts

\makeatletter % <@@@


\usepackage{blindtext} % generate text for testing
\usepackage[strict]{changepage} % for modulo 4
\usepackage{contour} % rounding words
\usepackage[nodayofweek]{datetime}
% \usepackage{DejaVuSans} % seems buggy for sffont font for symbols
\usepackage{enumitem} % <list>
\usepackage{etoolbox} % patch commands
\usepackage{fancyvrb}
\usepackage{fancyhdr}
\usepackage{float}
\usepackage{fontspec} % XeLaTeX mandatory for fonts
\usepackage{footnote} % used to capture notes in minipage (ex: quote)
\usepackage{framed} % bordering correct with footnote hack
\usepackage{graphicx}
\usepackage{lettrine} % drop caps
\usepackage{lipsum} % generate text for testing
\usepackage[framemethod=tikz,]{mdframed} % maybe used for frame with footnotes inside
\usepackage{pdftexcmds} % needed for tests expressions
\usepackage{polyglossia} % non-break space french punct, bug Warning: "Failed to patch part"
\usepackage[%
  indentfirst=false,
  vskip=1em,
  noorphanfirst=true,
  noorphanafter=true,
  leftmargin=\parindent,
  rightmargin=0pt,
]{quoting}
\usepackage{ragged2e}
\usepackage{setspace} % \setstretch for <quote>
\usepackage{tabularx} % <table>
\usepackage[explicit]{titlesec} % wear titles, !NO implicit
\usepackage{tikz} % ornaments
\usepackage{tocloft} % styling tocs
\usepackage[fit]{truncate} % used im runing titles
\usepackage{unicode-math}
\usepackage[normalem]{ulem} % breakable \uline, normalem is absolutely necessary to keep \emph
\usepackage{verse} % <l>
\usepackage{xcolor} % named colors
\usepackage{xparse} % @ifundefined
\XeTeXdefaultencoding "iso-8859-1" % bad encoding of xstring
\usepackage{xstring} % string tests
\XeTeXdefaultencoding "utf-8"
\PassOptionsToPackage{hyphens}{url} % before hyperref, which load url package

% TOTEST
% \usepackage{hypcap} % links in caption ?
% \usepackage{marginnote}
% TESTED
% \usepackage{background} % doesn’t work with xetek
% \usepackage{bookmark} % prefers the hyperref hack \phantomsection
% \usepackage[color, leftbars]{changebar} % 2 cols doc, impossible to keep bar left
% \usepackage[utf8x]{inputenc} % inputenc package ignored with utf8 based engines
% \usepackage[sfdefault,medium]{inter} % no small caps
% \usepackage{firamath} % choose firasans instead, firamath unavailable in Ubuntu 21-04
% \usepackage{flushend} % bad for last notes, supposed flush end of columns
% \usepackage[stable]{footmisc} % BAD for complex notes https://texfaq.org/FAQ-ftnsect
% \usepackage{helvet} % not for XeLaTeX
% \usepackage{multicol} % not compatible with too much packages (longtable, framed, memoir…)
% \usepackage[default,oldstyle,scale=0.95]{opensans} % no small caps
% \usepackage{sectsty} % \chapterfont OBSOLETE
% \usepackage{soul} % \ul for underline, OBSOLETE with XeTeX
% \usepackage[breakable]{tcolorbox} % text styling gone, footnote hack not kept with breakable


% Metadata inserted by a program, from the TEI source, for title page and runing heads
\title{\textbf{ Li livres dou tresor }}
\date{1267}
\author{Brunetto Latini}
\def\elbibl{Brunetto Latini. 1267. \emph{Li livres dou tresor}}
\def\elsource{Brunetto Latini, \emph{Li livres dou trésor} (1267). Édition critique par Francis J. Carmody, Berkeley: University of California Press, 1948.}

% Default metas
\newcommand{\colorprovide}[2]{\@ifundefinedcolor{#1}{\colorlet{#1}{#2}}{}}
\colorprovide{rubric}{red}
\colorprovide{silver}{lightgray}
\@ifundefined{syms}{\newfontfamily\syms{DejaVu Sans}}{}
\newif\ifdev
\@ifundefined{elbibl}{% No meta defined, maybe dev mode
  \newcommand{\elbibl}{Titre court ?}
  \newcommand{\elbook}{Titre du livre source ?}
  \newcommand{\elabstract}{Résumé\par}
  \newcommand{\elurl}{http://oeuvres.github.io/elbook/2}
  \author{Éric Lœchien}
  \title{Un titre de test assez long pour vérifier le comportement d’une maquette}
  \date{1566}
  \devtrue
}{}
\let\eltitle\@title
\let\elauthor\@author
\let\eldate\@date


\defaultfontfeatures{
  % Mapping=tex-text, % no effect seen
  Scale=MatchLowercase,
  Ligatures={TeX,Common},
}


% generic typo commands
\newcommand{\astermono}{\medskip\centerline{\color{rubric}\large\selectfont{\syms ✻}}\medskip\par}%
\newcommand{\astertri}{\medskip\par\centerline{\color{rubric}\large\selectfont{\syms ✻\,✻\,✻}}\medskip\par}%
\newcommand{\asterism}{\bigskip\par\noindent\parbox{\linewidth}{\centering\color{rubric}\large{\syms ✻}\\{\syms ✻}\hskip 0.75em{\syms ✻}}\bigskip\par}%

% lists
\newlength{\listmod}
\setlength{\listmod}{\parindent}
\setlist{
  itemindent=!,
  listparindent=\listmod,
  labelsep=0.2\listmod,
  parsep=0pt,
  % topsep=0.2em, % default topsep is best
}
\setlist[itemize]{
  label=—,
  leftmargin=0pt,
  labelindent=1.2em,
  labelwidth=0pt,
}
\setlist[enumerate]{
  label={\bf\color{rubric}\arabic*.},
  labelindent=0.8\listmod,
  leftmargin=\listmod,
  labelwidth=0pt,
}
\newlist{listalpha}{enumerate}{1}
\setlist[listalpha]{
  label={\bf\color{rubric}\alph*.},
  leftmargin=0pt,
  labelindent=0.8\listmod,
  labelwidth=0pt,
}
\newcommand{\listhead}[1]{\hspace{-1\listmod}\emph{#1}}

\renewcommand{\hrulefill}{%
  \leavevmode\leaders\hrule height 0.2pt\hfill\kern\z@}

% General typo
\DeclareTextFontCommand{\textlarge}{\large}
\DeclareTextFontCommand{\textsmall}{\small}

% commands, inlines
\newcommand{\anchor}[1]{\Hy@raisedlink{\hypertarget{#1}{}}} % link to top of an anchor (not baseline)
\newcommand\abbr[1]{#1}
\newcommand{\autour}[1]{\tikz[baseline=(X.base)]\node [draw=rubric,thin,rectangle,inner sep=1.5pt, rounded corners=3pt] (X) {\color{rubric}#1};}
\newcommand\corr[1]{#1}
\newcommand{\ed}[1]{ {\color{silver}\sffamily\footnotesize (#1)} } % <milestone ed="1688"/>
\newcommand\expan[1]{#1}
\newcommand\foreign[1]{\emph{#1}}
\newcommand\gap[1]{#1}
\renewcommand{\LettrineFontHook}{\color{rubric}}
\newcommand{\initial}[2]{\lettrine[lines=2, loversize=0.3, lhang=0.3]{#1}{#2}}
\newcommand{\initialiv}[2]{%
  \let\oldLFH\LettrineFontHook
  % \renewcommand{\LettrineFontHook}{\color{rubric}\ttfamily}
  \IfSubStr{QJ’}{#1}{
    \lettrine[lines=4, lhang=0.2, loversize=-0.1, lraise=0.2]{\smash{#1}}{#2}
  }{\IfSubStr{É}{#1}{
    \lettrine[lines=4, lhang=0.2, loversize=-0, lraise=0]{\smash{#1}}{#2}
  }{\IfSubStr{ÀÂ}{#1}{
    \lettrine[lines=4, lhang=0.2, loversize=-0, lraise=0, slope=0.6em]{\smash{#1}}{#2}
  }{\IfSubStr{A}{#1}{
    \lettrine[lines=4, lhang=0.2, loversize=0.2, slope=0.6em]{\smash{#1}}{#2}
  }{\IfSubStr{V}{#1}{
    \lettrine[lines=4, lhang=0.2, loversize=0.2, slope=-0.5em]{\smash{#1}}{#2}
  }{
    \lettrine[lines=4, lhang=0.2, loversize=0.2]{\smash{#1}}{#2}
  }}}}}
  \let\LettrineFontHook\oldLFH
}
\newcommand{\labelchar}[1]{\textbf{\color{rubric} #1}}
\newcommand{\milestone}[1]{\autour{\footnotesize\color{rubric} #1}} % <milestone n="4"/>
\newcommand\name[1]{#1}
\newcommand\orig[1]{#1}
\newcommand\orgName[1]{#1}
\newcommand\persName[1]{#1}
\newcommand\placeName[1]{#1}
\newcommand{\pn}[1]{\IfSubStr{-—–¶}{#1}% <p n="3"/>
  {\noindent{\bfseries\color{rubric}   ¶  }}
  {{\footnotesize\autour{ #1}  }}}
\newcommand\reg{}
% \newcommand\ref{} % already defined
\newcommand\sic[1]{#1}
\newcommand\surname[1]{\textsc{#1}}
\newcommand\term[1]{\textbf{#1}}

\def\mednobreak{\ifdim\lastskip<\medskipamount
  \removelastskip\nopagebreak\medskip\fi}
\def\bignobreak{\ifdim\lastskip<\bigskipamount
  \removelastskip\nopagebreak\bigskip\fi}

% commands, blocks
\newcommand{\byline}[1]{\bigskip{\RaggedLeft{#1}\par}\bigskip}
\newcommand{\bibl}[1]{{\RaggedLeft{#1}\par\bigskip}}
\newcommand{\biblitem}[1]{{\noindent\hangindent=\parindent   #1\par}}
\newcommand{\dateline}[1]{\medskip{\RaggedLeft{#1}\par}\bigskip}
\newcommand{\labelblock}[1]{\medbreak{\noindent\color{rubric}\bfseries #1}\par\mednobreak}
\newcommand{\salute}[1]{\bigbreak{#1}\par\medbreak}
\newcommand{\signed}[1]{\bigbreak\filbreak{\raggedleft #1\par}\medskip}

% environments for blocks (some may become commands)
\newenvironment{borderbox}{}{} % framing content
\newenvironment{citbibl}{\ifvmode\hfill\fi}{\ifvmode\par\fi }
\newenvironment{docAuthor}{\ifvmode\vskip4pt\fontsize{16pt}{18pt}\selectfont\fi\itshape}{\ifvmode\par\fi }
\newenvironment{docDate}{}{\ifvmode\par\fi }
\newenvironment{docImprint}{\vskip6pt}{\ifvmode\par\fi }
\newenvironment{docTitle}{\vskip6pt\bfseries\fontsize{18pt}{22pt}\selectfont}{\par }
\newenvironment{msHead}{\vskip6pt}{\par}
\newenvironment{msItem}{\vskip6pt}{\par}
\newenvironment{titlePart}{}{\par }


% environments for block containers
\newenvironment{argument}{\itshape\parindent0pt}{\vskip1.5em}
\newenvironment{biblfree}{}{\ifvmode\par\fi }
\newenvironment{bibitemlist}[1]{%
  \list{\@biblabel{\@arabic\c@enumiv}}%
  {%
    \settowidth\labelwidth{\@biblabel{#1}}%
    \leftmargin\labelwidth
    \advance\leftmargin\labelsep
    \@openbib@code
    \usecounter{enumiv}%
    \let\p@enumiv\@empty
    \renewcommand\theenumiv{\@arabic\c@enumiv}%
  }
  \sloppy
  \clubpenalty4000
  \@clubpenalty \clubpenalty
  \widowpenalty4000%
  \sfcode`\.\@m
}%
{\def\@noitemerr
  {\@latex@warning{Empty `bibitemlist' environment}}%
\endlist}
\newenvironment{quoteblock}% may be used for ornaments
  {\begin{quoting}}
  {\end{quoting}}

% table () is preceded and finished by custom command
\newcommand{\tableopen}[1]{%
  \ifnum\strcmp{#1}{wide}=0{%
    \begin{center}
  }
  \else\ifnum\strcmp{#1}{long}=0{%
    \begin{center}
  }
  \else{%
    \begin{center}
  }
  \fi\fi
}
\newcommand{\tableclose}[1]{%
  \ifnum\strcmp{#1}{wide}=0{%
    \end{center}
  }
  \else\ifnum\strcmp{#1}{long}=0{%
    \end{center}
  }
  \else{%
    \end{center}
  }
  \fi\fi
}


% text structure
\newcommand\chapteropen{} % before chapter title
\newcommand\chaptercont{} % after title, argument, epigraph…
\newcommand\chapterclose{} % maybe useful for multicol settings
\setcounter{secnumdepth}{-2} % no counters for hierarchy titles
\setcounter{tocdepth}{5} % deep toc
\markright{\@title} % ???
\markboth{\@title}{\@author} % ???
\renewcommand\tableofcontents{\@starttoc{toc}}
% toclof format
% \renewcommand{\@tocrmarg}{0.1em} % Useless command?
% \renewcommand{\@pnumwidth}{0.5em} % {1.75em}
\renewcommand{\@cftmaketoctitle}{}
\setlength{\cftbeforesecskip}{\z@ \@plus.2\p@}
\renewcommand{\cftchapfont}{}
\renewcommand{\cftchapdotsep}{\cftdotsep}
\renewcommand{\cftchapleader}{\normalfont\cftdotfill{\cftchapdotsep}}
\renewcommand{\cftchappagefont}{\bfseries}
\setlength{\cftbeforechapskip}{0em \@plus\p@}
% \renewcommand{\cftsecfont}{\small\relax}
\renewcommand{\cftsecpagefont}{\normalfont}
% \renewcommand{\cftsubsecfont}{\small\relax}
\renewcommand{\cftsecdotsep}{\cftdotsep}
\renewcommand{\cftsecpagefont}{\normalfont}
\renewcommand{\cftsecleader}{\normalfont\cftdotfill{\cftsecdotsep}}
\setlength{\cftsecindent}{1em}
\setlength{\cftsubsecindent}{2em}
\setlength{\cftsubsubsecindent}{3em}
\setlength{\cftchapnumwidth}{1em}
\setlength{\cftsecnumwidth}{1em}
\setlength{\cftsubsecnumwidth}{1em}
\setlength{\cftsubsubsecnumwidth}{1em}

% footnotes
\newif\ifheading
\newcommand*{\fnmarkscale}{\ifheading 0.70 \else 1 \fi}
\renewcommand\footnoterule{\vspace*{0.3cm}\hrule height \arrayrulewidth width 3cm \vspace*{0.3cm}}
\setlength\footnotesep{1.5\footnotesep} % footnote separator
\renewcommand\@makefntext[1]{\parindent 1.5em \noindent \hb@xt@1.8em{\hss{\normalfont\@thefnmark . }}#1} % no superscipt in foot
\patchcmd{\@footnotetext}{\footnotesize}{\footnotesize\sffamily}{}{} % before scrextend, hyperref


%   see https://tex.stackexchange.com/a/34449/5049
\def\truncdiv#1#2{((#1-(#2-1)/2)/#2)}
\def\moduloop#1#2{(#1-\truncdiv{#1}{#2}*#2)}
\def\modulo#1#2{\number\numexpr\moduloop{#1}{#2}\relax}

% orphans and widows
\clubpenalty=9996
\widowpenalty=9999
\brokenpenalty=4991
\predisplaypenalty=10000
\postdisplaypenalty=1549
\displaywidowpenalty=1602
\hyphenpenalty=400
% Copied from Rahtz but not understood
\def\@pnumwidth{1.55em}
\def\@tocrmarg {2.55em}
\def\@dotsep{4.5}
\emergencystretch 3em
\hbadness=4000
\pretolerance=750
\tolerance=2000
\vbadness=4000
\def\Gin@extensions{.pdf,.png,.jpg,.mps,.tif}
% \renewcommand{\@cite}[1]{#1} % biblio

\usepackage{hyperref} % supposed to be the last one, :o) except for the ones to follow
\urlstyle{same} % after hyperref
\hypersetup{
  % pdftex, % no effect
  pdftitle={\elbibl},
  % pdfauthor={Your name here},
  % pdfsubject={Your subject here},
  % pdfkeywords={keyword1, keyword2},
  bookmarksnumbered=true,
  bookmarksopen=true,
  bookmarksopenlevel=1,
  pdfstartview=Fit,
  breaklinks=true, % avoid long links
  pdfpagemode=UseOutlines,    % pdf toc
  hyperfootnotes=true,
  colorlinks=false,
  pdfborder=0 0 0,
  % pdfpagelayout=TwoPageRight,
  % linktocpage=true, % NO, toc, link only on page no
}

\makeatother % /@@@>
%%%%%%%%%%%%%%
% </TEI> end %
%%%%%%%%%%%%%%


%%%%%%%%%%%%%
% footnotes %
%%%%%%%%%%%%%
\renewcommand{\thefootnote}{\bfseries\textcolor{rubric}{\arabic{footnote}}} % color for footnote marks

%%%%%%%%%
% Fonts %
%%%%%%%%%
\usepackage[]{roboto} % SmallCaps, Regular is a bit bold
% \linespread{0.90} % too compact, keep font natural
\newfontfamily\fontrun[]{Roboto Condensed Light} % condensed runing heads
\ifav
  \setmainfont[
    ItalicFont={Roboto Light Italic},
  ]{Roboto}
\else\ifbooklet
  \setmainfont[
    ItalicFont={Roboto Light Italic},
  ]{Roboto}
\else
\setmainfont[
  ItalicFont={Roboto Italic},
]{Roboto Light}
\fi\fi
\renewcommand{\LettrineFontHook}{\bfseries\color{rubric}}
% \renewenvironment{labelblock}{\begin{center}\bfseries\color{rubric}}{\end{center}}

%%%%%%%%
% MISC %
%%%%%%%%

\setdefaultlanguage[frenchpart=false]{french} % bug on part


\newenvironment{quotebar}{%
    \def\FrameCommand{{\color{rubric!10!}\vrule width 0.5em} \hspace{0.9em}}%
    \def\OuterFrameSep{\itemsep} % séparateur vertical
    \MakeFramed {\advance\hsize-\width \FrameRestore}
  }%
  {%
    \endMakeFramed
  }
\renewenvironment{quoteblock}% may be used for ornaments
  {%
    \savenotes
    \setstretch{0.9}
    \normalfont
    \begin{quotebar}
  }
  {%
    \end{quotebar}
    \spewnotes
  }


\renewcommand{\headrulewidth}{\arrayrulewidth}
\renewcommand{\headrule}{{\color{rubric}\hrule}}

% delicate tuning, image has produce line-height problems in title on 2 lines
\titleformat{name=\chapter} % command
  [display] % shape
  {\vspace{1.5em}\centering} % format
  {} % label
  {0pt} % separator between n
  {}
[{\color{rubric}\huge\textbf{#1}}\bigskip] % after code
% \titlespacing{command}{left spacing}{before spacing}{after spacing}[right]
\titlespacing*{\chapter}{0pt}{-2em}{0pt}[0pt]

\titleformat{name=\section}
  [block]{}{}{}{}
  [\vbox{\color{rubric}\large\raggedleft\textbf{#1}}]
\titlespacing{\section}{0pt}{0pt plus 4pt minus 2pt}{\baselineskip}

\titleformat{name=\subsection}
  [block]
  {}
  {} % \thesection
  {} % separator \arrayrulewidth
  {}
[\vbox{\large\textbf{#1}}]
% \titlespacing{\subsection}{0pt}{0pt plus 4pt minus 2pt}{\baselineskip}

\ifaiv
  \fancypagestyle{main}{%
    \fancyhf{}
    \setlength{\headheight}{1.5em}
    \fancyhead{} % reset head
    \fancyfoot{} % reset foot
    \fancyhead[L]{\truncate{0.45\headwidth}{\fontrun\elbibl}} % book ref
    \fancyhead[R]{\truncate{0.45\headwidth}{ \fontrun\nouppercase\leftmark}} % Chapter title
    \fancyhead[C]{\thepage}
  }
  \fancypagestyle{plain}{% apply to chapter
    \fancyhf{}% clear all header and footer fields
    \setlength{\headheight}{1.5em}
    \fancyhead[L]{\truncate{0.9\headwidth}{\fontrun\elbibl}}
    \fancyhead[R]{\thepage}
  }
\else
  \fancypagestyle{main}{%
    \fancyhf{}
    \setlength{\headheight}{1.5em}
    \fancyhead{} % reset head
    \fancyfoot{} % reset foot
    \fancyhead[RE]{\truncate{0.9\headwidth}{\fontrun\elbibl}} % book ref
    \fancyhead[LO]{\truncate{0.9\headwidth}{\fontrun\nouppercase\leftmark}} % Chapter title, \nouppercase needed
    \fancyhead[RO,LE]{\thepage}
  }
  \fancypagestyle{plain}{% apply to chapter
    \fancyhf{}% clear all header and footer fields
    \setlength{\headheight}{1.5em}
    \fancyhead[L]{\truncate{0.9\headwidth}{\fontrun\elbibl}}
    \fancyhead[R]{\thepage}
  }
\fi

\ifav % a5 only
  \titleclass{\section}{top}
\fi

\newcommand\chapo{{%
  \vspace*{-3em}
  \centering % no vskip ()
  {\Large\addfontfeature{LetterSpace=25}\bfseries{\elauthor}}\par
  \smallskip
  {\large\eldate}\par
  \bigskip
  {\Large\selectfont{\eltitle}}\par
  \bigskip
  {\color{rubric}\hline\par}
  \bigskip
  {\Large TEXTE LIBRE À PARTICPATION LIBRE\par}
  \centerline{\small\color{rubric} {hurlus.fr, tiré le \today}}\par
  \bigskip
}}

\newcommand\cover{{%
  \thispagestyle{empty}
  \centering
  {\LARGE\bfseries{\elauthor}}\par
  \bigskip
  {\Large\eldate}\par
  \bigskip
  \bigskip
  {\LARGE\selectfont{\eltitle}}\par
  \vfill\null
  {\color{rubric}\setlength{\arrayrulewidth}{2pt}\hline\par}
  \vfill\null
  {\Large TEXTE LIBRE À PARTICPATION LIBRE\par}
  \centerline{{\href{https://hurlus.fr}{\dotuline{hurlus.fr}}, tiré le \today}}\par
}}

\begin{document}
\pagestyle{empty}
\ifbooklet{
  \cover\newpage
  \thispagestyle{empty}\hbox{}\newpage
  \cover\newpage\noindent Les voyages de la brochure\par
  \bigskip
  \begin{tabularx}{\textwidth}{l|X|X}
    \textbf{Date} & \textbf{Lieu}& \textbf{Nom/pseudo} \\ \hline
    \rule{0pt}{25cm} &  &   \\
  \end{tabularx}
  \newpage
  \addtocounter{page}{-4}
}\fi

\thispagestyle{empty}
\ifaiv
  \twocolumn[\chapo]
\else
  \chapo
\fi
{\it\elabstract}
\bigskip
\makeatletter\@starttoc{toc}\makeatother % toc without new page
\bigskip

\pagestyle{main} % after style

  
\chapteropen
\part[{Livre premier}]{Livre premier}\phantomsection
\label{tresor\_1}\renewcommand{\leftmark}{Livre premier}


\chaptercont

\chapteropen
\chapter[{.I.I. Cis livres est apielés tresors et parole de la naissance de toutes coses}]{\textsc{.I.I.} Cis livres est apielés tresors et parole de la naissance de toutes coses}\phantomsection
\label{tresor\_1-1}

\chaptercont
\noindent Cis livres est apielés Tresors. Car si come li sires ki vuet en petit lieu amasser cose de grandisme vaillance, non pas pour son delit solement, mes pour acroistre son pooir et pour aseurer son estat en guerre et en pais, i met les plus chieres choses et les plus precieus joiaus k’il puet selonc sa bonne entencion ; tout autresi est li cors de cest livre compilés de sapience, si come celui ki est estrais de tous les membres de philosophie en une sonme briement. Et la premiere partie de cest tresor est autresi comme de deniers contans, pour despendre tousjours es coses besoignables ; c’est a dire k’ele traite dou comencement du siecle, et de l’ancieneté des vielles istores et de l’establissement dou monde et de la nature de toutes coses en some.\par
Et çou apertient a la premiere science de philosophie, c’est a theorika, selonc ce ke li livres parole ci aprés. Et si come sans deniers n’aroit nule moieneté entre les oevres des gens, ki adreçast les uns contre les autres, autresi ne puet nus hons savoir des autres coses plainement s’il ne set ceste premiere partie du livre.\par
La seconde partie ki traite des vices et des viertus est de precieuses pieres, ki donent a home delit et vertu, c’est a dire quex coses on doit faire et quels non, et moustre la raison pour quoi ; et çou apiertient a la seconde et a la tierce partie de philosophie, c’est a pratike et a logike.\par
La terce partie du tresor est de fin or, c’est a dire k’ele ensegne a home parler selonc la doctrine de retorike, et coment li sires doit governer ses gens ki souz li sont, meismement selonc les us as ytaliens ; et tout ce apertient a la seconde sience de philosophie, c’est pratike. Car si comme li ors sormonte toutes manieres de metal, autresi est la sience de bien parler et de governer gens plus noble de nul art du monde. Et por ce ke li tresors ki ci est ne doit pas iestre donés se a home non ki soit souffissables a si haute richece, la baillerai jou a toi biaus dous amis, car tu en ies bien dignes selonc mon jugement.\par
Et si ne di je pas que le livre soit estrais de mon povre sens ne de ma nue science ; mais il ert ausi comme une bresche de miel coillie de diverses flours, car cist livres est compilés seulement des mervilleus dis des autours ki devant nostre tans ont traitié de philosophie, cascuns selonc çou k’il en savoit partie ; car toute ne la puet savoir hons terriens, pour çou ke philosophie est la rachine de qui croissent toutes les siences ke hom puet savoir, tot autresi comme une vive fontaine dont maint ruisssiel issent et decourent ça et la, si ke li un boivent de l’une et li autre de l’autre ; mais c’est diversement, car li un en boivent plus et li autre mains, sans estancier la fontaine.\par
Pour çou dist Boesces el livre de la Consolation que il le vit en samblance de dame, en tel abit et en si trés mervilleuse poissance qu’ele croissoit quant il li plaisoit, tant ke son chief montoit sor les estoiles et sour le ciel, et porveoit amont et aval selonc droit et selonc verité. A ce commence mon conte, car aprés bon comencement ensiut bone fin. Et nostre empereres dist el Livre de Loi que comencement est grignour partie de la chose.\par
Et se aucuns demandoit pour quoi cis livres est escris en roumanç, selonc le raison de France, puis ke nous somes italien, je diroie que c’est pour \textsc{.ii.} raisons, l’une ke nous somes en France, l’autre por çou que la parleure est plus delitable et plus commune a tous langages.
\chapterclose


\chapteropen
\chapter[{.I.II. De philosophie et de ses parties}]{\textsc{.I.II.} De philosophie et de ses parties}\phantomsection
\label{tresor\_1-2}

\chaptercont
\noindent Philosophie est verais enchiercemens des choses naturaus, et des divines, et des humaines, tant comme a home est possible d’entendre. Dont il avint ke aucun ki s’estudierent a enquerre et a savoir la verité de ces \textsc{.iii.} coses ki sont dites en philosophie, c’est a dire de la divinité et des choses de nature et des humaines choses, furent droit fiz de philosophie, et por ce furent il apielé philosophe.\par
Et il fu voirs que au commencement du siecle, quant les gens qui soloient vivre a loi de biestes conurent premierement la dignité de la raison et de la cognoissance, ke Dieus lor avoit donee, et il volrent savoir la verité des choses ki sont en philosophie, il cheirent en \textsc{.iii.} questions ; une estoit de savoir la nature de toutes choses celestiaus et terrienes, la seconde et la tierce sont des humaines choses. Dont la premiere est de savoir quex choses on doit faire et quex non, la tierche est de savoir raison et prueve pour quoi on doit les unes faire et les autres non.\par
Et puis ke ces \textsc{.iii.} questions furent traities et ventilees longuement entre les autres sages clers et entre les philosophes, il troverent en philosophie lor mere \textsc{.iii.} principaus membres, c’est a dire \textsc{.iii.} manieres de sciences pour ensegnier et prover la veraie raison des \textsc{.iii.} questions que jou ai devisees ci devant.
\chapterclose


\chapteropen
\chapter[{.I.III. Coment nature de toutes choses est devisee en .iii. manieres, selonc theorique}]{\textsc{.I.III.} Coment nature de toutes choses est devisee en \textsc{.iii.} manieres, selonc theorique}\phantomsection
\label{tresor\_1-3}

\chaptercont
\noindent Toute la premiere ce est theorike, et est cele propre science ki nous ensegne la premiere question, de savoir et de conoistre la nature de toutes coses celestiaus et terrienes. Mes pour çou ke ces natures sont vaires et diverses, a ce que autre nature est des choses ki n’ont point de cors ne ne conversent entor les corporaus choses, autre nature est des choses qui ont cors et conversent entre les corporaus choses, et une autre nature est des choses ki n’ont point de cors et sont entour les corporaus choses. Pour ce fu il bien raisnable cose ke ceste sience de theorike fesist de son cors \textsc{.iii.} autres sciences, pour demoustrer les \textsc{.iii.} diverses natures que jou ai devisees. Et ces sciences sont apielees en lor non theologie, phisike, et mathematike.\par
La premiere et la plus haute des \textsc{.iii.} sciences ki sont estraites de theorique si est theologie, ki trespasse le ciel et nos moustre les natures des choses ki n’ont point de cors ne se conversent entour les corporaus choses, en tel maniere ke par li cognoissons nous Deu le tot poissant. Par lui creons nous la Sainte Trinité du pere et du fil et du saint esperit en une seule personne. Par li avons nous la foi catholique et la loi de sainte eglise, et briement nous ensegne tout ce que a divinité apertient.\par
La seconde si est phisique, par qui nous savons la nature des choses ki ont cors et conversent entor les corporaus choses, c’est a dire des homes et des biestes, des oisiaus, des poissons, des plantes et des pieres, et des autres corporaus choses ki sont entre nous.\par
La tierce est matematike, par qui nous savons les natures des choses ki n’ont point de cors et sont entor les corporaus choses ; et ces choses sont de \textsc{.iiii.} diverses manieres, et pour çou sont \textsc{.iiii.} sciences el cors de mathematique, et sont apielees par droit non arismetique, musique, geometrie, et astronomie.\par
La premiere de ces \textsc{.iiii.} siences c’est arismetike, ki nous ensegne conter, nombrer, et joindre l’un conte sour l’autre, et mutepliier l’un parmi l’autre, et les uns oster des autres et partir et deviser en plusour parties, et de ce sont li ensegnement de l’abaque et de l’augorisme.\par
La seconde est musique, ki nous ensegne faire vois, sons en chant et en cytoles, et en orghenes et en autres estrumens acordables uns contre les autres, pour delit des gens, u en eglise por le service Nostre Signeur.\par
La tierce si est geometrie, par qui nous savons les mesures et les proportions des choses par lonc et par lé et par hautece, c’est la sience par quoi li anchien sage s’efforchierent par soutillité de geometrie de trover la grandeur dou ciel et de la tiere, et de la hautece entre l’un et l’autre et maintes autres proportions ki a mervillier font.\par
La quarte science est astronomie, ki nous ensegne trestot l’ordenement du ciel et du firmament, et des estoiles, et le cours des \textsc{.vii.} planetes par son zodiaque, c’est par mi les \textsc{.xii.} signaus, et comment se muent li tans a caut ou a froidure, ou a pluie ou a sech, ou a vent, et par raison ki est establies es estoiles.
\chapterclose


\chapteropen
\chapter[{.I.IIII. Des choses qe l’on doit faire et quex non, selonc pratiqe}]{\textsc{.I.IIII.} Des choses qe l’on doit faire et quex non, selonc pratiqe}\phantomsection
\label{tresor\_1-4}

\chaptercont
\noindent Pratique est la seconde science de philosophie, ki nos ensegne ke l’en doit faire et quoi non. Et a la verité dire ce puet estre en \textsc{.iii.} manieres ; car une maniere est de faire aucune chose et eschiver autre por governer li meisme, et une autre maniere est pour gouverner sa maisnie et sa maison et son avoir et son eritage, et une autre maniere est por governer gens u \textsc{.i.} regne u \textsc{.i.} peuple ou une cité, en pais et en guerre.\par
Mais puis ke li ancien sage cognurent ces \textsc{.iii.} diversités, il covint k’il trovaissent en pratique \textsc{.iii.} manieres de science pour adrecier les \textsc{.iii.} manieres de governer soi et autrui, ce sont etique, iconomike, politique.\par
La premiere des \textsc{.iii.} sciences c’est etique, ki nos ensegne governer nos premierement, a ensivre honeste vie, et faire les vertueuses oevres, et garder des vices ; car nus ne poroit vivre au monde, ne bien ne honestement, ne profiter a soi ne as autres, s’il ne governoit sa vie et adreçoit ses meurs selon les vertus.\par
La seconde est yconomique, ki nos ensegne a garder et nos fiz et nos gens et nos meismes, et si nos ensegne garder et acroistre nos possessions et nos iretages, et aver meuble et chatel por despendre et pour retenir, selonc ce que lieus et tens muent.\par
La tierce est politique ; et sans faille c’est la plus haute science et dou plus noble mestier ki soit entre les homes, car ele nos ensegne governer les estranges gens d’un regne et d’une vile, un peuple et une comune en tens de pes et de guerre, selonc raison et selonc justice.\par
Et si nous ensegne tous les ars et toz les mestiers ki a vie d’ome sont besonable. Ce est en \textsc{.ii.} manieres, car l’une est en oevre et l’autre en paroles. Cele ki est en oevre sont li mestier ke l’en oevre tousjors des mains et des piés, ce sont sueurs, drapiers, cordewaniers, et ces autres mestiers ki sont besoignable a la vie des homes, et sont apielés mecaniques. Cele ki est en paroles sont celes ke l’en oevre de sa bouche et de sa langue, et sont en \textsc{.iii.} manieres, sor qui sunt establies \textsc{.iii.} sciences, gramatique, dialetique, et rettorique.\par
Dont la premiere est gramatique, ki est fondement et porte et entree des autres sciences ; ki nos ensegne a parler et escrire et lire a droit sans vice de barbarisme et de solercisme.\par
La seconde est dyaletike, ki nous ensegne prover nos dis et nos paroles, par tele raison et par teus argumens ki donent foi as paroles ke nous avons dites, si k’eles samblent voires et provables a estre voires.\par
La tierce science est retorique, cele noble science ke nous ensegne trover et ordener et dire paroles bonnes et bieles et plaines de sentences selonc ce ke la nature requiert. C’est la mere des parliers, c’est l’ensegnement de diteours, c’est la science ki adrece le monde premierement a bien fere, et ki encore l’adresce par les predications des sain homes, par les divines escriptures, et par la loi ki les gens governe a droit et a joustice. C’est la science de qui Tulles dit en son livre que celui a hautisme chose conquise ki de ce trespasse les homes dont li home trespassent tous les autres animaus, c’est de la parleure.\par
Por ce devroit cascuns pener de savoir le, se sa nature li suefre et li aide, car sans nature et sans ensegnement ne la puet nus conquerre. Et a verité dire de li avons nous mestier en toutes besoignes tousjours, et maintes choses grans et petites poons nous faire par solement bien dire çou ki covient, ke nous ne le poriens faire par force d’armes ne par autre engin.
\chapterclose


\chapteropen
\chapter[{.I .V. Porquoi on doit les unes choses faire et les autres non, selonc logique}]{\textsc{.I .V.} Porquoi on doit les unes choses faire et les autres non, selonc logique}\phantomsection
\label{tresor\_1-5}

\chaptercont
\noindent Logike est la tierce science de philosophie, cele proprement ki ensegne prover et moustrer raison pour quoi on doit les unes choses faire et les autres non. Et ceste raison ne puet nus hom bien moustrer se par paroles non, donc est logique science par laquele prover et dire raison l’en puet, et coment çou que nous disons est ausi voirs com nous metons avant. Et c’est en \textsc{.iii.} manieres, et ausi sont \textsc{.iii.} sciences, dialetique, et fidique, sophistique.\par
Dont la premiere est dyaletique, et ensegne tencier, contendre, et desputer, les uns contres les autres, et faire questions et deffenses.\par
La seconde est fidike, et ensegne prover ke ses paroles k’il a dites sont veritables et que la cose est ensi come il dist par droites raisons et par verais argumens.\par
La tierce science de logike est sophistique, ki ensegne prover ke les paroles ke l’en dist soient veraies ; mais ce prueve il par mal engin et par fausses raisons et par sophismes, c’est par argument ki ont samblance et coverte de verité, mais n’i a cose se fause non.\par
Jusques ci a devisé li contes assés briefment et apertement ki est philosophie, et toutes les sciences que l’en puet savoir, dont philosophie est mere et fontaine. Desormés s’en wet torner a sa matire, c’est a theorike, ki est la premiere partie de philosophie, por demoustrer \textsc{.i.} poi de la nature des choses du ciel et de la tiere. Mais ce sera au plus briefment que li mestres pora.
\chapterclose


\chapteropen
\chapter[{.I.VI. Coment diex fist totes choses au commencement}]{\textsc{.I.VI.} Coment diex fist totes choses au commencement}\phantomsection
\label{tresor\_1-6}

\chaptercont
\noindent Li sage dient que Nostre Sires Dieus, ki est commencement de toutes choses, fist et crea le monde et toutes autres choses, en \textsc{.iiii.} manieres ; car tout avant ot il en sa pensee l’ymage et la figure coment il fist le monde et les autres choses, et ce ot il tousjors eternaument, si ke cele pensee n’ot onques comencement. Et ceste ymagination est apelee mondes arquetipes, c’est a dire mondes en samblance.\par
Aprés ce fist il de noient une grosse matire, que n’estoit de nule figure ne d’aucune samblance, mais ele estoit de si faite forme et si apareillie que il en pooit forgier et retraire ce k’il voloit, et ceste matire est apielee ylem. Puis k’il ot ce fait si comme a lui plot, mist il en oevre et en fait son proposement, et fist le monde et les autres criatures, selonc sa porveance. Et ja soit ce k’il le peust faire faire tost et isnielement, il n’i volt onques corre, ançois i mist \textsc{.vi.} jours, et au septime se reposa.\par
Que la Bible nous raconte que au comencement Nostre Sires commanda que li mondes fust fais, c’est a dire ciel, tiere, euue, jors et clarté, et les angeles, et ke la clartés fust devisee des tenebres. Et puis k’il le commanda, fu fait de noient, et ce fu le premier jor dou siecle ; de quoi tesmoignent li plusor que celi jour est \textsc{.xiiii.} jours a l’issue du mois de mars.\par
Au secont jour fu establis le firmament. Au tierç jour commanda ke la tiere fust devisee de la mer et des autres euues, et que toutes choses ki sont racinees en terre fust faite celui jour. Au quart jour commanda ke le soleil et la lune et les estoiles et tous les luminaires fussent faites. Au quint jour commanda ke poisson fussent fait, et toutes les creatures ki vivent en euues. Au \textsc{.vi.}te jour commanda ke tous animaus fussent fait, et lors fist il Adam a sa samblance, et puis fist Eve de la coste Adan, et lors crea il ames de noient et les mist en lor cors.
\chapterclose


\chapteropen
\chapter[{.I.VII. Coment chascunes choses furent faites de nient}]{\textsc{.I.VII.} Coment chascunes choses furent faites de nient}\phantomsection
\label{tresor\_1-7}

\chaptercont
\noindent Par ces paroles poons nous entendre ke Dieus fist seulement home, car de toutes les autres choses commanda il k’eles fuissent faites, et plus est a faire que a commander. Mais comment k’il fust, il i a \textsc{.ii.} manieres ; car aucunes choses furent faites de noient, ce sont li angele et li mondes et la clarté et ylem, ki furent faites au commencement ; mais l’ame est criee de noient, et tousjors crie il novieles ames. Et l’autre maniere est ke toutes les autres coses furent faites d’aucune autre matire.
\chapterclose


\chapteropen
\chapter[{.I.VIII. De l’office de nature}]{\textsc{.I.VIII.} De l’office de nature}\phantomsection
\label{tresor\_1-8}

\chaptercont
\noindent Or avés oii \textsc{.iii.} manieres coment Dieus fist toutes choses. La quarte maniere fu que quant il ot tout fait il ordena la nature de cascune chose par soi, et lor establi ciertain cours coment eles devoient naistre et comencier et finer et morir, et la force et la propriété et la nature de chascune.\par
Et sachés que totes choses ki ont comencement, c’est a dire ki furent faites d’aucune matire, si aront fin ; mais celes ki furent criees de noient n’aront pas fin. Et sor ceste quarte matire est li office de Nature, ki est viaire de son verai pere. Il est criatour ele est criature. Il est sans comencement ele fu commencie. Il est commanderes ele est obeissant. Il n’aura ja fin ele finera o tout son labour. Il est tous puissans ele n’a pooir, se de ce non que Dieus li a otroiet. Il set toutes choses passees, presentes et futures, ele ne set se ce non que il li moustre. Il ordena le monde ele ensit ses ordenemens. Ensi poons nous connoistre ke chascune chose est sousmise a sa nature ; et nanporquant cil ki tot fist puet remuer et cangier le cours de nature par divin miracle, si com il fist en la glorieuse virgene Marie, ki conchut le fil Dieu sans carneil cognoissement, et fu virgene nete devant et apriés, et il meismes resuscita de la mort. Ces et autres divins miracles ne sont mie contre nature.\par
Et se aucuns deist ke Dieus ordena certain cours a nature et puis fait contre le cors, k’il remue son premier talent ; et s’il remue son talent donc n’est il mie parmanables  : je li diroie ke nature n’a ke faire en chose que Dieus retint en sa poesté, et ke tosjors ot li peres en la volenté la naissance du fil et la passion et la resurrection si comme ele avint.
\chapterclose


\chapteropen
\chapter[{.I.VIIII. Ke en dieu n’a nul tens}]{\textsc{.I.VIIII.} Ke en dieu n’a nul tens}\phantomsection
\label{tresor\_1-9}

\chaptercont
\noindent Car la eternité Deu est devant tostans, en lui n’est pas divisions dou tans alé ou dou present ou de celui ki est a venir. Mais toutes choses sont presentes a lui, pour çou k’il les embrache toutes par sa eternité. Mais ces \textsc{.iii.} tens sont en nous, raison coment : l’en dit dou tens ki est alés, je ai doné ; et du tens ki est a venir dist li hons, je donrai ; et du tens ki est presens dist li hons, je done. Mais Dieus le comprent si universaument ke tout ce ke il fist ou k’il fet est a lui ausi comme present.\par
Et sachiés que tens n’apartient pas a creatures ki sont sor le ciel, mais a celes ki sont desous. Et devant le coumencement du monde n’estoit nul tens, por ce ke tens fu fait et establis a celi commencement, et por ce est apielé commencement ke totes choses furent lors commencies.\par
Mais li tans n’a nule espasse corporaument ; car por \textsc{.i.} poi s’en vont ançois k’il viegnent, et pour çou n’a il en aus point de fermeté, ke toutes creatures se muevent, et muent isnielement Pour ce di je que ces \textsc{.iii.} tens, c’est le preterit, le present, et celi ki est a venir, ne sont pas se en la pensee non, ki se sovient des choses alees et esgarde les presentes et atent les futures.
\chapterclose


\chapteropen
\chapter[{.I.X. Que en dieu n’a nul movement}]{\textsc{.I.X.} Que en dieu n’a nul movement}\phantomsection
\label{tresor\_1-10}

\chaptercont
\noindent Ce n’est pas ensi en Dieu, mais toz ensamble presentiaument. Por ce faillent cil ki dient ke en lui fu li tans mués quant il li vint noviele pensee dou monde faire. Mais je di ke ceste fachon fu en son conseil eternalment, et que devant le comencement n’estoit nul tens, mes sa eternité ; car le tens fu comenciés par le creature, non pas la creature par le tens.\par
Aucun demandent ke Dieus faisoit ains ke le monde fust fais et ke soudainement li vint en volenté de faire le monde ; et por ce quident il k’il volt aucune fois ce k’il ne voloit primes. Mais je di ke novele volenté ne fu ele pas ; car ja soit ce que li mondes ne fust encore fais, toutes fois estoit il en son eterneil conseil.\par
Et d’autre part Deus est sa volenté et sa volenté est Dieus ; mais Dieus est eternel et sans remuance, dont est sa volenté eternel et sans remuance.\par
Cele matire de quoi ces choses furent formees les desvance de naissance, non mie de tens, autresi comme li sons devant le chant ; car li sons est devant le chant por ce que la douçor dou chant apertient au son, mais li sons si n’apertient pas a la douçour dou chant, et neporquant andeus sont il ensamble.\par
Et de cele matire fu dit ça ariere k’ele n’avoit figure ne samblance nule, por ce ke encore n’estoient forgies les choses ki devoient estre faites, mais cele matire estoit de noient.\par
Je di ke au commencement clarté fu devisee de tenebres ; et ja soit ce ke Dieus dit par la bouche dou prophete, je sui cil ki fait la clarté et crie les tenebres, ne doit nus hom croire que tenebres aient cors ; mais la nature des angeles ki ne trespasserent est apielee clarté, et la nature de ciaus ki trespasserent est apelee tenebres. Et pour ce dist la Bible que au comencement fu la clartee devisee des tenebres, c’est a dire que Dieus cria les angeles, et de l’un de ciaus fist il la clarté et des autres les tenebres ; més les bons cria il et les aprist, et les mauvais cria il et ne les aprist. Dieus fist toutes choses molt bonnes ; donc n’est il nule chose malvaise par nature ; mais se nous usons d’eles malvaisement, eles devienent malvaises, et ensi change la bonne nature.
\chapterclose


\chapteropen
\chapter[{.I.XI. Coment le mal fu trovés}]{\textsc{.I.XI.} Coment le mal fu trovés}\phantomsection
\label{tresor\_1-11}

\chaptercont
\noindent Le mal fu trovés par le diauble, non pas criés ; et por ce est il noiens, car ce ki est sans Dieu est noient, et Deus ne fist pas le mal. Mais li hons quide ke Dieus feist le bien et li diaubles feist le mal, et ensi croient k’il soient \textsc{.ii.} natures, une de bien et une autre de mal. Mais il sont deceu, por ce que mal n’est pas par nature, ains fu trovés par le diauble lors que li angeles ki bons estoit par son orguel devint malvés et trova le mal.\par
Et que mal ne soit par nature apert tot clerement, car toute nature ou ele est parmanable, c’est Dieus, ou ele est remuable, c’est criature ; mais par nature n’est il mie, por ce ke s’il vient sor la bone nature si la fet il vicieuse, et quant il s’en depart la nature demeure. Et cest mal n’est pas en nul lieu, et d’autrepart nule chose anuie ki est naturel.\par
Aucun demandent por coi Dieus laissa nestre le mal, et je di por ce ke la biauté de la bone nature fust cogneue par son contraire. Car \textsc{.ii.} choses contraires quant eles sont ensamble l’une contre l’autre, eles sont plus aparissans : se tu rooignes les sorcis d’un home, tu en ostes petites choses, mais toz li cors en devient plus lais. Tot autresi se tu blasmes entre toutes les criatures une petite vermine k’ele soit malvaise par nature, certes tu fez tort a toutes criatures. \par
Touz maus sont venu a tot l’umain linage por le pechié du premier home ; et por ce tous maus ki sont en nous, u il sont par naissance, ou il sont par nostre coupe. Plusor dient ke li mal sont es criatures, c’est ou fu por ce k’il art, ou fer por ce k’il ocit ; mais il ne consirent mie que ces choses soient bones par nature, mais par le pechié de l’home sont devenues nuisans, car devant le pechié li estoient sozmises dou tout.\par
Ensi sont eles nuisans as homes par lor pechiés, non pas par nature ; si come la clarté, qui est bone par nature, mais ele est malvaise as ieus malades, et ce avient par vices des ieus, non pas de la clarté.\par
Et l’om fait mal en \textsc{.ii.} manieres, u en la pensee ou en l’uevre ; et cil ki est en la pensee est apielee iniquités, et est en \textsc{.iii.} manieres ou en temptation ou en delit ou en consentir. Cil ki est en oevre est apielés pechiés, et est autresi en \textsc{.iii.} manieres, ou en paroles ou en fais ou en perseverance.\par
Mais li prophetes David au commencement dou sautier ne nome ke \textsc{.iii.} manieres de pechiés, la premiere ki est en la pensee, ki vient par temptation et par mauvais conseil, la seconde est en oevre, la tierce est en la perseverance dou mal, en quoi on donne as autres essample de malfaire ; c’est senefiiet par les \textsc{.iii.} mors que Jhesucris resuscita, une ki estoit dedens la maison, c’est la pensee, l’autre ki estoit a l’huis de la maison, c’est l’uevre, l’autre ki estoit en voie, et c’est ce ki pardurront en malfaire devant tous.
\chapterclose


\chapteropen
\chapter[{.I.XII. De la nature des angles}]{\textsc{.I.XII.} De la nature des angles}\phantomsection
\label{tresor\_1-12}

\chaptercont
\noindent Angele sont esperit naturelement. Et lor nature est muable ; mais la charités perdurable les garde sans corruption, et ensi sont parmanable par grace non mie par nature ; car se ce fust par nature, le diauble ne cheist.\par
Devant toutes criatures dou monde furent fait li angele ; et devant les autres fu fais celui ki est diaubles - non mie par quantité de tans, mais par ordre de signourie k’il ot sour les autres. Et par cele seurté chef il sans retour, et il chei primes que li hom ne fust fais ; car maintenant que li diables fu criés, monta il en orguel, et chei du ciel ; puis descendi il a Adan et le fist cheoir et ensi chei li hom et li diaubles. Mais li hom retorna a Dieu pour ce k’il s’en repenti et reconut k’il estoit desous Dieu ; mais li diables dist k’il estoit pareil a Dieu et graindres ke Dieu, et por ce k’il ne se repenti mie ne trova pardon.\par
Mais je di ke li hom trueve pardon pour ce ke la foiblece de pechier vint en lui de par le cors, ki est de boe et de tiere moiste, et li angele pechierent ki n’estoient chargié de nule charnel maladie.\par
Puis que li mauvés angele furent cheus furent confermé li autre en bienfere, et de ce dist la Bible que au secont jour fu establis le firmament, et fu le ciel apielé firmament.\par
.ix. sont li ordre des bons angeles, et tuit sont establi par gré et par dignité, et chascuns obeist a l’autre selonc son office. Ces ordenes sont angeles, arcangeles, trosnes, dominations, virtus, principaus, poestés, cherubin et sceraphin.\par
Li angele sevent toutes choses par la parole Dieu avant k’eles soient fetes et ki encor sont a venir as homes. Et ja soit ce que li mauvais angele perdissent la sainteé, il ne perdirent la vertu du sens ki fu doné as angeles. Et ce k’il puent savoir devant est en \textsc{.iii.} manieres, ou por soutillité de nature, ou par esperiance dou tens, ou par revelation de poestés ki mainent deseure.\par
Quant Dieus se courouce au monde, il envoie les mauvais angeles en office de vengance, mais toutesfoies les constraint k’il ne facent tant de mal com il desirent. Mais les bons angeles envoie Deus en office de salut des homes, et por ce dient li plusor ke tot home ont angeles ki sont provost por aus garder. Dont cist vier sont dit : angele qui custos meus es virtute superna, me tibi commissum serva deffende guberna.
\chapterclose


\chapteropen
\chapter[{.I.XIII. Aprés parole de l’ome}]{\textsc{.I.XIII.} Aprés parole de l’ome}\phantomsection
\label{tresor\_1-13}

\chaptercont
\noindent Toutes choses del ciel en aval sont faites por l’omme, mais li hom est fais por lui meismes. Et ke li hom soit en plus haute dignité que toutes autres criatures il apert tout clerement par la reverence de Dieu ke de toutes autres choses commanda : soit ensint et ensint fait ce et ce ; mais de l’home mostre il k’il en pensa diligement en son conseil quant il dit, faisons home a l’ymage et a la similitude de nous.\par
Adan cria il ; mais la feme fu puis formee de la coste son baron ; li hom fu fais a l’ymage de Deu, mais la feme fu faite a l’ymage de l’ome, et por ce sont femes sousmises as homes par loi de nature ; encore fu li hom fais por soi meismes et la feme por lui aidier. Li hom por son pechié fu bailliés au diauble quand il fu dit : tu ies tiere et en tiere iras. Lors fu dit au serpent, c’est au diauble, tu mangeras la tiere, c’est a dire les malvais homes.
\chapterclose


\chapteropen
\chapter[{.I.XIIII. De la nature de l’ame}]{\textsc{.I.XIIII.} De la nature de l’ame}\phantomsection
\label{tresor\_1-14}

\chaptercont
\noindent Âme est vie de l’home, et Dieus est vie de l’ame. Et l’ame de l’home n’est pas home, mais son cors ki fu fais de moiste tiere solement est hom se ame abite dedens. Et par cestui conjungement de la char est apielé hom ; selonc ce que li apostle dist, ke l’ame fu faite en la char a l’ymage de Deu. Et por ce sont en erreur ciaus ki croient ke ame ait cors, car a l’ymage de Deu est ele faite, non pas en tel maniere k’ele soit muable, mais k’ele soit sans cors ausinc comme Dieu est et si angele.\par
Et sachiés que ames ont commencement mais n’aront fin ; car il i a choses en \textsc{.iii.} manieres, unes ki sont temporaus lesqués commencent et fenissent, les autres sont perpetueles ki conmencent mais ne definent, ce sont li angele et les armes, les autres sont sempiterneles ki ne commencent ne ne definent, c’est Dieus en sa divinité.\par
Mais ame n’est pas divine substance ne divine nature, ne n’est mie faite ançois ke son cors, mais a cele eure meismes est ele criee quant ele est mise dedens le cors. Maintes noblece sont en l’ame par nature, mais ele oscurcist pour le mellement du cors ki est decheable.
\chapterclose


\chapteropen
\chapter[{.I.XV. Des nons de l’ame et du cors}]{\textsc{.I.XV.} Des nons de l’ame et du cors}\phantomsection
\label{tresor\_1-15}

\chaptercont
\noindent At nous devancissons les autres animaus non mie par force ne par sens mais par raison. Et raison est en l’ame, mais force et sens est el cors, et au sens des corporaus choses ne soufist pas bien li sens de la char. Et sachiés que raison est en l’ame.\par
Et ame a maint office, et par chascune office est ele apielee par tel non comme a celui office couvient ; car en ce k’ele done vie au cors de l’home est apielee ame, et en ce k’ele a volenté d’aucune chose est apielee corages, et en ce qu’ele juge droitement est apelee raison, et en ce k’ele espire est apielé esperit, et en ce k’ele sent est apielé sens, mais en ce k’ele a sapience est apelee entendemens. Et a la verité dire, l’entendement est la plus haute partie de l’ame, par qui nous vient raison et cognoissance, et par qui li hom est apelés l’ymage de Dieu. Et raisons est un movement de l’ame ki asoutille la veue de l’entendement et trie le voir dou faus.\par
Mais li cors a \textsc{.v.} autres sens, veoir, oïr, odorer, gouster, touchier. Et si come li uns devancist l’autre par honorableté de son estage, tout autresi devancist il l’un de l’autre de vertu. Car flerier sormonte au goster et de leu et de viertu, por ce k’il est plus en haut et oevre sa vertu plus de long. Autresi l’oreille sormonte au flerier, car nous oons plus de long ke nous ne flerons. Et le veoir les sormonte trestous, et de leu et de dignité.\par
Mais toutes ces choses sormonte l’ame, ki est assise en la maistre forterece dou chief et esgarde par son entendement de raison ce ke li cors ne touche et ki ne vient jusques as autres sens dou cors. Por ce dient li sage ke le chief ki est ostel de l’ame a \textsc{.iii.} celles, une devant por aprendre, l’autre est el milieu por conoistre, la tierce est deriere por la memore. Por ce sont maintes choses en l’entention de l’home k’il ne poroit dire en langue ; et c’est la raison por quoi li enfant sont innocent dou faire, non mie dou penser, car il n’ont pas le pooir d’acomplir le movement dou corage ; ensi ont il foiblece por aage, non pas par entention.
\chapterclose


\chapteropen
\chapter[{.I.XVI. De memore et de raison}]{\textsc{.I.XVI.} De memore et de raison}\phantomsection
\label{tresor\_1-16}

\chaptercont
\noindent Memoire est tresoriere de toutes choses et garderesce de tout ce ke l’en trueve novelement par son engien ou k’il aprent des autres ; car tout ce que nous savons est par celes \textsc{.ii.} manieres, ou ke nous le trovons de noviel, ou k’eles nous furent ensegnies.\par
Memore est si tenans ke se aucune chose sera ostee devant le cors, ele laissera en la memore la samblance de soi. Mais de la beatitude ne se sovient ele par ymagination comme des autres, mais par lui meisme, autresi comme de leesce ; car se ce ne fust par lui meisme, ele s’en oublieroit.\par
Et memore est commune as homes et as bestes et as autres animaus, mais entendemens de raison n’est mie en nul autre animal se en home non ; car en tous autres animaus est une pensee par le sens dou cors, non mie par entendement de raison. Por ce fist Deu home en tel maniere que sa veue esgarde tozjors en haut, par senefiance de sa dignité ; mais les autres animaus fist il tous enclins a la tiere, pour demoustrer l’abaissance de lor condition qui ne font se ensivre lor volenté non, sans nul esgardement de raison.
\chapterclose


\chapteropen
\chapter[{.I.XVII. Comment loi fu premiers establie}]{\textsc{.I.XVII.} Comment loi fu premiers establie}\phantomsection
\label{tresor\_1-17}

\chaptercont
\noindent Puis ke li mauvés angeles ot trové le mal et decheu le premier home, son pechié enrachina sor l’umain linage, en tel maniere ke les gens ki aprés nasquirent estoient assés plus courant au mal ke au bien. Et por restraindre le mal k’il faisoient contre la reverence de Dieu en destruction de l’umanité, covint ke loi fust faite en tiere, et ce fu en \textsc{.ii.} manieres, c’est loi divine et loi humaine.\par
Moyses fu li premiers ki bailla la loi as ebreus, et li rois Foroneus fu li premiers ki le bailla as grezois, Mercurius as egyptiens, Salon as athenes, Ligurgus as troiens, Numma Pompalie, ki regna aprés Romolus en Rome, et puis ses fius, bailla et fist lois as romains premierement. Mais \textsc{.x.} sage home translaterent puis dou livre de Salon la loi des \textsc{.xii.} tables. Mais cele loi envielli si k’ele n’estoit pas en cort. Mais li empereres Constentins recommença a fere noviele loi, et autresi firent li autre empereour jusques au tans l’empereour Justiniien, ki toutes les adreça et ordena mieus et plus enterinement, si come il est encore.
\chapterclose


\chapteropen
\chapter[{.I.XVIII. De la divine loi}]{\textsc{.I.XVIII.} De la divine loi}\phantomsection
\label{tresor\_1-18}

\chaptercont
\noindent La divine loi est par nature ; et neporquant ele fu mise en escrit et fu confermee premierement par les prophetes, et c’est le viel testament. Puis fu li noviaus testament, et fu confermé par Jhesukrist et par ses disciples. Mais une maniere de gent blasment le viel testament porce k’il dist autre chose ke le noviel ; mais ne consirent que Dieus por sa grant porveance bailla a l’un tens et a l’autre ce ke covenable fu.\par
Car en la vielle loi comanda les mariages, mais en l’ewangile prise il virginité ; et en la vielle loi comanda il a oster oil pour oil, pié por pié, mais en la novele commanda il a baillier l’autre joe quant l’une estoit ferue. Et a la verité dire, itele fu la vielle loi por la foiblece des gens, et itele la noviele por lor perfection ; car au premier tans estoient li pechié de menour coupe, porce ke lors n’estoit seue verités, mais la samblance de verité ; por ce est la loi plus estroite.\par
Por ce avenoit il au viel tans ke quant aucuns hons saluoit les angeles il ne li rendoient salus, ains le despisoient. Mais el nuef testament lisons nous que Gabriel salua Marie ; et quant Jehans salua l’angele, il li respondi en tel maniere : garde, fist il, ne faire, car je sui tes siers et de tes freres.\par
Or vous ai devisé le conte dou viel testament et dou noviel, et de la divine loi et de l’humaine ; mais por çou ke comander u establir loi poi vaut entre les homes s’il ne fust aucuns ki le peust constraindre a garder la loi, covient ke pour essaucier justice et pour mortefiier le tortfait fuissent establiz rois et signour de maintes manieres ; por çou est il bon a deviser le comencement et la naissance des rois de la tiere, et de lor roiaumes.
\chapterclose


\chapteropen
\chapter[{.I.XVIIII. Coment roi et roiaume furent premierement}]{\textsc{.I.XVIIII.} Coment roi et roiaume furent premierement}\phantomsection
\label{tresor\_1-19}

\chaptercont
\noindent Deus regnes furent en tiere principaument, ki de hautece et de force et de noblece et de signourie orent sormonté tous les autres, en tel maniere ke tout li autre roi et roialme du monde furent ausi come apendans a ces \textsc{.ii.}, c’est le regne des assiriens premierement, et puis celui des romains. Mais il furent devisé en tans et en liu ; car tot avant fu celui des assiriens, et en sa fin comença celui des roumains. Celui des assiriens fu en orient, si com est en Egytpe, car c’est tout \textsc{.i.} regne des assiriens et des egyptiens. Mais li regne as roumains est en occident, ja soit ce que li uns et li autres tenist la monarchie de tot le monde.\par
Mais pour ce ke li mestres ne poroit bien dire la droite naissance des rois s’il ne comence les lignies del premier home, si tornera il cele part son conte, selonc l’ordre des aages dou siecle, por plus apertement moustrer les estas et les contenemens des gens de lors jusc’a nostre tens.\par
Et sachiés que li aage dou siecle sont \textsc{.vi.}, dont li premiers fu d’Adan jusques a Noé, li secons fu de Noé jusques a Abraham, li tiers d’Abraham jusques a David, li quars de David jusques au tans Pharaon, quant il deffist Jherusalem et prist les juis, le quint aage fu de lors jusques a la naissance Jhesucrist, li sisime aage fu ore de la venue Jhesucrist dusc’a la fin dou monde.
\chapterclose


\chapteropen
\chapter[{.I.XX. Del premier aage du siecle}]{\textsc{.I.XX.} Del premier aage du siecle}\phantomsection
\label{tresor\_1-20}

\chaptercont
\noindent El premier aage fist Nostre Soverain Pere le monde et ciel et tiere et toutes les autres choses, selonc ce que li contes devise ça ariere. Et sachiés ke \textsc{.xxx.} ans aprés ce que Dieus cacha Adan hors de paradis terrestre, engendra il en Eve sa feme Caym, et puis une fille ki ot non Calmanam.\par
Et quant Adan fu de l’aage de \textsc{.xxxii.} ans engendra il Abel et puis une fille ki ot non Delcora. Celui Abel fu de bone vie et de gracieuse a Deu et au siecle, tant que Caim son frere l’ocist de male mort, par envie k’il avoit envers lui, et ce fu en l’an de lor pere \textsc{.c.} et \textsc{.xxx.} ans. Lors engendra Adan un autre fiz, ki fu apielés en son non Seth. De celui Seth et de sa lignie nasqui Noé, selonc çou que l’on pora veoir en ce conte meismes.\par
Aprés ce ke Caim ot ocis Abel son frere, il engendra Enoch ; et pour l’onour d’Enoch son fil fist il une cité ki ot non Effraim, mais li plusour l’apieloient Enocham por le non d’Enoch ; et sachiés que cele fu la premiere cité du monde. Celui Enoch le fius Caym, engendra Irad. D’Irad nasqui Mathusael. De Mathusael nasqui Mathusale.\par
De Mathusale nasqui Lamech ; cil Lamech ot \textsc{.ii.} femes, dont la premiere ot a non Adan, en qui il engendra \textsc{.ii.} fius, Jubabel et Anon. Cil Jubabel et cil ki de lui issirent firent premierement tentes et loges pour aus reposer. Anon ses freres fu li premiers hons ki onques trova citoles et orghenes et ces autres estrumens.\par
La seconde feme Lamech ot a non Sellam, en qui il engendra Tubacain ki fu li premiers fevres du monde, et de lui issirent puis maintes malvaises lignies ki deguerpirent Dieu et ses commandemens. Et puis ke Lamech fu de si grant viellece k’il ne veoit ja goute, ocist par aventure Cain d’une saiete de son arc. Mais ki ceste ystore volra savoir plus apertement, si s’en aille au grant conte du viel testament, ou il les trouvera diligemment.\par
Et sachés ke quant Adan fu en aage de \textsc{.iic.} et \textsc{.xxx.} ans, ot il un autre fil de sa femme ki fu apelés Seth. Et quant Adan fu en l’eage de \textsc{.ixc.} et \textsc{.xxx.} ans, il morut, si con plot a celui ki fait l’avoit de vil tiere. De Seth le fil Adan nasqui Enoz. De Enoz nasqui Cainan. De Cainan nasqui Malaleel. De Malaleel nasqui Jareth. De Jareth nasqui Enoch, de qui nul home set la fin, car Dieus l’enmena la u il volt, et il sera ses tiesmons au jor del jugement. Et dient li plusour k’il est au leu meisme dont Adan fu cachiés lors ke li vieus anemis de l’umain linage le dechut. \par
De celui Enoch naquit Mathusala. De Mathusala nasqui Lamech, ki fu peres Noé. Et cil Noé fu preudom et de bone vie et de bone foi, et crut et ama Dieu tant ke Nostre Sires l’eslut quant il manda le deluge sur la tiere por la destruction de gens ki ne faisoient se mal non. Et lors defina li premiers aages dou siecle, ki dura mil et \textsc{.iic.} et \textsc{.lxii.} ans, selonc ke l’escriture le tiesmoigne.
\chapterclose


\chapteropen
\chapter[{.I.XXI. Des choses qi furent ou secont aage}]{\textsc{.I.XXI.} Des choses qi furent ou secont aage}\phantomsection
\label{tresor\_1-21}

\chaptercont
\noindent Noé ki fu li noviaus descendans des Adan le premier home, vesqui \textsc{.viiic.} ans. Et quant il fu de l’aage de \textsc{.vc.} ans, engendra il ses \textsc{.iii.} fiz, Sem, Cam, et Jafeth. Et puis k’il ot vescu \textsc{.vic.} ans fist il la grant arche par le comandement Nostre Signeur.\par
Et dedens cele arche garanti il soi et sa mesnie o toute cele compaignie des gens et des biestes et de tous autres animaus ke Dieus volt, quant li deluges vint sor toutes terrienes choses. Et sachiés ke cele arche ot de lonc \textsc{.iiic.} coutes, et de large ot ele \textsc{.l.}, et si en ot \textsc{.xxx.} de haut. Et plut euue du ciel \textsc{.xl.} jors et \textsc{.xl.} nuis, et dura cent et \textsc{.l.} jours avant k’ele commençast a descroistre.\par
Et quant li deluges fu trespassés et la tiere fu descoverte, si ke cascuns animaus pooit aler la u il voloit, lors commença le secont aage du siecle. Et Noé engendra un autre fil ki ot non Jonitus, ki tint la tiere de Eritaine jouste le fleuve d’Eufrates en orient ; et fu li premiers hom ki trova astronomie et ki ordena la science dou cors des estoiles.\par
Mais de lui se taist ore li contes, ke plus n’en dira en ceste partie ; et dist que quant le deluge fu trespassés, li \textsc{.iii.} premier fil Noé departirent la tiere et le deviserent en \textsc{.iii.} parties, en tel maniere ke Sem li fiz Noé ki estoit li ainsnés tint toute Aise le grant, Cam tint toute Aufrike, et Jafet tint Europe, si con l’en pora veoir ça avant, la u li mestres dira des parties de la tiere.
\chapterclose


\chapteropen
\chapter[{.I.XXII. Des gens qi nasquirent du fil noé le premiers}]{\textsc{.I.XXII.} Des gens qi nasquirent du fil noé le premiers}\phantomsection
\label{tresor\_1-22}

\chaptercont
\noindent Sem engendra \textsc{.v.} fiz, Azur, Ludin, Aram, et Arphaxat, et Elam. Aram li fius Sem ot \textsc{.iiii.} fiz, Us, Ul, Gesar, et Mesa. De Arphaxat, le derrenier fiz Sem, nasqui Salemen . De Salemen nasqui Eber. De Eber nasquirent \textsc{.ii.} fiz, Falech et Jethan. De Jecham nasquirent \textsc{.xiii.} fils, Elmada, Saleph, Samoth, Jareth, Aduram, Izach, Declam, Ebal, Abimelech, Saba, Ophir, et Villa et Jobal. De Phalech son frere, le fiz Heber, nasqui Reus. De Reus nasqui Seruch. De Seruch nasqui Nacor. De Nacor nasqui Thares. De Thares nasquirent Abraham et Aram et Nacor. De Aram nasqui Loth, cil ki eschapa de Sodome et de Gomorre, par la volenté Deu.
\chapterclose


\chapteropen
\chapter[{.I.XXIII. Des gens qi nasquirent dou secont fil noé}]{\textsc{.I.XXIII.} Des gens qi nasquirent dou secont fil noé}\phantomsection
\label{tresor\_1-23}

\chaptercont
\noindent Cam le secont fiz Noé, engendra \textsc{.iiii.} fiz, Cus, Mesaran, Phus, et Canaam. De Cus, le premier fiz Cam, nasquirent \textsc{.vi.} fiz, Saba, Evilach, Sabathat, Reuma, Sabatacha et Nembroth le jaiant, ki fu li premiers rois. De Reuma le fiz Cus nasqui Saba et Dadam. De Meserem le fiz Cam nasquirent \textsc{.vi.} fiz, Ludin, Amazin, Labim, Nefectim, Phetusim, et Celosim. De Canaam le fiz Cam nasquirent \textsc{.xi.} fiz, Sades, Eteus, Gebuseus, Amorreus, Gergeseus, Eveus, Aracus, Sireneus, Aradius, Samarites, et Amatheus.
\chapterclose


\chapteropen
\chapter[{.I.XXIIII. Des generations dou tierç fil noé}]{\textsc{.I.XXIIII.} Des generations dou tierç fil noé}\phantomsection
\label{tresor\_1-24}

\chaptercont
\noindent Japhet, le tierç fiz Noé, ot \textsc{.vii.} fiz, Gomer, Magoz, Mathal, Junam, Tubal, Mosoc, et Tiros. Gomer le fiz Japhet engendra Assenos, Raphan, et Tegorman. Junam fiz Jafeth engendra Elisam, Tharsim, Ceteon, Domamen. Mais ci se taist ore li contes a parler des fiz Noé et de lor generations, car il wet ensivre sa matire por deviser le comencement des rois ki furent ancienement, dont li autre sont estrait, jusq’a nostre tens. Et nos est bien notee ce ke li contes devise ci devant, coment Nembrot nasqui de Cus le fiz Cam, ki fu fius Noé.\par
Et sachiés ke au tens Phalech, ki fu de la lignie Sem, cil Nembrot edefia la tor Babel en Babilone, ou avint la diversités des parleures et de la confusion des langues. Neis Nembrot meismes mua sa langhe de ebreu en caldeu. Lors s’en ala il en Perse, et a la fin s’en repaira il en son païs, c’est en Babilone, et ensigna as gens novele loi, et lor faisoit aourer le fu autresi comme Deu et de lors commencierent les gens aourer les deus. Et sachiés que la cités de Babilone gire environ \textsc{.lxm.} pas, et que la tor Babel avoit en cascune quarrure \textsc{.x.} liues, dont chascune avoir \textsc{.iiiim.} pas, et si avoit li murs de large \textsc{.l.} coutes, et \textsc{.iic.} en avoit de haut, dont cascune coute ert \textsc{.v.} pas, et li pas a \textsc{.iii.} piés.\par
Aprés ce comença li regnes des assiriens et des egyptiens, dont Belus, ki nasqui de la lignie Nembroth, fu premier rois et sires toute sa vie. Mais aprés sa vie en fu rois Ninus son fiz. Et il fu voirs que Assur, fius Sem li fiz Noé, avoit comencie en celui païs une cité ; mais li rois Ninus l’acompli et estora de grant guise, et en fist chief de son regne ; et pour le non de lui est ele apielee Ninive. Et sachiés que Ninus fu li premiers ki onques assambla gens ne ost en fuerre et en guerre, car il assega Babilone et prist la cité et la tour Babel a fine force. Lors fu il navrés d’une saiete dont il morut a la fin ; mais avant k’il fust deviés et k’il avoit tenu son regne \textsc{.xlii.} ans, Thares, li fiz Nacor de la lignie Sem le fiz Noé, engendra \textsc{.iii.} fius, Abraham, Nacor, et Aram, ki cultiverent le vrai Deu. De Aram le frere Abraham nasqui Loth et \textsc{.ii.} filles, Saram la feme Abraham et Melca la feme Nacor. Aprés la naissance Abraham vesqui li rois Ninus \textsc{.xv.} ans en son regne.\par
En celui tens comença li regnes de Sicione, et uns maistres ki avoit non Çoroastres trova l’art magike des encantemens et de teus autres choses. Ces et maintes autres choses furent ou segont aage du siecle, ki fina au tans Abraham, dont aucun dient k’il dura \textsc{.ixc.} et \textsc{.xlii.} ans, li autre dient de mil et lxix. ans, mais cil ki plus touchent de la vérité dient que du deluge jusques a Abraham furent \textsc{.m.} et \textsc{.lxxxii.} ans.
\chapterclose


\chapteropen
\chapter[{.I.XXV. Des choses qi furent ou tierç aage}]{\textsc{.I.XXV.} Des choses qi furent ou tierç aage}\phantomsection
\label{tresor\_1-25}

\chaptercont
\noindent Li tiers aages du siecle comença a la nativité Abraham, selonc l’oppinion de plusors ; mais li autre dient k’il comença a \textsc{.lxxv.} ans de sa vie, quant Dieus parla a lui et k’il fu dignes de sa grace, et ke Nostre Sires li promist a lui et a ses oirs et a sa lignie la tiere de promission. Li autre dient k’il commença a centime an d’Abraham, quant il engendra Ysaac en Sarra sa feme, ki ausi estoit de grant aage, car ele avoit \textsc{.iiiixx.} et \textsc{.x.} ans.\par
Et sachiés que devant ce ke Ysaac fust engendrés, Abraham par la volenté de sa feme, ki ne pooit porter fiz, jut charnelment avec Agar sa camberiere, si en ot un fiz ki fu apielés Ysmael. Et quant Ysaac fu nés, ses peres le fist circonciser \textsc{.viii.} jors aprés sa nativité, et ensi le font encore li juis. Lors fist il ausi circonciser Ysmael, ki ja estoit de l’aage de \textsc{.xiii.} ans, ensi le font encore li sarrasin et ceaus ki abitent en Arrabe, ki sont estrait de la lignie Ysmahel.\par
Puis vesqui Abraham \textsc{.lxxii.} ans ; et sachiés k’il fist premiers autel en l’onour de Dieu. Mais d’Abraham ne de ses fiz ne dira plus ci li contes, ains tornera au roi Ninus et a sa roiauté, car a lui font les istores chief des premiers rois.
\chapterclose


\chapteropen
\chapter[{.I.XXVI. Dou roi ninus et des autres rois aprés sivant}]{\textsc{.I.XXVI.} Dou roi ninus et des autres rois aprés sivant}\phantomsection
\label{tresor\_1-26}

\chaptercont
\noindent Li rois Ninus tint en sa signorie toute la tiere d’Aise le grant fors ke Ynde ; et quant il trespassa de cest siecle, il laissa un jovene fiz ki avoit non Zaraeis, mais il fu apelés Ninus por le non son pere. Semiramis sa mere tint le regne et la signorie tote sa vie ; car ele fu plus caude ke nul home, et plus tiere, et aprés ce ele fu la plus cruele feme dou monde.\par
Et quant ele fu finee et son regne remest sans oir, li païsant eslurent \textsc{.i.} roi ki avoit a non Arrius, mais il fu apielés Diastones, et por lui furent puis li autre roi d’Egypte apelés Diastones. Et cil nons dura jusc’a \textsc{.xvii.} rois ki furent aprés, lors canga li nons, et furent li autre roi apelé Thebei. Encore furent remué lor non, et furent apelé Pastors ; mais a la fin furent apelés Pharaons.\par
De celui non furent puis \textsc{.xvii.} rois, ki durerent jusques au tans Cambises, fius Cirrus le roi de Perse, ki premierement prist Egypte et le sosmist a sa signorie, et encacha hors le roi Natanabon, ki puis fu maistres a Alixandre le grant. De lors remest Egypte sans propre roi, sous la signorie au roi de Perse, jusques au tans Alixandre, ki venki Perse. Et quant Alixandres fu mors et que li \textsc{.xii.} prince de sa cort deviserent entr’aus son regne, Sopter fu rois d’Egypte, et ot a non Tholomeu. Aprés lui regna li secons Tholomeu, ki avoit non Philadelphus. Aprés lui regna li tiers Tholomeu, ki avoit a non Everites. Aprés lui regna li quars Tholomeus, ki avoit non Philopater.\par
Lors estoit Anthiocus premiers rois et empereres d’Anthioce ki par fine force venki toute la tiere d’Egypte et de Perse et de Jude, et ocist Philopater Tholomeu ki lors estoit rois d’Egypte, et regna \textsc{.xxvi.} ans. Aprés la mort le roi Anthiocus, regna Seleucus, ki ot en sornon Epiphanes ; a son tans furent les batailles des Machabeus, dont l’escripture parole en la Bible. Aprés la mort Seleucus regna Eupater ses fiz.\par
Quant Eupater fu mors tint le regne Demetrius li fiz Seleucus ; a son tans fu ocis Judas Machabeus en bataille. Lors vint Alixandres, uns grans sires de haute puissance, encontre le roi Demetrion, et si l’ocist et venki en bataille et ot la signourie de son regne, et la tint quitement, tant ke Demetrius Creticus le fiz celui Demetrius ocist Alixandre et tint la signorie de tous regnes.\par
Puis vint Anthiocus li fiz celui Alixandre meismes, ki par le conseil et par l’ayde Triphon vainki Demetrion Creticon et le cacha hors dou regne, dont il fu puis rois et sires. Mais cil Trifon l’ocist en traison, et il fu rois au tans Symon Machabeu. Et sachiés ke encore vivoit Demetrius Creticus que Anthiocus li fiz Alixandre avoit hors cachié dou regne, si com li contes le devise devant. Trifon ne demora gaires en signorie, ains en fu hors caciés, et cil Demetrius Creticus fu receus en la signourie, et la tint si con rois et enpereours. Lors estoit Jeans Ircanus, li fiz Symon Macabeu, soverains prestres en Jherusalem, et son fil Aristobolus fu esleus rois des juis aprés la transmigration de Babilone \textsc{.iiiic.} et \textsc{.xliiii.} ans.\par
Quant Aristobolus defina sa vie, Alixandres fu rois des juis, et aprés lui en fu rois Aristobolus son fiz. Celui Aristobolus fu ocis par la force Pompei le consillier de Rome, ki establi procureur en Judee Antipater le pere Herode. Et Anthioce estoit ja conquise et sousmise a la signourie des romains. Et quant Antipater fu mors, Erodes ses fius fu esleus par les roumains rois des juis. A cel tans nasqui Jhesucris en Bethleem.
\chapterclose


\chapteropen
\chapter[{.I.XXVII. Du regne de babilone et de egypte}]{\textsc{.I.XXVII.} Du regne de babilone et de egypte}\phantomsection
\label{tresor\_1-27}

\chaptercont
\noindent Le regne de Babilone est conté sor celui des assiriens et des egyptiens. Mais il avint chose ke Nabucodonosor en fu rois, non mie par droit, car il n’estoit pas de roial linage, ançois fu uns estranges mesconeus ki nasqui d’avoutire celeement. Et a son tans comença li empires de Babilone a haucier et a monter en hautece, dont il s’enorghilli viers Deu et vers le siecle, tant k’il destruist Jherusalem et enprisona tous les juis, et maintes autres perversités fist il. Pour coi il avint par divine vengance k’il perdi soudainement sa signorie, et son cors fu remués en buef, et abita \textsc{.vii.} ans en desiers avec les bestes sauvages.\par
Aprés lui regna Nabuchodonosor ses fius, et puis regna Evilmeradap fiz dou premier Nabugodonosor. Aprés lui regna Ragiosar son fiz, et puis Labussar fiz Evilmeradap, et puis Baltazar son frere. Cil Baltazar rois de Babilone fu ocis par Daire roi des mediiens et par Cirrus roi des perse, qui conquisent le regne de Babilone.\par
Aprés la mort le roi Cirrus ot il \textsc{.xiii.} rois en son regne l’un aprés l’autre, jusques au tans ke Daires en fu rois, non mie celui Daire de qui li contes a devisé ça en arieres, ki fu au tans le roi Cirrus, mais ce fu Daire fiz Arçami, ki fu rois et sires de Perse, et avoit grandesime pooir de gens et de terre ; mais Alixandres li grans le venki et ocist et tint son regne.\par
Et sachiés ke Alixandres avoit ja regné \textsc{.vii.} ans, et puis regna il \textsc{.v.} ans, tant k’il defina en Babilone, et lors avoit d’aage entor \textsc{.xxxvi.} ans. Et sachiés que Alixandres fu fiz le roi Phelippe de Macedoine, ja soit ce ke Olimpias sa mere, por essauchier son fil, disoit k’ele l’avoit conceu d’un dieu ki avoit geu a li en samblance de dragon. Et certes il demana si haute vie ke l’en pooit bien croire k’il fust fiz d’un deu : il ala triumphant par le monde, et avoit por sa mestrie Aristotle et Calistene ; il estoit victorieus sor totes gens, mais il estoit vencus par vin et par luxure ; il venki \textsc{.xxii.} nations de barbarie et \textsc{.xxxii.} de Grece, et a la fin morut par venin, ke si privé li donerent desloiaument.\par
Et sachiés ke Alixandres nasqui \textsc{.iiic.} et lxxxv. ans aprés ce ke Rome fu comencie ; et se nous raconte l’istore ke des Adam jusques a la mort Alixandre ot \textsc{.vm.} et c. et lvxii. ans. Et quant il fu mors, si fut Tholomeu Souter li premiers rois d’Alixandre et de toute la terre d’Egypte, si com li contes le devise ça ariere. Et ensi i ot \textsc{.xii.} rois l’un aprés l’autre, dont cascuns avoit en sornon Tholomeu, por le non dou premier Tholomeu, ki en fu roi aprés la mort dou roi Alixandre.\par
De ces \textsc{.xii.} rois fu li derreniers Tholomeu Cleopatra. Quant il avoit ja tenu son regne entor \textsc{.iii.} ans, Julle Cesar fu empereour des roumains, par qui tot li autre empereour de Rome furent apelé Cesar. Mais ci se taist li contes a parler des egyptiiens, porce que ci define lor roiautés, et vint as romains, et ensivra sa matire des autres rois.
\chapterclose


\chapteropen
\chapter[{.I.XXVIII. Des rois de grece}]{\textsc{.I.XXVIII.} Des rois de grece}\phantomsection
\label{tresor\_1-28}

\chaptercont
\noindent Nembrot, cil meismes ki fist la male tour, ot plusors fiz, dont li uns fu apelés Cres, ki fu li premiers rois de Crete. Et son regne comença ou l’ille de Crete, et par le non de lui fu apelés l’ille de Crete, ki siet vers Rommanie. Aprés lui i fu roi Celum son fiz.\par
Aprés i fu rois Saturnus ses fius. Aprés i fu Jupiter ses fiz, ki regna en la cité d’Athenes k’il fist et fonda premiers. De Saturnus et de Jupiter quidoient les gens ki lors estoient k’il fussent deu et par ce non estoient il nomé, dont il ont encore issi a non deus planetes. Aprés fu rois Cicros.\par
Et sachiés ke Jupiter ot \textsc{.ii.} fius, Danaum et Dardanum. Cil Danaus fu rois en l’ille de Crete et de Micene et de Grece la environ, et ot guerre contre Trous le roi de Troie et contre Ilum et Ganimedem son fiz, et ocist celui Ganimedem. Ce fu la premiere haine des troiens et des grezois.\par
Et aprés la mort Danaum regna en Grece Pelops son fiz. Aprés lui fu rois Atrius son fiz, et puis li rois Menelaus ses fiz, li maris Elaine ki fu ravie par Paris le fiz au roi Priamus de Troie. Aprés le roi Menelaus regna Agamenon son frere. Et tant ala de rois en rois, ke Phelippes de Macidoine en fu rois, et puis Alixandres ses fiz, ki fu rois et empereour de tote Grece. Et des lors en avant furent apelé empereour de Grece, non mie roi.
\chapterclose


\chapteropen
\chapter[{.I.XXVIIII. Du regne de sicione}]{\textsc{.I.XXVIIII.} Du regne de sicione}\phantomsection
\label{tresor\_1-29}

\chaptercont
\noindent Li regne de Sicione comença au tans Nacor, ki fu aious d’Abraham ; dont Agrileons fu li premiers rois ; et dura celi regnes \textsc{.ixc.} et lxxi. ans, jusques au tans Ely le prestre, de qui li mestres dira la vie ça avant entre les prophetes ; et furent en some \textsc{.xxxi.} roi en Sicione.
\chapterclose


\chapteropen
\chapter[{.I.XXX. De femenie}]{\textsc{.I.XXX.} De femenie}\phantomsection
\label{tresor\_1-30}

\chaptercont
\noindent Li regne des femes comença lors ke li rois de Sciete o tout les homes de sa tiere ala sor les egyptiens u il furent ocis trestuit ; et quant lor femes sorent çou, elles fisent une dame de lor gent roine dou païs, et establirent que jamés nus hom peust habiter en lor tiere, et ke les filles fussent norries et li malle non, et ke cascune copast la senestre mamelle por mieus porter escu et armes ; et por ce sont eles apielees Amaçoines, c’est a dire sans l’une mamelle. Et cestes vinrent secorre Troie.\par
Et ce fist Pantasilee lor roine, ke l’en dit k’ele ama Hector par amors. Mais de ce ne sot onques hom la certaineté, fors k’ele i morut avec grant partie de ses damoiseles.
\chapterclose


\chapteropen
\chapter[{.I.XXXI. Du regne des arginois}]{\textsc{.I.XXXI.} Du regne des arginois}\phantomsection
\label{tresor\_1-31}

\chaptercont
\noindent Li regne des Arginois comença en celui tens meismes que Jacob et Esau li fiz Ysaac furent nés ; dont Ynacus fu premiers rois. Aprés lui fu rois Foroneus son fiz, ki premierement dona la loi as grezois en la cité d’Athenes, et ki establi ke les causes et li jugement fussent devant lui jugiés. Et li leus u l’en fet les jugemens est apelés foron, por le non de lui. Et sachiés que cis regnes des Arginois dura \textsc{.iic.} et lxiiii. ans, et fu destruis au tans Danai le roi de Grece, de qui li contes parole ci devant.
\chapterclose


\chapteropen
\chapter[{.I.XXXII. Des rois de troie}]{\textsc{.I.XXXII.} Des rois de troie}\phantomsection
\label{tresor\_1-32}

\chaptercont
\noindent Li contes a dit ça ariere ke li rois Jupiter ot \textsc{.ii.} fiz, Danaum et Dardanum. Et de celui Danaum vous a dit li contes totes la generation. Or dist li contes ke li autres fius, c’est Dardanus, fist en Grece une cité k’il apiela Dardaine por son nom \textsc{.iiim.} et \textsc{.iic.} et \textsc{.xlii.} ans dou comencement dou siecle. De Dardanus nasqui Eritanius ki aprés lui en fu rois. D’Eritanius nasqui Trous li rois ki estora la cité de Troie, et par son non fu ele apelee Troie.\par
Dou roi Trous nasqui Ilus ki fist la maistre forterece de Troie, ki par lui fut apelee Ylion. Et son frere Ganimedes fu ocis par les grezois, selonc ce ke li contes devise ci devant. Dou roi Ylus nasqui Lamedon, ki vea les pors a Jason et a ses autres compaignons ki aloient por la toison d’or, por vengance de la mort Ganimeden son oncle. Dont il avint puis ke Jason et Ercules o toutes les os des grezois vinrent a Troie et destruisent la terre et ocistrent le roi Lamedon, et si en amenerent Exionam sa fille.\par
Du roi Lamedon nasqui li rois Priant, et Ancises li peres Eneas. Cil Prians rois de Troie fu peres au bon Ector et a Paris ki ravi Elaine le feme Menelaus le roi de Grece, por vengance de çou ke jou ai devisé ; por quoi Troie fu destruite finalment, et li rois et ses fils tot en furent ocis, selonc ce ke vous porés trover el grant livre des troiens. Et ce fu fait \textsc{.ixc.} et lxii. ans aprés le comencement de Troie.
\chapterclose


\chapteropen
\chapter[{.I.XXXIII. Coment eneas ariva en ytaile}]{\textsc{.I.XXXIII.} Coment eneas ariva en ytaile}\phantomsection
\label{tresor\_1-33}

\chaptercont
\noindent Quant Troie fu prise et mise a feu et a ruines, et ke l’en ocioit les uns et les autres, Eneas et Anchises son pere et Ascanius son fils s’en issi hors et enporta grandesime trezor et avec lui tout plain de gens et s’en alerent a sauveté. Et por ce recontent li auctour k’il seut la traison et k’il en fu compains. Mais li plusor dient k’il n’en sot rien se a la fin non quant la chose ne pooit estre destornee. Mais comment ke la chose fust, il et sa gent s’en alerent par mer et par terre, une eure ça et autre la, tant qu’il ariva en Ytaile.
\chapterclose


\chapteropen
\chapter[{.I.XXXIIII. Coment eneas fu rois en ytaile et ses fil apriés}]{\textsc{.I.XXXIIII.} Coment eneas fu rois en ytaile et ses fil apriés}\phantomsection
\label{tresor\_1-34}

\chaptercont
\noindent Et il fu voirs ke Icarus ki fu fiz Nembrot, ki fist la tor Babel, vint en Ytaile, et si en fu sires toute sa vie. Aprés le tint Ytalus ses fiz, et por le non de lui fu apelés li païs Ytaile. Après le tint Janus ses fiz ; lors avint selonc ce ke les istores recontent que Saturnus roi de Grece fu essilliés de son regne, et s’enfui en Ytaile, et la fu il rois et sires de la tiere. Aprés le tint li rois Picus son fiz, et puis li roi Famus fius au roi Picus.\par
Dou roi Famus nasqui li rois Latins, ki lors estoit rois en Ytaile quand Eneas o ses gens i arriverent. Et ja soit ce ke au comencement li rois fu dous et debonaires, et li volsist doner a feme Laviniam sa fille, dont il n’avoit plus d’enfans, la roine ne consenti pas au mariage, ains la volt doner a un autre grant riche home dou païs. Por ce fu entr’eus haine grant si come mortel guerre.\par
Mais a la fin le venki Eneas a force d’armes et prist Laviniam a feme, et ensi fu li rois d’Ytaile, et regna \textsc{.iii.} ans et \textsc{.vi.} mois ; et quant il morut il laissa \textsc{.i.} petit enfant de sa feme, ki ot a non Julius Silvius, porce ke sa mere le faisoit priveement norrir en silves, c’est en bois, por poour d’Ascanius son frere ; mais n’avoit garde, k’il l’ama tenrement. Et ce fu au tans dou roi David, au commencement du quart aage du siecle.
\chapterclose


\chapteropen
\chapter[{.I.XXXV. De la lignie des rois de rome et d’engletere}]{\textsc{.I.XXXV.} De la lignie des rois de rome et d’engletere}\phantomsection
\label{tresor\_1-35}

\chaptercont
\noindent Quant Aschanius trespassa de cest siecle, Silvius ses freres fu rois aprés lui, et ot \textsc{.ii.} fius, Eneas et Bruthum. Et quant Silvius li rois morut, Eneas son ainsné fiz tint la tiere. Aprés sa mort Bruthus son frere passa en une tiere ki par le non de lui fu apelee Bretaigne, ki ore est Engletere clamee ; et il fu li commencemens des rois de la Grant Bretaigne. Et de ses generations nasqui puis li rois Artus, de qui li romant parolent, k’il fu rois coronés a \textsc{.iiiic.} et \textsc{.iiiixx.} et \textsc{.iii.} ans de l’incarnation Jhesucrist, au tans que Zeno fu empereres de Rome, et regna entor \textsc{.l.} ans.\par
Dou roi Eneas ki fu fius au roi Silvius nasqui Latin. Du roi Latin nasqui Alban, ki fist la cité d’Albani. Du roi Alba nasqui Egypte. Du roi Egypte nasqui Carpenaces. Dou roi Carpenaces nasqui Tyberius. Dou roi Tyberius nasqui Agripa. Dou roi Agripa nasqui Aventinus. Dou roi Aventinus nasqui Procas. Dou roi Procas nasqui Numentor et Amilio. Cil Numentor en fu rois aprés la mort son pere. Et avoit une fille ki avoit non Emilia. Mais Amilio li toli son regne, et cacha Numentor et sa fille en essil, et il se fist faire rois endementiers.\par
Et Emilia conçut \textsc{.ii.} fiz, Remum et Romolon, en tel maniere ke nus ne sot ki fu lor pere. Mais li plusor disoient ke Mars li dieus de batailles les engendra. Et de lors en avant fu cele feme apelee Reate ; et puis fist ele une cité el milieu d’Ytaile, ki por le non de li fu apelee Reate.\par
Et por ce ke maintes ystores devisent ke Romolus et Remus fu norris par une lue, il est drois ke j’en die la verité : il fut voirs ke quant il furent nés on les gieta sus une riviere, por ce ke les gens ne s’aperchussent pas ke la mere eust conceu. Entor cele riviere manoit une femme ki servoit a tous communalment. Et teles femes sont apelees lues en latin. Cele feme prist les enfans et les norri mout doucement. Et por ce fu dit k’il estoient fius d’une lue. Més ce ne pot estre.
\chapterclose


\chapteropen
\chapter[{.I.XXXVI. De romulus et des roumains}]{\textsc{.I.XXXVI.} De romulus et des roumains}\phantomsection
\label{tresor\_1-36}

\chaptercont
\noindent Romulus fu mout fiers et de grant corage ; et quant il fu en son aage il conversoit avec les jovenes bachelers et les legiers homes maufetours dont il estoit mestres et chevetains. Et quant on li descovri sa naissance, il ne fina onques de quellir gens de diverses manieres et de guerrier contre Amilion, ki le regne avoit tolu a son aieul. Et tant fist par sa proece k’il le venki et li toli le regne et le rendi a Numentor.\par
Aprés ce ne demora gramment que il le fist morir, et il fu rois en son leu. Et puis fist il Rome ki ensi est apelee por le non de lui. Puis fist il morir Remum son frere, et puis le pere sa feme, ki estoit sires dou temple des sacrefisses dou païs ; et il fu oirs de trestot et ot il soul tote la signorie de Rome. Ensi fu Rome comencie \textsc{.iiiim.} et \textsc{.iiiic.} et \textsc{.iiiixx.} et \textsc{.iiii.} ans aprés le comencement dou monde, ce fu \textsc{.iiic.} et \textsc{.xiiii.} ans aprés la destruction de Troie.\par
Et quant Romolus passa de ceste vie, le regne tint Noma Pompilius son fiz, et puis Tulius Ostilius, et puis en fu rois Ancus Marcus, et puis Tarquinius premiers, et puis li rois Servius, et puis regna Tarquins li orgilleus, le cui fil par son orguel fist honte et outrage a un noble dame de Rome et de haute lignie, por gesir a li charnelment. Cele dame avoit non Lucrece, une des millours dame del monde et la plus chaste.\par
Pour ceste aquoison fu cil Tarquinus chacié de son regne, et fu establis par les romains ke jamés n’i eust roi, mais fust la cité governee et tot son regne par les sinatours, par consoles, et patrices et tribuns et dicteours, et par autres officiaus, selonc ce ke les choses sont grans et dedens la vile et dehors.\par
Et cele signorie dura \textsc{.iiiic.} et lxv. ans, jusk’a tant ke Catelline fist a Rome la conjurison encontre ciaus ki governoient Rome, por l’envie des signatours. Mais cele conjurison fu descoverte au tans ke li tres sages Marcus Tullius Cicero, li mieus parlans hom del monde et maistres de rectorike, fu consoles de Rome, ki par son grant sens venqui les conjurés, et en prist et fist destruire une grant partie par le conseil dou bon Caton ki les juga a mort, ja soit ce ke Julius Cesar ne consilla pas k’il en fussent jugié a mort, mais fussent mis en diverses prisons.\par
Et pour ce disent li plusour k’il fu compains de cele conjuroison. Et a la verité dire il n’ama onques les signatours ne les autres officieus de Rome, ne il lui, car il estoit estrais de la lignie as fius Eneas. Aprés ce il estoit de si haut corage k’il ne baioit fors que a la signorie avoir dou tout, selonc ce ke ses anciestres avoient eu.
\chapterclose


\chapteropen
\chapter[{.I.XXXVII. De la conjurison catelline}]{\textsc{.I.XXXVII.} De la conjurison catelline}\phantomsection
\label{tresor\_1-37}

\chaptercont
\noindent Quant la conjuroison fu descoverte et le pooir Catelline fu affoibloié, il s’enfui en Toschane en une cité ki avoit non Fiesle et la fist reveler contre Rome. Mais li romain i envoierent grandesime ost, et troverent Catelline au pié des montaignes o toute son ost et sa gent cele part ou est ore la cité de Pistoire. La fu Catelline vencus en bataille, et mort lui et li sien ; neis une grant partie des romains i fu ocise. Et par le pestrine de cele grant occision fu la cités apelee Pistoire.\par
Aprés ce assegerent li romain la cité de Fiesle, tant k’il le venkirent et misent en sa subjection ; et lors firent il enmi les plains ki est au pié des hautes montaignes u cele cités seoit une autre cité, ki ore est apelee Florence. Et sachés que la place de tiere ou Florence est fu jadis apelee chiés Mars, c’est a dire maisons de batailles ; car Mars, ki est une des \textsc{.vii.} planetes, est apelés deus de batailles, ensi fu il aourés ancienement.\par
Por ce n’est il mie merveille se li florentin sont tozjors en guerre et en descort, car celui planete regne sor aus. De ce doit maistre Brunet Latin savoir la verité, car il en est nés, et si estoit en exil lors k’il compli cest livre por achoison de la guerre as florentins.
\chapterclose


\chapteropen
\chapter[{.I.XXXVIII. Comment julles cesar fu premiers empereor de rome}]{\textsc{.I.XXXVIII.} Comment julles cesar fu premiers empereor de rome}\phantomsection
\label{tresor\_1-38}

\chaptercont
\noindent Endementiers Julle Cesar porchaça tant amont et aval aprés ce k’il ot euues maintes victoires et mains païs sosmis au commun de Rome, k’il se combati contre Pompee et contre les autres ki lors governoient la cité, tant k’il les venki et chaça tos ses enemis, et il sol ot la signorie de Rome. Et por ce ke li romain ne pooient avoir roi, selonc l’establissement ki fu fais au tans Tarqinius, de qui li contes fist memore ça en ariere, se fist il apeler empereour. Et ensi Julle Cesar fu li premier empereor des romains, et tint son empire \textsc{.iii.} ans et \textsc{.vi.} mois, tant k’il fu ocis par traison sus la capitaile par les mains de ses ennemis.\par
Aprés la mort Julle Cesar fu empereres Octeviens son nevou, ki regna \textsc{.xlii.} ans et \textsc{.vi.} mois avant la naissance Jhesucrist et \textsc{.xiiii.} ans aprés, et tint la monarchie de trestot le monde ; et il fu mout sages et preus, mes il estoit fort luxurieus. A la fin destruist il tous ciaus ki ocisent Julle Cesar. Mes ci se taist li mestres a parler de lui et des empereours de Rome, et retorne a sa matire.
\chapterclose


\chapteropen
\chapter[{.I.XXXVIIII. Des rois de france et de lor vies}]{\textsc{.I.XXXVIIII.} Des rois de france et de lor vies}\phantomsection
\label{tresor\_1-39}

\chaptercont
\noindent Quant la cités de Troie fu destruite et ke li un s’enfuirent cha et li autre la, selonc ce ke fortune les conduisoit, il avint chose ke Priant li jovenes, ki fu fius de la seror Priant de Troie, et avec lui Antenor, s’en alerent par mer a tot \textsc{.xiiim.} homes armés, tant k’il arriverent la u est ore la cités de Vinese, k’il comencierent premierement et fonderent dedens la mer, por ce k’il ne voloient abiter en tiere ki fust a signour.\par
Puis s’en parti Antenor et Priant a grant compaignie de gent, et s’en alerent en la marche de Trevise et non mie loins de Vinese, et la firent il une autre cité ki est apelee Padue, ou gist le cors Antenor, et encor i est son sepulture. De la s’en partirent puis une gent et s’en alerent en Sicambre, une cité k’il firent a la fin, dont il furent appelez cycambriens. En trespassement de tens s’en alerent il en Germanie, et por ce furent il apelés germains. Et quant il furent en Germanie, il firent roi et signour d’aus Priant ki fu de la lignie Priant le joene, ki fu puis ocis en la bataille k’il ot contre les romains, et laissa un fil ki ot a non Arcomedes.\par
De Arcomedes nasqui Faramont, ki puis fu rois des germaniens. Aprés lui regna li rois Crinitus ses fius. Lors commença Rome a abeissier et a descroistre, et France commença a croistre et a enhaucier, tant k’il enchacierent les romains, ki lors abitoient jouste le fleuve dou Rin. Et quant li rois Crinitus fu mors, si en fu rois Gildebours, et engendra en la roine Bisine Glodeveum, ki fu rois de France. Aprés lui regna Miroveus son fiz. Aprés lui regna l’autre Mirovem son fiz. Aprés lui regna Ildris son fiz.\par
Aprés lui regna Glodeveum son fiz, ki fu li premiers rois de France ki onques fust crestiiens, car Sains Remis le baptiza. Il meismes sosmist les alemans a sa signourie, et venki les Gascons. En l’an de l’incarnation Jhesucrist \textsc{.viic.} et li. commencierent les ainsné a avoir la signorie dou regne de France, dont Arnous fu li premiers, ki puis fu evesques de Mes. Aprés regna Angigious son ainsné fiz. Aprés lui regna Pepins son fiz ki ot en sornon Croisus. Aprés regna Karlemartiaus son fiz. Aprés lui regna li rois Pepins ki fu peres Carlemaine, ki fu rois de France et empereres de Rome, selonc ce ke li contes devisera ça avant.\par
Mais or se taist li contes a parler des rois et de la terre et de lor regnes ; et por ce k’il a devisé assés clerement comment furent li premier, et ki il furent, et en quel terre. Des romains meismes a il devisé la droite istore et jusques au commencement de lor empire ; por ce n’en dira il ore mie avant, ains retornera a sa matire, c’est a dire dou tierç aage dou siecle, dont il s’est longhement teus.
\chapterclose


\chapteropen
\chapter[{.I.XXXX. Encore dou tierç aage dou siecle}]{\textsc{.I.XXXX.} Encore dou tierç aage dou siecle}\phantomsection
\label{tresor\_1-40}

\chaptercont
\noindent Or dist li contes, que quant li tiers aages fu comenciés au tans Abraham ki nasqui au tans le roi Ninus, ke Abraham engendra Ysaac. Et Ysaac engendra Esau et Jacob, et encore vivoit Abraham, mais il avoit bien \textsc{.c.} et lv. ans d’aage. Jacob engendra Joseph, et lui et ses autres freres, de qui l’escripture dist, et de qui furent estraites les \textsc{.xii.} lignies li sont apelees li fiz Israel. Car il fu voirs que Jacob se combati et luta de nuit contre l’angele, tant que a la fin le venki Jacob ; lors fu il beneois et li fu changiés son non, et fu apelés Israel, c’est a dire prince de Dieu.\par
Joseph fu vendus par ses freres, et a la fin fu il grans mestres en la cort Pharaon le roi d’Egypte au tans ke la grant famine fu en tiere. Lors i fist il son pere venir o tot ses freres, ki puis demorerent en Egypte juskes au tans Moysen, selonc ce que li contes dira ci aprés.\par
Li tiers frere Josep le fil Jacob, qui ot nom Levi, engendra Cahat. De Caat nasqui Aran. De Aran nasqui Moyses. Et quant Moyses fu nés, sa mere l’enclost diligemment en un petit escring, et le geta el fleuve, por ce ke uns autres Pharaons ki lors estoit rois d’Egypte avoit commandé ke tot li fil malle as ebreus fuissent gietés el fleuves, et les femes fuissent gardees. A la rive de celui fleuve le trova la fille le roi Pharaon, ki l’osta de l’euue et le fist norrir ensi come son fil, et por ce ot il celui non ; car Moys vaut autant a dire comme euue. Et quant Moyses fu en aage de \textsc{.lxxx.} ans, il amena tout le peule Israel hors d’Egypte en la terre que Dieus avoit promise a Abraham.\par
Et sachiés ke de lors ke Dieus avoit promise a Abraham la terre de promission jusc’a l’issue d’Egypte, ot \textsc{.iiiic.} et \textsc{.xxx.} ans. Et ensi fu Moyses maistres et sires dou peule Israel par la volenté de Deu, ki li dona la loi, et par lui comanda il k’ele fust gardee. Aprés sa mort furent puis maint autre governeour du peuple, juskes au tans David, ki en fu rois et sires, et ce fu \textsc{.vi.} cens et . ans aprés l’issue d’Egypte, quant Moyses en amena le peuple.\par
Lors defina li tiers aages ; et ja estoit Troie conquise et destruite, et Eneas et ses fiz avoient ja conquis le regne dou roi Latin. Et sachiés que li tiers aages, ki fu des Abraham jusques a David le roi, dura \textsc{.ixc.} et lxxiii. ans.
\chapterclose


\chapteropen
\chapter[{.I.XXXXI. Dou quart aage dou siegle}]{\textsc{.I.XXXXI.} Dou quart aage dou siegle}\phantomsection
\label{tresor\_1-41}

\chaptercont
\noindent Le quart aage commença lors ke Saul rois de Jherusalem fu ocis, et David en fu rois et sires. Aprés sa mort en fu rois Salemons ses fiz, ki fu plains de sens et de sapience, et ki fonda et fist le temple de Jerusalem. Puis i furent maint autre roi li uns aprés l’autre jusques a tant ke Sedeychias en fu rois. Et quant il ot regné entor \textsc{.xii.} ans, Nabugodonosor li rois de Babilone, de qui li contes parla ça ariere, le prist et li osta les ieux de la teste, et le mena en prison en Babilone, lui et tous les juis, c’est les gens ki estoient de la lignie Israhel.\par
Et le temple Salemon fu ars en feu et en flambe, ki ne dura ke \textsc{.iiiic.} et \textsc{.ii.} ans, lors defina le quart aage. Dedens le quart aage furent li prophete desquex l’escripture parole, et Romolus fonda Rome. Et sachiés que Tarquinius Priscus estoit rois des romains quant li juis furent en prison en Babilone, et cest aage dura \textsc{.vc.} et \textsc{.xii.} ans.
\chapterclose


\chapteropen
\chapter[{.I.XXXXII. Dou quint aage du siecle}]{\textsc{.I.XXXXII.} Dou quint aage du siecle}\phantomsection
\label{tresor\_1-42}

\chaptercont
\noindent Le quint aage commença a la transmigration de Babilone, c’est a dire quant li juis en furent mené en chaitivison. Et quant il estoient en la prison, Cirrus li premiers rois de Perse ocist Baltasar le roi de Babilone, et prist sa terre et son regne, selonc ce ke li contes devise ça arieres. Cil rois Cirrus delivra de la prison des juis bien \textsc{.lm.} por estorer le temple. Mais puis vint li rois Daires, ki tint la terre aprés lui, et tous les delivra ; et ce fu \textsc{.lxxii.} ans aprés ce k’il furent pris. Lors meismes fu cil Tarquinus Superbus rois des romains chaciés de sa signorie, si con nous trovons enconté ça arieres.\par
Cestui aage dura dusk’a la naissance Jhesucrist de la Glorieuse Virgene Marie, ce furent \textsc{.vc.} et \textsc{.xlviii.} ans dedens lequel furent Platons, Aristotles, et Demostenes, ki furent li soverain philosophe. Et regna Alixandres li grans, et li romain conquisent Grece, Espaigne, et Aufrike, Trace, Sirie, et maintes autres terres. Et cestui meismes aage bailla Marcus Tullius as romains la rectorique, et Pompee conquista la terre de Judee, et Catelline fist la conjuroison en Rome et Julles Cesar devint premiers empereour de Rome, et apré lui Octeviens. Et Nostre Sires prist char en la Virgene Marie \textsc{.vm.} et \textsc{.vc.} ans dou commencement du monde, mais li plusor dient k’il n’i ot ke \textsc{.vm.} et iic. et lviii. ans.
\chapterclose


\chapteropen
\chapter[{.I.XXXXIII. Dou sisime aage dou siecle}]{\textsc{.I.XXXXIII.} Dou sisime aage dou siecle}\phantomsection
\label{tresor\_1-43}

\chaptercont
\noindent Li \textsc{.vi.} aages comença a la naissance Jhesucrist, et durra tresk’a la fin du monde. Et sachiés que Nostre Sires fu en tiere avec ses apostles, et commença le novel testament et defina le viel, car a \textsc{.xxx.} ans de son aage se fist il baptizier por les mains Jehan Baptiste, pour moustrer que li crestiien celebrassent le baptesme la u la vielle loi faisoit la circoncision. Et por ce ke nous gardons la vieille loi, la u ele ne fu pas remuee, est il bien drois ke li contes devise les maistres de cele loi, et la vie de chascun, en ceste maniere.
\chapterclose


\chapteropen
\chapter[{.I.XXXXIIII. De david roi et phophete}]{\textsc{.I.XXXXIIII.} De david roi et phophete}\phantomsection
\label{tresor\_1-44}

\chaptercont
\noindent David fix Jessé, ki fu estrais de la lignie Juda, nasqui en Bethleem, ocist Goliam le grant, ki enemis estoit au roi Saul, ki sires fu de Jerusalem et de tous les juis. Il venki sans coutel le lien et le ourse, et venki le jaiant, et maintes autres grans choses fist il ; por quoi Saul le haoit et le chaçoit por lui tolir la vie, car il doutoit k’il ne li tolsist son regne. Mais si come a Dieu plot, Saul morut et David fu rois aprés lui, et fu mout victorieus. Et Deus volt k’il fust rois et prophetes. Et ja fust il perchieres, il revenoit tost et volentiers en penitance. Il ama mout Bersabee la feme Urie son connestable, et a la fin fist il aler Urie en une bataille u il morut, et puis tint la feme, et en li engendra il Salemon le sage ki fu rois aprés lui.\par
Et sachiés ke David fu li soverains prophetes de tous, car il ne prophetiza pas a la maniere des autres. Car prophetie est en \textsc{.iiii.} manieres, ou en fait u en dit u en vision ou en songe. En fait fu l’arche Noé, ki fu senefiance de sainte eglise. En dit fu ce ke li angeles dit a Abraham, en sa semence seroient toutes gens beneoites. En vision fu le rouge ke Moyses vit ardoir, ki ne definoit. En songe furent les \textsc{.vii.} vaches et li \textsc{.vii.} espi ke Pharaon songa, sor quoi Joseph prophetiza.\par
Mais ore de ces \textsc{.iiii.} prophetiza David par seule interpretation de Dieu et del Saint Esperit, ki li ensigna a dire tote la naissance Jhesucrist et sa mort et sa resurrection. Il descovri ce ke li autre avoient dit covertement, selonc ce ke l’on puet veoir en son livre, ki est apelé sautier, en samblance d’un estrument ki a autresi non, ki a \textsc{.x.} cordes : autresi parole li livres des \textsc{.x.} commandemens en cent et \textsc{.l.} psalmes ki sont el sautier. Et sachés ke David regna \textsc{.xl.} ans et passa de cest siecle en l’aage de \textsc{.lx.} ans.
\chapterclose


\chapteropen
\chapter[{.I.XXXXV. De salemon roi}]{\textsc{.I.XXXXV.} De salemon roi}\phantomsection
\label{tresor\_1-45}

\chaptercont
\noindent Salemons rois, fu fiz au roi David, home tres glorieus, plain de totes sapiences, riches de trezor et de tres haute chevalerie. Deus l’ama au commencement, mais puis le haï por ce k’il aoura les ydoles, et ce fist il por les amors d’une dame. Il fu rois en Jherusalem sor les \textsc{.xii.} lignies de Ysrahel\textsc{.xl.} ans, et fu ensevelis avec ses ancestres en Bethleem.
\chapterclose


\chapteropen
\chapter[{.I.XXXXVI. De helyas prophete}]{\textsc{.I.XXXXVI.} De helyas prophete}\phantomsection
\label{tresor\_1-46}

\chaptercont
\noindent Elyas Tesbitem fu grans prestres et prophetes, que tousjors habita seul es desiers, replains de foi et de sainte pensee. Il ocist le tyrant. Il resplendi des grans signes de vertu, car il clost \textsc{.iii.} ans le ciel de pluie, et puis par ses orisons retorna la pluie. Il resusita un home mort. Par sa vertu ne defali la farine ki en l’ydres estoit. Et d’un vaissel d’oile fist il une fontaine, de quoi tousjors sourt oiles. Par sa parole descendi li fus du ciel sor les sacrefices, et par ses paroles ardirent \textsc{.ii.} princes o tout lor chevaliers. Il ovri les fleuves Jordan et le passa a sés piés. Il monta el chiel en \textsc{.i.} char de fu,\par
Malachias prophetes dit ke Elias doit encore retorner a la fin du monde devant Antecrist, o grant signe de merveilles, ensi vendront Elyes et Enoch son compaignon. Mais Antecrist les fera ocire et gieter lor charoignes parmi les places sans sepulture. Mais Nostre Sires les resuscitera, et destruira Antecrist et son regne et tous ceus ki l’aourront.\par
Cist Elyas fu de la lignie Aaron. Et quant il vint a sa naissance, Sobi son pere songa ke hons vestus de blanche robe prenoient Helyas et l’envolepoient en blans dras, et puis li donoient fu a mangier. Et quant il s’esvilla, il enquist des prophetes ke ce pooit estre ; et il disent, ne doutes pas, car tes fius sera lumiere et parliers de sciences, et il jugera Israel a fu et a coutel.
\chapterclose


\chapteropen
\chapter[{.I.XXXXVII. De eliseus}]{\textsc{.I.XXXXVII.} De eliseus}\phantomsection
\label{tresor\_1-47}

\chaptercont
\noindent Elyseus vaut autant a dire come fil de Mon Deu. Il fu prophetes et disciples Elye, et fu d’un chastel ki avoit a non Amelmoat, de la lignie Ruben. Et lors k’il nasqui une petite vache d’or s’esmueta si fort, ki estoit en Galga, ke sa vois resona en Jherusalem. Et lors dist uns prophetes, hui est nés uns prophetes, ki destruira les ydoles.\par
Et certes il fist hautes merveilles, car il devisa les fleuves de Jordan et les fist retorner contremont, et il passa parmi la turchie dou fons. Il estora les euues de Jhericho ki seches estoient. Il fist corre euue de sanc, por destruire les enemis des juis. Une feme ki onques n’avoit porté fiz fist il par sa parole enchaindre et concevoir \textsc{.i.} fiz. Et celui meismes resuscita il de mort. Il aatempra l’amertume des viandes. Il saoula cent homes de \textsc{.x.} pains d’orge. Il gari Naaman de la lepre. Il fist noer la coignie de fier, ki estoit el fons de Jordan. Le enemi de Surie fist il avugler. Au signor de Samarie dist il sa mort devant. Il chaça l’ost des enemis, ki estoient sans nombre. Il cacha en \textsc{.i.} jor les grans famines. Il resuscita la charoigne d’un home. Elyseus morut en la cité dou Sebaste, ou son sepulchre est encore en grandesime reverence.\par
Eliseus ot \textsc{.ii.} esperis, le sien et le celui Helias ; por ce fist il plus hautes merveilles ; car Helias quant il vivoit resuscita l’omme mort, mais Elyseus quant il estoit ja mors en resuscita \textsc{.i.} autre. Helyas enmena famine et secheresse, mais Elyseus delivra en \textsc{.i.} jor tot le peuple de la grant famine.
\chapterclose


\chapteropen
\chapter[{.I.XXXXVIII. De ysaias}]{\textsc{.I.XXXXVIII.} De ysaias}\phantomsection
\label{tresor\_1-48}

\chaptercont
\noindent Isaias vaut autant a dire comme salus dou signeur, et fu fiz Amoz, non mie de Amoz prophete, ki fu né de pastors, mais Amoz le pere Ysaie fu nobles hom de Jerusalem. Ysaies fu hom de grant sainteté, ke par le commandement Nostre Signor conversoit entre le peuple tozjors nu cors et nus piés. Deus par sa proiere alonga la vie \textsc{.xv.} ans au roi Ezechie, ki ja devoit morir. Manasses fist partir parmi le cors Ysaie une sie de fust.\par
Et dient li juis k’il fu livrés a mort pour \textsc{.ii.} raisons, l’une por ce k’il les apiela pueples de Sodome et princes de Gomorre, l’autre ke quant Deus ot dit a Moysen, tu ne poras veoir ma face, cil Ysaias osa dire k’il avoit veu Dameldeu. Et sa sepulture est desous le chesne de Joel.
\chapterclose


\chapteropen
\chapter[{.I.XXXXVIIII. De jeremie}]{\textsc{.I.XXXXVIIII.} De jeremie}\phantomsection
\label{tresor\_1-49}

\chaptercont
\noindent Jeremies fu del linage des prestres, et fu nés en \textsc{.i.} chastel ki a non Anatout, a \textsc{.iii.} liues de Jherusalem. Il fu prestres en Judee, et fu sacrés a prophete. Avant k’il nasquist fu il conneus, et li fu commandé qu’il maintenist virginité. De s’enfance comença il a preechier et oster les gens de pechié, et a connorter les a penitance. Maint mal li furent fait dou cruel peuple, car il fu mis en chartre et fu getés en \textsc{.i.} lac et fu chains de chaines. Et a la fin fu il lapidés en Egypte ; et fu ensevelis la u Pharaon li rois manoit, et son sepulchre est en grant reverence entre les egyptiiens, por ce ke il les delivra des serpens.
\chapterclose


\chapteropen
\chapter[{.I.L. D’ezechiel}]{\textsc{.I.L.} D’ezechiel}\phantomsection
\label{tresor\_1-50}

\chaptercont
\noindent Ezechiel vaut autant a dire con force de Deu ; fiz fu Buzi, et fu prestres, et fu pris o Joachim son roi et menés en Babilone avec les autres qui la estoient enprisonné. Il prophetiza en Babilone et blasmoit les babiloniens de lor malvaistié. Mais li pueples d’Israhel l’ocist en traison, por ce k’il les reprenoit des crimes et des diablies k’il faisoient. Et fu mis ou sepulchre dou fil le fiz Noé, ki ot a non Arphazat, en lons chans des Mors.
\chapterclose


\chapteropen
\chapter[{.I.LI. De daniel}]{\textsc{.I.LI.} De daniel}\phantomsection
\label{tresor\_1-51}

\chaptercont
\noindent Daniel prophete vaut autant a dire comme onghement de Deu u home amable. Il fu estrais de la lignie Juda, et ses ancestres furent nobles, si com rois et preus . Il fu enportés en Babilone avec le roi Joachin avec les \textsc{.iii.} enfans, et la fu il fais sires et prinches de tous les caldeus. Et il fu hons glorieus et de grant biauté, et ot \textsc{.i.} noble corage et chaste cors. Et fu parfés en foi, et connoissoit des sacres choses, et savoit cele ki avenir devoit.
\chapterclose


\chapteropen
\chapter[{.I.LII. De achias}]{\textsc{.I.LII.} De achias}\phantomsection
\label{tresor\_1-52}

\chaptercont
\noindent Achias prophete fu de la cité Helye. Il dist lonc tans devant le roi Salemon k’il deguerpiroit la loi Deu por une feme. Et quant il morut son cors fu mis en tiere, juste \textsc{.i.} chesne en Silo.
\chapterclose


\chapteropen
\chapter[{.I.LIII. De jagdo}]{\textsc{.I.LIII.} De jagdo}\phantomsection
\label{tresor\_1-53}

\chaptercont
\noindent Jagdo prophete nasqui en Samarie. Il fu envoiés a Jeroboam, ki sacrefioit les vielles a Dieu, k’il demorast avec lui ; mes il n’i demora, et pour ce li avint ke, quant il s’en repairoit, un lion l’estrangla, por ce k’il avoit fali a son compaignon. Et puis fu entieré Jadgo en Bethleem.
\chapterclose


\chapteropen
\chapter[{.I.LIIII. De thobie}]{\textsc{.I.LIIII.} De thobie}\phantomsection
\label{tresor\_1-54}

\chaptercont
\noindent Thobias prophete vaut autant a dire comme bien de Deu. Et fu fiz Ananie, de la lignie Neptalim. Et nasqui en la cité de Chial de la region Galilee. Salmanassar le prist, por ce demora il en exil en la cité de Ninive. Il fu justes en totes choses. Il dona çou k’il avoit as prisons et as povres. Il ensevelissoit les mors de sa main. Puis avugla par fiens d’une arondele ki li chef es ieux. Mais a la fin Dieus li rendi la veue \textsc{.x.} ans aprés, et li dona grant richece. Et fu enterrés en Ninive.
\chapterclose


\chapteropen
\chapter[{.I.LV. Des trois enfans}]{\textsc{.I.LV.} Des trois enfans}\phantomsection
\label{tresor\_1-55}

\chaptercont
\noindent Li troi enfant furent estrait de roial linage, et furent de glorieuse memore, sage de science, et parloient de la foi. Et quand il furent bouté enmi les cheminees del fu ardant, il n’ardirent pas, ains estaindrent le fu, chantant et glorifiant Dameldeu. Et quant il passerent de vie il furent enterré ensamble en Babilone. Ces enfans furent apelé en ebreu Ananias, Azarias, Misael. Mais puis Nabugdonosor les apela Sidrac, Misaac, et Abdenago, c’est a dire Deus glorieus et victorieus sor les roiaumes.
\chapterclose


\chapteropen
\chapter[{.I.LVI. De esdras}]{\textsc{.I.LVI.} De esdras}\phantomsection
\label{tresor\_1-56}

\chaptercont
\noindent Esdras vaut autant a dire comme edefiemens de Jherusalem. Et plusour dient k’il ot a non Malachies, c’est a dire angele de Deu. Il fu prestres et prophetes, il estora les istores des saintes escriptures, il fu li secons ki dona la loi aprés Moyset. Il renovela la loi dou viel testament, ki avoit esté arse par les caldeus au tens de la chetivoison. Il trova les figures des letres as ebreus, et lor ensigna a escrire de diestre vers senestre, ki premiers escrivoient ore avant ore ariere, autresi con li buef font quant il arent la terre. Il ramena le peuple d’Israel, et fist redefiier Jherusalem, et la fu il enterrés.
\chapterclose


\chapteropen
\chapter[{.I.LVII. De zorobabel}]{\textsc{.I.LVII.} De zorobabel}\phantomsection
\label{tresor\_1-57}

\chaptercont
\noindent Zorobabel et Nemias, dou linage Juda, furent provoire et prophete. Et reedefiierent le temple Deu au tans ke Daires li fiz Istapis fu rois de Perse. Il firent les murs de Jerusalem et retornerent Israel en son premier estat, et estorerent le contenement de la religion et la raison des prevoires. Et furent enseveli en Jherusalem.
\chapterclose


\chapteropen
\chapter[{.I.LVIII. De hester}]{\textsc{.I.LVIII.} De hester}\phantomsection
\label{tresor\_1-58}

\chaptercont
\noindent Hester fu roine, et fu fille dou frere Mardochei, et fu menee en prison, de Jherusalem en la cité de Sussi. Et pour sa grant beauté fu ele mariee a Asuerres roi de Perse. Ele soufri la mort por le peuple sauver, et crucifia Amam ki voloit destruire le peuple Israel, et ensi le delivra de mort et de servage. Puis fu ensevelie en Sussi u ele avoit regné.
\chapterclose


\chapteropen
\chapter[{.I.LVIIII. De judit}]{\textsc{.I.LVIIII.} De judit}\phantomsection
\label{tresor\_1-59}

\chaptercont
\noindent Judith fu une veve dame, fille Merari de la lignie Symeon, et fu de haut corage et plus forte ke nul home. Ele ne douta pas le furor des rois, ains s’offri a la mort por sauver le peuple ; car ele ocist Olofernem quant il dormoit, et sans honte de son cors porta son chief a ses citeins, par quoi il orent victore contre ceus de l’ost. Ele veski \textsc{.c.} et \textsc{.v.} ans, et fu ensevelie en la spelonche Manasses son mari, en la cité de Manapullia en la terre Juda, entre Dotaim et Balim.
\chapterclose


\chapteropen
\chapter[{.I.LX. De zacharias}]{\textsc{.I.LX.} De zacharias}\phantomsection
\label{tresor\_1-60}

\chaptercont
\noindent Zacharias vaut autant a dire comme memore de Dameldeu, et fu prophetes et provoires et fiz Joiade le prestre, ki avoit en sornon Branchias, ki fu lapidés du peuple par le commandement du roi de Jude encoste l’autel du temple. Més li autre provoire l’ensevelirent en Jerusalem.
\chapterclose


\chapteropen
\chapter[{.I.LXI. De machabeu}]{\textsc{.I.LXI.} De machabeu}\phantomsection
\label{tresor\_1-61}

\chaptercont
\noindent Machabeu vaut autant a dire comme noble et triumphans ; et furent \textsc{.v.} Machabei, fiz Mathathie, ce sont Joan, Judas, Eleazar, Machabeu, et Jonathas. Et ki voudra savoir les victores k’il eurent sor les rois de Perse et les grans choses k’il fisent, si lise l’istore, ki le conte mot a mot diligement en la grant Bible.
\chapterclose


\chapteropen
\chapter[{.I.LXII. Des livres dou viel testament}]{\textsc{.I.LXII.} Des livres dou viel testament}\phantomsection
\label{tresor\_1-62}

\chaptercont
\noindent Or vous ai jou només les sains peres dou viel testament et lor vie briefment. Mais ki plus larghement le voudra savoir si s’en aille a la grant Bible, ou eles sont toutes escrites apertement. Et sachiés jadis anchienement, tant com li caldeu prisent les juis et les menerent en chetivison, c’est en essil et en prison, lors furent tuit ars li livre de la vielle loi. Mais Esdras par ensegnement du saint esperite, quant li peuples revint de cele chetiveté, renovela tote la loi et le mist en escrit et en fist \textsc{.xxii.} volumes de livres, autresi comme les letres sont \textsc{.xxii.} ; et il escrit le livre de la sapience Salemon. Mes les livres des ecclesiastes escrist Jhesu li fiz Siraac, ki les latins ot en ramembranche pour ce k’il fu samblables a Salemon. Del livre Judith et de Thobie et des Machabeus ne set on pas ki les escrist.
\chapterclose


\chapteropen
\chapter[{.I.LXIII. Ci aprés commence la novele loi}]{\textsc{.I.LXIII.} Ci aprés commence la novele loi}\phantomsection
\label{tresor\_1-63}

\chaptercont
\noindent Après ce ke li contes a dit de la vielle loi, il est bien drois k’il die de la novele, ki commence lors ke Jhesucris vint en terre por nos raembre. Mais ains k’il die autre chose, devisera son linage et son parenté, et puis dire de chascun disciple, autresi com il a dit des peres dou viel testament. Et nous trovons en l’ewangile de St. Matheu le commencement dou linage Jhesucrist en Abraham, ki fu li princes des St. Peres el commencement du tierç aage. Et ki voldra savoir la naissance d’Abraham, il la trovera ça ariere ou conte des premiers homes et des fiz Noé.\par
Abraham engendra Ysaac. D’Ysaac nasqui Jacob. De Jacob nasqui Judas et si frere. De Juda nasqui Phares et Zaram, de Thamar. De Phares nasqui Esrom. De Esrom nasqui Aaram. D’Aaram nasqui Aminadap. De Aminadap nasqui Naazon. De Naason nasqui Salmon. De Salmon nasqui Booz. De Booz nasqui Obeth. De Obeth nasqui Jessé. De Jessé nasqui David rois. De David nasqui Salmon. De Salomon nasqui Roboam. De Roboam nasqui Abias. De Abias nasqui Asa. De Asa nasqui Josaphat. De Josaphat nasqui Joras. De Joras nasqui Ozias. De Ozias nasqui Jonatham. De Jonatham nasqui Achaz. De Achaz nasqui Ezechias. De Ezechias nasqui Manasses. De Manasses nasqui Amon. De Amon naqui Josias. De Josias nasqui Jeconias en le transmigration de Babilone . Et aprés la transmigration, de Jeconias nasqui Salatiel. De Salatiel nasqui Zorobabel. De Zorobabel naquit Abiut. De Abiut nasqui Eliequin. De Eliekin nasqui Azor. De Azor nasqui Sadoc. De Sadoc nasqui Achim. De Achim nasqui Eliut. De Eliut nasqui Eliazar. De Eliazar nasqui Matham. De Matham nasqui Jacob. De Jacob nasqui Joseph le marit Marie, de qui nasqui Nostre Sires Jhesucris. Et sachés ke toutes generations de Abraham jusques a Jhesucrist sont .xli.\par
Et se aucuns demandoit por quoi l’escripture devise la lignie Joseph, puis k’il ne fu peres Jhesucrist, ja soit ce k’il fust maris Marie, et que on devoit conter le parenté Marie, ki fu sa mere, non pas celui de Joseph, ki noient ne li fu, je diroie ke en la vielle loi li ebreu ne se marioient se a ciaus non de lor parenté, et encore le font li juis qui sont en nostre tens. Et a la verité dire Marie fu de celi linage meisme de par son pere. Mais li ancien metoient en escrit les homes non pas les femes, por ce nome Joseph l’istore, et non sa feme ; car en conte de lignie li hom est plus dines ke la feme n’est. Et neporquant je dirai un poi du parenté Marie de par sa mere, en tel maniere que chascuns sache les parens et les cousins Jhesucrist.
\chapterclose


\chapteropen
\chapter[{.I.LXIIII. Du parenté nostre dame}]{\textsc{.I.LXIIII.} Du parenté nostre dame}\phantomsection
\label{tresor\_1-64}

\chaptercont
\noindent Or dist li contes ke Annam et Esmeria furent \textsc{.ii.} serours charneus. De cele Esmerie nasqui Elizabeth et Elivith. De Elivith ki fu freres Elizabeth nasqui Emman. De Emman nasqui Sains Servais, le qui cors gist en terre a Tret sour Mouse en l’eveschié de Liege, en une croute del mostier. De Elizabeth la feme Zacharie le prevoire nasqui Jehans Baptistes en Jherusalem.\par
De l’autre suer, c’est Anna la feme Joachim, nasqui Marie la mere Jhesucrist. Et quant Joachim son mari fu deviés, ele se maria a Cleophas, et Marie sa fille espousa Joseph frere Cleophas. De celui Cleophas et d’Anna nasqui l’autre Marie ki fu feme Alphei, de qui nasqui Jakes Alphei, et Joseph et Judas ; por ce l’apele l’escripture Jakes Alphei, c’est a dire fiz Alphei. Et sa mere est apelee Marie de Jakes, por ce k’ele fu sa mere ; autresi est ele apelee la mere Joseph ; et tout çou avint pour la diversité des ewangiles.\par
Quant Cleophas fu mors, Anna se maria a Salomé, de qui nasqui l’autre Marie la feme Zebedeu, de qui nasqui Jehans l’ewangelistes et Jake son frere ; por ce est ele apelee Marie Salomé de par son pere, autresi est ele apelee mere des fiz Zebedeu, por les diversités des euuangiles.\par
Et ensi veés vous ke Anne ot \textsc{.iii.} maris, et de chascun ot une Marie. Et ensi furent \textsc{.iii.} Maries, dont la premiere fu mere Jhesucrist, la seconde fu mere de Jake et de Joseph, la tierce fu mere de l’autre Jake et de Jehan ewangeliste.
\chapterclose


\chapteropen
\chapter[{.I.LXV. De la mere dieu}]{\textsc{.I.LXV.} De la mere dieu}\phantomsection
\label{tresor\_1-65}

\chaptercont
\noindent La premiere Marie fille Joachim de la lignie David, son non vaut autant a dire come estoile de mer, et dame de clarté et lumiere. Li angeles Gabriel le salua, et li noncha que Dieus prendroit char en li, et lors meismes dist il ke Elizabeth sa cousine, ki avec li estoit, et ja estoit en grant aage, ele et Zacharias son mari, et n’avoient onques engendré fiz ne fille, k’ele auroit un fiz, et si ot ele Jehan Baptiste, de qui li contes dira ça avant.\par
Maint home dient que Sainte Marie mere de Deu morut corporaument au siecle, et ce dient il pour çou ke Symeon dist a li, li coutel passera ta alme. Mais l’en doute de quel coutel il dist, ou de coutel de fer ou de la parole Deu, ki trence plus ke nul coutel. Mais a la verité dire, escripture ne tesmoigne sa mort, ou par coutel ou en autre maniere, ja soit ce ke l’en trueve sa sepulture dans la valee Josaphat.
\chapterclose


\chapteropen
\chapter[{.I.LXVI. De jehan babtiste}]{\textsc{.I.LXVI.} De jehan babtiste}\phantomsection
\label{tresor\_1-66}

\chaptercont
\noindent Elizabeth cousine Marie engendra de Zacharie son mari \textsc{.i.} fil ki ot a non Jehans. Cil fu noncieres Jhesucrist, et fu li definemens des prophetes : il prophetiza Deu ains k’il nasquist, et le salua dedens le ventre sa mere ; il connut Jhesu a la colombe et le monstra au doi. Il meismes le baptiza, et por ce est il apelés Jehans Baptiste. Son vestement fu de laine de cameus, et habita en hermitage et en desers. Sa viande fu miel et locustes.\par
Et a la fin rois Herodes le mist en chartre, por çou k’il li blasmoit de la feme son frere, k’il avoit prise a feme. Et Erodia la fille au roi pria son pere, un jour k’il estoit yvres, k’il li donast le chief Jehan, et il si fist, et ele le fist decoler, et presenta le chief a sa mere. Puis fu il ensevelis en Sebastia, c’est une vile de Palestine, ki jadis fu apelee Samarie : Herodes li fiz Antipater l’apeloit Auguste en grezois, pour reverence de Cesar Auguste l’empereour de Rome.
\chapterclose


\chapteropen
\chapter[{.I.LXVII. De saint jaqeme}]{\textsc{.I.LXVII.} De saint jaqeme}\phantomsection
\label{tresor\_1-67}

\chaptercont
\noindent Jakemes Alphei fu fiz de la seconde Marie, serour de la mere Deu, et pour ce est il apelés freres Damedeu. Et son non vaut autant a dire comme justes, et ensi ot il en sornon aucunefois. Il fu evesques de Jerusalem, et fu de si haute vertu ke li pueples aloit autresi come a eschieles por atouchier ses dras. A la fin le lapiderent et le tuerent li juis, et fu ensevelis delés le temple. Por ce dient li plusour ke Jerusalem en fu destruite. La feste de sa naissance est le premier jour de mai.
\chapterclose


\chapteropen
\chapter[{.I.LXVIII. De st. jude}]{\textsc{.I.LXVIII.} De st. jude}\phantomsection
\label{tresor\_1-68}

\chaptercont
\noindent Judas fu freres Jake, por çou est il apelés Judas de Jake ; et autresi fu il freres Deu. Il ala preechier l’ewangile en Mespotame et en Ponte, et converti les crueus gens et les mauveses. Et fu ensevelis en une cité d’Ermenie, ki a non Eriton, \textsc{.v.} jours devant la toussains.
\chapterclose


\chapteropen
\chapter[{.I.LXVIIII. De st. jehan ewangeliste}]{\textsc{.I.LXVIIII.} De st. jehan ewangeliste}\phantomsection
\label{tresor\_1-69}

\chaptercont
\noindent Jehans Ewangelistes fu fiz Zebedeu de la tierce Marie, et fu freres Jakeme. Son non vaut autant a dire comme grace de Deu. C’est Jehans Ewangelistes ki est figure en samblance d’aigle, pour ce k’il sormonta tous autres en hautece d’ewangile ; car lors k’il se reposa sor le pis Jhesucrist, en but il autresi come a une fontaine la haute soutillité d’ewangile. Deus l’eslut et ama mout, tant que a sa mort li commanda sa mere. Quant il fu enprisonés en l’ille de Patmos, fist il l’Apocalipse, et puis ke li empereour Domitiens morut, issi de prison et s’en ala en Efesun et la le derrenier fist il ewangile.\par
Ses miracles sont teles, k’il mua les verges du bois en fin or. Il fist les pieres d’une riviere devenir precieuses. Une veve dame morte resuscita il por la priiere du pueple. Autresi resuscita il \textsc{.i.} joene home ki estoit deviés. Il but le venin sans damage, et resuscita \textsc{.i.} home ki mors estoit de celui meisme venim.\par
Et sachiés que Jehans vesqui \textsc{.iiiixx.} et \textsc{.xix.} ans. Lors entra il en sa sepulture et se coucha tout vif, autresi comme en son lit, et ce fu \textsc{.lxvii.} ans aprés la passion Jhesucrist. Pour ce dient li plusour k’il ne morut mie et k’il vit encore et se repose et se dort laiens, car on voit apertement tousjours croller la terre desus le sepulchre, et movoir en amont, et la pourre boulir autresi comme par espiremens d’omes ki solent dedens. Car sachiés k’il se coucha en ceste maniere prés d’Efesun \textsc{.vi.} jours devant an renuef, c’est l’endemain de la nativita Jhesucrist.
\chapterclose


\chapteropen
\chapter[{.I.LXX. De saint jaqeme}]{\textsc{.I.LXX.} De saint jaqeme}\phantomsection
\label{tresor\_1-70}

\chaptercont
\noindent Jakes li fiz Zebedei freres Jehan fu quars en l’ordene des disciples. Il escrist des epistles as gens des \textsc{.xii.} lignies ki sont en la dispertion, et preecha l’ewangile en Espaigne en les parties d’occident. Puis le fist ocire a \textsc{.i.} coutel Herodes li tetrarches \textsc{.viii.} jors devant les kalendes d’aoust.
\chapterclose


\chapteropen
\chapter[{.I.LXXI. De saint piere}]{\textsc{.I.LXXI.} De saint piere}\phantomsection
\label{tresor\_1-71}

\chaptercont
\noindent Pieres ot \textsc{.ii.} nons, car il ot non Simons Pieres. Et Symon vaut autant a dire come obeissant, pour çou k’il obei a Deu lors ke il dist, vien aprés moi. Pieres vaut autant a dire come connoissans, pour çou k’il connut Dieu quant il dist, tu ies Crist fiz de Deu vivant. Il nasqui en Galilee, en une vile ki a non Bethsaida.\par
Il est li fermemens de la piere de l’eglise, a qui Deus dist, tu ies Piere, et sor ceste piere fonderai l’eglise. Il est li princes des apostles. Il fu li premiers confiesseres et disciples Jhesucrist. Il tient les clés du ciel. Il preecha l’euuangile en Ponthe, en Capadoce, en Galathas, en Bithine, en Aise, en Ytaile. Il ala par mer sus a sés piés. Il resuscita les mors par son ombre quant il passoit prés d’aus. Il resuscita une veve morte. Il fist degloutir a la terre Ananiam et Safiran. Il fist cheir a la terre Simon Magues ki s’en aloit au ciel contremont.\par
Il tint l’office d’apostoile \textsc{.viii.} ans en Anthioce et \textsc{.xxv.} ans en Rome. Mais a la fin l’empereour Noiron le fist crucefiier le chief desous et les piés contremont \textsc{.xxxviii.} ans aprés la passion Jhesucrist \textsc{.ii.} iors a l’issue du mois de jung ; et fu ensevelis en Rome vers soleil levant.
\chapterclose


\chapteropen
\chapter[{.I.LXXII. De st. paul}]{\textsc{.I.LXXII.} De st. paul}\phantomsection
\label{tresor\_1-72}

\chaptercont
\noindent Paul vaut autant a dire come mervilleus, ki premierement avoit a non Saulos. Il fu angeles des homes, et avocas des juis. Quant dieus l’apela il chei a terre et perdi la veue des ieux, mais il vit la venté Deu et s’esdreça et recovra sa veue, et si com il estoit persecuteur de l’eglise, devint il puis vaissiaus d’election. Il fut li plus novel entre les apostles, mais a preechier fu il premiers et soverains.\par
Et fu nés en Judee dou linage Benjamin, et fu baptiziés le secont an aprés l’ascention Jhesucrist. Il preecha de Jerusalem dusk’en Espaigne et par toute Ytaile, et a ceaus descovri le non Dieu ki ne le savoient.\par
Ses merveilles sont teles k’il fu portés jusques a haut ciel. Il resuscita \textsc{.i.} enfant mort. Il fist avugler \textsc{.i.} mague. Il fist devenir mut l’esperit du diable. Il sana les boçus. Il ne douta le mors de la vipre, ains l’ardi en feu. Il sana par ses orisons le pere Publii de fievre k’il avoit. Il soufri por le non Deu fain, soif, froit, nuece, et demora el parfont de la mer un jor et une nuit. Il soufri la rage des bestes sauvages et maintes ferues et tormens de chartre. Li juis le traïrent, et fu lapidés a mort. Il fu enchaenés en une prison dont il fu desloiés par un termuet. A la fin le fist l’empereour Noiron decoler, le jour que Sains Pieres fu crucefiiés.
\chapterclose


\chapteropen
\chapter[{.I.LXXIII. De st. andrieu}]{\textsc{.I.LXXIII.} De st. andrieu}\phantomsection
\label{tresor\_1-73}

\chaptercont
\noindent Andreas vaut autant a dire en grezois comme biaus. Il fu li secons entre les apostles, et preecha en Scite et en Achaie, ou il fu crucefiiés quant il ot faites maintes merveilles ; et morut le derrain jor de novembre. Sa sepulture est a Patras, ou il morut.
\chapterclose


\chapteropen
\chapter[{.I.LXXIIII. De saint phelippe}]{\textsc{.I.LXXIIII.} De saint phelippe}\phantomsection
\label{tresor\_1-74}

\chaptercont
\noindent Phelipe vaut autant a dire come bouche de lampe. Il fu nés en cele meisme cité dont Pieres fu nés, et preeça en Gaille jouste la mer occeane ; a la fin fu il lapidés et crucefiiés en Girople, une cité ki est en Frise, u il morut le premier jour de mai, et fu ensevelis avec ses filles.
\chapterclose


\chapteropen
\chapter[{.I.LXXV. De saint thomas}]{\textsc{.I.LXXV.} De saint thomas}\phantomsection
\label{tresor\_1-75}

\chaptercont
\noindent Thomas vaut autant a dire com abisme, et il ot en sornon Didimus, ki vaut autant a dire comme douteus, car il se douta de la resurrection Jhesucrist jusk’a tant k’il bouta sa main dedens la plaie. Et il preecha en Parthe et en Mede et en Perse et en Orcanie et en Inde vers orient. A la fin fu il navrés de glaves et de lances, tant k’il morut \textsc{.xi.} jors a l’issue de decembre, en une cité d’Inde ki avoit non Calamia ; la fu il ensevelis honorablement.
\chapterclose


\chapteropen
\chapter[{.I.LXXVI. St. bartholomeu}]{\textsc{.I.LXXVI.} St. bartholomeu}\phantomsection
\label{tresor\_1-76}

\chaptercont
\noindent Bartholomeu preecha entre les juis et translata l’ewangile de Matheu en lor langage. A la fin fu il escorchiés par les barbarins en Inde le grant en la cité de Alboger, et puis par commandement du roi Astrages li fu la tieste copee \textsc{.vi.} jours a l’issue d’aoust.
\chapterclose


\chapteropen
\chapter[{.I.LXXVII. De st. matheu}]{\textsc{.I.LXXVII.} De st. matheu}\phantomsection
\label{tresor\_1-77}

\chaptercont
\noindent Matheu fu apostles et ewangelistes, et ot en sornon Levi. Il fist ses ewangiles en Judee et puis preeça en Machedoine, et soufri martire en Perside, et fu enterrés es Mons de Pastours \textsc{.x.} jours a l’issue de septembre.
\chapterclose


\chapteropen
\chapter[{.I.LXXVIII. De st. mathias}]{\textsc{.I.LXXVIII.} De st. mathias}\phantomsection
\label{tresor\_1-78}

\chaptercont
\noindent Mathias fu uns des \textsc{.lxx.} disciples. Més puis fu il uns des \textsc{.xii.} apostres en lieu de Jude Escariot. Il preecha en Judee. La feste de sa nativité est \textsc{.v.} jours a l’issue de fevrier.
\chapterclose


\chapteropen
\chapter[{.I.LXXVIIII. De st. luc}]{\textsc{.I.LXXVIIII.} De st. luc}\phantomsection
\label{tresor\_1-79}

\chaptercont
\noindent Lucas ewangelistes vaut autant a dire comme mires ou luisans ; et a la verité dire il fu fisiciens et bons mires. Nés fu de Sirie. Il sot bien le langage de Grece, et aucun dient k’il fu proselites et k’il ne sot le langage d’ebreu. Mais il fu disciples de Pol et tozjors li tint compaignie. Et morut a \textsc{.lxxiii.} ans de son aage, et fu ensevelis en Bethanie \textsc{.xiiii.} jors avant la toussains. Mais ses os en furent porté en Constantinoble au tans de l’empereour Coustentin.
\chapterclose


\chapteropen
\chapter[{.I.LXXX. De saint simon}]{\textsc{.I.LXXX.} De saint simon}\phantomsection
\label{tresor\_1-80}

\chaptercont
\noindent Symon Zelotes vaut autant a dire comme Cananeus ou possession. L’en quide k’il soit et fust pareil a Piere en congoissance et en honour, car il tint la dignité en Egypte. Et aprés la mort de Jake le filz Alphei fu il evesques de Jerusalem. A la fin fu il crucefiiés, et son cors gist en Bossoffre. La feste de sa nativité est \textsc{.iiii.} jors devant la toussains.
\chapterclose


\chapteropen
\chapter[{.I.LXXXI. De saint marc}]{\textsc{.I.LXXXI.} De saint marc}\phantomsection
\label{tresor\_1-81}

\chaptercont
\noindent Marcus euuangelistes vaut autant a dire comme grans. Il fu fiz de Piere en baptesme et fu ses disciples. Et por ce dient li plusour ke son ewangile fu dite par la bouche de Piere. Et dient k’il se colpa le gros doi pour ce k’il ne voloit mie ke l’en le feist prestre, et toute fois fu il le premier ki ot siege de dignité en Alixandre. Et fonda premierement eglise en Egypte, et morut au tans Noiron \textsc{.vi.} jors a l’issue d’avril.
\chapterclose


\chapteropen
\chapter[{.I.LXXXII. De saint barnabé}]{\textsc{.I.LXXXII.} De saint barnabé}\phantomsection
\label{tresor\_1-82}

\chaptercont
\noindent Barnabé avoit a non premierement Joseph, et vaut autant a dire come feel. Nés fu en la cité de Cypre, et tint l’apostoillage avec Paul, puis le laissa et ala preechant. La feste de na nativité est \textsc{.xiii.} jors a l’entree de jung.
\chapterclose


\chapteropen
\chapter[{.I.LXXXIII. De st. tymothee}]{\textsc{.I.LXXXIII.} De st. tymothee}\phantomsection
\label{tresor\_1-83}

\chaptercont
\noindent Thimotheu fu li secons disciples Paul, car il meisme le mena de s’enfance avec lui, et le baptiza. Et cil garda virginité et chasteé. Nés fu de la cité de Listenois, et fu ensevelis en Efesun \textsc{.x.} jors a l’issue d’aoust.
\chapterclose


\chapteropen
\chapter[{.I.LXXXIIII. De saint tythus}]{\textsc{.I.LXXXIIII.} De saint tythus}\phantomsection
\label{tresor\_1-84}

\chaptercont
\noindent Thitus fu li disciples Paul et ses fiz en baptisme. Nés fu de Greze. Il seul fu circoncis aprés l’ewangile par les mains de Paul meisme, et il le laissa pour destruire les ydoles de Grece et por edefiier les eglises, et la mor—t il, et fu ensevelis en pés.
\chapterclose


\chapteropen
\chapter[{.I.LXXXV. Des dis commandemens de la loi}]{\textsc{.I.LXXXV.} Des dis commandemens de la loi}\phantomsection
\label{tresor\_1-85}

\chaptercont
\noindent Or vous ai je només les mestres dou novel testament. Et sachiés que les \textsc{.iiii.} euuangiles furent faites par les \textsc{.iiii.} euuangelistes. Paul fist et escrist ses epistles, dont il envoia les \textsc{.vii.} as eglises, les autres envoia a ses disciples, c’est a Thimotheu et a Thitus et a Polomeu. Més de cele ki fu envoie as ebreus sont li latin en discorde, car li un dient ke Barnabas le fist, et li autre dient de Clement.\par
Pierres fist \textsc{.ii.} epistles en son non. Jakes fist la soue. Jehans fist \textsc{.iii.} epistles ; mais li plusor dient ke uns prestres ki ot a non Johans fist les \textsc{.ii.} Judes fist la soue. Lucas ewangelistes escrist les vies des apostles, selonc ce k’il vit ou k’il out Johans escrist l’Apocalipse quant il estoit en prison ; et chascuns d’aus escrist par divin espirement, et ensi ordenerent les \textsc{.x.} commandemens, selonc ce ke nous devons vivre.\par
Et sachiés ke li commandement de la loi sont \textsc{.x.}, dont li premiers dist, ayme et cultive ton seul Deu. Li secons dist, ne reçoi pas en vain le non Deu. Li tiers dit, soviegne toi de saintefiier le sabat. Li quars dit, honeure ton pere et ta mere. Li quins dit, non faire avoutire. Li \textsc{.vi.} dit, non ocire. Li \textsc{.vii.} non faire larrecin. Li \textsc{.viii.} non faire faus tesmong. Li \textsc{.ix.} dit, ne covoite pas la chose de ton proisme. Li \textsc{.x.} dist, ne desire pas la feme ton prochain.\par
Et ja soit ce k’il soient devisé en \textsc{.x.} parties, on les poroit tous comprendre par les \textsc{.ii.} solement, c’est ayme Deu de tout son cuer et de tote ta vie et de toute ta vertu, et ayme ton prochain ausi comme toi meisme. Deus commandemens sont la some sans plus de toz \textsc{.x.} ; car en eus sont la lois et les propheties. Uns autres commandemens est en l’Escripture ki tous comprent les \textsc{.x.}, c’est deguerpissiés le mal et faites le bien. Et un autre est samblable a cestui ki dist, ce ke tu ne wés a toi, ne le faire pas a autrui. Mais ci se taist li contes a parler de le vie des peres de l’un testament et de l’autre, et tornera a sa matire, la u il laissa a Julle Cesar et de Octevian son nevou, ki furent li premier empereour de Rome.
\chapterclose


\chapteropen
\chapter[{.I.LXXXVI. Coment la premiere loi fu commencie}]{\textsc{.I.LXXXVI.} Coment la premiere loi fu commencie}\phantomsection
\label{tresor\_1-86}

\chaptercont
\noindent Ci endroit dit li contes ke Nostre Sires Jhesucris nasqui en cest siecle por raembre l’umain linage, el tens Octevien empereour de Romme. Et sachiés que le premier an de sa naissance li \textsc{.iii.} roi le vinrent aourer ; et el tierç en furent decolé li petit enfant innocent ; au septime an revint il d’Egypte o sa mere et Joseph, ki l’i porterent por paour d’Erode ; le .xiime. an de son aage sist il ou temple de Jherusalem ou il demostra sa grant sapience, si ke tous li mondes s’en esmervilloit.\par
A \textsc{.xxx.} ans fu il baptiziés, et lors commença il a preechier la novele loi et la droite creance et la cognoissance de la Sainte Trinité, c’est a dire la unité des \textsc{.iii.} persones, dou Pere ki est segnefiiés par la poissance, dou Fil ki est segnefiiet par la sapience, dou Saint Esperit ki est segnefiiet par la bienweillance. Pour ce devons nous croire ke ces \textsc{.iii.} persones soient une sustance, ki est tous poissans, tous sachans, et tous bienweillans.\par
Et quant Nostre Sires ot vescu \textsc{.xxxii.} ans et \textsc{.iii.} mois, il fu mors par les juis et par les traison Judes, selonc ce ke le euuangile le tesmoigne. Et ensi fu Nostre Sires Jhescris li premiers evesques et apostles et ensegnieres et maistres de la sainte crestiene loi.\par
Et quant Nostre Sires s’en ala es cieus, il laissa Saint Piere, son vigeour, en lieu de lui, et li dona pooir de liier et de desliier en terre. Et ensi tint Sains Pieres la chaiere et la dignité apostoliel es parties d’orient \textsc{.iiii.} ans ; puis s’en vint en Anthioce u il fu evesques \textsc{.viii.} ans, aprés ce s’en vint il a Rome u il preecha et mostra la loi Jhesucrist as gens, et la fu il evesques et mestres de la crestiieneté \textsc{.xxv.} ans et \textsc{.vii.} mois et \textsc{.viii.} jors, jusques au tans Noiron, ki lors estoit empereour ; ki par sa grant cruauté le fist crucefiier, et fist decoler Saint Pol tout en \textsc{.i.} jor.\par
Et quant Pieres dut morir, il ordena un de ses disciples, ki avoit a non Clemens, a tenir la chaiere aprés lui ; mais il ne le volt onques tenir, ains constrainst Linum son compaignon, ki le tint tant com il vesqui, et puis constrainst il Cleton, ki autresi la tint toute sa vie. Et quant il furent mort ambedeus, Clemens meismes tint la chaiere aprés lui, et fu aspotoiles de Rome. Et ce fu aprés la mort Titus empereour de Rome, cil meismes ki au tans Vaspassien empereour son pere, ki regna aprés Noiron, avoit conquise Jherusalem et les juis mors et pris et vengie la mort Jhesucrist \textsc{.xl.} ans aprés la passion.
\chapterclose


\chapteropen
\chapter[{.I.LXXXVII. Coment crestientés essauça au tens silvestre}]{\textsc{.I.LXXXVII.} Coment crestientés essauça au tens silvestre}\phantomsection
\label{tresor\_1-87}

\chaptercont
\noindent Et pour ce ke nature ne suefre mie que aucuns, coment k’il soit grans ne de haute dignité, trespasse le jour de sa fin, covint il ke li apostole et li empereour de Rome alaissent a la mort, et autre refussent establi en leu d’aus. Et pour ce ke la lois des crestiiens estoit novelement venue, si ke li un estoient en doute, li autre mescreant, avint il maintes fois ke li empereour et li autre ki governoient les viles faisoient maintes grans persecutions contre les crestiiens, et lor faisoient a soufrir divers tormens, jusques au tans ke Coustentin le Magne fu empereres et Silvestres fu evesques et apostoiles de Rome. Et sachiés ke devant aus avoient esté \textsc{.xxxv.} empereour aprés Julle Cesar et \textsc{.xxxiii.} apostoles aprés Jhesucrist.\par
Or avint chose ke Silvestres ot grant compaignie des crestiiens, si s’en estoit fuis ensus une haute montaigne pour eschiver les persecutions ; et Coustentins li empereres, ki estoit malades d’une lepre, l’envoia querre ; car a ce ke l’en disoit de lui et de son ancestre, il voloit oïr son conseil. Et tant ala la chose, ke Silvestre le baptiza selonc la foi des crestiiens, et le monda de sa lepre ; lors maintenant devint il crestiiens o tous les siens. Et pour enhaucier le non de Jhesucrist doa il Sainte Eglise et li dona toutes les imperiales dignités, et ce fu fait en l’an de l’Incarnation \textsc{.iiic.} et \textsc{.xxiii.} ; et ja estoit trovee la Sainte Crois poi devant.\par
Lors s’en ala Coustentins en Constantinoble, ki por son non est ensi apelee, car primes avoit ele non Besans ; et tint l’empire de Gresse, k’il ne sousmist mie as apostoles selonc ce k’il fist celui de Rome\par
Et sachés que les persecutions des crestiiens dura jusques au tans Silvestre. Et por ce furent tuit saint li apostoile ki devant lui soufrirent martire pour la foi. Mais quant li empereres dona si grant honor a Silvestre et as pastours de Sainte Eglise, toutes persecutions furent definees ; mais lors commencierent les errours des erites, ki se desvoierent contre Silvestre, por quoi maint empereour aprés et maint des rois de Lombardie furent corrompu de male creance, jusques au tans Justinien, ki fu empereor en l’an de l’Incarnation \textsc{.vc.} et ix.\par
Cist Justinien fu de mout grant sapience et de grant pooir, ki par son grant sens abrega les lois dou Coude et du Digest, ki premiers estoit en tante confusion ke nus n’en pooit a chief venir. Ja soit ce k’il fust au comencement en l’erreur des erites, a la fin reconut il son erreur de par le conseil Agapite, ki lors estoit apostoiles, et lors fu la crestiene lois confermee et fu devee la creance des erites, selonc ce ke l’en puet veoir sus les livres de lois qu’il fist. Et il regna \textsc{.xxxviii.} ans.\par
Et sachiés ke devant lui avoient esté \textsc{.xvi.} empereour des ke Coustentins fu empereour de Rome. Et de Silvestre jusques a cestui Agapite furent \textsc{.xxviii.} apostoiles.
\chapterclose


\chapteropen
\chapter[{.I.LXXXVIII. Comment sainte eglise essaucha}]{\textsc{.I.LXXXVIII.} Comment sainte eglise essaucha}\phantomsection
\label{tresor\_1-88}

\chaptercont
\noindent Des lors essaucha la force de Sainte Eglise, long et prés, et de cha mer et de la, jusques au tans Eracle, ki empereours fu en l’an de l’Incarnation \textsc{.vic.} et \textsc{.xviii.}, et regna \textsc{.xxxi.} an au tans Coustentin et son fiz, ki regna aprés lui. Car au tans d’eus, li sarrasin de Perse orent grant force contre les crestiiens, et gasterent Jerusalem et ardirent les eglises, et enporterent le fust de la Crois, et enmenerent le patriarche et maint des autres en chetivison, ja soit ce ke a la fin Eracles meismes i ala et ocist le roi de Perse et ramena les prisoniers et la Crois et sosmist les persans a la loi de Rome. Et puis i fu li mauvais preechieres Mahommés ki fu moines, ki les retraist de la foi et les mist en erreur.
\chapterclose


\chapteropen
\chapter[{.I.LXXXVIIII. Li rois de france comme il fu empereor}]{\textsc{.I.LXXXVIIII.} Li rois de france comme il fu empereor}\phantomsection
\label{tresor\_1-89}

\chaptercont
\noindent Or avint si comme si il plot a Nostre Signor ke Sainte Eglise enhauça et crut de jor en jour, meismement pour la force et por la signorie ki fu aquise au tans Silvestre ; mais li autre empereour ki aprés Costentin furent n’estoient mie si dous ne si debonaires comme il fu, ains recovraissent volentiers ce ke Costentins avoit fet s’il en eussent le pooir, mais Deus ne lor soufri pas. Et ensi estoient li empereor tousjors li un aprés l’autre, teus fois bons et teus fois malvais, et tenoient l’un empire et l’autre, jusques au tans Lyon empereour et Costentin son fil.\par
Cist empereres prist toutes les ymages des eglises de Rome et les emporta en Constantinoble en despit de l’apostoile, et les fist ardré en fu ; et fist contre lui une conjurison, entre lui et Telofle roi des lombars, por quoi Estienes li apostoles ki lors estoit les escomenia, et li toli Puille et l’establi qu’ele fust tousjours de Sainte Eglise.\par
Et quant il vit k’il ne pooit avoir contre aus longhe duree, il en ala en France au bon Pepin, ki lors en estoit rois et sires, et consecra li et ses fius a estre tozjors rois de France, et maudist et escomenia toz ceaus ki jamés feroient roi d’autre linage ke de la Pepin. Puis ala li rois avec l’apostoile a toutes ses os en Lombardie, et se combati contre Telofre tant k’il le venki, et li fist fere l’amende de Sainte Eglise, selonc ce ke li apostoles et si frere voudrent commander. Et par sa force fu establie la besoigne dou roiaume de Puille et dou patrimone Saint Piere, en icele maniere k’il deviserent.\par
Mais quant Pepins s’en fu alés en son païs, ne demora mie gramment ke Constantins li fiz Lyon, quant il fu empereres aprés la mort son pere, fist tout le pis k’il pot contre l’eglise de Rome. Et Desdiers rois de Lombardie recomença d’autre part la guerre, grignour ke Telofre son pere n’avoit onques fait en sa vie ; tant ke Adrians li apostoiles pria Karlemaine le fiz Pepin, ki lors estoit rois de France, tant k’il vint en Ytaile et venki la cité de Pavie ou le roi manoit et prist Desdier et sa feme et ses fiz et lor fist jurer la feauté de Sainte Eglise, puis l’envoia prison en France.\par
Mais Algife le fiz Desdier le roi s’enfui par mer en Constantinoble, ki puis fist mout de guerre. Et quant Charles ot Lombardie tote conquise et tote la terre de Ytaile sousmise a soi et a Sainte Eglise,il ala en Rome a grant triumphes. La fu il coronés empereres des romains, et tint la dignité de l’empire toute sa vie. Puis ot il maintes hautes victores contre les sarrasins et les enemis de sainte Eglise, et sousmist a sa signorie Alemaigne et Espaigne et maint autres païs.\par
Et quant Lions pepe, ki fu apostoiles de Rome aprés la mort Adrian, fu essilliez par les romains, Charles le ramena en Rome en sa dignité, et lors conferma il ce ke ses peres avoit fet, et establi toutes les besoignes de Sainte Eglise et de l’empire et de clers et de lais, et dona a mon signour St. Piere la duchee d’Ispolite et de Bonivent. Et puis k’il ot ce fait, et maintes grans choses et hautes, il trespassa dou siecle, en l’an de l’incarnation \textsc{.viiic.} et \textsc{.xiiii.} Et sachés ke devant lui orent esté \textsc{.xvi.} empereour de Justiniien et \textsc{.xl.} apostole d’Agapite jusques a cesti.
\chapterclose


\chapteropen
\chapter[{.I.LXXXX. Comment li empire de rome revint as ytaliens}]{\textsc{.I.LXXXX.} Comment li empire de rome revint as ytaliens}\phantomsection
\label{tresor\_1-90}

\chaptercont
\noindent En ceste maniere que li contes devise ci devant vint la dignité de l’empire de Rome as françois, et li romain n’en orent onques puis la signorie k’il orent devant. Et quant Karlemaine trespassa de cest siecle, Loys ses fiz fu aprés rois et empereres, ki regna \textsc{.xxv.} ans. Et quant il morut il laissa \textsc{.iiii.} fiz ; mais ains k’il fust deviés devisa il entre ses fiz ke Carle Cauf eust le roiaume de France, et ke Lotiers eust l’empire de Rome, et ke Pepins eust Alemaigne, Loys Aquitaine.\par
Or avint chose que quant Lotiers ot la signorie de l’empire, et il vit sa force et son pooir, il se pensa k’il iroit en France conquerre le roiaume son pere, et ensi s’en ala o tous les os des ytaliiens et passa les mons et venki la terre jusques a la cité de Rains ; la trova il ses freres ki li venoient a l’encontre, a tout si grant effor de gent k’il vit apertement k’il ne le poroient vaincre. Et quant il connut ke ses proposemens estoit defaillans, il se rendi moines en l’abeie de St. Marc de Saissoigne et laissa l’empire de Rome a son fiz Loys.\par
Cil Loys vesqui en son empire \textsc{.ii.} ans. Et quant il fu deviés il ne laissa k’une fille ki fu mariee au roi de Puille. Lors vint a Rome Karle Cauf le roi de France, et fu empereour mains de \textsc{.ii.} ans.\par
Mais por çou ke les guerres crurent diversement en Ytaile, et ke li empereour ki estoient françois n’aidoient mie les romains contre les lombars et contre les autres ki les damagoient sovent et menu, avint il ke par sentence de romains la dignité de l’empire fu tolue as françois et revint as ytaliiens ; dont li premiers fu Loys le joene, ki estoit fiz au roi de Puille et de sa feme, ki fu fille Loys, de qui l’istore parole ce devant.\par
Et dient li plusor que uns angeles comanda au desrenier roi des françois ki fu empereor k’il ne s’entremeist jamés de l’empire as romains, et k’il le quitast au joene roi de Puille ; et sor ce fu sentence fermee por ce ke li françois n’aidoient les romains ne ne deffendoient l’empire contre les ylaliiens ne contre les maufetours.
\chapterclose


\chapteropen
\chapter[{.I.LXXXXI. De l’empereour berengier}]{\textsc{.I.LXXXXI.} De l’empereour berengier}\phantomsection
\label{tresor\_1-91}

\chaptercont
\noindent En tel maniere come je vous ai devisé revint l’empire de Rome des françois as lombars, en l’an de grace \textsc{.ixc.} et \textsc{.ii.}, dont li joenes Loys fu li premiers. A son tens comença une divisions en l’empire, car un en estoit empereour en Ytaile et un autre en Alemaigne. Et ce dura aprés lui au tans de \textsc{.v.} empereours, ki estoient li uns aprés l’autre jusques au tans Berengier et Aubert son fil, ki furent li derrenier lombart ki l’empire tenissent ; et que Agapite fu apostoiles, ki maintesfois se combati contre les romains, por maintenir les drois de sainte eglise. Mais aprés lui fu apostoiles Jehans fiz a celui Berengier empereour. Et sachés que devant lui avoient esté \textsc{.xi.} empereour puis Karle ; et \textsc{.xli.} apostoiles de Lyon jusques a cestui Johan pape.\par
Et sachés que cestui Berengier fu coronés en l’an de grace \textsc{.ixc.} et \textsc{.xl.}, et regnerent entre lui et Aubert son fil en Ytaile \textsc{.xi.} ans. Celui Aubert avoit \textsc{.i.} frere a clerc ki avoit non Octevian. Il porcacha tant avec les grans mestres de Rome que, aprés la mort Agapite, ki lors estoit pape, Octevian son frere fu fet apostoile et fu apelés Jehans.\par
Or dient li maistre ki la cronike firent et ki misent en escrit les istores de cel tens, ke Berengier fu malvais tyrans et cruel a Dieu et au monde, et k’il prist une grant dame ki avoit esté feme Lotier empereor devant lui, il le prist et le gardoit en chartre et faisoit toutes les diaublies et les cruautés. Aubert son fiz faisoit d’autre part tous les maus k’il onques pooit, et que Johans ses freres, qui fu apostoiles, ki d’assés estoit pires ke son pere et ke son frere ; et il furent rnaistres et signors de par sainte eglise et de par le siecle ; lors crut mal sor mal et cruauté sor cruauté.
\chapterclose


\chapteropen
\chapter[{.I.LXXXXII. De l’empereor octe}]{\textsc{.I.LXXXXII.} De l’empereor octe}\phantomsection
\label{tresor\_1-92}

\chaptercont
\noindent Or dist l’istore, et li registre de sainte eglise le tesmoignent, ke por la malvaisté Berengier li preudome de sainte eglise et du comun de Rome et dou païs environ manderent a Octe de Saissoigne, ki estoit rois d’Alemaigne, k’il les venist aidier contre cel diauble. Et dont il vint puissamment en Ytaile, et venki Berengier, et le cacha hors de la terre, et osta de prison la veve dame de qui li contes a parlé ci devant, et le prist a feme. Puis s’acorda il a Berengier, et li rendi Lombardie trestoute, se ce ne fu la marche de Trevise et de Veronne et d’Aquilee, k’il ne li rendi pas ; et ensi s’en rala en Alemaigne, et regna longhement en grant pooir.\par
Or avint ke Berengier et Aubert faisoient mal et pis ke devant, et li apostoiles Jehans tenoit les femes apertement, et faisoit çou k’il voloit, non mie çou k’il devoit ; pour la quel chose aucun des cardenaus et des preudomes de Rome envoierent priveement a Octe meismes, k’il venist aidier l’eglise et preist le governement de l’empire et de la terre, ançois k’il le destruisist du tout.\par
Et quant il ot oï ce, il se mist a la voie, et vint en Lombardie et en Toscane, et entra en Rome, et fu receu par tout honorablement, et fu coronés a roi et empereor de Rome en l’an Nostre Signor \textsc{.ixc.} et lv. ; et regna \textsc{.xii.} ans, et fu li premiers empereour nés d’Alemaigne ; et maintes fois revint a Rome, por les biens de l’une terre et de l’autre. Et por ce ke l’apostoiles Jehans ne voloit laissier ses maus et torner a bone vie, fu il desposés par la volonté de la clergié et de tous autres de sa dignité, et fu esleu uns autres ki ot a non Lyon.\par
Cestui apostoile establi, por le malice des romains, que pape ne peust estre fés ne esleu sans l’assentement des empereors. Or avint une fois que Octes li empereres estoit alés en Alemaigne, li romain par lor malice esleurent un autre pape ki avoit non Benoit, et Lion fu getés de son office\par
Ensi estoient a celui tens \textsc{.iii.} apostoiles vivans, c’est a dire Jehan, Lion, et Benoit. Mais Benoit ne tint mie la chaiere plus de \textsc{.ii.} mois, ke l’empereor revint et assist Rome a tote son ost, tant ke l’en li rendi la terre et ke Lion refu mis en sa dignité ; et apaisa le païs et les gens, et s’en rala ariere en Alemaigne, et en amena avec lui Benoit pape en Sassoigne u il morut. Et l’autre pape Johan trespassa de cest siecle sans repentance et sans confession.\par
Et li empereres ot de sa feme \textsc{.i.} fiz ki autresi ot a non Octes, et fu empereor aprés lui en l’an de grasce. ixc. et lxviii. Et fu preudome et vaillant, et fist bones oevres et grans, et ot a feme la fille l’empereor de Costantinoble, en qui il engendra \textsc{.i.} fis ki autresi ot a non Octes li tiers. Et fu coronnés a empereour par les mains du cinkime Grigore pape en l’an de Nostre Signeur \textsc{.ixc.} et iiiixx. et ix. Et tint la terre et l’empire bien et vaillamment, ja soit ce k’il fist maintes persecutions contre les romains. Puis trespassa du siecle, si comme il plot a Dieu glorieus et beneoit.\par
Aprés fu esleu Frederik. Cestui Frederik fu vaillant home, et tint les lombars en molt grant destrece, et destruist la cité de Melan, et la fist arer et semer de sel, et ot guerre a l’apostole Innocent le tierç, et le cacha de Rome. Et li apostoiles et si frere s’enfuirent dusk’a Venise, et la les assega li empereour ; et afama la vile de Venisse, en tel maniere ke les citeins vindrent a l’apostoile et li disent k’il voloient mieus k’il alast hors de la vile k’il i moruissent de faim ; et l’apostoile et si frere se revestirent des armes de sainte eglise, et se misent en \textsc{.i.} bac, et alerent a l’ost l’empereour.\par
Et quant l’empereor les vit, il erranment s’en vint a l’apostoile a merci, et se mist et rendi a ses piés, et l’apostoile li mist le pié desous la goule et dist, Super aspidem et basiliscum ambulabis, et conculcabis leonem et draconem. Et le empereour li respondi, Non tu set Xpc. Et je sui son vicaire, dist li apostoles. Et la li commanda li apostoles k’il alast en la Terre Sainte por son meffet, et la ala il par terre, et chei de son cheval en une petite riviere, et la se neya.
\chapterclose


\chapteropen
\chapter[{.I.LXXXXIII. Comment li empires vint as alemans}]{\textsc{.I.LXXXXIII.} Comment li empires vint as alemans}\phantomsection
\label{tresor\_1-93}

\chaptercont
\noindent Mais puis ke la hautece et la signorie de l’empire de Rome crut et enhaucha sor toutes les dignités des crestiens, et ke l’envie croissoit et engendroit maintes haines entre les nobles lombars, et nus n’estoit ki se mellast de maintenir la chose commune se li prince d’Alemaigne non, fu establi ausi comme par necessité plaine de droit que la naissance et la election de l’empire fust faite par ciaus ki en estoient deffendeours et gardes : en tel maniere que li empereor fussent esleu por bonté et por proece, non mie par iretage, si comme li \textsc{.iii.} Octe avoient esté.\par
Et ensi vint la hautece d’eslire enpereour as \textsc{.vii.} princes d’Alemaigne ki sont officieus de l’empire, c’est a dire le archeveske de Magance, ki est cancelier an Alemaigne la u ele est Germonie. Li secons est li archevesques de Trieves, ki est cancelier en la terre de ça vers France. Li tiers est est li archeveske de Coloigne, ki est canceliers en Ytaile. Li quars est li marcis de Brondebourk, ki est chambrelens de l’empire. Li cinkimes est li quens palatins ki siert du premiers més. Li sisimes est li dus de Saissoigne ki porte l’espee. Li viiimes est li rois de Boen, ki est boutilliers l’empereour.
\chapterclose


\chapteropen
\chapter[{.I.LXXXXIIII. De l’empereor henri}]{\textsc{.I.LXXXXIIII.} De l’empereor henri}\phantomsection
\label{tresor\_1-94}

\chaptercont
\noindent Après ce fu esleus a roi et empereor Henris en l’an de grace \textsc{.mcc.} et \textsc{.iii.} Et puis k’il fu deviés fu esleus Octes dus de Saissoigne, et ot guerre a sainte eglise, et se combati a Phelippe roi de France, et fu desconfit, et puis fu desposés par sainte eglise.\par
Aprés fu li secons Frederis, ki fu fiz a l’empereour Henri et l’emperreis Constance, ki autresi estoit roine de Sezille et de Puille et de par son pere ki rois en fu. Cestui Frederik fu coronés par les mains Honouré pape en l’an de grace \textsc{.m.} et \textsc{.cc.} et .xx.\par
Et sachiés ke devant cestui Honoré pape avoient esté \textsc{.lii.} apostoles, de Jehan de qui li contes parole en la fin des lombars. Et dou premier empereour de Rome, ce fu Cesar, avoit jusc’a cestui Frederik \textsc{.iiiixx.} et \textsc{.xv.} empereour sans plus. Et se Merlins u la Sebille dient verité, l’en troeve en lor livres que en cesti doit definer la imperiale dignités. Més je ne sais se c’est a dire de son linage solement, u des alemans, u se ce dist de tous communalment.
\chapterclose


\chapteropen
\chapter[{.I.LXXXXV. De la hautece frederiq}]{\textsc{.I.LXXXXV.} De la hautece frederiq}\phantomsection
\label{tresor\_1-95}

\chaptercont
\noindent Cestui Frederik fu hom de grant cuer sor tous homes, et si estoit mervilleusement sages et artilleus et trop bien letrés, et savoit tous langages ; et ses cuers ne baoit a autre chose fors k’a estre sires et signor de tout le monde. Et ja soit ce k’il ot plusors femes et enfans de droit mariage, toutefoies aloit il as gentiz femes dou païs, si k’il en ot fiz et filles a grant plenté, ki vindrent en grant honour et en grant pooir. Et il quidoit bien par lui et par ses fius k’il sorprendroit l’empire et la terre toute, en tel maniere k’ele n’istroit jamés de lor subjection.\par
Mais hons pense une chose et Deus repense tout autre ; car quant il volt tourbillier un home, il li tolt tot avant la veue, c’est a dire son sens et sa bone porveance. Et ce veons nous apertement en cestui empereour ; car po aprés ce k’il fu coronés et ke sainte eglise li avoit fet toz les biens k’ele pooit, et devant k’il fust en aage et puis longhement, il esdreça soi contre sainte eglise et contre ses drois, et fist grant damage et persecutions contre l’apostoile et contre toz clers. Por laquel chose li pape Honorés, celui meismes ki coroné l’avoit, l’escomenia et dona sentence contre lui et assoust tous les barons dou sairement k’il li avoient fet sor la feauté de son empire et de la corone.\par
Et puis ke pape Honoré ot vescu en sa dignité entor \textsc{.xi.} ans et ke sa ame en ala au saint siecle, uns autres en fu esleus en son leu, ce fu li novimes Grigoires ; en l’an Nostre Signeur \textsc{.m.} .iic. et \textsc{.xxvi.}, ki a son tans fist les noveles decretales en \textsc{.i.} livre par les mains frere Raimont son chapelain et son peneanchier, et osta et rabati tous autres decretaus. Et a son tens recommença la guerre et la rebellations de l’empereour, en tel maniere k’il escomenia de rechief, et dona sentence contre lui, et envoia \textsc{.ii.} legas, ce furent \textsc{.ii.} cardinals de Rome, outre les mons, por avoir secours et aide contre Frederik.\par
Endementiers l’apostoiles establi a faire un concile a Rome, et manda ses letres et ses legaz as grans prelaz de sainte eglise, et as princes ki a lui se tenoient, k’il venissent au concille au jour nomé. Et comme li doi cardenal revenoient de France o tot grant compaignie d’archevesques et de evesques et d’abbés et des uns et des autres grans signors, il furent pris sor la mer de Pise parles pisans et par la force l’empereour.\par
Cestui empereor ala a ost entor Rome, et tint le siege longhement, et se porchaça tant as nobles romains par dons et par promesses k’il en avoit la grignor partie a sa volenté.\par
Et quant li apostoiles connut apertement ke Rome ne se pooit longuement deffendre, il prist le chief Saint Piere et St. Pol en ses \textsc{.ii.} mains, et asambla tot le comun de Rome, et ala a procession de Lateran jusk’a St. Piere, et sermona devant aus de maintes choses. Et tant fist que por poi tout li romain se croisierent contre l’emreor. Et quant ce fu chose seue en l’ost, li empereres desloga et s’en ala avec tote sa gent ariere, la u il plus forment pensoit entrer dedens Rome, et sosmetre l’apostole et la terre a sa signorie.\par
Et quant l’apostoile Grigoires ot vescu en sa chaiere \textsc{.xiiii.} ans il morut, et sa ame ala au beneoit leu u est la perpetuel glore, se Deu plaist. Aprés sa mort li cardenal s’acorderent a un viel home sage et de bone vie, ki estoit chardenal et evesque de Savine, et fu apostoles, et ot a non Celestin li quars. Mais il ne veski ke \textsc{.xvii.} jours ; et porce ke li mons estoit tous corrous et triboulés por les oevres Frederik et de ses fiz, por le grant pooir k’il avoient en toutes pars, demoura sainte eglise sans pape et sans pastor \textsc{.xx.} mois ; car entre les prelas de sainte eglise avoit discort et irour, tant k’il ne se pooient acorder. D’autrepart li chardenal, ki estoient li uns cha et li autres la, ne se pooient assembler en \textsc{.i.} leu, por les voies ke Frederis tenoit closes et sierees.\par
Mais si come il plot a Deu Nostre Signeur, tant alerent ambaisseur et legaz et d’une part et d’autre, et tant fu la chose menee par sens et par soutillité, la u totesvoies quidoit l’uns engignier l’autre, que monsigneur Sinebaut de Genes, ki fiz fu au conte de la Vaigne, et estoit chardenaus, uns des plus privés amis que li empereour avoit en toute clergié, fu esleus a pape par le consentement comun de tos, en l’an de grace \textsc{.m.} .iic. et \textsc{.xli.}, et ot a non Innocens li quars. Mais ci se taist ore li contes a parler de lui, tant que sa matire se tourt a ce point.
\chapterclose


\chapteropen
\chapter[{.I.LXXXXVI. De l’empereor et del pape innocent}]{\textsc{.I.LXXXXVI.} De l’empereor et del pape innocent}\phantomsection
\label{tresor\_1-96}

\chaptercont
\noindent En ceste partie dist li contes ke l’empereour se porchaça tant vers les princes d’Alemaigne ke Henris son ainsnés fiz fu esleus rois d’Alemaigne, et devoit estre empereour de Rome aprés la mort son pere. Cestui Henri crut en aage et en sapience, et vit les choses du siecle, et conut bien le pooir de sainte eglise, et apercevoit tot clerement ke le pooir son pere ne pooit longhement durer contre ceus ki le contrarioient. Et il en parloit sovent plus ke ses pere ne volsist par aventure ; si ala tant la chose ke li peres le fist metre en chartre en Puille et le fist morir a male fin. Et il laissa \textsc{.ii.} fiz, ke l’empereour fist garder et norrir avec ses autres fiz ses neveus.\par
Cestui empereour, en cel tans meismes k’il estoit escomeniiés, passa outre mer ; et la u on quidoit k’il aidast as crestiiens et as pelerins, il fist son traitement avec le soudan ; et en some, il ordena plus de mal ke de bien. Et soudainement s’en revint en son regne de cha, et recomença novele tribulation. Ke vous iroie disant ? nus ne poroit deviser de bouce ne metre en escrit les maus et les guerres ki longhement durerent entre l’empereour et sainte eglise, et entre lui et les lombars, ki tenoient la partie de st. eglise.\par
Mais entre tant de maus et des descors, assés fu ki traitoient paroles d’acort et de pais entre lui et l’apostoile Innocent. Et en ce ke l’en traitoit et entendoit a ces choses, li apostoiles fist venir soudainement grant fuison de nés et de vaissiaus de Genes parmi la mer es parties de Rome, et il i entra une nuit sans autre gent, et s’en ala a Genes, et de ki s’en ala a Lion sor le Rosne, c’est a dire outre les mons, ou il ne cremoit l’empire ne l’empereour ne son pooir. Iki assambla tout le general concile et dona perpetuel sentence contre lui et contre ses oirs a tousjors, et l’escomenia et maudist, et les desposa en privet de l’empire et dou regne et de toutes dignités.
\chapterclose


\chapteropen
\chapter[{.I.LXXXXVII. De l’empereour et de mainfroi}]{\textsc{.I.LXXXXVII.} De l’empereour et de mainfroi}\phantomsection
\label{tresor\_1-97}

\chaptercont
\noindent Après çou li empereours, ki fiers et sages estoit, et de grant porveance, pourchaça tant as prinches d’Alemaigne ke Conras son fiz fu esleus a roi et a empereour aprés la mort son pere.\par
Puis ordena li rois Frederis son fiz de porchaç vicaire en Toschane, ki par le comandement son pere faisoit tous les maus k’il pooit as Guelfs et a tous ceaus ki se tenoient a la partie l’apostoile. Et chaça les Guelfs de Florence, le jor de la candeler en l’an de l’incarnation \textsc{.m.} et \textsc{.iic.} et \textsc{.xlvii.}, dont maint maus sont puis avenu, si com li mestres ki cest livre fist puet bien tesmoignier.\par
Tout autresi establi il le roi Henri son fil de loiien vicaire en Lombardie, et tot feist il assés de maus as feus de Lombardie. A la fin ala il a ost sor Boloigne la grasce, u il et ses gens furent desconfis en plains chans de batailles, et il fu mis en chartre dedens Boloigne, u il demora en mal et en povreté entor \textsc{.x.} ans, k’il defina sa vie.\par
L’empereour meismes, aprés ceste desconfiture, assambla grandesime ost en Lombardie, et ferma siege entor la cité de Parme, et iki demoura longhement a grant force et a grant pooir. Mais si come il plot a Deu, un jour avint k’il estoit alés en bois a la chace, si comme il avoit a costume, car c’estoit uns des homes del monde ki plus se delitoit en chiens et en oiseaus et en tous deduis terriens. Li citein de Parme issirent hors a \textsc{.i.} cri. et a une vois, si fierement et si asprement k’il desconfirent l’ost et ardirent et prisent et gaignierent tout quank’il avoit. Donc li empereour s’en ala en Cremone, et rasambla ses gens, et fist assés de choses. Mais a la fin s’en ala il en Puille, ou il ne demora longhement k’il amaladi trop durement en une terre ke on apele Florentin.\par
Et il n’avoit entor lui de sez fiz ke Mainfroi, k’il avoit engendré en une gentil dame ki fu fille au marchis Lance de Lombardie. Et ne cuidiés mie k’ele fust sa feme par mariage, mais il l’ama sor toutes autres, et pour son sen et por sa biauté. Autresi amoit il Mainfroi son fil, car il estoit sages et cler veans, et mout se fia de lui ses peres en sa maladie.\par
Et quant il vit son pere ki si malades estoit, il commença tot belement a prendre les tresors son pere et a tenir la signorie sor les autres. Ke vous diroie ? Il se pensa k’il auroit tout, et pour ce il entra \textsc{.i.} jour en la chambre ou ses peres gisoit malades et prist \textsc{.i.} grant coussin et le mist sor la face son pere, et il se coucha sor le coussin, et le fist morir en tel maniere come vous entendés ; et ce fu le jor sainte Lucie devant Noel, en l’an de grace \textsc{.m.} et iic. et l.\par
En cestui tens rentrerent li Guelfs dedens Florence, dont il estoit chaciés, selonc ce ke li contes devise ci devant. Et Manfroi prist les trezors et le pooir de la terre, et comença a trere les cuers des gens a lui, tant ke li rois Conras ses freres, ki estoit en Alemaigne et ki estoit esleu a empereour, selonc ce ke nous avons dit ça ariere, vint en Puille et prist et ot la signorie de Puille et de Sesille.\par
Mais l’en dist ke Manfroi, ki n’avoit pas changié son cuer ne son propos, fist tant que le roi Conrat ne vesqui longhement, ains morut de venin ; et laissa \textsc{.i.} fiz de sa feme en Alemaigne, ki autresi ot a non Conrat, mais il estoit petis enfes. Lors se fist Manfroi baillis de la terre de par le petit Conradin son nevou, et prist la signorie et la force des viles et des fortereces et des gens du regne. Et les \textsc{.ii.} fiz le roi Henri son frere, de qui li contes parole ça ariere, fist il autresi morir de venim, selonc ce ke li plusor dient.\par
Aprés ce il envoia de ses privés une fois en Alemaigne au petit Conradin, por faire lui envenimer ; mais il fu si gardés ke ce ne pot mie estre. Toutesvoies li messagier revinrent par mer a unes voiles noires, et aporterent noveles ke le petit Conrad estoit mors. Si en fist Mainfrois grant samblant de dolor ; et la u les gens de la terre estoient assamblé por savoir la mort de lor signeur, li ami Manfroi et cil de son conseil distrent ke Manfroi estoit bien dignes d’estre rois de Puille, puis que tot li autre estoient mort. Ke vous iroie disant ? il fu esleu a roi et a signeur par le commun assentement de tous les barone dou roiaume, et tint la signorie grant tens, selonc ce ke li contes dira ça avant, la u il en sera leus et tens.
\chapterclose


\chapteropen
\chapter[{.I.LXXXXVIII. De mainfroi et dou roi qarle}]{\textsc{.I.LXXXXVIII.} De mainfroi et dou roi qarle}\phantomsection
\label{tresor\_1-98}

\chaptercont
\noindent Or dist li contes que quant le pepe Innocent ot desposé par sentence l’empereour Frederik, selonc ce que li contes a dit ça arieres, il porchaça tant que li Lantgrave de Thuringe, \textsc{.i.} haut prince d’Alemaigne, fu esleus d’estre rois d’Alemaigne et empereour de Rome. Mais si comme il plot a Nostre Signor, il morut po aprés. Et puis fu esleus li contes de Hollande Guillaumes, mais il trespassa de cest siecle avant k’il parvenist de sa dignité.\par
Mais li apostoiles aprés la mort Frederik s’en vint en Pulle et assambla grant ost contre Manfroi, por conquerre la terre ki devoit estre de sainte eglise. Toutesfoies Manfrois deffendi bien la terre, et li pape ne veski gaires, ains morut a Naples en l’an de grace \textsc{.m.} et iic. et liii.\par
Aprés sa mort fu esleus apostoile Alixandres li quars, et a son tens se fist Manfroi coroner en Puille, selonc ce que nous avons dit ça ariere. Et por ce ke son couronement estoit contre les drois ke sainte eglise devoit avoir ou regne, fu il tot avant escomeniiés et desposés par sentence. Et puis envoia li apostoiles grand effort contre lui, mais il n’i gaignierent rien. Encor au tens cestui apostoile vint une divisions entre les princes d’Alemaigne ; car li un esleurent a roi et a empereour monsigneur Alfons, roi de Chastele et d’Espaigne, li autre esleurent le conte Richart de Cornuaille, frere au roi d’Engleterre.\par
Et quant il pleut a Nostre Signor, li apostoiles trespassa de cest siecle, et uns françois de la cité de Troies fu fet apostoles, et ot a non Urbains li quars ; et ce fu en l’an de grasce mil et iic. et lxi.\par
Et quant cesti apostoiles fu en si haute chaiere comme d’estre vicaires Jhesucrist en terre, il pensa ke Mainfroi, par sa tyrannie, avoit occupé le regne de Sesille et de Puille, ki a sainte eglise apertienent par droit, et k’il avoit les prelas et les eglises mises en servage, et k’il avoit envoié sus le patrimoine saint Piere l’ost des sarrasins, et ke l’annee devant k’il fust apostoiles les gens Mainfroi entrerent en Toschane et chacierent les Guelfs de Florence hors de la vile et du païs ; et pensa bien en son cuer, et li preudome le tesmoignierent, ke Mainfroi aroit et penroit bien Ytaile tote, se ne fust ki le contrariast. Par ceste chose establi il ke Karles quens de Provence et freres le roi de France fust rois de Sezille et de Puille, et k’il traisist la terre des mains Manfroi.\par
En cel tens aparut l’estoile comee ou firmament, ki espandoi environ cors luisans, et dura bien \textsc{.iii.} mois. De cele estoile dient li sage astronomien que quant ele apert, ele segnefie remuemens de regnes u mort de grant signor. Et ja soit ce par aventure k’ele segnefiast assés de choses es autres parties dou monde, totesvoies savons nous bien que la nuit proprement k’ele desaparut et k’ele s’en ala, cela nuit proprement morut li apostoiles Urbains, dont il fu grans damages ; mais il i a maint de gent ki dient k’ele segnefia la mort Manfroi et la victore ke Karles ot de lui.\par
Car aprés la mort Urbain fu esleu Clemens li quars, en l’an de grace \textsc{.m.} et iic. et lxiiii. En l’an aprés, Karles vint par mer droit a Rome, dont il estoit signator, et vinrent ses gens par terre et passerent Lombardie et les autres païs, la u Karles les atendoit, et avec lui s’en alerent en Puille et se combatirent a Mainfroi et a son ost. Et ja soit ce ke la bataille fust grans et perilleuse, toutesvoies li champion Jhesucrist orent la victore et le regne et la corone et la terre, et Manfrois i perdi la vie et le regne tot a un cop en l’an de Nostre Signor mil et iic. et lxv.\par
Ensi ot li rois Karles la victoire de ses enemis, et fu rois et sires de la terre par la volenté de sainte eglise. Mais il ne demora mie longhement ke li petis Conras, niés l’empereour Frederik, de qui li contes a longhement parlé ci desoure, vint d’Alemaigne o tout grant oster de tyois et de lombars et de toschans ki avoient esté de la partie son aioul, et parvint a Rome ou il fu honorablement receus. Et de ki s’en ala en Pulle, et li rois Karles li fu a l’encontre prés d’une vile ki est apelee Taillecouz.\par
Et puis ke li dui ost furent assamblé, il ne fet a dire se la bataille fu grant et perilleuse, ne se il i ot chevaliers d’une part et d’autre, ki fierement se combatissent ; car il n’a plus aspre gent ou siecle ke alemans et françois. Mais sans faille en l’ost Conrat avoit assés plus de gens ke en celui dou roi Karle ; et neporquant il avoit entor lui teus deus chevaliers françois ke l’en ne quide ke en. tot le monde soient \textsc{.ii.} millours, c’est monsignor Erart de Valeri et monsignor Johan Britaut : cil doi sostindrent tot le fais de la bataille. Il faisoient çou ke cuers d’omme ne deust croire, ke vous diroie jou ? tous les cols et toutes les assamblees. C’est la some et la fins de la mellee, ke l’ost Conrat perdi del tout et ala a desconfiture, et Conrat meismes et li dus d’Osterriche et maint autre grant signor furent pris, et lor furent les testes copees. Ensi defina li linages a l’empereor Frederik, en tel maniere ke de lui ne de ses fiz n’est demoret en terre nule semence. Mais de ce se taist ore li mestres, et torne a sa matire, dont il s’est mout esloigniés.
\chapterclose


\chapteropen
\chapter[{.I.LXXXXVIIII. Comment nature de totes choses fu establie par .iiii. complexions}]{\textsc{.I.LXXXXVIIII.} Comment nature de totes choses fu establie par \textsc{.iiii.} complexions}\phantomsection
\label{tresor\_1-99}

\chaptercont
\noindent Ci endroit dist li contes ke la principale matire est a traitier en ces livres de la nature des choses du monde, laquele est establie par \textsc{.iiii.} complexions, c’est de chaut, de froit, de sech, de moiste, dont toutes choses sont complexionees.\par
Neis li \textsc{.iiii.} eliment, ki sont ausi come soustenement dou monde, sont enformees de celles \textsc{.iiii.} complexions. Car li fus est chaus et sés, l’euue est froide et moiste, li airs est chaus et moistes, la terre est froide et seche. Autresi sont complexioné li cors des homes et des bestes et de tous autres animaus, car en aus a \textsc{.iiii.} humors, colera, ki est chaude et seche, fleuma, ki est froide et moiste, melancolie ki est froide et seche, sanc ki est chaus et moistes. L’annee meisme ki est devisee en \textsc{.iiii.} tens ki sont ausi complexioné ; car printans est chaus et moistes, estés est chaus et sés, aupton est froit et sech, yvier est froit et moiste. Ensi poés vous connoistre ke li fus et la colre et li estés sont d’une complexion, et l’euue et le fleugme et li yvier sont d’une autre, mais li airs et li sans et li printans sont atempré de l’un et de l’autre. Et por ce sont il de millour complexion ke ne sont tout li autre ; et lor contraire sont la terre et melancolie et auptonne, et por ce ont il tres mauvaise nature.\par
Or est il legiere chose a entendre coment l’office de Nature est en acorder ces choses descordans et enywer les desigueles, en tel maniere ke toutes diversités retornent en unité, et les ajoste et les assamble en \textsc{.i.} cors et en une substance ou en autre chose ki le fait naistre au monde tousjors, ou de plantes, ou de semences, ou par conjungemens de malles et de femelles ; dont li un engendrent oes, ki sont replains de creatures, l’autre engendrent figures encharnees, selonc ce ke li contes devisera ça avant la u il en sera et leus et tans.\par
Par ces paroles apert il ke Nature est a Deu autresi com li martel au fevre, ki ore forge une espee, ore un elme, or fait \textsc{.i.} clo, or une aguille, ore une chose ore un autre, selonc ce ke li fevres wet. Et tout autresi comme il est une maniere de forgier elme et une autre de forgier une aguille, tot autresi oevre Nature es estoiles autrement ke es plantes, et autrement es homes et es bestes et es autres animaus.
\chapterclose


\chapteropen
\chapter[{.I.C. Comment totes choses furent faites et du mellement de complexion}]{\textsc{.I.C.} Comment totes choses furent faites et du mellement de complexion}\phantomsection
\label{tresor\_1-100}

\chaptercont
\noindent Et il fu voirs ke Nostre Sires au comencement fist une grosse matire sans forme et sans figure ; mais ele estoit de tel maniere k’il en pooit former et faire ce k’il voloit. Et sans faille de ce fist il les autres choses ; et por ce qu’ele fu faite de neant devance ele les autres choses, non mie de tens ne de eternité mais de naissance, autresi come li sons devant le chant, car Nostre Sires fist toutes choses ensamble. Raison coment : quant il cria cele grosse matire dont ces autres choses furent estraites, donc fist il toutes choses ensamble ; mais selonc la distinction et le devisement de chascune par soi, le fist il en \textsc{.vi.} jors, selonc ce que li contes devise ça ariere et la meisme dist il que cele grosse matire est apelee ylem.\par
Et por ce ke li \textsc{.iiii.} eliment ke l’en puet veoir sont estrait de cele matire, et sont apelés eliment por le non de lui, c’est por ylem, ensi s’entremellent ces elimens es criatures : car li doi sont legier et isnel, c’est l’air et le fu, li autre doi sont grief et pesant, c’est terre et euue. Et chascuns d’aus a \textsc{.ii.} estremités et \textsc{.i.} moien. Raison coment : li feus a une estremité desus ke tousjors vet amont, et cele est la plus delie et la plus isnele et legiere. Et l’autre estremité est desous, ki est mains legiere et mains delie ke l’autre. Le moien est entre deus, ki est atemprés de l’un et de l’autre.\par
Tot autresi est des autres \textsc{.iii.} elimens et des \textsc{.iiii.} complexions : ces choses s’entremellent es cors et es autres criatures car en ce ke le pesant se conjunt au legier, et le chaut avec le froit et le sech avec le moiste, en aucune creature, il covient ke la force de l’un sormonte les autres : je ne di pas des estoiles, car eles sont dou tout de nature de fu ; mais es autres creatures ou li eliment et les autres complexions sont entremellees avient il ke les estremités desoure sormontent les autres en aucune criature.\par
Et lors covient il que celes criatures soient plus legieres et plus isneles, et por ce vont il par l’air, ce sont oiseaus ; mais il i a difference, car tout autresi comme li oisel sormontent toutes autres criatures de legierté et d’isneleté, por les estremités des elimens desoure ki habondent en iaus, autresi l’un oisel sormonte l’autre por ce ke la estremité legiere et isnele habonde plus en lui, et por ce vole celi oisel plus haut ke li autre, c’est li aigle. Et cil en qui en abonde le mains ne volent pas si en haut, c’est la grue. Et cil en qui habonde le estremité desous sont plus grief et plus pesant, c’est l’oove et l’ane. Autresi devés vous entendre en tous autres animaus et poissons et arbres et plantes, selonc le devisement des oisiaus.
\chapterclose


\chapteropen
\chapter[{.I.CI. Des .iiii. complexions de l’ome et des autres choses}]{\textsc{.I.CI.} Des \textsc{.iiii.} complexions de l’ome et des autres choses}\phantomsection
\label{tresor\_1-101}

\chaptercont
\noindent Autresi avient il des \textsc{.iiii.} complexions quant eles s’entremellent en aucune criature, car chascuns ensiut la nature de son eliment. Et por çou covient il ke a l’entremeller des humors ke l’un sormonte l’autre, et que sa nature i soit plus forte de grignor pooir. Por ce avient il que une herbe est plus froide ou plus chaude que l’autre, et que l’une des natures est de complexion sanguine, l’autre de melancolique ou de flegme ou de colre, selonc ce ke humours habondent plus, pour ce sont li fruit les erbes les blés et les semences, l’une plus melancolieuse que l’autre, ou plus colerique, ou d’autre complexion ; autresi di je des homes et des bestes ou des oiseaus et des poissons.\par
Dont il avient que unes choses sont bonnes a mengier et les autres non, et que les unes sont douces et les autres ameres, les unes verdes ou rouges et les autres blanches ou noires, selonc le colour des elimens u des humours ki i sormontent ; les unes sont venimeuses, les autres valent en medecines. Car ja soit ce que en chascune chose soient mellees tot li \textsc{.iiii.} eliment et les \textsc{.iiii.} complexions et les \textsc{.iiii.} qualités, il covient que la force des uns i soit plus forte, selonc ce que plus i abonde.\par
Et par cele nature ki plus i abonde est tous apelés de cele nature. Raison coment : se flegme abonde plus en \textsc{.i.} home, il est apelés flegmatiques, pour la force k’ele a en sa nature. Car en ce ke flegme est froide et moiste, et est de nature d’ewe et d’ivier, covient il que cil hom soit lens et mols et pesans et froideilleus et dormilleus et non mie bien sovenans des choses alees, et c’est la complexion ki plus apertient as vieillars. Ele a son siege au pomon, et est purgie par la bouche. Ele croist en yvier, por ce que ce est de sa nature ; por ce sont en celui tans deshaitié li flegmatique vieus, mais li colerique sont haitiés, et li joene ausi. Et les maladies ki sont par ochoison de flegme sont tres malvaises en yvier, si come est cotidiane ; mais celes ki sont par colre sont mains males, si come est tiercaine ; pour ce est il bien que flegmatique usent en yvier choses chaudes et seches.\par
Sanc est chaut et moistes, et a son siege el foie. Et croist en printans, porce sont lors tres malvaises maladies de sanc et de synoche ; et en cel tans sont mieus haitiés li vieus que li joenes, por ce doivent il user choses seches et froides. Et li hom en qui ceste complexions habonde est apelés sanguins, c’est la millour complexions ki soit, dont il avient home gras, chantant, liés, hardis, et benigne.\par
Colre est chaude et seche, et a son siege el fiel, et est purgie par les oreilles. C’est complexions de nature de fu et d’esté et de chaude joenece, et por ce fait ele home ireus, engigneus, agut, fier, legier movant. Et si croist en esté, pour ce sont lors li colerique mains hetiés que li flegmatique, et mains li joene que li vieus ; porce doivent il user choses froides et moistes. Quant les maladies vienent par colre, si sont perilleuses en esté, plus ke celes ki sont par fleume\par
Melancolie est une humours que li plusour apelent colre noire, et est froide et seche, et a son siege en le splen, et est de nature de terre et de auptonne ; pour ce fet les homes melancolieus plains d’ire et de maintes malvaises pensees, et paourous, et ki ne puet bien dormir. Aucunefois est purgie par les oils, et croist en auptonne ; por ce sont en celui tens plus haitié li sanguin que li melancolieus, et plus li garchon que li vieus. Et lors sont plus greveuses maladies celes ki sont par melancolie ke celes ki sont par sanc, por ce fait il bon user choses chaudes et moistes.
\chapterclose


\chapteropen
\chapter[{.I.CII. Des .iiii. vertus qi soustienent les animaus en vie}]{\textsc{.I.CII.} Des \textsc{.iiii.} vertus qi soustienent les animaus en vie}\phantomsection
\label{tresor\_1-102}

\chaptercont
\noindent Et sachés ke en chascun cors ki a les soufissans membres sont \textsc{.iiii.} vertus establies et enformees par les \textsc{.iiii.} elimens et par leur nature, ce sont apetitive, retentive, digestive, et expulsive.\par
Car quant li \textsc{.iiii.} eliment sont ajosté et assamblé en aucuns cors, acomplis de drois membres, li fus por ce k’il est chaus et sés fet la vertu appetitive, c’est ki done talent de mengier et de boivre. Et la terre ki est froide et seche fait la vertu retentive, c’est k’il retient la viande. Et li airs ki est chaus et moistes fet la vertu digestive, c’est ki fait quire et moistir la viande. L’euue est froide et moiste et fet la vertu expulsive, c’est k’ele chace hors la viande quant ele est quite.\par
Ces \textsc{.iiii.} vertus servent a cele vertu ki norrist et paist le cors. Et la vertu dou nourissement sert a la vertu ki engendre, par qui li uns engendre l’autre selonc lor nature et lor samblance.\par
Et si comme li atemprement ki acorde la diversité des elimens fait le cors engendrer et naistre et vivre, tot autresi li desatemprement d’iaus les corront et le fait devier. Car se li cors fust d’un eliment sans plus, il ne poroit desatemprer jamés, por ce k’il n’auroit contraires, et ensi ne morroit il.\par
Mais ci se taist li contes de la nature as animaus et retornera a sa droite vote, car il doit dire premierement des choses ki primes furent faites, et por ce tornera il a dire dou monde et dou firmament, et dou ciel et de la terre.
\chapterclose


\chapteropen
\chapter[{.I.CIII. De l’eliment orbis}]{\textsc{.I.CIII.} De l’eliment orbis}\phantomsection
\label{tresor\_1-103}

\chaptercont
\noindent Li contes a devisé ça arieres de la nature des \textsc{.iiii.} elemens, c’est dou fu, de l’air, de l’euue, de la terre. Mais Aristotles li grans philosophes dit k’il est un autre eliment hors de ces \textsc{.iiii.}, ki n’a point de nature ne de complexion as autres, ançois est si nobles k’il ne puet pas estre esmeus ne corrompus si comme sont li \textsc{.iiii.} eliment.\par
Et por ce dist il meismes que se nature eust formé son cors de celui eliment, k’il se tenroit asseur de la mort, por ce k’il ne poroit morir en nule maniere. Cist elimens est apelés orbis ; c’est un ciel reont ki environne et enclot dedens soi tos les autres elimens et ces autres choses ki sont hors de la divinité. Et est autresi au monde comme l’eschaille de l’oef, ki enclot et ensierre ce ki est dedens. Et por ce k’il est tous reons, covient il a fines force que la terre et la forme dou mont soient reont.
\chapterclose


\chapteropen
\chapter[{.I.CIIII. Comment li mondes est reont et comment li .iiii. eliment sont establi}]{\textsc{.I.CIIII.} Comment li mondes est reont et comment li \textsc{.iiii.} eliment sont establi}\phantomsection
\label{tresor\_1-104}

\chaptercont
\noindent En ce fu nature bien porveant quant ele fist l’orbe tout reont, car nule chose puet estre si fermement serree en soi meismes comme cele ki est reonde. Raison porquoi et coment : garde ces carpentiers ki font ces tonniaus et cuves, k’il ne les poroient en autre maniere former ne joindre se par reondece non. Neis une voute, quant l’en le fait en une maison u un pont, covient k’il soit fermé par son reont, non mie par lonc ne par lé ne en nule autre forme.\par
D’autre part il n’est nule autre forme ki puisse tant de chose tenir ne porprendre con cele ki est reonde. Raison coment : il ne sera ja nus si soutils mestres ki de tant de merien seust faire un vaissel lonc u quarré ou d’autre forme ou l’en peust metre tant de vin d’assés comme en un tonel reont. D’autre part il n’est nule autre figure ki soit si atorné a movoir et atornoier comme la reonde.\par
Et il covient ke le ciel et le firmament se tornent et muevent tozjors ; et s’il ne fust reont, quant il se tornoie il covendroit a fine force k’il revenist a autre point ke au premier dont il estoit esmeus. D’autre part covient il a fine force que li orbis soit toz plains dedens sol, si ke l’une chose sousteigne l’autre, car sans soustenement ne poroient il estre mie. Et se ce fust que li mondes eust forme longue u quarree, il ne poroit iestre toz plains, ains li covendroit estre vuides en aucune part, et ce ne puet pas estre.\par
Par ces et par maintes autres raisons covient ausi come par necessité que li orbis eust forme et figure toute reonde, et que totes choses ki sont recloses dedens lui i fuissent mises et establies reondement, en tel maniere ke l’une environne l’autre et l’enclot dedens soi si igaument, et si a droit, k’ele n’i atouche plus d’une part ke d’autre.\par
Por ce poés vous entendre ke la terre est tote reonde, et autresi sont li autre eliment ki s’entretienent en ceste maniere ; car quant une chose est enclose et environnee dedens \textsc{.i.} autre, il covient que cele ki enclot tiegne celi ki est enclose, et covient ke cele ki enclose est soustiegne celi ki l’enclot. Raison coment : se le blanc d’un oef ki avironne le moieul ne le tenist enclos dedens soi, il cherroit sus son eschaille ; et se li moieul ne soustenist le blanc, certes il cherroit el fons de l’oef.\par
Et por ce covient il en toutes choses que celi ki est plus dure et plus grief soit tozjors el mileu des autres, por ce ke de tant comme ele est plus dure et de plus soude substance, de tant puet ele mieus soustenir les autres ki sont environ li. Et de tant comme ele est plus grief et plus pesant, covient k’ele se tire el mileu et el parfont des autres ke entor lui sont, c’est en tel leu k’ele ne puisse avaler ne monter ne aler ça u la. Et c’est la raison por quoi la terre, ki est li plus grief eliment et plus soude substance, est assise el mileu de toz cercles et de tot avironement, c’est el fons des cieus et des elimens. Et por ce ke l’euue est, aprés la terre, li plus grief elimens, est il assis sus la terre, ou ele se soustient. Mais li airs avironne et enclot l’euue et la terre tot entor en tel maniere que la terre et li euue n’ont pooir k’eles se remuent du leu u nature les a establi.\par
Environ cesti air ki enclot la terre et l’euue est assis le quart eliment, c’est le feu, ki est sor tous les autres. Or poés vous entendre que la terre est au plus bas leu de tous les elimens, c’est ou milieu dou firmament et dou quint eliment, ki est apelés orbis, ki enclot toutes choses.\par
Et a la verité dire, la terre est ausi come li point dou compas, ki tousjors est ou mileu de son cercle, si k’il ne s’esloigne plus d’une part ke d’autre. Et pour çou est il necessaire chose ke la terre soit toute reonde ; car s’ele fuste d’autre forme, ja seroit ele plus prés du ciel et du firmament en \textsc{.i.} leu k’en \textsc{.i.} autre.\par
Et ce ne puet estre. Car s’il fust chose possible que l’en peust chevillier la terre et fere \textsc{.i.} puis ki alast d’outre en outre, et par cest puis getast on une grandesime piere ou autre chose pesant, je diroie que cele piere ne s’en iroit pas outre, ains se tenroit tousjours el mileu de la terre, c’est sor le point dou compas de la terre. Si k’ele n’iroit ne avant ne ariere, por ce ke li airs ki avironne la terre enterroit par les pertruis d’une part et d’autre, et ne souferroit pas k’ele alast outre le mileu ne k’ele revenist arieres ; se ce ne fust \textsc{.i.} poi par la force du cheoir, et maintenant revenroit a son mileu autresi come une piere quant ele est getee en l’air contremont.\par
Et d’autre part toutes choses se traient et vont tozjors au plus bas. Et la plus basse chose et la plus parfonde ki soit au monde est li poins de la terre, c’est le mileu dedens, ki est apelés abisme, la u infier est assis. Et tant come la chose est plus pesans, tant se tire plus vers abisme. Et por ce avient il, ki plus cheville la terre en parfont tousjours la trueve plus grief et plus pesant.\par
Encore i a autre raison par quoi il apert que la terre est reonde : ke s’il n’eust sus la face de la terre nul enpechement, si ke on peust aler partot, certes il iroit tot droitement environ la terre tant k’il revenroit au leu meisme dont il estoit esmeus. Et se \textsc{.ii.} homes d’un leu et a un jor alaissent li uns vers soleil levant et li autres vers soleil couchant, certes il s’entreconterroient en celui leu ki fu d’autre part la terre tot droit encontre le leu dont il seroient meu.
\chapterclose


\chapteropen
\chapter[{.I.CV. De la nature de l’euue}]{\textsc{.I.CV.} De la nature de l’euue}\phantomsection
\label{tresor\_1-105}

\chaptercont
\noindent Sour la terre, de qui li contes a tenu long parlement, est assise l’euue, c’est la mer grignor, ki est apelee la mer ocheaine, de qui toutes les autres mers et bras de mer et fleuves et fontaines ki sont parmi la terre issent et naissent premierement, et la meisme retornent a la fin. Raison coment : la terre est toute petuisie dedens et plaine de vaines et de cavernes, par quoi les euues ki de la mer issent vont et vienent parmi la terre, et dedens et dehors sordent, selonc ce ke les vaines l’amainent ça et la ; autresi come le sanc de l’ome, ki s’espant par ces vaines, si k’ele encherche tot le cors amont et aval.\par
Et il est voirs que la mers est sus la terre, selonc ke ce li contes devise ça en ariere el chapitle des elimens. Et se ce est voirs k’ele siet sur la terre, donc est ele plus haute ke la terre ; donc n’est il mie merveille des fontaines ki sordent sor les hautes montaignes, car c’est la propre nature des euues k’ele monte tant comme elle avale.\par
Et sachiés que l’euue mue savour, colour, et qualité, selonc la nature de la terre ou ele court. Car la terre n’est mie d’une maniere, ançois est de diverses coulors et de diverses complexions. Car en \textsc{.i.} leu est ele douce, et en un autre amere ou salee, ou en \textsc{.i.} leu est ele blanche. et en \textsc{.i.} autre noire, ou rouge ou bloie ou d’autre colour. Et en \textsc{.i.} leu sont vaines de soufre, et en \textsc{.i.} autre d’or ou d’autre metal. Une terre est mole, une autre est dure, et ensi sont les vaines vaires et diverses par ou les euues courent. Et selonc la nature de la voie covient ke les euues remuent lor qualités et k’eles deviegnent de la savour et de la nature de la terre en quoi ele converse.\par
D’autre part a il en aucune partie de la terre aucune cavernes porries, ou por sa nature, ou por aucunes males bestes ki i repairent ; et por ce est aucune fois ke l’euue est malvaise et venimeuse ki court entre les vaines. Et ces cavernes par ou les euues vienent, covient il por le debatement des euues ke vens si esmueve. Et quant il se fiert es vaines soufrees, li soufres s’eschaufe et esprent de si grant chalour ke l’euue ki cort par iceles vaines devient si chaude com fu, et de ce sunt li chaut baing que on trueve en plusors terres.\par
Et quant celui vent boute l’air ki est enclos parmi ces cavernes et le debat a la terre, il covient a fine force, se cele terre est foible, par la force de celui debatement rompe et dequasse, si ke li airs s’en isse hors. Et lors covient il que la terre chee et fonde o tous les murs et les edifices ki sont sor li. Mais s’ele est si forte et grosse k’ele ne fent, lors covient il a fine force de celui deboutement de l’air et des vens ki sont a destroit la dedens face croller et muer toute la terre environ.
\chapterclose


\chapteropen
\chapter[{.I.CVI. De l’air et de la pluie et des autres choses qui sont en l’air}]{\textsc{.I.CVI.} De l’air et de la pluie et des autres choses qui sont en l’air}\phantomsection
\label{tresor\_1-106}

\chaptercont
\noindent Li contes dist ça ariere que li air environne la terre et l’euue et les enclot et sostient dedens soi. Neis li home et les autres animals vivent en l’air, ou il respirent ens, et font autresi come les poissons en l’euue. Et ce ne poroit il mie fere s’il ne fust moistes et espés. Et se aucuns disoit ke li airs ne fust espés, je diroie ke s’il movoit roidement une verge en l’air, ele sonnera et pliera maintenant por l’espois de l’air k’ele encontre.\par
Li airs meismes soustient les oiseaus par son espés. En cestui air naissent les nues et les pluies et li espar et li tonnoires et autres choses samblables, et orés raison coment. Li contes dit ça en arieres ke li airs environne la terre et l’euue, et les enclot et sostient dedens soi, et les homes et les autres animaus, et ke la terre est coverte et replaine de diverses euues.\par
Et quant li chaut du soleil ki est chief et fondement de toutes chalours, il se fiert en la moistour de la terre ou des choses baignies, il les essue et en oste les humours autresi come se ce fust \textsc{.i.} drap moilliet. Et lors s’en issent fors unes vapours comme fumees, et s’en vont en l’air amont, ou eles s’acueillent poi a poi et angroissent tant k’eles devienent oscures et espesses, si k’eles nos toilent la veue dou soleil, et ce sont les nues. Mais eles n’ont pas si grant oscurté k’eles nous toilent la clarté du jour, car li soleil reluist deseure autresi come fait une candeille dedens une lanterne, ki alume dehors, et si ne le puet veoir.\par
Et quant la nue est bien creue et noire et moiste, k’ele ne puet plus soufrir l’abondance des euues ki i sont vapourees, il les estuet cheoir sour la terre, et c’est la pluie. Lors retrence la moistour de la nue, ki maintenant devient blanche et legiere, et le soleil espant ses rais parmi ces nues, et fet de son resplendissement \textsc{.i.} arc de \textsc{.iiii.} colours diverses, car chascuns eliment i met de sa colour. Et ce seut avenir quant la lune est plaine. Et quant la nue est auques esmeue ou legiere, ele monte en haut tant que la chalours du soleil la confont et la gaste, en tel maniere que li hom voit l’air cler et pur et de bele colour.\par
Et sachiés que li airs ki sor nous est en haut est plus frois tousjors ke celi ki est en bas. Raison coment : tant comme la chose est plus grosse et de plus espesse nature, de tant s’i prent le feu plus fort. Et por ce ke li air ki est en bas est plus gros et plus espés ke celui ki est en haut, la chalours dou soleil se prent mains en haut k’en bas. D’autre part li vent muent et fierent sovent en bas air plus k’en haut, et totes choses ki demeurent coies sont plus froides ke celes ki sont en movement.\par
D’autre part en yvier, quant le soleil eslongue desor nous, por ce est li airs amont assés plus froit que devant. Et por ce avient il souvent que la moistour, avant k’ele soit engroissie en goutes, vient en celui air froit, et engiele et chiet aval tot engelé, et c’est noif, ki onques ne chiet en haute mer. Mais en esté quant li solaus revient et aproche de l’air froit, s’il trueve aucune vapour engelee, il les ensierre et endurcist, et en fait grelle mout grosses, et les enchace par sa chalour jusques a tiere. Mais au cheoir k’eles font, par l’espois de l’air amenuisent eles et devienent petites, et sovent aneantissent avant k’eles viegnent sur la terre.\par
Or avient par maintes fois ke li vent s’encontrent desus les nues, et s’entrefierent et boutent si fort en lor venir ke fus en naist en l’air. Et lors se cis feus trueve la amont ces vapors montees et engroissies, il les enflame et les fait ardoir, et c’est la foudre. Mais li fors deboutemens des vens les destraint et chace si roidement k’ele fent et passe les nues, et fait toner et espartir, et chiet aval de cel air, por les grans vens ki l’encauchent, que nule rien n’a contre lui duree. Et bien sachiés vraiement, que quant ele se muet a venir, ele est si grans ke c’est merveille ; mais ele amenuise en son venir, por le deboutement de l’air et des nues. Et maintes fois avient, quant ele n’est a prime molt grant ne trop dure, et ke les nues sont bien grosses et moistes et chargies d’euwe, ke la foudre n’a pooir k’ele s’en passe, ains s’estaint en la nue et pert son feu.\par
Et quant li vent ki s’entrecombatent si mervilleusement entrent dedens les nues et sont enclos dedens lor cours, il les esmuevent et font ferir l’un contre l’autre ; et por çou ke lor nature ne suefre pas k’il soient enclos, les rompent a fine force et lors font il le tonoire. Et il est de nature de toutes choses ki se puent ferir et bouter ensamble, ke feu en puet naistre. Et quant cel fort encontrement est des nues et des vens et li despuetemens dou tonnoire, nature en fet nestre fu, ki gete grandesime clarté, selonc ce ke vous veés sovent quant li espars gete sa lumiere ; c’est la propre ochoisons por quoi sont li espar et li tonoire. Et se aucuns me demandoit por quoi on voit plus tost l’espars ke le tonnoire, je diroie, por ce ke le veoir est plus prest ke l’oïr.\par
Tot autresi avient il sovent ke aucune vapours seche, quant ele est montee tant qu’ele s’esprent pour le chalour ki est amont, et maintenant k’ele est esprise ele avare vers la terre tant k’ele estaint et amortist ; dont.aucune gent dient que c’est le dragon, ou ke c’est une estoile ki chet.\par
Et sachiés k’en l’air sont environ la terre \textsc{.iiii.} vent principal es \textsc{.iiii.} parties du monde ; et chascun vent a sa nature et son office de quoi il oevrent, selonc ce ke li marenier le sevent ki l’esporvoient de jor et de nuit. Mais des nons et de la diversité des vens ne dira ore plus li mestres, por ce ke les gens du monde changent et devisent les nons selonc les usages et selonc la diversité des langages. Et d’autre part l’en trueve maintenant que uns meismes vens fait en \textsc{.i.} leu pluie en \textsc{.i.} autre non, selonc ce ke li vens vient devers la parfonde mer plus prés a cele terre. Et nanporquant on dist communalment ke celui ki vient de droit levant, et celui ki li vient de droit occident a l’encontre, ne sont pas de grant peril, por ce ke lor venue fiert plus a la terre que a la mer. Mais celi ki vient de droite tramontaine et celi ki vient de droit midi sont de trés fiers perilz, car li cors s de l’un et de l’autre fiert a la mer trop durement.\par
Et ce sont li \textsc{.iiii.} vent principal du monde, et chascuns d’aus en a \textsc{.ii.} autres entor li ki sont autresi comme bastars ; car vens de levant ki est atemprés, selonc ce que li contes devise ci devant, a devers tramontaine \textsc{.i.} vent ki seche toutes choses et est apelés Vulturne, mais li marenier le nomment Grec por ce k’il vient devers Grece. D’autre part devers midi en a \textsc{.i.} autre ki engendre nues, et a non Eore, mais li marenier en dient Siloc, mais je ne sai la raison por quoi il l’apelent ensi.\par
L’autre principal vent de midi est chaut et moistes, et sovent fait foudres et tempestes, et a de chascune part environ li vent chaut, ki tout font sovent et menu tempestes et crol de tiere.\par
L’autre principal ki vient de couchant oste et enchace froit et yver et amaine flors et fuelles et printans. Et devers midi a un vent de la nature as autres de midi, et a non Affriques ; et par cel non l’apelent li marenier aucune fois, mais par \textsc{.ii.} autres nons l’apelent li marenier encore ; car quant il est dous et souef si l’apelent Garbin, por ce ke celui païs que l’Escripture dist Affrike, l’en dit en vulgar parleure le Garb ; mais quant il vient de grant ravine et o fortune, les nagans en dient Libex. Mais devers tramontaine en a il \textsc{.i.} autre plus debonaire ki a non Chorus, cist apelent li marenier Mestre, por \textsc{.vii.} estoiles ki sont en celi meisme leu.\par
Li autre principaus ki vient de tramontaine done nues et froidures, et cil ki li est encoste vers soleil couchant done noif et grelle et a non Circé ; mais li autres ki vient devers levant restraint pluies et nuees. Ce puet on connoistre bien et briement, ke tot vent ki vienent d’orient devers midi, jusques en occident, donent tempeste ou pluies et teus choses samblables, selonc leu et tens.\par
Et li autre ki sont d’orient vers tramontaigne, jusques vers occident, font le contraire des autres, ja soit ce ke la nature de chascun puet changier selonc divers païs. Mais coment k’il soit, je di ke vens n’est autre chose ke deboutement de l’air. Mais fors de ce ki sont només maintenant, en sont \textsc{.ii.} autres de foible movement de l’air, dont li \textsc{.i.} est apelés en tiere Oria, et li autres Altain.
\chapterclose


\chapteropen
\chapter[{.I.CVII. Dou feu}]{\textsc{.I.CVII.} Dou feu}\phantomsection
\label{tresor\_1-107}

\chaptercont
\noindent Après l’avironement de l’air est assis le quart element, c’est le feu ; c’est un air de feu sans nule moistor, ki s’estent jusk’a la lune et avirone cestui air ou nous somes.\par
Et sachés ke deseure le feu est la lune premierement et les autres estoiles, ki toutes sont de nature de fu. Et li fus ki siet dessus les autres elemens ne touche pas as autres elimens : c’est orbis. Car deseure le fu est \textsc{.i.} air pur et cler et net, ou sont les \textsc{.vii.} planetes ; et encore desus cel air est li firmamens, ki tozjors tornoie et environne le monde o totes les estoiles, des orient juskes en occident, si con li contes devisera ça avant, la u il en sera lieus et tens.\par
Et sachiés que desus le f�rmament est uns cieus mout biaus et luisans, de coulor de cristal, et por ce est il apelés ciel cristalim ; c’est li leus dont li malvés angele cheirent. Encore i a desus celui \textsc{.i.} autre ciel de coulor de porpre ki est apelés ciel empire, ou maint la sainte glorieuse Divinités avec toz ses angeles et ses secrés, de qui li mestres ne s’entremet en cel livre, ains les laisse as mestres divins et as signors de sainte eglise a qui il apertient ; si tornera a son conte, c’est au devisement dou monde.
\chapterclose


\chapteropen
\chapter[{.I.CVIII. Des .viii. cercles}]{\textsc{.I.CVIII.} Des \textsc{.viii.} cercles}\phantomsection
\label{tresor\_1-108}

\chaptercont
\noindent Et li contes devise ça en arieres ke sor les \textsc{.iiii.} elimens est uns airs purs et clers sans nule oscurté, ki environne le fu et les autres \textsc{.iii.} elimens dedens soi, et s’estent jusques au firmament. En cesti pur air sont assisses les \textsc{.vii.} planetes, li \textsc{.i.} sor l’autre ; dont li premiers ki est plus prochain a la terre sor le fu est la lune. Desus la lune est Mercures, et puis Venus, et puis le soleil, et puis Mars, et puis Jupiter ter, et puis est Saturnus, ki maint sor tous les autres planetes emprés du firmamant.\par
Et sachés que chascune pianete a son cercle dedens le pur air, par quoi il vait et fait son cours environ la terre, li un plus haut et li autre plus bas, selonc ce k’il sont assis l’un cercle dedens l’autre. Et li contes dist apertement ça en ariere que li mondes est tous reons et compassés diligemment.\par
Et si come la terre est toute reonde a son compas, dont li point est el parfont de la terre, c’est en son mileu, ke les gens apelent abisme ; tout autresi sont compassé li cercle des elimens et des planetes et du firmament, si k’il sont trestout reont, li un dedens l’autre, et li uns environne l’autre. Et li cercles ki est dedens est maindres que cil ki est sor li. Por ce n’est il mie merveille se li uns planetes cort plus tost ke l’autre ; car tant comme son cercle est plus petit, de tant le puet il corre plus tost. Et cil qui vait entour le plus grant vait plus lentement, selonc ce ke li contes dira ça avant, la ou il traitera de chascune planete par soi.
\chapterclose


\chapteropen
\chapter[{.I.CVIIII. De la grandeur de la tiere et du ciel}]{\textsc{.I.CVIIII.} De la grandeur de la tiere et du ciel}\phantomsection
\label{tresor\_1-109}

\chaptercont
\noindent Et se c’est la verité ke la terre et li autre cercle sont formé a compas, donques covient il par necessité k’il soient tout fet a nombre et a mesure. Et se ce est, nous devons bien croire ke li anchien philosophe, ki savoient arismetike et geometrie, c’est la science de toz les nombres et de totes mesures, puet bien trover larghement la grandor des cercles et des estoiles ; car sans faille li cercles est environ \textsc{.vi.} fois tant come li compas a de large, c’est a dire k’il gire \textsc{.iii.} tans come il a d’espés. Et par ceste raison maintenant comme il truevent combien la terre gire puent il bien savoir combien ele a d’espés. Et por la mesure de la terre troverent il bien par la raison dou compas et par les aleures des planetes et des estoiles, combien li un cercle sont plus en haut que li autre, et la grandor de cascun.\par
Raison coment : la terre gire tot environ \textsc{.xxm.} et \textsc{.iiiic.} et \textsc{.xxvii.} lieues lombardes, ja soit ce ke li italiien ne dient pas liues, ains dient milliers de terre, por ce que un millier de terre sont \textsc{.m.} pas, et chascuns pas contient \textsc{.v.} piés, et chascun piés contient \textsc{.xii.} pous. Mais la liue franchoise est bien \textsc{.ii.} tans u \textsc{.iii.} tans ke le millier n’en est. Puis ke l’en sot la grandeur du cercle de la terre, lors fu il chose provee ke son espés est la tierce partie de sa grandor, et son compas est le moitié de son espés, c’est la siste partie de son cercle.\par
Et il est voirs ke les planetes ki sont en pur air, et totes les estoiles ki sont el firmament, courent tozjors par lor cercles entor la terre sans reposer ; mais ce n’est mie d’une maniere, car li firmamens cort de orient en occident entre jor et nuit une fois, si roidement et si fort que sa pesantour et sa grandor le feroit tot tressalir se ne fuissent les \textsc{.vii.} planetes, ki vont autresi come a l’encontre dou firmament, et atemprent son cors selonc son ordene. Et por ce n’est il mie merveille se les planetes vont lentement, que lor aleure est samblable au frommis quant il vait a l’encontre d’une grant roe torniant.
\chapterclose


\chapteropen
\chapter[{.I.CX. Dou firmament et dou cours des .xii. signaus}]{\textsc{.I.CX.} Dou firmament et dou cours des \textsc{.xii.} signaus}\phantomsection
\label{tresor\_1-110}

\chaptercont
\noindent Sor Saturnus, ki est la .viime. pianete amont, est le firmament, ou les autres estoiles sont assises. Et sachés que de terre jusques au firmament a \textsc{.xm.} et lxvi. fois autant com la terre a d’espés. Et pour la hautece ki est si grant n’est il mie merveille se ces estoiles nous samblent estre petites. Mais a la verité il n’a el firmament jusc’au soleil nule estoile ki ne soit grignour ke tote la terre.\par
Et sachiés que les estoiles ke l’om puet reconnoistre et chosir clerement el firmament sont \textsc{.m.} et \textsc{.xxii.}, selonc ce ke l’on trueve el livre de Almagestes. Mais entre les autres en i a il \textsc{.xii.} ki sont apelés les \textsc{.xii.} signaus, ce sont Aries, Taurus, Gemini, Cancer, Leo, Virgo, Libra, Scorpio, Sagittarius, Capricornus, Aquarius, et Pisces. Cist \textsc{.xii.} signaus ont el firmament \textsc{.i.} cercle, en quoi il se tornoient environ le monde, ki est apelés zodiake. Et chacuns i a \textsc{.xxx.} gré, et ensinc est li zodiake plaine de degré ; car il en i a \textsc{.xii.} fois \textsc{.xxx.} ki font \textsc{.iiic.} et \textsc{.lx.} grés. Cis cercles est li chemins as planetes, par ou il les covient errer parmi le firmament, les unes en haut les autres en bas, chascuns selonc sa voie et selonc son cors.\par
Car Saturnes ki est li soverain de tous, et est cruel et felon et de froide nature, vait par tot les \textsc{.xii.} signaus en un an et \textsc{.xiii.} jors. Et sachiés que a la fin de cel tens ne revient il pas el leu et el point meismes dont il s’estoit meus, ains retorne en l’autre signal aprés, ou il recommence sa voie et son cors, et ensi fait tozjors jusc’a \textsc{.xxx.} ans, poi mains. Lors s’en vient il al point meismes dont il s’estoit meus au premier jor dou premier an, et refait son cours come devant. Et por ce puet chascuns entendre que Saturnus parfait et complist son cours en \textsc{.xxx.} ans poi s’en faut, en tel maniere k’il revient au premier point dont il se mut.\par
Jupiter, ki est desous li et est dous et piteus et plains de tous biens, vet par tot les \textsc{.xii.} signaus en \textsc{.i.} an et \textsc{.i.} mois et \textsc{.iiii.} jors poi s’en faut, mes son cours parfet en \textsc{.xii.} ans poi s’en faut.\par
Mars est chaus et batilleres et mauvés, et est apelés deus de batailles, vet par tout les \textsc{.xii.} signaus en \textsc{.ii.} ans et en \textsc{.i.} mois et en \textsc{.xx.} jors poi s’en faut, mais son cors acomplist et parfet en \textsc{.i.} an et \textsc{.x.} mois et \textsc{.xxii.} jors.\par
Soleil ki est bone planete imperiale vet par tot les \textsc{.xii.} signaus en un an et \textsc{.vi.} eures, mais son cors parfait en \textsc{.xxviii.} ans poi s’en faut.\par
Et Venus vait par les \textsc{.xii.} signaus en \textsc{.i.} an et \textsc{.vii.} mois et \textsc{.viii.} jors, et son cours fet il en \textsc{.vii.} mois et \textsc{.xi.} jors ; et siut tousjors le soleil, et est bele et douce, et est apelee deus d’amors.\par
Mercures vait par les \textsc{.xii.} signaus en \textsc{.iii.} mois et \textsc{.xxvi.} jors poi s’en faut, et acomplist son cors en \textsc{.i.} mois et \textsc{.xvii.} jors ; et il se mue de legier selonc le bonté et le malice des planetes a qui il se joint.\par
La lune vet par les \textsc{.xii.} signaus en \textsc{.xxvii.} jors et en \textsc{.xviii.} eures et tierce partie d’une heure, mais sa revolution fait ele tant k’ele apert en \textsc{.xxviiii.} jors et \textsc{.vii.} heures et demi heure et quinte partie d’une heure, et si acomplist tot son cors en \textsc{.xviii.} ans et \textsc{.ix.} mois et \textsc{.xvi.} jors et demi, en tel maniere k’ele revient au point et au leu dont ele estoit esmeue au comencement.
\chapterclose


\chapteropen
\chapter[{.I.CXI. Dou cours dou soleil par les .xii. signaus}]{\textsc{.I.CXI.} Dou cours dou soleil par les \textsc{.xii.} signaus}\phantomsection
\label{tresor\_1-111}

\chaptercont
\noindent A ce poés vous entendre ke li solaus, ki est plus beaus et plus dignes des autres, siet enmi des planetes, car il en a \textsc{.iii.} desus lui et \textsc{.iii.} desous. Et il vet chascun jour poi mains un degré ; car li grés du cercle sont \textsc{.iiic.} et\par
x., selonc ce ke li contes dit ça ariere. Et il met a aler par tout \textsc{.iiic.} et lxv. jours et \textsc{.vi.} eures, c’est un an.\par
Et pour les \textsc{.vi.} heures ki sont chascun an en son cours outre les entiers jors, avient il ke en \textsc{.iiii.} ans en croissent un jour, ce sont \textsc{.xxiiii.} heures. Et lors a celui an \textsc{.iiic.} et lxvi. jors, que nous apelons bissexte ; et cil jors est mis el mois de fevrier \textsc{.v.} jors a l’issue, et lors a fevriers \textsc{.xxix.} jors. Et pour ce nous covient il au kalendier demorer \textsc{.ii.} jors sur une letre, et est li .f. ki est li quinte letre a la fin de fevrier.\par
Et quant li solaus a fet \textsc{.vii.} bissextes en son cors, en tel maniere que chascun jors de la semaine a esté en bissexte, lors a li solaus tot son cours acompli entierement, et torne a son premier point et par ses premieres voies. Et pour ce fu dit ça ariere k’il parfait son cors en \textsc{.xxviii.} ans, car lors a il fet \textsc{.vii.} bissextes.\par
Et sachiés que au premier jour du siecle entra li solaus el premier signal, c’est en Aries, et ce fu \textsc{.xiiii.} jors a l’issue du mois de mars ; et autresi fet il encore. Et quant il a celui passé, il s’en entre en l’autre, et puis en l’autre, tant k’il acomplist \textsc{.i.} an ; car il li covient demorer en chascun signal \textsc{.i.} mois, c’est \textsc{.xxx.} jours et \textsc{.i.} poi plus.\par
Mais por ce k’il estoit grief a savoir as communes gens ce poi ki est outre les \textsc{.xxx.} jors, fu il establi par les anciens sages ke li \textsc{.i.} des mois eussent \textsc{.xxx.} jors et li autre en eussent \textsc{.xxxi.}, ja soit ce ke fevriers n’en a més ke \textsc{.xxviii.} quant il n’a bissexte. Et ce fu fet pour le depiecement dou jour sauver.
\chapterclose


\chapteropen
\chapter[{.I.CXII. Dou jour et de la nuit et dou chaut et dou froit}]{\textsc{.I.CXII.} Dou jour et de la nuit et dou chaut et dou froit}\phantomsection
\label{tresor\_1-112}

\chaptercont
\noindent La voie dou soleil et son cours est a aler chascun jour de orient en occident par son cercle environ la terre, en tel maniere k’il fait entre jour et nuit un tour. Et sachiés ke en chascun leu du monde a son droit orient cele part u li solaus lieve, et son occident est devers couchant. Car ou ke tu soies sur la terre, ou ça ou la, dois tu savoir ke de toi jusc’a ton orient a \textsc{.lxxxx.} grés, et autretant a de toi jusc’a ton occident. Et de ton occident jusc’a ceaus ki sont desos toi encontre tes piés droitement, a autresi \textsc{.lxxxx.} grés, et autretant jusc’a lor occident ki est le tien orient. Ensi sont \textsc{.iiii.} fois \textsc{.lxxxx.} grés, ki montent \textsc{.iiic.} et lx. degrés ki sont el cercle, si come li contes a devisé ça arieres.\par
Et pour ce dois tu bien croire k’il est toute fois jour et nuit ; car quant li solaus est desous nous, et il alume ci u nous somes, il ne puet pas alumer de l’autre part la terre. Et quant il alume deça, il ne puet pas alumer dela, por la terre ki est entre nous et eus, ki ne laisse passer s’esplendour. D’autre part, se mon occident est le orient a ceaus ki abitent encontre mes piés, et mon orient est le leur occident, donc covient que totes fois soit jour et nuit ; car quant nous avons les jours il ont les nuis, car jour ne est autre chose que soleil sour terre, ki sormonte totes lumieres.\par
Et por sa grandisme resplendissour ne poons nous de jor veoir les estoiles, car lor lumiere n’ont nul pooir devant la clarté dou soleil, car sans faille li solaus est fondemens de totes lumieres et de toutes chalours.\par
Et por ce ke sa voie se trait plus a cele partie ke nous apelons midi, avient il que celui païs est plus chaut que nul autre, ou il a grandesime terre deserte ou nule gent abitent, pour la fierté de la chalour. D’autre part com il se tire plus en bas midi et s’esloigne de nous, tant avons nous plus grant froit et grandes nuis, et cele part est la nuis petite et la chalours grignors : raison coment.
\chapterclose


\chapteropen
\chapter[{.I.CXIII. Del cercle des .xii. signes}]{\textsc{.I.CXIII.} Del cercle des \textsc{.xii.} signes}\phantomsection
\label{tresor\_1-113}

\chaptercont
\noindent Li cercles des \textsc{.xii.} signaus ki environent tot le monde est devisé en en \textsc{.iiii.} parties, dont il a \textsc{.iii.} signaus en chascune. Et li premiers signaus est Aries ou li solaus entre \textsc{.xiiii.} jors a l’issue de mars ; ce fu li premiers jors dou siecle. Et por ce ke Dieus fist lors toutes choses ygués et droites et en bon poins, fu li jors ausi grans comme la nuis, si k’il n’ot entr’aus nule difference ; et autresi est il tozjors. Et li manoir d’Aries et des autres \textsc{.ii.} signaus ki sont aprés n’est pas en bas midi, ne n’est pas haut desour nos chiés vers mienuit, c’est vers la tramontaigne ki siet vers septentrion ; ains est enmi entre \textsc{.ii.} Et por ce est li tens plus atemprés et plus naturaus as engendremens de totes choses.\par
En ceste maniere commence li solaus son cors, et s’en vet tozjors plus en amont sor nous vers le plus haut du firmament. Et por ce couvient lors les jors a croistre, et amenuisier les nuis, tant k’il passe ces \textsc{.iii.} premiers signaus, jusc’a \textsc{.xv.} jors a l’issue du mois de jung ; lors il a coru la quarte partie du cercle, c’est par Aries et par Taurus et par Geminis.\par
L’autre jor recomence il a aler par l’autre quarte partie et entre el quart signal, c’est en Cancre, et lors est il si haut com il puet plus estre. Por quoi il covient que celui jor soit plus grans ki soit en tot l’an, et la nuis plus petite, et nos avons lors chalor grant. Més el parfont midi, ou li solaus s’esloigne com il puet plus, est grandisme la nuit. Et en septentrion, ou li solaus se trait plus prés, sont les jours grandesimes.\par
Ensi s’en vait li solaus faisant son chemin avalant tozjors de haut en bas petit a petit, en tel maniere ke autresi com li jours croist de Aries jusc’a Cancre, por la montee dou soleil, tot autresi recomencent il a peticier par son avalement, tant come il vet par Cancre et par Lion et par Virgene, jusq’au quinzime jour a l’issue de septembre.\par
L’autre jor aprés entre il en l’autre quartier, c’est en Libra et lors est il ou droit mileu du cercle, ce est ou septime signal, tot droit contre Ariete. Et por ce covient il que celui jours soit ingal a la nuit et pareil, autresi com il fu de l’autre part dou cercle contre lui Mais c’est diversiment, car ceste paroiletés avient en septembre por l’abregement des jors et por l’acroissement des nuis, mais l’autre avient en mars porl’acroissance des jors et por l’abregement des nuis.\par
Ensi court li solaus par Libre et par Escorpion et par Sagittaire, tozjors abaissant et esloingant soi de nous. Et por ce decline li tens vers la froidour, tot autresi comme en mars vers la chalor. Et cil tens dure par les \textsc{.iii.} signaus devant només jusques au quinzime jor a l’issue du mois de decembre.\par
L’autre jor aprés rentre il ou derrain quartier, c’est en Capricorne, ki est tous contraires a Cancre. Et pour ce covient il que autresi come il fu lors la plus grant jours, tout autresi est lors la plus grans nuis et plus petit li jour, por ce que li solaus est esloignié de nous, por quoi il nous estuet avoir defaute de jour et de chalour. Mais la grans chalours et li grant jour sont lors en parfont midi, et les grandesimes nuis sont lors en septentrion a tout le grant froit.\par
Et ensi s’en passe li solaus par Capricorne et par Aquare et par Poisson, et amenuisent les nuis petit a petit, tant que a la fin de l’an vient a la fin du cercle. Et puis recomence son tour par Aries, selonc ce que li contes devise ci devant.
\chapterclose


\chapteropen
\chapter[{.I.CXIIII. De la difference entre midi et septentrion}]{\textsc{.I.CXIIII.} De la difference entre midi et septentrion}\phantomsection
\label{tresor\_1-114}

\chaptercont
\noindent A ce poons nous conoistre que tot autresi comme il a en midi grant terre deserte, por la prochaineté du soleil ki vet cele part, en i ra il autant u plus vers mienuit, c’est sous tramontaine, ou nules genz habitent, por les tres grans froidures ki i sont pour le desevrance du soleil, ki est loins de cele terre.\par
Ce meisme est l’achoison por quoi il avient aucune fois ke en tramontaine ne dure li jours que \textsc{.i.} sol pichot, que a paines i poroit on messe charter. Et lors dure autresi poi la nuit ou parfont midi. Et teus fois dure li jors en midi prés d’un’an, et en tramontaine dure la nuit autretant, et si avient une heure que li jors dure \textsc{.vi.} mois en autre liu, et la nuit autretant, et en la contraire partie revient li contraires.\par
Et toutes ces differences, por quoi et coment eles avienent, puet on apertement veoir et entendre, cil ki diligement consire le aleure dou soleil par son cercle, selonc ce que li contes devise apertement.\par
Et ja soit ce que li contes dit que nous avons une fois le jour plus grant que la nuis, et une autre fois la nuit plus grant, et une autre fois la nuit plus que le jor, totefois di je que tozjors, coment k’il soit, il a autretant eures en chascune nuit comme en cascun jour ; car il en a \textsc{.xii.} en chascun, por ce que li nombres des eures ne croist ne apetice. Mais quant li jours est grignours, les heures sont grignors, et cele de la nuit sont plus briés ; autresi quant la nuis est graindre, k’ele a plus grant les heures.
\chapterclose


\chapteropen
\chapter[{.I.CXV. De la grandor dou soleil et de la lune}]{\textsc{.I.CXV.} De la grandor dou soleil et de la lune}\phantomsection
\label{tresor\_1-115}

\chaptercont
\noindent Et sachiés que li solaus et toutes les planetes et les estoiles ki sor li sont assises, sont grignor que toute la terre ; car li solauz est plus grans \textsc{.c.} et lxvi. fois et \textsc{.iii.} vintaines que toute la terre ne soit, selonc ce que li philosophe proverent par maintes raisons droites et necessaires. Et de la terre jusques au soleil a \textsc{.vc.} et iiiixx. et \textsc{.v.} tant com li espés de la terre est grans. Mais il disent bien que les autres planetes ki sont dou ciel en aval, c’est Venus et Mercurius et la lune, sont plus petit que la terre, car terre est plus grans \textsc{.xxxix.} tans et \textsc{.i.} poi plus ke la lune, et si est ele en haut \textsc{.xxiiii.} tans et demi et \textsc{.v.} dousaines comme tote la terre a d’espés parmi.\par
Et dient que la lune est tote reonde, dont li plusor dient que l’une moitié de son cors est resplendissant et l’autre moitié est oscure. Et selonc ce k’ele cort environ, demonstre sa clarté et sa oscurté une fois plus et autre mains, selonc qu’ele tornoie. Mais a la verité ele n’a point de sa propre lumiere, més ele est clere en tel maniere k’ele puet recevoir enluminement d’autri, autresi comme une espee brunie, et cristal, et autres choses samblables. Tot autresi fait la lune, ki par soi ne luist mie tant que nos puissons veoir sa clarté ; mais quant li solaus le voit, il l’enlumine de tant com il le puet veoir, et la fait ausi resplendissant com ele apert a nous.\par
Raison coment : la lune se renovele tozjors en celui meismes signal ou li solaus maint, et ele cort chascun jour \textsc{.xiii.} grés. Et vos avés bien oii ça en arieres que \textsc{.i.} signal a \textsc{.xxx.} grés, et ensi passe la lune .n signal en \textsc{.ii.} jors et tierce, poi s’en faut. Et quant ele vient en \textsc{.i.} signal u tout le soleil est, ele est alumee de la partie deseure dont li solaus l’esgarde, a ce k’ele cort desous lui, et por ce n’en poons nos point veoir. Mais au tierç jour, quant ele ist de celui signal ou est auques eslongie de lui, et il l’esgarde de costé, lors apert le croissant a nostre veue a \textsc{.ii.} cornes. Et de tant comme ele s’esloigne plus du soleil, tant croist ele plus et plus, car il en voit plus ; tant k’ele vient au septime signal de l’autre part du cercle, tot droit contre le soleil, c’est aprés les \textsc{.xiiii.} jors ; lors le voit li solaus tot clerement, et por ce devient ele toute resplendissant quant ele est reonde.\par
Et quant ele a ce fait, maintenant comence a avaler de l’autre part du cercle, et se torne vers le soleil, et lors a primes comence a descroistre d’autre part, dont li solaus ne le puet remirer. Et tant fet qu’ele vient a son mestre, et le trueve en l’autre signal aprés, ou ele l’avoit laissié ; car en tant comme li solaus met a ler tot \textsc{.i.} signal cort la lune par toutes les \textsc{.xii.} environ.
\chapterclose


\chapteropen
\chapter[{.I.CXVI. Comment la lune emprunte sa lumiere dou soleil, et des eclipses}]{\textsc{.I.CXVI.} Comment la lune emprunte sa lumiere dou soleil, et des eclipses}\phantomsection
\label{tresor\_1-116}

\chaptercont
\noindent Et il soit ensi voirs que la lune enprompte sa clarté dou soleil, et k’ele soit maindre de lui et de la terre, est prové certainement par les eclipses et par les oscurtés de l’un et de l’autre. Raison coment : veés ci la lune entrer en celui meisme signal ou li solaus maint, lors est ele entre lui et la terre, mais ne luist mie devers nous.\par
Et il puet bien estre que s’ele soit en celi point u li solaus est si droitement k’ele cuevre nos ols, en tel maniere que nous ne veons pas le soleil, sa clarté n’a nul pooir sor nous. Mais por ce que li solaus iest plus grans que la lune et que la terre, et por ce ke la terre est grignour ke la lune, n’avient pas cele oscurté par tote la terre, se en tant non comme l’ombre de la lune puet covrir et contretenir le rai du soleil.\par
Et quant la lune s’en est alee au septime signal de l’autre part du cercle, puet il estre aucune fois k’ele est si droitement contre le soleil que la terre entre enmi et contretient l’esplendissour du soleil, en tel maniere que la lune oscurcist et pert sa lumiere a celui point k’ele en doit plus avoir. Et l’ochoison por quoi ce avient est pour ce ke l’ombre de la terre fiert tozjors tot droit encontre le leu ou li solaus maint, si com on puet veoir de lui et del feu apertement as ombres ki sont a l’encontre.\par
Et vous devés croire que li ombres de la terre si apetice tozjours, tant com ele s’esloingne, por ce k’ele est maindre que li solaus et k’il mande ses rais tot environ. Et a ce poons nous entendre que li eclipses dou soleil ne puet estre se a la lune novele non. Et celui de la lune n’a pooir k’ele aviegne se en sa reondece non.\par
Par ces et par autres raisons proverent li sage que la lune enprunpte dou soleil la resplendissant lumiere ki vient jusc’a nous ; car en ce ke lune est une estoile, il covient k’ele ait sa propre lumiere, car totes estoiles sont reluisans. Mais la lueur de la lune n’aroit pooir k’ele enluminast sur la terre, se ce ne fust de par le soleil.
\chapterclose


\chapteropen
\chapter[{.I.CXVII. De la proprieté de la lune}]{\textsc{.I.CXVII.} De la proprieté de la lune}\phantomsection
\label{tresor\_1-117}

\chaptercont
\noindent Mais pour ce que la lune est plus en bas des autres estoiles, et plus prochaine a la terre, nous samble il k’ele soit grignor ke les autres ; car nostre veue ne puet soufrir de veoir ce ki est a long de nous. Et totes choses quant eles sont lontaines nos resamblent a estre maindres k’eles ne sont.\par
D’autre part nous veons apertement que por sa prochaineté ele oevre tousjors es choses ki sont cha aval plus apertement que les autres ; car quant ele croist, il covient a croistre toutes mooles dedens les os, et cancres et escrevises et tous animaus et poissons croissent en lor mooles ; neis la mers croist et boute lors grandisme flos. Et quant ele apetice, totes choses apeticent et sont maindres ke devant.\par
D’autre part nous veons qu’ele court plus tost que les autres planetes, et ce ne poroit pas estre se li cercles de sa voie ne fust maindres des autres, et maindres ne poroit il estre s’il ne fust plus en bas. Raison coment : la lune vet par tot les \textsc{.xii.} signaus et parfait son cours des \textsc{.iiic.} et \textsc{.lx.} degrés ki sont en lor cercle en \textsc{.xxvii.} jours et \textsc{.viii.} eures et tierce, en quoi li solaus met a aler \textsc{.i.} an, selonc ce que li contes a devisé ça ariere.\par
Mais nous devons savoir que li ans est en \textsc{.ii.} manieres, car li uns est selonc le cours du soleil, en \textsc{.iiic.} et lxv. jors et quarte d’un jor, et li autres est de la lune, c’est quant ele a couru par le cercle du signal \textsc{.xii.} fois, et ce fait ele en \textsc{.iiic.} et liiii. jors : raison coment.
\chapterclose


\chapteropen
\chapter[{.I.CXVIII. Ci devise la conposte de la lune et del soleil et dou premier jor dou siecle et dou bissexte et des epactes et des autres raisons de la lune}]{\textsc{.I.CXVIII.} Ci devise la conposte de la lune et del soleil et dou premier jor dou siecle et dou bissexte et des epactes et des autres raisons de la lune}\phantomsection
\label{tresor\_1-118}

\chaptercont
\noindent Nous lisons en la Bible que au comencement du siecle, quant Nostre Sires crea et fist toutes choses, que totes estoiles furent faites au quart jor, c’est \textsc{.xi.} jors a l’issue du mois de mars. Et por ce dient li plusour que lors est la droite paroiletés entre jor et nuit, et selonc ce est la lune apelee prime et novele par aucunes gens.\par
Mais selonc les observances de sainte eglise est ele apelee prime \textsc{.ix.} jors a l’issue de mars, c’est a dire quant li hom le puet veoir, et qu’ele a parut hors dou premier signal, ou ele estoit avec le soleil, selonc ce que li contes a devisé ça arieres. Et sachiés que li arrabien dient que li jors comence lors que la lune apert, c’est au couchier dou soleil.\par
Et vous avés bien oii que de l’une acention de la lune a l’autre sont \textsc{.xxix.} jors et \textsc{.vii.} heures et demie et quinte partie d’une heure, et c’est li drois mois de la lune, ja soit ce que li conteour de sainte eglise dient qu’il i a \textsc{.xxix.} jors et demi. Et por esclarcir le nombre dient il que li \textsc{.i.} mois a \textsc{.xxx.} jors et l’autre \textsc{.xxix.} Et de ce avient que li \textsc{.xii.} mois de la lune sont \textsc{.iiic.} et liiii. jors.\par
Et issi est li ans du soleil graindres que celui de la lune \textsc{.xi.} jors enterin, et par os \textsc{.xi.} jors de remanant avient li embolisme, c’est a dire l’an ki a \textsc{.xiii.} lunes. Raison coment : en \textsc{.iii.} ans i a de remanant \textsc{.xxxiii.} jors, ki sont une lunee et \textsc{.iii.} jors et plus, et autresi font en avant d’un an en autre, tant k’il acomplissent \textsc{.vii.} embolismes, por les \textsc{.vii.} jors de la sesmaines.\par
Et c’est tout fait en \textsc{.xviii.} ans et \textsc{.ix.} mois et \textsc{.xvi.} jors et demi, selonc les arabiiens ; mais selonc les conteours de sainte eglise, ki welent amender tous depiecemens, sont \textsc{.xix.} ans et \textsc{.i.} jor, ki est outre dou remanant. Lors retorne la lune a son premier point dont ele estoit esmeue premiers, et recourt comme devant.\par
Ore veés que tous li contes de la lune et ses raisons definent et acomplissent son cours dedens \textsc{.xix.} ans. Et que chascuns ans de la lune est maindres que celui du soleil \textsc{.xi.} jors, dont il avient que la u de la lune est ou an prime, ele sera l’an ki doit venir \textsc{.xi.} jors plus ariere au rebours dou kalendier et de l’annee.\par
De cestui meismes \textsc{.xi.} jors naist uns contes ki est apelés li epacte, pour trover la raison de la lune. Raison coment : au premier an du siecle que les planetes comencierent lor cors en \textsc{.i.} meisme jour n’ot nul remanant des ans de la lune ou dou soleil. Et por ce dient que li premiers ans des \textsc{.xix.} devant dis les epactes sont nules.\par
Et en celui an est la lune prime au novime jour a l’issue de mars, si come ele fu au comencement, et toute cele annee est come lors. Au secont an ke li remanans comença a primes, sont les epactes \textsc{.xi.}, car tant croist la lune ; et la u ele fu au premier an prime, au secont ara \textsc{.xi.} jors, au tierç an sont les epactes \textsc{.xxii.}, au quart an montent .xxxiii.\par
Mais por ce k’il i a un embolisme, c’est une lunee, tu dois oster les \textsc{.xxx.} jors, por ce que toutes lunes d’embolisme ont \textsc{.xxx.} jours ; et dois retenir le remanant, c’est \textsc{.iii.}, ki sont epacte du quart an. Ensi dois tu maintenir les riules, que tu joindras chascune anee \textsc{.xi.} ; et quant li nombres monte sor \textsc{.xxx.}, tu en osteras les \textsc{.xxx.} et te tenras au remanant, et ce feras jusc’a \textsc{.xix.} ans, que les epactes sont .xviii.\par
Et quant il sont finés, il i a de remanant \textsc{.i.} jor, selonc ce que li contes a dit ça ariere, ki est apelés li saut de la lune. Lors dois tu penre celui jor et les \textsc{.xi.} remanant, et joindre sor .xviii, et font \textsc{.xxx.}, c’est une lune embolisme, ki doit estre mise en l’anee .xixime. ; et tu n’en as aucun de remanant, por quoi les epactes sont nules, si come devant.\par
Et sachiés que les epactes muent tozjors en septembre, mais sa chaiere est \textsc{.x.} jors a l’issue de mars ; car en celui jor que la lune n’estoit encore veue, et sainte eglise ne le met en conte, si come vous avés oï ci devant, et ses jornees estoient nules, segnefie que en celui an sont les epactes nules. Mais la seconde anee, que la lune ot a celui jour \textsc{.xi.} jors, segnefie que les epactes sont \textsc{.xi.} Autresi est et sera tousjours ; tant comme la lune a d’aage a celui jour, tant seront les epactes de cele annee.\par
Et sachiés ke la premiere anee du siecle la lune ot le premier jor d’avrilh \textsc{.x.} jors, et en mai \textsc{.xi.}, et en jung \textsc{.xii.}, et en jungnet \textsc{.xiii.}, et en aoust \textsc{.xiiii.}, et en septembre \textsc{.v.}, et en octobre \textsc{.v.}, et en novembre \textsc{.vii.}, en decembre \textsc{.vii.}, et en jenvier \textsc{.ix.}, et en fevrier \textsc{.x.}, et en mars \textsc{.ix.} jors.\par
Cist conte sont apelé concurrent, a qui nous nous devons tenir tousjours la premiere annee quant les epactes sont nules ; mais du premier en avant dois tu joindre les epactes de celui an concurrens de celi mois que tu voldras, et tant aura la lune le premiers jour de celui mois, sauf ce que, se li nombres monte plus de \textsc{.xxx.}, tu les osteras et retenras le remanant.\par
Mais garde toi ou disenuevime an du saut de la lune, c’est a dire dou jour ki croist en tous les \textsc{.xix.} ans, selonc ce ke li contes dit ci deseure. Car de ce avient une erreur el mois de jule ; car la u la lune doit estre jugie de \textsc{.xxx.} jours, selonc les epactes ele est prime. Tout autresi te covient il garder en l’octime an et en l’onzime, pour ce que le raison des epactes i faut en \textsc{.ii.} lunees, por achoison de l’embolisme.\par
Et sachiés que la paske de la resurrection Jhesucrist mue selonc le cours de la lune. Raison comment : il fu voirs que jadis quant li pules d’Israhel fu amenés en chetivison en Babilone, k’il furent delivré \textsc{.i.} jor de plaine lune, c’est a dire k’ele avoit \textsc{.xiiii.} jors, et ce fu puis ke li solaus fu entrés en Aries.\par
Et vous avés bien oï ça arieres por quoi la chaiere de l’epacte est chascun an el disime jor a l’issue de mars ; et ensi observent li juis, que en celui jour ou de li avant, u k’il truevent la lune quatorzime, il celebrent lor Paske, en ramembrance de lor delivrance. Mais sainte eglise fait ses Paskes le premier diemence ki vient aprés cele lune plaine, por ce que Jhesucris resuscita de mort en celi jour. Et sachiés que la vielle loi gardoit le septime jor que Deus se reposa, quant il ot fait le monde et ces autres choses, ce est le samedi ; mais en la novele loi gardons nous l’octime jor c’est le diemence, por la reverance de la resurrection.\par
Et sachiés que \textsc{.xl.} jors aprés la resurrection, Nostre Signour s’en ala el ciel, et por ce celebrons nous la feste de l’Ascention. Et de lors a \textsc{.x.} jours vint li Sains Esperis sor les disciples, porquoi nous gardons la feste de la Pentecouste. Ces et maint autres choses puet on savoir par les raisons de la lune et du soleil, et por ce fet il bon a savoir les.\par
Mais ki voudra savoir quant l’anee cort el conte des \textsc{.xxviii.} ans du soleil, il prendera les ans de Nostre Signeur, et si ajoindra \textsc{.ix.} ans avec, car tant s’en estoient ja alé quant il nasqui ; et de toute cele some ostera toz les \textsc{.xxviii.} k’il pora, et le remanant sera son conte. Tout autresi ki savoir vuet quele anee court el nombre des \textsc{.xix.} ans de la lune, prenge les ans Nostre Signeur et \textsc{.i.} an plus et puis en oste tous les \textsc{.xix.} k’il puet, et li remanans est çou k’il quiert.
\chapterclose


\chapteropen
\chapter[{.I.CXVIIII. Des signaus des planetes et des .ii. tramontaines}]{\textsc{.I.CXVIIII.} Des signaus des planetes et des \textsc{.ii.} tramontaines}\phantomsection
\label{tresor\_1-119}

\chaptercont
\noindent Or est il bien legiere chose de savoir en quel signal maint le soleil ; et puis ke l’en set çou, il puet legierement savoir ou la lune est. Car ele s’esloigne du soleil chascun jor \textsc{.xiii.} degrés poi s’en faut. D’autre part se tu doubles l’aage de la lune et joindras \textsc{.v.} et la some partiras en \textsc{.v.}, sachés que tantefois con tu i troveras \textsc{.v.}, tant signal a couru la lune de celui u ele se renovela. Et tant com il i a de remanant, tant est ele ja dedens celui signal.\par
Et sachés que celui signal en quoi le soleil maint lieve tozjors au matin, c’est la premiere heure dou jor, et couche o tout le soleil la premiere eure de la nuit. Raison coment : li solaus se tornie tousjours des orient en occident, selonc ce que li firmamens tornie o tous les signaus et les autres estoiles, chascun selonc son cours ; més li solaus et les autres planetes ensivent toutes voies lor cercles des \textsc{.xii.} signaus. Et pour ce covient il que quant li solaus est en Aries, li solaus lieve et couche selonc ce que Aries fet ; et ensi lieve Aries la premiere heure du jor, et Taurus lieve l’eure seconde, Gemini la tierce heure, et puis tint li un aprés l’autre tant k’il sont trestous levés. Et quant li derrains est levés, lors couche li premiers, et vet tant toute nuit d’eure en heure, k’il revient a son levant.\par
Mais pour ce que li cercles du soleil est plus briés que celui des signaus, li avient il a fere plus tost son cours, tant k’il passe tout le jour plus avant que son signal poi main ke un degré, dont il en i a \textsc{.xxx.} en chascun signal. Et por ce garde que tant come li solaus a avancié son cors dedens son signal, autretant lieve celui signal devant le soleil, c’est a dire devant la premiere heure dou jor. Raison coment : se le soleil est ore entrés el chief d’Arles, certes il comence a lever soi el commencement de la premiere heure ; més quant il est couru jusc’au mileu d’Aries, lors est la moitié d’Aries ja levee quant li solaus lieve ; autresi di je vers la fin, et de tous autres signaus.\par
Or avés oï a quel heure del jor et de la nuit lieve chacun signal ; or est il bon a savoir ki est li sires de chascune heure. Et en some sachiés que la premiere heure de chascun jor est desous celui pianete por qui celui jour est només ; raison coment : la premiere heure du samedi est soz Saturne, et cele dou diemence est du soleil, et cil du lundi est soz la lune ; autresi sont les autres. Dont il covient que se la prime eure est de Saturne, que la seconde soit de Jupiter et la tierce de Mars et la quarte du soleil et la quinte de Venus et la siste de Mercure et la septime de la lune. Puis comence de rechief ; car l’uitime est celi meisme que la premiere, et la nuevime a celi ke la seconde. Ft ensi vet toustans par ordre jour et nuit, selonc ce ke li firmamens tornie tozjors sans definer de orient en occident, sor les \textsc{.ii.} eissiaus, ki sont l’un en midi l’autre en septentrion, et cil et cel ne se remuënt pas, autresi comme les eissiaus d’une charete.\par
Por ce nagent li marenier as ensegnes des estoiles ki i sont, k’il apelent tramontaines, et les gens ki sont en Europe et en ceste partie nagent a la tramontaine de septentrion et li autre nagent a la tramontaine de midi. Et ke ce soit verités, prenés une piere d’aimant : vous troverés k’ele a \textsc{.ii.} faces, une ki gist vers la tramontaine de midi, et l’autre gist vers l’autre. Et chascune des \textsc{.ii.} faces alie la puinte de l’aguille vers cele tramontaine a qui cele face gisoit. Et por ce seroient li marenier sovent deceu s’il ne s’en presissent garde. Et por ce que ces \textsc{.ii.} estoiles ne se muevent, avient il ke les autres estoiles ki sont enki entour ont plus petis cercles, et les autres grignours, selonc ce que les unes i sont plus prés et les autres plus loing.
\chapterclose


\chapteropen
\chapter[{.I.CXX. De nature quelle elle est et coument elle oevre es choses du monde}]{\textsc{.I.CXX.} De nature quelle elle est et coument elle oevre es choses du monde}\phantomsection
\label{tresor\_1-120}

\chaptercont
\noindent Par ces \textsc{.ii.} raisons que li contes devise ci devant et plus en arieres, poés vous bien entendre comment li firmamens tornie tozjors environ le monde, et coment les \textsc{.vii.} planetes courent par les \textsc{.xii.} signaus, dont il ont si grant poesté sor les choses terrienes k’il les couvient aler et venir selonc lor cours ; car autrement n’auroient eles nule force de naistre ne de croistre ne de finer ne d’autre chose.\par
Et a la verité dire, se li firmamens ne torniast tozjors environ la terre si com il fait, il n’est nule creature au monde ki se peust movoir par nule maniere del monde ; et plus encore, que se li firmamens demorast \textsc{.i.} seul pichot, k’il n’alast, il covendroit fondre et anientir totes choses ; por ce devons nos amer et criembre Nostre Signeur, ki est sires de tout ce, et sans qui nul bien ne nule poesté ne puet estre. Il establi nature desoz soi, ki ordeine totes les choses del ciel en aval selonc la volenté dou Soverain Pere.\par
Dont Aristotles dist, ke nature est cele chose par qui totes choses s’esmuevent ou se reposent par aus meismes. Raison comment : li feus va tozjors amont par soi meismes, et la piere se pose tozjors par soi meisme ; mais ki enclot le feu si k’il ne puisse monter, ou ki gete la piere, c’est a force et par autrui, non mie par aus meismes, dont n’est ce selonc nature. Et sor ce dist li philosophes, que les oevres de la nature sont en \textsc{.vi.} manieres, ce sont generation, corruption, acroissement, diminution, alteration, et muemens de l’un leu en l’autre.\par
Raison coment : generation est cele oevre de nature par qui totes choses sont engendrees, selonc ce k’ele fait engendrer d’un oef un oisel ; ce ne feroit trestoz li mondes ensamble, se force de nature ne le fait ; autresi di je des homes et des autres choses.\par
Corruption est cele chose de nature por quoi totes choses sont corrompues et menees a son definement ; car la mort de l’home et des autres choses n’avient se por ice non que ses humors ki le tienent en vie sont corrompues en tel maniere k’il n’ont plus de pooir ; lors covient il que cele chose viegne a sa fin ; mais quant on l’ocist a force, ce n’est mie corruptions de nature.\par
Acroissemens est cele oevre de nature ki fait croistre \textsc{.i.} petit enfant ou autre chose de sa generation jusc’a tant k’il doit croistre, car toutes choses sunt abonnees dedens lor terme, outre quoi ele ne puet pas croistre.\par
Diminutions est cele oevre de nature ki fait amenuisier un home u une autre chose ; car quant uns hom est alés jusc’a ses bonnes, et k’il a tant creu come il doit, lors recomence il a descroistre et amenuisier sa force, jusc’a sa fin\par
Alteration est cele oevre de nature ki mue une chose en autre, si come nous veons, en une figue ou autre fruit ki naist de color vert, ke nature mue cele coulor en autre, et les fet noires ou rouges ou d’autre coulor.\par
Muement est cele oevre de nature por quoi nature fet muer le firmament, les estoiles, les vens, les euues, et maintes autres choses, de l’un leu en autre, par aus meismes. Ce sont les oevres de nature, ja soit ce que li contes devise ci petit d’exampleres ; mais il souffissent bien a bon entendeour par toutes choses ki par nature sont et por ce est il chose provee a savoir k’est nature et quoi non.\par
Mais ci se taist li contes a parler du firmament et des estoiles et des choses desus, et tornera a deviser la nature des choses ki sont en tiere ; mais il devisera premiers les parties et les habitations de la terre.
\chapterclose


\chapteropen
\chapter[{.I.CXXI. Ci commence mapamunde, et devise briement toutes les regions et les lieus des habitans}]{\textsc{.I.CXXI.} Ci commence mapamunde, et devise briement toutes les regions et les lieus des habitans}\phantomsection
\label{tresor\_1-121}

\chaptercont
\noindent Terre est chainte et avironnee de mer, selonc ce que li contes a devisé ça arieres, la ou il parole des elimens. Et sachiés que çou est la grant mer, ki est apelee Ocheaine, de quoi sont estraites totes les autres ki sont parmi la terre en diverses parties et sont ausi come bras de celui. Dont cil ki vient par Espaigne en Ytaile et en Grece est graindres que les autres, et por ce est ele apelee la mer grant Mitteterraine, por ce k’ele fent par mileu de la terre jusques vers orient, et devise et depart les \textsc{.iii.} parties de la terre.\par
Raison coment : toute la terre est devisee en \textsc{.iii.} parties, ce sont Aise, Aufrique, et Europe. Mais ce n’est mie a droit, por ce ke l’une n’est pas igal a l’autre, car Aise tient bien une moitié de la terre toute, dés le leu ou le fleuve de Nile chiet en mer en Alixandre, et de ce leu ou le fleuve de Thanaim chiet en mer au Bras Saint George vers orient, tot jusc’a la mer ocheaine et au paradis terrestre.\par
Les autres \textsc{.ii.} parties sont le remanant de la terre vers occident partot jusc’a la mer ocheaine, mais eles sont devisees par la mer grant ki est entre deus. Et cele partie ki est par dela vers midi jusk’en occident est Aufrike, et l’autre terre ki est par de ça vers tramontaine, c’est en septentrion vers soleil couchant, est Europe.\par
Et por mieus moustrer le païs et les gens du monde, traitera li contes briement de cascune partie par soi, et premierement d’Aise, ki est la premiere et la grignor ; et comencera de celui chief ke est vers midi, ou el se part d’Aufrique au fleuve de Nile et au fleuve de Tigre, ce est en Egypte.
\chapterclose


\chapteropen
\chapter[{.I.CXXII. D’aise qi siet en la partie d’orient}]{\textsc{.I.CXXII.} D’aise qi siet en la partie d’orient}\phantomsection
\label{tresor\_1-122}

\chaptercont
\noindent En Egypte est la cité de Babilone et Lou Caire et Alixandre et plusors autres viles. Et sachiés ke Egypte est une terre ki siet contre midi et s’estent vers soleillevant ; et par deriere li est Etyope ; et par desus cort li fleuves de Niles, c’est Geon, ki comence desous sus la mer ocheaine, ou il fait maintenant un lac ki est apelés Nilides, et est de toutes choses samblables a celes ke nos veons au fleuves de Niles.\par
Et d’autre part quant il a en Mauritaine grans pluies ou grant noif ki decheent en celui lac, lors croist li fleuves de Niles et enonde la terre d’Egypte ; et por çou dient li plusour que cil fleuves est de celui lac. Mais l’euue del lac s’en entre par terre et court par voies closes et par pertruis privés dedens la terre, tant k’ele apert en Cesaire, ou ele se demoustre toute samblable au premier lac. Et iluec s’en entre de rechief souz terre, et s’en vet par lontaines terree, k’il n’en ist fors jusques a la terre d’Etyope, ou il apert et fet \textsc{.i.} fleuve ki a non Tigre, de quoi li contes dit k’il devise Affrique d’Aise. A la fin se part il en \textsc{.vii.} et s’en vet outre par midi en la mer d’Egypte, et en ist uns fleuves ki baigne et arose toute la terre d’Egypte, car il n’i a autres fleuves ne pluies.\par
Raison coment : quant li solaus est ou signal de Cancre \textsc{.x.} jors a l’issue de jung, cil fleuves comence a croistre, et tozjours croist jusk’il entre ou Lion. Et quant li solaus entre ou Lion et est dedans, il a si grant force, les \textsc{.iii.} jours devant les kalendes d’aoust jusc’a \textsc{.xi.} jors a l’entree, k’il ist outre le lit de son cours ça et la, tant k’il arouse toute la terre ; et ensi fait tant que li solaus maint en Lyon. Et quant il entre en Virge, il comence a descroistre chascun jour plus.et plus, tant que le solaus entre en Libre et que li jorz et la nuis sont ygal en septembre, lors retorne li fleuves dedens ses rives et se reclot en son canel.\par
Por ce dient li egyptiien ke en cele anee que li fleuves de Nile croist trop en haut et que son acroissement se desmesure outre \textsc{.xviii.} piés que li champ ne gaignent mie tant por le moistour des euues, ki i gisent trop longhement. Et quant il croist mains de \textsc{.xiiii.} piés, lors ne puent li champ estre baignié par tot ou il besoigne, et por ce avienent les famines en cele terre et la defaute du blé. Mais quant il est en \textsc{.xvi.} piés ou enki entour, lors est la plenté de toz biens.\par
C’est le fleuves d’Egypte de qui dient li plusour que sa naissance ne puet restre trovee outre celui leu ou le Tigre se part en \textsc{.vii.} et que Nile comence. Sa voie est le païs d’Arrabe, ki s’apertient a la mer Rouge. Et sachiés ke cele mers est rouge non mie par nature mais par accident, por les teres ki sont rouges dont il fait son cors. Et cil est \textsc{.i.} goufre de la mer occheaine ki est devisee en \textsc{.ii.} bras, un ki est de Perse et l’autre ki est d’Arrabe.\par
Et sachiés que en la riviere de la mer Rouge est une fontaine de tel nature ke se les brebis en boivent, tot maintenant comencent a muer la coulour de la toison des berbis dedens la pel ; et cele coulor croist et vient, et l’autre coulor, quant la toison est escreue, s’en vet o tout la toison. En celui païs croist li encens et la mirre et la canele et un oisel ki est apelés fenix, dont il n’a ke \textsc{.i.} en tout le monde, selonc ce que nous troverons ça avant ou livre des oisiaus.\par
Encore est outre celui leu meisme mont Casse, ou est Jafe la tres anciene vile de tout le monde, si come elle fu fete devant le delouge. Encore i est Sorie et Judee, c’est une grant province ou li basme naist, et si i est la cité de Jerusalem et Bethleem et le fleuve de Jordan, ki est ensi apelés por \textsc{.ii.} fontaines dont l’une a non Jor et l’autre Dan, ki se joignent ensamble. Et ces fleuves naissent sez le mont Liban, et devise le païs de Judee de celui d’Arrabe, et en la fin chiet en la mer Morte pres de Jericho.\par
Et sachiés que mer Morte est issi apelee pour çou k’ele n’engendre ne ne rechoit nule chose vivant, et toutes choses ki sont sans vie cheent en parfont, ne nus vens ne’l puet movoir ; et est toute samblans a burre bien tenant, por ce est ele apelee la mer Salmaire et le lac d’Asfalt. Et sachiés que le burre de celui lac est si tenant et si glutineus que, se uns hom en preist une filailles, eles ne se depecheroient jamés, ains s’en venroient totes ensamble, se l’en ne le touchast au sanc menstruel des femes, ki tantost le depieche. Et cil lac est es parties de Judee.\par
Aprés est Palestine, ou est la cités d’Escalona, ki jadis furent apelé Philistiem. Long de Jherusalem sont les \textsc{.ii.} cités de Sodome et de Gomorre. Dedens Jude vers soleil couchant sont les hesseniiens, ki par lor grant savoir se desovirent des gens por eschiver deliz, car entr’aus n’a nule feme, ne pecune n’i est ja conneue ; il vivent del palmis ; et ja soit ce que nus i naisse, la multitude des gens n’i faut, et se maint des gens i vont, nus n’i puet manoir longhement, se chasteté et foi et innocence n’est avec lui, car Deus ne le suefre mie.\par
Aprés vient li païs de Seluice, ou il a \textsc{.i.} autre mont Casse pres d’Anthioce, ki est si haut ke l’en puet veoir le soleil dedens la quarte partie de la nuit, et ensi puet on veoir jour et nuit a une heure, et puet on veoir la luour devant que li jors apere.\par
Par enki cort li fleuves d’Eufrate, ki naist en Ermenie la grant, sor Zizame, au pié del mont Catoten, tres parmi Babilone ; et s’en vet en Mespotaigne, et la baigne il et enonde tout le païs, tot ausi com Nile fet en Egypte en celui tens meismes. Salustes dist que Tigres et Eufrates naissent en Ermenie d’une meisme fontaine.\par
Tigre est uns fleuves ki alieve son chief en Ermenie d’une noble fontaine ki est només Elogies ; et au comencement cort lentement sans son non ; et quant il touche la marche des mediiens, maintenant est apelés Tigres, tant k’il chiet el lac ki est apelés Aretuse, ki sostient toutes choses coment k’eles soient griés et pesans. Et son cors est en tel maniere parmi le lac ke li poisson de l’une partie n’entrent pas en l’autre, et court si fort que c’est une merveille, et sa coulor est devisee de celui du lac. En ceste maniere s’en vet Tigre courant come foudre tant k’il treuve Montor a l’encontre, lors entre desous terre, et ist de l’autre part a Zomode. Puis s’en entre il desous terre, et cort dedens tant k’il apert en la terre des jabeniiens et des arabiiens.\par
Aprés vient Celice, une grant terre ou Montor siet. A destre esgarde septentrion ; et cele part est Caspie et Orcanie. A senestre esgarde midi, et en cele partie est Amazonie li regnes des femes, et Achaie et Escite, et ses frons esgarde occidant. Tant comme ce mont esgarde midi, eschaufe il fort par le soleil ; mais d’autre part, ki esgarde septentrion, n’a ke vens et pluies. La est la terre de Scite ou li mons de Cimere siet, ki par nuit fait grans fumees ; et la terre d’Aise la petite, ou est Epessim et Troie et la terre Galate de Bitine et la terre de Pasfegloine et cele de Capadoce, et la terre de Asire, en quoi est Arbelite, une regions u Alixandres venki Daire le roi, et la terre de Mede.\par
Encore sont a destre de Montor les portes de Caspe, ou l’en ne puet aler fors ke par \textsc{.i.} petit sentier ki fu fais a force par mains d’omme ; ki a de lonc bien \textsc{.viiim.} pas. Puis i a une espasse de \textsc{.xxviiim.} pas de terre par lonc, u il n’a point de puis ne de fontaines. Et sachiés que maintenant que printans vient, li serpent du païs s’enfuient cele part, par quoi nus hom puet aler. As portes de Caspe vers orient est uns leus li plus plentevieus de totes choses ki sont sus terre, et cil leus est apelés Direu. Enki enprés est la terre de Termegire, ki est si tres douce et delitable ke Alixandres i fist la premiere Alixandre, ki est ore apelee Cileuce.\par
Aprés est Bautie, uns païs ki fiert contre la terre d’Inde. Outre les bautriens est Pande, une vile de sodianiens, ou Alixandres fist la tierce Alixandre, por demostrer le fin de ses aleures. C’est li leus ou premierement Liber, et puis Ercules, et puis Semiramis, et puis Cire firent autel, por signe k’il avoient la terre conquise jusques la, et plus avant n’avoit point de gent. Par enki se torne la mers de Scite et celi de Caspe en occheaine, et au commencement sont les tres grans nois et parfondes. Aprés i est la grant deserte. Aprés i sont Antropofagi, une gent mout aspres et mout fieres.\par
Aprés i a une grandesime terre ki tote est plaine de bestes sauvages, si cruels ke l’on n’i puet pas aler. Et sachiés que ceste male aventure avient par le grandesime joug ki est sor la mer, que li barbarin apelent Tabi. Aprés sont les grandesimes solitudes et les terree desabitees vers soleil levant.\par
Aprés cel liu et outre totes habitations des homes trovons nous tot avant homes ki sont apelé Sere, ki de fuelles et d’escorce d’aubres par force d’euue font une laine dont il font vestimenz. Et sont humbles et paisibles entr’aus, et refusent compaignie d’autre gent ; més li nostre marcheant passent lor premier fleuve et trovent sor la riviere totes manieres de marchandises ki la puet estre trovee ; et sans nul parlement il esgardent a l’oeil le pris de chascune, et quant il l’ont veu, il en portent çou k’il welent et laissent la vaillance ou leu meismes. En ceste maniere vendent il lor marchandise ne des nostres ne welent ne poi ne mout.\par
Aprés ce est la terre de Arace sor la mer, ou li airs est mout atemprés ; et entre cele terre et Inde siet li païs de Symicoine entre deus.\par
Aprés cele terre siet Inde, ki dure des les montaignes de Medes jusc’a la mer de midi, ou li airs est si bons k’il i a \textsc{.ii.} fois esté et \textsc{.ii.} moissons en une annee, et en leu d’yvier i est uns vens clous et souef. En Inde avoit \textsc{.vm.} viles bien pueplees et habitees de gens ; et ce n’est pas merveille, a ce ke li indiiens ne furent onques remués de lor terre. Li tres grant fleuve ki sont en Ynde sont Ganges et Yndus et Ypanus, li tres noble fleuves ki detint les aleures Alixandre, selonc ce ke les bonnes k’il i ficha sor la riviere demoustrent apertement.\par
Gambaride sont li derrenier peuple ki sont en Inde. En une ille de Ganges est la terre de Pras et de Paliborte, et mont Martel. Les gens ki habitent entor le fleuve Indus devers midi sont de verde coulor. Hors de Inde sont \textsc{.ii.} illes, Eride et Argite, ou il a si tres grant chose de metal que li plusor quident ke toute la terre soit or et argent.\par
Et sachiés k’en Ynde et en celui païs la outre a maintes diversités de gens. Car il i a de teus ki ne vivent fors ke de poissons ; et de teus i a ki ochient lor peres avant k’il decheent por viellece ou por maladie, et si les menguënt, et ce est entr’aus une chose de grant pitié. Cil ki habitent el mont Niles ont les piés envers, c’est la plante desus, et en chascun \textsc{.viii.} dois. Autres i a ki ont testes de chien. Et li autre n’ont chief, mais lor oeil sont es espaules. Une gent i a ke, maintenant k’il naissent, sont chenus et blans de cheviaus, et en viellece noircissent. Li autre n’ont c’un oeil, et li autre une jambe. Et si i a femes ki portent enfans \textsc{.v.} ans, mais il ne vivent outre l’aage de \textsc{.viii.} ans. Tous arbres ki naissent en Inde ne sont jamés sans fueilles. En Inde comence mont Caucassus, ki de son joug esgarde grandesime partie du monde. Et sachés que en cele partie dou terre par ou li solaus lieve naist li poivres.\par
Encore a en Ynde un autre ille ki est apelee Oprobaine, dedens la Rouge mer ; ou il cort parmi uns grans fleuves, et d’une part sont li olifant et autres bestes sauvages, de l’autre part sont home o grant plenté de pieres precieuses. Et sachiés que en celui païs ne servent pas nules estoiles, car n’en i a nules ki luisent se n’est une grant et clere ki a non Canopes. Ne la lune ne voient il sor terre fors de l’octime jour jusc’au sesime.\par
Cele gent ont a destre soleil levant. Et quant il welent aler par mer, il portent oisiaus ki sont norris cele part ou il welent aler, et puis conduisent lor nef selonc ce que li oisel demostrent. Et sachiés que li yndiien sont grignor que nule gent ; et grandesime partie de cele isle est deserte et desabitee por la chalour.\par
Aprés les yndiiens sont les autres montaignes u habitent les Ictiofagi, une gent ki ne manguënt se poisson non ; mais quant Alixandres les conquist, il lor vea k’il ne les mangaissent jamés. Outre cele gent est la deserte de Karmanie, ou il a une terre rouge ou nule gent ne vont, car nule chose vivant n’i entre k’il ne muire tantost.\par
Puis vient la terre de Perse, entre Inde et la mer Rouge et entre Mede et Carmenie. Puis i a \textsc{.iii.} illes en quoi naissent les caucatrix, ki ont \textsc{.xx.} piés de lonc. Puis est la terre de Parte, et puis la terre de Caldee ou la cités de Babilone siet, ki a \textsc{.lxm.} piés environ ; et si i cort le fleuves d’Eufrates.\par
En Inde est li paradis terrestres, ou il a de totes manieres de fust et des arbres et des pomes, et de toz les fruis ki sont en terre, et li arbres de vie ke Dieus vea au premier home. Et si n’i a froit ne chaut, mes perpetuel atemprance. Et el mileu est la fontaine ki trestot l’arouse, et nest en \textsc{.iiii.} fleuves. Et sachiés que aprés le pechié du premier home, cil leus fu clos a toz homes. Ces et maint autres terres et fleuves sont en Inde et en tote cele partie ki est vers soleil levant. Mes li contes n’en dira ore plus que dit en a, ains voldra escrire de la seconde partie, c’est Europe.\par
Et sachiés que en ceste partie oriental nasqui Jhesucris, en une province ki est apelee Judee, prés de Jherusalem en une cités ki est apelee Bethleem. Et por ce comença premierement la crestiiene loi en celui païs, selonc ce que li contes a devisé ça arieres la u il parole de lui et de ses apostres. En celui païs a mains patriarches et archevesques et evesques, selonc l’establissement de sainte eglise, ki sont par conte \textsc{.c.} et \textsc{.iii.} Mais la force des sarrasins mescreans en une grant partie sorprise, par quoi la sainte loi Jhesucrist n’i puet estre cultivee.
\chapterclose


\chapteropen
\chapter[{.I.CXXIII. De europe}]{\textsc{.I.CXXIII.} De europe}\phantomsection
\label{tresor\_1-123}

\chaptercont
\noindent Europe est une partie de la terre ki est devisee de celui d’Aise la u est li estrois dou Bras St. George et es parties de Constantinoble et de Grece ; et s’en vient vers septentrion par tote la terre de ça la mer jusk’en Espaigne sur la mer ocheaine.\par
En ceste partie de terre est la cités de Romme, ki est li chiés de tote crestiienté. Et por ce dira li contes tot avant d’Ytaile, c’est li païs en quoi Rome siet ki a devers midi la grant mer encoste, et devers septentrion bat la mer de Venisse, ki est apelee la mer Adriane por la cité d’Adrie ki fu fondee dedens la mer ; et son mileu est es chans de la cité de Reate.\par
Et sachiés que Ytaile fu jadis apelee Grece la grant, quant li grezois la tenoient ; et est finee vers soleil couchant au joug des montaignes ki sont vers Provence et vers France et vers Alemaigne, u il a une grant terre entre les autres, ki a \textsc{.ii.} fontaines. De l’une devers Lombardie naist uns fleuves mout grans, ki s’en passent par Lombardie et reçoit en soi \textsc{.xxx.} fleuves, et s’en entre en la mer Adriane prés de la cité de Ravene ; et c’est Po, que li grieu apelent Eridaine, mais en latin est il apelés Padus.\par
De l’autre fontaine vers France ist Rosnes, ki s’en vet d’autre part vers Borgoigne et par Provence, tant k’il s’en entre en la grant mer de Provence si roidement k’il enporte les nés dedens la mer, et bien \textsc{.v.} lieus et plus est euue douce ; et por ce dient il k’il est uns des \textsc{.iii.} grignors fleuves d’Europe.\par
En Ytaille a maintes provinces, dont Toschane est la premiere, ou Rome est tot avant. Et parmi Rome cort Toivre, et s’en entre en la grant mer. Et sachiés que li apostoles de Rome a desous lui \textsc{.vi.} evesques ki sont chardenals, celui d’Ostie et de Albani et de Portes et de Savine et de Tosquelain et de Penestraine. Ce furent bonne cités ancienement, mais Rome les a sosmises a sa signorie, car eles sont totes enki prés. Et dedens la cité de Rome a \textsc{.xlvi.} eglises, ou il a \textsc{.xxviii.} prestres et \textsc{.xviii.} dyacres, ki tout sont chardinals de Rome.\par
Aprés ce sont en Toschane \textsc{.xxi.} eveschiés, sans Pise, ki a archeveschié, et a \textsc{.iii.} eveschiés desous lui. Et sachiés que la derraine eveschiés de Toschane est celi de Lune, ki marchist as genevois.\par
Outre Rome est la terre de Champaigne, ou est la cités d’Anaigne et de Gaite et .vii : autres eveschiés. Aprés est la terre d’Abruz, ou il a \textsc{.vii.} evesqués. Aprés est la duchees d’Ispolite ou est la cités de Assise et Reate et \textsc{.viii.} autres eveschiés. Aprés est la marche d’Anquone, ou est la cités de Esqule et Orbins et \textsc{.xi.} autres eveschiés. Aprés est terre de Labour, ou est la cités de Benevent et Salerne et maintes grans terres, ou il a \textsc{.vii.} archeveschiés et cinquante et \textsc{.i.} eveschiés.\par
Aprés est li regnes de Puille, ou est la cités d’Aceronte, sus la senestre corne d’Ytaile. Et sachiés que en Puille a \textsc{.viii.} archeveschiés et \textsc{.xxxvii.} eveschiés. Aprés est Calabre, ou est l’archeveschié de Cosens et \textsc{.ii.} autres archeveschiés et \textsc{.xvi.} eveschiés.\par
Aprés est l’ille de Sezille, entre la mer Adriane et le nostre, ou est l’archeveschié de Palerne et celi de Messine et de Mont Roial, a tot \textsc{.ix.} eveschiés. Et si est mont Gibel, ki tozjors giete fu par \textsc{.ii.} bouches, et nanporquant il a noif desus tozjors. Et si i est la fontaine d’Aretuse. Et sachiés que entre Sezille et Ytaile si a \textsc{.i.} petit bras de mer enmi ki est apelé Far de Mechine ; por quoi li plusor dient que Sesille n’est mie en Ytaile, ains est un païs par soi. Et en la mer de Sezille sont les illes Vulcaines, ki sont de nature de fu. Et toute la terre de Sezille n’est que \textsc{.iiim.} estages ; et estages est en grezois ce que nous apelons milliers, et ke li françois apelent liue, mais il ne sont mie pareil.\par
Encore est en Ytaile la terre de Romaigne sus la mer Adriane, ou est la cités de Arimine et Ravene et Ymole et \textsc{.x.} autres eveschiés. Aprés i est Lombardie, ou est Boloigne la crasse et \textsc{.iii.} autres cités, et l’archeveschié de Melan, ki dure dusc’a la mer de Gene, et la cités de Saone et de Albinge, et puis jusc’a la terre de Ferrere, ou il a \textsc{.xviii.} eveschiés.\par
Aprés est la marche de Trevise, ki est en la patriarchié d’Aquilee, ou il a \textsc{.xviii.} eveschiés, ki touchent les parties d’Alemaigne, et de Jare et de Dalmache sus la mer.\par
Encore est en Ytaile l’archeveschié de Jene o tout \textsc{.iii.} eveschiés, et puis i est l’ille de Sardaigne et Corsique, ou il a \textsc{.iii.} archeveschiés et \textsc{.xv.} eveschiés.\par
La ou Ytaile fenist a la mer de Venisse, si est la terre d’Istre d’autre part la mer, ou est l’archeveschié de Jadre et \textsc{.ii.} autres archeveschiés et \textsc{.xviii.} eveschiés. Aprés est la terre d’Esclavonie, ou il a \textsc{.ii.} archevesquiés et \textsc{.xiii.} evesquiés. Aprés ce est la terre de Hongrie, ou il a \textsc{.ii.} archeveschiés et \textsc{.x.} eveschiés. Aprés i est la terre as poulains, ou il a \textsc{.iiii.} archeveschiés o tout \textsc{.viii.} eveschiés. Mais de ce ne dira plus li contes, ains retornera a sa matire la ou il laissa Sezille a la fin d’Ytaile.\par
Outre Sesille est dedens Europe la terre de Greze, ki comence as mons Ceraumes et define sus Elespons. La est la terre de Tezaille ou Julle Cesar se combati contre Pompee ; et Macedoine, en quoi est la cités d’Athenes et mont Olimpe, ki tozjors reluist, et est plus haut que cist air en quoi li oisel volent, selonc ce que li anciien dient, ki i monterent aucune fois.\par
Puis est la terre de Trace, ou li barbarin sont, et Romenie et Constantinoble. Et sachiés que en la fin de Trace vers septentrion cort le Danoise, c’est le grant fleuve d’Ailemaigne. Puis est dedens le nostre mer l’ille de Crete ou li rois Cres regna premiers, selonc ce que li contes dit ça en arieres el catalogue des rois de Grece. Puis est Calistos, et l’ille Ciclade, ki est apelee Ortige, ou les greches coturnix furent premierement trovees. Puis est l’ille d’Ebua et Minoia et Naxon et Melon et Carpate et Lemnos, ou est mont Athos qui est plus haut ke les nues.\par
A ce puet on entendre que en Greze a \textsc{.viii.} païs ; li premiers est Dalmace, vers occident, la seconde est Pirus, la tierce est Elados, la quarte est Tesaille, la quinte Macedoine, la sisime est Achaie, et \textsc{.ii.} en mer, ce sont Creta et Ciclades. Et si a en Greze \textsc{.v.} diversités de langages.\par
De ci comence une autre partie de Europe, sus Elespons, c’est \textsc{.i.} leu en la mer ki depart Aise et Europe, et n’a plus de large que \textsc{.vii.} estages, ou li rois Serses fit un pont de nés ou il passa. Puis s’eslargist la mers desmesureement ; mais ce n’est gaires, car po aprés devient il si estrois que ce n’est outre ke \textsc{.vc.} pas, et c’est apelé le Bofre de Trace, par u Daires li rois porta la grant habondance. Et sachés que le Danaon est uns grans fleuves, ki est apelé Istre, ki naist es grans mons d’Alemaigne en occident vers Lombardie, et reçoit \textsc{.lx.} fleuves, si grans trestous ke nefs i puent aler ; tant k’il se depart en \textsc{.vii.} et s’en entre en mer vers oriant, dont li \textsc{.iiii.} i entrent si roidement ke ces euues maintienent lor douçor bien \textsc{.xx.} liues, ke ne se mellent a l’euue de la mer.\par
Outre ce leu a l’entree d’orient est la terre d’Escite. Desouz est mont Rifet, et l’Iparborei ou li oisel grifon naissent. Mais on prueve par les sages que la terre d’Escite siet en Aise, selonc ce que li contes devise ci devant, ja soit o que les illes de Scite sont de cha le Danon \textsc{.lxxm.} pas loing dou Bofre de Trace, ou est la mer congelee et perecheuse ke li plusour apelent la mer Morte.\par
Aprés la terre de Scite est Alemaigne, ki comence a la montaigne de Seune sus le Danon, et dure dusc’al Rin : c’est uns fleuves ki departoit jadis Alemaigne et France, mais or dure dusk’en Loheraigne. Et sachiés k’en Alemaigne est l’archeveschié de Magance et de Trieves, et \textsc{.vii.} autres archeveschiés et bien \textsc{.liiii.} eveschiés, jusc’a Mes et a Verdun, el contree de Loheraigne.\par
Aprés Alemaigne outre le Rin est France, ki jadis fu apelés Galle ; en quoi est premierement Borgoigne, ki commence as montaignes entre Alemaigne et Lombardie au fleuve del Rosne a l’archeveschié de Tarentasne et de Besençon et de Vienne et de Ombron ; ou il a \textsc{.xvi.} eveschiés.\par
Puis commence la droite France a le cité de Lion sor le Rosne, et dure dusk’en Flandres sor la mer d’Englenterre, et en Picardie et en Normendie et a la petite Bretaigne et Angio et Poito, jusc’al Bordiaus au fleuve de la Gironde, jusques au Pui Nostre Dame ; ou il a \textsc{.vii.} archeveschiés et bien cinquante et \textsc{.i.} eveschiés. Aprés est Provence jusc’a la mer, ou est l’archeveschié d’Ais et d’Arle o tout \textsc{.xii.} eveschiés. D’autre part est Gascongne, ou il a \textsc{.vii.} archeveschiés et \textsc{.x.} eveschiés ; et marchist a l’archeveschié de Nerbonne, ou est la contree de Toulouse et de Monpellier et \textsc{.ix.} eveschiés.\par
Aprés cest terre comence li païs d’Espaigne, ki dure par toute la terre du roi d’Arragon et du roi de Navare et du roi de Portugales et de Chastele jusc’a la mer ocheaine ; ou est la cités de Tolete et de Compostele, ou gist le cors monsignor St. Jake. Et sachiés k’il a en Espaigne \textsc{.iiii.} archeveschiés o \textsc{.xxxvii.} eveschiés de crestiens, sans les sarrasins, ki i sont encore.\par
Iki est la fins de la terre, selonc ce que les ancienes gens proverent ; et meismement le tesmoignent li tertre de Calpe et de Albinna (ou Ercules ficha ses colombes quant il venki tote la terre), ou leu ou la nostre mer ist de la mer ocheaine, et s’en vient parmi ces \textsc{.ii.} mons (ou sont les illes Gades et les colombes Ercules), en tel maniere k’il laisse les mors et toute la terre d’Aufrique a destre, et Espaigne et tote Europe a senestre, ou il n’a pas \textsc{.viim.} pas de large et \textsc{.xvm.} de lonc, et ne fine jusques as parties d’Aise, et k’il se conjunt a la mer ocheaine.\par
D’autre part la terre de France vers septentrion bat la mer ocheaine. Et por ce jadis i fu la fins des terres habitees, jusc’a tant que les ges crurent et multepliierent et k’il passerent en un ille ki est en mer, et a de lonc \textsc{.viiic.} mil pas, c’est la grant Bretaigne, ki ore est Engleterre dite ; en quoi est l’archeveschiés de Cantorbiere et celui d’Ebruic et \textsc{.xviii.} eveschiés.\par
Aprés i est Irlande, ou est l’archeveschié d’Armachie et de Duveline et de Casseles et de Tuem, o \textsc{.xxxvi.} eveschiés. Aprés est Escoche u il a \textsc{.ix.} eveschiés. Et aprés i est la terre de Norbe, ou il a \textsc{.iii.} archeveschiés o tot \textsc{.x.} eveschiés. Et sachiés que en la plus grande partie de toutes ces illes, et especialment en Irlande, n’i a nul serpent ; et por ce dient li païsant que, la ou on portast des pieres ou de la terre d’Irlande, nul serpent n’i poroit demorer.\par
Ces et maintes autres terree et illes sont outre Bretaigne et outre la mer de Norvee ; mais l’ille de Tille est la derraine, ki est si durement el parfont de septentrion que en esté quant le soleil entre el signal de cancre, as tres grans jors, la nuis i est si tres petite k’ele samble autresi come nient ; et en yver quant le soleil entre en Capricorne a la tres grant nuit, li jors i est si tres petis k’il n’a nule espasse entre la levee et la couchee dou soleil. Et outre Tilen est la mers congelee et tenans, ou il n’a nul devisement ne conjungement de levee ne de couchee, selonc ce ke li contes dit la ou il traita dou cours dou soleil. Encore i est l’ille d’Ebrides, ou li home ki la habitent n’ont nul blé, mais il vivent de poisson et de lait. Encore i sont les illes Orcades, ou nules gens n’abitent. Mais ci se taist li contes a parler d’Europe, ki define en Espaigne, et dire de la tierce partie, c’est Auffrike.
\chapterclose


\chapteropen
\chapter[{.I.CXXIIII. D’aufriqe}]{\textsc{.I.CXXIIII.} D’aufriqe}\phantomsection
\label{tresor\_1-124}

\chaptercont
\noindent En Espaigne est li trespas en Libbe, une terre d’Affrike ou est la regions de Mauritaine, c’est la terre des mors. Et sont \textsc{.iii.} Mauritaines, une ou fu la cités de Sutin, l’autre ou fu Cesaire, la tierce ou est la cités de Tingi. Et en Maritaine fenist la haute mers d’Egypte et comence cele de Libe, ou il a trop fiere merveille : car la mers i est d’assés plus haute que la terre, et si se tient dedens ses marges en tel maniere k’ele ne dechiet ne decort sor la terre. En celui païs est Atlans le mont, enmi les harenes, ki est plus haut que les nues, et dure dusc’a la mer ocheaine. Puis est Numide, la terre as numidiiens.\par
Et sachiés que tote Affrike comence sur la mer ocheaine as colombes Ercules ; et de ki se torne vers Tunis et vers Bugee et vers la cité de Septis, tot contreval vers Sardaigne, jusc’a la terre ki siet contre Sezille ; de ci se devise en \textsc{.ii.} parties, une qui est apelee la terre Cane, et l’autre ki s’en vet outre l’ille de Crete jusques es parties d’Egypte ; et s’en vet entre les \textsc{.ii.} sirteis des terres, ou l’en ne puet aler en nule maniere, pour les flos de mer ki ore croissent et ore descroissent si perilleusement ke nef n’i aroient nul pooir d’aler por diversitet des flos, ki ne vienent pas ordeneement, mais sans certaineté.\par
En ceste maniere dure tote la partie d’Affrique entre Egypte et la mer d’Espaigne, tozjors encoste la nostre mer. Mais par deriere vers midi sont les desers de Etyope sus la mer ocheaine, et le fleuve de Tygre, ki engendre Nile, ki devise la terre d’Affrike et celui d’Etyope, ou li etyopiien habitent. Et sachiés que tote la terre ki garde vers midi est sans fontaines, et nue d’euues, et povre terre ; mes devers midi est ele grasse et plentevieuse de tous biens.\par
Dedens les parties d’Aufrique sont contees les \textsc{.ii.} cirtés, de quoi li contes fist mention ci desoure, et l’ille de Menne ou est li fleuves Letheu ; de qui les anchienes istores dient ke c’est li fleuves d’infier, et que les ames qui en boivent perdent la remembrance des coses alees, en tel maniere qu’eles n’en ont plus memoire quant eles rentrent es autres cors, selonc la opinion as mescreans. La sont les gens de Namazoine et de Trogodite et les gens des Amans, ki font les maisons de sel. Puis est Garesmans, une vile ou on trueve une mervilleuse fontaine dont les euues sont de jour si tres froides ke nus ne les suefre, et de nuit sont si tres chaudes ausi, et c’est par une meisme vaine.\par
Encore i est la terre d’Etyope et del mont Athalant, ou sont les gens noires come meure, et por ce sont il apelé mors, por la prochaineté du soleil. Et sachiés que les gens d’Etyope et de Garremans ne sevent que mariages soit, ains ont entr’aus femes communes a tous ; et por ce avient ke nus n’i conoist peres, se meres non, por quoi il sont apelé les mains nobles gens du monde. Et sachiés que en Etyope sus la mer vers midi est un grant tertre ki giete grant plenté de feu ardant tozjors sans estanchier. Outre ces gens sont li tres grant desiert, ou nules gens ne reperent, jusques en Arrabe.\par
Or avés oï coment li contes devise briement et apertement les regions de la terre, et comment ele est avironnee de la grant mer ki est apelee ocheaine, ja soit ce que ses nons change et mue en plusor leus selonc les nons du païs ou ele bat : car premierement ou ele bat a la terre d’Arrabe est ele apelee la mers d’Arrabe, et puis la mers de Perse, et puis la mers d’Inde, et puis la mer de Yrcanie et de Caspe, et puis la mers de Escite et d’Alemaigne, et puis la mers de Galles, c’est d’Engleterre, et puis la mers de Athlans et de Libe et d’Egipte.\par
Et sachiés que es parties d’Inde ceste mers croist et descroit mervilleusement, et fait grandesime flos, ou por ce que la force du chaut le sostient en haut autresi come pendans, ou por ce que en celui païs a grant habondance de fleuves et de fontaines.\par
Et sor ce doutent li sage por quoi ce est que la mers ocheaine fait ce flot, et mande les, et puis les retrait grant piece, et puis les retrait \textsc{.ii.} fois entre jor et nuit sans definer. Li un dient que li mondes a ame, a ce k’il est fais de \textsc{.iiii.} elimens, et por ce covient il k’il ait esperit. Et dient que cil esperis a ses voies ou parfont de la mer, par ou il espire, ausi com l’en fait par les narilles. Et quant il espire hors et ens, il fet les euues de mer aler sus et trere et revenir ariere, selonc ce que son espiremens vet ens et hors.\par
Mais li astronomien dient que ce n’est se par la lune non, a ce ke l’en voit les flos croistre et apeticier selonc la croissance et la descroissance de la lune, de \textsc{.vii.} en \textsc{.vii.} jours que la lune fait ses \textsc{.iiii.} voutes en \textsc{.xxviii.} jors par les \textsc{.iiii.} quartiers de son cercle, de qui li contes a dit tout l’iestre.
\chapterclose


\chapteropen
\chapter[{.I.CXXV. Comment hom doit eslire terre gaaignable}]{\textsc{.I.CXXV.} Comment hom doit eslire terre gaaignable}\phantomsection
\label{tresor\_1-125}

\chaptercont
\noindent Et puis ke nostre contes a devisé la terre selonc ses habitations, il volt \textsc{.i.} po dire de la terre meisme, selonc ce k’ele est gaignable, car c’est la chose por quoi les vies des gens sont maintenues. Et por çou n’est il fors biens a moustrer quex chans on doit elire et en quel maniere.\par
Pallades dist que on doit garder \textsc{.iiii.} choses, c’est l’euue et l’air et la terre et maistrie, dont les \textsc{.iii.} sont par nature, et li uns est en volenté et en pooir. Par nature est que nous devons garder l’air qu’il soit sains et nés et dous, et ke li euue soit bone et legiere et la terre plentevieuse et bien seant.\par
Et ore orés raison coment : li sains airs puet estre conneus en ceste maniere, ke li leus ne soit es parfondes valees, et k’il soit purs de tenebreuses nuees, et ke les gens ki i abitent soient bien sain de lor cors et cler et apert, et ke la veue et l’oiie et la vois d’iaus soient bien cleres et purefiiet.\par
La bonté de l’euue pués tu aperçoivre, s’ele ne nest de palus ou de mal estanc ou de leu de soufre ou de coivre, et que sa colours soit luisans,et sa savours et son odours ne soient viciiet, et k’il n’i ait nul limon dedens, et soit en yvier chaude et en esté froide.\par
Et la naissance de son cors soit vers orient, un poi declinans vers septentrion, et bien courant et isnele sour petites pierres ou sor bele araine, ou au mains sor crete bien monde, ki ait sa colour rouge u noire. Car c’est signe que ceste euue soit bien legiere et soutil, ki tost eschaufe au feu et au soleil, et tost refroide quant ele en est eslongie, par sa legierté, ki le fait bien movant de l’une qualité en l’autre, a ce k’il n’i a nule chose terrestre. Mais sor toute maniere d’euue est cele ki est novelement coillie de pluie, s’ele est bien monde et mise en cisterne bien lavee netement sans totes ordures, por ce k’ele a mains de moistor que les autres, et est un petit stitique, non mie tant k’ele nuise a l’estomac, ains le conforte.\par
Aprés cesti est l’euue de fleuve lonc de vile, ki soit bien clere et courans sour sablon u sus pieres ; mais sour pieres est millour por le hurter des pieres, ki le fait plus delie ; et cele ki cort sor net sablon est millor que euue vielle en cisterne, ki prent males fumees de la terre par trop manoir dedens. Et tous fleuves et ruissiaus ki courent devers soleil levant sont mieldres que devers septentrion.\par
Et sachiés que euue est nuisans au pis, as nerfs, et a l’estomac, et engendre color en ventre, et fait estroit pis, pour çou s’en doivent garder toz ciaux ki ont froide complexions ; mais mout se doit garder d’euue sause et nitreuse, car ele eschaufe et seche et enpire le cors dedens. Euue de mer est forz sause et trenchans et poignans,et por ce mondefie le ventre de flegme grosse et viscouse.\par
Et generalment toutes euues sont froides et moistes, et por ce ne donnent au cors d’omme norrissement ne nule croissance, s’ele n’est composte d’autre chose. Et sa bonté poons nous perçoivre as gens ki i abitent et ki en boivent useement, s’il ont la bouche dedens saine et pure, et bone teste, o totes les vaines du pomon, et k’il n’aient dolours ou enflures el cors dedens, et la vessie nete et sans vice.\par
Terre dois tu connoistre s’ere n’est blanche ou nue, et ke ne soit de megre sablon sans compaignie de terre, et ke ne soit couverte de poudre clere ou de poudre doree, ne coverte de pieres, et k’ele ne soit sause ne amere, ne ne soit oligineuse ne genitoise, ou plaine d’araines, ne ne soit en oscure valee trop pendans ; mais soit grasse ausi come noire, et ki soit bien soufissans a covrir totes semences et racines. Et ce ki i naist ne soit pas boçu ne retort, ne sans propre jus ; mais doit engendrer des herbes, ki senefient bon forment.\par
Et en some doit on garder ke terre soit douce et grasse, car de color ne puet il gaires chaloir. Se tu vieus prover k’ele soit grasse, tu prenderas une poignie de terre et le moilleras bien d’euue douce ; et puis, s’ele est glutineuse et bien tenans, sachés k’ele est grasse. Et d’autre part tu feras une petite fosse, et puis le rempliras de la terre meismes que tu en aras ostee : s’il n’i a de remanant, sachés que la terre est grasse, et s’ele faut la terre est maigre, et s’il n’i a remanant ne faute cele terre est moiene entre \textsc{.ii.} Et se tu voldras connoistre terre douce, tu en meteras \textsc{.i.} petit en \textsc{.i.} pot avec euue douce, et puis l’aissaieras a ta langue.\par
Le leu de ton champ ne doit pas estre si plains k’il face estanc ne si pendans k’il decoure, ne si tres haut k’il reçoive toutes chalours et toutes tempestes ; mais il doit tenir le mileu, en tel maniere k’il soit profitables et bien estans. Et s’il est en froide terre, tu dois eslire teus chans ki soient contre oriant ou contre midi, sans tertre ki contretiegne les rais du soleil. Et se c’est en chaut païs, il est bon ke ton champ soit contre septentrion.
\chapterclose


\chapteropen
\chapter[{.I.CXXVI. Coment on doit maisonner et en quel lieu}]{\textsc{.I.CXXVI.} Coment on doit maisonner et en quel lieu}\phantomsection
\label{tresor\_1-126}

\chaptercont
\noindent Et por ce ke les gens maisonent sovent et volentiers sor bone terre, voldra li mestres ensegnier coment on le doit faire. Et tot avant doit chascuns warder que son edifiement ne trespasse outre la dignité ou outre la richesce de lui, ou il a grant peril, selonc ce que li contes devisera ça avant el livre des \textsc{.iiii.} vertus, el capitle de richesse ; et por ce ne dira il pas ci aleques de cele matire. Ains dist li mestres, ke li sires doit premierement esgarder la nature de l’euwe k’il doit user, et connoistre sa nature ; car on doit eschivre males euues et palus et estanc, et meismement s’il sont contre occident ou contre midi, et s’il ont en coustume de sechier en esté, por çou k’il sont pestilencie et engendrent malvais animaus.\par
Et li frons de ta maison doit estre contre midi, en tel maniere que li premiers angles soit contre le soleil de printans. Et d’autre part vers soleil couchant doit la maisons \textsc{.i.} po decliner vers soleil d’yvier, dont il avient ke cele maisons a tozjors la chalour dou soleil en yvier, et ne le sent en esté.\par
Et tot le merien de ton edifiement soit tailliés en novembre jusques a la lune novele ; et au mains jusc’a mooules, en tel maniere que tote la moistours s’en isse, ki est es vaines. Et sachiés que tout merien ke l’en taille vers midi son millor, ja soit ce que devers septentrion soient il plus haut, mais il devienent vicié plus legierement.\par
Et la chaus soit de piere blanche, ou rouge ou tiburtine, ou d’espoignes, ou au mains chenues, ou a la fin noires, ki pies valent. Mais araine de mer met trop a sechier ; et por ce garde ke ton edifiement ne soit pas fait tot ensamble, car ce seroit paine perdue ; et si doivent estre premiers baignié de euue douce, ke en oste l’amertume de la mer.\par
Ton celier doit estre contre septentrion, froit et oscur et loins de bains et d’estable et de four et de cisternes et d’euue et de toutes choses ki ont fieres odours. Le grenier degire cele partie meismes, a ce k’il soit loins de fiens et de toute moistour. Li leus des oeilles soit contre midi et soit bien garnie por le froit.\par
L’estables des chevaus et des bués garde vers midi, et ait aucune fenestre por alumer devers septentrion, en tel maniere que tu le puisses en yvier clore pour la froidour eschiver, et en esté ovrir por refroidier. Et si doit l’estable estre pendans, por decourre toutes humors, k’eles ne nuisent as piés des bestes.
\chapterclose


\chapteropen
\chapter[{.I.CXXVII. Comment on doit faire puis et fontaines}]{\textsc{.I.CXXVII.} Comment on doit faire puis et fontaines}\phantomsection
\label{tresor\_1-127}

\chaptercont
\noindent Se il fust chose k’il n’eust euue entor ton manoir, tu la dois querre en ceste maniere : le matin devant que soleil lieve en aoust, tu demorras encontre orient, le menton sor terre, et regarde tout droit la ou tu verras lever l’air crespe, autresi come une nue delie en samblance d’espandre rosee ; car c’est signe d’euue qui est reposte souz terre, se ce n’est leu ou il ait coustume d’avoir lac ou estanc ou autre moistor, selonc ce que demonstre le jonc et li saus sauvage et toz arbres ki de moistour naissent.\par
Et quant tu auras veu cel signe, tu dois chevillier la terre \textsc{.iii.} piés de large et \textsc{.v.} de haut. Et quant li solaus est couchiés tu dois metre dedens \textsc{.i.} vaissel de coivre ou de plonc ki soit oins dedens, et puis covrir le fosse, et estouper de busche et de terre, et au matin oster ; et se le vaissel sue dedens et il i a goutes d’euue, ne doutes ja que ci aura bon puis.\par
Encore se tu mes dedens le fosse un pot de terre sechiet, non mie cuit, s’il i a vaine d’euue, il sera fondus au matin. Encore se tu i mes \textsc{.i.} toison de faine et tu la troveras l’endemain baigné, u une lucerne ki au matin soit estainte, sachiés k’il i a euue a plenté ; et pour ce dois tu chevillier ton puis. Mais es piés du mont de septentrion habondent euues a grana fuison, et sont plus saines.\par
Et pour ce ke terre engendre sovent soufre et alun et teles choses perilleuses, doit li hons ki fait le puis avoir entor soi une lucerne ardant ; car s’ele dure sans estaindre, c’est bons signes, mais s’ele ne dure et amortist sovent, c’est signe de peril ou li chevilliers poroit devier tost et legier.\par
La bontés de l’euue doit estre essaie en ceste maniere : tu la metras en vaissel de coivre bien net, et s’ele n’engendre aucune male teche, c’est bon ; encore quant ele est cuite en \textsc{.i.} petit pot de coivre, k’ele ne face arene ou limon ; encore s’ele cuist tost le legum et est tres luisans et pure sans nue et sans toutes ordures.
\chapterclose


\chapteropen
\chapter[{.I.CXXVIII. Comment on doit faire cisternes}]{\textsc{.I.CXXVIII.} Comment on doit faire cisternes}\phantomsection
\label{tresor\_1-128}

\chaptercont
\noindent Mais se li lieus est teus que l’on n’i puisse trover euue ne chevillier puis, tu feras unes cisternes ki ait plus de lonc que de lé, et soit bien pavee en haut, et enointe sovent et menu de bon lart quit. Et quant ele est bien ointe et essuee longhement, l’euue soit mise dedens, et anguilles et poissons de fluns avec, ki por lor roer facent l’euue movoir de laiens.\par
Et se l’euue s’en ist d’aucune part, tu prenderas de bone pois liquide, et autant de bon lart ou de siu, et le feras quire ensamble, tant k’il espungnent ; lors l’en osteras, et quant il est refroidiés, tu meteras de bon chauc menuement, et le melleras ensamble, et puis le meteras au lieu par ou l’euue s’en ist.
\chapterclose


\chapteropen
\chapter[{.I.CXXVIIII. Coment on doit sa maison garnir}]{\textsc{.I.CXXVIIII.} Coment on doit sa maison garnir}\phantomsection
\label{tresor\_1-129}

\chaptercont
\noindent Quant ta maisons est acomplie et garnie de ses edifices, selonc l’estat du leu et du tens, tu dois fere chambres et cheminees la u li chans de la maison demousterra ke mieus soit. Et si penseras de molin et de four et de vivier, et de colombiers et d’estable a berbis et a pors et a gelines et a chapons et a ouwes et a anes, ke tu esliras selonc ce que li mestres dire ça avant el chapitle de la nature as animaus.\par
Mais en maisoner covient porveoir se li tens et li leus est en guerre ou en pais, ou se ce est dedens la vile ou dehors ou loins de gens. Car li ytaliien, ki sovent guerroient entr’aus, se delitent en faire tours et autres maisons de pieres. Et se c’est hors de vile, il font fosses et palis de mur et torneles et pons et portes couleices ; et sont garni de mangoniaus et de pieres et de saietes et de toutes choses ki a guerre besoignent, por deffendre et por offendre, et por la vie des homes ens et hors maintenir.\par
Mais li françois font maisons grans et plenieres et paintes et chambre lees, por avoir joie et delit sans guerre et sans noise, et por ce sevent il mieus faire praiaus et vergiés et pomiers et arbres entor lor habitacles ; car c’est une chose ki mout vaut a delit de home.\par
Et si doit li sires norrir grans mastins por les gardes de ses brebis, et petis chienos por la garde de sa maison, et levriers et braches et oisiaus por vener, quant il se wet en ce solachier. Et toute la maisons soit garnie de harnois ki soient besoignable, et en cuisine et partout, selonc ce que au signour affiert.\par
Et la maisnie soit bien ensegnie et ordenee a ce k’il doivent faire chascun a son office dedens et defors, en tel maniere que li sires soit frains et mestres de toz, et k’il voie sovent coment vait la chose de son ostel, si k’il puisse mener sa vie honestement, selonc son estat, a la maniere que li mestres ensegne ça avant ou livre de vertus.\par
Mais coment li sires doit garder son preu en labourer terres et vignes et en planter arbres et en semer et en quellir et en garder son blé et les toisons de ses oeilles et let et fromage et en norrir poulains et chevaus et en croistre son moble et son chatel, li mestres n’en dira nient plus que dit en a ; car li \textsc{.i.} le tenroient a desdaing, et li autre diroient que cel est avarice. Et cor ce laisse il ceste matire, et tornera a son conte, c’est a deviser la nature des animaus, et premierement des poissons, ki premier furent fait selonc l’ordene des \textsc{.vi.} jors.
\chapterclose


\chapteropen
\chapter[{.I.CXXX. Chi dist de la nature des animaus et premierement des poissons}]{\textsc{.I.CXXX.} Chi dist de la nature des animaus et premierement des poissons}\phantomsection
\label{tresor\_1-130}

\chaptercont
\noindent Poissons sont sans nombre, ja soit ce ke Plinius en conte \textsc{.c.} et \textsc{.xliii.} nons. Et sont de diverses manieres. Li un vivent en l’euue solement. Li autre conversent en terre et en euue, et vivent en chascun. Li autre conçoivent oes et les boutent en l’euue, et li euue les reçoit et les fet engendrer et lor done vie et norrissement. Li autre engendrent fiz vivans, ce sont balene et cete et delfin et maint autres ; et quant il les voient nés, il les gardent diligemment en tot lor tendre aage, en tel maniere que, s’il aperçoivent aucun malvais agait, la mere oevre la bouche et trait son fiz dedens son cors, la ou il avoit esté conceus ; et puis le giete hors de li quant ele wet, sans peril.\par
Et sachiés que poisson ne sevent avoutire, c’est a dire que l’une maniere ne se joint a l’autre charnelment, selonc ce que li asnes fet avec le jument, ou \textsc{.i.} cheval avec une asnesse. Ne ne puënt vivre sans euue, ne eslongier soi de sa lignie. Et si ont dens fors et agus et desous et deseure, por maintenir lor viande contre le fort cors de l’euue. Dont li \textsc{.i.} manguënt herbes et petites vermines, et li autre menguënt autres poissons ; et ce est par une tele maniere ke tousjours li maindres est viande del grignour, et ensi vit li uns de l’autre.\par
Balenes sont de fiere grandour, et gete l’euue plus en haut ke nule maniere de poisson. Son masle est la muscle, dont ele conçoit.\par
Serre est uns poissons ki a une creste a maniere de sie, dont il brise les nés par desous. Et ses eles sont si grans ke ele en fet voiles, et vet bien \textsc{.v.} liues u \textsc{.viii.} contre la nef ; mais a la fin k’ele ne puet pus soufrir chiet el parfont de la mer.\par
Pors sont une maniere de poissons ki chevillent la terre desouz les euues por querre lor viande, autresi come nostre porceaus font en terre ; car lor bouche est entour lor gorge, et en tele part k’il ne poroient lor viande queillir se le bek ne fust fichié dedens terre.\par
Glaive est un poisson ki a le bek autresi come une espee, dont il pertuise les nefs et les fait fondre.\par
Escorpion est apelé porce k’il laidist les mains des homes ki le prenent ; de qui dient li plusour que, se tu lies \textsc{.x.} cancres d’une herbe ki a non ozimi, ke tout li escorpion ki sont enki enprés s’asambleront as cancres.\par
Anguille est nee de limon, et por ce avient, ki plus l’estraint, plus fuit ; de qui dient li ancien ke ki buvroit du vin en quoi anguille fust noie n’auroit plus talent de boivre vin.\par
Murene est apelee pour ce k’ele se plie en mains cercles, de qui li pescheour dient que toutes murenes sont femeles, et k’ele conçoit de serpent et por ce la claiment il au flaüt, en guise de la vois au serpent, et ele vient et ensi est prise. Et sa vie n’est se en la coue non ; car ki le fiert sor le chief ou sor le dos, ele ne muert mie, mais des cols de la coue define maintenant.\par
Echinus est uns petiz poissons de mer ; mais il est sages, kar il aperçoit devant la tempeste, et maintenant prent une piere et porte la avec soi, autresi comme une ancre, et porte la por maintenir soi contre la force des tempestes ; por ce s’en prennent garde sovent les mareniers.
\chapterclose


\chapteropen
\chapter[{.I.CXXXI. Dou cocodril c’on apele caucatrix}]{\textsc{.I.CXXXI.} Dou cocodril c’on apele caucatrix}\phantomsection
\label{tresor\_1-131}

\chaptercont
\noindent Cocodril est un animal a \textsc{.iiii.} piés et de jaune coulour, ki naist el fleuve de Nile, c’est li fleuves ki arouse la terre de Egypte, selonc ce ke li contes a devisé ça en arieres, la ou il parole de celi terre. Il est grans plus de \textsc{.xx.} piés, armés de grans dens et de grandes ongles ; et son quir est si dur k’il ne sentira ja cop de piere. Le jor abite en terre, mais la nuit se repose dedens l’euues du fleuve. Et ses oes ne fait se en terre non, en tel leu que li fleuves n’i puisse parvenir. Et sachiés k’il n’a point de langhe ; et si est li animal au monde sans plus ki mueve la maisselle deseure, et celi desous tient ferme ; et s’il vaint l’omme, il le manguë en plorant.\par
Or avient que quant \textsc{.i.} oisel ki a non strophilos i vieut avoir charoigne por mangier, il se boute a la bouche dou cocodril, et li grate tout belement, tant k’il ovre tote la gorge, por le delit du grater ; lors vient \textsc{.i.} autre poisson ki a non idre, et li entre dedens le cors, et s’en ist d’autre part, brisant et desrompant son costé, en tel maniere k’il l’ocist.\par
Neis le delfin meismes, ki ont autresi comme une sie sor le dos, quant il le voient noer, si se mueent desous et le fiert enmi le ventre, et le font devier maintenant.\par
Et sachiés que cocatrix, ja soit ce k’il nest en euue et vit dedens le Nil, il n’est mie poissons, ains est serpens d’euue ; car il ocist home qui il puet ferir, se fiens de buef ne le garist.\par
Et en cele terre habitent homes molt petis, mais il sont si hardis qu’il osent bien contrester au cocodril ; car il est de tel nature k’il cachent ciaus ki fuient et si crient ciaus ki se deffendent, dont il avient k’il est pris aucunefois ; et quant il est pris et dontés, il oublie tote fierté, et devient si privés que li hons le chevauce et li fait faire çou k’il voet. Et quant il est dedens le fleuve, il ne voit gaires bien ; més en terre voit il mervilleusement. Et tot yvier ne manguë, ains endure et suefre fain tous les \textsc{.iiii.} mois de brume.
\chapterclose


\chapteropen
\chapter[{.I.CXXXII. De cete}]{\textsc{.I.CXXXII.} De cete}\phantomsection
\label{tresor\_1-132}

\chaptercont
\noindent Cete est gres poisson que li plusor apelent balaine ; c’est uns poissons si grans comme une terre, ki mainte fois remaint en sech, k’il ne puet aler la u la mers est haute plus de \textsc{.cc.} piés. C’est le poisson ki rechut Jonam le prophete dedens son ventre, selonc çou que l’istore dou Viel Testament nous recontent ; k’il quidoit estre alés en infier, por la grandor du lieu u il estoit.\par
Cist poissons lieve son dos enmi haute mer, et tant demeure en \textsc{.i.} leu que li vens aporte le sablon et ajoustent desor lui, et tant k’il i naissent petis arbissiaus. Par quoi li marenier sont deceu maintes fois : la u il quident que ce soit \textsc{.i.} ille, si i descendent et fichent palix et font feu ; mais quant li poissons sent la chalor, il nel puet soufrir, si s’en fuit dedens la mer et fait afondrer tot quank’il a desor lui.
\chapterclose


\chapteropen
\chapter[{.I.CXXXIII. De coquille}]{\textsc{.I.CXXXIII.} De coquille}\phantomsection
\label{tresor\_1-133}

\chaptercont
\noindent Conquille est uns poissons de mer enclos en charsoit comme une escrevise, et est toute reonde ; mais ele oevre et clot quant ele voet. Et son manoir est el parfont de la mer ; \par
mais ele vient le matin en haut, et le soir, et reçoit la rosee dedens soi. Et les rais du soleil ki se fierent sur le coquille font auques endurcir les goutes de la rosee, chascune partiement, selonc ce que eles i cheirent ; non pas en tel maniere qu’eles soient pieres, tant come eles sont en mer ; \par
mais quant on les oste de mer et oevre les, on en trait hors les goutes éndurcies, maintenant devienent pieres blanches petites et precieuses ke l’en claime perles ou margarites. Et sachiés que se la rosee est pure et nete et par matin, les perles seront blanches et luisans, autrement nient. Et nules perles ne sont grignor de demi pouç.\par
Une autre coquille est en mer ki a non murique ou conche ; et li plusor l’apelent oistre, purce que quant ele est taillie environ lui il en issent larmes, de quoi l’en taint les porpres ; et cele tainture est de son charçois.\par
Une autre conquille est que l’on apele cancre, por ce k’il a jambes et est reont ; il est anemis as oistres, car il menguë sa char par mervillous engin, et ore orés comment : il porte une petite piere, et ensit l’oistre, tant k’ele oevre son charçois ; lors vient li cancre et giete la piere dedens, en tel maniere k’il n’a pooir de reclore, et en ceste maniere se paist.
\chapterclose


\chapteropen
\chapter[{.I.CXXXIIII. Dou delfin}]{\textsc{.I.CXXXIIII.} Dou delfin}\phantomsection
\label{tresor\_1-134}

\chaptercont
\noindent Delfin est uns grans poissons de mer, ki ensit les vois des homes. Et est la plus isnele chose ki soit en mer, car il trespasse la mer d’outre en outre, autresi comme s’il volast. Mais il ne vait legierement seul a seul, ains vont plusor ensamble. Et par eus aperçoivent li marenier la tempeste ki doit venir, quand il voient les delfies fuir parmi la mer, et trebuchier soi en fuiant, autresi come la foudre le chaçast.\par
Et sachiés ke delfin engendrent fiz, non pas oes, et le portent \textsc{.x.} mois, et le garde et norrist de son let. Et quant si filz sont en sa joenece, il les aquellent dedens sa gorge, por mieus garder les. Et vivent \textsc{.xxx.} ans, selonc ce que gens le dient ki l’ont assaié as coues, k’il lor taillent. Et sa bouce n’est pas la ou les autres poissons les ont, ains est aprés le ventre. Contre la nature des bestes d’euue, nule ne muet la langue se le delfin non solement.\par
Et son esperit ne puet li atraire tant come il soit souz l’euue, s’il ne vient en haut en l’air. Et sa vois est samblable a home plorant.\par
En printans en vont plusor a la mer de Ponto, ou il norrissent lor fiz por la plenté des euues douces ; et lor entree est a destre et lor issue a senestre, por ce k’il ne voient gaires bien dou senestre oeil, mais dou destre voient il apertement.\par
Et sachiés que el fleuve de Nile sont une maniere de delfin ki ont sus le dos une eschine autele comme une sie, dont il ochient le cocodril.\par
Et si trovons es ancienes istoires que uns enfes de Campaigne norri le delfin de pain longhement, et le fist si privé k’il le chevauchoit, et tant que li dalfins le porta jusk’en haute mer, et enqui fu noiez ; et en la fin se laissa morir le delfin, quant il aperchut le mort de l’enfant.\par
Un autre en ot en Jace de Babilone, ki tant ama \textsc{.i.} enfant que aprés çou k’il ot jué avec lui, et le garçon s’en fui, il le voloit ensivre, si remest sor le sablon ou il fu pris. Ces et maintes autres merveilles sont veues de ceste beste, por les amours k’il portent as homes.
\chapterclose


\chapteropen
\chapter[{.I.CXXXV. De ypotame}]{\textsc{.I.CXXXV.} De ypotame}\phantomsection
\label{tresor\_1-135}

\chaptercont
\noindent Ipotame est uns poissons ki est apelés cheval fluel, pour çou k’il naist el fleuves de Nil. Et son dos et ses crins et sa vois est comme de cheval, ses ongles sont fendues comme de buef, et dens comme sengler, et la coue retorte, et manguë blés de champ, ou il vet a reculons, por les agais des homes ;\par
et quant il manguë trop et il aperchoit k’il est enfondus par sormangier, il s’en vet par sus les canons novelement tailliés, tant que li sans s’en ist par ses piés a grant fuison, et par tel maniere garist il de sa maladie.
\chapterclose


\chapteropen
\chapter[{.I.CXXXVI. Des sieraines}]{\textsc{.I.CXXXVI.} Des sieraines}\phantomsection
\label{tresor\_1-136}

\chaptercont
\noindent Serene, ce dient li auctor, sont de \textsc{.iii.} manieres, ki avoient samblance de feme dou chief jusk’as quisses, mais de cel leu en aval avoient samblance de poisson, et avoient eles et ongles. Dont la premiere chantoit mervilleusement de sa bouche comme vois de feme, l’autre en vois de flaüt et de canon, la tierce de citole ; et que par lor dous chans faisoient perir les nonsachans ki par la mer aloient.\par
Mais selonc la verité, les seraines furent \textsc{.iii.} meretrix ki dechevoient toz les trespassans et les metoient en povreté. Et dist l’estoire k’eles avoient eles et ongles, por senefiance de l’amor ki vole et fiert ; et conversoient en euue, por ce que luxure fu faite de moustor.\par
Et a la verité dire, il a en Arrabe une maniere de blans serpens ke l’on apiele seraine, ki courent si mervillousement ke li plusor dient k’il volent. Et lor venim est si tres cruel que, s’ere mort aucun home, il le covient devier ançois k’il sente nule dolour.\par
Mais des diversités des poissons ne de lor nature ne dira ore li contes plus ke dit en a, ains dira des autres animaus ki sont en terre, et premierement des serpens, por ce k’il sunt resamblables as poissons de maintes proprietés.
\chapterclose


\chapteropen
\chapter[{.I.CXXXVII. Des serpens}]{\textsc{.I.CXXXVII.} Des serpens}\phantomsection
\label{tresor\_1-137}

\chaptercont
\noindent Serpens sont de maintes generations ; et tant comme il sont de maintes manieres devisé, et tant ont il de diverses natures. Mais generalment tous serpens sont de froide nature, ne ne fierent s’il n’escaufent. Et por ce nuist li venins d’aus plus de jor ke de nuit ; car de tens de nuit se refroident il en aus tout coiement, por la froidour de la rosee. Et tot yvier se gisent en lor nos, et en esté s’en issent.\par
Et tout venin sont froit ; et por ce avient que li hom, quant il en est ferus, a poour tot avant ; car l’ame, ki est caude et de nature de fu, fuit la froidure du venin. Et il est apelés venin por çou k’il entre dedens les vaines ; et n’a pooir de mal faire s’il ne touce le sanc de l’home, et lors, quant li venins s’escaufe et art dedens, maintenant ocist l’omme.\par
Les natures des serpens sont iteles : quant il enviellist et lor oils sont plain de tenebres, il jeune longhement, et se garde de mengier tant k’il amaigrist et sa pel est large et pleniere en son dos ; lors s’en entre par fine force entre l’estroit de \textsc{.ii.} pieres, tant k’il se despoille de sa vielle eschaille, et devient joenes et frés et bien veans ; mais il use fenoil au mangier por avoir clere veue. Quant il voet boire, il laisse son venin en aucun leu repostement.\par
Et crient home nu ; et s’il menguë le crachet d’un home jeun, il muert. Et sa vie est ou chief, en tel maniere que, se la teste eschape vive outre \textsc{.ii.} dois solement de son cors, il vit, et ja pour çou ne muert ; et c’est la chose por quoi il met tout son cors en peril por deffendre la teste.\par
Tot serpent ont corte veue, et non gardent en travers se poi non ; car lor oils ne sont pas el front devant, ains sont d’encoste dedens les oreilles, et por ce ont il plus preste l’oïe ke le veue. Il maine la langhe plus tos que nule chose vivant, et por ce quident maintes gens k’il aient \textsc{.iii.} langhes, mais ce n’est c’une.\par
Et son cors est si moistes que nés la voie par ou il vet desegne par sa moistour. Et por ce ke serpens use ses costés en leu de jambes, et les eschames en leu des ongles, avient il ke s’il est ferus en aucune partie de la gorge, a la fin dou ventre, il pert cele force, en tel maniere k’il ne puet courre si com il soloit.
\chapterclose


\chapteropen
\chapter[{.I.CXXXVIII. De l’aspis}]{\textsc{.I.CXXXVIII.} De l’aspis}\phantomsection
\label{tresor\_1-138}

\chaptercont
\noindent Aspis est une maniere de venimeus serpens ki ocist home de ses dens. Ja soit ce, k’il sont de plusours manieres ; et chacuns a une proprieté de mal faire ; car celi ki est apelés aspis fait morir de soif l’ome que ele mort ; et l’autre ki a non prialis le fait tant dormir k’il muert ; et l’autre ki est apelés emorois li fait fondre tot son sanc jusc’a sa mort ; celi ki a non preste vait tozjours bouche overte, et quant il estraint nuli a ses dens, il enfle tant k’il devie, et maintenant porrist si malement que c’est diaublie.\par
Et sachiés que aspis porte la trés luisant et la precieuse piere que l’on claime carboncle ; et quant li enchanteour ki li veut oster la piere dist ses paroles, et maintenant ke la fiere beste s’en aperchoit, fiche l’une de ses oreilles dedens la terre et l’autre clot de sa coue, en tel maniere k’ele devient sourde et non oïans des paroles conjurans.
\chapterclose


\chapteropen
\chapter[{.I.CXXXVIIII. Du serpent a .ii. testes}]{\textsc{.I.CXXXVIIII.} Du serpent a \textsc{.ii.} testes}\phantomsection
\label{tresor\_1-139}

\chaptercont
\noindent Anfemeine est une maniere de serpens ki a \textsc{.ii.} testes, l’une en son lieu et l’autre en la coue, et de chascune part vieut ele corre ; et cort isnelement, et ses oils sont luisans comme chandeille. Et sachiés que c’est li serpens au monde sans plus ki maint a la froidure, et tozjors vait devant les autres comme chievetains et guierres.
\chapterclose


\chapteropen
\chapter[{.I.CXXXX. Des basiliques}]{\textsc{.I.CXXXX.} Des basiliques}\phantomsection
\label{tresor\_1-140}

\chaptercont
\noindent Basiliques est li rois des serpens. Et est si trés plains de venim k’il en reluist toz par dehors, nés le veoir et le flairier de lui porte venin et loins et prés, par quoi il corront l’air, et si estaint les arbres ; et tel est que de son odor ocist les oiseaus volans, et de son veir les homes quant il les esgarde ; ja soit ce que li ancien dient k’il ne nuist pas a celui ki voit primes les basilikes que il aus. Et sa grandor est de mi pié, et a blanches taches, et creste comme cok. Et vet droit contremont la moitié devant, et l’autre moitié comme autres serpens.\par
Et tot soit il si fiers, les belotes l’ocient, c’est une beste plus longhe que soris, et est blanche el ventre. Et sachiés que Alixandres les trova, et fist faire grandes ampoles de voirre ou homes entroient dedens ki veoient les basiliques, mais il ne veoient ceaus, ki les ocioient des saietes ; et par itel engin en fu delivrés il et son ost.
\chapterclose


\chapteropen
\chapter[{.I.CXXXXI. Des dragons}]{\textsc{.I.CXXXXI.} Des dragons}\phantomsection
\label{tresor\_1-141}

\chaptercont
\noindent Dragons est li trés grans serpens de toz, neis une des grans bestes dou monde, ki abite en Inde et Etyope, ou il a tozjours grant esté. Et quant il ist de son espelonche, il court parmi l’air si roidement et par si grant air que li airs en reluist aprés lui autresi come fu ardant.\par
Et cil a une grant teste et bouche petite, ou il a pertuis overs par ou il tret la langhe et son esperit. Et sa force n’est pas en la bouche mais en la coue, dont il nuist par batre plus ke par navrer ; ou il a si grant force que nus, coment k’il soit grans et fors, se li dragons l’estraint de sa coue, k’il en eschape sans mort, neis l’olifant meismes en covient il morir, a ce k’il a entr’aus mortel haine, selonc ce ke li mestres dira ça avant el conte de l’olifant.
\chapterclose


\chapteropen
\chapter[{.I.CXXXXII. De scitalis}]{\textsc{.I.CXXXXII.} De scitalis}\phantomsection
\label{tresor\_1-142}

\chaptercont
\noindent Scitalis est uns serpens ki vait mout lentement. Mais il est si bien tachiés de diverses coulors cleres et luisans ke les gens l’esgardent volentiers tant k’il les aproche, et la puour de lui les detient et les sosprent. Et sachiés k’il est de si chaude nature ke neis en yver despoille il sa piel pour le chaut k’il a.
\chapterclose


\chapteropen
\chapter[{.I.CXXXXIII. De la vipre}]{\textsc{.I.CXXXXIII.} De la vipre}\phantomsection
\label{tresor\_1-143}

\chaptercont
\noindent La vipre est une maniere de serpent de si fiere nature ke, quant li malles se conjoint a la femele, il met son chief dedens la gorge de la femele ; et quant ele sent le delit de la luxure, ele l’estraint as dens et li trence la teste, et l’englout, et de çou conchoit. Et quant si faon ont vie et k’il en voelent issir, lors desrompent et brisent a fine force le cors de lor mere, et vont hors, en tel maniere que son pere et sa mere muerent par aus. De cest serpent dist Sains Ambroses k’ele est la tres cruel chose du monde, et plus sans pitié et plaine de malisce.\par
Et sachiés que cist serpens, quant il a talent de luxure, s’en vet as euues ou la murene repaire, et l’apele de sa vois en samblance de flaüt, et ele vient a lui maintenant ; et par itel engin est ele sovent prise par les pescheours, selonc ce que li contes devise ça arieres el chapitre des poissons.
\chapterclose


\chapteropen
\chapter[{.I.CXXXXIIII. Des lisardes}]{\textsc{.I.CXXXXIIII.} Des lisardes}\phantomsection
\label{tresor\_1-144}

\chaptercont
\noindent Lisarde est de \textsc{.iii.} manieres, une grant, une petite, et une autre ki eschaufe en esté et prent homes as dens malement. Mais quant la petite lisarde enviellist, ele s’en entre parmi un estroit pertruis d’une paroit contre le soleil, et despoille la nue de ses iex et toute sa viellece.\par
Salemandre est samblable a petite laisarde de vaire coulor, et son venin est tres fors sor les autres ; car les autres fierent une seule chose, mais cestui en fiert plusours ensamble ; car s’ele monte par \textsc{.i.} pomier, ele envenime totes les pomes du pomier et ocist toz ceus ki en manguënt ; et s’ele chiet en \textsc{.i.} puis, la force de son venin ocist toz ceaus ki en boivent. Et sachiés que salamandre vit enmi les flames du feu sans color et sans damage de son cors, neis ele estaint le fu par sa nature.\par
Mais ci se taist li mestres a conter des serpens et de lor natures, et des vermines, comment il sont de diverses manieres et comment il naissent en terre et en euue et en air et en fueilles et en fust et en dras et en homes et en autres bestes vivans, ens et hors, sans assamblement de malle et de femeles, ja soit ce k’il naissent d’ués aucunefois. Si n’en devisera ore plus li contes, car ce seroit une longue matire sans grant proufit ; et ensivra son conte pour parler des animaus.
\chapterclose


\chapteropen
\chapter[{.I.CXXXXV. De l’aigle}]{\textsc{.I.CXXXXV.} De l’aigle}\phantomsection
\label{tresor\_1-145}

\chaptercont
\noindent Aigle est li mieus veans oiseaus du monde ; et vole si en haut k’ele n’apert pas a la veue des homes, mais il voit si clerement, que neis les petites bestes connoist ele en terre et les poissons es euues, et les prent en son descendre.\par
Et sa nature est de garder contre le soleil si fermement que ses oils ne remuent goute ; et pour çou prent l’aigle ses fiz et les tient a ses ongles droit contre le rai du soleil ; et cil ki esgarde justement sans croller est tenus et norris si come dignes, et celui ki les iex remue est refusés et getés du nit comme bastars.\par
Et ce n’est pas cruauté de nature, mais pour jugement de droiture, car li aigle ne la chace pas pour son fiz, mais comme autrui estrange. Et sachiés que \textsc{.i.} vil oisel, ki est apelés fulica, acomplist la fierté du roial oisel ; car ele reçoit celui entre ses fiz et le norrist comme ses fiz.\par
Et sachiés que aigle vit longhement por ce k’ele se renovele et depose sa viellece. Et dient li plusour k’ele vole en si haut leu vers la chalour du soleil que ses pennes ardent ; et oste toute l’oscurité de ses iols, et lors se laisse cheir en aucune fontaine ou ele se baigne \textsc{.iii.} fois, et maintenant est joene come a son comencement.\par
Li autre dient ke le bek de l’aigle croist et plie en son grant aage, en tel maniere k’ele ne puet mais penre de os bons oiseaus ki la maintenoient en vie et en jounece ; lors le fiert maintenant et aguise tant as roides pieres que le sorplus s’en oste, et son bek vient plus gens et plus esmolus que devant, si k’ele menguë et renovele et prent ce ke li plest.
\chapterclose


\chapteropen
\chapter[{.I.CXXXXVI. Des ostours}]{\textsc{.I.CXXXXVI.} Des ostours}\phantomsection
\label{tresor\_1-146}

\chaptercont
\noindent Ostours est uns oiseaus de proie, si come sont faucons et esperviers et autres oiseaus que l’en tient par delit a penre autres oiseaus. Ki tot sont mout fiers contre loz fiz ; car maintenant k’il les voient auques escreus, et k’il ont aucun pooir de voler, il ne les paissent pas de lors en avant, ains les chacent hors dou nit et les constraignent a porchacier lor viande en lor jounece ; car il ne voelent que lor fis oublient le propre mestier de lor ancissours, ne k’il aprenent a estre pereçous, en tel maniere les entrelaissent a norrir por metre a ravir.\par
Et sachiés ke ostours sont de \textsc{.iii.} manieres, petis et grans et moiens. Li petis est maindres des autres, a lois de terzel, et est preus et mainiers et tost volans et desirans de mangier, et legier en oiseler.\par
Les moiens ont les eles rouges, piés cours, ongles petites et mavaises, et les oils gros et oscurs, et est trop durs a fere le domesche ; et por ce ne vaut il gaires la premiere annee, mais a la tierce est bons et debonaires.\par
Li grans ostours est graindres des autres, et plus gros et plus meiniers et millors, et a oils biaus et clers et luisans, et gros piés et grans ongles et liet visage. Et est hardis, ke por nul oisel ne s’alentist, neis li aigle ne li fait nule paour.\par
Por ce dist li mestres que en eslire bon ostor l’en doit garder k’il soit grans et bien furnis par tout ; car a la verité dire, en trestoz oiseaus chaceours li grignor sont femeles, et li petit, c’est a dire li terzel, sont malles. Et sont si caut por la masculinité ki en aus regne, et si orgilleus, ke a paine prennent autre chose se tant non come il welent. Mais la femele, ki est froide, por la feminité ki en li est, est tozjors covoitouse et desirans de prendre, por çou ke froidure est rachine de toute covoitise.\par
Et ce est la nature por quoi li grant oisel veneour sont millor, car il n’ont desdaing de prendre, ains desirent tozjors la proie plus et plus ; en tel maniere ke quant il sont fors, en prendent aucun mal vice, ja soit ce k’il les perdent a le mue, u il muent et remieudrent pennes et abis ; més li terzel prendent a chascune mue aucun mal vice.
\chapterclose


\chapteropen
\chapter[{.I.CXXXXVII. De connoistre bien ostours}]{\textsc{.I.CXXXXVII.} De connoistre bien ostours}\phantomsection
\label{tresor\_1-147}

\chaptercont
\noindent Et quant l’en trueve ostour grant, garde k’il ait longue teste et plate a samblance d’un aigle, et ke sa chiere soit lie et un poi encline endementiers k’il soit adoubés ; car puis doit son vout estre autresi come courouciés et plains d’ire. Et ait les nés et les narilles bien jausne, et le moien ki est environ entre les oils soit bien lonc, et le sorciz pendant. Et les oils soient hors et grans assés par raison, et coulourex bonement, car c’est signes k’il soit fiz d’ostour ki fu mués plus de \textsc{.iii.} fois ; dont il vit mieus et plus longhement quant il est engendrés de viel pere.\par
Son col doit estre lonc et soutil et serpentin, et le pis gros et reont come coulons. Les panons, c’est les \textsc{.ii.} pennes des eles, que li plusour apelent espoetes, doivent estre serrees as eles, si k’eles n’aparissent. Les eles briés auques et bien joignans, pennes franchies et bien tenans, jambes grosses, jaunes, et briés, piés grans et lés et overs, et lonc talon, et tot l’arteil bien gros, non pas de char, mais de ners avec les os ; ongle grosse et fort et dure, et l’orteil de mi lonc amesureement : c’est la maistrie pour conoistre bien ostours. Mais tant sachiés que cil ki ont longhes jambes prenent plus legierement, et a paines falent dou prendre, mais il ne tienent si bien ne si fort con cil ki ont cortes jambes et briés, ja soit ce k’il ne pregnent si legierement come ciaus ki les ont longhes.\par
Et quant tu voudras savoir s’il est sains ou malades en aucune part, tu le dois lever sus la senestre main, et remirer legierement et diligeament haut et bas ; et s’il esgarde haut, et k’il ne bate fort, et ne bate son bee, et sa coue ne tiegne eslaischie, sachiés vraiement k’il est sains de son cors.\par
Et quant tu as ce fait, tornoie ton poing desus ses piés, et garde s’il revient mainterrant et demeure desus fermes, si drois k’il ne s’a fiche plus de l’un pié ke de l’autre, car s’est signes k’il soit fors et sains de ses piés. Et s’il gete tost et isnel sa jambe contre la char, et quant il l’a prise il baisse bien son bek et le prent et trait de grant force, et destent le col et ferme ses piés et ses talons, tu pués bien dire k’il soit sains de jambes et de quisses sans faille. Lors le detrai parle coutiaus de l’une des eles, et puis par l’autre, et s’il les laisse tirer et les retrait en son leu tost et igal, c’est senefiance k’il soit bien sains de eles.\par
Aprés ce garde s’il esmautist bien et delivrement, selonc le quantité dou past, blanc et noir, non pas entremellé, mais que l’un soit parti des autres ; ne k’il n’i ait sanc ne moustour clere, ne piere ne vers ne nule autre melleure ; car ce demoustre k’il soit bien sains dedens le cors. Et se aprés mangier il netoie son bek, et esmouche sovent, et gete euue par le nés, et fert son bek ça et la, et ne se tient ja en leu, sachiés k’il est sains de son chief. Meismement se sus la main u sur perche il se paroint et atorne sa plume, et demeure drois, et manguë et cuist la viande bien et gent, lors est il bien sains et haitiés de cors et de membres.
\chapterclose


\chapteropen
\chapter[{.I.CXXXXVIII. Des esperviers}]{\textsc{.I.CXXXXVIII.} Des esperviers}\phantomsection
\label{tresor\_1-148}

\chaptercont
\noindent Esperviers doit estre esleus de tel maniere : k’il ait petite teste, et les oils forains, jouant et tornant legier sor la main, gros pis et bien overs, piés grans et blans auques apers, jambe lee et fort, coue cloant et soutille, eles longues au tierç tour de la coue. Et le braier, c’est la plume desous la coue, soit tachié ausi comme par mailles : car teus esperviers doit estre bons par raison, meismement s’il a les jambes autresi comme roinuses, et s’il ont crossete enmi le moien arteil destre, la ou l’escaille se part, car c’est signe de grandesime bonté.\par
Et sachiés tant d’espervier, que cil a longue coue est couars, mais il vole tost ; et cil ki a \textsc{.xiii.} penes en la coue est tozjours mieudres des autres et mieus volans, et plus tost aconsiut sa proie.\par
Mais ki wet muer et avoir sain son espervier, il le doit tot yvier garder k’il ne prenge pie ne autre oisel ki li face mal ; mais cil ki prent coulons en tor se navre et brise et gaste legierement, pour le grant tour k’il fet au devaler.\par
Et sachiés que toz oiseaus veneour sont de \textsc{.iii.} manieres, niais ramans et grifains. Dont niais est celui que l’en trait dou nit et norrist en son ostel de sa javence, et est plus hardis et plus covoiteus de penre, et crie sovent por la seurté k’il a des gens ou il a esté norris. Ramans est celui ki a ja volé et vené selonc sa nature, mais il est puis pris en rain d’aubre ou en autre leu par son engien. Grifaing est uns oiseaus ke l’en prent a l’entree d’yvier, et a les oils rouges et vermaus comme feu.\par
Et tant sachiés que, s’il avoit ja mangié sus glace devant ce k’il fust pris, a paine puet estre k’il vive, car sa forcele ne puet estre delivree dou froit ; mais s’il fust en son pooir, il auroit tozjours chaudes viandes et fresches, ki li aideroient a quire son past. Et pour çou avient il ke li niais n’aura ja si beaus oils ke li autres ; ki manguë chascune fois novele viande, et gist hors a l’air, et fait quank’a lui plest ; ne ne prent si bien sa proie, ja soit ce k’il ait plus de covoitise.\par
Et sachiés que oisiaus joenes engendre fiz rouges o grosses mailles et oils descoulourés et hardis, mais il n’a pooir de vivre en mains d’ome plus de \textsc{.v.} ans. Oiseaus vieus engendre fils noirs o menue maille et oils coulourés et sont millor et de longue vie.
\chapterclose


\chapteropen
\chapter[{.I.CXXXXVIIII. Des faucons}]{\textsc{.I.CXXXXVIIII.} Des faucons}\phantomsection
\label{tresor\_1-149}

\chaptercont
\noindent Faucons sont de \textsc{.vii.} lignies. Dont la premiere est faucon lanier, ki est autresi come vilains entre les autres ; et cist meismes est devisés en \textsc{.iii.} manieres, dont li \textsc{.i.} ki ont la teste petite ne vaut rien du monde, l’autre ki a gros le chief et le bek et eles longues et coue brief et piés aglentins est bons, ja soit il durs a adouber. Mais ki le fait muer \textsc{.iii.} fois, il puet penre tous oiseaus.\par
La seconde lignie est faucon ke l’en apiele pelerins, pour çou que nus ne trueve son nit, ains est pris autresi comme en pelerinage ; et est mout legiers a norrir et mout cortois et vaillans et de bone maniere.\par
La tierce lignie est faucon montardis ; assés est connus par tous lieus ; et puis k’est privés, il ne s’enfuira jamés.\par
La quarte lignie est faucon gentil ou gruier, qui vaut mieus que li autre ; més il n’a mestier a home sans cheval, car trop li covient ensivre. Et tant sachés que de ces \textsc{.iiii.} lignies vous devés eslire celi ki a plus petit chief.\par
La quinte lignie est girfauq, ki sormonte toz oiseaus de son grant, et est fors et aspres et fiers et engigneus, et bieneureus en cachier et en prendre.\par
La sisime lignie est sorpoint ; cist est molt grans et resamble aigle blanche, més des oils et des eles et dou bek et d’orguel est il samblable a girfauc, ja soit ce que je n’ai home trové ki le veist onques.\par
La septime lignie est breton, que li plusour apelent rodion ; c’est le roi et le sire de toz oiseaus, car il n’est nus ki ose voler devant lui, ains chiet toz estordis, en tel maniere que on le puet penre comme s’il fust mors. Neis l’aigle meismes, por la poour de lui, n’ose aparoir la ou il est.\par
Et ensement tous faucons ki ont le pié gros, et les genous noouses autresi comme sor os, et sauvage regart et flameant et les oils terribles, et les eles grosses par desus, et les ongles noues et longhes et planes et bien aguës par mesure et luisans, est bons s’il a la teste aguë par mesure, meismement s’il est bien espés par le pis.
\chapterclose


\chapteropen
\chapter[{.I.CL. Des esmerillons}]{\textsc{.I.CL.} Des esmerillons}\phantomsection
\label{tresor\_1-150}

\chaptercont
\noindent Esmerillons sont de \textsc{.iii.} manieres, uns ki a l’eschine grise ; et li autres ki a l’eschine noire, et sont petit et fort ravineours ; li autres est graindres et resamble faucon lanier blanchet, et est millours de toz autres esmerillons, et plus tost devient privés. Mais il lor avient une maladie, porquoi il se manguë toz les piés, se l’en ne le fait demourer en tant de semence de lin ou de mil que li arteil n’aparissent par dehors .\par
Mais ci se taist li contes a parler des oiseaus chaceours, et coment on les doit norrir et enoiseler et ensegnier a penre proie as cans et a riviere, et coment on les doit curer quant il ont aucune maladie, car ce n’apertient pas a ce livre ; ains veut ensivre la nature des autres animaus.
\chapterclose


\chapteropen
\chapter[{.I.CLI. De l’alcion}]{\textsc{.I.CLI.} De l’alcion}\phantomsection
\label{tresor\_1-151}

\chaptercont
\noindent Alcion est uns oiseaus de mer, a qui Diex a doné grandesime grasce, et ore orés comment. Ele pose ses oës enprés la mer sus le sablon, et ce est el cuer de l’yvier quant les tempestes et orribles fortunes suelent sordre par mi la mer ; et acomplist la naissance de ses fiz en \textsc{.vii.} jors, et en \textsc{.vii.} autres les norrist. Et ce sont \textsc{.xiiii.} jors ki sont de si hautes vertus, selonc çou ke li maronier, ki maintefois l’ont esprové, le tesmoignent, que totes tempestes se departent et l’air esclarcist et li tens est dous et soués tant com li \textsc{.xiiii.} jor durent.
\chapterclose


\chapteropen
\chapter[{.I.CLII. De ardea}]{\textsc{.I.CLII.} De ardea}\phantomsection
\label{tresor\_1-152}

\chaptercont
\noindent Ardea est uns oiseaus ke li plusor apelent tantalus ou hairon ; et ja soit ce ke ele pregne a euue sa viande, toutefois fait ele son nit en haut arbres. Et sa nature est tele ke maintenant k’ele aperçoit ke tempeste doit cheoir, ele vole en haut et s’enfuit en l’air amont la ou la tempeste n’a pooir de monter ; et par lui cognoissent mainte gent ke tempeste vient, lorsque il la voient monter contremont le ciel.
\chapterclose


\chapteropen
\chapter[{.I.CLIII. De anes et de owes}]{\textsc{.I.CLIII.} De anes et de owes}\phantomsection
\label{tresor\_1-153}

\chaptercont
\noindent Anes et owes, de tant comme eles sont plus blanches, sont millours et plus domesches ; car oës noires ki sont tachies d’autre colour sont estraites de campestres, et pour çou n’engendrent pas si larghement comme les blanches. Et sachiés que anes et owes ne poroient vivre sans erbe ne sans euue ; mais trop nuisent a terre gaignable, et mout enpirent totes semences dou bek et dou fiens. Et le tens k’il s’entrecouchent charnelment dure des kalendes de març jusques as tres granz jors de esté.\par
Et a la vois des owes puet on connoistre toutes les heures de la nuit et les vigiles. Et il n’est nus animaus au monde ki sente si bien homes comme elles font. Et a lor cri furent aperceu li françois, quant il voloient penre le capitaille de Rome, selonc ce que l’istore nous reconte.
\chapterclose


\chapteropen
\chapter[{.I.CLIIII. Des besenes}]{\textsc{.I.CLIIII.} Des besenes}\phantomsection
\label{tresor\_1-154}

\chaptercont
\noindent Besenes sont les mousches ki font le miel, ki naissent sans piés et sans eles, mais il les recovrent aprés lor naissance. Ces mousches gardent grant diligence a lor miel faire ; car de la cire qu’eles acueillent de diverses flours edefient par mervilleus engin maison et estages, dont chascune a son propre leu ou ele repaire tozjours sans changier. Et si ont roi et ost, et font batailles, et fuient la fumee, et s’afichent por le son des pieres et des timbres et de teus chose ki font grant tumulte.\par
Et si dient cil ki esprové l’ont, k’eles naissent de caroigne de buef, en ceste maniere : k’il batent mout la char d’un viel mort, et quant son sanc est porris si en naissent vermines ki puis devienent besenes ; autresi naissent escharbot de cheval, et fuse de mul, et guespes d’asne.\par
Et tant sachiés que en trestouz animal du monde solement les besenes ont en totes lor lignies totes choses communes, a ce que totes abitent dedens une maison, et issent dedens la marche d’un païs ; et l’oevre de chacune est commun as autres, et la viande commune, et toz usages et fruit et pomes sont commune de tous. Quoi plus ? la generation en est commune, et lor fiz commun ; car a ce que toutes sont chastes et virgenes, sans nule corruption de luxure font eles soudainement fius a grant fuison. Eles ordeinent lor peuple et maintienent lor communes et lor borgoiseries.\par
Eles eslisent lor roi, non mie par sort, ou il a plus de fortune ke en droit jugement ; mais celui a qui nature done signe de noblece, ki est grignor et plus beaus et de millour vie, est esleus a roi et sires des autres. Et ja soit il rois et graindres, il en est plus humles et de grant pité, nés son aguillon n’use il en vengance de nule chose. Et neporquant s’il en est rois, les autres sont toutes franches et ont delivre signourie ; mais la bone volenté ke nature lor done les fait amables et obeissans a lor signour, en tel maniere que nule n’en ist de maison devant çou que lor rois en isse et pregne la signorie de voler cele part ou lui plest.\par
Nés les noveles mouschetes ne s’osent poser devant ce que lor maistres soit assis la u il wet, puis s’assient environ lui, et ensivent diligeaument sa loi. Et quant aucune d’iaus fet contre la loi son signour, ele meisme en fait la vengance de soi ; car ele oste et brise son aguillon, selonc ce que li persant soloient faire ; car quant aucuns brisoit la loi, il n’atendoit le jugement le roi, ains s’ocioient aus meismes por la vengance de son trespassement.\par
Et en some sachiés que les besenes ayment lor roi a si grant coer et a tant de foi k’eles quident ke bien soit a morir por lui garder et deffendre. Et tant comme li rois est avec eles sains et haitiés, ne sevent muer foi ne sentence ; mais quant il est mors ou perdus, eles perdent foi et jugement, en tel maniere qu’eles perdent et brisent lor mel et gastent lor habitacles.\par
Et sachiés que li office sont entr’aus departis, de quele chose chascune doit servir ; car les unes porchacent la viande, les autres gardent le miel et la cire et les boches, et les autres consirent le muement dou tens et les aleures des nues ; les autres atirent la cire des flours, et les autres queillent la rosee par desous les floretes, ki puis devient mel coulant, et s’avale par ces pertuisos ki sont laiens.\par
Et ja soit ce que chascune s’efforce selonc son pooir a bien faire, por ce n’est pas envie entr’aus ne haine. Mais se aucuns lor fait mal, eles espandent une mauvaise amertume dedens le miel. Et volentiers se metent a la mort por vengance faire de ciaus qui lor ennuient poi ne grant.
\chapterclose


\chapteropen
\chapter[{.I.CLV. Des caladres}]{\textsc{.I.CLV.} Des caladres}\phantomsection
\label{tresor\_1-155}

\chaptercont
\noindent Caladres est uns oiseaus toz blans, et son pomon garist de l’oscurté des oils ; de qui la Bible commande, que nus n’en menguë pas.\par
Et sa nature est que quant il voit homes deshaitiés ki doivent morir de cele maladie, maintenant estort sa face, et ne le regarde point. Mais celui qui ne doit mie morir regarde ele seurement sans son viaire remuer.\par
Et dient li pluisor que par son regart reçoit ele en soi toutes maladies, et les porte en l’air amont, ou li feus est, qui consume toutes maladies.
\chapterclose


\chapteropen
\chapter[{.I.CLVI. Ici parle dou colomp}]{\textsc{.I.CLVI.} Ici parle dou colomp}\phantomsection
\label{tresor\_1-156}

\chaptercont
\noindent Colomps sont oisiaux domesche de maintes colors, et converssent entour les homes ; et n’ont neent d’amer si com les autres animaux ont prés dou foie. Et esmuevent luxure par baisier, et plorent en lieu de chant, et font niz en pertuis entre les pierres ou aucuns fluns soit voisins. Et quant il perdent la veue par veillesce ou par autre maladie, il la recovrent. Et vont granz torbes enssemble.\par
Et cil qui les ont en lor maison ont une pointure de colons, la plus bele que l’en puet portraire, devant les niz des colons, porce que il engendrent fiz a la semblance de la pointure, que il voient devant eax. Mais qui prent le lien ou la hart d’un home pendu, et en giete devant touz les pertuis, sachiez certainement que nus ne s’enfuera jamais de son gré.\par
Et se l’en lor done a mangier comin sovent, ou se l’on en oint sovent les eles de baume, il amenront leens grant torbe des autres. Et se l’en lor done orge cuit chaut, il engendreront faons et moteplieront a grant foison. Mais on doit metre rains de ronces en mains lieus dou colombier, por deffensse de male beste.\par
Et sachez que nos trovons en la Sainte Escripture \textsc{.iii.} colons, un de Noé, qui raporta l’olivier, l’autre de David, la tierce qui aparut au baptisme Jhesu Crist.
\chapterclose


\chapteropen
\chapter[{.I.CLVII. Dou corbel}]{\textsc{.I.CLVII.} Dou corbel}\phantomsection
\label{tresor\_1-157}

\chaptercont
\noindent Corbel est \textsc{.i.} oisel noir, qui tant se doute de ses petiz fiz, que il ne les norrist ne ne cuide que il ne soient sien, tant que il lor voit la noire plume ; lors les aime et paist diligaument. Il manjue charoigne, mais tout avant quiert l’oill, et par enqui endroit manjue la cervele .\par
Ce est li oysiaux qui ne revint pas a l’arche Noé, ou porce que il trova granz charoines, ou porce que il morut es eiues parfondes.
\chapterclose


\chapteropen
\chapter[{.I.CLVIII. De la cornaille}]{\textsc{.I.CLVIII.} De la cornaille}\phantomsection
\label{tresor\_1-158}

\chaptercont
\noindent Cornaille est \textsc{.i.} oisiau de longue vie, de qui li ancien dient que ele devine les choses que a homes doivent avenir, et les demostrent a celui par maintes ensseignes, que il puet bien aparcevoir se il en set la maistrie. Et a la foiz poons nos conoistre la pluie qui vient, quant ele ne fine de crier, et esbat sa voiz.\par
Et aime tant ses fiz que grant tens aprés ce que il sont issuz de son niz, ele les ensiut touz jors o tot le past, que ele lor done sovent et menu.
\chapterclose


\chapteropen
\chapter[{.I.CLVIIII. Dou coturnix}]{\textsc{.I.CLVIIII.} Dou coturnix}\phantomsection
\label{tresor\_1-159}

\chaptercont
\noindent Cortunix est \textsc{.i.} oisiau que li françois claiment greosche, porce que il fu premiers trovez en Grece. Et en esté s’en revont outremer grant torbe ensemble ; et porce que li ostors prent touzjors la premiere qui vient en terre, si eslisent lor chevetaine d’un autre qui est d’estrange ligniee, parce que li ostors set que prendre, et les autres s’en aillent quitement.\par
Et sachiez que luer bones viandes sont vermineuses semences, por quoi li ancien sage veerent que nus n’en manjast, car c’est l’animal ou monde seulement qui chiet par epilence, ausi com li hons fait. Il criement fort le vent de midi por la moistor, mais mult s’aseurent a celui de septemptrion, qui est sec et hysnel.
\chapterclose


\chapteropen
\chapter[{.I.CLX. Ici parle de la cygoine}]{\textsc{.I.CLX.} Ici parle de la cygoine}\phantomsection
\label{tresor\_1-160}

\chaptercont
\noindent Cygoine est \textsc{.i.} oisiau sans lengue ; et por ce dient les genz que ele ne chante pas, mais bat son bec et fait grand tumulte. Et est enemie au serpent, porquoi veerent li ancien que on les occeist.\par
Et au comencement dou premier tens revienent entre nos et font entor nos lor niz et lor faons, ou il metent si grant estude au guarder et au norrir, que toute sa plume li chiet de son ventre et par desouz lui, si que aucunes foiz n’ont eles nul poeir de voler, ains covient que lor fiz les norrissent et gardent autant comme il furent paüs par lor pairons, et que lor plumes soient recovrees.\par
Et quant l’esté decline et li tens comence a changier pour yver, eles s’assemblent a granz eschieles, et passent la mer, et s’en vont en Aise, en tel maniere que les corneilles vont touzjors devant come guierres et chevetaines. Et bien sachez que la derraine qui vient en Aise, ou lieu ou eles s’amassent, icele est desplumee et depeciee par les autres trop cruelment.\par
Et porce poons nos conoistre que oisiaux et bestes ont esperit d’aucune conoissance ; car il avint chose que \textsc{.i.} lombart de l’eveschié de Melan osta \textsc{.i.} oeff d’un niz d’une cygoine priveement, et si i mist \textsc{.i.} autre, qui estoit de corbel, en ce liu. Et quant vint li tens que li faon nasquirent, et le corbel comença a moustrer sa colour et son devisement, li masles s’en ala et mena tant de cygoines que ce fu merveilles a veoir. Et quant tuit orent regardé le noir oisel qui estoit entre les autres, il corurent sur la femele a mort.\par
Et en la riviere dou Nil naissent une maniere d’oisel qui sont ressemblable a cygoine, que l’en apele ibes, qui ne quierent se petit poisson non, ou oes de serpent, ou d’autres bestes morticines qui sont entour la riviere : car dedenz l’eiue n’oseroit ele porter ses piez, car ele ne set nouer.\par
Et quant ele sent aucune maladie ou troublement de son ventre, por les males viandes que ele manjue, maintenant s’en va en la mer et gorjoie de cele eiue a grant foison, puis met son bec parmi la derraine part et versse l’eiue dedenz son cors, et fait espurgier son buel de toutes ordures. Et dient li pluisor que Ypocras li granz fysiciens fist premiers la clistere par l’example de cel oisel.\par
Et sachez que Ovides li tres bons poetes, quant li empereres le mist em prison, fist un livre el quel il apeloit l’empereor par le non de celui oisel, car il ne savoit pensser plus orde creature.
\chapterclose


\chapteropen
\chapter[{.I.CLXI. Dou cygne}]{\textsc{.I.CLXI.} Dou cygne}\phantomsection
\label{tresor\_1-161}

\chaptercont
\noindent Cygnes est \textsc{.i.} oisiau touz blans de plume, mais sa char est noire. Et use as fluns ; et quant il noë parmi l’eiue, il porte touzjors la teste levee, et ja nule foiz ne la metra dedenz l’eiue ; por coi li maronier dient que c’est bon encontre a trover. Et sa voiz fait douz son, porce que son col est lonc et pleé\par
Et si dient li pluisor paisant que es montaignes de Iperboré en Grece, quant l’en chante et cytole, que granz torbes vienent entour lui por le delit dou chant.\par
Dont li pluisor dient que quant il doit morir, une des pennes de son chief est fichee en sa cervele, et donc aperçoit sa mort ; lors comence a charter si docement que merveille est a oïr, et einsi chantant define sa vie.
\chapterclose


\chapteropen
\chapter[{.I.CLXII. Dou fenix}]{\textsc{.I.CLXII.} Dou fenix}\phantomsection
\label{tresor\_1-162}

\chaptercont
\noindent Fenix est \textsc{.i.} oisiau en Arrabe, dont il n’i a plus en tout le monde ; et est bien si granz com un aigle. Mais ele a creste sonz la maissele d’une part et d’autre, et la plume de son col et iqui entour est reluisanz come fin or arrabien. Mais en aval jusques a la coue est de colour de porpre, et la coue de rose, selonc ce que li arrabien le tesmoignent, qui maintes foiz l’ont veu.\par
Et dient aucun que il vit \textsc{.vc.} et \textsc{.xl.} anz, et li autre dient que sa vie dure bien \textsc{.m.} anz et plus ; mais li pluisor dient que il enveillist en \textsc{.vc.} anz.\par
Et quant il a jusques la vescu, sa nature la semont et atise a sa mort. Et por avoir vie, ele s’en va as bons arbres savouroux et de bone odour, et en fait \textsc{.i.} moncel ou ele fait le feu esprendre, et puis entre dedenz tout droit contre le soleil levant. Et quant il est ars, celui jor meisme de sa cendre sort une vermine qui a vie. Au secont jor de sa naissance est faiz come petiz pocins. Au tierz jors est touz granz et creuz tant comme il doit, et vole et s’en vait en son lieu ou sa habitation est.\par
Et li auquant dient que ce feu est fait par le provoire d’une cyté qui a non Eliopolis, ou la fenix renaist, selonc ce que li contes devise ici devant.
\chapterclose


\chapteropen
\chapter[{.I.CLXIII. De la grue}]{\textsc{.I.CLXIII.} De la grue}\phantomsection
\label{tresor\_1-163}

\chaptercont
\noindent Grues sont oisiaux qui vont a eschieles en maniere de chevaliers qui vont a bataille. Et touz jors va l’une devant l’autre ausi come confanoniers et guierre des autres, et les moine et conduist et chastie de sa voiz. Et tuit li autre sivent celui et obeissent a sa loy. Et quant la chevetaines est enroee, et sa voiz est auques defaillie, ele n’a pas honte que une autre soit mise en son lieu, et ele vait par derrieres avuec les autres.\par
Et s’il avenist chose que aucune fust lassee et n’eust pooir d’aler avuec ses conpaignes, lors entrent toutes desouz li, et la portent sor elles tant que ele recuevre sa force.\par
Et sachez que quant eles doivent movoir por aler ou lieu qui est entre Carrabin et Crium, tout avant engorgent dou sablon, et si prent chascune a son pié une petite pierre por voler plus seurement contre la force dou vent ; puis volent contremont le ciel au plus haut que puënt, por miauz veoir le lieu que eles desirent. Et tant sachez que quant eles ont tant alé que eles aperçoivent que ont passé la moitié de la mer, maintenant delivrent lor piez des pierres que eles portent, selonc ce que li maronier le tesmoignent, qui maintes foiz ont veu les pierres cheoir sur aux et environ. Mais le sablon ne vomissent eles pas devant ce que eles soient asseures de lor habitations.\par
Et tout ausi com eles observent bonne garde et diligente en cheminant, la observent a lor herberge, et encore plus fort ; car ertre toutes la disime veille et garde les autres qui dorment. Dont il en y a de telx qui veillent, mais ne se muevent de lor lieu ; car totes foiz tient une pierre dedenz son pié, qui ne la laisse mie si endormir.\par
Les autres vont environ gardant et remirant que il n’avenist nul encombrier. Et quant les premieres gardes ont tant veillié com eles doivent, eles reposent et dorment, et autres vienent a la gaite selonc l’ordre de leur loy .\par
Et quant il aperçoivent chose ou il y a perill, maintenant crient et font esveillier les autres por eschaper a sauveté. Et sachez que a la colour puet l’on apercevoir son aage, car eles nercisent por veillesce.
\chapterclose


\chapteropen
\chapter[{.I.CLXIIII. De la huppe}]{\textsc{.I.CLXIIII.} De la huppe}\phantomsection
\label{tresor\_1-164}

\chaptercont
\noindent Huppe est \textsc{.i.} oisiau qui a une creste sur son chief ; et manjue fyens et choses puanz : porce est sa alene mauvaise et porrie.\par
Mais tant font par lor nature que quant li fiz voit son pere enveillir, et que il est ja griés et pesanz, et que sa veue est auques obscurcie, il le desplument tout dedenz lor niz, et oignent lors ses yauz, et puis le paissent et norrissent et eschaufent souz leurs eles, tant que sa plume est renovelee et que il puet aler et venir ou il vuet seurement.
\chapterclose


\chapteropen
\chapter[{.I.CLXV. De l’arondele}]{\textsc{.I.CLXV.} De l’arondele}\phantomsection
\label{tresor\_1-165}

\chaptercont
\noindent Arondelle est \textsc{.i.} petit oiselet ; mais ja ne volera droite voie, ains vole a voutes et tours diverssement, et prent sa viande touz jors en volant, non pas en estant ; et si n’est la proie as autres oiseaus chaceor, mais tozfois por sa seurté habite entre les genz et fait son niz es maisons, dedenz ou desouz la coverture, non pas dehors .\par
Et dient li pluisor que cist oysiaux devine, car il deguerpist les maisons qui doivent fondre. Et fait son niz de boue et de festuz ; et por ce que ele ne puet mie porter la boue o ses piez, si baigne ses eles en tel maniere que la poudre se joint as eles mouliees et devient boue, dom ele ferme son hedefice.\par
Et quant ses fiz perdent la veue por aucune ochoison, il aportent une herbe que l’en apele chelidoine, qui les garist et luer rent la veue, si com li pluisor le tesmoignent, qui esprové l’ont aucunes foiz.\par
Mais l’on doit mult garder ses yauz dou fiens de l’arondele, car l’on trueve en la Bible que Thobie li granz en perdi la veue.
\chapterclose


\chapteropen
\chapter[{.I.CLXVI. Dou pelican}]{\textsc{.I.CLXVI.} Dou pelican}\phantomsection
\label{tresor\_1-166}

\chaptercont
\noindent Pelican est \textsc{.i.} oisel en Egypte, de cui li ancien dient que luer faons fierent o les eles luer pere enmi le visage, por quoi il se corroucent en tel maniere que il les occient. Et quant la mere les voit tuez, ele prore et fait grandisme color \textsc{.iii.} jors, tant que en la fin ele naffre ses costez a son bec et fait le sanc espandre sur ses fiz, tant que por ochoison dou sanc resordent et tornent en vie.\par
Mais aucunes genz dient que il naissent ausi come sans vie, et lor pere et lor mere les garissent de lor sanc. Mais coment que il soit, sainte yglise le tesmoigne bien, la ou Nostre Sires dist, je suis venuz de pelican par semblance.\par
Et sachiez que pelicans sont de \textsc{.ii.} manieres, uns de riviere, qui manjue poisson, et \textsc{.i.} autre qui est champestre et manjue serpenz et lisardes et autres bestes venimouses.
\chapterclose


\chapteropen
\chapter[{.I.CLXVII. Des perdrix}]{\textsc{.I.CLXVII.} Des perdrix}\phantomsection
\label{tresor\_1-167}

\chaptercont
\noindent Pertrix est uns oiseaus ki sovent est quis en proie et en venison por la bonté de sa char. Mais mout est trecheresse et luxurieuse ; car por la chalor de sa luxure stentrecombatent de lor femeles et a le fois oublient la cognoissance de nature, en tel maniere que li malles gist avec le malle. Et si dient maintes gens que quant la femele a chaude volenté, k’ele conchoit dou vent solement, ki le fiert devers le malle.\par
De ses baras dient k’ele emble les oes as autres pertrix, et si les met avec les siens ; mais quant li pertrisot sont nés, et il oënt la vois de lor droite mere, maintenant s’en vont a li et deguerpissent lor fausse mere. Et sachiés ke pertrix garnissent lor nit d’espines et de petites folles, et coevrent lor oes de poudre, et vont et revienent a lor niz priveement, et aucune fois tresporte la mere ses fius d’un leu en autre por engignier son malle. Et quant on vient prés de son nit, ele fait samblant k’ele ne puisse voler, pour çou k’ele puisse home esloignier de son repaire mieus et plus coiement.
\chapterclose


\chapteropen
\chapter[{.I.CLXVIII. De papegal}]{\textsc{.I.CLXVIII.} De papegal}\phantomsection
\label{tresor\_1-168}

\chaptercont
\noindent Papegal est uns oiseaus vers, mais son bec et ses pié si sont rouges ; et a plus grant langue et plus lee ke nus oiseaus, par quoi il dit paroles articuleres en samblance d’ome, se l’en li ensegne de sa jovenece, dedens le secont an de son aage ; car de lors en avant est il durs et oublieus, en tel maniere k’il n’aprent chose ke l’en li mostre. Et si le doit on chastier a une petite verge de fier.\par
Et si dient li indien que cist oiseaus ne naist ailleurs k’en Inde, et ke de lor nature sevent il saluer selonc l’usage de cele terre. Et cil qui ont \textsc{.v.} dois sont plus nobles, mais cil ki en ont \textsc{.iii.} sont de vilain linage. Et toute lor force est ou bec et ou chief, ou il rechoit plus volentiers toutes cheoites et ferues quant il ne les puet eschiver.
\chapterclose


\chapteropen
\chapter[{.I.CLXVIIII. Des paons}]{\textsc{.I.CLXVIIII.} Des paons}\phantomsection
\label{tresor\_1-169}

\chaptercont
\noindent Paons est uns biaus oiseaus simples en sa aleure. Mais il a chief serpentin, et vois de diauble, et pis de saphir, et riche coue de diverse coulour, ou il se delite mervilleusement, tant que la ou il voit les homes ki remirent sa beauté, il drece sa coue contremont pour avoir les los des homes, et descuevre le laide part deriere, k’il lor moustre vilainement. Molt desprise le laidece de ses piés. Et sa char est dure fierement et de grant odour.
\chapterclose


\chapteropen
\chapter[{.I.CLXX. Des tortreles}]{\textsc{.I.CLXX.} Des tortreles}\phantomsection
\label{tresor\_1-170}

\chaptercont
\noindent Torterele est uns oiseaus de grant chasteté, ki abite volentiers loins de gens ; et tout yvier maint es pertuis des arbres, por la plume ki li est cheoite. Son nit coevre des fueilles d’esquille, por le leu ki ne touche ses faons, car leu n’ose aler la u cele erbe soit. Et sachiés que tortrele est si amable vers son compaignon ke s’il est perdus par aucune maniere, ele ne quiert jamés autre mari, et garde sa foi ou par vertu de chasteté ou por çou k’ele quide que son mari reviegne.
\chapterclose


\chapteropen
\chapter[{.I.CLXXI. Des voutoirs}]{\textsc{.I.CLXXI.} Des voutoirs}\phantomsection
\label{tresor\_1-171}

\chaptercont
\noindent Voutart si est uns grans oiseaus samblables a aigle ki connoist odour dou nés plus long ke nul animal dou monde neis d’outre la mer flaire il la charoigne. Et si dient cil ki l’ont a coustume k’il sivent les ost des homes la ou il doit avoir grant fuison de charoigne ; et ensi devinent ke en celi ost sera grant occision d’omes ou de bestes.\par
Et si dient li plusour que entr’aus n’a nule conjunction de malle et de femele, et sans gesir engendrent et font fiz ki vivent longhement, si ke a paines definent en \textsc{.c.} ans. Et plus volentiers vont par terre sans voler por lor pesantour ; et ne menguë de nule charoigne, s’il ne le lieve devant desor terre en haut.
\chapterclose


\chapteropen
\chapter[{.I.CLXXII. Des ostrisses}]{\textsc{.I.CLXXII.} Des ostrisses}\phantomsection
\label{tresor\_1-172}

\chaptercont
\noindent Ostrisses est une grant beste ki a eles et plumes a samblance d’oisel, et piés comme chamel. Et ne vole pas, ains est griés et pesans par sa complexion, ki le fait si oublieuse malement, k’il ne li sovient des choses passees.\par
Por ce li avient autresi comme par amonestement de nature, ke en esté entor le mois de jung, quant il li covient penser de sa generation, il esgarde une estoile ki a non Virgile. Et quant ele comence a lever, depose ses oes et les coevre de sablon, et s’en vet porchacier son affere, et oublie ses oes en tel maniere ke jamés ne l’en sovient ne po ne grant. Mais la chalour dou soleil et li atemprement du tens acomplist son office et escaufe ce ke la mere devoit eschaufer, tant ke ses faons naissent si grans ke maintenant porchacent lor besoingnes.\par
Et nan porquant lor pere, quant il les truevent, la u il les devroit norrir et ensegnier si lor anuie et fet tant de cruauté come il puet. Et sachiés ke contre la peresce ke nature lor dona, si li fist ele \textsc{.ii.} ongles et eles dont il se bat et fiert il meismes quant il vieut aler, autresi comme se ce fussent \textsc{.ii.} espourons.\par
Et sachiés que son estomak, c’est la gorge, ou il retient son pas. Et est de si trescaude nature k’il englotist le fier et l’enduist et consomme dedens soi. Et son gras est mout profitables a totes dolours de menbres.
\chapterclose


\chapteropen
\chapter[{.I.CLXXIII. Del coq}]{\textsc{.I.CLXXIII.} Del coq}\phantomsection
\label{tresor\_1-173}

\chaptercont
\noindent Cok est un oisel domesche ki maint entres les homes tousjours, et par sa vois moustre les heures dou jour et de la nuit et les muemens dou tens. Ja soit ce que dedens la nuit il chante plus haut et plus orguilleusement mais vers le jour chante plus cler et plus souef, mais il bat son cors et ses eles avant k’il chante.\par
C’est li oiseaus au monde solement a qui l’en oste les coillons et en fait l’en chapon, ki mout sont sains et bons en esté. Mais geline est millour en yvier por mangier, car en esté sont toutes gelines covesses, et beent a lor oes et a ses poucins garder, et por la color d’aus se deplume et enmaladist et enviellist durement. Pour ce doit li sires de sa maison eslire gelines noires ou grises et eschivre les blanches ; et si lor doit doner a mengier orge demi cuit, ki les fait engendrer et porter oes assés grans et larghement. Et quant yviers passe, ke li sires voet avoir pouchins, il doit ensigruer sa maisnie que la nombre des oes soit nompers et que il soient mis a lune croissant, c’est a dire dou novime jour jusc’au quinsime jour de la lune.\par
Mais si ce taist li contes a parler des oiseaus et de lor nature pour dire \textsc{.i.} po de la nature des bestes, et premierement dou lion ki en est rois et sires.
\chapterclose


\chapteropen
\chapter[{.I.CLXXIIII. Del lion}]{\textsc{.I.CLXXIIII.} Del lion}\phantomsection
\label{tresor\_1-174}

\chaptercont
\noindent Lion est apelés selonc la langue as grizois, ki tant vaut a dire comme rois en nostre parleure. Car lions est apelés rois des bestes, pour ce que la ou il crie toutes bestes s’enfuient comme se la mors les cachast ; et la u il fait cercle de sa coue, nule beste n’ose passer par enki.\par
Et nanporquant lyon sont de \textsc{.iii.} manieres, car li \textsc{.i.} sont brief et ont les crins crespes et sont sans bataille, li autre sont lons et grans et ont les crins simples et sont de mervilleuse fierté. Et lor corage sont demoustré par le front et par sa coue, et sa force est en son pis, et sa fermeté est en son chief .\par
Et ja soit ce k’il est redoutés de tous animaus, neporquant il crient le blanc cok et la tumulte des roës, et feus li fet grant paour. Et d’autre part li escorpions li fet mal trop grant se il le fiert, nés le venin dou serpent l’ocist.\par
Car cil ki ne soufri pas que nule chose fust sans contraire volt bien ke li lions ki est orgilleus et fors sor toutes choses, et ke par sa grant fierté ensit proie tozjours, eust des choses ki l’enpechent contre sa cruauté, dont il n’a pooir k’il s’en deffende ; et entre ce est il si malades autresi con de fievre les \textsc{.iii.} jours de la sesmaine, ki mout amenuise son orguel. Et nanporquant nature li ensegne a mangier le siguë, ki le garist de sa maladie.\par
Et ja soit lions de haut corage et de si fiere nature comme li contes devise ci devant, totes voies ayme il home mervilleusement, et volentiers maint avec lui, et ne sera ja courouchiés a home se il ne li fet mal premierement. Mais merveilles est de sa pitié, ke la ou il est plus courouciés et plus plains d’ire et de mal talent contre lui, lors li pardonne plus tost se li hom se giete a la terre et fait samblant de crier merci. A paine se courece contre feme, ne petit enfant ne touche, se por grant desirier de mangier non.\par
Li ordene de sa vie est a mangier l’un jour et boivre l’autre ; pour ce que lyon est de si grant viande que sovent ne le puet il quire en son estomac, donc la bouche li put trop malement. Mais quant il aperçoit que le remanant de sa viande n’est pas quite dedens la forcele et k’ele li fet anui, il le prent a ses ongles et le oste de sa gorge. Et quant il a mout mengié que son ventre est bien soolés, et li veneour le cachent, il revomist tout pour delivrer soi de la pesantour de son cors. Autresi fet il sovent, ke quant il menguë trop, por sa santé recovrer, k’il ne manguë l’autre jor ne poi ne grant, et si ne touche char de beste ki fust morte le jour devant.\par
Et sachiés ke lyon gisent envers, li malle avec le femele, autresi come les cers et comme cameus et olifans et unicors et tygres, et si engendrent \textsc{.v.} fiz a la premiere porture ; mais la force k’il ont es ongles et es dens et en tout le cors empire molt la matrice sa mere, tant come il sont dedens. Et a lor naistre issent en tel maniere que a la seconde fois li leus ou la mere reçoit la semence son malle n’a pooir que il engendrent que \textsc{.iiii.}, et a la tierce fois trois, et a la quarte fois \textsc{.ii.}, et a la quinte \textsc{.i.} De lors en avant est cil leus si gastes k’il ne conçoit jamés tote sa vie.\par
Pour ce dient li plusour ke pour la trés grant dolour ki est en lor naissance, li lionceaus sont si esbahi k’il en gisent en pasmison \textsc{.iii.} jors, autresi come s’il fussent sans vie ; tant que lor peres vient au chief de \textsc{.iii.} jors, ki les escrie si fort de sa vois que li fiz s’esdrecent et ensivent lor nature.\par
L’autre maniere de lyon sont engendré d’une beste ki a non parde, et teus lyons sont sans crins et sans noblece et sont conté entre les autres vils bestes.\par
Mais totes manieres de lyons tienent les oils ouvers quant il dorment, et ou k’il aillent tousjours cuevrent les traches de lor piés a l’engien de la coue.\par
Et quant il cachent lors courent et sallent fort et isnel ; mais quant il sont chacié il n’ont pooir de salir. Et lor ongles gardent en tel maniere k’il ne les portent se enverses non. Et lor aage sont conneu as fautes de lor dens.
\chapterclose


\chapteropen
\chapter[{.I.CLXXV. De l’antelu}]{\textsc{.I.CLXXV.} De l’antelu}\phantomsection
\label{tresor\_1-175}

\chaptercont
\noindent Antelu est une trés fiere beste que nus hom ne puet consivre ne penre par aucun engien, car ses cornes sont grans et a maniere de sie, ki taillent et brisent tous las, et trenchent les grans arbres. Mais il avient aucunefois k’il vet boivre au fleuve d’Eufrates, ou il a buissons et arbrissiaus loins et deliés, ki se ploient et crollent en diverses manieres, si ke por la foiblece d’iaus il ne les puet trenchier, si comme il fait des fors, ki se tienent fermement contre ses cornes, et por ce les fiert il et se combat a aus. Et la ou il les quide taillier et porter a la terre, il envolope son chief en ces bastonnés, ki le lient et tienent si fermement k’il n’a pooir k’il s’en aille, ains crie et dolousist. Et quant il pense avoir aide, li hom vient a l’ensegne de sa vois, et le fiert tant k’il l’ocist.
\chapterclose


\chapteropen
\chapter[{.I.CLXXVI. Des asnes}]{\textsc{.I.CLXXVI.} Des asnes}\phantomsection
\label{tresor\_1-176}

\chaptercont
\noindent Asne sont de \textsc{.ii.} maniere, domesches et sauvages ; mais el domesche n’a chose ki face a ramentevoir en conte, se ce non que de sa negligence et de sa soutie dit on mains proverbes ki donent grans essamples as homes de bien fere.\par
L’autre ki sont sauvage troeve on en Aufrike, et sont si fiers que hom ne les puet donter, et si en soufist uns malles a plusours femeles. Cil est si jalous que quant il aperçoit ke aucun de ses poulains sont malles, maintenant lor court sus et lor trence lor coillons, se la mere ne s’en prent garde, si k’ele le tiegne en repost sauvement.\par
Et sachiés que cist asne sauvage, que l’en apele onagre, a chascune heure dou jour et de la nuit crie une fois, si ke l’en i poroit bien connoistre les eures et savoir certainement quant la nuit est igal au jour u quant il sont desiguel.
\chapterclose


\chapteropen
\chapter[{.I.CLXXVII. Des bués}]{\textsc{.I.CLXXVII.} Des bués}\phantomsection
\label{tresor\_1-177}

\chaptercont
\noindent Buef sont de maintes manieres, \textsc{.i.} ki naist es parties d’Aise et est apelés bonacon, pour çou k’il a crins comme cheval, et ses cornes sont si grans et vautices entor sa teste que nus hom ne les puet ferir se sus les cornes non. Et quant hons u autre beste le chace, il deslie son ventre et giete par deriere \textsc{.i.} fiens si puant et ardant k’il brise çou k’il touche.\par
Un autre buef est en Inde, ki n’a c’une corne sans plus, et ses ongles sont soudes et enterines comme de cheval. Autre buef sauvage naissent en Alemaigne, ki ont grans cors et bons por soner et por porter vin. Li autres s¢nt apelés bufles, ki dorment el parfont des grans fleuves, et vont autresi bien parmi le fons des euues aval comme li buef par terre. Més li buef domesche ki cultivent la terre sont dous et pitieus et ayment lor compaignons tenrement et de bone foi, selonc ce k’il demostrent as cris k’il font sovent et menu quant lor compaignon sont perdut.\par
Et pour çou k’il sont mout pourfitable a gaaing de terre doit li sires en ses maisons bués joenes eslire, ki set toz membres grans et quarrés, et grant oreilles, et le front large et crespe, oils et levres noires, cornes noires non mie voutices comme lune, les narilles overtes et grans, gorge grant et pleniere et pendant jusques as genous, larges pis et grans espaules, grandisme ventre et lonc et lé, l’eschine droite et plaine, jambes soudes et dures et nerveuses, petites ongles, coue grant et bien pelouse, et tous les pols dou cors briés et espés, meismement de rouge coulour.\par
Mais la vache doit estre mout haute et longue, de grandesime cors, ki set le front en haut, et les oils grans et noris, beles cornes et noires, oreilles pelouses, et tout le poil dou cors brief et espés, gorge et coue longue et grandesime, et petites ongles, jambes noires et briés, ki soit en aage de \textsc{.iii.} ans. Car de lors dusc’au disime an porte fiz plus pourfitables que devant ou que aprés. De ceste beste dient li grezois, que se te vieus engendrer malles, tu dois liier le senestre coillon dou torel quant il gist avec sa femele, et se tu voudras engendrer vache, tu lieras le diestre.
\chapterclose


\chapteropen
\chapter[{.I.CLXXVIII. Des brebis}]{\textsc{.I.CLXXVIII.} Des brebis}\phantomsection
\label{tresor\_1-178}

\chaptercont
\noindent Berbis est une simple beste, plaine de pais et de paour, ki reconnoist son filz, et ses fiz li, entre grant torbe d’oeilles, au baler solement et a la cognoissance de sa vois.\par
Et por çou ke ce sont bestes de grant proufit, a ce k’il donnent lait et formage et char a mangier, et laine pour vestir, et la pel por maint furniment d’ome, est il bien avenable que li sires de l’ostel eslise premierement monton haut et legier et viste, de grandesime cors, bien covert de blanche laine et espesse fierement, o longhe coue, o grans coillons, o lé front, et de bon aage, car il puet bien engendrer dusc’a \textsc{.viii.} ans.\par
Mais molt doit bien garder li sires sa laine, car selonc çou que lor laine est tachie, tot autretel engendrent il fiz et filles tachiés, por ce que de blanc mouton puet bien naistre fiz d’autre coulor, mais de noir ne puet nestre blans fiz.\par
Et sor ce dient li plusour que la vois du noir est devisee de celui dou blanc, en tel maniere que li plusor le sevent bien conoistre, a ce que li noirs dist mé, et li autres dist bee. Por ce doit on avoir berbis grans, ki ait laine blanche, et molle et gentil, et doit estre \textsc{.ii.} ans jusques a \textsc{.v.} Au septime an faut ele et ne puet engendrer.\par
Dont Aristotles dist, que se au tens que brebis doit assambler as montons, se l’en les garde et les paist on vers septentrion contre le vent ki vient de cele partie, k’eles engendrent malles ; mais devers oistre engendrent femeles.
\chapterclose


\chapteropen
\chapter[{.I.CLXXVIIII. Des bellotes}]{\textsc{.I.CLXXVIIII.} Des bellotes}\phantomsection
\label{tresor\_1-179}

\chaptercont
\noindent Bellote est une petite beste plus longue k’une soris, et ensiut soris et coluevres ; mais quant ele se combat a la coluevre, ele se torne sovent au fenoil et le manguë por la paour du venin, puis revient a sa bataille.\par
Et sachiés que belotes sont de \textsc{.ii.} manieres, une ki abite es maisons, et une autre champestre ; mais chacune conchoit par les oreilles et enfante par le bouche, selonc ce que aucune gent tesmongnent, mais li plusor dient que ce est chose fausse.\par
Mais coment k’il soit, sovent remue ses fiz d’un leu en autre, pour çou ke nus ne s’en aperçoive. Et s’ele les trueve mors, maintes gens dient k’ele les fait resusciter, mais ne sevent dire coment ne par quel medecine.
\chapterclose


\chapteropen
\chapter[{.I.CLXXX. De chameus}]{\textsc{.I.CLXXX.} De chameus}\phantomsection
\label{tresor\_1-180}

\chaptercont
\noindent Chameus sont de \textsc{.ii.} manieres, un ki sont arabiiens, et ont \textsc{.ii.} boches en sus l’eschine, li autre sont barriens, ki n’ont c’une boche et sont trés fors, et lor piés ne puent estre gastés pour cheminer. Li grant chamel sont bons por porter grandesimes somes. Li autre petit ki sont apelé dromadaire sont bons por aler tost et longuement, mais li \textsc{.i.} et li autre sont anemi as chevaus, et molt enpirent por assambler as femeles, si k’il covient grans estuves a rescaufer les de dras et de feu aprés le conjungement.\par
Et tant sachiés de sa propre nature, ke c’est li animaus dou monde ki mieus connoist sa mere, en tel guise k’il est de si gentil maniere k’il ne toucheroit jamés a li charnelment, si con font li autre animal, ki de ce n’ont nule ramenbrance.\par
Et si suefrent bien \textsc{.iii.} jours soif, mais quant il sont a l’euue il boivent autant comme s’il eussent beu es jors alés ; et il quident k’il lor besoigne por les jors ki sont a venir. Et se li euue fust clere, il la torblent a lor piés, car autre n’en buveroient il goute. Et sachiés que chameus vivent bien cent ans en lor païs, mais le muement de l’air les fet enmaladir et devier plus tost k’il ne deussent.
\chapterclose


\chapteropen
\chapter[{.I.CLXXXI. Del castore}]{\textsc{.I.CLXXXI.} Del castore}\phantomsection
\label{tresor\_1-181}

\chaptercont
\noindent Castore est une beste ki converse a la mer de Ponto ; pour ce est ele apelee chien pontique, car il est aukes resamblables a chien.\par
Lor coillons sont mout chaut et proufitables en medecines, por ce l’ensivent li païsant et le chacent por avoir ses coillons. Més nature, ki ensegne a toz ses proprietés, lor fait connoistre la propre ochoison por quoi on les chace ; car la u il aperçoivent k’il ne s’en puent aler, il meisme trenche as dens sa coille, et le giete devant les veneours, et ensi raambrist son cors pour icele partie ki millour est ; et de lors en avant se l’en le sit, il descovre ses quisses et moustre bien k’il est escoilliés.
\chapterclose


\chapteropen
\chapter[{.I.CLXXXII. Des chevris et des bisches}]{\textsc{.I.CLXXXII.} Des chevris et des bisches}\phantomsection
\label{tresor\_1-182}

\chaptercont
\noindent Chevrel et bisches sont une maniere de bestes ki sont de si bone connoissance que de loins cognoissent les gens ki vienent, s’il sont veneour u non. Autresi cognoissent il les bones herbes et les mauvaises, por solement le veoir. Et tousjors vont paissant de haut en haut. Et tant sachiés que se l’en le fiert ou navre en aucune maniere, maintenant court il a une herbe ki est apelee dyptame, et la touce la u ele le trueve, et ensi garist de ses plaies.
\chapterclose


\chapteropen
\chapter[{.I.CLXXXIII. Des cers}]{\textsc{.I.CLXXXIII.} Des cers}\phantomsection
\label{tresor\_1-183}

\chaptercont
\noindent Cerf est une beste sauvage, de quoi li ancien dient k’il n’escaufe de fievre a nul jour de sa vie ; et por ce sont aucunes gens ki manguënt de sa char chascun jor avant disner, et sont asseur de fievre tant comme il vivent ; et certes il vaut a ce s’il fust tués d’un cop sans plus. Neis en son cuer a \textsc{.i.} os ki mout vaut en medechine, selonc ce que li fisicien tesmoignent.\par
Li cerf meisme nous ensegnierent le dyptame, c’est une herbe k’il manguënt, la ou l’en les fiert. Car la vertu de cele herbe lor oste la saiete dou cors et le garist de la plaie.\par
Et ja soit ce que cerf soient generalment anemis as serpens, toutesvoies lor valent il a grans medecines, et ore orés coment. Il vait au pertuis du serpent a toute la bouche plaine d’euue et le boute dedens. Et quant il a ce fait, il atret a soi par les espiremens de son nés et de sa bouche, tant k’il le fet hors issir maugré sien, lors le fiert et ocist des piés. Mais quant li cers veut deposer sa viellece ou maladie k’il ait, il menguë la serpente, et pour la paour dou venim cort a la fontaine et boit assés : en ceste maniere mue le poil et les cornes et giete toute viellece.\par
Por ce vivent cerf longhement, selonc ce que Alixandres esprova quant il fist penre maint cerf, et lor fist metre a chascun \textsc{.i.} cercle d’or u d’argent en son col, ki puis furent trové en bone vie lonctens aprés, plus de cent ans.\par
Et ce sachiés que quant cherf tienent les oreilles enclines, k’il n’oënt goute, mais quant il les drechent amont il oënt aguement. Et quant il passent uns grans fleuves, cil deriere porte tousjours son chief sour le cors et sor la crupe a celui devant, et ensi s’entresostienent en tel maniere k’il ne se travaillent se mout petit non.\par
Et ja soit ce que li malle soient esmeu de fiere luxure quant li tans en est, neporquant la femele ne conchoit pas jusques a tant ke une estoile lieve, ki est apelee Arcton. Et quant li tens est de lor fiz, ki doivent nestre, lor couverie ne sera ja se en leu repost non, la ou li bois est parfons et espés, ou il ensegnent lor fiz a corre et a fuir et a aler par roches et par montaignes.\par
Et lor nature est que la ou il aperçoivent le glatissement des chiens ki les enchaucent, il adrecent lor aleure a l’autre vent, pour ce ke li odors d’aus ne soit pas porté vers les chiens. Et nanporquant la u li veneour ki le chacent le tienent si cort k’il se despere et ne quide ke jamés puisse son cors garantir, il recule ariere courant et batant cele part u li veneours vient, por morir devant lui plus legierement.
\chapterclose


\chapteropen
\chapter[{.I.CLXXXIIII. Des chiens}]{\textsc{.I.CLXXXIIII.} Des chiens}\phantomsection
\label{tresor\_1-184}

\chaptercont
\noindent Chiens naist non veans mais puis recuevre sa veue selonc l’ordre de sa nature. Et ja soit ce ke li chiens ayme mout home plus ke beste dou monde generalment, il ne conoissent pas estrange gent, se ceaus non entor qui il habitent ; et si entent son nom, et reconnoist la vois son mestre.\par
Ses plaies garist il a sa langhe. Sovent vomist son past, et puis le remanguë. Et quant il porte char ou autre chose en sa goule, et il passe aucun fleuve, maintenant k’il voit l’ombre de sa char en l’euue, il laisse çou k’il porte por prendre ce ki est noiens.\par
Et sachiés que de chien et de leu quant il assamblent naist une maniere de chiens ki mout sont fiers, més li trés fier naissent por assamblement de chien et de tigre, ki sont si isnel et si aspre que çou est droite diaublie.\par
Li autre chien de domesce nascion sont de maintes manieres ; car il a petis chiens gouços ki sont bons a garder maisons, et si i a trés petis camusos pour garder cambres et les lis as dames. Et s’il sont engendrés de petis patrons, l’en les puet en lor joenece norrir de mout petit de viande, ou en \textsc{.i.} petit pot, si k’il seront si petis et si briés ke merveille. Et si doit on trere lor oreilles souvent et menu, car lors sont plus gens quant eles vont pendant contreval la terre. Li autre sont brachés a oreilles pendans, ki conoissent l’odour des bestes et des oiseaus, pour ce sont il bons a la chace ; et ki en ce delite son corage, si les doit mout amer et garder de faus assamblement, car chien n’ont pas la cognoissance dou nés se par linage non. Pour ce dist li proverbes as vilains, que chiens chasse par nature.\par
Li autre sont loimiers, et sont apelés segus, pour çou k’il ensivent la beste jusc’a la fin ; dont il en i a teus que ce ke l’en li aprent en sa joenece, a ce se tient touzjors, si ke li \textsc{.i.} chacent cers et autres bestes champestres, li autres chacent loutres et lievres et autres bestes ki conversent en euues. Li autre sont levrier, ki sont plus legier et isnel a courre et a prendre beste a sa bouche. Li autre sont mastin grant et gros et de mout grant force, ki chacent ours et senglers et lous et toutes grans bestes, neis contre home se combatent il fierement.\par
Por çou trovons nous en ancienes istores que uns rois avoit esté pris par ses enemis, et si chien assamblerent grandesime compaignie d’autres chiens, et se combatirent si fort contre ciaus qui le roi detenoient, ke il le rescousent a fine force.\par
Et si n’a mie gramment ke en Champaigne s’asamblerent toz les chiens dou païs en \textsc{.i.} leu, ou il s’entrecombatirent si asprement que a la fin n’en eschapa un seul, que tout iten fussent mort en la place de terre ou il estoient.\par
Et pour ce que li contes devise ci devant ke chien ayment home plus ke best ki soit, si vous en dirai aucune chose de ce ke no maistre recontent en lor livres. Sachiés que quant Jase Lice fu ocis, son chien ne volt puis mengier ne poi ne grant, ains morut a dolour. Et la u li rois Lisemache fu mis en feu por son pechié ke fet avoit, son chiens se geta dedens avec son signor et se laissa cremer avec lui. Et uns autres chiens entra en prison avec son mestre, et puis quant on le geta el fleuve de Toivre ki cort a Rome, li chiens se geta aprés, et porta la charoigne sor l’euue tant com il pot. Ces et maintes autres natures sont trovees ke chien ont, mais tant come li contes en dist ci ilueques en puet bien soufire, pour abregier cest livre.
\chapterclose


\chapteropen
\chapter[{.I.CLXXXV. De camelion}]{\textsc{.I.CLXXXV.} De camelion}\phantomsection
\label{tresor\_1-185}

\chaptercont
\noindent Camelion est un beste ki naist en Ayse, et si en i a grant plenté. Et sa face est samblable a lizarde, mais ses jambes sont droites et longues, et les ongles fieres et aguës, et coue grant et voutice, et vet autresi lentement comme tortue, et sa pel est dure comme de cocodril ; et ses oils fiers et durement encavés dedens la teste, et ne les muet pas ça et la, por ce ne voit il en travers, ains regarde tot droit devant soi. Et sa nature est de fiere merveille, car il ne manguë ne ne boit chose del monde, ains vit solement de l’air k’il atire.\par
Et sa coulour est si muable que tot maintenant k’il atouche a aucune chose, prent sa coulor et devient d’autretel taint, se ce n’est rouge ou blanc, car ce sont \textsc{.ii.} coulours k’il ne puet faindre. Et sachiés que son cors est sans char et sans sanc, se ce n’est ou cuer ou il en a \textsc{.i.} petit. En yvier maint repost, et en esté vient et uns oiseaus l’ocist ki a non corax ; mais s’il manguë de lui il le covient a morir, se fueille de lorier ne l’en delivre.
\chapterclose


\chapteropen
\chapter[{.I.CLXXXVI. Des chevaus}]{\textsc{.I.CLXXXVI.} Des chevaus}\phantomsection
\label{tresor\_1-186}

\chaptercont
\noindent Chevals est une beste de mout grant cognoissance ; car a ce k’il repaire tozjors entre les homes, lor done aguete de sen et de raison, tant qu’il connoissent lor signor, et sovent muent il meurs et abis quant il muent signor.\par
Il flaire la bataille, il se courece, et est liés au son des buisines. Il sont liés quant il ont victore, et sont dolant quant il perdent. Et bien puet on aperçoivre se la bataille doit estre gaaignie ou non, au samblant que li cheval font de joie ou de courouch. Dont il en i a teus ki connoissent bien les enemis lor maistres, car il les mordent et fierent trop angoisseusement.\par
Et teus i a ki ne portent se lor droit sire non, selonc ce ke fist li chevals Julle Cesar et Bucifalas son signor Alixandre le grant, ki premiers se laissa donter et chevaucier comme nice beste ; mais aprés ce que li rois i monta, il ne daigna puis soufrir que ame du monde l’atouchast pour monter. Et sachiés que Bucifalas avoit chief de tor et mout fier esgart, et si avoit \textsc{.ii.} boches autresi comme \textsc{.ii.} cornes.\par
Et le cheval Cracarci le duc de Galatas, que quant ses sires fu mors et que li rois Anthiocus i monta por combatre, li chevaus courut au deval dou grant tertre, et se trebucha soi en tel maniere k’il ocist soi et son chevaucheour.\par
Et quant li rois de Sithe se combati contre son enemi cors a cors, et il fu ocis a la bataille, et li autres le voloit despoillier et coper lui la teste, li chevaus le deffendi vistement et le garda jusques a sa mort, car il ne volt onques puis mangier.\par
Car il est chose provee de mains chevaus, ki plorent et gietent larmes pour la mort lor signour, et si n’est nule autre beste ki le face. Et sachiés que cheval malle sont de longue vie, car nous lisons d’un cheval ki vesqui. \textsc{.lxx.} ans ; més les femeles ne vivent longhement. Et lor luxure puet on refraindre se l’en lor roegne les crins ; mais en son part naist \textsc{.i.} venefice d’amour, ki vient en mi les front dou poulain, mais la mere toste maintenant a ses dens, car ele ne vieut pas que cele chose viegne en main d’ome ; et nanporquant se tu l’en ostoies, sachés que la mere ne li donroit jamés son let. Et sa nature est que tant comme le cheval est plus sain et de millour cuer, tant met il plus son nés et sa bouche dedens l’euue quant il boit.\par
En cheval doit on garder \textsc{.iiii.} choses selonc l’opinion as anciens sages, ce sont forme, beauté, bonté, et coulor. Car en la forme dou cheval doit on essaier que sa char et son cors soit fort et dur et soude, et k’il soit bien haut selonc sa force ; li costé doivent estre lonc et pleniers et la croupe grandisme et reonde, lees quisses, grant pis et large et mout overs, et tout son cors tachiés d’espoisseté, piés sés et bien cavés par desour.\par
En biauté dois tu garder k’il ait petit chief et sec, si ke le cuir soit bien tenant aprés les os, oreilles briés et drechies en haut, grand oils et larghe nés, la teste droite ou auques resamblable a teste de monton, crins espés et coue bien velue, ongles soudes et fermes et reondes.\par
En bonté garde k’il ait hardi corage, lie aleure, membres crollans, bien courans et tenans a ta volenté. Et sachiés que la isneleté du cheval est cogneue as oreilles, et sa force as membres crollans tramblans.\par
En coulour consire le bai, ou ferrant pomelé, ou noir ou blanc ou cervin ou vairon, ou d’autre maniere que tu poras eslire millor et plus avenable.\par
Et por ce k’il i a chevaus de plusours manieres, a ce que li \textsc{.i.} sont destrier grant por combatre, li autre sont palefroi por chevaucier a l’aise de son cors, li autre sont roncin por sommes porter, ou mul ki sont estrait d’assamblement de cheval et d’asne : dois tu estre bien sovenans a eslire celui cheval a ton oés, ki set les proprietés et les teches ki besoignables sont a ce de quoi il doit servir ; car les uns covient bien corre, les autres bien aler au pas ou ambler, et autre chose que lor nature requiert.\par
Mais generalment garde en tous chevaus que lor membre soient bien ordenés, et les uns bien respondans as autres, et k’il ait oils sains et toz les membres avec, et k’il ne soit de tel aage k’il ne soit affolés par jovence ne par viellece.\par
Et por ce que vices et maladies de chevaus sont sans nombre, dont les unes sont dedens et les autres dehors, les unes aparissans et les autres privees, si ke nus ne puet estre ki n’en ait ou poi ou mout, sachiés que cil sont millor ki mains en ont.
\chapterclose


\chapteropen
\chapter[{.I.CLXXXVII. Des olifans}]{\textsc{.I.CLXXXVII.} Des olifans}\phantomsection
\label{tresor\_1-187}

\chaptercont
\noindent Olifant est la plus grant beste ke l’en sache. Ses dens sont yvoire, et son bec est apelés premoiste, ki est samblables a \textsc{.i.} serpent. A celui bec prent sa viande et met la en sa bouche, et pour çou que la premoiste est garnie de son servoire, est ele de si grant force k’il brise quank’il fiert.\par
Et si dient li cremonnois ke le secont empereour Frederik en amena \textsc{.i.} en Cremonne, ke li envoia li prestres Jehans d’Inde, k’il li virent ferir \textsc{.i.} asne chargié, si fort k’il le geta desus une maison. Et ja soient olifans mout fiers, neporquant il devient privés tantost comme il est pris. Mais il n’enterra jamés en nef por passer la mer, se ses mestres ne li fiance de ramener ariere.\par
Et si le puet on chevauchier et mener ça et la, non pas a frain, mais a \textsc{.i.} crochet de fier ; et por ce fait on sor li mangoniaus et tour de fust por combatre. Mais Alixandres fist faire a l’encontre ymages de coivre plaines de charbons ardans, en tel maniere qu’eles quisoient et brisoient le bek de l’olifant, si qu’il ne le referoit plus por la paour dou feu.\par
Et sachiés ke en aus a mout grant sens ; car il conservent la discipline dou soleil et de la lune, autresi comme li ome. Et vont grant torbe ensamble as eschieles, dont li ainsnés est chevetain par devant tous ; et li autres ki est aprés lui d’aage les guie et les constraint par deriere. Et quant il sont a la mellee, il n’usent que l’un de lor dens, et l’autre gardent au besong ; et neporquant la u il sont vencu, il s’efforcent por gaster les ambesdous.\par
La nature des olifans est que la femele devant \textsc{.xiii.} ans et li malles devant \textsc{.xv.} ne sevent pas que luxure soit ; et neporquant il sont si chaste chose ke entr’aus n’a mellee por femele, chascuns a la soue a qui il se tient tous les jors de sa vie, en tel maniere que se li uns pert sa moilher ou ele lui, il ne se joint jamés a autre, ains vont touzjors seul parmi les desers.\par
Et por ce que luxure n’est pas en aus si chaude k’il s’asamblent comme les autres bestes, si lor avient par amonestement de nature que li doi compaignon s’en vont contre orient enprés du paradis terrestre, tant que la femele trueve une herbe que l’on apele mandragore, si en manguë, et si atice tant son malle k’il en manguë avec ; et maintenant eschaufe la volenté de chascun, et s’entregisent a envers, et engendrent \textsc{.i.} fil sans plus, et ce n’est c’une fois sans plus toute lor vie (et si vivent bien \textsc{.iiic.} ans).\par
Et quant li tans vient du part, c’est \textsc{.ii.} ans aprés lor assamblement, il s’en vont dedens \textsc{.i.} estanc jusc’au ventre, et la mere depose son fiz, et li peres si agaite tozjours a la rive, pour la paour du dragon, ki est lor enemis, por la covoitise de lor sanc, ke olifant ont plus froit et a plus grant fuison ke beste du monde.\par
Et si dient cil ki veu l’ont sovent ke olifans, quant il chiet, ne puet relever sus por tot son pooir, car il n’a es genous nule jointure ; mais Nature, ki toz guie, li ensegne a crier a haute vois, tant que uns autres viegne et crie avec lui si roidement que tout li autre dou païs vienent, ou au mains jusques a \textsc{.xii.}, ki tout crient ensamble tant que li petit olifant vienent, ki le relievent a la force de lor bec et de lor bouche, k’il metent desoz lui.
\chapterclose


\chapteropen
\chapter[{.I.CLXXXVIII. Des formis}]{\textsc{.I.CLXXXVIII.} Des formis}\phantomsection
\label{tresor\_1-188}

\chaptercont
\noindent Formis est petite chose mais de grant porveance ; car ele se pourchace en esté çou que li est besoing en yvier, et eslist le forment et refuse orge, k’ele connoist a l’odour ; et ses grains brise parmi, pour çou k’il ne puissent renaistre a la moistour de l’yvier.\par
Et si dient li etyopiien k’il a formis en un ille, grans comme chienos ki chevillent or dou sablon a lor piés, et le gardent si fierement que nus n’en puet avoir sans mort. Mais li païsant envoient en cele ille a paistre jumens ki ont poulains, et sont chargie de bons coffres. Et quant les formis aperçoivent les coffres, il metent dedens lor or, car eles quident que ce soit leus de sauveté. Et quant vient au soir ke la jumens est bien peue et bien chargie et eles oënt lor poulains ke lor mestre ont la amenés henir et braire d’autre part la riviere, les jumens se fierent en l’euue courant et batant, et s’en passent outre o tout l’or ki est es coffres
\chapterclose


\chapteropen
\chapter[{.I.CLXXXVIIII. De yena}]{\textsc{.I.CLXXXVIIII.} De yena}\phantomsection
\label{tresor\_1-189}

\chaptercont
\noindent Iena est une beste ki une fois est malle et autrefois femele, et habite en cymentieres as homes, et manguë les cors des mors. Et li os de son eschine est si roit que son col ne puet ele plier, s’ele ne se torne toute cele part k’ele vieut. Et ensiut les maisons et les estables et contrefait la vois as homes, et ensi deçoit sovent les homes et les chiens et les deveure.\par
Et dient li plusour ke en ses oils a une piere de tel vertu, que se nus l’eust desous la larghe, il poroit deviner toutes les choses ki sont a avenir. Et pour çou que nule beste qui touche l’ombre de yena ne se puet movoir du leu, si dient li ancien que ceste beste est replenie d’encantement et d’art magique.\par
Et sachiés que en Etyope gist ceste beste avec la femele du lion, et engendre une beste ki a non cocote, ki autresi ensiut la vois des homes. Et en sa bouche n’a pas gencives ne dens devisés ke autres bestes ont, fors que tout est uns dens tout enterin, et les reclot comme une boiste.
\chapterclose


\chapteropen
\chapter[{.I.CLXXXX. Des lup}]{\textsc{.I.CLXXXX.} Des lup}\phantomsection
\label{tresor\_1-190}

\chaptercont
\noindent Lup habonde en Ytaile et en maintes autres terres. Et sa force est en sa bouche et ou vis. Et es rains ita point de force, et son col ne puet pas pliier ariere. Et si dient li pastour k’il vivent aucune fois de proie, aucune fois de terre, et aucune fois de vent. Et quant li tens de sa luxure vient, plusor malle ensivent par route la lue. Mais a la fin ele regarde en trestoz, et eslit le plus lait, ki gise o li, ja soit ce que en toute l’anee ne se joingnent se \textsc{.xii.} jors non. Et n’engendrent pas fiz se en mai non, quant li tonnoires vient. Pour la garde de ses caiaus ne prent il pas proie es contrees ki sont voisines a sa covace.\par
Et sachiés que la u il voit home premiers que il lui, li hom ne puet pas crier. Mais se li hom voit premiers lui, il depose toute fierté et ne puet corre. Et en la fin de sa coue a une laine d’amour, que li leus oste a ses dens quant il crient k’il soit pris. Et quant il usle de sa vois, il maine tousjours son pié devant sa bouche, pour moustrer que ce soit de plusors leus.\par
Une autre maniere de lous sont ke l’en apele cerviers u luberne, ki sont pomelés de noires taches, autresi come longe ; mais des autres choses est il samblable au lup. Et est de si clere veue ke ses oils perchent les murs et les mons. Et ne porte c’un filz ; et est la plus oublieuse chose du monde, car la u il manguë son past, et il regarde par aventure une autre chose, oublie ce maintenant k’il mengoit, en tel maniere k’il n’i set revenir, ains le pert dou tout. Et si dient cil ki veu l’ont, ke de son pissat naist une piere precieuse ki est apelee liguires ; ce connoist bien la beste meismes, selonc ce que li home li ont veu covrir sa orine dou sablon, par une envie de nature que tel piere ne parviegne as homes.
\chapterclose


\chapteropen
\chapter[{.I.CLXXXXI. De la lucrote}]{\textsc{.I.CLXXXXI.} De la lucrote}\phantomsection
\label{tresor\_1-191}

\chaptercont
\noindent Lucrote est une beste es parties d’Inde ki de isneleté passe tous autres animaus. Et est grant comme asne, et a croupe de cerf, et pis et jambes de lyon, chiés de cheval, piés de buef, et bouche grant jusc’as oreilles, et si dent sont tout d’un os.
\chapterclose


\chapteropen
\chapter[{.I.CLXXXXII. Des manticores}]{\textsc{.I.CLXXXXII.} Des manticores}\phantomsection
\label{tresor\_1-192}

\chaptercont
\noindent Manticores est une beste en celui païs meismes, ki a face d’ome et coulour de sanc, oils jaunes, cors de lyon, coue d’escorpion, et court si fort que nule beste ne puet eschaper devant lui. Mais sor toutes viandes aime char d’ome. Et si dent s’assamblent en tel maniere, que ore maint li uns desous et ore li autres.
\chapterclose


\chapteropen
\chapter[{.I.CLXXXXIII. De panthere}]{\textsc{.I.CLXXXXIII.} De panthere}\phantomsection
\label{tresor\_1-193}

\chaptercont
\noindent Panthere est une beste tachie de petis cercles blans et noirs autresi comme petit oil, et est amés de toz animaus fors que dou dragon. Et sa nature est que tout maintenant qu’ele a sa viande prise, se rentre en son espelonce, et se dort \textsc{.iii.} jours ; lors se lieve et oevre sa bouche, et flaire s’alaine si dous et si souef ke totes bestes ki sentent l’odour s’en vont devant lui, fors solement le dragon, ki s’afiche es pertuis desous terre, pour la puour k’il en a, k’il set bien que a morir le covient.\par
Et sachiés que li panthere ne porte fiz en trestote sa vie que une seule fois, et ore orés porquoi. Ses cheaus quant il sont escreu dedens le cors a la mere, ne welent pas soufrir jusc’au tens de lor droite naissance, ains s’efforcent nature et debrisent as ongles les entrailles a lor mere, et s’en issent hors en tel maniere que la mere n’engendre plus en nule maniere du monde par semence de son malle.
\chapterclose


\chapteropen
\chapter[{.I.CLXXXXIIII. De parande}]{\textsc{.I.CLXXXXIIII.} De parande}\phantomsection
\label{tresor\_1-194}

\chaptercont
\noindent Parande est une beste en Etyope, bien grant come buef, et a chief et cornes comme chierf, et coulour d’ours ; mais li etyopiien dient que parande mue sa droite coulour por paour selonc la tainture de la chose ki li est plus prochaine. Ce meisme font polpes en mer et camelion en terre, de quoi li contes fist mention ça arieres.
\chapterclose


\chapteropen
\chapter[{.I.CLXXXXV. Des singes}]{\textsc{.I.CLXXXXV.} Des singes}\phantomsection
\label{tresor\_1-195}

\chaptercont
\noindent Singes est une beste ki volentiers contrefait çou k’ele voit faire as homes. Et mout s’esleece a la nueve lune, mais de la reonde lune se dolousist et torble de grant melancolie.\par
Et sachiés que singes porte \textsc{.ii.} fiz a une portee, dont ele ayme l’un forment avers l’autre ; por quoi il avient quant on le chace k’ele porte son fiz k’ele ayme entre ses bras, et l’autre sor ses espaules, et s’enfuit tant com il puet. Mais la ou la chace aproche et on le tient si cort k’ele crient de son cors meismes, il li covient a deguerpir son chier fiz meismes ; car li autres se tient si fermement au col sa mere k’il eschape dou perii la ou la mere s’enfuit. Et si dient li etyopiien que en lor terre a singes de \textsc{.v.} diverses manieres.
\chapterclose


\chapteropen
\chapter[{.I.CLXXXXVI. Des tygres}]{\textsc{.I.CLXXXXVI.} Des tygres}\phantomsection
\label{tresor\_1-196}

\chaptercont
\noindent Tigre est une beste ki plus naist es parties d’Orcanie, et est menuement tachie de vaires taches. Et sans faille tigre est une des plus courans bestes du monde, et de grant fierté. Et sachiés que la u ele troeve remué son lit et vuit de ses filz c’on li a emblés, ele ensiut tost et isnel les traces dou veneor ki les enporte. Mais li hom ki les a si redoute mout sa cruauté, car bien set ke fuir de cheval ne d’autre chose ne le poroit garantir ; si giete enmi le voie par ou la beste vient plusors miroirs, les uns cha et les autres la ; et quant la tigre voit sa ymage dedens le miroir aparoir et aperçoit la figure et la samblance de son cors, ele quide que ce soit ses faons, si le torne et retorne, mais c’est nient ; puis s’en vet outre jusqu’a \textsc{.i.} autre miroir et garde et regarde, por la pitié de ses filx, or a l’un ore a l’autre, tant ke li chaceours s’en passe a sauveté.
\chapterclose


\chapteropen
\chapter[{.I.CLXXXXVII. De la taupe}]{\textsc{.I.CLXXXXVII.} De la taupe}\phantomsection
\label{tresor\_1-197}

\chaptercont
\noindent Taupe est une petite beste qui vait tonzjours sonz terre et cheville en diversses parties, et manjue les racines que ele trueve, ja soit ce que li pluisor dient que ele vit de terre soulement. Et bien sachez que taupe ne voit goute, car nature ne volt pas ouvrir la pel qui est sur les yauz, et einsi ne valent ele neent, porce que il ne sont descouverz.
\chapterclose


\chapteropen
\chapter[{.I.CLXXXXVIII. Ja parle de l’unicorne}]{\textsc{.I.CLXXXXVIII.} Ja parle de l’unicorne}\phantomsection
\label{tresor\_1-198}

\chaptercont
\noindent Unicorne est une fiere beste auques ressemblable a cheval, de cors ; mais il a piez d’olyfant, coue de cerf, et sa voix est fierement espoentable ; et enmi sa teste a une corne, sans plus, de merveillouse resplendor, et.a bien \textsc{.iiii.} piez de lonc, mais il est si forz et aguz que il perce legierement quantque il ataint.\par
Et sachez que unicor est si aspres et si fiers que nul ne la puet attaindre ne prendre par nul laz dou monde : bien puet estre qu’il soit occis, mais vif ne le puet on avoir. Et neporquant li veneor envoient une vierge pucele cele part ou li unicor converssent ; car c’est sa nature que maintenant s’en vait a la pucele tout droit et depose toute fierté, et se dort souef en son sain et en ses dras. En ceste maniere le deçoivent li veneor.
\chapterclose


\chapteropen
\chapter[{.I.CLXXXXVIIII. De l’hours}]{\textsc{.I.CLXXXXVIIII.} De l’hours}\phantomsection
\label{tresor\_1-199}

\chaptercont
\noindent Ours a molt foible chief, mais sa force est es jambes et es longes, et por ce vait il souvent tout droit en estant. Et sachez que se ours est desheitiez de cop ou de maladie, ele manjue une herbe qui a a non flomus, qui la conduist a sa garison. Mais se ele manjue pomes de mandragore, a morir la covient, se ne fuissent fourmiz, que ele menjue contre celui mal. Miel manjue volentiers sur toutes autres choses.\par
Et sa nature est que cn yver eschaufe de luxure et gisent ensemble come li home font avuec les femes, et engendrent fiz, que la femele ne porte que \textsc{.xxx.} jors ; et por la brieté dou tens, nature n’a pooir d’acomplir la forme ne la façon d’aux dedenz le cors de luer mere, ançois naist une piece de char blanche sans nule figure qui soit, se ce non que il y a \textsc{.ii.} oils. Et neporquant la mere la conforme et adrece a sa lengue selonc la semblance de soi, et puis l’estraint a son piz por doner li chalor et esperit de vie.\par
Et endementiers se dort la mere bien \textsc{.xiiii.} jors sans boivre et sanz mangier, si fermement que l’en la porroit batre ou occirre avant que ele s’esveillast. En ceste maniere maint la mere priveement avuec ses fiz en repost bien \textsc{.iiii.} mois, por quoi ses oils sont si tenebroux que ele ne voit se molt pou non quant ele ist de sa covace. De ceste beste dient li pluisor qu’ele amende et engraisse par bateures.
\chapterclose


\chapteropen
\chapter[{.I.CC. Le derinier chapitre dou .i. livre}]{\textsc{.I.CC.} Le derinier chapitre dou \textsc{.i.} livre}\phantomsection
\label{tresor\_1-200}

\chaptercont
\noindent Ci fenist la premiere partie de cestui livre, qui devise briement la generation dou monde et le comencement des roys et des terres, et l’establissement de l’une loy et de l’autre, et la nature des choses dou ciel et de la terre, ét l’ancieneté des vieilles ystoires, et briement raconte l’estre de chascune. Car se li maistres les vousist plus largement metre en escrit et moustrer de chascune chose por coi et coment, li livres seroit sans fin, car a ce besoigneroit toutes les ars et toute philosophye.\par
Et porce dit li maistres que la premiere partie de son tresor est en deniers contanz. Car si comme les genz ne porroient mie chevir lor besoignes ne lor marcheandies sans monoie, tout autresi ne porroient il savoir la certaineté des humaines choses, se il ne seussent ce que ceste premiere partie raconte.\par
Mais ici se taist li maistres des choses qui apartienent a theorique, qui est la premiere science dou cors de phylosofie ; car il viaut torner as autres \textsc{.ii.} sciences, pratique et logique, por amasser la seconde partie de son tresor, qui doit estre des pierres preciouses.
\chapterclose

\chapterclose


\chapteropen
\part[{Livre second}]{Livre second}\phantomsection
\label{tresor\_2}\renewcommand{\leftmark}{Livre second}


\chaptercont

\chapteropen
\chapter[{.II.I. Cis secons livres parole de la nature des visces et des vertus selonc etique}]{\textsc{.II.I.} Cis secons livres parole de la nature des visces et des vertus selonc etique}\phantomsection
\label{tresor\_2-1}

\chaptercont
\noindent Quant li mestres ot finee la premiere partie de son livre, et k’il ot mis en escrit de theorike çou ke s’en apertenoit a son proposement, il volt maintenant ensivre sa matire selonc la promesse k’il fist en son prologue de devant, pour dire des \textsc{.ii.} autres sciences dou cors de philosophie, c’est de pratique et de logique, ki ensegne a home çou k’il doit faire et quex non, et la raison pour quoi on doit les unes faire et les autres laissier. Mais de ces \textsc{.ii.} sciences traitera li mestres auques melleement, pour çou que li leur argument sont si entremellé que a paine poroient il estre devisé. Et ce est la seconde partie dou tresor, ki doit estre de pieres precieuses, ce sont les vertus, ce sont les mos et li ensegnement des sages, dont chascuns vaut a la vie des homes, et por beauté et pour delit et pour vertu ; car nule piere n’est chiere se por ces \textsc{.iii.} choses non.\par
Cist ensegnemens sera sor les \textsc{.iiii.} principaus vertus. Dont la premiere est prudence, ki est segnefiee par le carboncle, ki alume la nuit et resplendist sour toutes pieres. La seconde est atemprance, ki est segnefiee par le saphir, ki porte celestial coulor, et est plus gracieuse que piere du monde. La tierce est force, ki est segnefiee par le diamant, ki est si fort k’il ront et perce toutes pieres et tous metaus, et por poi il n’est chose ki le puisse donter. La quarte vertu est justice, ki est segnefiee par l’esmeraude, ki est la plus vertuouse et la plus bele chose que oil d’ome poisse veoir.\par
Ce sont les trés chieres pieres dou tresor, ja soit ce que il soit toz plains d’autres pieres ki ont aucune vaillance, selonc ce que li bons entendeour pora veoir et conoistre as paroles ke mestre Brunet Latin escrist en cest livre. Mais tot avant volt il fonder son edifice sor le livre Aristotle, et si le translata de latin en romanç, et le posera au commencement de la seconde partie de son livre.
\chapterclose


\chapteropen
\chapter[{.II.II. Ci commence etiqe, li livres aristotles}]{\textsc{.II.II.} Ci commence etiqe, li livres aristotles}\phantomsection
\label{tresor\_2-2}

\chaptercont
\noindent Tous ars et toutes doctrines et totes oevres et tous triemens sont por querre aucun bien ; dont disent bien li philosophe que çou ke toutes choses desirent est le bien. Selonc divers ars les fins sont diverses, car teus fins sont oevres et teus sont ki ensivent par les oevres. Et por çou que maint sont les ars et les oevres, chascuns a sa fin ; car medecine a une fin, c’est a fere santé ; et l’art de bataille a sa fin, por quoi ele fu trovee, c’est victore ; et les arz de faire nés a une autre fin, c’est nagier ; et la science ki ensegne a home governer sa maison et sa mesnie a \textsc{.i.} autre fin, c’est richece.\par
Et sont aucun art ki sont generaus, et aucun ki sont especiaus, c’est particulers, et aucun sont sans division ; et pour ce sont les uns souz les autres. Si comme est la science de chevalerie, ki est generaus, et desous li sont autres sciences particuleres, c’est la science de fere frains selles et espees et toutes les autres choses ki ensegnent fere chose ki a bataille besoignent. Et ceste art universele sont plus dignes que les autres, pour çou que les particuleres sont trovees par les universeles.\par
Et tot autresi comme les choses ki sont faites par nature, est une derraine chose a quoi la nature entent finalment ; autresi ens es choses ki sont faites par art est une finale chose a quoi sont ordenees trestoutes les choses de celui art. Et si con celui ki trait de son arc a \textsc{.i.} signe au berzail par son adrecement, tot autresi a chascun art une finale chose ki adrece ses oevres.
\chapterclose


\chapteropen
\chapter[{.II.III. De governemens de cités}]{\textsc{.II.III.} De governemens de cités}\phantomsection
\label{tresor\_2-3}

\chaptercont
\noindent Donques li ars ki ensegne la cité governer est principale et soveraine et dame de tous ars, pour ce que desous lui sont contenues maintes honorables art, si come est retorique, et la science de fere ost et de governer sa maisnie. Et encore est ele noble pour ce k’ele met en ordre et adrece toutes ars ki sous li sont, et li sien compliement ; et sa fin si est fin et compliement des autres. Donques le bien ki de ceste science vient si est le bien de l’home, pour ce k’ele le constraint de bien fere et le constraint de mal non faire.\par
Le droit ensegnement si est que l’en aille selonc que la nature le puet soufrir, c’est a dire que celui ki ensegne geometrie doit aler par ses argumens, ki sont apelés demoustracion, et en rettorique doit aler par argumens et par raisons voirsamblables ; et ce avient pour ce que chascuns artiers juge bien et dist la verité de ce ki apertient a son mestier, et en ce est son sens soutil.\par
La science de cité governer ne s’afiert pas a enfant ne a home ki ensive ses volentés, pour ce que ambesdeus sont nonsachans des choses du siecle ; car cestui art ne quiert pas la science de l’home, mais k’il se torne a bonté.\par
Et sachés que enfant sont en \textsc{.ii.} manieres, car on puet bien estre vieus d’aage et enfes de meurs, et puet estre enfant par aage et vieus par honeste vie. Donques la science de governer cités afiert a home ki n’est pas enfant de ses meurs, et ki ne sive ses volentés se lors non ki se covient et tant come il doivent et la u il se covient et si com il est covenable.\par
Il i a choses ki sont conneus a nature, et sont choses ki sont conneues a nous ; car ki se vieut estudier a savoir ceste science, il se doit user as choses justes et bones et honestes, ou il li covient avoir l’ame naturelement ordenee a ceste science. Mais celui ki n’a ne l’un ne l’autre, regart a ce que Omers dist, la premiers est bons, li autres est apareilliés a estre bons. Mais ki dedens ne set noient et ki n’aprent de ce que hom li ensegne, il est dou tout mescheans.
\chapterclose


\chapteropen
\chapter[{.II.IIII. Des .iii. vies}]{\textsc{.II.IIII.} Des \textsc{.iii.} vies}\phantomsection
\label{tresor\_2-4}

\chaptercont
\noindent Les vies nomees, ki font a conter, sont \textsc{.iii.} vies : l’une vie est de concupiscense et de covoitise, l’autre est vie citeine, c’est de sens et de proece et d’onour, la tierce est contemplative. Et li plusor vivent selonc la vie des bestes, ki est apelee vie de concupiscence, por çou k’il ensivent lor volenté et lor delit. Et chascune de ces \textsc{.iii.} vies a sa propre fin diverse des autres, tout autresi comme medecine a sa fin diverse de la science de combatre : car cele bee a fere santé, et cele autre a victore.
\chapterclose


\chapteropen
\chapter[{.II.V. De .ii. manieres de bien}]{\textsc{.II.V.} De \textsc{.ii.} manieres de bien}\phantomsection
\label{tresor\_2-5}

\chaptercont
\noindent Le bien est en \textsc{.ii.} manieres ; car une maniere de bien est ki est desirrés par lui meismes, et une autre maniere de bien est ki est desirrés par autrui. Bien par lui est beatitude, ki est nostre fin a quoi nous entendons. Bien par autri sont les honours et les vertus, car ce desire l’ome por avoir beatitude.\par
Naturele chose est a l’home k’il soit citeins et k’il se converse entre les homes et entre les artiers ; et contre nature seroit habiter es desers ou n’a point de gent, por ce ke l’ome se delite naturelement en compaignie. Beatitude est chose complie, si k’ele n’a nule besoigne d’autre chose hors de lui : par qui la vie des homes est prisable et glorieuse. Donques est la beatitude le grignour bien de tous, et la plus soveraine chose et la trés millour de tous les biens qui soient.
\chapterclose


\chapteropen
\chapter[{.II.VI. Des poissances de l’ame}]{\textsc{.II.VI.} Des poissances de l’ame}\phantomsection
\label{tresor\_2-6}

\chaptercont
\noindent La ame de l’home a \textsc{.iii.} poissances : l’une est vegetative, et ce est comun as arbres et as plantes, car il ont ame vegetative ausi con li home ont ; la seconde est apelee sensitive, et c’est comun a totes bestes, car eles ont ame sensitive ; la tierce est apelee raisonable, et por ce est li hom divers de toutes choses, por ce que nule autre chose n’a ame raisonable se l’ome non.\par
Et ceste poissance raisnable est aucune fois en oevre et aucune fois en pooir ; més beatitude est, quant ele est en oevre, et non pas en pooir solement, car s’il ne le fait il n’est mie bons. Totes les oevres de l’home, ou eles sont bones ou eles sont malvaises ; et cil ki fait bonnes oevres, il est dignes d’avoir le compliement de la vertu de cele oevre, car celi ki bien citole est dignes d’avoir le compliement de lor mestier, et cil ki mal le fait, le contraire.\par
Donc se la vie de l’home est selonc l’oevre de raison, lors est il prisables quant il la maine selonc la propre vertu ; mais quant maintes vertus sont en l’ome, sa vie est besoignable et honoree et mout digne, si ke plus ne poroit estre ; pour ce que une seule vertus ne poroit fere l’ome beate et parfet dou tout. Car une seule arondele ki viegne ne uns seus jours atemprés ne donent pas certaine ensegne dou printens. Et por ce en poi de vie d’omme ne en poi de tens k’il fait bones oevres ne poons nous pas dire k’il soit beates.
\chapterclose


\chapteropen
\chapter[{.II.VII. Des .iii. manieres de bien}]{\textsc{.II.VII.} Des \textsc{.iii.} manieres de bien}\phantomsection
\label{tresor\_2-7}

\chaptercont
\noindent Le bien est divisé en \textsc{.iii.} manieres ; car li uns est biens de l’ame, li autres du cors, li autres dehors le cors ; mais li biens de l’ame est plus dignes de nul des autres, car ce est biens de Deu, et sa forme n’est pas cogneue se par les oevres vertueuses non.\par
Et sans faille beatitude est en querre les vertus et en user les. Mais quant beatitude est en l’abit et en pooir de l’home et non en ses fais, c’est a dire quant il poroit bien fere et ne le fet mie, lors est il vertueus ausi con celui ki se dort, car ses oevres et ses vertus ne se monstrent. Mais l’ome ki est beate, covient ausi come par necessité k’il face le bien en oevre, et si comme li sages canpions et fors ki se combat et vaint et enporte la courone de victore, tot autresi li hons bons et beates a le guerredon et le loenge de le vertu k’il fait et moustre veraiement par ses oevres, por çou que le guerredon de la beatitude est le delit que l’en a tant com il oevre la vertu. Car chascuns se delite en ce k’il ayme, li justes se delite en justice, et li sages en sapience, et li vertueus en vertu. Et toute oevre ki est par vertu est bele et delitable en soi meisme.\par
Beatitude est la chose au monde ki est trés millour et trés joiouse et trés delitable. Mais la beatitude ki est en terre abesoigne des biens de hors, car il est dure chose que l’en face beles oevres se il n’a partie grant des choses avenables a bonne vie, abondance d’avoir et de amis et de parens et prosperité de fortune. Et por ce la sapience a besoigne d’aucune chose ki face connoistre sa valour et ses honours.\par
S’aucuns donne as homes du monde, Dieus glorieus et soverains le fait. L’en doit bien croire que celui don soit beatitude, por ce k’ele est la millour chose ki puisse estre au monde, car ele est mout honorable chose et est li compliemens et la forme des vertus. Ne il n’est pas dit du cheval ne des autres bestes ne des enfans k’il soient beates por çou k’il ne font oevre de vertu.\par
Beatitude est chose ferme et estable tozjors en une fermeté, si k’ele ne se remue pas ; et si est une fois bien et autre mal, mais toutefois est bien, porce ke li muemens de bonté et de malice n’est pas se es oevres des homes non.\par
Le piler de beatitude est l’oevre que l’om fait selonc vertu, et la colombe du contraire est l’oevre que l’om fait selonc vice, et la vertu ferme et estable est en l’ame de l’omme. Li hom vertuous ne se conturbe ne ne s’esmaie por nule tempore chose ki li aviegne, car il n’auroit beatitude s’il s’esmaiast ; car dolour et paour abatent l’oevre de vertu et la jole de beatitude. Et aucunes sont mout grevables a soustenir ; mais quant on l’a bien soustenue, lors apert et se monstre la hautece de son corage. Et sont autres choses ki ne sont griés a soustenir, ne li hom ki le suefre ne moustre pas que en lui soit fortece.\par
Et ja soit ce que mort et maladies de fiz soient griés a soustenir, ne doivent pas home remuer de sa felicité ; car bien et felicité et home felix et Dieus glorieus et beneoit sont tant digne chose et tant honorable, ke nul pris ne nule loenge ne lor soufist pas. Et nous devons reverir et magnefiier et glorifier Dieu sor toutes choses, et si devons croire que en lui sont toz biens et toute felicité, pour çou k’il est comencement et ochoison de tous biens.\par
Felicité est une chose ki vient par vertu de l’ame, non pas dou cors ; car tout ensi con li bons mires enquiert la nature de l’home por maintenir la en santé et por donner medecine en totes les maladies, ausi doit li home et li governeour des cités veillier et estudier k’il puisse proufiter a ses citeins, et maintenir la felicité, ki apertient a l’ame entellectuel, et amonester les a fere oevre de vertu, pour ce que lor fruit est felicité.
\chapterclose


\chapteropen
\chapter[{.II.VIII. De la premiere poissance de l’ame}]{\textsc{.II.VIII.} De la premiere poissance de l’ame}\phantomsection
\label{tresor\_2-8}

\chaptercont
\noindent L’ame de nous a maintes puissances : l’une n’est pas raisonable, c’est l’ame des plantes et des autres animaus, por ce n’est ele propre puissance de l’home, car ele puet ovrer en dormant.\par
Et l’autre poissance est intellective, par le qui oevre li hom, est dis bons ou mauvés ; et ne moustre pas ses oevres en dormant, et por ce fu dit que l’ome felix ne se dessamble pas dou malvais par moitié de lor vie, car en dormant teus est li bons com li mauvais, por ce k’en dormant se reposent les oevres ki font la vie bone u malvaise. Mais ce n’est voirs en tout generaument, por ce que l’ame des bons voit sovent en songe bones ymaginations et proufitables, les quex ne puet veoir l’ame dou mauvés.\par
Et est une autre puissance de l’ome ki n’est pas raisonable, mais a part en raison, pour çou k’ele doit estre enobeissant a raison, et est apelee vertus concupiscibles. Et tu dois savoir que en l’ame sont aucune fois contraire movement, ausi comme el cors quant li uns menbres se muet et est paralitiques, qui il covient movoir contre nature ; mais ceste contrarietés est manifestee el cors et privee en l’ame.\par
La raisonable puissance est en \textsc{.ii.} manieres, une ki est raisonable veraiement, ki nous fet aprendre et conoistre et jugier ; l’autre est apelee concupiscible, et si est ele raisonable, tant comme ele est obeissant a la puissance ki est raisonable veraiement : autresi com li bons fiz ki reçoit le chastiement son pere, et ne rebelle contre lui.
\chapterclose


\chapteropen
\chapter[{.II.VIIII. Des .ii. manieres de vertu}]{\textsc{.II.VIIII.} Des \textsc{.ii.} manieres de vertu}\phantomsection
\label{tresor\_2-9}

\chaptercont
\noindent Pour ce apert il que \textsc{.ii.} manieres sont de vertu : l’une est de l’entendement de l’home, c’est sapience, science, sens ; l’autre est de moralité, c’est chasteté et larghece, et autres samblables. Et ce puet chascuns veoir clerement : car quant nous volons un home prisier de vertu entellectuele, si dist on, c’est uns hons sages et soutiz ; mais quant nous le volons prisier de moralités, nous disons, c’est uns hons chastes et larges.
\chapterclose


\chapteropen
\chapter[{.II.X. Comment vertu naissent es homes}]{\textsc{.II.X.} Comment vertu naissent es homes}\phantomsection
\label{tresor\_2-10}

\chaptercont
\noindent La vertu de l’entendement est engendree et escreue en l’ome par doctrine et par ensegnement ; et por ce li covient experience et lonc fans. La vertu de moralité naist et croist par bon us et honeste ; car ele n’est pas en nous par nature, a ce que chose naturele ne puet estre muee de son ordene par usage contraire.\par
Raison coment : la nature de la piere est d’aler tousjors aval : nus ne le poroit tant geter amont k’ele seust sus aler. Et la nature du feu est d’aler amont : nus ne le poroit tant avaler que il seust en aval metre la flame ; et generalment nule naturele chose ne puet par usage aprendre a fere le contraire de sa nature. Et ja soit ce que ceste vertus n’est en nous par nature, certes la puissance d’aprendre la est en nous par nature, et le compliement est en nous par usage. Por quoi je di ke ces vertus ne sont pas en nous dou tot par nature, ne dou tout sans nature, més la racine et le commencement de reçoivre les vertus sont en nous par nature, et li lor compliemens est en nous par usage.\par
Et toutes choses ki sont en nous par nature sont premierement en pooir et puis en fait : ausi come li sens de l’omme, car tout avant a li hons pooir del veoir et de l’oiir, et par celui pooir voit et ot ; et nus ne voit devant k’il en a le pooir, donques savons nous que li pooirs est devant le fere. Mais es choses de moralité est le contraire, car l’oevre et le fait est devant le pooir.\par
Raison coment : aucuns hom a la vertu de justice, por ce k’il a devant fet maintes oevres de justice ; et uns autres a la vertu de chasteté, pour çou k’il a devant fet maintes oevres de chasteté.\par
Tot autresi est il des choses de mestiers et d’art, car on set faire maisons, por ce c’on en a maintes fetes premierement ; car autrement ne le seust il, se il ne l’eust ovré devant plusours fois. Autresi set aucuns bien citoler, pour çou k’il l’a mout usé.\par
Et li hons est bons pour bien fere et mauvais por malfaire. Et par une meisme chose naissent en nous et se corrompent les vertus, se cele chose est menee en diverses manieres ; tout autresi come la santé, car travillier atempreement engendre santé el cors de l’ome, mais travillier plus u mains que mestier n’est corront la santé, mais moieneté la garde et acroist.\par
Ausi est de vertu, car ele corront et gaste par poi et par trop, et si se conserve et maintient par la moieneté. Raison coment : paour et hardement corrompent la proece de l’home, car li hom ki a paour s’enfuit pour toutes choses, et li hardis enprent a fere toutes choses et les quide mener a fin. Ne l’un ne l’autre n’est pas proece ; mais proece est aler par mi entre hardement et poour, et doit on fuir les choses ki font a fuir et envaïr celes ki doivent estre envaïes.\par
Et cis abis est aquis par usage de desprisier les terribles choses, et li abis de chasteté est aquis par usage de retenir soi contre ces covoitises ; autresi devés entendre de toutes vertus.
\chapterclose


\chapteropen
\chapter[{.II.XI. Comment vertus est en abit}]{\textsc{.II.XI.} Comment vertus est en abit}\phantomsection
\label{tresor\_2-11}

\chaptercont
\noindent Or covient que nous devisons la difference ki est entre abit de vertu et abit ki est sans vertu, pour dolour ou pour leece, ki font lor oevres ; ce est a dire ke l’ome ki s’astient de charnele volenté et de cele astinence est liés, certes il est chastes, més celui ki s’en astient et de cele astinence est dolans, certes il est luxurieus.\par
Tout autresi est il d’un home ki sostient et suefre maintes terribles choses dont il ne se conturbe point, certes il est preudome et fort ; mais celui ki se conturbe est paourous en totes oevres, et en toutes meurs ensivent dolour ou leece. Donques chascune vertu est en delit ou en courouç. Et pour ce li governeour des viles honeurent ceaus ki se delitent de ce k’il doivent, et metent en divers tormens ceus ki se delitent si com il ne doivent.
\chapterclose


\chapteropen
\chapter[{.II.XII. Des choses que l'om desire}]{\textsc{.II.XII.} Des choses que l'om desire}\phantomsection
\label{tresor\_2-12}

\chaptercont
\noindent Des choses ke l’en desire et voet sont \textsc{.iii.}, l’une est proufitable, l’autre est delitable, et la tierce est bone ; et le contraire sont ausi \textsc{.iii.}, non proufitable, non delitable, malvaise. En ces \textsc{.iii.} choses, ki use raison est bons et ki ne l’use par raison est malvais. Et meismement en delit, car delit est norris avec nous de nostre naissance, pour quoi il sera grandisme chose a avoir mesure ou adrecement en delit. Donques trestoute l’entention de nostre livre est entor delit ; car Eraclitus dist que ens es griés choses covient avoir art. Donques tote l’entention de l’ome ki governe les cités est k’il face les citeins deliter es choses ki se coviegnent, et lors et ou et tant con il se covient. Et celui ki use bien ces choses selonc ce k’il covient, est bons et ki fet le contraire si est mauvés
\chapterclose


\chapteropen
\chapter[{.II.XIII. Comment hom est justes}]{\textsc{.II.XIII.} Comment hom est justes}\phantomsection
\label{tresor\_2-13}

\chaptercont
\noindent Et poroit aucuns demander coment est hons justes faisant oevre de justice, et atemprés faisant oevres d’atemprance ; et uns autres diroit que tot est ausi comme de gramatike, car li hons est apelés gramatiques se il parole selonc gramatique. Mais a la verité dire, il n’est pas des vertus ensi come des ars, ançois est tout autrement ; car ki wet estre bons en aucun art, il ne li covient autre chose que savoir le, mais en vertu le savoir n’est pas soufissant sans l’oevre, pour ce li couvient faire et eslire l’ovre de vertu, et que sa volenté soit parmanable.\par
Et cil qui quide bons estre par solement savoir, sans fere l’oevre, est samblables au malade, ki bien set l’amonestement de son fisicien, mais il n’en conserve nul. En tout autresi comme teus malades sont loins de garison, autresi sont teus homes loins de felicité.
\chapterclose


\chapteropen
\chapter[{.II.XIIII. Quex vertus sont en abit}]{\textsc{.II.XIIII.} Quex vertus sont en abit}\phantomsection
\label{tresor\_2-14}

\chaptercont
\noindent En l’ame de l’home sont \textsc{.iii.} poissances, c’est abit, pooir, et passion. Passions sont si come amour, leece, et misericorde ; et totes choses de qui ensiut volenté et moleste sont sous ces choses de passion. Pooir est la nature par cui nous poons courouchier ou joïr ou avoir misericorde. Abit est cele chose par qui li hom est prisiés ou blasmés, porquoi je di que vertu n’est pooir ne passion, ançois est abis ; car pour passion ne pour pooir n’est pas li hons blasmés ne loés, mais por l’abit est il prisiés et desprisiés s’il est fermes et parmanans en son corage. Et nous avons ja ensegnie la voie ki a ce nous amaine, car ki bien set la nature de vertu, il a bien la voie pour aler i.
\chapterclose


\chapteropen
\chapter[{.II.XV. De vertu qi ele est et coment}]{\textsc{.II.XV.} De vertu qi ele est et coment}\phantomsection
\label{tresor\_2-15}

\chaptercont
\noindent Vertus est trovee en celes choses ki ont milieu et estremités, c’est a dire plus et mains. Et ce mileu est en \textsc{.ii.} manieres, li uns est selonc nature, et li autres est par comparison de nous. Le mi ki est selonc nature et par soi ce est celui ki en toutes choses est une meisme chose. Raison coment : se \textsc{.x.} sont trop et \textsc{.ii.} sont poi, le mi est \textsc{.vi.}, pour ce que \textsc{.vi.} est tant plus ke \textsc{.ii.} comme il est mains ke .x.\par
Le mi leu ki est par comparison de nous est celui ki n’est ne poi ne trop. Raison coment : se mangier une petite viande est poi et mengier une grande viande est trop, il ne doit mie prendre le milieu ; car se mangier \textsc{.ii.} pains est poi et mangier \textsc{.x.} pains est trop, il ne doit mie pour ce mangier \textsc{.vi.} pains, pour çou k’il ne prendroit pas le milieu en comparison de soi, ains prendroit le mi par soi ; car mi selonc nous est mangier ke ne soit poi ne trop.\par
Et tot artier s’efforcent de tenir le mi en lor ars et deguerpir les estremités, c’est poi et trop. Et la vertu moral est en iceles choses en qui le poi et le trop est desprisable et le mileu est prisable. Donques est vertu \textsc{.i.} abit par volenté, ki par certes raisons et determinees demeure el mi ki est selonc nous.
\chapterclose


\chapteropen
\chapter[{.II.XVI. Encore de ce meismes}]{\textsc{.II.XVI.} Encore de ce meismes}\phantomsection
\label{tresor\_2-16}

\chaptercont
\noindent Le bien ne puet pas estre fet se par une guise non ; més mal fet hom en plusours guises, et pour ce est forte chose et penible a estre bons et legiere chose a estre mauvais ; et c’est l’achoisons por quoi avient que plus des gens sont mauvais ke bons. Et aucunes sont si mavaises dou tout que il n’i puet estre trovés aucun mi, pour çou qu’eles sont mavaises en tout, si com est larrecin, murtre, avoutire. Et autres choses sont mi tout purement, en quoi n’est nule estremité, si comme est vertu, c’est atemprance et fortece. Et ce avient por çou que le droit mi n’a dedens soi nule estremité.\par
Force est le mi entre poour et hardement ; et chasteté est le mi entre sivre ses volentés et non ensivre nules.\par
Et largece est le mi entre avarice et prodigalité, car prodigues est celui ki se desmesure en despendre et ki faut en prendre. Et avers est celui ki se desmesure en prendre et faut en despendre, mais li largues se tient entre ces \textsc{.ii.} estremités.\par
Et sachiés que liberalités et largece, avarice et prodigalités, sont entor les petites choses moienes, més le mi ki est es dignités et es hautes choses et grans est apelés magnificense, et ses estremités sont sanz propres nons.\par
Le mi en covoitise de dignité et d’onour est droiture de corage, et celi ki trop en desire est apelés magnanimes, et c’est a dire de grant corage ; et celui ki poi en desire est apelés pusillanimes, c’est a dire de povre cuer. Et li hom ki se courece de ce k’il doit et lors et tant et coment il se covient, il est mansuetes ; et celui ki se courece si con il ne doit est apelés iracondus ; et celui ki se courece mains k’il ne doit se est apelés non courouçables.\par
La verité est le mi entre \textsc{.ii.} estremités, c’est dou poi et dou trop ; et celui ki tient le mi lieu entre deus choses est apelés verais, et celui ki se desmesure est veniteres, et celui ki en ce faut est apelés humles.\par
Celui ki tient le mi en choses de jeu et de solas est apelés en grezois eutrepelos, et celui ki se desmesure est apelés jougleour e menestrier, et cil ki i faut est \textsc{.i.} forestier champestre.\par
Celui ki tient le mi a vivre entre les gens est apelés amis et home plaisant, e cil ki en ce se desmesure sans proufit est apelés biscortois, et s’il le fet por son proufit est apelé losengers ; et cil ki en ce faut est apelés home sans escole.\par
Vergoigne est une passions de l’ame, non pas vertu ; et celui ki tient le mi en vergoigne est vergongnous, et ki en ce se desmesure est apelés en grezois tacophia, et cil ki i faut es apelés sans vergoigne et sans front.\par
Et en toutes les passions a mi et estremités ; car quant il avient a aucun nostre voisin bien ou mal, cil garde le mi ki est liés dou bien ki avient as bons et ki n’est pas dolans dou mal ki avient as mauvais ; mais li envieus se duelent de tous biens a qui k’il aviegne.
\chapterclose


\chapteropen
\chapter[{.II.XVII. Ci ensegne por conoistre vertus}]{\textsc{.II.XVII.} Ci ensegne por conoistre vertus}\phantomsection
\label{tresor\_2-17}

\chaptercont
\noindent Trois ordres sont es oevres et es passions, c’est mi et plus et mains, \textsc{.ii.} mauvés et \textsc{.i.} bon. Mais trestout sont contraire entr’aus ; car les estremités sont contraires entre eles, a ce ke poi est contre trop et le mi est contre andeus les estremités, c’est de poi et de trop. Dont il avient ke se tu fés comparison entre le mi et le po, certes le mi entr’aus est li trop ; et se tu feras comparison entre mi et trop, certes le mi entr’aus est le poi. Et si te dirai coment : se tu voloies fere comparison entre proece et paour, certes la proece sera hardemens ; et se tu faisoies comparison entre proece et hardement, certes le proece sera poour.\par
Mais il a grignor contrarietés entre les \textsc{.ii.} estremités ke entre le mi et les \textsc{.ii.} estremités. Aucunes estremités sont plus propinques au mi que autres, car hardement est plus prés a force que a paour, et prodigalités est plus prés a largece que a avarice, et non sivre nule charnele volenté est plus prés de chasteté ke luxure.\par
Et ce avient par \textsc{.ii.} raisons, l’une est selonc la nature dè la chose, l’autre est par nous ; par nature de la chose est que paour est plus contraire a force que n’est li hardemens ; de par nous est par ce ke cele estremités, a qui nous somes plus cheable par usage, est plus lointaine dou mi. Et pour çou que nous somes plus naturelement atornés a ensivre la volenté de la char, covient il que covoitise soit plus contre chasteté que a son contraire.\par
Donques puis ke vertus est en prendre le mi, a qui prendre besoignent maint et grant consideration. Grief chose est a l’home k’il puisse vertueus estre, pour ce ke prendre le mi en tous ars n’afiert pas a chascun home, mais a celui proprement ki sages est et complis en cel art ; c’est a dire que tot home ne sevent trover le point en mi le compas d’un cercle, se celui non ki est sages en geometrie. Tout autresi est des autres oevres, car fere une chose est legiere, mais a fere les en tel maniere com on doit n’apertient pas se a celui non qui est sages en cele oevre.\par
Et chascune oevre ki tient le mi est bele et digne d’avoir merite. Et pour çou devons nous encliner nostre ame au contraire des nos desirs, jusc’a tant ke le mi viegne. Et il est mout grevable d’aler aprés la droite chose, de quoi li plusor forvoient ; més en toutes choses le mi est plus prisable, a quoi nous devons entendre, une heure faisant plus et une autre moine, jusc’a tant ke nous vignons a la certaineté de lui.
\chapterclose


\chapteropen
\chapter[{.II.XVIII. Comment li hom fet bien et mal}]{\textsc{.II.XVIII.} Comment li hom fet bien et mal}\phantomsection
\label{tresor\_2-18}

\chaptercont
\noindent Des oevres ke l’en fet, les unes sont par volenté et les autres sont naturaus, et unes autres sont composees des unes et des autres. Les naturaus contre volenté sont ce a quoi nos constraint nostre corage a fine force, encontre volenté ou por ignorance, ausi comme \textsc{.i.} vent levast \textsc{.i.} home et le portast en autre liu : et a teus homes pardone l’en sovent, mais aucune fois chiet la coupe et la honte sor lui meismes.\par
Et les oevres ke l’en fait par sa propre volenté sont quant uns hom esmuet ses membres ou son corage par son arbitre a aquerre les vertus et les vices, en quoi il est prisiés et desprisiés.\par
Les autres oevres ki sont composees par volenté et par nature sont autresi comme d’un home ki est en une nef tempestee, ki giete hors ses choses pour garantir sa vie.\par
Tout autresi avient il dou commandement dou tirant, ki commande a un home ki est desous sa signourie k’il ocie son pere et sa mere ; car tel commandement k’il obeist est par volenté et contre volenté. Mais plus s’acorde as oevres ki sont par volenté ke par force, pour ce que se tu l’ocis, c’est volenté puis que tu fais le murtre, ja soit ce que tu le faces par le commandement ton signor. Et por ce que en teus choses a peril et ledece, l’en se devroit ançois laissier tuer que faire si ledes oevres.\par
Povretés de sens et de discretion est achoison de mal. Et tot home mauvais ont poi de sens, et n’ont cognoissance de çou k’il doivent fere et quoi non, et par ceste guise mouteplient li mauvais home. Més pensent les gens ke li yvre et home couroucié, quant il font mal, il le facent par ignorance, c’est par non savoir.\par
Et ja soit ce k’il soient non sachant en lor afaire, totefois l’ochoison du mal n’est pas hors de l’ome ; por ce que la science de l’homme ne puet desevrer de lui, s’il n’est forsenés ou en autre maniere par quoi son sens s’en puet aler.\par
Et l’achoisons de ce est concupiscence et ire, ki sont achoison de totes mauvaises oevres que l’en fet par volenté ; car il n’est mie possible chose que l’en face les bones oevres par volenté et les mauvaises sans volenté.\par
Et volenté est plus commune et plus general ke n’est elections, por ce ke volenté est commune as enfans et as autres animaus, mais elections n’apartient pas se a celui non ki se garde d’ire et de concupiscence.\par
Et tel fois desire l’om chose ki n’est pas possible, mais il n’eslist pas chose non possible. Encore la volenté est fin, mais elections est devant le fin ; car nous desirons santé et felicité, mais primes eslisons les choses ki a ce nous amainent.\par
Encore la oppinion n’est pas election, car oppinion vet devant l’election et va aprés ausi, et li hons est apelés bons ou malvés selonc ses elections et non selonc sa oppinion.\par
Encore oppinion est de verité ou de fauseté, mais elections est eslire le bien ou le mal.\par
Encore oppinion est des choses que l’on ne set fermement, mais elections est des choses que l’en set a de certes.\par
Encore ne doit on eslire toutes les choses que l’en desire, mais celes solement sour quoi il a eu conseil devant.\par
Encore ne doit on faire conseil sour toutes choses, mais de celes sour coi conseillent li sage et li cognoissant home ; car des choses as foz et as simples gens ne doit on conseil fere.\par
Mais de griés choses que nous fere poons, dont nous somes en doute de la fin coment ele puisse aler, doit on conseil avoir ; si come est de baillier medecine a \textsc{.i.} malade, et ces autres choses samblables.\par
Encore de chose ki n’apertient a nous ne doit on fere conseil ; car nus ne doit consillier comment les gens puissent abiter de Godimoine.\par
Encore ne nous devons consillier des choses necessaires et perpetueles, c’est du soleil s’il se lieve au matin ou non, et s’il pluet ou non. Encore ne devons nous consillier des choses ki sont en doute, si com est de trover \textsc{.i.} tresor.\par
Encore ne se doit on consillier de la fin, mais des choses devant la fin, c’est a dire que li fisiciens ne se conseille pas de la santé, ne li rectorique de faire croire ses dis, ne celui ki fist la loi ne se conseille de felicité. Mais chascuns d’aus ferme en son corage la fin de cele chose, et prent conseil coment il puisse venir a cele fin, ou par soi, ou par ses amis ; car ce ke l’en fet par ses amis fet l’en par soi. En ceste maniere use il toutes choses ki l’amainent a cele fin, et deguerpist les autres.\par
Volenté est fin, si com il est devisé ça arieres. Et quident aucun que le bien soit ce que l’en desire, et aucun sont qui quident ke les choses que l’en desire soient celes ki samblent estre bonnes. Mais a la verité dire le bien est ce ke bon samble au bon home ; car li bons hom juge des choses ausi come le sain home juge des savours, ki juge le dous si come dous et l’amer pour amer ; mais li malades juge le dous por amer et l’amer pour dous.\par
Tout autresi est dou malvais home, a qui les bones oevres samblent estre malvaises et les malveises li samblent estre bonnes ; et ce avient por ce que au mauvais home samblent estre bonnes iceles choses selonc ce k’il se delitent, et celes malvaises ki ne li delitent.\par
Et maint home sont malade de ceste maladie pour ce que les oevres dou bien et dou mal sont en lor arbitre et en lor election ; car fere bien est en nous, et fere mal ausi. Mais il avient aucunefois des oeuvres autresi comme dou pere a qui samble que ses malvés fiz soit bons.\par
Et que la verité soit que bien fere et malfere soit en nous, apert clerement par ceaus ki la loi firent, car il tormentent ceaus ki malfont et honorent ceaus ki bien font. Et la loi nos conorte de bien fere et de garder nos de males oevres ; més nus ne connorte \textsc{.i.} autre se de ce non ki est en lor poesté, c’est qu’il n’eust dolor des choses ki font doloir, et que l’en ne s’eschaufe pour le feu, et ke l’en n’ait faim ne soif a la defaute des viandes.\par
Ceaus ki la loi firent pugnissent home de cele ignorance, de quoi il est aquoison par la negligence.\par
Car \textsc{.ii.} manieres sont de ignorance, l’une est de qui li hom ki la fet n’est pas ochoison, c’est li forsenés, et de ce ne doit il estre punis ; l’autre ignorance est cele de qui li hom est ochoison, c’est de l’yvre, dont il doit estre punis.\par
Et tout home qui trespassent le commandement de la loi doivent estre punis, ce sont tout li malvés et li non juste, car il meisme welent teus estre. Mais il n’est pas voirsamblables k’il soit non justes contre sa volenté, car il sevent bien les oevres ki si les font malvais estre, et si est en sa signorie dou fere et dou non fere.\par
Ausi comme dou sain ki devient malades por ce k’il ne croit a son fisicien de fere çou ki le maintiegne en santé, et ensi devient il malades par sa volenté : puis k’il est malades ne puet il avoir santé pour ce qu’il le weille.\par
Et celui ki gete une piere, avant k’il le giete est il en sa volenté de geter u non, mais puis k’ele est alee il n’est pas en sa volenté dou reprendre ne dou rechevoir. Tout autresi est il de l’home, car au comencement est il en sa volenté d’estre bons ou malvais, mais puis k’il est mauvés devenus il n’est pas en sa volenté de retorner a bonté et estre bons.\par
Et mauvaistés par volenté ne sont en l’ame solement, més el cors ausinc ; si come est d’un home ki est boisteus ou avugles par nature, a qui les gens doivent fere misericorde ; mais se çou est par sa coupe si come est de celui ki pert ses oils par trop boire ou par larrecin, nul n’en doit avoir misericorde.\par
Donques se chascuns hons est achoisons de son abit et de sa ymagination, il covient ke sans son esprovement il set aucun naturel comencement cognoissable entre bien et mal et ki li face voloir le bien et eschiver le mal.\par
Car cele est sovent bone chose que l’en ne puet avoir par usage ne par ensegnement, mes est en l’ame par nature et est bone et complie par nature ; por ce est il donc provee chose ke vertus ne sont pas par volenté ne contre volenté plus u mains que li vice.\par
Et sachiés que oevre et abit ne sont en l’omme en une meisme maniere més en diverses, car l’oevre dè le comencement jusc’a la fin de l’home est volenté de l’home, més abit n’est pas a la volonté de l’home se au commencement non.
\chapterclose


\chapteropen
\chapter[{.II.XVIIII. De force}]{\textsc{.II.XVIIII.} De force}\phantomsection
\label{tresor\_2-19}

\chaptercont
\noindent Et nous dirons huimés de chascun abit, et premierement de force, ki est mi entre paour et hardement. Car il sont choses de qui on doit avoir paour raisnablement, ce sont vices et toutes choses ki metent home en diffame et en blasme. Ki ces choses ne crient, il est sans vergoigne et sans front, et est dignes d’estre deshonorés ; mais ki en a paour, l’en le doit mout prisier.\par
Et sont aucun home ki sont couart en bataille et sont hardi en despendre deniers. Mais li homme fort ne doute ne plus ne mains k’il li besoigne, et est apareilliés de soufrir ce que mestiers est et tant com li covient ; mais li hons hardis se desmesure en ces choses, et li paourous i faut et est mauvés et chetif. Car les choses ki a douter font ne sont pas d’une maniere mais de plusours ; car choses sont ke chascuns hom doit douter, s’il a saine congnoissance ; et sont autres choses ke chascuns hom redoute, si comme est mort, doleur, povreté, car c’est la propre nature de paour. Mais a criembre ce ki fait une moleste, c’est coer de feme.\par
Et il sont \textsc{.v.} autres manieres de force. L’une est cyteine, pour ce que li home des cités orent force par le commandement de la loi ou por honour conquerre et por eschiver honte.\par
La seconde est par sens et par soutillité, laquele ont li home entor l’office et le mestier k’il oevrent ; car nous veons home bien endoctrinet de bataille ki font oevre de grant proece, pour ce k’il se fient de lor science : ja soit ce k’il ne soit pas fors selonc la verité, car puis k’il connoist mortel peril en la bataille, il s’en fuit et doute plus la mort que vergoigne ; mais cil ki est fors veraiement prise mains sa mort ke sa honte.\par
La tierce maniere de fortece est par furor, si comme nous veons de feres bestes, ki sont fors et hardis par lor furor. Et ceste n’est pas veraie fortece ; car l’ome quant il met tout con cors en peril par ire et par furor, il n’est mie fors, mais cil qui se met en peril par droite cognoissance est fort.\par
La quarte maniere est par fort movement de concupiscence, si comme font les bestes au tans k’il se muevent a luxure ; et maint home fait grant hardement ki ayme par amour.\par
La quinte maniere de fortece est par seurté que aucuns a de ce k’il a eut maintes victores ; et ce avient a home quant il se combat contre \textsc{.i.} autre k’il a maintes fois vencu ; mais quant il se combat contre \textsc{.i.} autre, il s’enfuit et pert son hardement. Mais ki bien dure es choses perilleuses est veraiement fort, ja soit ce ke les circonstances de force soient contristables. Et fortece est plus digne chose et plus noble que chasteté, porce que plus legiere chose est soufrir soi de charnel delit que soustenir les dolereuses choses.
\chapterclose


\chapteropen
\chapter[{.II.XX. De chasteté}]{\textsc{.II.XX.} De chasteté}\phantomsection
\label{tresor\_2-20}

\chaptercont
\noindent Chasteté : est mi entre les delis dou cors, et si n’est pas en trestout ; car ki se delite des choses dont il se doit deliter, et en cele guise et en celui tans et tant come il est covenable, il est chastes. Car cil ki se delite de veoir bele coulour ou bele pointure, ou oïr beles noveles, fables, et chant, et flairier bones odours, n’est pas chastes ne non chastes, ja soit ce k’il se delite tant comme covenable chose est, et lors et en guise ki se coviegne.\par
Et ces \textsc{.iii.} delis n’ont pas bestes. Mais chastité et non chastité est en autre \textsc{.ii.} sens, c’est gouster et touchier ; et ce ont les homes et les bestes communalment, c’est deliter soi en chose k’il menguënt et boivent et es choses k’il touchent ; car en touchier est grandisme delit, et por ce est bestiale chose a ensivre trop le delit dou touchier.\par
Et certes en gouster n’a pas si grant delit comme en touchier ; car celui delit est en eslire les savours ; mais en ce que gouster est autresi come \textsc{.i.} touchier ou por lui norrir soufist le delit ; mais il en sont plusor qui raemplent lor ventre en maniere de beste.\par
Et sont aucun delit ki ne sont de nature, en qui l’om puet bien pechier ; més la non chasteté est uns trespassement de delis corporaus, non pas en choses dolereuses, car en teus est fortece. Et aucune fois est hons non chastes, por ce k’il se dolousist trop quant il ne puet avoir çou k’il desire ; car a paines poroit estre trovés hons ki se delite mains k’il ne doit de corporaus deliz.\par
Dont celui est chastes ki tient le mi entre delit, c’est ki trop ne se delite quant il les a, et ki ne se courouce trop quant avoir ne les puet, ançois se delite tempreement selonc ce ke soufissable est a bone vie de home.\par
On doit contrester au desirier de delit ; car ki se laisse vaincre, la raisons remaint sous le desirier ; et toutefois le desirier a son fain, por ce covient il a home avoir maistre de s’enfance, par qui ensegnement il vive.\par
Et se ce n’est, le desirier sera tozjors avec lui jusc’a son grant aage. Par quoi on se doit estudiier que raisons soit sor la concupiscence, en tel maniere que l’un et l’autre desirent de bien fere.
\chapterclose


\chapteropen
\chapter[{.II.XXI. De largece}]{\textsc{.II.XXI.} De largece}\phantomsection
\label{tresor\_2-21}

\chaptercont
\noindent Largece est mi entre doner et reçoivre ; donques est celui largues et liberaus ki use sa pecune covenablement, c’est a dire ki done chose avenable a qui s’en covient et en celui tans et en cele guise et en cele quantité k’il covient. Mais prodighe est celui ki se desmesure en doner et faut en reçoivre, et li avers fet le contraire.\par
Et digne chose est que larghece soit plus en doner k’en reçoivre, por ce que plus legiere chose est non reçoivre que doner, et plus prisable est cil ki done ce k’il covient que celui ki ne reçoit çou que covenable soit.\par
Et generalment il est plus digne chose en vertus fere bien et droit que garder soi de fere ce dont il se doit garder ; mais totefois ces choses sont en voies de moieneté.\par
Poi fet a loer celui ki atempreement rechoit ; mais celui ki done est prisiés por le proufit que l’om a de ce k’il done. Et celui ki done est tozjors amés, mais a celui ki droitement reçoit est aucune fois mal volu.\par
Cil ki se dieut de ce k’il done n’est mie larghes, por ce k’il ne done pas pour larghece més por vergongne ou par autre passion ; donques celui ki droitement done est larghes. Et li hons larges se paie en soi par poi de chose, pour k’il puisse aidier a maint autres ; et poi ou mout k’il ait, totefois s’efforce de fere oevre de larghece selonc son pooir. Et poi se truevent de larghes homes ki soient riche, pour ce ke richece ne croist por doner mais por amasser et pour garder. Et richece ki est sans travail sieut fere son signour large.\par
Et merveille est que cil ki est prodiges est mains mauvés que li avers ; car il fet proufit a mains homes, et por ce l’aiment li plusour, més li avers ne fait proufit a soi ne a autrui, et pour çou le mesaiment toz homes. Encore i a plus, ke li prodighes puet estre castiiés mais li avers non. Et toute defaute nous atrait a avarice. Naturelement est li hons plus cheables a avarice que a prodigalité, et pour ce s’eslonge ele plus dou mi, c’est de larghece.\par
Maintes manieres sont de largece et petit sont trovees en \textsc{.i.} seul home ; car aucunefois est hom avers en garder ses choses et non est avers en desirer l’autrui, et sont ausi come aver non en garder lor choses mais en l’autrui covoitier. Et ceste covoitise ne puet estre saoulee, et por ce s’efforcent de gaaignier de male part, de putains, de caveterie, et prendre usure, de prester a jeu de dés. Et de ceste maniere sont li poissant home ki gastent les cités et robent les eglises et les chemins, et c’est plus grans pechiés que prodigalités.
\chapterclose


\chapteropen
\chapter[{.II.XXII. De magnificense}]{\textsc{.II.XXII.} De magnificense}\phantomsection
\label{tresor\_2-22}

\chaptercont
\noindent Magnificense est une vertus ki cevre par richece grans despenses et grans meisons ; et l’ome ki est magnifiques est ententis par sa nature que ses afferes soient fet a grant honor, et a grant despens plus volenters que a petis ; et qui en ce faut est apelés parvifiques.\par
Et cest vertu, c’est magnificence, est entour les grans choses mervilleuses, c’est edefiier temples, eglises, et autres hauteces pour l’onour de Nostre Signor ; autresi est ele en fere grans noces, et doner as gens grant herbegerie et grans viandes et grans presens, et a cestui ne covient penser de ses despenses solement, més des autrui.\par
Et en magnificense n’a mestier solement grant richece d’avoir, mais avec la richece covient home ki le sace deffendre et mener les choses en cele guise ki covenable soit, ou il ou autres de son conseil ; et s’il li faut une de ces choses ou andeus, on le doit et puet gaber s’il s’entremet de oevre de magnificense.\par
L’ome ki en ces choses se desmesure est cil ki despent plus ke mestiers n’est, et la ou soufist petite despense il le fet grant ; et ce font cil qui donnent as jougleours et as menestriers et gete en voie le porpre et dras dorés ; et ne fait ce pour amour de vertu mais por sambler as gens k’il soit mervilleus et glorieus.\par
Parvifiques est celui ki es grans choses mervilleuses se paine de po despendre, et ensi corront et gaste la beauté de son afaire ; et par poi d’avoir qu’il garde pert grant honour et grant despens. Ce sont les \textsc{.ii.} estremités de magnificence ; mais ne font trop a blasmer, por ce k’il ne damage ses voisins.
\chapterclose


\chapteropen
\chapter[{.II.XXIII. De magnanimité}]{\textsc{.II.XXIII.} De magnanimité}\phantomsection
\label{tresor\_2-23}

\chaptercont
\noindent Magnanimes est celui ki est atornés a grandismes afferes et s’esleece et s’esjoïst a fere les hautes choses. Mais celui ki s’entremet, s’il n’est atornés a ce fere, il est apelés vanaglorieus. Et cil ki est dignes d’avoir honour et dignité, s’il a poour dou rechoivre et d’entremetre soi de si hautes choses, il est apelés povres de corage.\par
Et magnanimités est extremités en comparison de la chose, més en comparison de l’oevre est mi. La droite magnanimetés n’est pas se es grandismes choses non, si comme est servir a Nostre Soverain Pere ; et de ce naist grant honour.\par
Et a la verité dire, celui ki est magnanimes est li plus grans hom et li plus honorables ki soit ; et il ne sera ja esmeus por petites choses ne n’obeist son cuer a chose laide. Donques est magnanimités courone et clartés de toutes vertus, car ele n’est se par vertu non. Et por ce n’est pas legiere chose a estre magnanimes, ançois est mout fort, car il li covient estre bons a soi et a maint autre.\par
Et se aucuns est drois magnanimes, je di k’il ne quidera ja que l’onor ke l’en li fait soient trop grant, pour ce que nule reverence ne puet estre comparee a ses merites, et a droite balance suefre ce ki li avient par dehors ; car il ne s’orguillist de sa prosperité, ne ne dechiet de ses mescheances.\par
Et noblesce de naissance, signorie, et richece, aident mout a home a estre magnanimes. Et cil est veraiement magnanimes ki a en soi \textsc{.ii.} choses por les queles il est honourés ; et ce est ki est devant dit, en bonté, et compte. Et l’ome ki est magnanimes a toz periz por noient ; car il n’a doute de finer sa vie s’en bien non, puis ke li besoins avient ; et s’esjoïst de bien fere as autres et se vergoigne de reçoivre, pour ce que plus noble chose est doner ke reçoivre. Et quant il reçoit, il se porchace dou rendre et dou contrechangier.\par
Et est negligens en petit despens, més es grans choses et honorables n’est il mie pereceus. Et aime et desaime apertement, non pas en repost ; car chetive chose li samble celer sa volenté, et se moustre aspres selonc droit as gens, se ce n’est as choses de geu et de solas ; et converse bien avec les hommes ki aiment geu et solas, et het toz losengiers si comme gens ki servent a loier.\par
Encore se recorde bien en son cuer de tort fait, mais il s’en estort fet samblant que rien ne li soit. It ne loe pas soi et poi loe les autres, et ne dist vilenie de nului, ne de ses enemis ausi. Et plus cure des grans choses que des petites, pour çou k’il est soufissans a soi meismes. Et si est lasches en ses movemens et en ses paroles, pensans et amesurés en parler ; c’est la sentence dou magnanimes.\par
Et celui ki en ces choses se desmesure est vaneglorieus et beubenciers, et celui ki s’entremet des grans afferes et des hautes choses ausi, comme s’il en fust dignes, et non est, et por ce fait il biaus dras et autres choses aparissans et de grant renomee, por quoi il quide estre enhauchiés : les sages le tienent por fol et por vain homme.\par
Pusillanimes est celui ki est dignes d’avoir les grant dignités et si ne s’en ose entremetre ne reçoivre les, ançois s’enfuit et repont ; et ce est mal, pour çou que chascuns doit desirer l’onour et le bien ki est covenables, dont chascuns erre de tant come il se part dou mi, mais ne sont trop malvés. Et en honor, ki est entour les petites choses, a mi et extremité, por ce k’en eles est trovés plus et mains et mi ; car hom puet bien plus desirer honour que on ne doit et k’a lui n’afiert, et ce font les communes gens.\par
Jusques ci est devisee la comparison ki est entre l’ome larghe et l’omme magnanime, et entre l’omme ki est magnifikes et celui ki ayme honor, et entr’aus et lor estremités. Car les unes sont entor les grignors choses et les autres sont entor les mendres, et le mi est honorable et les estremités sont deshonorables.
\chapterclose


\chapteropen
\chapter[{.II.XXIIII. D’ire et de mansuetude}]{\textsc{.II.XXIIII.} D’ire et de mansuetude}\phantomsection
\label{tresor\_2-24}

\chaptercont
\noindent En ire a mi et estremités ; et hom ki tient le mi est apelés mansuetes, et celui ki se desmesure est apelés iracondes, et celui ki se courece mains k’il ne doit est apelés neant courouçables. Mais cil est vraiement mansuetes ki a ire de ce k’il doit et en cele quantité et en celui leu et en cele maniere ki est covenable ; et iracondes est celui ki en ces choses se desmesure et tost cort en ire mais plusors fois retorne tost et legier, et ce est la millour chose ki en lui soit. Car se tot li visce s’ajoustaissent en \textsc{.i.} seul home, il ne poroit estre soustenus.\par
Li hom ki noient se courece est cil ki n’a point d’ire la ou il se devroit, ne es choses ne en la saison k’il devroit acourouchier. Et tés hom ne fet a loer ne a prisier, por ce que soufrir outrage et vilenie soit malvaisement fete a lui ou a ses amis est chose deshonourable. Et teus home sont aucune fois prisié ki ne font grans manaces, autresi comme li iracondes, ke l’en prise maintes fois k’il soit preus et hardis.\par
Mais il est trop grief chose a deviser par paroles les circonstances de l’ire, se de tant non que chascuns sache ke tenir le mi est chose prisable et tenir les estremités est chose desprisable.
\chapterclose


\chapteropen
\chapter[{.II.XXV. Des choses qi afierent as compagnies des gens}]{\textsc{.II.XXV.} Des choses qi afierent as compagnies des gens}\phantomsection
\label{tresor\_2-25}

\chaptercont
\noindent Après ce deviserons des choses ki afierent a compaignie des gens et en la conversation des homes et en lor parleure, pour ce que tenir le mi en ces choses fet a loer, et tenir les estremités fet a blasmer. Et en tenir le mi doit on estre plaisant en parler, et en demorer avec les gens et en converser entre les homes, et k’il soit de bele compaignie et soit communaus as choses qui se covient et en maniere et en leu et en tans k’il covient.\par
Et ceste conversation est samblable a amistié ; mais tant i a de difference, k’en amistié covient avoir compassion et unité de corage, mais en conversation n’a nule compassion, car on puet bien converser avec \textsc{.i.} home que on ne connoist.\par
Et cil ki en ceste chose se desmesure, et ki trop s’en entremet, et se laisse traitier et mener avec les estranges k’il ne conoist, et avec ses voisins et avec ses non voisins, et sa nature le trait a ce faire, il est apelés bisplaisant. Et celui ki ce fait por gaaignier est apelés losengier ; et ki converse et use mains k’il ne doit est apelés sauvages et hom de male escole.
\chapterclose


\chapteropen
\chapter[{.II.XXVI. De verité et de ses contraires}]{\textsc{.II.XXVI.} De verité et de ses contraires}\phantomsection
\label{tresor\_2-26}

\chaptercont
\noindent Verité et fauseté et mençoigne sont entr’aus contraires en toutes manieres ; et sont en paroles et en oevres. Et li hom honorables et de haut corage use verité en ses dis et en ses oevres, et l’ome vil et de petit corage le contraire.\par
Home verais est cil ki tient le mi entre celui ki se vante et moustre k’il face grans choses et s’enhauce plus k’il ne doit, et entre celui ki se desprise et humelie et ki wet celer et abaissier le bien ki est en lui ; mais li verais reconoist et conferme de lui tant de bien come il a en lui, et non plus ne mains.\par
Por quoi je di que celui ki s’umelie est mieus aornés ke celui ki se vante, por ce que li vanteres ment en dit et en fet. Mais en toz ciaus ki sont dit deseure, est pires de foi de tous cil ki pense de soi çou ki n’en est, por çou k’il ne connoist soi meismes ; et por ce fet il plus a blasmer que uns autres.\par
Et li hom verais est bons et loables, et li mençoigniers est malvais et blasmables, car chascuns dit teus paroles come il est. Et celui ki est verais seulement pour amour de la verité fet plus a loer ke celui ki verais est par concession des choses qi apertienent a torfet. Et celui ki se vante, et monstre de soi plus k’il n’en est, non mie por gaaing d’or ou d’argent, il fet a blasmer si come hons vains ; més cil ki se vante par honor ou por chose profitable ne fet tant a blasmer, car selonc sa oppinion est mervilleus et bobenciers.\par
Et aucun home dient mençongne pour solas, et \textsc{.i.} autre le dient por çou que l’en les tiegne a plus grans, u por gaaignier, ou por aucun delit avoir. Et cil est bas ki s’umelie et çoile de soi les grans choses, por eschivre descorde et travail ; si comme fist Socrates por mener sa vie en repost. Et celui ki se hauce de petite choses est apelés nient.
\chapterclose


\chapteropen
\chapter[{.II.XXVII. Coment hom est cogneus par ses movemens}]{\textsc{.II.XXVII.} Coment hom est cogneus par ses movemens}\phantomsection
\label{tresor\_2-27}

\chaptercont
\noindent Et sont li home jugié selonc les movemens et les meurs de lor cors. Cil qui trop rit si est blasmés, et ki pas ne rit si est sauvages et crueus ; mais li hom liés se traite avec ses compaignons selonc ce que covenable est a bone vie, et li tristes par son joiant n’esmovera ja jeu entre ses compaignons, por çou que jeus decline maintefois en courous u a deshonour, et est commencemens de luxure et des choses ki sont devees en loi ; més en bone compaignie a mout de concorde et d’amour.\par
Vergoigne est passions, et sa naissance est samblable a la poor des choses terribles ; car celui ki se vergoigne por toutes choses en rougist, et cil ki a paour devient pales. Et vergoigne est covenable as enfans et as joenes, por çou qu’ele les trait de pechié. Mais ele est blasmable en viel home, pour çou qu’il ne doit fere chose dont il se puisse vergoignier ; et por çou n’afiert il a preudome de faire laide chose de quoi li coviegne avoir honte.
\chapterclose


\chapteropen
\chapter[{.II.XXVIII. Ci parole de justice}]{\textsc{.II.XXVIII.} Ci parole de justice}\phantomsection
\label{tresor\_2-28}

\chaptercont
\noindent Justice est un loable abit par qui est hons justes et fait oevre de justice et aime les justes choses. Et si com il est abit de vertu, autresi le tort est abit de vice, car l’un des \textsc{.ii.} contraires est congneus par l’autre.\par
Et sont \textsc{.iii.} manieres de justice et de tort ; et li hons est non justes en \textsc{.iii.} manieres, l’une est ki fet contre la loi, l’autre ki trespasse la nature de l’igaillance, l’autre est home avers. Tot autresi est hons justes en \textsc{.iii.} manieres, l’une est ki se maintient selonc la loi, l’autre est ki maintient la nature de l’igaillance, l’autre est qui se paine de gaignier bien et justement.\par
La loi est chose juste, et toutes ses choses sont justes, por çou k’ele commande que l’on face oevre de vertu ; lesqueles oevres font l’ome felix, et conservent en lui oevres de felicité, et vee les mauvaises oevres de cités, et commande que les citeiens facent les hautes oevres et nobles, si come est ordener les eschieles en ost bien et fermement a la mellee, et commende ke l’en se garde de fornication et de luxure, et ke l’en se tiegne en pais, et ke li uns ne fiere l’autre et ne li die vilaines paroles et k’il se garde de tout vilan parlement ; et en some ele comande que l’on face oevre de vertu et se garde on de vices ki sont par volenté.\par
Justice est la plus noble vertu et la plus fors ki soit, et tot home ayment ses oevres, et se merveillent de sa bonté plus k’il ne font de la clarté dou soleil ou des estoiles, pour çou k’ele est plus enterine et plus complie que nule des autres.\par
Et li hom juste use justice en soi et en autrui ; car cil ki n’est bons a soi ne a autrui, il est pesmes et trés mauvais. Et a la verité dire, a estre bons a soi n’est pas assés, mais il li covient estre bons por soi et por ses amis. Et justice n’est pas partie de vertu, ançois est trestoute vertus ; et tort n’est mie partie de vice, ançois est trestous vices.\par
Et sont manieres de vices ki sont veé manifestement, si comme est larrecin, avoutire, enchantement, faus tesmoignage, et traison et mal engin des grans homes ; et une autre maniere de vices ki sont mout cruel et espoentable, si comme est navrer et ocire homes, et autres samblables choses.\par
L’ome juste est ygailleour et tel fois amoiour ; je di k’il est ygailleour entre \textsc{.ii.} et amoiour entre plusours choses. Avons en \textsc{.iiii.} choses, pour çou que \textsc{.ii.} persones sont entre qui il fet justice et de \textsc{.ii.} choses ; car en cele chose meisme puet il avoir ygaillance et desygaillance, por ce ke s’il n’i pooit avoir desygaillance, ja n’i seroit ygaillance ; et ensi est justice en conte de nombre. Et si comme la justice est chose ygal, autresi est la non justice desigal.\par
Et pour çou que li sires de la justice s’efforce d’ygaillier les choses ki ne sont ygaus, donc il li covient l’un ocire, l’autre navrer, l’autre chacier en exil, jusk’a tant k’il soit satisfet a celui ki a l’outrage receu. Et li sires de la justice s’efforce de recovrer, au mi de droite ygaillance, le plus et le mains es choses profitables ; et por ce tolt il a \textsc{.i.} et done a \textsc{.i.} autre, jusq’a tant k’il sont ygal.\par
Dont il covient savoir en quel maniere il doit tolir au plus grant et baillier au maindre, et comme il face fere satisfation des torsfés quant il avienent, issi que ses subtés vivent en bone fermeté d’ygaillance.
\chapterclose


\chapteropen
\chapter[{.II.XXVIIII. Encore de ce meismes}]{\textsc{.II.XXVIIII.} Encore de ce meismes}\phantomsection
\label{tresor\_2-29}

\chaptercont
\noindent Li citein, et cil ki habitent ensamble en une vile, s’entreservent li uns as autres ; car li uns si a mestier des choses d’un autre, si en reçoit et li rent son guerredon et son paiement, selonc la qualité de la chose, jusc’a tant k’il sont en droite moieneté entr’aus. C’est a dire se li fevres a une chose ki vaille \textsc{.i.}, et li cordoaniers a une ki vaut \textsc{.ii.}, et li charpentiers en ait \textsc{.i.} autre ki vaille \textsc{.iii.}, et li uns ait mestier des choses a l’autre, il covient entr’aus avoir aucune ygaillance, porquoi l’une vaut plus ke l’autre, si k’eles tornent a \textsc{.i.} mi ki soit igal entr’aus.\par
Por ce fu deniers trovés premierement, por çou k’il igaillassent les choses ki desigaus estoient ; et deniers est ausi comme justice sans ame, por ce k’il est un mi par quoi les choses desigaus tornent a igaillité ; et puet on baillier et prendre les grans choses et les petites par deniers. Et il est uns estrumens par qui li juges puet fere justice, car deniers est loi sans ame, més li juges est loi ki a ame, et Diex glorieus est loi universale de toutes choses.\par
Le jugeour d’ygaillance juge en fermeté por le maintenement de la loi, et croissent li citein, et muteplient li abiteour en champ et laboreur et terres et vignes. Et pour les mauvaistés ki sont fet es cités avienent li contraire, et a la fin tornent iaus et lor choses a desers et a bois.\par
Et li sires en est cil ki maintient justice, et quant il maintient droit et ygaillance et ne donne dou bien commun plus a soi k’a autrui ; et pour çou dient li plusor sage que les dignités et les signories font connoistre les homes. Les menues gens dient que cortoisie est achoison de la signorie, et li autre dient que l’achoison en est richece, les autres dient de nobilité de sanc, mais li sage dient que la veraie ochoison porquoi li hom est dignes d’estre princes et sires si est la vertu k’il a en soi.\par
Et justice est en \textsc{.ii.} manieres, l’une est selonc nature, l’autre selonc la loi, et cele ki est naturel a une meisme nature en totes pars, si comme avient dou feu, ki monte en haut ou k’il soit. La justice ki est selonc la loi a maintes diversités, si com nous veons des sacrefices que l’om fet, l’un por les ames des mors, l’autre por les aubres engendrer et acroistre ; et andeus ces justices entendent a ygaillance. L’ome ki rent la chose ki li est baillie en garde, por paour non pas por volenté, non est justes par soi més par autrui ; mais cil ki le rent par volenté et par honesteté est drois justes.\par
Li damage ki avienent en la compaignie des homes est en \textsc{.iii.} manieres, l’une est pour erreur et por non savoir, l’autre est par negligence, sans volenté de damagier, la tierce est par malice pensee et par volenté de damagier. Par erreur et par non savoir est quant on fait aucunes malvaises choses et ne quide fere çou, si comme est ocire son pere quant il quide tuer son anemi. Por negligence est quant on n’a cure de sa besoigne ne de l’autrui, selonc ce k’il poroit et devroit.\par
Et ces \textsc{.ii.} manieres ne sont dou tout malvaises, pour çou qu’eles ne sont par malice. Mais quant on fait damage par malice devant pensee ou par sa propre volenté, si k’il n’i a nule circonstance ki escuser le puisse de sa malice, il est mauvés voirement et blasmables, et est hors de la nature d’atemprance.\par
Ignorance, c’est a dire non savoir, est en \textsc{.ii.} manieres, l’une est par nature si come est d’un home ki est fol par nature, l’autre ignorance a l’om par sa propre achoison, si com est de l’yvre ki par yvrece pert la cognoissance de verité.\par
La seure justice est millour ke justice ; mais, a la verité, ou mi verai ne puet estre trovés plus ne mains, por ce que droit mi ne puet estre devisé. Et la voire justice n’est pas cele ki est en la loi, ançois est en Damedeu Nostre Signour et est donee as homes ; et por ceste justice est li hom samblables a Dieu.
\chapterclose


\chapteropen
\chapter[{.II.XXX. De vertu moral et intellectuele}]{\textsc{.II.XXX.} De vertu moral et intellectuele}\phantomsection
\label{tresor\_2-30}

\chaptercont
\noindent Vertu est en \textsc{.ii.} manieres ; l’une est apelee moral, ki s’apertient a l’ame sensible, en qui n’est veraie raison ; l’autre vertu est intellectuelle, ki s’apertient a l’ame raisonable, en qui est intellec et discretion et raison. Donques l’ame sensible sent et eschive et ensiut ce ke li plaist, sans nule porveance de sens. Et por ce dient que concupiscence ne desire, més intellec conferme, et sans lui ne puet estre aucune election. Donques le commencement de l’election est intellec, et election si est desirier intellectuel par achoison d’aucune chose. Et nus ki bien use election ne conseille des choses ki sont alees, car ce ki fet est ne puet estre a fere. Autresi elections n’a pas leu es choses ki sont par necessité ou ki ne sont possibles.
\chapterclose


\chapteropen
\chapter[{.II.XXXI. Les oevres de l’ame}]{\textsc{.II.XXXI.} Les oevres de l’ame}\phantomsection
\label{tresor\_2-31}

\chaptercont
\noindent En ame sont \textsc{.v.} choses par qui ele dist verité en afermer ou en niier, ce sont art, science, prudence, sapience, et intellec. Et la science est par tel demoustration ke autrement ne puet estre, et a chose ke l’en fait est necessaire ne non engendrable ne comparable. Et toutes sciences et disciplines et chascune cose ke l’en set puet on ensegnier, et tout ce que l’en puet apenre est des choses seues, c’est a dire, par comencement ki sont manifestes, et par aus meismes ; et est science por demostrance. Et demoustrance est touzjors veraie, si k’ele ne ment en aucun tans, car autrement ne puet estre, pour çou k’ele est des choses necessaires. Et est aucuns ordeneour de l’art o raison veraie.\par
Preudome et sage est celui ki puet consillier soi et autri es bones choses et es mauvaises ki a home apertienent. Donques est prudence celui abit par qui on puet consillier a veraie raison entor les bonnes et les mauvaises choses de l’home.\par
Sapience est la dignité et l’avantage de l’ome en son mestier ; car quant on dist d’un home k’il est sages en son art, lors est demoustré sa bonté et sa vaillance en celui art. Intellec est cele chose par qui on entent le commancement des choses et forme la fin et le compliement. Raison et science et intellec sont des choses ki naturelement sont nobles.\par
Et bien sont trovees des joenes homes engigneus et ki sont sage par discipline, mais par proudence non, por ce que prudence est es choses particuleres, que nus ne puet savoir se par longue experience non, et a grant experience covient lonc fans ; mais joenes hom a petit tans et poi experience.\par
Prudence amesure les commencemens et la fin et l’issue des choses. Par intellec nous vient solerce et astuce ; et solerce est uns sens par quoi on juge tost et isnel \textsc{.i.} droit jugement et consent legierement et tost a bon conseil ; mais astuce est tozjors encoste le proposement. Et quant li proposement est bons proprement, lors est il apelés astuce ; mais quant il est mauvais, lors est il apelés malices ; et de lui est enchantemens et devinailles, et cil ki ces choses ont ne sont mie sachant ne sage, ançois sont solers et consillié par intellec de nature. Sapience est felicité ke l’en doit eslire por lui, non pas comme chose ki amaine santé, mais comme santé meismes.\par
Les oevres de l’ame sont selonc la mesure de la vertu moral et selonc la mesure de prudence et de solerce et d’astuce. Donques la vertu adrece le proposement de l’ome au droit, et prudence, c’est a dire le sens, conferme les choses et les fet bones et les amaine a justice ; mais malice les corront et les amaine a non justice.\par
Les vertus moraus sont ausi comme meurs de nature ; car nous trovons aucunefois l’ome fort et chaste et juste de s’enfance, pour quoi il apert ke teus vertus sont par nature et sans intellec. Mais la signorie de trestoutes vertus doit estre baillie a la vertu intellectuele, pour çou que nule elections ne poroit estre fait par home sans intellec, et ne poroit estre complie se par vertu morale non. Et ensi la prudence nous ensegne fere ce ke covenable est en cele maniere k’il covient, mais la morale vertu mainne la chose a fin et a compliement par oevres.
\chapterclose


\chapteropen
\chapter[{.II.XXXII. De fortesce}]{\textsc{.II.XXXII.} De fortesce}\phantomsection
\label{tresor\_2-32}

\chaptercont
\noindent Fortece est un abis loables et bons entre hardement et paour. L’ome fort veraiement soustient maintes choses terribles et de grant outrage ; et pour en prendre çou ki covient et pour laissier çou ki fet a laissier despite mort et fet oevre de fortece, non mie por son delit ne por honour conquerre, més pour amour de vertu.\par
Home sont ki oevrent de fortece en lor cité solement, plus por vergoigne ke pour eschivre honte et reproce ; et eslisent mieus a soufrir les grans periz ke vivre vergoigneus.\par
La force de fieres est cele que l’on fet por furor quant il angoisse durement d’aucun tort ki fet li soit et court a fere vengance. Force des animaus est cele ke l’on fet pour acomplir sa covoitise que il fortement desire.\par
Force esperituele est cele que l’om fet pour aquerre pris et honour et hautece. Force divine est cele que li fort home ayment par aus meismes, et li home de Dieu sont mout fort.
\chapterclose


\chapteropen
\chapter[{.II.XXXIII. De chasteté}]{\textsc{.II.XXXIII.} De chasteté}\phantomsection
\label{tresor\_2-33}

\chaptercont
\noindent Chastetés est atempremens en mangier et en boivre et en robes et en tous autres corporaus delis dou siecle. Et celi ki oevre atempreement entor ces choses fet molt a loer ; et li sourplus est blasmables, més le poi ne se trueve gaires. Chasteté est bele chose, pour ce que li chastes se delite de covenables choses, et au tens er en leu et a la quantité et a la guise ki covient. Mais li delis du siecle, desevrés de nature, est desmesureement blasmables plus ke avoutire, et ce est gesir avec les malles. Non chasteté puet estre en \textsc{.ii.} manieres, c’est en boivre et en mangier et en toutes manieres de luxure.
\chapterclose


\chapteropen
\chapter[{.II.XXXIIII. De mansuetude}]{\textsc{.II.XXXIIII.} De mansuetude}\phantomsection
\label{tresor\_2-34}

\chaptercont
\noindent Mansuetudes est uns abis loables entre le poi et le trop de l’ire ; et celui ki trop dure c’est par malice, et la malicieuse ire quert grant vengance por poi d’ofension ; mais celui ki ne se courece ne a ire par offension ke l’en face a lui ou a ses amis, est home li qui sentemens est mors.
\chapterclose


\chapteropen
\chapter[{.II.XXXV. Encore de largece}]{\textsc{.II.XXXV.} Encore de largece}\phantomsection
\label{tresor\_2-35}

\chaptercont
\noindent Liberalité et magnificence et magnanimité ont entr’aus communité, car tous \textsc{.iii.} sont en doner et en reçoivre pecune a qui s’afiert, et dont et coment et lors k’il covient. Et plus bele chose est a home ki a ces vertus doner ke prendre, car il eschive le lait gaaing, mais li hons avers covoite forment l’avoir. Et pour çou avient il ke hom liberal n’a pas tant de possession com li avers.
\chapterclose


\chapteropen
\chapter[{.II.XXXVI. Encore de magnanimité}]{\textsc{.II.XXXVI.} Encore de magnanimité}\phantomsection
\label{tresor\_2-36}

\chaptercont
\noindent Li magnanimes desiert bien enterine vertu, car il lor fait grant honour ; et sont covenables a lui, car il apareille s’ame a hautes choses et despise les vils persones et de petit afere ; mais celui ki despent et gaste por nient les grans choses est prodigues. Envieus est cil ki se contriste de la prosperité et de tous biens, des bons et des mauvais sanz difference nule ; et li contraires de lui est liez en prosperité, et des bons et des mauvais. Et li mi entre l’un et l’autre est joians de la prosperité des bons et angoissent de la prosperité des mauvais. Cil ki de toute chose se vergoigne est non aparissans. Cil ki se vante et mostre d’avoir tous biens et desprise les autres, est apelés superbes et orguilleus.
\chapterclose


\chapteropen
\chapter[{.II.XXXVII. Encore de compaignie}]{\textsc{.II.XXXVII.} Encore de compaignie}\phantomsection
\label{tresor\_2-37}

\chaptercont
\noindent Et il a une maniere de gent qui vivre est mout grevable, pour çou k’il ont nature que l’on ne puet traitier ; li autre sont losengiers, ki samblent estre amis de chascun ; et autre sont ki tienent le mi entre ces \textsc{.ii.}, car il sevent estre igaus entre la gent selonc çou k’il covient, et ou et coment, et çou est mout loable. Gengleour est celui ki gengle entre les gens a ris et a gieu, et moke soi et sa feme et ses fiz et tous autres. Et son contraire est cil ki tousjors se moustre cruel et sa face torblee, et ne s’esleece avoec les autres, et ne parle et ne demeure avoec ceaus ki s’esleecent ; mais cil ki tient le mi entr’aus use la moieneté amesureement.
\chapterclose


\chapteropen
\chapter[{.II.XXXVIII. Encore de justice}]{\textsc{.II.XXXVIII.} Encore de justice}\phantomsection
\label{tresor\_2-38}

\chaptercont
\noindent El home justes est apelés igaus pour çou k’il igaillist les choses ; et c’est en \textsc{.ii.} manieres, l’une est departir pecune et dignité, l’autre est sauver et apoier ceus ki ont recheu tort, et çou k’il doivent fere l’un a l’autre.\par
Et li fet ke li home s’entredoivent fere sont en \textsc{.ii.} manieres ; l’une est par propre volenté des le commencement, l’autre est contre volenté, ce sont les choses que l’on fet a force, si comme est par decevance ou par rapine ou par larrecin. Cil ki fet la loi, sanne et adrece les choses ki sont entre poi et trop, et li justes adreceours depart la pecune et la dignité et fet partison entre \textsc{.ii.} au mains.\par
Et justice le fet en \textsc{.iiii.} choses, dont la premiere a proportion a la seconde, et la tierce a proportion a la quarte, et l’adrecement d’eles est selonc la proportion a soi meismes. Et justice juge entr’aus selonc la quantité de la vertu et de la desierte.\par
Et cil ki saine et sauve les fais et les choses ki entre les homes sont est cil ki fist la loi, et esgarde et fet justice entre ciaus ki font les torsfés et ciaus ki les reçoivent, et rent les iretages a ceaus qui il doivent estre, et si le tolt a ciaus ki le tenoient contre justice ; et aucunes choses commande il en persones et aucunes en avoir, et ensi adrece le poi a trop. Car cil ki fet le torfet a plus ke sien n’en est, et cil qui il est fete en a mains k’a li n’apertient. Et li juges adrece entr’aus selonc mesure d’arismetike, et li home vont devant le juge, pour çou k’il est justice plaine d’ame, a çou k’il atorne la justice selonc ce ki est possible.\par
Et justice n’est pas en chascun lieu, en tel maniere k’a celui ki fet soit tant fet come il fet, et a celui ki tolt soit tant tolu comme il tolt, pour ce que li adrecement n’est pas entour ce toutesfoies.\par
Et si comme li justes est mieudres ke li non justes, autresi li hons igaus est mieudres que li non igaus. Et le mi est aucunefois plus contraire a l’une estremité que a l’autre, et l’une estremité plus contraire a l’autre ke au mi. Justice est mi entre gaaignier et perdre, et ne puet estre sans doner et prendre et changier ; car li drapiers done drap pour autre chose dont il a mestier, et li fevres done son fier por autre chose. Et pour ce ke en ces choses avoit grant paine fu une chose trouvee ki l’adreçast, ce sont denier, por ce que l’uevre de celui ki fet la maison se puisse adrecier a l’oevre dou cordoanier par deniers.\par
Seure justice est millour que justice, donques celui ki est millours que bons est bons en toutes les manieres ki puent estre. Et cil ki est plus justes ke li justes est justes en toutes manieres ki puent estre.\par
Et justice naturel est millour que cele ki est mise par homes ; ausi comme li mius ki est dous par nature, et pour çou est il plus dou ke oximel ki est fait par art. Et li hom justes vit par divine vie, por le grant delit k’il a de la naturel justice ; et use les justes choses et les aime par eles meismes. Et cil ki met la loi ne le doit pas metre general en totes oevres, pour çou k’il n’est mie possible que universel regle soit maintenue en chascune chose partie. Donques doivent les paroles de la loi estre particuleres, pour çou k’eles ; iugent des choses particuleres devisees et corruptibles.
\chapterclose


\chapteropen
\chapter[{.II.XXXVIIII. Des visces en moralité}]{\textsc{.II.XXXVIIII.} Des visces en moralité}\phantomsection
\label{tresor\_2-39}

\chaptercont
\noindent Des visces en moralité que l’en doit mout eschiver sont \textsc{.iii.}, malice cruauté et luxure. Car \textsc{.iii.} vertus sont lor contraires, benignité clemence et chasteté. Aucun home sont de nature divine par le trés grant vertu ki en aus habonde ; et cist abis est proprement contraires a cruauté. Et teus homes sont apelés angeliques u divins por l’abondance des vertus ki est en aus outre les us des autres en toutes choses, autresi comme la bonté de Dieu sormonte la bonté des homes.\par
Autres homes sont crueus en lor meurs, et sont de nature de feres bestes, et sont mout lontain de vertu. Et sont home de nature de beste en parsivre lor volentés et lor delis, et sont samblable as singes et au porcel. Et li home ki parsivent lor volentés sont apelés epichures, c’est a dire k’il pensent dou delit dou cors solement.\par
Et des homes ki sont de nature divine ou de nature de bestes en toutes choses sont poi el monde, mais cil ki vivent a loi de beste habitent ens estremités de la terre ki puplee est ; car en droit midi sont li etyopien, et par devers septentrion sont li esclavon.\par
Et est li hons apelés de divine nature pour çou k’il est chastes et continens, en ce k’il se suefre des mauvaises concupiscences dou cors par la force de la vertu entellectuele ; mais celui ki ne s’en suefre est vencus par ses desiriers et trespasse les bonnes de la loi.\par
Car li homme ont lor bonnes a quoi il s’esmuevent naturelement et entre quoi il se regirent et tornoient dedens le mi, se autre oquoison ne vient en lor nature ki le face decliner a vie de beste. Car les bestes sont desliees, et pour çou ensivent les movemens de lor covoitises, et vont par mi les pastures, et ne se suefrent des choses a quoi lor nature les amaine. En cest maniere ist li hons hors de ses bonnes, et ensi est il presk’une beste, por la mauvaise vie k’il a eslevee, a ce ke la sience de l’omme est veritable en ses oevres.\par
Li hom ki set et ki aprent et ki use son sens entor la vertu moral et entour vertu divine et vertu entellective, il vet a ses bonnes et se tornoie entor son mi est use proposition universele, ki concloient saine conclusion.
\chapterclose


\chapteropen
\chapter[{.II.XXXX. De delit}]{\textsc{.II.XXXX.} De delit}\phantomsection
\label{tresor\_2-40}

\chaptercont
\noindent Aucunes choses sont delitables par necessité et autres par election, ou il en a aucunes ke l’en doit eslire par aus meismes et aucunes doit on eslire pour grasce d’autres choses. Li delit par necessité sont en mangier et en boivre et en habiter avec femes et en tous delis corporaus en quoi on vit chastement ; l’autre delit que l’om eslit par aus meismes sont cest : intellec, certaineté, sapience, et divine raison. Mais li delit ki sont esleut par grasce d’autre chose sont cest : victore, richece, honour, et tous autres biens en quoi les bestes ont aucune comunité a nous. Et celui ki tient le mi en ces choses est loables, en celi ki en fet trop u poi est blasmables.\par
Et sont aucun delit por nature, et aucun sont de manieres de bestes ou de feres, ou par raison de tans ou de maladie, ou par usage u par male nature. Et delit de fere est celui ki se delite en ovrir le cors de dames grosses pour saouler soi des fiz k’eles portent dedens lor cors, et en celui ki manguë char d’ome ou char crue.\par
Delit ki est par maladie ou par mal usage, si est oster soi les paus des sorciz ou mangier ses ongles ou boe ou charbons. Delit par male nature est gesir avec les malles, et des autres choses deshonorables.\par
Et sont aucunes crueles malisces a guise de feres sauvages par maladie, si come avient de frenetiques et de forsenés et de melancoliques. L’omme furibondes tient a sentence tout çou ke li plaist, et ne li chaut se c’est contre les autres gens. Et se raisons vieut k’ii set ire \textsc{.i.} poi, maintenant court a la grant ire et fet autresi comme vallés mout isniaus de son cors, ki se haste de fere çou ke commandé li est ançois que li commandemens soit acomplis ; et fet autresi comme li chiens ki brait a chascune vois k’il puet oïr, et ne pense s’il est vois d’ami u d’anemi. Et ceste incontinence ki est en l’ire si vient de chaude nature et d’isnel movement.\par
Et porce doit on plus pardoner a cestui, ke a celui ki n’est pas continens en ses covoitises, pour ce que maintenant k’il voit chose ki li delite, il n’atent mie jugement de raison, ançois s’estudie a avoir çou k’il desire. Donques la incontinance de l’ire est plus naturele chose, mais cele de concupiscence est plus en la volenté de l’home ; et concupiscence quiert lieus oscure, par quoi l’om dit k’ele abat et deçoit son fil. Li hom ki mal fet et ne se repent ne puet estre amendés, mais de celui ki mal fet et puis s’en repent puet hom avoir esperance qu’il se puisse amender.\par
Cil ki n’ont intellec sont millour ke ceaus ki l’ont et ne l’oevrent. Car ki se laisse vaincre as petites concupiscenses par foiblece d’intellec est samblables a celui ki devient yvres de po de vin par la foiblece de son cerebre. Li hom ki est continens ki a intellec conferme soi et pardure en veraie raison et en saine election, et ne se desoivre de drois atempremens. Remuer les meurs et les us est plus legiere chose que remuer nature. Et neporquant remuer usage est grief chose pour çou k’il est samblables a nature.\par
Home sont qui cuident ke nul delit soit bons, ne par soi ne par accident. Autre sont qui cuident ke aucun delit soient bons et li plusour non. Li autre quident que tot delit soient bons. Deliz sans respit ne est mie bons, pour çou k’il est de nature de sensualité, ki est commune es bestes ; et pour çou n’est ele samblable as choses complies.\par
Et li hom sages eschive le delit, pour çou k’il encombre et enpeche l’intellec, et fet l’ome oublier son sens, car li enfant et les bestes querent delit. Et li sont de teus delis ki font l’ome enmaladir et li font avoir moleste, mais li sages ne quiert les corporaus deliz se amesureement non.
\chapterclose


\chapteropen
\chapter[{.II.XXXXI. De chasteté}]{\textsc{.II.XXXXI.} De chasteté}\phantomsection
\label{tresor\_2-41}

\chaptercont
\noindent Chasteté et continence ne sont pas une chose, car chastetés est uns abis ki a eues maintes victores contre les batailles des charnaus covoitises, en tel maniere k’il ne redoute jamais aucun assaut de temptation, mais continence est uns abis ki soustient maintes temptations mauvaises, mais toutesvoies il ne se laisse pas vaincre por le raisons et por le sens ki est avoec lui. Donques chasteté et continence ne sont une meisme chose, et de tant se dessamblent comme vaincre et non estre vencus.\par
La non chastetés est uns habis ki fet home pechier es delitables choses, sans grant effort de temptations par ochoison de sa propre mauvaistié, autresi sont comme veneour des delis. Mais celui est non continens ki se laisse vaincre au delit ki le tempte forment. Et non chastes est cil ki sousmet soi mesmes au delit ki ne le tempte pas.\par
Et est hons non continens por la foiblece de la raison et por petite esperance. Dont il n’est mie dou tout mauvais, mais par moitié, et puet estre amendés par cognort de la raison et par longue prueve, maus li non chastes a paines puet estre amendés. Et vertus et malice sont conneues a ce que en la vertu est la raisons saine et en le malice est la raisons corrompue, et maintes fois est ele corrompue par trop de concupiscenses et de mauvais desiriers.
\chapterclose


\chapteropen
\chapter[{.II.XXXXII. De constance}]{\textsc{.II.XXXXII.} De constance}\phantomsection
\label{tresor\_2-42}

\chaptercont
\noindent Constance c’est a dire parmanance est en \textsc{.iii.} manieres, l’une est en l’ome ki est parmanans et ferm en toutes ses opinions, soient voires soient fausses, la seconde maniere est ki n’a nule fermeté ne nule constance, la tierce est de celi ki est parmanans ou bien, et legier depart dou mal.\par
Mais simplement li constans est millour que li mouvables, porce ke li movables se torne a chascun vent, mais li hom ferm et constans ne sera ja esmeus par force de desiriers. Ja soit ce que aucunefois par noble delit il remue sa fausse creance et se consent a verité\par
Il n’est mie cose possible que uns hom soit sages et non continens ensamble, por çou ke prudence n’est en savoir solement, més en ovrer. Mais astuce et non continence maintefoies sont ensanble pour ce que astuce est diverse de prudence, a ce que prudence est solement entor les bones choses, mais astuce est entor les bones et entor les mauvaises.\par
Et li hons sages ki oevre selonc son sens est samblables a celui ki veille, et celui ki n’en oevre selonc sa sience est samblables a celui qui dort u a l’yvre. Car en home est l’abisme des charnaus desiriers, en quoi il ensevelist et noie et transglotist l’oevre de la raison. Et est autresi comme de celui ki dort, car son sens est loiés en son cerebre pour les vapours ki montent en sa teste, autresi comme l’ivre en qui le sorplus dou vin abat le droit jugement.\par
Li hom malicieus est cil ki fet maus as autres gens penseement et par mauvais conseil k’il a porpensé devant et por mauvaisement eslire raison ; et çou est si pesme chose que l’om n’i puet conseil metre.
\chapterclose


\chapteropen
\chapter[{.II.XXXXIII. D’amistié}]{\textsc{.II.XXXXIII.} D’amistié}\phantomsection
\label{tresor\_2-43}

\chaptercont
\noindent Amistiés est une des vertus de Dieu et de l’home, et c’est mout besoignable chose a la vie de l’home ; car homs a besoigne d’amis autresi comme des autres biens, et li poissant home et riche et prinche de la terre ont besoigne de lor amis, a qui il facent bien et de qui il reçoivent service et honour et grace.\par
Et grant seurté ont li home por lor amis ; et de tant comme il est de grignour afaire plus li besoignent avoir amis, por çou que lors ke li degrés de sa grandece est plus en haut puet il plus legierement cheoir ; et li cheoirs est plus perilleus as grignours. Donques sont amis mout besoignable en ce et en toutes angoisses et aversités ke l’on puet avoir, por ce ke bons amis est trés bons refuis et segurs pors.\par
Et cil ki est sans son ami, il est tous seus en ses affaires ; et quant il est avec son bon ami il est acompaignié et en a parfete aide a acomplir ses oevres. Car de \textsc{.ii.} parfais et bons naist parfete oevre et parfaite ententions. Cil ki fet la loi connorte plus citains a avoir charité et amour ensamble que en justice ; por ce que se tout home fuissent juste, encor lor covient avoir avec amistié et carité, pour ce que charités est garderesse d’amistié selonc sa nature, et le deffent de tous assaus de discorde, et destruist totes mellees et malevoillance.\par
Les manieres d’amistié sont cogneues par les manieres des choses amees. Et ces choses sont \textsc{.iii.}, bien, proufit, et delit. Car chascuns ayme çou ki li samble profitable et bon et delitable. Et il covient que nos weillons bien a nos amis ; mais amistié n’est pas sauve en ce solement, car li chasteour welent bien as autres, més il ne li covient pas estre amis por çou ; més chastiemens est une amours qui requiert guerredons samblables a son oevre, et il covient k’il s’entrechastient et s’entreportent amor selonc la maniere de lor amistié. Et en chascune des \textsc{.iii.} manieres couvient couvenables guerredons, et non pas en repost, en tel guise k’il s’entrewelent bien, selonc la maniere de lor amours.\par
Et cil ki s’entraiment por profit et por delit non aiment vraiement, ançois aiment les choses par quoi il sont amis, et c’est delit et proufit. Et pour ce avient que entr’aus dure l’amistié tant come le delit et son profit, et pour çou devienent il tost amis et enemis. Et cele amistié ki naist par proufit est entre homes vieus, et cele ki naist por delit est entre les jovenes ; mais la droite amistié bonne et complie est entre les homes bons, ki sont sensibles en vertus et s’entraiment et welent bien par la samblance des vertus ki entr’aus est.\par
Et ceste amistié est divine, dedens ki sont tot li bien. Entr’aus n’est aucune decevance ne chose d’aucune mauvaistiet, et por cele amistié ne puet ja estre entre bons et mauvais ne entre les mauvais ensamble, mais entre les bons solement. Mais l’amistié ki naist par delit ou par proufit si puet bien estre entre les mauvés et entre les bons, mais ele est toute fois perdue, selonc la perse dou delit et dou proufit, car ce est amistiés par accident.\par
Amistiés est uns loables aornemens entre ciaus ki ensamblent conversent et ki ont compaignie ensamble, et est une trés bele vie par laquele il vivent en pais et en repos. Et celui abit ki est entr’aus n’est pas brisiés par aversités de leu ou des cors. Et nepourquant se la desevrance dure trop longuement ele fet refroidier et oublier amistié. Et por ce dist li proverbes que pelerinages et longe voiage departent amistié.\par
La chose amee en soi a aucun noble bien por quoi ele est amee. Li bons hons ki est amis devient bons amis, et li un aiment l’autre, non mie por passion, mais pour abis. Et chascuns des amis ayme son bien, et li uns fet gueredons a l’autre par sa bone volenté, selonc ygaillance ; et cele est veraie amistié.\par
La participation de ciaus ki communent ensamble, en bien et en mal et en marchandises et en user les uns avec les autres, suelent estre commencement d’amistié ; et selonc la quantité de ces choses est la quantités de lor amistiés. Et çou que li ami ont doit estre comun entr’aus, por çou que amistiés est ausi comme une communités, et chascune comunités desire chose ki vient a lui en concupiscense et en victore et en sapience.\par
Et pour ce furent premierement ordenees les sollempnités de Paskes et les offrandes des sacrefisces et li assamblement des cités, que compaignie et amours nasquist entre les proismes et honours a Damedeu.\par
Et li ancien soloient fere ces sollempnités aprés les meissons des blés, por ce que en celui tans sont li home apareillié a aquerre amistié et a rendre grasces a Dieu des benefices k’il ont receu.
\chapterclose


\chapteropen
\chapter[{.II.XXXXIIII. Des signories}]{\textsc{.II.XXXXIIII.} Des signories}\phantomsection
\label{tresor\_2-44}

\chaptercont
\noindent Seignouries sont de \textsc{.iii.} manieres, l’une est des rois, la seconde est des bons, la tierce est des communes, laquele est la trés millour entre ces autres. Et chascune maniere a son contraire ; car la signourie dou roi a contre lui la signorie dou tirant, a ce que li tirans se chace de fere son proufit solement, mais li rois se pourchace de fere çou que porfitable soit a son peuple, non pas a soi ; et cil est rois vraiement. Car maintenant ke li rois se pourchace de fere son preu et laisse le bien dou peuple, il devient tirans, et sa tirannie n’est pas autre chose ke corruption de sa signorie.\par
Tot autresi quant li bons et li haus hom laissent a fere çou ke bien soit, por quoi la signorie n’isse de sa lignie, et ne consirent lor honor ne lor merite ne lor dignité, lors se change lor signorie et torne a la signorie de lor comune. La signorie de la commune est corrompue por deguerpir les bons us et la lois ki est bone et loable.\par
Li governemens de l’omme a sa maisnie est samblable dou governement dou roi a son peuple, car la conversations dou pere a ses filz est samblable au roi entour les gens de son regne ; més la signorie des bons homes et des grans est ausi comme la signourie des freres, po ce ke li frere ne sont mie divers se par aage non.\par
Et chascune de ces manieres de signorie et de subjection covient amour et justice, selonc le mesure de sa bonté, car le bon signour s’enforce de bien fere a ses subtés ausi com li pastours de son fouc. Mais tant i a de difference entre la signorie dou roi et cele dou pere, que li rois est sires de plus grant nombre de gens que li peres, et li peres est ochoisons d’engendrer et norrir ses fiz et d’aus aprendre ; doncques est li peres sires et signors de ses fiz naturelement, et il les aime de grant amor. Et pour ce doit li peres et li rois estre honourés de cele honorableté ki est a chascun covenable.\par
La justice de chascun est selonc sa vertu, donques doit chascuns avoir plus de bien et d’onour, selonc ce k’il est millour. L’amor des freres est autel comme l’amour des compaignons, pour çou k’il sont vescu et norri ensamble et ont samblance de passion.\par
Mais quant tyrannie sorvaint, la justice est perdue et li amors faut. Li sires et si subtés ont relation ensamble, ausi comme uns arthiers a a son estrument et con li cors a l’ame ; et celui ki use son estrument en fait son proufit, et por ce l’ayme il, mais li estrumens n’aime pas lui. Et li cors n’aime pas l’ame, et estrumens est ausi come uns sers sans ame.\par
Li peres ayme son fil, et li fiz ayme son pere, pour çou que li uns est fés de l’autre ; mais l’amours dou pere est plus forte que l’amours de son fil, pour çou que li peres conoist que li fiz est estrais de lui et fais, maintenant k’il est nés ; mais li fiz ne conoist pas lui por son pere se premierement n’est passé lonc fans, tant que li sens sont compli et la discretions conortee. Et encore ayme li peres son fil si comme uns autres soi meismes, mais ses fiz aime lui si comme chose de quoi il a son nestre et par qui il est.\par
Li frere s’entraiment comme ciaus ki sont estrait d’un principe ; et pour çou est dit que li freres sont d’un sanc et d’une racine, et k’il sont une chose, ja soient il partis et sevrés. Et la chose ki plus conferme l’amor des freres est k’il sont norri et conversé ensamble et k’il sont d’une maison.\par
L’amour c’on porte a Dameldeu et celui ke li fiz porte a son pere sont d’une nature ; car l’une amour et l’autre sont por ramembrance du bien recheu et pour don de grasce. Mais l’amours Deu doit passer l’amour dou pere, por ce que li bien ki de lui vienent sont plus grant et plus noble.\par
L’amistié des parens et des freres et des compaignons et des voisins est graindre que celui des estranges, car de tant come l’achoison de l’amour est graindre, iert l’amistié grignour et les oevres millours. L’amor ki est entre le mari et sa feme est amour naturel, et est plus anciene que celi ki est entre les citains ; et en cesti amour est grant proufit, pour ce ke l’uevre de l’ome est diverse de cele de le feme. Et çou que ne puet fere li uns fet li autres, et ensi chavissent lor afaire. Li fiz sont li liien ki lient mari et moillier ensamble en une amor, pour çou que li fiz est li communs biens d’aus deus.\par
La communité conjoint les bons en une amour, car par ochoison de vertu s’entraiment de bien fere l’un a l’autre. Entr’aus n’a point de chaloigne ne de descorde, ne volenté de vaincre l’un l’autre, ne de revengier soi en non servir, por ce que lors est li amis liés et joians quant il a fet chose ki plaist a son ami.\par
Amistiés sont que l’on apele gaignables, quant li uns ensiut l’autre por entention de gaaign et de proufit. A la fin, s’il n’en puet avoir point, il naist entr’aus grant discorde ; car li uns dist, je te fis ce service et cel autre, de quoi tu ne m’as contrechangié et li autres li redist autre tel, et cele amistiés ne puet gaires durer.\par
Amistiés est samblables a la justice ; et si comme justice est en \textsc{.ii.} manieres, une de nature et autre de loi, tot autresi est amistiés une de loi et autre de nature. Cele ki est de loi est amistiés particulere et marchandable, et en baillier et en reçoivre maintenant sans respit et sans terme.\par
Maint home sont a qui il plest faire bien et covenable, mais toutefois se tient au profitable et laissent ce ke bon est. Et bone chose et covenable chose est a fere bien as autres sans nule esperance d’avoir change ; mais proufitable chose est de servir as autres por esperance de guerredon. Et cestui service fet hom a celui ki a pooir de changier ce k’il rechoit.\par
Li honours n’est pas autre chose que guerredons de vertu et merci dou bien receu ; mais gaains est aides as besoinsgneus, por quoi je di que li plus grant doivent doner gaaing as meneurs, et li meneur doivent fere honour et reverence as plus grans, ce doit estre selonc ce que afiert a chascun.\par
Car en tel maniere se conserve amistiés. L’onour ke l’en doit fere a Deu et a son pere n’est mie samblable as autres honours ; car nus hom n’est soufissans a fere ne l’un ne l’autre, ja soit ce k’il s’en efforcent a lor pooir. En ce doit chascuns metre toute sa force en obeir et en servir et en garder soi de cheoir en aucune malevoillance.\par
Li covenable adrecement d’amistié adrece les manieres d’amistié ki diverses sont, si comme avient parmi les viles, car li cordoaniers vent ses sollers selonc ce k’il valent, autresi sont li autre. Entr’aus est une chose commune amee, par qui il s’apareillent et conferment la marchandise, c’est or et argent.\par
Quant li amis ayme s’amie por son delit et ele ayme lui por son profit, non ayme li uns l’autre par droite amour, certes cele amours sera tost desevree. Et toute amistiés ki est par legiere ochoison tost se part ; mais achoison fort et ferme font durer amistié longhement.\par
Donques amistiés ki est por bien et por verité dure lonc tens, pour çou que vertus ne puet estre remuee legierement ; mais amistiés ki est par proufit se sevre maintenant ke le proufit s’en est osté.\par
Car a dire ke se uns hons chante por esperance de gaaignier, se tu li rens chanter en eschange, il ne se tenra pas apaiés, pour çou k’il atendoit autre guerredon. Donques n’aura il concorde en marchandise se par volenté ne sont concordees, et ce avient quant chascuns reçoit çou k’il desire en eschange de cele chose k’il done.\par
Et tel fois vaut plus reverence d’onour ke pecune selonc ce que voloit Pitagoras, k’il voloit de ses disciples reverence d’onor et non pas de pecune ; mais es autres ars mecaniques l’om demande pectine. Ce n’avient pas en philosophie, car de tant com li affere sont plus noble, tant il covient avoir plus noble guerredon, pour ce ke a celui ki nous ensegne science ne devons nous pecune mais honor et reverence, autresi comme a Deu et au pere.\par
Donques covient il conoistre les dignités de chascun home, por fere a tous honour et reverence selonc son degré ; car autre honour doit on fere a son pere que a son frere, et autre au signour de l’ost que au peuple, et autresi as voisins et as compaignons que as estranges.\par
Et celui ki fait aucune fauseté en amistié est tant pires ke celui ki fause or et argent, comme amistié est le millour tresor ki puist estre, et si comme li faus deniers est tost conneus, tout autresi la coverte amistié est tost desevree. Li justes despensiers de tous biens est Dameldeus, ki done a chascun son covenable. L’ome ki bons est desire le bien ki est covenable a se nature, et quiert chose samblable a lui, por ce k’il est bons. Et li bons hom se delite en soi meismes pensant es bones choses ; autresi se delite il avec son ami qui il tient et repute si c’un autre soi meismes.\par
Mais li mauvais hom tozjours est en paour et s’esloigne de bonnes oevres ; et s’il est mout mauvés, il s’esloigne de soi meismes, car il ne puet seus demorer sans tristece, por çou k’il li membre de ses males oevres k’il a faites, et blasme sa conscience ; pour çou het il soi et tous homes.\par
Et çou avient por çou que la rachine de tous biens est mortefiee en lui ; et en son mal ne se puet deliter plainement, car tot maintenant k’il se delite en une chose maufete, la nature de son mal l’atret au contraire de celui delit. Et a ce ke li mauvais est partis en soi meismes, si covient k’il soit en continuel travail de penser et plains de mout d’amertume et yvre de laidece et de perversité ; et k’il soit destors par misere nient ordenee.\par
Dont nus ne puet estre amis de tel home, pour çou k’il n’a en lui nule chose ki a amer face. Certes itel misere et itel male aventure n’aura ja medecine, par quoi ele puisse a bien venir. Donques chascuns se gart k’il ne se laisse cheoir en tel trebuchement de malice et d’iniquité ke l’om ne puet raembre, ançois se doit efforcier chascuns k’il viegne a la fin de bonté, par qui il se puisse deliter en soi meismes et avec son ami.\par
Confors n’est pas amistiés, ja soit ce k’il le samble estre. Mais li commencemens d’amistié est uns delis asavourés par cognoissance sensible ; et ce poons nous veoir d’un home ki ayme par amours une dame, car tout avant passe uns delitables regars, mais li ferm liens ki tozjors est avec l’amistié et ki point ne se desoivre, si est delis.\par
Celui abis dont premierement naist le confort puet estre apelés amistiés par samblance jusques a tant k’il croist par longhece de tens. Et l’office dou confort s’afiert au preudome et ferme ki soit grief en moralité de sa vie et a proece acoustumés et de toutes vertus, et plains de bone sapience et de bone opinion et de concorde esirans d’amour.\par
Por ce doivent estre ostees toutes descordes et malvaises pensees d’entre les nobles compaignies des homes, si k’il puissent vivre en pais et en concorde de propre volenté, cele chose ki plus aide a maintenir et governer les dignités des vertus et ses oevres.\par
Et la concorde des opinions est es bons homes et pour çou sont parmanans dedens aus et es choses de hors, car toutesfois jugent bien et welent bien. Li mauvés home poi s’acordent a lor oppinion, car il n’ont en amistié nule part, et por acomplir lor desiriers suefrent il maintes painnes et maint travail, non mie por amistié. Et sont es mauvais homes maintes males soutilleces por engignier ciaus ki a aus ont afaire ; pour ce sont il tousjours en angoisse et en paine.
\chapterclose


\chapteropen
\chapter[{.II.XXXXV. De service}]{\textsc{.II.XXXXV.} De service}\phantomsection
\label{tresor\_2-45}

\chaptercont
\noindent Li bienfaitour ki font bien as autres ayment plus ciaus a qui il font les biens k’il ne sont amés de ceaus ki les reçoivent ; pour ce que li bienfaitour ayment par droite liberalité, mais ciaus ki reçoivent ayment par dete de grasce. Car bienfaitour est en leu de presteour, et cil ki le benefice reçoit est en lieu dou depteour. Li presteours ayme plus son depteour que il lui, et maintefois se contourbe li depteres quant il encontre son presteour, por çou k’il li sovient de çou k’il doit rendre et de ce qu’il a receu ; mais li presteors est liés quant il l’encontre, por çou k’il achate sa bienweillance et ayme son salu et son preu. Et aucunefois avient que cil ki a receu moustre k’il ayme plus son bienfaitour que il lui. Et ce font il pour çou ke l’on ne le blasme de non connoistre le benefice.\par
Et encore le recet dou benefice est faiture dou bienfaiteur ; et chascuns ayme plus sa alture k’ele lui, et especiaument les choses ki ont ame ; nés li poetes ayme ses vers durement. Et l’achoison por quoi li home ayment naturelement sa faiture si est que la derreniere perfection de ce que l’on fet est sa oevre. Et quant une chose est sans oevre, ele est comme possible et est usee par oevre.\par
En \textsc{.iii.} manieres se delitent li home, ou pour çou k’il usent presentielement, ou en l’esperance k’il ont en aucunes choses ki doivent avenir, ou en recordant d’aucune chose ki est alee.\par
Les bones oevres et nobles ont delitable ramembrance par lonc tans aprés, mais les oevres charneus et les oevres profitables poi durent en memores. Et ce avient por çou que li hons ayme plus çou k’il a aquis a grant paine et a travail que une autre a qui il est venus legierement, si con chascuns puet bien veoir d’un home ki ait grant avoir gaaigniet par son grant travail et par sa paine, k’il le garde plus et despent mains que celui ki l’a gaaignié sans nul travail. Et pour ce la mere ayme plus son fil pour la grant paine k’ele i soufri quant ele l’enfanta et le norri.\par
Recevoir benefice avenablement est sans travail, mais fere le selonc ce k’avenable est est grant travail ; por quoi il avient que li bienfaitour ayment plus ceaus a qui il le font ke cil ki le bien reçoivent n’aiment eus.\par
Aucun home sont ki ayment trop eus meismes ; et ce est laide chose, pour ce ke li mauvés home font toutes choses a lor proufit, mais li bon home font lor oevres par entention de vertu et de bien ; et lor oevres sont totes plaines de vertu et en aus croissent oevres de vertu. Et sont home ki ont amé si noble oevre et si noble entention k’il laissent lor profit pour celui de lor ami, pour ce que nobles oevres sont en ramenbrance de lonc tans.\par
Et que ton ami soit uns autres toi se prueve par le proverbe, ki dist k’entre amis est uns sans et une ame, et ont toutes choses communes selonc droit, et est li \textsc{.i.} a l’autre si comme li genous a la jambe et si come li nés a la face et comme li doi a la main. Et por ce doit on amer son ami autant comme soi meismes, par amistié de verité, non pas de delit corporel, ki s’afiert a l’ame bestiele et as choses ki sont hors de verité.\par
Donques celui ki aime soi meisme vraiement face oevre ki apertient a la propre vertu de son estre, selonc les millours choses et plus hautes ordenees selonc verité. Et li bons hom fet bien a son ami et li done argent et iretage, et s’il li besoigne met soi meismes por lui a la mort, et obeist a la raison por conquerre bien a soi.\par
Li compliemens de la felicité des homes est en amis conquerre, car il n’est nus hom ki volsist avoir tot le bien dou monde por vivre solitairement, c’est a dire tout seul ; donques covient il a home felix avoir gens a qui il face bien et a qui il departe sa felicité. Et porce que naturelement li hom converse avec les autres, lor covient acomplir lor defautes par ses voisins et par ses amis, car il ne porroient chevir par aus. Delitable chose est de mener sa vie avec ses amis et partir ses biens avec aus.\par
Bien fere est noble chose et delitable en toutes manieres. Home bien esleu et vertueus et ki facent bien sont poi, mais de proufit et de delit en sont il a grant nombre. Et poi des amis ki sont por aquoison de delit sont assés a la vie de l’ome, car il doivent estre autresi com li condiemens as viandes.\par
Mais amis vertueus amés par li meismes est uns, car il n’est pas possible que uns amans set plus c’une amie, por ce que amors est une sorhabondance ke ne covient se a \textsc{.i.} non ; mais conseil et honestés et covenablece covient a chascun por dete de vertu.\par
Li hom besoigne d’amis au tens de sa prosperité, c’est a dire quant il a tous biens, et au tens de l’aversité, c’est a dire quant fortune li vient contrere. Mais au tans de sa prosperité li couvient il avoir amis ki aient aucune part de son bien et ki le sacent, et au tens de sa adversité li covient il avoir amis par qui il soit aidiés et maintenus.\par
Et la vie des amis ensamble est mout jocunde et plaine de toute leece, et por ce usent il et conversent ensamble au jeus et au vener et en toutes oevres samblables, por user le bien ki est comun entr’aus, et por quoi li uns deviegne mieudres por la compaignie de l’autre, por la resamblance que chascuns a a son compaignon del bien k’il voit en lui et des nobles oevres ki plaisent a chascun l’un de l’autre.
\chapterclose


\chapteropen
\chapter[{.II.XXXXVI. Encore de delit}]{\textsc{.II.XXXXVI.} Encore de delit}\phantomsection
\label{tresor\_2-46}

\chaptercont
\noindent Delis est nés et norris avec nous de le comencement de nostre naissance, pour ce doit on apenre ses enfans k’il se delitent et se couroucent selonc ce que covenable est. Et ce est le fondement de la vertu moral, que puis l’acroissement dou tens acroist la bonté de sa vie, pour ce que chascuns prent çou que a lui delite et eschive ce qui le contriste. Mais maint home sont serf de covoitise, por quoi il covient que lor ententions soient contraires a aus meismes, por ce li uns amis est loables a l’autre, et delitables.\par
Cil ki blasment les covoitises puis les ensivent font croire de lui k’il les aiment et k’il ne les blasment, a certes. Et paroles bonnes et creables proufitent a la conscience de celui ki les dit et meillourent les meurs de sa vie ; mais plus doit li hom croire a l’oevre que au parler. Et li hom discrés enforme et atorne sa vie par teus oevres ki soient acordans a ses dis et a ses fés.\par
La chose qui est desiree par soi meismes est trés bone ; a la vie delitable avec entendement est bone par lui ; et delis est desirés par soi, donques est il bons. Tristece et moleste sont choses mauvaises, et sont contraires au delit ; donques est delit des bones choses. Et l’en fuit tristece et moleste torce qu’eles sont mauvaises.\par
Delis est desirés porce que il est bons et loables ; et se il est joint a bone chose il la fait meillour ; et chascune chose qui fait autre meillor est trés bone. Mais Platons dist que nul delis est bons, et par aventure il ne dist mie verité, porce que en totes choses est aucune bonté naturelement. Et il puet bien estre que li uns maus est contraires de l’autre, et andeus font a eschivre ; mais li bien sont tous samblables, et le doit on penre et eslire. Et li habis de vertus reçoit plus et moins, car li hom puet estre justes et chastes plus et mains, autresi avient il de la santé du cors et dou delit k’il reçoivent plus et mains.\par
Delis n’est mie movemens, car chascune chose ki se puet movoir par lui a sa propre tardeté et sa propre isneleté, mais es choses relatives n’a nul movement par soi. Et chascune chose puet estre corrumpue par ce dont ele naist ; car celui la qui naissance se delite, sa corruption contriste.\par
Delis est en \textsc{.ii.} manieres, li uns est sensibles, et est de par l’ame sensible, li autres est d’entendement, et est de par l’ame entellectuele.\par
Delis est la u li sens est, et sens n’est pas sans l’ame sensible, donques est cis delis de l’ame sensible ; autresi est delis la u l’entendement est, mais entendemens est de l’ame rationable ; donques est celui delis de l’ame de raison. Maintefois vient tristece avant que delis sensibles ; car devant le mangier a esté faim, ki est triste, mais es deliz ki sont par veoir, par oïr, et par odorer, ne vient tristece devant. Autresi avient dou delit de science, et de tous ensegnemens entellectuel. Mais choses ki sont delitables a ceus ki ont perverse nature ne sont pas delitables selonc verité, ausi comme la chose ki samble douce a malade, ou d’autre savour, més a la verité n’est ele pas ensi.\par
Chascuns oevre a son propre delit et son propre deliteour, car justice delite juste, et sapience le sage, et amistiés l’ami. Chascuns s’efforce de faire oevres beles ki li esloignent aucune moleste, mais cil plus ki ont le delit joint avec aus, si comme est la pensee des vertus et lor oevres.\par
Delis est forme complie en tel maniere que a son compliement n’a pas mestier ne tens ne movemens, car movemens ne puet pas estre complis en sa forme en aucun tens ; ja soit ce que aucun movement sont en tens, toutevoies son compliemens est dehors le tens, s’il ne fust circuleres. Et tot home reçoivent delis en oevres et en tens et en movemens.\par
Le delit sensible est selonc la forme dou sens et de la bonté sensible ; por çou est li delis millours quant li sens sont plus fort, et la chose mieus apareillie a estre sentie, ou l’un et l’autre ensamble. Car la bontés de l’uevre est en \textsc{.iii.} choses, ou en la force de celui ki le fet, et en l’acointance de la chose k’il sent, et en la comparison de l’un et de l’autre.\par
Le millour delit ki soit est celui ki est plus parfais et plus complis. Delis est li compliemens des oevres, et il est complis par le sens, et delis est trovés en chascun sens. Et la parmanance dou delit est k’il puisse complir ses oevres selonc ce que beauté est jointe en jovence, tant comme la volenté dou fere dure entr’aus, ki est fermee dedens lor cuers.\par
Et tel delit dure tant comme la beauté de la chose dure ki doit fere le delit. Et quant ces choses falent, si faut le delit ; et por ce ne puet on avoir delit tous tens continuelement, et amenuisent en viellece. Li home ki desirent vivre desirent delit, por ce que delis complist la vie de l’home.\par
Li delis entellectuel est divers du sensible ; et chascuns delis acroist et milloure sa oevre ; et por ce sont multepliet les ars et les sciences que li home s’i delitent. Mais aucun delit enpechent les autres oevres ; car cil ki se delite au son d’une citole oublie sovent çou k’il a entre ses mains, c’est poi et assés, selonc ce que li delis est grans. Delis ki est de nobles oevres fet mout a loer ; delit de villes choses doit on mout deguerpir.\par
Et sont teus delis ki sont divers en generalité, si comme est li sensibles et li intelligibles, et autres delis sont divers en especialité, si comme est de veoir et d’oïr. Et toutes manieres d’animaus ont lor propre delit en qui il se delitent naturelement.\par
La plus noble oevre ki soit est cele d’intellec, et por ce a en lui le plus noble delit. Sor ce disent li ancien que la aprension de l’intellec est plus delitable que ors.\par
Li delit des homes sont divers de grandisme diversité, més drois delis est celui ki plest au bon home de saine nature et de saine vertu ; et por ce fu dit que vertus est atempremens de toutes choses. Donques males choses et laides ne sont mie delitables se a ceaus non ki ont la nature corrumpue, car es homes a maintes corruptions et maintes desigaillances et maint trespassement de nature.\par
Delis ki delite au preudome compli apertient as homes, et sa certainité est conneue quant les oevres ki sont propres a lui sont congneues, c’est a dire l’oevre ki est compliemens de totes humaines oevres.
\chapterclose


\chapteropen
\chapter[{.II.XXXXVII. De felicité}]{\textsc{.II.XXXXVII.} De felicité}\phantomsection
\label{tresor\_2-47}

\chaptercont
\noindent Apres çou que nous avons dit et traitié de vertu et de delit, covient il a dire de felicité et de beatitude, ki est compliemens de tous les biens c’on fait. Et ceste felicité n’est pas en abit, ançois est en cele chose ki est desiree par lui meismes, por ce que felicité est chose complie et soufissant ki n’a besoigne de nule autre chose hors de soi.\par
Li hom ki n’a dedens soi la soufissance des choses, pour çou k’il n’a pas asavouree la douçour dou propre delit de nature, ki es l’oevre intellectuel et ki apertient a la plus noble partie de l’omme, se torne et court au delit dou cors ; dont experience est plus prochaine as choses ki a ce samblent delitables estre, et ne le sont pas ala verité, autresi comme les choses li sont esleues par enfans, ne ne sont mie a eslire selonc voir, mais celes ki sont esleues por home noble et sage ; et vil chose covient a vil home.\par
Felicité n’est pas en gieu ne en choses ki sont de par gieu, més en celes en qui est grant estude et sollicitude et travail ; car repos n’est mie beatitude, por çou que repos est quis por mieus soufrir paine et travail, non pas pour soi. Més la vie dou felix est avec vertu et est es choses bien ordenees ; et pour ce fu dit que entendemens est millours dou ris, car li plus nobles membres fait plus nobles oevres, li millour home font millors choses.\par
Et puis ke felicité est oevre de vertu, est il bien digne chose qu’ele soit de la millour et de la plus complie et ki naturaument est en nous devant les autres, et ce est vertus divine. Et felicité est la fermeté et la constance des oevres propres de vertu. Et nos ancestres ont dit que l’ovrage de ceste puissance est continuele, pour ce ke l’intellec oevre continuelement. La plus perfecte oevre, et la plus delitable ki soit, est felicité ; més li trés millour delit sont trové en philosophie, por la sollicitude de eternité et pour la soutillance de verité, ki sont trovees en ses oevres.\par
Et li delit des sciences est plus savourous et plus delitable au sachant que a ciaus ki le vont querant ; dont il prueve que la verité de la devant dite vertu est trés grans felicités.\par
Et li sage besoignent des choses necessaires a lor vie, autresi comme chascuns autres. Neis les vertus meismes ont mestier des choses dehors, car justice et chasteté et force et toutes vertus ki font et ki oevrent ont besoigne des choses dehors, pour ce que la matire des oevres est dehors ; mais l’uevre de sapience est dedens. Et nanporquant li sages hom oevre plus parfetement quant il a aucuns ki li aident. Donques felicité n’est pas autre chose se l’oevre de ceste poissance non, c’est de sapience.\par
Et d’autre part la felicité de qui nous traitons ore est por achoison de salut ou de pais ; et ce apert bien manifestement et es vertus moraus et en tous citeins que nous combatons por aquerre pais et repos, et a nos et a nos citeins, autresi en toutes les autres vertus citeines, car totes fois beons nous a autre chose dehors, neis l’uevre speculative et de haute pensee est tousjors en pais et en tranquillité. Et covient que tous homes aient complie espasse de vivre, car avec felicité ne doit estre chose ki ne soit complie. Et quant hons vient a ses degrés de felicité, il ne vit par divine vie, mais por la humanité ki est en l’ome ne sont parfet.\par
Et li hom ki a en soi la vie ki est ensi beates ne doit ja penser des humaines choses ne continuer les morteus choses, ançois s’en doit deviser tant com il puet plus, et mener vie noble. Car ja soit ce que li hom set petit le cors, il est trés grans de pooir et d’onour, car chascuns a trés noble vie et trés digne par intellec. Donques la plus delitable chose ki en l’omme soit naturelement si est le oevre de intellec.\par
Les vertus moraus et les citaines sont en grignor paine et en travail ke les vertus intellectives, por çou que a home large et liberal covient avoir richece par quoi il puisse fere l’oevre de largece. Et li justes hom est en grant paine de rendre droit a ceaus ki le demandent, car justice n’est pas en volenté solement, mais en oevre de baillier a chascun son droit. Autresi li hons fors sostient grant ahan pour contrester as choses paourouses, et li chastes est en poine de soi deffendre des charneus desiriers.\par
Mais la vertus speculative ki est de l’entendement n’a mestier des choses dehors a acomplir ses oevres, ançois en sont sovent encombré li bien parfet home. Et li hom ki avenir ne puet a ceste vie ki est si bele et si grant, si se doit vivre a la commune vie des homes.\par
La complie et la parfete oevre de l’intellec speculative si est la fin de la vie de l’ome. Et felicité est example de veraie beatitude ; et c’est manifeste pour çou que nos somes samblable a Dieu et a ses angeles en ceste oevre d’intellec, por çou que Diex et si angele ont la plus noble oevre ki estre pusse, c’est la vie d’intellec, ki tousjors entent continuelement sans nul travail. Et ceste vie beate a plus compliement cil ki est plus samblables a Dieu, ki est veraiement beates.\par
Li hom felix besoigne a avoir plenté des choses dehors, pour ce que nature ne done pas souffisance de teus choses, si comme est santé, service, et teus autres choses qui tozjors sont besoignables. Mais tempree quantités de ces choses sont bien soufissans a home pour estre felix et a fere oevre de felicité, ja ne soit il sires de la terre et de la mer. Et poroit bien estre ke teus sont sousmis a autri, ki mieus sont atorné a felicité ke ciaus ki le segnorient. Et por çou dist bien Anassagoras que felicités n’est pas en richeces ne en signories.\par
Digne chose est que la parole de l’ome sage soit creue quant ses oevres tesmoignent ses dis ; car cil est verais et dist voir, et ses paroles font a croire, quant ses oevres s’acordent a ses paroles. Li hom ki fet ses oevres mout ordeneement, selonc la balance d’intellec, et ayme Deu. Nous devons bien croire ke se Dieus a nule cure d’ome terrien, que il l’a plus grant de celui ki plus s’efforce d’estre samblables a lui, et li done millour benefices, et se delite de lui ensi comme li uns amis aime l’autre.
\chapterclose


\chapteropen
\chapter[{.II.XXXXVIII. Encore de ce meisme}]{\textsc{.II.XXXXVIII.} Encore de ce meisme}\phantomsection
\label{tresor\_2-48}

\chaptercont
\noindent Ki vieut estre felix, il ne li soufist mie savoir ce que en cestui livre est escrit, mais il li covient user toutes les choses qui devisees sont ça en arieres, por ce que ces choses, ki doivent estre complies par oevres, ne est pas soufissant ke l’en les sache ou que on les die, ançois li covient ouvrer ; et en ceste maniere est complie la bonté des homes, c’est por savoir et pour ouvrer.\par
La science des vertus conduist l’omme. Et fet oevres vertueuses celi, di je, ki est bien nés et ki ayme bien selonc verité ; mais celui ki n’est pas a ce atornés ne s’esmuet a garder soi des vices pour amour de la vertu, més pour la paour del torment et de la paine : car ki bien ne pense ne l’oevre mie. Et n’est mie legiere chose trestorner par paroles ciaus ki endurcis sont en lor malice.\par
Home sont ki sont bon par nature, et home ki sont bon par doctrine. Et cil ki bon sont par nature ne sont par vertu, mais par grasce que Dieus lor a donee ; et il sont vraiement boneureus. Eticil ki sont bon par doctrine sont teus ke premierement avoient l’ame ordenee a haïr le mal et a amer le bien. Et ki tés est, il puet avenir a oevre de vertu par amonestement de doctrine, si comme la bone terre fet mutepliier la semence ki en li est getee. Et por ce covient ke li home soient acoustumés et amonesté de lor enfance a amer les vertus et haïr les visces. Et li norrissemens des enfans doit estre nobles, en tel maniere k’il soit apris a fere et user bones oevres par chasteté non mie par continance, car continance n’est mie covenable chose as gens.\par
Et on ne doit oster cest usage ne ces chastiemens maintenant k’il ont enfance passee, mais maintenir le jusques a tant que li drois aages soit complis. Il sont homes ki puent estre governees par chastiement de paroles, et autresi i a ki ne puent estre chastiiés par paroles més par manaces de torment. Et autres homes sont c’on ne puet chastiier, ne par l’un ne par l’autre ; et teus homes doivent estre chaciés, si k’il ne demeurent avec les autres gens.
\chapterclose


\chapteropen
\chapter[{.II.XXXXVIIII. Des governemens de la cité}]{\textsc{.II.XXXXVIIII.} Des governemens de la cité}\phantomsection
\label{tresor\_2-49}

\chaptercont
\noindent Li nobles governemens de la cité fet les citeins nobles et les fet bien ovrer et garder la loi et contrarier as autres ki ne le gardent. Ja soit ce k’il facent bien, maintes cités sont ou li governeour de la vie des homes sont destruit et vivent dessoluement, car chascuns va aprés sa volenté.\par
Li plus covenable governemens ki soit en la vie de l’ome, et a mains de paine et de travail, est celui ke l’om consire de maintenir soi et sa mesnie et ses amis. Et cil puet covenablement maintenir gens ki a la science de cest livre, por çou k’il sauura joindre les ensegnemens universés avec les particulers, car citainnance commune est diverse de la particulere, aussi come en tous mestiers ; car en cescune chose convient il conoistre les particuleres et les universaus choses, pour çou que seule esperiance n’est mie soufissans en ce.\par
Et savoir les universaus choses n’est pas seure chose sans la experiance ; si come nous veons mains mires ki par seule experience sevent maint bien fere a lor mestier mais ensegnier ne le poroient as autres, pour çou k’il n’ont science d’universeles. Donques sera celui parfés mestres de la loi ki set les particuleres choses par experiance et ki set les choses universeles.\par
Home furent qui quidierent ke rectorike et la science de metre loi fussent une chose, et penserent que ceste science fust legiere ; més la verité n’est pas ensi, pour ce que li mestres de la loi doit estre samblables a ses cyteins, et doit savoir cesti art ki les sauura proufitables iert, et autrement nenil. Et s’il commenchast a fere loi sans ceste science, il ne poroit droitement connoistre ne jugier la bonté de sa nature, ne complir la defaute de sa science. Mais pour çou que nous quidons consirer toutes humaines choses par guise de philosophie, si meterons tout avant les dis des anciens sages. Et en ce penserons nous ke les males manieres de vivre corrompent les bons us des cités, et li couvenable les redrecent ; et mousterra en ces livre ki est l’achoisons de male vie dedens la cité, et de la bone, et por quoi la loi est samblable as coustumes.
\chapterclose


\chapteropen
\chapter[{.II.L. Ci fenist le livre aristotle et commence les ensegnemens des visces et des vertus}]{\textsc{.II.L.} Ci fenist le livre aristotle et commence les ensegnemens des visces et des vertus}\phantomsection
\label{tresor\_2-50}

\chaptercont
\noindent Après çou ke li mestres ot mis en roumanç le livre Aristotle, ki est autresi comme fondemens de cest livre, volt li parsivre sa matire sor les ensegnemens de moralités, por mieus descovrir les dis d’Aristotle selonc ce que l’on trueve par mains autres sages. Car de tant comme l’on amasse et ajouste plus de bones choses ensamble, de tant croist celui bien et est de plus haute vaillance. Et c’est prové que tous ars et toutes oevres vont a aucun bien, mais por la diversité des choses covient il que li bien soient divers ; selonc ce que chascune chose requiert, son bien ki est apropriiés a sa fin.\par
Et entre tant diversités de biens, celui est trés millor de tous ki aquiert plus de bonté et de grignor vaillance ; car si comme li hom ot la signorie des autres creatures, tot autresi humaine compaignie ne puet estre sans signor, més plus noble signor ne poroit estre ke home, et ensi est de tous homes ke ou il est sour autri ou il est desous.\par
Et si comme les autres criatures sont por l’ome, tout autresi est li hom por l’ome ; car li sires est por garder ses subtés, et il sont por obeir a lor signours, et li uns et li autre beent au proufit de la comune compaignie des gens, sans tort et sans honte. Et ja soit ce que li uns soient clers, dont li uns nos moustre la religion et la foi Jhesucrist et la glore des bons et l’infier des mauvais, li autre sont juges u mires ou d’autre mestier de clergie ; et li autre soient lais, dont li un font les maisons, li autre coutivent terre gaiaignable, li autre sont fevre u cordewarliers ou d’autre mestier ; que k’il soient, je di k’il sont tout entendant a celui bien ki apertient a la paisible communité des homes et des cités. Pour quoi il apert que li biens ou entent li governeours des autres est plus nobles et plus honorables de tous autres, car il les adrece tous, et tout sont por adrecier lui.
\chapterclose


\chapteropen
\chapter[{.II.LI. Des .iii. manieres de bien}]{\textsc{.II.LI.} Des \textsc{.iii.} manieres de bien}\phantomsection
\label{tresor\_2-51}

\chaptercont
\noindent D’autre part il i a \textsc{.iii.} manieres de bien, un de l’ame, un autre du cors, et une autre de fortune. Mais si comrne l’ame est la plus noble partie de l’omme, ki li done vie et cognoissance et memore, selonc ce que li mestres dist ou premier livre ou chapistre de l’ame, autresi sont si bien sor tous autres, car chascuns officieus ensiut la nature son mestre.\par
Et Aristotles dit que il sont en l’ame \textsc{.ii.} puissances, une ki est sans raison, et c’est comun a tous animaus, et une autre par raison, ki est en l’entendement de l’ome, en quoi est la poissance de volenté, ki puet estre apelee raisnable tant comme ele est obeissant a raison. D’autre part tout bien, u il est honestes ou il est proufitables ou il est entremellés de l’un et de l’autre.\par
Mais coment qu’il soit, ou le bien est desirés par lui meismes, ou il est desirés par autre chose ki ensiut par lui ; car chascuns desire les vertus por avoir beatitude, ce est la boneeurtés et la glore des vertus et des vertueuses oevres, et est la fin et le compliement por quoi on fet les oevres de vertus ; mais cele beatitude n’est pas desiree par autre fin que par lui meismes.\par
Mais cele n’iert ja complie par volenté seulement, ains covient k’il ait compliement de oevre avec la bone volenté ; car si come cil ki fet oevres de chastité contre son talent ne doit pas estre contés por chaste, tot autresi ne parvient on a beatitude par oevre des vertus k’il fait outre son gré. Autresi est cil ki ensit sa volenté sans frain de raison ; il vit a loi de beste et sans vertu.
\chapterclose


\chapteropen
\chapter[{.II.LII. Ci prueve que vertus est le meillour bien de tous}]{\textsc{.II.LII.} Ci prueve que vertus est le meillour bien de tous}\phantomsection
\label{tresor\_2-52}

\chaptercont
\noindent Par ces et par maintes autres raisons apert il tout clerement que en trestoutes manieres de bien celui ki est honeste est trés millour, si comme celi ki governe humaine compaignie et maintient ne honorable ; car vertus et honestés sont une meisme chose, ki nous atret par sa force et nous alie par sa dignité. Tuilles dit que vertus est si gracieuse chose que neis li mauvés ne se puent soufrir de loer les millours choses. Pour çou doit on eslire et prendre les vertus, car li compliemens de la raison de l’ome est a prisier chascune chose tant comme ele fet a prisier.\par
Car en moralités a \textsc{.iii.} parties, une ki devise les dignités, et la vaillance meismement dou proufitable, l’autre ki restraint les covoitises, et la tierce ki governe les oevres. Senekes dist, nule chose n’est plus besoignable que conter chascune chose selonc sa vaillance. Tuilles dist, celui est honestes ki n’a nule laide tache, car honestés n’est autre chose que honours estable et parmanans.\par
Senekes dist, vertus est dou tout acordans a raison. Et Sains Bernars dist, vertus est us de la volenté, selonc le jugement de raison. Senekes dist, la riule des humaines vertus est la droite raison. Tulles dist, li commencemens des vertus sont enracinés dedens nous, en tel maniere que, s’eles peussent croistre, certes nature nos amenroit a beatitude ; mais nous estaignons les brandons que nature nos a donés.\par
Sains Bernars dist, totes vertus sont en home par nature, et por ce ke vertus est par nature, s’ajoste ele avec l’ame. Senekes dist, vertus est selonc nature, mais visce sont si anemi. Aristotles dist, vertus est abis de volenté governés par moienetés ; selonc nos vertus est la moienetés entre \textsc{.ii.} malices dou sorplus et de la defaute. Boeces dit, vertus tient le mi.\par
Augustins dist, vertus est la bone maniere dou corage par qui nus ne fet mal, et que Dieus fet en nos sans nous ; c’est a dire k’il le met en nous sans nostre aide, mais l’oevre i est par nous ; autresi comme, se tu ovroies une fenestre, certes li solaus alume la maison sans toi, car sa clarté est sans t’aide, més l’uevre i est par l’aide de toi.\par
Seneques dit, sachiés que cil n’est pas vertueus ki le resamble, mais celui ki est bons en son cuer ; car li sages establist toutes choses dedens soi. Il fu ja uns jors que uns preudons s’enfuioit tous seus nus de sa cité, ki fu prise et arse, ou il avoit perdu sa feme et ses enfans et quanqu’il avoit ; quant \textsc{.i.} autres li demanda s’il avoit rien perdu, nenil, f�st il, car mi bien sont avec moi. Li Apostles dist, trés bone grasce est a establir coer aus bones choses ; bones choses sont apelees celes ki sont communes a nous et a bestes, si comme est ore biauté, santé, et les autres bontés de cors ; meillours sont les bontés de l’ame, si comme est ore clergie, science, et teus autres choses ki nous millourent l’ame par necessité ; mais li trés millour sont vertu et grasce ; et chascuns doit eslire celes ki plus ont de bontes.\par
Seneques dit, \textsc{.i.} seul jour de sage est plus seu que lonc age de fol. Seneques dit, es sages homes est honestetés, mais a la commune gent est la samblance d’onesteté ; car si comme li fus porris samble k’il resplendisse en leu oscur, tot autresi est la bone oevre ki est contre talent. Pour ce dist Sains Matheus, se ta lumiere est tenebres, les tenebres de toi ke seront ? Sains Bernars dist, mieus vaut torbles ors que reluisans cuevres. Et a la verité dire, l’ame de celui ki fait teus oevres est comme cors sans vie et comme l’ome riche ki n’a noient.\par
Boesces dist, nus visces n’est sans paine et nule vertus sans loier. Seneques dist, li loiers des choses honestes est en aus meismes, c’est a dire la leece du cuer. Senekes dist, li vrai fruit des choses bien fetes est en aus, car dehors n’a nul loier soufissable as vertus. Sains Bernars dist, nous ne perdons le delit, mais il sont remué dou cuer a l’ame et dou sens a la conscience.\par
Augustins dist, leece de vertu est autresi comme fontaine de leece ki naist dedens la maison. Seneques dist, tu quides que je te toille maint delit quant je te blasme les choses de fortune ; mais ce n’est pas ensi, ançois te done parmanable leece quant je wel qu’ele n’isse de ta maison, c’est de ton corage, por ensonnier as biens de forains. Seneques dist, tu quides ke cil soit liés ki se rit, mais li corages doit estre joïans. Salemons dist, il n’est nus delis grignours que cil dou cuer. Salemons dist, despis ces choses ki resplendissent dehors, et esleece toi de toi. Macrobes dist, vertus solement font home boneureus. Seneques dist, droite raisons acomplist la boneeurté de l’omme.\par
Vertus est apelee por çou qu’ele deffent son signor a force ; por çou envoia Ihesucris ses disciples a soufrir les grans perils aprés sa passion, avant que lor vertus fussent amenuisies. Saint Luc dist, asseés vous en la cité tant que vos iestes vestu de vertu. Senekes dist, nus murs est deffensables dou tout contre fortune ; por çou se doit on armer dedens, car s’il est asseur dedens, touchies puet il estre, mais vencus non. Tuilles dist, li corages des sages est barrés de vertus autresi com de mur et de forterece. Augustins dist, si comme orgoils u une haine u uns autres visces abat \textsc{.i.} regne, tot autresi le met vertus en pais et en gloire ; car vertus et bonseureus movemens en l’ame, car ele fet d’estable temple, et de desiers fait ele praaus et vergiés. Sains Bernars dist, je croi que, se bestes parlaissent, eles desissent a Adan, vés ci un des nos.\par
Por ce dist li mestres que la biauté de vertu sormonte le soleil et la lune. Mais il i a fiere chose que Augustins dit, li mauvés ont totes beles choses, mais il sont lait. Por ce fist bien Diogenes quant uns lais hons li moustra sa maison, aornee d’or et de pieres, en tous lieus : il li cracha en mi le vis, car il n’i veoit plus vil chose. Salemons dit, li hom sages a precieus esperit ; et ailleurs dit il meismes, mieus vaut uns preudons que mil mauvés ; et dist encore aprés mieus, vaut chien vif que lion mors.
\chapterclose


\chapteropen
\chapter[{.II.LIII. Ci dist tulles des vertus}]{\textsc{.II.LIII.} Ci dist tulles des vertus}\phantomsection
\label{tresor\_2-53}

\chaptercont
\noindent Des vertus dist Tulles ke ancienement ne fu conneue se force non, ke foibleces des homes ne savoit encore nient des autres. Mais toutefoies fu tenus preudome cil ki biense maintenoit contre dolour ; mais li prueve et li assaiemens des choses ki avenoient de saison en saison les aprist puis des autres.\par
Et les ancienes istores le tiesmongnent. Premierement Abel, ki vint por mostrer la non nuisance. A mostrer neteté vint Enoch. A mostrer parmenableté de foi et de oevre vint Noé. A moustrer obeissance vint Abraham. A moustrer chasteté de mariage vint Ysaac. A mostrer soufrance de travail vint Jacob. A rendre bien por mal vint Joseph. A moustrer mansuetude vint Moyses. A moustrer fiance contre le mescheance vint Josué. A moustrer patience contre torment vint Job. A moustrer humilité et charité vint Jhesucris.\par
Sains Matheus dit, aprendés de moi ki sui humles. Sains Jehans devise la charité Jhesucris et sa humilité, quant il lava les piés des apostres.\par
Et por ce ke vertus sont si bons ensegneours et ke li fruit sont si proufitables comme tot sage le tiesmoignent, di je ke l’ame ki en est bien raemplie enterinement est en la joie de paradis terrestre ; car el leu des \textsc{.iiii.} fleuves ki arousent paradis et li donent plenté, l’ame a \textsc{.iiii.} vertus ki l’arousent et li donent maint secours contre la covoitise de la char, el leu que la Bible dist que il est mout en haut por grignor forterce avoir. Encore est l’ame plus haute selonc ce que Senekes dist, li cuers des sages est ausi comme li mondes sor la lune, ou il a tousjours clarté.\par
Autresi puet tele ame estre resamblable au paradis celestiel, pour \textsc{.iii.} raisons. L’une est por çou k’ele est maisons Deu, selonc ce que St. Jeromes dit, que nule chose n’est plus coie ne plus pure que li cuers ou Dieus habite ; k’il ne se delite pas es grans mostiers aorné d’or et de pieres, mais en ame aornee de vertus. L’autre por ce k’il est lieus de clarté : Job dist, savés vos la voie por quoi vient clartés, c’est par les vertus L’autre por çou k’il est lieus de leece, selonc ce que Salemons dist, et li contes meismes en a dit assés ci devant, et dira encore ci aprés.
\chapterclose


\chapteropen
\chapter[{.II.LIIII. Chi semont homme a oevre de vertu}]{\textsc{.II.LIIII.} Chi semont homme a oevre de vertu}\phantomsection
\label{tresor\_2-54}

\chaptercont
\noindent Tous ensegnemens ki connortent home a oevre de vertu si vienent par icele meisme voie a garder soi des visces, mais meismement joenes hom a paines puet estre sages. Mais vertueus n’en ert il ja, selonc ce k’Aristotles dist, pour çou k’il ne le puet pas estre sans lonc assaiement de maintes choses, et lonc assaiement requiert lonc aage. Pour çou trovons nos el premier livre de la Bible, ke sens et pensee d’omme est preste as visces de s’enfance\par
Salemons dist, mar i est a la tere ki a joene roi : il ne puet chaloir, soit joenes d’aage et vieus de sens et de vertus ; povres est de sens et de vertu ki fet aprés la volenté du joene ; et est samblables au roi Roboam, ki se tint plus au conseil des joenes que as bons viellars.\par
Certes volenté ne doit pas estre dame sor la raison, car ele est sa ancele. Salemon dit, serf ne doit pas iestre signor ne avoir signorie sor le prince ; por ce dit il meismes, li mauvés est pris par ses iniquites et chascuns est liiés as cordes de ses pechiés. Augustins dist, je ploroie liiés non par des autrui chaines mais de mon gré, car la ou ma volenté est dame, li us croist et delite, ki puis torne a necessité. Car quant li hom use sa vie as visces, il li samble trop grief li joug des vertus et est samblables a celui ki ist dou lieu tenebreus ke il ne puet soufrir la veue dou soleil, ki est si luisans.\par
Gregoires dit, li mauvés sont tormenté dedens lor cuers por les males covoitises. Augustins dit, corages mal ordenés est poine de soi. Senekes dist, ja soit ce que ton meffet ne soi lors seu des autres, neporquant li travail de ton cuer ne se dessoivre de ce k’il set sont mal. Poetes dit, c’est la miere vengance que chascuns dampne soi de son vice\par
La conscience del malfaisant est tousjors en poine, pour çou ke les oevres de vertu sont moienes choses, et nature meisme se conorte en moineté et se conturbe dou sorplus et de la defaute, si come la veue ki se connorte de la verde coulour, ki est moiene entre blanche et noire\par
Et si comme la preudefeme fu esleechie quant ele engendre \textsc{.i.} biau filz et se dolousist se ce fust uns chas ou autre chose contre nature ; tout autresi s’esjoïst l’ame des oevres de vertu, si comme de son fruit et s’esmaie des vices ki sont contre li. Use adonc tousjors de bien fere. Car Tuilles dit, l’en doit eslire la trés millor voie de vivre, car li us le feront plus legiere.\par
Et por ce que la foiblece des homes est si decheables as visces, dit Senekes, ha, comme li hom est vil chose et despisable, s’il ne s’eslieve sor les humaines choses. Et quant il si est eslevés donc est il nobles, lors est il gentil et de trés haute nature ; car la u la volentés est obeissans a raison, lors di je que la plus noble partie de celui est dame et roine dou roiaume dou cuer. Et cis hom est apelés nobles por les nobles oevres de vertu, et de ce nasqui premierement la noblece de gentiz gens, non pas de lor ancestres ; car a estre de chetif cuer et de haute lignie est autresi comme pot de tiere ki est covert de fin or par dehors.\par
De ce dit Salemons, boneeuree est la terre ki a noble signour ; car la raisons ki li donne noblece abat toutes mauvestiés. Seneques dist, ki est nobles ? celui ki est par nature establis a vertus. Jeromes dist, sovraine noblece est la clartés des vertus.
\chapterclose


\chapteropen
\chapter[{.II.LV. Ci devise de vertu en .ii. parties}]{\textsc{.II.LV.} Ci devise de vertu en \textsc{.ii.} parties}\phantomsection
\label{tresor\_2-55}

\chaptercont
\noindent Vertus sont en \textsc{.ii.} manieres, une contemplative et une autre morale. Et si comme Aristotles dist, toutes choses desirent aucun bien, ki est lor fin. Je di que vertus contemplative establist l’ame a la soveraine fin, c’est au bien des biens ; mais la morale vertus establist le corage a la vertu contemplative. Et por ce volt li mestres deviser tot avant de la vertu moral que de la vertu contemplative, por ce k’ele est autresi comme matire par qui on parvient a la contemplative.\par
Raison comment : vés chi \textsc{.i.} maistre ki veut avoir \textsc{.i.} estrument por percier, et certes il prent matere dure, c’est fier, et puis li fait pointe pour percier ; car autrement, se la matire ne fust dure et il n’eust pointe, il ne poroit venir a sa fin, c’est a percier. Et tout home ki welent aucune chose faire eslist tout avant cele matire ke covenable est a la fin de so entention. Tout autresi doit chascuns eslire la vie active, ki est aquise par les vertus moraus por governer soi entre les temporaus choses ; car puis est il ordenés et apareilliés a Dieu amer, et a ensivre sa divinité.
\chapterclose


\chapteropen
\chapter[{.II.LVI. De la vertu moral en .iiii. parties}]{\textsc{.II.LVI.} De la vertu moral en \textsc{.iiii.} parties}\phantomsection
\label{tresor\_2-56}

\chaptercont
\noindent Tout sage sont en acorde ke vertus contemplative a \textsc{.iii.} parties, ce sont foi, esperance, et charités ; et ke vertus moraus est devisee en \textsc{.iiii.} membres, ce sont prudence, atemprance, force, justice. Mais ki bien consire la verité, il trovera que prudence est le fondement des unes et des autres ; car sans sens et sans sapience ne poroit nus bien vivre, ne a Dieu ne au monde.\par
Por ce dist Aristotles que prudence est la vertu de l’entendement et de la cognoissance de nous, et est la force et li governemens de raison ; nés les autres moraus sont por adrecier les covoitises et les oevres dehors, et ce ne puet on fere sans le conseil de prudence.\par
Mais il est ensinc ke totes ces \textsc{.iiii.} vertus sont si conjointes ensamble que nus hom del monde ne puet avoir l’une compliement sans les autres, ne les autres sans chascune ; car comment puet estre hom sages s’il n’est atemprés et fors et justes ; et coment puet il estre justes s’il n’est sages, fors, et atemprés. Ausi ne puet nus estre fors ne atemprés sans les autres.\par
Or est ce donc une mache quarree por garder l’ome tout environ ; car deriere nous sont posees les douteuses choses que nous ne poons pas veoir certainement. De cele part nous garde prudence, ke tot establist par son sens ; et a destre sont les joies et les leeces et toutes bones oevres, contre qui est assise atemprance ki ne nous laisse pas desmesurer par orguel ne par leece. A senestre sont posees les adversités et les dolours, contre qui nous deffent force, ki nos connorte et asseure contre tous periz ; mais toutes choses c’on volt et set sans nule doute sont autresi comme devant nos oils. Por ce est justice posee par devant nous, car sa vertus n’est pas se es choses certaines non.
\chapterclose


\chapteropen
\chapter[{.II.LVII. De la premiere vertu, c’est prudence}]{\textsc{.II.LVII.} De la premiere vertu, c’est prudence}\phantomsection
\label{tresor\_2-57}

\chaptercont
\noindent Par ces paroles poons nous entendre ke ceste vertus, c’est prudence, n’est pas autre chose que sens et sapience. De qui Tulles dist que prudence est cognoissance de mal et de bien, et de l’un et de l’autre. Et pour çou dist il meismes k’ele vait par devant les autres vertus et porte la lumiere et moustre as autres la voie ; car ele done le conseil, mais les autres \textsc{.iii.} font l’oevre. Et le conseil doit tousjours aler devant le fet, si comme dist Salustes, ains que tu commences, te conseilles, et quant tu te seras conseilliés fai tost le oevre. Car Lucans dist, oste toutes demeures, car tousjors fet mal li atendres a ciaus ki sont apareilliés. Salemons dist, tes oils aillent devant tes piés, ce est a dire que li consaus aille devant tes oevres. Tuilles dist, poi valent les armeures dehors se li consaus n’est dedens.\par
Et li contes a dit ça ariere que prudence est cognoissance de bones choses et de mauvaises, et des unes et des autres, car par ceste vertu set on deviser le bien dou mal, et l’un mal de l’autre. De qui dit Alanus que la cognoissance dou mal nous a mestier por nous garder, car nus ne puet le bien connoistre se par cognoissance de mal non, et chascuns eschive le mal por cognoissance de bien.\par
Por ce di je que sens est si digne chose, k’il n’est nus hom ki ne desire a estre sages ; car il nous est avis que bele chose soit sormonter les autres de sens ; et nous samble male chose et laide de cheoir et de foliier et d’estre non sachans et d’estre decheus.\par
Salemons dist, por toutes tes possessions achate sapience, ki est plus precieuse ke nul trezors. Salemons dist, plus vaut sapience que toutes richeces ; et nule chose amee ne puet estre comparee a lui.
\chapterclose


\chapteropen
\chapter[{.II.LVIII. Encore de prudence}]{\textsc{.II.LVIII.} Encore de prudence}\phantomsection
\label{tresor\_2-58}

\chaptercont
\noindent Ki vieut prudence sivre, il aura par raison, et vivera droiturierement s’il pense toutes choses devant et s’il met en ordene la dignité des choses selonc lor nature, non pas selonc ce que maint home pensent ; car choses sont ki samblent a estre bones et non sont, et autres sont bones ki ne le samblent mie.\par
Toutes choses que tu as transitores, ne les quides pas grandes. Choses que tu aies a toi, ne les garde pas comme s’eles fuissent a autrui, mais par toi comme toues. Se tu vieus avoir prudence, soies uns en toz lieus ; non muer toi por les varietés des choses, mais atourne toi si comme la mains fait ki est une meisme tousjours quant ele est close et quant ele est overte.\par
La nature des sages homes est examiner et penser en son conseil avant k’il coure a chose fausse par legiere creance. Des choses ki sont douteuses ne dones jugement mais tien ta sentence pendant et ne la ferme, pour çou que toutes choses voirsamblables ne sont pas voires, chascune chose ki samble non creable n’est pas fausse. La verités a maintes fois face de mençoigne, et mençoigne est couverte en samblance de verité ; car tout autresi comme li losengiers coevre son maltalent par bele chiere de son vis, puet la fauseté recevoir couleur et samblance de verité, por mieus dechevoir.\par
Se tu vius estre sages, tu dois consirer les choses futures, et penser en ton corage tot çou k’avenir puet. Nule soudaine chose t’aviegne ke tu n’aies devant porveue ; car nus preudom ne dit, ce ne quidoie je mie ; ançois atent et non doute. Au commencement de toutes choses pense la fin ; car hom ne doit teus choses comencier en quoi fet mal perseverer.\par
Li sages hom ne vieut engignier autrui ne ne puet estre engigniés. Les toies opinions soient autresi comme sentences. Les vaines pensees ki sont samblables a songes ne le reçoivre pas ; car se tu te delitoies, quant tu penseroies toutes choses tu seroies tristes ; mais ta cogitations soit ferme et certaine en penser et en consillier et en querre. Ta parole ne soit friole, mais toutefoies soit ele ou pour ensegner ou por penser ou por commander. Loe plus atempreement, mais plus atempreement blasme, pour çou que loer est ausi blasmable comme trop blasmer ; car en trop loer puet avoir suspection de losenges, et en trop blasmer puet avoir suspection de malevoillance. Donne ton tesmoing a verité non pas a amistié ; ta promesse soit par grant consideration et soit le don grignour ke la proumesse.\par
Se tu ies sages hom, tu dois ordener ton corage selonc \textsc{.iii.} tans en ceste maniere : tu ordeneras les presentes choses, et porverras celes ki sont a avenir, et soviegne toi de celes ki sont alees. Car cil ki ne pense des choses passees piert sa vie si comme non sages ; et cil ki ne porvoit les futures si chiet en toutes non sages, si comme hom ki ne se garde. Propose en ton corage les choses ki a avenir sont, et les bones et les mauvaises, si ke tu puisses soufrir les males et atemprer les bonnes.\par
Ne soies tozjors en oevre, mais aucune fois laisse reposer ton corage ; mais garde ke celui reposer soit plains de sapience et de penser honeste. Li sages hom n’enpire de reposer  ; et se aucunefois ses cuers est un poi lasches, il ne sera ja desliiés ne ne brisera le liien dou sens. Il haste les choses tardives, les entrapees delivre et adoucist les aspres ; et por çou k’il fet de quele partie c’on doit coumencier les choses et coment on les doit parfere. Et par les overtes choses dois tu entendre les oscures, et par les petites les grans, et par les prochaines les lontaines, et par une partie dois tu entendre tout.\par
Ne tesmoigne l’autorité de celui ki dist, mais garde a ce k’il dit. Demande tel chose ki puist estre trovee, et desire en toi tel chose ke tu puisses desirer devant tot homes. Ne monter en si haut lieu dont il te couviegne descendre. Lors te besoigne consaus quant tu as vie de prosperité, et ensi te remandra ta prosperité, et en bon leu fermement. Ne te muet trop hastivement, mais garde le leu u tu dois aler et jusc’a ou.
\chapterclose


\chapteropen
\chapter[{.II.LVIIII. Encore de ceste vertu}]{\textsc{.II.LVIIII.} Encore de ceste vertu}\phantomsection
\label{tresor\_2-59}

\chaptercont
\noindent Et pour çou que en ceste vertu sont mis tous sens et tous ensegnemens, apert il qu’ele connoist tous tens ; c’est le tens alé par memoire, de quoi Seneques dist, ki ne pense noient des choses alees a sa vie perdue ; et du tens present par cognoissance ; et du tens a venir por porveance. Et pour ce dient li sage, prudence a \textsc{.iiii.} menbres por governer sa vertu, chascune selonc son office, ce sont porveance, garde, cognoissance, et ensegnement. Et li mestres devisera l’office de tous, et premierement de pourveance.
\chapterclose


\chapteropen
\chapter[{.II.LX. De porveance}]{\textsc{.II.LX.} De porveance}\phantomsection
\label{tresor\_2-60}

\chaptercont
\noindent Porveance est uns presans sens ki enquiert la venue des futures choses, c’est a dire ke porveance est en \textsc{.ii.} manieres et qu’ele a \textsc{.ii.} offices. L’une est k’ele pense et remire les choses ki sont presens, et par icele consire et voit devant tous çou k’il en puet avenir, et qu’ele puet estre la fins dou bien et du mal. Et puis k’ele a ce fait, si se garnist et se conseille par son savoir contre la mescheance ki avient ; por ce doit on veoir devant le mescheance ki avient, por ce doit on veoir devant le mal ki avenir puet, car s’il vient il pora plus legierement trespasser et soufrir.\par
Grigoires dist, por ce ne puet on eschiver les perils ki n’en fu porveus devant. Juvenaus dist, tu as aquise grant deité se prudence est avec toi ; car celui est bonseurés ki puet connoistre la fin des choses. Boesces dist, il ne soufist pas a l’home k’il voie u connoisse les choses ki sont devant ses oils, mais prudence mesure la fin des choses. Tuilles dit, il apertient a bon engien a establir devant çou k’il puet avenir en l’une partie et en l’autre, et que ce soit a faire quant ce sera avenus ; si ke l’en ne face chose, k’il coviegne dire aucunefois, je ne le quidoie.\par
Senekes dist, li consillieres doit amonester home k’il ne se fie de rien en son bon cuer, et k’il oste la fole creance k’il a de sa poissance, qu’ele doie durer tozjors, et k’il li ensegne que toutes les choses que fortune li a donees sont movables et k’eles s’enfuient grignour oire k’eles ne vienent ; et que l’en avale par ces degrés par quoi il monta en hautece ; et k’il n’a point de difference entre la plus haute fortune et la plus basse.\par
De quoi dist Boesces, fortune ne fera ja que les choses soient toues ki sont estranges de toi par l’uevre de nature ; mais li faus ami portent flaterie en lieu de conseil, et toute lor ententions est en decevoir souef. Tuilles dit, maint pechiet vienent quant li home enflent des opinions, puis en sont escharnis laidement. Seneques dit, por çou sont plusour ki ne connoissent pas lor forces ; et quant il quident estre si grant comme il oënt dire, il comencent guerre et choses superbes que puis retornent a grant peril.\par
Li mestres dit, pour ce se doit porveoir et garder chascuns de fausses paroles et de flateries, ki souef dechoivent ; ausi comme li dous sons dou flaüt ki engiegne les oisiaus tant k’il sont pris. Et maintefois li pesme venim est desous le miel ; por ce sont pieur li mal ki sont covert de bien. Catons dist, ne croire de toi plus ke toi meismes. Salemons dist, a paines getera ja larmes li oils de ton anemi ; et quant il verra son. Tans, ne se pora ja saouler de ton sanc Mais Juvenaus dist, il pleure quant il voit larmoier son ami, més de son mal ne se diut il noient.
\chapterclose


\chapteropen
\chapter[{.II.LXI. De garde}]{\textsc{.II.LXI.} De garde}\phantomsection
\label{tresor\_2-61}

\chaptercont
\noindent Garde est garder soi des visces contraires. Ses office est sivre le mi en toutes choses ; c’est a dire que l’en doit si garder son avoir, que por fuir avarice il ne deviegne wasteres, et k’il se doit si departir dou fol hardement, qu’il ne chee en poour Car cil est vraiement hardis ki enprent ce k’il fait a enprendre, et ki fuit ce k’il fait a fuir ; més li paourous ne fait ne l’un ne l’autre, et li fols hardis fet et l’un et l’autre.\par
Salemons dist, garde ton cuer en toute garde. Il dist, en toute garde que tu ne cloes a tes anemis d’une part les portes et d’autre part lor oevre l’entree ; c’est a dire que por garder toi d’un vice tu n’en faces \textsc{.i.} autre plus grant ; car il n’est mie bon a descovrir \textsc{.i.} autel pour covrir \textsc{.i.} autre.\par
Garde toi donc de toutes estremités, ne desirer pas desmesuree prudence, ne sachés plus ke covenable ne soit ; mais sachiés tant comme il vous soufist. Autresi garde toi d’ignorance ; car ki rien ne set ne bien ne mal, ses cuers est aveules et nonveans, il ne puet consillier ne soi ne les autres. Car s’uns avugles en voet guier un autre, certes il meismes chiet en la fosse tout avant, et li autres aprés lui.\par
Ensi donc prudence ki est le mi entre \textsc{.ii.}, ki contrepoise et adrece les pensees et atempre les oevres et mesure les parables. Car si comme des oevres ki ne sont establies par vertu ensiut periz, tout autresi fet il dou parler quant il n’est selonc ordre de raison. Et por ce tout avant ke tu dies ne çou ne quoi, dois tu consirer principaument \textsc{.vi.} choses, ki tu ies, ke tu vius dire, et a qui et por quoi et coument en en quel tens. Raison coment.
\chapterclose


\chapteropen
\chapter[{.II.LXII. Des choses c’on doit garder ançois c’on parole et premierement qui tu es qui parler viaus}]{\textsc{.II.LXII.} Des choses c’on doit garder ançois c’on parole et premierement qui tu es qui parler viaus}\phantomsection
\label{tresor\_2-62}

\chaptercont
\noindent Tout avant que tu dies mot, consire en ton cuer ki tu ies ki vieus parler, et premierement garder se la chose apertient a toi ou a autrui. Et se c’est k’ele apertiegne a \textsc{.i.} autre, ne t’en melles ja ; car selonc loi est encoupable ki s’entremet de ce ki n’apertient pas a lui. Salemon dit, cil ki s’entremet des autrui mellees est samblables a celui ki prent le chien par les oreilles. Jesu Siraac dit, de la chose ki ne te grieve ne te combates.\par
Aprés garde se tu ies en ton bon sens paisiblement sans ire et sans troublement de corage, car autrement dois tu taire et constraindre ton courous. Tuilles dist k’il est grans vertus a constraindre les movemens dou cuer ki sont troblés, et faire tant que ses desiriers soient a raison. Seneques dist, quant ii hons est plains d’ire, il ne voit rien se de crime non. Catons dist, ire enpeche le corage si k’il ne puet trier la verité. Por ce dist uns sages, la lois volt bien l’ome ki est sorpris d’ire, mais il ne voit pas la loi. Ovides dist, vain ton corage et ta ire, tu ki vains totes choses. Tuilles dist, ire soit long de nous, car o lui nule chose puet estre bien fete ne bien pensee ; et ce que l’om fait par aucun troublement ne puet estre parmanable ne plaire a ciaus ki i sont. Pieres Alfons dist, c’est en l’umaine nature que quant li corages est commeus par aucun troublement il pert les oils de la cognoissance entre voir et faus.\par
Aprés garde que tu ne soies courans par desirier dou parler en tel maniere que ta volentés ne consente a raison ; car Salemons dit, cil ki ne puet constraindre son esperit en parlant est samblables a la cité overte ki n’est avironee de mur. Li mestres dit, ki ne se set taire ne set parler. Et a \textsc{.i.} home fu demandé por quoi il estoit si taisans, ou pour sens ou pour folie, et il respondi que fols ne se puet taire. Salemon dist, pose frain a ta bouche, et garde que tes levres et ta langue ne te facent cheoir, et que la cheoite ne soit a mort sanz garison. Catons dit, soverainne vertus est a constreindre la langue ; et cil est prochains de Deu ki se set taire par raison. Salemons dit, ki garde sa bouche il garde s’ame ; et ki ne consire ses dis, il sentira mal.\par
Et se tu vieus blasmer ou reprendre autrui, garde que tu ne soies entechiés de celui meisme visce ; car estrange chose est a veoir une deliee poudre en l’uel d’un autre, et el sien ne voit un grandesime rien. Li Apostles dit, O tu, hons ki juges, en ce que tu juges les autres dampnes tu toi meismes, car tu fés çou que tu juges. Et ailleurs dist il meismes, tu aprens les autres et n’ensegnes mie a toi, tu dis ke l’en ne face avoutire et tu le fais. Catons dist, ce ke tu blasmes garde que tu ne le faces, car laide chose est quant la coupe en chiet sor lui. Augustins dist, bien dire et mal ovrer n’est pas autre chose que dampner soi par sa vois\par
Aprés garde se tu sés ce que tu vieus dire ou non, car autrement ne poroies tu bien dire. Un home demanda son mestre coment il poroit estre bons parliers ; et son mestre li dist k’il desist seulement çou k’il bien parfitement seust. Jesu Siraac dit, se tu as entendement, respont maintenant ; autrement soit ta mains sor ta bouche que tu ne soies pris par niches paroles et soies confus.\par
Aprés garde le fin de tes dis et k’il en puet avenir ; car maintes choses samblent estre bones au comencement ki auront male fin. Jhesu Siraac dist, en tous biens a double mal ; pour ce consire le commencement et le fin et le siute. Panfiles dist, se l’en porvoit le chief et le fin ensamble, la fin en porte l’onour et le blasme. Et la u tu te doutes de ta parole s’il en aviegne ou bien ou mal, je loe ke tu te taises. Pieres Aufons dit, crien de dire çou dont tu te repentes, car a sage home afiert de taire por lui mieus que parler contre soi. Nus home taisans est decheus, més sovent molt parlans est deceus. Et certes les paroles sont samblables as saietes, que l’om puet traire legierement més retraire non : paroles vont sans retour. Tuilles dit, ne fere çou dont tu ies en doute s’il est bien ou mal, car bontés reluist par soi meismes et doute a senefiance de mauvaistié. Senekes dit, folie ne soit ton conseil.
\chapterclose


\chapteropen
\chapter[{.II.LXIII. De connoistre le voir de la mençoigne}]{\textsc{.II.LXIII.} De connoistre le voir de la mençoigne}\phantomsection
\label{tresor\_2-63}

\chaptercont
\noindent Tout çou que tu vieus dire consire se çou est voirs ou mençoigne selonc çou que nous ensegne Jhesu Siraac ki dist, devant ta oevre soit veritable parole et parmanable consel. Pour çou doit on garder verité sor totes choses, car ele nous fait prochain a Deu ki est toute verités. Di donc la verité tousjors et te garde de mençoigne. Salemons dist, lerres fet plus a loer que celui ki tousjours ment.\par
Apense toi a la verité quant ele est dite, ou par ta bouche ou par autrui ; car Cassidorus dit que pesme chose est a despire verité, et verités est tousjors bonne s’ele n’est mellee de fauseté. Senekes dist, les paroles de celui ki ensiut verité covient estre simples, sans coverture nule.\par
Di donc la verité en tel maniere k’ele soit autresi come sairement. Seneques dit, le qui dit n’a fermeté de serement, certes son serement est vil chose ; car ja soit ce que tu ne claimes le non Dieu, ou k’il n’i ait tesmoing, neporquant garde verité et ne trespasse la loi de justice.\par
Et s’il te covient raembre la verité par mençoigne, tu ne mentiras, mais escuseras la ou il a honeste ochoison. Car bons hom ne cuevre pas ses secrés, il taist ce ki ne fait pas a dire, et dist ce ki covient. Salemons dist, je te pri, Deus, de \textsc{.ii.} choses, que vanités et parole de mençoigne soient loing de moi. Li Apostres dit, ne fait nient contre verité, més pour verité. Li mestres dit, di donc tele verité k’il soit creable ; car verité ki n’est pas creue est en leu de mençoigne, autresi comme mençoigne creue tient leu de verité. Et cil ki ment et si quide voir dire n’est pas mençoignier, car tant com en lui est, il ne deçoit pas mais il est decheus, mais ki ment a ensient, il est bien mençoignier.\par
Por ce di je k’il i a \textsc{.vii.} manieres de mengoignes. La premiere est ens ensegnemens de la foi et de religion, et ceste est trés mauvaise. La seconde est pour anoier autrui sans le proufit de nului. La tierce est por aucun avoir por le preu d’un autre. La quarte est por volenté de falir, et c’est droite mençoigne. La quinte est de biaus dis por envoiseure ou por plaire as gens. La sisime est por proufit d’aucun sans nului damagier. Et la septime est sans damage de tous homes, mais ele est dite pour garder \textsc{.i.} home k’il ne chee en pechiet. En ces \textsc{.vii.} manieres de boisdie est grignour pechiés de tant comme ele aproche plus a la premiere et mains a la derraine, car nule n’est sans pechié.\par
Aprés garde que tes paroles ne soient frioles, car nus ne doit dire mos ki ne soient proufitables d’aucune part. Seneques dit, ta parole ne soit por noient ; mais ou ele soit pour consillier ou por commander ou por amonester. Li Apostles dist, eschive les vaines paroles et les mauvaises.\par
Aprés garde se ton dit soit por raison ou sans raison, car chose ki n’est raisnable n’est parmanable. Por ce dist uns sages, se tu vieus vaincre tot le monde, sousmet toi a raison. Et ki bien ensiut raison, ele li fait connoistre tout bien ; et ki s’en desoivre il chiet en erreur.\par
Aprés garde que tes dis ne soient pas aspre, mais douç et debonaire. Jesu Siraac dist, cytoles et vieles font mout de melodies, mais andeus sormonte la langue souef, la douce parole monteplie amis et endoucist les enemis. Panfiles dist, douce parole aquiert et norrist les amis. Salemons dist que la mole response desront ire, et la dure parole fet courous.\par
Aprés garde que ta parole soit bone et bele, non pas laide ne male. Car li Apostres dist que males paroles corrompent bones meurs. Et ailleurs dist il meismes, nul mal mot ne isse de vostre bouche. Encore dist il en autre liu que bons hons ne doit pas amentevoir laidure ne foz dis Senekes dist, astien toi de laides paroles, car eles norrissent folie. Salemons dist, li hons ki est acoustumés de paroles de reproce n’iert pas amendés toz les jors de sa vie. Li Apostres dist, vostre parole soit tozjors condiee de sel de grasce, en tel maniere que vous sachiés tousjors comment il vous covient a chascun respondre.\par
Aprés garde que tu ne dies oscures paroles, mais entendables. De quoi la lois dist, il n’i a point de difference de nier ou de taire ou de respondre oscurement, se celui ki remaint ne remaint certain. Car l’escripture dist que plus seure chose est a estre mu que dire paroles ke nus n’entende.\par
Aprés garde que tes paroles ne soient sophistiques, c’est a dire k’il n’i ait desous aucun mal engien pour decevoir Car Jhesu Siraac dist, ki parole sosphistequeement, il sera haïs de tous homes et defaillans en toutes choses, et Dieus ne li done sa grasce.\par
Aprés garde que tu ne dies ne ne faces tort ne damage ne anui, car il est escrit, ki a mains manace, k’il fait tort a \textsc{.i.} home. Jhesu Syraac dist, ne te soviegne pas de chose ki apertiegne a anui. Cassidorus dist, pour \textsc{.i.} tortfet sont plusour commeu. Li Apostles dit, ki fet anui aura ce k’il fist de mal : atent d’autrui ce ke tu as fet a autrui. Tuilles dist, il n’est nus si chevetains torsfés comme de ciaus ki, lorsqu’il le font, welent resambler k’il soient bons. Jhesu Siraac dist, la roiautés est trespassee de gens as gens par les maus et par les torsfais.\par
Mais hom ne s’en doit garder seulement, ains doit contrarier a ceaus ki le font as autres. Tuilles dist ke \textsc{.ii.} manieres sont de torsfais, une est ki le fait, une autre est ki ne contrarie a ciaus ki le font, et ce est autresi blasmable comme cil ki n’aide son fil ne sa cité. Et nanporquant se l’en te dist mal ou anui, tu te dois taire ; car Augustins nous ensegne que plus bele chose est a eschivre \textsc{.i.} tortfait en taisant que vaincre en respondant.\par
Aprés garde que tes dis ne soit pour semer descorde, car il n’a si male chose entre les homes. Aprés garde ke en tes dis tu ne te gabes malement, ne de ton ami, ne de ton anemi, ne de nului. Car il est escrit, il n’afiert pas a gaber ton ami ; car se tu li fais anui, il se courece plus fort ; et ton anemi se tu le mokes vient tost a la mellee, car il n’est nus a qui il ne desplaise. Et amours est chose muable ; et s’il mue, tost faut et a paines revient. Salemons dist, ki done sentence des autres, par teus l’ora de lui meismes. Ce meisme conferme Martialis la ou il dist, ki descuevre les autrui visces, par teus ora les siens crimes ; car ki escharnist, il est escharnis a sa coupe, il n’a pas si general chose au monde.\par
Aprés garde ke tu ne dies malicieus mot ; car li Prophetes dit, Dieus destruise les levres malicieuses et langue vanteresce. Aprés garde que tu ne dies orguilleus mot ; car Salemon dist, ou orgoils est, la maint folie, et la ou est humilités, si est sens et leece. Job dist, se orgoils monte jusques au ciel et son chief touche les nues, a la fin le covient il cheoir et torne a perte et a noient. Jhesu Siraac dist, orgoils est heables devant Deu et devant les homes, et toute iniquités avec. Et aillours dist il que orgois et tortfais destruit la substance, et grandisme richeces vient a noient par superbe.\par
A la fin garde que tes paroles ne soient oiseuses, car il nous covenra rendre raison de toutes. Les ensegnemens estuet il garder en parlant, et en some tout çou ki enpire l’onnour de nous ; et tout çou ki est contre bonnes meurs, nus le doit pas dire ne metre en oevre. Socrates dist, ce ki est let a fere, je ne croi pas k’il soit bons a dire. Pour ce doit chascuns dire honestes parole ou k’il soit ; car ki wet honestement parler entre les estranges ne doit pas deshonestement parler entre les privés, car honestés est necessaire en toutes les parties de la vie des homes.
\chapterclose


\chapteropen
\chapter[{.II.LXIIII. De garder a qui tu paroles}]{\textsc{.II.LXIIII.} De garder a qui tu paroles}\phantomsection
\label{tresor\_2-64}

\chaptercont
\noindent Or estuet regarder a qui tu paroles, s’il est tes amis ou non ; car avec ton ami pués tu parler bien et droitement, a ce qu’il n’a pas nule si douce chose au monde come il est d’avoir \textsc{.i.} ami a qui tu puisses parler autresi comme a toi ; mais ne di pas chose ki ne doit estre seue, s’il ne devenist tes anemis. Senekes dist, parole avec tes amis, autresi comme se Dieus te ooit, et vif avec les homes ausi comme se Dieus te veist. Car ailleurs dist il meismes, tien ton ami en tel maniere ke tu ne criemes k’il deviegne tes enemis. Pieres Alfons dist, por les amis ke tu n’as assaiés, te porvois une fois des anemis et mil des amis, car par aventure li amis devendra enemis.\par
Li mestres dit, ton secré, de quoi tu ne dois consillier, ne le di pas a home vivant. Jhesu Siraac dist, a ton ami et a ton anemi ne descoevre pas çou que tu sés, meismement se ce est mal, car il te gabera et se mokera de toi, en samblanche de deffendre ton pechié. Li mestres dist, tant comme tu retiens ton secré il est autresi comme en ta chartre, mais quant tu l’as descovert, il tient toi en sa prison ; car plus seure chose est a taire que priier \textsc{.i.} autre k’il se taise. Pour çou dist Senekes, se tu ne commandes a toi de taire, coment en prieras tu \textsc{.i.} autre ? Et neporquant s’il t’estuet consillier de ton secré dire, di le a ton bon ami droit et loial que tu as assaiet de droite bienweillance. Salemons dit, aies amour et pais de plusours, mais \textsc{.i.} conseilleour entre mil. Catons dist commet ton secré a loial compaignon et ta maladie a loial mire.\par
Aprés garde que tu ne paroles trop a ton anemi, car en lui ne pués tu avoir nule fiance, neis s’il fust pacefiiés a toi. Esopes dist, ne vous afiiés pas en ciaus que vous avés guerroiés, car il ont tousjours el pis le feu de la haine. Seneques dist, la ou li feus a demouré longuement, tousjors i seront les fumees. Et ailleurs dist il meismes, mieus vaut a morir por ton ami que vivre avec ton anemi. Salemons dist, ne croire par a ton ancien anemi, car ja soit ce k’il s’umelie, ce n’est pas por amour, més pour prendre ce k’il ne pooit avoir devant. Et ailleurs dist il meismes, tes enemis pleure devant toi ; mais s’il veoit le tens, il ne poroit estre saoulés de ton sanc.\par
Pieres Autons dist, ne t’acompaigne pas a tes anemis, car se tu fés mal il le noteront, et se tu fés bien il le dampneront. Generaument entre toutes gens dois tu garder que tu dies ; car plusour portent samblance d’ami, ki sont enemi. Pieres Aufons dit, tous ceaus que tu ne conois, souspeçonne k’il soient tes enemis. Et s’il welent cheminer avec toi ou enquerre de tes aleures, faing toi que tu ailles plus long. Et s’il porte glave, si va a sa destre, et s’il porte espee, si va a sa senestre.\par
Aprés garde que tu a fol ne paroles pas ; car Salemons dist, es oreilles du fol ne dies mot, car il despit l’ensegnement de ta parleure. Et ailleurs dist il meismes, li sages hom, s’il tence avec le fol, ou k’il se courece ou k’il rie, ne trovera ja repos ; et fols ne rechoit pas le dit de sens se tu ne dis ce ki soit agreable a son cuer. Jhesu Siraac dist, cil parole a home dormant ki dist au fol sapience.\par
Aprés garde toi que tu ne paroles a home escarnisseour, et fui ses dis comme venin, car la compaignie de lui est lasse de toi. Salemon dist, ne chasties home gabeour, car il te haroit, mais chastie le sage, ki t’amera. Senekes dist, ki blasme l’escharnisseour fet anui a soi meismes, et ki blasme le mauvais aquiert de ses teches. Jhesu Siraac dist, ne te conseille pas avec fols, car il ne loent pas se ce non que lor plest.\par
Aprés garde que tu ne paroles a home gengleour et plain de descorde ; car li Prophetes dist, hom ki a langhe jengleresse n’ert amés sur terre. Jhesu Siraac dit, espoentables est en cité home discordeus et fol de paroles. Et ailleurs dist il meismes, ki het genglerie estaint malice.\par
Garde donc que tu ne paroles a homes descordables, que tu ne meches busches en lor feu. Tuilles dist, la raisons des chiens doit hom dou tout eschivre, c’est des homes ki tozjors abaient comme chien. Car de ciaus et des autres samblables dit Nostre Sires, ne getes pas perles entre les porciaus.\par
Aprés garde toi de tous mauvais homes, car Augustins dit que si comme li feus croist tousjors pour la croissance des busces, tot autresi li mauvés home quant il oënt grignour raison croissent il en plus fiere malice ; car en male ame n’entre pas sapience.\par
Aprés garde que de ton secré tu ne paroles a home ivre ne a male feme ; car Salemons dist, nul secré regne la ou est yvresce. Li mestres dist, femes sevent celer ce k’eles ne sevent. Et en some garde tousjors devant qui tu ies, et mout bien consire le lieu ; car a moustier covient a dire autre chose que a court, et a noces autre chose que a dolour, et a maison autre chose que a compaignie ou ke es places, et pour çou que li proverbes dit ke ki est encoste voie ne die pas folie, que li parleour doit bien prendre garde k’il ne die chose mauvaise se aucuns fust enki priveement.\par
Aprés garde tu paroles au signour que tu li rendes bonour et reverence, selonc ce ke tu dois a sa dignité, car es homes dois tu diligeaument consirer la dignité et le gré de chascun ; car autrement dois tu parler as princes que as chevaliers, et autrement a son per que a son menour, et autrement as religieus que as seculers.
\chapterclose


\chapteropen
\chapter[{.II.LXV. De garder por quoi tu paroles}]{\textsc{.II.LXV.} De garder por quoi tu paroles}\phantomsection
\label{tresor\_2-65}

\chaptercont
\noindent Après dois tu garder por quoi tu paroles, c’est a dire l’ochoison de tes dis ; car Seneques dist, commande ke tu enquieres l’ochoison de totes choses. Cassidorus dist que nule chose non puet estre fete sans achoison ; et achoison est en \textsc{.iii.} manieres, une ki fait, la seconde est la matire de quoi il le fet, la tierce est la fins pour quoi il le fait. Et tu dois garder por quoi tu dis ; car autrement dois tu parler por le service Dieu que por les homes, et autrement por ton preu que pour l’autrui.\par
Mais garde ke ton gaaing soit biaus et covenables, car lois nous a deveé lait proufit. Seneques dist, fuies laide gaigne come perte. Li mestres dist, proufiz ki vient con male renomee est malvés, et jou ameroie mieus a despendre que laidement a gaignier. Et si doit li gaains estre amesurés, car Cassidorus dist que se li gaains ist de covenable mesure, il n’aura pas la force de son non. Et si doit estre naturel, c’est a dire d’un preudome a \textsc{.i.} autre ; car la loi dist k’il est drois de nature ke nus n’enrichisse d’autrui damages. Tuilles dist, ne paour ne dolour ne mort ne nule autre chose dehors n’est si fierement contre nature comme enrichir dou damage des autres, meismement de la povreté as povres. Cassidorus dist, sour toutes manieres de cruauté est enrichir de la povreté as besoigneus.\par
Pour achoison de ton ami dois eu bien dire, mais ke ce soit biens ; car Tuilles nous ensegne que la lois d’amistié commande k’il ne s’entreprient des choses vilaines, et cil ki en est priiés ne le face ; car amours n’est deffense dou pechié ke l’om fait pour son ami, mais mout peche cil ki donne oevre au pechié. Seneques dist, pechiers est chose laide, et deguerpist Deu \textsc{.ii.} fois. Cassiodorus dist : cil est bons deffenderes ki deffent sans tort faire.
\chapterclose


\chapteropen
\chapter[{.II.LXVI. Consirer comment tu dois parler}]{\textsc{.II.LXVI.} Consirer comment tu dois parler}\phantomsection
\label{tresor\_2-66}

\chaptercont
\noindent Or te covient il consirer comment tu paroles, car il n’est nulle chose ki n’ait besoing de ses manieres et de sa mesure, et tout çou ki est demesuré est de mal, et tout sorplus tornent a anui. Por ce doit la maniere et la mesure du parler estre de \textsc{.v.} choses, c’est en parleure, en isneleté, en tardeté, en quantité, et en qualité.\par
Parleure est la dignités dou mot et la porteure dou cors, selonc ce que matire requiert, et ce est une chose ki mout vaut a bien dire. Tuilles dist, ja soit ce ke tes dis ne soit biaus ne gaires polis, se tu le proferes gentement et de bele maniere et de biau port, si seroient il loés et s’il sont biaus et bons et tu ne le dies belement, si seroit il blasmé. Por ce dois tu atorner et atemprer ta vois et ton esperit et tous les movemens dou cors et de la langue, et amender les paroles a l’issue de ta bouche en tel maniere k’eles ne soient enflees ne dequassees au palais, ne trop resonans de fiere vois, ne aspres a la levee des levres, mais soient entendables et sonans par bele proference clere et souef, si ke chascune letre ait son son et chascuns mos son accent, et soit entre haut et bas ; et neporquant tu dois comencier plus bas que a la fin.\par
Mais tout ce t’estuet il muer selonc les muemens dou leu, des choses, des ochoisons, et dou tens ; car unes choses doit on conter simplement, unes autres doucement, les autres a desdaing, les autres par pitié, en tel maniere ke ta vois et tes dis et ta porteure soient tousjours acordables a la matire. En ta porteure garde que tu tiegnes ta face droite, non mie contremont le ciel ne contreval, les oils fichies en terre, ne torner les levres laidement ne grater les sourcis ne lever ses mains, ne ne soit en toi nul port blasmables.\par
En isneleté et en tardeté, garde moieneté tousjours, car au parler ne doit nus estre courans, mais auques lasches avenablement. Li Apostres dit, soies isniaus a oïr et tardis au parler et a ire. Salemon dit, quant tu vois \textsc{.i.} home isnel au parler, sachés k’il a en lui mains de sens ke de folie. Cassidorus dit, ce est sans faille roiaus vertus a corre lentement as paroles et hastivement a entendre. Je pense, fet uns sages, que cil soit bons juges ki entent tost et tart juge ; car demeure por conseil prendre est mout bone, car ki tost juge cort a sa repentance. Et li proverbes dit, demeure est haïe, mais ele fet home sage. Donc est ele bone, meismement en conseil ; car ce est bon conseil de quoi tu as pensé longhement ; et aprés hastif conseil vient repentance. Li mestres dist, \textsc{.iii.} choses sont contraires a conseil, ce est abstinence, ire, et covoitise.\par
Mais aprés conseil fet bon haster. Seneques dist, di mains que tu ne fais et longhement conseille ; mais fai tost et isnel. Salemons dist, cil ki est isnel en toutes ses oevres demorra devant le roi, non pas entre le menu peuple. Jhesus Siraac dist, soies isniaus en toutes tes oevres, mais garde que ta isneletés n’enpeche la perfection de ta oevre, car li vilains dist ke hastive lisse fet chiens avules.\par
En la quantité de tes dis dois tu sor toute chose garder de trop parler, car il n’est nule chose ki tant desplaise comme grant parleure. Tais toi : tu plairas a tous se tu dis poi. Salemons dist, di poi et fai assés de bien. Et por ce ke lons dis ne puet estre sans pechié, dois tu apeticier ton conte au plus brief que tu poras ; mais que cele brietés n’engendre oscurté.\par
En la qualité de tes dis garde que tu dies bien, car biens dire est la rachine de l’amistié et maudire est le commencement d’ennemistié. Di donc bones paroles, lies, honestes, cleres, simples, bien ordenees, a plaine bouche, le visage coi, sans trop rire et sans mout crier. Salemons dit que paroles bien ordenees sont bresches de miel, douçours de l’ame, et santé des os.
\chapterclose


\chapteropen
\chapter[{.II.LXVII. De prendre garde le tens que on doit parler}]{\textsc{.II.LXVII.} De prendre garde le tens que on doit parler}\phantomsection
\label{tresor\_2-67}

\chaptercont
\noindent Autresi dois tu garder le tens quant tu vieus en parler. Jhesu Siraac dit, li sages soi taist jusques a tens, et li fols ne esgarde saison. Salemon dit, et il i a tens de parler et tens de taire. Seneques dist, tais toi tant come tu auras mestier de parler. Li mestres dist, autresi dois tu taire tant que li autre oïent ta parole. Jhesu Siraac dist, n’espandes pas ton sermon la ou il n’a point de oïe, et ne moustre ton sens afforce, car c’est autresi comme citole en plour.\par
Autresi ne dois tu pas respondre devant çou ke li demande soit finee, car Salemons dit ke cil ki respont devant k’il ait oii moustre k’il soit fols, et ki parole avant k’il aprenge chiet en eschar. Car Jesu Siraac comande ke tu apareilles justice avant que tu juges, et que tu apreignes avant que tu dies. Mais ci se taist li mestres des ensegnemens de parler, et n’en dira ore plus jusques a tant k’il venra ou tierç livre, ou il ensegnera tout l’ordre de rectorike ; et tornera a la tierce partie de prudence, c’est a cognoissance.
\chapterclose


\chapteropen
\chapter[{.II.LXVIII. De cognoissance}]{\textsc{.II.LXVIII.} De cognoissance}\phantomsection
\label{tresor\_2-68}

\chaptercont
\noindent Cognoissance est a conoistre et deviser les vertus des visces ki ont samblance des vertus ; et de ce nous covient il garder pour ce ke maintefois, si comme dist Seneques, les visces entrent desous le non de vertu, car fols hardemens entre en samblance de force, et mauvestié est apelee atempremens, et li couars est tenus pour sages ; et pour falir en ces choses somes nous en grant peril, et pour çou i devons metre certains signes.\par
Isidorus nous maine a l’office de ceste vertu quant il dist, il i a visces ki portent samblance de vertu, pour quoi il deçoivent plus perilleusement ciaus ki les sivent, pour çou qu’eles se cuevrent desous coverture de vertus ; car desous demoustrance de justice est fete crualté, et paresce lasche est apelee debonairetés. Tuilles dist, nul agait ne sont si repost comme cil ki s’atapiscent en samblance de service. Li mestres dist, uns chevaus de fust dechut ciaus de Troie, pour çou k’il fainst la forme de Minerve, c’estoit lor dieuesse.
\chapterclose


\chapteropen
\chapter[{.II.LXVIIII. L’ensegnement d’aprendre les non sachans}]{\textsc{.II.LXVIIII.} L’ensegnement d’aprendre les non sachans}\phantomsection
\label{tresor\_2-69}

\chaptercont
\noindent Ensegnement est apenre soi et les nonsachans. Son office est ke l’en doit premierement ensegnier soi meisme et puis les autres ; selonc ce que dist Salemons, biaus fiz, boi l’euue de ta cisterne et ce qui degoute de ton puis, et li ruissel de tes fontaines aillent hors et arouse tes euues par miles places.\par
Li mestre a dit, boivre l’euue de ta cisterne ou de ton puis, c’est a dire que l’en doit espandre sa science en ensegnant les autres. Salemons dist, je ce pri, Dieus, ke tu me doinses cuer ensegnable. Senekes dit, il a ja grant part en bonte ki vieut devenir bons, et bontés de cuer n’iert ja enpruntee ne vendue. Senekes dist, vertus ne puet estre sans estude de soi, car mauvaistiés nous prent legierement.\par
Vertus est aquise par grant travail, eles desirent governeour ; mais visce aprent on sans mestre. Gregoires dist, il nous covient souvenir sovent de çou ke li mondes nous fet oublier. Senekes dit, je n’iert trop dit çou ki n’est dit assés. Augustins dit, cil sont maleureus ki tienent vil çou k’il sevent et tousjors quierent noveles choses. Vieus tu bien savoir ensegne ; car ensi s’apreste doctrine, s’ere est tenue destroit. Anteglaudianus dist, close faut overte revient. Seneques dist, aprent çou ke tu ne sés, si que tu n’en soies ensegnieres nient proufitables. Catons dit que laide chose est au mestre quant il est entechiés de la coupe.\par
Li mestre dit, mais la nature de tous homes est tele que il jugent plus d’autrui choses que des lor ; et çou avient pour çou que en la nostre chose nous somes enpeecié ou de trop grant joie ou de trop grant dolour ou d’autre chose samblable, par quoi nous ne poons jugier la chose selonc çou k’ele est. Pour çou commande la lois de Roume que on doit querre avocat en sa propre cause. Mais il avient ne sai coment que nous veons en autrui s’il fet mal plus tost que en nous, et en l’uel d’un autre puet on veoir \textsc{.i.} petit festu, c’om ne verroit un grant tref en son oil. Autresi voit on la male de son compaignon ki li vait au devant, mais il ne voit pas la sue ki est deriere lui.\par
En toute ceste vertu dist Tulles ke l’on doit eschivre \textsc{.ii.} visces, li uns est que nous, ce que nous ne savons, pour seues, et que nous n’i asentons folement, car ce est presumptions. Et ki voldra eschivre ce visce, il metra tens et pensee a consirer les oscures choses. Li autres visces est metre grant estuide es oscures choses et griés ki ne sont pas necessaires ; et cist visces est apelés curiosités ; c’est quant on met toute sa cure et trop grant entente a çou u il n’a grant preu, autresi comme se tu laissasses la science de vertu et mesisses grant estuide a lire astronomie ou augorisme. Seneques dit k’il est mieus se tu tiens \textsc{.i.} poi des ensegnemens de sapience et les as prestement en usage, que se tu en avoies apris mout et ne les eusses a main.\par
Li mestres dit, autresi come on apele bon luiteour non pas celui ki set mout de tours de qui il use poi, mais celui ki en \textsc{.i.} ou en \textsc{.ii.} se travaille diligemment (il n’i a point de force combien k’il en sace, fors tant k’il gaigne victore), autresi est il es disciplines, il i a mout de choses ki poi aident et mout delitent. Car ja soit ce que tu ne saces par quel raison la mers s’espant, et pour quoi li enfant jumel ki sont conceu ensamble et ne naissent mie ensamble, et por quoi diverses destinees sont a ciaus ki ensamble naissent : il ne te nuist gaires a trespasser ce k’il ne te loist a savoir ne ne proufite. Tuilles dist, sens ki est sans justice doit mieus estre apelés malices que science.
\chapterclose


\chapteropen
\chapter[{.II.LXX. Ce dist encore de prudence}]{\textsc{.II.LXX.} Ce dist encore de prudence}\phantomsection
\label{tresor\_2-70}

\chaptercont
\noindent En prudence se doit on garder dou trop et dou poi et sivre le mi, selonc ce que fu dit arieres ou livree d’Aristotle. Car la ou vertus s’efforce outre son pooir sans retenement de raison, lors chiet ele perilleusement. Grigoires dit, ki roidement garde le rai dou soleil, il entenebrist si k’il ne voit goute. Salemons dist, ki n’a prudence, il destruit son trezor ; mais garde toi de porveoir çou ki nos est devee. Il n’est pas de nostre licence, dis Jhesucris, a savoir le tens et les eures ke li Peres retint en sa poesté. Li Apostres dist, li sens de la char est enemis a Dieu, et la sapience de ce monde est folie a Dieu.\par
\par
Seneques dit, se prudence passe outre ses bonnes, tu seras tenus por engigneus, d’espoentable soutillece se tu enquiers les choses secrees ; et chascune petite chose tu voudras savoir, tu seras tenus envieus, suspicieus et solliciteus et plains de paour et de pensers. Et se tu més toute ta soutillité en trover une petite chose perdue, on te mousterra au doi, et dira chascuns que tu es mout engigneus et plains de malice et enemis de simpleté, et generaument seras tenus pour mauvés par tous homes. Et en teles mauvestés amaine l’ome la desmesuree prudence, donques doit on aler par le mi, si k’il ne soit trop gros ne trop soutiz.
\chapterclose


\chapteropen
\chapter[{.II.LXXI. De la seconde vertu, c’est atemprance}]{\textsc{.II.LXXI.} De la seconde vertu, c’est atemprance}\phantomsection
\label{tresor\_2-71}

\chaptercont
\noindent Après les ensegnemens de prudence, ki est li premiere des autres, et ki est dame et ordeneresse, si come cele ki est par la force de raison, ki devise les homes des autres animaus ; wet li mestres dire des autres \textsc{.iii.}, et premierement d’atemprance et de force que de justice, por çou ke l’un et l’autre est por adrecier le corage de l’home as oevres de justice. Raison coment : covoitise et paour enpeecent l’office de justice, se ne fust atemprance ki coustraint l’une et efforce l’autre.\par
Et toutefois dist li mestres d’atemprance avant que de force, pour çou k’atemprance establist le corage as choses ki sont avec nous ; ce est as biens ki servent au cors, mais force les establist as choses contraires. Et d’autrepart par atemprance governe li hom soi meismes, et par force et par justice governe il les autres, et mieus est a governer toi que autrui.
\chapterclose


\chapteropen
\chapter[{.II.LXXII. Encore d’atemprance}]{\textsc{.II.LXXII.} Encore d’atemprance}\phantomsection
\label{tresor\_2-72}

\chaptercont
\noindent Atemprance est cele signorie que l’om a contre luxure et contre les autres movemens ki sont desavenans. C’est la trés noble vertus ki refraint les carneus delis et ki nous donne mesure et atemprement quant nous somes en prosperité, si ke nous ne montons en superbe ne ne concevons la volenté ; car quant volentés vaint le sens, li hons est en male voie. Tuilles dit, ceste vertus est li aornemens de tote vie et li apaiemens de toz torblemens. Pour ce se doit chascuns widier son corage de la volenté dou charnel delit, car autrement n’i poroit abiter vertus ; selonc ce que dist Orasces, se li vaissiaus n’est purs, quanque tu i meteras enegrira.\par
Por ce dois tu despire delis, car trop anuie delis ki est achatés par dolor, li avers a tozjors besoigne ; met donc certaine fin en ton desirier. Li envieus amaigrist tousjours des grasces choses as autres. Ki n’atemperra sa ire, il aura dolour, et voudra k’il n’eust onques fait ce k’il avoit pensé. Ire est une corte forsenerie en quoi tu dois governer ta volenté, car se tu ne le fés obeir ele commande, refrain le donc o frain et o chaines.\par
Li mestres dit, desous atemprance sont toutes les vertus, ki ont signorie sor les laides meurs et sor les mauvés delis, ki nuisent a home trop perilleusement, car il sont sovent ochoison de mort et de maladie. Senekes dit, por les desiriers perist la grignour partie des cors. D’autre part, ki sert a ses desiriers est sousmis au joug de servage ; il est orguilleus, il a Deu deguerpi, il pert son sens et sa vertu. Salemons dit, sapience n’ert ja trovee en la terre de ciaus ki vivent delitablement.
\chapterclose


\chapteropen
\chapter[{.II.LXXIII. Des delis et des desiriers}]{\textsc{.II.LXXIII.} Des delis et des desiriers}\phantomsection
\label{tresor\_2-73}

\chaptercont
\noindent Delis et desiriers sont acomplis et mis en oevre par les \textsc{.v.} sens dou cors, dont gouster et touchier sont principaus. Mais li autre troi sont establi pour ces \textsc{.ii.}, car nous cognoissons la chose de long, par veoir et par oïr et par flerier, ce ke le gouster ne touchier ne puet connoistre se de prés non Pour ce sont tous oisiaus de proie de grant veue, car il lor covient de loins connoistre lor past ; autresi vit la premiere feme le fruit tot avant k’ele le mangast, et David vit Bersabee nue ki se lavoit, avant k’il fest l’avoutire.\par
Nous lisons el livre de la Nature as Animaus que touchier et goster sont plus poissans en l’ome que en nule autre beste ; mais le veoir et l’oïr et le flairier sont de menour pooir en l’ome que es autres animaus. Et por çou ki jou ke li delit ki sont pour touchier et pour gouster sont plus perilleus des autres, et les vertus ki sont lor contraires sont de grignor vaillance.\par
Et pour ce que deliz sont en l’ame de nous et sont par les \textsc{.v.} sens dou cors, et chascun diversement selonc son office, avient il que cele vertus, ce est atemprance, soit devisee par plusours membres pour constraindre la vertu concupiscible et la vertu irascible, ce est le movement de covoitise et d’ire, pour governer la suite des \textsc{.v.} sens. Et cist membre sont \textsc{.v.}, mesure, honesteté, chasteté, sobrieté, et retenance.
\chapterclose


\chapteropen
\chapter[{.II.LXXIIII. De mesure}]{\textsc{.II.LXXIIII.} De mesure}\phantomsection
\label{tresor\_2-74}

\chaptercont
\noindent Mesure est une vertus ki tous nos aornemens, et tos nos movemens et tous nos afferes, fait estre sans defaute et sans outrage. Orasces dist, en toutes choses est certaine mesure et certaines ensegnes, si ke li drois ne puet faire ne plus ne mains. Tuilles dit, oste tous aornemens ki ne sont digne a home. Pour ce que Senekes dist, mauvés aornemens dehors est messages de mauvaise pensee. Tuilles dit, ta neteés doit tele estre k’ele ne soit haïe par trop aornement, mais tant que tu ostes la sauvage negligence et sa campestre laidece.\par
Il i a \textsc{.ii.} movemens, un de cors et autre de corage. En celui dou cors doit on garder ke sa aleure ne soit trop mole par tardeté, car c’est samblance de superbe contenance ; et k’ele ne soit trop hastive, tant k’ele te face engroissier l’alaine et muer la colour, car tes choses sont senefiance que li hom ne soit pas estables.\par
Li movemens dou corage est doubles, l’une est pensee de raison, l’autre desirier de volenté. Pensee est en enquerre le voir, et desiners fet fere les choses, dont doit on curer raisons soit dame par devant, et ke li desiriers obeisse. Car se la volentés ki naturaument est sousmise a raison ne li est obeissant, il fait sovent troubler cors et corage. Et on puet connoistre les viaires a ciaus ki sont courouciés ou esmeut par paour ou ki ont grant volenté d’aucun delit, a ce k’il muent et changent vout et coulour et vois et tot lor estat. Car li coers ki est enflamés d’ire bat fort, li cors tramble, la langhe s’enpeche, la face enflamme, li oil estincelent si ke il ne cognoissent lor amis ne lor acointes. La face moustre çou ki est dedens ; por ce dit Juvenaus, regardes les tourmens et les joies dou coer en la face, ki tousjors moustre en apert son abit.\par
Par les paroles ki sont dites puet on entendre que li desiriers de la volenté doivent estre restraint et aquites, car les besoignes et les afferes sont divers, selonc les diversités des meurs, des aages, et des choses. Autresi comme il a es cors grant diversité, car li \textsc{.i.} sont isnel pour corre, li autre sont fort por luitier. Autresi a il ou corage plus grant diversité des meurs, car li \textsc{.i.} ont cortoisie, li autre leece, li autre cruauté, li autre sont sage et voiseus et de celer lor pensee, li autre sont simple et apert, ki ne welent rien faire en repost ne par agait, ains ayment verité et gardent amistié et heent barat, k’en diroie jou ? autretant sont il de volenté comme il sont de figures.\par
Perses dit, il i a mil manieres d’omes, et lor usages sont dessamblables, chascuns a son voloir, et les gens ne vivent a une volenté. Tuilles dit, chascuns doit metre s’entente as choses a quoi il est covenables ; et ja soit ce que autre chose li seroit millour et plus honorable, toutefoies doit il amesurer ses estuides selonc sa riule. Raison comment : s’il est fiebles de son cors et il ait bon engien et vive memore, ne sive pas chevalerie mais l’estuide de letre et de clergie. Car nus ne doit aler contre nature ne sivre çou k’il ne puet consivre ; mais se besoigne nos fait meller as choses ki n’apartient a nostre engin, nous devons curer ke nous les fachons beles sans laidece ou a poi de deshonour, ne nous ne devons pas tant efforcier les biens ki ne nous sont doné comme de fuir les visces.\par
Les proprietés des aages nous raconte Orasces en ceste maniere : li enfant, maintenant k’il sevent parler et aler, welent joer o lor pers et se courecent et s’esjoissent et se muent par diverses heures. Li joenes ki n’a mais point de garde se delite es chevaus et es chiens et es chans ; il se flechist legierement as visces et se courece quant on le chastie ; il se porvoit a tart de son preu et gaste son iretage ; il est orguilleus et covoiteus, et laisse tout çou k’il ayme, car joenes hom n’a point de fermeté. Quant il vient en aage et en corage d’ome, il mue sa maniere, et aquiert richeces et amis et honour, et se garde de faire choses ke li coviegne muer. Li vieus a maint meschief : il quiert les choses, et quant il les a si a paour d’user les ; il fait toutes choses celeement et coardement, il met en delai, et covoite çou ki est a venir ; il se plaint de çou ki est present, et loe le tens ki est passés ; il wet chastiier les enfans et jugier les joenes.\par
Maximiens dist, li vieus loe les choses passees et blasme les presentes, por ce ke nostre vie empire continuelement : li aage dou pere sont piour que cil des aious, et nous somes pire ke nostre pere, et encore seront nostre enfant plus plain de visces. Juvenaus dit, terre norrist ore malvais homes et petis.\par
Et encore de ceste matire dist Tuilles que joenes hom doit porter reverence as ainsnés et entr’aus amer les mieus esprouvés et user de lor conseil. Car si comme dist Senekes, la ignorance et la folie des joenes doit estre governee par le conseil as vieus. Terrences dit, tant comme li corages est en doute, il se torne ça et la. Tuilles dit, en joenece a grant foiblece de conseil, car lors quide chascuns k’il doie vivre selonc ce ke plus li plaist ; ensi est il sozpris d’aucun sien cors de vivre ains k’il puisse le millour eslire.\par
Pour ce doit li joenes hom esgarder la vie des autres autresi comme en \textsc{.i.} miroir, et de ce prendre essample de vivre. Senekes dist, bone chose est de regarder en autri le mal c’on doit fuir. Juvenaus dit, cil est boneurés ki set garder soi por les autrui periz : quant li feus est pris chiés ton voisin dois tu garnir ta maison d’iaue. En cest aage se doit on garder sor toutes choses de luxure et d’autre lecherie, et faire selonc ce ke Juvenaus dist, quant tu fais vilaine chose, soit brieve, et retaille tes crimes o ta premiere barbe. Tuilles dit, li joenes se doit travillier de cuer et de cors, si ke ses ensegnemens vaille es offices de la cité. Ce est a dire k’il se doivent user dé s’enfance a bien faire, si k’il le retiegnent tous les jours de lor vie ; car li pos gardera longhement l’odour k’il prist quant il fu nués. Li enfant apreignent a souirir povreté et a mener lor vie as meillours choses.\par
Tuilles dit, quant il welent relaissier lor corages et metre entente a delit, gart soi de desatemprance, soviegne lor de vergoigne ; et ce sera plus legier se il suefrent que li ainsné soient au geu. Il loist bien a jouer aucune fois pour reposer soi, autresi comme de dormir ; car nature ne nous fist por geu ne por sens. Orasces dit, proufitable chose est as enfans joer en enfance, mais k’il s’estudie puis a avoir sens, et laissier çou que rien ne li vaut, et garder ke li gieu ne le facent cheoir ; car jeus engendre estrif et ire, et ire engendre maine et mortel bataille. Tuilles dit, \textsc{.ii.} manieres sont de joer li uns est vilains, mauvés et lais, li autres est biaus, cortois et engigneus.\par
Li offices de l’home qui a passé joenece sont cil qui Orasces a només ça arieres, en quoi il n’a k’amender ; pour ce s’en passe ore li contes briesment. As viellars doit on amenuisier les travaus dou cors et croistre ciaus dou corage, ou en apenre, ou en chastoier, ou en servir a Deu.\par
Terrences dist, nus ne fu onques si plains de sens que la chose u li aages u li usages ne requiere tozjors aucunes noveles choses, et qu’il ne quide savoir çou k’il ne set, et k’il ne refuse ce ki premierement li plaisoit quant il les a esprovés. Car maintes choses samblent estre bonnes avant ke l’en les assaie ; mais quant on les a assaies, on les trueve malvaises. Tuilles dit, li viellars doivent metre entente en consillier les joenes amis. Vieus hom ne se doit tant garder des choses comme d’abandonner soi a pereche ; autrement li dira on ce ke dit Orasces, tu apoies envie et laisses vertus. Tuilles dist, luxure est laide en toute aage, mais trop est laide en viellece ; et se desatemprance est avec, c’est double maus, car viellece reçoit la honte, et la desatemprance dou viellart fet le joene estre mains sage. Et en ce dist Juvenaus, li essample de nos privés nous corrompent plus tost, car nous somes legier a ensivre laidece et mauvaistié.\par
Orasces dit, li office des besoignés sont divers ; car li sires doit maintenir les besoignes de la cité, et garder la loi, et sovenir soi ke la loi est baillie en sa garde ; mais comme uns autres borgois doit vivre de droit, dont li autre vivent ; k’il ne se face trop haut ne trop bas, mais garde le comun bien en pais et en honesteté, si k’il ne chee el pechié Catelline. De qui dit Salustes, cil ki sont povres es cités ont tousjors envie des riches, et eslievent les mauvés et heent les vielles choses et ayment les noveles, et pour la malevoellance de lor choses desirent que li estat de la cité se remue tousjours.\par
Tuilles dit, li estrange ne se doivent entremetre de nule chose fors de lor besoigne, k’il ne s’entremetrent de l’aitrui besoigne. Vilains offices est a chiaus ki achatent des marchans por revendre maintenant, car il n’i puet rien gaaignier sans mentir, et nule chose n’est plus laide que vanité, et por ce on doit aquerre çou ke mestier est sans laidece et espargnier. Tuilles dist k’il n’est si grans gaains con de garder ce ke l’en a. Medecine et carpenterie sont honestes a ciaus qui eles covienent ; mais marchandise, s’ele est petite, on le dient a laide, se ele est des grans et bien gaignans et done a plusours sans vanité, ele ne doit estre blasmee.\par
Nul mestiers n’est millour que labourer de terre, ne plus plentive ne plus digne d’ome franc ; de qui dit Orasces, cil est bieneurés ki laisse toz mestiers, si comme firent li ancien, et cultive ses bués et ses chans, et est sans dete et sans usure.
\chapterclose


\chapteropen
\chapter[{.II.LXXV. De honesté}]{\textsc{.II.LXXV.} De honesté}\phantomsection
\label{tresor\_2-75}

\chaptercont
\noindent Honestés est garder honour es paroles et es meurs ; c’est a dire que l’om garde de fere et de dire chose dont il li coviegne puis vergoignier. Car nature meisme, quant ele fist l’ome volt ele garder honesté, ele mist en apert nostre figure, en quoi il a honeste samblance, et repost les parties ki sont donees as besoignes de l’home, pour çou k’eles estoient laides a veoir. Et li honeste home ensivent diligament ceste forge de nature ; car il nascondent çou que nature repost, et c’est honeste chose ke l’en ne monstre ses membres. Autresi doit on avoir vergoigne en paroles, car il ne doit pas noumer ces membres qu’il repost.\par
Vicieuse chose est es hautes besoignes dire noz de solas ; car quant Paricles et Çoflocles estoient compaignon en une provosté, et il traitoient ensamble de lor office, uns biaus enfes passa par devant aus ; si dist Çoflocles, ves ci bel enfant, Paricles respondi, provos doit avoir vergoigne, non tant solement es mains, mais es oils. Mais se Çoflocles eust ce dit au mangier, il n’en deust pas estre blasmé. Pour ce dit Orasces que a l’omme triste se covient tristes paroles, au couroucié paroles de manaces, a celui qui se joue paroles jolives, au sage paroles sages. Mais se la parole ert devisee, et desenblable de la fortune de celui ki l’a cit, toutes gens s’en gaberoient.\par
Le quart office ensegne Orasces la ou il dist, n’encercies, cas le secré aucun. Le quint office dist il meismes, se aucuns te dist son secré tu le celeras, ne ne le descoverras par ire ne par ivrece. Garde que tu diras et de qui ; et si te garde de celui ki demande, s’il est jengleres, car il ne puet celer ce k’il ot ne retenir ce ki li entre par les oreilles. Et puis que la parole est issue de bouche, ele vole en tel maniere ke jamés ne puet estre rapelee.\par
Li mestres dit, ne descoevre ton secré, car se tu meismes ne le pués celer, tu ne dois pas commander a autre k’il le ceile. Terrences dit, tien toi donques en ce ke tu oïes plus volentiers que ne paroles. Salemon dist, en mout parler ne faut pechiés. Sour toutes choses fuies tençon ; car douteuse chose est a estriver contre son per, forsenerie est tencier a plus haut de soi, laide chose est a tencier au plus bas, mais trés orde chose est tencier a \textsc{.i.} home fol et yvre.
\chapterclose


\chapteropen
\chapter[{.II.LXXVI. De chasteté}]{\textsc{.II.LXXVI.} De chasteté}\phantomsection
\label{tresor\_2-76}

\chaptercont
\noindent Chasteté est a donter les delis dou touchier par atemprement de raison. Salustes dit, se la volenté de luxure porsiut le corage et ele i a signourie, li corages n’a pooir de bien faire. Senekes dit, delis est frailles et brief ; et de tant comme il fait plus volenteusement, desplest il plus tost ; et a la fin covient k’il se repente ou k’il en ait honte. En luxure n’a nule autre chose ki soit avenant a nature d’omme, ains est basse chose et chetive ki vient de l’oevre au vilain membre. Tuilles dit, laide chose est, et ki mout fait a blasmer, d’encliner la franchise de l’ame au servage dou delit, et faire de son travail autrui delisces.\par
Il sovient tozjors a fort home et sage, combien nature d’omme sormonte as bestes ; car eles n’aiment fors ke delit et a ce metent tout lor effort ; mais coers d’ome entent a autre chose, c’est a penser ou a penre. Et por ce se aucuns est trop enclins a delit, garde soi k’il ne soit dou linage as bestes ; mais s’il est sages et volentés le sorprent, il repont son apetit pour vergoigne.\par
Gardons dont que li delis n’ait signorie sor nous, car il fet home mout desvoier de vertu. Pour ce dist la Sainte Escripture, se ta oevre n’est chaste, si soit privee : luxure et vins confundent la science de l’ome et le metent en erreur de la foi.\par
Car certes ki bien consire la nature de chasteé, ki est pour donter le delis dou toucier, il trovera ke delit sont en \textsc{.ii.} manieres, un ki est par luxure, et un autre ki est des autres membres ; si comme est ore de robes, de bains, et de harnois, de geu de dés, et de teus autres choses ki corrompent la vie de l’ome se ce est a desmesure. Mais ki le fait aucune fois atempreement et sans mauvaise covoitise, on le doit bien soufrir, s’il n’enpire soi ne ses honours ne ses choses.
\chapterclose


\chapteropen
\chapter[{.II.LXXVII. De delit par luxure}]{\textsc{.II.LXXVII.} De delit par luxure}\phantomsection
\label{tresor\_2-77}

\chaptercont
\noindent L’autre maniere de delit, ki est par luxure, est fierement contre bonne vie, se ce n’est chastement fait. Et ce puet estre par \textsc{.v.} raisons, l’une ke li assamblemens soit d’ome avec feme ; la seconde k’il ne soient parent ; la tierce k’il soient en droit mariage ; la quarte ke ce soit pour engendrer ; la quinte que çou soit fet selonc la humaine nature.\par
Par ces paroles poons nous entendre que mariage est sainte chose et plaisans a Dieu, et proufitant as homes en plusours manieres ; l’une maniere est pour çou que Dieus l’establi premierement, la seconde pour la dignité dou lieu ou il fu fait, c’est en paradis, la tierce que ce n’est pas novel establissement, la quarte k’Adam et Eve estoient nés de tous peciés quant li mariages fu fais, la quinte pour ce que cist ordre sauva Diex en l’arche dou deluge, la sisime est que Nostre Dame volt estre de cel ordre, la septime pour çou que Jhesucris ala as noces avec sa mere et avec ses disciples, l’octime pour çou k’il i fist de l’euue vin por senefiance de l’avantage ki vient dou mariage, la novime pour le fruit ki naist dou mariage, ce sont fiz, la disime est pour ce ke mariages est un des \textsc{.vii.} sacremens de sainte eglise, la onsime est pour le pechié de quoi on se garde pour le mariage, et pour maint autre proufit ki en sont aquis a l’ame et au cors.\par
Et tot cil ki se welent marier doivent consirer quatre choses, une por avoir fiz, la seconde k’il se marie a son pareil de linage de cors et d’aage, la tierce k’il soit estrais de bone gent ki ait prodome a pere et prodefeme a mere, la quarte k’ele soit bone et sage, car richece est donee de pure sapience de Deu. Garde donc tous clers et tous autres ki sont establi au service Jhesucrist, et les veves dames, et les puceles, k’il ne cheent en ces perilleus visces ki dampnent le cors et l’ame.
\chapterclose


\chapteropen
\chapter[{.II.LXXVIII. De sobrieté}]{\textsc{.II.LXXVIII.} De sobrieté}\phantomsection
\label{tresor\_2-78}

\chaptercont
\noindent Sobriété est a donter les delis dou goust et de la bouche par atemprance de raison. A ceste vertu nous semont nature quant ele fist si petite bouche a si grant cors ; et d’autre part li fist ele \textsc{.ii.} oils et \textsc{.ii.} oreilles, ja ne li fist ke une bouche. Més mout nous semont la brieté dou delit ki ne dure se mout petit non, tant comme il trespasse la gorge. Et la dolour des maladies ki en sieut avenir dure longuement.\par
Consire donc que totes choses maintenant k’ele est goustee est corrumpue ; ce n’est pas ensi des autres sens, car por veoiret por oïr une bele chose n’iert ele pour çou corrompue. Senekes dist, consire çou ke nature soufist, non pas çou ke lecherie requiert ; car si comme li poissons est pris au lameçon et li oisel au las ; tot autresi est li hom pris por mengier et pour boivre desmesureement, il pert son sens, il pert sa cognoissance, il en oublie toutes oevres de vertus.\par
En ceste vertu a \textsc{.iiii.} offices. Li uns est de non mangier devant l’eure establie. Senekes dist, nule chose est delitable s’ele est trop sovent. Orasces dit, çou ki est poi delite plus. Suefre donc jusques a tant que nature semoigne, car tout outrage le confont et mesure le conorte.\par
Le secont office est que l’on ne quiere trop precieuses viandes, car lecherie et yvresce ne sont sans ordure. Ha, come ci a laide chose de perdre sens et moralité et santé pour outrage de vin et de viande. Juvenaus dit, en ce visce chiet cil ki fait grant force comment on doit depechier le lievre et le geline.\par
Li tiers offices est de constraindre la rage du mangier. Senekes dist, soit ta vie de petit mangier et ton palais soit esmeus par fain non mie par savour. Soustien donques ta vie de tant comme nature requiert. Orasces dit, les viandes ki sont prises sans mesure devienent ameres. Seneques dit, tu dois mangier pour vivre, non pas vivre pour mangier. Orasces dit, il n’est chose k’yvrece ne face ; ele descuevre les secrés, ele amaine les desarmés a la bataille, et ensegne les ars. Jeronimus dit, ki s’enyvre, mors est et ensevelis. Augustins dist, quant li hons quide le vin boivre, et il est ja beus. Li mestres dist, plus honorable chose est que tu te plaignes de soif que d’yvrece. Li poetes dist, vertus est soufrir soi des choses ki delitent a male part.\par
Li quars offices est que pour mangier tu ne despendes desmesureement, car c’est laide chose que tes voisins te moustre au doi et dist, tu ies devenus povres par ta gloutenie. Orasces dist, aies mesure selonc ta bourse es grans choses et es petites ; garde toi donc de tavernes et de tous grans apareillemens de mengier, se ce n’est pour tes noces ou pour tes amis ou pour essaucier tes honours, selonc les ensegnemens de magnificence.
\chapterclose


\chapteropen
\chapter[{.II.LXXVIIII. De retenance}]{\textsc{.II.LXXVIIII.} De retenance}\phantomsection
\label{tresor\_2-79}

\chaptercont
\noindent Retenance est a constraindre les delis des autres \textsc{.iii.} sens, c’est dou veoir, de l’oïr, et de l’odourer, en tout ce ki soit visce. Salemon dist, ne regarde pas male feme. Isaias dit, ki clot ses oreilles et ses oils contre le mal, il habitera el ciel. Salemons dist, n’oles pas feme chan’ant ; et aprés dist il, clo tes oreilles et n’escoute pas langue mauvaise. Senekes dist, mais il est dure chose de non oïr les dous dis des flateours. Ysaias dit, en lieu de souef odour sera gandesime puour.\par
Mais si ce taist li contes a parler d’atemprance et de ses parties, et dire çou ke Senekes dist en son livre de ceste vertu meismes, k’il apele continence, et c’est toute une chose.
\chapterclose


\chapteropen
\chapter[{.II.LXXX. De continence}]{\textsc{.II.LXXX.} De continence}\phantomsection
\label{tresor\_2-80}

\chaptercont
\noindent Se tu aymes continence, oste le sorplus et le trop et destraing tes desiriers en estroit lieu. Consire avec toi combien soufisse a ta nature, non pas combien desire a ta concupiscense. Se tu ies continens, atent juskes a tant que tu soies quites et content de toi meismes ; car celui ki est content de soi, il est soufissans ou il est nés avec les richeces. Met frain a ta concupiscence, part de toi tous delis ki priveement esmuevent les corages a desiriers. Tant menguë que tu ne te saoules, et tant boi que tu ne t’enyvres.\par
Quant tu ies en compaignie de gent, garde que tu ne mesdies de ciaus ki de ta volenté ne sont. Ne te joindre as presenz deliz, et ne desirer ciaus ki present ne sont. Soustien ta vie de poi de chose. Ne sivre la volenté de la viande. Ton palés s’esmueve par fain non mie par savour. Ton desirier prise po, car tu dois pourchacier solement k’il defaillent. A l’example dou vin, te composte. Par toi dou cors et te joing a ton esperite.\par
\par
Se tu estudies en continence, tu habiteras en maison proufitable, non pas delitable. Et ne soit congneus li sires par la maison, mais les maisons soit congneue par lui. Ne te faindre pas d’estre çou que tu n’ies, mais veuilles sembler ce que tu es. Sour toutes choses garde ke tu ne soies povres de laide povreté, et ke tu n’aies abandonnee simplece, ne legierté non ferme, ne laide escharseté. Se tu as poi des choses, ne soient estroites ; tes choses ne plorer, et de l’autrui n’ales merveilles. Se tu aymes continence, fui toutes laides choses avant k’eles aviegnent. Croi toutes choses ki puent estre soustenues, se ce n’est laidure. Garde toi de laides paroles, et ti dit soient plus proufitable que courtois. Ayme les homes bien parlans mais plus ayme ciaus ki parlent droit.\par
\par
Entre ton afaire dois tu meller \textsc{.i.} po de jeu, si atempreement k’il n’i ait abaissement de dignité ne defaute de reverence ; car reprendable chose est non rire. Donques se sens fait de joer, porte toi selonc ta dignité sagement. Soies teus ke nus ne te repregne que tu soies aspres, ne nus te desprise comme vilh. En toi ne soit aucune vilenie, mais avenable cortoisie. Tes geus soient sans legierté, et tes ris sans huchier, et ta vois sans cri, et t’aleure sans rumor. Ton repost ne soit pas negligence. Quant li autre jeuent devant toi, pense aucune honeste chose.\par
\par
Et se tu vieus estre continens, tu auras toutes loenges, autretel te samblera estre loés par les mavaises gens comme estre loés par malvaises oevres. Soies liés que tu desplaises as malvais homes ; et quant il pensent u dient mal de toi, lors dois tu estre joians, et quides que ce soit tes pris.\par
\par
La plus grevable chose ki soit en continence est garder soi de douces paroles que losengiers dient, par qui li corages s’esmuet as grans delis. N’aquerre l’amistié d’aucun home par losenge. Ne soies hardis ne orguillous, humelie toi et baisse, et non te vanter greveusement. Ensegne volentiers as autres. Respont belement. Se aucuns te reprent por droite okison, sachés k’il le fait por ton preu. Les aspres paroles ne douter, mais aies paour des bones. Oste et dessoivre toi des visces, et de l’autri n’enquieres trop. Ne soies reprenderes trop aspres, mais ensegne sans reproce, en tel maniere que tozjours ait leece devant ton chastiement. Quant li hom erre, pardone li legierement.\par
\par
Entent quitement ciaus ki parolent, et retien fermement çou k’il dient. Se aucuns te demande aucune chose, tu dois respondre isnelement. A celui ki tençonne, done liu tost, et te part de lui. Se tu ies continens, destraing toz mauvés movemens de ton cors et de t’ame, et ne te chaut que li autre ne voient, car assés est que tu le voies tu. Soies movans et non pas mol. Soies constans, mais non partinaces. Tu quideras que tout home soient pareil de toi, se tu ne despises le plus petit par orgoil, et se tu ne doutes le plus grant par droiture de vie. Ne soies negligens au rendre benefices, et ne soies prés au reçoivre. A tous homes soies tu benignes, et a nului losengier, a poi familiier, a tous droituriers. Soies plus fiers en jugement que en paroles, et plus en ta vie que en ta face. Soies piteus vengieres et despis totes cruautés.\par
Raconte pris des autres et de toi non, et n’aies envie de l’autrui. Soies tousjours contraires a ciaus ki se soutillent d’engignier autres par samblance de simpleté. Soies lens a ire et isniaus a misericorde. Ens es aversités soies fermes et sages. Tu dois celer tes vertus ausi comme les autrui visces. Despis vaine gloire, et de ton bien ne soies crueus as autres. N’aies en despit le petit sens d’aucun home. Parle poi et enten quitement ceaus ki parolent. Soies ferm et liés et seurs et ayme sapience. Çou que tu sés, garde sans orguel, et çou que tu ne sés enquier doucement k’il te soit apris.\par
Continence soit constrainte dedens tes bonnes que tu soies trop escars ne trop despendables. Et ne metre trop ton penser es choses menues et petites, car c’est mout vergoigneuse chose. Donques en ceste maniere maintien la continence que tu ne soies donés a la charnele volenté. Et ne soies prodigues ne tachiés de male avarice. Mais ci se taist li contes d’atemprance et torne a la tierce vertu, ce est force.
\chapterclose


\chapteropen
\chapter[{.II.LXXXI. De force}]{\textsc{.II.LXXXI.} De force}\phantomsection
\label{tresor\_2-81}

\chaptercont
\noindent Force est une vertus ki fait homes fors contre les assaus d’aversités et done cuer et hardement de faire les grans choses ; de qui li contes a dit ça en arieres, qu’ele garde l’ome a senestre, si comme uns escus, contre les maus ki vienent. Et voirement est ele escus et deffense de l’ome, c’est son hauberc et son glave, car ele fet l’ome deffendre soi et offendre a ciaus k’il doit. De ceste vertu trovons nous el Livre des Rois ki dist, tu m’as garni de force a la bataille, et mes enemis soumis a moi. Saint Luc dist, li hom fors garde ses maisons et ses choses sont en pais. Salemon dist, la mains dou fort aquiert richeces, et tous perecheus sont en poverté ; la mains des fors a signorie, et la mains des couars sert a treus. Saint Matheus dit, fors hom aquiert le regne Deu.\par
Et sachiés k’il i a \textsc{.xii.} choses ki connortent en nous ceste vertu ; l’une est la droite fois de Jhesucrist, la seconde est l’amonestement des grans et des ainsnés de nous, la tierce est la memore des preudomes et de lor oevres, la quarte est volentés et us, la quinte est le guerredon, la sisime est paour, la septime esperance, la octime est bone compaignie, la nuevime est la verités et li drois, la disime est le sens, l’onsime est la foibleces de ton enemi, la dousime est la force meismes.\par
Couardie est en \textsc{.iii.} manieres, une por la paour dou mal ki est a venir, l’autre dou mal ki est present, l’autre par le cuer ki est parmanables. Et pour conoistre toutes manieres de foible cuers est ceste vertus devisee en \textsc{.vi.} parties, ce sont magnificence, fiance, seurté, magnanimité, patience, et constance. De cascune dira li contes ce ke lor en apertient, mais tot avant en dira il ce que Seneques dit de ceste vertu, c’est de force, ke il apele magnanimité, en cest maniere.
\chapterclose


\chapteropen
\chapter[{.II.LXXXII. De magnanimité}]{\textsc{.II.LXXXII.} De magnanimité}\phantomsection
\label{tresor\_2-82}

\chaptercont
\noindent Magnanimité, ki est apelee fortece, s’ele est en ton corage, tu viveras a grant esperance, franc et seur et liet. Grandisme bien est a home non douter mais estre parmanans a soi meismes et atendre la fin de sa vie seurement. Se tu es magnanimes, tu ne jugeras en aucun tens que honte te soit faite ; et de son anemi diras k’il ot cuer de damagier toi, mais ne le fist pas. Et lors ke tu le tenras en ton pooir, tu quideras avoir vengance prise, en çou ke tu as le pooir de toi revengier, pour ce que trés noble maniere de vengance est pardoner quant on puet faire sa vengance. Tu ne dois assalir home priveement mais tout en apert. Ne fere bataille se tu ne le dis avant, pour çou que traison et engien n’afiert se a home foible non et couart. Ne metre ton cors en peril contre fol, et non douter com paourous, pour ce que nule chose fet home paourous se la conscience de vie blasmable non.\par
\par
Or est il bien covenable que li contes die de \textsc{.vi.} parties de force, et premierement de magnanimité. Ceste parole vaut autant a dire comme grant corage ou hardement ou provosté, car ele nous fait par nostre gré envaïr raisonablement les griefs choses. Je di raisonablement pour ce que nul ne doit envaïr chose contre droit ; car ki envaïroit frere menour, ce ne seroit pas proece, ains seroit forsenerie.\par
A ceste vertu nous amoneste Virgiles quant il dist, ordenés vos corages as grans oevres de vertu et a grandisme travaus. Orasces dit, ceste vertu oevre le ciel et essaie a aler par la voie ki li est devee, et despite la menue gent, et desdaigne la terre et ne doute paine. Tuilles dit, ja soit ce ke vertus face home corageus as aspres choses, totefois garde ele plus au comun proufit ke au sien propre. Science ki est lontaine de justice doit estre apelee malice, non pas sen ; et li corages ki est aperellés a periz, s’il est plus covoiteus de son preu ke dou commun, ele a non folie, non pas force. Car ceste vertu oste coardie ou perrece.\par
Lucans dist, oste toutes demorees, car ele nuist tozjors a ceux ki sont appareilliés. Orasces dist, commence, car se tu prolonges l’eure de bien faire, tu seras comme li vilains ki voet tant atendre a passer outre le euue dou fleuve k’ele soit toute courue, mais ele cort et corra tousjors. Perses dit, quant on dist, demain sera ce fait, demain sera fet une grant chose, tu ne dones autre chose que \textsc{.i.} jour ; li autre jour vienent, et lors avons gasté cel demain, li ans passe, et tozjors remaint \textsc{.i.} poi outre. Tuilles dist, cil doivent estre tenus apreudomes et de grans corages ki boute ariere le tortfait, non pas ki ne le fait.\par
\par
Mais pour ce que ceste vertu done a home seur cuer et hardement et li fait avoir grant corage entour les hautes choses, covient il k’il se garde mout de \textsc{.iii.} visces ki tost le feroient trebuchier de son hardement et cheoir de sa pensee.\par
Li premiers visces est avarisces, car laide chose seroit que cil ki ne se laisse froissier par paour soit vaincus par avarice ou par convoitise ; nés que cil ki ne puet estre vaincus por travaill se laisse froissier par volenté.\par
Li secons est covoitise de dignité, car par grief travail aquiert hom clarté, c’est renomee. Et cil ki plus est laboureus est de grignour pris ; et a paines ert trovés ki de son travail ne desire glore ausi comme son loier. Seneques dist, li sages met le fruit de sa vertu en sa consience, mais li fols le met en vaine glore. Tuilles dist, il sont aucuns qui quident monter es grans dignités par lor renomee, mais celui ki vraiement est de grant corage vieut mieus estre princes ke sambler le. On ne doit pas aquerre les dignités par gloire, car il en seroit chaciés legierement. Pour ce dit Orasces que vertus ne sera chacie vilainement ; ele resplendist de grans honors, et ne lieve sa hache por le cri dou peuple. Il ne sera ja esmeus por \textsc{.i.} poi de vent.\par
Le tierç visce est fols hardement, c’est a dire quant uns hom est hardis a faire une fole mellee, car ce n’est pas proece, ançois est folie. Tuilles dist, ki folement cort es assamblees a combatre de sa main contre ses anemis, il est samblables as bestes sauvages, et ensiut lor fole fierté. Neporquant se necessité le requiert, tu dois bien combatre, et ains soufrir mort que deshonor ; nous ne devons pas fuir, car ce seroit mauvestiés et coardie. Et pour ce dist Lucans, fuie est une laide mauvestiés, en quoi ne chiet nus se par chetiveté et par defaute de cuer n’est.\par
Et nanporquant nous devons bien fuir quant uns grans periz sorvient que nous ne poriens soustenir. Et lors est grans proece de bien fuir, selonc ce que Tuilles dit, ne nous abandonons as periz sans raison, car grignor folie ne puet estre faite. Li mestres dit, cil ki est en pais et va querant la guerre est forsenés, mais li sages hom se maintient en pais tant comme il puet. Et quant il est constrains d’avoir guerre, il le fait droitement, autresi com li bons mires fait, ki aide a l’ome sain maintenir sa santé ; et s’il est malades legierement, il le cure par legieres medecines, et es plus griés maladies metent il les douteuses medecines.\par
A sa maniere doit on user sa force o sens non mie sans raison. Car Orasces dit, force ki est sans conseil dechiet par sa pesantour ; li dieus acroist force ki est atempree, et si het ciaus ki par lor sorquidance osent envaïr les grans choses seulement.
\chapterclose


\chapteropen
\chapter[{.II.LXXXIII. De fiance}]{\textsc{.II.LXXXIII.} De fiance}\phantomsection
\label{tresor\_2-83}

\chaptercont
\noindent Fiance est une vertus ki demeure entour l’esperance dou cuer, a ce k’il puisse mener a fin ce k’il commence. Ses offices est haster soi de parfaire la chose commencie, si comme Lucans dit de Julius Cesar, k’il li estoit avis k’il n’eust rien fait tant comme il avoit rien a faire. Car nule chose est si po avenant a ciaus ki sont ja avancié que desesperer soi de venir a bone fin.
\chapterclose


\chapteropen
\chapter[{.II.LXXXIIII. De seurté et de paor}]{\textsc{.II.LXXXIIII.} De seurté et de paor}\phantomsection
\label{tresor\_2-84}

\chaptercont
\noindent Seurtés est non redouter les damages ki vienent ne la fin des choses commencies. Et est seurtés de \textsc{.ii.} manieres, une ki est par folie, si comme est de combatre sans armes et dormir emprés le serpent, l’autre est par sens et par vertu. Et ses offices est de doner confort contre l’aspreté de fortune, selonc ce ke Orasces dit, cil ki a bien apareillié son pis se raseure en adversité et se doute en prosperité ; et Diex amaine yvier et il oste les choses ki furent ou sont males ne le seront pas tousjours ensi. Mais une bone eure venra de quoi on avoit esperance.\par
Contre ceste vertu se combat paour, en ceste maniere : Paour dist a l’ome, tu morras, et Seurtés respont, c’est humaine nature, non pas paine. Je entrai au monde par ceste covenance que je m’en istroie La lois comande que l’om rende çou ke l’om enprunte.\par
Et vie d’ome est uns pelerinages : quant tu auras mout alé, il te covenra revenir. Paour redist, tu morras, Seurtés respont, je quidoie que tu desises noveles choses, mais pour morir vif je, et a ce me maine nature chascun jour ; car des ke je nasqui, me mist ele ce terme. Je n’ai de quoi me courece, mais je di par mon sairement que fole cose est criembre chose que l’en ne puet eschiver. Lucans dist, mors est la derraine paine, si ne le doit on criembre.\par
Orasces dist, mors est li derrains termes de toutes choses. Senekes dist, ki prolonge la mort n’en eschape. Paour dist, tu morras, Seurtés respont, je ne serai le premiers ne li derreniers, et home sont alé devant moi et home me sivront, c’est la fins de l’umain linage.\par
Nus sages ne doit estre dolans de mort, ki est la fins des maus. Je ne sai que jou soie autre chose ke uns animaus raisnables ki doit morir. Nule chose n’est griés ki n’avient c’une fois. Par ceste condition sont toutes choses engendrees, ke toutes choses ki ont commencement auront f�n. Il n’est pas estrange chose de mourir, et je sai bien que je doi morir, et a ce ne puis contredire. Dieus fist trop bien, car nus ne m’en puet menacier, car mors igaillist le signour au serf et les couronés as fosseors ; ele enporte en une maniere ciaus ki sont mout divers.\par
Paours dist, tu seras decolés, Seurtés respont, il n’i a force que je muire tost, u pichot a pichot. Paour dist, tu auras maintes ferues, Seurtés respont, moi ke poise combien ke je aie de plaies ? de l’une me covient a morir. Paour dist, tu morras en estrange païs, Seurtés respont, nule chose n’est estrange a home mort, et la mors n’est pas plus griés hors de maison que ens. Paours dist, tu morras joenes, Seurtés respont, autresi vient la mors as joenes que as vieus, ele n’i fait nule difference ; mais tant di je bien k’il fait millour morir lors k’il delite de vivre, et trés bon morir est ançois que tu desirres la mort. Par aventure la mors m’osta d’aucun mal, au mains m’eschapa de viellece ki est mout grief.\par
Selonc ce que dist Juvenaus, ceste paine est donnee a ciaus ki longhement vivent, ke lor pestilence se renovele tozjors ; il enviellissent en paine et en pardurable dolour et porrissent en orde vesteure. Pour ce dist Senekes qu’il fet bon morir tant comme il plest a vivre. Lucans dist, se li derreniers jours venoit avec la fin des biens, et il n’estoit tristece par isnele mort, l’om averoit tristece de sa premiere fortune ; et si se meteroit en aventure desespereement, s’il n’atendoient la mort.\par
Pour ce ne m’en chaut se je muir joenes, car il n’est nus si pesans maus comme viellece. Juvenaus dist, hastive mors ne aigres definemens ne doit pas estre redoutee, mais viellece doit estre plus cremue que mors. Seneques dist, il ne puet caloir quand ans je aie ne quans j’en ai pris ;car se je ne puis plus vivre, c’est ma viellece, kiconques vient a son derrenier jor, il muert vieus.\par
Paour dist, tu ne seras pas ensevelis, Seurtés respont, petit damage est celui du sepulchre. Lucans dist, il n’i a force se la charoigne pourrist ou s’ele est arse, car nature prent tout a gré a qui li cors doivent sans fin ; mors n’a ke faire de fortune, la terre ki tout engendre tout reçoit ; et ki n’est covers de la terre, il est couvers du ciel. Dés ke li cors ne sent riens, ne li chaut s’il n’est enfouis ; et s’il se sentoit, toutes sepoutures li seroient tormens. Car sepulture ne fu pas trovee as oils des mors, més de vif, pour ce que la charoigne pourie fust ostee de la veue des homes. Por ce sont li \textsc{.i.} mis en terre et li autre en feu, et ce n’est fors que pour garder les iols des vivans.\par
Paours dist, tu seras malades, Seurtés respont, or voi je bien que la vertus des homes ne se moustre en mer ou en bataille seulement ; mais ele se moustre en \textsc{.i.} petit lit ou je laisserai la fievre u ele moi : la bataille est entre moi et la maladie, u ele sera vaincue u ele vaincera. Paour dist, les gens dient mal de toi, Seurtés respont je me troubleroie se li sage home mesdisoient de moi ; mais desplaire as mauvés est los et pris, car cele sentence n’a point d’auctorité en quoi cil blasme ki doit estre blasmés. Il ne me blasment par loial jugement, mais por la lor mauvestié ; et dient mal de moi pour çou k’il ne sevent bien dire. Il dient çou k’il welent non pas çou que je deserf, car il sont chiens ki ont si apris a abaier k’il ne le font pas par verité mais par coustume. Juvenaus dist, li home sage ne doutent les laidenges dou fol.\par
Paours dist, tu seras chaciés en essil, Seurtés respont, li païs ne m’est contredis mais li lieus, car tout çou ki est desous le ciel est mon païs ; la troverai viles, la troverai la mer, la troverai les pors. Toutes terres sont païs au preudome autresi comme la mers as poissons ; ou que jou aille serai jou en la moie terre, que nule terre ne m’est essilh, mais estranges leus, car bon estre apertient a l’home non pas au leu.\par
Paour dist, dolours te vient, Seurtés respont, s’ele est petite, soufrons la, car cy a petite souffrance ; se elle est grant, souffrons le, car ce sera grans glore. Et se aucuns disoit que dolours est dure chose, Seurtés respont que cil hom est trop maus ki soufrir ne set. Paour dist, il est poi de gent ki puissent soufrir doleur, Seurtés respont, soions de ces poi. Paour dist, nature nous fist sans force, Seurtés respont, n’en blasme la nature ki nous engendra fort. Paour dist, fuions delour, Seurtés respont, por quoi, il te sivra u que tu fuies. Paour dit, tu seras povres, Seurtés respont, li visces n’en est pas en la povreté, mais el povre ; il est povres pour çou k’il le quide estre. Paour dist, je ne sui poissans, Seurtés respont, ales joie tu le seras. Paours dist, cil a grans deniers, Seurtés respont, n’est pas omme ne signour, ains est une huche : nus ne doit avoir envie de bourse plainne. Paour dist, cil est mout riches hom, Seurtés respont, s’il est avers, il n’a riens, s’il est gasteres il n’ara longhement. Paour dist, mout de gent vont aprés lui, Seurtés respont, les mousches sivent le miel et li lou les charoignes, et li formis le forment : il sivent la proie non pas l’ome. Paours dist, je ai perdu mes deniers, Seurtés respont, par aventure il eussent perdu toi ; car il ont ja maint home mené a peril, mais de ceste part t’est bien avenut se tu as perdue avarisse avec. Et sachés que avant que tu eus ces deniers, uns autres les avoit perdus.\par
Paour dist, je ai perdu les oils, Seurtés respont, c’est pour ton bien, car la voie t’est close a mout de covoitises. Maintes choses seront dont tu deusses oster ces oils pour ce ke tu ne les veisses. Tu sés bien ke avugletés est une partie d’innocense ; car li oeil moustrent a \textsc{.i.} l’avoutire, et as autres covoitier maisons et cités. Paour dit, je ai perdu mes fiz, Seurtés respont, fols est ki pleure la mort as morteus, mort sont pour çou ke morir doivent. Deus ne les a pas tolus mais receus.\par
En ceste maniere Paours, ki onques ne dona bon conseil, se combat contre Seurté, mais l’ome seur ne le redoute pas. Selonc ce que dist Orasces, li malisces des citeins ki donent essample de mal faire, ne vout de felon tirant, n’esmuet pas preudome ki est de ferme proposement et de fort corage. Lucans dist, paour de mal avenir a mis maint home en grant peril, mais cil est trés fors qui puet soufrir les douteuses choses. Car il apertient a fort corage et ferme k’il ne soit troublés en adversités, et k’il ne soit pas abatus de son estat avant que li tumulte viegne, ains use del present conseil, et ne se depart pas de raison. Senekes dist, il est plus de choses ki nos espoentent que de celes ki nous grievent ; et nous somes plus sovent en doute par penser que par oevre. Et pour ce ne soies chetif devant le tens, car ce que tu criens n’avenra par aventure jamés.\par
Contre la poour de morir nous asseurent \textsc{.vi.} choses, une est la mors du cors ki est anemis a l’ame, la seconde est que elle pose fin au peril dou siecle, la tierce est la necessités de morir, la quarte est que nous veons tousjours morir les autres, la quintes est que Dieus morut pour nous, la \textsc{.vi.} est la perpetuel vie ki vient aprés. Mais ci se taist li contes de paour et de seurté, de quoi il a longhement parlé, et moustre mainte bonne raison ki font a avoir en memore ; et si tornera a l’autre partie de force, c’est magnificence.
\chapterclose


\chapteropen
\chapter[{.II.LXXXV. De magnificence au tans de pais}]{\textsc{.II.LXXXV.} De magnificence au tans de pais}\phantomsection
\label{tresor\_2-85}

\chaptercont
\noindent Magnificence vaut autant a dire comme grandour, et c’est une vertus ki nous fet acomplir les griés choses et nobles de grant afaire. Et ses offices est en \textsc{.ii.} manieres, l’une est en tans de pais, l’autre en tans de guerre. En choses de pais doivent tenir li signour les \textsc{.iii.} commandemens Platon.\par
Li uns est k’il gardent le proufit as citeins, k’il reportent a çou quanqu’il font, et n’entendent pas a lor propre preu, et k’il s’estudent qu’il aient plenté et habondance des viandes et des choses ki besoignent a la vie des gens, Li autres commandemens est k’il soient curieus de tout le cours de la cité, et qu’il gardent la chose commune et les possessions et les rentes dou commun au besong de tous, non pas d’aucun home privé.\par
Li tiers commandemens est qu’il tiegne justice entre ses subtés, et k’il rende a chascun çou ki est sien, et k’il garde les unes parties en tel maniere k’il n’abandonent les autres ; car cil ki aident as uns encontre les autres amainent en la cité perilleuse descorde. Aprés doivent li signor et li governeour d’une cité garder ke contens ne soit entr’eus. Car Platons dist que cil ki ont content ki mieus aministre la cité font autresi comme se li marenier estrivoient entr’eus li quex governe mieus la nef, et c’est morteus periz.
\chapterclose


\chapteropen
\chapter[{.II.LXXXVI. De magnificence au tens de guerre}]{\textsc{.II.LXXXVI.} De magnificence au tens de guerre}\phantomsection
\label{tresor\_2-86}

\chaptercont
\noindent Au tans de la guerre, quant il lor covient bataille faire, il doivent tout premierement commencier la guerre a cede entention qu’aprés la bataille il puissent vivre en pais, sans tort faire.\par
Aprés doivent il garder que, avant qu’il envaïssent les estris, il soient apareilliés diligement de toutes choses qui besoignent a aus deffendre et a assalir lor enemis. Senekes dit, loins apareillemens de bataille fet toustaine victore ; et cis apareillemens est en bateillier et en forteresces par despenses et par armes. Terrences dist, li sages hom doit esprover totes choses avant k’il se combate ; car mieus vient porveoir que reçoivre le damage et puis vengier.\par
Tuilles dist, li tiers offices est que tu ne te desperes trop par coardie ne ne te fie trop par covoitise. Car la desmesuree covoitise d’avoir maine l’ome en perils ; selonc ce que dist Orasces, li ors fet home aler par mi ses enemis et estre plus fier ke feu ne foudre ; li dons enlacent les felons princes.\par
Li quars offices est que en bataille doit on eschiver plus laide coardie que la mort et entendre plus a bonté que a autre proufit ne k’a eschaper ; car mieus vaut morir que laidement vivre. Neporquant on ne doit laissier son salu pour cri, c’est pour oster le blasme que l’en te porte a tort ou pour aquerre grant renomee.\par
Li quins offices est traveillier sovent son cors es choses ki sont a fere ; car Lucans dit, li hom huiseus mue sovent diverses pensees. Ovides dit, li euue ki souvent ne se muet devient porrie, autresi devient li hom chetis pour estre oiseus.\par
Li sisimes offices est que puis que l’om vient a combatre il doit metre grant vistece, et amonester les chevaliers et les bachelers a bien fere, et loer les de lor proeces et de lor ancissours, et dire tant qu’il les face enhardir et oster de coardie.\par
Li septimes offices est aler au premier assaut secourre et aidier ciaus ki sont affoibloiés, et soustenir ciaus ki cancielent ou ki fuient.\par
Li offices witimes est que quant on a victore on doit espargnier et garder ciaus ki ne furent cruel anemi.\par
Li nuevimes offices est que se l’om fait pais ou trives ou autres aliances as anemis, il les garde et maintiegne ; et non croie a ciaus ki dient que l’om doit porchacier de vaincre ses anemis ou par force ou par tricherie.\par
Ce nos moustre uns haus citeins de Rome ki fu pris en Cartage lors que li romain i furent a ost ; car ciaus de Cartage l’envoierent en Rome pour faire encangier les chetis, et li firent jurer k’il revenroit ; et quant il fu a Rome, il ne loa pas que li chetif fussent rendus. Et quant si ami le voldrent retenir, il volt mieus revenir a son torment que mentir sa foi k’il avoit donee as enemis.\par
Mais Alixandres li grans dit qu’il n’a point de difference coment que l’en ait victore ou par force ou par barat, car fet n’i doit avoir pitié, et cil est enemis de soi meismes ki prolongue la vie a son anemi.
\chapterclose


\chapteropen
\chapter[{.II.LXXXVII. Dou content qi est entre guerre et pais}]{\textsc{.II.LXXXVII.} Dou content qi est entre guerre et pais}\phantomsection
\label{tresor\_2-87}

\chaptercont
\noindent Or a devisé li contes de \textsc{.ii.} manieres de grandeur et en guerre et en pais. Mais pour amenuisier la creance de ciaus ki quident que li affere de guerre est plus grans que celui de la cité, li mestres dit que pais et li affaires de la cité est maintenue par sens et par conseil de corage, més li plusour ont quise bataille par aucune covoitise. Mais a la verité dire poi valent les armes dehors se li sens n’est dedens. Pour ce dist Salustes, tuit li home ki s’estuident d’avancier les autres animaus doivent garder k’il ne mainent lor vie en maniere de beste, ki naturaument sont obeissant as ventres. Mais toute nostre force et ou cors et ou corage, car li corages commande et li cors doit servir.\par
Il est plus drois ke l’on quiere gloire par engien que par force. Tuilles dit, toutes choses honestes que nous querons par haut corage est aquise par vertu de cuer, non pas par force de cors. Nonporquant on doit amener son cors si k’il puisse obeir a conseil et a raison.
\chapterclose


\chapteropen
\chapter[{.II.LXXXVIII. De constance}]{\textsc{.II.LXXXVIII.} De constance}\phantomsection
\label{tresor\_2-88}

\chaptercont
\noindent Constance est une estable fermetés de corage ki se tient en son proposement. Ses offisses est a retenir fermeté en l’une fortune et en l’autre, si ke l’om se s’eshauce trop par prosperité, et ne soit trop tourblés en adversité, mais tiegne le mi. Car noble chose est avoir en chascune aventure \textsc{.i.} front et \textsc{.i.} meisme vout. Senekes dit, la porveance dou corage est k’il soit bien ordenés quant il se puet estre et maintenir soi en \textsc{.i.} estat. Orasces dit, garde que es grans choses ton cuer soit tousjours igal. Aies atempree leece quant plus de biens te vient que tu n’aies acoustumé, car li sages et li hardis apert a la destrece. Et ailleurs dist Orasces, l’ome fort et ferme retret a bon vent sa voile quant ele est trop enflee.\par
La loi de fermeté est cele que nous ne soions pas fichié es maus ne movables es biens. Es maus meismes a fermeté, mais lors n’est ele pas vertus. Selonc ce que dist Orasces, une partie des homes s’esjoïssent des visces, et por ce se ferment en mal fere ; une autre partie vait flotant, car une fois fet bien et autre mal. Juvenaus dit, la nature des mauvés est tozjours vaires et movables quant il meffont, encore ont il fermeté tant qu’il commencent a connoistre bien et mal ; et quant il ont fet les crimes, nature se fiche es meurs d’aus ; et ne s’en set remuer. Ki est cil ki met fin en pechier ? puis que la rouge coulour s’en est alee une fois de son front, quel home vois tu ki se tiegne a \textsc{.i.} sol pechié ? puis ke sa face endurcist, et ne redoute vergoigne.\par
A ceste vertu est contraire \textsc{.i.} visces ki a non muableté, c’est a dire dou corage ki n’a nule fermeté, ains est sovent esmeus en diverses pensees. Et sont aucun si plain de ces visces que li autre quident que lor fermeté soit tozjors movable ; aucun sont si po estable ke, maintenant k’il lor avient \textsc{.i.} poi de mal, desprisent tous delis ; delis ; par dolour afoiblissent, et despisent gloire, et sont froissié par male renoumee. De çou dist \textsc{.i.} sages, quant je sui malades jou ayme Dieu et sainte eglise, mais quant je sui garis cele amours est oubliee.\par
Pour ce dist Orasces, ma sentence se combat o moi, car ele refuse ce k’ele avoit quis et requiert ce k’ele avoit refusé ; or fait edifisces or les despise, or mue les choses quarrees et les fet reondes. Quant je sui a Rome j’aime Tybur, et quant je sui aTybur je ayme Rome. Li corages est coupables, que nule fois ne fuit sa volenté. Cil ki vont outre la mer muent l’air, non pas le corage. Par quel neu tendrai je Proteum ki toute fois change son vout ?\par
Li mestres dit, de cestui vice avient que nus hom ne se tient apaiés de sa fortune ne de son estat. Orasces dist, mais chascuns desire choses diverses ; car li bués desire d’avoir frain et siele, et li chevaus desire arer la terre. Je jugerai que chascuns se tiegne a celui mestier a quoi il est livrés.\par
A ceste vertu apertienent \textsc{.v.} choses ; l’une est la parmanableté de l’entendement ki se sieut remuer en diverses pensees, la seconde est uns meismes corages es biens et es maus, la tierce est fermetés entour les choses desirees, la quarte est a endurer contre les tentations, la quinte est permanance es oevres.
\chapterclose


\chapteropen
\chapter[{.II.LXXXVIIII. De patience}]{\textsc{.II.LXXXVIIII.} De patience}\phantomsection
\label{tresor\_2-89}

\chaptercont
\noindent Patience est une vertus par qui nostre corages nous fait a soufrir les assaus d’aversités et les torsfais. Son office moustra Lucans quant il dist, patience s’esjoist es dures choses ; la plus grant leece k’ele puisse avoir est quant ele puet ovrer sa vertu. Li mestres dist, ceste vertus est remede de tortfet.\par
Orasces dit, tout li mal ki sont a venir devienent plus legier par patience. Boesces dit, par non soufrir te sera l’aventure plus aspre ke tu ne pués muer. Terrences dit, soufrons o bons corages ce que fortune nos aporte, car folie est de regiber contre aguilon. Senekes dist, li malades ki ne est obeissans fet enasprir son mire, car nule chose n’est si legiere ki ne te soit grief se tu le fés a ennui.\par
Et pour çou que ceste vertu est contre les passions covient il savoir ke les unes sont par volenté, les autres non. Et toutes soufrances que l’on fet de par son gré sont loables et sont dignes de merite, mais les uns ou les autres ou eles sont dedens ou dehors, et celes ki sont dedens sont por bien ou pour mal ki vient de hors si comme est ore leece, esperance, paour, et dolour. Celes ki sont dehors sont li anui et li tortfet que li autre nous font ou dient. Mais en toutes manieres de tribulations dois tu consirer la passion Jhesucrist et la maleurté Job, ki le sot si bien soufrir.\par
Aprés dois tu garder le travail que li mauvés suefrent por acomplir lor mauvestiés. Aprés consire se tu avoies deservi devant celui mal ou grignour, et consire la maniere dou mal ki vient et de celui ki le te fet ; car en chascune de ces choses pués tu prendre connort a bien soufrir toutes tribulations dou siecle.
\chapterclose


\chapteropen
\chapter[{.II.LXXXX. Encore de force}]{\textsc{.II.LXXXX.} Encore de force}\phantomsection
\label{tresor\_2-90}

\chaptercont
\noindent En ceste vertu, ce est force, et en toutes ses parties, de qui vous avés oï çou ke li contes en a dit, se doit on amesurer et garder soi de trop et de poi. Selonc ce que dist Senekes, sa magnanimité ist de sa mesure, ele fet home menaceour et enflé et tourblé sans repos, et courans a dire grans paroles sans nule honesteté, et par petites choses lieve et engroisse ses sourcis et commuet autrui, et chace et fiert. Et ja soit il si hardis et si fiers, certes il ara chetive fin entour les grans choses, et laissera de soi perilleuse ramembrance. Donques l’amesuree magnanimité est que l’en ne soit trop hardis ne trop paourous.\par
Mais ci se taist li contes a parler de force et de ses manieres, et tornera a la quarte vertu, c’est justice
\chapterclose


\chapteropen
\chapter[{.II.LXXXXI. Ci parole de la quarte vertu, c’est justice}]{\textsc{.II.LXXXXI.} Ci parole de la quarte vertu, c’est justice}\phantomsection
\label{tresor\_2-91}

\chaptercont
\noindent Justice vient aprés toutes les autres vertus. Et certes justice ne poroit rien faire se les autres vertus ne le faisoient ; car au comencement dou siecle, quant il n’avoit en tiere ne roi ne empereor, ne justice n’estoit conneue, les gens de lors vivoient en guise de bestes, li \textsc{.i.} en unes repostailles et li autre en \textsc{.i.} autre, sans loi et sans communité. Li home gardaissent volentiers la franchise que nature lor avoit donnee ; et n’eussent mie mis lor cos au joug des signories, se ne fust ce que les males oevres mouteplioient perilleusement et li maufetour n’estoient chastoiet. Lors furent aucun preudome ki par lor sens assamblerent et ordenerent les gens a abiter ensamble et a garder humaine compaignie et establirent justice et droiture ; dont pert il certainement que justice est celle vertu qui garde humaine compaignie et communité de vie. Car en ce que li home abitent ensamble, et li uns a terre gaignable ou autres possessions de quoi il a besoing, s’uns autres pour ce en fust commeus par envie et par descorde, se justice ne fust.\par
\par
Ceste vertu sormonte les aspres choses, car en ce que li uns est chevaliers et li autres marchans, li autres laboureres, et li pourchas de l’un enpire le gaaing de l’autre, les guerres et les haines naistroient et seroient a la destruction des homes, sa justice ne fust, ki garde et deffent la communité de vie ; de qui la force est si grans que cil ki se paissent de felonie et de meffet ne puent pas vivre sans aucune partie de justice, car li laron ki emblent ensamble welent que justice soit entr’aus gardee, et se lor mestre ne departent ygaument la proie, ou si compaignon l’ocient ou il le lient.\par
Tuilles dit, nus ne puet estre justes ki crient la mort ou dolour ou exil ou poverté ou ki met contre loiauté les choses ki sont contraires a ceste vertu ; c’est a dire ki est si liés d’avoir vie ou santé ou richece ou autre chose que l’om fait contre loiauté, cil ne puet estre justes. Tout li establissement de vie sont fet pour aidier as homes par force de justice, premierement que on ait a qui il puisse dire ses privees paroles ; et a ciaus ki vendent et achatent et prenent et baillent a loages, et ki s’entremetent de marchandises, est justice necessaire.\par
\par
De qui dist Senekes, en ceste maniere : Justice est jointe a nature et trovee par le bien et par le maintenement de maintes gens ; et n’est mie ordenemens d’ome, ançois est lois de Dieu et liiens d’umaine compaignie. Et en cestui ne covient a home penser que covenable soit, car ele le demoustre en ensegne.\par
Se tu vieus ensivre justice, premierement aime et criem Dieu Nostre Signour, si ke tu tu soies amés de lui. Et lui pués tu amer en ceste maniere, que tu faces bien a chascun et a nului mal ; et lors te clameront les gens justes et te sivront et feront reverence et toi ameront.\par
Se tu vieus estre justes, n’est mie assés a non damagier les autres, mais il te covient contrariier a ciaus ki damagier les welent, pour ce ke non damagier n’est pas justice. Non prendre a force les autrui choses, et rent celes que tu as prises, et chastie teus homes ki les prenent. Nule discorde ki soit devant toi n’entent par doubles paroles, mais garde la qualité dou corage. Une cose soit ton jurer et ton affermer, car ja n’i soit li non Dieu apelés, totefois i est il tesmoins ; et por ce ne trespasse la verité, a ce ke tu ne trespasses la loi de justice.\par
Et se aucunefois te covient mençoigne dire, tu le diras non mie por fauseté, mais por la verité deffendre. Se il te covient la verité raembre par mençogne, tu ne dois mentir, mais escuser ; la ou est honeste ochoison, l’ome juste ne descoevre pas les choses secrees, mais il taist ce que fait a taire et dist ce que fait a dire. Li hom justes est ensi que por apareillier pais et ensivre tranquillité ; mais quant li autre sont vaincus par mauvaises choses, il les vaint. Donques se tu faisoies tels choses tu atendroies ta fin liés et sans paour, et joians poras tu veoir les choses tristes, et quites seras en veoir les choses de rumour et seurs esgarderas les estremités.\par
Et pour ce que justice est li compliemens des autres vertus, apelent li plusor tout bien et toutes vertus ensamble par cest non, c’est justice apelee ; mais li mestres apele justice cele vertu solement qui rent a chascun son droit, as quex oevres nous semont nature en \textsc{.iii.} manieres, une ke Dieus fist l’ome tot droit pour senefiier le droit de justice, la seconde que por quoi çou ki apertient a justice est escrit en nos corages com par nature, la tierce est que tous autres animaus gardens justice et amour et pitié entre ciaus de lor maniere.\par
Autresi nous i semont li ensegnemens dou sage Salemon, amés justice, vous ki jugiés la tiere Salemon dist, combat toi pour justice jusc’a la mort. Salemons dit, devant la sentence apareille justice. St. Matheu dist, boneureus sont cil ki suefrent parsecution por justice. David dist, Diex serra les bouches des lyons pour çou que jou avoie justice. Salemons dist, justice enhauce les besoignans. Salemons dist, tresors de malisces ne proufitent noient, més justice garantira de mort. David dist, ma justice me armera devant toi. Salemons dist, justice est perpetuel et sans mort. Seneques dist, en justice est la trés grans resplendissours des vertus.\par
A justice apertienent \textsc{.ii.} choses, volenté de proufiter a trestous et de non anolier a nului, car ce sont li commandement de la loi naturele. Saint Matheu dit, faites as homes ce que vous volés k’il facent a vous. Li mestres dit, justice doit sivre le sens.\par
Mais \textsc{.ii.} volentés enpechent l’office de justice, ce sont paour et covoitise, et \textsc{.ii.} fortunes, c’est prosperité et adversité ; c’est a dire, se il sera aucuns ki par son sens soit dignes que tu li faces aucun bien, et li autre te dient que se tu li fes tu en auras haine d’aucun poissant home, vés ci que paours te fera cesser de l’office de justice. D’autre part sont aucuns vers qui tu dois estre larges, et je vuel garder mon avoir, vés ci que covoitise va contre justice. Pour quoi il covient que justice soit apoie de \textsc{.ii.} pilers, c’est de force contre paour et contre adversité, et d’atenprance contre covoitise et prosperité.\par
En fortune apert il, car contre prosperité doit on metre atemprance, et contre aversité doit on metre force ; car autrement la prosperité esleveroit trop home et l’adversité le baisseroit trop, si comme li contes a dit apertement ça en arieres. Por ce puet entendre chascuns que atemprance et force metent l’ome el siege de justice, et le tienent si fermement k’il n’en orguillist par prosperité ne ne crient par adversité.\par
La lois de Rome dist que justice est ferme et perpetuele volentés, et done a chascun son droit. Et pour ce poons nous entendre que toutes vertus et toutes oevres ki rendent ce k’eles doivent sont sous justice et desous ses parties. Mes il i a choses ke nous ne devons a tous homes, c’est foi et amour et verité ; et choses sont que nous ne devons pas a tous homes, mais a aucuns, si com li mestres devisera en son conte diligement. Mais tout avant dist il que justice est devisee principaument en \textsc{.ii.} parties, ce sont reddeur et liberalité.
\chapterclose


\chapteropen
\chapter[{.II.LXXXXII. De reddeur qi est la premiere branche de justice}]{\textsc{.II.LXXXXII.} De reddeur qi est la premiere branche de justice}\phantomsection
\label{tresor\_2-92}

\chaptercont
\noindent Reddeurs est une vertus ki restraint le torfet par digne torment. Et a \textsc{.iiii.} offices ; dont li premiers est que nus ne nuise a l’autre s’il n’ait avant receus le torfet, la seconde est que l’en use les communes choses si comme communes et les propres si comme propres. Et ja soit ce que nule chose soit propre par nature, mais commune, toutefois ce que chascuns en a est sien propre. Et se aucuns en demande plus, il brisera la droiture de l’umaine compaignie, et de çou vient tous descors, que tu t’esforces de torner mes choses en ta proprieté. Seneques dit, mais li home vesquissent mout en pais se ces \textsc{.ii.} paroles, mien et tien, fussent ostees dou mi.\par
Tuilles dist, li tiers offices de reddeur est oster les mauvés de la commune as homes ; car autresi comme on coperoit aucuns membres s’il commençoient a estre sans sanc et sans vie, k’il ne nuisissent as autres, doit on departir la felonie et la cruauté as mauvés de la compaignie des gens, car il sont homes non pas par oevre mais par non. Quele difference a il donc, se aucuns se mue en fiere sauvage, u s’il a samblance d’ome et cruauté de beste ?\par
Li ferues ki ne sentent garison pour nule medecine doivent estre malices de fier, dont ne doit on pas pardoner a teus homes. Senekes dit ke li juges est dampnés quant li maufetours est ostés. Tuilles dit, li juges se doit garder d’ire quant il juge, car o ire ne poroit il garder la moieneté ki est entre poi et trop. Catons dit, ire enpeeche le corage, si k’il n’a pooir de la verité connoistre. Orasces dit, quant li hom n’est sires de sa ire, il est raisons que çou k’il fet soit por non fet.
\chapterclose


\chapteropen
\chapter[{.II.LXXXXIII. Des juges}]{\textsc{.II.LXXXXIII.} Des juges}\phantomsection
\label{tresor\_2-93}

\chaptercont
\noindent Li juges doit tozjors sivre la verité, mais li advocas sivent aucune fois ce que samble verité, et le welent deffendre ja ne soit il verités. Salustes dit, tout cil ki jugent des choses douteuses, c’est a dire tot cil ki sont pour fere justice, doivent estre wit de haine d’amistié, d’ire, et de misericorde, car li corage a qui teus choses nuisent a paines puent veoir gaires de verité.\par
Tuilles dit, li riche tolent sovent au riche par envie, et donent au povre par misericorde. Senekes dist, maintenant que li hom ne vest persone de juge doit il vestir persone d’amis et garder que sa parole ne forcloe les autres autresi comme s’il fust venus en sa possession, il doit user communité en sa parole ausi comme es autres choses.
\chapterclose


\chapteropen
\chapter[{.II.LXXXXIIII. De liberalité, qui est la seconde branche de justice}]{\textsc{.II.LXXXXIIII.} De liberalité, qui est la seconde branche de justice}\phantomsection
\label{tresor\_2-94}

\chaptercont
\noindent Libéralités est une vertus ki donne et fet benefices. Ceste meisme vertus est apelee cortoisie ; mais quant ele est en volenté, nous l’apelons benignité, et quant ele est en oevre, nous l’apelons largece. Ceste vertus est toute en doner et en guerredoner, et par ces \textsc{.ii.} choses somes nous religieus vers Dieu Nostre Soverain Pere et pitieus a nostre pere et a nostre mere et as nos parens et a nostre païs et amables a tous et reverans as plus grans et misericordieus as besoigneus et non nuisans as plus foibles et concordans a nos voisins. Donques pert il bien que liberalités est devisee en \textsc{.vii.} parties, ce sont don, guerredon, religion, pitié, carité, reverence, et misericorde. Et pour çou ke chascuns rent çou k’il doit sont il vraiement partie et metre de justice.
\chapterclose


\chapteropen
\chapter[{.II.LXXXXV. Les ensegnemens de doner}]{\textsc{.II.LXXXXV.} Les ensegnemens de doner}\phantomsection
\label{tresor\_2-95}

\chaptercont
\noindent Or dira li contes de chascune partie de liberalité par sol, et premierement de don, ou il a ensegnement comment on se doit contenir en doner. Seneques dit, en doner garde que tu ne soies durs ; mais ki est li hom a qui il soufist d’iestre priés legierement a une seule fois ? ki est celui que quant il quide que tu li weilles demander aucune chose ki ne torne aucun po son front, ki n’endurcist sa face et fait samblant k’il est enbesoigniés ? Ce ke l’en done est deu par autretel corage comme il est donés ; et pour çou ne doit on pas negligemment doner. Nus ne guerredone volentiers çou k’il n’a receu de bon gré, ains la estors ; et cele chose doit on a soi meismes qu’il rechut dou nonsachant.\par
Li mestres dit, aprés ce garde toi de deloier ton don, de ce k’il l’a tenu en delai et laissiet en lonc atendre. Donques ne dois tu delaier çou que tu pués doner maintenant, car ki tost done, \textsc{.ii.} dois done : une fois done la chose, et une autre chose par samblanche que doner li plest. Senekes dit, on ne set gré dou don ki a longuement demoré entre les mains dou doneur, car ki mout doute est prochains a escondire, et ki tart donne longuement pense de non doner. De tant apetices tu la grace comme tu i més de demeure, pour çou que la face de celui ki te prie enrougist par honte. Mais cil ki ne se laisse demander longuement muteplie son don, pour ce que trés bonne chose est davancier le desirier de chascun. Senekes dit, cil n’a pas pour nient la chose ki par priere la requiert, nule chose ne couste plus chier que cele ki est achatee par priiere. Li mestres dist, c’est amere parole et anvieuse en qui doit baissier le vout, ke dire, le pri. Tobies dit, priere est vois de misere et parole de color.\par
Pour ce sourmonte toutes manieres de don cil ki vient a l’encontre et ki est fete sans requeste Tuilles dist, plus est gracieus uns petis dons fais isnelement que uns autres grans ki est a paines donés. La grace de celui ki donne amenuise s’il l’en covient priier a autres. Nule chose est si amere comme longuement atendre ; et maint home sevent millour gié d’escondire les tost ke metre les en delai.\par
Tuilles dit, aprés garde que ton don ne nuisent a ciaus qui tu les donnes, ou as autres ; car ki done autrui chose ki li nuise ne fait pas benefice mais malefice. Plusour sont si covoiteus de glore qu’il tolent as uns ce k’il donent as autres. Ki prent mauvaisement por bien despendre plus fait de mal ke de bien, et nule chose est si contraire a liberalité. Senekes dist, car il done a vaine glore non pas a moi. Tuilles dist, usons donc liberalité en tel maniere k’il vaille a nos amis et ne nuise a nului\par
Li mestres dit, aprés garde ke ton don ne soit grignour ke ton pooir. Senekes dist ; car en tele liberalité covient avoir covoitise de prendre l’autrui por avoir a doner.\par
Li mestres dit, garde que tu ne reproces ce que tu as doné, car tu le dois oublier, més celui ki le reçoit s’en doit ramenbrer. Tuilles dist, la loi de bien fere entre \textsc{.ii.} est tele que li uns doit tantost oublier ce k’il a doné, a l’autre doit tousjours souvenir que il a recheu. Il ne souvient pas a bon homme de ce qu’il a donné, se cil qui le guerdonne ne l’en fait sovenir. Estroitement lie cil ki si debonairement donne k’il li est avis qu’il gaigne çou qu’il done ; il done sans esperance d’avoir guerredon, et reçoit comme se il n’eust onques doné. Cil ki reprocent ou asprement ou debonairement ou ki se repentent de lor don brisent toute la grace. A qui Tuilles dit, ha, orgueil, a nul home plaist riens prendre de toi, car tu corrons quanque tu dones.\par
Li mestres dit, aprés gar toi de malicieus engien d’escondire, si comme fist li rois Antigonus, ki dist a \textsc{.i.} menestrer ki li demanda un besant qu’il li demandoit plus k’a lui n’afroit ; et quant il li demanda \textsc{.i.} denier, si li dist que rois ne doit si povre don donner : ce fu malicieus escondit, car il li pooit bien doner, \textsc{.i.} besant pour ce k’il estoit rois, u \textsc{.i.} denier pour çou k’il estoit menestrés. Mais Alixandres le fist mieus ; car quant il dona une cité a \textsc{.i.} home, ki li dist k’il estoit de trop bas afaire a avoir cité, Alixandres li respondi, je ne preng pas garde quele chose tu dois avoir, mais quele chose je doi doner.\par
Li mestres dist, aprés garde ke tu ne te plaignes de celui ki ne te set gré de ce que tu l’as servi : il en ert de millour se tu t’en tais, mais se tu t’en plains il en enpirra, car il est tousjours en doute de sa honte, mais maintenant ke tu t’en plaindras, sa honte est alee, et dira chascuns k’il n’est pas teus comme nous quidiens. Ne soies pas samblables a aus ; s’il ne te set gré d’un bienfet, il le te saura d’un autre, et s’il en oublie \textsc{.ii.}, li tiers li ramenra a memore ciaus k’il avoit oubliés. Quel raison est de couroucier celui a qui tu as doné grant chose, si ke cil ki est tes amis deviegne tes enemis ? Soies larges en doner et ne soies pas aigres en demander, car quant les laidenges montent plus hautes que les merites, celui a qui il plest s’en oublie ; et ki s’en dieut l’amenuise.\par
Li mestres dit, en liberalité devons nous sivre les dieus, ki sont signour de toutes choses : il comencent a doner a ciaus ki gré n’en sevent, et ne cessent de doner. Et lor volentés est de profiter, car li solaus luist sor les escomeniés, et la mers est habondee as larons. Donc se tu vieus ensivre les dieus, done neis a ciaus ki ne te sevent gré. Car se aucuns ne me set gré de ce que je li doins, il ne fet pas tort a moi mais a lui. Car a celui ki en set gré delite tozjours li benefices, mais a celui ki n’en set gré ne delite il c’une fois.\par
Ce n’est mie grant chose, doner et perdre ; mais perdre et doner apertient a grant corage. Vertu est donner sans atendre le change : je ameroie mieus non reçoivre que non donner. Cil ki ne done çou k’il promet meffet plus que cil ki ne set gré de ce ke l’en li donne. Rechevoir don n’est autre chose ke vendre sa franchise ; et por ce se tu promés a celui ki n’est pas dignes, donne li non mie por don mais por raembre ta parole. Lucans dist, franchise ne seroit pas bien vendue por tot l’or dou monde.\par
Tuilles dit, ja soit ce que tu dois g doner a chascun ki te demande, toutefoies on doit eslire ki en est dignes ; en ce doit on regarder les meurs de celui a qui il done, et quel corage il a vers nous et avec queles gens il abite, et en quele compaignie il vit, et le service k’il nous fet, ou que cil avoec qui il vit soient parfet ou aient samblance de vertu. Car je ne croi que nus doit estre despis en qui apert aucun signe de vertu. Et tu dois croire de chascun k’il soit bons, se le contraires n’en est provés.\par
Li mestres dit, chascuns doit estre honorés plus en tant comme il est aornés de plus legieres vertus, c’est de mesure et d’atemprance ; car fort corage et plus ardant est en celui sovent ki n’est parfetement sages. La premiere chose en servir est que nous devons plus a celui qui plus nous ayme. Mais il i a plusours ki font maintes choses par soudaine haste autresi comme s’il fussent esmeu par \textsc{.i.} po de vent : itel bienfet ne doivent estre tenus pour si grant comme s’il fuissent fet apenseement.\par
Il est autrement de celui ki a mesaise que de celui ki a tot bien et demande mieus : on doit plus tost bien fere a ciaus ki ont mesaise s’il ne sont digne d’avoir le mesaise ; mais nous nous devons dou tout escondire a ciaus ki beent de monter plus haut. Encor croi jou ke bienfet soit mieus emploiés as bons povres que as mauvés riches ; car cil ki sont riche ne voellent estre obligiés por bienfet, ains quident a toi fere grant bien quant il reçoivent de toi, ou il quident que tu atendes aucune chose d’aus. Se tu fés bien au mauvés riche, tu n’auras gré, fors que de lui et de sa mesnie ; mais se tu fés bien au bon povre, il li est avis que tu regardes a lui, non pas a sa fortune, ensi en auras gré et grasces de toz les bons povres, car chascuns le tenra en s’aide. Et pour ce se la chose vient en content, se ensivras Domistocles, ki dist quant il volt marier sa fille, j’aime mieus, fist il, home ki set soufreté de deniers ke les deniers ki aient soufreté d’ome.\par
Nous devons teus dons doner ki ne soient pas oiseus ; car a femes ne doit on doner armes as chevaliers. Senekes dist, nous donrons teus choses ki ne reprocent a home ne a sa maladie, c’est a dire ke l’om ne doit a l’yvre doner vin. On a dit li contes et ensegniet ce ki apertient a doner ; desormais dire il de guerredoner.
\chapterclose


\chapteropen
\chapter[{.II.LXXXXVI. De guerredoner}]{\textsc{.II.LXXXXVI.} De guerredoner}\phantomsection
\label{tresor\_2-96}

\chaptercont
\noindent Quant on a receu don ou autre bienfet par quoi il est obligiés a rendre le guerredon, nule chose est si necessaire comme rendre grasces, c’est a dire ke tu reconnoisses le bien que tu as receu, non mie par paroles solement, mais par oevre. Car se Ysodorus commande que tu rendes a grignour mesure que tu n’as emprunté, ke devons nous faire quant aucuns nous fet bien de son gré ?\par
Certes nous devons ensivir les chans gaignables ki apartent mout plus que l’om ne lor baille. Car se nous ne doutons a servir a ciaus que nos quidons ki nous vaudront, ke devons nous faire a ciaus ki nous ont ja valu ? Il est en nostre poesté doner u non doner, mais il ne loist pas a bon homme qu’il ne rende guerredon de ce k’il a receu, s’il le puet faire sans forfait.\par
Sor toutes choses garde que tu n’oublies le bien que chascuns te fait, car tout heent celui a qui il ne sovient dou bienfet k’il a receu et lor est avis que aussi oublieroient il le bien s’il le faisoient. Seneques dit, cil est mauvais ki denie le bienfet k’il a receu. Senekes dist, cil est mauvés ki n’en fet samblant, plus mauvais ki n’en rent guerredon, trés mauvés ki l’oublie. Cil ne puet gré savoir dou bienfet ki l’a tost oblié ; il pert bien k’il n’apensoit gaires a rendre, et cil ki l’oublie samble celui ki gete le don si long de soi k’il ne le puisse veoir ; car l’om n’oublie rien fors ce ke l’om ne voit sovent. Pour çou di je ke tu n’oublies pas le bienfet ki est trespassés. Nus ne tient por bienfet ce ki est trespassé, ains le tient autresi comme chose perdue ; s’il en est trais en court par devant juge, lors n’est il pas dons ne bienfés, ains commence a estre ausi comme dete enpruntee. Et ja soit il trés honeste chose de rendre grasces, ele devient deshoneste s’ele est fete par force.\par
Aprés te garde a que tu n’aproces a benefice par tortfet ; car il sont aucun ki rendent trop grant grasces, si k’eles sont mauvaises, car il voldroient que cil a qui il sont obligiés eussent aucune besoigne pour moustrer coment il se recordent dou bien k’il lor ont fet ; li lor corages est autresi comme cil ki sont eschaufé de mauvaise amour ; il desirent que lor amis soient essilliés por fere li compaignie quant il s’enfuira ; ou k’il soit povres por doner lui a son besoing ; ou k’il soit malades pour seoir soi devant lui : si ami desirent ce que si anemi voldroient de lui. Et par poi la fins de hayne est autretele comme de mauvaise amour ; car estrange felonie est de plongier \textsc{.i.} home en l’euue pour retraire l’en ou d’abatre por le relever ou enclore pour metre le hors. Car la fins de torfet n’est pas benefices, ne ce n’est pas services ki oste le mal k’il a fet.\par
Aprés garde ce que dist Tuilles ke tu ne te hastes trop a demoustrer ke tu saches gré. Cil ki devancist le tens de guerredoner peche ausi bien comme cil ki le passe. Car ce que tu ne vieus ki demeure entor toi samble que ce soit charge non mie don. Et est signes de geter ariere le don, quant on l’envoie maintenant \textsc{.i.} autre en lieu. Et a qui poise k’il n’a encore guerredon rendu, se repent dou don k’il a receu.\par
Aprés garde que tu ne rendes grasces en repost ; car cil ne set gré dou bienfait ki en rent grasces en tel maniere que nus ne l’ot. Mais sortot garde que tu reçoives benignement ; car en ce que tu reçois debonnairement, as tu rendu grasces ; mais n’en quides pas pour çou estre quites, ains estes plus seurement tenus a rendre ; car nous devons rendre volenté contre volenté et chose contre chose et parole contre parole.
\chapterclose


\chapteropen
\chapter[{.II.LXXXXVII. Encore de liberalité}]{\textsc{.II.LXXXXVII.} Encore de liberalité}\phantomsection
\label{tresor\_2-97}

\chaptercont
\noindent Encore est liberalités devisee en autre maniere, car l’une est en l’oevre, li autre en pecune ; et ki en a le pooir doit servir de chascune ou de l’une u de l’autre. Et cele ki est en pecune est plus legiere meismement a riche home, mais cele ki est en oevre est plus noble et plus digne a bonhome.\par
De celui dist Senekes, vertus n’est close a nului, ele est a tous overte, ne ele n’aquiert maisons ne terres, ele se tient pour apaie de l’home nu. Et ja soit ce que l’une maniere et l’autre de liberalité ou cele ki est en oevre ou cele ki est en pecune facent home plaisant et agreable, neporquant l’une vient de la huche, li autre vient de vertu ; et cele ki est de pecune apetice plus tost et ainsi est ostee par benignité. Car de tant comme tu en uses plus, de tant en poras tu mains user : kiconques plus done et despent ses deniers, tant en aura il mains.\par
Li autre maniere ki vient de vertu fet home plus riche et plus apareillié de bien fere, de tant comme l’om si acoustume plus. Quant Alixandres se porchaçoit d’avoir la bone volenté de ciaus dou regne son pere, ce ert de Macedoine, par deniers qu’il lor donoit, ses pere li rois Phelippes li envoia letres en tel maniere : quele erreur t’amaine en ceste esperance que tu quides que celui soient loiaus envers toi ke tu as corrompus par deniers ? tu fais tant que cil de Macedoine ne te tienent pas pour roi més por ministreor et por doneour. Cil ki reçoit en devient pires ; car tozjours est en atendance que tu li doignes ; neporquant on ne se doit dou tout retraire de doner, car a proudes gens ki ont besoigne doit on bien doner, més diligement et atempreement, pour ce que plusour ont gasté lor patrimoines por doner folement.\par
Li mestres dist, nule grignour folie n’est que faire tant que tu ne puisses longuement durer a fere ce que tu fais volentiers ; aprés les grans dons vienent les rapines, car quant on devient povre et besoigneus par doner, on est constraint de prendre de l’autrui, et lors a il grignours haines de ciaus a qui il tolt k’il n’a amour de ciaus a qui il dona. Catons dist, ki gaste les sienes choses, quiert les autrui quant il n’a plus ke gaster. Li mestres dit, pour ce que doner n’a font doit chascuns garder son aise et son pooir ; et generalement plus sont de ciaus ki se repentent de trop doner que de trop restraindre.\par
Mais entour ceste matere sont \textsc{.iii.} manieres de gens, car li i sont destruiseour, li autre sont aver, et li autre sont liberaus. Destruiseour sont cil ki a jeu de dés et en viandes et en doner as jougleours et as lecheours despent ce k’il at, de quoi n’est nule ramembrance ; et en some il despent çou k’il devroit retenir et garder. Avers est cil ki garde çou k’il devroit doner et despendre. Liberaus, c’est a dire larges, est cil ki de son chatel rachate les enprisonés ou aident a lor amis a marier lor filles ; et en somme cil est larges ki despent volentiers la ou il doit.\par
Encore doit on aidier as autres et par conseil et par paroles, et en court se mestiers est ; mais il se doit garder d’aidier en tel maniere as uns k’il ne nuise as autres ; car maintes fois grievent ciaus k’il ne doivent pas grever ; et s’il le font as fols ce est negligence, s’il le font as sages, c’est folie.\par
Quant tu grieves aucun maugré tien, tu te dois escuser et moustrer por quoi tu ne le pués autrement faire, et restorer lor par autres services ce en quoi tu le grieves. Mais pour çou ke toutes causes sont en acuser ou en deffendre, je di que le deffendement est plus loable ; et neporquant aucune fois doit hom et puet acuser, mais ce soit une fois sans plus. Tuilles dist que c’est home cruel, ou il n’est pas home, ki plusors acuse de chose dont il soit en peril. Vil renomee est que tu soies acuseour ; garde donc diligement que tu n’acuses home ki n’est coupables de chose dont il soit en periz, car ce ne puet estre fet sans felonie. Tuilles dist, il n’est nule si deshumaine chose comme de torner a la gravance des bons homes la parleure ki fu donnee por le salut des homes.\par
Li mestres dit, garde que ta parole ne moustre k’il ait visce en tes meurs, et ce siut avenir quant aucuns detrait a autrui et quant il se gabe et quant il mesdit. Tuilles dit, nous devons faire samblant que nous doutons et amons ceaus a qui nous parlons. Et maintefois covient il chastoier les gens ki sous li sont, par necessité, lors doit il parler grossement et dire aigres paroles ; ce devons nous faire si k’il samble que nous soions iriet. Mais que por chastoier et por vengier, neporquant a ceste maniere de chastoiement devons nous venir po et non liement. Mais ire soit loins de nous, avec qui nule chose non puet estre faite a droit.\par
Li mestres dit, on doit moustrer ke la cruauté k’il a en chastoiement soit par le meffet de celui k’il chastie, et contens que nous avons envers nos enemis devons nous soufrir de croire griés paroles, car il est drois de retenir atemprance et oster ire ; et les choses que l’om fet par aucuns troublemens ne puent pas estre droitement faites ne loees de ciaus ki les oënt. Laide chose est dire de soi meismement choses fauses et ensivre o gabois les chevaliers ki quierent vaineglore.\par
En toutes ces choses covient il ensivre les meurs as homes, non pas lor nature ne lor fortune. Mais ki est cil ki plus volentiers soustient la cause au povre que cele au riche et au poissant ? car nostre volentés se trait plus la dont nous quidons avoir grignour gueredon et plus tost.
\chapterclose


\chapteropen
\chapter[{.II.LXXXXVIII. De religion}]{\textsc{.II.LXXXXVIII.} De religion}\phantomsection
\label{tresor\_2-98}

\chaptercont
\noindent Jusques a ci a devisé li contes des \textsc{.ii.} parties de liberalité, c’est de doner et de guerredoner, ke l’en doit faire, et quoi non, et l’un et l’autre. Or wet aler outre as autres \textsc{.vii.} parties, mais tot avant dira de religion por ce qu’ele est la plus digne, car toutes choses et toutes vertus ki apertienent a divinité et ki nous amaine a faire oevre por aler a la vie pardurable sormonte toutes les autres choses.\par
Et religions est cele vertus ki nous fait curieus de Dieu et rendre li son service ; ceste vertus est apelee la foi de sainte eglise, c’est la creance ke li home ont en Deu. Et kiconques n’est fors et fiers en sa loi et en sa religion, a poines poroit estre loiaus hom, car ki n’est loiaus vers son Dieu, comment sera il loiaus vers les homes ?\par
Et li premiers offices de religion est repentir soi de tous ses meffés. Orasces dist, cil ki est bien repentans doit errachier de son cuer la mauvaise covoitise et les pensees ki font trop pechier, et en former la plus aspres estuides.\par
Le secont office de religion est poi prisier la mouvableté des choses temporaus, car aprés biau jour vient la noire nuit. Orasces dist, li uns jours reclot l’autre, et la nueve lune cort tousjours a son definement, pour ce ne dois tu avoir esperance es morteus choses, car l’un an tolt l’autre, et une heure fet perdre toz les jours. Nous somes ombre et poudre, tout somes doné a la mort et nous et nos choses ; porquoi as tu hui joie, que por aventure demain morras.\par
Li tiers offices est que l’en doit commetre sa vie dou tot a Dieu, selonc ce que dist Juvenaus, se tu as conseil, tu lairas a Deu despendre le tens et pourveir ke nous covient et ki est proufitable a nos choses ; car en lui de joieuses choses nous donra il les covenables. Il aime plus l’ome que li hom meismes ne fait ; por ce devons nous priier que nos pensees soient saines. Car Salustes dit, l’aide de Deu n est pas gaaignie par solement desirer ou par veu de feme ; mais por veillier et por faire prendre lor conseil vient toute boneeurtés. Quant tu seras abandonés a mauvaistié et a perrece, ne prie pas Dieu, car il est courouciés a toi.\par
Seneques dist, sachés que tu seras lors delivrés de toutes covoitises quant tu ne prieras Dieu de chose fors çou que tu poroies demander en apert. Il est grans derverie de l’ome consillier a Deu de ses vilains desiriers ; et se aucuns le wet ascouter, il se taist, et demande a Deu ce k’il ne vieut que li home sacent. Pour ce dois tu vivre avec les homes autresi comme se Diex te veoit, et parler a Dieu ausi comme se li home t’ooient.\par
Li quars offices de religion est garder verité et loiauté. Seneques dist que verités desoivre et trie la persone de l’home franc de celui dou serf, mais mençoigne le joint et melle.\par
Tuilles dit, por ce quident aucun ke ceste vertus soit apelee fois et loiautés, por ce que par li fet hons ce k’il doit. Neporquant on ne doit fere tozjors ce que on proumet, quant la chose n’est proufitable a celui qui ele est promise, ou se la chose nuist plus a toi qu’ele ne vaut a lui ; car il est plus drois eschivre le grignour damage ke le menour. Car se tu as promis a \textsc{.i.} home que tu seras ss advocas en sa cause, et dedens ce tes fiz s’acouche griement malades, il n’est contre l’office de la foi ne contre verité se tu ne fés ce ke tu as proumis. Et se aucune chose t’est baillie en garde, ele puet bien tele estre que tu ne le dois rendre tousjours ; car se aucuns quent il estoit sages et de bonne pensee te baillast a garder \textsc{.i.} glave, et puis quant il est forsenés le te demandast, tu feroies pechié se tu li randoies, et est vertus se tu ne li rens pas. Se cil ki te bailla deniers a garder commence bataille contre ton païs, ne li rendes pas ce ki te bailla, car tu feroies contre ton commun, c’est contre la communité de ta vile ou de ton païs, que tu dois avior mout chier.\par
Autresi avient il que maintes choses ki samblent honestes par nature, devienent deshonestes par trespassement de tans. Encontre ceste vertu font mortelment li papelart et li faus ypocrite ki moustrent ce k’il ne sont, pour decevoir Dieu et le monde.
\chapterclose


\chapteropen
\chapter[{.II.LXXXXVIIII. De pitié}]{\textsc{.II.LXXXXVIIII.} De pitié}\phantomsection
\label{tresor\_2-99}

\chaptercont
\noindent Pitiés est une vertus ki nous fet amer et servir diligeaument nos parens et nostre païs. Et ce nos vient par nature, car nous naissons premierement a Deu et puis a nos parens et a nostre païs. Catons dit, fius, combat toi pour ton païs, on doit faire tot son pooir por le commun profit de son pais et de sa vile.\par
A ces choses faire nous amaine force de nature, non pas force de loi. Senekes dist, autresi comme nus ne doit estre destrains d’amer sol, autresi ne commande pas la loi que l’en ayme pere et mere et ses enfans, car ce seroit oiseuse chose ke l’om fust constraint de fere ce k’ele fet. Li mestres dit, sor toutes choses devons nous esgarder que nous ne lor faisons aucun mal ne aucun tortfait. Salustes dit, se tu ies anemis as tiens, comment seront ti ami li estrange ? Terrences dit, ki ose decevoir son pere, comment le fera il as autres ? Ki ne pardone a soi, comment pardonera il a toi ne as autres ?
\chapterclose


\chapteropen
\chapter[{.II.C. De innocense}]{\textsc{.II.C.} De innocense}\phantomsection
\label{tresor\_2-100}

\chaptercont
\noindent Innocense est pureté de corage ki het a faire tous torsfais. Par ceste vertu est apaiés Dieus. Orasces dist, se mains nete d’omme, ki ne nuise nului, touche l’autel, nul sacrefisse plus delitable n’apaie Dieu. Tuilles dist, ki voldra garder ceste vertu, tiegne toz ses mesfais por grans, comment k’il soient petit. Orasces dist que nus ne naist sans visce ; mais cil est trés bons ki mains en est chargiés. Juvenaus dist, nus ne croit que ce soit assés s’il meffet tant comme il a loisir ; ensi enprent chascuns larguement le pooir.\par
Li offices de ceste vertu est a liier plusours a soi sans grevance de nului. Tuilles dit que ki fet tort a \textsc{.i.} il menace plusours, et fait paour a maintes gens. Li autres offices est non fere vengance Senekes dist que laide chose est perdre innocense par la haine d’un nuisant, et felonnie ne doit pas estre vengie par felonie. Cil met plusours sous ses piés ki trop aigrement wet vengier. Ovides dist, en vengier devient il trop nuisans.
\chapterclose


\chapteropen
\chapter[{.II.CI. De charité}]{\textsc{.II.CI.} De charité}\phantomsection
\label{tresor\_2-101}

\chaptercont
\noindent Charités est la fins des vertus ; ki naist de fin cuer et de droite conscience et non de fause foi. Ses commandemens est teus, ayme Deu et ton proisme autresi comme toi meismes. A ce nous connortent plusours raisons. Premierement sainte eglise ki tousjours crie, ayme ton proisme, et ayme les estranges si comme toi.\par
La seconde raisons est l’amours ke chascune beste a as autres de sa maniere. La tierce est la parenté de nature, ke tot somes estrait d’Adan et de Eve. Le quarte est le parenté de l’esperit ki est por la foi de sainte eglise, ki est mere de nous tous. La quinte est l’amors Jhesucris, ki volt morir pour l’amour des homes. La sisime est li essamples, car ja soit çou que tu aymes le filz ton ami, neporquant tu aymes mieus celui ki plus se resamble a ton ami, por ce dois tu amer toz homes, car il furent fait a la samblance Dieu.\par
La septime est le profit ki ensiut d’amor et de compaignie. Salemons dit, mieus vaut a estre \textsc{.ii.} ensamble que \textsc{.i.} sans plus, car li freres ki est aidiés par son frere est autresi comme ferme cités. Ambroses dist, bataille quant ele est enprise par commune volenté aquiert victore ; por ce donques li uns le charge de l’autre. Car Salemon dist que li coers se delite par dous oignemens et par bonnes espisses, mais l’ame s’esleece au bon conseil son ami. Tuilles dist, cil ostent le conseil dou monde ki ostent amistié des homes ; car a ce ke les humaines choses sont frailles et decheables, nous devons tousjours aquerre amis ki nous ayment et ki soient amés de nous, pour ce que la u la charités de l’amour est ostee, toute leece de vie est morte.\par
La octime raisons est les trés crueus damages ki avienent des guerres et de la haine du proisme ; et ja soit ce que amer et estre amés soit bone chose, toutefoies vaut il mieus a amer que estre amés, por ce ke grignour vertu est doner ke prendre.
\chapterclose


\chapteropen
\chapter[{.II.CII. Des choses qi aident amistié}]{\textsc{.II.CII.} Des choses qi aident amistié}\phantomsection
\label{tresor\_2-102}

\chaptercont
\noindent Et por ce que ceste vertus vaut a la vie des homes plus ke tote richesces dist li mestre k’il i a maintes raisons ki nous aident a ce ke l’en soit amés, et premierement avoir mesure en parler. Salemon dist, cil ki est sages en parler aquert amis, et la grasce du fol iert perdue. La seconde ert bonté et vertu. Tuilles dist, il n’est plus amable chose ke vertu, et nule chose ki tant nos atise a amour ; neis nos enemis, et ciaus que nous ne veismes encore, amons nous par la renomee de sa vaillance.\par
La tierce est humilités. Salemons dist, fai tes oevres par humilité, et tu seras amés sor toutes choses. La quarte est loiautés. Salemons dist, se tes sers est loiaus, soit autresi comme t’ame. Et ailleurs dist il meisme que loial amis est medecine de vie\par
La quinte est a encomencier. Seneques dist, ayme se tu vieus estre amés. La sisime est a servir, mais je ne di pas que services maintiegne amour s’il n’est fés sagement ; car sapience est mere de bonne amour. Salemons dist, il covient avoir sens a servir as amis. Seneques dist, cil ki se fie solement en ses services, ne a nul si perilleus mal comme cil ki quide que cil soient ses amis k’il n’aiment pas.
\chapterclose


\chapteropen
\chapter[{.II.CIII. Comment nos devons amer nos amis}]{\textsc{.II.CIII.} Comment nos devons amer nos amis}\phantomsection
\label{tresor\_2-103}

\chaptercont
\noindent Nous devons amer tous homes, meismement ciaus ki s’acointent de nous, en \textsc{.iii.} manieres. La premiere est ke nous les amons de bon gré, non pas por loier ou por achat, ke nous les amons non solement por le profit de nous més por le bien de nos acointes Senekes dit, por ce que amis ki est aquis par ochoison de proufit plest tant comme il est proufitables. Jeronimes dist, amistiés est vertus non pas marchandise. Ambroses dist, amistiés ne quiert chose, mais volentés ; et quant nous les amons sagement, c’est a dire bien faisant et ostant visces.\par
Car si comme Tuilles dist, ce n’est pas raisnable escusations que tu faces mal par achoison d’amistié, et que nous les amons de trés grant amour ; car il n’est nul grignour delit com de metre ta ame por ton ami et que nous les amons proufitablement, et de langue et de oevre ensamble.\par
Amistié fet aide de dis et de largece, car l’oevre est preuve d’amour, et que nous les amons pardurablement. Gregoires dist, quant hom bonseurés est amés, c’est mout douteuse chose de savoir se ses cors est amés u sa bonneeurtés. Seneques dit, ce que tu ne pués savoir par ton benefice, tu le sauras par ta poverté. Boesces dist, fortune descuevre la certaineté des amis ; car la u ele s’en va, ele te laisse les tiens et s’enporte ciaus ki tien n’estoient. Tuilles dist, n’eschive pas le viel ami por le novel. Et aprés dist il, il n’est nule si laide chose comme de combatre contre ciaus ki ont vescu avec nous.\par
La seconde maniere est que nos les amons autretant comme nous meismes et non mie plus ; car nule lois ne commande que tu aymes nului plus de toi, mais ki ne set amer soi ne set amer les autres. Ayme donc tes amis outre les choses cheables, et non pas outre toi ne ton Deu.\par
La tierce maniere est ke nous les amons autresi comme nos membres s’entraiment l’un l’autre, premierement que l’un membre n’a pas envie de l’autre, que chascuns membres depart son office as autres et que se li uns fet mal a l’autre, il n’en fet vengance, et que li uns se dieut du mal de l’autre et s’esjoist de son bien, et que li uns membres se trait avant pour deffendre l’autre, et que toz li cors se dieut por la perse de l’un des membres ; et ce que li uns reçoit, il le depart as autres ; et s’il le tient c’est son damage.
\chapterclose


\chapteropen
\chapter[{.II.CIIII. De la veraie amistié}]{\textsc{.II.CIIII.} De la veraie amistié}\phantomsection
\label{tresor\_2-104}

\chaptercont
\noindent Amistiés ki est sous charité est de \textsc{.x.} manieres. L’une est par droite foi et par veraie bienweillance ; et por ce dure tozjours en sa fermeté, ne ne puet estre desevree par adversité ne par chose ki aviegne ; et c’est avant tout le tresor du monde, por ce que nus hom ne puet venir a compliement de bien par soi solement ; et cele amistiés n’est autre chose que bone volentés envers aucuns pour achoison de lui. Salustes dit, li offices de ceste vertu est voloir et desvoloir une meisme chose, més qu’ele soit honeste.\par
Seneques dist, l’autre office est en chastoier en secré et loer en apert. Tuilles dist, la loi d’amistié est que nous ne demandons vilaines choses et que nous ne les faisons se aucuns nous en prie. Senekes dist, l’autre loi est que tu te conseilles de toutes choses o ton ami, mais premierement te conseille de lui.\par
Li tiers offices est que tu ne t’entremetes de savoir ce k’il te veut celer. Plus humaine chose est non faire samblant de la chose que de metre entente a savoir çou pour quoi tes amis te welent mal.\par
Li quars offices est que maleeurtés ne departe pas amistié, selonc ce que Lucans dit, il n’est pas avenant que l’en faille a son ami en adversité, car fois ne wet pas demorer avec le chetif ami.\par
Le quint office est la communités des choses ; por ce dist li philosophes quant il oï dire de \textsc{.ii.} homes qu’il estoient amis, por quoi donc est cil povres quant li autres est riches ? Et nanporquant Tulles dist, done selonc ton pooir, et non pas tout, més tant que tu puisses soustenir ton ami ; mais laide chose est, ce dist Tuilles, de metre le service a conte l’un parmi l’autre.\par
Li sisimes offices est de garder paroiletés, car amistiés resovire nul degré. Tuiles dit, grandisme chose est en amistié que li graindres se face pareil dou menour. Salemons dist, ki desprise son ami, est de povre vertu.\par
Li septimes est perpetualités. Salemons dist, tozjours ayme celui qui est tes amis ; et il meismes dist aprés, maintien foi a ton ami en sa povreté.\par
Li octimes est a non descovrir le secré ton ami, et celer son pechié.\par
Li nuevimes offices est a faire tost sa priere. Salemons dist, ne di pas a ton ami, va et revien demain.\par
Li disimes offices est a dire ce que li doit proufiter, ançois que ce ki li doit plaire. Li mauvais hom alaicte son ami et le deçoit de sa bouche.\par
De la veraie amistié dist Salemons, bonseureus est ki trueve son ami. Tuilles dist, amistiés doit estre mise devant toutes humaines choses. De ce dist Tuilles meismes que de tant vaut amistiés mieus que parentés, que amors puet perir en tous parens et tozjours remaint le non de parenté ; mais s’il perist entre les amis, li nons d’amistié perist avec. Salemons dist, li hom amables en compaignie iert plus amis que tes freres. Tuilles dist, veoir ton ami ou sovenir toi de lui est autresi comme veir toi meismes en \textsc{.i.} miroir ; et de ce avient que cil ki est loing de nous est autresi comme en present et cil ki est mors est autresi comme vivans.\par
Pour ce doit on consirer \textsc{.iiii.} choses quant om vieut ami conquerre, premierement s’il est sages, car Salemons dist que li amis des foz vient samblables a aus. Aprés garde s’il est bons, car Tuilles dist, je sai bien que amistiés ne dure se entre les bons non. Aprés garde qu’il soit debonnaires : Salemons dist, ne soies amis a home courouçable, car ire art et point. Aprés garde k’il soit humles. Salemons dit, ou il a orguel, vient courous et haine.
\chapterclose


\chapteropen
\chapter[{.II.CV. D’amistié par proufit}]{\textsc{.II.CV.} D’amistié par proufit}\phantomsection
\label{tresor\_2-105}

\chaptercont
\noindent Cil ki t’aime por son proufit est samblables au corbel et au voutoir, ki tozjors sivent les caroignes ; il t’aime tant comme il puet avoir dou tien, donc ayme il tes choses, non pas toi. Et se tes choses faillent que tu viegnes en povreté ou en adversité, il ne te connoist jamés, ains fait a la maniere dou rossignol ki au printans quant li solaus prent sa force et vienent flours et arbres verdoient, il demeure entor nous et chante et se solace sovent ; mais quant la froidure revient il s’enfuit et se part de nous hastivement.
\chapterclose


\chapteropen
\chapter[{.II.CVI. D’amistié par delit}]{\textsc{.II.CVI.} D’amistié par delit}\phantomsection
\label{tresor\_2-106}

\chaptercont
\noindent Et cil ki ayme pour son delit, fet autresi comme le tercelez de sa femele, que maintenant k’il a faite sa volenté charnelment, il s’enfuit au plus tost qu’il puet, ja plus ne l’ayme. Mais il avient maintefois qu’il n’ont nul pooir d’aus meismes, ançois abandonent cuer et cors a l’amour d’une feme ; en ceste maniere perdent il lor sens, si k’il ne voient goute. Si com Adan fist pour sa feme, de quoi tout humain linages est en peril, et sera tousjours ; David le prophete ki por la beauté Bersabee fist murtre et avoutire ; Salemons ses fis aoura les ydoles et fausa sa loi por amour Ydumee ; et Sanson li fors descovri a s’amie la force k’il avoit en ses cheviaus, dont il perdi puis la force et la vertu et la vie, et morut il et li siens. De Troie comment fu destruite sevent uns et autres, et de maintes autres terres, et de haus princes ki sont destruit por amer folement. Neis Aristotles li tres sages philosophes et Merlins furent deceu par feme, selonc ce que les ystores nous racontent.
\chapterclose


\chapteropen
\chapter[{.II.CVII. De reverence}]{\textsc{.II.CVII.} De reverence}\phantomsection
\label{tresor\_2-107}

\chaptercont
\noindent EVERENCE est cele vertus ki nous fet rendre honour as nobles persones et a celes ki ont aucune signourie. Et est son office por porter reverence a ses ainsnés et as grignours de lui. Trés bonne chose est a ensivre les traces des grignours s’il sont a la droite voie. Nous devons eslire \textsc{.i.} bon home, et avoir le tozjours devant les oils, si ke nous vivons autresi comme s’il fust presens, et faisons autresi comme sil nous veoit ; car grant partie de ces pechiés remaint a faire s’il i a tesmoing. Tuilles dit, tu dois croire que nul liu soit sans tesmong.\par
Mais pense ce que Juvenaus dist : quant tu vius faire vilainne cose, ne quide pas estre sans tesmoing. Et nous devons aprés Deu et aprés ses menistres honorer ciaus ki sont en plus haute dignité ; selonc ce que li Apostres commande que l’en rende honour a celui qui doit estre honouret. Sains Pierres dit, fetes honour au roi ; autresi devons nous honourer les plus anciens. Levitici dit, lieve toi encontre le chief chenu, honeure la persone dou viellart, autresi le devons nous honourer par dignité de nature. Exodi dist, honeure ton pere et ta mere. Et generaument devons nous honourer tous ciaus ki nous sormontent en aucune grace ou en aucune bonté. Et pour ce que nous devons croire que chascuns soit millour que nous ne sommes ou dou tout ou de partie, devons nous rendre honor covenablement.\par
Mais li hom ki sert, certes il doit servir et obeir volentiers, car il n’i a nule doute que celui ki s’offre a servir devant ce que l’en li commande n’aquiere plus de grace que cil ki le fait aprés le commandement. Saint Bernars dist que la obeissance des griés commandemens est plus loable que la contumace ne seroit dampnable, mais es legiers commandemens la contumace est plus dampnable que la obeissance ne seroit loable ; car la contumace Adam de tant fu ele plus dampnable comme li commandemens fu plus legiers et sans nule grevance.\par
Aprés doit chascuns obeir simplement sans noise et sans question. Saint Bernars dist, quant tu as oï le commandement, ne fere nule demande. Deuteronomio dist, fai ce que je te commant, non pas plus ne mains.\par
Aprés doit on servir liement. Li Apostles dist, Dieus ayme celui ki liement done. Jesu le fiz Syraac dist, en ton don soit lie ta chiere et ton visage\par
Apré doit on obeir vistement et humlement, si com Sains Pieres fist, que tot maintenant laissa ses rois et ala aprés Jhesucrist. Ensi doit chascuns obeir vistement, humelement, et pardurablement, en tel maniere qu’il aquiere grasce et qu’il le maintiegne quant il l’a aquise ; car assés puet on amis aquerre, et grasce, mais poi valent s’il ne le gardent.
\chapterclose


\chapteropen
\chapter[{.II.CVIII. De concorde}]{\textsc{.II.CVIII.} De concorde}\phantomsection
\label{tresor\_2-108}

\chaptercont
\noindent Concorde est une vertus ki lie en \textsc{.i.} droit et en une habitation ceaus d’une cité et d’un païs. Platons dist, nous ne somes pas ne por nous solement, mais une partie en a nostre païs et une autre nos amis. Et dient une maniere de philosophe ki furent apelé stoici, toutes choses sont criees as usages des homes, et li home sont engendré li uns pour achoison de l’autre, c’est a dire li \textsc{.i.} valent as autres, et por ce devons nous ensivre nature et metre avant tout le commun profit, et garder les compaignies des homes par services, c’est donnant et prenant de ses mestiers et de ses ars et de sa richece et en doner et en laissier de son droit debonairement as autres ; car doner dou sien aucune fois n’est solement cortoisie, mais puet estre grans proufiz.\par
Li mestres dit que pais fait maint bien et guerre le gaste. Salustes dist, par concorde croissent les petites choses et par discorde se destruisent les grandismes. Salemons dist, chascuns regnes qui est partis, en soi meismes sera destruis.
\chapterclose


\chapteropen
\chapter[{.II.CVIIII. De misericorde}]{\textsc{.II.CVIIII.} De misericorde}\phantomsection
\label{tresor\_2-109}

\chaptercont
\noindent Misericorde est une vertus par qui li corages est esmeus sour les mesaisiés et sor la poverté des tormentés. Terrences dist, ceste vertus ne quide que aucune chose humaine soit estrange de lui, et tient les autrui damages et proufiz por siens. Virgiles dist, je n’ai pas les maus, mais je wel secorre les tormentés. Senekes dist, ki a misericorde des malhaitiés, il li sovient de soi, més la cure des autrui choses est greveuse.
\chapterclose


\chapteropen
\chapter[{.II.CX. De .ii. manieres de tortfet}]{\textsc{.II.CX.} De \textsc{.ii.} manieres de tortfet}\phantomsection
\label{tresor\_2-110}

\chaptercont
\noindent En ariere a devisé li contes de justice et de tous ses membres, et comment ele est devisee en \textsc{.ii.} parties principaument, c’est en reddeur et en liberalité, et de chascun a il dit soufissablement, selonc ce que l’om trueve par auctorité des sages, ki sont alé. Donc il est bien covenable que il die de \textsc{.ii.} manieres de tort ki sont contraires a justice, de qui nous covient mout garder, c’est cruautés et negligence.\par
Cruautés est uns tors ki desloiaument fet mal a celui ki ne l’a pas deservi. Negligence est quant on puet bouter ariere ou vengier le tortfet, et ne le fet ; et c’est contraire a reddeur, car deffendre et non deffendre sont \textsc{.ii.} choses contraires, autresi est contraire cruauté a liberalité. Tuilles dit que droitfet et tortfet sont \textsc{.ii.} choses contraires.\par
Il i a \textsc{.iii.} causes por quoi on fet cruauté, ou pour paour, ou por avarice, ou por covoitise de dignité. Pour paour fet on cruauté quant on crient que s’il ne fet mal a \textsc{.i.} autre, k’il n’en set damage. Salustes dit, por avarice fet on cruauté quant on fet tort a \textsc{.i.} autre por avoir ce k’il covoite. Salemons dit, covoitise de dignité a constraint plusours mortels de devenir faus, car il portent une chose en lor pis et \textsc{.i.} autre en lor bouche ; il ne sevent eslire amistié et haine par la chose mais par le preu, et ayment plus le vout que la volentés ne ke engien. Tules dist, mais il i a une male chose, que maintesfois la covoitise de dignité sorprent les hardis et les larghes homes ; car hardement fet les homes plus prest a guerroier, et largesce lor done grans aides, et por ce vient de lor covoitises grans tormens.\par
Lucans dist, entre \textsc{.ii.} rois d’un roiaume n’a point de foi car nus ki en poesté soit ne puet soufrir compaignons. Couvoitise de dignité est chose forsenee et avuglee, nule fois ne nule pitiés n’est en ciaus ki sivent l’ost. Les mains, ki n’entendent se a vendre non, quident ke la soit le droit ou il a grignour loier. Li mestre dist, la cours est mere et norrice de mauvaises oevres, car ele reçoit le mauvais ausi comme le bon, et le honeste ausi com le deshoneste.\par
Cruautés est devisee en \textsc{.ii.} manieres, l’une est force, l’autre est boisdie. Force est comme de lion, boisdie est comme de gourpil. L’une et l’autre est pesme chose et deshumaine ; mais boisdie est plus haïe, car en toute desloiauté n’a grignour pestilence que de ciaus que, quant il deçoivent, si s’efforcent de resambler bons ; nul agait n’est si perilleus comme cil ki est covers en samblance de service.\par
Orasces dist, garde ke ne te deçoivent li corage ki s’atapissent sous le gorpil. Juvenaus dist, li membre velu et les dures soies es bras mostrent la cruauté dou corage ; el front n’a nule foi : ki est ce donc, ki ne soit plains de visces tristes et ors ? Li mestres dist, garde toi de l’euue plaine, et entre en la rade seurement.
\chapterclose


\chapteropen
\chapter[{.II.CXI. De la negligence des justes}]{\textsc{.II.CXI.} De la negligence des justes}\phantomsection
\label{tresor\_2-111}

\chaptercont
\noindent Autresi sont \textsc{.iiii.} causes en negligence, c’est en non deffendre le tortfet ; car il sont aucun ki ne welent pas avoir haine ou travail ou despens en deffendre, ou il sont si encombré de lor besoignes, ou si plain de haines k’il guerpissent ciaus k’il doivent deffendre. Tuilles dist, mais plus seure chose est a estre negligens envers les bons que envers les mauvais. Salustes dist que les bons en devienent plus pereceus en bien fere, mais li mauvais en devienent plus engriés de maufaire.\par
Li mestres dit, autresi di je que plus seure chose est a estre negligens envers les riches que envers les povres mesaisiés. Terrences dit que tout cil ki ont aversité ou mesaise et ne sevent por quoi, souspeçonnent que tout çou que on fait soit por lor mal ; et lor est tousjors avis que l’en les despit por lor nonpoissance. Tuilles dit, en toute desloiauté, a mout grant difference se li tors est fés par troublement de corage ou penseement ; car troublemens est briés et ne dure c’un petit, et toutes choses ki avienent par soudain movement sont plus legieres ke les pensees devant.
\chapterclose


\chapteropen
\chapter[{.II.CXII. Comment on doit moderer justice}]{\textsc{.II.CXII.} Comment on doit moderer justice}\phantomsection
\label{tresor\_2-112}

\chaptercont
\noindent En justice doit on garder dou trop et dou poi, et faire moienement. Selonc ce que dit Senekes, en joustice te covient avoir mesure, por ce que tu ne dois estre negligens en governer les grans choses et les petites. Ta face ne doit estre trop mole ne trop cruele, ton vis ne soit trop aspres si qu’il n’ait en soi aucun samblant d’umilité ; donc dois tu ensivre l’ordene de justice, en tel maniere que ta doctrine ne deviegne vil par trop grant humilité, ne ne te moustre si dur et si cruel que tu en perdes la grasce de la gent.
\chapterclose


\chapteropen
\chapter[{.II.CXIII. De la comparison des vertus}]{\textsc{.II.CXIII.} De la comparison des vertus}\phantomsection
\label{tresor\_2-113}

\chaptercont
\noindent Li contes a devisé ça arieres ke en celes sciences ki ensegnent a home governer soi et autrui, puet il avenir que celui bien ke l’en i desire est solement honeste, u que li uns est plus honestes que li autres. Et il a moustré jusc’a ci liquel bien sont honeste, ce sont les \textsc{.iiii.} vertus, et li lor membre briement et apertement. Et or dira des biens ki sont plus honeste li \textsc{.i.} que li autre.\par
Et li mestres a dit au comencement que prudence, c’est sens et cognoissance, doit tozjours aler devant les oevres, et dist que les autres \textsc{.iii.} vertus sont pour faire l’oevre. Mais il i a choses es quex l’oevre doit avancier le sens, por ce qu’ele est lors plus honeste. Raison coment : se aucuns est mout desirans de conoistre les natures des choses, et comme il mete en ce savoir tout son sens, uns autres vient a lui et li aporte noveles soudainement que sa cités et son païs sont en peril, s’il ne lor aide ; et il en a le pooir d’aidier le ; dont est il plus honeste chose delaier s’estude et aler deffendre son païs. En ceste maniere vois tu que prudence est ariere les autres vertus.\par
Entre les autres \textsc{.iii.} vertus, doit atemprance estre mise par devant les autres \textsc{.ii.}, car par li governe hom lui meismes, més par force et par justice governe il sa mesnie et sa cité. Et mieus vaut a l’ome avoir signourie de soi que d’autri, selonc ce que dist Orasces, plus grant regne governe cil ki donte sa volenté que s’il eust la signorie d’occident jusk’en orient et de midi en septentrion. Senekes dist, se tu vieus sousmetre a toi toutes choses, si sousmet avant toi meismes a raison ; car se raisons te governe, tu seras governeour de plusours, mais riens n’est bons a home s’il n’est bons avant. Tuilles dist, on ne doit rien faire contre atemprance por amor des autres vertus. Mais aucunes choses sont si vilaines que nus sages ne le saroit, car neis a noumer sont eles laides.\par
Entre les autres \textsc{.ii.} vaut mieus justice que force, car en justice a degrés d’offices : li premiers est a Dieu, li secons est au païs, li tiers as parens, et li autres aprés, selonc ce que li contes devise la ou il dist des parties de justice entour la fin de liberalité.\par
Et en some, en cele vertu ki est apelee force, se aucuns est de si grant corage k’il despit les communes gens, c’est cruauté et fierté s’il ne fust justiciet a droit ; donques est justice plus honeste que force. Mais ci se taist li contes a perler des choses honestes, dont il a mout longement traitié, si tornera a dire des biens dou cors et des dons de fortune.
\chapterclose


\chapteropen
\chapter[{.II.CXIIII. Des biens dou cors}]{\textsc{.II.CXIIII.} Des biens dou cors}\phantomsection
\label{tresor\_2-114}

\chaptercont
\noindent Li bien dou cors sont \textsc{.vi.}, biauté, noblece, isneleté, force, grandeur, et santé. Ce sont li bien de par le cors, dont li \textsc{.i.} ont plus et li autre mains ; et teus sont qui mout si delitent et efforcent a la fois, et li \textsc{.i.} plus que li autre ; mais sovent en puet plus avenir de mal que de bien, et plus honte que honours ; car par le delit d’aus il refusent et chacent la vertu. Et por çou dist Juvenaus que biauté ne s’acorde gaires a Dieu ne a chasteté, et que pris de biauté ne delite les chastes ; mais il dist que cele est chaste ki onques ne fu requise, donques pert il bien, a ce que biautés de cors est contraire a chasteté.\par
Et cil ki se delitent en noblesce de lignie, et ki se vantent de haut antecessours, s’il ne font les vertueuses oevres, il ne pensent bien que li los de lor parens torne plus a lor honte que a lors pris. Car quant Catelline faisoit la conjurison de Rome priveement, et n’ouvroit se mal non, et il disoit devant les signatours la bonté son pere et la hautece de son linage et le bien ke ses linages fist a la commune de Rome, certes il disoit plus sa honte que s’onnour. Et en ce dist Juvenaus que tant est li hom blasmés plus de malfaire comme les gens quident qu’il soit de plus grande hautece. Seneques dist, la vie des antecessours est autresi comme lumiere de ciaus ki vendront aprés, tele que lor mal ne suefre k’il soient en repost.\par
Li mestres dist, tot visce sont plus seu de tant comme cil ki peche est graindres. Mais de la droite nobilité dist Orasces qu’ele est vertus solement. Et pour çou dist Alixandres que noblesce n’est autre chose se cele non ki adorne les corages a bonnes meurs. Donques n’a en celui nule noblesce ki use vie deshonestes. Et pour çou dist Juvenaus, je aime mieus que tu soies filz Tercides, et tu resambles Hector, que se Hector t’eust engendré et tu resamblaisses a Tercides ki fu li plus chetis hom del monde.\par
Li mestres dit, por quoi je di que li millours fruis ki en noblesce d’antecessours soit, si est cele qui Tuilles dit, li grandesime eritage que li fiz ont de lor peres, et ki sormonte toz patrimoines, c’est glore de vertu et des oevres k’il ont fetes.\par
Or vous ai je dit \textsc{.i.} poi coment beauté et gentillece sont contraires a oevre de vertu ; que vous diroie je d’isneleté et de grandeur ou de force de cors ? de qui Boesces dist, vous ne sormonterés pas les olifans pour grant cors, ne le tors pour force, ne les tygres pour isneleté : la seule mors moustre quex sont li cors des homes et coment il sont decheable.
\chapterclose


\chapteropen
\chapter[{.II.CXV. Des biens de fortune}]{\textsc{.II.CXV.} Des biens de fortune}\phantomsection
\label{tresor\_2-115}

\chaptercont
\noindent Li bien de Fortune sont \textsc{.iii.}, richece, signourie, et gloire. Et vraiement sont il bien de Fortune, car il vont et vienent d’eure en heure, ja n’auront point de fermeté ; car Fortune n’est pas chose raisnable, ne son cors n’est mie par droit ne par raison, si comme ele moustre tozjours de maint home ki sont noient de sens ne de valour, et si montent en grandisme richece ou en grant dignité de signorie, ou en loenge de grant pris ; et uns autres, ki sera li plus vaillans hons del monde, ne poroit avoir seul petit bien de Fortune. Por ce dient li plusor que fortune est avuglee, et qu’ele tournoie tousjours sa roe en non veant. Mais nous en devons croire ce que li sage en dient, que Dieus abaisse le puissant et enhauce le foible. Et toutefois en dira li mestres aucune chose, tant comme il en covient a bon home.
\chapterclose


\chapteropen
\chapter[{.II.CXVI. De la premiere branche de fortune ce est richesce}]{\textsc{.II.CXVI.} De la premiere branche de fortune ce est richesce}\phantomsection
\label{tresor\_2-116}

\chaptercont
\noindent Richece est en avoir iretage, sers, et pecune. En heritage sont conté edifices et terres gaignables. De ce nous ensegne Tulles, garde, fait il, se tu edifies, que tu ne faces pas trop grant despenses, car on doit garder moieneté. Orasces dist, ki ayme droite moieneté si ne face trop vil maison ne trop grant. Tuilles dist, li sires ne doit pas estre ennoblis par sa maison, mais sa maison par lui. Seneques dist, nule maisons n’est trop petite ki rechoit assés d’amis ; car grans maisons u nus n’entre est honte au signour, meismement se plusour i entroient au tans de l’autre signour.\par
Vilaine chose est quant li trespassant dient. Ha, maisons, comme tu as malement changié signour. Pour ce dist Orasces, ne te chaut de grant maison, car en petite maison pués tu mener roial vie. Lucans dist de Julle Cesar, il ne voloit mangier fors por vaincre son faim, ne maisoner fors por le froit et por la pluie ; mais on doit loer grant vaisselemente en petite maison. La grandour des maisons n’oste pas les oevres, selonc ce que dist Orasces, se tu ies si riches que tu aies tous les deniers dou monde, et soies de noble linage, rien ne te vaut a la fin ; nient plus ke se tu fusses de basse gent povres sans maison, car tu morras, a ce ne pués tu contrester, nus sacrefices ne te vaudront a la mort ne tost ne tart. Ja maisons ne terre ne monciaus d’or n’osteront les fievres dou cors lor signor, car quant il est malades, cil qui est covoiteus de gaaignier et a paour de perdre ; ausi li aident ses maisons, ou ses avoirs, comme les tables pointes aident a celui ki a mal en ses oils. La voire mors boute igaument as petites maisons des povres et as grans cours des rois.
\chapterclose


\chapteropen
\chapter[{.II.CXVII. Des sers}]{\textsc{.II.CXVII.} Des sers}\phantomsection
\label{tresor\_2-117}

\chaptercont
\noindent Vous avés oï d’iretage, or poés oir de la seconde maniere de richece, c’est des sers et ce que li serf doivent faire. Premierement l’ordone ce que mestiers lor est, et puis demande lor service. Seneques dist, li sires est deceus quant il quide que li servages descende en tout l’ome, car la millour partie en est ostee : li cors sont tenu as signours, mais la pensee est franche, qu’ele ne puet estre tenue en la chartre ou ele est enclose, qu’ele n’aille a sa volenté.\par
Li maistres dist, tu dois donques vivre avec celui qui est plus bas de toi, ensi comme tu voldroies que cil ki est plus haus vesquist avec toi ; et toutes les fois k’il te sovendra combien tu as de pooir sour ton sergant, soviegne toi que autretel pooir a tes sire sor toi.\par
Li offices as sergans est confermer soi premierement a la maniere son signour, selonc ce que dist Orasces, li triste heent les joieus et li joieus les tristes, et li isnel les pereceus et li pereceus les isniaus, li beveur heent ciaus ki ne welent boivre. Ne soies donques orguilleus, car li amesurés desmesure aucune fois, et li paisibles prent samblanche de folour. Cil ki crera que tu consentes a sa maniere te loera et amera plus. Orasces dist, li services as poissans est dous a ciaus qui ne l’ont esprové, cil ki l’a esprové le crient ; pour çou garde que quant ta nef est en haute mer tu la governes en tel maniere que se li vens change, il ne le porte en leu perilleus.\par
Le second office est, loer les bons et garder soi des mauvés. Orasces dit, garde que tu loeras, et que autrui pechiés ne te faces honte ; nous somes aucunefois deceu quant nous loons celui ki n’en est dignes : laisse donc a deffendre celui ki sa coupe aprient, car par aventure, quant il wet aucun maufaire, il se fie en ta deffense. Mais la tue maisons est en perilh se tu ne le secours, quant tu vois ardre chiés ton voisin.\par
Li tiers office est de refrener avarice et luxure. De quoi dist Orasces, ne te demant covoitise, ne soies desirans de la biauté d’une meschine ou d’un enfant.\par
Li quars offices est oster orgueil. Et ce dist uns sages hom, ne loe pas tes oevres ne ne blasme les autri ; soies soués en servir ton puissant ami.\par
Le quint office est qu’il ne se plaigne pas. Orasces dist, cil ki devant lor signour se taisent de lor povreté enportent plus que cil ki demandent. Il i a difference entre prendre honestement et ravir ; car se li corbaus se peust taire quant il menguë, il eust plus a mangier et mains de noise et d’envie.\par
Le sisime office est acomplir ce que ses sires li comande, et k’il n’ait aucune soufreté. Lucans dist, li besoins as sergans n’est pas griés a lui, mais au signour ; li sergans se doit mout garder k’il ne soit gengleres. Juvenaus dist, la langue est la pire partie dou mauvais sergant ; mais il doit tel signour eslire, s’il puet, k’il soit dignes que l’on le serve, car por la dignité as signors sont ennoblis li services as sergans.
\chapterclose


\chapteropen
\chapter[{.II.CXVIII. De pecune}]{\textsc{.II.CXVIII.} De pecune}\phantomsection
\label{tresor\_2-118}

\chaptercont
\noindent Or a dit li contes des \textsc{.ii.} parties de richece, si dira de la tierce, c’est de pecune. Et en pecune sont conté deniers, tresor, adornement, et tous meubles. De qui dist Tuilles, nule chose n’est de si petit corage comme d’amer richesce, Seneques dist, pour çou est cil grans qui use de vaissiaus d’or comme de vaissiaus de terre ; et cil n’est pas maindres ki use des vaissiaus de fiere comme d’or. Juvenaus dist, nule chose n’est plus haute ne plus honeste ke despire pecune se l’en ne l’a, et d’estre larges ki les a.\par
De trop covoitier ces choses nous rapelent plusors causes. La premiere est por ce que vie d’ome est corte. Orasces dist, la brieté de vie nos moustre que nous ne devons commencier chose de grant esperance ; tu ne ses se tu vivras demain. Ne pense donc de demain, car Dieus ne vieut que nous sachons ce ki est a avenir, mais l’ordere de la chose presente. Car cil doit estre liés ki puet dire, je ai bien vescu \textsc{.i.} jour ; car se li jours d’ui est clers, cil de demain est oscurs, car nule chose est bieneuree de toutes pars. Seneques dit, en ce somes nous tout deceu, ke nous ne pensons a la mort ; car grant partie en est ja passee, ele tient tout ce ki est alé de nostre aage. Perses dit, pense toute voies que tu morras maintenant : mors enporta le noble Hector lors k’il vivoit glorieusement, et viellece amenuisa la grant renomee del grant Tytonus.\par
La seconde cause est por ce que covoitise de richece abat les vertus. Orasces dit, cil pert s’arme et laisse les vertus ki tozjours se haste d’acroistre son chatel ; il dechiet por avoir, car joies et leeces ne vienent solement as riches home ; ne cil ne vesqui mal ki se morut en naissant. Juvenaus dist, nus ne demande dont çou vient k’il a, més qu’il le puisse avoir. Orasces dist, ne lynages ne vertus n’est prisie sans richesce ; nule chose n’est mie assés, car tu vieus chascun autant de foi comme il a deniers en la huche.\par
Nule plus dure chose n’est en poverté que ce que l’en le gabe. Orasces dist, richece donne biauté et gentillece, por ce que vertus et renomee et honours et toutes choses divines et humaines obeissent a richesce ; et ki les aura, il sera nobles, fors, loiaus, et sages et rois. Mais ce lor torne a l’encontre, car pecune aporte visces et male renomee en lieu de vertu.\par
La tierce chose est que denier font home vicieus. Selonc ce que Juvenaus dist, richesce amena premierement mauvaises meurs et raempli le monde d’orgoil, car cil ki eurent premierement richesces soillierent mariage et linage et maisons, dont puis sont avenu mains perils au peuple et au païs. Mais Orasces dit apertement ke noblesce ne vient pas por avoir, la u il dist, ja soit ce que tu ailles o les orguilleus por ton avoir, fortune ne mue pas gentillece ; car se \textsc{.i.} pos de terre estoit tous covers d’or, ja por l’or ne remaint que ce ne fust boe.\par
La quarte causa est que nul conquest ne saoule covoitise. Orasces dist, richesces croissent engriessement, et tousjours faut aucune chose. Tant comme li avoirs croist, tant croist la cure et la covoitise ; qui mout aquiert mout li faut. Cil est bien riches ki se tient apaiés, et cil est povres ki bee a grant richece. Cil n’est pas povres a qui soufist ce k’il a a sa vie : se tu ies bien peus et vestus et chauciés, toutes les richesces au roi ne te poroient croistre de rien.\par
La quinte cause est la poour que li avoirs t’aporte. Juvenaus dist, ja soit ce que tu ne portes que \textsc{.i.} poi d’argent, se tu vas par nuit tu auras paour des larrons, et se tu vois a la lune \textsc{.i.} petit rainsel movoir, tu auras paour ; mais cil ki rien ne porte va chantant devant les larons. Penible chose est de garder grant avoir.\par
La sisime cause est ke pecune vieut que l’en soit ses sers. Orasces dist, la pecune ou ele sert ou ele est servie, mais il est plus digne chose qu’ele ensive la corde son signour qu’ele tire lui. Por ce dist Orasces, je wel sousmetre mes coses a moi, non pas moi a mes choses. Tulles dist, et por ce que ensi est la chose, ne croire que celui soit bonseureus ki porsiut mout de choses, mais celui ki use sagement ce que Dieus li a donné et ki bien suefre sa povreté, et ki plus crient visce que mort. Car riche chose et honeste est lie povretés, et dolerous usages est grans povretés. Senekes dist, cil n’est pas povres qui liés est ; et ki bien s’acorde a sa poverté est riches ; ne cil n’est povres ki a petit mais ki plus covoite.\par
Se tu vieus enrichir, tu ne dois croistre ton chatel mais apeticier ta covoitise. La corte voie a enrichir est despire richesce ; car on puet bien tout despire mais tout ne puet on avoir. Et por ce dist Senekes, Dyogenes li povres fu plus riches que li grans Alixandres, car plus valoit ce qu’il ne volsist recevoir que ce que Alixandres pooit donner ; car poi valoit ce k’il avoit en sa huche ou en ses greniers, puisk’il ne baoit se a l’autrui non ; et ne contoit par çou k’il avoit conquis, mais ce ki remanoit a conquerre.\par
Et se aucuns demandoit quele est la mesure de richese, je diroie que la premiere est ce que necessités requiert, la seconde ke tu t’apaies de ce ki est assés ; car ce que nature desire est bien, se tu ne li donnes outrages. Boesces dist, Nature se tient apaie de petites choses. Mais ci endroit se taist li contes a parler de richece ; si tornera a dire dou secont bien de Fortune, cest signorie.
\chapterclose


\chapteropen
\chapter[{.II.CXVIIII. De signourie}]{\textsc{.II.CXVIIII.} De signourie}\phantomsection
\label{tresor\_2-119}

\chaptercont
\noindent Segnorie est uns des biens ki vient par Fortune. Et ja soient signories de maintes manieres, sor les autres est la plus digne cele des rois et de governer cités et gens ; c’est li plus nobles mestiers c’on puisse avoir au monde, et entour ce est la tierce science de pratike, si comme li mestres devise ça en ariere ou conte de philosophie. Et de ceste science ne dira ore li contes plus, se ce non k’a moralité s’en apertient ; més en avant dira li mestres ce que s’en apertient a signorie et au governement de cité, selonc ce que requiert l’usage de son païs et la lois de Rome.\par
Selonc le commandement de moralité et de vertu l’en doit atemprer les desiriers de signorie. Juvenaus dit, poissance fait maint cheoir. Lucans dist, l’ordre des destinés est envieus, car il est devee a hautes choses qu’eles ne durent longuement, et il est grief a cheoir desous pesant faisel. Les grans choses decheent par eles meismes, et ce est li termes jusques a qui Dieus laisse croistre les leeces ; et il done legierement les grans choses, mais a paine le garantissent.\par
Seneques dist, tu troveras plus legierement fortune que tu ne la tendras. Orasces dist, grans arbres est sovent crollés par vent, et les hautes tours cheent plus pesamment, et le foudre chiet es hautes montaignes. Autresi fet Fortune, ki sovent change les geus en dolour et faut de haut bas ; quant ele bat ses eles, il me covient laissier ce k’ele m’a doné. Senekes dist, Ha, Fortune, tu n’ies pardurablement bonne.\par
Aprés doit on atemprer le desirier de signorie, por ce qu’ele descuevre faintise et ypocrisie, car il est grant chose obeir a la signorie de ciaus qui fainsent k’il fuissent bons por la covoitise d’avoir cele signorie. Il i a plusours ki aucunefois sont humle et autrefois orguilleus, et c’est selonc fortune, non pas de corage. Terrences dist, il est ensi de nous, que nous somes grans et petis, selonc ce que fortune se porte.\par
Li offices de signour est k’il atraie le peuple a lor preu. Et Tuilles dist, il n’est nule chose ki face plus a tenir signorie que estre amés, ne nule plus estrange que estre cremus. Salustes dist, plus seure chose est commander a ciaus ki vuelent obeir que a cels qui en sont constraint. Seneques dist, li sousmis heent celui k’il criement, et chascuns desire que ce k’il het perisse. Tuilles dist, paour ne garde longhement son signour. Juvenaus dist, poi de tirant muerent s’il ne sont ocis. Mais bienweillance est bone garderesse de signour, et perpetualment le fait renomer aprés sa mort. Cil ki welent estre cremus, covient k’il criement ceaus de qui welent estre cremus. Boesces dit, ne quides tu que cil soit poissans ki tozjours amaine gardes avec lui, car il crient ciaus a qui il fet paour.\par
Tuilles dist que uns hom ki avoit non Denis cremoit tant les rasoirs as barbiers que il brulloit ses paus. Et Alixandres, quant il voloit gesir avec sa feme, il mandoit devant ses sergans pour enchercier ses huches et ses dras, k’il n’i eust coutel repost : c’estoit mauvaistiés a fier soi plus d’un sergant ke de sa feme ; ne pour ceste suspection ne fu il trais de sa feme, més de ses sergans.\par
Soviegne au signour k’il fu sans dignité. Seneques dist que cil ki sont monté a ce k’il n’esperoient, conçoivent sovent mauvaises esperances. Terrences dist, nous enpirons tout quant nous avons le loisir. Staces dist, nule cure n’est si griés a l’home comme longue esperance. Mais or se taist a parler li contes de signorie jusc’a tant k’il en dira plus apertement, car il veet premiers dire dou tierç bien de fortune, c’est glore.
\chapterclose


\chapteropen
\chapter[{.II.CXX. De glore}]{\textsc{.II.CXX.} De glore}\phantomsection
\label{tresor\_2-120}

\chaptercont
\noindent Gloire est la bone renomee, ki cort par maintes terres, d’aucun home, de grant afere, ou de savoir bien son art. Ceste renomee destre chascuns, pour ce que sans lui ne seroit pas congneue. Selonc ce que Orasces dit, vertus ceree ne se devise pas de mauvaistié reposte, et cil ki traitent de grans choses tesmoignent que glore done au preudome une seconde vie ; c’est a dire que aprés sa mort la renomee ki maint de ses bones oevres fait sambler k’il soit encore en vie.\par
Orasces dist, la glore deffent que cil ne soit mors qui est dignes de loenge. Mais contre glore dist il meismes Orasces, quant tu seras bien congneus a la piace d’Agripe et en la voie d’Apius, encore te covenra il aler la ou alerent Numma et Ancus ; c’est a dire quant ta renomee sera alee ça et la, encore te covenra il aler aillours, c’est a la mort. Boesces dist, mort despite toute glore, et envolepe les haus et les bas, et igaillist tous. Mais nous querons glore si desmesureement, que nous volons mieus sambler bons que estre le, et mieus estre mauvais que sambler le.\par
Por çou dist Orasces, fausse honours delite, et renomee mençoigniere espavente. Li fruis de glore est souvens orgoils, de quoi Boesces dit, gloire et maint millier de homes n’est tout fors uns enflemens d’oreilles. Mais en glore n’a point de fruit, s’il n’a autre bien avec, selonc ce que Juvenaus dist, conbien que gloire soit grans, ele ne vaut rien se ele est seule.\par
Et ce dist Tuilles, ki vieut avoir gloire, face k’il soit teux comme il wet resambler ; car cil ki quide gaaignier glore par fausse demoustrance ou par faintes paroles ou par samblance de sa chiere est vilainement deceus, pour ce que vraie gloire a racine et fermeté ; mais la fainte chiet tost comme la flor, por ce que nule chose fainte ne puet durer longhement. Li mestres dit, ou monde n’a si fause chose come vois, mais mençoigne a cours piés.
\chapterclose


\chapteropen
\chapter[{.II.CXXI. De la comparison entre les biens dou cors et de fortune}]{\textsc{.II.CXXI.} De la comparison entre les biens dou cors et de fortune}\phantomsection
\label{tresor\_2-121}

\chaptercont
\noindent Vous avés bien oii en ceste partie çou que li contes a devisé des biens de fortune, et en arieres avoit il devisé les biens dou cors, et li un et li autre sont proufitable a la vie de l’home. Mais si comme il est dit autrefois, li uns est plus profitables que li autres ; car se tu wes acompaignier les biens dou cors a ceaus de fortune, je di ke santé est millour que richesse ; et d’autre part di je que richesce vaut mieus que force de cors ; et se tu vieus acompaignier les biens dou cors entr’aus, je di que bonne santés est millours que grandours, et force ke ysneleté ; et se tu vieus comparer les biens de fortune entr’aus, je di que glore vaut mieus que richesce, et rente de cités mieus que de chans.
\chapterclose


\chapteropen
\chapter[{.II.CXXII. De la querele entre honesté et proufitableté}]{\textsc{.II.CXXII.} De la querele entre honesté et proufitableté}\phantomsection
\label{tresor\_2-122}

\chaptercont
\noindent Après ce que li mestres a mostré apertement li quel bien sont honeste et li quel proufitable, et li quex sont plus honeste et plus profitable li \textsc{.i.} que li autre, encore remaint la quinte question entre honeste et proufitable, a quoi on se doit plus tenir a l’un qu’a l’autre. Car se aquerre est profitable et doner est honeste, il avient sovent que nostre corage est en doute lequel il fera. De quoi dist Juvenaus, force et licence sont a plusours maufere ; més tant comme le ciel se devise de la terre et li feus de l’euue, tant se devise proufis de droiture ; car toute la force des signours dechiet maintenant k’il commencent a perdre justice, et vertu et signourie ne s’entracordent gaires bien. Mais en ceste maniere dist Tuilles que ces \textsc{.iii.} choses, bien et honestés et proufit, sont si entremellet que tout çou ki est bon est tenut a proufitable, et tout ce ki est honeste est tenu por bon. Et de ce s’ensiut il que toute chose honeste est proufitable.\par
Tien donc a certes et ne doute pas que honestés est si profitable que nule n’est proufitable s’ere n’est honeste, ne il n’a nule difference en la generalité de ces \textsc{.ii.} choses, mais en lor proprietés. Raison coment : cist hom si est animai en generalité, non pas en cognoissance ; car a estre animaus n’a mestier autre chose se tant non qu’il soit une substance mortel ki ait ame et sentement ; mais a çou k’il soit home covient k’il cognoisse raison et soit morteus ; donques est la difference en la proprieté solement.\par
Tout autresi honeste et proufitable sont une chose en generalité ; mais a çou que une chose soit profitable, covient il k’ele ait fruit, et a ce k’ele soit honeste covient il k’ele nous atraie par sa dignité. C’est donc une meisme chose, por quoi il s’ensiut que nule chose n’est proufitable ki se descorde de vertu. Por ce port il manifestement k’il n’a point de contraire entre profit et honesté ; mais por ce que les gens quident k’il soit proufit a user les temporaus choses et k’il ne loist a fere contre honesté, por ce il propose la question entre profit et honesté.\par
Tuilles dist, mais il samble a l’home que proufitable chose soit d’acroistre son preu dou damage d’un autre, et que li uns toille a l’autre. Mais c’est plus contre droit de nature que poverté ou dolour ou ke mort, car il oste tot avant la comune vie des homes ; car se pour gaaignier nous avons volenté de despoillier et efforcier autrui, il covient que la compaignie des homes, ki est selonc nature, soit departie. Raison coment : se aucuns membres quidoit mieus valoir s’il atraisist a lui la santé del prochain membre, il coviendroit que tous li cors afoibloiast et morust. Autresi bien est il en humaine compaignie ; car autresi comme nature otroie que chascuns aquiere ce que mestier li est, pour soi mieus que pour autrui, autresi n’otroie pas nature que nous acroissons nos richesses por despoillier les autres.\par
Et cil ki grieve autrui por aquerre aucun preu, ou il ne quide rien faire contre nature, ou il li est avis que l’om se doit garder de povreté plus que de fere tort a aucun. Mais s’il ne quide rien faire contre nature, il n’est pas humain ; et s’il li est avis que tort soit mal, mais croit que mors ou poverté soit encor pire, il est deceus, que plus grief est li visces d’un cuer, c est tort, que cil dou cors ou de fortune, c’est mort u povreté.\par
Et se aucuns me demandoit, se uns sages muert de fain, ne doit il tolir sa viande a \textsc{.i.} autre ke rien ne vaille ? je di menil, por ce que vie ne m’est plus proufitable que cele volenté, porquoi je me garde de faire tort a autri pour mon preu. Quant on pert la vie, li cors est corrompus de mort ; mais se je laisse cele volenté, je cherai en vice de corage. Et si comme li visce de corage est plus griés que cil du cors, autresi li bien dou corage est millour que celi du cors, car mieus vaut vertus que vie.\par
Il n’afiert pas a bon home mentir et maldire ne decevoir por son gaaing. Tu ne dois donc tant prisier nule chose, ne tant covoitier ton proufit, que tu en perdes non de bon home, car tel proufit ne te puet reporter tant comme il te ravist s’il te tolt non de bon home et amenuise en toi foi et justice. Porquoi donc voient li home le gaaing des choses, et ne voient pas la trés grant paine de loi et de laidece ? Laissons donc ceste pensee, et gardons ce ke nous volons ensivre, s il est honeste ou se nous faisons mal a escient ; car solement dou penser est contre vertu, ja soit ce que l’om ne viegne jusc’au fet. La volenté de maufere, seulement por la pensee suefre tel paine comme s’il eust le mal acompli. Et el malpenser ne doit nus croire que sa pensee soit ceree solement, et ki peust celer a Dieu ? si ne devroit om meffere, ne par avarice ne par covoitise ne par autre chose ki soit desavenant.\par
Tuilles dist, choses ki sont si corrompues de visces ne puent estre proufitables. Et se uns sages hom avoit \textsc{.i.} anel de tel force k’il ne peust estre veus tant comme il le portast, ja pour çou ne seroit il mains que s’il ne l’eust. Li bon home doivent querre choses honestes non pas repostes ; car prodom n’oseroit chose voloir qu’il n’osast preecier. Li mestres dist, mais se tu t’atiens de mal fere, que les gens ne le sacent, tu n’aimes pas bonté, mais tu criens la paine. Et en ce ensius tu la nature que Orasces dist des bestes, li leus a paour de la fosse et li esperviers des rois et li escoufles dou lamechon. Autresi li mauvais laissent a pechier por paour de la paine, et li bon por amour de vertu.\par
Et por ce k’il apert par ce qui est devant dit, ke solement honeste chose est proufitable, se aucuns proufis avenist et tu voies que aucune laidece i est jointe, je ne di pas que celui proufit tu laisses ; mais tu dois entendre que la u laidece est ne puet avoir point d’avantage. Mais se nous volons jugier veraiement, toutefois que laidece nos moustre samblance de preu, sieut ele estre blasmee a la fin de la chose ; car nous veons aucunefois que d’une chose honeste, ki ne samble proufitable, avient a la fin tel preu ke l’en n’espere.\par
Raison coment : Damon et Ficus furent si bons amis que quant Danis li tyrans ot jugié l’un a mort, cil demande \textsc{.i.} po de terme k’il peust aler a ordener son testament et ses choses ; et li autres fu engagés dedens, ce fu par covenance que se cil ne venoit cil morroit. Et quant cil fu revenus au jor, li tirane s’esmervilla de lor amor, si lor requist k’il le receussent a estre li tiers de lor amistié.\par
Or gardes coment il fu proufitable chose que cil remest por son ami, et que li autres revint por son ami, ja soit ce que l’un et l’autre semblast au comencement perilleuse chose, ensi avient d’onestet proufitable fin, dont on ne se donne garde. Et de laidece avient fin mauvaise et perilleuse. Et por ce quant une chose ki porte samblance de proufit est acomparisie a cele ki samble honeste, certes la samblance dou proufit doit conchiier et cele d’onnesté doit valoir, pour ce que honestés est vertus del cuer et de l’ame, ki tozjors maint avec toi, mais bien de fortune sont vain et decheable, sans nule fermeté.\par
Por ce dist li Apostles, trés bone chose est a establir le cuer. Augustins dist, la trés millour chose est celi ki fait l’ame trés bone, c’est vertus. Jhesu li fiz Syraac dist, se tu es riches tu ne seras sans pechié. Senekes dist, griés chose est non estre corrompus par la multitude de richesse. Li mestres dist, mais les gens de nostre tans n’ont nule cure de bonté, mais que lor choses soient bones. Seneques dit, mais li hom n’a nule plus vil chose que soi.\par
Augustins dist, tu vieus avoir bonnes choses et si ne vieus estre bons, ne ne vieus male feme ne mauvais fil, ne malvaise cote ne mauvaise chause, et si vieus avoir male vie : ke t’a donc ta vie forfait, que entre tous biens tu vieus estre mauvaise ? mais je te pri que tu aymes plus ta vie que ta chause. Senekes dist, il ne puet chaloir combien de gent te saluent, ne de grant lit, ne de grande et precieuse viande, mais que tu soies bons ; car es temporaus choses n’a point de bien se ce non que l’en use a droit et sans pechié, et ce apertient a vertu.\par
Seneques dist, fols n’a mestier de nule chose, car il n’en set user nules. Jhesu li filz Siraac dist, richesce est bone a ki n’a mauvaise entention, ausi comme le sanc est bon el cors de l’ome s’il n’est corrompus de maladie. Salemons dist, fols desire tousjours ce ki torne a son damage. Seneques dist, il n’est pas biens de vivre, mais de bien vivre. Tuilles dist, je croi que ce soit bon sans plus ki est droit et honeste et avec vertu, car vertus est li bien de nous proprement, mais li bien de fortune sont estrange.\par
Tuilles dist, toutes autres choses sont decheables, mais vertus est fichie a parfondes rachines ; di donc que ce ki est posé dedens toi soit tien, et quide que humaines cheoites soient maindres que vertus. Seneques dist, il n’est pas tien ce ke fortune te baille : certes il doit perir. Il n’est si fole chose comme de loer en toi les autri choses, ne nule si niche sorquidance comme remirer en toi ce que maintenant s’en puet aler aillours ; car frain d’or ne fait millour le cheval.\par
Abacuc dist, mar i est a celui ki amasse ce ki n’est pas sien. Seneques dist, ce desire et ce adresce tes pensers que tu soies apaiés de toi et de ce ki de toi naist, car quant li hom pourchace des choses dehors, maintenant commence a estre sousmis a fortune. Seneques dist, il est maindres que sers ki crient les sers ; car li sages se tient apaiés non pas de vivre mais de bien vivre. Boesces dist, O estroites et chetives richeces quant li plusour ne les puent avoir trestotes, et ne vont as uns sans povreté des autres. Jhesu li fiz Syraac dist, li sordemens de bon corage est a non deliter soi es vaines choses.\par
Gregorius dist, il n’a pas tant de delit es visces comme il a es vertus. Boesces dist, li honours des vertus ne fu pas aquise por les dignités, mais li honors de dignité avint por les vertus, car vertus a sa propre dignité.\par
Et se aucuns me demandast por quoi Dieus volt que ces maus et biens temporeus fussent communs et as bons et as mauvais, je diroie ce que Augustins dist, que Dieus le volt pour ce que li bien que li mauvés ont sovent ne fussent trop desirés, et que li mal ki a avienent as bons ne fussent trop despit. Pour ce est il grandisme sens de prisier poi les biens et les maus, ki sont commun as bons et as mauvais, et aquerre les biens ki proprement sont des bons, et eschiver les maus ki proprement sont des mauvés Augustins dist, por çou donne Dieus biauté as mauvés, que li bon ne quident que ce soit grans biens.\par
Mais ci se taist li contes a parler des biens de l’ame et des biens dou cors, et de ciaus de fortune et de la comparison de l’un et de l’autre, de quoi il a longuement parlé, et si tornera a autre chose.
\chapterclose


\chapteropen
\chapter[{.II.CXXIII. De la vertu contemplative}]{\textsc{.II.CXXIII.} De la vertu contemplative}\phantomsection
\label{tresor\_2-123}

\chaptercont
\noindent Li contes devise ça en ariere, la u il comença a dire de vertu, et premierement que prudence, justice, force, et atemprance sont vertu actives por adrecier les meurs des homes et por ovrer ce q’a honeste vie apertient : et de ce a il dit assés diligement. Et la meismes dit il qu’il sont \textsc{.iii.} vertus contemplatives, c’est foi, esperance, et carité. Mais plus n’en dist en celi partie ; por çou est il bien raisons k’il en die aucune chose.\par
Une vie est active, l’autre et contemplative. La vie active est la innocence des bones oevres, selonc ce que li mestres dist jusques ci el conte des \textsc{.iiii.} vertus ; la contemplative est li penseement des celestiaus choses, cele est acointe a pluseurs, c’est a petit. La vie active use bien les mondaines choses, la contemplative refuse le monde et se delite en Deu solement. Car ki bien s’esprueve en la vie active puet bien monter puis a la contemplative ; mais cil ki encore desire la temporel glore et la charnele covoitise est deveé de la contemplative, por quoi il li estuet demorer en la active tant qu’il soit purgiés : la doit il oster toz visces par usages de bones oevres, si k’il set l’entention et la pensee pure et nete quant il venra a contempler Dieu.\par
Car tot ensi com celui ki est en l’active vie est ostés de tous terriens desiriers, autresi cil ki vit en contemplation se retrait de toutes oevres actives ; et por ce vois tu que la vie active sormonte a la mondaine, et la contemplative sormonte a l’active. Et si comme li aigle fiche tozjors ses oils contre les rais dou soleil et ne les torne se por son past non, tot autresi li saint home se tornent aucunefois a la vie active por ce qu’ele est besoignable as homes.\par
Mais ces \textsc{.ii.} vies sont diverses entr’eus ; car se li hom desvoie de la contemplative aucunefois et puis i wet revenir et renoveler sa droite entention, il est bien receus ; mais s’il se dessoivre de la vie active, maintenant il est sospris en desvoiement des visces. Li doi oeil de l’omme segnefient ces \textsc{.ii.} vies ; et por ce quant Dieus commande que le destre oeil ki escandelizast fust ostés et getés hors, dist il de la vie contemplative s’ere courust en erreur, por ce que mieus vaut a oster l’oeil de la contemplative et garder celi de l’active, si k’il aille por ses oevres a la vie pardurable, que aler au feu d’infier.\par
Par eurreur de la contemplative, Dieus abaisse sovent maint home es charneus choses por sa grasce k’il enhauce en la grandor de la contemplation, et mains autres oste il de contemplation par droite sentence et les abandonne as terrienes choses.
\chapterclose


\chapteropen
\chapter[{.II.CXXIIII. Des sains homes}]{\textsc{.II.CXXIIII.} Des sains homes}\phantomsection
\label{tresor\_2-124}

\chaptercont
\noindent Li saint home qui cest monde refusent laissent le siecle en tel maniere k’il ne se delitent a vivre se en Deu non ; et tant com il se desoivrent de la conversation du siecle, tant contemplent il a la presence de Deu a la veue de la pensee dedens. Mais les praves oevres as mauvais sont si manifestees ke cil qui desirent le pais desoivrent et fuient lor meurs et lor compaignies.\par
Aucun se departent des mauvés por ce k’il ne soient envolepé de lor mauvaisté, mais plusor sont que ja soit ce qu’il ne se puissent partir de lor compaignie corporaument, toutesvoies s’en departent il par spirituele entention ; et se la compaignie est commune, les coers et les oevres sont divers ; et ja soit ce que Dieus deffent la vie des sains en mi les carnés choses, a paines sera aucuns ki entre les deliz du siecle parmaigne sans vice. Por ce est il bien que l’om se parte corporaument dou monde, et mieus vaut a desevrer la volenté ; mais cil ki en depart cors et volenté est tous complis.
\chapterclose


\chapteropen
\chapter[{.II.CXXV. De commandement}]{\textsc{.II.CXXV.} De commandement}\phantomsection
\label{tresor\_2-125}

\chaptercont
\noindent Autre commandement sont doné as bons, ki demeurent a la commune vie du siecle, et autre sont donné a ciaus ki du tout le refusent, car a ciaus ki sont doné au siecle est commandé generaument k’il facent bien en toutes lor choses. Mais a ciaus ki le refusent est commandé k’il abandonnent toutes lor choses. Et encore font il plus que ce ; car a ce k’il soient parfet ne soufist pas k’il renoie ses choses, mais il li covient renoier soi meismes.\par
Et certes renoier soi n’est pas autre chose que refuser ses volentés, en tel maniere que cil ki estoit superbes deviegne humles ; et cil ki estoit plains d’ire deviegne mansuetes, car ki refuse ces choses et ne refuse ses volentés, il n’est pas disciples Deu. Par ce dist il, ki vieut venir aprés moi renoit soi meismes. Mais de ce taist ore li contes, et torne a dire des \textsc{.iii.} vertus contemplatives, et premierement de foi.
\chapterclose


\chapteropen
\chapter[{.II.CXXVI. De foi}]{\textsc{.II.CXXVI.} De foi}\phantomsection
\label{tresor\_2-126}

\chaptercont
\noindent Autresi homs ne puet venir beatitude se par foi non. Et cil est droitement beates, qui croit droitement et garde la droite foi. Et lors est Dieus bien glorifiiés quant il est creus veraiement, et lors puet il bien estre requie et proiés. Sans foi ne puet nus hom plaire a Dieu, car tout ce ki n’est par foi est pechiés ; si comme li hom ki a arbitre et delivre signorie de soi par sa volenté se depart de Dieu ; tot autresi retorne il par droite creance de son coer.\par
Mais Dieus regarde la foi enmi le cuer, ou cil ne se puet escuser ki moustrent samblance de verité, et ont ou cuer malice de grant erreur. Et si comme la fois ki est en la bouche, et n’est creue dedens le cuer, ne proufite de rien, tout autresi la fois ki est ou cuer ne vaut rien s’ele n’est mostree par la bouche ; car cele fois est wide ki est sans oevre, por ce sont plusour home ki sont crestiien solement par foi, mais en l’oevre se descordent mout de la crestiiene verité.
\chapterclose


\chapteropen
\chapter[{.II.CXXVII. De charité}]{\textsc{.II.CXXVII.} De charité}\phantomsection
\label{tresor\_2-127}

\chaptercont
\noindent Ja soit ce que aucuns samble estre bons par foi et par oevre, je di k’il n’ont point de vertu s’il sont wit de charité et d’amour as homes. Car ce dist li apostles, se je bailloie mon cors a ardoir, ne me vaudroit noient se je n’avoie charité. Sans amor de charité ne puet nus venir a beatitude, ja soit ce qu’il ait droite creance, por ce que la vertu de la charité est si trés grans que nus guerredons ne se puet comparer ; ele est dame et roine de toutes vertus et liiens de la perfection, car ele lie les autres vertus.\par
Charités est amer Dieu et son proisme, et l’amour de Dieu est samblables a la mort. Salemons dit, amours est autresi fors comme la mors, car si comme la mors desoivre l’ame dou cors, tout autresi l’amours Dieu depart l’ame du monde et de la charnele amour. Cil n’aime pas Dieu ki despit ses commandemens, autresi n’aime cil le roi ki het sa loi. Cil garde charité ki ayme son proisme. Jhesucris est Dieus et hom, donc cil ki het home n’aime pas dou tout Jhesucrist. Mais la cognoissance des bons et de non haïr les personnes mais lor coupes.
\chapterclose


\chapteropen
\chapter[{.II.CXXVIII. D’esperance}]{\textsc{.II.CXXVIII.} D’esperance}\phantomsection
\label{tresor\_2-128}

\chaptercont
\noindent Cil ki ne finent de maufere, por noient ont esperance en la pitié Dieu et en sa misericorde requerre. Mais s’il cessaissent de males oevres, il le poroient bien priier. Et lors doit on avoir esperance en Dieu, k’il li pardoinst ses meffais, mais on doit mout douter que par esperance que Dieus promet de son pardonement il ne soit perseverancié en pechier. Autresi ne se doit il desesperer, por ce que li torment sont establi selonc les pechiés. Mais il doit eschiver l’un peril et l’autre, en tel maniere k’il se gart de malfaire tant comme il pora, et k’il ait esperance en la misericorde de Dieu.
\chapterclose


\chapteropen
\chapter[{.II.CXXVIIII. Des justes homes}]{\textsc{.II.CXXVIIII.} Des justes homes}\phantomsection
\label{tresor\_2-129}

\chaptercont
\noindent Li juste home sont tozjours en paour et en esperance, car une fois s’eshaucent par esperance de la perpetuale leece, une autre fois doutent pour paour dou feu de gehenne.
\chapterclose


\chapteropen
\chapter[{.II.CXXX. Des pechiés}]{\textsc{.II.CXXX.} Des pechiés}\phantomsection
\label{tresor\_2-130}

\chaptercont
\noindent En ariere vous est moustré k’est vertus active et contemplative : mais de la contemplative briefment, pour ce k’ele requiert grans sollempnités. Or est il covenable a dire \textsc{.i.} poi des pechiés et des visces, car, se l’on connoist lor naissance et lor norrissement, il s’en puet mieus prendre garde. Por quoi je di que pechiés n’est autre chose que trespasser la divine loi et non obeir as celestiaus mandemens ; car pechiés ne seroit se li deveemens ne fust, se pechiés ne fust ne seroit vertus, ne malices ne porroit estre se aucunes semences de lui ne fussent, ne nous n’oons pas les celestiaus mandemens par les oreilles dou cors, mais l’oppinion de bien et de mal vint en nous en tel maniere que nous savons naturelement que nous devons faire bien et eschivre mal. Donc di je bien que les commandemens de Dieu ne sont pas escrit en nous par letre d’autre, mais il est fichié dedens nos cuers par divin esperit.\par
Pour ce puet chascuns entendre que la oppinions de l’home vient de divine loi ; et por ce avient il que maintenant que l’en pense de malfaire suefre on la paine et le torment de sa conscience ; car toutes choses puet on fuir, mais son cuer non, pour ce que nus hom ne puet desevrer soi de soi meismes : ou k’il aille, li malisces de sa conscience ne le deguerpist pas. Et ja soit ce que aucuns ki maufet eschape dou jugement as homes, il n’en eschapera dou jugement de sa conscience ; car a soi ne puet nus hom celer çou k’il çoile as autres, il set bien k’il fet mal. Ensi chiet sor li double sentence, une en cest siecle par sa consience, et l’autre en celi dou perpetuel torment.\par
Pour çou dist Ysidorus que l’entention de l’oevre est oils et lumiere de l’home ; car se l’entention de l’oevre est bonne, certes l’oevre sera bonne, mais l’oevre de malvaise entention ne puet estre se malvaise non, ja soit ce que ele se resamble a estre bone, por ce que chascuns est jugiés bons ou mauvais, selonc sa entention. Cil ki font bones oevres ou malvais entendement sont avuglés par cele oevre dont il pooient estre enluminé. Et chascuns face donc le bien par bonne entention, car autrement seroit il perdus. Mais puis ke li contes nous a dit comment li hom se doit garder que sa oppinion ne soit corrompue, et k’il ait bien bonne entention, si vieut il dire des pechiés ki en l’oevre sont.
\chapterclose


\chapteropen
\chapter[{.II.CXXXI. Des criminaus pechiés}]{\textsc{.II.CXXXI.} Des criminaus pechiés}\phantomsection
\label{tresor\_2-131}

\chaptercont
\noindent Les criminaus pechiés sont \textsc{.vii.}, superbe, envie, ire, luxure, covoitise, mescreance, et avarice. Encor sont maint autre pechié ki tout naissent et muevent de ces \textsc{.vii.} que je vous ai només ; mais de tous pecchiés est superbe la mere et la racine ki toz les engendre.\par
Et nanporquant chascuns de ces \textsc{.vii.} engendre autres pecchiés ; car de superbe vient orghieus, despit, vantance, ypocrisie, contention, descorde, perdurabletés, et contumace.\par
D’envie naist traine, decevance, leece dou mal de son proisme et tristece de son bien, maudire et abaissier le bien.\par
De ire muet tençons, gros cuers, complainte, cris, desdains, blasme, tort, non soufrance, cruautés, folie, malignités, et murtre.\par
De luxure vient avugletés de cuer, non fermeté, amour de soi meismes, traine de Deu, volenté de cest siecle, et despit de l’autre, fornication, avoutire, et pechiés contre nature.\par
De covoitise naist chetive leece, laidece, mout parler, vain parler, forsenerie, yvrece, prodigalités, desmesurance, deshonestetés, et desvergoigne.\par
De mescreance naist malisces, petit corages, desesperance, perrece, descognoissance, non porveance, soutie, et delit dou mal. De avarice vient traisons, fausetés, forjurer, force, dur cuer, symonie, usure, larrecin, mençogne, ravine, non justice, et decevance.\par
Cist pechiés et maint autre sont engendré par superbe generaument ; et si comme les vertus mantienent humaine compaignie en bone pais et en bone amour et amainent l’ame a sauveté, tout autresi li pecchieg desrompent la compaignie des homes, et l’ame conduisent anfier. Car orgoils engendre envie, et envie engendre mençoigne, et mençoigne engendre decevance, et decevance engendre ire, et ire engendre malevoeillance, et malevoellance engendre ennemistié, et ennemistiés engendre bataille, et bataille desront la loi et gaste la cité.
\chapterclose


\chapteropen
\chapter[{.II.CXXXII. C’est le derrain ensegnement de cest livre}]{\textsc{.II.CXXXII.} C’est le derrain ensegnement de cest livre}\phantomsection
\label{tresor\_2-132}

\chaptercont
\noindent En cest livre nos a moustré li mestres les ensegnemens de vertu et des vices, les uns por ovrer, les autres pour eschiver ; car ce est l’achoisons por quoi on doit savoir bien et mal. Et ja soit ce que li livres parole plus longuement des vertus que des visces, nanporquant la u li bien sont comandé a faire doit chascuns entendre que li maus soit devees a fere ; selonc ce qu’Aristotles dist, uns meismes ensegnemens est de \textsc{.ii.} contraires choses.\par
Et certes cil ki wet atorner sa vie au proufit de lui et des autres, Senekes li commande k’il use la forme des \textsc{.iiii.} vertus par lor droit mi et amesureement, selonc la diversité du lieu et dou tens et des personnes et des achoisons. Por ce doit on ensivre les traces as millours et faire ce k’il font ; car si comme la cire reçoit la figure dou saiel, tout autresi la moralités des homes est fermeté por exampleres. Gardent soi tout home de malfaire, et soit tot asseur que quant hons est entechiés de male renomee une fois, il li covient mout d’euue a bien laver.\par
Mais ci se taist li contes a parler de ceste matere, car il wet encomencier a parler la tierce partie de son livre, pour ensegnier la science de bone parleure, selonc ce k’il dist en son prologhe devant.
\chapterclose

\chapterclose


\chapteropen
\part[{Tiers livre}]{Tiers livre}\phantomsection
\label{tresor\_3}\renewcommand{\leftmark}{Tiers livre}


\chaptercont

\chapteropen
\chapter[{.III.I. Ci commence li livres de bone parleure}]{\textsc{.III.I.} Ci commence li livres de bone parleure}\phantomsection
\label{tresor\_3-1}

\chaptercont
\noindent Après ce ke mestre Brunés Latins ot complie la seconde partie de son livre, en quoi il demoustre assés bonement quex hom doit estre en moralités et comment il doit vivre honestement et governer soi et sa mesnie et ses choses selonc la science d’etike et de iconomike, dont il fist mention la u il devisa les membres de philosophie ; et k’il ot dit quel chose derront la loi et gaste la cité, il li fu avis que tot çou estoit une oevre copee s’il ne desist de la tierce science, c’est politike, ki ensegne coment on doit governer la cité. Car cités n’est autre chose ke unes gens assamblees por vivre a une loi et a \textsc{.i.} governeour.\par
Et Tuilles dist que la plus haute science de cité governer si est rectorique, c’est a dire la science du parler ; car se parleure ne fust cités ne seroit, ne nus establissemens de justice ne de humaine compaignie. Et ja soit ce que parleure soit donee a tous homes, Catons dit que sapience est donee a poi. Et por ce di je que parleures sont donnees de \textsc{.iiii.} manieres : car li \textsc{.i.} sont garni de grant sens et de bone parleure, et c’est la flour dou monde li autre cont wit de bonne parleure et de sens et c’est la trés grant mesceance ; et li autre sont wit de sens mais il sont trop bien parlant, et c’est trés grant periz ; li autre sont plain de sens, mais il se taisent por la povreté de lor parleure et ce requiert aide. Et pour ceste diversité furent li sage en content de ceste science, s’ele est par nature ou s’ele est par art.\par
Et a la verité dire, devant ce que la tour Babel fust faite tout home avoient une meisme parleure naturelement, c’est ebreu ; mais puis que la diversités des langues vint entre les homes, sor les autres en furent \textsc{.iii.} sacrees, ebrieu, grieu, latin. Et nous veons par nature que ciaus ki abitent en orient parolent en la gorge si comme li ebreu font ; li autre ki sont ou milieu de la terre parolent ou palais si comme font li grezois ; et cil ki abitent es parties d’occident parolent es dens si comme font les ytaliiens.\par
Et ja soit ce que ceste science ne soit en parleure seulement, mais en bien parler, neporquant Platons dit k’ele est par nature non pas par art, a ce ke l’on trueve mains bons parliers naturelement, sans nul ensegnement. Aristotles dist k’ele est art, mais mauvaise, por ce que por parleure estoient avenu as gens plus de mal que de bien.\par
Tulle s’accorde bien que la seule parleure est par nature ; mais en la bonne parleure covient \textsc{.iii.} choses, nature, us, art, car us et art sont plain de grant ensegnement, et ensegnement n’est autre chose que sapience.\par
Et sapience est a comprendre les choses selonc ce qu’eles sont, pour ce est ele apelee amoieneresse des choses, car ele les porvoit toutes devant et lor met certaine fin et certaine mesure. Et la u sapience est jointe a parleure, ki dira k’il en puisse naistre se biens non ?\par
Tuilles dit que au comencement que li home vivoient a loi de bestes, sans propres maisons et sans cognoissance de Dieu, parmi les bois et parmi les repostailles champestres, si ke nus n’i regardoit mariage, nus ne connissoist peres ne fiz. Lors fu uns sages hom bien parlans, ki tant consilla les autres et tant lor moustra la grandour de l’ame et la dignité de la raison et de la discretion, qu’il les retraist de ces sauvegines, et les aombra a abiter en \textsc{.i.} lieu et a garder raison et justice. Et ensi por la bonne parleure ki en lui estoit acompaignie o sens fu cesti ausi comme \textsc{.i.} secons Dieus, ki estora le monde par l’ordene de l’umaine compaignie.\par
Et si nous raconte l’istoire que Amfions, ki fist la cité d’Athenes, i faisoit venir les pieres et le merien a la douçor de son chant, c’est a dire par ses bonnes paroles. Il retraist les homes des sauvages roches ou il abitoient, et les amena a la comune habitation de cele cité.\par
Et d’autre part s’acorde bien Tuilles a ce k’Aristotles dist de parleure, qu’ele est mauvaise art ; mais ceste parleure sans sapience, quant uns hom a bone langue dehors et il n’a point de conseil dedens, sa parleure est fierement perilleuse a la cité et as amis.\par
Or est il dont prové que la science de rectorique n’est pas dou tout aquise par nature ou par us mais par ensegnement et par art, por quoi je di que chascuns doit estudiier son engien a savoir le.\par
Car Tuilles dist que li hom, ki en mout de choses est maindres et plus foibles des autres animaus, les devance tous de ce k’il puet parler ; dont part il manifestement que celui aquiert trés noble chose qui de ce devancist les homes de quoi li hom sormonte as bestes.\par
Neporquant dist li proverbes que noureture passe nature ; car selonc ce que nous trovons en la premiere et en la seconde partie de cest livre, l’ame de tous homes est bone naturelement, mais ele mue sa nature por la mauvaisté dou cors en quoi ele maint enclose, autresi com le vin ki enpire por la malvaistié du vaissel. Et quant li cors est de bone nature, il conorte s’ame et aide a sa bonté, et lors li valent ars et us. Car art li ensegne le commandement ki a ce covient, et us le fait prest et apert et esmolut a l’oevre.\par
Et pour ce vieut li mestres ramentevoir a son ami le riule et l’ensegnement de l’art de rectorique ki mout li aideront a la soutillece ki est en lui par sa bone nature. Mais tot avant dira il qui est rectorique et desous qui ele est, et puis de son office et de sa fin et de sa matire et de ses pars ; car ki bien set ce, il entent mieus le compliement de cestui art.
\chapterclose


\chapteropen
\chapter[{.III.II. De rectoriqe}]{\textsc{.III.II.} De rectoriqe}\phantomsection
\label{tresor\_3-2}

\chaptercont
\noindent Rectorique est une science ki nous ensegne bien plainement et parfitement dire es choses communes et es privees, et toute sa entention est a dire paroles en tel maniere que l’en face croire ses dis a ceaus ki les oient.\par
Et sachiés que rectorique est desoz la science de cité governer, selonc ce q’Aristotles dist en son livre, ki est translaté en romans ça en ariere, autresi com art de frains faire et de selles est sous l’art de chevalerie.\par
Li office de cestui art, selonc ce que Tuilles dist, est de parler penseement por faire croire ce qu’il dist. Entre l’office et la fin a tel difference que en l’office consire le parleurs çou ki covient a sa fin, c’est a dire k’il parolt en tel maniere qu’il soit creus ; et en la fin consire il ce ki covient a son office, c’est a faire croire par sa parleure.\par
Raison coment : li office de fisicien est a faire cures et medecines penseement por saner, et sa fins est saner par ses medecines et briefment. L’office de rectorique est a parler penseement selonc les ensegnemens de l’art, et la fins est cele chose pour quoi il parole.\par
La matire de retorique est ce de quoi li parliers dist, ausi comme les maladies sont matires dou fisicien. Dont Gorgias dist que toutes choses de quoi covient dire sont matire de cestui art.\par
Ermagoras dist que ceste matire est es causes et es questions, et disoit que causes sont ce de qui li parleours sont en contens d’aucunes certaines gens ou d’autres choses certaines, et de ce ne disoit il mie mal ; mais il disoit que questions est çou sour quoi li parleour sont en content sans nomer certaines gens en autre chose ki apartiegne as besoignes certaines, si comme est ore de la grandour dou soleil et de la fourme dou firmament, de ce disoit il trop mal, car teus choses ne covient pas a governeours de cités, ains sont des philosophes ki s’estudent en parfonde clergie.\par
Por ce sont il deceu, cil ki quident que raconter fables ou ancienes istores ou quanque on puet dire soient matire de rectorique. Mais çou que l’om dist de bouche ou que l’om mande par ses letres apenseement por faire croire, ou par contençon de loer ou de blasmer ou de conseil avoir sor aucune besoigne ou de choses qui requierent jugement, tout çou est de la maniere de rectorique.\par
Mais tout ce que l’on ne dist artificielement, c’est a dire par nobles paroles, griés et replaines de bonnes sentences, ou par aucunes choses davant dites, est hors de ceste science et loins de ses riules. Pour ce dist Aristotles que la matire de cestui art est sour \textsc{.iii.} choses seulement, c’est demoustrement, conseil, et jugement.\par
A ce meisme s’acorde bien Tulles, et dist que demoustrement est quant li parleours loënt et blasment home generaument ou partiement. Raison comment : je loe mout biauté de femmes dist li uns, et je les blasme dist li autres, et c’est dit generaument ; més partiement dist li uns, Julles Cesar fu mout preus et mout vaillans, fet li autres, non fu, mais traitres et desloiaus ; et ceste question n’a pas lieu se es choses passees et presentes non, car de ce ki est a venir ne puet nus estre blasmés ne prisiés.\par
Conseil est quant li parleour conseillent sour une chose ki est proposee devant aus generalment ou partiement, por mostrer liquel soit voir, ou non profitable, et quel non, en ceste maniere : dist li uns, proufitable chose est garder pais entre crestiens ; non est, fet li autres. Mais partiement dist li uns, proufitable chose est la pais entre le roi de France et celui d’Engleterre ; non est, fet li autres ; et ceste questions n’a pas lieu se es choses futures non. Et quant chascuns a doné son conseil, on s’en tient a celui ki moustre plus ferme raison et plus creable.\par
Jugement est en acuser et en deffendre ou en demander ou en refuser, pour moustrer de l’ome ou d’autre chose generaument ou partiement k’ele soit juste u non. Raison coment : je di generaument, fet li uns, que tout larrons doivent estre pendus ; fait li autres, non doivent pas. Ou dist li uns, cil ki governent bien la cité doivent avoir grant guerredon ; fait li autres, non doivent. Mais partiement dist li uns, je di que on doit pendre Gouliam, pour ce qu’il est lerres atains ; non est fet li autres. Ou je demant gueredon, por ce que jou ai fet le proufit dou commun ; non as, fet li autres ; ou respont par aventure que tu as deservi paine. Et ceste questions n’a pas lieu se des choses passees non, car nus ne doit estre dampnés ne guerredonés se par les choses passees non k’il a ja fetes. Mais de ce se taist ore li mestres, pour deviser les parties de rectorique.
\chapterclose


\chapteropen
\chapter[{.III.III. Des .v. parties de rectoriqe}]{\textsc{.III.III.} Des \textsc{.v.} parties de rectoriqe}\phantomsection
\label{tresor\_3-3}

\chaptercont
\noindent En ceste science ce dist Tuilles sont \textsc{.v.} parties, ce sont truevement, ordre, parables, memores, et parleure. Boesces dist que ces \textsc{.v.} choses sont si de la substance du parler que se aucune i faut il n’est pas compli, tout autresi comme li fondemens et la parois et la coverture sont parties d’une maison, sans coi ele n’est pas enterine maisons.\par
Troevemens est uns apensemens de trover choses voires ou voirsamblables et a prover sa matire ; c’est le fondement et la fermeté de toute ceste science que, tot avant que on die ou que on escrive mot, doit il trover ses raisons et ses argumens et prover ses dis por fere croire a ciaus a qui il parole.\par
Ordres est establir ses dis et ses argumens k’il a trovés chascun en son lieu, selonc ce k’il puisse mieus valoir ; c’est a dire que tot avant doit il metre les bonnes raisons, et ou milieu les foibles, mais a la fin doit il metre les trés fors argumens, en quoi il plus se fie, et que son aversaire ne puisse contrester.\par
Parables est li atornemens de paroles et de sentences avenables a ce k’il a trové ; car trover et penser poi vaudroient sans les paroles acordans a sa matire, car les paroles doivent servir a la matire, non pas la matere as paroles ; car uns beaus mos et une bonne sentence et uns proverbes et une similitude, u uns essamples ki soit samblables a la matire conferme trestous tes dis et les fait biaus et creables. Pour ce doit li parliers, quant il traite de ost ou de fuerre, dire paroles de guerre ou de victore, et en dolour paroles de courouch, et en joie paroles de leece.\par
Memore est a sovenir soi fermement de ce k’il a pensé et mis en ordre, car tout seroit ausi comme noiens s’il ne s’en sovenist quant il est au parler venus ; et si ne quide nus que ce soit la naturele memore, ki est une vertus de l’ame, ki se sovient de ce que nous aprendons par aucun sens dou cors, ains est memore artificiel que l’en aquiert par ensegnement des sages a retenir ce qu’il pense et ce k’il aprent.\par
Parleure est a dire ce k’il a trové et establi en sa pensee, et avenableté du cors et de la vois et des meurs, selonc la dignité des choses et des parables. Et a la verité dire quant li parliers vint a dire son conte, il doit mout consirer sa matire et son estre ; car autrement doit il porter ses membres et sa chiere et son esgart en dolour que en leece, et autrement en guerre que en pais, et en \textsc{.i.} lieu autrement k’en l’autre. Por ce doit chascuns garder qu’il ne lieve ses mains ne ses oils ne son front en maniere qui soit blasmable : et sor ceste matire vaut la doctrine ki est ça ariere ou livre de visces et de vertus ou chapistre de garde.
\chapterclose


\chapteropen
\chapter[{.III.IIII. De deus manieres de parler, u de bouche u par letres}]{\textsc{.III.IIII.} De deus manieres de parler, u de bouche u par letres}\phantomsection
\label{tresor\_3-4}

\chaptercont
\noindent Or dist li mestres que la science de rectorique est en \textsc{.ii.} manieres, une ki est en disant de bouche et une autre que l’om mande par letres ; mais li enseignement sont commun, car il ne puet chaloir que l’on die un conte ou que on le mande par letres. Mais l’une et l’autre maniere puet estre diversement, c’est par content et sans content. Et ce ki est dit ou escrit sans content n’apertient pas a rectorique, selonc ce k’Aristotles et Tuilles dient apertement ; mais Gorgias dist que tout ce ke li parleour dient apertient a rectorique. Boesces meismes s’acorde bien a çou que quanq’a dire covient puet estre matire du diteour.\par
Et ki bien wet consirer la soutillece de cestui art, il covenra que la premiere sentence soit de grignour pesantour car kiconques dit de bouche u envoie letres a aucun home, ou il le fet por movoir le corage celui a croire et a voloir ce qu’il dist, ou non. Et s’il ne le fait mie pour ce, je di sans faille que ces dis n’apartienent as ensegnemens de rectorique, ains est la commune parleure des homes ki sont sans art et sans mestrie, et ce soit loins de nous et remaigne a la nicheté des femes et du menu peuple, car il n’ont que faire des citaines choses.\par
Mais s’il le fait artificielment por movoir le cuer a celui a qui il parole ou mande par ses letres il covient que ce soit en proiant et en demandant ancune chose, ou par conseil ou par manaces ou par conort ou par commander ou par amonester ou par autres choses samblables. Et il set bien que celui, a qui il envoie ses letres, a ses deffenses contre ce que il li mande ; et por ce li sages ditieres conferme ses letres par beles et par bones raisons et par fors argomens ki aident a ce k’il voet, autresi comme s’il fust a la contention devant lui. Et celes letres apertienent a rectorique, autresi comme la chançons dont li uns amans parole a l’autre autresi com s’il fust devant lui a la contençon.\par
Et por ce poons nous entendre que contençons puet estre en \textsc{.ii.} manieres, ou en apert quant on se deffent de bouche a autre ou par letres, ou non en apert quant li uns mande letres garnies de bons argumens contre la deffense qu’il quide ke li autres ait. Et tous contens sont apertenant a rectorique, meismement se c’est des choses citaines et des besoignes as princes de la terre et des autres gens ; non mie des fables et des movemens de la mer, ne dou compas de la terre, ne dou cours des estoiles, car de teus contens ne s’entremet pas ceste science.
\chapterclose


\chapteropen
\chapter[{.III.V. Des contens qi naissent des paroles escrites}]{\textsc{.III.V.} Des contens qi naissent des paroles escrites}\phantomsection
\label{tresor\_3-5}

\chaptercont
\noindent Par ce pert il tot clerement que tous contens ou il sont par paroles escrites ou parole que l’om dit sans escripture nule, selonc ce que Tuilles dist en son livre. Et celui ki est par paroles escriptes si puet estre en \textsc{.v.} manieres ; car aucunefois la parole ne s’acorde pas a la sentence de celui qui l’escrit, et aucunefois est que \textsc{.ii.} paroles ou \textsc{.ii.} lois ou plusours se descordent entr’eles meismes, et aucunefois samble que ce ki est escrit segnefie \textsc{.ii.} choses ou plusours, et aucunefois avient que de ce ki est escrit retrait on sens et example que on doit faire en une autre chose qui n’estoit pas escrite, et aucune fois est li contens sor la force d’une parole escrite por savoir qu’ele doit segnefiier.
\chapterclose


\chapteropen
\chapter[{.III.VI. Comment tous contens naissent par .iiii. raisons}]{\textsc{.III.VI.} Comment tous contens naissent par \textsc{.iiii.} raisons}\phantomsection
\label{tresor\_3-6}

\chaptercont
\noindent D’autre part nous ensegne Tuilles que tout contens, ou sont de bouche ou sont d’escripture, naissent dou fait ou dou non de celui fait ou de sa qualité ou de sa remuance ; car se l’une de ces \textsc{.iiii.} ne fust, ne poroit donc naistre li contens. Raison coment : je dirai que tu as aucune chose faite, et si metrai sus aucune ensegne pour demoustrer que tu l’as fait, en ceste maniere, tu oceis Jehan, car je te vi oster le coutel sanglant de son cors ; mais tu le nies et di ke tu ne l’as pas ocis. Et ensi naist li contens dou fait entre moi et toi, ki mout est fors et griés a prover, pour ce que chascuns a autresi fort argoment li uns que li autres.\par
Li contens ki naist dou non est quant ambedeus les parties reconnoissent le fait, mais il sont en discorde de son non en ceste maniere. Je di que cis hom ci a fet sacrilege por ce k’il embla \textsc{.i.} cheval dedens le mostier ; ce n’est pas sacrileges, fait li autres, mais larrecin ; et ensi naist li contens par le non dou fait. Et sor ce covient il consirer que monte l’un nom et quoi l’autre ; car sacrileges est embler chose sacree d’un lieu sacré, mais tot autre maniere d’embler est larrecin. En ce content reconnoist home le fet, mais il sont en discorde dou non de celui fet solement.\par
Li contens ki naist de la qualité est quant on reconnoist le fait et le non, mais il se descorde de la maniere de celui fet, c’est de la force ou de la quantité ou dela comparison. Raison coment : je di que c’est uns crueus meffais ou que cist est plus cruel que cist autres, ou ke c’est bien fet selonc droit et selonc raison, et li autres dist que non est. Et quant Catelline disoit que Tulles n’avoit mie tant valu au commun de Rome comme il avoit : et quant \textsc{.i.} signatour disoit, mieus vaut a destruire Cartage que laissier la ; et quant Julle Cesar disoit, je chace Pompee justement ; je di que tous ces contens naissent de la qualité dou fait, non pas dou fait ne de son non.\par
Li contens ki naist de la remuance est quant li uns commence une question et li autres dist qu’ele doit estre remué, ou pour çou k’ele n’apertient pas a celui ki l’esmuet, u pour çou k’il ne l’esmuet contre celui qu’il doit, u non, cf\par
devant ceaus qui i doivent estre, ou non, en celui tans k’il covient, u non, de cele loi u de celui pechié ou de cele paine k’il deust.
\chapterclose


\chapteropen
\chapter[{.III.VII. Du content qi naist de la qualité du fet}]{\textsc{.III.VII.} Du content qi naist de la qualité du fet}\phantomsection
\label{tresor\_3-7}

\chaptercont
\noindent Li contens ki naist de la qualité dou fait, coment k’il soit, Tuilles dist k’il est devisee en \textsc{.ii.} parties, l’une partie est de droit ki consire des choses presentes et des futures, selonc les us et les drois dou païs. Et a çou prover travaillent mout li parleour pour la comparison k’il lor estuet a fere des samblables choses ou des contraires.\par
L’autre est de loi ki consire seulement des choses alees selonc loi escrite, et en ce soufist assés a dire ce ki est escrit en la loi ; et selonc ce sont les choses jugies, se eles sont justement faites et selonc justice, et d’un home s’il est dignes de paine u de merite.\par
Et ceste meisme ki est de loi est double : une clere, ki por sa clarté moustre maintenant se cele chose est bone u male, ou de raison u de tort ; et une autre enprompteresse, ki par soi n’a nule deffense ferme s’ele ne l’enprompte dehors. Et ses enpruns sont en \textsc{.iiii.} manieres, ou par recognoissance ou par removance ou par vengance ou par recomparison.\par
Recognoissance est quant on ne nie ne ne deffent pas le fait, mais il demande que l’en li pardoinst ; et ce puet estre en \textsc{.ii.} manieres, une sans coupe et autre par proiere.\par
Sans coupe est quant il dist k’il ne le fist pas a escient, mais ce fu par non savoir u par necessité ou par autre enpeechement. Par priere est quant on prie que on li pardoinst ses meffés, et ce n’avient pas sovent.\par
Removance est quant li hom se vieut oster dou meffet qu’il ne le fist pas, et k’il n’i ot nule coupe, ains le met sor \textsc{.i.} autre et ensi s’efforce de removoir le fait et la coupe de soi a \textsc{.i.} autre ; et ce puet il faire en \textsc{.ii.} manieres, ou metant sor l’autre l’achoison et la coupe, ou metant le fait.\par
Et certes l’achoison et la coupe met il sor l’autre quant il dist que ce ki est avenu vint par la force et par la signorie que cil autres avoit sor li, ki se deffent. Le fait puet il metre sor \textsc{.i.} autre quant il dist de soi k’il ne le fist pas, ne ne fu fait par sa coupe, ne por achoison de lui, mais il moustre que cil autres le fist, pour ce k’il le pooit et le devoit fere.\par
Vengance est quant li hom reconnoist bien qu’il fist ce que l’en dist de lui, mais il moustre que c’est fet raisnablement et por vengance, pour ce que devant ce eut il receu le por quoi.\par
Comparison est quant on reconnoist k’il fist çou que l’om li met sus, més il moustre que ce fist il por acomplir une autre chose honeste et proufitable, que autrement ne pooit estre menee a bone fin.
\chapterclose


\chapteropen
\chapter[{.III.VIII. Des choses qu’on doit consirer en sa matire}]{\textsc{.III.VIII.} Des choses qu’on doit consirer en sa matire}\phantomsection
\label{tresor\_3-8}

\chaptercont
\noindent Encore nous ensegne Tulles que nous regardons nostre matire sor quoi nous devons parler et escrire letres, s’ele est toute simple d’une chose seulement, ou s’ele est de plusours. Et aprés çou que nous avons bien consiré diligemment la naissance dou content et tot son estre et ses manieres, encore nos couvient savoir coi et coment est la question et la tençon et le jugement et le confermement dou content.
\chapterclose


\chapteropen
\chapter[{.III.VIIII. Ke est contens et comment doist estre establis par parties}]{\textsc{.III.VIIII.} Ke est contens et comment doist estre establis par parties}\phantomsection
\label{tresor\_3-9}

\chaptercont
\noindent Par ces ensegnemens que li mestres devise ça en arieres devons nous entendre que contens n’est autre chose que la descorde ki est entre \textsc{.ii.} parties ou entre \textsc{.ii.} ditteors, tant que li uns dit qu’il a droit et li autres dit que non a. Et quant il sont a ce venus, des lors covient il veoir s’il a droit u non, et c’est la question sor le content. Mais por ce ke poi vaut a dire k’il a droit s’il ne moustre raison pour quoi, li covient k’il die maintenant cele propre raison por quoi il quide avoir droit en sa question, car s’il ne le desist sa deffense seroit frivole.\par
Et quant il a dit la raison por quoi il fist ce, ses aversaires dist ses autres argumens pour afoibloier la raison que li autres li moustre, por apeticier sa deffense, et lors naist le jugement sor les dis de l’un et de l’autre, por jugier se cil a droit pour la raison k’il moustre. Et quant il sont jusques la venut, maintenant met il son confermement, c’est a dire les trés bones raisons et les trés fors argumens ki plus valent au jugement. En ceste maniere establissent li sage lor letres, pour moustrer lor droit et por affermer lor raison.\par
Et sachiés que toutes manieres de contens, tant comme il i a de descordes et des chapistres tensonables, autretant il covient avoir de questions et de raisons et de jugement et de confermement. Sauve ce que quant li contens naist dou fet, que l’en ne reconnoist pas, certes li jugemens sor la raison ne puet pas nestre, por ce que cil ki nie n’ensegne pas nule raison de sa negation.\par
Et lors est le jugement sour la question solement, c’est a dire s’il fist ce ou non ; et si ne doit nus folement quidier que ces ensegnemens soient baillié solement par le content ki sont en plet et en court, ains sont en tous les dis que hom dist en consillant ou proiant ou en message ou en autre maniere.\par
Neis es letres que l’om envoie as autres observe il cest ordre meismes, car tot avant demande il ce qu’il wet, et c’est autresi comme question, car il est en doute que li autres se deffende par aucune raison contre sa requeste ; et por ce joint il maintenant la raison por quoi li autres doit faire ce k’il li requiert et pour quoi li autres ne puisse afoibloier cele raison, met il encore les trés fors argumens de quoi il se fie mieus.\par
Et a la fin de sa letre met il la conclusion, la ou il li mande que s’il fet ce k’il li requiert que ce et ce en sera, et c’est en lieu de jugement et de confermement.\par
Mais de ces devisemens des contens se taist ore li contes, por dire des autres parties de bone parleure ki sont besoignables en conte ; car a la verité dire on ne doit consirer devant solement ce qu’il doit au devant conter ; mais il estuet a establir des premieres paroles les derrenieres s’il vieut que si dit soient bien acordant a sa matire.
\chapterclose


\chapteropen
\chapter[{.III.X. De .ii. manieres de parler, ou en prose ou en risme}]{\textsc{.III.X.} De \textsc{.ii.} manieres de parler, ou en prose ou en risme}\phantomsection
\label{tresor\_3-10}

\chaptercont
\noindent La grant partison de tous parliers est en \textsc{.ii.} manieres, une ki est en prose et \textsc{.i.} autre ki est en risme. Mais li ensegnement de rectorique sont commun d’ambes \textsc{.ii.}, sauve ce que la voie de prose est large et pleniere, si comme est ore la commune parleure des gens, mais li sentiers de risme est plus estrois et plus fors, si comme celui ki est clos et fermés de murs et de palis, c’est a dire de pois et de nombre et de mesure certaine de quoi on ne puet ne ne doit trespasser.\par
Car ki bien voudra rimoier, il li covient a conter toutes les sillabes de ses dis, en tel maniere que li vier soient acordables en nombre et que li uns n’en ait plus que li autres. Aprés li covient il amesurer les \textsc{.ii.} derraines sillabes de ses dis en tel maniere ke toutes les letres de la derraine sillabe soient samblables, et au mains la vocal de la sillabe qui va devant la derraine. Aprés ce li covient il contrepeser l’accent et la vois, si ke ses rimes s’entracordent en lor accens. Car ja soit ce que tu acordes les letres et les sillabes, certes la risme n’ert ja droite se l’accent se descorde.\par
Mais comment que ta parleure soit, ou par rime ou par prose, esgarde que ti dit ne soient maigre ne sech, mais soient replain de jus et de sanc, c’est a dire de sens et de sentence. Garde que ti mot ne soient nices, ains soient grief et de grant pesantour, mais non mie de trop grant, ki les fesist atrebuchier. Garde qu’il n’aportent laidures nules, mais la bele coulour soit dedens ou dehors, et la science de rectorique soit en toi peinturiere, ki mete la coulour en risme et en prose. Mais garde toi de trop poindre, car aucunefois est couleur a eschiver couleur.
\chapterclose


\chapteropen
\chapter[{.III.XI. Ci define de truevement et commence a deviser de l’ordre}]{\textsc{.III.XI.} Ci define de truevement et commence a deviser de l’ordre}\phantomsection
\label{tresor\_3-11}

\chaptercont
\noindent En ceste partie ki est passee a devisé li mestres le fondement et la nature de cestui art, et coment on doit establir sa matire par ordres et par parties. Més por mieus esclarcir çou k’il en a dit, dira il en ceste partie les riules ki apertienent a l’ordre de cestui art ; car il ne vieut pas faire si comme Ciclicus fist, de qui parole Orasces : il ne vieut torner la lumiere en fumee, car tot ce k’il dist par riules mousterra aprés par essamples.\par
Et vous avés bien ou ça arieres entor le commencement de cest livre, que aprés ce que on a trové et pensé en son cuer ce c’on doit dire, lors maintenant doit il establir ses dis par ordre, c’est a dire k’il die cascune chose en son lieu. Mais cis ordres est en \textsc{.ii.} manieres, une ki est naturel et \textsc{.i.} autre ki est artificiel.\par
Li ordre ki est apelee naturel s’en va droite voie par mi le grant chemin, k’il n’en ist ne d’une part ne d’autre, c’est a reconter et a dire les choses selonc ce qu’eles furent dé le commencement jusc’a la fin, ce devant devant, et ce du milieu u milieu de son conte, et a la fin ce qui fu derieres. Et ceste maniere de parler est sans grant mestrie d’art, et por ce ne s’en entremet cis livres.\par
Li ordres artificiel ne se tient pas au grant chemin, ains s’en va par sentier et par adrecemens ki l’aimainent plus delivrement la u il vieut aler. Il ne dist pas chascune chose selonc ce qu’ele fu, mais il remue ce devant deriere ou el milieu de son dit, non pas desavenablement mais tot sagement, por affermer sa entention. Et pour ce remue li parliers sovent son prologue et sa conclusion et les autres parties de son conte ; et les met non pas en son naturel lieu, mais en autre ki plus vaut, por ce que les plus fermes choses doivent tozjours estre mises au commencement, et a la fin, et les plus foibles en milieu.\par
Et quant tu vieus respondre a ton adversaire, tu dois commencier ton conte a sa derraine raison, en quoi il se confie plus par aventure. Neis cil ki wet raconter une istore vielle et usee, il est bon de rebourser son droit cours et varier son droit ordre, en tel maniere qu’ele samble toute novele. Ce meisme vaut molt en sermoner et en toutes causes, car l’om doit tousjors garder a la fin ce ki plus plaise et ki plus esmueve les corages as oïans.\par
Et cis ordres artificiels est devisé en \textsc{.viii.} manieres. La premiere est a dire au commencement ce ki avoit esté a la fin. La seconde si est a commencier a ce ki fu ou milieu. La tierce est a fonder tot son conte sor \textsc{.i.} proverbe, selonc ce ke segnefie le commencement de celui proverbe. La quarte est a fonder le selonc ce que segnefie le mi dou proverbe. La quinte est a fonder le selonc la fin dou proverbe. La siste est a fonder tot son conte sor \textsc{.i.} essample, selonc ce ki est segnefiet par le commencement de l’essample. La septime est a commencier le selonc la segnefiance dou mi de l’example. La \textsc{.viii.} est a fonder tot son conte sor la segnefiance de la fin de l’essample.\par
Raison comment : a la fin de la chose commence cil ki dist, ja soit ce que li solaus couchans nous laisse la noire nuit, totefois revient il au matin plus luisans ; et cil ki dit, Abrahans quant il voloit ocire son fil por rendre a Dieu son sacrefice, li angeles li moustra \textsc{.i.} monton a sacrefiier. Ce meisme fist Virgiles quant il voloit raconter l’istore de Troie, car il commença son livre a Eneas quant il s’enfui de la destruction de Troie.\par
Et au milieu de la chose commence cil : Abraham laissa son serf avec l’asne au pié du tertre, car il ne voloit pas k’il seust sa covine.\par
A la segnefiance dou commencement dou proverbe commence cil ki dist ensi, mout desert grant merite cil ki de bone foi sert volentiers et hastivement, si comme fist Abraham, ke lorsque Dieus li commanda a ocire son fil, maintenant ala por acomplir son commandement.\par
A la segnefiance dou mi dou proverbe commence cil ki dist ensi, sers ne doit pas savoir le secré de son signor, por ce laissa Abraham son serf quant il monta a son sacrefice.\par
Selonc la fin dou proverbe commence cil ki dist ensi, il n’est pas digne chose que enterine fois perde ses merites ; por ce garanti Nostre Sires a Abraham son fil, ki ja estoit mis sor l’autel dou sacrefisse.\par
Selonc ce ki est segnefiiet par le commencement d’un essample commence cil ki dit ensi, bons arbres engendre bon fruit, por çou volt Dieus que li fiz Abraham fust mis sor son autel, et k’il n’i morust.\par
A la segnefiance dou mi de l’essample commence cil ki dist ensi, hom doit oster d’entre le forment toutes males semences, en tel maniere que li pains ne soit amers ; por ce laissa Abraham son serf, k’il n’enpechast son sacrefisce.\par
A la segnefiance de la fin de l’essample commence cil ki dist ensi, si comme li solaus ne pert sa clarté par la nuit, tot autresi li fiz Abraham ne perdi la vie au sacrefisse son pere, ains revint biaus et clers comme soleil levant.\par
Or avés oii diligemment comment li parliers puet dire son conte selonc ordre naturel, et comment il le puet dire en \textsc{.viii.} manieres selonc l’ordre artificiel. Et sachiés que proverbes et examples ki sont avenables et acordans a la matire sont trop bons ; mais k’il ne soient trop sovent, car lors seroient il gravables et souspecenous.
\chapterclose


\chapteropen
\chapter[{.III.XII. Des .iiii. choses que li parleors doit consirer en sa matire devant q’il die u q’il escrise son conte}]{\textsc{.III.XII.} Des \textsc{.iiii.} choses que li parleors doit consirer en sa matire devant q’il die u q’il escrise son conte}\phantomsection
\label{tresor\_3-12}

\chaptercont
\noindent Aprés ce covient que tu regardes en ta matire \textsc{.iiii.} choses se tu vieus estre bons parliers ou diter sagement une letre. La premiere est que se la matire est longue et oscure, tu le dois apeticier a moz briés et entendables. La seconde est que se la matire est briés et oscure, tu le dois auques acroistre et ovrir tout belement. La tierce est que la u la matire est longue et overte, tu le dois abrevier et enforcier et covrir de bons dis. La quarte est que quant la matire est briés et legiere, tu le dois eslongier briefment et aorner avenablement.\par
En ceste maniere dois tu consirer en toi meismes et connoistre se la matire est longue ou briés ou s’ele est legiere et oscure a entendre, si ke tu puisses governer chascun selonc la loi. Car matire est samblable a la cire, ki se laisse mener et apeticier et croistre a la volenté du mestre.
\chapterclose


\chapteropen
\chapter[{.III.XIII. Comment on puet acroistre son conte en .viii. manieres}]{\textsc{.III.XIII.} Comment on puet acroistre son conte en \textsc{.viii.} manieres}\phantomsection
\label{tresor\_3-13}

\chaptercont
\noindent Et se ta matire doit estre escreue par paroles, je di que tu le pués acroistre en \textsc{.viii.} manieres, ki sont apelees coulour de rectorique. Dont la premiere est apelée aornemens ; que tout çou que on poroit dire en \textsc{.iii.} mos ou en \textsc{.iiii.} ou a mout poi de paroles, il l’acroist par autres paroles plus longues et plus avenans ki dient ce meismes. Raison comment : Jhesucris nasqui de la Virgene Marie ; mais li parleour ki ce wet adorner dira ensi, li Beneois Fiz Dieu prist char en la Glorieuse Virgene Marie, ki autant vaut a dire comme ce poi devant. Ou se je disoie, Julles Cesar fu empereres de tot le monde ; li parleour ki ce dit voudra acroistre dira ensi, li sens et la vaillance dou bon Julle Cesar sousmist tot le monde en sa subjection, et fu empereres et sires de la terre.\par
La seconde est apelee tourn ; car la ou ta matire est tote briés, tu changeras les propres mos et remueras les nons des choses et des persones en plusors paroles tot belement environ le fait, et feras point a tes dis, et reposeras ton esperit tant que tu esloignes ton conte et de tens et de paroles.\par
Et cil torn puet estre en \textsc{.ii.} manieres, ou k’il dist la verité tot clerement, raison coment : tu vieus dire, il ajourne, di donc, ja commence li solaus a espandre ses rais parmi la terre ; ou k’il eschive la verité par son tourn, ki autant vaut selonc ce que li Apostles dist, il ont remué, fist il, les us de nature en cel usage ki est contre nature. Por ce tourn eschiva li Apostres \textsc{.i.} lait mot k’il voloit dire, et dit ce ki otant valut.\par
La tierce coulour por acroistre tes dis est apelee comparison, et c’est la plus bele croissance et avenable que parleur facent. Mais ele est devisee en \textsc{.ii.} manieres, car ou ele est coverte ou ele est descoverte.\par
Et cele ki est descoverte se fait connoistre par \textsc{.iii.} mos ki segnefient comparison, ce sont plus, mains, et autant. Raison coment : por ce mot plus, dist on ensinc, cist est plus fors de lyon ; por ce mot mains, dit on ensinc, cist hom est mains courouçables que palombe ; par ce mot autant, dist on ensinc, cist est autretant couars comme lievres.\par
La seconde maniere, ki est coverte, ne se fait pas connoistre a ses signes ; ele ne vient pas en son abit, ains moustre une autre samblance dehors ki est si comme jointe a la verité dedens comme s’ele fust de la matire meismes. Raison comment : d’un home perreceus je dirai, c’est une tortue, et d’un isnel je dirai, c’est uns vens ; et sachiés que ceste maniere de parler est mout bele et mout bonne et cortoise et de bone sentence, et mout le puet on trover es dis de sages homes.\par
La quarte couleur est apelee clamour, pour ce que l’on parole ensi comme en criant ou plaignant de courous, ou par desdaing, ou par autre chose samblable. Raison comment : Ha Nature, Nature, por quoi faisoies tu le joene roi si replain de tous bons abis quant tu l’en devoies si tost oster. Ha male mort, car fussiés vos morte quant vous en avés porté la flour dou monde.\par
La quinte coulour est apelee fainture, pour ce que l’om faint une chose ki n’a pooir de parler ne nature n’en a, autresi comme s’ele parlast ; si come nous poons oïr tousjours des gens qui dient ou de bestes ou d’autre chose en samblance qu’ele eust parlé et dit ancune chose. Et c’est si entendable que li mestres ne s’entremet de mostrer aucun example de ce.\par
La sisime colour est apelee trespas, por ce que quant li parliers a encommenciet son conte, il s’en desoivre \textsc{.i.} petitet et trespasse a une autre chose ki est resamblable a sa matire, et lors est il bons et profitables, mais se li trespas n’est bien dou tout acordans a la matire, certes il sera mauvais et desprisables. Pour ce fist bien Julle Cesar quant il volt deffendre les conjurés, il fist son trespas au pardon que li ancien avoient fais a ceus de Rodes et de Cartages. Autresi fist Catons quant il les volt jugier a mort, il ramentut Maliun Torcatum coment il juga son fis a la mort. Autresi trespasse on sovent a la fin u au milieu de sa matire, por renoveler ce ki sambloit estre vieus ou par autre bonne raison.\par
La septime coulor est apelee demoustrance, pour ce que li parleours demoustre et dist ses proprietés et les ensegne d’une chose ou d’un home, par ochoison de prover aucune chose ki apertient a sa matire ; si comme l’escripture dist, il avoit, fist il, en la tere Hus \textsc{.i.} home ki avoit a non Job, simple, droiturier, juste, et ki cremoit Dieu.\par
Autresi fist Tristans quant il devisa la biauté dame Yseude. Ses cheviaus, fist il, resplendissent plus que fil d’or, son front sormonte la flour de list, ses noirs sourcis sont ploié comme petis arconciaus, et une petite voie de let les desoivre parmi la ligne du nés et si par mesure k’il n’i a ne plus ne moins ; ses oils ki sormontent toutes esmeraudes reluisent en son front comme \textsc{.ii.} estoiles ; sa face ensiut la biauté dou matinet, car ele est de vermeil et de blanc mellés ensamble en tel maniere que l’une colour et l’autre ne resplendissent malement ; la bouche petite et les levres auques espessetes et ardans de bele coulour, et les dens plus clers que perles, ki sont establi par ordre et par mesure ; mais ne pantere ne espisse nule ne se puet comparer a la trés douce alaine de sa bouche. Li mentons est assés plus polis que nul marbre, lait done colour a son col, et cristal resplendissant a sa gorge. De ses droites espaules descendent \textsc{.ii.} bras grailles et lons, as blanches mains ou la chars est tenre et mole ; les dois grans, tenues et roons, sor quoi se reluist la beauté des ongles. Son trés beau pis est adornés de \textsc{.ii.} pomes de paradis, ki sont autresi comme une masse de noif ; et est si grelle en sa chainture que on la poroit porprendre de ses mains. Mais je me tairai des autres parties dedéns, desquex li corages parole mieus que la langue.\par
La octime colour est apelee adoublement, pour ce que li parliers adoble son conte et le dist \textsc{.ii.} fois ensamble. Et c’est en \textsc{.ii.} manieres, une ki dist de sa matire et maintenant le redist par le contraire de son dit. Raison coment : je voeil dire d’un home k’il est joenes, mais je adoublerai mes dis en tel maniere : cis hom est joenes et non pas vieus, ou ceste chose est douce et non pas amere. L’autre maniere dist de sa matire et maintenant redist autres paroles ki ostent le contraire de ce k’il avoit dit, en ceste maniere : voirs est que cis hom si est joenes mais il n’est pas fols, et ja soit il nobles il n’est par orgullous, et il est larges mais non pas gasteres.\par
Or avés oii comment on puet acroistre ses dis, et sa matere alongnier, car a poi de semence croist grans blés et de petites fontaines naist grans flueves ; por ce est il drois et raisons que li mestres ensegne a abregier son conte quant il est trop grans et trop lons ; et de ce mousterra il avant la u il dira du fet. Mais ci se taist li mestres des ensegnemens de la grant parleure por deviser ciaus de la petite parleure, c’est a dire d’un conte u d’une epistle ke tu vieus faire sor aucune matire ki vient ; car li mestres apele parleure le general non de tous dis, mais contes est uns seus dis u une soule letre u une autre chose que l’en conte sor sa matire.
\chapterclose


\chapteropen
\chapter[{.III.XIIII. Des branches du conte et comment li parleour doit establir ses dis par ordre}]{\textsc{.III.XIIII.} Des branches du conte et comment li parleour doit establir ses dis par ordre}\phantomsection
\label{tresor\_3-14}

\chaptercont
\noindent Les parties dou conte, selonc ce que Tulles nous ensegne, sont \textsc{.vi.}, le prologue, le fait, le devisement, le confermement, le deffermement, et la conclusion. Mais li ditteour ki ditent letres et epistles par maistrie de rectorique dient que une letre n’a que \textsc{.v.} parties, c’est la saluence, le prologue, le fait, la demande, et la conclusion.\par
Et se aucuns demandoit pour quoi il i a descorde entre Tuille et les ditteours, puisque chascuns ensiut l’ensegnement de rectorique, je li diroie que la descorde est par samblance, non pas sor la verité. Car la u li ditteour dient que la saluence est la premiere branche de la letre et du message, Tuilles entendi et volt que la saluence soit sous le prologhe ; car tot ce que l’on dist devant le fait, autresi comme por apareillier sa matire, est apelé prologue ; mais li ditteor dient que la saluence est li huis et l’entree dou conte et oils et lumière de lui, et por ce li donent il l’onor de la premiere partie des epistles et des messages, car envoier letres et envoier messages tot cort par une voie.\par
D’autre part cele partie ke Tuilles apele devisement, li ditteour le comprent sor le fait ; et ce que Tulles apele confermement et deffermement, li ditteour le comprendent sous le demande. Et por mieus entendre les nons des uns et des autres et por conoistre l’entention de Tulle et des autres diteors vieut li mestres esclarcir maintenant la senefiance de chascune partie et de son non.
\chapterclose


\chapteropen
\chapter[{.III.XV. Des .vi. branches dou conte en parlant de bouche}]{\textsc{.III.XV.} Des \textsc{.vi.} branches dou conte en parlant de bouche}\phantomsection
\label{tresor\_3-15}

\chaptercont
\noindent Prologues est le commencement et la premiere partie dou conte ki adrece et apareille l’oïe et le corage de ciaus a qui tu paroles a entendre ce que tu diras.\par
Le fait est a conter les choses ki furent u ne furent autresi comme s’eles fuissent. Et ce est quant li hom dist ce sor quoi il a fermé son conte.\par
Devisemens est lors quant on conte le fait et maintenant commence a deviser les parties, et dist, ce fu en tel maniere et ce en cele autre, et aquieut celes parties ki soient plus porfitables a lui et plus contraires a ses adversaires, et les afiche au plus k’il puet au cuer de celui a qui il parole. Et lors samble qu’il si ait conté le fait, et c’est l’achoison por quoi li diteour content le devisement sor le fait.\par
Confermement est la u li parleour moustrent lor raisons, et assigne tous les argumens k’il puet a prover a sa entention et acroistre foi et creance a ses dis.\par
Deffermemens est quant li parleours moustre sa bone raison et ses fors argomens. ki afoibloient et apeticent et destruient le confermement son adversaire.\par
Conclusion est la derraine partie et la confirmations dou conte. Ce sont les \textsc{.vi.} parties dou conte, selonc la sentence de Tuille. Or fait il bien a dire les parties que li diteour dient et premierement des salus.
\chapterclose


\chapteropen
\chapter[{.III.XVI. Des .v. parties des letres escrites que l’en envoie as gens les uns as autres}]{\textsc{.III.XVI.} Des \textsc{.v.} parties des letres escrites que l’en envoie as gens les uns as autres}\phantomsection
\label{tresor\_3-16}

\chaptercont
\noindent Salus est le commencement de la letre, ki nome ciaus ki mandent et ciaus ki reçoivent les letres, et la dignité de chascun, et la volenté de coer ke cil ki envoie a contre celui ki reçoit, ce est a dire que s’il est ses amis, il li mande salus et autres dous mos ki autant valent et plus ; et se c’est anemis il se taist ou il li mande aucun autre mot covert ou descovert de mal ; et s’il est graindres, il li mande paroles de reverence ; autresi doit on mander as pers et as menours ce ki avenable est a chascun en tel maniere qu’il n’i ait visces dou plus ne defautes du mains. Et sachiés que le non de celui ki est graindres et en plus haute dignité doit tozjours estre devant, se ce n’est ou par cortoisie u par humilité ou par autre chose samblable.\par
Du prologue et du fait et de lor force a dit li mestres ci devant la segnefiance, et por ce n’en dira il ore plus ke dit en a, car li dicteour s’en acordent bien a la sentence de Tuille.\par
Mais de la demande dist li mestres k’ele est cele partie en quoi la letre ou li messages demande le fait de ce k’il wet, en priant ou en commandant ou menachant ou consillant ou en autre maniere de chose a quoi il bee atrere le cuer de celui a qui il a envoié. Et quant li ditteours a finee se demande et moustré ses confermemens et deffermemens, il fet la conclusion, c’est la fin de ses dis ; en quoi il conclust la somme de son conte, comment il est et comment il en puet avenir.
\chapterclose


\chapteropen
\chapter[{.III.XVII. Des ensegnemens de toz prologues selonc les diversités des matieres}]{\textsc{.III.XVII.} Des ensegnemens de toz prologues selonc les diversités des matieres}\phantomsection
\label{tresor\_3-17}

\chaptercont
\noindent Et pour ce que li prologues est sires et princes de tot le conte, selonc ce que Tuilles prueve en son livre, est il bien covenable que sour ce doinst li mestres son ensegnement. De qui Tuilles dit ke prologues est uns dis ki aquiert veraiement le corage de celui a qui tu paroles a oïr ce que tu diras. Et ce puet estre en \textsc{.iii.} manieres, ou en aquerant sa bienweillance, ou por li doner volenté d’oïr, ou volenté de savoir tes dis.\par
Por quoi je di que quant tu vieus bien faire ton prologue, il te covient tot avant consirer ta matire et connoistre la nature dou fait et sa maniere. Fai donc a l’essample de celui ki vieut maisoner, car il ne cort pas a l’oevre hastivement, ains le mesure tot avant a la ligne de son cuer, et comprent en sa memore trestot l’ordre et la figure de la maison.\par
Et tu gardes que ta langue ne soit courans a parler, ne la mains a l’escrire, ne comence pas l’un ne l’autre a cours de fortune. Mais ton sens tiegne en sa main l’office de chascune, en tel maniere que la matire soit longuement a la balance de ton cuer. Et dedens lui pregne l’ordre de sa voie et de sa fin, car a ce que les besoignes du siecle sont diverses, te covient a parler diversement, et a chacune selonc sa matire\par
Tuilles dist que tout dit sont de \textsc{.v.} manieres, ou il est honestes u contraires, ou vil ou douteus, ou il est oscur ; et pour ce pense que tu dois autrement comencier et ensivre ton conte en l’un k’en l’autre.\par
Et sachiés que honestes est ce ke maintenant plest a ciaus ki l’entendent, sans ton prologue et sans nul aornement de paroles.\par
Contraire est ce que maintenant desplest par sa malice.\par
Vilh est ce a quoi cil ki doit oïr n’entent gaires, por la vilté et por la petitece de la chose.\par
Douteus est en \textsc{.ii.} manieres, ou pour çou que l’on doute de sa sentence ou pour çou k’il est d’une part honeste et d’autre part deshoneste, en tel maniere k’il engendre bienweillance et haine.\par
Oscur est çou que cil ki le doit oïr ne puet entendre legierement, ou por ce k’il n’est bien sages ou k’il est travilliés, ou por çou que tes dis sont si oscur et couvert ou envelopés, k’il ne les puet bien conoistre.
\chapterclose


\chapteropen
\chapter[{.III.XVIII. Des .ii. manieres de prologue, une coverte et autre descoverte}]{\textsc{.III.XVIII.} Des \textsc{.ii.} manieres de prologue, une coverte et autre descoverte}\phantomsection
\label{tresor\_3-18}

\chaptercont
\noindent Pour les diversités des dis et des choses sont li prologue autresi divers ; et sour çou dist Tuilles que toz prologues sont en \textsc{.ii.} manieres, uns ki est apelés commencemens et uns autres ki est apelés coverture.\par
Et commencement est ce ki apertement et a poi de paroles aquiert la bienvoeillance u la volenté de ciaus qui oïr doivent.\par
Coverture est quant li parleour met en son prologue mout de paroles entor le fait, et fait samblant qu’il ne weille çou k’il vieut, pour aquerre covertement la bienvoeillance de ciaus a qui il parole. Et por ce nos covient savoir li quex des \textsc{.ii.} prologues doit estre mis sor chascune matire de nostre conte.
\chapterclose


\chapteropen
\chapter[{.III.XVIIII. Quel prologue covient sor honeste matire}]{\textsc{.III.XVIIII.} Quel prologue covient sor honeste matire}\phantomsection
\label{tresor\_3-19}

\chaptercont
\noindent La ou nostre matire est honeste chose, il ne nos covient coverture nule. Mais tot maintenant comencier nostre conte et deviser nostre afere, por ce que la honesté de la chose a ja aquise la volenté des oïans, en tel maniere que poi nous en covient travillier. Et nanporquant aucunefois est bons uns biaus prologues, non mie por aquerre bienweillance mais por acroistre la ; et se nous volons deguerpir le prologue, il est bon a comencier premierement as bons dis et as seurs argumens.
\chapterclose


\chapteropen
\chapter[{.III.XX. Quel prologue covient sor contraire matire}]{\textsc{.III.XX.} Quel prologue covient sor contraire matire}\phantomsection
\label{tresor\_3-20}

\chaptercont
\noindent Et quant la matire est contraire ou cruel ou contre droit u ke tu viens demander une grant chose ou chiere ou estrange, lors dois tu consirer se li oïeres est escommeus contre toi, ou s’il a porpensé en son cuer k’il ne face rien de ta requeste. Car se ce fust lors te covient il fuir a la coverture, et colourer paroles en ton prologue pour abaissier son courouç et adoucier sa durece, en tel maniere que son cuer soit apaissiés et tu en aquieres sa bienweillance. Mais quant son cuer n’est gaires torblés contre toi, lors t’en poras tu passer legierement pour \textsc{.i.} po de bel comencement.
\chapterclose


\chapteropen
\chapter[{.III.XXI. Quel prologue covient sor vilh matire}]{\textsc{.III.XXI.} Quel prologue covient sor vilh matire}\phantomsection
\label{tresor\_3-21}

\chaptercont
\noindent Quant la matire est vil et petite et que li oïeres ne bee pas a ce se poi non, lors covient il que ton prologue soit adornés de teus paroles ki li donent talent d’ooïr et ki enhaucent ta matire et ostent de sa viutance.
\chapterclose


\chapteropen
\chapter[{.III.XXII. Quel prologue covient sor douteuse matire}]{\textsc{.III.XXII.} Quel prologue covient sor douteuse matire}\phantomsection
\label{tresor\_3-22}

\chaptercont
\noindent Et quant la matire est douteuse pour çou que tu demandes \textsc{.ii.} choses, et on doute de la sentence, laquele des \textsc{.ii.} doit estre fermee, lors dois tu commencier ton prologue a la sentence meisme de la chose que tu vieus, et de la raison en quoi tu te confies plus. Mais s’ele est doutense, por ce que la cose est d’une part honeste et d’autrepart deshoneste, lors dois tu aorner ton prologue por aquerre l’amour et la bienweillance des oïans, en tel maniere k’il lor samble que toute la chose soit tornee a honesté.
\chapterclose


\chapteropen
\chapter[{.III.XXIII. Quel prologue covient sor oscure matere}]{\textsc{.III.XXIII.} Quel prologue covient sor oscure matere}\phantomsection
\label{tresor\_3-23}

\chaptercont
\noindent La ou la matire est oscure a entendre, lors dois tu commencier ton conte par teus paroles ki donent as oïans talent de savoir ce que tu vieus dire, et pnis deviser ton conte selonc ce que tu quideras ke mieus soit.
\chapterclose


\chapteropen
\chapter[{.III.XXIIII. Des .iii. choses qi besoignables sont a chascun prologue qi ne puet bons estre u sans l’une u sans l’autre}]{\textsc{.III.XXIIII.} Des \textsc{.iii.} choses qi besoignables sont a chascun prologue qi ne puet bons estre u sans l’une u sans l’autre}\phantomsection
\label{tresor\_3-24}

\chaptercont
\noindent Par ces ensegnemens poons nos savoir ke en toutes manieres de prologues, sor quel ke la matire i soit, nos covient il faire une de ces \textsc{.iii.} choses. ou aquerre la bienweillance de celui a qui nous parlons, ou doner talent d’oïr nos dis ou de savoir les ; car quant nostre matire est honeste ou mervilleuse ou douteuse, nostre prologue doit estre por aquerre la bienweillance. Et quant nostre matire est vil, lors doit il estre por doner talent d’oiir. Et quant li matire est oscure, lors doit li prologues estre por doner talent de savoir çou que nous dirons. Et por çou est il bien raisons que li mestres nous die comment ce puet estre fet, et en quel maniere.
\chapterclose


\chapteropen
\chapter[{.III.XXV. Des enseignemens por aquerre bienvoeillance as oians}]{\textsc{.III.XXV.} Des enseignemens por aquerre bienvoeillance as oians}\phantomsection
\label{tresor\_3-25}

\chaptercont
\noindent Bienweillance est aquise de \textsc{.iiii.} lieus, c’est par nostre cors, et de nostre adversaire, et des oïeurs, et de par la matire meisme.\par
Du cors de nous est ele aquise se nous ramentevons nos oevres et nos dignités cortoisement, sans nul orguel et sans nul outrage ki soit. Et quant on met sor nous aucuns blasmes ou coupe ou autre meffet, se nous disons que nous ne le fesimes pas et que ce ne fu mie de par nous, et se nous moustrons les maus et les dolours et les mescheances ki ont esté et ki puent avenir a nous et as nostres, et se vostre priiere est douche et debonaire ou de pitié ou de misericorde, et se nous nos offrons debonairement as oïans par cestes et par autres samblables proprietés de nous et des nostres, est aquise bienweillance selonc ce k’a rectorike apertient.\par
Et sachiés que chascuns cors d’omme et chascune chose a sa proprieté par lesquex on puet aquerre bienweillance et malevoeillance. Et de ce dira li mestres ça avant la ou il en sera lieu et tens.\par
Par le cors de ton adversaire aquerras tu bienvoeillance se tu recontes les proprietés de lui et le met en haine ou en envie ou en despit des oïours. Car sans faille tes adversaires est en haine se tu dis que ce k’il a fet est contre droit de nature, ou par son grant orguel, ou par son fiere cruauté, ou par trop grant malice.\par
Autresi chiet il a envie se tu ramentois la force et le hardement ton adversaire, et son pooir et sa signorie et ses rikeces et ses homes et son parenté et son linage et ses amis et son trezor et ses deniers et sa fiere maniere, ki n’est pas soustenable, et k’il use tozjours son pooir et son sens en malice, et k’il se fie plus de ce ke de son droit.\par
Autresi chiet il en despit se tu moustres que ton adversaires soit niches et sans art, home perreceus et lasche, et ki n’estuide se en choses fiebles non, et k’il met tot son cuer en lecherie et en luxure et en gieuet en tavernes.\par
Par les cuers des oïeurs est aquise bienweillance se tu dis lor bones teches et les proprietés de lor bonté, et loe aus et lor ouvres, et di k’il ont tozjours coustume de fere toutes choses bien et sagement et hardiement, selonc Diu et selonc justice, et ke tu te fies d’aus et que tous li mondes en a bone creance, et ce k’il feront ore de ceste besoigne sera tozjours en ramenbrance et en example des autres.\par
Par la matire aquerras tu la bienweillance se tu dis les proprietés et les apertenances de la cose dont tu paroles, ki enhancent et enforcent ta partie, et qui confondent la partie de ton adversaire et le metent en despit. Mais ci se taist li contes a parler de la bienweillance, pour moustrer coment on donne as oïeurs talent d’oïr les dis.
\chapterclose


\chapteropen
\chapter[{.III.XXVI. Les enseignemenz por doner as oieurs talent d’oir nos diz}]{\textsc{.III.XXVI.} Les enseignemenz por doner as oieurs talent d’oir nos diz}\phantomsection
\label{tresor\_3-26}

\chaptercont
\noindent Quant tu paroles devant aucune gent ou home u feme quele qu’ele soit, u tu li envoies la letre, se tu li wes donner talent k’ele entende tes dis, por ce que ta matire est anques petite ou desprisable, tu dois dire au commencement de ton prologue ke tu conteras grans novieles ou grans choses, ou que ne samble pas creable, ou k’il touchent a tous homes ou a ciaus ki sont devant toi, ou des homes de grant renonmee ou de divines choses ou dou comun proufit ; ou se tu proumés que tu diras briement et en poi de paroles, u se tu touches au commencement \textsc{.i.} pichot de la raison en quoi tu plus te confies.
\chapterclose


\chapteropen
\chapter[{.III.XXVII. Les enseignemenz por doner talent de savoir nos dis}]{\textsc{.III.XXVII.} Les enseignemenz por doner talent de savoir nos dis}\phantomsection
\label{tresor\_3-27}

\chaptercont
\noindent Et quant tu voldras ke li oïeres ait talent de savoir ce que tu vieus dire, pour ce que la matire est oscure, u por une okison u pour autre, lors dois tu au commencement de ton conte dire la some de ta entention briefment et apertement, c’est a dire celui point en quoi est la trés grant force de toute la besoigne. Et sachés que tous homes ki ont talent de savoir, certes il ont talent d’oïr ; mais tot home ki ont talent d’oïr n’ont pas talent de savoir, et c’est la difference entre l’un talent et l’autre.
\chapterclose


\chapteropen
\chapter[{.III.XXVIII. Des prologues qi sont par coverture}]{\textsc{.III.XXVIII.} Des prologues qi sont par coverture}\phantomsection
\label{tresor\_3-28}

\chaptercont
\noindent Jusques ci a devisé li mestres comment on doit comencier son conte sans prologue ou par tel prologue ki n’ait couverture nule ; desoremais vieut il deviser comment on doit faire son prologue par mestrie et par coverture. Car a la verité dire quant la matire dou parleour est honieste ou vil u douteuse ou oscure, il s’en puet passer legierement outre et commencier son conte a poi de coverture ou sans nule coverture, selonc ce que li ensegnement devisent ce deseure.\par
Mais quant la matire est contraire et laide et que li corages des oïeurs sont commeut contre lui, lors li estuet torner a la maistral coverture, et ce puet estre par \textsc{.iii.} okisons, ou pour ce ke la matire ou çou de quoi il voet parler ne siet pas a celui ki le doit oïr, ains li desplaist ; ou por ce que ton adversaire u \textsc{.i.} autre quel k’il soit li fait entendre autres choses si k’il les croit dou tout ou de la grignor partie, ou pour ce que li oïeres est abesoigniés ou travilliés de maint autre ki ont parlé a lui devant.
\chapterclose


\chapteropen
\chapter[{.III.XXVIIII. Comment on doit commencier son prologue quant la matire desplest as oians}]{\textsc{.III.XXVIIII.} Comment on doit commencier son prologue quant la matire desplest as oians}\phantomsection
\label{tresor\_3-29}

\chaptercont
\noindent Et se ce est lee ta matire desplaise, il te covient covrir ton prologue en tel maniere, que se ce est cors d’ome ou autre chose ki li desplest ou k’il n’aime pas, tu t’en tairas, et nommeras \textsc{.i.} home u autre chose qui li soit agreable et amable a lui, si comme Catelline fist quant il noumoit ses ancestres et lor bones oevres devant les signatours quant il se voloit covrir de la conjurison de Rome, et quant il lor disoit que çou n’estoit mie por mal mais por aidier les febles et les nonpoissans, si comme il avoit acoustumé tousjours, ce disoit.\par
Et ensi dois tu faindre ta volenté, et en lieu de l’ome ki desplaist ramentevoir \textsc{.i.} autre home ou autre chose ki soit amee ; et en lieu de la chose ki est laide nomeras \textsc{.i.} bon home u une bone chose plaisable, et en tel maniere que tu retraies son corage, de çou ki ne li siet mie a ce ki li doit plaire. Et quant ce sera fait tu te dois faindre ke tu ne voeilles pas ce ke l’on quide ke tu voeilles ou ke tu ne deffendes mie çou que tu vieus deffendre, selonc çou ke Julle Cesar fist quant il volt deffendre les conjurés.\par
Lors commence auques a endurcir les coers des oïeurs, et tu dois maintenant entrer poi a poi a touchier ta entention, et moustrer que tous ce ke plaist as oïans plest a toi, et ce que lor desplaist ne te soit pas a gré. Et quant tu auras apaiet celui a qui tu paroles, tu diras que de cele besoigne a toi ne tient ne ce ne quoi, c’est a dire que tu ne li fesis mie le mal que \textsc{.i.} autre li fist.\par
Ce dist la premiere amie Paris en ses letres k’ele li envoia, puis k’il l’avoit deguerpie por l’amour Elaine : je ne demande, fist ele, ton argent ne tes joiaus por loier de mon cors ; et ce vaut autant comme s’ele desist, tout çou te quist Elaine.\par
Aprés ce dois tu niier que tu ne dies de lui ce meisme que tu en dis, selonc ce que Tulles dist contre Verres, je ne dirai, fist il, que tu ravisses le catel tes compaignons, ne ke tu desrobaisses maisons et viles ; et ce vaut autant a dire comme s’il desist, tout çou as tu fet.\par
Mais tu te dois mout garder que tu ne dies ne l’un ne l’autre, en tel maniere ke soit descovertement, contre la volenté des oïans ou contre ciaus k’il ayme, ains soit si covertement k’il meisme ne s’en aperçoive, et que tu eslonges son cuer de ce k’il avoit proposé, et le conjoingnes a ton desirier. Et quant la chose iert a ce venue, tu dois ramentevoir \textsc{.i.} essample samblable, ou proverbe ou sentence ou auctorité des sages homes, et mostrer ke ta besoigne soit dou tout resemblable a celui.\par
Selonc ce que Catons dit contres les conjurés : je di, fist il, ke ancienement Mallius Torquatus dampna son fil a mort, por ce k’il avoit combatu contre le commant de l’empire ; autresi doivent iestre dampné cil conjuré ki voloient Rome destruire, car il ont pis fait que celui.
\chapterclose


\chapteropen
\chapter[{.III.XXX. Comment on doit commencier son prologue quant li oieur croient a son adversaire}]{\textsc{.III.XXX.} Comment on doit commencier son prologue quant li oieur croient a son adversaire}\phantomsection
\label{tresor\_3-30}

\chaptercont
\noindent Et quant celui a qui tu paroles croit ce que ton adversaire ou autre home li avoit fait entendant, lors dois tu au commencement de ton conte proumetre que tu vieus dire de ce meismes en quoi li adversaires se fie plus, meismement de ce ke li oieur avoient creu, u tu commences ton conte a une des raisons ton aversaire, meismement a celi k’il dist en la fin de ton conte ; ou tu dis que tu ies en doute comment tu dois commencier, ne a quoi ne comment tu dois respondre, et faire samblant autresi comme d’une merveille. Car quant li oïeres voit ke tu ies fermement apareilliés de contredire la u tes adversaires te quident avoir tourblé, il quidera k’il ait folement creu et que li drois en soit devers toi.
\chapterclose


\chapteropen
\chapter[{.III.XXXI. Comment on doit commencier son prologue quant li oieres est travilliés u enbesoigniés}]{\textsc{.III.XXXI.} Comment on doit commencier son prologue quant li oieres est travilliés u enbesoigniés}\phantomsection
\label{tresor\_3-31}

\chaptercont
\noindent Mais se li oieres est travilliés ou enbesoigniés de maint autres parleours devant, lors dois tu avant promettre que tu ne diras se poi non et que ton conte sera plus brief ke tu n’avoies pensé, et que tu ne vieus ensivre la matire des autres ki parolent longuement. Et ancunefois dois tu commencier a une novele chose u ki le face rire, mais qu’ele soit apertenant a ton conte u a une fable ou a \textsc{.i.} essample ou a autre parole pensee u non pensee ki soit de maniere de ris u de sollas.\par
Mais se la cose est decorous, lors est il bon a commencier a une dolereuse novele ou autre orrible parole. Car si comme li estomas chargiés de viandes est relevés par une amere chose ou asouagiés par une douce, tout autresi li corages ki est travilliés de trop oïr est renovelés ou par merveilles ou par ris.\par
Mais ci se taist li contes a parler des prologues ki sont par coverture ou sans coverture, car il en a dit partiement tous les ensegnemens de l’un et de l’autre par soi ; mais or vieut il moustrer les communs ensegnemens de chascun ensamble.
\chapterclose


\chapteropen
\chapter[{.III.XXXII. Les ensegnemens de tous prologues ensamble}]{\textsc{.III.XXXII.} Les ensegnemens de tous prologues ensamble}\phantomsection
\label{tresor\_3-32}

\chaptercont
\noindent En tous prologues, de quel matere k’il soit, dois tu metre, ce dist Tuilles, assés de bons mos et de bones sentences. Et par tout doit il estre garni d’avenableté, pour çou que sor toutes coses te covient a dire ce ki te mete a la grace des oïans. Mais il doit avoir petit de doreure et de jeu et de consonances, por ce que de teus choses naist souvent une suspection comme de chose pensee par grant mestrie, en tel maniere ke li oïeres se doute de toi ne ne croit pas a tes paroles.\par
Et certes, ki bien consire la matire dou prologue, il trovera k’il n’est pas por autre cose ke pour apareillier le corage celui a qui tu paroles a oïr diligement tes dis et croire les, et k’il face a la fin ce que tu li fais entendant ; pour quoi je di k’il doit estre garnis de moz creables et de sentences, c’est a dire des ensegnemens as sages homes ou de proverbes ou de bons essamples. Mais que trop n’en i ait car il ne doit estre dorés de losenges u de mos coviers, si k’il samble une chose pensee felenessement et par malisce, et si ne dois tu dire trop de paroles de jeu et de vanité, mais fermes et de bon savoir ; et garder k’il n’i ait consonances, c’est a dire plusours mos ensamble l’un apriés l’autre, ki tot commencent u fenissent en une meismes letres et en une meisme sillabe, car c’est une laide maniere de conter.
\chapterclose


\chapteropen
\chapter[{.III.XXXIII. Des .vii. visces dou prologue et premierement du general}]{\textsc{.III.XXXIII.} Des \textsc{.vii.} visces dou prologue et premierement du general}\phantomsection
\label{tresor\_3-33}

\chaptercont
\noindent Apriés les vertus dou prologue est il bien raisnable chose de traitier des vices, ki sont \textsc{.vii.}, ce dist Tuilles, generaus, communs, muables, lons, estranges, divers, et sans ensegnement.\par
Generaus prologues est cil que l’on puet metre en maint conte avenablement. Comuns est cil que ton adversaire puet ausi bien metre comme tu. Muables est cil ki par poi de remuance sera bons a ton adversaire. Lons est celui u il a trop de paroles ou de sentences outre çou ki est covenable. Estranges est cil ki en nule maniere dou monde n’apertient a ta matire. Divers est cil ki fait autre cose que ta matire requiert, c’est ke la u tu dois aquerre la bienweillance tu ne le fais pas, ains donnes talent d’oïr u de savoir ; ou quant tu dois parler par coverture, tu paroles tout descoviert. Sans ensegnement est cil ki ne fait nient de ce que li mestres ensegne, ne n’aquiert bienweillance ne talent d’oïr ne de savoir, ains enquiert tout le contraire, ke pis vaut.\par
De toz ces visces nous covient il garder fierement et ensivre les ensegnemens, en tel maniere que nul salut ne nule partie dou prologue soit blasmable, mais tout soit agreable et de bonne maniere.
\chapterclose


\chapteropen
\chapter[{.III.XXXIIII. Ci met example por mieus moustrer ce qi est dit devant}]{\textsc{.III.XXXIIII.} Ci met example por mieus moustrer ce qi est dit devant}\phantomsection
\label{tresor\_3-34}

\chaptercont
\noindent Or avés oï les ensegnemens ki apertienent as prologues et comment li parleour doivent commencier lor conte, selonc la diversité des matires, ki avienent tousjours as besoignes dou siecle, mais por ce ke li mestres vieut plus apertement moustrer ce k’il a dit de grant auctorité, sor quoi fu dit par plusours sages.\par
Et il fut voirs, lors que Catelline fist en Rome la conjuroison selonc ce ke les istoires dient, Marcus Tullius Cicero, celui meismes ki ensegne la science de rectorike, estoit adonc consilliers de Rome, ki par son grant sens enquist et trova la conjuroison et prist plusors des conjurés, meismement des puissans homes de Romme.\par
Et quant il les ot mis en chartre, et ke la conjuroison fu descoverte et seue certainement, Marcus Tullius apela tous les signatours et le conseill de Rome por jugier que on feroit des prisons. Salustes dit ke Decius Sillanus, uns nobles signator ki estoit esleus a estre consoles l’an apriés, dist premiers sa sentence, ke li prisonier devoient estre livré a mort, et il et li autre ke l’en puet penre. Et quant il ot son conte finé et ke tout li autre pour poi s’acordoient a sa sentence, Julle Cesar ki voloit les prisons deffendre parla par coverture maistrielement en ceste maniere :
\chapterclose


\chapteropen
\chapter[{.III.XXXV. Le dit julle cesar}]{\textsc{.III.XXXV.} Le dit julle cesar}\phantomsection
\label{tresor\_3-35}

\chaptercont
\noindent Segnour peres escris, tout cil ki welent droit conseil doner des choses douteuses ne doivent esgarder a ire ne a haine n’a amour n’a pitié, car ces \textsc{.iiii.} choses puënt faire l’omme laissier la voie de droiture et desvoier de droit jugement. Sens ne vaut riens la u on vieut dou tout sivre sa volenté.\par
Je poroie assés princes noumer ki droite voie laissent a tenir pour ce ke ire les avoit sorpris u pitiés sans raison. Mais je voeil mieus parler de çou que li sage home de ceste cité ont fait aucunefois quant il laissoient la volenté de lor cuers et tenoient ce que bons ordenes ensegnoit et ki tornoit au comun proufit.\par
La cité de Rodes se tint contre nous en bataille ke nous eumes contre Persé le roi de Macedoine. Quant la bataille fu finee, li senat et li consolet jugierent que cil de Rodes ne fussent pas destruit pour ce que nus ne desist que covoitise de lor richece les fesist destruire plus ke l’ochoison de lor tort.\par
Cil de Cartages nos forfirent jadis en guerres que nous eumes contre ciaus d’Aufrike et brisierent trives et pais neporquant nostre mestre ne garderent pas a ce k’il devoient faire d’iaus ; car cil les peussent bien destruire, ains les retint douçours et debonairetés. Ce meismes devons nous porveir, signor peres, ke la felonie et les forfés de ciaus ki sont pris ne sormonte nostre dignité ne noustre douçour, plus i devons regarder nostre bonne renomee que nostre courous.\par
Cil ki ont avant moi sentence donnee ont assés belement moustré ce ke puet de mal avenir par lor conjuroison, cruauté de bataille, prendre puceles a force, esrachier les enfans des bras as peres et as meres, fere force et honte as dames, despoillier temples, ocire gens, et maisons ardoir, et emplir la cité de caroignes et de sanc et de plour. Et de ce ne covient il ja parler, car plus puet movoir les cuers la cruauté de tel meffait ke li recort de l’oevre. Nus n’est qui il ne griet de son damage, teus i a ki le portent plus grief ke mestiers n’est, mais il loist a \textsc{.i.} ce ki ne loist mie a \textsc{.i.} autre. Se je sui uns bas hom et je mespreng aucune chose par mon courous, poi le sauront, mais tout sevent çou ke uns grans hom mesprent, ou en justice ou en autre chose. Que quant uns bas hom mesprent, on le torne a ire, le forfait a \textsc{.i.} grant home on le torne a orguel, por ce devons nous garder nostre renomee\par
Et si di je bien en droit de moi ke le forfet as conjurés sormonte toute paine. Mais quant on tormente aucun home, se li tormens est auques aspres, tex i a ki bien sevent blasmer le torment, et du meffait ne tienent nule parole. Je croi bien que Decius Sillanus a dit ce qu’il a dit por bien dou commun, et k’il n’i regarde ne amour ne haine, tant connois je de ses meurs et sa atemprance. Ne sa sentence ne m’est pas cruel, car on ne poroit fere nule cruauté en teles gens. Mais totefois di je que sa sentence n’est pas covenable a nostre commun ; par quoi Sillanus est fort home et nobles esleus a consoles, il les a jugiés a mort pour paour del mal ki en puet avenir ki les lairoit vivre. Paour n’a ci point de liu, car Ciceron nostre consoles est si discrés et si garnis d’armes et de chevaliers ke nous ne devons rien douter.\par
De la paine dirai je si comme il est : se l’en les ocit, mors n’est pas tormens, ains est fins et repos de toute chetiveté, mors consome toutes paines terrienes, aprés la mort ne cures ne joies. Por quoi ne dist Sillanus se viaus que l’en les batist et tormentast tot avant ? Se aucune lois deffent que l’on ne fustast home jugié a mort, aucunes lois redient que l’om n’ocie pas citain dampné, ains les envoie on en essil a tozjours.\par
Segnour peres escris, gardés que vous faites, car on fait sovent teus choses por bien dont maus avient puis. Quant li lacedoniens orent pris Atenes, il establirent \textsc{.xxx.} homes ki estoient mestre del commun ; cil ocioient au commencement les piesmes desloiaus homes tot sans jugement, li peules en estoit liés et disoient ke bien faisoient. Aprés crut la coustume et la licence petit a petit, si k’il ocioient bons et mauvais a lor volenté, tant que li autre en estoient espoenté, et fu la cités en tel servage que bien s’aperchurent ke lor joie lor revertissoit en plour.\par
Luce Sillaire fu mout loés de ce k’il juga et ocist Damacipe et autres ki avoient esté contre le commun de Rome ; mais cele chose fu commencement de grant mal, car apriés si comme chascuns covoitoit la maison la vile le vaissel ou la robe d’autrui, il se penoit de dampner celui la qui cose il voloit avoir, et estoient maint home dampné a tort por lor avoir. Ensi cil ki furent liet de la mort Damacipe en furent puis couroucié. Car Cilla ne fina en ceste maniere d’ocire juskes la que tot si chevalier fureint plain d’avoir.\par
Neporquant de teu chose n’ai jou doute en cest tens meismement tant comme Marcus Tullius Cicerons est consoles, mais en si grant cité a mains divers homes et plain d’engien. El tens as autres consoles poroit aucuns metre avant faus por voir, et se li consoles ocioit, lors, por le decré del senat, home encoupé a tort, donc en poroit nul mal avenir.\par
Cil ki furent devant nous orent sens et hardement ; orgueil ne lor toli pas, k’il ne presissent bien example de raison es estranges ; quant il trovoient loins en lor enemis aucune teche, il savoient bien metre a oevre a lor ostel : mieus amoient sivre le bien que avoir envie, il fustoient le citein forfet a la guise de Grece. Quant li mal commencierent a monter, lors furent lois donnees que li dampnés alaissent en essil. Porquoi donques prendons nous conseil noviel ? ensi le fisent nostre antecissours, et grant vertu et sapience ot plus en aus que en nous, car il estoient poi et ce aquisent a poi de rikeces que nous poons a paines tenir et garder.\par
Que ferons nous donques ? laisserons nous ces prisons aler por acroistre l’ost Catelline ? nenil ; ains est ma sentence que lor avoirs soit publiiés et mis en aoust, lor cors soient mis en diverses prisons fors de Roume en forteresces bien garnies, ne nus ne parolt ja por aus au senat ne au peuple : ki autrement le fera soit mis en prison comme \textsc{.i.} de ceus.
\chapterclose


\chapteropen
\chapter[{.III.XXXVI. Comment cesar parla selonc cest art}]{\textsc{.III.XXXVI.} Comment cesar parla selonc cest art}\phantomsection
\label{tresor\_3-36}

\chaptercont
\noindent Sor ces sentences poés vous entendre ke li parleour, c’est Decius Sillanus, s’en passa briement a poi de paroles sans prologue et sans coverture nule, pour ce que sa matire estoit de honeste chose, si comine de livrer a mort les traitors dou commun de Rome. Mais Julle Cesar qui repensoit autre chose se torna as covertures et as moz dorés, por ce ke sa matire estoit contraire, car il savoit bien ke li cuer des oïeurs estoient commeu contre sa entention, et por ce li covient il aquerre la lor bienweillance ; et d’autrepart restoit sa matiere doutouse et oscure par plusours sentences et couvertures k’il voloit consillier, et sor ce li estevoit il doner as oïeurs talent de savoir et d’oïr ce k’il voloit dire.\par
Mais pour ce que doreure de paroles est auques souspecenouse, ne se volt il au commencement sevrer de la bienweillance aquerre, ains toucha la some de sa entention por doner as oïans talent d’oïr et d’entendre ses dis, la u il dist des \textsc{.iiii.} choses que bon conseilleour doivent garder.\par
Et neporquant sans bienweillance ne fu pas son prologue, la u il les apela segnor peres escris, et la u il enhaucha sa matire et le conferma par beles raisons et par examples des vielles istores k’il amentut. Ensi tot belement, en lieu de la chose ki desplaisoit, noma il choses ki deussent plaire, por retraire les corages des oïeurs de ce que lait estoit, a ce que fust honeste et raisnable.\par
En ceste maniere se passa a dire le fait sor quoi il voloit fonder son conte, c’est dou conseil ki doit estre pris sor le meffait des conjurés, et fist samblant k’il ne volsist pas deffendre lor mal, mais il voloit garder la dignité et l’onour del senat.\par
Lors commença la tierce partie de son conte, c’est devisement, et devisa les dis des autres et la cruauté dou forfait par parties, et aqueilli cele partie ki plus li aidoit contre ciaus ki avoient parlé, et les afficha as cuers des oïeurs tant comme il onques pot.\par
Et quant il ot ensi conté le fait, il commença la quarte partie de son conte, c’est confermement, la u il dit k’il devoient garder lor renomee ; et se fainst de loer les sentences des autres, mais mout les blasme ; et sor ce conferma son conte par maintes raisons ki donoient foi a son conseil et l’acueilloit a la sentence des autres.\par
Et puis k’il ot fermé son conte par les bons argumens, il ala maintenant a la quinte partie, c’est au deffermement, por affoibloier et pour destruire le confermement des autres ki avoient dit devant lui, la u il dit, gardés ke vous faites. Et maintenant ramentut plusors examples et auctorités et sentences des sages homes ki estoient samblables a sa matire ; et puis quant vint vers la fin il conferma ses dis par les millors argumens et par les plus fortes raisons k’il onques peut.\par
Et vint a la sisime partie c’est a conclusion, et dist sa sentence et mist fin a son conte. Et puis ke Cesar ot ensi parlé, li un disoient \textsc{.i.}, li autre disoient autre, tant que Marcus Catons se leva et parla en ceste maniere :
\chapterclose


\chapteropen
\chapter[{.III.XXXVII. Le jugement caton}]{\textsc{.III.XXXVII.} Le jugement caton}\phantomsection
\label{tresor\_3-37}

\chaptercont
\noindent Segnour peres escris, quant je regart la conjurison et le peril, et contrepois en moi meismes la sentence de cascun ki a parlé, je pense tot autre chose que Cesar n’a dit ne aucun des autres. Il ont parlé tant solement de la paine as conjurés ki ont apareillie bataille a lor païs et a lor parens, a lor temples et a lor maisons destruire. Mais graindre mestiers est prendre garde comment on se pora garder d’iaus et de peril ke prendre conseil comment il soient livré a paine et dampnés. Se l’om ne se porvoit que cis periz n’aviegne, por noient iroit on a conseil quant il sera avenu. Se la cités est prise a force, li vencu n’ont rien d’atente, tot sera en la mine. Or parlerai a vous, ki bien entendés raison et ki baés a avoir maisons, viles, ensegnes, et tables d’or et d’argent, plus que au preu du commun.\par
Se vos ces choses que vous tant amés volés garder et retenir et vous volés maintenir par ordre vos delis et par repos, esveilliés vos ci et pensés del commun garantir. Se li communs perist comment escaperois vos ? Ceste besoigne n’est pas de tolleu ne de paage, ne de queriele de compaignons, ançois est de nostre franchise deffendre et de nos cors ki sont en peril.\par
Segnour, je ai maintesfois parlé et meu plaintes par devant vos de l’avarice, de la luxure, de la covoitise, a vos citeins, dont j’ai la malevoeillance d’aucuns, car je ne pardone pas volentiers autri le meffet dont je ne sentoie nule teche en moi. Et de nul forfait pardoner, je ne querroie autrui grasce avoir. S’il ne vos caloit de ce, et nostre richesce nos faisoit maintes choses metre en nonchaloir, toutefoies estoit li communs en droit estat et plus fors c’ore.\par
Mais ci endroit ne parlons pas de nostre bienvivre ou de nostre malvivre, ne de la signorie as romains acroistre et essauchier, ains nous covient veoir se ce ke nos avons nos puet remanoir et estre nostre, u il sera nos enemis. Ci ne doit nus parler de debonaireté ne de misericorde, nos avons pieç’a perdut le droit non de pitié et de merci ; car doner autrui biens c’est nostre debonairetés, estre osés de mal faire c’est nostre vertus, por ice vet nostre communs ausi comme a declin.\par
Or poés donc estre debonaires et metre le pule en aventure, or poés estre piteus en ciaus ki ne nos quident rien laissier et quident le commun trezor rober. Donés lor nostre sanc, si ke tot li preudome aillent a destruction ; et en ce que vos espargniés \textsc{.i.} po de malfaiteurs, si destruisiés une grant torbe de bone gent.\par
Cesar a parlé biel et afaitiement, oïant nous, de la vie et de la mort, quant il dist, aprés la mort ne cure ne joie ; mais quant il en parla ensi, je croi k’il quide a faus ce que l’on trueve de ciaus d’enfier, li mauvés sont desevré des bons et entrent en noirs lieus et en oribles puans et espoentables. Aprés il juga que lor avoirs fust publiés et il fesissent gardes en prison et en diverses fortereces fors de Rome : doutoit il que s’on les gardoit a Rome, que cil de la conjuroison u autres gens loees les getassent a force de prison ? N’a il donc male gent se en ceste cité non ? partout puet on trover des mauvais homes.\par
De noient se redoute Cesar se il crient que on ne les puisse garder en Romme ausi bien comme dehors. Et s’il seul n’a pas paour k’il escapaissent de cele prison u il dist k’il soient, et il seul ne crient pas le peril dou commun, je sui cil ki ai paour et de moi et de vous et des autres. Pour ce vous devés savoir ke ce ke vos jugerois de ces prisoniers, ce doit estre jugié de tote la compaignie Catelline. Et se vos fetes de ces aspre justice, tuit cil de l’ost Cateline en seront espoenté. Se vous le fetes foiblement et mollement, vos les verrés venir crueus et fiers contre nous.\par
Ne quidés pas que nostre antecessors aient acreu la signorie du commun tant solement par armes ; car s’il alast ensi, donc le peussons nous enmillorir, car plus avons nos signorie et compaignons et grignor habondance de chevaus et d’armes k’il n’orent. Mais il eut en aus autre chose par quoi il furent de grant pris et de renon, et ces choses ne sont gaires en nous. Il estoient en lor osteus sage et apercevant, et donoient droituriers commandemens a ceaus dehors. Lor cuer estoient franc esperit et delivre a donner conseil sans subjection de pechié k’il creinssissent ne sans sivre mauvaise volenté.\par
En lieu de ce on puet trover en nous luxure, et avarisse, commune poverté, et privee richesse. Nos loons les richesces, nous ensivons les perreces, nous ne faisons nule difference des bons ne des mauvais, tot est torné a covoitise, c’est li loiers de vertu. Et ce n’est pas merveille, car chascuns tient sa voie et son conseil par soi meismes ; vos servés en vos osteus a vos delis et entendés a vos volentés sivre ; fors de vos osteus servés a avoir amasser ou a grasce d’autrui conquerre, de ce avient que l’en guerroie le commun et ke li conjuret le welent destruire\par
Més de ces choses que vos en tel maniere faites ne vous dirai ore plus. Nobles citains ont ensamble jurés k’il ardront la vile, et atraient avec aus por movoir bataille la gent de France, ki pas n’aiment la signorie ne le non de Rome. Catelline li dus de nos enemis nous vient sor nos testes avec tot son effort. Demourés vos donc et doutés que vos devés faire de nos enemis que vos avés pris dedens ces murs ? Or soit que je juge que vos en aiés merci, dites ke joene home sont par folie, et par mauvaise covoitise l’ont fet, et les laissiés aler toz armés : mais certes je croi ke ceste pitié et ceste douçours vos tort a misere et a amertume.\par
La chose est aspre et perilleuse, n’en avés vos cremour ? Oïl voir ! Mais la perrece, la malvestiés, la molestes de vos cuers fait que li uns s’atent a l’autre. Vos metés vostre fiance en vos dieus, et dites k’il ont le commun garde et delivre de maint peril. La aide de Dieu ne vient pas a la volenté de ciaus ki welent vivre comme femes, mais a totes choses aident a ciaus ki voelent veillier en bien faire et en doner bon conseil. Por noient apele Dieu ki s’abandone a perrece et a malvestié. Mallius Torquatus, uns de nos anciens dus, commanda a ocire son fil por tant solement k’il envaï en une bataille de France ses enemis contre son commandement : por tel forfet morut ce noble jouvenciel, et vos demourés a fere justice de ces crueus paricides ki welent la cité destruire.\par
Laissiés les vos por lor bonne vie ; ne muire pas Lentulus por le dignité de son linage, s’il ama onques chasteté, s’il ama onques bonne renomee, s’il ama onques Dieu, s’il espargna onques home ; ne muire pas Chetegus, ait on pitié de sa joventé, s’il ne meut onques noise ne bataille en ce païs. Gabinus et Statillius et Separius, quex sont il, qu’en dei jou dire ? s’il eussent en aus raison ne mesure, il n’eussent pas tel conseil pris contre le commun.\par
Au derrain vos di, signour peres, ke por Dieu se moi en leust eschaper, je vos en laissasse bien covenir et soufrisse bien que vous en fuissiés chastoié par leur outrage, quant vous conseil n’en volés croire. Mais pour ce le di ke nos somes enclos en grant perils de toutes pars, Catelline a toute son ost nous est es oils la dehors et nos quide engloutir, li autre sont en mi ceste vile partout, nous ne poons rien apareillier ne consillier que nostre ennemi ne sachent, dont nous nous devons plus haster.\par
Et pour ce dirai ceste sentence : voirs est que le commun est en peril par le maudit conseil des citeins escomeniiés et desloiaus ; cist ont regehi et sont convencu pour le dit des messagiers de France et de Tite Vultier, k’il voloient la vile ardoir, ocire le millour, le païs destruire, dames et puceles honir, autres cruatés faire ; por ce doins sentence et jugement ke l’en doit faire d’iaus comme de traiteurs et d’omecides atains.
\chapterclose


\chapteropen
\chapter[{.III.XXXVIII. Comment catons parla selonc cest art}]{\textsc{.III.XXXVIII.} Comment catons parla selonc cest art}\phantomsection
\label{tresor\_3-38}

\chaptercont
\noindent Ce est li contes et la sentence Caton. Mais por mieus entendre son dit et comment il parla selonc les riules de rectorike nous covient il a garder tot avant la maniere de son dit et la nature de sa matire : dont li plusor dient k’ele est douteuse et \textsc{.i.} poi oscure, por ce que sa matire est d’une part honeste et d’autre par deshoneste.\par
Car a dire le profit dou commun et deffendre le bon estat de Rome et a destruire tous traiteurs c’est honeste chose, et a jugier a mort une gent de nobles citeins et a dire contre Cesar ki avoit ensi establi fierement son jugement par si bones raisons ke a paine le peust hom contredire et ke li oïant estoient auques commeus a croire son dit, certes ele sambloit a estre crueus chose et mervilleuse ; et por ce li estavoit il aorner son prologue en tel maniere que il acuellist la bienvoeillance des oïans et k’il lor donast talent de savoir ce k’il voloit dire por retraire les de la sentence Cesar, selonc ce que li mestres devise ci derieres la u il ensegne la diversité des prologues.\par
Et por ce toucha en son comencement briement et apertement le point u estoit la grant force de toute la besoigne, meismement celui que li oïeur avoient creu, quant il dist k’il pensoit tote autre chose que Cesar n’avoit dit ne aucuns des autres. Ensi lor dona talent de savoir et d’oïr ce k’il diroit, et fist samblant de consillier solement de la garde du commun, non pas de la mort as conjurés.\par
Et maintenant se porchaça d’aquerre l’amour et la bienwellance des oïans, por apaisier lor cuers, por tourner toute la chose a honesté, et por acroistre la bienweillance k’il avoit de cele part que la matire estoit honeste ; selonc ce que li bon entendour pora savoir et connoistre s’il consire et garde diligement les ensegnemens ki sont ça ariere. Por ce s’en taist ore li mestres, car il volra dire autre doctrine bones et profitables.
\chapterclose


\chapteropen
\chapter[{.III.XXXVIIII. De la seconde branche dou conte, c’est le fait}]{\textsc{.III.XXXVIIII.} De la seconde branche dou conte, c’est le fait}\phantomsection
\label{tresor\_3-39}

\chaptercont
\noindent Après la doctrine des prologues vient la secunde branche dou conte, ce est le fait. De quoi Tuilles dit que le fait est quant li parleour dist les choses ki ont esté, ou celles ki n’ont pas esté autresi comme, se eles n’eussent esté ; c’est a dire quant il laisse les prologues et vient au fait et dist les propres choses sor quoi est l’achoison et la matire de tot son conte.\par
Et ceste en \textsc{.iiii.} manieres, une citeins ki dit tot proprement le fait et la chose de quoi est li contes et la questions, et devise les raisons porquoi cele chose puet estre provee ; et ceste maniere apertient a celui art, pour çou k’ele ensegne tencier l’un parleur contre l’autre, selonc ce que li livres dit ça ariere entor le commencement. Mais ci n’en dira ore plus, car il en dira larghement ça avant, ains vieut ichi dire des \textsc{.ii.} autres manieres dou fait, ki n’apertient pas si povrement a cestui art.
\chapterclose


\chapteropen
\chapter[{.III.XXXX. Dou conte qui trespasse hors de sa matire}]{\textsc{.III.XXXX.} Dou conte qui trespasse hors de sa matire}\phantomsection
\label{tresor\_3-40}

\chaptercont
\noindent La seconde maniere do fet dire est quant l’en se desevre \textsc{.i.} poi de sa propre matire et trespasse a autre chose hors de sa principale cause, ou por blasmer le cors u la cause, u por acroistre le mal u le bien k’il dist, ou pour moustrer que \textsc{.ii.} choses resamblables entres lor, ou por faire deliter les oïans d’aucun gabois ki soit apertenans a sa matire. Et ceste maniere dou fait dire usent sovent li parleur por mieus prover ce k’il welent dou cors et de la chose.
\chapterclose


\chapteropen
\chapter[{.III.XXXXI. Dou conte qi est par geu ou par envoiseure}]{\textsc{.III.XXXXI.} Dou conte qi est par geu ou par envoiseure}\phantomsection
\label{tresor\_3-41}

\chaptercont
\noindent La tierce matire dou fait dire n’apertient pas as choses citaines, ains est por gieu et por solas ; et nanporquant il est bone chose que l’on s’i acoustume a bien conter, car on devient mieus parlant en grant besoignes, et por ce en dira li mestres toute la nature.\par
Tulles dit que ce ke l’en dit en ceste derreniere maniere, ou il devise les proprietés et les meurs dou cors, u il devise les proprietés d’un autre chose. Et s’il devise les proprietés d’une chose, il covient a fine force que son dit soit fable ou istore ou argument ; et por ce fait il bon a savoir ke monte a dire l’un et koi l’autre.\par
Et certes fables est \textsc{.i.} conte ke l’en dit des choses ki ne sont pas voires ne voirsamblables, si comme li fable de la nef ki vola parmi l’air longuement. Istores est de raconter les ancienes choses ki ont esté veraiement, mais eles furent devant nostre tens, loins de nostre memore. Argumens c’est a dire une chose fainte ki ne fu pas, mais ele pot bien estre, et le dist hom por resamblance d’aucune chose, et se li parleour devise les proprietés dou cors ; et il covient que par ses dis reconnoisse les meurs et les proprietés dou cors et del corage ensamble, c’est a dire s’il est vieus u joenes u s’il est cortois u vilains, u des autres teus proprietés.\par
A cestes choses covient avoir grant aournement, et ki soit formés de la diversité des choses, et de la dessamblance des corages, et de fierté et de debonaireté, d’esperance, de paour, de suspection, de desirier, de fainture, d’erreur, de misericorde, de remuement de fortune, de periz ke l’om ne quidoit, de soudaine leece, et de bone fin, selonc ce ke cis livres devisera ça avant, la u il ensegnera prendre les aornemens et la biauté des paroles Por ce n’en dist ore plus ke dit en a, ains tornera a la premiere maniere dou fet dire, ki est apelee citeine.
\chapterclose


\chapteropen
\chapter[{.III.XXXXII. Dou conte qi est es choses cytaines}]{\textsc{.III.XXXXII.} Dou conte qi est es choses cytaines}\phantomsection
\label{tresor\_3-42}

\chaptercont
\noindent Or dist li contes ke la citeine maniere dou fet dire, ki devise la cause proprement, doit avoir \textsc{.iii.} choses, k’ele soit brief, k’ele soit clere, et k’ele soit voirsamblable. De tous dira li mestres, et premierement de la brieté.
\chapterclose


\chapteropen
\chapter[{.III.XXXXIII. De conter le fet briement}]{\textsc{.III.XXXXIII.} De conter le fet briement}\phantomsection
\label{tresor\_3-43}

\chaptercont
\noindent Tuilles dist que lors est li fais contés briement quant li parleors se commence a droit commencement de sa matire, non pas a une longue conmençaille ki noient proufite a son conte. Si comme fist Salustes quant il volt raconter l’istore de Troie ; il commença a la creation du ciel et de la tiere, mais bien li soufissoit commencier a Paris quant il ravi Helaine.\par
Ausi est il briés se, la u il est assés a dire la some del fait, il ne le devise par parties ; car il soufist bien a dire ensinc, cist hom tua cel autre, ne di pas, il le prist, il traist le cotiel de sa gaaine, il le leva et fist ensi et ensi une chose et autre. Car plusors fois est il assés a dire ke fait est, non pas coment et en quel maniere.\par
Autresi est il briés se l’en ne dist plus des choses que mestier ne soit dou savoir, et s’il ne trespasse a dire autres choses estranges et ki de rien n’apertient a sa matire, et s’il ne redist ce ke l’en puet entendre par ce k’il a dit. Et se tu dis, il aloient la u il pooient, il n’estuet pas dire, il n’aloient la u il ne pooient. Et se je di, Aristotles dit tel chose, il ne covient que je die, il le dist de sa bouche, car bien le puet chascuns entendre par ce ki estoit dit devant.\par
Autresi est il briés s’il ne raconte ce ki li puet aidier ne ce ki puet anuier, ou ce qui ne li puet aidier ne enuier ; et s’il dist chascune chose une fois sans plus, et s’il ne recommence sovent a la parole meismes k’il a dit maintenant. Et si comme li parleours se doit garder de la multitude des moz et des paroles, se doit il garder k’il ne die trop des choses ; car il i a mainte gent ki sont deceu ; et la u il s’estudient de poi dire, il dient trop longuement, por ce k’il se porchacent de dire plusors choses a poi de paroles ; mais il ne s’efforcent de dire poi des choses, tant comme li besoignent et non pas plus.\par
Raison coment : tu quideras ja briement dire se tu dis en tel maniere, je alai chiés vous et huchai vostre garçon ; il respondi ; je quis de vous ; il dist que vous n’i estiés pas. Ja soit ce ke tu dies briés moz, neporquant tu recontes plus des choses ke mestiers n’est, car assés estoit a dire, on me dist ke vous n’estiés pas en maison. Pour ce se doit bien garder chascuns ke sans les briés moz il ne die tant des choses que son conte en soit lons ne anuieus a escouter.
\chapterclose


\chapteropen
\chapter[{.III.XXXXIIII. De conter le fet entendablement}]{\textsc{.III.XXXXIIII.} De conter le fet entendablement}\phantomsection
\label{tresor\_3-44}

\chaptercont
\noindent Apriés ce doit li parliers estudiier k’il die clerement ce k’il dist et ke si dit soient ouvert et entendable. Tuilles dist ke li fais est contés clerement quant li parleour et li ditteour commence son dit a ce ki devant a esté et s’il ensiut l’ordene de la chose et de la saison tot ensi comment il fu ou coment ele peut estre ; en tel maniere ke si dit ne soient pas tourble ne confus, et k’il ne soient envolepé sous estranges paroles, et k’il ne trespasse a autres choses ki soient dessamblables et loins de sa matire, et k’il ne commence a trop longue commençaille, et k’il ne la prolongue de son conte tant comme il poroit dire, et k’il ne laisse rien dece ki a conter face.\par
Et en somme il doit tout çou garder que li mestres aprent ci devant sor la briefté dou fet ; car il avient maintefois ke li contes est plus confus par longuement parler ke por l’oscurté des paroles, et sor tout ce doit li parleor user mos propres, biaus, et acoustumés, selonc ce que li mestres devisa ça ariere el conte de parleure.
\chapterclose


\chapteropen
\chapter[{.III.XXXXV. A conter le fait voirsamblablement}]{\textsc{.III.XXXXV.} A conter le fait voirsamblablement}\phantomsection
\label{tresor\_3-45}

\chaptercont
\noindent Apriés ce doit li parleour conter le fait en tel maniere k’il soit voirsamblables ; c’est a dire k’il die teus choses ke li oïant puissent croire k’il die la verité. Tuilles dit ke a ce faire li covient a dire les proprietés ; dou cors, s’il est vieus u jovenes, u perreceus ou courouçables, u chachieres u des autres samblables proprietés ki tiesmognent a son dit.\par
Apriés li covient moustrer l’achoison dou fait, ce est a dire l’achoison por quoi et coment l’en poroit et devroit faire, et k’il ot pooir et loisir de ce faire, et k’il ot avenable tens de ce faire, et k’il ot assés de tens et de saison a ce faire, et ke li lieus fu bons et soufissans a faire ce ke li parleours dist avant.\par
Aprés ce doit il moustrer que li hons u la chose de quoi il parole sont de tel nature k’il peust et seust bien faire çou, et ke la renomee et la vois du peuple en est sor li ; et quant il a tele foi et tele creance et tele opinion k’il feroit bien une si faite cose.
\chapterclose


\chapteropen
\chapter[{.III.XXXXVI. Des visces dou fait dire}]{\textsc{.III.XXXXVI.} Des visces dou fait dire}\phantomsection
\label{tresor\_3-46}

\chaptercont
\noindent Or avés oii comment li parliers doit le fait dire en tel maniere k’il soit briés et clers et voirsamblables, car ces \textsc{.iii.} choses sont fierement besoignables a bien dire. Et si comme li parliers doit ensivre les vertus ki apertienent a bien dire, tot autresi se doit il garder des visces ki enpechent et honissent son parlement.\par
Et sont \textsc{.iiii.} : li uns est quant il est son damage a conter le fait, li secons est quant il ne li proufite de rien a dire, li tiers est quant le fait n’est pas conté en cele maniere k’il doit, li quars est quant il ne dist en cele partie dou conte ce ke mestiers est.\par
Sachiés donques que lors est il damages dou parleour a dire le fait selonc ce k’il a esté, quant cele chose desplest as oïans, et k’il en soient esmeu contre lui a ire et a haine s’il ne s’adoucist par bons argumens ki conferment sa cause. Et quant ce avient, tu ne dois pas conter le fait mot a mot ensamble si comme il fu, ains le te covient deviser par parties et dire une branche chi et autre la, et tot maintenant joindre la raison de chascune partie en son lieu, en tel maniere que ton cop ait tantost sa medecine et la bone deffense adoucist la haine des oïans.\par
Autresi bien sachiés k’il ne proufite de riens a conter le fait quant ses adversaires, u uns autres ki a parlé devant toi, a dit toute la cause en tel maniere ke il ne besoigne pas que tu la redies ne ainsi ne autrement de lui, ou quant cil a cui tu paroles sevent la chose en tel maniere ke tu n’as pas mestier de moustrer k’ele soit d’autre guise. Et quant ces choses avienent, Tuilles commande que tu te tais et que tu ne dies pas le fait.\par
Li tiers visces est quant le fait n’est pas conté en cele maniere k’il doit, c’est quant ce ki doit proufiter a ton adversaire et tu meismes le devises et bien et biel, u quant ce ki doit proufiter a toi tu le dis troubles et pereceusement. Tuilles dit ke por eschiver ce visce tu dois mout sagement torner toutes choses au profit de ta cause et taire le contraire tant comme tu poras ; et s’il te covient rien dire de ce ki apertient a l’autre partie, tu t’en passeras legierement ; et toutefois de ta partie di diligenment et apertement et fermement.\par
Li quars visces est quant le fait n’en est pas dit en cele partie dou conte ke mestiers est, et c’est des choses ki apertienent a ordre : por ce se taist ore li mestres jusques la u il traitera de l’ordene coment hom doit establir son conte et ses parties.
\chapterclose


\chapteropen
\chapter[{.III.XXXXVII. De la tierce branche dou conte c’est devisement}]{\textsc{.III.XXXXVII.} De la tierce branche dou conte c’est devisement}\phantomsection
\label{tresor\_3-47}

\chaptercont
\noindent Après la doctrine dou fait vient la tierce branche dou conte, c’est devisement : De quoi Tulles dit ke devisement, quant li parleour le dist selonc son droit, certes trestous li contes en est plus riches et plus biaus et mieus entendables. Et ja soit ce que ces \textsc{.ii.} brankes, c’est le fait et le devisement, soient por dire la cause, neporquant il a difference entr’eus ; car devisement dist tot a certes le point en quoi li parleours s’aferme et k’il vieut prover, més le fet ne le dist pas ensi.\par
Les parties dou devisement sont \textsc{.ii.}, une ki devise ce que li adversaires reconnoist et ce k’il nie, en tel maniere que chascuns puet bien entendre le point que li parliers vieut prover ; l’autre est quant li parleours devise par parties briement tous les poins k’il voudra prover, si que li oïeres le set en son corage, et entent bien k’il a dit toute la force de la sentence. Por ce est il drois de veoir les ensegnemens de l’un et de l’autre, et comment li parleours le doit user en son conte
\chapterclose


\chapteropen
\chapter[{.III.XXXXVIII. Dou premier devisement}]{\textsc{.III.XXXXVIII.} Dou premier devisement}\phantomsection
\label{tresor\_3-48}

\chaptercont
\noindent Et li premiers devisement ki raconte ce que li adversaires reconnoist et ce k’il nie : doit li parleour tot avant torner cel recongnoissance au proufit de sa cause. Si comme fist li adversaires Horreste : il ne dist pas ke Horrestes reconneust k’il eust tué Clistemestrem, ains dist autres paroles ki plus afermoient sa cause contre Horreste. Il est bien reconneu, fist il, que la mere fu tuee par les mains son fiz. Car a dire que fiz ocie sa mere est plus cruens que dire les nons de l’un et de l’autre.\par
Autresi fist Catons en sa sentence ; il ne dist pas k’il eussent reconneue la conjurison solement, car maintes gens disoient k’il ne l’avoient pas fet contre le commun de Rome, mais por bien contre aucuns ki governoient malement le commun. Pour ce torna Catons la lor recognoissance au profit de la cause, et dist contre aus fieres merveilles, k’il voloient la vile ardoir, ocire les millours, le païs destruire, et honir dames et pucieles. A ce vois tu que li uns et li autres dist ce ki estoit recongneu, mais chascuns le torna a son millour.\par
Et quant tu auras ce meisme fet en ton conte, tu dois dire ce que tes adversaires nie, et establir la question soz jugement por savoir ent le droit. Rais¢n comment : Horreste reconnoist le murtre, mais il nioït que il ne le fist pas a tort, mais a droit, et c’est la question ki remaint sous jugement, por savoir s’il le fist u a tort u a droit.
\chapterclose


\chapteropen
\chapter[{.III.XXXXVIIII. Dou secont devisement}]{\textsc{.III.XXXXVIIII.} Dou secont devisement}\phantomsection
\label{tresor\_3-49}

\chaptercont
\noindent Et le secont devisement, ki nombre par parties les poins k’il vondra prover, dois tu garder k’il soit briés et delivres et cours. Briés est que tu ne dies mot oiseus, se teus non ki besoignent a ta cause, car tu ne dois lors travillier les cuers des oïans par paroles et par mervilleus aornemens quant tu devises ton fait et tes parties.\par
Delivrance est quant tu dis generaument tot ce ki comprent toutes choses de quoi tu vieus dire, et sor ce te covient fierement garder que tu ne laisses a ramentevoir nule general chose ki te soit profitable, et que tu ne le dies a tart, c’est hors de ton devisement, car c’est mal dit et vicieus.\par
Cours est li devisemens la u tu dis le general mot de la cause, tu ne redies avec l’especial mot ki est compris soz le general que tu avoies ja dit.\par
Et sachiés que general mot est celui ki comprent maintes choses sous son non ; car cist mot animal comprent homes, bestes, oiseaus et poissons. Especial mot est celui ki est compris desous \textsc{.i.} autre, car ce mot Pieres ou Jehans u Jakes est bien compris soz ce general non, c’est home. Mais il i a mos ki sont general sor \textsc{.i.} et sont especial sor \textsc{.i.} autre, car cis moz home est especial sous ce mot animal, mais il est general sor ce mot Pieres u Jehans.\par
Ces ensegnemens dou general ou de l’especial dist li mestres por ce ke li parleours se garde que, en son general devisement, il ne mete les especiales parties ; car cil ki devisa son fet en ceste maniere : jou mousterrai, fist il, que pour la covoitise et por la luxure et por l’avarice de nos enemis, tous maus sont avenus as communs il n’entendi pas bien k’en son devisement il mellast les especials rnos avec son general ; car sans faille covoitise est le general non de tous desiriers, et luxure et avarice sont parties de lui.\par
Garde dont que quant tu as devisé le general, tu ne redies ses parties autresi comme se ce fussent autres choses estranges. Mais en l’autre branche qui vient apriés, c’est ou confermement, poras tu bien metre les especiaus parties dou general devant dit, por mieus afermer ton fet et ton devisement. Raison comment : tu vieus prover que Horreste fist murtre ; di dont apriés le devisement, Horreste ocist Clistemestrem, donc fist il omecide.\par
Aprés garde en ton devisement que tu ne devises plus parties ke mestiers soit a ta cause ; car se tu devisoies en tel maniere : je mousterroie que mon adversaire avoit bien le pooir de ce faire et k’il le voloit fere et k’il le fist, certes itex devisemens est grevables, car il i a trop de choses ; et il soufist assés a dire, je mousterroie k’il le fist. Autresi garde ke la u la chose est simple et d’une chose sans plus, il ne covient se poi deviser non, car il est assés a dire le point de la question.\par
Et neporquant il avient sovent que une chose puet estre provee par pluseurs raisons ; et quant ce est, li parleour doit deviser ses prueves en ceste maniere : je mousterrai ke tu fesis cele chose par teus et par teus raisons, et par chartres et par tiesmoing. Sour ceste branke dist Tuilles que il trouva en philosophie mains ensegnemens, mais il laissa ciaus ki n’estoient si fierement besoignables a bien parler comme ciaus ki ci sont.\par
Encore commande il une autre chose que l’en ne doit pas oublier en son conte, mais quant il aura finé son devisement et il commence l’autre branche, c’est confermement, por prover ce k’il a dit, soviegne lui que tot autresi il conferme devant ce k’il devisa devant et puis l’autre et puis l’autre, chescun en son lieu, en tel maniere que quant il voldra finer son conte il n’ait oublié neant de ses confermemens ; car ce seroit laide chose a recomencier \textsc{.i.} autre plet aprés la fin de son parlement.
\chapterclose


\chapteropen
\chapter[{.III.L. De la quarte branche, c’est confermement}]{\textsc{.III.L.} De la quarte branche, c’est confermement}\phantomsection
\label{tresor\_3-50}

\chaptercont
\noindent Après la doctrine dou devisement vient la quarte branche dou conte, c’est confermement ; de qui Tuilles dist que confermement est apelés quant li parleours dist ses argumens bons ki acroissent sa auctorité et fermeté a sa cause.\par
Et por ce que a diverses causes covient divers confermement, voldra li mestres tot avant moustrer et aprendre les lius desquex li parleour puet retraire ses argumens ; et puis quant il en sera lieus et tens, il dira comment l’om doit former son conte sor chascune maniere des causes.\par
Et sachiés que nules sciences dou monde n’ensegnent lieus de prover ses dis, se dyaletique et rectorike non ; mais tant i a de difference entre l’un et l’autre, ke rectorique consire especiaus choses selonc le sens dou non et selonc la vois solement, mais dyaletique consire les generaus choses selonc la senefiance dou non et de la vois. Ja soit ce ke cil ki sivent loy et divinité ou les autres ars font prouvance par lieus, je di ke c’est ou par dialetique ou par rectorique.
\chapterclose


\chapteropen
\chapter[{.III.LI. Des arguments por prover ce qe li parleres dit}]{\textsc{.III.LI.} Des arguments por prover ce qe li parleres dit}\phantomsection
\label{tresor\_3-51}

\chaptercont
\noindent Toute science est confermee par argument ki sont retrait des proprietés dou cors ou des proprietés de la chose. Et sachés que Tuilles apiele cors celui por le qui dit ou par le qui fet naist la question, mais chose apele il celui dit u celui fet de quoi la question naist. De ses proprietés dira li mestres tous les ensegnemens, et premierement dou cors.
\chapterclose


\chapteropen
\chapter[{.III.LII. Des proprietés dou cors qui donent argumenz de prover}]{\textsc{.III.LII.} Des proprietés dou cors qui donent argumenz de prover}\phantomsection
\label{tresor\_3-52}

\chaptercont
\noindent Les proprietés dou cors sont teus ke par eus puet li parleours dire et prover ke celui cors est atournés a ancune chose faire ou a non faire. Tuilles dit que ces proprietés sont \textsc{.xi.}, li nons, la nature, la norreture, la fortune, l’abit, le volenté, l’estuide, le conseil, le oevre, le dit, et la cheoite.\par
Non est une propre et certaine vois ki est mise a chascune chose comment ele soit apelee, dont li uns sont nons et li autres sornons, et de l’un et de l’autre puet li parleour fermer ses argumens. Raison coment : je di que cis hom ci doit bien estre fiers car il a a non Lyonés. Autresi dist sovent l’escripture : je di, fist li angeles, k’il ara a non Jhesu pour ce k’il sauvera le peuple.\par
Nature est mout grief chose a descrire son estre, car li un dient que Nature est le commencement de toutes choses, li autre dient que non est : car s’ele fust, donques eust Diex commencement de par Nature, mais Platons dist que Nature est le volenté Dieu, et par ce pert que Diex et Nature sont ensamble. Mais Nature est double, une ki fet naistre, et une de ce ki est net. Et des choses ki sont nees, les unes sont divines, les autres sont mondaines. Et des mondaines choses, les unes apertienent as homes, les autres apertienent as biestes.\par
Celes ki apertienent as homes par nature sont \textsc{.vi.} lieus, de koi li parleour puënt prendre lor argumens. Li premiers est s’il est malles u femiele ; raison comment : vous ne devés pas croire que ma dame fesist la bataille, car ce n’est pas oevre de feme. Li secons lieus est son païs ; raison coment : nous devons bien croire que cis hom soit sages, car il est grizois. Li tiers est sa vile ; raison coment : nous devons croire que cis hom soit bons drapiers por ce k’il est de Pronvins. Li quars est de sa lignie ; raison comment : bien doit estre Karles loiaus, car il fu fius le roi de France. Li quins est son aage ; raison coment : il n’est pas merveille se cist hom est legiers et muables, car il est fierement joenes.\par
Li sisimes est li biens et li maus ke l’om a par nature en son cors u en son corage ; el cors est s’il est sains u malades, grans u petit, biaus u lais, isniel u lent ; el corage s’il a dur engien u soutil, s’il est bien sovenables ou non, ou dous u aspres, ou soufrans u courouchables ensement. Et en some toutes choses ke l’en a par nature el cors et el corage sont conté sous le lieu de nature, mais celes que l’om aquiert par ensegnement sont contees sous le lieu d’abit, si comme li mestres dira ci apriés.\par
Norreture demoustre coment et entor queles gens et par qui li hom a esté norris et apris, c’est a dire ki furent si mestre et si ami et si compaignon, quel art il fait, et de quoi il s’entremet, et comment il governe ses choses et sa mesnie et ses amis, et comment il maine sa vie. Ces et autres samblables proprietés apertienent a norreture, et de tout puet on quellir ses argumens. Raison comment : Alixandres devoit bien estre sages por ce k’Aristotles fu ses mestres, u cis prestres ne doit pas estre evesques car il mainne sa vie en luxure.\par
Fortune comprent ce ki avient de bien u de mal, c’est a dire se cist kome est franc u serf, riches u povres, provost u sans prevosté, et s’il a cele provosté a droit u a tort, et s’il est bonseurés u de bone renomee u non, et quex fieus il a et quele feme. Mais se tu paroles d’ome mort, consire ses proprietés, c’est a dire quex hom il fu et comment il morut. Car de toutes ces choses poés vos prendre tes argumens por les lieus de fortune. Si comme dist Juvenaus, il n’a, fist il, au monde si grief chose comme riche fame.\par
Habis est uns compliemens que l’en a d’une parmenable chose en son cuer et en son cors. El cuer est li compliemens de vertus, ki sont devisees en l’autre livre, et les compliemens des ars et des sciences que l’on set et aprent en son cuer. El cors sont li compliement que l’en aquiert non mie par nature mais par son estude u par son ensegnement, si comme est ore de bien combatre u de bien luitier et de bien chevauchier.\par
Volentés est \textsc{.i.} legier movement ki aucunefois avient au cors et au corage par aucune okison, si comme est ore leece, covoitise, poour, courous, maladies, foibleces, et autres samblables choses.\par
Estude est une continuel enprise que li corages fet o grant volenté, si comme est a estudiier soi en philosophie et en clergie. De ce puet li parleours former ses argumens en ceste maniere : cis hom sera bons advocas, car il estudie mout fierement en loi.\par
Conseil est une sentence longuement pensee sor une chose faire u non faire. Mais il i a difference entre conseil et pensement, car pensement est consirier l’une partie et l’autre, mais conseil est la sentence quant il prent l’une des \textsc{.ii.} parties ; et por ce covient a tous conseil que la matire et le conseilleour et le temps soient avenable a ce que on vuet prover ; car se je disoie, cist home a bien bargeignié son cheval por ce k’il s’en consilla avec son provoire, certes le conseilleour n’est pas avenable ; mais se je di, cist hom est bien repentans por ce k’il a longhement consillié a son provoire, certes c’est bons argumens et creables.\par
Oevre en cest conte n’est pas la propre cause sor quoi l’om parole, ains est li usages ke uns hom siut avoir d’une chose faire u de non faire ; et de ce puet li parleour prendre son argument a moustrer de cel home s’il fist cele chose ou s’il la fet maintenant u s’il le fra ; si comme \textsc{.i.} des chevaliers Catelline dist : je croi bien, fist il, que Catelline fera la conjurison contre nous, car il en est acoustumés.\par
Dit est li usages que l’en siut avoir d’une chose dire u non dire, et ensinc de tout la nature ki est comme devisee de l’oevre ci desus. L’argument fet on en ceste maniere : je ne croi pas ke cis hom mesdie de moi, por ce k’il ne siut pas dire vilenie de nului.\par
Cheoite est des choses ki sont par aventure non mie penseement, et ensiut la nature de l’oevre et dou dit ; car on puet retraire son argument de ce ki est a venir et de ce ki est avenu et de ce ki avient, en ceste maniere : vous devés bien croire que cist hom tua cest autre, por çou k’il tenoit \textsc{.i.} coutel sanglant en ses mains. Ou en ceste autre maniere : il n’est par merveille se cist hom rit, por ce k’il a trové un grant moncel d’or. Mais ci se taist li contes des proprietés dou cors, por deviser les proprietés de la chose.
\chapterclose


\chapteropen
\chapter[{.III.LIII. Des proprietés de la chose}]{\textsc{.III.LIII.} Des proprietés de la chose}\phantomsection
\label{tresor\_3-53}

\chaptercont
\noindent En ceste partie dist li mestres que les proprietés de la chose sont teus ke par eus puet li parleours dire et prover sa entention. De cele chose Tulles dist que ces proprietés sont en \textsc{.iiii.} manieres, une ki se tient en toutes les choses, une autre ki est en la chose faisant, autres ki sont jointes a la chose, et une autre ki est environ la chose.\par
Les proprietés ki se tienent en totes les choses sont \textsc{.iii.} manieres, ce sont la some dou fait, l’achoison, et l’apareil. La some dou fait est quant li parleres dit le nom dou fait et de la chose ki a esté u ki est maintenant ou ki est a venir, en une somme briement, en ceste maniere : cist hons si fist murtre, cist autres fist larrecin, l’autres fera traison.\par
L’achoisons de la chose est double, une pensee et l’autre non pensee. Et l’achoisons ki est pensee est quant on fait une chose penseement et par conseil ; la non pensee est quant hom cort a faire une chose par aucun soudain movement et sans conseil.\par
L’apareil est en \textsc{.iii.} manieres. Une ki est devant le fait, en ceste maniere : cist hom agaita cel autre et le chaça longuement l’espee nue en sa main. Et li autre apareil est sor le fait, en ceste maniere : et quant il l’ot aconseu, il le gieta a la tiere et le feri tant que il morut. Li tiers apareill est aprés le fait, en ceste maniere : et quant il l’ot tué, il l’enseveli enmi le bois. Ces autres samblables proprietés se tienent o toutes les choses si fermement que a paine puet estre une chose faite sans aus, et por ce en puet li parleour establir ses argumens et prover la chose et bien et fermement.\par
Les proprietés ki sont en la chose faisant sont \textsc{.v.}, li lieus, li tens, la maniere, la saisons, et le pooir. Li lieus est cele part u la chose fu faite ; et certes il afiert mout a bien prover son dit que li parleours esgarde bien toutes les proprietés dou lieu, c’est se li lieus est grans u petis u long u prés u desiers u habités, et de quel nature est li lieus et tout le pais environ, c’est a dire s’il i a mons u valees ou rivieres u fleuves u sans enues, u se li lieus est bons u mauvés, et si li lieus est sacrés u non, u s’il est communs ou privés, et s’il est u fu a celui ki fist la chose, u non.\par
Tens est li espasse ke l’en ot en la chose faisant, c’est a dire par annees u par mois u par sesmaines u par jors u par heures u novelement u ancienement ou tart u tost ; car on doit garder se une grant chose peut estre fete en celui tans. Et sachiés que ces \textsc{.ii.} proprietés, ce sont lieu et tens, sont si proufitable a la chose prover ke neis cil ki misent en escrit les ancienes istores, et cil ki font chartres et letres, escrisent les lieus et les tens por mieus affermer la besoigne.\par
Saison est comprise soz le tens ; mais tant i a de difference entre l’un et l’autre ke tens esgarde l’espasse et la quantité dou tens alé et dou present et de celi ki est a venir, mais la saisons esgarde la maniere dou tens, c’est a dire s’il est nuit u jour, u s’il fet cler u oscur tens, u s’il est jor de feste u tens de vendenge u de moissons, u se cil hom dort u s’il fet noces u ensevelist son pere. Vois donc que une saisons apertient a tout le païs, si comme est de meissons, une autre apertient a toute la vile, si comme sont li jour de lor festes et de lor jeus acoustumés u por eslire provost u eveske, un autre apertient a \textsc{.i.} seul home, ce sont noces et sepultures.\par
Maniere est a moustrer coment on fist cele chose et a quel corage, c’est a dire s’il le fist a essient u non u par son gré u contre son gré.\par
Pooir est en \textsc{.ii.} manieres, une ki aide a faire la chose plus legierement, et un autre sans quoi ne poroit estre fait. De ce puet li parleours establir ses argumens, en ceste maniere : il n’est pas merveille se cis chevaliers gaaigne la jouste, por ce k’il est mieus montés que li autre ; u ensinc : cist hom ci ne fera pas la jouste, por ce k’il n’a point de cheval, et cil ne fist le coutel por ce k’il n’avoit fier.\par
Des proprietés ki sont jointes a la chose establist li parleour ses argumens en ceste maniere, quant il le retrait d’une autre chose plus grant u plus petit u samblable u d’une contraire u de son general u especial u de la fin u de la chose.\par
Et sachiés que chose pareille et plus grant et plus petite est consiré par la force et par le nombre et par la figure de lui. Raison coment : force est en \textsc{.ii.} manieres, une ki est u cors et un autre ki est en la chose. Ou cors est la force quant son non segnefie les proprietés de lui, car a estre apelés Salemons segnefie sens u savoir, et a estre apelés Noiron segnefie cruauté et folie. En la chose est la force quant le non de la chose segnefie les proprietés de lui, car a dire pericide et mericide segnefie grant cruauté et a Dieu et as homes. Autresi est consirés le nombre quant li parleor dist un u \textsc{.ii.} u \textsc{.iii.} gens u s’il dist u un u \textsc{.ii.} u plusours choses. Autresi est consirés la figure dou cors quant on dist k’il est grans u petis, et la figure de la chose quant ele a plus des proprietés. Car plus est a dire, ci hons ci ocist un priestre sour l’autel au jour de Paskes, que a dire, il ocist un home privé en lieu privé.\par
Samblable chose n’est mie pareille, car pareille chose segnefie la grandour et la mesure, mais samblable ne segnefie autre chose ke la qualité ; car samblanche est la proprieté ki fait \textsc{.ii.} diverses choses estre samblables entre lor. Raison coment : cist hom est isnel comme tigre, et cis prestres deveroit sermoner au pule comme Sains Pieres.\par
Contraires choses sont celes ki tot droit, front a front, sont l’une contre l’autre ; si comme est froit contre chaut et vie contre mort, mal contre bien et veillier contre dormir et orguel contre humilité. De quoi li parleour puet ses argumens former en tel maniere : se tu anuies celui ki te garanti de mort, ke feras tu dont a ciaus ki te welent ocire ?\par
Generale chose est cele ki est desus, c’est a dire celi ki comprent maintes choses desur lui ; car vertu est general, pour ce que desous lui sunt justice, sens, et atemprance, et maintes autres bontés ; et animal est general por ce que desous lui sont homes et bestes.\par
Especiale chose est cele ki est desous son general, car avarice est especiale pour ce k’ele est sous covoitise, et sens est sous vertu.\par
La fin de la chose est ce ki ja en est avenu et ki en avient et ki en est a avenir, et de ces choses retrait li parleour son argument quant il moustre ce ki est avenu u ce ki doit avenir u ce que avenir sieut des choses samblables, en ceste maniere : par orguel vient outrages et par outrage avient haine.\par
La quarte maniere de la proprieté de la chose est de ce ki avient environ la chose, non pas si dedens comme les autres devant dites ; en quoi on doit tot avant garder comment cele chose iert apelee et de quex non, et quel furent li chievetain et li troveour de la chose, et ki li aida a fere. Aprés doit il garder quel loi et quel us et quel jugement il a sous cele chose u quel art u quele science ou quel mestier. Autresi doit il garder se cele chose siut avenir sovent u par nature u non, u s’ele siut desplaire as gens u non, et ces proprietés, et maint autres ki suelent avenir aprés le fait maintenant u tart, u se c’est honeste u proufitable. Et doit tousjours li parleour consirer en tel maniere ke de toutes proprietés il sache confermer ses dis et retraire ses argumens et prover sa cause, por ce que mal s’entremet de parler ki ne prueve ses paroles raisnablement, si k’il soit creus de qank’il dist u de la grignour partie. Et por ce vaut li mestres moustrer comment li parleour doit moustrer ses argumens.
\chapterclose


\chapteropen
\chapter[{.III.LIIII. De .ii. manieres de touz argumens}]{\textsc{.III.LIIII.} De \textsc{.ii.} manieres de touz argumens}\phantomsection
\label{tresor\_3-54}

\chaptercont
\noindent Tous argumens que li parleour font par les proprietés devant dites, Tuilles dist k’il doivent estre necessaire u voirsamblables ; car argumens est uns dis trovés sour aucune matire ki le moustre voirsamblablement u ki le prueve necessairement.
\chapterclose


\chapteropen
\chapter[{.III.LV. Des argumens necessaires}]{\textsc{.III.LV.} Des argumens necessaires}\phantomsection
\label{tresor\_3-55}

\chaptercont
\noindent Li necessaires argomens est celui ki moustre la chose en tel maniere que autrement ne puet ele pas iestre. Raison coment : ceste feme gist d’enfant, donc a ele connut home. Et sachiés que li argumens ki prueve la chose par necessité puet estre dit en \textsc{.iii.} manieres, u par reploiement u par nombre u par simple conclusion.\par
Reploiement est quant li parleours dist choses, deus u trois u plusours parties, desquex, se son adversaire conferme l’une quele qu’ele soit, certes il sera conclus. Raison coment : je di que Thomas, u il est bons u il est malvais. Se tu dis k’il est bons, je di, por quoi le blasmes tu donc ? et se tu dis k’il est mauvais, je diroie, por quoi converses tu donques avec lui ? Et ensi vient de reploiement, ke laquele partie que tu preignes, je en trai maintenant mon argument ki te conclust par necessité.\par
Et sachés que cis argumens est en \textsc{.ii.} manieres, une ki est par la lorce de \textsc{.ii.} contraires coses ke l’on dist tot ensamble l’un aprés l’autre, si comme est ci en l’essample avant dit ; l’autre maniere est par la force de \textsc{.ii.} choses ki sont contraires entr’aus par la force d’une negation, en ceste maniere : je di que cist hom ci a deniers u il n’en a nus. Iteus argumens fist Sains Augustins contre les juis quant il lor dist, li Sains des sains est venus u non est : s’il est venus, donc est perdus vostre enoignemens, et s’il n’est venus, n’est li enoignemens pas perdus. Et se li enoignemens n’est pas perdus, donc avés vous roi ; et se vos avés roi, donc u çou est Jhesucris u uns autres ; mais d’autre roi n’avés vous point, donc est il Jhesucris.\par
Nombres est quant li parleours conte en son dit maintes choses et maintenant les oste trestotes fors que une seulement, laquelle il prueve par necessité. Raison comment : je dirai ensi, il covient a fine force ke se cist hom tua cel autre k’il le fist u par haine ki entr’aus fust u par paor u par esperance ou par l’amor d’aucun sien ami ; et s’il n’i a nule de ces ochoisons, donc ne l’ocist il pas, car sans oquoison ne puet estre fait un tel malfet. Mais je di ke entr’aus n’avoit nule haynne ne poour et k’il n’i avoit esperance k’il deust estre son oir u d’avoir aucun autre proufit de sa mort, ne il ne nus de ses amis, donc di je k’il ne l’ocist pas.\par
Et ceste maniere d’argument est proufitable a celui ki deffent sa queriele, si comme li essamples moustre ci deseure. Autresi est il proufitable a celui ki acuse. Raison comment : je di ke mon argent u il fu ars u il est en la huche u tu l’as emblé ; mais ars ne fu il pas ne en la huche n’est il mie, donc remaint ceste partie que tu l’as emblé.\par
Simple conclusions est quant li parleour conclust necessairement ce k’il wet prover par la force d’une chose ki est dite devant. Raison comment : vous dites que cest murtre je fis en aoust, mais en celui tens estoie je outre mer, donc apert par necessité que je ne le fis pas.\par
Or avés oii les \textsc{.iii.} manieres de necessaires argomens, es quex li parleours se doit mout fierement garder que son argument n’ait mie solement la coulor et la samblance de necessité, ains soit dou tout de si necessaire raison que li adversaires ne puisse rien dire a l’encontre.
\chapterclose


\chapteropen
\chapter[{.III.LVI. D’argument voirsamblable}]{\textsc{.III.LVI.} D’argument voirsamblable}\phantomsection
\label{tresor\_3-56}

\chaptercont
\noindent Li voirsamblables argumens est des choses ki ont acoustumé a venir sovent, ou des choses que l’on quide ki soient u des choses ki ont aucune samblance k’eles soient voires u voirsamblables.\par
Des choses ki ont acoustumé a venir prent li parleours son argument en tel maniere : se ceste feme est mere, donc aime ele son fis, et ce cist hom est avers, donc ne redoute il sairement.\par
Et des choses ke l’om quide ki soient prent li parleor son argument en ceste maniere : se cist hom est pechieres, s’ame ira a la pardurable mort, et se cist hom est philosophes, donc ne croit il es Diex.\par
Des choses ki ont aucune samblance prent li parleours son argument en trois manieres, u par ses contraires, u par ses pareilles, u par celes ki sont d’une meisme raison. Par ses contraires fet on son argument en ceste maniere : se li pecheour vont en infier donc li religieus vont en paradis.\par
Par les pareilles ensi : si comme li lieus sans port n’est pas seurs as neis, tot autresi li corages sans foi n’est parmanables as amis, car lieus sans port et coer sans foi sont samblables en muableté, et nef et amis sont samblables en seurté.\par
Par les choses ki sont d’une meisme raison prent li parleour ses voirsamblables argumens en ceste maniere : s’il n’est laide chose au chevalier de doner ses robes, donc n’est il laide chose as menestriers s’il les viestent. Or sachiés que cis argumens et li autre samblable sont necessaire en ceste maniere : s’il a le marge, donc fu il navrés ; mais li voirsamblables est ensi : s’il a mout de poudre sor ses sollers, donc a il alé longue voie. Ités argumens est provables mais il ne sont pas necessaires, car on poroit bien avoir poudre sour sa chancemente sans estre alé ne poi ne grant, més marge ne poroit nus avoir sans navreure.\par
Por quoi je di que toz argumens voirsamblables, ou il est signe u il est creable, u il est establis u il est samblables.\par
Et signes est une demostrance ki done presumption ke la chose fust u sera selonc la segnefiance de celui signe, mais il n’est pas certaine prueve, et por ce requiert encore grignour confermement. Et cist signes sont selonc les \textsc{.v.} sens dou cors, de l’oïr, dou flairier, dou savoir, dou touchier. Car je di, il a enki environ charoigne pour ce k’il i a grant puantour, certes c’est \textsc{.i.} signes ; mais il n’est pas certains, k’il ne coviegne encore grignor prueve.\par
Creables est ce ke sans nul tiesmoigne done foi et creance, en ceste maniere : il n’est nus hom ki ne desire ke ses fiz soient sains et bonseureus.\par
Establis est en \textsc{.iii.} manieres, ou par loi u par commun usage u par establissement des homes. Par loi est establie la paine des larrons et des murtriers. Par commun usage est establis que l’en rende honour as vieus et as plus grans de lui. Par establissement des homes est quant il establissent par aus meismes sor une chose douteuse, k’il en doit estre. Raison comment : Grasces, quant il fu censor de Rome, ne fist rien del monde sans le sens son compaignon ; se li tornoient li \textsc{.i.} a sens et li autre a folie ; mais li commun dou peule establi k’il fu consoles en l’an apriés, et ensi fu fermé k’il avoit fait grandisme sens.\par
Samblables est ce ki demoustre aucune samblance de raison entre deus diverses choses, et c’est par \textsc{.iii.} manieres, u par ymage u par comparison u par essample. Ymages est çou ki dist ke \textsc{.ii.} u plusours diverses choses ont aucune samblance entre eles selonc les proprietés dou cors et de la nature, en ceste maniere : cist hom est plus hardis que lions et cil autres couars comme lievres. Comparisons est ki moustre ke aucunes diverses choses aient entre eles resamblance selonc les proprietés des corages, en ceste maniere : cist hom est engignous comme Aristotles et cil autres est gros comme un asnes. Examples est celi argumens ki moustre aucune resamblance es choses pour les dis et par les commandemens ke l’om trueve es livres des sages, u par ce ki est avenu as homes u as choses ki furent de cele semblance. Mais de ces argumens se taist ore li mestres, car il voldra torner as autres choses ki apertienent a confermement.
\chapterclose


\chapteropen
\chapter[{.III.LVII. Des argumens de long et de priés}]{\textsc{.III.LVII.} Des argumens de long et de priés}\phantomsection
\label{tresor\_3-57}

\chaptercont
\noindent Après çou que li mestres ot ensegniés les lieus et les proprietés et les raisons desquex et comment li parleours puet prendre argoment de prover sa matire et son dit, et il li fu avis que, se il ces argumens devisast par parties, que la raisons en seroit plus bele et plus entendable, meismement por ce que c’est une science que poi de parleour sevent, car ele est grief a savoir et a moustrer. Et por ce dist il en ceste partie que totes manieres d’argumens, de quel proprieté u de quel raison k’il soient, certes il covient k’il soit pris u de long u de priés ; car aucunefois est tele la matire dou parleour k’il ne le poroit prover s’il ne se fesist de loing. Et por ce est il drois a deviser les ensegnemens de l’un et de l’autre.
\chapterclose


\chapteropen
\chapter[{.III.LVIII. De l’argument qi est pris de long}]{\textsc{.III.LVIII.} De l’argument qi est pris de long}\phantomsection
\label{tresor\_3-58}

\chaptercont
\noindent De long est pris celui argumens ki par la samblanche des certaines choses de loins amaine son adversaire a consentir et a connoistre cele chose que li parleours vieut moustrer.\par
Raison comment : je parlai a Audebrant ki pas n’amoit sa feme, ne ele lui, en ceste maniere : dites moi, Audebrant, se vostre voisin eust mieudre cheval de vous, lequel ameriés vous mieus u le sien u le vostre ; le sien, fist il. Et s’il eust plus bele maison de vous, le quel volriés vous avoir, u la vostre u la soue ; la soue, fist il. Et s’il eust millour fame de vous, laquele volrois vous, u sa soue u la vostre : a ce ne dist il ne ce ne quoi.\par
Maintenant je m’en alai a sa fame et li enquis en tel maniere : se vostre voisine eust millour trezor de vous lequel volriés vous, u le sien u le vostre ; le sien, fist ele. Et s’il eust millours dras et plus riches harnois de vous, lequel voldriés vos, u le sien u le vostre ; le sien, fist ele. Et s’ele eust millour mari de vous, lequel volriés vous, u le sien u le vostre ; a ceste parole ot la feme honte, k’ele ne dist noient.\par
Et quant je fui a ce venus, maintenant je lor dis, por ce que nus de vous ne respont pas a ce ke je voloie oïr, je diroie ce que chascuns pense : vos volriés avoir trés bone feme et vous trés bon mari ; pour ce vous di que se vous ne fetes tant que chascuns soit le millour, vous ne finerés jamés de querre ce que trés bon samble. Donc vous covient il penser que vous soiés trés bons maris et tu trés bonne fame.\par
Gardes donc, ke por la resamblance des certaines choses de long je les amenoie a consentir ce que je voloie ; car se je demandaisse tout simplement se celui volsist millor feme, u cele meillor mari, certes il ne se fussent pas consenti a ma demande. Iteus commandemens usa mout Socrates en ses dis et tote foies k’il voloit rien prover metoit il avant teus raisons que l’en ne pooit pas nier, et lors faisoit il sa conclusion de ce ki estoit en son proposement.\par
Mais en ces argumens doit li parleor garder \textsc{.iii.} choses. Premierement que ce que il prent de long par samblance de sa cause soit certaine sans doute, car chose douteuse doit estre provee par certaines raisons.\par
Aprés doit garder k’ele soit dou tout samblable a ce k’il wet prover, car s’ele fust estrange u dessamblable il ne poroit pas former sa prueve.\par
Aprés doit il garder ke li oïeres ne sace pas a quoi li parleor bee ne por quoi il li fait ces demandes, car s’il s’en aperceust, u il s’en tairoit u il nieroit u il responderoit par contraire.\par
Et quant tu auuras a ce mené ton adversaire, il covient a fine force k’il face une de ces \textsc{.iii.} choses, u k’il se taise u k’il nie ou k’il conferme ta prueve. Et se ce est que il le nie, et tu la reprueves par la samblance de ce ke tu avoies devant dit, ou d’autres samblables choses ke tu redies maintenant. Mais s’il le conferme u il se taist, maintenant dois tu conclure la querele et poser fin a ton dit. Car Tuilles dit ke cil ki se taist est samblables a celui ki conferme.\par
Par ces ensegnemens poés vous entendre ke en cesti argument de long covient avoir \textsc{.iii.} choses, la premiere est la samblance que li parleour dit tot avant, la seconde est cele propre chose k’il vieut prouver, la tierce est cele conclusion ki moustre ce ki ensiut de ses argumens et conferme sa prueve.\par
Mais por ce k’il i a mainte de gent de si dur cervice que par les ensegnemens ki sont doné sor ancune science ne le poroient pas entendre s’il ne le veissent par essample, por ce vieut encore li mestres escrire \textsc{.i.} essample d’un plait ki dura longuement entre les grezois, o il avoit jadis une loi ki disoit ke se li viens connestables ne renvoiast toz les chevaliers au noviel connestable, k’il deust perdre la teste. Or avint chose ke Epaminaus n’envoia pas ses chevaliers au noviel connestable lors k’il devoit, ains s’en ala a toute son ost contre les lacedoniens et les venki par force d’armes. Et quant il fu encusés, il disoit que cil ki fist la loi entendi que se li vieus connestables detenist les chevaliers por le proufit dou commun, k’il ne fust dampnés ; mais son adversaire faisoit ses argument contre lui, et ceste maniere :\par
Segnour juges, çou ke Epaminaus dit, et k’il vieut joindre a la loi outre ce ke vous i troverés escrit, soufrirés le vous ? nenil. Et se ce fust que vous le soufrissiés por la victore k’il a aquise, ce seroit mout contre les dignités de vous et contre vostre honorabletés ; mais quidiés vos ke le peuple le suefre ? nenil certes. Et se c’est tort a joindre a la loi, quidiés vous que il soit droit a fere le ? certes je conois tant le sens et le savoir ki en vous est, que ja ne vous sambleroit. Por quoi je vous di que se la lois ne puet estre amendee, ne par vous ne par autre, donc ne poés vous remuer la sentence de ce que vous ne poés remuer \textsc{.i.} sol mot. Mais ci se taist li mestres a parler des argumens de loing, de quoi il a dit assés, si torne son conte as argumens de prés.
\chapterclose


\chapteropen
\chapter[{.III.LVIIII. De l’argument qi est pris de prés}]{\textsc{.III.LVIIII.} De l’argument qi est pris de prés}\phantomsection
\label{tresor\_3-59}

\chaptercont
\noindent De priés est pris celui argument ki par aucune des proprietés dou cors u de la chose mousterrai ke son dit soit voirressamblables, et le conferme par sa force et par sa raison sans nul argument de loins. De cestui argument dit Aristotles et Teoforastes k’il i a \textsc{.v.} parties. Dont la premiere est le proposement, c’est a dire quant tu proposes briement la somme de ton argument. Raison comment : tu dis ke totes choses sont mieus governees par conseil ke sans conseil, c’est ton proposement, et est la premiere partie de ton argument.\par
Or te covient aler a la seconde, c’est a confermer la par maintes raisons, en ceste maniere : la maisons ki est establie par raisons est mieus governee de toutes choses ke cele ki est governee folement. Li hom ki a bon chievetain et bon signour est plus sagement menés ke celui ki a fol signor et niche. La nef meismes fet bien son cours quant ele a sage governeour.\par
Or est acomplie la seconde partie de son argument, c’est le confermement dou premier proposement, si te covient aler a la tierce partie, c’est a prendre ce que tu vieus prover par le premier proposement, en ceste maniere mais nule rien n’est si bien governee par conseil comme tous li mondes ; c’est la prise ke tu vieus prover.\par
Et maintenant te covient passer a la quarte partie de l’argument, c’est a confermer prise par maintes raisons, en ceste maniere, que nous veons que li cours des signaus et des planetes et de toutes estoiles est establi a son ordre, li muement du tans sont chascun an par necessité, ou por profit de toutes terrienes choses ; ne li ordre dou jour et des nuis n’est mie por le damage de nuli : totes ces choses sont signes ke li mondes n’est mie governés sans grandisme conseil.\par
Or est acomplie la quarte partie des argumens, c’est li confermemens et la prise ; si te covient maintenant aler a la cinkime partie de l’argument, c’est a conclusion, ki puet estre dite en \textsc{.ii.} manieres : ou sans redire noient du premier proposement ne de la prise, en ceste maniere : donc di je que li mondes est governez par conseil ; ou en redisant le premier proposement et la prise, en ceste maniere : car se toutes choses sont mieus governees par conseil ke sans conseil, ne nule rien n’est si bien governee par conseil come tous li mondes ; donc di je que tous li mondes est governés par conseil. Et ce sont les \textsc{.v.} parties des argumens de priés, c’est le proposement et son deffermement, la prise et son confermement, et la conclusion.\par
Mais il i a maint des gens ki dient ke cestui argument n’a que \textsc{.iii.} parties sans plus, car il quident ke le proposement et son confermement ne soit que une meisme chose, et la prise et ses confermemens, une chose, et la conclusion soit une autre chose. Mais il sont trop malement deceu, et orés raison por coi.\par
Çou sans quoi une chose puet estre n’est pas de cele chose, ains est d’un autre tot par lui, et ensi sont \textsc{.ii.} choses et non pas une, et ore orés comment. Se je puis estre home sans savoir lire, donc sui je une chose et la letre est une autre. Autresi est d’un proposement ki puet estre ferm et estables sans nul confermement. Se le jour que ce murtre fu fés a Rome je estoie a Paris, donc ne fu je pas a ce murtre ; ci n’a mie mestier de nul confermement.\par
Maintenant feras ta prise, et diras en ceste maniere, mais a Paris estoie je sans faille. Et quant tu auras ce dit, tu le dois confermer et prover et faire puis ta conclusion, et dire, donc ne fui je pas a ce murtre. Tot autresi est d’une prise ki puet estre ferme et estable sans nul confermement, en ceste maniere : se li hom wet estre sages, il se doit estudier en philosophie ; c’est le premier proposement ki requiert a estre confermee, pour ce que mainte gent quident que l’estuide de philosophie soit mauvaise. Et quant tu l’auras confermé par bones raisons, tu feras ta prise en ceste maniere : mais tot home desirent a estre sage ; ceste prise est si certaine ki ne le covient confermer. Mais tot maintenant fai ta conclusion en ceste maniere, donc se doit chascuns estudier en philosophie. Par ces raisons et par ces examples pués tu bien connoistre k’il i a de teus proposement et de teus prises ki requierent a estre confermé, et de teus ke non.\par
Por ce s’acorde bien Tuilles a la sentence d’Aristotle, et dit que en cestui argument a \textsc{.v.} parties, et que cil sont en erreur ki quident qu’il n’en i ait que \textsc{.iii.} solement. Et neporquant il puet estre aucunefois li argomens de tel nature k’il n’i a que les \textsc{.iiii.} u les \textsc{.iii.} parties sans plus. Et a la verité li argumens a totes les \textsc{.v.} parties quant il dist le proposement et son confermement, et la prise et son confermement, et la conclusion. Mais quant li proposement ou la prise est si estable que li uns d’aus n’a mestier de nul confermement, lors n’a que \textsc{.iiii.} parties li argumens. Et se li proposemens u la prise sont teus que l’un ne l’autre ne quiert confermement, lors n’a li argumens que \textsc{.iii.} parties, c’est le confermement ou la prise et la conclusion.\par
Mais il i a mainte gent qui quident que cest argumens puet estre de \textsc{.ii.} parties u d’une solement ; car se li proposement et la prise sont si estables que la conclusion naist toute clere, si k’il ne le covient pas dire, lors n’a il que \textsc{.ii.} parties ; et se li proposemens est si fors que li parleres en puet maintenant confermer sa conclusion sanz prise, lors n’a il que \textsc{.ii.} parties. Autresi en ceste maniere : ceste feme gist d’enfant, dont connut ele home carnelment. Et si li proposemens est si fors et si estables que l’en entent bien la conclusion sans dire la, lors n’i a il c’une partie ; car se tu dis, ceste feme est grosse, chascuns entent bien k’ele a conneu malle, si k’il ne le t’estuet pas dire.\par
Sor ces parties dist Tuilles k’il ne quide pas ke drois argumens puist estre fait selonc cestui art au mains de \textsc{.iii.} parties. Car ja soit ce que diverses sciences aloigne divers ensegnemens, neporquant la science de rectorique requiert argument clers et certains ki se facent croire as oïans. Por ce a li mestres diligement devisé de toutes manieres de prover ce ke l’en wet dire et de confermer selonc ce que s’en apertient a la quarte branche du conte, c’est a confermement ; si tornera a sa matire por dire de la quinte branche, c’est dou deffermement.
\chapterclose


\chapteropen
\chapter[{.III.LX. De la quinte branche du conte, c’est deffermement}]{\textsc{.III.LX.} De la quinte branche du conte, c’est deffermement}\phantomsection
\label{tresor\_3-60}

\chaptercont
\noindent Après la doctrine dou conffermement vient la quinte branke dou conte, c’est deffermement. De quoi Tulles dit que deffermement est apielés quant li parleours apetice et destruit les argumens son adversaire dou tout ou de la grignor partie. Et sachiés que deffermement ist de cele meisme fontaine que le confermement, car si comme une chose puet estre confermee par les proprietés du cors et de la chose, tot autresi puet ele estre deffermee, et por ce dois tu prendre l’ensegnement meismes que li mestres devise cha ariere el chapistre dou confermement. Et neporquant il en dira aucune chose por mieus mostrer la force et la nature dou confermement, et chascuns le pora entendre plus legierement quant li \textsc{.i.} contraires est mis aprés l’autre.\par
Toz argumens sont deffermés en \textsc{.iiii.} manieres. Premierement se tu nies la prise de ton adversaire, ce meismes k’il wet prouver. Aprés se tu confermes la prise, tu nies la conclusion. Aprés se tu dis que son argument soit vicieus. Aprés se contre ton argument tu en redis \textsc{.i.} autre autresi ferme u plus. Por ce volt doner li mestres les ensegnemens ki covient a chascun de ces \textsc{.iiii.} manieres.
\chapterclose


\chapteropen
\chapter[{.III.LXI. Dou deffermement qi nie le voirsamblable argument}]{\textsc{.III.LXI.} Dou deffermement qi nie le voirsamblable argument}\phantomsection
\label{tresor\_3-61}

\chaptercont
\noindent Le premier deffermement est a nier ce que ton adversaire prent a prouver par argument necessaires ou par argumens voirsamblable. Et se ce que il dist est argumens voirsamblables, tu les poras niier en \textsc{.iiii.} manieres.\par
Dont la premiere est quant il a dit d’une chose qu’ele est voirsamblable, et tu dis que non est, por ce que son dit est tout clerement faus, en ceste maniere : tes adversaires dist qu’il n’est nus hom ki ne soit plus covoiteus de deniers que de sens ; certes de ce ne dist il mie voir, car il en i a plus ki ayment sens que catel.\par
U se son dit est teus que son contraire soit autresi creables comme son dit, en ceste maniere : tes adversaires dist k’il n’est nus hom ki ne soit plus covoiteus de signorie que de deniers, certes autresi pués tu dire son contraire, k’il n’est nus ki ne covoite plus deniers que signorie.\par
U se son dit n’est pas creable, en ceste maniere : uns hom ki est fierement avers dit que por \textsc{.i.} petit service de son ami il a laissié un sien grandisme proufit.\par
U se ce ki siut avenir aucunefois, tes adversaires dist k’il avient tozjours useement, u en ceste maniere : il dist que tot povre covoitent plus deniers que signorie, certes il avient bien aucunefois que uns povres covoite plus deniers, mais il en resont des autres ki mieus ayment la signorie ; si comme en aucuns lieus desers suelt on fere murtre, mais non pas en tous.\par
U se son dit n’est pas creable, en ceste maniere : tes adversaires dist de ce ki avient aucunefois k’il n’avient en nule maniere dou monde, en ceste maniere : il dist que nus ne puet estre pris d’amour de feme por un seul regart, car c’est chose ki puet avenir que par une seule vene l’en ayme par amours.\par
La seconde maniere de niier le dit ton adversaires est quant il dist le signe d’une chose, et tu le deffermes par cele meisme voie k’il a confermee. Car en toz signes covient il a moustrer \textsc{.ii.} choses, une ke celui signe soit voir, l’autre k’il soit propre signe de la chose k’il vieut prouver : si comme sanc qui est signe de meslee, et charbons sont signe de feu. Et puis covient a moustrer que fait soit ce qui ne covient pas, ou que ce ne soit ki covenoit, et que li hom de qui li parleour dient sa loi et la coustume de ceste chose, car toutes ces choses apertienent a signe et au samblant.\par
Et pour çou que tu vieus deffermer les signes ton adversaire, tu dois esgarder comment il le dist ; car s’il dist ke ce soit signes de cele chose, tu diroies ke non est, en ceste maniere : il dist ke la cote est toute sanglante que tu aportes, et est signes que tu as esté a la mellee, et tu diroies que non est ; ou tu dis que c’est legiers signes, car la cote sanglante puet estre signes que tu aies esté signiés.\par
U tu dis que celui apertient plus a toi que a lui ; car s’il dist que fait soit ce ki ne covient pas, en ceste maniere : tu en rougis ou visage pour ce que tu avoies coupe en cest meffet, tu dis que ce ne fu mie por mal més por honesteté et par droit. U tu dis que celui signe soit dou tout faus ; car s’il dist que tu tenoies le coutel sanglant, tu dis que sanglant n’estoit il pas, mais il ert enrueilliés.\par
Ou tu dis que celui soit apiertenans a autre suspection que tes adversaires ne dist ; car s’il dist que ce ne soit pas fet ki covient, en ceste maniere : tu t’en alas sans congié prendre, c’est samblance de larrecin, et tu dis que ce ne fu mie por mal, mais por ce que tu ne voloies esvillier le signeur de laiens.\par
La tierce maniere de niier le dit de ton adversaire est quant il fait en son dit une comparison entre \textsc{.ii.} choses, et tu dis que cele chose n’est pas samblable a cele autre, por ce k’eles sont de diverses manieres. Car s’il dist, tu voldroies avoir millour cheval que ton voisin, donques voldroies tu avoir millour feme, et tu denies son dit por ce que fame est d’autre maniere ke cheval. Ou pour çou k’eles sont de diverses matires ; car s’il dist que l’om le doit redouter comme lyon, tu nies son dit, por ce que hom est d’autre nature ke lyon.\par
Ou por ce k’il sont de diverses forces ; car s’il dist que Pyrus doit estre dampnés a mort por la feme Hirestis k’il ravi, autresi com Paris ravi Helaine, et tu nies son dit, por ce ke le forfait Paris fu plus fort que celui de Pyrrus.\par
Ou por ce que eles ne sont pas d’un grant ; car s’il dist, cist home ci a \textsc{.i.} home tué, il doit estre jugiés a mort autresi comme cil autres ki en ocist \textsc{.ii.} ; et tu nies son dit, por ce k’il ne fist si grant mal comme cel autre. Si di je en somme des diversités dou tens, dou liu, dou cors, et de l’opinion, et de toutes diversités ki sont es hommes et es choses. Car de chascune puet li bons parliers reprendre son adversaire, et deffermer son confermement.\par
La quarte maniere de niier le dit a ton adversaire est quant il ramentoit aucun jugement des sages homes. Car teux argumens puet il confermer en \textsc{.iiii.} manieres.\par
Ou par la louange de ciaus ki le jugement donerent, si comme Julle Cesar dist que li ancien sage de Rome par lor grant sen avoient pardonné a ciaus de Cartages.\par
Ou il le puet confermer par la resamblance que celui jugement a a la chose de qui il parole, si comme fist \textsc{.i.} preteor de Rome quant il dist, si comme nos anciestres pardonerent a ciaus de Cartage, tot autresi devons nos pardoner a ciaus de Greze.\par
Autresi le puet il confermer por ce k’il dist que le jugement k’il ramentut fust confermés par trestous homes, u par toz ciaus ki l’oïrent ou ki confermer le peurent, por ce ke celui jugement fu grignour et plus grief de la chose de qui il parole ; si comme fist Catons quant il dist que Mallius Torcatus juga son fis por itant solement k’il envaï les françois contre son commandement.\par
Ce sont les \textsc{.iiii.} manieres por confermer le jugement, et tu soies maintenant apareilliés, et defferme ce k’il a dit par le contraire de ses confermemens, se tu onques pués. C’est a dire que se il les loe tu les blasmes, et s’il dist que le jugement fu confermés, et tu dis que non fu. Autresi fai de toutes les raisons k’il dist sor le jugement, tu dies les contraires raisons.\par
Mais por ce ke li ensegnement dou parleor doivent estre communs entre l’un parleour et l’autre, dist li mestres ke li parleour ki ramentoit le jugement doit mout regarder que le jugement ne soient dessamblables de cele chose de qui il parole por ce que son adversaire le poroit legierement reprendre.\par
Aprés ce doit il garder k’il ne ramentoive tel jugement ki ait esté contre les oïans, por ce k’il crieront et diront maintenant que ce fu contre droit, et ke li juges en deust estre dampnés.\par
Aprés ce doit il garder que, quant il puet ramentevoir mains bons jugemens loés et seus, il n’en die \textsc{.i.} estrange ne mesconneu, car c’est une chose de quoi son adversaire le puet legierement reprendre et deffermer son dit. Or avés oii comment on puet deffermer toz voirsamblables argumens ; si fet huimés bon dire dou deffermement de necessaires.
\chapterclose


\chapteropen
\chapter[{.III.LXII. Dou deffermement qi nie les necesaires argumens}]{\textsc{.III.LXII.} Dou deffermement qi nie les necesaires argumens}\phantomsection
\label{tresor\_3-62}

\chaptercont
\noindent Et se ton adversaire fait sor son dit argumens necessaires, tu dois maintenant consirer se li argumens est necessaires u s’il porte samblance de necessité. Car s’il est droitement necessaires tu n’as pooir de contredire, mais s’il porte la samblance et il ne soit pas necessaires, lors li poras tu deffermer par icele voie meisme ki fu devisee ça ariere el chapitle des necessaires argamens, c’est par reploiement et par nombre u par simple conclusion.\par
Reploiement est quant li parleours devise \textsc{.ii.} u \textsc{.iii.} u plusours parties desquex, se tu confermes l’un quex k’il soit, certes il te conclust s’ele est voire. Mais s’ele est fausse tu le pués deffermer en \textsc{.ii.} manieres u en \textsc{.ii.} deffermemens, ou en deffermant toutes les parties, u en deffermant l’une sans plus.\par
Raison coment : tes adversaires vieut conclure que tu ne dois pas chastoier ton ami, et sor ce devise \textsc{.ii.} parties, en ceste maniere : u il crient, u il ne crient pas ; se il crient ne le chastoies mie, car il est bons, s’il ne crient, ne le chastoies pas, car il a por nient ton ensegnement. Cis argumens n’est pas necessaires, més il le resamble ; et tu dois maintenant deffermer andeus les parties en ceste maniere ; mais je les doi chastiier, car s’il crient honte, il ne despira pas més dis, et s’il ne crient, de tant le doi je plus chastiier, por ce k’il ne soit mie bien sages. Et se tu vieus deffermer l’une de ces parties sans plus, tu diras ensi : mais s’il crient voirement le doi je chastiier, car il sera amendés par mes dis, et deguerpira son erreur.\par
Nombres est quant li parleor conte en son dit maintes choses pour l’une prover selonc ce que li contes devisera el chapistre des necessaires argumens. Lors maintenant te covient deffermer son nombre, car il puet avoir \textsc{.iii.} visces.\par
Dont li premiers est se il ne nombre pas cele partie que tu vieus affermer. Raison coment : tes adversaires dit ensi : u tu as achaté ce cheval u il te fu donés, u il fu norris en ta maison u il t’eschei d’iretage, u se ce n’est, donc tu l’as emblé ; mais je sai bien, tu ne l’as achaté, ne ne te fu pas donnés, ne ne t’eschei d’iretage, ne ne nasqui en ta maison, donc l’as tu emblé sans faille. Et quant il a ensi conclus, tu dois maintenant dire la partie k’il laissa en son nombre, et di ke tu l’as gaaignié au tornoiement, car son argument i est toz deffermés se c’est la verité k’il n’avoit pas conté.\par
Li secons visces est quant il nombre une chose que tu pués contredire, car s’il dist que chil chevaus ne t’eschei d’iretage, et tu pués dire que si fist, certes son argument est toz depechiés.\par
Li tiers visces est quant une des choses k’il nombre tu le pués bien reconnoistre et affermer sans laidure. Raison coment : tes adversaires dist ensi : tu demeures enki por luxure u por agait u por le proufit ton ami, certes tu pués bien affermer ke tu i soies por le porfit d’aucun.\par
Simple conclusion est quant li parleour conclust ce k’il vieut par la force d’une chose ki est dite devant moi ; c’est de \textsc{.ii.} manieres, u il le prueve par necessité, u par samblance de necessité. Et se c’est par necessité, tu ne le pués pas contredire ; car s’il dist, ceste feme est enchainte, donc conut ele home, u se cist hom espire, donc vit il, certes tu ne pués rien dire a l’encontre. Mais se c’est par samblance de necessité, en ceste maniere : s’ele est mere, donc ayme ele son fiz ; certes tu le poras bien reprendre, et moustrer que ce ne soit par necessité, ains puet estre tot autrement.
\chapterclose


\chapteropen
\chapter[{.III.LXIII. Dou secont deffermement qi nie la conclusion}]{\textsc{.III.LXIII.} Dou secont deffermement qi nie la conclusion}\phantomsection
\label{tresor\_3-63}

\chaptercont
\noindent Li secons deffermemens est quant tu reconnois que li proposemens u la prise de ton adversaire soit veritable, mais tu nies sa conclusion por ce k’ele naist de ce que tu avoies reconneu, ains conclust autre chose qu’ele ne doit ne ne puet. Raison coment : les gens de ta vile alerent a ost, et il avint chose que quant tu i aloies une maladie te prist enmi la voie ke ne te laissa pas aler jusques a l’ost, si ke ton adversaire t’en acuse, et conclust en ceste maniere : se vos fussiés venus en l’ost, nostres connestables vos eust veu ; mais ne vous virent, donques ni volsistes vous pas venir.\par
Or garde que en cestui argument tu affermes bien le proposement ton adversaire, c’est que se tu eusses esté a l’ost li connestable t’eussent veu, et affermes sa prise, c’est k’il ne te virent pas, mais sa conclusion ne naist mie de ce, car la u il dist que tu ni volsis aler, il ne dist mie voir, por ce ke tu i voloies bien aler mais tu ne pooies.\par
Cis essamples est si clers et si ouvers que c’est legiere chose a connoistre son visce, et por ce vieut li mestres moustrer \textsc{.i.} autre essample plus oscur a entendre, por mieus ensegnier ce ki apertient a bon parleour ; car la u li visces est oscurs a entendre, il puet bien estre prouvés autresi comme s’il fust veritables. Et ce puet estre en \textsc{.ii.} manieres, ou por ce k’il quide ke tu affermes a de certes une chose douteuse, ou por ce k’il quide qu’il ne te soviegne pas de ce que tu as affermé et recongneu.\par
Et se c’est k’il quide que tu aies affermé a de certes une chose douteuse por quoi ton adversaire te conclust, lors maintenant te covient il moustrer l’entendement que tu avoies quant tu affermas cele chose, et dire k’il a reploiet son argument a autre chose. Raison comment : tes adversaires dist ensi, vos avés mestier d’argent ; et tu affermes bien son dit selonc ta entention, c’est a dire que tu en voldroies avoir plus grant somme que tu n’as. Mais tes adversaires pense tot autre chose, et dist ensi, mais vous avés mestier d’argent, car se ce ne fust vous feriés marchandise, donc iestes vous povres. Garde donc k’il te conclust par autre entention, et por ce pués tu deffermer son argoment k’il reploia et mua ce que tu entendoies.\par
Mais s’il quide que tu aies oblié çou que tu as recogneu, et coment, il en fera une mauvaise conclusion contre toi, en ceste maniere : se li iretages dou mort apertient a toi, chascuns doit croire que tu l’ochesis.\par
Et sor ce mot dist tes adversaires maintes paroles et assigne plusours raisons a prover sa cause. Et quant il a ce fait il prent son argument, et dist, mais sans faille li iretages apertient a toi, donc l’as tu ocis. Garde donc que ceste conclusion ne siut pas de ce que li iretages apertient a toi. Et por ce te covient il diligement regarder la force de son argument et dont il le trait et coment.
\chapterclose


\chapteropen
\chapter[{.III.LXIIII. Comment on doit deffermer l’argument qi est vicieus}]{\textsc{.III.LXIIII.} Comment on doit deffermer l’argument qi est vicieus}\phantomsection
\label{tresor\_3-64}

\chaptercont
\noindent Li tiers deffermemens est quant tu dis que li argumens ton adversaire est vicieus. Et ce puet estre en \textsc{.ii.} manieres, ou por ce k’il a visce en l’argument meismes, ou por ce qu’il n’apartient pas a ce que li parleres propose. Et sachiez que vices est en l’argument quant il est dou tout faus, ou s’il est communs u universaus u legiers u lontains u mal appropriez u doutous u certains u non affermés u lais u anoieus u contraires u mouvables u adversaires.\par
Faus est celui ki est apertenant de mençoignes. Raison coment : nus ne poroit estre mout sages ki desprise deniers, mais Socrates desprisoit les deniers, donc ne fu il pas sages.\par
Comun est celui ki n’apertient pas a toi plus k’a ton adversaire. Car se tu dis ensi, je dirai briefment por ce que j’ai droit, autresi le puet dire ton adversaire comme tu.\par
Universel est celui ki puet estre retrait sor une autre chose ki n’est pas veritable, en ceste maniere : Segnor juges, je ne m’en fusse pas mis sor vous se je ne quidasse que li drois en fust tousjours devers moi.\par
Legiers est en \textsc{.ii.} manieres une ki est dite a tart, si comme li vilains dist, se je quidasse que l’en emblast mes bués, je eusse fermé l’estable. L’autre maniere est a covrir une laide chose de legier covertour, si comme fist le chevaliers ki deguerpi son roi quant il estoit en sa haute signourie, et puis ke son roi fu essilliés son chevalier l’encontra un jour et li dist, Sire, fist il, vous me devés pardonner ce que je vous deguerpi, por ce ke je m’apareil a aler tous seus a vostre secors.\par
Lontains est celui argument ki est pris trop de long ; selonc ce que fist la chamberiere Medee : Diex vosist, fist ele, ke l’en n’eust pas taillié le mairien de quoi les nés furent faites.\par
Mal apropriés est en \textsc{.iii.} manieres. Une ki dist les proprietés ki autresi sont communs a une autre chose ; car se tu me demandes les proprietés de l’omme ki est descordables, et je disoie que descordables est cil ki est mauvais et anuieus entre les homes, certes ces proprietés ne sont plus dou descordable que de l’orguilleus ne ke d’un fol ne que d’un autre mal home. La seconde maniere dist teux proprietés ki ne sont mie voires, mais fausses ; car se tu demandes des proprietés de sapience, et je disoie que sapience n’est autre chose que gaaignier argent, certes je diroie fausses proprietés. La tierce maniere dist aucunes proprietés mais non pas trestotes ; car se tu me demandes des proprietés de folie, et je disoie que folie est de covoitier haute renomee, certes ja soit ce folie d’aucune part, ne di je mie totes les propriétés de folie.\par
Douteus est celui argumens ki par douteuses causes vieut prover une douteuse chose, en ceste maniere : Segnour princes de la tiere, vous ne devés avoir guerre l’un contre l’autre, por ce que li diex, ki governent les movemens du ciel, ne s’entrecombatent mie.\par
Certains est quant li parleour conclust ce meismes que son adversaire conferme, et laisse ce k’il deust prover. Si comme fist li adversaires Horrestis ; quant il devoit moustrer que Horrestis avoit murdrie sa mere, il moustra k’il l’avoit ocise ; et ce ne besoignoit pas, pour ce k’il ne le denioit mie, ains disoit k’il l’avoit a droit ocise.\par
Non affermé est quant li parleours dist maintes paroles et confermemens sor une chose ke son adversaire nie tot plainement. Raison coment : Ulixes fu acusés k’il avoit ocis Ajacem, mais il disoit que non avoit ; et totesvoies son adversaire disoit grans mos et grans paroles, et ce estoit mout laide chose quant uns vilains ocist \textsc{.i.} si noble chevalier.\par
Lais argumens est celui ki est deshonestes par raison dou leu, c’est a dire lais moz devant l’autel ; ou par raison de celui ki les dist, c’est s’uns evesques parole des femes et de luxure ; u par raison dou tens, c’est se au jour de Paskes on desist que Diex ne resurrexist mie ; ou par raison des oïans, c’est se devant les religieus l’om parole de vanités et de delit dou siecle ; u par raison de la chose, c’est a dire, ki parole de la Sainte Crois ne doit pas dire que ce soient forques.\par
Anoieus est celui ki anuie a la volenté des oïans ; car se par devant le prestre je loaisse la loi ki dampne luxure, certes mon argument anuieroit as oïans.\par
Contraires argumens est quant li parleours dist contre ce meismes que li oïant firent. Raison coment : je vais devant Alixandre, et acuserai aucun preudome ki avoit une cité vaincue a force d’armes, et dirai k’il n’i a au monde si cruel chose con de prendre cités et gaster les ; certes iteus argumens est bien contraires, por ce que li oïans, c’est Alixandres, deforfist plusors viles et cités.\par
Muables est quant li parleors dist d’une meisme chose \textsc{.ii.} diversités ki sont l’un contre l’autre ; selonc ce que uns hom dist, kiconques a les vertus, il n’a pas mestier d’autres a bien vivre, et poi aprés dist il meismes que nus ne puet bien vivre sans santé. Et uns autres quant il ot dit k’il servoit a son ami par amours, et puis aprés si dist il k’il atendoit de lui grans services.\par
Adversaires est celui argument ki plus fait contre le parleour que por sa partie ; car se je voloie connoistre le chevalier a la bataille, et je desisse que nos enemis sont grans et fors et bonseurés, certes ce seroit plus contre moi que por moi\par
Or covient a dire de l’autre maniere de l’argument ki est vicieus, c’est quant il n’apertient pas a ce que li parleor propose. Et ce puet estre en maintes manieres ; c’est se li parleours promet k’il dira de plusors choses, et puis ne dist que de l’une ; u se il doit moustrer tout, et il ne moustre c’une partie, c’est a dire se li parleours vieut moustrer que toutes femes sont averes et il ne le moustre que d’une u de \textsc{.ii.} ; u se il ne se deffent de çou dont il est blasmés, selonc ce que fist Pankuves quant il volt deffendre musike, ki estoit blasmee par plusours, il ne le deffendi pas mais il loa molt sapience. Autresi fist cil ki estoit blasmés de vaine gloire, car il ne s’en deffendi pas, ains dist k’il estoit mout fiers et hardis as armes ; u se la chose est blasmee par le visce de l’home, si comme sont cil ki dient mal de sainte eglise por la malvestié des prelas ; u se je voloie loer \textsc{.i.} home, je diroie k’il est mout riches et bonseureus, mais je ne di pas k’il ait aucune vertu ; u se je faç comparison entre deus homes u entre deus choses en tel maniere k’il ne quide pas que je puisse loer l’un sans blasmer l’autre.\par
U il loe l’une seulement et ne fait de l’autre nule mention comme se nous fuissons a conseil por establir liquex vaut miex u la pais u la guerre, je ne fineroie de loer la pais, mais de la guerre ne diroie ne ce ne coi ; u se je demandoie d’une certaine chose et tu me respons d’une general, car se je te demande de l’ours s’il cort, et tu dis que \textsc{.i.} home et \textsc{.i.} animal cort.\par
Ou se la raisons que li parleors rent est fausse, car s’il dist que deniers sont bons por ce k’il donent plus bonneeureuse vie que rien dou monde, certes la raisons est fausse, por ce que denier donent a home grandesime travail et maleurté, selonc Dieu et selonc le siecle.\par
U se li parleours rent foible raison de son dit, selonc ce ke Plauctus fist : il n’est mie bon, fist il, ke l’en chastist son ami de ses meffais dev nt tens, por coi je ne voeil pas hui chastoier mon ami des maus k’il a fais.\par
Ou se li parleours rent tel raison de son dit que ce est ce meismes que son dit, car s’il dist k’avarisce est trop male chose por ce que covoitise d’argent a ja fés mains grans damages a maintes gens, certes avarice et covoitise sont une chose.\par
Ou se li parleours rent une petite raison la u il le poroit rendre grant, car s’il dist, bone chose est amistié por ce que l’en en a maint delit, certes il poroit dire millour raison et dire k’il i a maint proufit et honesteté et viertu.
\chapterclose


\chapteropen
\chapter[{.III.LXV. Du quart deffermement qi dist autresi fermes raisons que ses adversaires u plus}]{\textsc{.III.LXV.} Du quart deffermement qi dist autresi fermes raisons que ses adversaires u plus}\phantomsection
\label{tresor\_3-65}

\chaptercont
\noindent Li quars deffermemens est quant ton adversaire a dit son argument, tu redis a l’encontre \textsc{.i.} autre autresi fort u plus ; et teus argumens apertienent plus a content ki sont sour conseil prendre que a autres choses.\par
Et sachiés que cest deffermement puet estre fet en \textsc{.ii.} manieres. La premiere est quant mes adversaires dit une chose que je consenc, et ensi est fermee, mais tot maintenant je redi a l’encontre une autre plus estable raison ki est fermee par necessaires argumens. Car la u Cesar disoit, nous devons, fist il, pardonner as conjurés pour ce k’il sont nos citains, voirs est, dist Catons, k’il sont nos citeins, mais s’il ne sont dampnés, il covient a fine force que Rome en soit destruite por aus.\par
La seconde maniere est quant mes adversaires dit d’une chose qu’ele est proufitable et je di ke voirs est, mais je moustre tout maintenant que ce que je di est honeste chose, car sans faille honestés est plus ferme chose que proufit, ou autant. Mais ci se taist li mestres a parler de la quinte branche dou conte, c’est dou deffermement, de quoi il a dit ce k’il en savoit dire ; si dira il de la sisime branche, c’est de conclusion.
\chapterclose


\chapteropen
\chapter[{.III.LXVI. De la sisime branche, c’est conclusion}]{\textsc{.III.LXVI.} De la sisime branche, c’est conclusion}\phantomsection
\label{tresor\_3-66}

\chaptercont
\noindent Après la doctrine dou deffermement et de toutes les \textsc{.v.} premieres brankes dou conte, vient la derraine branche, c’est la conclusion, la u li parleours conclust les raisons et pose fin a son conte. Et neporquant nous trovons ke Ermagoras dist en ses livres que devant la conclusion doit estre le trespas : ensi faisoit \textsc{.vii.} branches au content. Mais li trés sages Tuilles Cicerons, ki de bone parleure passa toz homes, blasme trop la sentence Ermagoras.\par
Et vous avés bien oii ça ariere que trespas est quant li parleours ist un pichot de sa propre matire et trespasse a \textsc{.i.} autre, por achoison de loer soi et sa partie u de blasmer son adversaire et sa partie u por achoison de confermer u de deffermer non mie por argument mais por acroistre sa cause, selonc ce que li mestres devise ça ariere el chapistre la u il dist coment on puet acroistre sa matire en maint autres lieus. De cestui trespas dist Tuilles k’il n’est ne ne doit estre tot par lui branche dou conte.\par
Et por ce s’en taist ore li mestres, et dist que conclusion est l’issue et la fin dou conte ; ains est sousmis as argumens des branches dou conte. Et sachiés ke la conclusion a \textsc{.iii.} parties, ce sont reconte, desdaing, et pité ; et vous orés de chascun par soi diligement, et premierement de reconte.
\chapterclose


\chapteropen
\chapter[{.III.LXVII. De raconte}]{\textsc{.III.LXVII.} De raconte}\phantomsection
\label{tresor\_3-67}

\chaptercont
\noindent Raconte est celui fin dou conte en quoi li parleours briement et en somme reconte tous ses argumens et les raisons qu’il avoit contees parmi son dit, les unes cha et les autres la ; et les ramentoit en briés mos por torner les a memore des oïans plus fermement. Mais por ce que se li parleours faisoit tozjours son reconte d’une maniere solement, li oïant en seroient souspecenous et quideroient que ce fust chose apensee ; por ce covient il sovent varier et reconter ore en une maniere ore en une autre, selonc ce que tu poras veoir ci desous.\par
Et tu pués bien aucunefois ramentevoir la some de chascun de ces argumens par soi, car c’est assés legiere chose a dire et a entendre. Aucunefois pués tu bien reconter toutes les parties que tu dois en ton devisement et que tu promeis de prover, et ramentevoir toutes les raisons comment tu les as provees et confermees.\par
Aucunefois pués tu demander as oïans en ceste maniere : Segnour, que querés que volees autre chose ? et quoi plus ? je ai ce dit et si ai prouvé ce et cel autre. En celi maniere ramentoive et tes dis et tes argumens, car li oïant sauront mieus et quideront k il n’i ait plus a prover.\par
Aucunefois pués tu ramentevoir tes raisons et prover, sans noient dire des raisons ton adversaire ; et aucunefois dire des argumens avec les tiens, en tel maniere que, quant tu dis un des argumens ton adversaire, tu dois maintenant dire comment tu l’as deffermee et defacie ; car c’est une maniere de reconter por quoi les oïans se sovienent de tout çou que tu as confermé et deffermé.\par
Aucunefois pués tu amonester les oïans de ta bouche que il lor soviegne de ce que tu as dit en quel lieu et comment. Aucunefois pués tu nomer \textsc{.i.} autre home autresi comme s’il parlast, et metre sor lui ton reconte, en ceste maniere : je vous ai ce apris et moustré ce et cel autre : mais s’il fust enki Tuilles, ke li demanderois vous plus ?\par
Aucunefois pués tu nomer une autre chose ki ne soit pas home, si comme est loi u un liu u une cité et teus autres choses samblables, et metre sor lui ton reconte, en ceste maniere : se la loi peust parler, ne se plainderoit ele devant vous, et diroit, que querés, que demandés plus quant on prueve ce et cel autre et moustre si clerement com vous avés oii conter ?\par
Et sachiés k’en ces \textsc{.ii.} manieres, c’est d’un autre cors d’ome et d’une autre chose, pués tu ensivre toutes les varieté ki sont posees ci deseure. Mais le general ensegnement de toutes manieres de raconter est que de chascun de tes argumens tu saches triier et prenre ce que plus vaut, et reconter le au plus brief que tu onques poras, en tel maniere k’il samble que la memore soit renovelee, non pas le parlement.
\chapterclose


\chapteropen
\chapter[{.III.LXVIII. De desdaing}]{\textsc{.III.LXVIII.} De desdaing}\phantomsection
\label{tresor\_3-68}

\chaptercont
\noindent Desdaing est celui fin dou conte en quoi li parleour met un cors d’ome u d’autre en grant haine et en grief malevoeillance. Et sachiés ke cestui desdaing naist de cel lieu meismes de quoi naissent confermement et deffermement et des proprietés du cors et de la chose, selonc ce que le livre devise cha en ariere en ses chapistres ; car ce sont li lieu par quoi on puet acroistre les crimes et les forfés et tout desdaing. Et neporquant li mestres devisera enki les ensegnemens ki apertienent tout droit au desdaing.\par
Le premier lieu de desdaing est pris par auctorité ; c’est a dire quant je dis ke cele chose a esté de grant estuide a Deu, u as homes de grant auctorité. Et ce puet estre moustré par raison de sort u de divin mandement, u des prophetes u de mervilles et de teus choses samblables. Autresi puet il estre mostré par la raison de lor ainsnés u de nos signeurs u des cités u des gens u des trés sages homes u del signat u del peuple u de ciaus ki firent la loi.\par
Raison coment : il fu voirs que Judas deguerpi les disciples par traison. Li autre apostle geterent sort pour veir ki deust estre mis en son lieu ; le sort en vint sor Mathie ki fu apostles en lieu de Judas. Mais s’il s’en fust encondis et ne le volsist pas estre, on peust metre sor lui desdaing, en ceste maniere : nus ne te doit amer quant tu refuses ce que Diex nous a moustré par sort. De cestui essample se passe li mestres, car il soufist bien a entendre tous les autres lieus devant dis.\par
Li secons lieus de desdaing est pris quant li parleours croit le forfait por courous, et moustrent a qui il apertient ; car s’il est contre tous homes ou contre plusours, c’est grant cruauté ; et s’il est contre les graindres et ki sont plus digne de nous, c’est grant desdaing ; et se c’est contre nos pers c’est grans mauvaistiés, et ce c’est contre les fiebles c’est grans fiertés.\par
Li tiers lieus de desdaing est pris quant li parleour dit autresi comme en demandant le mal ki en puet avenir se li autre faisoient ce que ses adversaires a fet, et que l’en li pardone ce meffet : maint desdaing oseront faire teus et pieurs oevres dont il puet avenir grans periz.\par
Le quart lieu est quant li parleours dist as juges que maintes gens regarderont a ce k’il establiront sor celui meffet, por savoir que il lor loist a fere se il pardonent a lui.\par
Li cinquime lieus est quant li parleour dist que toz autres jugemens, s’il fussent contre droit, poroient estre amendés, mais cil crimes est de tel nature ke ce ki en sera jugiés une fois iert si estable k’il ne pora pas estre remué por autres sentences ne par jugement de nului.\par
Le sisime lieu est quant li parleour dist que son adversaire a ce fait penseement et par conseil, et que nus ne doit pardoner le tortfet que l’en fait de son gré, ja soit ce que on puet aucunefois pardonner ciaus qui avienent contre lor gré et nonsachant.\par
Le septime lieu est quant le parleour dist ke son adversaire par sa poissance et par ses richesces a fait une si cruel chose et si desaperte comme c’est a oïr.\par
Le octime lieu est quant li parleours dist que une si piesme chose ne fu onques oïe ne poi ne grant, et que nus tyrans, nule beste, ne sarrasins ne juis, ne l’osa onques faire ; et nome ciaus contre qui il a ce fait, c’est contre son pere u contre ses fiz ou contre sa feme u contre ses parens u ses subtés u ses ainsnés u contre son oste u son voisin u son compaignon, son ami, son mestre, u contre les mors u contre les chetis et les foibles, u contre ciaus ki ne se puënt aidier, si comme sont enfant, vieillars, et femes et malades ; car de toutes tens choses naist \textsc{.i.} cruel desdaing par qui li oïant sont fierement commeu contre ciaus ki font teus et samblables oevres.\par
Li nuevimes lieus est quant li parleours ramentoit une grant mauvaisté provee, et dist que ce que son adversaire fist est d’assés plus grrief et de grignour periz que cel autre.\par
Li \textsc{.x.} lieus est quant li parleours ramentoit tote la besoigne par ordre, si comme il fu en la chose faisant et ki fu aprés a la fin, et croist le desdaing et la cruauté de chascune chose par sai tant comme il puet, et le demoustre as oïans autresi comme s’il l’eussent veu en sa presence.\par
Li onsimes lieus est quant li parleour dit de celui ki a ce fait : ne le devoit pas faire, ains devoit metre et cuer et cors por deffendre ke ce ne fust pas fait. Li dousimes argumens est quant li parleours dist autresi comme par courous que l’en a ce fait a lui tous premiers ki onques ne fu fet a nului.\par
Li trezimes lieus est quant li parleours dit que outre ce mal que son adversaires li a fet il li dist mains crueus moz et reproces et manaces.\par
Li quatorzimes lieus est quant li parleours prie les oïans k’il tornent sor aus le torfait que on li a fait, c’est a dire que se li mal est des enfans k’il le tornent sor ses fiz, et s’il est de femes k’il le tornent sor leur femes.\par
Li quinzimes lieus est quant li parleors dit ke ce ke li est avenu sieut sambler grief et cruel ad son adversaire. Et en somme, ce que li parleours dist par desdaing, il le doit dire au plus griement k’il onques puet, si k’il mue les corages des oïans contre son adversaire ; car c’est une chose ki mout proufite a sa cause quant li oïant sont commeu par courous contre son adversaire.
\chapterclose


\chapteropen
\chapter[{.III.LXVIIII. De pitié}]{\textsc{.III.LXVIIII.} De pitié}\phantomsection
\label{tresor\_3-69}

\chaptercont
\noindent Pitié est un dit ki a la fin aquiert la misericorde des oïans. Et pour ce li parleour ki vieut finer et clore son dit par pitiet doit faire \textsc{.ii.} choses : une ki adoucisse les corages des oïans en tel maniere k’il n’aient nul troublement contre lui, et s’il l’ont, k’il tornent a debonnairetet ; l’autre si est k’il face tant que li oïant aient misericorde de lui, c’est a dire k’il lor poise de son damage, por ce que quant li oïant soient a ce venus k’il sont debonnaires et k’il n’aient nul torblement contre lui et k’il lor poise de ton mal, certes il sont legierement esmeu a pité.\par
Et a ce faire doit li parleours torner as communs lieus, c’est a la force de fortune et a la foiblece des homes ; car la u tu dis bien ces choses, il ne seront ja de si dur cuer k’il ne tort a misericorde, meismement quant il consire que l’autrui mal puet venir sor lui et sor ses choses.\par
Et sachiés que li lieus ki apertient a aquerre pitié sont \textsc{.xvi.} Li premiers est quant li parleours conte les biens k’il soloit avoir jadis et moustre le mal k’il suefre maintenant.\par
Li secons lieus est quant li parleours monstre le mal k’il a eu jadis et ciaus k’il a maintenant et ciaus k’il aura ça avant.\par
Li tiers lieus est quant li parleours se plaint et nome tous ses maus ; si comme li peres se plainsist de la mort son fil, et nomast le delit k’il avoit de sa joventé et l’esperance k’il avoit de lui et la trés grant amour k’il li portoit et le sollas et la noreture et les autres choses samblables.\par
Le quart lieu est quant li parleours se plaint k’il a soufiert u k’il li covient soufrir laides choses u vils de siervage, lesquex il ne deust soufrir por la raison de son aage u de son linage u de sa fortune u de sa signorie u por le bien k’il a ja fet.\par
Li cinkimes lieus est quant li parleours devise par devant les oils des oïans tous les maus ki lui sont avenu autresi comme s’il les veissent ; car c’est une maniere par qui li parleour sont commeu, autresi bien por la force dou fais comme par la force dou dit.\par
Li sisimes lieus est quant li parleours moustre que hors de son esperance il est venus en maleurté, et que la u il atendoit que de cel home u de cele chose li deust avenir grandisimes proufiz, il n’en a noient ançois en est cheus en grant mal aventure.\par
Li septimes lieus est quant li parleours torne ses maus viers les oïans et lor prie que quant il esgardent lui k’il lor souviegne de ses fiz et de ses parens et de ses amis.\par
Li witimes lieus est quant li parleors moustre ke fait soit ancun desavenant u que ce ki estoit avenant ne fu pas fait ; selonc ce que dist Cornele la feme Pompei : lasse, fist ele, que je né fui pas a son definement ; je ne le vi, je ne oï son derrenier mot, ne ne rechiu son esperit ; en ceste maniere se plaignoit sa feme et moustroit ke ce ki estoit avenant ne fu pas fet ; et maintenant moustra comment fu fet le desavenant ; la u ele dist, il morut, fist ele, es mains de ses anemis, il jut vilainement en la tiere de ses guerriés, il n’ot onques sepulture ne point d’onnorableté a sa mort, et sa charoigne fu longuement trainee par les bestes sauvages.\par
Li \textsc{.ix.} lieus est quant li parleors torne son dit sor aucune bieste u sor une autre chose sans sens et sans parler ; car c’est une maniere de parler ki mout entre ens cuers des oïans ; selonc ce que fist la feme Pompei : gardés, fist ele, coment son ostel pleure, sa robe et son harnois se plaignent, son cheval et ses armes recontent ses torsfais.\par
Li \textsc{.x.} lieus est quant li parleours se plaint de sa povreté, de sa maladie, de sa foiblece, de sa sollitude ; selonc que fist la feme Pompei : ha lasse ! comme je sui desoremais povre et nue sans nul pooir, je serai tote seule sans signor et sans nul conseil.\par
Li onsimes lieus est quant on parole de ses enfans u de son pere et de son cors entierer, selonc ce que Eneas dist a ses gens quant il fu chaciés de Troie : je ne sai, fist il, comment il sera de ma vie u de ma mort entre tant de periz, mais je laisse mon fil en vos mains, je vos prie de lui et de mon pere et que mon cors soit entierés honorablement se je muir.\par
Li dousimes liens est quant l’om se desoivre de ciaus ke l’om ayme tenrement et moustre quel doleur, quel damage il avoient a lui u a ciaus de sa desevrance.\par
Li \textsc{.xiii.} lieus est quant li parleours se plaint ke teus gens li font mal et anui ki li devroient bien faire et honour.\par
Li quatorsimes lieus est quant li parleours humlement prie les oïans autresi comme en plorant k’il aient pitié de lui et de ses maus.\par
Li \textsc{.xv.} lieus est quant li parleours ne se dieut de son mal, mais il se plaint de la mescheance son ami u de son parent ; selonc ce que Catons disoit contre les conjurés de Rome : il ne me causist de moi, fist il, mais il me poise de la destruction de nostre commun, de nos fis, et de nos gens.\par
Li \textsc{.xvi.} lieus est quant li parleors dist k’il li poise mout fierement du mal des autres, et neporquant il mostre k’il ait grant cuer et franc de soufrir tous periz ; car il avient sovent as princes de la tiere et as autres ki ont auctorité de signorie u des vertus, ke s’il dient hautes paroles et mostrent franc corage, li oïant en sont commeu a misericorde plus tost et miex que par proieres u par humilités. Et sachiés que c’est une maniere de parler a quoi se tornent tuit connestable et li signour des os quant il welent les lor connorter a la bataille.\par
Or avés oï toz les lieus por aquerre la misericorde des oïans. Mais li parleours doit mout garder que la u il aperçoit que li corage sont commeu de pitié, k’il ne demeurt plus en sa plainte, mais tout maintenant fine son dit devant ce que li oïant issent de pitié ; car Appolles dist, nule rien seche si tost comme larmes.
\chapterclose


\chapteropen
\chapter[{.III.LXX. De la diversité qi est entre ditteors et parleours, et de la conclusion}]{\textsc{.III.LXX.} De la diversité qi est entre ditteors et parleours, et de la conclusion}\phantomsection
\label{tresor\_3-70}

\chaptercont
\noindent Ci sont les \textsc{.iii.} parties de la droite conclusion, ki apertient a bien parler, selonc les ensegnemens de Tuille. Mais li ditteour s’en descordent \textsc{.i.} pichot ; car en la conclusion ki est en parlant comprent li parleours sa demande et la some de ses raisons, et fine son conte ; més es letres ke l’om envoie as autres, quant li ditteor ont escrit les premieres branches, c’est la saluence, le prologue, le fait, la demande, et k’il a priiet u demandé ce k’il vieut, il escrit maintenant le bien ki en puet avenir se l’en fait sa requeste, u le mal se l’om ne le fait, et pose fin a sa letre ; et c’est sa conclusion. Mais ci se taist li mestres de conclusion, por mostrer des autres doctrines.
\chapterclose


\chapteropen
\chapter[{.III.LXXI. Coment li contes puet estre a mains de .vi. branches}]{\textsc{.III.LXXI.} Coment li contes puet estre a mains de \textsc{.vi.} branches}\phantomsection
\label{tresor\_3-71}

\chaptercont
\noindent Jusques ci a devisé li mestres les branches dou conte, et a moustré diligement toz les ensegnemens ki a ce covienent selonc l’auctorité de Tuille et des autres mestres de rectorique. Et ja soit ce k’il die que uns contes de bouche a \textsc{.vi.} branches et c’une letre en a \textsc{.v.}, selonc ce ke vous avés oï ça arieres, neporquant li contes poroit bien estre de tel maniere k’il ne requerroit pas toutes les branches et les parties devant dites, ains seroit assés d’une branche sans plus, u de \textsc{.ii.} u de \textsc{.iii.} u de \textsc{.iiii.} u de \textsc{.v.}, selonc la nature dou fait. Et pour mieus connoistre comment ce est, te couvient il savoir que les unes branches sont si sustancieles que l’en ne puet rien dire se par eles non, si comme est le fait et la demande ; car sans une de ces \textsc{.ii.} ne puet iestre nus contes de bouche ne d’escripture.\par
Mais les autres branches, sont la saluence, li prologues, et li devisemens, et confermemens, et li deffermemens, et conclusion, ne sont mie dou tout de la substance dou conte. Car letres et messages puet bien aucunefois estre sans saluence, u por ce que s’uns autres ovrist les letres que il ne seust les nons, u por ce que li messages est de tel maniere que li messagiers nomera et les uns et les autres plusour fois en son conte, et lors n’a en celi letre ne en celui message que \textsc{.iiii.} branches de remanant.\par
Mais quant la matire est si honieste k’ele par sa dignité plaist as oïans sans nule doreure de prologue, lors se puet l’om bien taire dou prologue, et dire sa besoigne selonc ce ke vos avés oï cha ariere el chapitle des prologues. Autresi bien puet on laissier le devisement et le confermement et le deffermement et la conclusion, et dire simplement le fait u sa demande. A ce poés vous entendre ke ancunefois est a dire assés le fait seulement, en ceste maniere : sachiés que nous somes en Franche. Et aucunefois soufist a dire la demande sans plus, en ceste maniere ; je te pri ke tu soies prodom en ceste guerre. Aucunefois est assés a dire et l’un et l’autre, en ceste maniere : vous veés bien ke nous somes a la bataille venu (c’est le fait), dont je vos pri que vos soiés preu et hardi contre nos enemis (c’est la demande).\par
Et si comme uns contes puet estre de \textsc{.ii.} branches u de l’une sans plus, tot autresi puet il estre que l’une des \textsc{.ii.} u andeus soient acompaignies a une u as \textsc{.ii.} u a plus des autres branches devant dites, selonc ce que li sages parleours voit k’il covient a sa matire.
\chapterclose


\chapteropen
\chapter[{.III.LXXII. Des branches qi ont estable lieus determinés}]{\textsc{.III.LXXII.} Des branches qi ont estable lieus determinés}\phantomsection
\label{tresor\_3-72}

\chaptercont
\noindent Et si comme il a au conte unes branches sans quoi il ne puet pas iestre et \textsc{.i.} autre sans coi il puet bien estre, tot autresi ont les unes si propre lieus et si certains sieges que ailleurs ne poroient pas iestre, et les autres sont si muables que li parleours le puet remuer de lieu en lieu si comme il vieut. Car la saluance ne puet estre mise se au commencement non, et la conclusion a la fin, mais totes les autres parties puet li parleour metre hors de son lieu, selonc sa porveance. Mais de ce se taist li mestres, et tourne a autre chose.
\chapterclose


\chapteropen
\chapter[{.III.LXXIII. Dou governement des cités}]{\textsc{.III.LXXIII.} Dou governement des cités}\phantomsection
\label{tresor\_3-73}

\chaptercont
\noindent Es premiers livres devant sont devisees les natures et li commencemens des choses dou siede, et li ensegnement des visces et des vertus, et la doctrine de bone parleure. Mais en ceste derraine partie vieut mestre Brunet Latin acomplir a son ami ce ke li avoit promis entour le commencement dou premier livre, la u il dist que son livre defineroit en politique, c’est a dire des governemens des cités, ki est la plus noble et la plus haute science et li plus nobles offices ki soit en tiere, selonc ce que Aristotles prueve en son livre.\par
Et ja soit ce ke politike compregne generaument tous les ars ki besoignent a la communité des homes, neporquant li mestres ne s’entremet se de ce non ki apertient au cors dou signor et a son droit office. Car des lor ke gens commencierent premierement a croistre et a mouteplier et ke li pechié dou premier home s’enracina sor son linage et ke li siecles enpira durement si ke li un covoitoient les choses a lor voisins - li autre par lor orgueil sousmetoient les plus foibles au jug de servage - il covint a fine force ke cil ki voloient vivre de lor droit et eschiver la force des maufeteurs se tornaissent ensamble en \textsc{.i.} lieu et en \textsc{.i.} ordre. Dé lors commencierent a fonder maisons et fermer viles et fortereces et clore les de murs et de fosses.\par
Dé lors commencierent a establir les coutumes et la loi et les drois ki estoient communs par trestous les borgois de la vile. Por ce dist Tuilles ke cités est uns assamblemens de gens a abiter en un lieu et vivre a une loi. Et si comme les gens et les habitations sont diverses, et li us et li droit sont divers parmi le monde, tot autresi ont il diverses manieres de signories ; car dé lors ke Nembrot li jaians sorprist premierement le roiaume dou païs, et ke covoitise sema la guerre et les morteus haines entre les gens du siecle, il covint as homes k’il eussent signors de plusors manieres, selonc ce que li un furent esleu a droit et li autre i furent par lor pooir. Et ensi avint que li uns fu sires et rois dou païs, li autre furent chastelain et gardeour de chastiaus, li autres furent dus et conduisieres des os, li autre furent quens et compaignons le roi, li autre avoient des autres offices, dont chascuns avoit sa tiere et ses homes a governer.\par
Mais tous signours et tous officiaus, u il sont perpetuaus a tousjours par lui et par ses oirs, si comme sont rois et quens et chastelains et li autre samblable ; ou il sont a tous les jours de lor vie, si comme est mesire li apostoiles ou l’empereour de Rome et li autre esleu a lor vie ; u il sont par annees, si comme sont maieur et prouvost et la poestés et la eschievins des cités et des viles ; u il sont sor aucunes especiaus choses, si comme sont li legas et li delegas et li vigueres et li officiaus a qui li plus grant signor baillent a faire aucunes choses ou sor quoi l’om se met a faire lor questions. Mais de tous se taist li mestres en cest livre, que il n’en dit noient de la signourie des autres se de ciaus non ki governent les viles par annees.\par
Et cil sont en \textsc{.ii.} manieres ; uns ki sont en France et es autres païs, ki sont sozmis a la signorie des rois et des autres princes perpetuens, ki vendent les provostés et les baillent a ciaus ki plus l’achatent (poi gardent sa bonté ne le proufit des borgois) ;\par
l’autre est en Ytaile, que li citain et li borgois et li communité des viles eslisent lor poesté et lor signour tel comme il quident qu’il soit plus proufitables au commun preu de la vile et de tous lor subtés. Et sor ceste maniere parole li mestres, car l’autre n’apertient pas a lui ne a son ami ; et nonporquant tot signour, quel signorie k’il aient, en poroient prendre mains bons ensegnemens.
\chapterclose


\chapteropen
\chapter[{.III.LXXIIII. De signorie et de ses pilers}]{\textsc{.III.LXXIIII.} De signorie et de ses pilers}\phantomsection
\label{tresor\_3-74}

\chaptercont
\noindent Toutes signories et toutes dignités nous sont baillies par le Souverain Pere, ki entre les sains establissemens des choses dou siecle volt que li governemens des viles fust fermés de trois pilers, c’est de justice et de reverence et d’amor.\par
Justice doit estre si establement fermee dedens le cuer au signor, k’il doinst a chascun son droit, ne que il soit ploiés ne a diestre ne a senestre ; car Salemons dit que justes rois n’aura ja mescheance.\par
Reverence doit estre en ses borgois et en ses subtés ; car c’est la seule chose au monde ki porsiut les merites de foi et ki sormonte toz sacrefices ; por ce dist li Apostres : honorés, fet il, Nostre Signeur.\par
Amor doit estre et en l’un et en l’autre ; car li sires doit amer ses subtés de grant cuer et de clere foi, et veillier de jour et de nuit au commun proufit de la vile et de tous homes. Tot autresi doivent il amer lor signour a droit cuer et a veraie entention, et doner li conseil et aide a maintenir son office ; car a ce ki n’est k’un seul entr’aus, il ne poroit rien fere se par aus non.
\chapterclose


\chapteropen
\chapter[{.III.LXXV. Quex hons doit estre esleus a signour et a governeor}]{\textsc{.III.LXXV.} Quex hons doit estre esleus a signour et a governeor}\phantomsection
\label{tresor\_3-75}

\chaptercont
\noindent Et por ce que li sires est autresi comme li chiés des citeins, et que tot home desirent a avoir saine teste, por ce que quant le chief est deshaitiés, toz les membres en sont malades : por ce doivent il sor totes choses estudier k’il aient tel gouverneour ki les conduise a bone fin selonc droit et selonc justice. Il ne le doivent pas eslire par sors ne par cheance de fortune, mais par grant porveance de sage conseil, en quoi il doivent consirer \textsc{.xii.} choses.\par
La premiere est k’Aristotles dit ke par longue prueve de maintes choses devient hom sages, et longue prueve ne puet nus avoir se par longue vie non. Donques pert il que joenes hom ne puet pas iestre sages, ja soit ce k’il puet avoir bon engien de savoir ; et por ce dist Salemons que mar est a la terre ki a joene roi. Et nanporquant on puet bien estre de grant aage et de petit sen ; car autant vaut a estre joene de sens comme d’aage. Por çou doivent li borgois eslire tel signour ki ne soit joenes ne en l’un ne en l’autre mieus vaut k’il soit vieus en chascun. Ne por nient ne devea la lois que nus ne deust avoir dignités dedens les \textsc{.xxv.} ans, ja soit ce que les decretales de sainte eglise les donent aprés les \textsc{.xx.} ans d’aage.\par
La seconde est k’il ne regardent a la puissance de lui ne de son linage, mais a la noblece de son cuer et a la honorableté de ses meurs et de sa vie et as vertueuses oevres qu’il siut faire en son ostel et en ses autres segnories ; car la maisons doit estre honoree par son bon signour et non le signor par sa bonne maison ; mais s’il est nobles de cuer et de lignie, certes il en vaut trop mieus en toutes choses.\par
La tierce est k’il aient justice ; car Tuilles dist que sens sans justice si n’est pas sens, mais malisce, ne nule chose puet valoir sans justice.\par
La quarte est k’il ait bon engien et soutil entendement a connoistre toutes les verités des choses et a entendre et a savoir legierement ce ki covient, et aperçoivre la raison des choses ; car c’est laide chose a estre deceus por poverté de cognoissance.\par
La quinte est k’il soit fors et estables de grant corage, non pas de mauvais et de vaine gloire, et k’il ne croie pas legierement au dit de tous. Il fu jadis une cités dont nus ne pooit estre sires se li mieudres non, et tant comme cele coustume dura il n’avint au païs mescheance ne nule povreté. Que cil puet tant comme il vaut ki ne quide de soi plus k’il en soit. Et nus n’est tenus a preudome par sa dignité, mais par ses oevres, car li sages hom ayme mieus a estre sires que sambler le.\par
La sisime est k’il ne soit covoiteus d’argent ne de ses autres volentés ; car ce sont \textsc{.ii.} choses ki tost le gieteroient de sa chaiere, et il est mout deshonorable chose que celui ki ne se laisse ploier par paour soit depechiés par deniers, et ki ne se laisse vaincre as grans travaus, k’il soit vencus par ses volentés. Mais mout doit hom garder k’il ne soit trop desirans de dignités avoir, car maintesfois vaut il mieus a laissier que aprenre les.\par
La septime est k’il soit trés bons parliers, car il afiert as signors k’il parole mieus que li autre, por ce que toz li mondes tient a plus sage celui ki sagement dist, meismement s’il est joenes hom. Mais sor toutes choses covient il a garder qu’il ne parole trop, por ce que en trop dire ne faut pechiés ; et si com une seule corde descorde toute une citole, tout autresi por \textsc{.i.} mauvais mot dechiet s’onnour et son dit.\par
L’uitime est k’il ne soit desmesurés en despendre ne gasteour de ses choses, car tout home ki ce font covient cheoir a ravine et a larrecin ; et neporquant il ne doit pas eschiver ce visce en tel maniere k’il en soit eschars ne avers, car c’est la chose ki plus vilement honnist cors de son signour.\par
La \textsc{.ix.} est k’il ne soit trop courouçables et k’il ne dure trop en sa ire et en son mautalent ; car ire ki abite en signorie est samblable a foudre, et ne laisse la verité cognoistre ne droit jugement jugier.\par
La disime est k’il soit riches et manans, car s’il est garnis des autres vertus c’est samblant k’il ne soit corrompus par deniers. Et nanporquant je loe plus bon povre que malvais riche.\par
La onsime k’il n’ait lors autre signorie, k’il n’e pas creable que nul home soit soufissans a \textsc{.ii.} choses de si grant pesantour comme governemens de gens est.\par
Li .xiimes, c’est la some de toutes choses, est k’il ait droite foi a Dieu et as homes, car sans foi et sans loiauté n’est ja droiture gardee.\par
Ciaus et les autres vertus doivent li bon citein garder avant k’il eslisent lor signor, en tel maniere k’il ait en lui tant de bones teches comme il en puet plus avoir ; car li plusor n’esgardent pas a lor meurs, ançois se tienent a la force de lui u a son linage, u a sa volenté u a l’amour de la vile dont il est.\par
Mais il en sont deceu, car a ce ke guerre et haine est si mutepliee entre les ytaliens au tans d’ore, et parmi le monde en maintes terres, k’il a devision en trestoutes les viles et enemistié entre les \textsc{.ii.} parties des borgois, certes, kiconques aquiert l’amour des uns il li covient avoir la malevoeillance de l’autre. D’autrepart se cis provos n’est bien sages, il chiet en despit et en mautalent de ciaus meismes ki l’eslurent, en tel maniere que en ce que chascuns esperoit son bien, il trueve son damage.
\chapterclose


\chapteropen
\chapter[{.III.LXXVI. Comment et en quel maniere li sires doit estre esleus}]{\textsc{.III.LXXVI.} Comment et en quel maniere li sires doit estre esleus}\phantomsection
\label{tresor\_3-76}

\chaptercont
\noindent Et quant li sage gent de la vile asquex apertient la election sont en acorde d’aucun preudome, il doivent maintenant regarder les us et la loi et les constitutions de la vile et selonc çou doivent eslire lor poesté el non de celui ki done toz honors et toz biens. Et maintenant doit on escrire les letres bien et sagement ; et segnefiier au preudome comment il l’ont esleu, et establi comment il soit sires et provos l’an aprés de lor terre ; et mander lui briement la sonme de tout son office, et esclairies toutes choses au commencement, si ke nule erreur n’en puisse pas sourdre. Et por ce doit on nomeement mander le jour k’il doit estre corporaument dedens la vile et faire son sairement as constitutions des choses, et k’il doit amener avec soi juges et notaires et autres officiaus por faire ses choses et ses atours, et quant jours il li covenra demorer aprés la fin por rendre son conte, et la raison de ce c’on li voldra demander contre lui, et quel loier il doit avoir et comment et quans chevaus il doit amener et coment, et que tous periz de lui et de ses choses soient sor lui, ces covenances et autres ki apertienent a la besoigne, doit on mander es letres selonc les us et selonc la loi de la vile.\par
Mais une chose ne doit pas estre oubliee, ains la doit on clerement escrire, ke il reçoive la signorie ou k’il le refuse dedens \textsc{.ii.} jours u dedens trois ou plus u mains, selonc la costume de la vile, et s’il ne fait ce, que la election ne vaille rien del monde.\par
Et il avient sovent que li conseilleor establissent sovent demander a mon signour l’apostole et a l’empereour que il lor mande \textsc{.i.} bon governeour cele annee. Et quant ce est, toutefois on doit mander totes covenances si clerement k’il n’i ait matire de nul courous. Et quant ces letres sont faites et saielees, il les doivent mander au preudome par bons messages ki bien entendent la besoigne et ki aportent ariere les letres de sa response.\par
Il ne li doivent pas au comencement envoier trop grans homes ne de grant affaire ; car s’il ne receust la provosté, ce torneroit a grant honte a aus et a lor vile. Et nanporquant s’il le reçoit, il li poront bien envoier honorables messages au tans k’il doit venir por faire lui compaignie, ja soit ce que c’est une chose souspecenouse, car en cele voie devienent il acointes au signour et a sa maisnie plus ancunefois que mestier n’est. Et il n’afiert pas a governeour k’il soit privés de ses borgois por \textsc{.ii.} raisons, l’une por ce ke sa dignités en abaisse, l’autre pour souspecion ke les gens ont de lui et de ses acointes.
\chapterclose


\chapteropen
\chapter[{.III.LXXVII. Ci demoustre plus clerement la forme de la letre}]{\textsc{.III.LXXVII.} Ci demoustre plus clerement la forme de la letre}\phantomsection
\label{tresor\_3-77}

\chaptercont
\noindent Et por faire l’ensegnement plus cler et plus apert volra li mestres en ceste partie escrire une petite forme de la letre a celui ki est esleu a governeour et a signour, en ceste maniere : A HOME de grant vaillance et de grant renomee, mon signor KARLE conte d’Ango et de Provence, li governeour de Rome et tous lor consaus, salus et acroissance de toutes honours. Ja soit ce que totes humaines gens communalment desirent la franchise ke nature lor dona premierement, et volentiers eschivent le joug de servage, toutefois por ce que la suite des males covoitises et les loisirs des males oevres ki n’estoient p s chastes tornoit a periz des homes et a destruction d’umaine compaignie, esgarda la justice de ciaus et dreça sor le pule governeour en diverses manieres de signories, por enhaucier la renoumee des bons et por confondre le malice des mauvés. Et ensi covient il autresi comme par necessité que nature fust sous justice et ke franchise obeist a jugement ; et de ce avient, por les desiriers ki sont ore plus corrompus et por les perversités ki croissent a nostre tens, nule chose ne puet estre plus proufitable a chascun peuple et a toutes communes que avoir droit signour et sage governeour.\par
Et comme nous pensissiens ensamble d’un homme ki nous condue l’an aprés ki vient et ki garde le commun et maintiegne les estranges et les privés et sauve les choses et les cors de tous, en tel maniere que drois n’apetice pas en nostre vile : il nous avint ausi com par divin demoustrement que en trestous les autres que on tient ore a sages et a vaillans a si haute chose comme a signourie des gens, vous fustes criés et receus por le millour. Et pour ce, sire, nous par le commun assentement de la vile avons establi que vous soiés signatour et governeour de Rome de cesti prochaine feste de la Tossains jusques a \textsc{.i.} an.\par
Et nous ne doutons pas, et toz li mondes le crie, ke vos savés et volés metre jugement en pais, justice a la mesure, et ferir la spiet dou droit a la vengance des maufetours. Et por ce, sire, ke tot se tienent apaiés, grans et petis, si vous prions et requerons de toute foi et de toz nos desiriers que vos preigniés et recevés la signorie que nos vos offrons plus volentiers que nul autre, a solaire de \textsc{.xm.} lib’ de prou’, et as covenances que vous verrois en la chartre des tabellions ki est enclose dedens ses letres, et as chapistres des constitutions de Rome.\par
Et sachiés que vos devés amener avec vous \textsc{.x.} juges et \textsc{.xii.} notaires bons et loables, venir et demorer et raler o tout vostre mesnie sor vos despens et sor vostre perilz de cors et des choses, et estre venus dedens Rome le jor Nostre Dame en septembre. Et lors maintenant que vous i enterrois, sans aler a l’ostel, vous ferois les seremens de vostre office sor les livres de constitutions clos et saielés, ançois k’il soit overs ; et les ferois ausi fere a vos chascuns selon son office, dedens le capitoile de Rome. Mais une chose sachiés, ke dedens les trois jours ke l’om vos baillera les letres, vous devés prendre u refuser la signorie ; et se ce ne fesissiés, ce soit tout por noient, et sa election soit frivole.
\chapterclose


\chapteropen
\chapter[{.III.LXXVIII. Coment li sires doit prendre conseil de prendre u de laissier la signorie}]{\textsc{.III.LXXVIII.} Coment li sires doit prendre conseil de prendre u de laissier la signorie}\phantomsection
\label{tresor\_3-78}

\chaptercont
\noindent En ceste maniere u en autre que li sages ditteours voldra seront les letres envoies as signours toutes les chartres des covenances, et li messagiers ki porte li baillera cortoisement et priveement, sans cri et sans noise. Et li sires les doit prendre a maniere de sage, et aler tot coiement en aucun lieu privé et brisier le saiel et veoir les letres et savoir ce ki est dedens, et penser en son cuer diligenment ce ke fere li covient, et enquerre le conseil de ses bons amis, et veoir s’il est saufissables a tel chose.\par
Tuilles dist, ne desire pas que tu soies juges sor les gens se tu n’ies tex ke ta vertus puisse baissier les iniquités. Et nonporquant il ne se doit pas desesperier meismement por covoitise, ains doit toutes choses contrepeser a la balance de son cuer, au conseil de ses amis, et l’onour et la honte et le bien et le mal, ke mieus vaut metre conseil devant que repentir a la fin.\par
Et se c’est chose k’il le refuse, certes il doit honorer le messagier selons la maniere de lui, et renvoier la response par biaus dis et par cortoises paroles.\par
Et tot avant fera li ditteour la saluence de biaus mos, et puis la letre en ceste maniere : POUR çou que la dignité des poestés sormonte a hancurs dou siecle, ne puet la cité ne li pueples faire grignour reverence a home ne metre la en plus haut k’eslire lui entre aus et sousmetre aus de bon gre a sa signorie : c’est signe de tres grant amor et de seure fiance, c’est la glore ki enhauce le non de lui et de ses nations a tozjors. Icele grace et honor cognoissons nous que vous nos avés faite, et de tant plus haute et plus large comme la signorie de vous et de vostre vile est la plus honorable dou monde. Et ja soit ce que nous ne soions pas soufissant a rendre les honorables grasses, toutefois vos en mercions nous de tout nostre cuer et de tout nostre desirier, si comme celui ki est tousjors més obligiés a vous et a vostre commune. Mais por çou que nous somes maintenant enpechié de maintes choses ki requierent nostre presance, nous vous prions et requerons en don de grasse que vos nous pardonnés, biau signor, ce que nos ne recevons pas vostre governement ; car la besoigne ki nous detient est si grans que demorer nos covient.
\chapterclose


\chapteropen
\chapter[{.III.LXXVIIII. Des choses que li sires doit faire quant il reçoit la signourie}]{\textsc{.III.LXXVIIII.} Des choses que li sires doit faire quant il reçoit la signourie}\phantomsection
\label{tresor\_3-79}

\chaptercont
\noindent Mais se son conseil li loe k’il reçoive la signorie ke l’en li mande, consire molt en son cuer comme il prent haute chose, et k’il sousmet ses espaules a si grant charge. Et por ce se doit il porveoir de grant apareil ; c’est li propres guerredon de signorie connoistre que il doit avoir la cure de la cité et maintenir ses honours et ses dignités, garder la loi et fere droit, et que totes choses soient baillies a sa fai.\par
Et tot maintenant doivent honorer le messagier si comme il afiert a l’un et a l’autre, et esclaircir avec lui toutes covenances, s’il en a le pooir ; en tel maniere k’il en ait bones chartres pour oster toutes manieres de debat.\par
Et quant ce iert fait, il li baillera unes letres, la saluence devant, et puis en ceste maniere : VOIRS est que nature fist tous homes igaus ; mais il est avenu, non mie par visce de nature mais par le malisce des oevres, ke por restraindre les iniquités li hom ait la signorie des homes, non pas de lor nature mais de lor visces. Et sans faille, cil ki seulement est dignes de si tres honorable chose, ki set adevancir les autres par ses merites et par ses vertus, a celui seulement doit estre bailliés li governemens, ki por sa bonté vaut au lieu et a l’onour, et ki n’a pas les espaules fiebles a si charghables faissiaus. Car ja soit signorie de grant honour, neporquant ele a en soi grieté de periz et de charge.\par
Mais por ce que la soule soufissabletés de Jhesucrist fet home soufissant a ces offices, non mie por la soule fiance de lui, non por bonté ki soit en nous : el non dou Soverain Pere, par le commun conseil de tous nos amis, prenons et rechevons l’onour et l’office de vostre governement, selonc le devisement de vos letres ; meismement sour icele fiance que nos cuidons vraiement, que le sens et le savoir des chevaliers et dou peuple et la force et la loiautés de toz les citeins nos aidera a porter partie de nos charges et alegier nos feissiaus par bone obeissance.\par
Et quant il aura les letres renvoies ariere et le messagier, lors maintenant comence ses apareillemens ; et se porchace d’avoir chevaus et harnois bons et honorables ; mais sor toutes choses soit s’estuide en avoir son juge et son acesseur discret et sage et aprovés, ki criement Diu et ki soit biau parliers, non pas vergoignous, chaste de son cors contre femes, non pas orguilleus ne courouçable ne paourous, ne de \textsc{.ii.} langages, et ki ne desire pris de grant fierté ne de grant pitié ; mais soit fort et droit, et justes et de bone foi, et religious a Dieu et a sainte eglise.\par
Car en la loi est li juges apelés sacrés, au commencement dou Digest, la u ele dist, l’on vous apele dignement provoires. Et ou Code de jugement et de sacrement et en maint autre lieu dist la lois que li juges est consacrés de la presence de Dieu et k’il est en tierre autresi comme \textsc{.i.} Dieu.\par
Mais s’il ne le trueve pas si compli de toutes choses, por ce que tous blans oiseaus ne sont pas cignes, soit au mains loiaus et permanables, k’il ne puisse estre corrompus, et soit de bone foi non pas simples trop, ne envelopés de maus visces. Garde donc li sires k’il ne laisse bon juge por argent la ou il le trueve ; car il est escrit, mar i est celui ki va seul ; car s’il chiet, il n’est ki le relieve. Por quoi je di que celui sires vait en la signorie por honor conquerre, mieus que por covoitise de deniers, certes il doit garder par qui li drois sera governés ; car si comme la nef est governee par les tesnons, tot autresi est la cités menee par le savoir dou juge.\par
Autresi doit il avoir ses notaires trés bons et sages de loi, ki sachent bien parler el bien lire et bien escrire chartres et letres, et ki soient bons ditteours, et chaste de lor cors ; car maintesfois la bonté dou tabellion amende et acomplist la defaute dou juge, et porte grant charge de tot l’office. Autresi doit il amener en sa compaignie chevaliers sages et bien apris, ki ayment l’onour son mestre, et senescal et vallés et sergans, et toute la mesnie sage et amesuree sans orgoil et sans felonnie, et ki volentiers obeissent a lui et a ciaus de l’ostel.\par
Aprés ce siut on faire noveles robes por lui et por son compaignon, et vestir la mesnie d’une taille, et renoveler ses armeures, ses banieres, et ses autres choses ki covienent a la besoigne. Et puis, quant li tens aproce, il doit envoier son senescal a la vile por garnir l’ostel des choses besoignables ; car li sages dit que mieus vaut a porveir devant ke querre conseil aprés la fin.
\chapterclose


\chapteropen
\chapter[{.III.LXXX. Que li sires doit faire quant il est au chemin}]{\textsc{.III.LXXX.} Que li sires doit faire quant il est au chemin}\phantomsection
\label{tresor\_3-80}

\chaptercont
\noindent Or sieut il avenir aucunesfois que, au tans que li sires doit aler sa voie, li commun de la vile li envoient des honorables citains de la cité jusc’a son ostel, por faire li compagnie au chemin u por priier le commun de sa vile que il le laissent aler a son office, u por autres okoisons. Mais comment k’il soit, il les doit honorer et festiier mervilleusement et envoier grans presens et aler les veoir a lor ostel. Mais bien se gart k’il ne parolt priveement a nul d’aus, car de ce parlement naist sovent male soupection ; et por ce est ore demorés cist usages ke poi des cités envoient tens embaisseours a l’encontre.\par
Et quant li sires a apareilliet son oire, si se met a la voie el non dou vrai cors Dieu, et s’en aille tout droit a son office enquerant tozjours et espiant des us et des conditions de la vile et de la nature des gens, si k’il les sache tot ançois k’il i entre. Et quant il aproche la vile a une journee, il doit envoier au devant son senescal o tout les keux ki atornent son mangier et l’ostel, autresi doit il envoier a la vile les letres de sa venue. Et le matin k’il doit entrer en la vile doit il sans faille oïr la messe et le service Nostre Signeur.\par
D’autrepart ses devanciers, c’est a dire cil ki tenoit lors la signorie de la vile, maintenant k’il reçoit les letres de la venue au novel signor, face crier parmi la vile que tot chevalier et borgois de la vile voisent a l’encontre, et il meismes i doit aler avec l’evesque de la cité s’il i est u s’il i vieut aler.\par
Et certes li noviaus sires et li devanciers la u il s’entretruevent doit li uns honorer l’autre por oster toutes suspections, et saluer les gens debonairement. En ceste maniere s’en doit il aler tot droit a la mestre eglise, et estre devant l’autel a genous et priier a Dieu humlement de tot son cuer et de toute sa foi, et metre de ses deniers sor l’autel honorablement, et puis monter ariere et aler la u il doit.
\chapterclose


\chapteropen
\chapter[{.III.LXXXI. De la diversité des viles}]{\textsc{.III.LXXXI.} De la diversité des viles}\phantomsection
\label{tresor\_3-81}

\chaptercont
\noindent A ce point a plusors diversités ; kar il i a viles qui ont a coustumes que li sires si s’en vait a son ostel, et on li baille les livres des establissemens de la vile ançois k’il face son sairement ; et en ce a il grant avantage, car il se puet mieus proveir contre les chapistres ki sont contre lui.\par
Autres i a ki ont en usage ke maintenant que li sires est dedens la vile, et k’il a esté devant l’autel, on l’en maine au conseil de la vile, u devant la commune des gens la u il sont assamblé ; et la li fet on jurer lui et les siens ançois que li livres des chapistres soit overs ne k’il soit livrés a lui u a son juge.\par
Mais li sires ki est sages, ançois k’il mete la main sor les sains, il requiert la commune k’il li donent arbitre sor le malefice, non mie por son preu mais por le bien de la vile et por le mal des maufetours. Se l’en li baille, c’est bon, se ce non, il prie que s’il i eust aucun malicieus chapistle contre lui u contre l’onour dou commun u de sainte eglise il puist iestre amendés par les conseilleours de la vile. Et se il ce li font, c’est bon si le fait escrire en chartre de tabellion ; et se ce non, il fera serement selonc ce que l’om li devise de par le commun.\par
La forme dou sairement est icele : Vos mesire .A. jurés sor sains de governer les choses et les besoignes de ceste vile ki apertienent a vostre office, et guier et conduire et maintenir et sauver la cité et tote la conté et toz homes et fames, grans et petiz, chevaliers et borgois, et lor droit maintenir et deffendre et garder, et faire ce que la lois commune et les constitutions commandent, et faire k’il soient fait, et gardés por toutes gens meismement orfelins et veves femes et as autres petites gens, et a trestous homes ki seront a plet devant vos et devant les vos.\par
Et garder et maintenir et deffendre eglises, temples, hospitaus, et toutes maisons de religion, les chemins, les pelerins et les marchans ; et de faire quanqu’il a escrit en ces livres des establissemens de ceste vile, a quoi vos jurrois a bonne et a loiale conscience. Remués soiez a amour et a haine, pri et loier et toute malice, selonc la vostre vraie entention, des le prochain jour de la Toussains jusc’a \textsc{.i.} an, et tot le jour de ceste Toussains.\par
En ceste maniere fera li sires son sairement ; sauve ce que s’il i a nule choses ki doivent estre ostees dou sairement, qu’il l’oste devant k’il fiere la main sor les sains. Et quant il a juré, lors maintenant doivent jurer ses juges et li chevalier et li notaires, chascuns endroit de lui, de fere bien et loiaument son office, et doner a lor signour bon conseil, et de tenir creence ce ki doit estre privé.
\chapterclose


\chapteropen
\chapter[{.III.LXXXII. Que li sires doit faire quant il est a la vile venus}]{\textsc{.III.LXXXII.} Que li sires doit faire quant il est a la vile venus}\phantomsection
\label{tresor\_3-82}

\chaptercont
\noindent A ce point il i a plusors diversités. Car il i a viles ki ont a coustumes ke tout maintenant que li sires a fait son sairement, il parra devant les gens de la vile ; et autres en i a u il ne parole mie, ains s’en vait tout belement a son ostel, meismement se la vile est en bone pais.\par
Encore i a il autres diversités ; car u la vile a guerre dehors contre ses voisins, u il a guerre dedens entre les borgois, u ele est en pais dedens et dehors. Por quoi je di que li sires se doit tenir as usages dou païs ; car se li us de la vile requiert, il pora dire la parole bien et cortoisement sans rien commander. Car tant comme ses devanciers est en signorie, il ne li loist pas a metre la faus en autrui meissons, mais il puet bien priier et amonester les gens sans commander ou deveer nule rien.\par
Et se la tiere est en pais, il puet parler en ceste maniere : AU commencement de mes dis cri je le non Jhesucrist le tot puissant roi ki done toz biens en toutes poestés ; et la glorieuse Virge Marie, et mon signor Saint Jehan, ki est pairon et guiere de ceste vile : k’il par sa sainte pitié me donent grasce et pooir que je hui en cest jour, et tant comme je serai en vostre service, die et face toute chose ki soit honour et glore de sa maiesté et reverence et honorabilité monsignour l’apostoile et l’empereour de sainte eglise et de l’empire de Rome ; et ki soit honor et pris monsignor .A. ki a esté vostre signor et est encore, et ki soit acroissance et amendement et boneurtés de l’estat de vous et de ceste vile et de toz vos amis.\par
Se je voloie former la matire de mon parlement sor la loenge de si trés noble cité comme est ceste vile, et nomer l’onour, le sens, le pooir, et les autres oevres, de vous et de vos anciestres, certes je ne poroie a chief venir, tant il a a conter et de haute chevalerie et dou franc pueple de ceste vile. Et pour ce m’en tairai jou a tant. De mon signeur .A. meismes, et de ses bonnes oevres k’il a fet ceste annee en vostre signourie, et ou governement dou commun et de toutes ces choses, n’en dirai je noient, car eles resplendissent par mi le monde comme la clarté dou soleil.\par
Il est voirs que vous m’avés esleu et fet poesté et signor de vous ; et ja soit ce que je n’en soie pas dignes, ne por mes merites ne por ma bonté, neporquant a la fiance Jhesucrist et des preudomes de ceste vile je reçui l’onour que vous me fesistes, sor itel cuer et en tele entention que je mete por vous et cuer et cors, sans eschiver travail dou cors et damage d’avoir.\par
Et puiske vous m’avés fet le plus grant honour ke gens puissent faire en cest siecle vivant, c’est a faire de moi conduiseour et segnor de vous par vostre bon gré, je espoir et croi veraiement que vous serés estables et obeissant a mes honours et a mes commandemens, meismement por le proufit et por le governement de vos et des vostres. Et tant sachiés que tos ciaus ki ensi le feront, je les amerai et ferai grans honours ; més les autres ki mefferont contre m’onour ou ki feront tort u desraison a nului qui k’il soit, grans u petis, je les dampnerai et tormenterai de cors et d’avoir en tel maniere que la paine d’un sera poor de plusours.\par
Je ne sui ci venus por covoitise de gaaignier argent, més por conquerre los et pris et honour, a moi et a tous les miens. Et por ce m’en irai je parmi le droit et parmi le cors de justice, en tel maniere que je n’abaisse ne a diestre ne a senestre ; car tant connois je bien, chascuns le doit savoir, que la cités ki est governee selonc droit et selonc verité, si ke chascuns ait ce k’il doit avoir, et ke li maufetour soient li \textsc{.i.} chacié hors, li autre livrés a paine : certes, ele croist et mouteplie des gens et d’avoir et dure tousjours en bone pais a l’onour de lui et de ses amis. Por quoi je me torne a celui que je ai commencié, c’est Dieu le tot puissant, k’il doinst a vous et a moi et a tous les citeins, et justiciables de ceste vile, ki ci sont et ailleurs, grasce et pooir de faire et de dire ce ki soit honour de lui et essaucement de nous et dou commun de la vile et de tous ciaus ki nous ayment de bon cuer.\par
En ceste maniere puet li noviaus poestés dire la parole de sa venue ; mais li sages parleours doit mout garder les us et l’estat et la condition de la vile, si k’il puisse muer ses paroles et trover autres selonc le lieu et selonc le tens.\par
Més se la cités a guerre dedens, por la descorde ki sont entr’aus, lors covient il que li sires parole de ceste matire, et si puet bien ensivre ce ki est dit devant. Et la u il voit que mieus soit, en son dit puet il ramentevoir comment Nostre Sires commanda ke pais et bone volentés fust entre les gens, et comment il en seroit liés ke il les eust trovés en bone amour. Car il afiert mout a signour que ses subtés soient en acorde, et s’il ne sont, k’il les i torne. Et die comment concorde essauce les viles et enrichist les borgois, et guerre les destruit ; et ramentevoir Romme et les autres bonnes viles ki por la guerre dedens sont decheues et mal alees ; et coment guerre citaine amainne mains maus, si come est rober temples et chemins, ardoir maisons, murtre, avoutire et larrecin, traison et perdition de Dieu et dou siecle.\par
Teus et autres paroles dira li sires a sa venue, priant et amonestant les gens de faire bien et d’avoir pais et laissier haine ; et die coment il aura le conseil des preudomes, et establi la besoigne bien et honourablement.\par
Et quant la cités a guerre dehors contre aucune autre cité, ciertes li sires a sa venne puet bien ensivre la matire ki est devisee ci devant.\par
Et la u il voit ke mius estoit, si puet joindre teus autres paroles : ET IL est voir, et tous li mondes le set, que por le mal et por le torfait ki ne devoient ne ne pooient estre plus soufert, guerre est venue entre vous et vous enemis a grant tort et a grant desloiauté d’iaus et de lor partie.\par
Et ja soit ce une besoigne ki requiert maintes choses, neporquant je ne parlerai ore se petit non, car il covient k’il soit plus dou fait ke dou dit. Mais s’il i a en cest siecle vivant chose u on puisse ouvrer sa force et son pooir et aquerre haute renomee de sa vertu, je di que en cele guerre sormonte totes besoignes, car ele fet home preu as armes, franc de corage, vighereus ; plain de viertus, fors au travailh, veillables a agais, soutil et engigneus en toutes choses.\par
Estuide donc chascuns soi meismes es choses devant dites. Soiés trestot garni de beles armes et de bons chevaus, car teus choses donent talent as homes de combatre, et siute de victore, et font as anemis paour de perdre et talent de fuir. Soiés d’un coer et d’une volenté. Soiés parmanables a l’asamblee ; alés estroit a la bataille et ne vous en desevrés sans congié. Soviegne vos de vos anciestres et de lor victorieuses batailles : et je me fi tant en la vaillance et en la bonté de vous et de vos gens, et en droit ke vous avés contre vos anemis, ke vous aurés la victore et l’onour que vos desirrois.\par
Teus et autres paroles, que li sages parleour saura trover a la matire, doit il dire entre ses citeins, en tel maniere k’il voit k’il soit plus agreable, et poser fin a son dit.\par
Et quant il est assis, son devancier, s’il i est, doit maintenant lever et faire son prologue bien et sagement, et respondre a ce que li autres a dit, et loer lui et son sens et ses oevres et sa lignie, et fere li grasces dou bien et de l’onour k’il li a fet en son dit. Et a la fin de son parlement doit commander a tous k’il obeissent au novel signor, et k’il metent a oevre ses ensegnemens ; et quant il a ce dit, si done congié a ses gens, a chascun k’il s’en aille a sa maison.\par
Or sieut il avenir aucunesfois que avec le novel signor vienent gentiz gens de sa vile de par le commun de la cité, ki parolent en cel lieu meismes, et portent salus, et devisent l’amor ki est entre l’un commun et l’autre, et loent la cité et les citeins, et le vielh poesta et sa bone signorie ; autresi loënt il le novel signour et sa lignie et lor bonnes oevres, et moustrent comment tot le commun de lor vile le tienent a grant honour et a amour ce k’il en ont esleu lor governeour. Et dient que le signor et le commun de lor cité li ont commandé, sor le periz de lor cors et de lor fiz et de quank’il a au monde, k’il face et die ce ki tort a honour et au pourfit de la vile k’il doit governer. Et por ce prient les gens de la cité ke il li obeissent et li doignent aide et conseil, en tel maniere k’il puisse honorablement finer son office.\par
Et quant il ont ce dit, li vieus governeour doit faire avenable response en ce parlement meismes k’il respont au novel signour si comme li contes devise ci devant ; u en autre maniere, se la condition l’aporte.
\chapterclose


\chapteropen
\chapter[{.III.LXXXIII. Que li sires doit faire quant il a fet son sairement}]{\textsc{.III.LXXXIII.} Que li sires doit faire quant il a fet son sairement}\phantomsection
\label{tresor\_3-83}

\chaptercont
\noindent Après les sairemens et les parlemens des uns et des autres s’en doit li sires aler a l’ostel et ovrir les livres des establissemens et des chapistles de la vile ; en quoi ses juges et si notaire doivent lire et estudiier, et de nuit et de jour et devant et deriere, et noter ce ki covient a faire, ce devant devant et ce deriere a la fin. Car c’est la trés grant bonté des juges et des notaires que il les lisent et relisent sovent, en tel maniere que il le retiegnent toute en lor cuer, et k’il sachent les puins et les lieus ki touchent a lor besoignes. Neis au signor meismes afiert k’il les sache bien, meismement les puins ki plus le lient ; et k’il s’en soviegne tousjours.\par
Et quant il ont diligeament regardé, lors maintenant doivent il noter la forme dou sairement et des ensegnemens ki doivent estre jurés par tous les justiciables, et mander tous ciaus ki sont devant en chascune parroche ce k’il jurent devant, et puis facent jurer toz ciaux armes portans, et metent en escrit les nons et les baillent as notaires.\par
Aprés ce doit il eslire son conseil selonc la loi de la vile, mais il doit porcachier que li conseilleours soient bons et sages et de bons aages, car de bone gent bon conseil ; et puis les autres officiaus et sergans de la cort bons et loiaus, ki li aident a porter le fais de son office.\par
Endementiers que li sires est a l’ostel et k’il fait et ces et autres apareillemens, ançois k’il monte sor la maison dou commun ne k’il soit en sa propre signorie, il se doit sovent et menu conseillier as preus gens de la vile des choses ki avienent a l’onor de lui et de la vile.\par
Et se la vile se descorde u dedens u dehors, il se doit molt travillier por avoir la pais, se ce ne fust de tel maniere que si citain ne voelent pas k’il s’en melle ; car li sires se doit mout garder k’il ne chie en la haine ou en la suspection de sa gens.
\chapterclose


\chapteropen
\chapter[{.III.LXXXIIII. Que li sires doit faire quant il entre premiers en sa signourie}]{\textsc{.III.LXXXIIII.} Que li sires doit faire quant il entre premiers en sa signourie}\phantomsection
\label{tresor\_3-84}

\chaptercont
\noindent Et quant li jors est venus k’il doit comencier son office, il doit le matin tot avant aler au moustier et oïr le siervice et orer a Dieu et a ses sains. Et puis maintenant s’en aille a la maison dou commun, et tiegne la chaiere de sa glore ; et por ce k’il est venut en usage que on laisse au governeour la porveance d’establir les paines, meismement sor les petites coupes, doit li sires par conseil de ses sages establir ses bans et ses ordenemens, teus k’il soient acordables as bons us de la vile : mais k’il ne contredient as chapistles k’il jura le premier jour.\par
Et le premier jour de fieste ki vient, il fra assambler les gens de la vile en lieu ki est acoustumés, et devant aus doit il parler si haut que chascuns entende sa parole ; et tiegne en son dit cele meismes voie k’il tint au premier jour, sauve ce k’il doit ore parler plus roidement et commander et deveer comme sires et manacier et prier et amonester, si comme il verra ke bien soit.\par
Et quant il a finé son conte, ses notaires lise les ordenemens a haute vois entendablement, et si ne suetre pas li sires que nus hom de la vile se lieve por dire noient au parlement ; car se uns i desist uns autres rediroit, et ensi seroit uns griés enpeechemens, meismement s’il a en la vile \textsc{.ii.} parties.
\chapterclose


\chapteropen
\chapter[{.III.LXXXV. Coment li sires doit amonester ses officiaus}]{\textsc{.III.LXXXV.} Coment li sires doit amonester ses officiaus}\phantomsection
\label{tresor\_3-85}

\chaptercont
\noindent Après ce doit li sires assambler ses juges, ses notaires, ses compaignons, et les autres officiaus de son ostel, et priier et amonester les de bien faire au plus doucement k’il onques puet ; et aprés les prieres lor comande k’il gardent et maintiegnent l’onour de lui et dou commun, et k’il veillent et estudient chascuns a son office, et k’il rendent a chascun son droit, et k’il delivrent toutes querieles au plus tos k’il onques puënt, salve soit l’ordre de raison ; k’il se gardent de ces visces et dou blasme de la gent, et k’il ne se couroucent as gens, ne ne voisent en tavernes ne chiés aucun home, por mangier ne por boivre ; et k’il ne soient privé de nului et qu’il gardent qu’il ne soient corrompu par deniers, ne par fenmes, ne par autre chose ki soit. Et se autrement le font, je di k’il les doit punir plus aigrement que les autres, car plus grief paine chiet sor les nos et sor ciaus ki doivent garder nos commandemens.
\chapterclose


\chapteropen
\chapter[{.III.LXXXVI. Comment li sires doit honourer son antecessor}]{\textsc{.III.LXXXVI.} Comment li sires doit honourer son antecessor}\phantomsection
\label{tresor\_3-86}

\chaptercont
\noindent Entre les autres choses ki coviennent au signour est k’il adoucisse les corages de ses devanciers, et k’il lor face honor et amour de quank’il puet. Et quant il a a rendre son conte, ne suefre pas c’on li face honte ne tort, car il afiert au signour de restraindre les iniquités des mauvés soz les bonnes de justices : et bien sache k’il vendra a ce point. Et si comme il maisonera a son pere, tot autresi li remaisonnera ses fiz ; car il est escrit que nous devons estre teux a nos peres comme nous volons ke nos fiz soient envers nous.
\chapterclose


\chapteropen
\chapter[{.III.LXXXVII. Coment li sires doit assambler le conseil de la vile}]{\textsc{.III.LXXXVII.} Coment li sires doit assambler le conseil de la vile}\phantomsection
\label{tresor\_3-87}

\chaptercont
\noindent Quant li sires est venus a son office et a sa signorie tenir, il doit mult penser et de jour et de nuit as choses ki apertienent a son governement. Et ja soit il chiés et gardeour dou commun, neporquant es grans besoignes.et douteuses doit il assambler les conseilleours de la vile et proposer et dire devant aus la besoigne et demander k’il li conseillent ce ke biau soit pour le bien de la vile, et oïr ce k’il diront.\par
Et se la besoigne fust grans, il s’en doit conseillier une fois u \textsc{.ii.} u \textsc{.iii.} u plus et plusour fois se mestier est, au petit conseil et o grant, et joindre a conseil des autres preudomes, des juges, des prieus des ars, et des autres bones gens ; car il est escrit ke de grant conseil vient grant salus. Et a la verité dire, li sires puet seurement aler selonc les establissement de conseil ; car Salemons dit, fait toutes choses par conseil, et, puis les fais, ne t’en repentiras.\par
Mais bien garde li sires que la proposition k’il fait devant les conseilleours soit briés et soit escrites a poi de chapistres ; car la multitude des choses engendre enpeechement, oscurcist les corages, et afoiblist les millours sens, car sens ki pense a maint des choses est maindre a chascune.\par
Et quant li notaires a leu les propositions devant les conseilleours, li sires se lieve et redie la besoigne comment ele est et comment ele est esmeue. Mais garde bien ke si dit et ses pors soient nu et simple, de tel maniere que nus hom li puisse dire k’il weille plus l’une partie que l’autre. Je ne di pas que li sires ne puisse aucunefois dire se ce ne fust chose ki engendre suspection ; car il i a maintes gens ki par envie et par haine de cuer dient plus contre le signeur ke por le bien dou commun.\par
Et quant li sires a dite la proposition, il doit maintenant commander que nus ne die autre chose se de ce non k’il lor a mis devant, et que nus ne se melle de loer ne lui ne les siens, et k’il escoutent ciaus ki parolent. Lors doit il commander a ses notaires k’il metent diligeaument en escrit les dis des parleours, non mie tout çou k’il dient, mais sou sans plus ki touche au point dou conseil. Et si ne suefre pas ke trop de gent se lievent a consillier.\par
Et quant il ont dit et d’une part et d’autre, li sires se lieve a deviser les dis par parties les uns contre les autres. Cil a quoi s’acorde la grignour partie des gens ki sont assamblé au conseil doit estre ferme et estable, et tout ensi les doit escrire les tabellions ; et se mestiers est pour mieus esclarcir la besoigne, il puet bien escrire trestous les conseilleours comment il s’acordent a l’un conseil et a l’autre. Et quant c’est tot fait bien et diligemment, li sires lor done congié ; et se mestiers est, si commande creance ; et ki ne le tient, il doit estre dampnés comme traitres.\par
Entre les autres choses doit li sires mout honourer les gens dou conseil, car il sont si membre, et çou k’il establissent doit estre sans remuance, se ce ne fust por certain meillourement dou commun. Mais on ne doit pas assambler conseil por toutes choses, mais por iceles soulement ki en ont bon mestier.
\chapterclose


\chapteropen
\chapter[{.III.LXXXVIII. Coment li sires doit honorer les messagiers et les ambaisseors estranges}]{\textsc{.III.LXXXVIII.} Coment li sires doit honorer les messagiers et les ambaisseors estranges}\phantomsection
\label{tresor\_3-88}

\chaptercont
\noindent Et quant li ambaisseour des estraingnes terres vienent a lui por aucune besoigne ki touche a l’une tiere et a l’autre, ciertes li sires les doit volentiers veoir et honorer et reçoivre debonairement. Et ançois k’il lor assamble conseil, il se doit mout travillier de savoir l’achoison por quoi il vienent, s’il onques puet ; car ele puet estre de tel maniere k’il n’assamblera ja point de conseil, et tel poroit estre k’il assambleroit le petit conseil sans plus, ou par aventure le grant, o tout le commun de la vile.\par
Mais s’il sont legaz monsignour l’apostoile ou de l’empereour de Rome ou de ses grans honours, il ne doit pas veer conseil, ains lor doit aler a l’encontre, et convoier les, et honorer de tout son pooir. Et quant il ont parlé au conseil, li sires doit respondre bien et cortoisement, et dire k’il sont signours de l’aler u dou demorer, et que li sage de la vile penseront ce ke sera covenable. Et quant li ambaisseour sont issu dou conseil, li sires doit oïr les volentez des consilleours, et si comme il establissent doivent faire le fait et la response.
\chapterclose


\chapteropen
\chapter[{.III.LXXXVIIII. Coment li sires doit envoier ses ambaisseours}]{\textsc{.III.LXXXVIIII.} Coment li sires doit envoier ses ambaisseours}\phantomsection
\label{tresor\_3-89}

\chaptercont
\noindent Quant il avient aucune chose por quoi on doit envoier messages u ambaisseours hors de la vile, certes se la besoigne ne fust de grant pesantor, il les doit eslire par briés entre les conseillors de la vile, ou autrement, selonc les us de la vile. Mais s’il doivent estre envoiés a l’apostoile ou a l’empereor de Rome u autre part ki requiert grans sollempnités, je lo que li sires meismes eslise tous les millours de la vile, se c’est la volentés dou conseil.
\chapterclose


\chapteropen
\chapter[{.III.LXXXX. Coment li sires doit oir les causes et les advocas}]{\textsc{.III.LXXXX.} Coment li sires doit oir les causes et les advocas}\phantomsection
\label{tresor\_3-90}

\chaptercont
\noindent Pour oïr les desiriers des gens et por apaier le clamour des citeins, afiert il a bon poesté k’il soit sovent a oïr les extraordinaires querieles, et k’il les delivre et amenuise les plais de totes les gens ; car c’est de grant bonté ke li sires constraigne ses subtés dedens les bonnes de drois, k’il ne viegnent a la descorde, por ce que feus ki n’est estains prent aucunefois grant force.\par
Mais s’il avient aucun fort point dont il se doute, je lo k’il i amaine son juge et use son conseil, ou k’il i mete jour, jusc’a tant k’il s’en soit conseilliés. Mais molt est bele chose et honeste au signor, comme il siet a cort, k’il entende et coiement les uns et les autres, meismement les advocas et les paires des causes ; car il descuevrent la force des plés et manifestent la nature des questions, por quoi la lois dist que lor office est fierement bons et besoignables a la vie des homes, et tant ou plus come il se combatissent a l’espee ou au coutel por lor parens et por lor païs. Car nous ne quidons pas, fait l’empereres, que seulement cil soient chevaliers ki usent l’escu et le hauberc ; mais en chevalerie sont li advocas et li pairon des causes, et por ce doit bien li sires porveir, por son office, que se ancuns povres ou autres est en plet devant li ki ne puisse avoir advocat u par sa foiblece ou par la force son adversaire, il doit constraindre aucun bon advocat ki li soit aide et k’il le conseille de son droit et de sa parole.\par
Et quant li sires a oïes les parties, lors se doit molt apenser comme il responde, ne il ne doit rien dire comme fols, mais sagement et penseement ; et tout ce k’il commande et k’il establist soit par conseil, et soit estables, si k’il samble drois et sages en oevre et en parole, autrement son dit seroit en lieu de mokerie, et chacuns le tendra por noient.\par
Porquoi je di que s’il trespasse ancunefois outre ce que bon soit, u en ses dis u en ses commandemens, il n’est pas honte d’amender le, ains est grans vertus que chascuns chastie son erreur et retourt au millor ; et ce doit li sires faire, selonc ce que la lois commande.
\chapterclose


\chapteropen
\chapter[{.III.LXXXXI. Ke le signor doit faire sor le malefice}]{\textsc{.III.LXXXXI.} Ke le signor doit faire sor le malefice}\phantomsection
\label{tresor\_3-91}

\chaptercont
\noindent Sour toute chose doit la poesté faire ke la vile ki est a son governement soit en bon estat, sans noise et sans forfait. Et ce ne puet pas estre s’il ne fait tant que li païs soit widiés et nés de larons et de murtriers et de tos maufeteours ; car la lois commande bien que li sires espurge bien le paIs de male gent, et por ice a il la signorie sor les estranges et sor les privés ki font les crimes en sa judicion. Et neporquant il ne doit pas livrer a paine ciaus ki sont sans coupe, car il est plus sainte chose d’asoudre \textsc{.i.} nuisant que dampner \textsc{.i.} nonnuisant, et laide chose est que tu perdes le non d’innocense por haine d’un nuisant.\par
Sor le malefice doit li sires et ses officiaus sivre les us du païs et l’ordre de raison, en ceste maniere : premierement doit cil ki acuse jurer sor sains de dire voir en acusant et en deffendant, et k’il n’amenra pas faus testimoine a son esciant, lors baillera son acuse u denonciation en escrit, u se ce non, li notaires le doit escrire mot a mot si comme il a devisé, et enquerre de lui meismes diligemment ce k’il ot.\par
Li sires u li juges meismes quideront ke ce soit des apertenances dou fait, voir de la chose ; et puis aprés semondre celui ki est acusés dou malefice, et s’il vient, si le facent jurer et aseurer la court des pleges ; et metent en escrit la confession ou sa negation si comme il a dit ; et s’il ne donne pleges ou se li malefices est trop grevables, on doit arrester en bonne garde. Lors doit li sires ou li juge juges metre jour de prover et oïr le tiesmong ki welent venir, et constraindre ciaus ki ne vienent, et examiner toute chose bien et sagement, et metre les dis en escrit.\par
Et quant li tesmong sont tot receu, li juges et li notaires doivent semondre les parties devant aus, et, se il i sont, il doivent ovrir et publiner les dis des tesmoins, et baillier les a ciaus, si k’il puissent consillier et moustrer ses raisons.\par
Or avient aucunefois es grans crimes k’il ne puet estre seus ne prouvés certainement, mais on trueve bien contre celui ki est acusés aucunes ensegnes et fors argumens de suspection. A ce point le puet on metre en la gehine, por faire li gehir sa coupe, autrement nenil. Et si di ge que, a la gehine, li juges ne doit pas demander se Jehans fist le murtre, mais generalment doit demander ki le fist.
\chapterclose


\chapteropen
\chapter[{.III.LXXXXII. Comme li sires doit dampner et asoudre les acusez}]{\textsc{.III.LXXXXII.} Comme li sires doit dampner et asoudre les acusez}\phantomsection
\label{tresor\_3-92}

\chaptercont
\noindent En ceste maniere doit on reçoivre les acusés et les prueves des malefices. Et quant ambedeus les parties ont moustré ce k’il welent, lors maintenant sans nul delaiement doit li sires estre en une des chambres avec les juges et les notaires de son ostel, et veoir et oïr et enchierchier diligement tot le plet, d’amont et d’aval, tant k’il connoissent la verité, selonc ce k’il est moustré devant aus. Et se il sont certain dou malefice, por la confiession dou malfaitour meismes, de son gré sans torment, ou par tiesmong, u par bataille de champion, ou par sa contumace, il le doivent dampner u de cors u d’avoir, selonc la maniere dou meffait, et selonc la loi u l’usage dou païs.\par
Mais mout doit garder li sires que ce ne soit ne plus aigrement ne plus molement que la nature de la chose requiert, por renomee de fierté ne de pitié. Et ja soit ce que en griés malefices covient grief paine, neporquant li sires doit avoir aucun atemprement de benignité ; mais cil ki sont a nostre tens ne le font pas ensi, ains le dampnent et tormentent au plus fierement k’il puënt. Mais ciaus ki ne sont pas coupables, on les doit assondre. Li notaires mete en escrit les dampnés d’une part et les assous d’autre.\par
Aprés ce doit li sires assambler le conseil a la coustume du païs, et commande que nus n’i face noise ne cri. Et s’il wet il puet bien en \textsc{.i.} poi parler et amonester les gens k’il se gardent des meffais, et que nus n’i regarde si petites paines k’il met maintenant sor aucun des maufetours ; car une autre fois les fera il plus fieres, et tozjours les acroistra jusc’a la fin de ses offices.\par
Lors doit il maintenant ciaus ki doivent estre dampnés de cors, k’il soient enki en presence por oïr lor sentence, por ce que sentence de cors ne puet estre donnee contre nului s’il ne fait presens. Lors lieve li notaires et lise tot belement ces sentences, tes absols devant et les dampnés apriés. Et quant il a tout leu, li sires le conferme : il commande que ceaus dou cors soient maintenant danné et li autre paient a jour nomé, et baille l’essamplaire as cambrelains dou commun, et donne congié as gens.
\chapterclose


\chapteropen
\chapter[{.III.LXXXXIII. Comment li sires doit garder la chose dou commun}]{\textsc{.III.LXXXXIII.} Comment li sires doit garder la chose dou commun}\phantomsection
\label{tresor\_3-93}

\chaptercont
\noindent Et quant li jours trespasse que li dampnés doivent paier, s’il ne paient li sires doit mout constraindre de paier, car poi vaut a dampner les s’il ne les fait paier ; et d’autrepart il doit estudier que li chambrelenc dou commun soient bien garni d’argent, por faire les grans despenses et les petites, ki vienent sor le commun.\par
Mais il doit sovent et menu veoir le conte des chambrelains et l’entree et l’issue, et garder ke li avoirs dou commun ne soit pas despendus desmesureement ; car s’il doit garder soi meismes de trop larghement despendre, certes il doit assés mieus espargnier la chose dou commun, pour çou que laide chose est estre aver du sien et large de l’autrui.\par
Et ja fust il grans despenderes de son avoir, si doit il estre garderes du commun, et sauver et maintenir les drois dou commun, les detes, les justices, les signouries, les chastiaus, les viles, les maisons, les cors, les officiaus, les places, les voies, les chemins, et toutes choses ki apertienent au commun de la vile, en tel maniere que li honours et li proufis dou commun n’apetice pas, mais croissent et amendent a son tens. Autresi doit li sires garder et faire la vile dedens et defors, meismement de nuit por les larrecins et por les autres maus crimes.
\chapterclose


\chapteropen
\chapter[{.III.LXXXXIIII. Comment li sires doit garder la chose de son ostel et de sa gent}]{\textsc{.III.LXXXXIIII.} Comment li sires doit garder la chose de son ostel et de sa gent}\phantomsection
\label{tresor\_3-94}

\chaptercont
\noindent Dedens son ostel doit li sires establir sa mesnie bien et sagement, chascun en son lieu et en son office, et chastoier l’un de parole, l’autre de verge ; et amoner son senescal k’il soit amesurés en despendre, non pas en tel maniere k’il soit blasmés d’avarice, mais k’il maintiegne l’avoir de lui, et k’il souffice as gens de son ostel, et ke rien ne faille a la mesnie, por ce que la faute des choses besoignaibles le poroit amener a vilaine pensee.
\chapterclose


\chapteropen
\chapter[{.III.LXXXXV. Comment li sires doit consillier avec ses sages}]{\textsc{.III.LXXXXV.} Comment li sires doit consillier avec ses sages}\phantomsection
\label{tresor\_3-95}

\chaptercont
\noindent Pour ce doit il honourer et amer tous ciaus de sa mesnie, et rire et esbatre aucunefois avec aus. Mais sor tous doit il amer et honorer les juges et les notaires de son ostel, car il ont en lor main la grignor partie de s’onor et de sa bonté ; et por ce doit li sages poestas sovent et menu, meismement as jours des festes ou le soir en yver tens, assambler les en sa cambre u ailleurs, et parler a aus des choses qui apertienent a lor office, et encerchier k’il font, que les querieles il ont devant aus, et enquerre la nature des ples, et prendre conseil des choses k’il doivent faire. Car c’est une chose de grant sens de sovenir soi des choses alees, et establir les presentes, et porveir des futures.\par
Autresi les doit il priier et amonester k’il soient la droite balance ki contrepoise les drois et les tors, selonc Dieu et selonc justice ; et k’il gardent que drois ne soit vendus ne changiés por deniers ne por amour ne por haine ne por autre chose vivant ; mais soviegne lors que Nostre Sires commande : amés justice, vous ki jugiés la tiere. Mais de ce se taist li mestres, et torne as autres choses.
\chapterclose


\chapteropen
\chapter[{.III.LXXXXVI. De la discorde qi est entre ciaus qi voelent estre cremus et amés}]{\textsc{.III.LXXXXVI.} De la discorde qi est entre ciaus qi voelent estre cremus et amés}\phantomsection
\label{tresor\_3-96}

\chaptercont
\noindent En ceste partie dist li mestres ke entre les governeours des viles siut avoir une tele difference, que li un aiment miex estre cremus ke amés, et li autre desirent plus a estre amés ke cremus. Et cil ki ayment mieus a estre cremus que amés desirent a avoir renomee de grant fierté ; et, por çou k’il welent sambler fiers et crueus, metent tres fieres paines et aspres tormens, et de ce quident que l’en les redoute plus et que la vile en soit mieus apaisie.\par
Et ce pruevent il par les dis Seneque, ki dist k’escharseté de poine corront les cités, et ke l’abondance des pecheours amainent les usages de pecchier, et ke cil pert le hardement de sa malice ki est fierement tormentés, et que li prince soufrant conferme les visces, et la douçcur dou signor oste la vergoigne dou maufetour, et plus est redoutee la paine ki est establie de par son signor ke de par son ami, et de tant comme li torment sont plus apert proufitent il plus par example, et toz li mondes crient les fiers et les hardis, et la paine donne paour as plusours.\par
Contre ce dist li autres ke mieus vaut a estre amés que cremus, por çou c’amours ne puet estre sans cremour, mais cremours si puet bien estre sans amour : Tuilles dit que au monde n’a plus seure chose a deffendre ses choses ke d’iestre amés, ne nule plus espoentable ke d’iestre cremus ; car chascuns het celui k’il crient, et ki de tous est haïs a perir le covient, que nule richesce puet contrester as haines de plusors. Longue paour est male garde, cruauté est enemie de nature.\par
Il covient que chacuns crieme celui ou ciaus de que il wet estre cremus, et force ki est par paour n’aura ja longue duree, et toute paine doit estre mise sans tort, non mie par le signor, mais por le bien dou commun ; ne paine ne doit estre grignour ke la coupe, ne nus ne doit estre dampnés por les crimes d’un autre et tout governement doivent estre sans folie et sans peresce. Tuilles dit, garde ke tu ne face rien dont tu ne puisses moustrer raison por quoi. Et Senekes dit ke cil maufet ki plest plus a sa renomee ke a sa conscience, et cruauté n’est pas autre chose que fiertés de corage es grans paines ; pourquoi je di ke cil est cruel ki n’a mesure en daner quant il en a l’ochoison.\par
Platons dist que nus sages ne dampne de çou que li pechiés est fais mais por çou k’il ne soit fais de lors en avant. Quele difference a il entre roi et tirant ? il sont pareil de fortune et de pooir, mais li tirans fet oevres de crualté par son gré, ce ne fet pas li rois sans necessité : li uns est amés et li autres est cremus. Et cil est tenus a mauvais peres ki tozjors bat et fiert son enfant asprement.\par
Li plus seurs garnimens dou monde est li amours des citains, car il done la plus bele chose en cest siecle, c’est que chacuns desire que tu vives.\par
Par ceste parole puet on bien entendre ceste querele ; car clemence ki est contre cruauté est uns atempremens de corage sor la paine ki li puet establir. Tuilles dist que la plus bele chose ki est en signorie si est clemence et pitiés, s’ele est jointe avec droit, sans coi la cités ne puet pas estre governee. Seneques dist, quant je sui a curer la cité, je i truis tant de visces entre tant de gens, ke pour garir les maus de chascun il couvient que li uns soit sanés par ire et li autres par essil et par pelerinage, et li autres par dolour et li autres par poverté et li autres par fier ; et tout me coviegne il aler por aus dampner, je n’irai o furor ne en la cruauté, mais jou ai a une voie de loi par l’oevre, de sage vois sans orgueil, jugement sans ire. Et iont les mauvais autel samblant et autel corage comme font les serpens et les autres bestes ki portent venin.\par
Il ne covient pas que li sires soit dou tout cruel ne dou tout plains de clemence ; car autresi bien est il cruauté de pardoner a toz comme non pardoner a nului ; mais ce est oevre de haute clemence a confondre les maus en pardonant. Por coi je di que nus ne doit pardoner le maufetor, car li juges est dampnés quant li malfaitour est assous. Autresi ne doit il estre trop cruel, por çou que nule paine ne doit estre grignor que le meffet ne cheoir sor le non nuisant ; car se la paine est du cors, donc est il omecides, et stele est de deniers, a rendre li covient.
\chapterclose


\chapteropen
\chapter[{.III.LXXXXVII. Des choses que li sires doit consirer en sa signorie}]{\textsc{.III.LXXXXVII.} Des choses que li sires doit consirer en sa signorie}\phantomsection
\label{tresor\_3-97}

\chaptercont
\noindent Soviegne toi donc, tu ki governes la cité, dou sairement que tu fesis sor sains quant tu presis l’office de ta signorie. Soviegne tai de la loi et de ses commandemens, et n’oublie pas Dieu et ses sains, mais va sovent au mostier et prie Dieu de toi et de tes subtés. Car David le prophete dist que se Dieus ne garde la cité, por noient se travaillent cil ki le gardent.\par
Honoures les pastours de sainte eglise, car Dieus dist de sa bouche : ki vos reçoit moi reçoit. Soies religieus et mostres la droite foi, pour çou k’il n’a plus bele chose ou prince de la tiere ke avoir droite foi et veraie creance. Et il est escrit : quant li justes rois siet en sa chaiere, nul mal ne puet cheoir contre lui.\par
Et por ce garde les eglises, les maisons Diu ; garde les veves les orfenes, car il est escrit : soies deffenderes des orfenins et des veves, c’est que tu deffendes lor drois contre la mauvestié des poissans, non pas en tel maniere que li poissant perdent lor droit por les larmes du foible ; car tu as en ta garde les grans et les petits et les moiens.\par
Donques covient il des le commencement ke tu pregnes l’office a net cuer et a pure entention, et ke tes mains soient netes, a Dieu et a la loi, de tous gaains outre les loiers dou comun, et ke tu deffendes les choses dou commun et donnes a chascun çou ke sien est ; et que tu porvoies a ton pooir k’il n’ait haine ne descorde entre ses subtés, et s’ele i est que tu ne soies ploiés as uns plus c’as autres, ne por argent ne por femes ne por chose ki soit, et ke tu entendes diligemment les plés et les plaintes ; et que tu delivres les petites querieles tost et legierement sans respit ; et que tu faces tout çou ki est escrit es livres des constitutions de la vile ; et que tu maintiegnes les oevres et les edifices du commun, et faces afaitier les voies et les pons et les portes et les murs et les fosses et les autres choses.\par
Ne suefre pas que li maufetour eschapent sans paine, ke nus dou païs les detiegne. Les murtriers les traitres et ciaus qui efforcent les puceles et ki font ces autres crimes, dois tu dampner fierement selonc la loi et l’us dou païs. Tien tes officieus en tel maniere k’il ne facent ne tort ne anui. Aies entor toi tel conseilleour ki soit bons et sages et loiaus a toi et a raison. Soies tex que tu sambles terribles as mauvais et agreables as bons. En some regarde la seconde partie dou livre, la u il parole ça ariere des visces et des vertus, et gardes ke tu soies garnis des vertus et nés de visces.
\chapterclose


\chapteropen
\chapter[{.III.LXXXXVIII. Des choses dont li sires se doit garder por achoison de soi}]{\textsc{.III.LXXXXVIII.} Des choses dont li sires se doit garder por achoison de soi}\phantomsection
\label{tresor\_3-98}

\chaptercont
\noindent Or dist li mestres k’il ne vieut pas en ceste derraine partie nomer la seurté de quoi li sires doit estre garnis, por çou k’il en a assés longuement dit en la seconde partie dou livre,et por ce s’en taist a tant. Et neporquant il dira aucun des visces dont li sires se doit fierement garder, et il et ses sages. Car sans faille il se doit mout garder de ces choses dont il commande que li autre se gardent, selonc ce que li Apostles dist : je chasti, fist il, tot avant mon cuer, et le met en siervage, si ke je ne soie dampnés en chastoiant les autres. Et Catons dist que laide chose est au mestre quant la coupe en torne sor lui. Mais bien dire est loable s’il le fait ; car bien dire et mau fere n’est autre chose que dampner soi par sa parole.\par
Aprés ce se doit il garder d’yvrece, d’orguel, d’ire, de dolour, d’avarisce, et d’envie, et de luxure ; car chascuns de ces pechiés est mortex a Dieu et as homes, et fait le prinche cheoir legierement de son siege. Mais mout se doit garder de trop parler, car s’il parole po et bien, on le tient a plus sage, et molt parler n’est sans pechié. Autresi se doit il garder de trop rire, car il est escrit que ris est en la bouche des fols. Et nanporquant il puet bien rire et jouer et esbatre aucunefois, non pas a maniere d’enfant ne de feme, ne ki samble faus ris ne orguilleus. Et s’il est bons des autres, il sera cremus s’il ne moustre liet le visage, meismement quant il est assis a oïr plet.\par
Autresi ne doit il loer soi meismes, por ce k’il soit loés des bons, et ne li chaut s’il est loés des mauvais. Et garde soi de gengleours ki le loent devant lui. Croie de soi plus a soi que as autres, et soit autresi tristes quant il est loés par les malvais comme s’il fust loés par males oevres. Autresi se doit il garder des espieurs, k’il ne die ne ne face chose, s’ele est seue, k’il en soit blasmés.\par
Autresi garde que justice ne soi pas vendue por deniers ; car la lois dist k’il doit estre dampnés comme lerres. Autresi garde k’il ne soit privés de ses subtés, pour çou k’il en chiet en despit et en souspeçon.\par
Autresi garde k’il ne reçoive nul present de nuli ki soit sous son governement, por ce que tous homes ki reçoivent don u service a sa franchise vendue, et est obligiés comme par dette. Autresi garde k’il ne se conseille a nului de la vile priveement, ne ne chevauce avec nului, ne ne voise a sa maison por mangier ne por autre chose, por ce que de ce naist suspections de lui entre ses citains.
\chapterclose


\chapteropen
\chapter[{.III.LXXXXVIIII. Des choses dont li sires se doit garder por ochoison dou commun}]{\textsc{.III.LXXXXVIIII.} Des choses dont li sires se doit garder por ochoison dou commun}\phantomsection
\label{tresor\_3-99}

\chaptercont
\noindent Autresi se doit li sires mout garder k’il, par le comun k’il a en sa garde,ne face nule conjurison ne compaignie avec les autres cités et viles dou païs ; et se a fere le covient, si le face par le conseil de la vile et par le commun assentement des gens. Car en teus choses doit on penser et repenser longuement k’il ne face tel lien ki puis li coviegne brisier sa foi ; ou s’il ne le brise ke li periz en viegne sour lui.\par
Autresi garde k’il ne mete a son tans ne taille ne chartre de vante ne dette ne de nul ligement dou commun, se ce ne fust por manifest proufit de la vile et par commun establissement dou conseil.
\chapterclose


\chapteropen
\chapter[{.III.C. Que li sires doit fire au tens de guerre}]{\textsc{.III.C.} Que li sires doit fire au tens de guerre}\phantomsection
\label{tresor\_3-100}

\chaptercont
\noindent En ceste partie dist li mestres ke en signorie a \textsc{.ii.} saisons, une de pais, autre de guerre. Et por ce k’il dist assés bien de l’un et de l’autre el livre des visces et des vertus el chapistre de manificence, n’en dira il ore autre chose se ce non k’il covient au signeur por son office. Et certes se li sires, quant il vieut governer la vile, et il le trueve en pais, il en doit estre mout liés et joians, et garder k’il ne commence guerre a son tens s’il onques puet, car en guerre a trop de periz.\par
Mais se a commencier le covient, ce soit fet par les communs assentemens des citeins, et par establissement de conseil et des sages gens de la vile. Més se la guerre estoit commencie au tens de ses anciestres, je lo que il porchace la pais, ou au mains les trives ; ou se ce non, il doit requerre sovent et menu le conseil des sages homes, et espiier le pooir de sa partie et de ses enemis, et estudiier que la vile soit bien gardee dedens et dehors, et les chastiaus et les viles ki sont baillies en sa garde.\par
Et si doit avoir entor lui une gent de sages et de vaillans homes de la vile, ki se sachent meller de guerres et ki soient tousjors a son conseil et ki soient aprés lui chievetains et guieurs de la guerre. Et requerre toz les amis et compaignons et les subtés de la vile, les uns par letres les autres par bouche et les autres par messagiers ki soient apareilliés as armes et a la guerre.\par
Aprés ce doit il assambler a la mestre place de la vile, ou en autres lieus acoustumés as gens de la vile, et dire devant aus paroles de guerre, et ramentevoir les tors des enemis et les drois des citains, nomer la proece et les valours de lor ancestres et les lor vertueuses batailles, semonre les gens a la guerre et conorter les a bataille, et commander que chascuns face grant apareil d’armes et de chevaus et de tentes et de pavillons et de toutes choses ki besoignent en guerre.\par
Teux et autres paroles doit li sires dire por acuser les corages des gens au plus k’il onques puet ; mais bien garde k’il ne die nul foible mot, ains soit sa chiere de courouç et d’ire, le samblant terrible, la vois menachable, et son cheval hennisse et fiere les piés a la tierre ; et face tant que maintes fois, ançois k’il fine son dit, la noise lieve et li cris entre les citeins, autresi comme s’il fussent a la mellee. Et nanporquant il doit mout consirer la maniere de la guerre, por ce que autre samblant covient entre les graindres et autre contre les pers et autre contre les menours.\par
Aprés son parlement face lire par la bouche de son notaire, ki ait clere vois et entendable, les ordenemens et les chapistres de la guerre, et porchace s’il onques puet k’il ait arbitre sor le malefisce de l’ost. Et quant tout çou est fet, il doit de ses mains baillier les gonfanons et les banieres, selonc les coustumes de la vile. De lors en avant ne fine d’apareillier li sires soi et tous ses subtés a la guerre, en tel maniere ke rien n’i faille au point de l’ost et de la bataille.\par
Mais coment il doit guier l’ost et metre les chans des pavillons et garder l’ost tout environ, de jour et de nuit, ne comment il doit establir les eschieles et comment il doit estre en toz lieus, ore de cha ore de la, et comment il doit garder son cors, k’il ne combate se ce n’est par necessité, coment il doit gaitier sa vile s’ele est assegie, ne de maintes autres choses ki covienent a guerre, li mestres n’en dist ore plus ; ains les laisse a la porveance dou signour et dou conseil.
\chapterclose


\chapteropen
\chapter[{.III.CI. Le general ensegnement des provostés}]{\textsc{.III.CI.} Le general ensegnement des provostés}\phantomsection
\label{tresor\_3-101}

\chaptercont
\noindent Par les ensegnemens de cest livre puet bien chascuns ki droitement le regarde governer la cité, au tans de pais et de guerre, a l’aide de Dieu et de bon conseil. Et ja soit ce k’il ait assés d’ensegnemens, neporquant il a es signories de choses tant et de diversités ke nus hom vivans ne les poroit escrire ne dire de bouche ; mais en some il doit ensivre la loi commune et les us de la vile a bone foi et conduire son office selonc la coustume dou païs. Por ce que le vilains dist, quant tu ies a Rome vif comme a Rome, car de deus tiere tex pos.\par
Sor le malefice doit il ensuir la maniere des mires, ki a petites maladies metent petites medecines, et as grignours metent il grignours et plus fors, et as trés grans metent il le feu et le fer. Tout autresi doit il dampner les maufetours selonc la maniere de son meffait, sans pardoner a ciaus ki ont coupe et sans grever a ciaus ki ne l’ont.
\chapterclose


\chapteropen
\chapter[{.III.CII. Comment le novel governeor doit estre esleu}]{\textsc{.III.CII.} Comment le novel governeor doit estre esleu}\phantomsection
\label{tresor\_3-102}

\chaptercont
\noindent Et quant vient li tens ke l’en doit penser dou novel signor por l’anee ki vient apriés, li sires doit assambler le conseil de la vile, et par aus trover selonc la loi de la vile les preudomes ki doivent amender les constitutions de la vile. Et quant il les a trovés et il ont fait lor sairement, il doivent estre en \textsc{.i.} lieu priveement, tant k’il acomplissent ce ki apertient a lor office. Et maintenant que li livres est establis et acomplis, il doit estre clos et saielés et mis en garde jusques a la venue dou novel signeur.\par
Et quant ces choses sont diligement acomplies et mises en ordre, l’en doit eslire le noviel signor selonc l’ordre que li mestres devisa el commencement de cest livre. Mais se li citein te welent avoir a signour pour l’annee ki vient, je lo que tu ne le prengnes, car a paine puet bien estre la seconde signorie finee.
\chapterclose


\chapteropen
\chapter[{.III.CIII. Coment li sire se doit porveir entor l’issue de sa signorie}]{\textsc{.III.CIII.} Coment li sire se doit porveir entor l’issue de sa signorie}\phantomsection
\label{tresor\_3-103}

\chaptercont
\noindent Après ce dois tu assambler les juges et les notaires et tes autres officiaus, et priier les et amonester, que toutes querieles et tous plés ki sont devant aus, il les delivrent selonc droit jugement, et k’il ne laissent noient a autrui amendement. Tu meismes te conseilles avec aus, et te pense en ton cuer se tu as nului grevé plus u mains que drois commande. Et se tu as laissié a fere noient de ce ki est el livre de la vile, maintenant t’en porvoit en tel maniere que tu amendes et acomplisses et torne a point ce que tu pués, ou par toi ou par establissement dou conseil ; car li sages governeors se porvoit au devant, ou par ciaus ki amendent les constitutions u par le conseillors meismes, et se fait asoudre de toutes choses ki sont parvenues as chambrelains dou commun, et des autres chapistres ki sont demoré.\par
Autresi dois tu dedens ton tens, s’il est mestiers, trover ambaisseours, par la volenté du commun, ki te facent compaignie jusques a ton ostel et ki portent grasces et salus et bons tesmong de toi et de tes oevres au commun de ta vile. Autresi te porvoit, par le conseil de la vile, de maison en quoi tu demeures aprés la fin por rendre ton conte.\par
Mais n’oublie pas une chose, que \textsc{.x.} jors u \textsc{.viii.} jours devant ta fin tu faces crier sovent et menu ke chascuns ki doit avoir, de toi ne des tiens, k’il viegnent prendre lor paiement, et fai tant que tout soient bien paiet. Autresi garde que tu retiegnes l’essample de tous les chapistles et des establissemens dou conseil ki touchent a toi u a ton sairement, en tel maniere que tu t’en puisses aidier se l’en mesist sor toi nule calenge.
\chapterclose


\chapteropen
\chapter[{.III.CIIII. Comment tu dois parler et assambler les gens ou derrenier jor de ton office}]{\textsc{.III.CIIII.} Comment tu dois parler et assambler les gens ou derrenier jor de ton office}\phantomsection
\label{tresor\_3-104}

\chaptercont
\noindent Et quant vient ou derrenier jor de ton office, tu dois assambler les gens de la vile, et devant aus dire des grans paroles et agreables por aquerre l’amor et la bienvoeillance des citains, et ramentevoir des bones oevres,les honours, et les proufis dou commun ki sont avenues a ton tens, et amercoer les de l’amor et de l’onour k’il ont fait a toi et as tiens, et offrir toi et tout ton pooir en lor service en tote ta vie.\par
Et por mieus atrere les corages des gens, tu pués dire que se aucuns a mespris jusques lors contre son sairement, u par perrece u par non savoir ou par autres choses, que tu li pardoignes, se ce n’est murtriers ou lerres ou autres maufetours ou dampnés de la vile.\par
Mais toutefois retien a toi toute ta signorie jusques a la mienuit ou tu la commetes au novel provost.\par
Aprés cest parlement, le jour meismes ou l’autre aprés, selonc la maniere dou païs, dois tu rendre au novel signour ou chambrelains tous les livres et toutes choses que tu avoies de par le commun ; et puis t’en iras al’ostel en quoi tu dois herbergier tant comme tu demorras a rendre ton conte
\chapterclose


\chapteropen
\chapter[{.III.CV. Coment tu dois demorer en la vile a rendre ton conte}]{\textsc{.III.CV.} Coment tu dois demorer en la vile a rendre ton conte}\phantomsection
\label{tresor\_3-105}

\chaptercont
\noindent Quant tu ies a ce venus, il te covient iestre cindeés et rendre conte de ton office a toi et as tiens. S’il i a nul ki se plaingne de toi, tu te dois faire baillier le libel de sa demande et avoir conseil de tes sages et respondre si comme il te conseillent. En ceste maniere dois tu demorer a la vile jusques au jour ki fu establis quant tu presis la provosté. Lors se a Dieu plaist tu seras assols honorablement, et prenderas congié dou conseil et dou commun de la vile, et t’en iras chiés toi a gloire et a honour.\par
Amen.
\chapterclose

\chapterclose

 


% at least one empty page at end (for booklet couv)
\ifbooklet
  \pagestyle{empty}
  \clearpage
  % 2 empty pages maybe needed for 4e cover
  \ifnum\modulo{\value{page}}{4}=0 \hbox{}\newpage\hbox{}\newpage\fi
  \ifnum\modulo{\value{page}}{4}=1 \hbox{}\newpage\hbox{}\newpage\fi


  \hbox{}\newpage
  \ifodd\value{page}\hbox{}\newpage\fi
  {\centering\color{rubric}\bfseries\noindent\large
    Hurlus ? Qu’est-ce.\par
    \bigskip
  }
  \noindent Des bouquinistes électroniques, pour du texte libre à participation libre,
  téléchargeable gratuitement sur \href{https://hurlus.fr}{\dotuline{hurlus.fr}}.\par
  \bigskip
  \noindent Cette brochure a été produite par des éditeurs bénévoles.
  Elle n’est pas faîte pour être possédée, mais pour être lue, et puis donnée.
  Que circule le texte !
  En page de garde, on peut ajouter une date, un lieu, un nom ; pour suivre le voyage des idées.
  \par

  Ce texte a été choisi parce qu’une personne l’a aimé,
  ou haï, elle a en tous cas pensé qu’il partipait à la formation de notre présent ;
  sans le souci de plaire, vendre, ou militer pour une cause.
  \par

  L’édition électronique est soigneuse, tant sur la technique
  que sur l’établissement du texte ; mais sans aucune prétention scolaire, au contraire.
  Le but est de s’adresser à tous, sans distinction de science ou de diplôme.
  Au plus direct ! (possible)
  \par

  Cet exemplaire en papier a été tiré sur une imprimante personnelle
   ou une photocopieuse. Tout le monde peut le faire.
  Il suffit de
  télécharger un fichier sur \href{https://hurlus.fr}{\dotuline{hurlus.fr}},
  d’imprimer, et agrafer ; puis de lire et donner.\par

  \bigskip

  \noindent PS : Les hurlus furent aussi des rebelles protestants qui cassaient les statues dans les églises catholiques. En 1566 démarra la révolte des gueux dans le pays de Lille. L’insurrection enflamma la région jusqu’à Anvers où les gueux de mer bloquèrent les bateaux espagnols.
  Ce fut une rare guerre de libération dont naquit un pays toujours libre : les Pays-Bas.
  En plat pays francophone, par contre, restèrent des bandes de huguenots, les hurlus, progressivement réprimés par la très catholique Espagne.
  Cette mémoire d’une défaite est éteinte, rallumons-la. Sortons les livres du culte universitaire, cherchons les idoles de l’époque, pour les briser.
\fi

\ifdev % autotext in dev mode
\fontname\font — \textsc{Les règles du jeu}\par
(\hyperref[utopie]{\underline{Lien}})\par
\noindent \initialiv{A}{lors là}\blindtext\par
\noindent \initialiv{À}{ la bonheur des dames}\blindtext\par
\noindent \initialiv{É}{tonnez-le}\blindtext\par
\noindent \initialiv{Q}{ualitativement}\blindtext\par
\noindent \initialiv{V}{aloriser}\blindtext\par
\Blindtext
\phantomsection
\label{utopie}
\Blinddocument
\fi
\end{document}
