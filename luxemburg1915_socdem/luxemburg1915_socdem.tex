%%%%%%%%%%%%%%%%%%%%%%%%%%%%%%%%%
% LaTeX model https://hurlus.fr %
%%%%%%%%%%%%%%%%%%%%%%%%%%%%%%%%%

% Needed before document class
\RequirePackage{pdftexcmds} % needed for tests expressions
\RequirePackage{fix-cm} % correct units

% Define mode
\def\mode{a4}

\newif\ifaiv % a4
\newif\ifav % a5
\newif\ifbooklet % booklet
\newif\ifcover % cover for booklet

\ifnum \strcmp{\mode}{cover}=0
  \covertrue
\else\ifnum \strcmp{\mode}{booklet}=0
  \booklettrue
\else\ifnum \strcmp{\mode}{a5}=0
  \avtrue
\else
  \aivtrue
\fi\fi\fi

\ifbooklet % do not enclose with {}
  \documentclass[french,twoside]{book} % ,notitlepage
  \usepackage[%
    papersize={105mm, 297mm},
    inner=12mm,
    outer=12mm,
    top=20mm,
    bottom=15mm,
    marginparsep=0pt,
  ]{geometry}
  \usepackage[fontsize=9.5pt]{scrextend} % for Roboto
\else\ifav
  \documentclass[french,twoside]{book} % ,notitlepage
  \usepackage[%
    a5paper,
    inner=25mm,
    outer=15mm,
    top=15mm,
    bottom=15mm,
    marginparsep=0pt,
  ]{geometry}
  \usepackage[fontsize=12pt]{scrextend}
\else% A4 2 cols
  \documentclass[twocolumn]{report}
  \usepackage[%
    a4paper,
    inner=15mm,
    outer=10mm,
    top=25mm,
    bottom=18mm,
    marginparsep=0pt,
  ]{geometry}
  \setlength{\columnsep}{20mm}
  \usepackage[fontsize=9.5pt]{scrextend}
\fi\fi

%%%%%%%%%%%%%%
% Alignments %
%%%%%%%%%%%%%%
% before teinte macros

\setlength{\arrayrulewidth}{0.2pt}
\setlength{\columnseprule}{\arrayrulewidth} % twocol
\setlength{\parskip}{0pt} % classical para with no margin
\setlength{\parindent}{1.5em}

%%%%%%%%%%
% Colors %
%%%%%%%%%%
% before Teinte macros

\usepackage[dvipsnames]{xcolor}
\definecolor{rubric}{HTML}{800000} % the tonic 0c71c3
\def\columnseprulecolor{\color{rubric}}
\colorlet{borderline}{rubric!30!} % definecolor need exact code
\definecolor{shadecolor}{gray}{0.95}
\definecolor{bghi}{gray}{0.5}

%%%%%%%%%%%%%%%%%
% Teinte macros %
%%%%%%%%%%%%%%%%%
%%%%%%%%%%%%%%%%%%%%%%%%%%%%%%%%%%%%%%%%%%%%%%%%%%%
% <TEI> generic (LaTeX names generated by Teinte) %
%%%%%%%%%%%%%%%%%%%%%%%%%%%%%%%%%%%%%%%%%%%%%%%%%%%
% This template is inserted in a specific design
% It is XeLaTeX and otf fonts

\makeatletter % <@@@


\usepackage{blindtext} % generate text for testing
\usepackage[strict]{changepage} % for modulo 4
\usepackage{contour} % rounding words
\usepackage[nodayofweek]{datetime}
% \usepackage{DejaVuSans} % seems buggy for sffont font for symbols
\usepackage{enumitem} % <list>
\usepackage{etoolbox} % patch commands
\usepackage{fancyvrb}
\usepackage{fancyhdr}
\usepackage{float}
\usepackage{fontspec} % XeLaTeX mandatory for fonts
\usepackage{footnote} % used to capture notes in minipage (ex: quote)
\usepackage{framed} % bordering correct with footnote hack
\usepackage{graphicx}
\usepackage{lettrine} % drop caps
\usepackage{lipsum} % generate text for testing
\usepackage[framemethod=tikz,]{mdframed} % maybe used for frame with footnotes inside
\usepackage{pdftexcmds} % needed for tests expressions
\usepackage{polyglossia} % non-break space french punct, bug Warning: "Failed to patch part"
\usepackage[%
  indentfirst=false,
  vskip=1em,
  noorphanfirst=true,
  noorphanafter=true,
  leftmargin=\parindent,
  rightmargin=0pt,
]{quoting}
\usepackage{ragged2e}
\usepackage{setspace} % \setstretch for <quote>
\usepackage{tabularx} % <table>
\usepackage[explicit]{titlesec} % wear titles, !NO implicit
\usepackage{tikz} % ornaments
\usepackage{tocloft} % styling tocs
\usepackage[fit]{truncate} % used im runing titles
\usepackage{unicode-math}
\usepackage[normalem]{ulem} % breakable \uline, normalem is absolutely necessary to keep \emph
\usepackage{verse} % <l>
\usepackage{xcolor} % named colors
\usepackage{xparse} % @ifundefined
\XeTeXdefaultencoding "iso-8859-1" % bad encoding of xstring
\usepackage{xstring} % string tests
\XeTeXdefaultencoding "utf-8"
\PassOptionsToPackage{hyphens}{url} % before hyperref, which load url package

% TOTEST
% \usepackage{hypcap} % links in caption ?
% \usepackage{marginnote}
% TESTED
% \usepackage{background} % doesn’t work with xetek
% \usepackage{bookmark} % prefers the hyperref hack \phantomsection
% \usepackage[color, leftbars]{changebar} % 2 cols doc, impossible to keep bar left
% \usepackage[utf8x]{inputenc} % inputenc package ignored with utf8 based engines
% \usepackage[sfdefault,medium]{inter} % no small caps
% \usepackage{firamath} % choose firasans instead, firamath unavailable in Ubuntu 21-04
% \usepackage{flushend} % bad for last notes, supposed flush end of columns
% \usepackage[stable]{footmisc} % BAD for complex notes https://texfaq.org/FAQ-ftnsect
% \usepackage{helvet} % not for XeLaTeX
% \usepackage{multicol} % not compatible with too much packages (longtable, framed, memoir…)
% \usepackage[default,oldstyle,scale=0.95]{opensans} % no small caps
% \usepackage{sectsty} % \chapterfont OBSOLETE
% \usepackage{soul} % \ul for underline, OBSOLETE with XeTeX
% \usepackage[breakable]{tcolorbox} % text styling gone, footnote hack not kept with breakable


% Metadata inserted by a program, from the TEI source, for title page and runing heads
\title{\textbf{ La crise de la social-démocratie (Brochure de Junius) }}
\date{1915}
\author{Luxemburg, Rosa}
\def\elbibl{Luxemburg, Rosa. 1915. \emph{La crise de la social-démocratie (Brochure de Junius)}}
\def\elsource{\{bibl\}}

% Default metas
\newcommand{\colorprovide}[2]{\@ifundefinedcolor{#1}{\colorlet{#1}{#2}}{}}
\colorprovide{rubric}{red}
\colorprovide{silver}{lightgray}
\@ifundefined{syms}{\newfontfamily\syms{DejaVu Sans}}{}
\newif\ifdev
\@ifundefined{elbibl}{% No meta defined, maybe dev mode
  \newcommand{\elbibl}{Titre court ?}
  \newcommand{\elbook}{Titre du livre source ?}
  \newcommand{\elabstract}{Résumé\par}
  \newcommand{\elurl}{http://oeuvres.github.io/elbook/2}
  \author{Éric Lœchien}
  \title{Un titre de test assez long pour vérifier le comportement d’une maquette}
  \date{1566}
  \devtrue
}{}
\let\eltitle\@title
\let\elauthor\@author
\let\eldate\@date


\defaultfontfeatures{
  % Mapping=tex-text, % no effect seen
  Scale=MatchLowercase,
  Ligatures={TeX,Common},
}


% generic typo commands
\newcommand{\astermono}{\medskip\centerline{\color{rubric}\large\selectfont{\syms ✻}}\medskip\par}%
\newcommand{\astertri}{\medskip\par\centerline{\color{rubric}\large\selectfont{\syms ✻\,✻\,✻}}\medskip\par}%
\newcommand{\asterism}{\bigskip\par\noindent\parbox{\linewidth}{\centering\color{rubric}\large{\syms ✻}\\{\syms ✻}\hskip 0.75em{\syms ✻}}\bigskip\par}%

% lists
\newlength{\listmod}
\setlength{\listmod}{\parindent}
\setlist{
  itemindent=!,
  listparindent=\listmod,
  labelsep=0.2\listmod,
  parsep=0pt,
  % topsep=0.2em, % default topsep is best
}
\setlist[itemize]{
  label=—,
  leftmargin=0pt,
  labelindent=1.2em,
  labelwidth=0pt,
}
\setlist[enumerate]{
  label={\bf\color{rubric}\arabic*.},
  labelindent=0.8\listmod,
  leftmargin=\listmod,
  labelwidth=0pt,
}
\newlist{listalpha}{enumerate}{1}
\setlist[listalpha]{
  label={\bf\color{rubric}\alph*.},
  leftmargin=0pt,
  labelindent=0.8\listmod,
  labelwidth=0pt,
}
\newcommand{\listhead}[1]{\hspace{-1\listmod}\emph{#1}}

\renewcommand{\hrulefill}{%
  \leavevmode\leaders\hrule height 0.2pt\hfill\kern\z@}

% General typo
\DeclareTextFontCommand{\textlarge}{\large}
\DeclareTextFontCommand{\textsmall}{\small}

% commands, inlines
\newcommand{\anchor}[1]{\Hy@raisedlink{\hypertarget{#1}{}}} % link to top of an anchor (not baseline)
\newcommand\abbr[1]{#1}
\newcommand{\autour}[1]{\tikz[baseline=(X.base)]\node [draw=rubric,thin,rectangle,inner sep=1.5pt, rounded corners=3pt] (X) {\color{rubric}#1};}
\newcommand\corr[1]{#1}
\newcommand{\ed}[1]{ {\color{silver}\sffamily\footnotesize (#1)} } % <milestone ed="1688"/>
\newcommand\expan[1]{#1}
\newcommand\foreign[1]{\emph{#1}}
\newcommand\gap[1]{#1}
\renewcommand{\LettrineFontHook}{\color{rubric}}
\newcommand{\initial}[2]{\lettrine[lines=2, loversize=0.3, lhang=0.3]{#1}{#2}}
\newcommand{\initialiv}[2]{%
  \let\oldLFH\LettrineFontHook
  % \renewcommand{\LettrineFontHook}{\color{rubric}\ttfamily}
  \IfSubStr{QJ’}{#1}{
    \lettrine[lines=4, lhang=0.2, loversize=-0.1, lraise=0.2]{\smash{#1}}{#2}
  }{\IfSubStr{É}{#1}{
    \lettrine[lines=4, lhang=0.2, loversize=-0, lraise=0]{\smash{#1}}{#2}
  }{\IfSubStr{ÀÂ}{#1}{
    \lettrine[lines=4, lhang=0.2, loversize=-0, lraise=0, slope=0.6em]{\smash{#1}}{#2}
  }{\IfSubStr{A}{#1}{
    \lettrine[lines=4, lhang=0.2, loversize=0.2, slope=0.6em]{\smash{#1}}{#2}
  }{\IfSubStr{V}{#1}{
    \lettrine[lines=4, lhang=0.2, loversize=0.2, slope=-0.5em]{\smash{#1}}{#2}
  }{
    \lettrine[lines=4, lhang=0.2, loversize=0.2]{\smash{#1}}{#2}
  }}}}}
  \let\LettrineFontHook\oldLFH
}
\newcommand{\labelchar}[1]{\textbf{\color{rubric} #1}}
\newcommand{\milestone}[1]{\autour{\footnotesize\color{rubric} #1}} % <milestone n="4"/>
\newcommand\name[1]{#1}
\newcommand\orig[1]{#1}
\newcommand\orgName[1]{#1}
\newcommand\persName[1]{#1}
\newcommand\placeName[1]{#1}
\newcommand{\pn}[1]{\IfSubStr{-—–¶}{#1}% <p n="3"/>
  {\noindent{\bfseries\color{rubric}   ¶  }}
  {{\footnotesize\autour{ #1}  }}}
\newcommand\reg{}
% \newcommand\ref{} % already defined
\newcommand\sic[1]{#1}
\newcommand\surname[1]{\textsc{#1}}
\newcommand\term[1]{\textbf{#1}}

\def\mednobreak{\ifdim\lastskip<\medskipamount
  \removelastskip\nopagebreak\medskip\fi}
\def\bignobreak{\ifdim\lastskip<\bigskipamount
  \removelastskip\nopagebreak\bigskip\fi}

% commands, blocks
\newcommand{\byline}[1]{\bigskip{\RaggedLeft{#1}\par}\bigskip}
\newcommand{\bibl}[1]{{\RaggedLeft{#1}\par\bigskip}}
\newcommand{\biblitem}[1]{{\noindent\hangindent=\parindent   #1\par}}
\newcommand{\dateline}[1]{\medskip{\RaggedLeft{#1}\par}\bigskip}
\newcommand{\labelblock}[1]{\medbreak{\noindent\color{rubric}\bfseries #1}\par\mednobreak}
\newcommand{\salute}[1]{\bigbreak{#1}\par\medbreak}
\newcommand{\signed}[1]{\bigbreak\filbreak{\raggedleft #1\par}\medskip}

% environments for blocks (some may become commands)
\newenvironment{borderbox}{}{} % framing content
\newenvironment{citbibl}{\ifvmode\hfill\fi}{\ifvmode\par\fi }
\newenvironment{docAuthor}{\ifvmode\vskip4pt\fontsize{16pt}{18pt}\selectfont\fi\itshape}{\ifvmode\par\fi }
\newenvironment{docDate}{}{\ifvmode\par\fi }
\newenvironment{docImprint}{\vskip6pt}{\ifvmode\par\fi }
\newenvironment{docTitle}{\vskip6pt\bfseries\fontsize{18pt}{22pt}\selectfont}{\par }
\newenvironment{msHead}{\vskip6pt}{\par}
\newenvironment{msItem}{\vskip6pt}{\par}
\newenvironment{titlePart}{}{\par }


% environments for block containers
\newenvironment{argument}{\itshape\parindent0pt}{\vskip1.5em}
\newenvironment{biblfree}{}{\ifvmode\par\fi }
\newenvironment{bibitemlist}[1]{%
  \list{\@biblabel{\@arabic\c@enumiv}}%
  {%
    \settowidth\labelwidth{\@biblabel{#1}}%
    \leftmargin\labelwidth
    \advance\leftmargin\labelsep
    \@openbib@code
    \usecounter{enumiv}%
    \let\p@enumiv\@empty
    \renewcommand\theenumiv{\@arabic\c@enumiv}%
  }
  \sloppy
  \clubpenalty4000
  \@clubpenalty \clubpenalty
  \widowpenalty4000%
  \sfcode`\.\@m
}%
{\def\@noitemerr
  {\@latex@warning{Empty `bibitemlist' environment}}%
\endlist}
\newenvironment{quoteblock}% may be used for ornaments
  {\begin{quoting}}
  {\end{quoting}}

% table () is preceded and finished by custom command
\newcommand{\tableopen}[1]{%
  \ifnum\strcmp{#1}{wide}=0{%
    \begin{center}
  }
  \else\ifnum\strcmp{#1}{long}=0{%
    \begin{center}
  }
  \else{%
    \begin{center}
  }
  \fi\fi
}
\newcommand{\tableclose}[1]{%
  \ifnum\strcmp{#1}{wide}=0{%
    \end{center}
  }
  \else\ifnum\strcmp{#1}{long}=0{%
    \end{center}
  }
  \else{%
    \end{center}
  }
  \fi\fi
}


% text structure
\newcommand\chapteropen{} % before chapter title
\newcommand\chaptercont{} % after title, argument, epigraph…
\newcommand\chapterclose{} % maybe useful for multicol settings
\setcounter{secnumdepth}{-2} % no counters for hierarchy titles
\setcounter{tocdepth}{5} % deep toc
\markright{\@title} % ???
\markboth{\@title}{\@author} % ???
\renewcommand\tableofcontents{\@starttoc{toc}}
% toclof format
% \renewcommand{\@tocrmarg}{0.1em} % Useless command?
% \renewcommand{\@pnumwidth}{0.5em} % {1.75em}
\renewcommand{\@cftmaketoctitle}{}
\setlength{\cftbeforesecskip}{\z@ \@plus.2\p@}
\renewcommand{\cftchapfont}{}
\renewcommand{\cftchapdotsep}{\cftdotsep}
\renewcommand{\cftchapleader}{\normalfont\cftdotfill{\cftchapdotsep}}
\renewcommand{\cftchappagefont}{\bfseries}
\setlength{\cftbeforechapskip}{0em \@plus\p@}
% \renewcommand{\cftsecfont}{\small\relax}
\renewcommand{\cftsecpagefont}{\normalfont}
% \renewcommand{\cftsubsecfont}{\small\relax}
\renewcommand{\cftsecdotsep}{\cftdotsep}
\renewcommand{\cftsecpagefont}{\normalfont}
\renewcommand{\cftsecleader}{\normalfont\cftdotfill{\cftsecdotsep}}
\setlength{\cftsecindent}{1em}
\setlength{\cftsubsecindent}{2em}
\setlength{\cftsubsubsecindent}{3em}
\setlength{\cftchapnumwidth}{1em}
\setlength{\cftsecnumwidth}{1em}
\setlength{\cftsubsecnumwidth}{1em}
\setlength{\cftsubsubsecnumwidth}{1em}

% footnotes
\newif\ifheading
\newcommand*{\fnmarkscale}{\ifheading 0.70 \else 1 \fi}
\renewcommand\footnoterule{\vspace*{0.3cm}\hrule height \arrayrulewidth width 3cm \vspace*{0.3cm}}
\setlength\footnotesep{1.5\footnotesep} % footnote separator
\renewcommand\@makefntext[1]{\parindent 1.5em \noindent \hb@xt@1.8em{\hss{\normalfont\@thefnmark . }}#1} % no superscipt in foot
\patchcmd{\@footnotetext}{\footnotesize}{\footnotesize\sffamily}{}{} % before scrextend, hyperref


%   see https://tex.stackexchange.com/a/34449/5049
\def\truncdiv#1#2{((#1-(#2-1)/2)/#2)}
\def\moduloop#1#2{(#1-\truncdiv{#1}{#2}*#2)}
\def\modulo#1#2{\number\numexpr\moduloop{#1}{#2}\relax}

% orphans and widows
\clubpenalty=9996
\widowpenalty=9999
\brokenpenalty=4991
\predisplaypenalty=10000
\postdisplaypenalty=1549
\displaywidowpenalty=1602
\hyphenpenalty=400
% Copied from Rahtz but not understood
\def\@pnumwidth{1.55em}
\def\@tocrmarg {2.55em}
\def\@dotsep{4.5}
\emergencystretch 3em
\hbadness=4000
\pretolerance=750
\tolerance=2000
\vbadness=4000
\def\Gin@extensions{.pdf,.png,.jpg,.mps,.tif}
% \renewcommand{\@cite}[1]{#1} % biblio

\usepackage{hyperref} % supposed to be the last one, :o) except for the ones to follow
\urlstyle{same} % after hyperref
\hypersetup{
  % pdftex, % no effect
  pdftitle={\elbibl},
  % pdfauthor={Your name here},
  % pdfsubject={Your subject here},
  % pdfkeywords={keyword1, keyword2},
  bookmarksnumbered=true,
  bookmarksopen=true,
  bookmarksopenlevel=1,
  pdfstartview=Fit,
  breaklinks=true, % avoid long links
  pdfpagemode=UseOutlines,    % pdf toc
  hyperfootnotes=true,
  colorlinks=false,
  pdfborder=0 0 0,
  % pdfpagelayout=TwoPageRight,
  % linktocpage=true, % NO, toc, link only on page no
}

\makeatother % /@@@>
%%%%%%%%%%%%%%
% </TEI> end %
%%%%%%%%%%%%%%


%%%%%%%%%%%%%
% footnotes %
%%%%%%%%%%%%%
\renewcommand{\thefootnote}{\bfseries\textcolor{rubric}{\arabic{footnote}}} % color for footnote marks

%%%%%%%%%
% Fonts %
%%%%%%%%%
\usepackage[]{roboto} % SmallCaps, Regular is a bit bold
% \linespread{0.90} % too compact, keep font natural
\newfontfamily\fontrun[]{Roboto Condensed Light} % condensed runing heads
\ifav
  \setmainfont[
    ItalicFont={Roboto Light Italic},
  ]{Roboto}
\else\ifbooklet
  \setmainfont[
    ItalicFont={Roboto Light Italic},
  ]{Roboto}
\else
\setmainfont[
  ItalicFont={Roboto Italic},
]{Roboto Light}
\fi\fi
\renewcommand{\LettrineFontHook}{\bfseries\color{rubric}}
% \renewenvironment{labelblock}{\begin{center}\bfseries\color{rubric}}{\end{center}}

%%%%%%%%
% MISC %
%%%%%%%%

\setdefaultlanguage[frenchpart=false]{french} % bug on part


\newenvironment{quotebar}{%
    \def\FrameCommand{{\color{rubric!10!}\vrule width 0.5em} \hspace{0.9em}}%
    \def\OuterFrameSep{\itemsep} % séparateur vertical
    \MakeFramed {\advance\hsize-\width \FrameRestore}
  }%
  {%
    \endMakeFramed
  }
\renewenvironment{quoteblock}% may be used for ornaments
  {%
    \savenotes
    \setstretch{0.9}
    \normalfont
    \begin{quotebar}
  }
  {%
    \end{quotebar}
    \spewnotes
  }


\renewcommand{\headrulewidth}{\arrayrulewidth}
\renewcommand{\headrule}{{\color{rubric}\hrule}}

% delicate tuning, image has produce line-height problems in title on 2 lines
\titleformat{name=\chapter} % command
  [display] % shape
  {\vspace{1.5em}\centering} % format
  {} % label
  {0pt} % separator between n
  {}
[{\color{rubric}\huge\textbf{#1}}\bigskip] % after code
% \titlespacing{command}{left spacing}{before spacing}{after spacing}[right]
\titlespacing*{\chapter}{0pt}{-2em}{0pt}[0pt]

\titleformat{name=\section}
  [block]{}{}{}{}
  [\vbox{\color{rubric}\large\raggedleft\textbf{#1}}]
\titlespacing{\section}{0pt}{0pt plus 4pt minus 2pt}{\baselineskip}

\titleformat{name=\subsection}
  [block]
  {}
  {} % \thesection
  {} % separator \arrayrulewidth
  {}
[\vbox{\large\textbf{#1}}]
% \titlespacing{\subsection}{0pt}{0pt plus 4pt minus 2pt}{\baselineskip}

\ifaiv
  \fancypagestyle{main}{%
    \fancyhf{}
    \setlength{\headheight}{1.5em}
    \fancyhead{} % reset head
    \fancyfoot{} % reset foot
    \fancyhead[L]{\truncate{0.45\headwidth}{\fontrun\elbibl}} % book ref
    \fancyhead[R]{\truncate{0.45\headwidth}{ \fontrun\nouppercase\leftmark}} % Chapter title
    \fancyhead[C]{\thepage}
  }
  \fancypagestyle{plain}{% apply to chapter
    \fancyhf{}% clear all header and footer fields
    \setlength{\headheight}{1.5em}
    \fancyhead[L]{\truncate{0.9\headwidth}{\fontrun\elbibl}}
    \fancyhead[R]{\thepage}
  }
\else
  \fancypagestyle{main}{%
    \fancyhf{}
    \setlength{\headheight}{1.5em}
    \fancyhead{} % reset head
    \fancyfoot{} % reset foot
    \fancyhead[RE]{\truncate{0.9\headwidth}{\fontrun\elbibl}} % book ref
    \fancyhead[LO]{\truncate{0.9\headwidth}{\fontrun\nouppercase\leftmark}} % Chapter title, \nouppercase needed
    \fancyhead[RO,LE]{\thepage}
  }
  \fancypagestyle{plain}{% apply to chapter
    \fancyhf{}% clear all header and footer fields
    \setlength{\headheight}{1.5em}
    \fancyhead[L]{\truncate{0.9\headwidth}{\fontrun\elbibl}}
    \fancyhead[R]{\thepage}
  }
\fi

\ifav % a5 only
  \titleclass{\section}{top}
\fi

\newcommand\chapo{{%
  \vspace*{-3em}
  \centering % no vskip ()
  {\Large\addfontfeature{LetterSpace=25}\bfseries{\elauthor}}\par
  \smallskip
  {\large\eldate}\par
  \bigskip
  {\Large\selectfont{\eltitle}}\par
  \bigskip
  {\color{rubric}\hline\par}
  \bigskip
  {\Large TEXTE LIBRE À PARTICPATION LIBRE\par}
  \centerline{\small\color{rubric} {hurlus.fr, tiré le \today}}\par
  \bigskip
}}

\newcommand\cover{{%
  \thispagestyle{empty}
  \centering
  {\LARGE\bfseries{\elauthor}}\par
  \bigskip
  {\Large\eldate}\par
  \bigskip
  \bigskip
  {\LARGE\selectfont{\eltitle}}\par
  \vfill\null
  {\color{rubric}\setlength{\arrayrulewidth}{2pt}\hline\par}
  \vfill\null
  {\Large TEXTE LIBRE À PARTICPATION LIBRE\par}
  \centerline{{\href{https://hurlus.fr}{\dotuline{hurlus.fr}}, tiré le \today}}\par
}}

\begin{document}
\pagestyle{empty}
\ifbooklet{
  \cover\newpage
  \thispagestyle{empty}\hbox{}\newpage
  \cover\newpage\noindent Les voyages de la brochure\par
  \bigskip
  \begin{tabularx}{\textwidth}{l|X|X}
    \textbf{Date} & \textbf{Lieu}& \textbf{Nom/pseudo} \\ \hline
    \rule{0pt}{25cm} &  &   \\
  \end{tabularx}
  \newpage
  \addtocounter{page}{-4}
}\fi

\thispagestyle{empty}
\ifaiv
  \twocolumn[\chapo]
\else
  \chapo
\fi
{\it\elabstract}
\bigskip
\makeatletter\@starttoc{toc}\makeatother % toc without new page
\bigskip

\pagestyle{main} % after style

  \section[{Socialisme ou Barbarie ?}]{Socialisme ou Barbarie ?}\renewcommand{\leftmark}{Socialisme ou Barbarie ?}

\noindent La scène a changé fondamentalement. La marche des six semaines sur Paris a pris les proportions d’un drame mondial ; l’immense boucherie est devenue une affaire quotidienne, épuisante et monotone, sans que la solution, dans quelque sens que ce soit, ait progressé d’un pouce. La politique bourgeoise est coincée, prise à son propre piège : on ne peut plus se débarrasser des esprits que l’on a évoqués.\par
Finie l’ivresse. Fini le vacarme patriotique dans les rues, la chasse aux automobiles en or ; les faux télégrammes successifs ; on ne parle plus de fontaines contaminées par des bacilles du choléra, d’étudiants russes qui jettent des bombes sur tous les ponts de chemin de fer de Berlin, de Français survolant Nuremberg ; finis les débordements d’une foule qui flairait partout l’espion ; finie la cohue tumultueuse dans les cafés où l’on était assourdi de musique et de chants patriotiques par vagues entières ; la population de toute une ville changée en populace, prête à dénoncer n’importe qui, à molester les femmes, à crier : hourra ! et à atteindre au paroxysme du délire en lançant elle-même des rumeurs folles ; un climat de crime rituel, une atmosphère de pogrome, où le seul représentant de la dignité humaine était l’agent de police au coin de la rue.\par
Le spectacle est terminé. Les savants allemands, ces « lémures vacillants », sont depuis longtemps, au coup de sifflet, rentrés dans leur trou. L'allégresse bruyante des jeunes filles courant le long des convois ne fait plus d’escorte aux trains de réservistes et ces derniers ne saluent plus la foule en se penchant depuis les fenêtres de leur wagon, un sourire joyeux aux lèvres ; silencieux, leur carton sous le bras, ils trottinent dans les rues où une foule aux visages chagrinés vaque à ses occupations quotidiennes.\par
Dans l’atmosphère dégrisée de ces journées blêmes, c’est un tout autre chœur que l’on entend : le cri rauque des vautours et des hyènes sur le champ de bataille. Dix mille tentes garanties standard ! Cent mille kilos de lard, de poudre de cacao, d’ersatz de café, livrables immédiatement, contre payement comptant ! Des obus, des tours, des cartouchières, des annonces de mariage pour veuves de soldats tombés au front, des ceinturons de cuir, des intermédiaires qui vous procurent des contrats avec l’armée - on n’accepte que les offres sérieuses ! La chair à canon, embarquée en août et septembre toute gorgée de patriotisme, pourrit maintenant en Belgique, dans les Vosges, en Masurie, dans des cimetières où l’on voit les bénéfices de guerre pousser dru. Il s’agit d’engranger vite cette récolte. Sur l’océan de ces blés, des milliers de mains se tendent, avides de rafler leur part.\par
Les affaires fructifient sur des ruines. Des villes se métamorphosent en monceaux de décombres, des villages en cimetières, des régions entières en déserts, des populations entières en troupes de mendiants, des églises en écuries. Le droit des peuples, les traités, les alliances, les paroles les plus sacrées, l’autorité suprême, tout est mis en pièces. N'importe quel souverain par la grâce de Dieu traite son cousin, s’il est dans le camp adverse, d’imbécile, de coquin et de parjure, n’importe quel diplomate qualifie son collègue d’en face d’infâme fripouille, n’importe quel gouvernement assure que le gouvernement adverse mène son peuple à sa perte, chacun vouant l’autre au mépris public ; et des émeutes de la faim éclatent en Vénétie, à Lisbonne, à Moscou, à Singapour ; et la peste s’étend en Russie, la détresse et le désespoir, partout.\par
Souillée, déshonorée, pataugeant dans le sang, couverte de crasse ; voilà comment se présente la société bourgeoise, voilà ce qu’elle est. Ce n’est pas lorsque, bien léchée et bien honnête, elle se donne les dehors de la culture et de la philosophie, de la morale et de l’ordre, de la paix et du droit, c’est quand elle ressemble à une bête fauve, quand elle danse le sabbat de l’anarchie, quand elle souffle la peste sur la civilisation et l’humanité qu’elle se montre toute nue, telle qu’elle est vraiment.\par
Et au cœur de ce sabbat de sorcière s’est produit une catastrophe de portée mondiale : la capitulation de la social-démocratie internationale. Ce serait pour le prolétariat le comble de la folie que de se bercer d’illusions à ce sujet ou de voiler cette catastrophe : c’est le pire qui pourrait lui arriver. « Le démocrate » (c’est-à-dire le petit-bourgeois révolutionnaire) dit Marx, « sort de la défaite la plus honteuse aussi pur et innocent que lorsqu’il a commencé la lutte : avec la conviction toute récente qu’il doit vaincre, non pas qu’il s’apprête, lui et son parti, à réviser ses positions anciennes, mais au contraire parce qu’il attend des circonstances qu’elles évoluent en sa faveur. » Le prolétariat moderne, lui, se comporte tout autrement au sortir des grandes épreuves de l’histoire. Ses erreurs sont aussi gigantesques que ses tâches.  Il n’y a pas de schéma préalable, valable une fois pour toutes, pas de guide infaillible pour lui montrer le chemin à parcourir. Il n’a d’autre maître que l’expérience historique. Le chemin pénible de sa libération n’est pas pavé seulement de souffrances sans bornes, mais aussi d’erreurs innombrables. Son but, sa libération, il l’atteindra s’il sait s’instruire de ses propres erreurs. Pour le mouvement prolétarien, l’autocritique, une autocritique sans merci, cruelle, allant jusqu’au fond des choses, c’est l’air, la lumière sans lesquels il ne peut vivre.\par
Dans la guerre mondiale actuelle, le prolétariat est tombé plus bas que jamais. C'est là un malheur pour toute l’humanité. Mais c’en serait seulement fini du socialisme au cas où le prolétariat international se refuserait à mesurer la profondeur de sa chute et à en tirer les enseignements qu’elle comporte.\par
Ce qui est en cause actuellement, c’est tout le dernier chapitre de l’évolution du mouvement ouvrier moderne au cours de ces vingt-cinq dernières anses. Ce à quoi nous assistons, c’est à la critique et au bilan de l’œuvre accomplie depuis près d’un demi-siècle. La chute de la Commune de Paris avait scellé la première phase du mouvement ouvrier européen et la fin de la Ire Internationale. A partir de là commença une phase nouvelle. Aux révolutions spontanées, aux soulèvements, aux combats sur les barricades, après lesquels le prolétariat retombait chaque fois dans son état passif, se substitua alors la lutte quotidienne systématique, l’utilisation du parlementarisme bourgeois, l’organisation des masses, le mariage de la lutte économique et de la lutte politique, le mariage de l’idéal socialiste avec la défense opiniâtre des intérêts quotidiens immédiats. Pour la première fois, la cause du prolétariat et de son émancipation voyait briller devant elle une étoile pour la guider : une doctrine scientifique rigoureuse. A la place des sectes, des écoles, des utopies, des expériences que chacun faisait pour soi dans son propre pays, on avait un fondement théorique international, base commune qui faisait converger les différents pays en un faisceau unique. La théorie marxiste mit entre les mains de la classe ouvrière du monde entier une boussole qui lui permettait de trouver sa route dans le tourbillon des événements de chaque jour et d’orienter sa tactique de combat à chaque heure en direction du but final, immuable.\par
C'est le parti social-démocrate allemand qui se fit le représentant, le champion et le gardien de cette nouvelle méthode. La guerre de 1870 et la défaite de la Commune de Paris avaient déplacé vers l’Allemagne le centre de gravité du mouvement ouvrier européen. De même que la France avait été le lieu par excellence de la lutte de classe prolétarienne pendant cette première phase, de même que Paris avait été le cœur palpitant et saignant de la classe ouvrière européenne à cette époque, de même la classe ouvrière allemande devint l’avant-garde au cours de la deuxième phase. Au prix de sacrifices innombrables, par un travail minutieux et infatigable, elle a édifié une organisation exemplaire, la plus forte de toutes ; elle a créé la presse la plus nombreuse, donné naissance aux moyens de formation et d’éducation les plus efficaces, rassemblé autour d’elle les masses d’électeurs les plus considérables et obtenu le plus grand nombre de sièges de députés. La social-démocratie allemande passait pour l’incarnation la plus pure du socialisme marxiste. Le parti social-démocrate occupait et revendiquait une place d’exception en tant que maître et guide de la II° Internationale. En 1895, Friedrich Engels écrivit dans sa préface célèbre à l’ouvrage de Marx les luttes de classes en France :\par

\begin{quoteblock}
 \noindent « Mais, quoi qu’il arrive dans d’autres pays, la social-démocratie allemande a une position particulière et, de ce fait, du moins dans l’immédiat, aussi une tâche particulière. Les deux millions d’électeurs qu’elle envoie aux urnes, y compris les jeunes gens et les femmes qui sont derrière eux en qualité de non-électeurs, constituent la masse la plus nombreuse et la plus compacte, le " groupe de choc " décisif de l’armée prolétarienne internationale. »
\end{quoteblock}

\noindent La social-démocratie allemande était, comme l’écrivit la \emph{Wiener Arbeiterzeitung} le 5 août 1914 « le joyau de l’organisation du prolétariat conscient. » La social-démocratie française, italienne et belge, les mouvements ouvriers de Hollande, de Scandinavie, de Suisse et des États-Unis marchaient sur ses traces avec un zèle toujours croissant. Quant aux Slaves, les Russes et les sociaux-démocrates des Balkans, ils la regardaient avec une admiration sans bornes, pour ainsi dire inconditionnelle. Dans la II° Internationale, le « groupe de choc » allemand avait un rôle prépondérant. Pendant les congrès, au cours des sessions du bureau de l’Internationale socialiste, tout était suspendu à l’opinion des Allemands. En particulier lors des débats sur les problèmes posés par la lutte contre le militarisme et sur la question de la guerre, la position de la social-démocratie allemande était toujours déterminante. « Pour nous autres Allemands, ceci est inacceptable » suffisait régulièrement à décider de l’orientation de l’Internationale. Avec une confiance aveugle, celle-ci s’en remettait à la direction de la puissante social-démocratie allemande tant admirée : elle était l’orgueil de chaque socialiste et la terreur des classes dirigeantes dans tous les pays.\par
Et à quoi avons-nous assisté en Allemagne au moment de la grande épreuve historique ? A la chute la plus catastrophique, à l’effondrement le plus formidable. Nulle part l’organisation du prolétariat n’a été mise aussi totalement au service de l’impérialisme, nulle part l’état de siège n’est supporté avec aussi peu de résistance, nulle part la presse n’est autant bâillonnée, l’opinion publique autant étranglée, la lutte de classe économique et politique de la classe ouvrière aussi totalement abandonnée qu’en Allemagne.\par
Or, la social-démocratie allemande n’était pas seulement l’avant-garde la plus forte de l’Internationale, elle était aussi son cerveau. Aussi faut-il commencer par elle, par l’analyse de sa chute ; c’est par l’étude de son cas que doit commencer le procès d’autoréflexion. C'est pour elle une tâche d’honneur que de devancer tout le monde pour le salut du socialisme international, c’est-à-dire de procéder  la première à une autocritique impitoyable. Aucun autre parti, aucune autre classe de la société bourgeoise ne peut étaler ses propres fautes à la face du monde, ne peut montrer ses propres faiblesses dans le miroir clair de la critique, car ce miroir lui ferait voir en même temps les limites historiques qui se dressent devant elle et, derrière elle, son destin. La classe ouvrière, elle, ose hardiment regarder la vérité en face, même si cette vérité constitue pour elle l’accusation la plus dure, car sa faiblesse n’est qu’un errement et la loi impérieuse de l’histoire lui redonne la force, lui garantit sa victoire finale.\par
L'autocritique impitoyable n’est pas seulement pour la classe ouvrière un droit vital, c’est aussi pour elle le devoir suprême. Sur notre navire, nous transportions les trésors les plus précieux de l’humanité confiés à la garde du prolétariat, et tandis que la société bourgeoise, flétrie et déshonorée par l’orgie sanglante de la guerre, continue de se précipiter vers sa perte, il faut que le prolétariat international se reprenne, et il le fera, pour ramasser les trésors que, dans un moment de confusion et de faiblesse au milieu du tourbillon déchaîné de la guerre mondiale, il a laissé couler dans l’abîme.\par
Une chose est certaine, la guerre mondiale représente un tournant pour le monde. C'est une folie insensée de s’imaginer que nous n’avons qu’à laisser passer la guerre, comme le lièvre attend la fin de l’orage sous un buisson pour reprendre ensuite gaiement son petit train. La guerre mondiale a changé les conditions de notre lutte et nous a changés nous-mêmes radicalement. Non que les lois fondamentales de l’évolution capitaliste, le combat de vie et de mort entre le capital et le travail, doivent connaître une déviation ou un adoucissement. Maintenant déjà, au milieu de la guerre, les masques tombent et les vieux traits que nous connaissons si bien nous regardent en ricanant. Mais à la suite de l’éruption du volcan impérialiste, le rythme de l’évolution a reçu une impulsion si violente qu’à côté des conflits qui vont surgir au sein de la société et à côté de l’immensité des tâches qui attendent le prolétariat socialiste dans l’immédiat toute l’histoire du mouvement ouvrier semble n’avoir été jusqu’ici qu’une époque paradisiaque.\par
Historiquement, cette guerre était appelée à favoriser puissamment la cause du prolétariat. Chez Marx qui, avec un regard prophétique, a découvert au sein du futur tant d’événements historiques, on peut trouver dans les luttes de classes en France ce remarquable passage :\par

\begin{quoteblock}
 \noindent « En France, le petit bourgeois fait ce que, normalement, devrait faire le bourgeois industriel ; l’ouvrier fait ce qui, normalement, serait la tâche du petit-bourgeois ; et la tâche de l’ouvrier, qui l’accomplit ? Personne. On ne la résout pas en France, en France on la proclame. Elle n’est nulle part résolue dans les limites de la nation ; la guerre de classes au sein de la société française s’élargit en une guerre mondiale où les nations se trouvent face à face. La solution ne commence qu’au moment où, par la guerre mondiale, le prolétariat est mis à la tête du peuple qui domine le marché mondial, à la tête de l’Angleterre. La révolution, trouvant là non son terme, mais son commencement d’organisation, n’est pas une révolution au souffle court. La génération actuelle ressemble aux Juifs que Moïse conduit à travers le désert. Elle n’a pas seulement un nouveau monde à conquérir, il faut qu’elle périsse pour faire place aux hommes qui seront à la hauteur du nouveau monde. »
\end{quoteblock}

\noindent Ceci fut écrit en 1850, à une époque où l’Angleterre était le seul pays capitaliste développé, où le prolétariat anglais était le mieux organisé et semblait appelé à prendre la tête de la classe ouvrière internationale grâce à l’essor économique de son pays. Remplacez l’Angleterre par l’Allemagne et les paroles de Marx apparaissent comme une préfiguration géniale de la guerre mondiale actuelle. Cette guerre était appelée à mettre le prolétariat allemand à la tête du peuple et ainsi à produire un « début d’organisation » en vue du grand conflit général international entre le Capital et le Travail pour le pouvoir politique.\par
Et quant à nous, avons-nous présenté d’une façon différente le rôle de la classe ouvrière dans la guerre mondiale ? Rappelons-nous comment naguère encore nous décrivions l’avenir :\par

\begin{quoteblock}
 \noindent « Alors arrivera la catastrophe. Alors sonnera en Europe l’heure de la marche générale, qui conduira sur le champ de bataille de 16 à 18 millions d’hommes, la fleur des différentes nations, équipés des meilleurs instruments de mort et dressés les uns contre les autres. Mais, à mon avis, derrière la grande marche générale, il y a le grand chambardement. Ce n’est pas de notre faute : c’est de leur faute. Ils poussent les choses à leur comble. Ils vont provoquer une catastrophe. Ils récolteront ce qu’ils ont semé. Le crépuscule des dieux du monde bourgeois approche. Soyez-en sûrs, il approche ! »
\end{quoteblock}

\noindent Voilà ce que déclarait l’orateur de notre fraction, Bebel, au cours du débat sur le Maroc au Reichstag.\par
Le tract officiel du parti, \emph{Impérialisme ou Socialisme}, qui a été diffusé il y a quelques années à des centaines de milliers d’exemplaires, s’achevait sur ces mots :\par

\begin{quoteblock}
 \noindent « Ainsi la lutte contre le capitalisme se transforme de plus en plus en un combat décisif entre le Capital et le Travail. Danger de guerre, disette et capitalisme - ou paix, prospérité pour tous, socialisme ; voilà les termes de l’alternative. L'histoire va au-devant de grandes décisions. Le prolétariat doit inlassablement oeuvrer à sa tâche historique, renforcer la puissance de son organisation, la clarté de sa connaissance. Dès lors, quoi qu’il puisse arriver, soit que, par la  force qu’il représente, il réussisse à épargner à l’humanité le cauchemar abominable d’une guerre mondiale, soit que le monde capitaliste ne puisse périr et s’abîmer dans le gouffre de l’histoire que comme il en est né, c’est-à-dire dans le sang et la violence, à l’heure historique la classe ouvrière sera prête et le tout est d’être prêt.  »
\end{quoteblock}

\noindent Dans le Manuel pour les électeurs sociaux-démocrates de l’année 1911, destiné aux dernières élections parlementaires, on peut lire à la page 42, à propos de la guerre redoutée :\par

\begin{quoteblock}
 \noindent « Est-ce que nos dirigeants et nos classes dirigeantes croient pouvoir exiger de la part des peuples une pareille monstruosité ? Est-ce qu’un cri d’effroi, de colère et d’indignation ne va pas s’emparer d’eux et les amener à mettre fin à cet assassinat ? »\par
 « Ne vont-ils pas se demander : pour qui et pourquoi tout cela ? Sommes-nous donc des malades mentaux, pour être ainsi traités ou pour nous laisser traiter de la sorte ? »\par
 « Celui qui examine à tête reposée la possibilité d’une grande guerre européenne ne peut aboutir qu’à la conclusion que voici : »\par
 « La prochaine guerre européenne sera un jeu de va-tout sans précédent dans l’histoire du monde, ce sera selon toute probabilité la dernière guerre. »
\end{quoteblock}

\noindent C'est dans ce langage et en ces termes que nos actuels députés au Reichstag firent campagne pour leurs 110 mandats.\par
Lorsqu’en été 1911 le saut de panthère de l’impérialisme allemand sur Agadir et ses cris de sorcière eurent rendu imminent le péril d’une guerre européenne, une assemblée internationale réunie à Londres adopta le 4 août la résolution suivante :\par

\begin{quoteblock}
 \noindent « Les délégués allemands, espagnols, anglais, hollandais et français des organisations ouvrières se déclarent \emph{prêts à s’opposer avec tous les moyens dont ils disposent à toute déclaration de guerre}. Chaque nation représentée prend l’engagement \emph{d’agir} contre toutes les menées criminelles des classes dirigeantes, conformément aux décisions de son Congrès national et du Congrès international. »
\end{quoteblock}

\noindent Cependant, lorsqu’en novembre 1912 le Congrès international se réunit à Bâle, alors que le long cortège des délégués ouvriers arrivait à la cathédrale, tous ceux qui étaient présents furent saisis d’un frisson devant la solennité de l’heure fatale qui approchait et ils furent pénétrés d’un sentiment d’héroïque détermination.\par
Le froid et sceptique Victor Adler s’écriait :\par

\begin{quoteblock}
 \noindent « Camarades, il est capital que, nous retrouvant ici à la source commune de notre pouvoir, nous y puisions la force de faire ce que nous pouvons dans nos pays respectifs, selon les formes et les moyens dont nous disposons et avec tout le pouvoir que nous possédons, pour nous opposer au crime de la guerre. Et si cela devait s’accomplir, si cela devait réellement s’accomplir, \emph{alors nous devons tâcher que ce soit une pierre, une pierre de la fin}. »\par
 « Voilà le sentiment qui anime toute l’Internationale. »\par
 « Et si le meurtre et le feu et la pestilence se répandent à travers l’Europe civilisée - nous ne pouvons y penser qu’en frémissant et la révolte et l’indignation nous déchirent le cœur. \emph{Et nous nous demandons : les hommes, les prolétaires, sont-ils vraiment encore des moutons}, pour qu’ils puissent se laisser mener à l’abattoir sans broncher ?... »
\end{quoteblock}

\noindent \emph{Troelstra} prit la parole au nom des « petites nations » ainsi qu’au nom de la Belgique :\par

\begin{quoteblock}
 \noindent « Le prolétariat des petits pays se tient corps et âme à la disposition de l’Internationale pour tout ce qu’elle décidera en vue d’écarter la menace de guerre. Nous exprimons à nouveau l’espoir que, si un jour les classes dirigeantes des grands États appellent aux armes les fils de leur prolétariat pour assouvir la cupidité et le despotisme de leurs gouvernements dans le sang des petits peuples et sur leur sol - alors, grâce à l’influence puissante de leurs parents prolétaires et de la presse prolétarienne, les fils du prolétariat y regarderont à deux fois avant de nous faire du mal à nous, leurs amis et leurs frères, pour servir cette entreprise contraire à la civilisation. »
\end{quoteblock}

\noindent Et après avoir lu le manifeste contre la guerre au nom du bureau de l’Internationale, Jaurès conclut ainsi son discours :\par

\begin{quoteblock}
 \noindent « L'Internationale représente toutes les forces morales du monde ! Et si sonnait un jour l’heure tragique qui exige de nous que nous nous livrions tout entiers, cette idée nous soutiendrait et nous fortifierait. Ce n’est pas à la légère, mais bien \emph{du plus profond de notre être que nous déclarons : nous sommes prêts à tous les sacrifices} ! »
\end{quoteblock}

\noindent   C'était comme un serment de Rutli. Le monde entier avait les yeux fixés sur la cathédrale de Bâle, où les cloches sonnaient d’un air grave et solennel pour annoncer la grande bataille à venir entre l’armée du Travail et la puissance du Capital.\par
Le 3 décembre 1912, David, l’orateur du groupe social-démocrate, déclarait au Reichstag :\par

\begin{quoteblock}
 \noindent « Ce fut une des plus belles heures de ma vie, je l’avoue. Lorsque les cloches de la cathédrale accompagnèrent le cortège des sociaux-démocrates internationaux, lorsque les drapeaux rouges se disposaient dans le chœur de l’église autour de l’autel, et que le son de l’orgue saluait les délégués des peuples qui venaient proclamer la paix - j’en ai gardé une impression absolument inoubliable. ... \emph{Les masses cessent d’être des troupeaux dociles et abrutis}. C'est un élément nouveau dans l’histoire. Auparavant les peuples se laissaient aveuglément exciter les uns contre les autres par ceux qui avaient intérêt à la guerre, et se laissaient conduire au meurtre massif. Cette époque est révolue. \emph{Les masses se refusent désormais à être les instruments passifs et les satellites d’un intérêt de guerre, quel qu’il soit.}  »
\end{quoteblock}

\noindent Une semaine encore avant que la guerre n’éclate, le 26 juillet 1914, les journaux du parti allemand écrivaient :\par

\begin{quoteblock}
 \noindent « Nous ne sommes pas des marionnettes, nous combattons avec toute notre énergie un système qui fait des hommes des instruments passifs de circonstances qui agissent aveuglément, de ce capitalisme qui se prépare à transformer une Europe qui aspire à la paix en une boucherie fumante. Si ce processus de dégradation suit son cours, si la volonté de paix résolue du prolétariat allemand et international qui apparaîtra au cours des prochains jours dans de puissantes manifestations ne devait pas être en mesure de détourner la guerre mondiale, alors, \emph{qu’elle soit à moins la dernière guerre, qu’elle devienne le crépuscule des dieux du capitalisme}. » (Frankfurter Volksstimme.)
\end{quoteblock}

\noindent Le 30 juillet 1914, l’organe central de la social-démocratie allemande s’écriait :\par

\begin{quoteblock}
 \noindent « Le prolétariat socialiste allemand décline toute responsabilité pour les événements qu’une classe dirigeante aveuglée jusqu’à la démence est en train de provoquer. Il sait que \emph{pour lui une nouvelle vie s’élèvera des ruines}. Les \emph{responsables}, ce sont ceux qui \emph{aujourd’hui détiennent le pouvoir} ! »\par
 « Pour eux, il s’agit d’une question de \emph{vie ou de mort} ! »\par
 « \emph{L'histoire du monde est le tribunal du monde.} »
\end{quoteblock}

\noindent Et c’est alors que survint cet événement inouï, sans précédent : le 4 août 1914.\par
Cela devait-il arriver ainsi ? Un événement d’une telle portée n’est certes pas le fait du hasard. Il doit résulter de causes objectives profondes et étendues. Cependant ces causes peuvent résider aussi dans les erreurs de la social-démocratie qui était le guide du prolétariat, dans la faiblesse de notre volonté de lutte, de notre courage, de notre conviction. Le socialisme scientifique nous a appris à comprendre les lois objectives du développement historique. Les hommes ne font pas leur histoire de toutes pièces. Mais ils la font eux-mêmes. Le prolétariat dépend dans son action du degré de développement social de l’époque, mais l’évolution sociale ne se fait pas non plus en dehors du prolétariat, celui-ci est son impulsion et sa cause, tout autant que son produit et sa conséquence. Son action fait partie de l’histoire tout en contribuant à la déterminer. Et si nous pouvons aussi peu nous détacher de l’évolution historique que l’homme de son ombre, nous pouvons cependant bien l’accélérer ou la retarder.\par
Dans l’histoire, le socialisme est le premier mouvement populaire qui se fixe comme but, et qui soit chargé par l’histoire, de donner à l’action sociale des hommes un sens conscient, d’introduire dans l’histoire une pensée méthodique et, par là, une volonté libre. Voilà pourquoi Friedrich Engels dit que la victoire définitive du prolétariat socialiste constitue un bond qui fait passer l’humanité du règne animal au règne de la liberté. Mais ce « bond » lui-même n’est pas étranger aux lois d’airain de l’histoire, il est lié aux milliers d’échelons précédents de l’évolution, une évolution douloureuse et bien trop lente. Et ce bond ne saurait être accompli si, de l’ensemble des prémisses matérielles accumulées par l’évolution, ne jaillit pas l’étincelle de la volonté consciente de la grande masse populaire. La victoire du socialisme ne tombera pas du ciel comme fatum, cette victoire ne peut être remportée que grâce à une longue série d’affrontements entre les forces anciennes et les forces nouvelles, affrontements au cours desquels le prolétariat international fait son apprentissage sous la direction de la social-démocratie et tente de prendre en main son propre destin, de s’emparer du gouvernail de la vie sociale. Lui qui était le jouet passif de son histoire, il tente d’en devenir le pilote lucide.\par
Friedrich Engels a dit un jour : « \emph{La société bourgeoise est placée devant un dilemme : ou bien passage au socialisme ou rechute dans la barbarie.} » Mais que signifie donc une « rechute dans la barbarie » au degré de civilisation que nous connaissons en Europe aujourd’hui ? Jusqu’ici nous avons lu ces paroles sans y réfléchir et nous les avons répétées sans en pressentir la terrible gravité. Jetons un coup d’œil autour de nous en ce moment même, et nous comprendrons ce que signifie une rechute de la société  bourgeoise dans la barbarie. Le triomphe de l’impérialisme aboutit à l’anéantissement de la civilisation - sporadiquement pendant la durée d’une guerre moderne et définitivement si la période des guerres mondiales qui débute maintenant devait se poursuivre sans entraves jusque dans ses dernières conséquences. C'est exactement ce que Friedrich Engels avait prédit, une génération avant nous, voici quarante ans. Nous sommes placés aujourd’hui devant ce choix : ou bien triomphe de l’impérialisme et décadence de toute civilisation, avec pour conséquences, comme dans la Rome antique, le dépeuplement, la désolation, la dégénérescence, un grand cimetière ; ou bien victoire du socialisme, c’est-à-dire de la lutte consciente du prolétariat international contre l’impérialisme et contre sa méthode d’action : la guerre. C'est là un dilemme de l’histoire du monde, un ou bien - ou bien encore indécis dont les plateaux balancent devant la décision du prolétariat conscient. Le prolétariat doit jeter résolument dans la balance le glaive de son combat révolutionnaire : l’avenir de la civilisation et de l’humanité en dépendent. Au cours de cette guerre, l’impérialisme a remporté la victoire. En faisant peser de tout son poids le glaive sanglant de l’assassinat des peuples, il a fait pencher la balance du côté de l’abîme, de la désolation et de la honte. Tout ce fardeau de honte et de désolation ne sera contrebalancé que si, au milieu de la guerre, nous savons retirer de la guerre la leçon qu’elle contient, si le prolétariat parvient à se ressaisir et s’il cesse de jouer le rôle d’un esclave manipulé par les classes dirigeantes pour devenir le maître de son propre destin.\par
La classe ouvrière paie cher toute nouvelle prise de conscience de sa vocation historique. Le Golgotha de sa libération est pavé de terribles sacrifices. Les combattants des journées de Juin, les victimes de la Commune, les martyrs de la Révolution russe - quelle ronde sans fin de spectres sanglants ! Mais ces hommes-là sont tombés au champ d’honneur, ils sont, comme Marx l’écrivit à propos des héros de la Commune, « ensevelis à jamais dans le grand cœur de la classe ouvrière ». Maintenant, au contraire, des millions de prolétaires de tous les pays tombent au champ de la honte, du fratricide, de l’automutilation, avec aux lèvres leurs chants d’esclaves. Il a fallu que cela aussi ne nous soit pas épargné. Vraiment nous sommes pareils à ces Juifs que Moïse a conduits à travers le désert. Mais nous ne sommes pas perdus et nous vaincrons pourvu que nous n’ayons pas désappris d’apprendre. Et si jamais le guide actuel du prolétariat, la social-démocratie, ne savait plus apprendre, alors elle périrait « pour faire place aux hommes qui soient à la hauteur d’un monde nouveau ».
\section[{Devant le fait indéniable de la Guerre}]{Devant le fait indéniable de la Guerre}\renewcommand{\leftmark}{Devant le fait indéniable de la Guerre}


\begin{quoteblock}
 \noindent « Maintenant nous nous trouvons devant la réalité brutale de la guerre. Les affres d’une invasion ennemie nous menacent. Nous n’avons pas aujourd’hui à trancher pour ou contre la guerre, mais sur la question des moyens requis en vue de la défense du pays. La liberté future de notre peuple dépend pour beaucoup, sinon entièrement, d’une victoire du despotisme russe, qui s’est couvert du sang des meilleurs hommes de son propre peuple. Il s’agit d’écarter cette menace, de garantir la civilisation et l’indépendance de notre pays. Nous appliquons un principe sur lequel nous avons toujours insisté : à l’heure du danger, nous n’abandonnons pas notre propre patrie. Nous nous sentons par là en concordance de vues avec l’Internationale, qui a reconnu de tous temps le droit de chaque peuple à l’indépendance nationale et à l’autodéfense, de même que nous condamnons en accord avec elle toute guerre de conquête. Inspirés par ces principes, nous votons les crédits de guerre demandés. »
\end{quoteblock}

\noindent Par cette déclaration, le groupe parlementaire donnait le 4 août le mot d’ordre qui allait déterminer l’attitude des ouvriers allemands pendant la guerre. Patrie en danger, défense nationale, guerre populaire pour l’existence, la civilisation et la liberté - tels étaient les mots clés que proposait la représentation parlementaire de la social-démocratie.\par
Tout le reste en découla comme une simple conséquence : la position de la presse du parti et de la presse syndicale, le tumulte patriotique des masses, l’Union sacrée, la dissolution soudaine de l’Internationale, tout cela n’était que la conséquence inévitable de la première orientation qui fut adoptée au Reichstag.\par
Si réellement sont en jeu l’existence de la nation et la liberté, si celle-ci ne peut être défendue que par le fer meurtrier, si la guerre est la sainte cause du peuple, alors tout est clair et évident, alors il faut accepter en bloc. Qui veut le but doit vouloir les moyens. La guerre est un meurtre méthodique, organisé, gigantesque. En vue d’un meurtre systématique, chez des hommes normalement constitués, il faut cependant d’abord produire une ivresse appropriée. C'est depuis toujours la méthode habituelle des belligérants. La bestialité des pensées et des sentiments doit correspondre à la bestialité de la pratique, elle doit la préparer et l’accompagner. Dès lors, le \emph{Wahre Jakob} du 28 août avec l’image du « batteur » allemand, les feuilles du parti à Chemnitz, Hambourg, Kiel, Francfort et Cobourg, entre autres, avec leur excitation patriotique en vers et en prose, dispensèrent le narcotique spirituel dont le prolétariat avait besoin une fois qu’il ne pouvait plus sauvegarder son existence et sa liberté qu’en plongeant le fer meurtrier dans le sein de ses frères russes, français et anglais. Ces feuilles instigatrices sont donc plus logiques avec elles-mêmes que celles qui veulent réunir le jour et la nuit, concilier la guerre avec l’« humanité », le meurtre avec l’amour fraternel, l’approbation des moyens nécessaires à la guerre avec la fraternité socialiste des peuples.\par
  Mais si le mot d’ordre donné le 4 août par le groupe parlementaire était juste, alors serait prononcée contre l’Internationale ouvrière une condamnation sans appel, et qui ne vaudrait pas seulement pour cette guerre. Pour la première fois dans le mouvement ouvrier moderne, il y a une coupure entre les impératifs de la solidarité internationale des prolétaires et les intérêts de liberté et d’existence nationale des peuples, pour la première fois nous découvrons que l’indépendance et la liberté des nations exigent impérieusement que les prolétaires de pays différents se massacrent et s’exterminent les uns les autres. Jusqu’à présent, nous vivions avec la conviction que les intérêts des nations et les intérêts de classe du prolétariat concordaient harmonieusement, qu’ils étaient identiques, qu’on ne pouvait en aucun cas les opposer. C'était la base de notre théorie et de notre pratique, l’esprit qui animait notre agitation parmi les masses populaires. Etions-nous sur ce point essentiel de notre conception du monde, victimes d’une erreur monstrueuse ? Nous voilà devant la question vitale qui se pose au mouvement socialiste international.\par
La guerre mondiale n’est pas la première mise à l’épreuve de nos principes internationaux. Notre parti a subi sa première épreuve il y a quarante-cinq ans. A ce moment, le 21 juillet 1870, Wilhelm Liebknecht et August Bebel firent la déclaration suivante devant le parlement d’Allemagne du Nord :\par

\begin{quoteblock}
 \noindent « La guerre actuelle est une guerre dynastique, entreprise dans l’intérêt de la dynastie Bonaparte, de même que la guerre de 1866 fut entreprise dans l’intérêt de la dynastie Hohenzollern. »\par
 « Nous ne pouvons pas accepter les crédits que l’on exige du Reichstag pour la conduite de la guerre, parce que ce serait un vote de confiance au gouvernement prussien, lequel par la manière dont il a agi en 1866 a préparé la guerre actuelle. »\par
 « Mais nous pouvons tout aussi peu refuser les crédits demandés, car ce serait interprété comme une approbation de la politique insolente et criminelle de Bonaparte. «\par
 « En tant qu’ennemis par principe de toute guerre dynastique, en tant que sociaux-républicains et membres de l’Association internationale des travailleurs, qui lutte sans distinction de nationalité contre tous les oppresseurs et qui cherche à réunir tous les opprimés en une grande fraternité, nous ne pouvons nous déclarer ni directement ni indirectement pour la guerre actuelle, et nous nous abstenons donc de voter, en exprimant avec confiance l’espoir que les peuples d’Europe, instruits par les funestes événements actuels, mettront tout en oeuvre pour conquérir le droit à disposer d’eux-mêmes et pour éliminer la domination des armes et le pouvoir de classe, qui sont à l’origine de tout le mal politique et social. »
\end{quoteblock}

\noindent Par cette déclaration, les représentants du prolétariat allemand plaçaient clairement et sans ambages leur cause sous le signe de l’Internationale et ils refusaient carrément d’admettre que la guerre contre la France fût une guerre nationale au service de la liberté. On sait que Bebel affirme dans ses mémoires qu’il aurait voté contre l’approbation des emprunts si, au moment du vote, il avait eu connaissance de ce qu’on ne devait apprendre que dans les années qui suivirent.\par
Au cours de cette guerre, que l’opinion publique bourgeoise tout entière et l’énorme majorité du peuple influencée par les machinations de Bismarck considéraient alors comme l’intérêt vital de la nation allemande, les dirigeants de la social-démocratie soutenaient le point de vue suivant : les intérêts vitaux de la nation et les intérêts de classe du prolétariat international ne font qu’un, et tous les deux sont \emph{opposés} à la guerre. C'est seulement avec la guerre mondiale actuelle et avec la déclaration du groupe social-démocrate du 4 août 1914 qu’est apparu pour la première fois ce terrible dilemme : liberté nationale d’un côté, socialisme international de l’autre !\par
Le changement d’orientation dans les principes de la politique prolétarienne qui constituait le fait le plus important de la déclaration de notre groupe parlementaire fut donc, en tout état de cause, une illumination tout à fait soudaine. C'était une simple réplique de la version présentée le 4 août dans le discours du trône et dans celui du chancelier. « Nous ne sommes pas poussés par un désir de conquête, disait-on dans le discours du trône, nous sommes animés par la volonté inflexible de garder la place que Dieu nous a donnée, pour nous et pour toutes les générations à venir. Grâce aux documents qui vous sont communiqués, vous vous rendrez compte que mon gouvernement et avant tout mon chancelier se sont efforcés jusqu’au dernier moment d’éviter le pire. C'est en état de légitime défense, la conscience pure et les mains propres, que nous empoignons l’épée. » Et Bethmann-Hollweg déclara : « Messieurs, nous nous trouvons maintenant en état de légitime défense, et nécessité n’a point de loi. - Celui qui est menacé comme nous le sommes et qui combat pour son intérêt suprême ne doit songer qu’à la manière de se battre. - Nous combattons pour les fruits de notre travail pacifique, pour l’héritage de notre passé et pour notre avenir. » C'est exactement le contenu de la déclaration social-démocrate :\par
\textbf{1°} nous avons tout fait pour maintenir la paix, la guerre nous est imposée par d’autres ; \textbf{2°} maintenant que la guerre est là, nous devons nous défendre ; \textbf{3°} dans cette guerre tout est en jeu pour le peuple allemand. La déclaration de notre fraction parlementaire ne fait que répéter sous une forme un peu différente les déclarations du gouvernement. De même que celles-ci insistaient sur les tentatives diplomatiques de Bethmann-Hollweg pour maintenir la paix, et sur les télégrammes du Kaiser, le groupe rappelle les manifestations de paix organisées par la social-démocratie avant la déclaration de guerre. De  même que le discours du trône se défend de tout désir de conquête, le groupe refuse la guerre de conquête en se référant aux principes du socialisme. Et lorsque l’empereur et le chancelier s’écrient : nous combattons pour notre intérêt suprême ; je ne connais pas de partis, je ne connais que des Allemands - la déclaration social-démocrate répond en écho : pour notre peuple tout est en jeu, nous n’abandonnons pas notre patrie à l’heure du danger. Sur un point seulement, la déclaration social-démocrate s’écarte du schéma du gouvernement : elle place le despotisme russe au premier plan de son argumentation, comme mettant en danger la liberté de l’Allemagne. Dans le passage du discours du trône concernant la Russie, c’étaient des regrets : « Le cœur lourd, j’ai dû mobiliser mon armée contre un voisin avec lequel elle a combattu en commun sur tant de champs de bataille. C'est avec un chagrin sincère que je vois se briser une amitié loyalement respectée par l’Allemagne. » Le groupe social-démocrate a transposé la rupture douloureuse d’une amitié loyalement respectée avec le tsarisme russe en une fanfare de la liberté contre le despotisme, de sorte que sur le seul point où il se montre indépendant vis-à-vis de la déclaration gouvernementale, il se sert des traditions révolutionnaires du socialisme pour donner une caution démocratique à la guerre et pour lui forger un prestige populaire.\par
Comme nous l’avons dit plus haut, tout ceci apparut le 4 août comme l’effet d’une inspiration tout à fait soudaine. Tout ce que la social-démocratie avait dit jusqu’à ce jour, tout ce qu’elle avait dit la veille même du déclenchement de la guerre, tout cela s’opposait radicalement à cette déclaration. Ainsi le \emph{Vorwärts} écrivait-il, le 25 juillet, lorsque fut publié l’ultimatum autrichien à la Serbie qui provoqua la guerre :\par

\begin{quoteblock}
 \noindent « Ils veulent la guerre, les éléments sans scrupules qui font la pluie et le beau temps au palais de Vienne. Ils veulent la guerre - c’est ce qui ressort depuis des semaines des cris sauvages que fait entendre la presse fanatique jaune et noire. Ils veulent la guerre, l’ultimatum autrichien à la Serbie le montre clairement au monde entier. »\par
 « Parce que le sang de François-Ferdinand et de sa femme a coulé sous les coups d’un fanatique, il faut verser le sang de milliers d’ouvriers et de paysans, et un crime dément doit donner lieu à un crime bien plus dément encore ! ... L'ultimatum autrichien à la Serbie est peut-être l’allumette qui va mettre le feu aux quatre coins de l’Europe ! »\par
 « Car cet ultimatum est tellement exorbitant dans sa forme comme dans ses exigences que si le gouvernement serbe y cédait docilement, il devait s’attendre à être expulsé sur-le-champ par les masses populaires. »\par
 « C'était un crime de la part de la presse chauvine d’Allemagne, que de stimuler les désirs de guerre de son alliée fidèle jusqu’à la dernière extrémité, et sans aucun doute M. Bethmann-Hollweg a-t-il également assuré M. Berchtold de son soutien. Mais en procédant de la sorte, on joue un jeu tout aussi dangereux à Berlin qu’à Vienne... »
\end{quoteblock}

\noindent Le \emph{Leipziger Volkszeitung} écrivait le 24 juillet :\par

\begin{quoteblock}
 \noindent « [...] Le parti militaire autrichien mise tout sur une carte, puisque dans aucun pays au monde le chauvinisme national et militariste n’a rien à perdre [...]. En Autriche, les cercles chauvinistes sont en pleine faillite, leurs vociférations nationalistes doivent renflouer leur ruine économique et ils comptent sur la guerre pour remplir leurs caisses par le vol et le meurtre... »
\end{quoteblock}

\noindent Le même jour, le \emph{Dresdner Volkszeitung} s’exprimait ainsi :\par

\begin{quoteblock}
 \noindent « A l’heure qu’il est, les fauteurs de guerre de la salle de bal de Vienne nous doivent encore ces preuves décisives qui autoriseraient l’Autriche à poser des exigences à la Serbie. » \\
« Aussi longtemps que le gouvernement autrichien ne sera pas en mesure de les fournir, il se met dans son tort aux yeux de l’Europe entière en bousculant ainsi la Serbie de façon provocatrice et offensante, et même si les torts de la Serbie étaient prouvés, si l’attentat de Sarajevo avait bien été préparé sous les yeux du gouvernement serbe, les exigences contenues dans cette note dépasseraient toutes les limites. Seuls les desseins de guerre les plus frivoles peuvent expliquer qu’un gouvernement adresse de telles prétentions à un autre État. »
\end{quoteblock}

\noindent Voici le point de vue du \emph{Münchener Post} du 25 juillet :\par

\begin{quoteblock}
 \noindent « Cette note autrichienne est un document sans précédent dans l’histoire des deux siècles derniers. Sur la base d’un dossier d’enquête dont le contenu est resté caché jusqu’ici à l’opinion publique européenne, et sans justifier ses dires par un procès public contre les meurtriers du couple d’héritiers du trône, il pose à la Serbie des exigences dont l’acceptation équivaudrait à un suicide... »
\end{quoteblock}

\noindent Le \emph{Schleswig-Hollsteinische Volksteitung} déclarait le 24 juillet :\par

\begin{quoteblock}
 \noindent « L'Autriche provoque la Serbie, l’Autriche-Hongrie veut la guerre, elle commet un crime qui peut mettre l’Europe entière à feu et à sang... »\par
   « L'Autriche joue un jeu de va-tout. Elle ose adresser à l’Etat serbe une provocation à laquelle celui-ci ne peut consentir, à moins qu’il soit tout à fait sans résistance. »\par
 « Tout homme civilisé doit protester de la manière la plus énergique contre cette attitude criminelle des maîtres de l’Autriche ; ce doit être avant tout la tâche des ouvriers ainsi que de tous ceux qui gardent encore le moindre sens de la paix et de la civilisation, que de tenter le maximum pour enrayer la folie sanguinaire qui s’est déclarée à Vienne. »
\end{quoteblock}

\noindent Le \emph{Magdeburger Volkstimme} du 25 juillet déclarait :\par

\begin{quoteblock}
 \noindent « Tout gouvernement serbe qui ferait mine, si peu que ce soit, de prendre au sérieux l’une ou l’autre de ces exigences, serait chassé sur l’heure du Parlement par le peuple. »\par
 « Les agissements de l’Autriche sont d’autant plus répréhensibles que les Berchtold se présentent devant le gouvernement serbe et, de ce fait, devant l’Europe, avec des assertions qui ne reposent sur rien. »\par
 « On ne peut plus de nos jours machiner ainsi une guerre, qui deviendrait une guerre mondiale. On ne peut pas procéder ainsi, si on ne veut pas troubler la paix de tout un continent. Ce n’est pas comme cela qu’on peut faire des conquêtes morales ou qu’on peut persuader de son propre droit les États non engagés. Il est dès lors à supposer que la presse, et ensuite les gouvernements européens, vont rappeler à l’ordre ces hommes d’État viennois outrecuidants et insensés, d’une manière franche et énergique. »
\end{quoteblock}

\noindent Le \emph{Frankfurter Volksstimme} écrivait le 24 juillet :\par

\begin{quoteblock}
 \noindent « Poussés par les instigations de la presse ultramontaine qui pleurait son meilleur ami en François-Ferdinand et qui voulait venger sa mort sur le peuple serbe, et forts de l’appui d’une partie des fauteurs de guerre allemands, dont le langage est devenu de jour en jour plus menaçant et plus ignoble, les gouvernants autrichiens se sont laissés entraîner à adresser à l’État serbe un ultimatum qui non seulement est rédigé dans un langage d’une arrogance inouïe, mais contient certaines exigences que le gouvernement serbe ne peut absolument pas accepter. »
\end{quoteblock}

\noindent Le \emph{Elberfelder Freie Presse} écrivait le même jour :\par

\begin{quoteblock}
 \noindent « Un télégramme du Bureau officieux Wolf rapporte les exigences autrichiennes à l’égard de la Serbie. Il ressort de ce texte que les hommes au pouvoir à Vienne poussent à la guerre de toutes leurs forces, car ce qu’ils demandent dans la note remise hier soir à Belgrade n’est rien d’autre qu’une sorte de protectorat autrichien sur la Serbie. Il serait d’une nécessité urgente que la diplomatie de Berlin fasse entendre aux provocateurs viennois qu’elle ne peut lever le petit doigt pour soutenir des exigences d’une telle outrecuidance et que dès lors un retrait des prétentions autrichiennes s’impose. »
\end{quoteblock}

\noindent Et le \emph{Bergische Arbeiter Stimme} de Solingen :\par

\begin{quoteblock}
 \noindent « L'Autriche veut le conflit avec la Serbie, et utilise l’attentat de Sarajevo uniquement comme prétexte pour placer la Serbie dans son tort du point de vue moral. Mais l’affaire a quand même été trop grossièrement emmanchée pour que l’opinion publique européenne s’y laisse prendre... »\par
 « Néanmoins, au cas où les fauteurs de guerre de la salle de bal de Vienne croiraient que leurs alliés italiens et allemands devraient venir à leur aide dans un conflit où la Russie serait également entraînée, alors qu’ils abandonnent leurs vaines illusions. L'Italie verrait d’un oeil très favorable un affaiblissement de l’Autriche-Hongrie, qui est son concurrent sur l’Adriatique et dans les Balkans, et, de ce fait, elle ne se mouillera pas pour soutenir l’Autriche. En Allemagne, les gouvernants ne pourront pas - même s’ils sont assez fous pour le vouloir - oser risquer la vie d’un seul soldat pour soutenir la politique criminelle et autoritaire des Habsbourg sans déclencher contre eux la colère du peuple. »
\end{quoteblock}

\noindent Voilà comment notre presse tout entière, sans exception, jugeait la guerre une semaine encore avant son déclenchement. On le voit, il ne s’agissait pas pour elle de l’existence et de la liberté de l’Allemagne, mais de l’aventurisme criminel du parti de la guerre autrichien ; non pas de légitime défense, de défense nationale et d’une guerre qu’on est contraint de faire au nom de la liberté, mais d’une provocation frivole, d’une menace éhontée visant l’indépendance et la liberté d’un pays étranger, la Serbie.\par
Qu'est-ce qui a bien pu se passer le 4 août pour qu’une conception si nettement marquée, et aussi unanimement répandue, soit soudain bouleversée du tout au tout ? Un seul fait nouveau était intervenu : le Livre Blanc présenté le même jour au Reichstag par le gouvernement allemand. Et il contenait à la page 4 le passage suivant :\par

\begin{quoteblock}
 \noindent   « Dans ces conditions, l’Autriche devait se dire qu’il serait incompatible avec la dignité et la sauvegarde de la monarchie de tolérer plus longtemps sans agir cette agitation de l’autre côté de la frontière. Le gouvernement royal et impérial nous fit connaître son point de vue et nous demanda ce que nous en pensions. C'est de tout cœur que nous pûmes donner à notre allié notre accord quant à son appréciation de la situation et l’assurer que toute action qu’il jugerait nécessaire en vue de mettre fin en Serbie à un mouvement dirigé contre l’existence de la monarchie rencontrerait notre approbation. En disant cela, nous étions tout à fait conscients qu’une manœuvre de guerre éventuelle de l’Autriche-Hongrie contre la Serbie amènerait la Russie à intervenir, ce qui, conformément à notre devoir d’allié, pourrait dès lors nous impliquer dans la guerre. Connaissant les intérêts vitaux qui étaient en jeu pour l’Autriche-Hongrie, nous ne pouvions cependant pas conseiller à notre allié une modération qui aurait été incompatible avec sa dignité, ni lui refuser notre soutien dans un moment aussi difficile. Nous le pouvions d’autant moins que le travail de sape persistant des Serbes menaçait sensiblement nos propres intérêts. Si on avait toléré plus longtemps que les Serbes mettent en danger l’existence de la monarchie voisine avec l’aide de la Russie et de la France, cela aurait eu pour conséquence l’effondrement progressif de l’Autriche et l’assujettissement de tous les peuples slaves au sceptre russe, ce qui rendrait intolérable la position de la race germanique en Europe centrale. Une Autriche moralement affaiblie et qui s’effondrerait sous la poussée du panslavisme russe ne serait plus pour nous un allié sur lequel nous puissions compter et auquel nous puissions nous fier, ce qui est pour nous une nécessité en raison de l’attitude de plus en plus menaçante de nos voisins de l’Est et de l’Ouest. Nous laissions donc l’Autriche entièrement libre d’agir comme elle l’entend contre la Serbie. Nous ne sommes pour rien dans ce qui a préparé cette situation. »
\end{quoteblock}

\noindent Avec ce texte, qui constitue le seul passage important et décisif de tout le Livre Blanc, le groupe parlementaire social-démocrate avait sous les yeux les explications précises du gouvernement allemand, à côté desquelles tout autre livre, qu’il soit jaune, gris, bleu ou orange, est absolument dépourvu d’intérêt pour éclairer les antécédents diplomatiques et les causes immédiates de la guerre. Il tenait là la clé qui lui aurait permis de juger de la situation. Une semaine plus tôt, toute la presse social-démocrate s’écriait que l’ultimatum était une provocation criminelle et espérait que le gouvernement allemand agirait de manière à entraver l’action des fauteurs de guerre viennois et à modérer leur ardeur. La social-démocratie, ainsi que l’opinion publique allemande tout entière, était persuadée que le gouvernement allemand, depuis l’ultimatum autrichien, suait sang et eau pour maintenir la paix en Europe. La presse social-démocrate unanime supposait que le gouvernement avait été aussi surpris par cet ultimatum que l’avait été l’opinion publique allemande, chez qui il avait fait l’effet d’une bombe. Or, le Livre Blanc déclarait noir sur blanc : 1° que le gouvernement autrichien avait demandé l’accord de l’Allemagne avant de s’engager dans une démarche à l’égard de la Serbie ; 2° que le gouvernement allemand était pleinement conscient de ce que l’attitude de l’Autriche conduirait à une guerre avec la Serbie, et, dans un deuxième temps, à une guerre européenne ; 3° que le gouvernement allemand n’avait pas exhorté l’Autriche à la modération, mais qu’il déclarait au contraire qu’une Autriche conciliante et affaiblie ne serait plus un allié valable pour l’Allemagne ; 4° que, avant que l’Autriche n’agisse contre la Serbie, le gouvernement allemand l’avait, quoi qu’il arrive, fermement assurée de son assistance en cas de guerre, et enfin 5° que, malgré l’importance de l’enjeu, le gouvernement allemand n’avait pas gardé le contrôle sur l’ultimatum décisif adressé à la Serbie, mais qu’au contraire il « avait laissé entière liberté » à l’ Autriche.\par
Notre groupe parlementaire apprit tout cela le 4 août. Et, le même jour, il apprit encore un fait nouveau, de la bouche même du gouvernement : que les armées allemandes avaient déjà pénétré en Belgique. Disposant de tous ces éléments, le groupe social-démocrate en conclut qu’il s’agissait d’une guerre défensive de l’Allemagne contre une invasion étrangère, qu’il y allait de l’existence de la patrie et de la civilisation et que c’était une guerre libératrice contre le despotisme russe.\par
Est-ce que l’arrière-plan évident de la guerre et le décor qui le dissimulait péniblement, est-ce que tout le jeu diplomatique qui voilait la déclaration de guerre, les hauts cris pour ce monde d’ennemis qui attentaient à la vie de l’Allemagne, qui voulaient l’affaiblir, l’abaisser, la soumettre, tout cela pouvait-il être une surprise pour la social-démocratie allemande, était-ce trop demander à sa faculté de jugement, à son sens critique aiguisé ? Certainement pas ! Notre parti avait déjà eu l’expérience de deux grandes guerres allemandes et de chacune de ces guerres on peut tirer une leçon mémorable.\par
Même s’il ne connaît rien à l’histoire, chacun sait aujourd’hui que la première guerre de 1866 contre l’Autriche avait été préparée méthodiquement et de longue main par Bismarck, et que sa politique conduisait dès le premier instant au déclenchement de la guerre avec l’Autriche. Le prince héritier Frédéric, qui devint plus tard empereur, avait rapporté lui-même dans son journal à la date du 14 novembre de cette année ce point de vue du chancelier :\par

\begin{quoteblock}
 \noindent « En entrant en fonctions, il (Bismarck) aurait pris la ferme résolution d’amener la Prusse en guerre avec l’Autriche, mais il se serait bien gardé d’en parler alors avec Sa Majesté, il ne voulait pas en parler prématurément, avant qu’il ait jugé le moment opportun. »
\end{quoteblock}

\noindent   « Que l’on compare cet aveu - dit Auer dans sa brochure les Fêtes de Sedan et la social-démocratie - avec les termes de l’appel que le roi Guillaume lançait à son peuple :\par

\begin{quoteblock}
 \noindent « La Patrie est en danger ! »\par
 « L'Autriche et une partie considérable de l’Allemagne se dressent en armes contre elle ! »\par
 « Il y a quelques années à peine, spontanément et en oubliant tous les torts du passé, j’ai tendu la main de l’alliance à l’empereur d’Autriche, comme il le fallait pour libérer un pays allemand de la domination étrangère. - - Mais mon espoir a été déçu. L'Autriche ne veut pas oublier que ses princes ont régné jadis sur l’Allemagne : dans la Prusse, pays plus jeune qu’elle, mais qui se développe fortement, elle se refuse à voir son allié naturel ; elle ne veut voir en elle qu’un rival et un ennemi. Elle estime qu’il faut combattre toutes les aspirations de la Prusse parce que ce qui profite à la Prusse nuit à l’Autriche. La vieille jalousie funeste s’est rallumée et brille de tous ses feux ; la Prusse doit être affaiblie, anéantie, déshonorée. A son égard, aucun traité ne compte plus ; on ne dresse pas seulement les princes allemands contre la Prusse, on les pousse à rompre leur alliance avec elle. En Allemagne, nous avons des ennemis qui nous entourent de toutes parts, et leur cri de guerre à tous est : humilier la Prusse. »\par
 « Pour implorer la bénédiction du ciel sur cette guerre juste, le roi Guillaume décréta que le 18 juin serait un jour de prière et de repentir dans tout le pays. A cette occasion, il déclara : " Il n’a pas plus à Dieu de couronner mes efforts de succès ni de réaliser les souhaits de paix de mon peuple. " »
\end{quoteblock}

\noindent Si notre groupe n’avait pas complètement oublié l’histoire de son propre parti, est-ce que la fanfare officielle qui accompagnait la déclaration de guerre ne devait pas lui apparaître comme une réminiscence de certains airs et de certaines paroles qu’il connaît depuis bien longtemps ?\par
Mais ce n’est pas tout. Il y eut ensuite la guerre de 1870 avec la France. Et il est un document qui, dans l’histoire, reste inséparablement associé à son déclenchement : c’est la \emph{dépêche d’Ems}. Ce document est devenu le symbole de la politique bourgeoise en matière de « fabrication de guerre », et il représente également un épisode mémorable de l’histoire de notre parti. En effet, en la personne du vieux Liebknecht, la social-démocratie considéra à cette époque comme sa tâche et son devoir de dévoiler « comment les guerres sont fabriquées » et de le montrer aux masses populaires.\par
Ce n’est d’ailleurs pas Bismarck qui inventa ce moyen de fabriquer une guerre uniquement en la camouflant en une « défense de la patrie menacée ». Il ne faisait qu’appliquer, avec l’absence de scrupules qui lui était propre, une vieille recette de la politique bourgeoise, largement répandue et valant pour tous les pays.\par
Car, depuis que l’opinion dite publique joue un rôle dans les calculs des gouvernements, y a-t-il jamais eu une guerre où chaque parti belligérant n’ait pas tiré l’épée du fourreau d’un cœur lourd, uniquement pour la défense de la patrie et de sa propre cause juste, devant l’invasion indigne de son adversaire ? Cette légende appartient tout autant à l’art de la guerre que la poudre et le plomb. Le jeu est ancien. Le seul élément nouveau, c’est qu’un parti social-démocrate ait pris part à ce jeu.
\section[{Le développement de l’Impérialisme}]{Le développement de l’Impérialisme}\renewcommand{\leftmark}{Le développement de l’Impérialisme}

\noindent Toutefois, une cohérence encore plus grande et une connaissance encore plus approfondie préparaient notre parti à discerner la nature véritable et les buts réels de cette guerre et à ne se laisser surprendre par elle à aucun égard. Les événements et les forces motrices qui ont conduit au 4 août n’étaient un secret pour personne. La guerre mondiale avait été préparée pendant des dizaines d’années, avec la publicité la plus large, au grand jour, pas à pas et heure par heure. Et si aujourd’hui plusieurs socialistes imputent avec colère la catastrophe à la « diplomatie secrète » qui aurait fomenté cette diablerie derrière les coulisses, c’est bien à tort qu’ils prêtent au pauvre coquin une puissance occulte qu’il n’a pas, tout comme le Botocudo qui fouette son fétiche en l’accusant de l’orage. Ceux qui « dirigeaient » les destinées de l’État n’étaient alors, comme toujours, que des pions manœuvrés sur l’échiquier de la société bourgeoise par des processus et des mouvements qui les dépassaient. Et si quelqu’un s’était efforcé pendant tout ce temps de comprendre lucidement ces processus et ces mouvements, et était capable de le faire, c’était bien la social-démocratie allemande.\par
Deux lignes de force de l’évolution historique la plus récente conduisent tout droit à la guerre actuelle. L'une part de la période de la constitution des « États nationaux », c’est-à-dire des États capitalistes modernes ; elle a pour point de départ la guerre de Bismarck contre la France. La guerre de 1870, qui, suite à l’annexion de l’Alsace-Lorraine, avait jeté la République française dans les bras de la Russie, provoqué la scission de l’Europe en deux camps ennemis et inauguré l’ère de la folle course aux armements, a apporté le premier brandon au brasier mondial actuel. Alors que les troupes de Bismarck se trouvaient encore en France, Marx écrivit au comité de Braunschweig :\par

\begin{quoteblock}
 \noindent   « Celui qui n’est pas complètement assourdi par le tapage de l’heure présente, et qui n’a pas intérêt à assourdir le peuple allemand, doit comprendre que la guerre de 1870 donnera naissance à une guerre entre la Russie et l’Allemagne aussi nécessairement que celle de 1866 a amené celle de 1870. Nécessairement et inéluctablement, sauf au cas improbable du déclenchement préalable d’une révolution en Russie. Si cette éventualité improbable ne se produit pas, alors la guerre entre l’Allemagne et la Russie doit dès maintenant être considérée comme un fait accompli. Que cette guerre soit utile ou nuisible, cela dépend entièrement de l’attitude actuelle des vainqueurs allemands. S'ils prennent l’Alsace et la Lorraine, la France combattra contre l’Allemagne aux côtés de la Russie. Il est superflu d’en indiquer les conséquences funestes. »
\end{quoteblock}

\noindent A l’époque, on se moqua de cette prophétie : le lien qui unissait la Prusse et la Russie semblait si solide qu’il paraissait insensé de penser un seul instant que la Russie autocratique pût s’allier avec la France républicaine. Ceux qui soutenaient cette conception étaient tout simplement considérés comme fous à lier. Et cependant, tout ce que Marx a prédit s’est réalisé point par point. « On reconnaît bien là - dit Auer dans ses Fêtes de Sedan - la politique social-démocrate, qui voit clairement ce qui est, à la différence de cette politique au jour le jour qui ne voit pas plus loin que le bout de son nez. »\par
Mais, bien entendu, cet enchaînement d’une guerre à l’autre ne signifie pas que ce serait l’idée d’une revanche à prendre sur la mainmise opérée par Bismarck qui, depuis 1870, aurait poussé la France avec une fatalité inéluctable à l’épreuve de force avec le Reich allemand, et que la guerre mondiale actuelle consisterait essentiellement en cette « revanche » tant proclamée pour l’Alsace-Lorraine. Ce sont les fauteurs de guerre allemands qui ont forgé la légende nationaliste commode d’une France sinistre et assoiffée de vengeance qui « ne pouvait pas oublier » sa défaite, tout comme les organes de presse dévoués à Bismarck racontaient en 1866 l’histoire de la princesse Autriche qui, détrônée, « ne pouvait pas oublier » le rang qu’elle occupait jadis avant que n’arrive la charmante Cendrillon prussienne. En réalité, la revanche de l’Alsace-Lorraine n’était plus qu’un hochet grotesque agité par quelques pitres patriotiques, et le lion de Belfort, une vieille bête d’armoiries.\par
Dans la politique française, l’annexion était depuis longtemps dépassée ; elle avait été remplacée par de nouvelles préoccupations, et ni le gouvernement ni aucun parti sérieux en France ne songeaient plus à une guerre territoriale avec l’Allemagne. Si l’héritage de Bismarck fut en définitive le premier pas vers la conflagration actuelle, c’est bien plus en ce sens que, d’une part, il a poussé l’Allemagne et la France, et par là toute l’Europe, sur la pente glissante de la course aux armements, et que, d’autre part, il a eu comme conséquence inévitable l’alliance de la France avec la Russie, et de l’Allemagne avec l’Autriche. Par là, on obtenait un renforcement extraordinaire du tsarisme russe en tant qu’élément déterminant de la politique européenne. Et c’est précisément à partir de cette époque que la Prusse-Allemagne et la République française se mettent systématiquement à faire assaut de courbettes pour obtenir les faveurs de la Russie. Ainsi, on obtenait l’association politique du Reich allemand avec l’Autriche-Hongrie, qui, comme le montrent les mots qui figurent dans le Livre Blanc, trouve son couronnement dans la « fraternité d’armes » de la guerre actuelle.\par
Ainsi, la guerre de 1870 a eu comme conséquences : en politique extérieure, d’amener le regroupement politique de l’Europe autour de l’axe formé par l’opposition franco-allemande ; et dans la vie des peuples européens, d’assurer la domination formelle du militarisme. Cette domination et ce regroupement ont cependant donné ensuite à l’évolution historique un tout autre contenu.\par
La deuxième ligne de force qui débouche sur la guerre actuelle et confirme avec tant d’éclat la prédiction de Marx découle d’un phénomène à caractère international que Marx n’a plus connu : le développement impérialiste de ces vingt-cinq dernières années.\par
L'essor du capitalisme qui s’est affirmé après la période de guerre des années 60 et 70 dans l’Europe reconstruite et qui, notamment après qu’eut été surmontée la longue dépression consécutive à la fièvre de spéculation et au krach de 1873, avait atteint un sommet sans précédent dans la haute conjoncture des années 90, cet essor inaugurait, comme on le sait, une nouvelle période d’effervescence pour les États européens : leur expansion à qui-mieux-mieux vers les pays et les zones du monde restées non capitalistes. Déjà, depuis les années 80, on assistait à une nouvelle ruée particulièrement violente vers les conquêtes coloniales. L'Angleterre s’empare de l’Égypte et se crée un empire colonial puissant en Afrique du Sud ; en Afrique du Nord, la France occupe Tunis et, en Asie orientale, elle occupe le Tonkin, l’Italie s’implante en Abyssinie, la Russie achève ses conquêtes en Asie centrale et pénètre en Mandchourie, l’Allemagne acquiert ses premières colonies en Afrique et dans le Pacifique et finalement les Etats-Unis entrent également dans la danse en acquerrant avec les Philippines des « intérêts » en Asie orientale. Ce dépecement de l’Afrique et de l’Asie déroule, à partir de la guerre sino-japonaise de 1895, une chaîne presque ininterrompue de guerres sanglantes, qui culmine dans la grande campagne de Chine et s’achève avec la guerre russo-japonaise de 1904.\par
Ces événements, qui se succédèrent coup sur coup, créèrent de nouveaux antagonismes en dehors de l’Europe : entre l’Italie et la France en Afrique du Nord, entre la France et l’Angleterre en Égypte, entre l’Angleterre et la Russie en Asie centrale, entre la Russie et le Japon en Asie orientale, entre le Japon et l’Angleterre en Chine, entre les États-Unis et le Japon dans l’océan Pacifique - une mer mouvante, un flux et  reflux d’oppositions aiguës et d’alliances passagères, de tensions et de détentes, au milieu de laquelle une guerre partielle menaçait d’éclater à intervalle régulier entre les puissances européennes, mais, chaque fois, était différée à nouveau. Dès lors, il était clair pour tout le monde :\par
\textbf{1°} Que cette guerre de tous les États capitalistes les uns contre les autres sur le dos des peuples d’Asie et d’Afrique, guerre qui restait étouffée mais qui couvait sourdement, devait conduire tôt ou tard à un règlement de comptes général, que le vent semé en Afrique et en Asie devait un jour s’abattre en retour sur l’Europe sous la forme d’une terrible tempête, d’autant plus que ce qui se passait en Asie et en Afrique avait comme contre-coup une intensification de la course aux armements en Europe.\par
\textbf{2°} Que la guerre mondiale éclaterait enfin aussitôt que les oppositions partielles et changeantes entre les États impérialistes trouveraient un axe central, une opposition forte et prépondérante autour de laquelle ils puissent se condenser temporairement. Cette situation se produisit lorsque l’impérialisme allemand fit son apparition.\par
L'avènement de l’impérialisme s’étant produit en Allemagne sur une période très courte, il peut y être observé en vase clos. L'essor sans équivalent de la grosse industrie et du commerce depuis la fondation du Reich a donné lieu ici dans les années 80 à deux formes particulièrement caractéristiques de l’accumulation du capital : le plus fort développement de cartels en Europe ainsi que la plus grosse formation et la plus grosse concentration bancaires dans le monde entier. C'est le développement des cartels qui a organisé l’industrie lourde, c’est-à-dire précisément la branche du capital qui est directement intéressée par les fournitures d’État, les armements militaires et les entreprises impérialistes (constructions de chemins de fer, exploitations de mines, etc.) et en a fait le facteur le plus influent à l’intérieur de l’État.\par
C'est la concentration bancaire qui a comprimé le capital financier en une puissance bien distincte, d’une énergie toujours plus grande et toujours plus tendue, une puissance qui régnait souverainement dans l’industrie, le commerce et le crédit, était prépondérante dans l’économie privée comme dans l’économie publique, douée d’un pouvoir d’expansion souple et illimité, toujours en quête de profit et de zones d’activité, une puissance impersonnelle de grande envergure, audacieuse et sans scrupules, d’emblée internationale, et qui, dans sa structure même, était taillée à la dimension du monde, futur théâtre de ses exploits.\par
Qu'on y ajoute le régime personnel le plus fort, le plus versatile dans ses initiatives politiques, et le parlementarisme le plus faible, incapable de toute opposition, qu’on y joigne en outre toutes les couches de la bourgeoisie réunies dans l’opposition la plus abrupte à la classe ouvrière, et abritées derrière le gouvernement, et l’on pouvait prévoir dès lors que ce jeune impérialisme, plein de force, qui n’était gêné par aucune entrave d’aucune sorte, et qui fit son apparition sur la scène mondiale avec des appétits monstrueux, alors que le monde était déjà pour ainsi dire partagé, devait devenir très rapidement le facteur imprévisible de l’agitation générale.\par
Cela apparut déjà dans le changement radical intervenu dans la politique militaire de l’Empire à la fin des années 90 avec les deux projets de loi sur la force navale, qui furent présentés coup sur coup en 1898 et 1899. Fait sans précédent, ils allaient doubler brusquement les effectifs de la flotte de guerre, et ils comportaient un plan énorme de construction navale calculé sur près de deux décennies. Cela ne représentait pas seulement une vaste réorganisation de la politique financière et de la politique commerciale du Reich (le tarif douanier de 1902 n’était que l’ombre qui suivait les deux lois sur la force navale), laquelle était le prolongement logique de la politique sociale et des rapports entre les classes et entre les partis à l’intérieur de la société ; les lois sur la force navale indiquaient avant tout un changement manifeste dans la politique extérieure du Reich, telle qu’elle avait prévalu depuis sa fondation. Alors que la politique de Bismarck reposait sur le principe que l’Empire était une puissance terrestre et devait le rester, la flotte allemande étant considérée tout au plus comme un accessoire superflu de la défense des côtes - le secrétaire d’État Hollmann déclarait lui-même en mars 1897 à la commission du budget du Reichstag : « Pour la défense des côtes, nous n’avons certes pas besoin d’une marine : les côtes se défendent très bien toutes seules » -, c’est un tout autre programme que l’on fixa : l’Allemagne devait devenir la première puissance sur terre et sur mer. De ce fait, on passait de la politique continentale de Bismarck à la politique mondiale, les armements étaient désormais destinés à l’attaque et non plus à la défense. Le langage des faits était si clair que l’on fournit même le commentaire nécessaire au Reichstag. Le 11 mars 1896, après le fameux discours du Kaiser à l’occasion du vingt-cinquième anniversaire de l’empire allemand, discours dans lequel le Kaiser avait développé le nouveau programme en guise d’avant-première au projet de loi, le leader du Zentrum, Lieber, parlait déjà de « plans navals illimités » contre lesquels il fallait protester vigoureusement. Un autre leader du Zentrum, Schadler, s’écria au Reichstag, le 23 mars 1898, à l’occasion du premier projet de loi sur la flotte de guerre : « Le peuple considère que nous ne pouvons pas être à la fois la première puissance sur terre et sur mer. Si tout à l’heure on me crie qu’on ne veut absolument pas de cela, je répondrai : oui, messieurs, vous en êtes au début, et à vrai dire un très copieux début. » Et lorsque vint le second projet de loi le même Schadler déclarait au Reichstag, le 8 février 1900, après avoir fait allusion à toutes les déclarations antérieures qui disaient qu’il ne fallait pas songer à de nouvelles lois sur la force navale : « [...] et aujourd’hui cette loi dérogatoire qui inaugure ni plus ni moins la création d’une flotte mondiale et l’établissement d’une politique mondiale, en doublant le volume de notre flotte au moyen d’un programme qui doit s’étendre sur près de deux décennies. » D'ailleurs le gouvernement lui-même exposa  ouvertement le programme politique qui correspondait à la nouvelle orientation : le 11 décembre 1899, von Bülow, alors secrétaire d’État pour les Affaires étrangères, déclarait à l’occasion de la présentation du second projet de loi sur la force navale : « Si les Anglais parlent d’une Greater Britain, si les Français parlent d’une Nouvelle France, si les Russes se tournent vers l’Asie, de notre côté nous avons la prétention de créer une Grösseres Deutschland... Si nous ne construisions pas une flotte qui soit capable de défendre notre commerce et nos compatriotes à l’étranger, nos missions et la sécurité de nos côtes, nous mettrions en danger les intérêts les plus vitaux du pays. Dans les siècles à venir, le peuple allemand sera le marteau ou l’enclume. » Si on retire les fleurs de rhétorique de la défense des côtes, des missions et du commerce, il reste ce programme lapidaire : pour une Plus Grande Allemagne, pour une politique du marteau à l’égard des autres peuples. Contre qui ces provocations étaient-elles dirigées en premier lieu ? Cela ne faisait pas le moindre doute : la nouvelle politique agressive de l’Allemagne devait faire d’elle le concurrent de la première puissance navale au monde : l’Angleterre. Et c’est bien ainsi qu’on l’a compris dans ce pays. La réforme navale et les déclarations d’intentions qui l’accompagnaient suscitèrent en Angleterre la plus vive inquiétude, une inquiétude qui ne s’est pas calmée depuis lors. En mars 1910, Lord Robert Cecil déclarait à nouveau à la Chambre basse au cours du débat sur la flotte navale que chacun se demandait quelle raison plausible l’Allemagne pouvait bien avoir de construire une flotte gigantesque, sinon l’intention de rivaliser avec l’Angleterre. La rivalité sur mer qui durait des deux côtés depuis quinze ans, et finalement la construction fébrile des dreadnoughts et super-dreadnoughts, c’était déjà la guerre entre l’Allemagne et l’Angleterre. Le projet de loi maritime du 11 décembre 1899 était une déclaration de guerre de l’Allemagne, dont l’Angleterre accusa réception le 4 août 1914.\par
Bien entendu, cette rivalité sur mer n’avait rien à voir avec une quelconque rivalité économique au sujet de la maîtrise du marché mondial. Le « monopole anglais » sur le marché mondial, qui étranglait prétendument le développement économique de l’Allemagne, sur lequel on raconte aujourd’hui même tant de balivernes, c’est encore une de ces légendes patriotiques, parmi lesquelles on trouve aussi cette croyance indéracinable à la « revanche » d’une France furibonde. Dès les années 90, pour le malheur des capitalistes anglais, ce monopole n’était déjà plus que de l’histoire ancienne. Le développement industriel de la France, de la Belgique, de l’Italie, de la Russie, de l’Inde, du Japon, mais surtout de l’Allemagne et des États-Unis, y avait mis fin depuis la première moitié du XIXe siècle jusqu’aux années 60. Au cours des dernières décennies du siècle, tous les pays, les uns après les autres, faisaient leur entrée sur le marché mondial à côté de l’Angleterre, et le capitalisme se développait régulièrement et au pas de charge en direction d’une économie capitaliste mondiale. Quant à la suprématie maritime de l’Angleterre qui, aujourd’hui encore, provoque tant d’inquiétude même chez certains sociaux-démocrates allemands, et dont la destruction semble à ces braves être d’une nécessité urgente pour la prospérité du socialisme international, cette suprématie maritime - conséquence de l’expansion de l’Empire britannique sur les cinq continents - troubla si peu le capitalisme allemand jusqu’à présent que, sous son joug, il grandit tout au contraire avec une rapidité inquiétante pour devenir un robuste gaillard aux joues pleines de santé. C'est précisément l’Angleterre et ses colonies qui ont servi de tremplin à l’essor du gros capitalisme allemand, tout comme inversement l’Allemagne a été le client principal et le plus indispensable de l’Empire britannique. Bien loin de se contrarier mutuellement, les développements respectifs du gros capital anglais et du gros capital allemand étaient faits pour s’entendre et étaient enchaînés l’un à l’autre dans une vaste division du travail, ce qui fut favorisé dans une large mesure par le libre-échange anglais. Le commerce allemand des marchandises et ses intérêts sur le marché mondial étaient donc absolument étrangers au changement de front dans la politique allemande et à la construction de la flotte.\par
Quant aux possessions coloniales de l’Allemagne, elles n’étaient pas davantage susceptibles d’amener un affrontement mondial périlleux et une concurrence maritime avec l’Angleterre. La défense des colonies allemandes ne nécessitait pas que l’Allemagne détienne la suprématie maritime, car, de par leur nature, quasi-personne ne les enviait à l’Allemagne, et surtout pas l’Angleterre. Et si maintenant, au cours de la guerre, l’Angleterre et le Japon s’en sont emparés, il ne faut y voir qu’une mesure courante consécutive à l’état de guerre, tout comme l’appétit de l’impérialisme allemand se précipite maintenant sur la Belgique sans que personne ait jamais proposé avant la guerre d’annexer la Belgique : on l’aurait pris pour un fou. Jamais on n’en serait venu à une guerre, sur terre ou sur mer, à propos de l’Afrique du Sud ou du Sud-Est, de la Terre de Guillaume ou du bassin du Tsing-Tau ; tout juste avant la guerre, il y avait même un accord tout prêt entre l’Angleterre et l’Allemagne en vue d’assurer un partage équitable des colonies portugaises entre ces deux puissances.\par
Le développement de la puissance maritime et le déploiement de la bannière de la politique mondiale du côté allemand laissaient donc présager de nouvelles et considérables incursions de l’impérialisme dans le monde. Avec cette flotte offensive de première qualité et avec les accroissements militaires qui, parallèlement à sa construction, se succédaient à une cadence accélérée, c’était un instrument de la politique future que l’on créait, politique dont la direction et les buts laissaient le champ libre à de multiples possibilités. La construction navale et l’armement militaires constituaient en eux-mêmes l’affaire la plus colossale de la grosse industrie allemande, et, en même temps, ils ouvraient des perspectives infinies au capital des cartels et des banques qui brûlait d’étendre ses opérations au monde entier. Par là était acquis le ralliement de tous les partis bourgeois au drapeau de l’impérialisme. L'exemple des nationaux-libéraux, troupe de choc de l’industrie lourde impérialiste, fut suivi par le Zentrum, lequel, en  acceptant en 1900 le projet de loi sur la force navale qu’il avait si hautement dénoncé parce qu’il inaugurait une politique mondiale, devenait définitivement un parti gouvernemental ; le parti libéral lui emboîta le pas à l’occasion du tarif douanier de famine qui faisait suite à la loi sur la flotte de guerre ; le parti des junkers fermait la marche, lui qui, d’adversaire farouche qu’il était de l’« épouvantable flotte » et de la construction du canal, était devenu un zélote et un parasite du militarisme maritime, du brigandage colonial et de la politique douanière qui leur était liée. Les élections parlementaires de 1907, appelées « élections de Hottentots », montrèrent à nu l’Allemagne bourgeoise tout entière, dans un paroxysme d’enthousiasme impérialiste, solidement réunie sous un seul drapeau, l’Allemagne de von Bülow, qui se sentait appelée à jouer le rôle de marteau du monde. Et ces élections, avec leur atmosphère de pogrom, un prélude à l’Allemagne du 4 août, étaient également une provocation qui visait non seulement la classe ouvrière allemande, mais tous les autres États capitalistes, un poing brandi vers aucun État en particulier, mais vers tous à la fois.
\section[{La Turquie}]{La Turquie}\renewcommand{\leftmark}{La Turquie}

\noindent La Turquie devint le terrain d’opération le plus important de l’impérialisme allemand ; il eut pour promoteurs dans ce pays la Deutsche Bank avec ses entreprises gigantesques en Asie qui se trouvent au centre de la politique allemande pour l’Orient. Au cours des années 50 et 60, c’est surtout le capitalisme anglais qui entretenait des relations économiques avec la Turquie d’Asie ; il achevait le chemin de fer de Smyrne et avait également affermé le premier tronçon de la ligne d’Anatolie jusqu’à Ismid. En 1888, le capital allemand fait son apparition : Abdul Hamid lui confie l’exploitation du tronçon construit par les Anglais et la construction du nouveau tronçon entre Ismid et Angora avec des embranchements vers Scutari, BrussaKonnia et Kaisarile. La Deutsche Bank obtient en 1899 la concession et l’exploitation d’un port avec installation à Haidar Pascha et la direction exclusive sur le commerce et les douanes dans le port. En 1901, le gouvernement turc lui confie la concession pour le grand chemin de fer de Bagdad au golfe Persique, et en 1907, la concession pour l’assèchement de la mer de Karaviran et l’irrigation de la plaine de Koma.\par
Cette « oeuvre civilisatrice » grandiose et pacifique avait un revers : la ruine grandiose et « pacifique » de la paysannerie de l’Asie mineure. Les frais nécessaires à ces entreprises colossales sont évidemment avancés par la Deutsche Bank selon un système de dette publique aux multiples ramifications ; l’État turc devient à tout jamais le débiteur de MM. Siemens, Gwinner, Helfferich, etc., comme c’était déjà le cas auparavant pour le capital anglais, français et autrichien. Ce débiteur ne devait pas seulement se mettre à pomper constamment d’énormes sommes hors des caisses de l’État pour payer les intérêts des emprunts, mais devait aussi produire une garantie pour les bénéfices bruts du chemin de fer ainsi construit. Les moyens de transport et les méthodes de placement les plus modernes se greffent ici sur une situation économique tout à fait retardataire, et qui reste essentiellement fondée sur l’économie naturelle, à savoir sur l’économie paysanne la plus primitive. Le trafic et les profits nécessaires pour le chemin de fer ne peuvent évidemment pas provenir du sol aride de cette économie qui, sucée sans scrupules jusqu’à la moelle par le despotisme oriental depuis des siècles, produit à peine quelques miettes pour la nourriture des paysans eux-mêmes et de quoi payer des impôts à l’État. En raison de la nature économique et culturelle du pays, le commerce des marchandises et le transport des personnes sont très peu développés et ne peuvent augmenter que très lentement. Afin de compenser ce qui manque pour former le profit capitaliste requis, l’État accorde donc annuellement une « garantie kilométrique » aux sociétés de chemin de fer. C'est selon ce système que les lignes de la Turquie européenne furent construites par le capitalisme autrichien et français, et il fut également appliqué pour les entreprises de la Deutsche Bank en Turquie d’Asie. En guise de gage et d’assurance que le supplément sera bien payé, le gouvernement turc a cédé aux représentants du capitalisme européen, le « conseil d’administration de la dette publique », la source principale des revenus de l’État turc : les dîmes de toute une série de provinces. De 1893 à 1910, le gouvernement turc a versé ainsi, pour la ligne d’Angora et le tronçon Eskischehir-Konia, par exemple, un « supplément » d’environ 90 millions de francs. Les « dîmes » mises en gage par l’État turc à ses créanciers européens sont les impôts paysans archaïques, en nature : en blé, en moutons, en soie, etc. Les dîmes ne sont pas perçues directement, mais par l’intermédiaire de fermiers, analogues aux fameux receveurs d’impôts de la France de l’Ancien Régime : l’État leur vend aux enchères, c’est-à-dire au plus offrant et contre paiement comptant, le revenu probable de l’impôt de chaque vilayet (province). Si la dîme d’un vilayet est acquise par des spéculateurs ou par un consortium, ceux-ci revendent la dîme de chaque sandjak (district) à d’autres spéculateurs, qui cèdent à leur tour leur part à toute une série de petits agents. Comme chacun veut couvrir ses frais et empocher le plus de bénéfice possible, la dîme grossit comme une avalanche à mesure qu’elle se rapproche du paysan. Si le fermier s’est trompé dans ses comptes, il cherche à se dédommager aux dépens du paysan. Celui-ci attend avec impatience, presque toujours endetté, le moment de pouvoir vendre sa récolte ; mais quand il a fauché ses blés, il doit souvent attendre des semaines pour les battre, avant que le fermier ne daigne prendre la part qui lui revient. Le fermier qui, généralement, est en même temps négociant en blés, profite de cette situation où le paysan craint que la moisson entière ne se gâte sur le champ, pour lui extorquer sa récolte au plus bas prix, et il  sait s’assurer l’aide des fonctionnaires et spécialement du muktar (gouverneur local) pour faire face aux plaintes éventuelles des mécontents. Et si on ne parvient pas à trouver un fermier, les dîmes sont touchées directement en nature par le gouvernement, sont emmagasinées et cédées aux capitalistes et servent de compensation à la dette. Voilà comment fonctionne le mécanisme interne de la « régénération économique de la Turquie » effectuée par l’œuvre civilisatrice du capital européen !\par
Ces opérations permettent d’atteindre deux résultats différents : d’une part, l’économie paysanne de l’Asie mineure devient l’objet d’un processus bien organisé de succion pour le plus grand bien du capital bancaire et industriel européen et, en l’occurrence, surtout du capital allemand. Ainsi augmentent les « sphères d’intérêt » de l’Allemagne en Turquie, ce qui fournit le point de départ à une « protection » politique de la Turquie. En même temps, l’appareil de succion nécessaire à l’exploitation économique de la paysannerie, à savoir le gouvernement turc, devient l’instrument obéissant, le vassal de la politique extérieure allemande. Depuis longtemps déjà, les finances, la politique fiscale et les dépenses de l’État turc étaient placées sous contrôle européen. L'influence allemande, elle, s’est emparée tout spécialement de la organisation militaire.\par
Tout cela fait apparaître que l’impérialisme allemand a intérêt à ce que la puissance de l’État turc soit renforcée, pour que son effondrement n’intervienne pas trop tôt. Une liquidation accélérée de la Turquie conduirait à son partage entre l’Angleterre, la Russie, l’Italie et la Grèce entre autres, et, de ce fait, cette base unique en son genre pour les grandes opérations du capital allemand devrait disparaître. En même temps, il en résulterait un surcroît de puissance extraordinaire pour la Russie et l’Angleterre tout comme pour les États méditerranéens. Il s’agissait donc pour l’impérialisme allemand de conserver l’appareil commode de l’Etat turc « indépendant » et l’« intégralité » de la Turquie, assez longtemps pour que le pays soit dévoré de l’intérieur par le capital allemand, comme cela s’était passé auparavant pour l’Égypte avec les Anglais et, récemment encore, pour le Maroc avec les Français, et qu’il tombe comme un fruit mûr dans les mains de l’Allemagne. Le célèbre porte-parole de l’impérialisme allemand, Paul Rohrbach, déclare par exemple, avec la plus grande franchise :\par

\begin{quoteblock}
 \noindent « Il est dans la nature des choses que la Turquie, entourée de tous côtés de voisins pleins de convoitises, cherche un appui auprès d’une puissance qui n’ait, autant que possible, aucun intérêt territorial en Orient. Cette puissance, c’est l’Allemagne. De notre côté, nous subirions un grand dommage si la Turquie venait à disparaître. Si la Russie et l’Angleterre sont les héritiers principaux des Turcs, il est clair que ces deux Etats en recevront un surcroît de puissance considérable. Mais au cas où la Turquie serait partagée de telle sorte qu’un morceau important de son territoire nous échoirait, cela représenterait aussi pour nous des difficultés sans fin, car la Russie, l’Angleterre et, d’une certaine manière également la France et l’Italie, sont des voisins des possessions actuelles de la Turquie et sont à même de prendre possession de leur part et de la défendre tant sur mer que sur terre. Quant à nous, par contre, nous nous trouvons en dehors de toute communication avec l’Orient [...] Une Asie Mineure ou une Mésopotamie allemandes : ce projet ne pourra devenir une réalité qu’à une condition : c’est que la Russie et, du même coup, la France, soient obligées de renoncer à leurs buts et à leurs idéaux actuels, c’est-à-dire qu’au préalable, l’issue de la guerre actuelle se soit décidée dans le sens des intérêts allemands. » (La Guerre et la politique allemande, p. 36.)
\end{quoteblock}

\noindent L'Allemagne, qui jura solennellement le 8 novembre 1898, à l’ombre du grand Saladin, de garantir et de protéger le monde musulman et le drapeau vert du Prophète, mit donc tout son zèle à renforcer pendant dix ans le régime du sultan sanglant Abdul Hamid, et après une courte période de disgrâce, elle poursuivit son oeuvre sous le régime des Jeunes Turcs. En dehors des affaires lucratives de la Deutsche Bank, la mission allemande s’occupa principalement de la réorganisation et de l’entraînement du militarisme turc. La modernisation de l’armée créait naturellement de nouvelles charges qui retombaient sur le dos des paysans turcs, mais elle promettait également de nouvelles affaires brillantes pour Krupp et pour la Deutsche Bank. En même temps, le militarisme turc se plaçait sous la dépendance du militarisme prusso-allemand et devenait le point d’appui de la politique allemande en Asie mineure.\par
Que la « régénération » de la Turquie entreprise par l’Allemagne n’était qu’une tentative de réanimation artificielle d’un cadavre, c’est ce qui apparaît à travers les péripéties de la révolution turque. Tout d’abord, lorsque l’élément idéologique était prédominant au sein des Jeunes Turcs, alors qu’ils concevaient des projets grandioses et se berçaient d’illusions en croyant pouvoir donner une nouvelle jeunesse à la Turquie par un véritable renouveau interne, leurs sympathies politiques se tournaient résolument vers l’Angleterre, en laquelle ils voyaient l’idéal de l’État libéral moderne, tandis que l’Allemagne, qui depuis des années était le protecteur officiel du régime sacré du vieux sultan, faisait figure d’ennemi des Jeunes Turcs. La révolution de 1908 semblait marquer la faillite de la politique orientale de l’Allemagne, et en général c’est ainsi qu’on l’interpréta ; il semblait que la révocation d’Abdul Hamid, c’était aussi la révocation de l’influence allemande. Cependant, une fois arrivés au pouvoir, les Jeunes Turcs démontrèrent progressivement leur incapacité complète à réaliser une réforme économique, sociale et nationale de grande envergure, leur caractère contre-révolutionnaire montrait de plus en plus le bout de l’oreille, et ils ne tardèrent pas à en revenir, tout naturellement, aux méthodes de domination ancestrales qui étaient celles d’Abdul Hamid : organiser périodiquement des bains de sang en dressant les uns contre  les autres les peuples vassaux, et exploiter la paysannerie sans ménagements, à la mode orientale, ces deux méthodes constituant les deux piliers de l’État. Du même coup, la « Jeune Turquie » eut à nouveau comme souci essentiel de conserver artificiellement ce régime de violence, et elle fut ainsi amenée dans le domaine de la politique extérieure à renouer avec les traditions d’Abdul Hamid, c’est-à-dire à en revenir à l’alliance avec l’Allemagne.\par
Compte tenu de la multiplicité des questions nationales qui écartelaient l’Etat turc : les questions arménienne, kurde, syrienne, arabe, grecque (naguère encore la question albanaise et la question macédonienne) ; de la naissance d’un capitalisme puissant et vigoureux dans les jeunes États balkaniques voisins ; et surtout de la désagrégation économique que le capitalisme international et la diplomatie internationale avaient produite depuis des années en Turquie, tout le monde, et en premier lieu la social-démocratie allemande, voyait bien qu’une régénération réelle de l’État turc était une opération vouée à l’échec. Déjà à l’occasion du grand soulèvement de Crète en 1896, avait eu lieu dans la presse du parti allemand une discussion approfondie de la question d’Orient qui conduisit à réviser le point de vue jadis défendu par Marx du temps de la guerre de Crimée et à rejeter définitivement l’idée d’« intégrité de la Turquie », en tant qu’héritage de la réaction européenne. Et c’était bien une idée typiquement prussienne que de penser qu’il suffisait d’un chemin de fer stratégique susceptible d’amener une mobilisation rapide et de quelques instructeurs militaires énergiques pour rendre viable une baraque aussi vermoulue que l’Etat turc \footnote{ \noindent Le 3 décembre 1912, après la première guerre balkanique, l’orateur du groupe social-démocrate s’exprimait en ces termes au Reichstag : « Hier, on a fait remarquer à cette même tribune que la politique orientale de l’Allemagne n’était pas responsable de l’effondrement de la Turquie, que ç'avait été une bonne politique. M. le chancelier de l’Empire croyait que nous avions rendu bien de bons services à la Turquie et M. Bassermann disait que nous avions amené la Turquie à accomplir des réformes judicieuses. Sur ce dernier point, je ne suis au courant de rien (hilarité chez les sociaux-démocrates) ; et derrière les bons services je voudrais mettre aussi un point d’interrogation. Pourquoi la Turquie s’est-elle effondrée ? Ce qui s’est effondré là-bas, c’est un régime de junkers semblable à celui que nous avons à l’est de l’Elbe. (" Très juste ! " - sur les bancs sociaux-démocrates. Rires à droite) L'effondrement de la Turquie est un phénomène parallèle à l’effondrement du régime des junkers mandchous en Chine. Pour les régimes de junkers ça a l’air d’aller de plus en plus mal partout. (Acclamations des sociaux-démocrates : "Tant mieux !" ) Ils ne répondent plus aux exigences du monde moderne. »\par
 « Je disais que la situation en Turquie ressemblait jusqu’à un certain point à celle que nous connaissons à l’est de l’Elbe. Les Turcs sont une caste dirigeante de conquérants, ils ne sont qu’une petite minorité. A côté des Turcs, on trouve encore les non-Turcs qui ont adopté la religion musulmane, mais les véritables Turcs d’origine ne sont qu’une petite minorité, une caste guerrière, une caste qui, comme en Prusse, s’est emparée de tous les postes clés, dans l’administration, dans la diplomatie et dans l’armée ; une caste qui, vis-à-vis des papans bulgares et serbes, a poursuivi la même politique seigneuriale que nos spahis à l’est de l’Elbe. (Hilarité.) Aussi longtemps que la Turquie avait une économie naturelle, cela allait encore ; car alors, un tel régime de seigneurs est encore supportable dans une certaine mesure, parce que le seigneur ne pousse pas encore tellement l’exploitation de ses manants ; s’il a de quoi bien manger, et de quoi bien vivre, il est content. Mais au moment où la Turquie, entrant en contact avec l’Europe, devint une économie monétaire moderne, l’oppression des junkers turcs devint de plus en plus insupportable. La classe paysanne fut pressée comme un citron, et une grande partie des paysans réduite à la mendicité ; beaucoup se firent brigands. Voilà ce que sont les Komitaschis ! (Rires à droite) Les junkers turcs n’ont pas seulement fait la guerre contre l’ennemi extérieur, non, en dessous de cette guerre contre l’ennemi extérieur une révolution paysanne s’est accomplie en Turquie. Voilà ce qui a rompu l’échine des Turcs et voilà ce qui a provoqué l’effondrement de leur régime de junkers ! Et on dit maintenant que le gouvernement allemand a rendu de bons services dans ce pays ! Mais les meilleurs services qu’il aurait pu rendre à la Turquie, et aussi au régime des junkers, il ne les a pas rendus ! Il aurait dû leur conseiller d’accomplir les réformes qu’ils étaient tenus de faire en vertu du protocole de Berlin : de libérer véritablement leurs paysans, comme la Bulgarie et la Serbie l’avaient fait. Mais comment une diplomatie allemande de junkers en aurait-elle été capable ? »\par
 « [...] Les instructions que M. von Marschall recevait de Berlin ne pouvaient en tout cas l’amener à rendre réellement de bons services aux Jeunes Turcs. Ce qu’elles leur ont apporté - je ne veux même pas parler des choses militaires -, c’est un certain état d’esprit qu’elles ont introduit dans le corps des officiers turcs : l’esprit de l’« officier de garde élégant » (hilarité chez, les sociaux-démocrates), un esprit qui s’est avéré si extraordinairement funeste pour l’armée turque dans le combat. On raconte notamment qu’on a trouvé des officiers morts qui portaient des chaussures laquées. La prétention de dominer la masse du peuple et la masse des soldats en toutes choses, cette morgue de l’officier, cette façon de commander de haut, tout cela a pourri à la racine les rapports de confiance au sein de l’armée turque, et, dès lors, on peut donc comprendre que cet état d’esprit ait contribué à provoquer la débâcle de l’armée turque. »\par
 « Messieurs, nos avis divergent sur le point de savoir qui est responsable de l’effondrement de la Turquie. La transmission d’un certain esprit prussien n’est pas responsable à lui seul de l’effondrement de la Turquie, bien sûr que non, mais il y a contribué, il l’a précipité. L'effondrement lui-même était dû fondamentalement à des causes économiques, comme je l’ai montré. »
}.\par
En été 1912 déjà, le régime des Jeunes Turcs devait faire place à la contre-révolution. Le premier acte de la « régénération turque » dans cette guerre fut, fait significatif, le coup d’État, l’abolition de la Constitution, c’est-à-dire aussi à cet égard le retour formel au régime d’Abdul Hamid.\par
Le militarisme turc, qui avait été formé par l’Allemagne, fit déjà lamentablement faillite au cours de la première guerre des Balkans. Et quant à la guerre actuelle qui a entraîné la Turquie dans son tourbillon sinistre en tant que « protégée » de l’Allemagne, elle devra, quelle que soit son issue et avec une fatalité inéluctable, poursuivre ou même accomplir définitivement la liquidation de l’Empire turc.\par
  La position de l’impérialisme allemand en Orient, c’est-à-dire, au premier chef, les intérêts de la Deutsche Bank, avait fait entrer l’Empire allemand en conflit avec tous les autres États, et tout d’abord avec l’Angleterre. Non seulement celle-ci avait dû laisser des entreprises anglaises céder la place à leurs rivales allemandes en Anatolie et en Mésopotamie, perdant ainsi de copieux bénéfices, ce dont elle s’accommoda finalement, mais surtout la construction de lignes stratégiques et le renforcement du militarisme turc sous l’influence de l’Allemagne se produisait à l’un des points les plus sensibles pour l’Angleterre sur la carte politique mondiale : à un croisement entre l’Asie centrale, la Perse et l’Inde, d’une part, et l’Égypte, d’autre part.\par

\begin{quoteblock}
 \noindent « L'Angleterre - écrivait Rohrbach dans son livre le Chemin de fer de Bagdad - ne peut être attaquée sur terre et être sévèrement touchée qu’à un seul endroit en dehors de l’Europe : en Égypte. En perdant l’Egypte, l’Angleterre ne perdrait pas seulement la maîtrise du canal de Suez et la communication avec l’Inde et l’Asie, mais elle perdrait vraisemblablement aussi ses possessions en Afrique centrale et orientale. La conquête de l’Égypte par une puissance musulmane comme la Turquie pourrait en outre susciter des réactions dangereuses dans les Indes chez les 60 millions de sujets musulmans de l’Angleterre, ainsi qu’en Afghanistan et en Perse. Mais la Turquie ne peut envisager de conquérir l’Égypte qu’à plusieurs conditions : qu’elle dispose d’un réseau de chemin de fer complet en Asie Mineure et en Syrie ; qu’après avoir prolongé la ligne d’Anatolie elle puisse parer à une attaque de l’Angleterre sur la Mésopotamie, qu’elle améliore son armée et augmente ses effectifs ; et que sa situation économique générale et ses finances fassent des progrès. »
\end{quoteblock}

\noindent Et dans son livre paru au début de la guerre, la Guerre et la politique allemande, il dit :\par

\begin{quoteblock}
 \noindent « Le chemin de fer de Bagdad était tout d’abord destiné à mettre les points stratégiques principaux de l’Empire turc en Asie Mineure en communication immédiate avec la Syrie et les provinces arrosées par l’Euphrate et le Tigre. Naturellement, il était à prévoir que cette ligne de chemin de fer, rattachée aux lignes de Syrie et d’Arabie, qui sont en partie à l’état de projet, en partie en chantier ou déjà achevées, permettrait d’amener des troupes turques, prêtes à intervenir, en direction de l’Égypte. Personne ne niera qu’en supposant une alliance entre l’Allemagne et la Turquie, et à plusieurs autres conditions qu’il serait encore moins simple de réaliser que cette alliance, le chemin de fer de Bagdad représenterait pour l’Allemagne une assurance-vie politique. »
\end{quoteblock}

\noindent Les porte-parole mi-officieux de l’impérialisme allemand exposaient donc ouvertement les projets et les intentions de celui-ci en Orient. Ils définissaient les grandes lignes de la politique allemande : une tendance agressive qui compromettrait gravement l’équilibre qui avait existé jusqu’alors dans la politique mondiale, et un fer de lance visiblement dirigé contre l’Angleterre. La politique orientale de l’Allemagne devenait ainsi la traduction dans les faits de la politique maritime inaugurée en 1899.\par
En même temps, en soutenant le principe de l’intégrité de la Turquie, l’Allemagne entrait en conflit avec les États balkaniques, dont l’histoire et l’essor intérieur s’identifiaient avec la liquidation de la Turquie d’Europe. Enfin, elle entra en conflit avec l’Italie, dont les appétits impérialistes étaient dirigés en premier lieu contre les possessions turques. A la conférence marocaine d’Algésiras de 1905, l’Italie se trouvait déjà aux côtés de l’Angleterre et de la France. Et six ans plus tard, l’expédition tripolitaine de l’Italie qui faisait suite à l’annexion de la Bosnie par l’Autriche, et qui donna le départ à la première guerre des Balkans, c’était déjà le défi de l’Italie, l’éclatement de la Triple Alliance et l’isolement de la politique allemande.\par
Quant à la deuxième direction des efforts d’expansion de l’Allemagne, c’est à l’ouest qu’elle se manifesta, dans l’affaire du Maroc. Nulle part ailleurs, l’éloignement par rapport à la politique de Bismarck ne fut aussi net. Comme on le sait, Bismarck favorisait délibérément les aspirations coloniales de la France à seule fin de la détourner des points chauds de la politique continentale et notamment de l’Alsace-Lorraine. La nouvelle orientation politique de l’Allemagne, tout au contraire, s’en prenait directement à l’expansion coloniale de la France. Mais il y avait de sensibles différences entre la situation au Maroc et la situation en Turquie d’Asie. Il existait très peu d’intérêts capitalistes allemands véritables au Maroc. Sans doute, au cours de la crise du Maroc, les impérialistes allemands firent-ils grand bruit autour des revendications de la firme capitaliste Mannesmann de Remscheid, qui avait prêté de l’argent au sultan du Maroc et reçu en échange des concessions minières, jusqu’à en faire une affaire d’« intérêt vital pour la patrie ». Mais du fait que chacun des deux groupes capitalistes concurrents au Maroc - aussi bien le groupe Mannesmann que la société Krupp-Schneider - présentaient un mélange tout à fait international d’entrepreneurs allemands, français et espagnols, on ne peut pas parler sérieusement et avec quelque succès d’une « sphère d’intérêts allemands ». D'autant plus symptomatiques étaient la résolution et l’énergie avec lesquels l’Empire allemand fit connaître tout à coup en 1905 sa prétention à collaborer au règlement de l’affaire du Maroc et protesta contre l’hégémonie française dans le pays. C'était le premier accrochage avec la France sur le plan de la politique mondiale. En 1895 encore, l’Allemagne était tombée sur le dos du Japon victorieux, aux côtés de la France et de la Chine, pour l’empêcher d’exploiter sa victoire sur la Chine à Chimonoseki. Cinq ans plus tard, elle entra, bras dessus bras dessous avec la France, dans la grande phalange internationale formée en vue de l’expédition de pillage contre la Chine. Et maintenant, au Maroc, on assistait à un changement  radical dans les relations franco-allemandes. Par deux fois, au cours des sept années que dura la crise du Maroc, on frôla de justesse une guerre entre la France et l’Allemagne. Il ne s’agissait plus cette fois d’une « revanche » pour une quelconque rivalité continentale entre les deux États. Ici c’était un tout autre conflit qui prenait naissance, et qui provenait de ce que l’impérialisme allemand chassait sur les terres de l’impérialisme français. En définitive, au terme de cette crise, l’Allemagne accepta de se contenter du territoire congolais, et reconnut par là qu’elle ne possédait pas d’intérêts à défendre au Maroc. Mais c’est précisément pourquoi l’escarmouche allemande au Maroc avait une signification politique lourde de conséquences. Du fait que ses buts et ses revendications exactes restaient indéterminés, la politique de l’Allemagne au Maroc trahissait ses appétits illimités : on la voyait tâtonnant à la recherche d’une proie. Cette politique était généralement considérée comme une déclaration de guerre impérialiste à la France. L'opposition entre les deux Etats apparaissait là en pleine lumière. Là-bas, un développement industriel lent, une population stagnante, un État de rentiers qui investit de préférence à l’étranger et qui est encombré d’un grand empire colonial dont il ne parvient qu’à grand-peine à maintenir la cohésion ; de ce côté-ci, un capitalisme jeune et puissant qui s’installe au premier rang, qui court le monde pour y faire la chasse aux colonies. Il n’était pas question pour l’impérialisme allemand d’envisager la conquête des colonies anglaises. Dès lors, sa fringale dévorante ne pouvait se tourner, en dehors de la Turquie d’Asie que vers les possessions françaises. Ces possessions permettaient également de faire miroiter devant l’Italie la possibilité d’un dédommagement aux dépens de la France, au cas où elle se sentirait lésée par les appétits de conquête de l’Allemagne dans les Balkans - et de la retenir ainsi au sein de la Triple Alliance en l’associant à une entreprise commune. Il est clair que les prétentions de l’Allemagne sur le Maroc devaient inquiéter l’impérialisme français au plus haut point, si l’on songe qu’une fois établie en n’importe quel point du Maroc, l’Allemagne aurait eu à tout instant la possibilité de mettre le feu aux quatre coins de l’Empire français d’Afrique du Nord en procédant à des livraisons d’armes, car la population de cette région vivait dans un état de guerre chronique contre les conquérants français. Et si l’on aboutit à un compromis, si l’Allemagne consentit finalement à renoncer à ses prétentions, on n’avait fait qu’écarter le danger immédiat alors que persistaient l’inquiétude générale de la France et l’antagonisme politique qui avait été ainsi créé \footnote{ \noindent La bruyante excitation entretenue depuis des années dans les milieux impérialistes allemands autour de la question du Maroc n’était pas faite pour calmer les appréhensions de la France. L'association pangermanique défendait tout haut le programme d’annexion du Maroc, qu’elle considérait naturellement comme une « question vitale » pour l’Allemagne, et elle diffusa un tract de la plume de son président Heinrich Clatz sous le titre : \emph{L'Ouest du Maroc allemand} ! Lorsque le professeur Schiemann chercha à justifier l’arrangement conclu par le département des Affaires étrangères et son renoncement au Maroc en invoquant les intérêts du commerce au Congo, le Post s’en prit à lui de la manière suivante :\par
 « M. le professeur Schiedmann est russe de naissance, et peut-être n’est-il pas même de pure descendance allemande. Dès lors, personne ne peut lui en vouloir s’il considère d’un air froid et moqueur des questions qui piquent au plus vif la conscience nationale et la fierté que tout Allemand authentique porte en lui. Le jugement d’un étranger qui parle de ce qui est le battement de cœur patriotique et la palpitation douloureuse de l’âme inquiète du peuple allemand comme s’il s’agissait d’une fantaisie politique passagère ou d’une aventure de conquistadores doit provoquer à juste titre notre colère et notre mépris, d’autant plus que cet étranger jouit de l’hospitalité de l’État prussien en tant que professeur à l’Université de Berlin. Que l’homme qui ose insulter ainsi les sentiments les plus sacrés du peuple allemand dans l’organe directeur du parti conservateur soit le maître et le conseiller de notre Kaiser en matière politique, et qu’il soit considéré, à tort ou à raison, comme son porte parole, cela nous remplit d’une profonde tristesse.  »
}.\par
La politique marocaine de l’Allemagne n’amenait pas seulement celle-ci en conflit avec la France, mais indirectement aussi avec l’Angleterre. Comme Gibraltar est le deuxième carrefour le plus important de la politique mondiale de l’Angleterre, l’arrivée soudaine de l’impérialisme allemand au Maroc, à proximité immédiate de Gibraltar, avec les prétentions qu’il manifestait et le style brutal de son action, devait apparaître aux Anglais comme une manifestation hostile à leur égard. Sur le plan formel également, la première note de protestation de l’Allemagne s’en prenait directement à l’arrangement intervenu en 1904 entre la France et l’Angleterre au sujet du Maroc et de l’Égypte, et les exigences allemandes tendaient nettement à éliminer l’Angleterre du règlement de l’affaire du Maroc. L'effet que cette prise de position devait inévitablement produire sur les rapports anglo-allemands ne pouvait être un secret pour personne. Le correspondant à Londres du Frankfurter Zeitung dépeint clairement la situation ainsi créée dans l’édition du 8 novembre 1911 :\par

\begin{quoteblock}
 \noindent « Voilà le bilan : au total, un million de nègres au Congo, un amer déboire contre la " perfide Albion ". L'Allemagne digérera son amertume. Mais qu’adviendrait-il de nos rapports avec l’Angleterre, qui ne peuvent absolument se poursuivre sans changements, mais qui, selon toute vraisemblance historique, doivent conduire soit à une aggravation, soit même à la guerre, ou bien doivent rapidement s’améliorer... L'expédition du Panther était, comme un correspondant berlinois du Frankfurter Zeitung l’exprimait récemment de façon frappante, une bourrade qui devait montrer à la France que l’Allemagne n’avait pas cessé d’exister... Quant à l’impression que cette estocade devait produire à Londres, il est impossible que l’on ait jamais pu en douter un seul instant à Berlin, et que l’on soit resté dans l’incertitude à ce sujet ; du moins aucun correspondant ici n’a douté que l’Angleterre ne se porte énergiquement aux côtés de la France. Comment le Norddeutsche Allgemeine Zeitung peut-il encore s’accrocher à ce cliché selon  lequel l’Allemagne aurait à discuter " uniquement avec la France" ! Depuis quelque cent ans, la politique européenne s’est développée de telle sorte que, de plus en plus, les intérêts politiques sont enchevêtrés les uns aux autres. Si un pays est dans une mauvaise passe, la nature des lois politiques dans lesquelles nous vivons veut que les uns se frottent les mains et que les autres se désolent. Lorsqu’il y a deux ans les Autrichiens eurent des démêlés avec la Russie à propos de la Bosnie, l’Allemagne entra dans la lice " en armes étincelantes ", quoiqu’à Vienne, comme on le déclara plus tard, on eût préféré régler l’affaire tout seul... Il n’est pas concevable que l’on ait pu croire à Berlin que les Anglais, qui venaient à peine de sortir d’une période de climat tout à fait hostile à l’Allemagne, auraient tout à coup été d’avis que nos pourparlers avec la France ne les concernaient en rien. Il s’agissait en dernier ressort d’une question de puissance, car une bourrade, même si elle peut paraître amicale, est une voie de fait, et personne ne peut dire si, peu de temps après, elle ne sera pas suivie d’un coup de poing sur la mâchoire. Depuis, la situation est devenue moins critique. Au moment où Lloyd George prit la parole, existait de manière aiguë, nous avons là-dessus des informations très précises, le danger d’une guerre entre l’Allemagne et l’Angleterre... Est-ce que - compte tenu de cette politique suivie depuis longtemps par sir Edward Grey et ses partisans, et dont nous ne discutons pas ici le bien-fondé - on devait s’attendre de leur part à une autre attitude sur la question du Maroc ? Il nous semble que si Berlin y a compté, c’est toute sa politique qui est du même coup condamnée. »
\end{quoteblock}

\noindent Ainsi, la politique impérialiste de l’Allemagne en Asie comme au Maroc avait créé un antagonisme violent entre l’Allemagne d’une part, l’Angleterre et la France de l’autre. Où en étaient les rapports entre l’Allemagne et la Russie ? Comment s’était produit l’affrontement dans ce cas-ci ? Dans l’atmosphère de pogrom qui s’était emparée de l’opinion publique allemande pendant les premiers mois de la guerre, on gobait n’importe quoi. On croyait que les femmes belges crevaient les yeux des blessés allemands, que les Cosaques mangeaient des bougies de stéarine et qu’ils empoignaient les nourrissons par leurs petites jambes pour les mettre en pièces - on croit aussi que les buts de la Russie en cette guerre consistent à annexer l’Empire allemand, à anéantir la civilisation allemande et à implanter l’absolutisme de la Warthe jusqu’au Rhin, et de Kiel à Munich.\par
L'organe social-démocrate Chemnitzer Volksstimme écrivait le 2 août :\par

\begin{quoteblock}
 \noindent « En ce moment nous ressentons tous le devoir de lutter contre le règne de knout russe avant tout. Les femmes et enfants allemands ne doivent pas devenir les victimes des bestialités russes, l’Allemagne ne sera pas le butin des Cosaques. Car si le Triple Alliance l’emporte, ce ne sera pas un gouverneur français ou un républicain français mais le tsar russe qui règnera sur l’Allemagne. C'est pourquoi nous défendons en ce moment toute la culture allemande et toute la liberté allemande contre un ennemi barbare qui ne connaît pas de merci. »
\end{quoteblock}

\noindent Le Fränkische Tagespost s’écriait le même jour :\par

\begin{quoteblock}
 \noindent « Nous ne voulons pas que les Cosaques, qui ont déjà occupé toutes les localités frontalières, fassent irruption dans notre pays et apportent la ruine dans nos villes. La social-démocratie n’a jamais cru aux intentions pacifiques du tsar russe, pas même le jour où il a publié son manifeste de paix ; nous ne voulons pas que ce tsar, qui est déjà le pire ennemi du peuple russe, commande à un peuple de race allemande. »
\end{quoteblock}

\noindent Et le Königsberger Volkszeitung du 3 août écrivait :\par

\begin{quoteblock}
 \noindent « Mais aucun de nous, qu’il soit astreint ou non au service militaire, ne peut en douter un seul instant : aussi longtemps que durera la guerre, le devoir de chacun est de faire tout ce qu’il peut pour maintenir loin de nos frontières cet odieux régime tsariste. Si le tsarisme remporte la victoire, des milliers de nos camarades seront envoyés dans les geôles horribles de la Russie. Sous le sceptre russe, le droit des peuples à disposer d’eux-mêmes est réduit à néant ; pas la moindre trace là-bas d’une presse social-démocrate ; les syndicats sociaux-démocrates et les réunions social-démocrates sont interdits. Et c’est pourquoi, à cette heure, aucun de nous n’aurait l’idée de se désintéresser de l’issue de la guerre ; au contraire, tout en maintenant notre opposition à la guerre, nous voulons agir tous ensemble pour nous garder nous-mêmes des atrocités de ces canailles qui gouvernent la Russie. »
\end{quoteblock}

\noindent Examinons de plus près les rapports de la civilisation allemande avec le tsarisme russe, qui forment un chapitre entier dans l’attitude de la social-démocratie au cours de cette guerre. Pour ce qui est du désir que le tsar aurait d’annexer l’Empire allemand, on pourrait tout aussi bien admettre que la Russie envisageait d’annexer l’Europe ou même la Lune. Dans la guerre actuelle, il ne s’agit d’une question d’existence, en tout et pour tout, que pour deux États : la Belgique et la Serbie. Et les canons allemands furent dirigés contre eux parce qu’on criait de tous côtés que l’existence de l’Allemagne était en jeu. Avec des fanatiques du meurtre rituel, toute discussion est évidemment exclue. Toutefois, les gens qui prennent en considération non les instincts de la populace et les grands mots démagogiques et sublimes de la presse nationaliste provocatrice, mais plutôt les points de vue politiques, ceux-là doivent comprendre que le  tsarisme pouvait se fixer comme but aussi bien l’annexion de la Lune que celle de l’Allemagne. Ce sont de franches crapules qui dirigent la politique russe, mais pas des fous, et la politique de l’absolutisme a de toute façon ceci en commun avec toute autre politique qu’elle se meut non dans les nuages, mais dans le monde des possibilités réelles, dans un espace où les choses entrent rudement en contact. En ce qui concerne la crainte de voir nos camarades allemands arrêtés et déportés à vie en Sibérie, et de voir l’absolutisme russe s’introduire dans l’Empire allemand, les hommes politiques du tsar sanglant, malgré leur infériorité intellectuelle, ont mieux compris le matérialisme historique que les journalistes de notre parti : ces politiciens savent très bien qu’une forme de gouvernement donnée ne se laisse pas « exporter » à volonté n’importe où, mais que chaque forme de gouvernement correspond à certaines conditions économiques et sociales bien précises : ils savent, pour en avoir fait l’amère expérience, que même en Russie les conditions de leur domination ont presque fait leur temps ; ils savent enfin que le règne de la réaction se sert dans chaque pays de la forme qui lui convient, toute autre forme lui étant intolérable, et que la variante de l’absolutisme qui correspond aux rapports entre les classes et les partis que connaît l’Allemagne, c’est l’État policier des Hohenzollern et le suffrage censitaire de la Prusse. En examinant froidement les choses, on voit qu’il n’existait de prime abord aucune raison de craindre que le tsarisme russe se serait vraiment senti obligé d’ébranler ces produits de la civilisation allemande, même dans le cas improbable de sa victoire totale.\par
En réalité, c’était sur un plan tout à fait différent que la Russie et l’Allemagne entrèrent en opposition. Ce n’est pas dans le domaine de la politique intérieure qu’ils s’affrontèrent, domaine où, au contraire, grâce à leurs tendances communes et à leur affinité intime, une amitié ancienne et traditionnelle s’était établie depuis un siècle entre les deux États, mais, en dépit de la solidarité de leur politique intérieure, dans le domaine de la politique extérieure, sur les terrains de chasse de la politique mondiale.\par
Tout comme celui des États occidentaux, l’impérialisme russe est un tissu d’éléments de nature différente. Son fil le plus solide n’est pas constitué, comme en Allemagne ou en Angleterre, par l’expansion économique d’un capital affamé d’accumulation, mais par les intérêts politiques de l’État. Il est vrai que l’industrie russe, ce qui est absolument caractéristique de la production capitaliste, en raison de l’inaptitude de son marché intérieur, exporte depuis longtemps vers l’Orient, vers la Chine, la Perse, l’Asie centrale, et que le gouvernement tsariste cherche par tous les moyens à favoriser cette exportation qui lui donne le fondement rêvé pour sa « sphère d’intérêts ». Mais ici, la politique de l’État détient le rôle actif, elle n’est pas dirigée par les autres facteurs. Dans les tendances conquérantes du régime tsariste s’exprime, d’une part, l’expansion traditionnelle d’un Empire puissant dont la population comprend aujourd’hui 170 millions d’êtres humains et qui, pour des raisons économiques et stratégiques, cherche à obtenir le libre accès des mers, de l’océan Pacifique à l’est, de la Méditerranée au sud, et, d’autre part, intervient ce besoin vital de l’absolutisme : la nécessité sur le plan de la politique mondiale de garder une attitude qui impose le respect dans la compétition générale des grands États, pour obtenir du capitalisme étranger le crédit financier sans lequel le tsarisme n’est absolument pas viable. A tout cela s’ajoute l’intérêt dynastique qui, comme dans toutes les monarchies, en raison de l’opposition de plus en plus vive entre le régime et la grande masse de la population, avait besoin de maintenir en permanence son prestige à l’étranger, et d’y chercher une diversion aux difficultés intérieures : recette indispensable de la politique.\par
Toutefois, les intérêts bourgeois modernes entrent toujours davantage en ligne de compte comme facteur de l’impérialisme dans l’Empire des tsars. Le jeune capitalisme russe, qui sous le régime absolutiste ne peut naturellement pas parvenir à un épanouissement complet et qui, en gros, ne peut quitter le stade du système primitif de vol, voit cependant s’ouvrir devant lui un avenir prodigieux dans les ressources naturelles immenses de cet Empire gigantesque. Il ne fait aucun doute que dès qu’elle sera débarrassée de l’absolutisme, la Russie deviendra rapidement - et à supposer que la situation de la lutte des classes internationale lui en laisse encore le répit - le premier État capitaliste moderne. C'est parce qu’elle pressent cet avenir et qu’elle est, pour ainsi dire par avance, affamée d’accumulation, que la bourgeoisie russe est dévorée par une fièvre impérialiste violente, et qu’elle manifeste avec ardeur ses prétentions dans le partage du monde. Cette fièvre historique trouve en même temps un soutien dans les très puissants intérêts actuels de la bourgeoisie russe. Tout d’abord dans les intérêts concrets de l’industrie des armements et de ses fournisseurs ; en Russie également, l’industrie lourde fortement organisée en cartels joue un grand rôle. En second lieu, l’opposition à l’« ennemi intérieur », au prolétariat révolutionnaire, a particulièrement renforcé l’estime que la bourgeoisie porte au militarisme et à l’action de diversion que représente l’évangile de la politique mondiale, et elle a ainsi rapproché la bourgeoisie du régime contre-révolutionnaire. L'impérialisme des milieux bourgeois de la Russie, et surtout des milieux libéraux, grandi à vue d’œil dans l’air orageux de la Révolution et dans ce baptême du feu, il a donné une physionomie moderne à la politique étrangère traditionnelle de l’Empire des tsars.\par
Or, le but principal de la politique traditionnelle du tsarisme aussi bien que des appétits modernes de la bourgeoisie russe, ce sont les Dardanelles, qui, selon le mot célèbre de Bismarck, donnent la clé des possessions russes sur la mer Noire. C'est en vue de ce but que la Russie a mené depuis le XVIIIe siècle une série de guerres sanglantes contre la Turquie, qu’elle a entrepris de libérer les Balkans et qu’au service de cette mission elle a entassé des monceaux énormes de cadavres à Ismail, Navarin, Sinope, Sistrie et Sébastopol, à Plevna et à Schipka. Tout cela, disait-on, pour défendre les frères slaves et les chrétiens contre les atrocités des Turcs ; cette séduisante légende de guerre joua auprès des moujiks russes le même rôle  que la « défense de la civilisation et de la liberté allemandes contre les atrocités russes » elle joue maintenant auprès de la social-démocratie allemande. La bourgeoisie russe était plus enthousiaste pour les perspectives sur la Méditerranée que pour la mission civilisatrice en Mandchourie et en Mongolie. C'est pourquoi la guerre japonaise fut très critiquée par la bourgeoisie libérale qui la considérait comme une aventure insensée, parce que, selon elle, la politique russe se détournait de sa tâche essentielle : les Balkans. Mais, d’une autre manière encore, la guerre malheureuse contre le Japon a eu le même effet. L'extension de la puissance russe en Asie orientale et en Asie centrale jusqu’au Tibet et vers la Perse devait inquiéter vivement la vigilance de l’impérialisme anglais. Préoccupée pour son énorme empire indien, l’Angleterre devait suivre l’avance russe en Asie avec une méfiance croissante. Et l’opposition anglo-russe en Asie fut effectivement l’opposition politique la plus forte de la conjoncture internationale au début de ce siècle, et elle devrait très vraisemblablement devenir le nœud du futur développement impérialiste après la guerre actuelle. La défaite fracassante de la Russie en 1904 et l’éclatement de la révolution modifièrent la situation. L'affaiblissement visible de l’empire des tsars eut comme conséquence d’amener une détente dans ses rapports avec l’Angleterre, détente qui conduisit même à un arrangement sur un blocage commun de la Perse en 1907, et qui permit des relations de bon voisinage en Asie centrale. Par là, la Russie se voyait avant tout interdire l’accès à de grandes entreprises en Asie et elle rassembla toute son énergie en vue de son vieil objectif : la politique des Balkans. C'est dans cette région que la Russie tsariste, après un siècle d’amitié solide et fidèle avec la civilisation allemande, entra pour la première fois dans un conflit pénible avec elle. Le chemin des Dardanelles passe par le cadavre de la Turquie, mais l’Allemagne considérait l’intégrité de ce cadavre comme sa tâche politique principale. Il est vrai que les principes de la politique russe dans les Balkans avaient déjà changé plus d’une fois : irritée de l’« ingratitude » des Slaves des Balkans qu’elle avait libérés et qui cherchaient à s’arracher à leurs liens de vassalité vis-à-vis de l’empire du tsar, la Russie avait, elle aussi, défendu pendant tout un temps le programme de l’« intégrité » de la Turquie, et pour elle aussi, il était sous-entendu que le partage était remis dans l’attente d’une époque plus favorable. Cependant, la liquidation finale de la Turquie a maintenant sa place dans les plans de la Russie tout comme dans la politique anglaise. Celle-ci, en vue de renforcer sa propre position dans les Indes et en Égypte, s’efforce de réunir en un grand empire musulman sous le sceptre britannique les territoires qui séparent ces deux parties de son empire, à savoir l’Arabie et la Mésopotamie. Ainsi, l’impérialisme russe, tout comme auparavant l’impérialisme anglais, tomba en Orient sur l’impérialisme allemand, lequel, se considérant comme l’usufruitier attitré de la décomposition de la Turquie, montait la garde sur le Bosphore \footnote{ \noindent Au mois de janvier 1908, l’homme politique libéral russe Pierre Strouvé écrivait, d’après la presse allemande : « Maintenant, il est temps de dire qu’il n’existe qu’un moyen pour créer une grande Russie, c’est de concentrer toutes nos forces sur une seule région qui soit accessible à la civilisation russe et où elle pourra exercer une influence réelle. Cette région, c’est tout le bassin de la mer Noire, c’est-à-dire l’ensemble des pays européens et asiatiques riverains de la mer Noire. Là, nous disposons d’une base réelle pour asseoir solidement notre souveraineté économique : des hommes, du charbon et du fer. C'est sur cette base réelle, et sur elle seulement, que, par un travail civilisateur infatigable soutenu de tous côtés par l’État, on pourra édifier une grande Russie économiquement forte. »\par
 Au début de la guerre mondiale actuelle, le même Strouvé écrivait, encore avant l’intervention de la Turquie :\par
 « Chez les politiciens allemands apparaît une politique d’autonomie turque, dont l’idée maîtresse est le programme de l’égyptisation de la Turquie sous la sauvegarde de l’Allemagne. Le Bosphore et les Dardanelles devraient devenir un Suez allemand. Avant la guerre entre l’Italie et la Turquie qui délogea les Turcs de leurs positions en Afrique, et avant la guerre des Balkans, qui les chassa presque d’Europe, la tâche suivante apparaissait déjà clairement pour l’Allemagne : conserver la Turquie et maintenir son indépendance dans l’intérêt de la stabilité économique et politique de l’Allemagne. Après les guerres que nous venons de mentionner, cette tâche ne changea que dans la mesure où la faiblesse extrême de la Turquie s’était montrée au grand jour ; dans ces conditions, une alliance devait dégénérer aussitôt en un protectorat ou une tutelle qui devait finalement amener l’Empire ottoman au même point que l’Égypte. Or, il est absolument clair qu’une Égypte allemande sur la mer Noire et la mer de Marmara serait tout à fait intolérable pour la Russie. Dès lors, il ne faut pas s’étonner que le gouvernement russe ait aussitôt protesté contre les démarches qui préparaient une telle politique et notamment contre la mission du général Liman von Sanders, qui devait non seulement réorganiser l’armée turque, mais même commander un corps d’armée à Constantinople. La Russie obtint là-dessus des satisfactions formelles, mais, en réalité, la situation ne changea pas d’un pouce. Dans ces conditions, en décembre 1913, une guerre était imminente entre la Russie et l’Allemagne : l’exemple de la mission militaire Liman von Sanders avait révélé que la politique de l’Allemagne tendait à l"' égyptisation " de la Turquie. Cette nouvelle direction de la politique allemande aurait suffi à elle seule à provoquer un conflit armé entre l’Allemagne et la Russie. Nous entrions donc en décembre 1913 dans une époque de mûrissement d’un conflit qui devait inévitablement prendre le caractère d’un conflit mondial. »
}.\par
Mais la politique russe dans les Balkans se heurtait encore plus directement à l’Autriche qu’à l’Allemagne. L'impérialisme autrichien est le complément politique de l’impérialisme allemand, son frère siamois et son destin funeste tout à la fois.\par
L'Allemagne se retrouva isolée de tous côtés à cause de sa politique mondiale, et son seul allié était l’Autriche. Sans doute, l’alliance avec l’Autriche est-elle ancienne, c’est encore Bismarck qui l’a établie en 1879, mais elle a changé entièrement de caractère depuis lors. De même que l’opposition avec la France, cette alliance a pris un tout autre contenu au cours de l’évolution des dernières décennies. Bismarck songeait seulement à défendre les possessions acquises jusqu’en 1870 grâce à la guerre de 1864. La Triple Alliance qu’il avait conclue avait un caractère conservateur d’un bout à l’autre : elle signifiait que l’Autriche devait renoncer définitivement à entrer dans la confédération allemande, elle représentait la consécration  de la situation créée par Bismarck, la victoire de l’éparpillement national de l’Allemagne et de l’hégémonie militaire de la Grande Prusse. Les tendances de l’Autriche vers les Balkans déplaisaient à Bismarck tout autant que les acquisitions de l’Allemagne en Afrique. Dans ses Pensées et souvenirs, il dit :\par

\begin{quoteblock}
 \noindent « Il est naturel que les habitants du bassin du Danube aient des besoins et des projets qui dépassent les frontières actuelles de la monarchie : la constitution de l’empire allemand montre la voie par laquelle l’Autriche peut parvenir à réconcilier ses intérêts politiques et matériels qui sont compris entre la frontière orientale qui est de race roumaine et le golfe de Cattaro. Mais ce n’est pas le rôle de l’Empire allemand que de prêter main forte à ses sujets pour réaliser les vœux qu’ils peuvent entretenir quant à leurs rapports avec leurs voisins. »
\end{quoteblock}

\noindent Comme il l’avait exprimé un jour avec force dans un mot célèbre, la Bosnie ne valait pas pour lui l’os d’un grenadier de Poméranie. La meilleure preuve de ce que Bismarck ne pensait effectivement pas à mettre la Triple Alliance au service des efforts d’expansion de l’Autriche, c’est le Traité de réassurance conclu en 1884 avec la Russie, et aux termes duquel, au cas où une guerre éclaterait entre la Russie et l’Autriche, l’Empire allemand ne se porterait en aucun cas aux côtés de l’Autriche, mais conserverait une « neutralité bienveillante ». Depuis que s’est accompli le virage de la politique allemande vers l’impérialisme, ses relations avec l’Autriche se modifièrent également. L'Autriche-Hongrie se trouve située entre l’Allemagne et les Balkans, donc sur le chemin de ce centre de la politique orientale de l’Allemagne. Avoir l’Autriche pour adversaire équivaudrait, en raison de l’isolement général dans lequel s’est placée la politique allemande, à renoncer à tous ses projets sur le plan de la politique mondiale. Dans le cas d’un affaiblissement ou même de la ruine de l’Autriche-Hongrie, qui entraînerait aussitôt une liquidation de la Turquie et un renforcement énorme de la puissance de la Russie, des États balkaniques et de l’Angleterre, l’Allemagne réaliserait sans doute son unification et renforcerait sa puissance, mais il faudrait sonner le glas de la politique impérialiste de l’empire allemand \footnote{Dans le tract impérialiste Pourquoi la guerre allemande ?, nous lisons : « La Russie avait déjà éprouvé auparavant la tentation de nous offrir l’Autriche allemande, ces dix millions d’Allemands qui étaient restés en dehors de notre unification nationale en 1866 et en 1870-71. Si nous leur livrions la monarchie des Habsbourg, cette trahison pourrait recevoir salaire. »}.\par
Le sauvetage et la conservation de la monarchie habsbourgeoise devenait donc logiquement la tâche accessoire de l’impérialisme allemand, tout comme la conservation de la Turquie était sa tâche principale.\par
L'existence même de l’Autriche représente cependant un état permanent de guerre latente dans les Balkans. Depuis que le processus irrésistible de décomposition de la Turquie a conduit à la formation et à la consolidation des États balkaniques dans la proximité immédiate de l’Autriche, ce fut le début d’une opposition entre l’État habsbourgeois et ses jeunes voisins. Il est évident que la naissance à ses côtés d’États nationaux indépendants et viables devait accélérer la décomposition de cette monarchie déjà délabrée qui, étant elle-même constituée d’une mosaïque de pièces détachées de ces mêmes nationalités, ne sait les diriger que sous la férule des paragraphes dictatoriaux. La non-viabilité foncière de l’Autriche se manifeste précisément dans sa politique balkanique et tout spécialement dans ses rapports avec la Serbie. En dépit de ses appétits impérialistes qui se jetaient sans discernement tantôt sur Salonique, tantôt sur Durazzo, l’Autriche n’était pas en état d’annexer la Serbie le cas échéant, même si celle-ci n’avait pas reçu un surcroît de force et d’étendue à la suite des deux guerres balkaniques. En incorporant la Serbie, l’Autriche aurait nourri en son sein d’une manière dangereuse l’une des plus turbulentes parmi les nationalités slaves du sud qu’elle ne parvenait déjà à maîtriser qu’à grand-peine malgré le régime brutal et stupide de sa réaction \footnote{Le Kölnische Zeitung écrivait après l’attentat de Sarajevo, c’est-à-dire à la veille de la guerre, alors qu’on ne connaissait pas encore le dessous des cartes de la politique allemande officielle : « Celui qui n’est pas au courant de la situation posera la question : comment se fait-il que malgré les bienfaits qu’elle a prodigués à la Bosnie, non seulement l’Autriche n’est pas aimée dans ce pays, mais est même carrément détestée par les Serbes qui constituent 42 \% de la population ? Seul quelqu’un qui connaît le peuple et la situation comprendra la réponse à cette question : un non-initié, surtout s’il est accoutumé aux idées et aux réalités européennes, l’écoutera bouche-bée sans comprendre. Voici la réponse noir sur blanc : l’administration de la Bosnie fut un gâchis complet dans sa conception et dans ses principes fondamentaux et c’est l’ignorance absolument criminelle qui règne encore en partie aujourd’hui, après plus d’une génération (depuis l’occupation) au sujet de la situation réelle de ce pays, qui en porte la responsabilité. »}.\par
L'Autriche ne pouvait cependant pas non plus tolérer le développement normal autonome de la Serbie et en tirer profit par des relations économiques normales. En effet, la monarchie habsbourgeoise n’est pas une organisation politique d’État bourgeois, mais seulement un trust unissant par des liens assez lâches quelques coteries de parasites sociaux qui veulent se remplir les poches en exploitant au maximum les ressources du pouvoir tant que la monarchie tient encore debout. Pour favoriser les agriculteurs hongrois et pour maintenir artificiellement les produits agricoles à un prix élevé, l’Autriche interdit l’importation du bétail et des fruits à la Serbie, privant ainsi ce pays paysan du débouché principal de ses produits. Dans l’intérêt des cartels industriels autrichiens, elle contraignit la Serbie à obtenir à l’est l’accès de la mer Noire en concluant une alliance militaire avec la Bulgarie, et à l’ouest l’accès de la mer Adriatique en acquérant un port en Albanie. La politique balkanique de l’Autriche visait donc uniquement à étrangler la Serbie. Mais  en même temps, elle visait à empêcher tout rapprochement mutuel entre les États balkaniques et à entraver leur essor intérieur ; elle constituait à elle seule un danger permanent. Tantôt par l’annexion de la Serbie, tantôt en manifestant ses prétentions sur le sandjak de Novibazar et sur Salonique, tantôt en revendiquant la côte albanaise, l’impérialisme autrichien menaçait continuellement l’existence et les possibilités de développement des États balkaniques. Conformément aux tendances de l’Autriche et en raison de la concurrence de l’Italie, on allait même créer après la seconde guerre balkanique l’image dérisoire d’une « Albanie indépendante » sous un prince allemand qui, dès la première heure, ne fut rien d’autre que le jouet des intrigues des puissances impérialistes rivales.\par
L'Autriche ne pouvait cependant pas non plus tolérer le développement normal autonome de la Serbie et en tirer profit par des relations économiques normales. En effet, la monarchie habsbourgeoise n’est pas une organisation politique d’État bourgeois, mais seulement un trust unissant par des liens assez lâches quelques coteries de parasites sociaux qui veulent se remplir les poches en exploitant au maximum les ressources du pouvoir tant que la monarchie tient encore debout. Pour favoriser les agriculteurs hongrois et pour maintenir artificiellement les produits agricoles à un prix élevé, l’Autriche interdit l’importation du bétail et des fruits à la Serbie, privant ainsi ce pays paysan du débouché principal de ses produits. Dans l’intérêt des cartels industriels autrichiens, elle contraignit la Serbie à obtenir à l’est l’accès de la mer Noire en concluant une alliance militaire avec la Bulgarie, et à l’ouest l’accès de la mer Adriatique en acquérant un port en Albanie. La politique balkanique de l’Autriche visait donc uniquement à étrangler la Serbie. Mais en même temps, elle visait à empêcher tout rapprochement mutuel entre les États balkaniques et à entraver leur essor intérieur; elle constituait à elle seule un danger permanent. Tantôt par l’annexion de la Serbie, tantôt en manifestant ses prétentions sur le sandjak de Novibazar et sur Salonique, tantôt en revendiquant la côte albanaise, l’impérialisme autrichien menaçait continuellement l’existence et les possibilités de développement des États balkaniques. Conformément aux tendances de l’Autriche et en raison de la concurrence de l’Italie, on allait même créer après la seconde guerre balkanique l’image dérisoire d’une « Albanie indépendante » sous un prince allemand qui, dès la première heure, ne fut rien d’autre que le jouet des intrigues des puissances impérialistes rivales.\par
Ainsi, au cours des dernières décennies, la politique impérialiste de l’Autriche devint le carcan qui empêchait un développement normal vers le progrès dans les Balkans, et elle conduisait tout naturellement à ce dilemme inévitable : ou bien la monarchie habsbourgeoise, ou bien le développement des États balkaniques ! Les Balkans, qui s’étaient émancipés de la souveraineté turque, se voyaient confrontés à une nouvelle tâche : se débarrasser de l’obstacle que représentait l’Autriche. Historiquement, la liquidation de l’Autriche-Hongrie n’est que la continuation du démembrement de la Turquie et est, comme lui, imposée par l’évolution historique.\par
Mais ce dilemme ne laissait pas d’autre solution que la guerre, et même la guerre mondiale. En effet, derrière la Serbie, on trouvait la Russie, qui ne pouvait renoncer à son influence dans les Balkans et à son rôle de « protecteur » sans compromettre la totalité de son programme impérialiste en Orient. Exactement à l’opposé de la politique autrichienne, la politique russe avait pour objectif de réunir les États balkaniques, sous protectorat russe évidemment. La confédération balkanique, dont la victoire dans la guerre de 1912 avait presque entièrement liquidé la Turquie d’Europe, était l’œuvre de la Russie, et il entrait dans les intentions de celle-ci qu’elle soit principalement dirigée contre l’Autriche. Sans doute la confédération se disloqua-t-elle dès la première guerre balkanique malgré tous les efforts de la Russie, mais la Serbie qui sortit victorieuse de cette guerre était destinée à devenir l’alliée de la Russie de la même manière que l’Autriche devenait son ennemi mortel. L'Allemagne, enchaînée au destin de la monarchie habsbourgeoise, fut obligée de donner son soutien à sa politique archi-réactionnaire et ainsi d’entrer doublement en conflit avec la Russie.\par
La politique balkanique de l’Autriche l’amena également à entrer en conflit avec l’Italie, qui s’intéressait vivement à la fois à la liquidation de l’Autriche et à celle de la Turquie. L'impérialisme italien trouve à ses désirs d’expansion le prétexte le plus proche et le plus commode, parce que le plus populaire, dans les possessions italiennes de l’Autriche et, dans le nouveau partage des Balkans, ses prétentions visent surtout la côte albanaise de l’Adriatique qui fait face à l’Italie. La Triple Alliance qui avait déjà subi une rude épreuve dans la guerre de Tripoli fut complètement ravagée par la crise que connurent les Balkans depuis les deux guerres balkaniques, et ses deux puissances centrales étaient en conflit avec le reste du monde. L'impérialisme allemand, enchaîné à deux cadavres en décomposition, se dirigeait tout droit vers la guerre mondiale.\par
Ce cheminement était du reste tout à fait conscient. C'est surtout l’Autriche qui donnait l’impulsion, elle qui courait depuis des années à la catastrophe avec un fatal aveuglement. Sa clique dirigeante cléricale et militaire, avec à sa tête l’archiduc François-Ferdinand et l’homme de main de celui-ci, le baron von Chlumezki, était bel et bien à l’affût d’un prétexte pour lancer les opérations. En 1909, pour déclencher dans les pays allemands la fureur guerrière qu’elle recherchait, elle fit fabriquer tout spécialement par le professeur Friedmann les fameux documents qui dévoilaient une conspiration diabolique aux multiples ramifications dirigée contre la monarchie habsbourgeoise, documents qui n’avaient qu’un seul défaut : ils étaient faux de \emph{a} à \emph{z}. Quelques années plus tard, se répandit pendant des jours la nouvelle que le consul autrichien Prohaska avait subi un martyre atroce à Uestub, ce qui devait faire l’effet d’une bombe, alors que pendant ce temps-là, Prohaska, qui se portait comme un charme, se promenait en sifflotant dans les rues de  Uestub. Enfin, il y eut l’attentat de Sarajevo, enfin se produisit le crime révoltant et authentique que l’on attendait depuis si longtemps. « Si un sacrifice a jamais eu un effet libérateur et rédempteur, c’est bien celui-là », exultaient les porte-parole de l’impérialisme allemand. Les impérialistes autrichiens exultaient encore plus fort et décidèrent d’utiliser les cadavres des archiducs tant qu’ils étaient encore frais \footnote{ \noindent Pourquoi la guerre allemande ? p. 21. L'organe de la clique de l’archiduc « Grande-Autriche » écrivait semaine après semaine des articles incendiaires de ce style :\par
 « Si on veut venger dignement la mort de l’archiduc héritier François-Ferdinand en respectant ses volontés, alors il faut exécuter aussi rapidement que possible le testament politique de cette victime innocente du développement funeste de la situation au sud de l’Empire. »\par
 « Cela fait déjà dix ans que nous attendons d’être enfin délivrés de toutes les tensions accablantes qui se font si cruellement sentir dans toute notre politique. »\par
 « Nous savons que l’Autriche grandiose et nouvelle, la Grande-Autriche qui ira délivrer ses peuples dans l’allégresse, ne pourra naître que par une guerre, et c’est pourquoi nous voulons la guerre. »\par
 « Nous voulons la guerre parce que nous sommes profondément convaincus que seule une guerre nous permettra de réaliser d’une manière radicale et soudaine notre idéal d’une Grande-Autriche puissante où, dans l’éclat lumineux d’un avenir sublime et joyeux, pourront s’épanouir la pensée politique et les projets missionnaires de l’Autriche : apporter la liberté et la civilisation aux peuples des Balkans. »\par
 « Depuis la mort du grand homme dont la main puissante et l’énergie opiniâtre auraient fondé la Grande-Autriche du jour au lendemain, la guerre reste notre seul espoir. »\par
 « C'est notre dernière carte, sur laquelle nous misons tout. »\par
 « Peut-être l’énorme indignation que cet attentat a soulevée en Autriche et en Hongrie provoquera-t-elle une explosion contre la Serbie, et ultérieurement aussi contre la Russie. »\par
 « L'archiduc François-Ferdinand à lui seul n’a pu que préparer cet impérialisme, il n’a pu l’accomplir. Il faut espérer que sa mort aura été le sacrifice nécessaire qui provoquera l’embrasement impérialiste de toute l’Autriche. »
}. Ils s’entendirent rapidement avec l’Allemagne, la guerre fut conclue et on expédia le télégramme qui allait mettre le feu aux poudres à l’intérieur du monde capitaliste.\par
Mais l’incident de Sarajevo n’avait fait que fournir le prétexte.\par
Pour ce qui est des causes et des oppositions, tout était déjà mûr pour la guerre depuis longtemps, la configuration que nous connaissons aujourd’hui était déjà prête depuis dix ans. Chaque année qui s’écoulait et chaque nouvel événement politique qui s’est produit au cours de ces dernières années rapprochaient un peu plus l’échéance : la révolution turque, l’annexion de la Bosnie, la crise du Maroc, l’expédition de Tripoli, les deux guerres des Balkans. C'est dans la perspective de cette guerre que furent proposés tous les projets de loi de ces dernières années : on se préparait consciemment à l’inévitable conflagration générale. Cinq fois au cours de ces dernières années, il s’en est fallu d’un cheveu que la guerre n’ait éclaté : en été 1905, lorsque l’Allemagne fit connaître pour la première fois ses prétentions dans l’affaire du Maroc d’une manière péremptoire ; en été 1908, après la rencontre des monarques à Reval, lorsque l’Angleterre, la Russie et la France voulurent envoyer un ultimatum à la Turquie à cause de la question macédonienne, et que, pour défendre la Turquie, l’Allemagne était prête à se lancer dans une guerre qui ne fut empêchée que par l’éclatement soudain de la révolution turque \footnote{« Du côté de la politique allemande, on était évidemment informé de ce qui devait se passer, et aujourd’hui, on ne trahit plus un secret en disant que, comme d’autres flottes européennes, les forces navales de l’Allemagne se trouvaient alors sur le pied de guerre, prêtes à intervenir immédiatement. » (Rohrbach, La guerre et la politique allemande, p. 32.)}; au début de 1909, lorsque la Russie répondit à l’annexion de la Bosnie par une mobilisation, sur quoi l’Allemagne déclara en bonne forme qu’elle était prête à entrer en guerre aux côtés de l’Autriche ; en été 1911, lorsque le Panther fut envoyé à Agadir, ce qui aurait inévitablement provoqué le déclenchement de la guerre, si l’Allemagne n’avait pas renoncé à réclamer sa part du Maroc et ne s’était pas contentée du Congo. Et enfin, au début de l’année 1913, quand l’Allemagne, voyant que la Russie envisageait de pénétrer en Arménie, déclara pour la deuxième fois en bonne forme qu’elle était prête à faire la guerre.\par
C'est ainsi que la guerre mondiale actuelle était dans l’air depuis huit ans. Si, à chaque fois, elle fut différée, c’est uniquement parce que l’une des parties impliquées n’avait pas encore terminé ses préparatifs militaires. La guerre mondiale actuelle était déjà mûre dans l’aventure du Panther en 1911 - sans le couple d’archiducs assassinés, sans les aviateurs français au-dessus de Nuremberg, et sans l’invasion russe en Prusse orientale. L'Allemagne l’a simplement remise à une date qui lui conviendrait mieux. Ici aussi, il suffit de lire l’explication naïve des impérialistes allemands :\par

\begin{quoteblock}
 \noindent « Du côté "pan-germanique", on reproche à la politique allemande de s’être montrée trop faible durant la crise du Maroc en 1911 ; pour liquider cette idée fausse, il suffit de rappeler qu’au moment où nous avons envoyé le Panther à Agadir, l’aménagement du canal de la mer du Nord était encore en chantier, que l’aménagement d’Helgoland en une grande place forte maritime n’était pas encore terminé et que les rapports de force entre notre flotte et la puissance navale anglaise en dreadnoughts et en armements auxiliaires nous étaient nettement plus défavorables que trois ans après. Le canal, l’île d’Helgoland et la puissance de notre flotte étaient, comparativement à ce qu’ils sont aujourd’hui, en 1914, soit fortement périmés, soit  absolument inaptes pour la guerre. Dès lors, sachant que l’on rencontrerait un peu plus tard des chances de succès bien plus favorables, vouloir provoquer une guerre décisive aurait été une folie pure et simple \footnote{Rohrbach, La Guerre et la politique allemande, p. 41.}. »
\end{quoteblock}

\noindent Il fallait d’abord mettre la flotte allemande en état et faire passer au Reichstag les projets de lois militaires. En été 1914, l’Allemagne se sentit préparée pour la guerre, alors que la France en était encore à élaborer péniblement le service militaire de trois ans, et alors que la Russie n’avait pas encore accompli son programme, ni pour la force navale ni pour l’armée de terre. Le même Rohrbach - qui n’est pas seulement le porte-parole le plus sérieux de l’impérialisme allemand, mais, étant très proche des milieux dirigeants de la politique allemande, est presque leur voix officieuse - écrit au sujet de la situation en 1914 :\par

\begin{quoteblock}
 \noindent « Quant à nous, c’est-à-dire l’Allemagne et l’Autriche-Hongrie, notre crainte essentielle était que si la Russie adoptait pour quelque temps une attitude manifestement conciliante, nous aurions été moralement obligés d’attendre jusqu’au moment où la France et la Russie auraient été réellement prêtes. » (loc. cit. p. 83)
\end{quoteblock}

\noindent Autrement dit : la crainte essentielle en juillet 1914, c’était que l’« action de paix » du gouvernement allemand puisse être couronnée de succès, et que la Russie et la Serbie puissent se laisser fléchir. Il s’agissait cette fois de les contraindre à la guerre. Et cela réussit. « C'est avec une profonde douleur que nous voyons échouer nos efforts visant à maintenir la paix mondiale », etc.\par
Dès lors, lorsque les bataillons allemands pénétrèrent en Belgique, lorsque le Reichstag fut placé devant le fait accompli de la guerre et de l’état de siège, il n’y avait pas de quoi être frappé de stupeur, car ce n’était pas une situation nouvelle et inouïe, ce n’était pas un événement qui, compte tenu du contexte politique, pouvait surprendre la social-démocratie allemande. La guerre mondiale déclarée officiellement le 4 août était celle-là même pour laquelle la politique impérialiste allemande et internationale travaillait inlassablement depuis des dizaines d’années, celle-là même dont, depuis dix ans, la social-démocratie allemande, d’une manière tout aussi inlassable, prophétisait l’approche presque chaque année, celle-là même que les parlementaires, les journaux et les brochures sociales-démocrates stigmatisèrent à de multiples reprises comme étant un crime frivole de l’impérialisme, qui n’avait rien à voir ni avec la civilisation ni avec les intérêts nationaux mais qui, bien au contraire, agissait à l’encontre de ces deux principes.\par
Effectivement : ce n’est pas l’« existence et le développement libre » de l’Allemagne qui sont en jeu dans cette guerre, comme le dit la déclaration du groupe parlementaire social-démocrate, ce n’est pas la civilisation allemande, comme l’écrit la presse sociale-démocrate, mais ce sont bien plutôt les profits actuels de la Deutsche Bank en Turquie d’Asie et les profits futurs des Mannesmann et Krupp au Maroc, c’est l’existence du régime réactionnaire de l’Autriche, ce « monceau de pourriture organisée qui s’appelle la monarchie habsbourgeoise », comme l’écrivait le Vorwärts du 25 août 1914, ce sont les cochons et les pruneaux hongrois, c’est le paragraphe 14, ce sont les trompettes d’enfants et la civilisation de FriedmannProhasta, c’est le maintien de la domination turque des Bachibouzouks en Asie Mineure et de la contre-révolution dans les Balkans.\par
Une grande partie de la presse de notre parti était profondément choquée de ce que les « gens de couleur et les sauvages », les Nègres, les Sikhs, les Maoris, étaient poussés à la guerre par les adversaires de l’Allemagne. Or, ces peuples jouent à peu près le même rôle dans la guerre actuelle que les prolétaires socialistes des États européens. Et si on apprenait par les communiqués de Reuter que les Maoris de Nouvelle-Zélande brûlaient d’envie de se faire massacrer pour le roi d’Angleterre, ils manifesteraient le même discernement dans la conscience de leurs propres intérêts que celui dont a fait preuve le groupe parlementaire social-démocrate en confondant le salut de la monarchie habsbourgeoise, de la Turquie et de la caisse de la Deutsche Bank avec l’existence et la liberté du peuple allemand et de la civilisation allemande. Il est vrai qu’une grande différence les sépare malgré tout : il y a une génération, les Maoris pratiquaient encore le cannibalisme, et pas la théorie marxiste.
\section[{Mais le Tsarisme !}]{Mais le Tsarisme !}\renewcommand{\leftmark}{Mais le Tsarisme !}

\noindent Mais le tsarisme ! C'est lui indiscutablement qui a décidé de l’attitude adoptée par notre parti, surtout tout au début de la guerre. Dans sa déclaration, le groupe social-démocrate avait lancé le mot d’ordre : contre le tsarisme ! Dans la presse sociale-démocrate, c’est devenu aussitôt un combat pour la « civilisation » européenne tout entière.\par
Le \emph{Frankfurter Volksstimme} écrivait déjà le 31 juillet :\par

\begin{quoteblock}
 \noindent « La social-démocratie allemande a depuis longtemps accusé le tsarisme d’être le rempart sanglant de la réaction européenne, depuis l’époque où Marx et Engels poursuivaient tous les faits et gestes de ce régime barbare de leurs analyses pénétrantes, jusqu’à l’époque actuelle, où il remplit ses prisons de prisonniers politiques mais tremble pourtant devant tout  mouvement ouvrier. Puisse maintenant venir l’occasion d’en finir avec cette société effroyable sous les drapeaux de guerre allemands. »
\end{quoteblock}

\noindent Le \emph{Pfälzische Post} de Ludwigshafen, le même jour :\par

\begin{quoteblock}
 \noindent « C'est un principe qu’a forgé notre inoubliable August Bebel : il s’agit ici du combat de la civilisation contre la barbarie, auquel le prolétariat participe également. »
\end{quoteblock}

\noindent Le \emph{Münchener Post} du 1er août :\par

\begin{quoteblock}
 \noindent « Dans l’accomplissement du devoir de la défense du pays contre le tsarisme sanglant, nous ne voulons pas qu’on fasse de nous des citoyens de deuxième classe. »
\end{quoteblock}

\noindent Le \emph{Volksblatt} de Halle daté du 5 août :\par

\begin{quoteblock}
 \noindent « S'il est exact que nous sommes attaqués par la Russie - et c’est ce que toutes les dépêches nous ont donné à entendre jusqu’ici -, il va de soi que la social-démocratie approuve tous les moyens mis en oeuvre pour la défense. Nous devons de toutes nos forces chasser le tsarisme du pays ! »
\end{quoteblock}

\noindent Et le même journal, le 18 août :\par

\begin{quoteblock}
 \noindent « Maintenant, les dés en sont jetés, maintenant ce n’est plus seulement le devoir de défendre notre patrie et de sauvegarder l’existence de la nation qui nous fait prendre les armes, comme tous les autres Allemands, mais aussi la conscience que l’ennemi contre lequel nous nous battons à l’est est également l’ennemi de tout progrès et de toute civilisation... La défaite de la Russie équivaut à la victoire de la liberté en Europe. »
\end{quoteblock}

\noindent Le \emph{Volksfreund} de Braunschweig écrivait le 5 août :\par

\begin{quoteblock}
 \noindent « La pression irrésistible de la violence militaire emporte tout sur son passage. Mais les ouvriers conscients ne subissent pas seulement une contrainte extérieure, ils obéissent à leur propre conviction, lorsqu’ils défendent leur sol devant l’envahisseur venu de l’est. »
\end{quoteblock}

\noindent Le \emph{Arbeiterzeitung} d’Essen s’écriait déjà le 3 août :\par

\begin{quoteblock}
 \noindent « Si maintenant ce pays est menacé par les desseins de la Russie, alors, puisqu’il s’agit de combattre le militarisme russe dont les crimes contre la liberté et la civilisation ne se comptent plus, nous n’accepterons pas d’être en reste avec personne dans le pays quant à l’accomplissement du devoir et à l’esprit de sacrifice... A bas le tsarisme ! A bas le rempart de la barbarie ! Voilà le mot d’ordre. »\par
 De même, le \emph{Volkswacht} de Bielefeld daté du 4 août :\par
 « Le mot d’ordre est partout le même : Contre le despotisme russe et sa perfidie !»
\end{quoteblock}

\noindent Le journal du parti à \emph{Elberfeld} daté du 5 août :\par

\begin{quoteblock}
 \noindent « Il est de l’intérêt vital de toute l’Europe occidentale d’éliminer ce tsarisme abominable et assoiffé de crimes. Mais cette action qui concerne toute l’humanité est contrecarrée par l’avidité des classes capitalistes d’Angleterre et de France qui veulent priver le capital allemand des sources de profit qu’il exploitait jusqu’ici. »
\end{quoteblock}

\noindent Le Rheinische Zeitung de Cologne :\par

\begin{quoteblock}
 \noindent « Amis, faites votre devoir, tous autant que vous êtes, là où vous envoie le destin ! Vous luttez pour la civilisation européenne, pour la liberté de votre patrie et pour votre propre prospérité. »
\end{quoteblock}

\noindent Le Schleswig-Holsteinische Volkszeitung du 7 août écrivait :\par

\begin{quoteblock}
 \noindent « Il va de soi que nous vivons à l’époque du capitalisme et que nous aurons très certainement aussi des luttes de classes après la Grande Guerre. Mais ces luttes se dérouleront dans un Etat plus libre qu’aujourd’hui ; elles se limiteront beaucoup plus au domaine économique et il sera impossible à l’avenir, une fois que le tsarisme russe aura disparu, de traiter les sociaux-démocrates comme des réprouvés, comme des bourgeois de deuxième classe dépourvus de droits politiques. »
\end{quoteblock}

\noindent Le 11 août, le Hamburger "\emph{Echo}" s’écriait :\par

\begin{quoteblock}
 \noindent « Car nous n’avons pas seulement à mener une guerre de défense contre l’Angleterre et la France, nous avons à faire avant tout la guerre au tsarisme, et cette guerre-là nous la faisons avec un enthousiasme et sans réserve. Car c’est une guerre pour la civilisation. »
\end{quoteblock}

\noindent Et l’organe du parti à \emph{Lübeck} déclarait encore le 4 septembre :\par

\begin{quoteblock}
 \noindent   « Si la liberté de l’Europe est sauvegardée après le déchaînement de la guerre, l’Europe le devra à la puissance des armes allemandes. C'est contre l’ennemi mortel de toute démocratie et de toute liberté que tendent tous nos efforts dans ce combat ! »
\end{quoteblock}

\noindent Voilà l’appel qui retentissait de tous côtés dans la presse du parti social-démocrate allemand.\par
Au stade initial de la guerre, le gouvernement allemand accepta l’aide qu’on lui offrait : d’une main nonchalante, il piqua sur son casque le laurier du libérateur de la civilisation européenne. Il consentit à jouer le rôle de « libérateur de nations », quoique avec un malaise visible et une grâce un peu lourde. Le commandement général « pour les deux grandes armées » avait même - nécessité n’a point de loi appris à parler juif, et, dans la Pologne russe, il chatouillait les « mendiants et les conspirateurs » derrière les oreilles. De même, on promit monts et merveilles aux Polonais s’ils trahissaient en masse le gouvernement tsariste alors que, au Cameroun, accusé à tort de ce même crime de « haute trahison », le Duala Manga Bell était pendu sans tambours ni trompettes au milieu du vacarme de la guerre, sans avoir à subir de fastidieuses procédures judiciaires. Et la social-démocratie prenait part à tous ces sauts d’ours de l’impérialisme allemand en difficulté. Tandis que le groupe parlementaire couvrait le corps de ce chef de tribu du Cameroun d’un silence discret, la presse social-démocrate remplissait l’air de ses chants d’allégresse et louait la liberté qui était apportée par les « crosses de fusil » allemandes aux pauvres victimes du tsarisme.\par
L'organe théorique du parti, le \emph{Neue Zeit}, écrivait dans son numéro du 28 août :\par

\begin{quoteblock}
 \noindent « La population frontalière de l’empire du " Petit Père " a salué avec des clameurs de triomphe les troupes d’avant-garde allemandes car, pour tous les Polonais et tous les Juifs de ces régions, l’idée de patrie n’évoque que la corruption et le knout. Ce sont de pauvres diables et de vrais sans-patrie, ces sujets exploités du sanglant Nicolas, et même s’ils en éprouvaient le désir, ils n’auraient rien d’autre à défendre que leurs chaînes, et c’est pourquoi maintenant ils ne vivent que dans l’espoir que des crosses de fusil allemandes, brandies par des poings allemands, puissent sous peu fracasser tout le système tsariste... Tandis que la tempête fait rage au-dessus de leurs têtes, la classe ouvrière allemande est animée par une volonté politique consciente : se défendre à l’ouest contre les alliés de la barbarie orientale, conclure avec eux une paix honorable et poursuivre la destruction du tsarisme jusqu’au dernier souffle des chevaux et des hommes. »
\end{quoteblock}

\noindent Le groupe social-démocrate avait prêté à la guerre le caractère d’une défense de la nation et de la civilisation allemandes ; la presse social-démocrate, elle, la proclama libératrice des peuples étrangers. Hindenburg devenait l’exécuteur testamentaire de Marx et Engels.\par
La mémoire de notre parti lui a décidément joué un mauvais tour au cours de cette guerre : il oubliait complètement tous ses principes, tous ses serments et toutes les résolutions adoptées dans les congrès internationaux au moment même où il s’agissait de les appliquer, mais, pour comble de malchance, il se souvenait d’un « testament » de Marx et le ressortait de la poussière des temps au moment même où il aurait mieux valu qu’il y reste, pour en faire l’ornement du militarisme prussien que Marx voulait combattre « jusqu’au dernier souffle des chevaux et des hommes ». C'étaient les sonneries de trompettes refroidies du Neue \emph{Rheinische Zeitung} et de la révolution allemande de Mars, dirigées contre la Russie asservie à Nicolas Ier qui, soudain, en l’an de grâce 1914, vinrent frapper les oreilles de la social-démocratie et lui fourrer dans les mains les « crosses de fusil allemandes » pour partir en campagne, bras dessus bras dessous avec les junkers prussiens, contre la Russie de la grande Révolution.\par
C'est ici qu’il s’agit de procéder à une « révision », et de passer en revue l’expérience historique de près de soixante-dix ans avec à la main les maîtres mots de la révolution de Mars.\par
En 1848, le tsarisme russe était effectivement « le rempart de la réaction européenne ». Produit spécifique des rapports sociaux de la Russie, profondément enraciné dans un système médiéval reposant sur l’économie naturelle, l’absolutisme russe était l’appui en même temps que le guide tout-puissant de la réaction monarchique ébranlée par la révolution bourgeoise et affaiblie en Allemagne par le particularisme des petits États. En 1851, Nicolas Ier pouvait faire entendre à Berlin par l’intermédiaire de l’envoyé prussien, von Rochow, qu’il « aurait beaucoup aimé qu’en novembre 1848, la révolution ait été étouffée dans l’oeuf par l’entrée du général von Wrangel à Berlin » et qu’il « y aurait eu d’autres moments où on n’aurait pas dû donner une mauvaise Constitution ». Ou, une autre fois, dans une admonestation à Manteuffel : qu’il avait le ferme espoir que, sous sa noble conduite, le ministère royal défendrait de la façon la plus énergique les droits de la couronne devant les chambres, et qu’il ferait valoir les principes conservateurs. Le même Nicolas Ier pouvait aussi accorder l’ordre de Alexandre Nevski à un président du Conseil prussien pour le récompenser de ses « efforts constants... en vue de renforcer l’ordre légal en Prusse ».\par
Déjà avec la guerre de Crimée, les choses changèrent considérablement. Cette guerre entraîna la faillite militaire et, du même coup, la faillite politique du vieux système. L'absolutisme russe se vit obligé d’entrer sur la voie des réformes, de se moderniser, de s’adapter aux conditions bourgeoises, et ainsi il avait mis le doigt dans un engrenage diabolique qui, peu à peu, devait finir par le happer tout entier. Les événements de la guerre de Crimée nous permettent en même temps d’effectuer une mise à l’épreuve instructive du dogme de la libération que des « crosses de fusil » peuvent apporter à un peuple asservi.\par
  La faillite militaire de Sedan a donné la république à la France. Mais cette république n’était pas un cadeau de la soldatesque de Bismarck : hier comme aujourd’hui, la Prusse n’avait rien à offrir aux autres peuples que son propre régime de junkers. En France, la république était le fruit d’une maturation intérieure, résultait des luttes sociales qui eurent lieu depuis 1789 et des trois révolutions. Quant au krach de Sébastopol, il eut le même effet que celui d’Iéna : faute d’un mouvement révolutionnaire à l’intérieur du pays, il ne conduisit qu’à une rénovation intérieure et à la consolidation de l’ancien régime.\par
Mais les réformes des années 60 en Russie, qui ont ouvert la voie au développement bourgeois-capitaliste, ne pouvaient être réalisées qu’avec les moyens financiers d’une économie bourgeoise-capitaliste. Et ces moyens étaient fournis par le capital d’Europe occidentale - de France et d’Allemagne. C'est à ce moment que s’est nouée cette nouvelle situation qui dure jusqu’à nos jours : l’absolutisme russe est entretenu par la bourgeoisie d’Europe occidentale. Le « rouble russe » ne coule plus à flots dans les chambres diplomatiques et, comme le déplorait amèrement le prince Guillaume de Prusse encore en 1854, « jusque dans l’antichambre du roi », mais tout au contraire, c’est l’or allemand et français qui coule vers Saint-Pétersbourg pour y nourrir le régime tsariste qui, sans cette sève vivifiante, aurait déjà achevé sa mission depuis longtemps. Dès cette époque, le tsarisme n’est plus seulement un produit des conditions économiques de la Russie : le système capitaliste de l’Europe occidentale devient sa deuxième racine. Depuis, la situation se transforme de plus en plus à chaque décennie. A mesure que la racine originelle de la monarchie est rongée en Russie même par le développement du capitalisme russe, son autre racine, la racine occidentale, se fortifie de plus en plus. Au soutien financier s’ajouta d’une manière croissante le soutien politique, à cause de la concurrence que se livraient la France et l’Allemagne depuis la guerre de 1870. Plus des forces révolutionnaires se dressaient contre l’absolutisme au sein même du peuple russe, et plus elles rencontraient de résistances venant des pays d’Europe occidentale, qui assuraient le tsarisme menacé de leur appui moral et politique. Au début des années 80, le mouvement terroriste du vieux socialisme russe avait ébranlé pour un certain temps le régime tsariste, et il avait sérieusement ruiné son autorité aussi bien à l’intérieur qu’à l’extérieur du pays ; c’est ce moment que choisit Bismarck pour conclure avec la Russie son Traité de réassurance et lui donner un appui dans la politique internationale. D'autre part, plus la Russie était courtisée par la politique allemande, et plus les caisses de la bourgeoisie française lui étaient largement ouvertes. Puisant dans ces deux sources de revenus, l’absolutisme cherchait à prolonger son existence en luttant contre le flot désormais croissant du mouvement révolutionnaire à l’intérieur du pays.\par
Alors, le développement capitaliste que le tsarisme avait choyé de ses propres mains commença enfin à porter ses fruits : à partir des années 90, on assista au mouvement révolutionnaire de masse du prolétariat russe. Les fondements du tsarisme se mettent à trembler et à chanceler dans le pays même. L'unique « rempart de la réaction européenne » se voit bientôt contraint d’accorder « une mauvaise constitution » et doit désormais chercher à son tour un « rempart » salvateur devant le flot qui monte dans son propre pays. Et il le trouve : en Allemagne. L'Allemagne de Bülow acquitte la dette de reconnaissance que la Prusse de Wrangel et de Manteuffel avait contractée. Le scénario est complètement inversé : l’aide fournie par la Russie en vue de lutter contre la révolution allemande est remplacée par l’aide que fournit l’Allemagne en vue de lutter contre la révolution russe. Dénonciations, interdictions de séjour, extraditions... - comme au temps béni de la Sainte Alliance, une chasse en règle aux « agitateurs » se déchaîne en Allemagne contre les combattants russes pour la liberté, et les poursuit jusqu’au seuil de la révolution russe. La persécution trouve son couronnement dans le procès de Königsberg, mais de plus, ce procès illumine comme un éclair toute la période de l’évolution historique depuis 1848, le renversement complet des rapports entre l’absolutisme russe et la réaction européenne. \emph{Tua res agitatur} ! s’écrie un ministre de la Justice prussien à l’adresse des classes dirigeantes allemandes, en montrant du doigt les fondements chancelants du régime tsariste. « L'établissement d’une république démocratique en Russie devrait avoir des répercussions sensibles en Allemagne », déclare à Königsberg le premier procureur Schulze. - «  Si la maison de mon voisin est en flammes, la mienne aussi est en danger » Et son adjoint Caspar souligne : « Les intérêts publics de l’Allemagne sont évidemment concernés au premier chef par le sort du bastion de l’absolutisme. Il est indubitable que les flammes d’un mouvement révolutionnaire peuvent facilement rejaillir sur l’Allemagne... » Ici, on peut enfin saisir de manière tangible comment la taupe de l’évolution historique effectue son travail de sape et change les choses du tout au tout : elle avait enterré le mot d’ordre de « rempart de la réaction européenne ». Maintenant, c’est la réaction européenne, et en premier lieu celle des junkers prussiens, qui est le rempart de l’absolutisme russe ; c’est grâce à elle qu’il tient encore debout et c’est en elle qu’il peut être mortellement touché. Les événements de la Révolution russe devaient confirmer cela.\par
La révolution fut écrasée. Mais, si on les examine un peu plus profondément, les raisons de cet échec provisoire sont instructives pour la position de la social-démocratie allemande au cours de la guerre actuelle. Deux causes peuvent expliquer la défaite du soulèvement russe de 1905-1906, malgré la richesse extraordinaire de sa force révolutionnaire, malgré la lucidité et la ténacité dont il a fait preuve. La première est une cause interne, elle réside dans la nature même de la révolution : dans l’immensité de son programme historique, dans la masse de problèmes économiques et politiques qu’elle a soulevés, tout comme la grande Révolution française l’avait fait un siècle auparavant, et qui, comme la question agraire, par exemple, sont absolument insolubles dans l’ordre social actuel ; dans la difficulté de créer une forme  moderne d’État assurant la domination de classe de la bourgeoisie contre la résistance contre-révolutionnaire de toute la bourgeoisie de l’Empire. De ce point de vue, la révolution russe a échoué parce qu’elle était une révolution prolétarienne avec des tâches bourgeoises, ou, si l’on veut, une révolution bourgeoise avec des moyens de lutte socialistes prolétariens, le heurt violent de deux époques qui s’entrechoquent dans la foudre et le tonnerre, le produit aussi bien du développement attardé des rapports de classes en Russie que de leur développement avancé en Europe occidentale ; de ce point de vue, sa défaite en 1906 n’était pas sa faillite, mais seulement la conclusion du premier chapitre, qui doit être suivi par d’autres chapitres avec la nécessité d’une loi naturelle. Quant à la deuxième cause, il s’agit à nouveau d’une cause extérieure, et c’est en Europe occidentale qu’il faut la chercher. Une fois de plus, la réaction européenne se précipitait au secours de son protégé en détresse. Pas encore avec la poudre et le plomb, quoiqu’en 1905 déjà, les « crosses de fusil allemandes brandies par des poings allemands » n’attendaient qu’un signe de Saint-Pétersbourg pour pénétrer dans la Pologne voisine. Mais avec des remèdes simples qui étaient tout aussi efficaces : elle donna un coup d’épaule au tsarisme par des subsides financiers et des alliances politiques. Avec de l’argent français, le tsarisme acheta la mitraille avec laquelle il abattit les révolutionnaires russes, et il reçut de l’Allemagne le réconfort politique et moral qui lui permit de reprendre contenance après les affronts que lui avaient infligés les torpilles japonaises et les poings des prolétaires russes. En 1910, l’Allemagne reçut officiellement le tsarisme russe à bras ouverts. En recevant ce monstre sanguinaire devant les portes de la capitale du Reich, l’Allemagne ne donnait pas seulement sa bénédiction à l’étranglement de la Perse, mais surtout au travail de bourreau de la contre-révolution russe ; c’était le banquet officiel de la « civilisation » allemande et européenne sur le prétendu tombeau de la Révolution russe. Et, chose étonnante, au moment même où elle assistait dans son propre pays à ce festin funèbre célébré sur les hécatombes de la révolution russe, la social-démocratie allemande garda un silence complet et elle avait totalement oublié le « testament de nos maîtres » de l’année 1848. Alors que maintenant, au début de la guerre, depuis que la police le permet, la plus petite feuille du parti se grise d’expressions sanglantes contre le bourreau de la liberté russe, en 1910, au moment où le bourreau lui-même était fêté à Potsdam, elle n’a pas soufflé mot, n’a pas fait entendre la moindre protestation, n’a pas publié le moindre article de solidarité avec la liberté russe, n’a pas introduit de veto contre le soutien de la contre-révolution russe ! Et cependant, ce voyage triomphal du tsar en Europe en 1910 a montré, on ne peut mieux, que les prolétaires russes qui furent assassinés ne sont pas seulement des victimes de la réaction de leur pays natal, mais aussi de la réaction d’Europe occidentale, et que, tout comme les combattants de mars 1848, ils ne se sont pas seulement fracassé le crâne contre la réaction dans leur propre pays, mais aussi contre son « rempart » à l’étranger.\par
Et pourtant, la source vive de l’énergie révolutionnaire du prolétariat russe est aussi inépuisable que la coupe des souffrances qu’il a endurée sous le double régime de knout du tsarisme et du capital. Après une période de croisade barbare de la contre-révolution, la fermentation révolutionnaire a recommencé. Depuis 1911, depuis le massacre de Lena, la masse ouvrière s’est relevée et elle a repris le combat, le flot a recommencé à monter et à écumer. D'après les communiqués officiels, les grèves économiques en Russie ont compté 46.623 ouvriers et 256.385 jours de grève en 1910, 96.730 ouvriers et 768.556 jours de grève en 1911, 98.771 ouvriers et 1.214.881 jours de grève dans les cinq premiers mois de 1912. Les grèves politiques de masse, les protestations et démonstrations rassemblèrent 1.005.000 ouvriers en 1912 et 1.272.000 en 1913. En 1914, le flot continuait à monter, avec un grondement sourd, et il se faisait de plus en plus menaçant. Le 22 janvier, pour célébrer l’anniversaire du début de la révolution, il y eut une grève de masse comprenant 200.000 ouvriers. En juin, tout comme avant l’éclatement de la révolution, un foyer révolutionnaire prit naissance dans le Caucase, à Bakou. 40.000 ouvriers firent la grève de masse. Les flammes en rejaillirent jusqu’à Saint-Pétersbourg : là-bas, le 17 juillet, 80.000 ouvriers se mirent en grève ; le 20 juillet, 200.000 ouvriers ; le 23 juillet, la grève générale commença à s’étendre à tout l’Empire russe, on dressa des barricades, la révolution était en marche... Encore quelques mois, et elle faisait certainement son apparition, drapeaux au vent. Encore quelques années, et elle pouvait peut-être paralyser le tsarisme au point qu’il n’aurait plus pu servir à la danse impérialiste de tous les Etats, prévue pour 1916. Cela aurait peut-être modifié toute la configuration de la politique mondiale et bouleversé tous les plans de l’impérialisme...\par
Mais c’est l’inverse qui s’est produit : la réaction allemande a bouleversé tous les plans du mouvement révolutionnaire russe. La guerre fut déclenchée par Vienne et Berlin, et elle a enseveli la Révolution russe sous ses décombres - peut-être à nouveau pour des années. Les « crosses de fusil allemandes » n’ont pas écrasé le tsarisme, mais ses opposants. Elles ont fourni au tsarisme la guerre la plus populaire que la Russie ait connue depuis un siècle. Cette fois, tout contribuait à auréoler le gouvernement russe d’un prestige moral : le fait, évident partout sauf en Allemagne, que la guerre avait été provoquée par Vienne et par Berlin, l’Union sacrée proclamée en Allemagne et le délire nationaliste qu’elle déchaînait, le sort de la Belgique, la nécessité de courir au secours de la République française - jamais l’absolutisme n’avait eu une position aussi favorable dans une guerre européenne. Le drapeau de la révolution qui symbolisait tant d’espoirs fut englouti dans les remous tumultueux de la guerre - mais il sombra avec honneur, et il ressortira de cette boucherie immonde pour flotter à nouveau, malgré les crosses de fusil allemandes, malgré la victoire ou la défaite du tsarisme sur les champs de bataille.\par
  Les rébellions nationales que l’on attendait en Russie n’ont pas eu lieu. Les minorités nationales se sont évidemment moins laissé leurrer par la mission libératrice des cohortes de Hindenburg que la social-démocratie allemande. Pratiques comme ils le sont, les Juifs pouvaient faire ce calcul simple sur leurs doigts : si les « poings allemands » n’avaient pas réussi une seule fois à « écraser » la réaction dans leur propre pays, s’ils permettaient l’existence du suffrage censitaire, ils étaient encore bien moins à même d’« écraser » l’absolutisme russe. Les Polonais, en proie au triple enfer de la guerre, ne pouvaient en vérité pas répondre à haute voix au message de salut plein de promesses de leurs « libérateurs » venus de Wreschen, où on inculquait le Notre-Père allemand aux enfants polonais en les marquant de raies sanglantes, ou à celui des membres des commissions de colonisation prussiennes - mais ils pouvaient avoir traduit dans leur for intérieur, en un polonais encore plus énergique, la sentence allemande de Goetz von Berlichingen. Tous : Polonais, Juifs et Russes avaient fait très tôt la simple constatation que les « crosses de fusil allemandes » avec lesquelles on leur fracassait le crâne ne leur apportaient pas la liberté, mais la mort.\par
La légende de libération forgée dans cette guerre par la social-démocratie allemande avec le testament de Marx est plus qu’une mauvaise plaisanterie : c’est un acte frivole. Pour Marx, la révolution russe était un tournant de l’histoire. Toutes les perspectives politiques et historiques qu’il traçait étaient liées à cette restriction : « pour autant que la révolution n’éclate pas entre-temps en Russie ». Marx croyait à la révolution russe et il l’attendait, même lorsqu’il n’avait encore devant les yeux qu’une Russie asservie. Entre-temps, la révolution s’était produite. Elle n’avait pas remporté la victoire du premier coup, mais on ne peut plus ne pas en tenir compte, elle est à l’ordre du jour, elle vient précisément de se relever. Et voilà que tout à coup, les sociaux-démocrates allemands reviennent avec les « crosses de fusil allemandes » et déclarent la Révolution russe nulle et non avenue, ils la rayent de l’histoire. Ils ont soudain ressorti les archives de 1848 : Vive la guerre contre la Russie ! Mais en 1848, il y avait la révolution en Allemagne, et en Russie une réaction désespérément figée. En 1914 par contre, la Russie avait la révolution dans le corps, alors que l’Allemagne était dirigée par le régime des junkers prussiens. Ce n’est pas à partir des barricades allemandes, comme Marx en 1848, mais directement de la cave de Pandour, où un petit lieutenant les tenait enfermés, que les « libérateurs de l’Europe » allemands se sont lancés dans leur mission civilisatrice contre la Russie ! Dans une étreinte fraternelle avec les junkers prussiens, qui sont le rempart le plus solide du tsarisme russe ; bras dessus bras dessous avec des ministres et des procureurs de Königsberg avec lesquels ils avaient conclu une Union sacrée - ils se sont élancés contre le tsarisme et ont fracassé leurs « crosses de fusil »... sur le crâne des prolétaires russes ! On peut difficilement imaginer une farce historique plus sanglante, une insulte plus brutale à la Révolution russe et au testament de Marx. Elle forme l’épisode le plus sombre de l’attitude politique de la social-démocratie durant cette guerre.\par
Car la libération de la civilisation européenne ne devait être qu’un épisode : très vite, l’impérialisme allemand abandonna ce masque encombrant et s’attaqua ouvertement à la France et surtout à l’Angleterre. Une partie de la presse du parti le suivit également dans ce revirement. Au lieu de s’en prendre au tsar sanglant, elle se mit à livrer la perfide Albion et son esprit de boutiquier au mépris général, et à libérer la civilisation européenne de la domination maritime de l’Angleterre après l’avoir délivrée de l’absolutisme russe. La situation affreusement embrouillée dans laquelle le parti s’était placé ne pouvait pas se manifester d’une manière plus éclatante que dans les efforts convulsifs de la meilleure partie de sa presse, laquelle, effrayée par ce front réactionnaire, s’efforçait par tous les moyens de ramener la guerre à son but initial, en insistant sur le « testament de nos maîtres » c’est-à-dire sur un mythe qu’elle-même, la social-démocratie, avait forgé ! « C'est le cœur lourd que j’ai dû mobiliser mon armée contre un voisin avec lequel elle a combattu en commun sur tant de champs de bataille. C'est avec une douleur sincère que je vois se briser une amitié loyalement respectée par l’Allemagne. » C'était clair, simple, et honnête. Le groupe et la presse sociaux-démocrates avaient transposé cela dans un article du \emph{Neue Rheinische Zeitung}. Mais lorsque la rhétorique des premières semaines de la guerre fut balayée par le prosaïsme lapidaire de l’impérialisme, l’unique et faible explication de l’attitude de la social-démocratie allemande s’en alla en fumée.
\section[{La fin de la lutte des classes}]{La fin de la lutte des classes}\renewcommand{\leftmark}{La fin de la lutte des classes}

\noindent L'autre aspect de l’attitude de la social-démocratie était l’acceptation officielle de l’Union sacrée, c’est-à-dire la suspension de la lutte de classes pour la durée de la guerre. La déclaration du groupe lue au Reichstag le 4 août fut même le premier acte de cet abandon de la lutte de classes : le texte était d’accord à l’avance avec les députés du gouvernement et des partis bourgeois, l’acte solennel du 4 août était un numéro patriotique préparé en coulisse qui était destiné au peuple et à l’étranger, et dans lequel la social-démocratie jouait déjà à côté des autres participants le rôle qu’elle avait adopté.\par
Le vote des crédits par le groupe parlementaire montra l’exemple à toutes les instances dirigeantes du mouvement ouvrier. Les chefs syndicaux firent aussitôt cesser toutes les luttes de salaires et ils communiquèrent officiellement leur position aux entrepreneurs en invoquant les devoirs de l’Union sacrée. La lutte contre l’exploitation capitaliste fut spontanément interrompue pour toute la durée de la guerre. Ces mêmes chefs syndicaux prirent l’initiative de fournir aux agriculteurs de la main-d’œuvre des villes, de manière que la rentrée des récoltes ne soit pas interrompue. La direction du mouvement des femmes socialistes proclama l’union avec les femmes de la bourgeoisie et forma avec elles un « service national des  femmes » de sorte que la part la plus importante des effectifs du parti restée au pays après la mobilisation ne s’occupait pas de faire de l’agitation sociale-démocrate, mais était enrôlée dans des bonnes oeuvres d’intérêt national : distribuer de la soupe, donner des conseils, etc. Sous la loi des socialistes, le parti avait le plus souvent utilisé les élections parlementaires pour propager ses idées et affirmer sa position en dépit de tous les états de siège et des persécutions dont était l’objet la presse social-démocrate. Maintenant, au cours des secondes élections parlementaires pour le Reichstag, les diètes locales et les représentations communales, la social-démocratie renonça officiellement à toute lutte électorale, c’est-à-dire à toute agitation et à toute discussion idéologique dans le sens de la lutte de classe prolétarienne, et réduisit les élections à leur simple contenu bourgeois : amasser le plus de mandats possible, sur lesquels elle se mettait d’accord à l’amiable avec les partis bourgeois. Le vote du budget par les députés sociaux-démocrates dans les diètes locales et dans les représentations communales - à l’exception de la diète prussienne et de la diète d’Alsace-Lorraine -, accompagné d’un appel solennel à l’Union sacrée, souligna la rupture brutale avec la pratique d’avant le début de la guerre. La presse sociale-démocrate, à quelques rares exceptions près, exaltait tout haut le principe de l’unité nationale dans l’intérêt vital du peuple allemand. Au moment de la déclaration de guerre, elle mit même ses lecteurs en garde contre les retraits des sommes déposées dans les caisses d’épargne ; par là, elle contribua grandement à empêcher des troubles dans la vie économique du pays et elle permit aux fonds des caisses d’épargne de servir aux emprunts de guerre de façon notable ; elle conseillait les femmes prolétariennes de ne pas informer leurs maris envoyés au front de la misère où elles se trouvaient ainsi que leurs enfants, et de ne pas les mettre au courant de l’insuffisance des approvisionnements fournis par l’État, mais leur suggérait plutôt de « produire un effet apaisant et exaltant » sur les combattants en leur dépeignant les charmes du bonheur familial et « en leur décrivant avec bienveillance l’aide dont elles avaient bénéficié jusqu’ici \footnote{Voir l’article de l’organe du parti à Nuremberg, reproduit dans le Hamburger Echo du 6 octobre 1914.} ».\par
Elle louait le travail éducateur du mouvement ouvrier moderne comme fournissant une aide précieuse à la conduite de la guerre, comme par exemple dans ce morceau classique :\par

\begin{quoteblock}
 \noindent « C'est dans le besoin qu’on reconnaît ses vrais amis. Ce vieux proverbe se confirme à l’heure présente. En butte à tant de vexations et de tracasseries, les sociaux-démocrates se lèvent comme un seul homme pour défendre la patrie, et les centrales syndicales allemandes, à qui on a si souvent mené la vie dure en Allemagne prussienne, annoncent toutes unanimement que leurs meilleurs hommes se trouvent sous les drapeaux. Même des journaux d’entreprise du genre du Generalanzeiger, annoncent ce fait et ajoutent qu’ils sont persuadés que " ces gens " accompliront leur devoir comme les autres et que là où ils se trouveront, les coups tomberont peut-être le plus dru. »\par
 « Quant à nous, nous sommes persuadés que, grâce à leur instruction, nos syndiqués peuvent faire bien mieux que " rentrer dedans ". Avec les armées de masse modernes, les généraux n’ont pas la tâche facile pour mener la guerre : les obus modernes d’infanterie qui permettent de toucher une cible jusqu’à 3.000 mètres et avec précision jusqu’à 2.000 mètres font qu’il est tout à fait impossible aux chefs d’armée de faire avancer de grands corps de troupes en colonne de marche serrée. C'est pourquoi il faut au préalable " s’étirer ", et cet étirement exige à son tour un nombre beaucoup plus grand de patrouilles et une grande discipline et une grande clarté de jugement, aussi bien de la part des détachements que des hommes isolés, et c’est là qu’on voit quel rôle éducateur ont joué les syndicats et à quel point on peut compter sur cette éducation dans des jours aussi difficiles que ceux-ci. Le soldat russe et le soldat français peuvent bien accomplir des prodiges de bravoure, mais pour ce qui est de la réflexion froide et calme, le syndiqué allemand les surpassera. En plus de cela, il y a le fait que, dans les zones frontières, les gens organisés connaissent souvent tous les recoins du terrain comme leur poche, et que beaucoup de fonctionnaires syndicaux possèdent aussi leur connaissance des langues, etc. Ainsi donc, si on a pu dire en 1866 que la marche en avant des troupes prussiennes était une victoire du maître d’école, il faudra parler cette fois-ci d’une victoire du fonctionnaire syndical. » (Frankfurter Volksstimme du 18 août 1914).
\end{quoteblock}

\noindent L'organe théorique du parti, \emph{Neue Zeit} (numéro 23 du 25 septembre 1914), déclarait :\par

\begin{quoteblock}
 \noindent « Tant que la question est posée simplement sous la forme victoire ou défaite, on fait passer au second plan toutes les autres, y compris celle de la finalité de la guerre. Donc passent au second plan, à plus forte raison au sein de l’armée et de la population, toutes les différences entre les partis, les classes, les nations. »
\end{quoteblock}

\noindent Et dans le numéro 8 (du 27 novembre 1914), la même revue \emph{Neue Zeit} écrit dans un article intitulé « Les limites de l’Internationale » :\par

\begin{quoteblock}
 \noindent « La guerre mondiale divise les socialistes en camps différents et essentiellement en différents camps nationaux. \emph{L'Internationale est incapable d’empêcher cela}, c’est-à-dire qu’elle n’est  pas un instrument efficace en temps de guerre ; l’Internationale est essentiellement un instrument valable en temps de paix. »
\end{quoteblock}

\noindent Sa grande « mission historique » serait « la lutte pour la paix, \emph{la lutte de classes en temps de paix} ».\par
Ainsi donc, la social-démocratie déclare qu’à la date du 4 août 1914, et jusqu’à la conclusion future de la paix, la lutte de classes n’existe plus. Ainsi, dès qu’a tonné en Belgique le premier coup des canons de Krupp, l’Allemagne s’est métamorphosée en un pays de cocagne, le pays de la solidarité des classes et des harmonies sociales.\par
Mais comment peut-on, à la vérité, imaginer ce miracle ? On sait bien que la lutte de classes n’est nullement une invention, une création délibérée de la social-démocratie que celle-ci pourrait à son gré et de sa propre initiative supprimer pendant certaines périodes. La lutte de classes du prolétariat est plus ancienne que la social-démocratie ; c’est un produit élémentaire de la société de classes qui se déchaîne avec l’avènement du capitalisme en Europe. Ce n’est pas la social-démocrate qui a poussé le prolétariat moderne à la lutte de classes, c’est au contraire le prolétariat qui a suscité la social-démocratie afin qu’elle coordonne la lutte des fractions diverses, dans l’espace et dans le temps, de la lutte de classes et qu’elle fasse prendre conscience à tous du but à atteindre. Qu'est-ce que la déclaration de guerre a changé à cela ? Est-ce que par hasard la propriété privée, l’exploitation capitaliste, la domination de classe ont cessé ? Est-ce que par hasard, dans un accès de patriotisme, les possédants auraient déclaré : « Aujourd’hui, étant donné la guerre et tant qu’elle durera, les moyens de production : terrains, fabriques, usines, nous les remettons entre les mains de la communauté, nous renonçons à en tirer profit pour nous seuls, nous abolissons tous les privilèges politiques et nous les sacrifions sur l’autel de la patrie aussi longtemps qu’elle sera en danger » ? Hypothèse tout à fait absurde qui fait penser aux histoires que l’on raconte aux petits enfants. Et pourtant, ce serait la seule prémisse qui, logiquement, aurait pu entraîner la classe ouvrière à déclarer : « La lutte de classes est interrompue. » Mais bien sûr, rien de tel ne s’est passé. Au contraire, tous les rapports de propriété, l’exploitation, la domination de classe et même l’absence de droits politiques pour le prolétariat, sous les diverses formes qu’elle prend dans notre Reich germano-prussien, sont restés intacts. Le tonnerre des canons en Belgique et en Prusse orientale n’a pas changé d’un iota la structure économique, sociale et politique de l’Allemagne.\par
La suppression de la lutte de classes a donc été une mesure parfaitement unilatérale. Tandis que pour la classe ouvrière, « l’ennemi de l’intérieur », c’est-à-dire l’exploitation et l’oppression capitalistes, a continué d’exister, les dirigeants de la classe ouvrière : la social-démocratie et les syndicats, dans un mouvement de magnanimité patriotique, ont livré sans combat la classe ouvrière à son ennemi pour toute la durée de la guerre. Tandis que les classes dominantes restent sur le pied de guerre, en possession de tous leurs droits de propriétaires et de maîtres, la social-démocratie a ordonné au prolétariat de « désarmer ».\par
Le miracle de l’harmonie des classes, de la fraternisation de toutes les couches sociales, avait déjà un précédent dans une société bourgeoise : les événements de 1848 en France.\par

\begin{quoteblock}
 \noindent « Dans l’esprit des prolétaires - écrit Marx dans son ouvrage les luttes de classes en France - qui confondaient en général l’aristocratie financière avec la bourgeoisie, dans l’imagination de braves républicains qui niaient l’existence même des classes ou l’admettaient tout au plus comme une conséquence de la monarchie constitutionnelle, dans les phrases hypocrites des fractions bourgeoises jusque-là exclues du pouvoir, le pouvoir de la bourgeoisie se trouvait abolie avec l’instauration de la République. Tous les royalistes se transformèrent alors en républicains et tous les millionnaires de Paris en ouvriers. Le mot qui répondait à cette suppression imaginaire des rapports de classe, c’était la fraternité, la fraternisation et la fraternité universelle. Cette abstraction débonnaire des antagonismes de classes, cet équilibre sentimental des intérêts de classe contradictoires, cette exaltation enthousiaste au-dessus de la lutte de classes, la fraternité, telle fut vraiment la devise de la révolution de Février. [...] Le prolétariat de Paris se laissa aller à cette généreuse ivresse de fraternité. [...] Le prolétariat parisien, qui reconnaissait dans la République sa propre création, acclamait naturellement chaque acte du Gouvernement provisoire qui lui permettait de prendre pied plus facilement dans la société bourgeoise. Il se laissa docilement employer par Caussidière à des fonctions de police pour protéger la propriété à Paris, de même qu’il laissa régler à l’amiable les conflits de salaires entre ouvriers et maîtres par Louis Blanc. Il mettait son point d’honneur à maintenir immaculé, aux yeux de l’Europe, l’honneur bourgeois de la République. »
\end{quoteblock}

\noindent En février 1848, le prolétariat parisien avait lui aussi suspendu naïvement la lutte de classes, mais, bien entendu, il venait d’écraser la monarchie de Juillet par son action révolutionnaire et venait d’imposer la république. Le 4 août 1914, ce fut la révolution de Février mise sur la tête : la suppression de la lutte de classes non pas sous la république, mais sous la monarchie militaire, non pas après une victoire du peuple sur la réaction, mais après une victoire de la réaction sur le peuple, non pas par la proclamation de « liberté, égalité, fraternité », mais par la proclamation de l’état de siège, l’étranglement de la liberté de presse et la suppression de la Constitution ! Le gouvernement proclama solennellement l’Union sacrée et reçut de tous les partis l’engagement de la respecter scrupuleusement. Mais, en politicien expérimenté, il ne se fiait pas  aux promesses, et assura l’Union sacrée par les moyens tangibles de la dictature militaire. Cela aussi, la social-démocratie l’accepta sans broncher. Dans sa déclaration au Reichstag le 4 août, de même que dans celle du 2 décembre, le groupe parlementaire ne prenait pas la moindre précaution contre la gifle de l’état de siège. En plus de l’Union sacrée et des crédits de guerre, la social-démocratie approuvait par son silence l’état de siège qui la livrait pieds et poings liés au bon vouloir des classes dirigeantes. Elle admettait du même coup que l’état de siège, le musellement du peuple et la dictature militaire étaient des mesures nécessaires à la défense de la patrie. Mais l’état de siège n’était dirigé contre personne d’autre que contre la social-démocratie. C'était uniquement du côté social-démocrate que l’on pouvait s’attendre à rencontrer des difficultés, de la résistance, des protestations contre la guerre. Au moment même où, avec l’approbation de la social-démocratie, on proclamait l’Union sacrée, donc la suppression des oppositions de classes, la social-démocratie elle-même fut déclarée en état de siège, on proclamait le combat contre la classe ouvrière sous sa forme la plus aiguë : sous la forme de la dictature militaire. Comme fruit de sa capitulation, la social-démocratie reçut ce que, si elle avait pris le parti de résister, elle aurait dû subir dans le pire des cas, celui d’une défaite : l’état de siège ! La déclaration solennelle du groupe parlementaire en appelle, pour justifier son vote des crédits militaires, au principe socialiste du droit des nations à disposer d’elles-mêmes. La première étape de ce « droit » de la nation allemande à disposer d’elle-même, au cours de cette guerre, fut la camisole de force de l’état de siège dans laquelle on fourrait la social-démocratie ! Assurément, l’histoire a rarement vu un parti se tourner à ce point en dérision.\par
En acceptant le principe de l’Union sacrée, la social-démocratie a renié la lutte de classes pour toute la durée de la guerre. Mais par là, elle reniait le fondement de sa propre existence, de sa propre politique. Dans chacune de ses fibres, est-elle autre chose que lutte de classes ? Quel rôle peut-elle jouer maintenant pendant la durée de la guerre, après avoir abandonné son principe vital : la lutte de classes ? En reniant la lutte de classes, la social-démocratie s’est donné à elle-même son congé pour toute la durée de la guerre en tant que parti politique actif, en tant que représentant de la classe ouvrière. Mais par là, elle s’est privée de son arme la plus importante : la critique de la guerre du point de vue particulier de la classe ouvrière. Elle a abandonné la « défense nationale » aux classes dominantes, se bornant à placer la classe ouvrière sous leur commandement et à assurer le calme pendant la durée de l’état de siège, c’est-à-dire qu’elle joue le rôle de gendarme de la classe ouvrière.\par
Mais la social-démocratie a, par son attitude, compromis très gravement la cause de la liberté allemande pour une période qui déborde singulièrement la durée de la guerre actuelle, cause qui est actuellement confiée, si l’on en croit la déclaration du groupe parlementaire, aux canons de Krupp. Dans les cercles dirigeants de la social-démocratie, on compte beaucoup sur le fait que, après la guerre, la classe ouvrière verra s’élargir considérablement les libertés démocratiques et qu’on lui assurera l’égalité des droits avec la bourgeoisie, en récompense de son attitude patriotique pendant la guerre. Mais jamais encore dans l’histoire les classes dominantes n’ont concédé aux classes dominées des droits politiques à titre de pourboire en raison de l’attitude adoptée par ces dernières pour complaire aux classes dominantes. Au contraire, l’histoire est pleine d’exemples de dirigeants manquant brutalement de parole, même dans le cas où des promesses solennelles avaient été faites avant une guerre. En réalité, la social-démocratie ne garantit pas, par son comportement, l’élargissement des libertés politiques en Allemagne dans l’avenir, mais ébranle les libertés qui existaient avant la guerre. La manière avec laquelle l’état de siège et la suppression de la liberté de la presse, de la liberté d’association et de la vie publique sont supportés en Allemagne depuis des mois sans le sans le moindre combat, et sont même en partie approuvés du côté social-démocrate \footnote{La \emph{Chemnitzer Volkstimme} écrivait le 21 octobre 1914 : « En tout cas, la censure militaire en Allemagne est dans l’ensemble plus convenable et plus raisonnable qu’en France ou en Angleterre. Ies hauts cris au sujet de la censure, qui cachent souvent l’absence d’une position cohérente sur le problème de la guerre, ne font qu’aider les ennemis de l’Allemagne à répandre le mensonge selon lequel l’Allemagne serait une seconde Russie. Celui qui croit sérieusement ne pas pouvoir écrire selon ses opinions sous la censure militaire actuelle, il n’a qu’à déposer la la plume et se taire. »} - tout cela est sans exemple dans l’histoire de la société moderne.\par
En Angleterre règne une complète liberté de presse, en France la liberté de la presse est loin d’être aussi muselée qu’en Allemagne. Dans aucun pays, l’opinion publique n’a disparu aussi complètement comme en Allemagne pour être remplacée par la simple « opinion » officielle, c’est-à-dire par les ordres du gouvernement. Même en Russie, on ne connaît que les ravages du crayon rouge du censeur ; on n’y connaît absolument pas la disposition selon laquelle la presse de l’opposition doit imprimer tels quels des articles que lui remet le gouvernement et doit, dans ses propres articles, soutenir certaines conceptions qui lui sont dictées et imposées par les autorités gouvernementales au cours d’« entretiens confidentiels avec la presse ». Allemagne même, on n’a rien connu de comparable à la situation actuelle au cours de la guerre de 1870. La presse jouissait d’une liberté illimitée et, à la grande colère de Bismarck, les événements de la guerre faisaient l’objet de critiques parfois très vives et de conflits d’opinions très animés, notamment au sujet des objectifs de guerre, des questions d’annexion et des questions constitutionnelles. Lorsque Johann Jakoby fut arrêté, une vague d’indignation déferla sur toute l’Allemagne, et Bismarck lui-même désavoua cet attentat de la réaction comme étant une grave maladresse. Telle était la situation en Allemagne après que Bebel et Liebknecht, au nom de la classe ouvrière allemande, eurent nettement refusé de s’associer au patriotisme délirant qui régnait alors. Il fallut attendre la social-démocratie patriotique et ses 4 millions et  demi d’électeurs pour assister à cette fête de réconciliation attendrissante de l’Union sacrée et à l’approbation des crédits de guerre par le groupe social-démocrate, à la suite de quoi on infligea à l’Allemagne la dictature militaire la plus dure qu’un peuple majeur ait jamais eu à subir. Qu'une chose pareille soit possible actuellement en Allemagne, qu’elle soit acceptée non seulement par la presse bourgeoise, mais par la presse sociale-démocrate qui est si développée et si influente, d’une manière résignée, et sans la moindre résistance notable - ce fait en dit long, hélas, sur le destin de la liberté allemande. Cela prouve que, dans la société allemande, les libertés politiques ne reposent sur aucun fondement véritable, puisqu’on peut les enlever sans difficulté ni anicroche. N'oublions pas que la portion congrue des droits politiques qui existait dans l’Empire allemand avant la guerre n’était pas, comme en France ou en Angleterre, le fruit de luttes révolutionnaires importantes et répétées, qu’elle n’était pas ancrée dans la vie du peuple par des traditions, mais qu’elle était le cadeau de la politique de Bismarck après une contre-révolution victorieuse qui avait duré plus de deux ans. La Constitution allemande n’avait pas mûri sur les champs de bataille de la révolution, mais dans le jeu diplomatique de la monarchie militaire prussienne, c’était le ciment avec lequel fut construit l’Empire allemand. Les dangers que courait le « développement libre » de l’Allemagne ne se trouvaient donc pas en Russie, comme le pensait le groupe parlementaire, mais bien en Allemagne même. Ils résidaient dans cette origine contre-révolutionnaire particulière de la Constitution allemande, dans ces groupes réactionnaires de la société allemande qui, depuis la fondation de l’Empire, n’ont pas cessé de mener une guerre silencieuse contre la chétive « liberté allemande », à savoir : les junkers prussiens, les provocateurs de la grosse industrie, le Zentrum archi-réactionnaire, le libéralisme allemand en lambeaux, le régime personnel, et enfin, ce à quoi, tous ensemble, ils ont donné naissance : la domination du sabre, le cours de Saverne qui, juste avant la guerre, fêtait ses victoires en Allemagne. Il y a, il est vrai, une excuse vraiment libérale pour expliquer le calme de cimetière qui règne actuellement sur l’Allemagne : il ne s’agirait que d’une renonciation « provisoire » pour la durée de la guerre. Mais un peuple politiquement mûr peut aussi peu renoncer « provisoirement » à ses droits politiques qu’un homme vivant ne peut « renoncer » à respirer. Un peuple qui admet par son attitude que l’état de siège est une chose nécessaire pendant la guerre admet du même coup que la liberté politique n’est tout compte fait, pas si indispensable que cela. En se résignant à l’état de siège actuel - et elle ne faisait rien d’autre en approuvant inconditionnellement les crédits de guerre et en admettant le principe de l’Union sacrée -, la social-démocratie ne peut qu’exercer un effet démoralisateur sur les masses populaires qui sont le seul soutien de la Constitution, alors que dans une mesure égale elle stimule et encourage le parti de la réaction, qui est l’ennemi de la Constitution.\par
En renonçant à la lutte de classes, notre parti s’est du même coup ôté la possibilité d’exercer une influence réelle sur la durée de la guerre et sur la tournure que pourrait prendre la conclusion de la paix. Et par là, sa propre déclaration officielle se retourne contre lui. Un parti qui se déclarait solennellement contre toutes les annexions - et les annexions sont la conséquence logique de la guerre impérialiste, dès l’instant où il y a des succès militaires - se dessaisissait en même temps de toutes ses armes et de tous les moyens qui lui auraient permis de mobiliser les masses populaires et l’opinion publique, de les rallier à son point de vue, et, par leur intermédiaire, d’exercer une pression efficace et de contrôler la guerre pour contribuer au rétablissement de la paix. Au contraire : bien loin d’exercer un contrôle, la social-démocratie, en adoptant la politique d’Union sacrée, assurait au militarisme la tranquillité sur ses arrières et lui permettait d’aller de l’avant sans tenir compte d’autres intérêts que ceux des classes dominantes, elle lui laissait déchaîner sans entraves ses instincts impérialistes innés qui aspirent précisément à des annexions, et ne peuvent conduire qu’à des annexions. Autrement dit, en acceptant le principe de l’Union sacrée, et en désarmant politiquement la classe ouvrière, la social- démocratie a condamné sa propre protestation solennelle contre les annexions à rester lettre morte.\par
Mais, ce faisant, elle a obtenu encore autre chose : la prolongation de la guerre. Et ici, on peut toucher du doigt le dangereux piège que recèle, pour la politique du prolétariat, le dogme actuellement admis selon lequel nous ne pouvions lutter contre la guerre qu’aussi longtemps que celle-ci menace ; une fois que la guerre est là, le rôle de la politique sociale-démocrate serait terminé ; la seule question serait alors victoire ou défaite, autrement dit, la lutte de classes cesserait pour la durée de la guerre. En réalité, c’est après le déclenchement de la guerre que la tâche la plus importante de la politique sociale-démocrate commence. On lit dans la résolution adoptée à Stuttgart en 1907 par le Congrès de l’Internationale et confirmée à Bâle en 1912, résolution qui avait été adoptée à l’unanimité par les représentants du parti et des syndicats allemands :\par

\begin{quoteblock}
 \noindent « Au cas où la guerre éclaterait néanmoins, c’est le devoir de la social-démocratie d’agir pour la faire cesser promptement et de s’employer, de toutes ses forces, à exploiter la crise économique et politique provoquée par la guerre pour mettre en mouvement le peuple et hâter de la sorte l’abolition de la domination capitaliste. »
\end{quoteblock}

\noindent Or, qu’a fait la social-démocratie pendant cette guerre ? Exactement le contraire de ce qu’ordonnaient les congrès de Stuttgart et de Bâle. En votant les crédits, en maintenant la politique d’Union sacrée, elle s’emploie à empêcher par tous les moyens la crise économique et politique, à empêcher que la guerre n’amène les masses à se mettre en mouvement. De toutes ses forces, elle s’emploie à sauver la  société capitaliste de sa propre anarchie consécutive à la guerre, donc elle s’emploie à prolonger la guerre indéfiniment et à accroître le nombre de ses victimes.\par
Mais il paraît qu’il n’y aurait de toute façon pas eu un homme de moins tué sur le champ de bataille si la social-démocratie n’avait pas voté les crédits de guerre : voilà le raisonnement que l’on entend parmi nos parlementaires. Et la presse de notre parti défend généralement le point de vue suivant : nous devions participer à la défense du pays et la soutenir précisément pour réduire au minimum le nombre des victimes sanglantes de la guerre, dans l’intérêt de notre peuple. Mais la politique poursuivie par la social-démocratie a abouti exactement à l’inverse : c’est l’attitude « patriotique » de la social-démocratie, c’est l’Union sacrée assurée à l’arrière qui ont permis à la guerre impérialiste de déchaîner ses furies sans être inquiétée. Jusqu’alors, la peur de l’agitation intérieure, de la fureur du peuple misérable était le cauchemar perpétuel des classes dirigeantes et, du même coup, le garde-fou le plus efficace à leurs désirs de guerre. On connaît le mot de von Bülow, qui disait que c’était essentiellement par crainte de la social-démocratie que l’on s’efforçait autant que possible de différer toute guerre. Rohrbach écrit à la page VII de son livre \emph{la Guerre et la politique allemande} : « Si des catastrophes naturelles n’interviennent pas, la seule chose qui puisse contraindre l’Allemagne à la paix, c’est la famine des sans-pain. » Il songeait évidemment à une famine qui s’exprime, qui se met nettement en évidence, et qui oblige les classes dirigeantes à la prendre en considération. Ecoutons enfin ce que dit un homme militaire éminent, un théoricien de la guerre, le général von Bernhardi. Dans son grand ouvrage, \emph{De la guerre actuelle}, il écrit :\par

\begin{quoteblock}
 \noindent « Ainsi donc, les armées de masse modernes rendent la conduite de la guerre plus difficile à tous les points de vue. Mais en outre, elles présentent en elles-mêmes et pour elles-mêmes un facteur de danger qu’il ne faut pas sous-estimer. »\par
 « Le mécanisme d’une telle armée est si colossal et si compliqué qu’il ne peut rester opérationnel et maîtrisable que si ses rouages fonctionnent, au moins dans l’ensemble, d’une manière sûre et si on évite au maximum les fortes secousses morales. A vrai dire, on ne doit pas espérer l’élimination complète de tels phénomènes dans une guerre mouvementée, et il ne faut pas s’attendre à une campagne limpide et victorieuse. Toutefois, on peut les surmonter s’ils se manifestent sur une échelle réduite. Mais si de grandes masses échappent au contrôle du haut commandement, si elles sont prises de panique, si l’intendance fait défaut sur une grande échelle, si l’esprit d’insubordination s’empare des troupes, dans ce cas, de telles masses non seulement ne sont plus capables de résister à l’ennemi, mais elles deviennent un danger pour elles-mêmes et pour le commandement de l’armée ; elles font sauter les liens de la discipline, troublent arbitrairement le cours des opérations et placent ainsi le haut commandement devant des tâches qu’il n’est pas en mesure d’accomplir. »\par
 « La guerre menée avec des armées de masse modernes est donc, en tout état de cause, un jeu risqué, qui est excessivement éprouvant pour les forces personnelles et financières de l’Etat. Dans de telles conditions, il n’est que naturel que des dispositions soient prises partout pour mettre rapidement un terme à la guerre lorsqu’elle éclate et pour supprimer rapidement l’énorme tension que provoque la levée en masse de nations entières. »
\end{quoteblock}

\noindent Des politiciens bourgeois et des experts militaires considéraient donc la guerre moderne menée avec des armées de masse comme un « jeu risqué », et c’était là la raison essentielle qui pouvait faire hésiter les maîtres actuels du pouvoir à déclencher la guerre et les amener à tout faire pour qu’elle s’achève rapidement au cas où elle éclaterait. L'attitude de la social-démocratie au cours de la guerre actuelle - attitude qui, à tous points de vue, a eu pour effet d’amortir « l’énorme tension » - a dissipé leurs inquiétudes, elle a abattu les seules digues qui s’opposaient au torrent déchaîné du militarisme. Il s’est produit quelque chose que ni un Bernhardi, ni aucun homme politique bourgeois n’auraient jamais osé espérer : dans le camp de la social-démocratie a retenti le mot d’ordre de « tenir bon », c’est-à-dire de continuer le carnage. Et ainsi, ces milliers de victimes qui tombent depuis des mois et dont les corps couvrent les champs de bataille, nous les avons sur la conscience.
\section[{Invasion et lutte des classes}]{Invasion et lutte des classes}\renewcommand{\leftmark}{Invasion et lutte des classes}

\noindent Mais malgré tout - maintenant que nous n’avons pas pu empêcher que la guerre n’éclate, maintenant que la guerre est quand même une réalité, que le pays se trouve devant une invasion ennemie - devons-nous laisser notre propre pays sans défense, l’abandonner à l’ennemi - les Allemands abandonner leur pays aux Russes, les Français et les Belges aux Allemands, les Serbes aux Autrichiens ? Est-ce que le  principe socialiste du droit des nations à disposer d’elles-mêmes ne dit pas que chaque peuple a le droit et le devoir de protéger sa liberté et son indépendance ? Quand la maison est en flammes, ne doit-on pas avant tout éteindre le feu, au lieu de rechercher celui qui l’a mis ? Cet argument de la « maison en flammes » a joué un grand rôle dans l’attitude des socialistes, en Allemagne comme en France, et il a également fait école dans des pays neutres ; traduit en hollandais, cela devient : quand le bateau coule, ne doit-on pas avant tout chercher à boucher les fuites ?\par
Assurément, un peuple qui capitule devant l’ennemi venu de l’extérieur est un peuple indigne, tout comme est indigne le parti qui capitule devant l’ennemi intérieur. Les pompiers de la « maison en flammes » n’ont oublié qu’une chose : c’est que, dans la bouche d’un socialiste défendre la patrie ne signifie pas servir de chair à canon sous les ordres de la bourgeoisie impérialiste. Tout d’abord, en ce qui concerne « l’invasion », est-ce vraiment là l’épouvantail devant lequel toute lutte de classe à l’intérieur du pays devrait disparaître, comme envoûtée et paralysée par un pouvoir surnaturel ? D'après la théorie policière du patriotisme bourgeois et de l’état de siège, toute lutte de classes est un crime contre les intérêts de la « défense nationale », parce que, selon cette théorie, la lutte de classes met en péril et affaiblit la force armée de la nation. La social-démocratie officielle s’est laissé impressionner par ces hauts cris. Et pourtant, l’histoire moderne de la société bourgeoise montre sans cesse que, pour la bourgeoisie, l’invasion ennemie n’est pas la plus abominable de toutes les horreurs, comme elle la dépeint aujourd’hui, mais un moyen éprouvé dont elle se sert volontiers pour lutter contre l’« ennemi intérieur ». Est-ce que les Bourbons et les aristocrates de France n’ont pas fait appel à l’invasion étrangère contre les Jacobins ? Est-ce que la contre-révolution de l’Autriche et des États pontificaux n’a pas appelé en 1849 l’invasion française contre Rome, et l’invasion russe contre Budapest ? Est-ce que le « parti de l’ordre » en France n’a pas ouvertement brandi la menace d’une invasion des Cosaques, pour faire fléchir l’Assemblée nationale ? Et est-ce que par le fameux traité du 18 mai 1871 - conclu entre Jules Favre, Thiers et Cie. et Bismarck - on n’est pas convenu de la remise en liberté de l’armée bonapartiste et du soutien direct des troupes prussiennes en vue d’écraser la Commune de Paris ? Pour Karl Marx, cette expérience historique suffisait pour dénoncer, il y a quarante-cinq ans déjà, les « guerres nationales » des États bourgeois modernes comme une escroquerie. Dans sa fameuse Adresse du Conseil général de l’Internationale, il dit :\par

\begin{quoteblock}
 \noindent « Qu'après la plus terrible guerre des temps modernes, l’armée victorieuse et l’armée vaincue s’unissent pour massacrer en commun le prolétariat, cet événement inouï prouve, non pas comme le croit Bismarck, l’écrasement définitif de la nouvelle société montante, mais bien l’effondrement de la vieille société bourgeoise. Le plus haut effort d’héroïsme dont la vieille société soit encore capable est une guerre nationale ; et il est maintenant prouvé qu’elle est une pure mystification des gouvernements, destinée à retarder la lutte des classes, et qui est jetée de côté, aussitôt que cette lutte de classes éclate en guerre civile. La domination de classe ne peut plus se cacher sous un uniforme national, les gouvernements nationaux font bloc contre le prolétariat ! »
\end{quoteblock}

\noindent L'invasion et la lutte de classes ne sont donc pas contradictoires dans l’histoire bourgeoisie comme on le lit dans les légendes officielles, mais l’une se sert de l’autre pour s’exprimer. Si pour les classes dirigeantes, l’invasion représente un moyen éprouvé de lutter contre la lutte des classes, de même pour les classes révolutionnaires, la lutte de classe la plus violente s’est toujours révélée le meilleur moyen de lutter contre l’invasion. Au seuil des temps modernes, l’histoire turbulente des villes, et spécialement des villes italiennes, agitées par d’innombrables bouleversements intérieurs et par des hostilités extérieures, l’histoire de Florence, de Milan avec son combat séculaire contre les Hohenstaufen, montre déjà que la violence et le tumulte des luttes de classes intérieures non seulement n’affaiblissent pas la capacité de résistance de la société aux dangers extérieurs, mais qu’au contraire sa force se trempe dans le feu de ces luttes et qu’elle devient capable de braver tout affrontement avec un ennemi venu de l’extérieur. Mais l’exemple le plus prestigieux de tous les temps, c’est la grande Révolution française. Si jamais l’expression « des ennemis de tous côtés » a eu un sens, c’est bien pour la France de 1793, et pour le cœur de cette France, Paris. Si Paris et la France n’ont pas été submergés par le flot de l’Europe coalisée, des invasions qui déferlaient de tous côtés, et si au contraire ils leur ont opposé une gigantesque résistance, alors que le péril ne faisait qu’augmenter et que les attaques ennemies se multipliaient, s’ils ont mis en déroute chaque nouvelle coalition par le mirage à chaque fois renouvelé d’une ardeur combative inépuisable, ce n’était dû qu’aux forces illimitées que déchaînait le grand règlement de comptes des classes à l’intérieur de la société. Aujourd’hui, avec une perspective d’un siècle, nous voyons clairement que seule l’expression vive de ce règlement de comptes, seule la dictature du peuple parisien et son radicalisme brutal, ont été capables de tirer de la nation les moyens et les forces suffisantes pour défendre et affirmer la société bourgeoise qui venait à peine de naître contre un monde plein d’ennemis : contre les intrigues de la dynastie, les machinations traîtresses des aristocrates, les manigances du clergé, la rébellion de Vendée, la trahison des généraux, la résistance de soixante départements et chefs-lieux de province, et contre les armées et les flottes réunies de la coalition monarchique européenne. Une expérience séculaire démontre, par conséquent, que ce n’est pas l’état de siège, mais la lutte de classes pleine d’abnégation qui éveille le respect de soi-même, l’héroïsme et la force morale des masses populaires, qui est la meilleure défense, la meilleure protection d’un pays contre l’ennemi du dehors. Le même quiproquo tragique est arrivé à la social-  démocratie, lorsque, pour justifier son attitude dans cette guerre, elle s’est réclamée du droit des nations à disposer d’elles-mêmes. C'est vrai : le socialisme reconnaît à chaque peuple le droit à l’indépendance et à la liberté, à la libre disposition de son propre destin. Mais c’est une véritable dérision du socialisme que de proposer les États capitalistes actuels comme expression de ce droit de libre disposition. Dans lequel de ces États la nation a-t-elle donc pu disposer jusqu’ici des formes et des conditions de son existence nationale, politique ou sociale ?\par
Ce que signifie la libre disposition du peuple allemand, ce que suppose un tel principe, les démocrates de 1848, les défenseurs de la cause du prolétariat allemand, Marx, Engels et Lassalle, Bebel et Liebknecht l’ont proclamé et soutenu : \emph{c’est la grande République allemande}. C'est pour cet idéal que les combattants de Mars ont versé leur sang sur les barricades à Vienne et à Berlin ; c’est pour réaliser ce programme que Marx et Engels voulaient contraindre la Prusse à faire la guerre contre le tsarisme russe en 1848. Pour accomplir ce programme national, il était tout d’abord nécessaire de liquider ce « monceau de pourriture organisée » appelé monarchie habsbourgeoise, et d’abolir la monarchie militaire prussienne tout comme les deux douzaines de monarchies naines en Allemagne. La défaite de la révolution allemande, la trahison de la bourgeoisie allemande envers ses propres idéaux démocratiques conduisirent au régime de Bismarck et à son oeuvre politique : la grande Prusse actuelle, avec les vingt patries sous un seul casque à pointe, appelé Empire allemand. L'Allemagne actuelle est édifiée sur le tombeau de la révolution de Mars, sur les ruines du droit de libre disposition nationale du peuple allemand. La guerre actuelle, qui, en plus de la conservation de la Turquie, a pour buts la conservation de la bourgeoisie habsbourgeoise et le renforcement de la monarchie militaire prussienne, est un nouvel enterrement des morts de Mars et du programme national de l’Allemagne. Et il y a une véritable ironie diabolique de l’histoire dans le fait que les sociaux-démocrates, les héritiers des patriotes allemands de 1848, entrent dans cette guerre en brandissant l’étendard du « droit des nations à disposer d’elles-mêmes ». A moins que, par hasard, la IIIe République avec ses possessions coloniales dans quatre continents et ses atrocités coloniales dans deux continents soit l’expression de la « libre disposition » de la nation française ? Ou bien peut-être l’Empire britannique avec les Indes et la domination d’un million de Blancs sur une population noire de cinq millions d’habitants en Afrique du Sud ? Ou encore la Turquie ou l’Empire tsariste ?... Seul un politicien bourgeois, pour qui l’humanité est représentée par les races de seigneurs et une nation par ses classes dirigeantes, peut parler de « libre disposition » à propos d’États coloniaux. Au sens socialiste du concept de liberté, il ne saurait y avoir de nation libre, lorsque son existence nationale repose sur la mise en esclavage d’autres peuples, car les peuples coloniaux eux aussi sont des peuples et ils font partie de l’État. Le socialisme international reconnaît aux nations le droit d’être libres, indépendantes, égales. Mais lui seul est capable de créer de telles nations, lui seul est en mesure de faire du droit des peuples à disposer d’eux-mêmes une réalité. Ce mot d’ordre du socialisme est lui aussi, comme tous les autres, non pas une sanctification de l’état de choses existant, mais une indication et un stimulant pour la politique active du prolétariat qui s’emploie à opérer des transformations révolutionnaires. Tant qu’il existe des États capitalistes et en particulier tant que la politique impérialiste détermine et modèle la vie intérieure et extérieure des États, le droit des peuples à disposer d’eux-mêmes ne ressemble en rien à la manière dont il est pratiqué pendant la guerre comme en temps de paix.\par
Il y a plus. Dans le cadre impérialiste actuel, il ne saurait plus y avoir de guerre défensive, de guerre nationale, et les socialistes qui ne tiendraient pas compte de ce cadre historique déterminant, qui, au milieu du tumulte du monde, voudraient se placer à un point de vue particulier, au point de vue d’un pays, bâtiraient d’entrée de jeu leur politique sur du sable.\par
Précédemment déjà, nous avons essayé de montrer les arrière-plans du conflit actuel entre l’Allemagne et ses adversaires. Il était nécessaire d’éclairer les ressorts réels et les connexions internes de la guerre actuelle parce que, dans la prise de position de notre groupe parlementaire comme dans les arguments de notre presse, l’argument décisif a été : défense de la liberté et de la culture allemandes. Contre cette affirmation, il faut s’en tenir à la vérité historique : il s’agit ici d’une guerre préventive préparée depuis des années par l’impérialisme allemand, provoquée par les objectifs de sa Weltpolitik et déclenchée sciemment, dans l’été 1914, par la diplomatie allemande et autrichienne. Mais en outre, quand on veut porter un jugement général sur la guerre mondiale et apprécier son importance pour la politique de classe du prolétariat, la question de savoir qui est l’agresseur et l’agressé, la question de la « culpabilité » est totalement sans objet. Si l’Allemagne mène moins que quiconque une guerre défensive, ce n’est pas non plus le cas de la France et de l’Angleterre ; car ce que ces nations « défendent », ce n’est pas leur position nationale, mais celle qu’elles occupent dans la politique mondiale, ce sont leurs vieilles possessions impérialistes menacées par les assauts du nouveau venu allemand. Si les incursions de l’impérialisme allemand et de l’impérialisme autrichien en Orient ont sans aucun doute apporté l’étincelle, de leur côté l’impérialisme français en exploitant le Maroc, l’impérialisme anglais par ses préparatifs en vue de piller la Mésopotamie et l’Arabie et par toutes les mesures prises pour assurer son despotisme en Inde, l’impérialisme russe par sa politique des Balkans dirigée vers Constantinople, ont petit à petit rempli la poudrière. Les préparatifs militaires ont bien joué un rôle essentiel : celui du détonateur qui a déclenché la catastrophe, mais il s’agissait d’une compétition à laquelle participaient tous les États. Et si c’est l’Allemagne en 1870 qui, par la politique de Bismarck, a donné la première impulsion à la course aux  armements, la politique du Second Empire lui avait préparé le terrain et elle fut encouragée par la suite par la politique aventurière de la IIIe République, par ses expansions en Asie orientale et en Afrique.\par
Ce qui donna aux socialistes français l’illusion qu’il s’agissait de « défense nationale », c’est le fait que le gouvernement français et le peuple français tout entier n’entretenaient aucune intention belliqueuse en juillet 1914. « Aujourd’hui, tout le monde en France est pour la paix, sincèrement et loyalement, sans réserves et sans restrictions », attestait Jaurès dans son dernier discours, qu’il prononça à la veille de la guerre à la Maison du peuple de Bruxelles. Le fait est parfaitement plausible, et il peut expliquer psychologiquement l’indignation qui s’est emparée des socialistes français quand une guerre criminelle fut imposée de force à leur pays. Mais cela ne suffit pas pour juger la guerre mondiale en tant que phénomène historique et pour permettre à la politique prolétarienne de prendre position à son sujet. L'histoire qui a donné naissance à la guerre actuelle n’a pas commencé en juillet 1914, mais elle remonte à des années en arrière, pendant lesquelles elle s’est nouée fil après fil avec la nécessité d’une loi naturelle, jusqu’à ce que le filet aux mailles serrées de la politique mondiale impérialiste ait enveloppé les cinq continents - un formidable complexe historique de phénomènes dont les racines descendent dans les profondeurs plutoniques du devenir économique, et dont les branches extrêmes font signe en direction d’un nouveau monde encore indistinct qui commence à poindre, des phénomènes qui, par leur ampleur gigantesque, rendent inconsistants les concepts de faute et d’expiation, de défense et d’attaque.\par
La politique impérialiste n’est pas l’œuvre d’un pays ou d’un groupe de pays. Elle est le produit de l’évolution mondiale du capitalisme à un moment donné de sa maturation. C'est un phénomène international par nature, un tout inséparable qu’on ne peut comprendre que dans ses rapports réciproques et auquel aucun État ne saurait se soustraire.\par
C'est de ce point de vue seulement qu’on peut évaluer correctement dans la guerre actuelle la question de la « défense nationale ». L'Etat national, l’unité et l’indépendance nationales, tels étaient les drapeaux idéologiques sous lesquels se sont constitués les grands États bourgeois du cœur de l’Europe au siècle dernier. Le capitalisme est incompatible avec le particularisme des petits États, avec un émiettement politique et économique ; pour s’épanouir, il lui faut un territoire cohérent aussi grand que possible, d’un même niveau de civilisation ; sans quoi, on ne pourrait élever les besoins de la société au niveau requis pour la production marchande capitaliste, ni faire fonctionner le mécanisme de la domination bourgeoise moderne. Avant d’étendre son réseau sur le globe tout entier, l’économie capitaliste a cherché à se créer un territoire d’un seul tenant dans les limites nationales d’un État. Ce programme, étant donné l’échiquier politique et national tel qu’il avait été transmis par le féodalisme médiéval, ne pouvait être réalisé que par des voies révolutionnaires. Il ne l’a été qu’en France au cours de la grande Révolution. Dans le reste de l’Europe (tout comme la révolution bourgeoise d’ailleurs), ce programme est resté à l’état d’ébauche, il s’est arrêté à mi-chemin. L'Empire allemand et l’Italie d’aujourd’hui, le maintien de l’Autriche-Hongrie et de la Turquie jusqu’à nos jours, l’Empire russe et le Commonwealth britannique en sont des preuves vivantes. Le programme national n’a joué un rôle historique, en tant qu’expression idéologique de la bourgeoisie montante aspirant au pouvoir dans l’État, que jusqu’au moment où la société bourgeoise s’est tant bien que mal installée dans les grands Etats du centre de l’Europe et y a créé les instruments et les conditions indispensables de sa politique.\par
Depuis lors, l’impérialisme a complètement enterré le vieux programme bourgeois démocratique : l’expansion au-delà des frontières nationales (quelles que soient les conditions nationales des pays annexés) est devenue la plate-forme de la bourgeoisie de tous les pays. Certes, la phrase nationale est demeurée, mais son contenu réel et sa fonction se sont mués en leur contraire. Elle ne sert plus qu’à masquer tant bien que mal les aspirations impérialistes, à moins qu’elle ne soit utilisée comme cri de guerre, dans les conflits impérialistes, seul et ultime moyen idéologique de capter l’adhésion des masses populaires et de leur faire jouer leur rôle de chair à canon dans les guerres impérialistes.\par
La tendance générale de la politique capitaliste actuelle domine la politique des États particuliers comme une loi aveugle et toute-puissante, tout comme les lois de la concurrence économique déterminent rigoureusement les conditions de production pour chaque entrepreneur particulier.\par
Imaginons un instant - pour dissiper le fantôme de la « guerre nationale » qui domine actuellement la politique sociale-démocrate que, dans l’un des États contemporains, la guerre ait effectivement débuté comme une simple guerre de défense nationale ; nous voyons que des succès militaires conduisent avant toute chose à l’occupation des territoires étrangers. Mais en présence de groupes capitalistes hautement influents, qui sont intéressés à des acquisitions impérialistes les appétits d’expansion se réveillent au cours de la guerre, et la tendance impérialiste qui, au début de celle-ci, était en germe ou sommeillait, va se développer comme en serre chaude et va déterminer le caractère de la guerre, ses buts et ses conséquences. En outre, le système d’alliance entre les États militaires qui, depuis des dizaines d’années, domine les relations politiques des États implique nécessairement que chacune des parties belligérantes, d’un point de vue purement défensif, cherche à attirer des alliés dans son camp. De ce fait, la guerre entraîne sans cesse de nouveaux pays et ainsi, inévitablement, les intérêts impérialistes de la politique mondiale sont touchés, et de nouveaux intérêts se créent. L'Angleterre a entraîné le Japon dans la guerre, a fait passer la guerre d’Europe en Asie orientale et a mis les destinées de la Chine à l’ordre du jour, a attisé les rivalités entre le Japon et les États-Unis, entre elle et le Japon - et ainsi a accumulé de quoi alimenter de nouveaux conflits. De même, dans l’autre camp, l’Allemagne a entraîné la Turquie dans la guerre, ce qui  amène à liquider aussitôt la question de Constantinople, la question des Balkans et du Proche-Orient. Celui qui n’aurait pas compris que, dans ses causes et ses points de départ, la guerre mondiale était déjà une guerre purement impérialiste, peut apercevoir en tout cas, d’après ses effets, que la guerre devait, dans les conditions actuelles, se transformer en un processus impérialiste de partage du monde selon un enchaînement tout à fait mécanique et inévitable. C'est ce qui s’est produit pour ainsi dire depuis le début. Comme l’équilibre de forces reste constamment précaire entre les parties belligérantes, chacune d’elles est obligée d’un point de vue purement militaire de renforcer sa propre position et de se préserver du danger de nouvelles hostilités, et de tenir en laisse les pays neutres en procédant à toute une série de combines sur les peuples et les pays. Voir les « offres » germano-autrichiennes, d’une part, et anglo-russes, d’autre part, qui sont faites en Italie, en Roumanie, en Grèce et en Bulgarie. La prétendue « guerre de défense nationale » a donc comme conséquence chez les pays non engagés un déplacement général des possessions et des rapports de force, qui va expressément dans le sens de l’expansion. Enfin, comme aujourd’hui tous les Etats capitalistes ont des possessions coloniales et qu’en cas de guerre, même si celle-ci débute comme une « guerre de défense nationale », les colonies y sont attirées pour des raisons purement militaires, et comme chaque Etat belligérant cherche à occuper les colonies de l’adversaire ou du moins à y provoquer un soulèvement - voir la mainmise des colonies allemandes par l’Angleterre et les tentatives qui sont faites pour déclencher la « guerre sainte » dans les colonies anglaises et françaises -, toute guerre actuelle doit automatiquement se transformer en une conflagration mondiale de l’impérialisme.\par
Ainsi, cette idée d’une guerre modeste et vertueuse pour la défense de la patrie qui obsède aujourd’hui nos parlementaires et nos journalistes est une pure fiction qui empêche toute saisie d’ensemble de la situation historique dans son contexte mondial. L'élément déterminant quant à la nature de la guerre, c’est la nature historique de la société contemporaine et de son organisation militaire, et non pas les déclarations solennelles ni même les intentions sincères des « dirigeants » politiques.\par
Le schéma d’une pure « guerre de défense nationale » pouvait peut-être à première vue s’appliquer à un pays comme la Suisse. Mais, comme par hasard, il se fait que la Suisse n’est pas un État national et que, de plus, elle n’est pas représentative des États actuels. Sa « neutralité » et le luxe de sa milice ne sont précisément que des produits négatifs de l’état de guerre latent des grandes puissances militaires qui l’entourent et ils ne seront durables qu’aussi longtemps qu’elle pourra s’accommoder de cette situation. Une telle neutralité est foulée aux pieds en un clin d’œil par les bottes de l’impérialisme, au cours d’une guerre mondiale : c’est ce dont témoigne le sort de la Belgique. Ici, nous en arrivons tout spécialement à la situation des petits États. Le cas de la Serbie constitue aujourd’hui le meilleur moyen de mettre à l’épreuve le mythe de la « guerre nationale ». S'il est un État qui a pour lui le droit à la défense nationale d’après tous les indices formels extérieurs, c’est bien la Serbie. Privée de son unité nationale par les annexions de l’Autriche, menacée par l’Autriche dans son existence nationale, acculée à la guerre par l’Autriche, la Serbie mène une véritable guerre de défense nationale pour sauvegarder son existence et sa liberté. Si la position du groupe social-démocrate allemand est juste, alors les sociaux-démocrates serbes qui ont protesté contre la guerre devant le parlement de Belgrade et qui ont refusé les crédits de guerre sont tout simplement des traîtres : ils auraient trahi les intérêts vitaux de leur propre pays. En réalité, les Serbes Lapstewitch et Kazlerowitch ne sont pas seulement entrés en lettres d’or dans l’histoire du socialisme international, mais ont fait preuve d’une pénétrante vision historique des circonstances réelles de la guerre, et par là ils ont rendu un service à leur pays et à l’instruction de leur peuple. Formellement, la Serbie mène sans nul doute une guerre de défense nationale. Mais les tendances de sa monarchie et de ses classes dirigeantes vont dans le sens de l’expansion, comme les tendances des classes dirigeantes de tous les États actuels, sans tenir compte des frontières nationales, et prennent par là un caractère agressif. Il en est ainsi pour la tendance de la Serbie vers la Côte Adriatique, où elle a vidé avec l’Italie un véritable différend impérialiste sur le dos des Albanais, et dont l’issue se décida finalement en dehors de la Serbie, entre les grandes puissances. Cependant, le point capital est le suivant : derrière l’impérialisme serbe, on trouve l’impérialisme russe. La Serbie elle-même n’est qu’un pion sur le grand échiquier de la politique mondiale et toute analyse de l’attitude de la Serbie face à la guerre qui ne tient pas compte de tout ce contexte et de l’arrière-plan politique général n’est bâtie que sur du sable.\par
Ceci concerne également la dernière guerre des Balkans. Si on considère les choses isolément et d’une manière formelle, les jeunes États balkaniques étaient historiquement dans leur bon droit, ils accomplissaient le vieux programme démocratique de l’État national. Cependant, replacées dans leur contexte historique réel qui a fait des Balkans le centre de la politique mondiale impérialiste, les guerres des Balkans n’étaient objectivement qu’un détail du tableau d’ensemble des hostilités, un maillon de la chaîne fatidique des événements qui ont conduit à la guerre mondiale avec une fatale nécessité. La social-démocratie internationale a réservé à Bâle aux socialistes des pays balkaniques l’ovation la plus chaleureuse pour leur refus de toute collaboration morale ou politique à la guerre des Balkans et pour avoir démasqué le vrai visage de cette guerre par là, elle a condamné par avance l’attitude des socialistes allemands et français dans la guerre actuelle.\par
Tous les petits États se trouvent cependant aujourd’hui dans la même situation que les États balkaniques ; ainsi, par exemple, la Hollande. « Quand le bateau coule, il faut avant tout songer à boucher les fuites. » De quoi pourrait-il s’agir en effet pour la petite Hollande, sinon tout simplement de défense nationale, de la défense de l’existence et de l’indépendance du pays ? Si on prend uniquement en  considération les intentions du peuple hollandais, il ne serait question que de défense nationale. Mais la politique prolétarienne qui repose sur la connaissance historique ne peut tenir compte des intentions subjectives d’un pays particulier, elle doit se placer à un niveau international et s’orienter par rapport à la totalité de la situation de la politique mondiale. La Hollande, qu’elle le veuille ou non, n’est, elle aussi, qu’un petit rouage dans tout l’engrenage de la politique et de la diplomatie mondiales actuelles. Ceci apparaîtrait aussitôt d’une manière évidente au cas où la Hollande serait effectivement entraînée dans le Maelstrom de la guerre mondiale. Tout d’abord, ses adversaires chercheraient à frapper ses colonies ; la stratégie de la Hollande au cours de cette guerre aurait donc tout naturellement pour but la conservation de ses possessions actuelles, et la défense de l’indépendance nationale du peuple flamand de la mer du Nord déboucherait en fait sur la défense de son droit à dominer et à exploiter le peuple malais de l’archipel indonésien. Mais ce n’est pas tout : livré à lui-même, le militarisme hollandais se briserait comme une coquille de noix dans le tourbillon de la guerre mondiale ; la Hollande ferait aussitôt partie, qu’elle le veuille ou non, d’une des grandes associations d’États combattants, et de la sorte elle deviendrait aussi le support et l’instrument de tendances purement impérialistes.\par
Ainsi, c’est à chaque fois le cadre historique de l’impérialisme actuel qui détermine le caractère de la guerre pour chaque pays particulier, et ce cadre fait que, \emph{de nos jours, les guerres de défense nationale ne sont absolument plus possibles}.\par
C'est ce qu’écrivait également Kautsky il y a quelques années à peine dans sa brochure \emph{Patriotisme et social-démocratie} (Leipzig, 1907) :\par

\begin{quoteblock}
 \noindent « Si le patriotisme de la bourgeoisie et le patriotisme du prolétariat sont deux choses tout à fait différentes, et même opposées, il y a quand même des situations dans lesquelles ces deux sortes de patriotisme peuvent converger pour agir de concert, même dans le cas d’une guerre. La bourgeoisie et le prolétariat d’une nation sont l’un comme l’autre intéressés à son indépendance et à son autonomie, ils veulent tous deux l’élimination et l’éloignement de toute sorte d’oppression et d’exploitation par une nation étrangère ; au cours des luttes nationales naissant de pareilles aspirations, le patriotisme du prolétariat s’est toujours uni à celui de la bourgeoisie. Depuis lors, cependant, le prolétariat est devenu une force qui, chaque fois que l’Etat subit un grand ébranlement, se montre dangereuse pour les classes dirigeantes ; depuis lors, à la fin de toute guerre, la révolution menace, comme l’ont montré la Commune de Paris et le terrorisme russe après la guerre russo-turque ; et depuis lors, même la bourgeoisie des nations qui ne sont pas du tout ou trop peu indépendantes et unifiées a effectivement abandonné ses buts nationaux lorsqu’ils ne pouvaient être atteints que par le renversement du gouvernement, car elle déteste et redoute la révolution plus qu’elle n’aime l’indépendance et la grandeur de la nation. C'est pourquoi elle renonce à l’indépendance de la Pologne et laisse subsister des formes d’États aussi antédiluviens que l’Autriche et la Turquie, qui, il y a une génération déjà, semblaient destinés à disparaître. De ce fait, les problèmes nationaux qui, aujourd’hui encore, ne peuvent être résolus que par la guerre ou la révolution ne pourront désormais trouver une solution qu’après la victoire préalable du prolétariat. Car ils prendront aussitôt, en raison de la solidarité internationale, une toute autre forme aujourd’hui, dans la société de l’exportation et de l’oppression. Le prolétariat des États capitalistes n’aura plus à s’occuper comme aujourd’hui de ses luttes pratiques, il pourra consacrer toutes ses forces à d’autres tâches. » (pp. 12-14.)\par
 « Entre-temps, il devient de moins en moins vraisemblable que le patriotisme prolétaire et le patriotisme bourgeois puissent encore s’unir pour défendre la liberté de leur pays. » La bourgeoisie française, dit-il, s’est unie au tsarisme. La Russie n’est plus un danger pour la liberté de l’Europe occidentale, parce que affaiblie par la révolution. « Dans ces conditions, on ne doit plus s’attendre à assister encore à une guerre de défense nationale au cours de laquelle le patriotisme prolétarien et le patriotisme bourgeois pourraient s’allier. » (p. 16.)\par
 « Nous avons vu précédemment qu’avaient cessé les oppositions qui, au XIXe siècle encore, pouvaient obliger bien des peuples libres à entrer en conflit armé avec leurs voisins, nous avons vu que le militarisme actuel ne servait plus du tout la défense des intérêts essentiels du peuple, mais seulement du profit ; qu’il ne contribuait plus à maintenir l’indépendance et l’intégrité nationales qui ne sont menacées par personne, mais seulement à conserver et à étendre les conquêtes d’outre-mer qui favorisent uniquement le profit capitaliste. Les oppositions actuelles entre les États ne permirent plus de mener une guerre à laquelle le patriotisme prolétarien ne devrait pas s’opposer de la manière la plus catégorique. » (p. 23.)
\end{quoteblock}

\noindent Que résulte-t-il de tout cela en ce qui concerne l’attitude pratique de la social-démocratie dans la guerre actuelle ? Devait-elle déclarer : puisque cette guerre est une guerre impérialiste, puisque l’Etat dans lequel nous vivons ne répond pas au droit socialiste de libre disposition, ni à l’idéal national, nous ne nous en soucions pas, nous l’abandonnons à l’ennemi ? Jamais l’attitude passive du laisser faire, laisser passer ne peut être la ligne de conduite d’un parti révolutionnaire comme la social-démocratie. Le rôle de la social-démocratie, ce n’est pas de se placer sous le commandement des classes dirigeantes pour défendre la  société de classe existante, ni de rester silencieusement à l’écart en attendant que la tourmente soit passée, mais bien de suivre une politique de classe indépendante qui, dans chaque grande crise de la société bourgeoise, aiguillonne les classes dirigeantes à aller de l’avant et, par là, chasse la crise : voilà son rôle, en tant qu’avant-garde du prolétariat en lutte. Au lieu de draper la guerre impérialiste dans le vote fallacieux de la défense nationale, il s’agissait précisément de prendre au sérieux le droit de libre disposition des peuples et la défense nationale, de s’en servir comme de leviers révolutionnaires, et de les retourner contre la guerre impérialiste. L'exigence la plus élémentaire de la défense de la nation est que la nation prenne elle-même sa défense en main. La première étape dans cette direction est : la milice, à savoir : pas seulement l’armement immédiat de tous les hommes adultes, mais avant tout aussi la possibilité pour le peuple de décider de la guerre et de la paix, et encore le rétablissement immédiat de tous les droits politiques, car la plus grande liberté politique est le fondement indispensable de la défense nationale populaire. Proclamer ces véritables mesures de défense nationale et exiger leur application, c’était là la première tâche de la social-démocratie. Pendant quarante ans, nous avons expliqué aux classes dirigeantes et aux masses populaires que seule la milice était à même de défendre réellement la patrie et de la rendre invincible. Et voilà qu’au moment où arrivait la première grande épreuve, nous avons, comme si c’était l’évidence même, abandonné la défense du pays à l’armée permanente, cette chair à canon sous la férule des classes dirigeantes. Visiblement, nos parlementaires n’ont même pas remarqué qu’en accompagnant de leurs « vœux ardents » cette chair à canon qui partait au front et en reconnaissant qu’elle était la véritable défense de la patrie, en admettant sans aucun commentaire que l’armée royale prussienne permanente était sa sauvegarde à l’heure de la plus grande détresse, ils laissaient froidement tomber le point capital de notre programme politique : la milice, qu’ils réduisaient à néant la signification de quarante ans d’agitation sur la question de la milice, qu’ils en faisaient une fumisterie utopique que personne ne prendra plus jamais au sérieux \footnote{ \noindent « Si malgré tout le groupe parlementaire social-démocrate a voté à l’unanimité les crédits de guerre - écrivait l’organe du parti à Munich, le 6 août -, s’il a accompagné de ses vœux ardents tous ceux qui s’en allaient défendre le Reich allemand, ce n’était pas de sa part une " manœuvre tactique ", c’était la conséquence tout à fait naturelle de l’attitude d’un parti qui a toujours été prêt à confier la défense du pays à une armée populaire pour remplacer un système qui lui paraissait refléter la domination de classe plutôt que la volonté de la nation de se défendre contre des attaques insolentes. »\par
 Paraissait !... Dans le Neue Zeit, la guerre actuelle est même directement érigée en « guerre populaire », l’armée permanente en « armée populaire » (voir n° 20 et 23 de août-septembre 1914). L'écrivain militaire social-démocrate Hugo Schulz, dans un compte rendu de guerre du 24 août 1914, fait l’éloge du « puissant esprit de milice » qui « anime » l’armée habsbourgeoise !...
}.\par
Les maîtres du prolétariat international comprenaient autrement la défense de la patrie. Lorsque le prolétariat prit le pouvoir en 1871 dans la ville de Paris assiégée par les Prussiens, Marx commentait ainsi avec enthousiasme son action :\par

\begin{quoteblock}
 \noindent « Paris, siège central de l’ancien pouvoir gouvernemental, et, en même temps, forteresse sociale de la classe ouvrière française, avait pris les armes contre la tentative faite par Thiers et ses ruraux pour restaurer et perpétuer cet ancien pouvoir gouvernemental que leur avait légué l’Empire. Paris pouvait seulement résister parce que, du fait du siège, il s’était débarrassé de l’armée et l’avait remplacée par une garde nationale, dont la masse était constituée par des ouvriers. C'est cet état de fait qu’il s’agissait maintenant de transformer en une institution durable. Le premier décret de la Commune fut donc la suppression de l’armée permanente, et son remplacement par le peuple en armes. [...] Si la Commune était donc la représentation véritable de tous les éléments sains de la société française, et par suite le véritable gouvernement national, elle était en même temps un gouvernement ouvrier, et, à ce titre, en sa qualité de champion audacieux de l’émancipation du travail, internationale au plein sens du terme. Sous les yeux de l’armée prussienne qui avait annexé à l’Allemagne deux provinces françaises, la Commune annexait à la France les travailleurs du monde entier. » (\emph{Adresse du Conseil général de l’Internationale})
\end{quoteblock}

\noindent Et comment nos vieux maîtres concevaient-ils le rôle de la social-démocratie dans une guerre comme celle que nous connaissons aujourd’hui ? Friedrich Engels décrivait comme suit les lignes fondamentales de la politique que le parti du prolétariat doit adopter dans une grande guerre :\par

\begin{quoteblock}
 \noindent « Une guerre où les Russes et les Français envahiraient l’Allemagne serait pour celle-ci un combat de vie ou de mort dans lequel elle ne pourrait assurer son existence nationale qu’en recourant aux mesures les plus révolutionnaires. Le gouvernement actuel, s’il n’y est pas forcé, ne déclenchera certes pas la révolution. Mais nous, nous avons un parti fort qui peut l’y forcer, ou le remplacer, s’il le faut : le parti social-démocrate.» \\
« Et nous n’avons pas oublié l’exemple prestigieux que nous a donné la France de 1793. Le jubilé du centenaire de 1793 approche. Si l’ardeur conquérante du tsarisme et l’impatience chauviniste de la bourgeoisie française devaient retarder l’avance victorieuse mais pacifique des sociaux-démocrates allemands, ceux-ci - soyez-en sûrs sont prêts à prouver au monde que les prolétaires allemands d’aujourd’hui ne sont pas indignes des sans-culottes et que 1893 peut  être placé à côté de 1793. Et si les soldats étrangers mettent le pied en territoire allemand, ils seront accueillis par ces paroles de la Marseillaise : »\par
 « Quoi, ces cohortes étrangères\par
 Feraient la loi dans nos foyers ? »\par
 En bref : la paix signifie la certitude de la victoire du parti social-démocrate allemand en dix ans environ. La guerre lui apportera soit la victoire en deux ou trois ans, soit la ruine complète pour quinze à vingt ans au moins.»
\end{quoteblock}

\noindent Lorsqu’il écrivait cela, Engels envisageait une tout autre situation que la situation actuelle. Il avait encore sous les yeux le vieil Empire tsariste, alors que nous, depuis lors, nous avons connu la grande Révolution russe. De plus, il songeait à une véritable guerre de défense nationale de l’Allemagne attaquée simultanément à l’est et à l’ouest. Enfin, il avait surestimé le degré d’évolution de la situation en Allemagne et les perspectives d’une révolution sociale : les vrais militants ont souvent tendance à surestimer le rythme de l’évolution. Mais ce qui ressort en tout cas clairement de son analyse, c’est que par défense nationale dans le sens de la politique social-démocrate, Engels n’entendait pas le soutien du gouvernement des junkers prussiens et de son état-major, mais une action révolutionnaire qui suivrait l’exemple des jacobins français.\par
Oui, les sociaux-démocrates doivent défendre leur pays lors des grandes crises historiques. Et la lourde faute du groupe social-démocrate du Reichstag est d’avoir solennellement proclamé dans sa déclaration du 4 août 1914 : « A l’heure du danger, nous ne laisserons pas notre patrie sans défense », et d’avoir, dans le même temps, renié ses paroles. Il a laissé la patrie sans défense à l’heure du plus grand danger. Car son premier devoir envers la patrie était à ce moment de lui montrer les dessous véritables de cette guerre impérialiste, de rompre le réseau de mensonges patriotiques et diplomatiques qui camouflait cet attentat contre la patrie ; de déclarer haut et clair que, dans cette guerre, la victoire et la défaite étaient également funestes pour le peuple allemand ; de résister jusqu’à la dernière extrémité à l’étranglement de la patrie au moyen de l’état de siège ; de proclamer la nécessité d’armer immédiatement le peuple et de le laisser décider lui-même la question de la guerre ou de la paix ; d’exiger avec la dernière énergie que la représentation populaire siège en permanence pendant toute la durée de la guerre pour assurer le contrôle vigilant de la représentation populaire sur le gouvernement et du peuple sur la représentation populaire ; d’exiger l’abolition immédiate de toutes les limitations des droits politiques, car seul un peuple libre peut défendre avec succès son pays ; d’opposer, enfin, au programme impérialiste de guerre - qui tend à la conservation de l’Autriche et de la Turquie, c’est-à-dire de la réaction en Europe et en Allemagne -, le vieux programme véritablement national des patriotes et des démocrates de 1848, le programme de Marx, Engels et Lassalle : le mot d’ordre de la grande et indivisible République allemande. Tel est le drapeau qu’il fallait déployer devant le pays, qui aurait été véritablement national, véritablement libérateur, et qui aurait répondu aux meilleures traditions de l’Allemagne et de la politique de classe internationale du prolétariat.\par
La grande heure historique de la guerre mondiale réclamait manifestement une action politique résolue, une prise de position aux vues larges et étendues, une orientation supérieure du pays que seule la social-démocratie était appelée à proposer. Au lieu de cela, on assista à une faillite lamentable et sans exemple de la part de la représentation parlementaire de la classe ouvrière, qui avait la parole à ce moment. Par la faute de ses dirigeants, la social-démocratie n’a même pas suivi une fausse politique : elle n’a pas suivi de politique du tout, en tant que parti d’une classe doué de sa propre vision du monde, elle s’est mise complètement hors circuit ; elle a abandonné sans broncher le pays au sort redoutable de la guerre impérialiste et à la dictature du sabre et, par-dessus le marché, elle a assumé la responsabilité de la guerre. La déclaration du groupe parlementaire dit qu’il a seulement voté en faveur des moyens nécessaires à la défense du pays, mais que, par contre, il décline la responsabilité de la guerre. Or, c’est précisément l’inverse qui est vrai. Les moyens nécessaires à cette « défense nationale », c’est-à-dire à la boucherie humaine déclenchée par l’impérialisme au moyen des armées de la monarchie militaire, la social-démocratie n’avait pas du tout besoin de les voter, car leur mise en oeuvre ne dépendait pas le moins du monde du vote des sociaux-démocrates : ceux-ci étaient en minorité face à la majorité compacte des trois quarts du Reichstag bourgeois. Par son vote spontané, le groupe social-démocrate n’a abouti qu’à une chose : à attester l’unité du peuple tout entier pendant la guerre, à proclamer l’Union sacrée, c’est-à-dire la suspension de la lutte de classes, l’interruption de la politique d’opposition de la social-démocratie au cours de la guerre, donc à assumer la coresponsabilité morale de la guerre. Par son vote spontané, elle a marqué cette guerre du sceau de la défense démocratique de la patrie, a contribué à tromper les masses sur les vraies conditions et les vraies tâches de la défense de la patrie et contresigné cette mystification.\par
Ainsi le grave dilemme : intérêts de la patrie et solidarité internationale du prolétariat, le conflit tragique qui a incité nos parlementaires à rallier « d’un cœur lourd » le camp de la guerre impérialiste, n’est que pure invention, une fiction nationaliste bourgeoise. Au contraire, entre les intérêts du pays et les intérêts de classe de l’Internationale prolétarienne, il existe aussi bien pendant la guerre que pendant la paix une parfaite harmonie : la guerre, comme la paix, exige le développement le plus intense de la lutte de classes et la défense la plus résolue du programme social-démocrate.\par
  Mais que devait faire notre parti pour souligner son opposition à la guerre et ses exigences ? Devait-il proclamer la grève de masse ? Ou bien exhorter les soldats à refuser de servir ? C'est ainsi que l’on pose la question habituellement. Répondre oui à de telles questions serait tout aussi ridicule que si le parti se mettait à décréter : « Si la guerre éclate, alors nous faisons la révolution. » Les révolutions ne sont pas « faites », et les grands mouvements populaires ne sont pas mis en scène avec des recettes techniques qui sortiraient de la poche des dirigeants des instances du parti. De petits cercles de conspirateurs peuvent bien « préparer » un putsch pour un jour et une heure précis, ils peuvent au moment voulu donner le signal de l’« attaque » à quelques milliers de partisans. Mais dans les grands moments de l’histoire, les mouvements de masse ne sont pas dirigés par des moyens aussi primitifs. La grève de masse « la mieux préparée » peut dans certaines circonstances faire long feu lamentablement, juste au moment où un chef de parti lui donne « le signal », ou bien, après un premier élan, tomber à plat. Si de grandes manifestations populaires et des actions de masse ont effectivement lieu sous une forme ou une autre, ce qui en décide, c’est tout un ensemble de facteurs économiques, politiques et psychiques, la tension des oppositions de classe à un moment donné, le degré de conscience et de combativité des masses tous facteurs imprévisibles qu’aucun parti ne peut produire artificiellement. C'est là toute la différence entre les grandes crises de l’histoire et les petites actions de parade qu’en période calme un parti bien discipliné peut exécuter délicatement sous la baguette de ses « instances ». L'heure historique exige à chaque fois les formes correspondantes du mouvement populaire et en crée elle-même de nouvelles et improvise des moyens de lutte inconnus jusque-là, trie et enrichit l’arsenal du peuple, insouciante de toutes les prescriptions des partis.\par
Ce que les dirigeants de la social-démocratie avaient à proposer en tant qu’avant-garde du prolétariat conscient, ce n’était donc pas des prescriptions et des recettes ridicules de nature technique, mais le mot d’ordre politique, la formulation claire des tâches et des intérêts politiques du prolétariat au cours de la guerre. Ce qu’on a dit de la grève de masse à propos de la révolution russe peut s’appliquer à tout mouvement de masse :\par

\begin{quoteblock}
 \noindent « S'il est donc vrai que c’est à la période révolutionnaire que revient la direction de la grève de masse au sens de l’initiative de son déclenchement et de la prise en charge des frais, il n’est pas moins vrai qu’en un tout autre sens la direction dans les grèves de masse revient à la social-démocratie et à ses organismes directeurs. Au lieu de se poser le problème de la technique et du mécanisme de la grève de masse, la social-démocratie est appelée, dans une période révolutionnaire, à en prendre la direction politique. La tâche la plus importante de " direction " dans la période de la grève de masse, consiste à donner le mot d’ordre de la lutte, à l’orienter, à régler la tactique de la lutte politique de telle manière qu’à chaque phase et à chaque instant du combat soit réalisée et mise en activité la totalité de la puissance du prolétariat déjà engagé et lancé dans la bataille et que cette puissance s’exprime par la position du parti dans la lutte ; il faut que la tactique de la social-démocratie ne se trouve jamais, quant à l’énergie et à la précision, au-dessous du niveau du rapport des forces en présence, mais que, au contraire, elle dépasse ce niveau ; alors, cette direction politique se transformera automatiquement en une certaine mesure en direction technique. Une tactique socialiste conséquente, résolue, allant de l’avant, provoque dans la masse un sentiment de sécurité, de confiance, de combativité ; une tactique hésitante, faible, fondée sur une sous-estimation des forces du prolétariat, paralyse et désoriente la masse. Dans le premier cas, les grèves de masse éclatent " spontanément " et toujours " en temps opportun " ; dans le deuxième cas, la direction du parti a beau appeler directement à la grève - c’est en vain\phantomsection
\label{A2} \footnote{R. Luxemburg, Grève de masses, parti et syndicats, Hamburg, 1907.}. »
\end{quoteblock}

\noindent La preuve qu’il ne s’agit pas de la forme extérieure, technique, de l’action, mais de son contenu politique, c’est par exemple le fait que la tribune du Parlement, cet endroit unique d’où on peut se faire entendre librement et avoir une audience internationale, pouvait dans ce cas-ci devenir un outil prodigieux de stimulation du peuple si elle avait été utilisée par les députés sociaux-démocrates dans le but de formuler d’une manière claire et distincte les intérêts, les tâches et les exigences de la classe ouvrière dans cette crise.\par
Et les masses auraient-elles soutenu ces mots d’ordre de la social-démocratie par leur attitude ? Personne ne peut le dire dans le feu de l’action. Mais ce n’est pas du tout le point décisif. « Avec confiance », nos parlementaires ont bien laissé partir en guerre les généraux de l’armée prusso-allemande, sans exiger de leur part l’assurance qu’ils seraient vainqueurs et que la possibilité d’une défaite était exclue. Ce qui vaut pour les armées militaires vaut aussi pour les armées révolutionnaires : elles engagent le combat là où il se présente sans réclamer au préalable la certitude de la réussite. Dans le pire des cas, la voix du parti serait restée au début sans effet visible. Et l’attitude virile de notre parti lui aurait vraisemblablement valu les plus grandes persécutions comme c’avait été le cas en 1870 pour Bebel et Liebknecht. « Mais qu’est-ce que cela peut faire ? » - disait très simplement Ignaz Auer en 1895 dans son discours sur les Fêtes de Sedan - « un parti qui veut conquérir le monde doit maintenir bien haut ses principes, sans tenir compte des dangers que cela implique ; il serait perdu s’il agissait autrement ! »\par

\begin{quoteblock}
 \noindent   « Il n’est jamais facile de nager à contre-courant - écrivait le vieux Liebknecht - et lorsque le courant se précipite avec la vitesse et la masse impétueuse d’un Niagara, alors c’est encore moins une sinécure. »\par
 « Les camarades les plus âgés ont encore en mémoire la haine des socialistes dans l’année de la plus grande honte nationale : de la honte de la Loi des socialistes - 1878. Des millions de gens voyaient alors en tout social-démocrate un meurtrier et un criminel de droit commun, et, en 1870, un traître à la patrie et un ennemi mortel. De telles explosions de l’"âme du peuple" ont, par leur force élémentaire monstrueuse, quelque chose de déconcertant, de stupéfiant, d’oppressant. On se sent impuissant devant une puissance supérieure, une force majeure excluant toute hésitation. On n’a aucun adversaire saisissable. C'est comme une épidémie : elle est dans les hommes, dans l’air, partout. »\par
 « L'explosion de 1878 n’était cependant pas comparable en force et en sauvagerie à celle de 1870. Pas seulement l’ouragan de passion humaine qui plie, abat, détruit tout ce qu’il saisit, mais encore la machinerie redoutable du militarisme fonctionnant à plein rendement - et nous entre les rouages de fer qui mugissaient tout autour et dont le contact était synonyme de mort, et entre les bras de fer qui sifflaient tout autour de nous et pouvaient à tout instant nous saisir. A côté de la force élémentaire des esprits déchaînés, le mécanisme le plus complet de l’art du meurtre que le monde ait jamais connu. Et tout cela dans le mouvement le plus effréné - toutes les chaudières prêtes à exploser. Où reste alors la force individuelle, la volonté individuelle ? Surtout si on sait qu’on fait partie d’une minorité, et si on n’a même plus un point d’appui sûr dans le peuple.»\par
 « Notre parti était encore en formation. Nous étions soumis à l’épreuve la plus difficile qui se puisse concevoir, avant que l’organisation nécessaire ne soit créée. Lorsque vint la haine des socialistes, l’année de l’ignominie pour nos ennemis, l’année de la gloire pour la social-démocratie, nous avions déjà une organisation si forte et si ramifiée que chacun était réconforté par la conscience d’un appui puissant et que personne de sensé ne pouvait croire que le parti pût succomber. »\par
 « Ce n’était donc pas une sinécure, alors, de nager à contre-courant. Mais qu’y avait-il à faire ? Ce qui devait être, devait être. Cela voulait dire : serrer les dents et, quoi qu’il advienne, laisser venir. Ce n’était pas le moment d’avoir peur... Or, Bebel et moi... nous ne nous occupions pas un seul instant des avertissements. Nous ne pouvions pas battre en retraite, nous devions rester à notre poste, advienne que pourra. »
\end{quoteblock}

\noindent Ils restèrent à leur poste, et la social-démocratie allemande s’est nourrie pendant quarante ans de la force morale dont elle avait fait preuve alors contre un monde d’ennemis.\par
C'est ainsi que cela se serait aussi passé cette fois-ci. Au début, le seul résultat aurait peut-être été que l’honneur du prolétariat allemand aurait été sauf, que les milliers et les milliers de prolétaires qui périssent maintenant dans les tranchées, le jour, la nuit et dans le brouillard, ne seraient pas morts dans un désarroi spirituel accablant, mais en gardant à l’esprit cette faible lueur d’espoir : ce qui leur était le plus cher au monde, la social-démocratie internationale, libératrice des peuples, n’était pas une illusion. Mais déjà la voix courageuse de notre parti aurait eu pour effet de tempérer fortement l’ivresse chauvine et l’inconscience de la foule, elle aurait gardé du délire les cercles populaires les plus éclairés, elle aurait contrecarré le travail d’intoxication et d’abrutissement du peuple par les impérialistes. Et précisément, la croisade contre la social-démocrate aurait rapidement dégrisé les masses populaires. Par la suite, à mesure que les hommes de tous les pays sont pris d’un sentiment de nausée devant cette boucherie humaine lugubre et interminable, où le caractère impérialiste de la guerre se trahit de plus en plus, où le tohubohu de la spéculation sanguinaire devient de plus en plus insolent tout ce qu’il y a de vivant, de sincère, d’humain et de progressiste se rassemblerait autour du drapeau de la social-démocratie. Et surtout, dans le tourbillon, la ruine et la débâcle, la social-démocratie, comme un rocher au milieu de la mer mugissante, serait restée le grand phare de l’Internationale sur lequel tous les autres partis ouvriers se seraient bien-tôt orientés. L'énorme autorité morale dont jouissait la social-démocratie allemande dans tout le monde prolétaire jusqu’au 4 août 1914 aurait sans aucun doute provoqué rapidement un changement au milieu de cette confusion générale. Par là, l’atmosphère favorable à la paix et la pression des masses populaires en vue de la paix auraient été renforcées dans tous les pays, la fin de ce meurtre de masse aurait été accélérée, les guerres mondiales sous la direction de l’Angleterre seraient réduites le lendemain en raison du nombre de ses victimes. Le prolétariat allemand serait resté la sentinelle vigilante du socialisme et de la libération de l’humanité - et cela, c’était bien un acte patriotique qui n’était pas indigne des disciples de Marx, Engels et Lassalle.
\section[{La lutte contre l’Impérialisme}]{La lutte contre l’Impérialisme}\renewcommand{\leftmark}{La lutte contre l’Impérialisme}

\noindent   Malgré la dictature militaire et la censure de la presse, malgré la faillite de la social-démocratie, malgré la guerre fratricide, la lutte des classes ressort avec une force élémentaire de l’« Union sacrée » et la solidarité internationale des ouvriers surgit des vapeurs sanglantes des champs de bataille. Non pas dans les tentatives faiblardes pour galvaniser artificiellement la vieille Internationale, non pas dans les promesses qui sont renouvelées par-ci, par-là de faire à nouveau cause commune aussitôt que la guerre sera terminée. Non, c’est maintenant, pendant la guerre, et à partir de la guerre, qu’apparaît à nouveau, avec une force et une importance toutes nouvelles, le fait que les prolétaires de tous les pays ont un seul et même intérêt. La guerre mondiale réfute elle-même la mystification à laquelle elle avait donné lieu.\par
Victoire ou défaite ? Tel est le mot d’ordre lancé par le militarisme dominant dans chacun des pays belligérants, et auquel les dirigeants sociaux-démocrates ont fait chorus. Pour les prolétaires d’Allemagne, de France, d’Angleterre et de Russie de même que pour les classes dirigeantes de ces pays, tout devrait être suspendu à l’alternative de la victoire ou de la défaite sur les champs de bataille. Aussitôt que les canons se mettent à tonner, le prolétariat de chaque pays devrait être intéressé à sa victoire et à la défaite des autres pays. Voyons donc ce que la guerre peut apporter au prolétariat.\par
D'après la version officielle reprise telle quelle par les leaders de la social-démocratie, la victoire représente pour l’Allemagne la perspective d’un essor économique illimité et sans obstacle, et la défaite, au contraire, la menace d’une ruine économique. Cette conception s’appuie à peu près sur le schéma de la guerre de 1870. Or, la prospérité que connut l’Allemagne après la guerre de 1870 ne résultait pas de la guerre, mais bien de l’unification politique, même si celle-ci n’avait que la forme rabougrie de l’Empire allemand créé par Bismarck. L'essor économique découla de l’unification politique malgré la guerre et malgré les nombreux obstacles réactionnaires qu’elle entraîna. L'effet propre de la guerre victorieuse, ce fut de consolider la monarchie militaire de l’Allemagne et le régime des junkers prussiens, alors que la défaite de la France avait contribué à liquider l’Empire et à instaurer la République. Mais aujourd’hui il en va autrement dans tous les États impliqués. Aujourd’hui la guerre ne fonctionne plus comme une méthode dynamique susceptible de procurer au jeune capitalisme naissant les conditions politiques indispensables de son épanouissement « national ». A la rigueur peut-on admettre que la guerre possède ce caractère en Serbie, et seulement si on la considère isolément. Réduite à son sens historique objectif, la guerre mondiale actuelle est d’un point de vue général, une lutte de concurrence d’un capitalisme déjà parvenu à sa pleine maturité, pour la souveraineté mondiale et pour l’exploitation des dernières zones du monde restées non capitalistes. C'est pourquoi on assiste à un changement complet dans le caractère de la guerre elle-même et de ses effets. Le degré élevé du développement économique de la production capitaliste se manifeste aussi bien dans le niveau extraordinairement élevé de la technique c’est-à-dire de la puissance de destruction des armements de guerre, que dans son niveau approximativement égal pour tous les pays belligérants. L'organisation internationale de l’industrie de guerre se reflète actuellement dans l’équilibre des forces qui se rétablit sans cesse à travers les décisions et les hésitations partielles de la balance, et qui diffère sans cesse une décision générale. A son tour, l’indécision des opérations militaires a pour conséquence que de nouveaux effectifs sont constamment envoyés au feu : aussi bien de nouvelles masses de population dans les pays belligérants que de nouveaux pays qui étaient restés neutres jusque-là. La guerre trouve partout à profusion de nouveaux désirs impérialistes et de nouveaux conflits à exploiter, elle en crée elle-même de nouveaux et ainsi elle se propage et fait boule de neige. Mais plus il y a de masses colossales de population et de pays qui sont entraînés dans la guerre, et plus sa durée se prolonge. Tout cela ensemble fait qu’avant même qu’intervienne une décision militaire, la guerre produit un phénomène que les guerres précédentes des temps modernes n’ont pas connu : la ruine économique de tous les pays qui y prennent part et même d’une manière croissante des pays qui sont formellement non impliqués. A mesure que la guerre se prolonge, ce phénomène se confirme et se renforce : à chaque mois qui passe, la possibilité de récolter les fruits d’une victoire militaire s’éloigne encore de dix ans. Ni la victoire ni la défaite ne peuvent en fin de compte rien changer à ce phénomène, qui rend au contraire tout à fait douteuse une décision purement militaire : il est de plus en plus vraisemblable que la guerre s’achèvera finalement par l’épuisement extrême de tous les adversaires. Dans ces conditions, si l’Allemagne devait sortir victorieuse de la guerre - même si les fauteurs de guerre impérialistes accomplissaient leurs rêves ambitieux, s’ils réussissaient à poursuivre le massacre jusqu’à l’élimination complète de tous leurs adversaires -, elle ne remporterait qu’une victoire à la Pyrrhus. Elle aurait pour trophées : l’annexion de quelques territoires dépeuplés et réduits à la mendicité et, chez elle, le spectre ricanant de la ruine qui surviendra lorsque auront disparu le carton-pâte d’une économie financière soutenue par les emprunts de guerre et les villages de Potemkine du « bien-être inébranlable du peuple » maintenus en activité par les livraisons de guerre. Il crève les yeux que même l’État le plus victorieux ne peut espérer réparer, si peu que ce soit, avec les indemnités de guerre, les dégâts subis pendant la guerre. En guise de compensation et pour compléter sa « victoire », l’Allemagne assisterait à la ruine peut-être plus grande encore du camp opposé, de la France et de l’Angleterre, c’est-à-dire des pays avec lesquels elle est le plus étroitement liée du point de vue économique, son renouveau économique dépendant en grande partie de leur propre prospérité. C'est dans ce cadre qu’après la guerre - une guerre « victorieuse », bien entendu - il s’agira pour le peuple allemand de payer après coup la note des frais de guerre que les parlementaires patriotes ont « approuvés » par avance, c’est-à-dire de supporter à la fois la charge d’une série interminable d’impôts et le poids d’une réaction militaire renforcée : voilà quel sera le seul fruit durable et tangible de sa « victoire ».\par
  Si on cherche maintenant à se représenter les pires conséquences d’une défaite, on constate qu’à l’exception des annexions impérialistes elles ressemblent trait pour trait à l’ensemble des conséquences qui résulteraient inéluctablement d’une victoire : c’est que les effets de la guerre elle-même sont si profonds et si étendus que son issue militaire ne peut y changer grand-chose.\par
Imaginons néanmoins un instant que l’État victorieux entende malgré tout se décharger du plus gros de la ruine et en accabler son adversaire vaincu, et qu’il étrangle son développement économique par des entraves de toutes sortes. La classe ouvrière allemande peut-elle après la guerre aller de l’avant avec succès si l’action syndicale des ouvriers français, anglais, belges et italiens est entravée par un dépérissement économique ? Jusqu’en 1870, le mouvement ouvrier progressait encore indépendamment dans chaque pays, et les décisions tombaient dans des villes isolées. C'est sur le pavé de Paris que se sont livrées et décidées les batailles du prolétariat. Mais le mouvement ouvrier actuel, avec sa lutte quotidienne laborieuse et mesurée et son organisation de masse, est fondé sur la coopération de tous les pays qui connaissent la production capitaliste. S'il est vrai que la cause ouvrière ne peut prospérer que sur la base d’une vie économique saine et vigoureuse, alors cela ne vaut pas seulement pour l’Allemagne, mais aussi pour la France, l’Angleterre, la Belgique, la Russie, l’Italie. Et si le mouvement ouvrier stagne dans tous les États capitalistes d’Europe, si on y trouve partout des salaires bas, des syndicats faibles et peu de résistance de la part des exploités, alors il est impossible que le mouvement syndical soit florissant en Allemagne. A ce point de vue, le dommage est en fin de compte exactement le même pour la lutte économique du prolétariat si le capitalisme allemand se renforce aux dépens du capitalisme français ou si le capitalisme anglais se renforce aux dépens du capitalisme allemand.\par
Mais tournons-nous vers les conséquences politiques de la guerre. Ici, on devrait pouvoir trancher plus facilement que dans le domaine économique. Depuis toujours, les sympathies et le soutien des socialistes sont allés à celui des belligérants qui combattait pour le progrès historique et contre la réaction. Dans la guerre mondiale actuelle, quel camp représente le progrès et quel camp la réaction ? Il est clair qu’on ne peut juger de cette question d’après les étiquettes extérieures des Etats belligérants, telles que « démocratie » ou « absolutisme », mais uniquement d’après les tendances objectives de la position adoptée par chaque camp dans la politique mondiale. Avant de pouvoir juger de ce qu’une victoire allemande peut apporter au prolétariat allemand, nous devons envisager les conséquences qu’elle aurait sur la configuration d’ensemble de la situation politique en Europe. La victoire nette de l’Allemagne aurait comme première conséquence l’annexion de la Belgique et probablement aussi de quelques morceaux de territoires à l’Est et à l’Ouest et d’une partie des colonies françaises ; elle permettrait en même temps la conservation de la monarchie habsbourgeoise qui s’enrichirait de nouveaux territoires, et enfin la conservation de l’« intégrité » fictive de la Turquie sous protectorat allemand, c’est-à-dire pratiquement la transformation immédiate de l’Asie mineure et de la Mésopotamie en provinces allemandes, sous une forme ou une autre. La deuxième conséquence, ce serait l’hégémonie militaire et économique effective de l’Allemagne en Europe. S'il faut s’attendre à ce qu’une victoire nette de l’Allemagne produise tous ces résultats, ce n’est pas parce qu’ils correspondent aux souhaits des braillards impérialistes au cours de la guerre actuelle, mais bien parce qu’ils découlent inévitablement de la position adoptée par l’Allemagne dans la politique mondiale, des oppositions avec l’Angleterre, la France et la Russie dans lesquelles l’Allemagne est prise et qui au cours de la guerre se sont développées bien au-delà de leurs dimensions initiales. Il suffit cependant de se représenter ces résultats pour comprendre qu’en aucun cas ils ne pourraient donner lieu à un équilibre stable de la politique mondiale. Malgré la ruine qu’aura représentée la guerre pour tous les pays impliqués et plus encore peut-être pour les vaincus, le lendemain même de la conclusion de la paix, des préparatifs en vue d’une nouvelle guerre mondiale commenceront à se faire sous la direction de l’Angleterre, pour secouer le joug du militarisme prusso-allemand qui devrait peser sur l’Europe et l’Asie. Une victoire de l’Allemagne ne serait donc qu’un prélude à une deuxième guerre mondiale qui surviendrait aussitôt après, et de ce fait, ne serait que le point de départ à de nouveaux préparatifs militaires fiévreux ainsi qu’au déchaînement de la réaction la plus noire dans tous les pays, mais en premier lieu en Allemagne même. D'autre part, la victoire de l’Angleterre et de la France conduirait, très vraisemblablement, à la perte pour l’Allemagne d’une partie au moins de ses colonies et de sa métropole et à coup sûr à la faillite de la position de l’impérialisme allemand dans la politique mondiale. Ce qui signifie : le morcellement de l’Autriche-Hongrie et la liquidation complète de la Turquie. Ces deux États sont maintenant des produits si archi-réactionnaires, et leur chute correspond à ce point aux exigences de l’évolution du progrès, que, dans le cadre concret actuel de la politique mondiale, la chute de la monarchie habsbourgeoise et de la Turquie ne pourrait aboutir à rien d’autre qu’à la distribution de leurs territoires et de leurs populations entre la Russie, l’Angleterre, la France et l’Italie. Cette redistribution géographique de grande envergure et ce réajustement des forces dans les Balkans et dans la Méditerranée se prolongerait inévitablement en Asie par la liquidation de la Perse et par un nouveau démantèlement de la Chine. Par là, le conflit anglo-russe et le conflit anglo-japonais passeraient à l’avant-plan de la politique mondiale, ce qui, aussitôt après la liquidation de la guerre mondiale actuelle, entraînerait peut-être une nouvelle guerre mondiale dont l’enjeu pourrait être Constantinople, et ce qui ferait en tout cas de cette guerre une perspective ultérieure inévitable. De ce côté, la victoire conduirait donc aussi à de nouveaux préparatifs militaires fiévreux dans tous les Etats - l’Allemagne vaincue prenant évidemment la tête des opérations - et par là ouvrirait une ère  de domination incontestée du militarisme et de la réaction dans l’Europe entière, avec comme but final une nouvelle guerre mondiale.\par
Ainsi, si dans la guerre actuelle elle doit prendre position pour l’un ou l’autre des deux camps au point de vue du progrès et de la démocratie, en considérant globalement la politique mondiale et ses perspectives futures, la politique prolétarienne est, somme toute, coincée entre Charybde et Scylla, et la question : victoire ou défaite revient dans ces conditions, pour la classe ouvrière européenne, tant sur le plan politique que sur le plan économique, à un choix désespéré entre deux malheurs. Ce n’est donc qu’une funeste illusion de la part des socialistes français que de s’imaginer qu’en écrasant l’Allemagne par les armes ils vont frapper le militarisme ou même l’impérialisme en plein cœur et ouvrir la voie à la démocratie pacifique dans le monde. Tout au contraire : quel que soit le vainqueur de la guerre, l’impérialisme et son serviteur le militarisme y trouveront largement leur compte, sauf dans un cas : si, par son intervention révolutionnaire, le prolétariat vient brouiller leurs comptes.\par
En effet, la leçon la plus importante que la politique du prolétariat doit tirer de la guerre actuelle, c’est l’absolue certitude que ni en Allemagne, ni en France, ni en Angleterre, ni en Russie, le prolétariat ne peut faire sien le mot d’ordre : \emph{victoire ou défaite}, un mot d’ordre qui n’a de sens véritable que pour l’impérialisme et qui, dans chaque grand État, équivaut à la question : renforcement ou perte de sa puissance dans la politique mondiale, de ses annexions, de ses colonies et de sa prédominance militaire. Si on considère la situation actuelle globalement, la victoire ou la défaite de chacun des deux camps est tout aussi funeste pour le prolétariat européen, de son point de vue de classe. C'est la \emph{guerre} elle-même, et quelle que soit son issue militaire, qui représente pour le prolétariat européen la plus grande défaite concevable, et c’est l’élimination de la guerre et la paix imposée aussi rapidement que possible par la lutte internationale du prolétariat qui peuvent apporter la seule victoire à la cause prolétarienne. Et c’est uniquement cette victoire qui permettra de sauver réellement la Belgique et la démocratie européenne.\par
Dans la guerre actuelle, le prolétariat conscient ne peut identifier sa propre cause à aucun des deux camps. En résulte-t-il que la politique du prolétariat exige le maintien du statu quo ? Que nous n’ayons d’autre programme d’action que ce vœux : que subsiste l’ancien état de choses, que tout reste comme avant la guerre ? D'abord nous ne saurions jamais tenir pour idéal l’état de choses existant qui d’ailleurs ne résulte nullement de la libre détermination des peuples ; qui plus est, on ne peut plus en revenir à l’état de choses ancien parce qu’il n’existe plus, même si venaient à subsister les actuelles frontières entre les États. Même avant d’avoir épuisé toutes ses conséquences, la guerre a amené un changement si considérable dans les rapports de force et dans l’évaluation des forces antagonistes, dans les alliances et les oppositions politiques, elle a si radicalement modifié les relations des États entre eux et des classes à l’intérieur de la société, elle a anéanti tant de vieilles illusions et de vieilles lunes, créé tant de nouvelles urgences et de nouvelles tâches, que le retour à la vieille Europe telle qu’elle existait avant le 4 août 1914 est tout à fait exclu, tout comme il est impossible de retourner à la situation qui avait précédé une révolution, même lorsque cette révolution a été écrasée. D'ailleurs, la politique du prolétariat ne connaît pas de « retour en arrière », elle ne peut qu’aller de l’avant, il lui faut toujours aller au-delà de ce qui existe, dépasser ce qui vient d’être créé. C'est en ce sens seulement qu’il peut opposer sa propre politique à celle de chacun des deux camps impérialistes en guerre.\par
Mais pour les partis sociaux-démocrates, cette politique ne saurait consister à se retrouver dans des conférences internationales pour élaborer à qui-mieux-mieux des projets, chacun pour soi ou tous ensemble, et pour inventer des recettes subtiles à l’usage de la diplomatie bourgeoise : il ne s’agit pas de lui expliquer comment elle doit conclure la paix pour permettre à l’avenir une évolution pacifique et démocratique. Toutes les revendications qui tendent par exemple à un « désarmement » total ou partiel, à l’abolition de la diplomatie secrète, au démembrement de tous les grands États en vue de créer des petits Etats nationaux, et tutti quanti, relèvent toutes sans exception de l’utopie, tant que la classe capitaliste tient les rênes en main ; d’autant plus que, étant donné l’orientation impérialiste actuelle, la bourgeoisie ne saurait renoncer au militarisme, à la diplomatie secrète, au grand État multinational centralisé puisque aussi bien tous ces postulats reviennent au fond, si l’on veut être conséquent, à cette simple « exigence » l’abolition de l’État de classe capitaliste. La politique du prolétariat ne peut reconquérir la place qui lui revient en donnant des conseils utopiques ou en élaborant des projets qui permettraient, au moyen de réformes partielles, d’adoucir, de dompter, de modérer l’impérialisme dans le cadre de l’Etat bourgeois. Le problème réel que pose aux partis socialistes cette guerre mondiale, et de la solution duquel dépendent les destins du mouvement ouvrier, c’est la \emph{capacité d’action des masses prolétariennes dans leur lutte contre l’impérialisme}. Ce qui manque au prolétariat international, ce ne sont pas des postulats, des programmes, des mots d’ordre, ce qui lui fait défaut ce sont des actions, une résistance efficace, la capacité d’attaquer l’impérialisme au moment opportun, dans la guerre justement, et de mettre en pratique le vieux mot d’ordre « guerre à la guerre ». C'est ici qu’il faut faire le saut, c’est ici que se situe le nœud gordien de la politique du prolétariat et de son avenir. Il est vrai que l’impérialisme, avec toute la violence brutale de sa politique et la chaîne ininterrompue de catastrophes sociales qu’il provoque, est une nécessité historique pour les classes dirigeantes du monde capitaliste moderne. Rien ne serait plus funeste pour le prolétariat que de garder encore la moindre illusion et le moindre espoir au sortir de la guerre actuelle quant à la possibilité d’une évolution idyllique et paisible du capitalisme. Mais la conclusion qui résulte pour la politique  prolétarienne de la nécessité historique de l’impérialisme n’est pas qu’elle doit capituler devant l’impérialisme pour ronger ensuite à ses pieds l’os qu’il voudra bien lui jeter après sa victoire.\par
La dialectique de l’histoire progresse au milieu des contradictions et, avec chaque chose nécessaire, elle met au monde son contraire. La domination de classe bourgeoise est sans aucun doute une nécessité historique, mais le soulèvement contre elle de la classe ouvrière n’en est pas moins nécessaire. Le capitalisme est une nécessité historique, mais son fossoyeur aussi, le prolétariat socialiste ; la domination mondiale de l’impérialisme est une nécessité historique, mais également son renversement par l’Internationale prolétarienne. A chaque pas existent deux nécessités historiques qui se contestent l’une l’autre, et la nôtre, la nécessité du socialisme, a plus de souffle. Notre nécessité est pleinement justifiée dès le moment où l’autre, la domination de classe bourgeoise, cesse d’être porteuse de progrès historique et devient un carcan et un danger pour l’évolution ultérieure de la société. C'est précisément ce que la guerre actuelle a révélé à propos de l’ordre capitaliste.\par
La force impérialiste d’expansion du capitalisme qui marque son apogée et constitue son dernier stade a pour tendance, sur le plan économique, la métamorphose de la planète en un monde où règne le mode de production capitaliste, l’éviction de toutes les formes de production et de société périmées, précapitalistes, la transmutation de toutes les richesses de la terre et de tous les moyens de production en capital, tandis que les masses laborieuses de tous les pays sont transformées, elles, en esclaves salariés. En Afrique, en Asie, du cap Nord au cap Horn et à l’océan Pacifique, les derniers vestiges de communautés communistes primitives, de conditions de domination féodales, d’économies paysannes patriarcales, de productions artisanales séculaires sont anéantis, foulés aux pieds par le capitalisme qui extermine des peuples entiers et efface de la surface du globe des civilisations millénaires pour y substituer les moyens les plus modernes d’extorquer du profit. Cette marche triomphale au cours de laquelle le capitalisme fraie brutalement sa voie par tous les moyens : la violence, le pillage et l’infamie, possède un côté lumineux : elle a créé les conditions préliminaires à sa propre disparition définitive ; elle a mis en place la domination mondiale du capitalisme à laquelle seule la révolution mondiale du socialisme peut succéder. Tel était le seul aspect culturel et progressiste de ce que l’on a appelé la grande oeuvre civilisatrice du capitalisme dans les pays primitifs. Pour les économistes et les politiciens bourgeois libéraux, des chemins de fer, des allumettes suédoises, des canalisations de rue et des comptoirs de commerce représentent le « progrès » et la « civilisation ». Mais, en eux-mêmes, ces ouvrages greffés sur des conditions économiques primitives ne représentent ni le progrès, ni la civilisation, car ils sont vendus au prix de la ruine économique accélérée des pays où ils sont introduits, leurs peuples ayant à subir d’un seul coup la misère et l’épouvante de deux âges : celui des rapports de domination de l’économie naturelle traditionnelle et celui de l’exploitation capitaliste la plus moderne et la plus raffinée. C'est seulement en tant que conditions préliminaires à la suppression de la domination du capital et à l’abolition de la société de classes que, dans un sens historique plus large, les ouvrages de la marche triomphale du capitalisme portaient la marque du progrès. C'est en ce sens que l’impérialisme, en dernière analyse, travaillait pour nous.\par
La guerre mondiale est un tournant dans l’histoire du capitalisme. Pour la première fois, le fauve que l’Europe capitaliste lâchait sur les autres continents fait irruption d’un seul bond en plein milieu de l’Europe. Un cri d’effroi parcourut le monde lorsque la Belgique, ce précieux petit bijou de la civilisation européenne, ainsi que les monuments culturels les plus vénérables du nord de la France furent ravagés par l’impact d’une force de destruction aveugle. Le « monde civilisé » qui avait assisté avec indifférence aux crimes de ce même impérialisme : lorsqu’il voua des milliers de Hereros à la mort la plus épouvantable et remplit le désert de Kalahari des cris déments d’hommes assoiffés et des râles de moribonds, lorsque sur le Putumayo en l’espace de dix ans quarante mille hommes furent torturés à mort par une bande de chevaliers d’industrie venus d’Europe et que le reste d’un peuple fut rendu infirme, lorsqu’en Chine une civilisation très ancienne fut mise à feu et à sang par la soldatesque européenne et livrée à toutes les horreurs de la destruction et de l’anarchie, lorsque la Perse, impuissante, fut étranglée par les lacets toujours plus serrés de la tyrannie étrangère, lorsqu’à Tripoli les Arabes furent soumis par le feu et l’épée au joug du capital et que leur civilisation et leurs habitations furent rayées de la carte - ce même « monde civilisé » vient seulement de se rendre compte que la morsure du fauve impérialiste est mortelle, que son haleine est scélérate. Il vient de le remarquer maintenant que le fauve a enfoncé ses griffes acérées dans le sein de sa propre mère, la civilisation bourgeoise européenne. Et cette découverte se propage sous la forme de l’hypocrisie bourgeoise qui veut que chaque peuple ne reconnaisse l’infamie que dans l’uniforme national de son adversaire. « Les barbares allemands ! » - comme si chaque peuple qui se prépare au meurtre organisé ne se transformait pas à l’instant même en une horde de barbares ; « les horreurs cosaques ! » comme si la guerre n’était pas en soi l’horreur des horreurs et comme si le fait d’exalter la boucherie humaine comme une entreprise héroïque dans un journal de jeunesse socialiste n’était pas de la graine d’esprit cosaque !\par
Mais le déchaînement actuel du fauve impérialiste dans les campagnes européennes produit encore un autre résultat qui laisse le « monde civilisé » tout à fait indifférent : c’est \emph{la disparition massive du prolétariat européen}. Jamais une guerre n’avait exterminé dans ces proportions des couches entières de population ; jamais, depuis un siècle, elle n’avait frappé de cette façon tous les peuples civilisés d’Europe. Dans les Vosges, dans les Ardennes, en Belgique, en Pologne, dans les Carpates, sur la Save, des millions de vies humaines sont anéanties, des milliers d’hommes sont réduits à l’état d’infirmes. Or, c’est la population ouvrière des villes et des campagnes qui constitue les neuf dixièmes de ces millions de victimes. C'est notre  force, c’est notre espoir qui est fauché quotidiennement sur ces champs de bataille par rangs entiers, comme des épis tombent sous la faucille ; ce sont les forces les meilleures, les plus intelligentes, les mieux éduquées du socialisme international, les porteurs des traditions les plus sacrées, les représentants les plus audacieux, les plus héroïques du mouvement ouvrier moderne, les troupes d’avant-garde de l’ensemble du prolétariat mondial : les ouvriers d’Angleterre, de France, de Belgique, d’Allemagne, de Russie qui sont maintenant massacrés après avoir été bâillonnés. Ces ouvriers des nations capitalistes dirigeantes d’Europe sont ceux à qui incombe la mission historique d’accomplir la révolution socialiste. C'est seulement d’Europe, c’est seulement de ces pays capitalistes les plus anciens que peut venir, lorsque l’heure aura sonné, le signal de la révolution sociale qui libérera l’humanité. Seuls les ouvriers anglais, français, belges, allemands, russes et italiens peuvent ensemble prendre la tête de l’armée des exploités et des opprimés des cinq continents. Eux seuls peuvent, quand le temps sera venu, faire rendre des comptes au capitalisme pour ses crimes séculaires envers tous les peuples primitifs, pour son oeuvre d’anéantissement sur l’ensemble du globe, et eux seuls peuvent exercer des représailles. Mais pour que le socialisme puisse faire sa trouée et remporter la victoire, il faut qu’existent des masses dont la puissance réside tant dans leur niveau culturel que dans leur nombre. Et ce sont ces masses précisément qui sont décimées dans cette guerre. La fleur de l’âge viril et de la jeunesse, des centaines de milliers de prolétaires dont l’éducation socialiste, en Angleterre et en France, en Belgique, en Allemagne et en Russie, était le produit d’un travail d’agitation et d’instruction d’une dizaine d’années, d’autres centaines de milliers qui demain pouvaient être acquis au socialisme - ils tombent et ils tuent misérablement sur les champs de bataille. Le fruit de dizaines d’années de sacrifices et d’efforts de plusieurs générations est anéanti en quelques semaines, les troupes d’élite du prolétariat international sont décimées.\par
La saignée de la boucherie de Juin avait paralysé le mouvement ouvrier français pour une quinzaine d’années. La saignée du carnage de la Commune l’a encore retardé de dix ans. Ce qui a lieu maintenant est un massacre massif sans précédent qui réduit de plus en plus la population ouvrière adulte de tous les pays civilisés qui font la guerre à n’être plus composée que de femmes, de vieillards et d’infirmes. C'est une saignée qui risque d’épuiser mortellement le mouvement ouvrier européen. Encore une guerre de ce genre, et les perspectives du socialisme sont enterrées sous les ruines entassées par la barbarie impérialiste. C'est beaucoup plus grave que la destruction scandaleuse de Louvain et de la cathédrale de Reims. C'est un attentat non pas à la culture bourgeoise du passé, mais à la civilisation socialiste de l’avenir, un coup mortel porté à cette force qui porte en elle l’avenir de l’humanité et qui seule peut transmettre les trésors précieux du passé à une société meilleure. Ici, le capitalisme découvre sa tête de mort, ici il trahit que son droit d’existence historique a fait son temps, que le maintien de sa domination n’est plus compatible avec le progrès de l’humanité.\par
Ici encore la guerre actuelle s’avère non seulement un gigantesque assassinat, mais aussi un suicide de la classe ouvrière européenne. Car ce sont les soldats du socialisme, les prolétaires d’Angleterre, de France, d’Allemagne, de Russie, de Belgique qui depuis des mois se massacrent les uns les autres sur l’ordre du capital, ce sont eux qui enfoncent dans leur cœur le fer meurtrier, s’enlaçant d’une étreinte mortelle, chancelant ensemble, chacun entraînant l’autre dans la tombe.\par
« L'Allemagne, l’Allemagne au-dessus de tout ! Vive la démocratie ! Vive le tsar et le panslavisme ! » Dix mille toiles de tentes garanties standard, cent mille kilos de lard, d’ersatz-café livrables immédiatement ! Les dividendes grimpent et les prolétaires tombent, et avec chacun d’eux c’est un combattant de l’avenir, un soldat de la révolution, un de ceux qui délivreront l’humanité du joug du capitalisme, qui descend au tombeau.\par
Cette folie cessera le jour où les ouvriers d’Allemagne et de France, d’Angleterre et de Russie se réveilleront enfin de leur ivresse et se tendront une main fraternelle couvrant à la fois le chœur bestial des fauteurs de guerre impérialistes et le rauque hurlement des hyènes capitalistes, en poussant le vieux et puissant cri de guerre du Travail :\par

\labelblock{Prolétaires de tous les pays, unissez-vous !}

\section[{Annexe. Thèses sur les tâches de la social-démocratie}]{Annexe \\
Thèses sur les tâches de la social-démocratie}\renewcommand{\leftmark}{Annexe \\
Thèses sur les tâches de la social-démocratie}

\noindent Une majorité de camarades des quatre coins de l’Allemagne a adopté les thèses suivantes, qui présentent une application du programme d’Erfurt au problème actuel du socialisme international.\par
\textbf{1°} La guerre mondiale actuelle a réduit à néant les résultats du travail de quarante années de socialisme européen, en ruinant l’importance de la classe ouvrière révolutionnaire en tant que facteur de pouvoir politique, en ruinant le prestige moral du socialisme, en faisant éclater l’Internationale prolétarienne, en conduisant ses sections à un fratricide mutuel et en enchaînant les vœux et les espoirs des masses populaires dans les pays capitalistes les plus importants au vaisseau de l’impérialisme.\par
\textbf{2°} En votant les crédits de guerre et en proclamant l’Union sacrée, les dirigeants officiels des partis sociaux-démocrates d’Allemagne, de France et d’Angleterre (à l’exception du parti ouvrier indépendant) ont renforcé l’impérialisme sur ses arrières, ont engagé les masses populaires à supporter patiemment la misère et l’horreur de la guerre, et ainsi ont contribué au déchaînement effréné de la fureur impérialiste, au prolongement du massacre et à l’accroissement du nombre de ses victimes ; ils partagent donc la responsabilité de la guerre et de ses conséquences.\par
\textbf{3°} Cette tactique des instances officielles du parti dans les pays belligérants, et en tout premier lieu en Allemagne, qui était jusqu’ici le pays pilote de l’Internationale, équivaut à une trahison des principes les plus élémentaires du socialisme international, des intérêts vitaux de la classe ouvrière et de tous les intérêts démocratiques des peuples. A cause de cette tactique, la politique socialiste était également condamnée à l’impuissance dans les pays où les dirigeants du parti sont restés fidèles à leurs devoirs : en Russie, en Serbie, en Italie, et - avec une exception - en Bulgarie.\par
\textbf{4°} En abandonnant la lutte de classes pour toute la durée de la guerre, et en la renvoyant à la période d’après-guerre, la social-démocratie officielle des pays belligérants a donné le temps aux classes dirigeantes de tous les pays de renforcer considérablement leur position aux dépens du prolétariat sur le plan économique, politique et moral.\par
\textbf{5°} La guerre mondiale ne sert ni la défense nationale ni les intérêts économiques ou politiques des masses populaires quelles qu’elles soient, c’est uniquement un produit de rivalités impérialistes entre les classes capitalistes de différents pays pour la suprématie mondiale et pour le monopole de l’exploitation et de l’oppression des régions qui ne sont pas encore soumises au Capital. A l’époque de cet impérialisme déchaîné il ne peut plus y avoir de guerres nationales. Les intérêts nationaux ne sont qu’une mystification qui a pour but de mettre les masses populaires laborieuses au service de leur ennemi mortel : l’impérialisme.\par
\textbf{6°} Pour aucune nation opprimée, la liberté et l’indépendance ne peuvent jaillir de la politique des États impérialistes et de la guerre impérialiste. Les petites nations, dont les classes dirigeantes sont les jouets et les complices de leurs camarades de classe des grands États, ne sont que des pions dans le jeu impérialiste des grandes puissances, et, tout comme les masses ouvrières des grandes puissances, elles sont utilisées comme instruments pendant la guerre pour être sacrifiées après la guerre aux intérêts capitalistes.\par
\textbf{7°} Dans ces conditions, quel que soit le vainqueur et quel que soit le vaincu, la guerre mondiale actuelle représente une défaite du socialisme et de la démocratie; quelle que soit son issue, elle ne peut conduire qu’au renforcement du militarisme, des conflits internationaux et des rivalités sur le plan de la politique mondiale, sauf au cas d’une intervention révolutionnaire du prolétariat international. Elle augmente l’exploitation capitaliste, accroît la puissance de la réaction dans la politique intérieure, affaiblit le contrôle de l’opinion publique et réduit de plus en plus le Parlement à n’être que l’instrument docile du militarisme. En même temps, la guerre mondiale actuelle développe toutes les conditions favorables à de nouvelles guerres.\par
\textbf{8°} La paix mondiale ne peut être préservée par des plans utopiques ou foncièrement réactionnaires, tels que des tribunaux internationaux de diplomates capitalistes, des conventions diplomatiques sur le « désarmement », la « liberté maritime », la suppression du droit de capture maritime, des « alliances politiques européennes », des « unions douanières en Europe centrale », des Etats tampons nationaux, etc. On ne pourra pas éliminer ou même enrayer l’impérialisme, le militarisme et la guerre aussi longtemps que les classes capitalistes exerceront leur domination de classe de manière incontestée. Le seul moyen de leur résister avec succès et de préserver la paix mondiale, c’est la capacité d’action politique du prolétariat international et sa volonté révolutionnaire de jeter son poids dans la balance.\par
\textbf{9°} L'impérialisme, en tant que dernière phase et apogée de la domination politique mondiale du Capital, est l’ennemi mortel commun du prolétariat de tous les pays. Mais il partage aussi avec les phases antérieures du capitalisme le destin d’accroître les forces de son ennemi mortel à mesure même qu’il se développe. Il accélère la concentration du capital, la stagnation des classes moyennes, l’accroissement du prolétariat, suscite la résistance de plus en plus forte des masses, et conduit ainsi à l’intensification des oppositions entre les classes. Dans la paix comme dans la guerre, la lutte de classe prolétarienne doit concentrer toutes ses forces en premier lieu contre l’impérialisme. Pour le prolétariat international, la lutte contre l’impérialisme est en même temps la lutte pour le pouvoir politique dans l’État, l’épreuve de force décisive entre socialisme et capitalisme. Le but final du socialisme ne sera atteint par le prolétariat international que s’il fait front sur toute la ligne à l’impérialisme et s’il fait du mot d’ordre « guerre à la guerre » la règle de conduite de sa pratique politique, en y mettant toute son énergie et tout son courage.\par
\textbf{10°} Dans ce but, la tâche essentielle du socialisme consiste aujourd’hui à rassembler le prolétariat de tous les pays en une force révolutionnaire vivante et à créer une puissante organisation internationale possédant une seule conception d’ensemble de ses intérêts et de ses tâches, et une tactique et une capacité d’action politique unifiées, de manière à faire du prolétariat le facteur décisif de la vie politique, rôle auquel l’histoire le destine.\par
\textbf{11°} La guerre a fait éclater la IIe Internationale. Sa faillite s’est avérée par son incapacité à lutter efficacement pendant la guerre contre la dispersion nationale et à adopter une tactique et une action communes pour le prolétariat de tous les pays.\par
\textbf{12°} Compte tenu de la trahison des représentations officielles des partis socialistes des pays belligérants envers les objectifs et les intérêts de la classe ouvrière, compte tenu du fait qu’ils ont abandonné les positions de l’Internationale pour rallier celles de la politique bourgeoise-impérialiste, il est d’une nécessité vitale pour le socialisme de créer une nouvelle Internationale ouvrière qui se charge de diriger et de coordonner la lutte de classe révolutionnaire menée contre l’impérialisme dans tous les pays. Pour accomplir sa tâche historique, elle devra s’appuyer sur les principes suivants :\par
La lutte de classe à l’intérieur des États bourgeois contre les classes dirigeantes, et la solidarité internationale des prolétaires de tous les pays sont les deux règles de conduite indispensables que la classe ouvrière doit appliquer dans sa lutte de libération historique. Il n’y a pas de socialisme en dehors de la solidarité internationale du prolétariat, le prolétariat socialiste ne peut renoncer à la lutte de classe et à la solidarité internationale, ni en temps de paix, ni en temps de guerre : cela équivaudrait à un suicide.\par
L'action de classe du prolétariat de tous les pays doit, en temps de paix comme en temps de guerre, se fixer comme but principal de combattre l’impérialisme et de faire obstacle à la guerre. L'action parlementaire, l’action syndicale et l’activité globale du mouvement ouvrier doivent être subordonnées à l’objectif suivant : opposer dans tous les pays, de la manière la plus vive, le prolétariat à la bourgeoisie, souligner à chaque pas l’opposition politique et spirituelle entre les deux classes, tout en mettant en relief et en démontrant l’appartenance commune des prolétaires de tous les pays à l’Internationale.\par
Le centre de gravité de l’organisation de classe du prolétariat réside dans l’Internationale. L'Internationale décide en temps de paix de la tactique des sections nationales au sujet du militarisme, de la politique coloniale, de la politique commerciale, des fêtes de mai, et de plus elle décide de la tactique à adopter en temps de guerre.\par
Le devoir d’appliquer les décisions de l’Internationale précède tous les autres devoirs de l’organisation. Les sections nationales qui contreviennent à ses décisions s’excluent elles-mêmes de l’Internationale.\par
Dans la lutte contre l’impérialisme et la guerre, les forces décisives ne peuvent être engagées que par les masses compactes du prolétariat de tous les pays. La tactique des sections nationales doit par conséquent avoir pour objectif principal de former la capacité d’action politique des masses et leur sens de l’initiative, d’assurer la coordination internationale des actions de masse, de développer les organisations politiques, de telle sorte que par leur intermédiaire on puisse compter à chaque fois sur le concours rapide et énergique de toutes les sections et que la volonté de l’Internationale se concrétise dans l’action des masses ouvrières les plus larges dans tous les pays.\par
La première tâche du socialisme est la libération spirituelle du prolétariat de la tutelle de la bourgeoisie, tutelle qui se manifeste par l’influence de l’idéologie nationaliste. L'action des sections nationales, tant au Parlement que dans la presse, doit avoir pour but de dénoncer le fait que la phraséologie traditionnelle du nationalisme est l’instrument de la domination bourgeoise. Aujourd’hui, la seule défense de toute liberté nationale effective est la lutte de classe révolutionnaire contre l’impérialisme. La patrie des prolétaires, dont la défense prime tout, c’est l’Internationale socialiste.
 


% at least one empty page at end (for booklet couv)
\ifbooklet
  \pagestyle{empty}
  \clearpage
  % 2 empty pages maybe needed for 4e cover
  \ifnum\modulo{\value{page}}{4}=0 \hbox{}\newpage\hbox{}\newpage\fi
  \ifnum\modulo{\value{page}}{4}=1 \hbox{}\newpage\hbox{}\newpage\fi


  \hbox{}\newpage
  \ifodd\value{page}\hbox{}\newpage\fi
  {\centering\color{rubric}\bfseries\noindent\large
    Hurlus ? Qu’est-ce.\par
    \bigskip
  }
  \noindent Des bouquinistes électroniques, pour du texte libre à participation libre,
  téléchargeable gratuitement sur \href{https://hurlus.fr}{\dotuline{hurlus.fr}}.\par
  \bigskip
  \noindent Cette brochure a été produite par des éditeurs bénévoles.
  Elle n’est pas faîte pour être possédée, mais pour être lue, et puis donnée.
  Que circule le texte !
  En page de garde, on peut ajouter une date, un lieu, un nom ; pour suivre le voyage des idées.
  \par

  Ce texte a été choisi parce qu’une personne l’a aimé,
  ou haï, elle a en tous cas pensé qu’il partipait à la formation de notre présent ;
  sans le souci de plaire, vendre, ou militer pour une cause.
  \par

  L’édition électronique est soigneuse, tant sur la technique
  que sur l’établissement du texte ; mais sans aucune prétention scolaire, au contraire.
  Le but est de s’adresser à tous, sans distinction de science ou de diplôme.
  Au plus direct ! (possible)
  \par

  Cet exemplaire en papier a été tiré sur une imprimante personnelle
   ou une photocopieuse. Tout le monde peut le faire.
  Il suffit de
  télécharger un fichier sur \href{https://hurlus.fr}{\dotuline{hurlus.fr}},
  d’imprimer, et agrafer ; puis de lire et donner.\par

  \bigskip

  \noindent PS : Les hurlus furent aussi des rebelles protestants qui cassaient les statues dans les églises catholiques. En 1566 démarra la révolte des gueux dans le pays de Lille. L’insurrection enflamma la région jusqu’à Anvers où les gueux de mer bloquèrent les bateaux espagnols.
  Ce fut une rare guerre de libération dont naquit un pays toujours libre : les Pays-Bas.
  En plat pays francophone, par contre, restèrent des bandes de huguenots, les hurlus, progressivement réprimés par la très catholique Espagne.
  Cette mémoire d’une défaite est éteinte, rallumons-la. Sortons les livres du culte universitaire, cherchons les idoles de l’époque, pour les briser.
\fi

\ifdev % autotext in dev mode
\fontname\font — \textsc{Les règles du jeu}\par
(\hyperref[utopie]{\underline{Lien}})\par
\noindent \initialiv{A}{lors là}\blindtext\par
\noindent \initialiv{À}{ la bonheur des dames}\blindtext\par
\noindent \initialiv{É}{tonnez-le}\blindtext\par
\noindent \initialiv{Q}{ualitativement}\blindtext\par
\noindent \initialiv{V}{aloriser}\blindtext\par
\Blindtext
\phantomsection
\label{utopie}
\Blinddocument
\fi
\end{document}
