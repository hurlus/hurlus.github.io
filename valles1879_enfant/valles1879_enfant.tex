%%%%%%%%%%%%%%%%%%%%%%%%%%%%%%%%%
% LaTeX model https://hurlus.fr %
%%%%%%%%%%%%%%%%%%%%%%%%%%%%%%%%%

% Needed before document class
\RequirePackage{pdftexcmds} % needed for tests expressions
\RequirePackage{fix-cm} % correct units

% Define mode
\def\mode{a4}

\newif\ifaiv % a4
\newif\ifav % a5
\newif\ifbooklet % booklet
\newif\ifcover % cover for booklet

\ifnum \strcmp{\mode}{cover}=0
  \covertrue
\else\ifnum \strcmp{\mode}{booklet}=0
  \booklettrue
\else\ifnum \strcmp{\mode}{a5}=0
  \avtrue
\else
  \aivtrue
\fi\fi\fi

\ifbooklet % do not enclose with {}
  \documentclass[french,twoside]{book} % ,notitlepage
  \usepackage[%
    papersize={105mm, 297mm},
    inner=12mm,
    outer=12mm,
    top=20mm,
    bottom=15mm,
    marginparsep=0pt,
  ]{geometry}
  \usepackage[fontsize=9.5pt]{scrextend} % for Roboto
\else\ifav
  \documentclass[french,twoside]{book} % ,notitlepage
  \usepackage[%
    a5paper,
    inner=25mm,
    outer=15mm,
    top=15mm,
    bottom=15mm,
    marginparsep=0pt,
  ]{geometry}
  \usepackage[fontsize=12pt]{scrextend}
\else% A4 2 cols
  \documentclass[twocolumn]{report}
  \usepackage[%
    a4paper,
    inner=15mm,
    outer=10mm,
    top=25mm,
    bottom=18mm,
    marginparsep=0pt,
  ]{geometry}
  \setlength{\columnsep}{20mm}
  \usepackage[fontsize=9.5pt]{scrextend}
\fi\fi

%%%%%%%%%%%%%%
% Alignments %
%%%%%%%%%%%%%%
% before teinte macros

\setlength{\arrayrulewidth}{0.2pt}
\setlength{\columnseprule}{\arrayrulewidth} % twocol
\setlength{\parskip}{0pt} % classical para with no margin
\setlength{\parindent}{1.5em}

%%%%%%%%%%
% Colors %
%%%%%%%%%%
% before Teinte macros

\usepackage[dvipsnames]{xcolor}
\definecolor{rubric}{HTML}{800000} % the tonic 0c71c3
\def\columnseprulecolor{\color{rubric}}
\colorlet{borderline}{rubric!30!} % definecolor need exact code
\definecolor{shadecolor}{gray}{0.95}
\definecolor{bghi}{gray}{0.5}

%%%%%%%%%%%%%%%%%
% Teinte macros %
%%%%%%%%%%%%%%%%%
%%%%%%%%%%%%%%%%%%%%%%%%%%%%%%%%%%%%%%%%%%%%%%%%%%%
% <TEI> generic (LaTeX names generated by Teinte) %
%%%%%%%%%%%%%%%%%%%%%%%%%%%%%%%%%%%%%%%%%%%%%%%%%%%
% This template is inserted in a specific design
% It is XeLaTeX and otf fonts

\makeatletter % <@@@


\usepackage{blindtext} % generate text for testing
\usepackage[strict]{changepage} % for modulo 4
\usepackage{contour} % rounding words
\usepackage[nodayofweek]{datetime}
% \usepackage{DejaVuSans} % seems buggy for sffont font for symbols
\usepackage{enumitem} % <list>
\usepackage{etoolbox} % patch commands
\usepackage{fancyvrb}
\usepackage{fancyhdr}
\usepackage{float}
\usepackage{fontspec} % XeLaTeX mandatory for fonts
\usepackage{footnote} % used to capture notes in minipage (ex: quote)
\usepackage{framed} % bordering correct with footnote hack
\usepackage{graphicx}
\usepackage{lettrine} % drop caps
\usepackage{lipsum} % generate text for testing
\usepackage[framemethod=tikz,]{mdframed} % maybe used for frame with footnotes inside
\usepackage{pdftexcmds} % needed for tests expressions
\usepackage{polyglossia} % non-break space french punct, bug Warning: "Failed to patch part"
\usepackage[%
  indentfirst=false,
  vskip=1em,
  noorphanfirst=true,
  noorphanafter=true,
  leftmargin=\parindent,
  rightmargin=0pt,
]{quoting}
\usepackage{ragged2e}
\usepackage{setspace} % \setstretch for <quote>
\usepackage{tabularx} % <table>
\usepackage[explicit]{titlesec} % wear titles, !NO implicit
\usepackage{tikz} % ornaments
\usepackage{tocloft} % styling tocs
\usepackage[fit]{truncate} % used im runing titles
\usepackage{unicode-math}
\usepackage[normalem]{ulem} % breakable \uline, normalem is absolutely necessary to keep \emph
\usepackage{verse} % <l>
\usepackage{xcolor} % named colors
\usepackage{xparse} % @ifundefined
\XeTeXdefaultencoding "iso-8859-1" % bad encoding of xstring
\usepackage{xstring} % string tests
\XeTeXdefaultencoding "utf-8"
\PassOptionsToPackage{hyphens}{url} % before hyperref, which load url package

% TOTEST
% \usepackage{hypcap} % links in caption ?
% \usepackage{marginnote}
% TESTED
% \usepackage{background} % doesn’t work with xetek
% \usepackage{bookmark} % prefers the hyperref hack \phantomsection
% \usepackage[color, leftbars]{changebar} % 2 cols doc, impossible to keep bar left
% \usepackage[utf8x]{inputenc} % inputenc package ignored with utf8 based engines
% \usepackage[sfdefault,medium]{inter} % no small caps
% \usepackage{firamath} % choose firasans instead, firamath unavailable in Ubuntu 21-04
% \usepackage{flushend} % bad for last notes, supposed flush end of columns
% \usepackage[stable]{footmisc} % BAD for complex notes https://texfaq.org/FAQ-ftnsect
% \usepackage{helvet} % not for XeLaTeX
% \usepackage{multicol} % not compatible with too much packages (longtable, framed, memoir…)
% \usepackage[default,oldstyle,scale=0.95]{opensans} % no small caps
% \usepackage{sectsty} % \chapterfont OBSOLETE
% \usepackage{soul} % \ul for underline, OBSOLETE with XeTeX
% \usepackage[breakable]{tcolorbox} % text styling gone, footnote hack not kept with breakable


% Metadata inserted by a program, from the TEI source, for title page and runing heads
\title{\textbf{ L’enfant }}
\date{1879}
\author{Vallès, Jules}
\def\elbibl{Vallès, Jules. 1879. \emph{L’enfant}}
\def\elsource{\href{https://www.ebooksgratuits.com/details.php?book=699}{\dotuline{ebooksgratuits.com}}\footnote{\href{https://www.ebooksgratuits.com/details.php?book=699}{\url{https://www.ebooksgratuits.com/details.php?book=699}}}}

% Default metas
\newcommand{\colorprovide}[2]{\@ifundefinedcolor{#1}{\colorlet{#1}{#2}}{}}
\colorprovide{rubric}{red}
\colorprovide{silver}{lightgray}
\@ifundefined{syms}{\newfontfamily\syms{DejaVu Sans}}{}
\newif\ifdev
\@ifundefined{elbibl}{% No meta defined, maybe dev mode
  \newcommand{\elbibl}{Titre court ?}
  \newcommand{\elbook}{Titre du livre source ?}
  \newcommand{\elabstract}{Résumé\par}
  \newcommand{\elurl}{http://oeuvres.github.io/elbook/2}
  \author{Éric Lœchien}
  \title{Un titre de test assez long pour vérifier le comportement d’une maquette}
  \date{1566}
  \devtrue
}{}
\let\eltitle\@title
\let\elauthor\@author
\let\eldate\@date


\defaultfontfeatures{
  % Mapping=tex-text, % no effect seen
  Scale=MatchLowercase,
  Ligatures={TeX,Common},
}


% generic typo commands
\newcommand{\astermono}{\medskip\centerline{\color{rubric}\large\selectfont{\syms ✻}}\medskip\par}%
\newcommand{\astertri}{\medskip\par\centerline{\color{rubric}\large\selectfont{\syms ✻\,✻\,✻}}\medskip\par}%
\newcommand{\asterism}{\bigskip\par\noindent\parbox{\linewidth}{\centering\color{rubric}\large{\syms ✻}\\{\syms ✻}\hskip 0.75em{\syms ✻}}\bigskip\par}%

% lists
\newlength{\listmod}
\setlength{\listmod}{\parindent}
\setlist{
  itemindent=!,
  listparindent=\listmod,
  labelsep=0.2\listmod,
  parsep=0pt,
  % topsep=0.2em, % default topsep is best
}
\setlist[itemize]{
  label=—,
  leftmargin=0pt,
  labelindent=1.2em,
  labelwidth=0pt,
}
\setlist[enumerate]{
  label={\bf\color{rubric}\arabic*.},
  labelindent=0.8\listmod,
  leftmargin=\listmod,
  labelwidth=0pt,
}
\newlist{listalpha}{enumerate}{1}
\setlist[listalpha]{
  label={\bf\color{rubric}\alph*.},
  leftmargin=0pt,
  labelindent=0.8\listmod,
  labelwidth=0pt,
}
\newcommand{\listhead}[1]{\hspace{-1\listmod}\emph{#1}}

\renewcommand{\hrulefill}{%
  \leavevmode\leaders\hrule height 0.2pt\hfill\kern\z@}

% General typo
\DeclareTextFontCommand{\textlarge}{\large}
\DeclareTextFontCommand{\textsmall}{\small}

% commands, inlines
\newcommand{\anchor}[1]{\Hy@raisedlink{\hypertarget{#1}{}}} % link to top of an anchor (not baseline)
\newcommand\abbr[1]{#1}
\newcommand{\autour}[1]{\tikz[baseline=(X.base)]\node [draw=rubric,thin,rectangle,inner sep=1.5pt, rounded corners=3pt] (X) {\color{rubric}#1};}
\newcommand\corr[1]{#1}
\newcommand{\ed}[1]{ {\color{silver}\sffamily\footnotesize (#1)} } % <milestone ed="1688"/>
\newcommand\expan[1]{#1}
\newcommand\foreign[1]{\emph{#1}}
\newcommand\gap[1]{#1}
\renewcommand{\LettrineFontHook}{\color{rubric}}
\newcommand{\initial}[2]{\lettrine[lines=2, loversize=0.3, lhang=0.3]{#1}{#2}}
\newcommand{\initialiv}[2]{%
  \let\oldLFH\LettrineFontHook
  % \renewcommand{\LettrineFontHook}{\color{rubric}\ttfamily}
  \IfSubStr{QJ’}{#1}{
    \lettrine[lines=4, lhang=0.2, loversize=-0.1, lraise=0.2]{\smash{#1}}{#2}
  }{\IfSubStr{É}{#1}{
    \lettrine[lines=4, lhang=0.2, loversize=-0, lraise=0]{\smash{#1}}{#2}
  }{\IfSubStr{ÀÂ}{#1}{
    \lettrine[lines=4, lhang=0.2, loversize=-0, lraise=0, slope=0.6em]{\smash{#1}}{#2}
  }{\IfSubStr{A}{#1}{
    \lettrine[lines=4, lhang=0.2, loversize=0.2, slope=0.6em]{\smash{#1}}{#2}
  }{\IfSubStr{V}{#1}{
    \lettrine[lines=4, lhang=0.2, loversize=0.2, slope=-0.5em]{\smash{#1}}{#2}
  }{
    \lettrine[lines=4, lhang=0.2, loversize=0.2]{\smash{#1}}{#2}
  }}}}}
  \let\LettrineFontHook\oldLFH
}
\newcommand{\labelchar}[1]{\textbf{\color{rubric} #1}}
\newcommand{\milestone}[1]{\autour{\footnotesize\color{rubric} #1}} % <milestone n="4"/>
\newcommand\name[1]{#1}
\newcommand\orig[1]{#1}
\newcommand\orgName[1]{#1}
\newcommand\persName[1]{#1}
\newcommand\placeName[1]{#1}
\newcommand{\pn}[1]{\IfSubStr{-—–¶}{#1}% <p n="3"/>
  {\noindent{\bfseries\color{rubric}   ¶  }}
  {{\footnotesize\autour{ #1}  }}}
\newcommand\reg{}
% \newcommand\ref{} % already defined
\newcommand\sic[1]{#1}
\newcommand\surname[1]{\textsc{#1}}
\newcommand\term[1]{\textbf{#1}}

\def\mednobreak{\ifdim\lastskip<\medskipamount
  \removelastskip\nopagebreak\medskip\fi}
\def\bignobreak{\ifdim\lastskip<\bigskipamount
  \removelastskip\nopagebreak\bigskip\fi}

% commands, blocks
\newcommand{\byline}[1]{\bigskip{\RaggedLeft{#1}\par}\bigskip}
\newcommand{\bibl}[1]{{\RaggedLeft{#1}\par\bigskip}}
\newcommand{\biblitem}[1]{{\noindent\hangindent=\parindent   #1\par}}
\newcommand{\dateline}[1]{\medskip{\RaggedLeft{#1}\par}\bigskip}
\newcommand{\labelblock}[1]{\medbreak{\noindent\color{rubric}\bfseries #1}\par\mednobreak}
\newcommand{\salute}[1]{\bigbreak{#1}\par\medbreak}
\newcommand{\signed}[1]{\bigbreak\filbreak{\raggedleft #1\par}\medskip}

% environments for blocks (some may become commands)
\newenvironment{borderbox}{}{} % framing content
\newenvironment{citbibl}{\ifvmode\hfill\fi}{\ifvmode\par\fi }
\newenvironment{docAuthor}{\ifvmode\vskip4pt\fontsize{16pt}{18pt}\selectfont\fi\itshape}{\ifvmode\par\fi }
\newenvironment{docDate}{}{\ifvmode\par\fi }
\newenvironment{docImprint}{\vskip6pt}{\ifvmode\par\fi }
\newenvironment{docTitle}{\vskip6pt\bfseries\fontsize{18pt}{22pt}\selectfont}{\par }
\newenvironment{msHead}{\vskip6pt}{\par}
\newenvironment{msItem}{\vskip6pt}{\par}
\newenvironment{titlePart}{}{\par }


% environments for block containers
\newenvironment{argument}{\itshape\parindent0pt}{\vskip1.5em}
\newenvironment{biblfree}{}{\ifvmode\par\fi }
\newenvironment{bibitemlist}[1]{%
  \list{\@biblabel{\@arabic\c@enumiv}}%
  {%
    \settowidth\labelwidth{\@biblabel{#1}}%
    \leftmargin\labelwidth
    \advance\leftmargin\labelsep
    \@openbib@code
    \usecounter{enumiv}%
    \let\p@enumiv\@empty
    \renewcommand\theenumiv{\@arabic\c@enumiv}%
  }
  \sloppy
  \clubpenalty4000
  \@clubpenalty \clubpenalty
  \widowpenalty4000%
  \sfcode`\.\@m
}%
{\def\@noitemerr
  {\@latex@warning{Empty `bibitemlist' environment}}%
\endlist}
\newenvironment{quoteblock}% may be used for ornaments
  {\begin{quoting}}
  {\end{quoting}}

% table () is preceded and finished by custom command
\newcommand{\tableopen}[1]{%
  \ifnum\strcmp{#1}{wide}=0{%
    \begin{center}
  }
  \else\ifnum\strcmp{#1}{long}=0{%
    \begin{center}
  }
  \else{%
    \begin{center}
  }
  \fi\fi
}
\newcommand{\tableclose}[1]{%
  \ifnum\strcmp{#1}{wide}=0{%
    \end{center}
  }
  \else\ifnum\strcmp{#1}{long}=0{%
    \end{center}
  }
  \else{%
    \end{center}
  }
  \fi\fi
}


% text structure
\newcommand\chapteropen{} % before chapter title
\newcommand\chaptercont{} % after title, argument, epigraph…
\newcommand\chapterclose{} % maybe useful for multicol settings
\setcounter{secnumdepth}{-2} % no counters for hierarchy titles
\setcounter{tocdepth}{5} % deep toc
\markright{\@title} % ???
\markboth{\@title}{\@author} % ???
\renewcommand\tableofcontents{\@starttoc{toc}}
% toclof format
% \renewcommand{\@tocrmarg}{0.1em} % Useless command?
% \renewcommand{\@pnumwidth}{0.5em} % {1.75em}
\renewcommand{\@cftmaketoctitle}{}
\setlength{\cftbeforesecskip}{\z@ \@plus.2\p@}
\renewcommand{\cftchapfont}{}
\renewcommand{\cftchapdotsep}{\cftdotsep}
\renewcommand{\cftchapleader}{\normalfont\cftdotfill{\cftchapdotsep}}
\renewcommand{\cftchappagefont}{\bfseries}
\setlength{\cftbeforechapskip}{0em \@plus\p@}
% \renewcommand{\cftsecfont}{\small\relax}
\renewcommand{\cftsecpagefont}{\normalfont}
% \renewcommand{\cftsubsecfont}{\small\relax}
\renewcommand{\cftsecdotsep}{\cftdotsep}
\renewcommand{\cftsecpagefont}{\normalfont}
\renewcommand{\cftsecleader}{\normalfont\cftdotfill{\cftsecdotsep}}
\setlength{\cftsecindent}{1em}
\setlength{\cftsubsecindent}{2em}
\setlength{\cftsubsubsecindent}{3em}
\setlength{\cftchapnumwidth}{1em}
\setlength{\cftsecnumwidth}{1em}
\setlength{\cftsubsecnumwidth}{1em}
\setlength{\cftsubsubsecnumwidth}{1em}

% footnotes
\newif\ifheading
\newcommand*{\fnmarkscale}{\ifheading 0.70 \else 1 \fi}
\renewcommand\footnoterule{\vspace*{0.3cm}\hrule height \arrayrulewidth width 3cm \vspace*{0.3cm}}
\setlength\footnotesep{1.5\footnotesep} % footnote separator
\renewcommand\@makefntext[1]{\parindent 1.5em \noindent \hb@xt@1.8em{\hss{\normalfont\@thefnmark . }}#1} % no superscipt in foot
\patchcmd{\@footnotetext}{\footnotesize}{\footnotesize\sffamily}{}{} % before scrextend, hyperref


%   see https://tex.stackexchange.com/a/34449/5049
\def\truncdiv#1#2{((#1-(#2-1)/2)/#2)}
\def\moduloop#1#2{(#1-\truncdiv{#1}{#2}*#2)}
\def\modulo#1#2{\number\numexpr\moduloop{#1}{#2}\relax}

% orphans and widows
\clubpenalty=9996
\widowpenalty=9999
\brokenpenalty=4991
\predisplaypenalty=10000
\postdisplaypenalty=1549
\displaywidowpenalty=1602
\hyphenpenalty=400
% Copied from Rahtz but not understood
\def\@pnumwidth{1.55em}
\def\@tocrmarg {2.55em}
\def\@dotsep{4.5}
\emergencystretch 3em
\hbadness=4000
\pretolerance=750
\tolerance=2000
\vbadness=4000
\def\Gin@extensions{.pdf,.png,.jpg,.mps,.tif}
% \renewcommand{\@cite}[1]{#1} % biblio

\usepackage{hyperref} % supposed to be the last one, :o) except for the ones to follow
\urlstyle{same} % after hyperref
\hypersetup{
  % pdftex, % no effect
  pdftitle={\elbibl},
  % pdfauthor={Your name here},
  % pdfsubject={Your subject here},
  % pdfkeywords={keyword1, keyword2},
  bookmarksnumbered=true,
  bookmarksopen=true,
  bookmarksopenlevel=1,
  pdfstartview=Fit,
  breaklinks=true, % avoid long links
  pdfpagemode=UseOutlines,    % pdf toc
  hyperfootnotes=true,
  colorlinks=false,
  pdfborder=0 0 0,
  % pdfpagelayout=TwoPageRight,
  % linktocpage=true, % NO, toc, link only on page no
}

\makeatother % /@@@>
%%%%%%%%%%%%%%
% </TEI> end %
%%%%%%%%%%%%%%


%%%%%%%%%%%%%
% footnotes %
%%%%%%%%%%%%%
\renewcommand{\thefootnote}{\bfseries\textcolor{rubric}{\arabic{footnote}}} % color for footnote marks

%%%%%%%%%
% Fonts %
%%%%%%%%%
\usepackage[]{roboto} % SmallCaps, Regular is a bit bold
% \linespread{0.90} % too compact, keep font natural
\newfontfamily\fontrun[]{Roboto Condensed Light} % condensed runing heads
\ifav
  \setmainfont[
    ItalicFont={Roboto Light Italic},
  ]{Roboto}
\else\ifbooklet
  \setmainfont[
    ItalicFont={Roboto Light Italic},
  ]{Roboto}
\else
\setmainfont[
  ItalicFont={Roboto Italic},
]{Roboto Light}
\fi\fi
\renewcommand{\LettrineFontHook}{\bfseries\color{rubric}}
% \renewenvironment{labelblock}{\begin{center}\bfseries\color{rubric}}{\end{center}}

%%%%%%%%
% MISC %
%%%%%%%%

\setdefaultlanguage[frenchpart=false]{french} % bug on part


\newenvironment{quotebar}{%
    \def\FrameCommand{{\color{rubric!10!}\vrule width 0.5em} \hspace{0.9em}}%
    \def\OuterFrameSep{\itemsep} % séparateur vertical
    \MakeFramed {\advance\hsize-\width \FrameRestore}
  }%
  {%
    \endMakeFramed
  }
\renewenvironment{quoteblock}% may be used for ornaments
  {%
    \savenotes
    \setstretch{0.9}
    \normalfont
    \begin{quotebar}
  }
  {%
    \end{quotebar}
    \spewnotes
  }


\renewcommand{\headrulewidth}{\arrayrulewidth}
\renewcommand{\headrule}{{\color{rubric}\hrule}}

% delicate tuning, image has produce line-height problems in title on 2 lines
\titleformat{name=\chapter} % command
  [display] % shape
  {\vspace{1.5em}\centering} % format
  {} % label
  {0pt} % separator between n
  {}
[{\color{rubric}\huge\textbf{#1}}\bigskip] % after code
% \titlespacing{command}{left spacing}{before spacing}{after spacing}[right]
\titlespacing*{\chapter}{0pt}{-2em}{0pt}[0pt]

\titleformat{name=\section}
  [block]{}{}{}{}
  [\vbox{\color{rubric}\large\raggedleft\textbf{#1}}]
\titlespacing{\section}{0pt}{0pt plus 4pt minus 2pt}{\baselineskip}

\titleformat{name=\subsection}
  [block]
  {}
  {} % \thesection
  {} % separator \arrayrulewidth
  {}
[\vbox{\large\textbf{#1}}]
% \titlespacing{\subsection}{0pt}{0pt plus 4pt minus 2pt}{\baselineskip}

\ifaiv
  \fancypagestyle{main}{%
    \fancyhf{}
    \setlength{\headheight}{1.5em}
    \fancyhead{} % reset head
    \fancyfoot{} % reset foot
    \fancyhead[L]{\truncate{0.45\headwidth}{\fontrun\elbibl}} % book ref
    \fancyhead[R]{\truncate{0.45\headwidth}{ \fontrun\nouppercase\leftmark}} % Chapter title
    \fancyhead[C]{\thepage}
  }
  \fancypagestyle{plain}{% apply to chapter
    \fancyhf{}% clear all header and footer fields
    \setlength{\headheight}{1.5em}
    \fancyhead[L]{\truncate{0.9\headwidth}{\fontrun\elbibl}}
    \fancyhead[R]{\thepage}
  }
\else
  \fancypagestyle{main}{%
    \fancyhf{}
    \setlength{\headheight}{1.5em}
    \fancyhead{} % reset head
    \fancyfoot{} % reset foot
    \fancyhead[RE]{\truncate{0.9\headwidth}{\fontrun\elbibl}} % book ref
    \fancyhead[LO]{\truncate{0.9\headwidth}{\fontrun\nouppercase\leftmark}} % Chapter title, \nouppercase needed
    \fancyhead[RO,LE]{\thepage}
  }
  \fancypagestyle{plain}{% apply to chapter
    \fancyhf{}% clear all header and footer fields
    \setlength{\headheight}{1.5em}
    \fancyhead[L]{\truncate{0.9\headwidth}{\fontrun\elbibl}}
    \fancyhead[R]{\thepage}
  }
\fi

\ifav % a5 only
  \titleclass{\section}{top}
\fi

\newcommand\chapo{{%
  \vspace*{-3em}
  \centering % no vskip ()
  {\Large\addfontfeature{LetterSpace=25}\bfseries{\elauthor}}\par
  \smallskip
  {\large\eldate}\par
  \bigskip
  {\Large\selectfont{\eltitle}}\par
  \bigskip
  {\color{rubric}\hline\par}
  \bigskip
  {\Large TEXTE LIBRE À PARTICPATION LIBRE\par}
  \centerline{\small\color{rubric} {hurlus.fr, tiré le \today}}\par
  \bigskip
}}

\newcommand\cover{{%
  \thispagestyle{empty}
  \centering
  {\LARGE\bfseries{\elauthor}}\par
  \bigskip
  {\Large\eldate}\par
  \bigskip
  \bigskip
  {\LARGE\selectfont{\eltitle}}\par
  \vfill\null
  {\color{rubric}\setlength{\arrayrulewidth}{2pt}\hline\par}
  \vfill\null
  {\Large TEXTE LIBRE À PARTICPATION LIBRE\par}
  \centerline{{\href{https://hurlus.fr}{\dotuline{hurlus.fr}}, tiré le \today}}\par
}}

\begin{document}
\pagestyle{empty}
\ifbooklet{
  \cover\newpage
  \thispagestyle{empty}\hbox{}\newpage
  \cover\newpage\noindent Les voyages de la brochure\par
  \bigskip
  \begin{tabularx}{\textwidth}{l|X|X}
    \textbf{Date} & \textbf{Lieu}& \textbf{Nom/pseudo} \\ \hline
    \rule{0pt}{25cm} &  &   \\
  \end{tabularx}
  \newpage
  \addtocounter{page}{-4}
}\fi

\thispagestyle{empty}
\ifaiv
  \twocolumn[\chapo]
\else
  \chapo
\fi
{\it\elabstract}
\bigskip
\makeatletter\@starttoc{toc}\makeatother % toc without new page
\bigskip

\pagestyle{main} % after style

  \section[{Dédicace}]{Dédicace}\renewcommand{\leftmark}{Dédicace}


\begin{verse}
À TOUS CEUXqui crevèrent d’ennui au collège\\
ou\\
qu’on fit pleurer dans la famille\\
qui, pendant leur enfance,\\
furent tyrannisés par leurs maîtres\\
ou\\
rossés par leurs parents\\!

Je dédie ce livre.\\
Jules \surname{Vallès}.\\!
\end{verse}
\section[{1. Ma mère}]{1. \\
Ma mère}\renewcommand{\leftmark}{1. \\
Ma mère}

\noindent Ai-je été nourri par ma mère ? Est-ce une paysanne qui m’a donné son lait ? Je n’en sais rien. Quel que soit le sein que j’ai mordu, je ne me rappelle pas une caresse du temps où j’étais tout petit ; je n’ai pas été dorloté, tapoté, baisoté ; j’ai été beaucoup fouetté.\par
Ma mère dit qu’il ne faut pas gâter les enfants, et elle me fouette tous les matins ; quand elle n’a pas le temps le matin, c’est pour midi, rarement plus tard que quatre heures.\par
Mademoiselle Balandreau m’y met du suif.\par
C’est une bonne vieille fille de cinquante ans. Elle demeure au-dessous de nous. D’abord elle était contente : comme elle n’a pas d’horloge, ça lui donnait l’heure. « Vlin ! Vlan ! Zon ! Zon ! – voilà le petit Chose qu’on fouette ; il est temps de faire mon café au lait. »\par
Mais un jour que j’avais levé mon pan, parce que ça me cuisait trop, et que je prenais l’air entre deux portes, elle m’a vu ; mon derrière lui a fait pitié.\par
Elle voulait d’abord le montrer à tout le monde, ameuter les voisins autour ; mais elle a pensé que ce n’était pas le moyen de le sauver, et elle a inventé autre chose.\par
Lorsqu’elle entend ma mère me dire : « Jacques, je vais te fouetter !\par
— Madame Vingtras, ne vous donnez pas la peine, je vais faire ça pour vous.\par
— Oh ! chère demoiselle, vous êtes trop bonne ! »\par
Mademoiselle Balandreau m’emmène ; mais au lieu de me fouetter, elle frappe dans ses mains ; moi, je crie. Ma mère remercie, le soir, sa remplaçante.\par
« À votre service » répond la brave fille, en me glissant un bonbon en cachette.\par
Mon premier souvenir date donc d’une fessée. Mon second est plein d’étonnement et de larmes.\par
\bigbreak
\noindent C’est au coin d’un feu de fagots, sous le manteau d’une vieille cheminée ; ma mère tricote dans un coin ; une cousine à moi, qui sert de bonne dans la maison pauvre, range sur des planches rongées quelques assiettes de grosse faïence avec des coqs à crête rouge et à queue bleue.\par
Mon père a un couteau à la main et taille un morceau de sapin ; les copeaux tombent jaunes et soyeux comme des brins de rubans. Il me fait un chariot avec des languettes de bois frais. Les roues sont déjà taillées ; ce sont des ronds de pommes de terre avec leur cercle de peau brune qui imite le fer… Le chariot va être fini ; j’attends tout ému et les yeux grands ouverts, quand mon père pousse un cri et lève sa main pleine de sang. Il s’est enfoncé le couteau dans le doigt. Je deviens tout pâle et je m’avance vers lui ; un coup violent m’arrête ; c’est ma mère qui me l’a donné, l’écume aux lèvres, les poings crispés.\par
« C’est ta faute si ton père s’est fait mal ! »\par
Et elle me chasse sur l’escalier noir, en me cognant encore le front contre la porte.\par
Je crie, je demande grâce, et j’appelle mon père : je vois, avec ma terreur d’enfant, sa main qui pend toute hachée ; c’est moi qui en suis cause ! Pourquoi ne me laisse-t-on pas entrer pour savoir ? On me battra après si l’on veut. Je crie, on ne me répond pas. J’entends qu’on remue des carafes, qu’on ouvre un tiroir ; on met des compresses.\par
« Ce n’est rien, » vient me dire ma cousine, en pliant une bande de linge tachée de rouge.\par
Je sanglote, j’étouffe : ma mère reparaît et me pousse dans le cabinet où je couche, où j’ai peur tous les soirs.\par
Je puis avoir cinq ans et me crois un parricide.\par
Ce n’est pas ma faute, pourtant !\par
Est-ce que j’ai forcé mon père à faire ce chariot ? Est-ce que je n’aurais pas mieux aimé saigner, moi, et qu’il n’eût point mal ?\par
Oui – et je m’égratigne les mains pour avoir mal aussi.\par
C’est que maman aime tant mon père ! Voilà pourquoi elle s’est emportée.\par
On me fait apprendre à lire dans un livre où il y a écrit en grosses lettres, qu’il faut obéir à ses père et mère : ma mère a bien fait de me battre.\par
\bigbreak
\noindent La maison que nous habitons est dans une rue sale, pénible à gravir, du haut de laquelle on embrasse tout le pays, mais où les voitures ne passent pas. Il n’y a que les charrettes de bois qui y arrivent, traînées par des bœufs qu’on pique avec un aiguillon. Ils vont, le cou tendu, le pied glissant ; leur langue pend et leur peau fume. Je m’arrête toujours à les voir, quand ils portent des fagots et de la farine chez le boulanger qui est à mi-côte ; je regarde en même temps les mitrons tout blancs et le grand four tout rouge, – on enfourne avec de grandes pelles, et ça sent la croûte et la braise !\par
\bigbreak
\noindent La prison est au bout de la rue, et les gendarmes conduisent souvent des prisonniers qui ont les menottes, et qui marchent sans regarder ni à droite ni à gauche, l’œil fixe, l’air malade.\par
Des femmes leur donnent des sous qu’ils serrent dans leurs mains en inclinant la tête pour remercier.\par
Ils n’ont pas du tout l’air méchant.\par
Un jour on en a emmené un sur une civière, avec un drap blanc qui le couvrait tout entier ; il s’était mis le poignet sous une scie, après avoir volé ; il avait coulé tant de sang qu’on croyait qu’il allait mourir.\par
Le geôlier, en sa qualité de voisin, est un ami de la maison ; il vient de temps en temps manger la soupe chez les gens d’en bas, et nous sommes camarades, son fils et moi. Il m’emmène quelquefois à la prison, parce que c’est plus gai. C’est plein d’arbres ; on joue, on rit, et il y en a un, tout vieux, qui vient du bagne et qui fait des cathédrales avec des bouchons et des coquilles de noix.\par
À la maison, l’on ne rit jamais ; ma mère bougonne toujours. – Oh ! comme je m’amuse davantage avec ce vieux là et le grand qu’on appelle le braconnier, qui a tué le gendarme à la foire du Vivarais !\par
Puis, ils reçoivent des bouquets qu’ils embrassent et cachent sur leur poitrine. J’ai vu, en passant au parloir, que c’étaient des femmes qui les leur donnaient.\par
D’autres ont des oranges et des gâteaux que leurs mères leur portent, comme s’ils étaient encore tout petits. Moi, je suis tout petit, et je n’ai jamais ni gâteaux, ni oranges.\par
Je ne me rappelle pas avoir vu une fleur à la maison. Maman dit que ça gêne, et qu’au bout de deux jours ça sent mauvais. Je m’étais piqué à une rose l’autre soir, elle m’a crié : « Ça t’apprendra ! »\par
\bigbreak
\noindent J’ai toujours envie de rire quand on dit la prière. J’ai beau me retenir ! Je prie Dieu avant de me mettre à genoux, je lui jure bien que ce n’est pas de lui que je ris, mais, dès que je suis à genoux, c’est plus fort que moi. Mon oncle a des verrues qui le démangent, et il les gratte, puis il les mord ; j’éclate. – Ma mère ne s’en aperçoit pas toujours, heureusement ; mais Dieu, qui voit tout, qu’est-ce qu’il peut penser ?\par
Je n’ai pas ri pourtant, l’autre jour ! On avait dîné à la maison avec ma tante de Vourzac et mes oncles de Farreyrolles ; on était en train de manger la \emph{tourte}, quand tout à coup il a fait noir. On avait eu chaud tout le temps, on étouffait, et l’on avait ôté ses habits. Voilà que le tonnerre a grondé. La pluie est tombée à torrents, de grosses gouttes faisaient \emph{floc} dans la poussière. Il y avait une fraîcheur de cave, et aussi une odeur de poudre ; dans la rue, le ruisseau bouillait comme une lessive, puis les vitres se sont mises à grincer ; il tombait de la grêle.\par
Mes tantes et mes oncles se sont regardés, et l’un d’eux s’est levé ; il a ôté son chapeau et s’est mis à dire une prière. Tous se tenaient debout et découverts, avec leurs fronts jeunes ou vieux pleins de tristesse. Ils priaient Dieu de n’être pas trop cruel pour leurs champs, et de ne pas tuer, avec son plomb blanc, leurs moissons en fleur.\par
Un grêlon a passé par une fenêtre, au moment où l’on disait \emph{Amen}, et a sauté dans un verre.\par
\bigbreak
\noindent Nous venons de la campagne.\par
Mon père est fils d’un paysan qui a eu de l’orgueil et a voulu que son fils étudiât \emph{pour être prêtre}. On a mis ce fils chez un oncle curé pour apprendre le latin, puis on l’a envoyé au séminaire.\par
Mon père – celui qui devait être mon père – n’y est pas resté, a voulu être bachelier, arriver aux honneurs, et s’est installé dans une petite chambre au fond d’une rue noire, d’où il sort, le jour, pour donner quelques leçons à dix sous l’heure, et où il rentre le soir, pour faire la cour à une paysanne qui sera ma mère, et qui accomplit pour le moment ses devoirs de nièce dévouée près d’une tante malade.\par
On se brouille pour cela avec l’oncle curé, on dit adieu à l’Église ; on s’aime, on \emph{s’accorde}, on s’épouse ! On est aussi au plus mal avec les père et mère, à qui l’on a fait des sommations pour arriver à ce mariage de la débine et de la misère.\par
Je suis le premier enfant de cette union bénie. Je viens au monde dans un lit de vieux bois qui a des punaises de village et des puces de séminaire.\par
\bigbreak
\noindent La maison appartient à une dame de cinquante ans qui n’a que deux dents, l’une marron et l’autre bleue, et qui rit toujours ; elle est bonne et tout le monde l’aime. Son mari s’est noyé en faisant le vin dans une cuve ; ce qui me fait beaucoup rêver et me donne grand’peur des cuves, mais grand amour du vin. Il faut que ce soit bien bon pour que M. Garnier – c’est son nom – en ait pris jusqu’à mourir. Madame Garnier boit, tous les dimanches, de ce vin qui sent l’homme qu’elle a aimé : les souliers du mort sont aussi sur une planche, comme deux chopines vides.\par
On se grise pas mal dans la maison où je demeure.\par
Un abbé qui reste sur notre carré ne sort jamais de table sans avoir les yeux hors de la tête, les joues luisantes, l’oreille en feu. Sa bouche laisse passer un souffle qui sent le fût, et son nez a l’air d’une tomate écorchée. Son bréviaire embaume la matelote.\par
Il a une bonne, mademoiselle Henriette, qu’il regarde de côté, quand il a bu. On parle quelquefois d’elle et de lui dans les coins.\par
Au second, M. Grélin. Il est lieutenant des pompiers, et, le jour de la Fête-Dieu, il commande sur la place. M. Grélin est architecte, mais on dit qu’il n’y entend rien, que « c’est lui qui est cause que le Breuil est toujours plein d’eau, qu’il a coûté cinquante mille francs à la ville, et que, \emph{sans sa femme}… » On dit je ne sais quoi de sa femme. Elle est gentille, avec de grands yeux noirs, de petites dents blanches, un peu de moustache sur la lèvre ; elle fait toujours bouffer son jupon et sonner ses talons quand elle marche.\par
Elle a l’accent du Midi, et nous nous amusons à l’imiter quelquefois.\par
On dit qu’elle a des « amants ». Je ne sais pas ce que c’est, mais je sais bien qu’elle est bonne pour moi, qu’elle me donne, en passant, des tapes sur les joues, et que j’aime à ce qu’elle m’embrasse, parce qu’elle sent bon. Les gens de la maison ont l’air de l’éviter un peu, mais sans le lui montrer.\par
» Vous dites donc qu’elle est bien avec l’adjoint ?\par
— Oui, oui, au mieux !\par
— Ah ! ah ! et ce pauvre Grélin ? »\par
J’entends cela de temps en temps, et ma mère ajoute des mots que je ne comprends pas.\par
« Nous autres, les honnêtes femmes, nous mourons de faim. Celles-là, on leur fourre des places pour leurs maris, des robes pour leurs fêtes ! »\par
Est-ce que madame Grélin n’est pas honnête ? Que fait-elle ? Qu’y a-t-il ? pauvre Grélin ! Mais Grélin a l’air content comme tout. Ils sont toujours à donner des caresses et des joujoux à leurs enfants ; on ne me donne que des gifles, on ne me parle que de l’enfer, on me dit toujours que je crie trop. Je serais bien plus heureux si j’étais le fils à Grélin : mais voilà ! L’adjoint viendrait chez nous quand ma mère serait seule… Ça me serait bien égal, à moi. Madame Toullier reste au troisième : voilà une femme honnête ! Madame Toullier vient à la maison avec son ouvrage, et ma mère et elle causent des gens d’en bas, des gens de dessus, et aussi des gens de Raphaël et d’Espailly. Madame Toullier prise, a des poils plein les oreilles, des pieds avec des oignons ; elle est plus honnête que madame Grélin. Elle est plus bête et plus laide aussi.\par
\bigbreak
\noindent Quels souvenirs ai-je encore de ma vie de petit enfant ? Je me rappelle que, devant la fenêtre, les oiseaux viennent l’hiver picorer dans la neige ; que, l’été, je salis mes culottes dans une cour qui sent mauvais ; qu’au fond de la cave, un des locataires engraisse des dindes. On me laisse pétrir des boulettes de son mouillé, avec lesquelles on les bourre, et elles étouffent. Ma grande joie est de les voir suffoquer, devenir bleues. Il paraît que j’aime le bleu !\par
Ma mère apparaît souvent pour me prendre par les oreilles et me calotter. C’est pour mon bien ; aussi, plus elle m’arrache de cheveux, plus elle me donne de taloches, et plus je suis persuadé qu’elle est une bonne mère et que je suis un enfant ingrat.\par
Oui ingrat ! car il m’est arrivé quelquefois, le soir, en grattant mes bosses, de ne pas me mettre à la bénir, et c’est à la fin de mes prières, tout à fait, que je demande à Dieu de lui garder la santé pour veiller sur moi et me continuer ses bons soins.\par
\bigbreak
\noindent Je suis grand, je vais à l’école.\par
Oh ! la belle petite école ! Oh ! la belle rue ! et si vivante, les jours de foire !\par
Les chevaux qui hennissent ; les cochons qui se traînent en grognant, une corde à la patte ; les poulets qui s’égosillent dans les cages ; les paysannes en tablier vert, avec des jupons écarlates ; les fromages bleus, les \emph{tomes} fraîches, les paniers de fruits ; les radis roses, les choux verts !…\par
Il y avait une auberge tout près de l’école, et l’on y déchargeait souvent du foin.\par
Le foin, où l’on s’enfouissait jusqu’aux yeux, d’où l’on sortait hérissé et suant, avec des brins qui vous étaient restés dans le cou, le dos, les jambes, et vous piquaient comme des épingles !…\par
On perdait ses livres dans la meule, son petit panier, son ceinturon, une galoche… Toutes les joies d’une fête, toutes les émotions d’un danger… Quelles minutes !\par
Quand il passe une voiture de foin, j’ôte mon chapeau et je la suis.
\section[{2. La famille}]{2. \\
La famille}\renewcommand{\leftmark}{2. \\
La famille}

\noindent Deux tantes du côté de ma mère, la tante Rosalie et la tatan Mariou. On appelle cette dernière \emph{tatan} ; je ne sais pourquoi, parce qu’elle est plus caressante peut-être. Je vois toujours son grand rire blanc et doux dans son visage brun : elle est maigre et assez gracieuse, elle est femme.\par
Ma tante Rosalie, son aînée, est énorme, un peu voûtée ; elle a l’air d’un chantre ; elle ressemble au père Jauchard, le boulanger, qui entonne les vêpres le dimanche et qui commence les cantiques quand on fait le Chemin de la croix. Elle est \emph{l’homme} dans son ménage ; son mari, mon oncle Jean, ne compte pas : il se contente de gratter une petite verrue qui joue le grain de beauté dans son visage fripé, tiré, ridé. – J’ai remarqué, depuis, que beaucoup de paysans ont de ces figures-là, rusées, vieillottes, pointues ; ils ont du sang de théâtre ou de cour qui s’est égaré un soir de fête ou de comédie dans la grange ou l’auberge, ils sentent le cabotin, le ci-devant, le vieux noble, à travers les odeurs de l’étable à cochons et du fumier : ratatinés par leur origine, ils restent gringalets sous les grands soleils.\par
Le mari de la tatan Mariou, lui, est bien un bouvier ! Un beau laboureur blond, cinq pieds sept pouces, pas de barbe, mais des poils qui luisent sur son cou, un cou rond, gras, doré ; il a la peau couleur de paille, avec des yeux comme des bleuets et des lèvres comme des coquelicots ; il a toujours la chemise entrouverte, un gilet rayé jaune, et son grand chapeau à chenille tricolore ne le quitte jamais. J’ai vu comme cela des dieux des champs dans des paysages de peintres.\par
Deux tantes du côté de mon père.\par
Ma tante Mélie est muette, – avec cela bavarde, bavarde !\par
Ses yeux, son front, ses lèvres, ses mains, ses pieds, ses nerfs, ses muscles, sa chair, sa peau, tout chez elle remue, jase, interroge, répond ; elle vous harcèle de questions, elle demande des répliques ; ses prunelles se dilatent, s’éteignent ; ses joues se gonflent, se rentrent ; son nez saute ! elle vous touche ici, là, lentement, brusquement, pensivement, follement ; il n’y a pas moyen de finir la conversation. Il faut y être, avoir un signe pour chaque signe, un geste pour chaque geste, des réparties, du trait, regarder tantôt dans le ciel, tantôt à la cave, attraper sa pensée comme on peut, par la tête ou par la queue, en un mot, se donner tout entier, tandis qu’avec les commères qui ont une langue, on ne fait que prêter l’oreille : rien n’est bavard comme un sourd-muet.\par
Pauvre fille ! elle n’a pas trouvé à se marier. C’était certain, et elle vit avec peine du produit de son travail manuel ; non qu’elle manque de rien, à vrai dire, mais elle est coquette, la tante Amélie !\par
Il faut entendre son petit grognement, voir son geste, suivre ses yeux, quand elle essaye une coiffe ou un fichu. Elle a du goût : elle sait planter une rose au coin de son oreille morte, et trouver la couleur du ruban qui ira le mieux à son corsage, près de son cœur qui veut parler…\par
Grand-tante Agnès.\par
On l’appelle la « béate\footnote{Religieuse qui ne vit pas dans un couvent.} ».\par
Il y a tout un monde de vieilles filles qu’on appelle de ce nom-là.\par
« M’man, qu’est-ce que ça veut dire, une béate ? »\par
Ma mère cherche une définition et n’en trouve pas ; elle parle de consécration à la Vierge, de vœux d’innocence.\par
« L’innocence. Ma grand-tante Agnès représente l’innocence ? C’est fait comme cela, l’innocence ! »\par
Elle a bien soixante-dix ans, et elle doit avoir les cheveux blancs ; je n’en sais rien, personne n’en sait rien, car elle a toujours un serre-tête noir qui lui colle comme du taffetas sur le crâne ; elle a, par exemple, la barbe grise, un bouquet de poils ici, une petite mèche qui frisotte par là, et de tous côtés des poireaux comme des groseilles, qui ont l’air de bouillir sur sa figure.\par
Pour mieux dire, sa tête rappelle, par le haut, à cause du serre-tête noir, une pomme de terre brûlée et, par le bas, une pomme de terre germée : j’en ai trouvé une gonflée, violette, l’autre matin, sous le fourneau, qui ressemblait à grand-tante Agnès comme deux gouttes d’eau.\par
« Vœux d’innocence. »\par
Ma mère fait si bien, s’explique si mal, que je commence à croire que c’est malpropre d’être béate, et qu’il leur manque quelque chose, ou qu’elles ont quelque chose de trop.\par
Béate ?\par
Elles sont quatre « béates » qui demeurent ensemble – pas toutes avec des poireaux couleur de feu sur une peau couleur de cendre, comme grand-tante Agnès, qui est coquette, mais toutes avec un brin de moustache ou un bout de favoris, une noix de côtelette, et l’inévitable serre-tête, l’emplâtre noir !\par
On m’y envoie de temps en temps.\par
C’est au fond d’une rue déserte, où l’herbe pousse.\par
Grand-tante Agnès est ma marraine, et elle adore son filleul.\par
Elle veut me faire son héritier, me laisser ce qu’elle a, – pas son serre-tête, j’espère.\par
Il paraît qu’elle garde quelques vieux sous dans un vieux bas, et quand on parle d’une voisine chez qui l’on a trouvé un sac d’écus dans le fond d’un pot à beurre, elle rit dans sa barbe.\par
Je ne m’amuse pas fort chez elle, en attendant qu’on trouve son pot à beurre !\par
Il fait noir dans cette grande pièce, espèce de grenier soutenu par des poutres qui ont l’air en vieux bouchon, tant elles sont piquées et moisies !\par
La fenêtre donne sur une cour, d’où monte une odeur de boue cuite.\par
Il n’y a que les rideaux de lit qui me plaisent, – ils suffisent à me distraire ; on y voit des bonshommes, des chiens, des arbres, un cochon ; ils sont peints en violet sur l’étoffe, c’est le même sujet répété cent fois. Mais je m’amuse à les regarder de tous les côtés, et je vois surtout toutes sortes de choses dans les rideaux de ma grand-tante, quand je mets ma tête entre mes jambes pour les regarder.\par
La chasse – c’est le sujet – me paraît de toutes les couleurs. Je crois bien ! Le sang me descend à la figure ; j’ai le cerveau comme un fond de barrique : c’est l’apoplexie ! Je suis forcé de retirer ma tête par les cheveux pour me relever, et de la replacer droit comme une bouteille en vidange.\par
On fait des prières à tout bout de champ : \emph{Amen ! Amen} ! avant la rave et après l’œuf.\par
Les raves sont le fond du dîner qu’on m’offre quand je vais chez la béate ; on m’en donne une crue et une cuite.\par
Je racle la crue, qui semble mousser sous le couteau, et a sur la langue un goût de noisette et un froid de neige.\par
Je mords avec moins de plaisir dans celle qui est cuite au feu de la chaufferette que la tante tient toujours entre les jambes, et qui est le meuble indispensable des béates. – Huit jambes de béates : quatre chaufferettes – qui servent de boîte à fil en été, et dont elles tournent la braise avec leur clef en hiver.\par
Il y a de temps en temps un œuf.\par
On tire cet œuf d’un sac, comme un numéro de loterie et on le met à la coque, le malheureux ! C’est un véritable crime, un \emph{coquicide}, car il y a toujours un petit poulet dedans.\par
Je mange ce fœtus avec reconnaissance, car on m’a dit que tout le monde n’en mange pas, que j’ai le bénéfice d’une rareté, mais sans entrain, car je n’aime pas l’avorton en mouillettes et le poulet à la petite cuiller.\par
En hiver, les béates travaillent \emph{à la boule} : elles plantent une chandelle entre quatre globes pleins d’eau, ce qui donne une lueur blanche, courte et dure, avec des reflets d’or.\par
En été, elles portent leurs chaises dans la rue sur le pas de la porte, et les \emph{carreaux} vont leur train.\par
Avec ses bandeaux verts, ses rubans roses, ses épingles à tête de perle, avec les fils qui semblent des traînées de bave d’argent sur un bouquet, avec ses airs de corsage riche, ses fuseaux bavards, le \emph{carreau} est un petit monde de vie et de gaieté.\par
Il faut l’entendre babiller sur les genoux des dentellières, dans les rues de béates, les jours chauds, au seuil des maisons muettes. Un tapage de ruche ou de ruisseau, dès qu’elles sont seulement cinq ou six à travailler, – puis quand midi sonne, le silence !…\par
Les doigts s’arrêtent, les lèvres bougent, on dit la courte prière de l’Angelus. Quand celle qui la dit a fini, tous répondent mélancoliquement : \emph{Amen} ! et les \emph{carreaux} se remettent à bavarder…\par
\bigbreak
\noindent Mon oncle Joseph, mon \emph{tonton} comme je dis, est un paysan qui s’est fait ouvrier. Il a vingt-cinq ans, et il est fort comme un bœuf ; il ressemble à un joueur d’orgue ; la peau brune, de grands yeux, une bouche large, de belles dents ; la barbe très noire, un buisson de cheveux, un cou de matelot, des mains énormes toutes couvertes de verrues, – ces fameuses verrues qu’il gratte pendant la prière !\par
Il est \emph{compagnon du devoir}, il a une grande canne avec de longs rubans, et il m’emmène quelquefois chez la Mère des menuisiers. On boit, on chante, on fait des tours de force ; il me prend par la ceinture, me jette en l’air, me rattrape et me jette encore. J’ai plaisir et peur ! puis je grimpe sur les genoux des compagnons ; je touche à leurs mètres et à leurs compas, je goûte au vin qui me fait mal, je me cogne au \emph{chef-d’œuvre}, je renverse des planches, et m’éborgne à leurs grands faux-cols, je m’égratigne à leurs pendants d’oreilles. Ils ont des pendants d’oreilles.\par
« Jacques, est-ce que tu t’amuses mieux avec ces “messieurs de la bachellerie” qu’avec nous ?\par
— Oh ! mais non ! »\par
Il appelle « messieurs de la bachellerie » les instituteurs, professeurs, maîtres de latinage ou de dessin, qui viennent quelquefois à la maison et qui parlent du collège, tout le temps ; ce jour-là, on m’ordonne majestueusement de rester tranquille, on me défend de mettre mes coudes sur la table, je ne dois pas remuer les jambes, et je mange le gras de ceux qui ne l’aiment pas ! Je m’ennuie beaucoup avec ces messieurs de la bachellerie, et je suis si heureux avec les menuisiers !\par
Je couche à côté de tonton Joseph, et il ne s’endort jamais sans m’avoir conté des histoires – il en sait tout plein, – puis il bat la retraite avec ses mains sur son ventre. Le matin, il m’apprend à donner des coups de poing, et il se fait tout petit pour me présenter sa grosse poitrine à frapper ; j’essaie aussi le coup de pied, et je tombe presque toujours.\par
Quand je me fais mal, je ne pleure pas, ma mère viendrait.\par
Il part le matin et revient le soir.\par
Comme j’attends après lui ! Je compte les heures quand il est sur le point de rentrer.\par
Il m’emporte dans ses bras après la soupe, et il m’emmène jusqu’à ce qu’on se couche, dans son petit atelier, qu’il a en bas, où il travaille à son compte, le soir, en chantant des chansons qui m’amusent, et en me jetant tous les copeaux par la figure ; c’est moi qui mouche la chandelle, et il me laisse mettre les doigts dans son vernis.\par
Il vient quelquefois des camarades le voir et causer avec lui, les mains dans les poches, l’épaule contre la porte. Ils me font des amitiés, et mon oncle est tout fier : « Il en sait déjà long, le gaillard – Jacques, dis-nous ta fable ! »\par
Un jour, l’oncle Joseph partit.\par
Ce fut une triste histoire !\par
Madame Garnier, la veuve de l’ivrogne qui s’est noyé dans sa cuve, avait une nièce qu’elle fit venir de Bordeaux, lors de la catastrophe.\par
Une grande brune, avec des yeux énormes, des yeux noirs, tout noirs, et qui brûlent ; elle les fait aller comme je fais aller dans l’étude un miroir cassé, pour jeter des éclairs ; ils roulent dans les coins, remontent au ciel et vous prennent avec eux.\par
Il paraît que j’en tombai amoureux fou. Je dis « il paraît », car je ne me souviens que d’une scène de passion, d’épouvantable jalousie.\par
Et contre qui ?\par
Contre l’oncle Joseph lui-même, qui avait fait la cour à mademoiselle Célina Garnier, s’y était pris, je ne sais comment, mais avait fini par la demander en mariage et l’épouser.\par
L’aimait-elle ?\par
Je ne puis aujourd’hui répondre à cette question ; aujourd’hui que la raison est revenue, que le temps a versé sa neige sur ces émotions profondes. Mais alors, – au moment où mademoiselle Célina se maria, j’étais aveuglé par la passion.\par
Elle allait être la femme d’un autre ! Elle me refusait, moi si pur. Je ne savais pas encore la différence qu’il y avait entre une dame et un monsieur, et je croyais que les enfants naissaient sous les choux.\par
Quand j’étais dans un potager, il m’arrivait de regarder ; je me promenais dans les légumes, avec l’idée que moi aussi je pouvais être père…\par
Mais tout de même, je tressaillais quand ma tante me tapotait les joues et me parlait en bordelais. Quand elle me regardait d’une certaine façon, le cœur me tournait, comme le jour où, sur le Breuil, j’étais monté dans une balançoire de foire.\par
J’étais déjà grand : \emph{dix ans.} C’est ce que je lui disais :\par
« N’épouse pas mon oncle Joseph ! Dans quelque temps, je serai un homme : attends-moi, jure-moi que tu m’attendras ! C’est pour de rire, n’est-ce pas, la noce d’aujourd’hui ? »\par
Ce n’était pas pour de rire, du tout ; ils étaient mariés bel et bien, et ils s’en allèrent tous les deux.\par
Je les vis disparaître.\par
Ma jalousie veillait. J’entendis tourner la clef.\par
Elle me tordit le cœur, cette clef ! J’écoutai, je fis le guet. Rien ! rien ! Je sentis que j’étais perdu. Je rentrai dans la salle du festin, et \emph{je bus pour oublier.}\par
Je n’osai plus regarder l’oncle Joseph en face depuis ce temps-là. Cependant quand il vint nous voir, la veille de son départ pour Bordeaux, il ne fit aucune allusion à notre rivalité et me dit adieu avec la tendresse de l’oncle, et non la rancune du mari !\par
Il y a aussi ma cousine Apollonie ; on l’appelle la Polonie.\par
C’est comme ça qu’ils ont baptisé leur fille, ces paysans !\par
Chère cousine ! grande et lente, avec des yeux bleu de pervenche, de longs cheveux châtains, des épaules de neige ; un cou frais, que coupe de sa noirceur luisante un velours tenant une croix d’or ; le sourire tendre et la voix traînante, devenant rose dès qu’elle rit, rouge dès qu’on la fixe. Je la dévore des yeux quand elle s’habille, – je ne sais pas pourquoi, – je me sens tout chose en la regardant retenir avec ses dents et relever sur son épaule ronde sa chemise qui dégringole, les jours où elle couche dans notre petite chambre, pour être au marché la première, avec ses blocs de beurre fermes et blancs comme les moules de chair qu’elle a sur sa poitrine. On s’arrache le beurre de la Polonie.\par
Elle vient quelquefois m’agacer le cou, me menacer les côtes, de ses doigts longs. Elle rit, me caresse et m’embrasse ; je la serre en me défendant, et je l’ai mordue une fois ; je ne voulais pas la mordre, mais je ne pouvais pas m’empêcher de serrer les dents, comme sa chair avait une odeur de framboise… Elle m’a crié : Petit méchant ! en me donnant une tape sur la joue, un peu fort ; j’ai cru que j’allais m’évanouir et j’ai soupiré en lui répondant ; je me sentais la poitrine serrée et l’œil plus doux.\par
Elle m’a quitté pour se rejeter dans son lit, en me disant qu’elle avait attrapé froid. Elle ressemble par derrière au poulain blanc que monte le petit du préfet.\par
J’ai pensé à elle tout le temps, en faisant mes thèmes.\par
\bigbreak
\noindent Je reste quelquefois longtemps sans la voir, elle garde la maison au village, puis elle arrive tout d’un coup, un matin, comme une bouffée.\par
« C’est moi, dit-elle, je viens te chercher pour t’emmener chez nous ! Si tu veux venir ! »\par
Elle m’embrasse ! Je frotte mon museau contre ses joues roses, et je le plonge dans son cou blanc, je le laisse traîner sur sa gorge veinée de bleu !\par
Toujours cette odeur de framboise.\par
Elle me renvoie, et je cours ramasser mes hardes et changer de chemise.\par
Je mets une cravate verte et je vole à ma mère de la pommade pour sentir bon, moi aussi, et pour qu’elle mette sa tête sur mes cheveux !\par
Mon paquet est fait, je suis graissé et cravaté, mais je me trouve tout laid en me regardant dans le miroir, et je m’ébouriffe de nouveau ! Je tasse ma cravate au fond de ma poche, et, le col ouvert, la casquette tombante, je cours avoir un baiser encore. Ça me chatouillait ; je ne lui disais pas.\par
Le garçon d’écurie a donné une tape sur la croupe du cheval, un cheval jaune, avec des touffes de poils près du sabot ; c’est celui de ma \emph{tatan} Mariou, qu’on enfourche, quand il y a trop de beurre à porter, ou de fromages bleus à vendre. La bête va l’amble ta ta ta, ta ta ta ! toute raide ; on dirait que son cou va se casser, et sa crinière couleur de mousse roule sur ses gros yeux qui ressemblent à des cœurs de moutons.\par
La tante ou la cousine montent dessus comme des hommes ; les mollets de ma tante sont maigres comme des fuseaux noirs, ceux de ma cousine paraissent gras et doux dans les bas de laine blanche.\par
Hue donc ! Ho, ho !\par
C’est Jean qui tire et fait virer le cheval ; il a eu son picotin d’avoine et il hennit en retroussant ses lèvres et montrant ses dents jaunes.\par
Le voilà sellé.\par
« Passez-moi Jacquinou », dit la Polonie, qui est parvenue à abaisser sur ses genoux sa jupe de futaine et s’est installée à pleine chair sur le cuir luisant de la selle. Elle m’aide à m’asseoir sur la croupe.\par
\bigbreak
\noindent J’y suis !\par
Mais on s’aperçoit que j’ai oublié mes habits roulés dans un torchon, sur la table d’auberge pleine de ronds de vin cernés par les mouches.\par
On les apporte.\par
« Jean, attachez-les. Mon petit Jacquinou, passe tes bras autour de ma taille, serre-moi bien. »\par
Le pauvre cheval a le tricotement sec et les os durs ; mais je m’aperçois à ce moment que ce que dit la fable qu’on nous fait réciter est vrai.\par
Dieu fait bien ce qu’il fait !\par
Ma mère en me fouettant m’a durci et tanné la peau.\par
« Serre, je te dis ! Serre-moi plus fort ! »\par
Et je la serre sous son fichu peint semé de petites fleurs comme des hannetons d’or, je sens la tiédeur de sa peau, je presse le doux de sa chair. Il me semble que cette chair se raffermit sous mes doigts qui s’appuient, et tout à l’heure, quand elle m’a regardé en tournant la tête, les lèvres ouvertes et le cou rengorgé, le sang m’est monté au crâne, a grillé mes cheveux.\par
J’ai un peu desserré les bras dans la rue Saint-Jean. C’est par là que passent les bestiaux, et nous allions au pas. J’étais tout fier. Je me figurais qu’on me regardait, et je faisais celui qui sait monter : je me retournais sur la croupe en m’appuyant du plat de la main, je donnais des coups de talons dans les cuisses et je disais hue ! comme un maquignon.\par
Nous avons traversé le faubourg, passé le dernier bourrelier.\par
Nous sommes à Expailly !\par
Plus de maisons ! excepté dans les champs quelques-unes ; des fleurs qui grimpent contre les murs, comme des boutons de rose le long d’une robe blanche ; un coteau de vignes et la rivière au bas, – qui s’étire comme un serpent sous les arbres, bornée d’une bande de sable jaune plus fin que de la crème, et piqué de cailloux qui flambent comme des diamants.\par
Au fond, des montagnes. Elles coupent de leur échine noire, verdie par le poil des sapins, le bleu du ciel où les nuages traînent en flocons de soie ; un oiseau, quelque aigle sans doute, avait donné un grand coup d’aile et il pendait dans l’air comme un boulet au bout du fil.\par
Je me rappellerai toujours ces bois sombres, la rivière frissonnante, l’air tiède et le grand aigle…\par
J’avais oublié que j’étais le cœur battant contre le dos de la Polonie. Elle-même, ma cousine, semblait ne penser à rien, et je ne me souviens avoir entendu que le pas du cheval et le beuglement d’une vache…
\section[{3. Le collège}]{3. \\
Le collège}\renewcommand{\leftmark}{3. \\
Le collège}

\noindent Le collège. – Il donnait, comme tous les collèges, comme toutes les prisons, sur une rue obscure, mais qui n’était pas loin du Martouret, le Martouret, notre grande place, où étaient la mairie, le marché aux fruits ; le marché aux fleurs, le rendez-vous de tous les polissons, la gaieté de la ville. Puis le bout de cette rue était bruyant, il y avait des cabarets, « des bouchons », comme on disait, avec un trognon d’arbre, un paquet de branches, pour servir d’enseigne. Il sortait de ces bouchons un bruit de querelles, un goût de vin qui me montait au cerveau, m’irritait les sens et me faisait plus joyeux et plus fort.\par
Ce goût de vin ! – la bonne odeur des caves ! – j’en ai encore le nez qui bat et la poitrine qui se gonfle.\par
Les buveurs faisaient tapage ; ils avaient l’air sans souci, bons vivants, avec des rubans à leur fouet et des agréments pleins leur blouse – ils criaient, \emph{topaient} en jurant, pour des ventes de cochons ou de vaches.\par
Encore un bouchon qui saute, un rire qui éclate, et les bouteilles trinquent du ventre dans les doigts du cabaretier ! Le soleil jette de l’or dans les verres, il allume un bouton sur cette veste, il cuit un tas de mouches dans ce coin. Le cabaret crie, embaume, empeste, fume et bourdonne.\par
\bigbreak
\noindent À deux minutes de là, le collège moisit, sue l’ennui et pue l’encre ; les gens qui entrent, ceux qui sortent éteignent leur regard, leur voix, leur pas, pour ne pas blesser la discipline, troubler le silence, déranger l’étude.\par
Quelle odeur de vieux !…\par
\bigbreak
\noindent C’est mademoiselle Balandreau qui m’y conduit – ma mère est souffrante. – On me fait mon panier avant de partir, et je vais m’enfermer là-dedans jusqu’à huit heures du soir. À ce moment-là, mademoiselle Balandreau revient et me ramène. J’ai le cœur bien gros quelquefois et je lui conte mes peines en sanglotant.\par
\bigbreak
\noindent Mon père fait la première étude, celle des élèves de mathématiques, de rhétorique et de philosophie. Il n’est pas aimé, on dit qu’il est \emph{chien}.\par
Il a obtenu du proviseur la permission de me garder dans son étude, près de sa chaire, et je suis là, piochant mes devoirs à ses côtés, tandis qu’il prépare son agrégation.\par
Il a eu tort de me prendre avec lui. Les grands ne sont pas trop méchants pour moi ; ils me voient timide, craintif, appliqué ; ils ne me disent rien qui me fasse de la peine, mais j’entends ce qu’ils disent de mon père, comment ils l’appellent ; ils se moquent de son grand nez, de son vieux paletot, ils le rendent ridicule à mes yeux d’enfant, et je souffre sans qu’il le sache.\par
Il me brutalise quelquefois dans ces moments-là. « Qu’est-ce que tu as donc ? – Comme il a l’air nigaud ! »\par
Je viens de l’entendre insulter et j’étais en train de dévorer un gros soupir, une vilaine larme.\par
Il m’envoie souvent, pendant l’étude du soir, demander un livre, porter un mot à un des autres pions qui est au bout de la cour, tout là-bas… il fait noir, le vent souffle ; de temps en temps, il y a des étages à monter, un long corridor, un escalier obscur, c’est tout un voyage ; on se cache dans les coins pour me faire peur. Je joue au brave, mais je ne me sens bien à l’aise que quand je suis rentré dans l’étude où l’on étouffe.\par
J’y reste quelquefois tout seul, quand mademoiselle Balandreau est en retard. Les élèves sont allés souper, conduits par mon père.\par
\bigbreak
\noindent Comme le temps me semble long ! C’est vide, muet ; et s’il vient quelqu’un, c’est le lampiste qui n’aime pas mon père non plus, je ne sais pourquoi : un vieux qui a une loupe, une casquette de peau de bête et une veste grise comme celle des prisonniers ; il sent l’huile, marmotte toujours entre ses dents, me regarde d’un œil dur, m’ôte brutalement ma chaise de dessous moi, sans m’avertir, met le quinquet sur mes cahiers, jette à terre mon petit paletot, me pousse de côté comme un chien, et sort sans dire un mot. Je ne dis rien non plus et ne parle pas davantage quand mon père revient. On m’a appris qu’il ne fallait pas « rapporter ». Je ne le fais point, je ne le ferai jamais dans le cours de mon existence de collégien, ce qui me vaudra bien des tortures de la part de mes maîtres.\par
Puis, je ne veux pas que, parce qu’on m’a fait mal, il puisse arriver du mal, à mon père, et je lui cache qu’on me maltraite, pour qu’il ne se dispute pas à propos de moi. Tout petit, je sens que j’ai un devoir à remplir, ma sensibilité comprend que je suis un fils de galérien, pis que cela ! de garde-chiourme ! et je supporte la brutalité du lampiste.\par
J’écoute, sans paraître les avoir entendues, les moqueries qui atteignent mon père ; c’est dur pour un enfant de dix ans.\par
Il est arrivé que j’ai eu très faim, quelques-uns de ces soirs-là, quand on tardait trop à venir. Le réfectoire lançait des odeurs de grillé, j’entendais le cliquetis des fourchettes à travers la cour.\par
Comme je maudissais mademoiselle Balandreau qui n’arrivait pas !\par
J’ai su depuis qu’on la retenait exprès ; ma mère avait soutenu à mon père que s’il n’était pas une poule mouillée, il pourrait me fournir mon souper avec les restes du sien, ou avec le supplément qu’il demanderait au réfectoire.\par
« Si c’était elle, il y a longtemps que ce serait fait. Il n’avait qu’à mettre cela dans du papier. Elle lui donnerait une petite boîte, s’il voulait. »\par
Mon père avait toujours résisté – le pauvre homme. La peur d’être vu ! le ridicule s’il était surpris – la honte ! Ma mère tâchait de lui forcer la main de temps en temps, en me laissant affamé, dans son étude, à l’heure du souper. Il ne cédait pas, il préférait que je souffrisse un peu et il avait raison.\par
Je me souviens pourtant d’une fois où il s’échappa du réfectoire, pour venir me porter une petite côtelette panée qu’il tira d’un cahier de thèmes où il l’avait cachée : il avait l’air si troublé et repartit si ému ! Je vois encore la place, je me rappelle la couleur du cahier, et j’ai pardonné bien des torts plus tard à mon père, en souvenir de cette côtelette chipée pour son fils, un soir, au lycée du Puy…\par
\bigbreak
\noindent Le proviseur s’appelle Hennequin, – envoyé en disgrâce dans ce trou du Puy.\par
Il a écrit un livre : \emph{Les Vacances d’Oscar.}\par
On les donne en prix, et après ce que j’ai entendu dire, ce que j’ai lu à propos des gens qui étaient auteurs, je suis pris d’une vénération profonde, d’une admiration muette pour l’auteur des \emph{Vacances d’Oscar}, qui daigne être proviseur dans notre petite ville, proviseur de mon père, et qui salue ma mère quand il la rencontre.\par
J’ai dévoré \emph{Les Vacances d’Oscar}.\par
Je vois encore le volume cartonné de vert, d’un vert marbré qui blanchissait sous le pouce et poissait les mains, avec un dos de peau blanche, s’ouvrant mal, imprimé sur papier à chandelle. Eh bien ! il tombe de ces pages, de ce malheureux livre, dans mon souvenir, il tombe une impression de fraîcheur chaque fois que j’y songe !\par
Il y a une histoire de pêche que je n’ai point oubliée.\par
Un grand filet luit au soleil, les gouttes d’eau roulent comme des perles, les poissons frétillent dans les mailles, deux pêcheurs sont dans l’eau jusqu’à la ceinture, c’est le frisson de la rivière.\par
Il avait su, cet Hennequin, ce proviseur dégommé, ce chantre du petit Oscar, traîner ce grand filet le long d’une page et faire passer cette rivière dans un coin de chapitre…\par
\bigbreak
\noindent Le professeur de philosophie – M. Beliben – petit, fluet, une tête comme le poing, trois cheveux, et un filet de vinaigre dans la voix.\par
Il aimait à prouver l’existence de Dieu, mais si quelqu’un glissait un argument, même dans son sens, il indiquait qu’on le dérangeait, il lui fallait toute la table, comme pour une réussite.\par
Il prouvait l’existence de Dieu avec des petits morceaux de bois, des haricots.\par
« Nous plaçons ici un haricot, bon ! – là, une allumette. – Madame Vingtras, une allumette ? – Et maintenant que j’ai rangé, ici les vices de l’homme, là les vertus, j’arrive avec les FACULTÉS DE L’ÂME. »\par
Ceux qui n’étaient pas au courant regardaient du côté de la porte s’il entrait quelqu’un, ou du côté de sa poche, pour voir s’il allait sortir quelque chose. Les facultés de l’âme, c’était de la haute, du chenu ! Ma mère était flattée.\par
« Les voici ! »\par
On se tournait encore, malgré soi, pour saluer ces dames ; mais Beliben vous reprenait par le bouton du paletot et tapait avec impatience sur la table. Il lui fallait de l’attention. Que diable ! voulait-on qu’il prouvât l’existence de Dieu, oui ou non !\par
« Moi, ça m’est égal, et vous ? » disait mon oncle Joseph à son voisin, qui faisait chut, et allongeait le cou pour mieux voir.\par
Mon oncle remettait nonchalamment ses mains dans ses poches et regardait voler les mouches.\par
Mais le professeur de bon Dieu tenait à avoir mon oncle pour lui et le ramenait à son sujet, l’agrippant par son amour-propre et s’accrochant à son métier.\par
« Chadenas, vous qui êtes menuisier, vous savez qu’avec le compas… »\par
Il fallait aller jusqu’au bout : à la fin le petit homme écartait sa chaise, tendait une main, montrait un coin de la table et disait : « DIEU EST LÀ. »\par
On regardait encore, tout le monde se pressait pour voir : tous les haricots étaient dans un coin avec les allumettes, les bouts de bouchons et quelques autres saletés, qui avaient servi à la démonstration de l’\emph{Être suprême}.\par
Il paraît que les vertus, les vices, les facultés de l’âme venaient toutes \emph{fa-ta-le-ment} aboutir à ce tas-là. Tous les haricots y sont. Donc Dieu existe. C. Q. F. D.
\section[{4. La petite ville}]{4. \\
La petite ville}\renewcommand{\leftmark}{4. \\
La petite ville}

\noindent La porte de Pannesac.\par
Elle est en pierre, cette porte, et mon père me dit même que je puis me faire une idée des monuments romains en la regardant.\par
J’ai d’abord une espèce de vénération, puis ça m’ennuie ; je commence à prendre le dégoût des monuments romains.\par
Mais la rue !… Elle sent la graine et le grain.\par
Les culasses de blé s’affaissent et se tassent comme des endormis, le long des murs. Il y a dans l’air la poussière fine de la farine et le tapage des marchés joyeux. C’est ici que les boulangers ou les meuniers, ceux qui font le pain, viennent s’approvisionner.\par
\bigbreak
\noindent J’ai le respect du pain.\par
Un jour je jetais une croûte, mon père est allé la ramasser. Il ne m’a pas parlé durement comme il le fait toujours.\par
« Mon enfant, m’a-t-il dit, il ne faut pas jeter le pain ; c’est dur à gagner. Nous n’en avons pas trop pour nous ; mais si nous en avions trop, il faudrait le donner aux pauvres. Tu en manqueras peut-être un jour, et tu verras ce qu’il vaut. Rappelle-toi ce que je te dis là, mon enfant ! »\par
Je ne l’ai jamais oublié.\par
Cette observation, qui, pour la première fois peut-être dans ma vie de jeunesse, me fut faite sans colère, mais avec dignité, me pénétra jusqu’au fond de l’âme ; et j’ai eu le respect du pain depuis lors.\par
Les moissons m’ont été sacrées, je n’ai jamais écrasé une gerbe, pour aller cueillir un coquelicot ou un bluet ; jamais je n’ai tué sur sa tige la fleur du pain !\par
Ce qu’il me dit des pauvres me saisit aussi et je dois peut-être à ces paroles, prononcées simplement ce jour-là, d’avoir toujours eu le respect, et toujours pris la défense de ceux qui ont faim.\par
« Tu verras ce qu’il vaut. »\par
Je l’ai vu.\par
\bigbreak
\noindent Aux portes des allées sont des mitrons en jupes comme des femmes, jambes nues, petite camisole bleue sur les épaules.\par
Ils ont les joues blanches comme de la farine et la barbe blonde comme de la croûte.\par
Ils traversent la rue pour aller boire une goutte, et blanchissent, en passant, une main d’ami qu’ils rencontrent, ou une épaule de monsieur qu’ils frôlent.\par
Les patrons sont au comptoir, où ils pèsent les miches, et eux aussi ont des habits avec des tons blanchâtres, ou couleur de seigle. Il y a des gâteaux, outre les miches, derrière les vitres : des brioches comme des nez pleins, et des tartelettes comme du papier mou.\par
À côté des haricots ou des graines charnues comme des fruits verts ou luisants comme des cailloux de rivière, les marchands avaient du plomb dans les écuelles de bois.\par
C’était donc là ce qu’on mettait dans un fusil ? ce qui tuait les lièvres et traversait les cœurs d’oiseaux ? On disait même que les charges parfois faisaient balle et pouvaient casser un bras ou une mâchoire d’homme.\par
Je plongeais mes doigts là-dedans, comme tout à l’heure j’avais plongé mon poing dans les sacs de grain, et je sentais le plomb qui roulait et filait entre les jointures comme des gouttes d’eau. Je ramassais comme des reliques ce qui était tombé des écuelles et des sacs.\par
\bigbreak
\noindent Les articles de pêche aussi se vendaient à Pannesac.\par
Tout ce qui avait des tons vifs ou des couleurs fauves, gros comme un pois ou comme une orange, tout ce qui était une tache de couleur vigoureuse ou gaie, tout cela faisait marque dans mon œil d’enfant triste, et je vois encore les bouchons vernis de rouge et les belles lignes luisantes comme du satin jaune.\par
Avoir une ligne, la jeter dans le frais des rivières, ramener un poisson qui luirait au soleil comme une feuille de zinc et deviendrait d’or dans le beurre !\par
Un goujon pris par moi !\par
Il portait toute mon imagination sur ses nageoires !\par
J’allais donc vivre du produit de ma pêche ; comme les insulaires dont j’avais lu l’histoire dans les voyages du capitaine Cook.\par
J’avais lu aussi qu’ils faisaient des vitres à leurs huttes avec de la colle de poisson, et je voyais le jour où je placerais les carreaux à toutes les fenêtres de ma famille ; je me proposais de gratter tout ce qui « mordrait » et de mettre ce résidu d’écaille et de fiente dans ma grande poche.\par
Je le fis plus tard ; mais la fermentation, au fond de la poche, produisit des résultats inattendus, à la suite desquels je fus un objet de dégoût pour mes voisins.\par
Cela ébranla ma confiance dans les récits des voyageurs, et le doute s’éleva dans mon esprit.\par
Il y avait une épicerie dans le fond de Pannesac, qui ajoutait aux odeurs tranquilles du marché une odeur étouffée, chaude, violente, qu’exhalaient les morues salées, les fromages bleus, le suif, la graisse et le poivre.\par
C’était la morue qui dominait, en me rappelant plus que jamais les insulaires, les huttes, la colle et les phoques fumés.\par
Je lançais un dernier regard sur Pannesac, et je manquais régulièrement d’être écrasé, près de la porte de pierre.\par
Je me jetais de côté pour laisser passer les grands chariots qui portaient tous ces fonds de campagne, ces jardins en panier, ces moissons en sac. Ces chariots avaient l’air des voitures de fête dans les mascarades italiennes, avec leur monde d’enfarinés et de pierrots à dos d’Hercule !\par
\bigbreak
\noindent Là-haut, tout là-haut, est l’École normale.\par
\bigbreak
\noindent Le fils du directeur vient me prendre quelquefois pour jouer.\par
Il y a un jardin derrière l’école, avec une balançoire et un trapèze.\par
Je regarde avec admiration ce trapèze et cette balançoire ; seulement il m’est défendu d’y monter.\par
C’est ma mère qui a recommandé aux parents du petit garçon de ne pas me laisser me balancer ou me pendre.\par
Madame Haussard, la directrice, ne se soucie pas d’être toujours à me surveiller ; mais elle m’a fait promettre d’obéir à ma mère. J’obéis.\par
Madame Haussard aime bien son fils, autant que ma mère m’aime ; et elle lui permet pourtant ce qu’on me défend !\par
J’en vois d’autres, pas plus grands que moi, qui se balancent aussi.\par
Ils se casseront donc les reins ?\par
Oui, sans doute ; et je me demande tout bas si ces parents qui laissent ainsi leurs enfants jouer à ces jeux-là ne sont pas tout simplement des gens qui veulent que leurs enfants se tuent. Des assassins sans courage ! des monstres ! qui, n’osant pas noyer leurs petits, les envoient au trapèze – et à la balançoire !\par
Car enfin, pourquoi ma mère m’aurait-elle condamné à ne point faire ce que font les autres ?\par
Pourquoi me priver d’une joie ?\par
Suis-je donc plus cassant que mes camarades ?\par
Ai-je été recollé comme un saladier ?\par
Y a-t-il un mystère dans mon organisation ?\par
J’ai peut-être le derrière plus lourd que la tête !\par
Je ne peux pas le peser à part pour être sûr.\par
En attendant je rôde, le museau en l’air, sous le petit gymnase, que je touche du doigt en sautant comme un chien après un morceau de sucre placé trop haut.\par
Mais que je voudrais donc avoir la tête en bas !\par
Oh ! ma mère ! ma mère ! Pourquoi ne me laissez-vous pas monter sur le trapèze et me mettre la tête en bas !\par
Rien qu’une fois !\par
Vous me fouetterez après, si vous voulez !\par
\bigbreak
\noindent Mais cette mélancolie même vient à mon secours et me fait trouver les soirées plus belles et plus douces sur la grande place qui est devant l’école, et où je vais, quand je suis triste d’avoir vu le trapèze et la balançoire me tendre inutilement les bras dans le jardin !\par
La brise secoue mes cheveux sur mon front et emporte avec elle ma bouderie et mon chagrin.\par
Je reste silencieux, assis quelquefois comme un ancien sur un banc, en remuant la terre devant moi avec un bout de branche, ou relevant tout d’un coup ma tête pour regarder l’incendie qui s’éteint dans le ciel…\par
« Tu ne dis rien, me fait le petit de l’École normale, à quoi penses-tu ?\par
— À quoi je pense ? Je ne sais pas. »\par
Je ne pense pas à ma mère, ni au bon Dieu, ni à ma classe ; et voilà que je me mets à bondir ! Je me fais l’effet d’un animal dans un champ, qui aurait cassé sa corde ; et je grogne, et je caracole comme un cabri, au grand étonnement de mon petit camarade, qui me regarde gambader, et s’attend à me voir brouter. J’en ai presque envie.
\section[{5. La toilette}]{5. \\
La toilette}\renewcommand{\leftmark}{5. \\
La toilette}

\noindent Un jour, un homme qui voyageait m’a pris pour une curiosité du pays, et m’ayant vu de loin, est accouru au galop de son cheval. Son étonnement a été extrême, quand il a reconnu que j’étais vivant. Il a mis pied à terre, et s’adressant à ma mère, lui a demandé respectueusement si elle voulait bien lui indiquer l’adresse du tailleur qui avait fait mon vêtement.\par
« C’est moi », a-t-elle répondu, rougissant d’orgueil.\par
Le cavalier est reparti et on ne l’a plus revu.\par
Ma mère m’a parlé souvent de cette apparition, de cet homme qui se détournait de son chemin pour savoir qui m’habillait.\par
Je suis en noir souvent, « rien n’habille comme le noir », et en habit, en frac, avec un chapeau haut de forme ; j’ai l’air d’un poêle.\par
Cependant, comme j’use beaucoup, on m’a acheté, dans la campagne, une étoffe jaune et velue, dont je suis enveloppé. Je joue l’ambassadeur lapon. Les étrangers me saluent ; les savants me regardent.\par
Mais l’étoffe dans laquelle on a taillé mon pantalon se sèche et se racornit, m’écorche et m’ensanglante.\par
Hélas ! Je vais non plus vivre, mais me traîner.\par
Tous les jeux de l’enfance me sont interdits. Je ne puis jouer aux barres, sauter, courir, me battre. Je rampe seul, calomnié des uns, plaint par les autres, inutile ! Et il m’est donné, au sein même de ma ville natale, à douze ans, de connaître, isolé dans ce pantalon, les douleurs sourdes de l’exil.\par
\bigbreak
\noindent Madame Vingtras y met quelquefois de l’espièglerie.\par
On m’avait invité pendant le carnaval à un bal d’enfants. Ma mère m’a vêtu en charbonnier. Au moment de me conduire, elle a été forcée d’aller ailleurs ; mais elle m’a mené jusqu’à la porte de M. Puissegat, chez qui se donnait le bal.\par
Je ne savais pas bien le chemin et je me suis perdu dans le jardin ; j’ai appelé.\par
Une servante est venue et m’a dit :\par
« C’est vous, le petit Choufloux, qui venez pour aider à la cuisine ? »\par
Je n’ai pas osé dire que non, et on m’a fait laver la vaisselle toute la nuit.\par
Quand le matin ma mère est venue me chercher, j’achevais de rincer les verres ; on lui avait dit qu’on ne m’avait pas aperçu ; on avait fouillé partout.\par
Je suis entré dans la salle pour me jeter dans ses bras : mais, à ma vue, les petites filles ont poussé des cris, des femmes se sont évanouies, l’apparition de ce nain, qui roulait à travers ces robes fraîches, parut singulière à tout le monde.\par
Ma mère ne voulait plus me reconnaître ; je commençais à croire que j’étais orphelin !\par
Je n’avais cependant qu’à l’entraîner et à lui montrer, dans un coin, certaine place couturée et violacée, pour qu’elle criât à l’instant : « C’est mon fils ! » Un reste de pudeur me retenait. Je me contentai de faire des signes, et je parvins à me faire comprendre.\par
On m’emporta comme on tire le rideau sur une curiosité.\par
\bigbreak
\noindent La distribution des prix est dans trois jours.\par
Mon père, qui est dans le secret des dieux, sait que j’aurai des prix, qu’on appellera son fils sur l’estrade, qu’on lui mettra sur la tête une couronne trop grande, qu’il ne pourra ôter qu’en s’écorchant, et qu’il sera embrassé sur les deux joues par quelque autorité.\par
Madame Vingtras est avertie, et elle songe…\par
Comment habillera-t-elle son fruit, son enfant, son Jacques ? Il faut qu’il brille, qu’on le remarque, – on est pauvre, mais on a du goût.\par
« Moi d’abord, je veux que mon enfant soit bien mis. »\par
On cherche dans la grande armoire où est la robe de noce, où sont les fourreaux de parapluie, les restes de jupe, les coupons de soie.\par
Elle s’égratigne enfin à une étoffe criante, qui a des reflets de tigre au soleil ; – une étoffe comme une lime, qui exaspère les doigts quand on la touche, et qui flambe au grand air comme une casserole ! Une belle étoffe, vraiment, et qui vient de la grand-mère, et qu’on a payée à prix d’or. « Oui, mon enfant, à prix d’or, dans l’ancien temps. »\par
« Jacques, je vais te faire une redingote avec ça, m’en priver pour toi !… », et ma mère ravie me regarde du coin de l’œil, hoche la tête, sourit du sourire des sacrifiées heureuses.\par
« J’espère qu’on vous gâte, monsieur », et elle sourit encore, et elle dodeline de la tête, et ses yeux sont noyés de tendresse.\par
« C’est une folie ! tant pis ! on fera une redingote à Jacques avec ça. »\par
On m’a essayé la redingote, hier soir, et mes oreilles saignent, mes ongles sont usés. Cette étoffe crève la vue et chatouille si douloureusement la peau !\par
« Seigneur ! délivrez-moi de ce vêtement ! »\par
Le ciel ne m’entend pas ! La redingote est prête.\par
Non, Jacques, elle n’est pas prête. Ta mère est fière de toi ; ta mère t’aime et veut te le prouver.\par
Te figures-tu qu’elle te laissera entrer dans ta redingote, sans ajouter un grain de beauté une mouche, un pompon, un rien sur le revers, dans le dos, au bout des manches ! Tu ne connais pas ta mère, Jacques !\par
Et ne la vois-tu pas qui joue, à la fois orgueilleuse et modeste, avec des noyaux verts !\par
La mère de Jacques lui fait même kiki dans le cou.\par
Il ne rit pas. – Ces noyaux lui font peur !…\par
Ces noyaux sont des boutons, vert vif, vert gai, en forme d’olives, qu’on va, – voyez si madame Vingtras épargne rien ! – qu’on va coudre tout le long, à la \emph{polonaise} ! À la polonaise, Jacques !\par
Ah ! quand, plus tard, il fut dur pour les Polonais, quoi d’étonnant ! Le nom de cette nation, voyez-vous, resta chez lui cousu à un souvenir terrible… la redingote de la distribution des prix, la redingote à noyaux, aux boutons ovales comme des olives et verts comme des cornichons.\par
Joignez à cela qu’on m’avait affublé d’un chapeau haut de forme que j’avais brossé à rebrousse-poil et qui se dressait comme une menace sur ma tête.\par
Des gens croyaient que c’étaient mes cheveux et se demandaient quelle fureur les avait fait se hérisser ainsi. « Il a vu le diable », murmuraient les béates en se signant…\par
J’avais un pantalon blanc. Ma mère s’était saignée aux quatre veines.\par
Un pantalon blanc à sous-pieds !\par
Des sous-pieds qui avaient l’air d’instruments pour un pied-bot et qui tendaient la culotte à la faire craquer.\par
Il avait plu, et, comme on était venu vite, j’avais des plaques de boue dans les mollets, et mon pantalon blanc, trempé par endroits, collé sur mes cuisses.\par
« MON FILS », dit ma mère d’une voix triomphante en arrivant à la porte d’entrée et en me poussant devant elle.\par
Celui qui recevait les cartes faillit tomber de son haut et me chercha sous mon chapeau, interrogea ma redingote, leva les mains au ciel.\par
\bigbreak
\noindent J’entrai dans la salle.\par
J’avais ôté mon chapeau en le prenant par les poils ; j’étais reconnaissable, c’était bien moi, il n’y avait pas à s’y tromper, et je ne pus jamais dans la suite invoquer un alibi.\par
Mais, en voulant monter par-dessus un banc pour arriver du côté de ma classe, voilà un des sous-pieds qui craque, et la jambe du pantalon qui remonte comme un élastique ! Mon tibia se voit, – j’ai l’air d’être en caleçon cette fois ; – les dames, que mon cynisme outrage, se cachent derrière leur éventail…\par
Du haut de l’estrade, on a remarqué un tumulte dans le fond de la salle.\par
Les autorités se parlent à l’oreille, le général se lève et regarde : on se demande le secret de ce tapage.\par
« Jacques, baisse ta culotte », dit ma mère à ce moment, d’une voix qui me fusille et part comme une décharge dans le silence.\par
Tous les regards s’abaissent sur moi.\par
Il faut cependant que ce scandale cesse. Un officier plus énergique que les autres donne un ordre :\par
« Enlevez l’enfant aux cornichons ! »\par
\bigbreak
\noindent L’ordre s’exécute discrètement ; on me tire de dessous la banquette où je m’étais tapi désespéré, et la femme du censeur, qui se trouve là, m’emmène, avec ma mère, hors de la salle, jusqu’à la lingerie, où on me déshabille.\par
Ma mère me contemple avec plus de pitié que de colère.\par
« Tu n’es pas fait pour porter la toilette, mon pauvre garçon ! »\par
Elle en parle comme d’une infirmité et elle a l’air d’un médecin qui abandonne un malade.\par
Je me laisse faire. On me loge dans la défroque d’un petit, et ce petit est encore trop grand, car je danse dans ses habits. Quand je rentre dans la salle, on commence à croire à une mystification.\par
Tout à l’heure j’avais l’air d’un léopard, j’ai l’air d’un vieillard maintenant. Il y a quelque chose là-dessous.\par
Le bruit se répand, dans certaines parties de la salle, que je suis le fils de l’escamoteur qui vient d’arriver dans la ville et qui veut se faire remarquer par un tour nouveau. Cette version gagne du terrain ; heureusement on me connaît, on connaît ma mère ; il faut bien se rendre à l’évidence, ces bruits tombent d’eux-mêmes, et l’on finit par m’oublier.\par
J’écoute les discours en silence et en me fourrant les doigts dans le nez, avec peine, car mes manches sont trop longues.\par
À cause de l’orage la distribution a lieu dans un dortoir, – un dortoir dont on a enlevé les lits en les entassant avec leurs accessoires dans une salle voisine. On voyait dans cette salle par une porte vitrée, qui aurait dû avoir un rideau, mais n’en avait pas ; on distinguait des vases en piles, des vases qui pendant l’année servaient, mais qu’on retirait de dessous les lits pendant les vacances. On en avait fait une pyramide blanche.\par
C’était le coin le plus gai ; un malin petit rayon de soleil avait choisi le ventre d’un de ces vases pour y faire des siennes, s’y mirer, coqueter, danser, le mutin, et il s’en donnait à cœur joie !\par
Adossée à cette salle était l’estrade, avec le personnel de la baraque, je veux dire du collège : – Monseigneur au centre, le préfet à gauche, le général à droite, galonnés, teintés de violet, panachés de blanc, cuirassés d’or comme les écuyers du cirque Bouthors. Il n’y avait pas de chameau, malheureusement.\par
Je crus voir un éléphant ; c’était un haut fonctionnaire qui avait la tête, la poitrine, le ventre et les pieds couleur d’éléphant, mais qui était douanier de son état ou capitaine de gendarmerie, j’ai oublié. Il était gros comme une barrique et essoufflé comme un phoque : il avait beaucoup du phoque.\par
C’est lui qui me couronna pour le prix d’Histoire sainte. Il me dit : « C’est bien, mon enfant ! » Je croyais qu’il allait dire « Papa » et replonger dans son baquet.
\section[{6. Vacances}]{6. \\
Vacances}\renewcommand{\leftmark}{6. \\
Vacances}

\noindent Je m’amuse un peu pendant les vacances chez Soubeyrou, puis à Farreyrolles.\par
\bigbreak
\noindent M. Soubeyrou est un maraîcher des environs.\par
Trois fois par semaine, mon père donne quelques leçons au fils de ce jardinier, et comme l’enfant est maladif, sort peu, on a demandé que je vinsse lui tenir compagnie de temps en temps.\par
Je prends le plus long pour arriver.\par
\bigbreak
\noindent Je suis donc libre !\par
\bigbreak
\noindent Ce n’est pas pour faire une commission, avec l’ordre de revenir tout de suite et de ne rien casser ; ce n’est pas accompagné, surveillé, pressé, que je descends la rue en me laissant glisser sur la rampe de fer.\par
Non. J’ai mon temps, une après-midi, toute une après midi !\par
« Cela t’amuse d’aller chez M. Soubeyrou ? demande ma mère.\par
— Oui, m’man. »\par
Mais un \emph{oui} lent, un \emph{oui} avec une moue.\par
Tiens ! si je disais trop vite que ça m’amuse, elle serait capable de m’empêcher d’y aller.\par
Si une chose me chagrine bien, me répugne, peut me faire pleurer, ma mère me l’impose sur-le-champ.\par
« Il ne faut pas que les enfants aient de volonté ; ils doivent s’habituer à tout. – Ah ! les enfants gâtés ! Les parents sont bien coupables qui les laissent faire tous leurs caprices… »\par
Je dis : « Oui, m’man », de façon qu’elle croie que c’est \emph{non}, et je me laisse habiller et sermonner en rechignant.\par
\bigbreak
\noindent Je descends dans la ville.\par
Je ne m’arrête pas au Martouret, parce que ma mère peut me voir des fenêtres de notre appartement, perché là-haut au dernier étage d’une maison, qui est la plus haute de la ville.\par
Je fais le sage et le pressé en passant sur le marché ; mais, dans la rue Porte-Aiguière, je m’abrite derrière le premier gros homme qui passe, et j’entre dans la cour de l’auberge du \emph{Cheval-Blanc}.\par
De cette cour, je vois la rue en biais, et je puis dévorer des yeux la devanture du bourrelier, où il y a des tas de houppes et de grelots, des pompons bleus, de grands fouets couleur de cigare et des harnais qui brillent comme de l’or.\par
Je reste caché le temps qu’il faut pour voir si ma mère est à la fenêtre et me surveille encore ; puis, quand je me sens libre, je sors de la cour du \emph{Cheval-Blanc} et je me mets à regarder les boutiques à loisir.\par
\bigbreak
\noindent Il y a un chaudronnier en train de taper sur du beau cuivre rouge, que le marteau marque comme une croupe de jument pommelée et qui fait « dzine, dzine », sur le carreau ; chaque coup me fait froncer la peau et cligner des yeux.\par
Puis c’est la boutique d’Arnaud, le cordonnier, avec sa botte verte pour enseigne, une grande botte cambrée, qui a un éperon et un gland d’or ; à la vitrine s’étalent des bottines de satin bleu, de soie rose, couleur de prune, avec des nœuds comme des bouquets, et qui ont l’air vivantes.\par
À côté, les pantoufles qui ressemblent à des souliers de Noël.\par
Mais le fils du jardinier attend.\par
Je m’arrache à ces parfums de cirage et à ces flamboiements de vernis.\par
\bigbreak
\noindent Je prends le Breuil…\par
Il y a un décrotteur qui est populaire et qu’on appelle Moustache.\par
Mon rêve est de me faire décrotter un jour par Moustache, de venir là comme un homme, de lui donner mon pied, – sans trembler, si je puis, – et de paraître habitué à ce luxe, de tirer négligemment mon argent de ma poche en disant, comme font les messieurs qui lui jettent leurs deux sous :\par

\begin{quoteblock}
\noindent Pour la goutte, Moustache !\end{quoteblock}

\noindent Je n’y arriverai jamais ; je m’exerce pourtant !\par

\begin{quoteblock}
\noindent Pour la goutte, Moustache !\end{quoteblock}

\noindent J’ai essayé toutes les inflexions de voix ; je me suis écouté, j’ai prêté l’oreille, travaillé devant la glace, fait le geste :\par
\emph{Pour la goutte}…\par
Non, je ne puis !\par
Mais, chaque fois que je passe devant Moustache, je m’arrête à le regarder ; je m’habitue au feu, je tourne et retourne autour de sa boîte à décrotter ; il m’a même crié une fois :\par

\begin{quoteblock}
\noindent Cirer vos bottes, m’ssieu ?\end{quoteblock}

\noindent J’ai failli m’évanouir.\par
Je n’avais pas deux sous, – je n’ai pu les réunir que plus tard dans une autre ville, – et je dus secouer la tête, répondre par un signe, avec un sourire pâle comme celui d’une femme qui voudrait dire : « Il m’est défendu d’aimer ! »\par
\bigbreak
\noindent Au fond du Breuil est la tannerie avec ses pains de tourbe, ses peaux qui sèchent, son odeur aigre.\par
Je l’adore, cette odeur montante, moutardeuse, verte – si l’on peut dire verte, – comme les cuirs qui faisandent dans l’humidité ou qui font sécher leur sueur au soleil.\par
Du plus loin que j’arrivais dans la ville du Puy, quand j’y revins plus tard, je devinais et je sentais la tannerie du Breuil.\par
— Chaque fois qu’une de ces fabriques s’est trouvée sur mon chemin, à deux lieues à la ronde, je l’ai flairée, et j’ai tourné de ce côté mon nez reconnaissant…\par
\bigbreak
\noindent Je ne me souviens plus du chemin, je ne sais par où je passais, comment finissait la ville.\par
Je me rappelle seulement que je me trouvais le long d’un fossé qui sentait mauvais, et que je marchais à travers un tas d’herbes et de plantes qui ne sentaient pas bon.\par
J’arrivais dans le pays des jardiniers. Que c’est vilain, le pays des maraîchers !\par
Autant j’aimais les prairies vertes, l’eau vive, la verdure des haies ; autant j’avais le dégoût de cette campagne à arbres courts, à plantes pâles, qui poussent, comme de la barbe de vieux, dans un terrain de sable ou de boue, sur le bord des villes.\par
Quelques feuilles jaunâtres, desséchées, galeuses, pendaient avec des teintes d’oreilles de poitrinaires.\par
On avait déshonoré toutes les places, et l’on dérangeait à chaque instant un tourbillon d’insectes qui se régalaient d’un chien crevé.\par
\bigbreak
\noindent Pas d’ombre !\par
Des melons qui ont l’air de boulets chauffés à blanc ; des choux rouges, violets, – on dirait des apoplexies, une odeur de poireau et d’oignons !\par
\bigbreak
\noindent J’arrive chez M. Soubeyrou.\par
Je reste, avec le petit malade, dans la serre.\par
Il est tout pâle, avec un grand sourire et de longues dents, le blanc des yeux taché de jaune ; il me montre un tas de livres qu’on lui a achetés pour qu’il ne s’ennuie pas trop.\par
Un \emph{Ésope} avec des gravures coloriées.\par
Je me rappelle encore une de ces gravures qui représentait Borée, le Soleil et un voyageur.\par
Le voyageur avait de la sueur chocolat qui lui coulait sur le front et un énorme manteau lie-de-vin.\par
\bigbreak
\noindent « Veux-tu t’amuser, m’aider à arroser les choux ? » me dit le père Soubeyrou, qui tient un arrosoir de chaque main et qui marche le pantalon retroussé, les jambes et les pieds nus, depuis le matin.\par
Son mollet ressemble, velu et cuit par la chaleur, à une patte de cochon grillé ; il a sa chemise trempée et des gouttes d’eau roulent sur le poil de son poitrail.\par
Non, je ne veux pas m’amuser, aider à arroser les choux !\par
Si ça l’amuse lui, tant mieux !\par
Je ne veux pas priver M. Soubeyrou d’un plaisir, et je lui réponds par un mensonge.\par
« Je suis tombé hier, et je me suis fait mal aux reins. »\par
J’aime les choux, mais cuits.\par
Je ne fuis pas le baquet maternel, la vaisselle de mes pères, pour venir tirer de l’eau chez des étrangers.\par
Je tire assez d’eau comme cela dans la semaine, et je sens assez l’oignon.\par
Non, M. Soubeyrou, je ne vous suivrai pas à ce puits là-bas : je ne tournerai pas la manivelle, je ne ferai pas venir le seau, je ne me livrerai pas au travail honnête des jardins.\par
Je suis corrompu, malsain, que voulez-vous !\par
Mais je ne veux pas tirer d’eau !\par

\labelblock{DEVANT LES MESSAGERIES}

\noindent En revenant, je fais le grand tour et je passe devant le \emph{Café des Messageries}.\par
L’enseigne est en lettres qui forment chacune une figure, une bonne femme, un paysan, un soldat, un prêtre, un singe.\par
C’est peint avec une couleur jus de tabac, sur un fond gris, et c’est une histoire qui se suit depuis le \emph{C} de Café jusqu’à l’\emph{S} de Messageries.\par
Je n’ai jamais eu le temps de comprendre.\par
Il fallait rentrer.\par
Puis, tandis que je regardais l’enseigne, que ma curiosité saisissait le cotillon de la bonne femme, le grand faux-col du paysan, la giberne du soldat, le rabat du curé, la queue du singe, autour de moi on attelait les chevaux, on lavait les voitures ; les palefreniers, le postillon et le conducteur faisaient leur métier, donnaient de la brosse, du fouet ou de la trompe.\par
Les voyageurs venaient prendre leurs places, retenir un coin.\par
J’étais là quelquefois à l’arrivée : la diligence traversait le Breuil avec un bruit d’enfer, en soulevant des flots de poussière ou en envoyant des étoiles de boue.\par
Elle était assaillie par un troupeau de portefaix qui se disputaient les bagages, et vomissait de ses flancs jaunes des gens engourdis qui s’étiraient les jambes sur le pavé.\par
Ils tombaient dans les bras d’un parent, d’un ami, on se serrait la main, on s’embrassait ; c’étaient des adieux, des au revoir, à n’en plus finir.\par
On avait fait connaissance en route ; les messieurs saluaient avec regret des dames, qui répondaient avec réserve :\par
« Où aurai-je le plaisir de vous retrouver ?\par
— Nous nous rencontrerons peut-être. Ah ! voici maman.\par
— Voici mon mari.\par
— Je vois mon frère qui arrive avec sa femme. »\par
Il y avait des Anglais qui ne disaient rien et des commis-voyageurs qui parlaient beaucoup. Tout le monde remuait, courait, s’échappait comme les insectes quand je soulevais une pierre au bord d’un champ.\par
\bigbreak
\noindent J’en ai vu pourtant qui restaient là, à la même place, fouillant le boulevard et le Breuil du regard, attendant quelqu’un qui ne venait pas.\par
Il y en avait qui juraient, d’autres qui pleuraient.\par
\bigbreak
\noindent Je me rappelle une jeune femme qui avait une tête fine, longue et pâle.\par
Elle attendit longtemps…\par
Quand je partis, elle attendait encore. Ce n’était pas son mari, car sur la petite malle qu’elle avait à ses pieds, il y avait écrit : « Mademoiselle. »\par
Je la rencontrai quelques jours plus tard devant la poste ; les fleurs de son chapeau étaient fanées, sa robe de mérinos noir avait des reflets roux, ses gants étaient blanchis au bout des doigts. Elle demandait s’il n’était pas venu de lettre à telle adresse : poste restante.\par
« Je vous ai dit que non.\par
— Il n’y a plus de courrier aujourd’hui ?\par
— Non. »\par
Elle salua, quoiqu’on fût grossier, poussa un soupir et s’éloigna pour aller s’asseoir sur un banc du \emph{Fer-à-cheval}, où elle resta jusqu’à ce que des officiers qui passaient l’obligèrent, par leurs regards et leurs sourires, à se lever et à partir.\par
Quelques jours après, on dit chez nous qu’il y avait sur le bord de l’eau le cadavre d’une femme qui s’était noyée. J’allai voir. Je reconnus la jeune fille à la tête pâle…\par
\bigbreak
\noindent Je vais chez mes tantes à Farreyrolles.\par
J’arrive souvent au moment où l’on se met à table.\par
Une grosse table, avec deux tiroirs de chaque bout et deux grands bancs de chaque côté.\par
Dans ces tiroirs il traîne des couteaux, de vieux oignons, du pain. Il y a des taches bleues au bord des croûtes, comme du vert-de-gris sur de vieux sous.\par
Sur les deux bancs s’abattent la famille et les domestiques.\par
On mange entre deux prières.\par
C’est l’oncle Jean qui dit le bénédicité.\par
Tout le monde se tient debout, tête nue, et se rassoit en disant : « \emph{Amen ! »}\par
\emph{Amen} ! est le mot que j’ai entendu le plus souvent quand j’étais petit.\par
\emph{Amen} ! et le bruit des cuillers de bois commence ; un bruit mou, tout bête.\par
Viennent les grandes taillades de pain, comme des coups de faucille. Les couteaux ont des manches de corne, avec de petits clous à cercle jaune, on dirait les yeux d’or des grenouilles.\par
Ils mangent en bavant, ouvrent la bouche en long ; ils se mouchent avec leurs doigts, et s’essuient le nez sur leurs manches.\par
Ils se donnent des coups de coude dans les côtes, en manière de chatouillade.\par
Ils rient comme de gros bébés ; quand ils éclatent, ils renâclent comme des ânes ou beuglent comme des bœufs.\par
\bigbreak
\noindent C’est fini, – ils remettent le couteau à œil de grenouille dans la grande poche qui va jusqu’aux genoux, se passent le dos de la main sur la bouche, se balayent les lèvres, et retirent leurs grosses jambes de dessous la table.\par
Ils vont flâner dans la cour, s’il fait soleil, bavarder sous le porche de l’écurie, s’il pleut ; soulevant à peine leurs sabots qui ont l’air de souches, où se sont enfoncés leurs pieds.\par
Je les aime tant avec leur grand chapeau à larges ailes et leur long tablier de cuir ! Ils ont de la terre aux mains, dans la barbe, et jusque dans le poil de leur poitrail ; ils ont la peau comme de l’écorce, et des veines comme des racines d’arbres.\par
Quelquefois, quand leur tablier de cuir est à bas, le vent entrouvre leur chemise toute grande, et en dessous du triangle de hâle qui fait pointe au creux de l’estomac, on voit de la chair blanche, tendre comme un dos de brebis tondue ou de cochon jeune.\par
Je les approche et je les touche comme on tâte une bête ; ils me regardent comme un animal de luxe, – moi de la ville ! – quelques-uns me comparent à un écureuil, mais presque tous à un singe.\par
Je n’en suis pas plus fier, et je les accompagne dans les champs, en leur empruntant l’aiguillon pour piquer les bœufs.\par
J’entre jusqu’au genou dans les sillons, à la saison du labourage ; je me roule dans l’herbe au moment où l’on fait les foins, je piaule comme les cailles qui s’envolent, je fais des culbutes comme les petits qui tombent des nids quand la charrue passe.\par
Oh ! quels bons moments j’ai eus dans une prairie, sur le bord d’un ruisseau bordé de fleurs jaunes dont la queue tremblait dans l’eau, avec des cailloux blancs dans le fond, et qui emportait les bouquets de feuilles et les branches de sureau doré que je jetais dans le courant !…\par
Ma mère n’aime pas que je reste ainsi, muet, la bouche béante, à regarder couler l’eau.\par
Elle a raison, je perds mon temps.\par
« Au lieu d’apporter ta grammaire latine pour apprendre tes leçons ! »\par
Puis, faisant l’émue, affichant la sollicitude :\par
« Si c’est permis, tout taché de vert, des talons pleins de boue… On t’en achètera des souliers neufs pour les arranger comme cela ! Allons, repars à la maison, et tu ne sortiras pas ce soir ! »\par
Je sais bien que les souliers s’abîment dans les champs et qu’il faut mettre des sabots, mais ma mère ne veut pas ! ma mère me fait donner de l’éducation, elle ne veut pas que je sois un campagnard comme elle !\par
Ma mère veut que son Jacques soit un \emph{Monsieur}.\par
Lui a-t-elle fait des redingotes avec olives, acheté un tuyau de poêle, mis des sous-pieds, pour qu’il retombe dans le fumier, retourne à l’écurie mettre des sabots !\par
Ah oui ! je préférerais des sabots ! j’aime encore mieux l’odeur de Florimond le laboureur que celle de M. Sother, le professeur de huitième ; j’aime mieux faire des paquets de foin que lire ma grammaire, et rôder dans l’étable que traîner dans l’étude.\par
Je ne me plais qu’à nouer des gerbes, à soulever des pierres, à lier des fagots, à porter du bois !\par
Je suis peut-être né pour être domestique !\par
C’est affreux ! oui, je suis né pour être domestique ! je le vois ! je le sens ! ! !\par
Mon Dieu ! Faites que ma mère n’en sache rien !\par
J’accepterais d’être Pierrouni le petit vacher, et d’aller, une branche à la main, une pomme verte aux dents, conduire les bêtes dans le pâturage, près des mûres, pas loin du verger.\par
Il y a des églantiers rouges dans les buissons, et là-haut un point barbu, qui est un nid ; il y a des bêtes du bon Dieu, comme de petits haricots qui volent, et dans les fleurs, des mouches vertes qui ont l’air saoules.\par
On laisse Pierrouni se dépoitrailler, quand il a chaud, et se dépeigner quand il en a envie.\par
On n’est pas toujours à lui dire :\par
« Laisse tes mains tranquilles, qu’est-ce que tu as donc fait à ta cravate ? – Tiens-toi droit. – Est-ce que tu es bossu ? – Il est bossu ! – Boutonne ton gilet. – Retrousse ton pantalon, – Qu’est-ce que tu as fait de l’olive ? L’olive là, à gauche, la plus verte ! – Ah ! cet enfant me fera mourir de chagrin ! »\par
\bigbreak
\noindent Mais les grands domestiques aussi sont plus heureux que mon père !\par
Ils n’ont pas besoin de porter des gilets boutonnés jusqu’en haut pour couvrir une chemise de trois jours ! Ils n’ont pas peur de mon oncle Jean comme mon père a peur du proviseur ; ils ne se cachent pas pour rire et boire un verre de vin, quand ils ont des sous ; ils chantent de bon cœur, à pleine voix, dans les champs, quand ils travaillent ; le dimanche, ils font tapage à l’auberge.\par
Ils ont, au derrière de leur culotte, une pièce qui a l’air d’un emplâtre : verte, jaune ; mais c’est la couleur de la terre, la couleur des feuilles, des branches et des choux.\par
Mon père, qui n’est pas domestique, ménage, avec des frissonnements qui font mal, un pantalon de casimir noir, qui a avalé déjà dix écheveaux de fil, tué vingt aiguilles, mais qui reste grêlé, fragile et mou !\par
À peine il peut se baisser, à peine pourra-t-il saluer demain…\par
S’il ne salue pas, celui-ci…, celui-là… (il y a à donner des coups de chapeau à tout le monde, au proviseur, au censeur, etc.), s’il ne salue pas en faisant des grâces, dont le derrière du pantalon ne veut pas, mais alors on l’appelle chez le proviseur !\par
Et il faudra s’expliquer ! – pas comme un domestique – non ! – comme un professeur. Il faudra qu’il demande pardon.\par
On en parle, on en rit, les élèves se moquent, les collègues aussi. On lui paye ses gages (ma mère nomme ça « les appointements ») et on l’envoie en disgrâce quelque part faire mieux raccommoder ses culottes, avec sa femme qui a toujours l’horreur des paysans ; avec son fils… qui les aime encore…\par
\bigbreak
\noindent Je me suis battu une fois avec le petit Viltare, le fils du professeur de septième.\par
Ç’a été toute une affaire !…\par
On a fait comparaître mon père, ma mère ; la femme du proviseur s’en est mêlée ; il a fallu apaiser madame Viltare qui criait :\par
« Si maintenant les fils de pion assassinent les fils de professeur ! »\par
Le petit Viltare m’avait jeté de l’encre sur mon pantalon et mis du bitume dans le cou : je ne l’ai pas assassiné, mais je lui ai donné un coup de poing et un croc-en-jambe…, il est tombé et s’est fait une bosse.\par
On a amené cette bosse chez le proviseur (qui s’en moque comme de \emph{Colin Tampon}, qui se fiche de monsieur Viltare comme de monsieur Vingtras), mais qui doit « surveiller la discipline et faire respecter la hiérarchie » ; je les entends toujours dire ça. Il m’a fait venir, et j’ai dû demander pardon à M. Viltare, à Mme Viltare, puis embrasser le petit Viltare, et enfin rentrer à la maison pour me faire fouetter.\par
Ma mère m’avait dit d’être là au quart avant cinq heures.\par
\bigbreak
\noindent Ce n’est pas comme ça à Farreyrolles.\par
Je me suis battu avec le petit porcher, l’autre jour, nous nous sommes roulés dans les champs, arraché les cheveux, cognés, et recognés, il m’a poché un œil, je lui ai engourdi une oreille, nous nous sommes relevés, pour nous retomber encore dessus !\par
Et après ?\par
Après ? – nous avons rentré nos tignasses, lui, sous son chapeau, moi sous ma casquette, et on nous a fait nous taper dans la main. – On en a ri tout le soir devant le chaudron entre le Bénédicité et les Grâces, et au lieu de me cacher de mon oncle, je lui ai montré que j’avais du sang à mon mouchoir.\par
\bigbreak
\noindent C’est le jour du \emph{Reinage}.\par
On appelle ainsi la fête du village ; on choisit un roi, une reine.\par
Ils arrivent couverts de rubans. Des rubans au chapeau du roi, des rubans au chapeau de la reine.\par
Ils sont à cheval tous deux, et suivis des beaux gars du pays, des fils de fermiers, qui ont rempli leurs bourses ce jour-là, pour faire des cadeaux aux filles.\par
On tire des coups de fusil, on crie hourrah ! on caracole devant la mairie, qui a l’air d’avoir un drapeau vert : c’est une branche d’un grand arbre.\par
Les gendarmes sont en grand uniforme, le fusil en bandoulière, et mon oncle dit qu’ils ont leurs gibernes pleines ; ils sont pâles, et pas un ne sait si, le soir, il n’aura pas la tête fendue ou les côtes brisées.\par
Il y en a un qui est la bête noire du pays et qui sûrement ne reviendrait pas vivant s’il passait seul dans un chemin où serait le fils du braconnier Souliot ou celui de la mère Maichet, qu’on a condamnée à la prison parce qu’elle a mordu et déchiré ceux qui venaient l’arrêter pour avoir ramassé du bois mort.\par
En revenant de l’église, on se met à table.\par
Le plus pauvre a son litre de vin et sa terrine de riz sucré, même Jean le Maigre qui demeure dans cette vilaine hutte là bas.\par
On a du lard et du pain blanc, – du pain blanc !…\par
On remplit jusqu’au bord les verres ; quand les verres manquent, on prend des écuelles et on boit du vivarais comme du lait, – un vivarais qu’on va traire tout mousseux à une barrique qui est près des vaches…\par
Les veines se gonflent, les boutons sautent !\par
On est tous mêlés ; maîtres et valets, la fermière et les domestiques, le premier garçon de ferme et le petit gardeur de porcs, l’oncle Jean, Florimond le laboureur, Pierrouni le vacher, Jeanneton la trayeuse, et toutes les cousines qui ont mis leur plus large coiffe et d’énormes ceintures vertes.\par
Après le repas, la danse sur la pelouse ou dans la grange.\par
Gare aux filles !\par
Les garçons les poursuivent et les bousculent sur le foin, ou viennent s’asseoir de force près d’elles sur le chêne mort qui est devant la ferme et qui sert de banc.\par
Elles relèvent toujours leur coude assez à temps pour qu’on les embrasse à pleines joues.\par
Je danse la bourrée aussi, et j’embrasse tant que je peux.\par
\bigbreak
\noindent Un bruit de chevaux ! – Les gendarmes passent au galop…\par
C’est à la maison Destougnal dans le fond du village ; ceux de Sansac sont venus, et il y a eu bataille.\par
On se tue dans le cabaret.\par
— \emph{Anyn ! les gars} ! – ceux de Farreyrolles en avant !\par
On franchit les fossés, en se baissant dans la course pour ramasser des pierres ; en cassant, dans les buissons qu’on saute, une branche à nœuds ; j’en vois même un qui a un vieux fusil ! ils ne crient pas, ils vont essoufflés et pâles…\par
Voilà le cabaret !\par
On entend des bouteilles qui se brisent, des cris de douleur : « À moi, à moi ! » comme un sanglot.\par
C’est Bugnon\emph{ le Velu} qui crie !\par
Ils se sont jetés sur ce cabaret comme des mouches sur un tas d’ordures ; comme j’ai vu un taureau se jeter sur un tablier rouge, un soir, dans le pré.\par
Du rouge ! il y en a plein les vitres du cabaret et plein les bouches des paysans…\par
Est-ce du vin du Vivarais ou du sang de Farreyrolles qui coule ?\par
J’ai la tête en feu, car j’ai du sang de Farreyrolles aussi dans mes veines d’enfant !\par
Je veux y être comme les autres, et taper dans le tas !\par
Je me sens pris par un pan de ma veste, arrêté brusquement, et je tombe, en me retournant, dans les bras de ma tante, qui n’a pas empêché ses fils d’aller au cabaret de Destougnal, mais qui ne veut pas que son petit neveu soit dans cette tuerie.\par
Ça ne fait rien. Si je peux de derrière un arbre lancer une pierre aux gendarmes, je n’y manquerai pas. Comme j’aimerais cette vie de labour, de reinage et de bataille !
\section[{7. Les joies du foyer}]{7. \\
Les joies du foyer}\renewcommand{\leftmark}{7. \\
Les joies du foyer}

\noindent \emph{1\textsuperscript{er} janvier.}\par
Les collègues de mon père, quelques parents d’élèves, viennent faire visite, on m’apporte des bouts d’étrennes.\par
« Remercie donc, Jacques ! Tu es là comme un imbécile. »\par
\bigbreak
\noindent Quand la visite est finie, j’ai plaisir à prendre le jouet ou la friandise, la boîte à diable ou le sac à pralines ; – je bats du tambour et je sonne de la trompette, je joue d’une musique qu’on se met entre les dents et qui les fait grincer, c’est à en devenir fou !\par
Mais ma mère ne veut pas que je devienne fou ! elle me prend la trompette et le tambour. Je me rejette sur les bonbons et je les lèche. Mais ma mère ne veut pas que j’aie des manières de courtisan : « On commence par lécher le ventre des bonbons, on finit par lécher… » Elle s’arrête, et se tourne vers mon père pour voir s’il pense comme elle, et s’il sait de quoi elle veut parler ; – en effet, il se penche et montre qu’il comprend.\par
Je n’ai plus rien à faire siffler, tambouriner, grincer, et l’on m’a permis seulement de traîner un petit bout de langue sur les bonbons fins : et l’on m’a dit de la faire pointue encore ! Il y avait Eugénie et Louise Rayau qui étaient là, et qui riaient en rougissant un peu. Pourquoi donc ?\par
Plus de gros vernis bleu qui colle aux doigts et les embaume, plus le goût du bois blanc des trompettes !…\par
On m’arrache tout et l’on enferme les étrennes sous clef.\par
« Rien qu’aujourd’hui, maman, laisse-moi jouer avec, j’irai dans la cour, tu ne m’entendras pas ! rien qu’aujourd’hui, jusqu’à ce soir, et demain je serai bien sage !\par
— J’espère que tu seras bien sage demain ; si tu n’es pas sage, je te fouetterai. Donnez donc de jolies choses à ce saligaud, pour qu’il les abîme. »\par
Ces points vifs, ces taches de couleur joyeuse, ces bruits de jouet, ces trompettes d’un sou, ces bonbons à corset de dentelle, ces pralines comme des nez d’ivrognes, ces tons crus et ces goûts fins, ce soldat qui coule, ce sucre qui fond, ces gloutonneries de l’œil, ces gourmandises de la langue, ces odeurs de colle, ces parfums de vanille, ce libertinage du nez et cette audace du tympan, ce brin de folie, ce petit coup de fièvre, ah ! comme c’est bon, une fois l’an ! – Quel malheur que ma mère ne soit pas sourde !\par
Ce qui me fait mal, c’est que tous les autres sont si contents ! Par le coin de la fenêtre, je vois dans la maison voisine, chez les gens d’en face, des tambours crevés, des chevaux qui n’ont qu’une jambe, des polichinelles cassés ! Puis ils sucent, tous, leurs doigts ; on les a laissés casser leurs jouets et ils ont dévoré leurs bonbons.\par
Et quel boucan ils font !\par
\bigbreak
\noindent Je me suis mis à pleurer.\par
C’est qu’il m’est égal de regarder des jouets, si je n’ai pas le droit de les prendre et d’en faire ce que je veux ; de les découdre et de les casser, de souffler dedans et de marcher dessus, si ça m’amuse…\par
Je ne les aime que s’ils sont à moi, et je ne les aime pas s’ils sont à ma mère. C’est parce qu’ils font du bruit et qu’ils agacent les oreilles qu’ils me plaisent ; si on les pose sur la table comme des têtes de mort, je n’en veux pas. Les bonbons, je m’en moque, si on m’en donne un par an comme une exemption, quand j’aurai été sage. Je les aime quand j’en ai trop.\par
« Tu as un coup de marteau, mon garçon ! » m’a dit ma mère un jour que je lui contais cela, et elle m’a cependant donné une praline.\par
« Tiens, mange-la avec du pain. »\par
On nous parle en classe des philosophes qui font tenir une leçon dans un mot. Ma mère a de ces bonheurs-là, et elle sait me rappeler par une fantaisie, un rien, ce qui doit être la loi d’une vie bien conduite et d’un esprit bien réglé.\par
« Mange-la avec du pain ! »\par
Cela veut dire : Jeune fou, tu allais la croquer bêtement, cette praline. Oublies-tu donc que tu es pauvre ! À quoi cela t’aurait-il profité ! Dis-moi ! Au lieu de cela, tu en fais un plat utile, une portion, tu la manges avec du pain.\par
J’aime mieux le pain tout seul.\par

\labelblock{LA SAINT-ANTOINE}

\noindent C’est samedi prochain la fête de mon père.\par
Ma mère me l’a dit soixante fois depuis quinze jours.\par
« C’est la fête–\emph{de–ton–père}. »\par
Elle me le répète d’un ton un peu irrité ; je n’ai pas l’air assez remué, paraît-il.\par
« Ton père s’appelle Antoine. »\par
Je le sais, et je n’éprouve pas de frisson ; il n’y a pas là le mystérieux et l’empoignant d’une révélation. Il s’appelle Antoine, voilà tout.\par
Je suis sans doute un mauvais fils.\par
Si j’avais du cœur, si j’aimais bien mon père, ce qu’elle dit me ferait plus d’effet. Je me tords la cervelle, je me frappe la poitrine, je me tâte et me gratte ; mais je ne me sens pas changé du tout, je me reconnais dans la glace, je suis aussi laid et aussi malpropre. C’est pourtant sa fête, samedi.\par
\bigbreak
\noindent « As-tu appris ton compliment ? »\par
Je me trouve un peu grand pour apprendre un compliment, – je ne sais pas comment j’oserai entrer dans la chambre, ce qu’il faudra dire, s’il faudra rire, s’il faudra pleurer, si je devrai me jeter sur la barbe de mon père et la frotter en y enfonçant mon nez – bien rapproprié, par exemple ! – s’il sera filial que j’appuie, que j’y reste un moment, ou s’il vaudra mieux le débarrasser tout de suite, et m’en aller à reculons, avec des signes d’émotion, en murmurant : « Quel beau jour ! » À ce moment-là, je commencerai :\par
« \emph{Oui, cher papa}… »\par
J’en tremble d’avance. J’ai peur d’avoir l’air si bête… – Non, j’ai peur qu’on devine que j’aimerais que ce ne fût point sa fête…\par
La fête de mon père !\par
\bigbreak
\noindent Mes inquiétudes redoublent, quand ma mère m’annonce que je devrai offrir un pot de fleurs.\par
Comme ce sera difficile !\par
Mais ma mère sait comment on exprime l’émotion et la joie d’avoir à féliciter son père de ce qu’il s’appelle Antoine.\par
Nous faisons des répétitions.\par
D’abord, je gâche trois feuilles de papier à compliments : j’ai beau tirer la langue, et la remuer, et la crisper en faisant mes majuscules, j’éborgne les \emph{o}, j’emplis d’encre la queue des \emph{g}, et je fais chaque fois un pâté sur le mot « allégresse ». J’en suis pour une série de taloches. Ah ! elle me coûte gros, la fête de mon père !\par
Enfin, je parviens à faire tenir, entre les filets d’or teintés de violet et portés par des colombes, quelques phrases qui ont l’air d’ivrognes, tant les mots diffèrent d’attitudes, grâce aux haltes que j’ai faites à chaque syllabe pour les \emph{fioner} !\par
Ma mère se résigne et décide qu’on ne peut pas se ruiner en mains de papier ; je signe – encore un pâté – encore une claque. – C’est fini !\par
Reste à régler la cérémonie.\par
« Le papier comme ceci, le pot de fleurs comme cela, tu t’avances… »\par
Je m’avance et je casse deux vases qui figurent le pot de fleurs ; – c’est quatre gifles, deux par vase.\par
\bigbreak
\noindent Il est temps que le beau jour arrive : la nuit, je rêve que je marche pieds nus sur des tessons et qu’on m’empale avec des rouleaux de papier à compliment, ce qui me fait mal !\par
L’achat du pot provoque un grand désordre sur la place du marché. Ma mère prend les pots et les flaire comme du gibier ; elle en remue bien une centaine avant de se décider, et voilà que les jardiniers commencent à se fâcher ! – elle a dérangé les étalages, troublé les classifications, brouillé les familles ; un botaniste s’y perdrait !\par
On l’insulte, on a des mots grossiers pour elle – et même pour son fils – qu’on ne craint pas d’appeler « aztèque » et avorton. Il est temps de fuir.\par
Au bout de la place, ma mère s’arrête et me dit :\par
« Jacques, va-t’en demander au gros – celui qui est au bout, tu sais, – s’il veut te donner le géranium pour onze sous. »\par
Il faut que je retourne dans cette bagarre, vers ce gros-là ; c’est justement celui qui m’a appelé « avorton ».\par
J’en ai la chair de poule. J’y vais tout de même ; j’ai l’air de chercher une épingle par terre ; je marche les yeux baissés, les cuisses serrées, comme un ressort rouillé qui se déroule mal, et j’offre mes onze sous.\par
Il a pitié, ce gros, et il me donne le géranium sans trop se moquer de moi. Les autres ne sont pas trop cruels non plus, et je puis rejoindre ma mère avec cette fleur, emblème de notre allégresse :\par


\begin{verse}
Accepte cette fleur…\\
Qui poussa dans mon cœur.\\
\end{verse}


\labelblock{Vendredi soir.}

\noindent Vendredi soir, répétition générale, dans le mystère et l’ombre.\par
Mon père – Antoine – est censé ne plus savoir ce qui se passe. Il sait tout ; il a même hier soir renversé le géranium mal caché, et je l’ai vu qui le relevait à la sourdine et le refrisait d’un geste furtif.\par
Il a failli marcher sur le compliment raide, gommé, et qui en gardera la cassure. Je l’avais pourtant caché dans la table de nuit. Il sait tout, mais il feint, naïf comme un enfant et bon comme un patriarche, de tout ignorer. Il faut que ce soit une surprise.\par
Le matin du jour solennel, j’arrive : il est dans son lit.\par
« Comment ! c’est ma fête ? »\par
Avec un sourire, tournant un œil d’époux vers ma mère :\par
« Déjà si vieux ! Allons, que je vous embrasse ! »\par
Il embrasse ma mère qui me tient par la main comme Cornélie amenant les Gracques, comme Marie-Antoinette traînant son fils. Elle me lâche pour tomber dans les bras de son époux.\par
\bigbreak
\noindent C’est mon tour ; je croyais que je devais dire le compliment d’abord et qu’on n’embrassait qu’après le pot de fleurs. Il paraît qu’on embrasse avant.\par
Je m’avance.\par
Je tiens le géranium de onze sous et le rouleau, ce qui me gêne pour grimper.\par
Mon père m’aide, il me trouve lourd ; je monte une jambe, – je glisse. Mon père me rattrape, il est forcé de me saisir par le fond de la culotte, et je tourne un peu dans l’espace. Ce n’est pas ma figure qu’il a devant les yeux ; moi-même je ne trouve pas son visage. Quelle position ! Puis je sens le géranium qui file ; il a filé, et tout le terreau tombe dans le lit. La couverture était un peu soulevée.\par
On me chasse dans la chambre à coups de pied, et je n’ai pas la joie pure d’embrasser mon père, d’être embrassé par lui le jour de sa fête ; mais je n’ai pas non plus à lire le compliment. C’est entendu, bâclé, fini. Il y a un peu de fumier dans le lit.\par
\bigbreak
\noindent La fête de ma mère ne me produit pas les mêmes émotions : c’est plus carré.\par
Elle a déclaré nettement, il y a de longues années déjà, qu’elle ne voulait pas qu’on fît des dépenses pour elle. Vingt sous sont vingt sous. Avec l’argent d’un pot de fleurs, elle peut acheter un saucisson. Ajoutez ce que coûterait le papier d’un compliment ! Pourquoi ces frais inutiles ? Vous direz : ce n’est rien. C’est bon pour ceux qui ne tiennent pas la queue de la poêle de dire ça ; mais elle, qui la tient, qui fricote, qui dirige le ménage, elle sait que c’est quelque chose. Ajoutez quatre sous à un franc, ça fait vingt-quatre sous partout.\par
Quoique je ne songe pas à la contredire, mais pas du tout (je pense à autre chose, et j’ai justement mal au ventre), elle me regarde en parlant, et elle est énergique, très énergique.\par
Puis les plantes, ça crève quand on ne les soigne pas.\par
Elle a l’air de dire : on ne peut pas les fouetter !\par
\bigbreak
\noindent La grande distraction qu’elle m’offre est la messe de minuit, parce que c’est gratis.\par
La messe de minuit !\par
De la neige sur les toits et la crête des murs.\par
Elle a fondu sous les pieds des passants dans la rue et l’on patauge dans la boue.\par
C’est triste en haut, sale en bas.\par
Il y a un monde fou chez les charcutiers.\par
On commande du boudin pour la nuit ; et notre épicier a tué un cochon exprès l’autre soir.\par
L’odeur vive et crue des salaisons domine mes souvenirs de Noël.\par
Une satanée petite queue de cochon m’apparaît partout, même dans l’église.\par
Le cordon de cire au bout de la perche de l’allumeur, le ruban rose, qui sert à faire des signets dans les livres et jusqu’à la mèche d’un vicaire, qui tire-bouchonne, isolée et fadasse au coin d’une oreille violette ; la flamme même des cierges, la fumée qui monte en se tortillant des trous des encensoirs, sont autant de petites queues de cochon que j’ai envie de tirer, de pincer ou de dénouer ; que je visse par la pensée à un derrière de petit porc gras, rose et grognon, et qui me fait oublier la résurrection du Christ, le bon Dieu, Père, Fils, Vierge et C\textsuperscript{ie}.\par
J’aspire une odeur de sel comme au bord de la mer, et par la pensée je gratte la cire jaune pour en faire de la chapelure ou de la moutarde !\par
Je lâche ma mère pour aller avec les voisins à l’épicerie qui est à côté de chez nous.\par
Les acheteurs chez notre épicière sont des impies.\par
Ils ont attaqué un saucisson sur le comptoir en buvant une bouteille de vin blanc.\par
J’en ai une goutte, et le piquant du vin, la saveur de la charcuterie m’ont agaillardi.\par
Leur conversation est poivrée comme le reste.\par
Je n’y comprends rien, mais je vois qu’ils disent du mal du ciel et de l’Église, et qu’ils sont tout de même pleins d’appétit et de gaieté.\par
« Encore une rondelle, une hostie à l’ail ! – Versez toujours, madame Potin ! – Nous nous retrouverons en enfer, n’est-ce pas ? Toutes les jolies femmes y sont. Croyez-vous pas que saint Joseph était cocu »
\section[{8. Le Fer-à-Cheval}]{8. \\
Le Fer-à-Cheval}\renewcommand{\leftmark}{8. \\
Le Fer-à-Cheval}

\noindent Le Fer-à-cheval…\par
J’y vais avec ma cousine Henriette.\par
C’est pour voir Pierre André, le sellier du faubourg, qu’elle y vient.\par
Il est de Farreyrolles comme elle et elle doit lui donner des nouvelles de sa famille, des nouvelles intimes et que je ne puis pas connaître ; car ils s’écartent pour se les confier, et elle les lui dit à l’oreille.\par
Je le vois là-bas qui se penche ; et leurs joues se touchent. Quand Henriette revient, elle est songeuse et ne parle pas.\par
\bigbreak
\noindent Il y a aussi la promenade d’Aiguille, toute bordée de grands peupliers. De loin ils font du bruit comme une fontaine.\par
C’est l’automne ; ils laissent tomber des feuilles d’or qui ont encore la queue vivante et la peau tendre comme des poires.\par
Je m’amuse à bouleverser ces tas de feuilles sous mes pieds. Plus loin, de hauts marronniers, avec les marrons tombés. J’en ramasse plein mes poches pour en faire des chapelets ; mais je ne pensais pas au bon Dieu en les enfilant !\par
Je me figure que je troue des rognons, de ces beaux rognons frais, violets, luisants que j’entrevois chez les bouchers.\par
\bigbreak
\noindent Ce que j’aime, c’est le soleil qui passe à travers les branches et fait des plaques claires, qui s’étalent comme des taches jaunes sur un tapis ; puis les oiseaux qui ont des pattes élastiques comme des fils de fer, avec une tête qui remue toujours ; – et surtout cet air frais, ce silence !\par
On ne distingue que la cloche du couvent de Sainte-Marie, et le bruit que fait un attelage à grelots dans la route blanche, là-bas…\par
« Écoute, mademoiselle Balandreau, on n’entend que moi… »\par
Et je jette un cri, ou je lance une pierre bien haut, qui emplit tout l’horizon et retombe.\par
C’est comme un coup sur la poitrine.\par
Quelquefois sur les bancs du fond un monsieur et une dame s’asseyent et causent tout bas.\par
Mademoiselle Balandreau m’éloigne, mais je me retourne.\par
Comme ils s’embrassent !\par

\labelblock{LE PLOT}

\noindent Mes tantes y arrivent le samedi pour vendre du fromage, des poulets et du beurre.\par
Je vais les y voir, et c’est une fête chaque fois.\par
C’est qu’on y entend des cris, du bruit, des rires !\par
Il y a des embrassades et des querelles.\par
Il y a des engueulades qui rougissent les yeux, bleuissent les joues, crispent les poings, arrachent les cheveux, cassent les œufs, renversent les éventaires, dépoitraillent les matrones et me remplissent d’une joie pure.\par
Je nage dans la vie familière, grasse, plantureuse et saine.\par
J’aspire à plein nez des odeurs de nature : la marée, l’étable, les vergers, les bois…\par
Il y a des parfums âcres et des parfums doux, qui viennent des paniers de poissons ou des paniers de fruits, qui s’échappent des tas de pommes ou des tas de fleurs, de la motte de beurre ou du pot de miel.\par
Et comme les habits sont bien des habits de campagne !\par
Les vestes des hommes se redressent comme des queues d’oiseaux, les cotillons des femmes se tiennent en l’air comme s’il y avait un champignon dessous.\par
Des cols de chemise comme des œillères de cheval, des pantalons à ponts, couleur de vache, avec des boutons larges comme des lunes, des chemises pelucheuses et jaunes comme des peaux de cochons, des souliers comme des troncs d’arbre…\par
Les parapluies énormes, en coton sang-de-bœuf, les longs bâtons qui ont le bout comme un oignon, les petites poules noires qui se cognent contre les cages, les coqs fiers, piaffant sur leurs pattes à la hussarde…\par
C’est l’arche de Noé en plein vent, déballée sur un lit de fumier, de paille et de feuillage.\par
La fontaine claire vomit par la gueule de ses lions des nappes de fraîcheur.\par
\bigbreak
\noindent Un homme qui a une tête de belette, la mine triste, qui n’a pas l’air d’un paysan, ni d’un ouvrier, mais d’un mendiant endimanché ou d’un prisonnier libéré de la veille, montre dans un panier des petits loups vivants.\par
Prisonnier ? Mendiant ?\par
Il appartient, bien sûr, à cette race.\par
On ne veut pas de lui dans les fermes, parce qu’il y a quelque histoire dans sa vie.\par
Il est le fils d’un guillotiné ou d’un galérien ; ou bien il a lui-même eu affaire aux gendarmes.\par
Il rôde sur la marge des bois, sur le bord de la rivière, dans la montagne.\par
Quand il peut attraper un renard, un loup, – quelquefois il blesse un aigle, – il montre sa bête ou sa nichée pour deux sous à la ville ; pour un morceau de lard dans les villages.\par
J’ai eu peur de lui jusqu’au jour où mon oncle Joseph lui a donné dix sous et lui a parlé :\par
« Comment ça va, Désossé ? »\par
Et en s’en allant il a dit : « Pauvre bougre ! il ne mange pas tous les jours. »\par

\labelblock{SUR LE BREUIL}

\noindent J’ai eu bien des émotions au Breuil.\par
On a planté une tente de toile comme une grosse toupie renversée, et, en allant faire une commission, j’ai vu par-là un grand nègre.\par
C’est le cirque Bouthors, qui vient s’installer dans la ville.\par
Ils ont un éléphant et un chameau, une bande de musiciens à shakos et à tuniques rouges, avec des parements d’or et des épaulettes comme des pâtés.\par
Ils ont fait le tour de la ville en battant de la grosse caisse ; les écuyères sont en amazones et les écuyers en généraux.\par
Les paysans regardaient, la bouche ouverte ; les gamins suivaient en trottant.\par
Une écuyère a laissé tomber sa cravache.\par
Nous nous sommes jetés dix pour la ramasser, et on s’est battu à qui la rendrait. L’écuyère riait ; son œil a rencontré le mien ; et j’ai senti comme quand ma tante de Bordeaux m’embrassait…\par
J’veux \emph{la} revoir, \emph{cette femme} !\par
Puis je reverrai aussi le chameau et l’éléphant.\par
Sur l’affiche on les montre qui se mettent à genoux, dansent sur deux jambes, débouchent des bouteilles – avec un clown bariolé qui fait le saut périlleux par-dessus.\par
Je les ai revus, tous ; et même le clown m’a donné, en se jetant, par farce, sur le parterre, un coup de tête dans l’estomac.\par
\bigbreak
\noindent « C’est sur moi qu’il est tombé !\par
— Pas vrai, sur moi !\par
— À preuve qu’il m’a laissé du blanc sur ma veste !\par
— Il ne t’a pas écorché, toi, – j’ai du rouge à la joue, c’est lui qui m’a fait ça ! »\par
Et de là, dispute à qui a été bousculé, blanchi, ensanglanté par le clown !\par
\bigbreak
\noindent Au tour de l’écuyère !\par
Elle arrive ! – Je ne vois plus rien ! Il me semble qu’elle me regarde…\par
Elle crève les cerceaux, elle dit : Hop ! hop !\par
Elle encadre sa tête dans une écharpe rose, elle tord ses reins, elle cambre sa hanche, fait des poses ; sa poitrine saute dans son corsage, et mon cœur bat la mesure sous mon gilet.\par
« Qu’est-ce que tu as donc, Jacques, tu es blanc comme le clown ! »\par
\bigbreak
\noindent Je suis amoureux de Paola ! – c’est le nom de l’écuyère.\par
J’ai envie de la voir encore. Il le faut ! Mais je n’ai pas les dix sous, prix des troisièmes.\par
J’irai tout de même.\par
Je me fais beau, je prends en cachette dans l’armoire mon gilet des dimanches, je mets des manchettes de ma mère et je pars pour le Breuil, en disant que je vais jouer chez le petit Grélin.\par
Il fait nuit. Je traverse la place toute noire, jusqu’à ce que j’aperçoive les lampions qui brûlent rouge dans la brume. La musique est rentrée dans l’intérieur ; on a commencé. J’entends claquer la chambrière à travers la toile qui sert de mur.\par
\emph{Elle} est là !\par
Je n’ai pas dix sous, rien, rien !… que mon amour.\par
Je fais le tour du manège, je colle mon œil à des fentes, je me dresse sur mes orteils, à m’en casser les ongles ; pas un trou pour mon regard de flamme !\par
\bigbreak
\noindent Par ici…\par
Par ici la toile est plus courte. Elle est déchirée près du poteau, et en déchirant encore un peu…\par
J’ai élargi la déchirure, mis le pied – je veux dire passé la tête – dans le chemin qui conduit à l’écurie.\par
Je suis à plein ventre par terre, dans la boue, et je me glisse comme un voleur, comme un assassin, la nuit, dans un cirque habité !\par
M’y voici ! Je rampe sous les planches, je me racle au poteau, je me fais des écorchures aux mains ; mon nez, qui s’est aplati contre un madrier, ne donne plus signe de vie ; je ne le sens plus, j’ai peur de l’avoir perdu en route ; ce que je tiens n’y ressemble guère ; mais encore un effort, encore une blessure, et je pourrai \emph{la} voir en passant derrière cette grosse bonne.\par
Je vais grimper !… Je grimpe, – un point d’appui me manque… je me raccroche à ce que je trouve…\par
Un cri !… tumulte !\par
Une femme serre ses jupes, appelle au secours !\par
On croit que le cirque s’écroule !\par
J’ai pris la bonne à pleine chair, je ne sais où ; elle a cru que c’était le singe ou la trompe égarée de l’éléphant.\par
On me prend moi-même par la peau de ce qu’on peut, on me pousse comme du crottin dans l’écurie, on m’interroge, je ne réponds pas !\par
On m’entoure. ELLE est là près de moi. ELLE ! Je l’entends, mais je ne peux pas la voir à cause de mon nez qui gonfle.\par
\bigbreak
\noindent Je me retrouve à temps à la maison pour m’entendre avec madame Grélin, qui m’empêchera d’être fouetté, – (oh ! Paola !) et à qui je dis tout, – tout, moins le secret de mon amour ! Compromettre une femme ! J’ai tout mis sur le compte du chameau, qui a bon dos, et de l’éléphant dont on a soupçonné la trompe.\par
Et quand quelquefois je tâche de me rappeler le Breuil, c’est toujours Paola et le gras de la bonne que la mémoire empoigne. Le Breuil tient dans ce cirque, sous ce maillot et cette jupe…
\section[{9. Saint-Étienne}]{9. \\
Saint-Étienne}\renewcommand{\leftmark}{9. \\
Saint-Étienne}

\noindent Mon père a été appelé comme professeur de septième à Saint-Étienne, par la protection d’un ami. Il a dû filer \emph{dare-dare}.\par
Ma mère et moi, nous sommes restés en arrière, pour arranger les affaires, emballer, etc., etc.\par
\bigbreak
\noindent Enfin nous partons. Adieu le Puy !\par
\bigbreak
\noindent Nous sommes dans la diligence ; il fait froid, c’est en décembre. Nous avons pour compagnons de route un commis voyageur, une grosse femme et un petit vieux.\par
La grosse femme a une poitrine comme un ballon, avec une échancrure dans la robe qui laisse voir un V de chair blanche, douce à l’œil et qui semble croquante comme une cuisse de noix. Elle a des yeux dans le genre de ceux de ma tante, avec des cils très longs.\par
Une plaisanterie – à laquelle je ne comprends rien – dite par le commis voyageur, lui écarte les lèvres et lui arrache un bon gros rire. À partir de ce moment-là, ils ne font plus que rigoler et ils se donnent même des tapes, au grand scandale de ma mère, qui s’écarte et manque de m’écraser dans mon coin, à la grande joie du petit vieux qui se frotte les mains et cligne de l’œil en branlant la tête.\par
Quand on arrive aux relais, ils descendent ensemble et je les vois à travers les fenêtres de l’auberge qui se passent les radis – toujours en riant – et s’allongent des coups de coude.\par
Le commis voyageur offre à la grosse un bouquet qu’un mendiant lui a vendu et demande qu’elle le fourre dans son corsage ; elle finit par mettre le bouquet où il veut.\par
Comme elle est plus gaie que ma mère, celle-là !\par
Que viens-je de dire ?… Ma mère est une sainte femme qui ne rit pas, qui n’aime pas les fleurs, qui a son rang à garder, – son honneur, Jacques !\par
Celle-ci est une femme du peuple, une marchande (elle vient de le dire en remontant dans sa voiture) ; elle va à Beaucaire pour vendre de la toile et avoir une boutique à la foire. Et tu la compares à ta mère, jeune Vingtras !\par
\bigbreak
\noindent Nous arrivons à Saint-Étienne.\par
\bigbreak
\noindent Il fait nuit ; mon père n’est pas là pour nous recevoir.\par
Nous attendons debout entre les malles. Il y a de la neige plein les rues et je regarde l’ombre des réverbères se détacher sur ce blanc cru. Ma mère fouille la place d’un œil qui lance des éclairs ; elle va et vient, se mord les lèvres, se tord les mains, fatigue les employés de questions éternelles.\par
On lui demande si elle veut entrer ou sortir, se tenir dans le bureau ou sur le pavé, si elle persistera longtemps avec ses malles à encombrer la porte.\par
« J’attends mon mari qui est professeur au lycée. »\par
Ils ont l’air de s’en moquer un peu !\par
Je voudrais bien rester dans le bureau ; j’ai les pieds gelés, les doigts engourdis, le nez qui me cuit. J’en fais part à ma mère.\par
« Jacques ! »\par
Un « Jacques » qui inaugure mal notre entrée dans cette ville – et elle marmotte entre ses dents qui claquent :\par
« Il laisserait sa mère crever de froid, tenez, tandis qu’il se rôtirait les cuisses ! »\par
Mais elle peut se rôtir les jambes aussi ! Rien ne l’empêche, puisqu’on lui a demandé si elle voulait se mettre près du feu.\par
\bigbreak
\noindent Mon père arrive tout essoufflé.\par
« Je suis en retard… (Il s’essuie le front.) Vous avez fait un bon voyage ? » (Il tend les bras vers ma mère et la manque.)\par
Il se retourne vers moi.\par
« Ah ! voilà Jacques !\par
— Crois-tu pas que je t’en aurais amené un autre ? » dit ma mère.\par
Mon père dit : « Non, non ! » – c’est-à-dire – il ne sait plus trop.\par
Il va pour m’embrasser à mon tour, il me rate ; comme il a raté ma mère. Pas de chance pour les embrassades, pas de veine pour les baisers.\par
« J’étais avec l’économe, M. Laurier, tu sais… Je croyais que la diligence… »\par
On ne lui répond rien, rien, rien !\par
\bigbreak
\noindent Nous prenons un fiacre pour nous rendre à la maison.\par
Du silence tout le long de la route, du silence et de la neige. Mon père regarde à la portière, ma mère s’est accroupie dans un coin, je suis au milieu, n’osant bouger de crainte qu’on n’entende tourner mes os, virer ma tête. Je tourmente du bout du doigt un gland de parapluie ; à ce moment le parapluie m’échappe – je me penche pour le rattraper ; mon père se tournait – \emph{pan} ! – Nous nous cognons – nous nous relevons comme deux Guignols ! – Encore un faux mouvement – \emph{pan, pan} ! – c’est en mesure.\par
Le sourire jaune reparaît sur la face de mon père ; des changements visibles s’opèrent sur la mienne. C’était la lutte de l’œuf dur contre l’œuf mollet. Mon père a pu supporter le choc et il sourit. – Bonne nature ! Mais moi j’ai une bosse qui enfle, c’est pesant comme une maison. Mon père étend sa main dans l’obscurité, pour tâter, et aussi parce que mon front a l’air d’avancer et va le gêner tout à l’heure ; il étend la main, c’est mon nez qu’il attrape ; il croit de son devoir, plus paternel et plus gracieux, plus conforme à sa dignité ou meilleur à ma santé de rester un instant sur ce nez qu’il a l’air de bénir ou de consulter.\par
De ma mère on ne voit rien, on n’entend rien, qu’un grincement de soie : ce sont ses ongles qui en veulent à sa ceinture.\par
Ce grincement dans le silence a quelque chose de terrible. Pour des augures, c’eût été un présage ; pour mon pauvre père, c’en était un aussi ; il annonçait des malheurs. Il devait nous en arriver au moins un, en effet, dans cette ville que traversait, neigeuse et triste, notre fiacre muet.\par
\bigbreak
\noindent La maison où la voiture nous descend fait le coin de la rue.\par
L’entrée est misérable, avec des pierres qui branlent sur le seuil, un escalier vermoulu et une galerie en bois moisi à laquelle il manque des membres.\par
Nous faisons trembler ce bois sous nos mains, ces pierres sous nos pieds – ce qui gêne tout le monde. Il semblait qu’on devait rester muet jusqu’à la fin des siècles. Mon père fait l’affairé.\par
« Passe devant, dit-il. Il y a une marche ici. Prends garde, un trou là. Tiens-toi à la rampe. »\par
Il joue avec la clef pendue à son petit doigt ; le geste est isolé et saugrenu comme un geste de bébé.\par
Je traînais le parapluie.\par
Ordinairement, quand je laisse ce parapluie piquer la robe ou cogner le flanc de ma mère, c’est du « maladroit » par-ci, du « nigaud » par-là ; elle crie, je reçois une gifle.\par
Je donnerais beaucoup pour recevoir une gifle ; ma mère est contente quand elle me donne une gifle, – cela l’émoustille, c’est le frétillement du hoche-queue, le plongeon du canard, – elle s’étire et rencontre la joue de son fils. Quelle joie pour une mère de le sentir à sa portée et de se dire : c’est lui, c’est mon enfant, mon fruit, cette joue est à moi, – clac !\par
Mais non.\par
Elle a les bras croisés et les garde cachés sous son châle… Allons ! Elle n’est pas disposée à la bonne humeur.\par
Mon père use un tas d’allumettes ; elles se cassent et font un petit bruit sec qui est tout ce qu’on entend devant cette porte fermée, dans le corridor que glace le vent, avec ma mère et moi contre le mur comme des habits de la Morgue.\par
Jamais moment ne m’a paru plus long.\par
Enfin une des chimiques prend, et mon père peut introduire la clef dans la serrure…\par
Nous entrons dans une pièce immense où arrive, par des croisées énormes, la lumière d’un réverbère qui clignote dans la rue.\par
Elle tombe en plein sur ma mère, qui se tient immobile et muette, avec la rigidité d’une morte, l’insensibilité d’un mannequin et la solennité d’un revenant.\par

\astertri

\noindent Mais je sauve toujours les situations avec ma tête ou mon derrière, mes oreilles qu’on tire ou mes cheveux qu’on arrache, en glissant, m’accroupissant ou roulant, comme l’ahuri des pantomimes, comme l’\emph{innocent} des escamoteurs.\par
Je me sens tout d’un coup dégringoler, je tombe !\par
Il y avait une pelure d’orange sous mon talon ; ce dont on s’aperçoit en se penchant vers moi, comme sur un problème. Je déconcerte les mathématiciens par l’imprévu de mes opérations. – C’est ma mère tout d’un coup rappelée à l’amour de son fils, par cette chute à tournure de mystification, qui remarque la première cette peau d’orange.\par
Elle croise ses bras et avance sur mon père :\par
« On mange donc des oranges ici, on mange des oranges !… »\par
Et elle trépigne, trépigne… Je ne sais ce que cela veut dire.\par
Je suis à terre, forcé de lever la tête pour voir tout ce qui se passe ; ma situation d’historiographe ressemble à celle d’un cul-de-jatte qu’on a porté là et laissé tomber comme un sac trop lourd.\par
Je ne veux pourtant pas mourir à cette place ! Puis je ne dois pas écouter ma mère qui est debout, dans cette position indifférente, m’isolant d’elle avec l’apparence du mépris ; Jacques, tu as trop tardé déjà !\par
Relève-toi, et mets-toi entre le discours de ta mère et l’effroi de ton père. Relève-toi, fils ingrat.\par
Mais non, non !\par
J’ai voulu bouger… je ne puis…\par
Je suis tombé sur une gravure et j’ai cassé le verre.\par
On est forcé de reconnaître des lésions affligeantes, et quelques gouttes de sang qui traînent sur le plancher servent de prétexte à mon père – et à ma mère aussi – pour entrer dans des mouvements nouveaux. J’en tressaille d’aise (autant que je puis tressaillir sans trop de souffrance, entendons-nous). Mais je suis bien content tout de même d’avoir dérangé ce silence, \emph{cassé la glace}, et ma famille en arrache les morceaux.\par
On me lave comme une pépite ; on me sarcle comme un champ.\par
L’opération est minutieuse et faite avec conscience.\par
Dans le hasard de l’échenillage, les mains se rencontrent, les paroles s’appellent ; on se réconcilie sournoisement sur ma blessure, et je crois même que mon père fait traîner le sarclage pour laisser à la colère de sa femme le temps de tomber tout à fait. Je saigne bien un peu ; je suis tantôt à quatre pattes, tantôt sur le ventre, suivant qu’ils l’ordonnent et que les piquants se présentent ; mais je sens que j’ai rendu service à ma famille, et cela est une consolation, n’est-ce pas ?\par
Au lieu de pousser tant de haricots dans les coins, pourquoi M. Beliben ne dirait-il pas : « Voyez si Dieu est fin et s’il est bon ! que lui a-t-il fallu pour raccommoder l’époux et l’épouse qui se fâchaient ? Il a pris le derrière d’un enfant, du petit Vingtras, et en a fait le siège du raccommodement. »\par
On pouvait me montrer dans les cours de philosophie ou de catéchisme.\par
J’en fus malade, j’eus la fièvre. Mais l’orage avait été apaisé : on s’explique sur la peau d’orange, avec calme ; on donna une raison pour l’arrivée tardive à la diligence ; on mit les compresses sur la colère ; on m’en mit aussi ailleurs.\par
On s’expliqua sur la peau d’orange, mais il paraît qu’il y avait un mystère, tout de même…\par
Mon père avait menti en disant que M. Laurier l’avait retenu ; je le sus en l’entendant causer avec un collègue, qui vint le voir, à un moment où ma mère, fatiguée par le voyage, l’attente, l’orage et surtout l’échenillage, faisait un somme.\par
« Vous direz ceci, je dirai cela. Nous préviendrons Chose. – Pourvu qu’\emph{elles} ne s’avisent pas de nous reconnaître dans la rue. – Il n’y a pas de danger, au moins ? »\par
J’entendais tout de mon lit, où je reposais à plat ventre, un peu de côté, par instants, et je me demandais ce que ce \emph{elles} signifiait.
\section[{10. Braves gens}]{10. \\
Braves gens}\renewcommand{\leftmark}{10. \\
Braves gens}

\noindent Je pourrais à peine dire comment était fait l’appartement dans lequel nous entrâmes, ainsi que je l’ai conté, avec bris de cadre, clignotement de réverbère et raccommodement posthume – si posthume est le mot.\par
À peine étions-nous installés, qu’un grand événement arriva.\par
Ma mère dut repartir pour recueillir ou soigner une succession, – celle de la tante Agnès peut-être, et je restai seul avec mon père.\par
\bigbreak
\noindent C’est une vie nouvelle, – il n’est jamais là, je suis libre, et je vis au rez-de-chaussée avec les petits du cordonnier et ceux de l’épicière.\par
J’adore la poix, la colle, le tire-fil : j’aime à entendre le tranchet passer dans le gras du cuir et le marteau tinter sur le veau neuf et la pierre bleue.\par
On s’amuse dans ce tas de savates, et le grand frère ressemble à mon oncle Joseph. Il est compagnon du Devoir aussi, il a un grade, et quelquefois c’est moi qui attache les rubans à sa canne et brosse sa redingote de cérémonie. Les jours ordinaires, il me laisse planter des clous et prendre des coins de maroquin rouge.\par
Je suis presque de la famille. Mon père m’a mis en pension chez eux ; il dîne je ne sais où, au collège sans doute, avec les professeurs d’élémentaires. Moi, j’avale des soupes énormes, dans des écuelles ébréchées, et j’ai ma goutte de vin dans un gros verre, quand on mange le \emph{chevreton}.\par
Ils sont heureux dans cette famille ! – c’est cordial, bavard, bon enfant : tout ça travaille, mais en jacassant ; tout ça se dispute, mais en s’aimant.\par
On les appelle les Fabre.\par
L’autre famille du rez-de-chaussée, les Vincent, sont épiciers.\par
\bigbreak
\noindent Madame Vincent est une rieuse. Je les trouve tous gais, les gens que je vois et que ma mère méprise parce qu’ils sont paysans, savetiers ou peseurs de sucre.\par
Madame Vincent n’est pas avec son mari. On ne l’a vu qu’une fois, vêtu en Arabe, avec un burnous blanc, mais il n’est resté que deux heures, et est reparti.\par
Il paraît qu’ils sont séparés – judiciairement, je ne sais pas ce que c’est – et il vit en Afrique, en \emph{Algère}, dit Fabre.\par
Il était venu pour chercher un de ses fils. Madame Vincent, qui rit toujours, ne riait pas ce jour-là ! Il s’en fallait de tout ; on l’entendait qui disait : « Non ; non », d’une voix dure, à travers la porte – et le petit Vincent qui pleurait :\par
« Je veux rester avec maman !\par
— Je te donnerai un cheval, avec un pistolet comme celui-là. »\par
Un pistolet ! un cheval !\par
Si mon père m’avait promis cela, et, en plus de m’emmener loin de ma mère ! s’il m’avait pris avec lui, sans la redingote à olives et le chapeau tuyau de poêle, quel soupir de joie j’aurais poussé ! – à la porte seulement – de peur que ma mère ne m’entendît et ne voulût me reprendre !… Oh ! oui, je serais parti !\par
Le petit Vincent, au contraire, pleurait et s’accrochait aux jupes.\par
Il y eut encore du bruit… le père qui se fâchait, la mère qui parlait plus haut et l’enfant qui sanglotait… puis la porte s’ouvrit, le burnous blanc passa. Il ne reparut plus.\par
Il me fit de la peine tout de même. Je le vis qui se cachait au coin de la rue ; il regardait la maison d’où il sortait, où étaient sa femme, son enfant ; il resta un long moment, l’air triste, et je crus m’apercevoir qu’il pleurait.\par
Je trouve des pères qui pleurent, des mères qui rient ; chez moi, je n’ai jamais vu pleurer, jamais rire ; on geint, on crie. C’est qu’aussi mon père est un professeur, un homme du monde, c’est que ma mère est une mère courageuse et ferme qui veut m’élever comme il faut.\par
\bigbreak
\noindent Les Vincent, les Fabre et le petit Vingtras forment une colonie criarde, joueuse, insupportable.\par
« Vous êtes insupportables, Jacques, Ernest… »\par
C’est la mère Vincent qui veut faire la méchante et qui ne peut pas ; c’est le père Fabre qui le dit faiblement, avec un doux sourire de vieux.\par
« Insupportables ! Ah ! si je vous y reprends ! »\par
On nous y reprend sans cesse, et on nous supporte toujours.\par
Braves gens ! Ils juraient, sacraient, en lâchaient de salées ; mais on disait d’eux : « Bons comme le bon pain, honnêtes comme l’or. » Je respirais dans cette atmosphère de poivre et de poix, une odeur de joie et de santé ; ils avaient la main noire, mais le cœur dessus ; ils balançaient les hanches et tenaient les doigts écarquillés, parlaient avec des velours et des cuirs ; – c’est le métier qui veut ça, disait le grand Fabre. Ils me donnaient l’envie d’être ouvrier aussi et de vivre cette bonne vie où l’on n’avait peur ni de sa mère ni des riches, où l’on n’avait qu’à se lever de grand matin, pour chanter et taper tout le jour.\par
Puis, on avait de belles alènes pointues. On voyait luire sous la main le museau allongé d’une bottine, le talon cambré d’une botte, et l’on tripotait un cirage qui sentait un peu le vinaigre et piquait le nez.\par
Braves gens !\par
Ils ne battaient pas leurs enfants – et ils faisaient l’aumône. Ce n’était pas comme chez nous.\par
\bigbreak
\noindent Pendant toute mon enfance, j’ai entendu ma mère dire qu’il ne fallait pas donner aux pauvres : que l’argent qu’ils recevaient, ils l’allaient boire, que mieux valait jeter un sou dans la rivière, qu’au moins il ne roulait pas au cabaret. Je n’ai jamais pu cependant voir un homme demander un sou pour acheter du pain, sans qu’il me tombât du chagrin sur le cœur, comme un poids.\par
Mais comment cela se fait-il cependant ?\par
Madame Vincent était contente quand son fils tirait un des sous de sa petite bourse pour le mettre dans la main d’un malheureux. Elle embrassait Ernest et disait : « Il a bon cœur ! »\par
Madame Vincent voulait donc le malheur de son fils ? Elle l’aimait pourtant, sans cela elle l’aurait donné à l’homme au burnous blanc.\par
Ah ! elles me troublaient un peu, les braves femmes, la mère Vincent et la mère Fabre ! Heureusement cela ne durait pas et ne tenait pas une minute quand j’y réfléchissais.\par
Elles n’osaient pas battre leur enfant, parce qu’elles auraient souffert de le voir pleurer ! Elles lui laissaient faire l’aumône, parce que cela faisait plaisir à leur petit cœur.\par
Ma mère avait plus de courage. Elle se sacrifiait, elle étouffait ses faiblesses, elle tordait le cou au premier mouvement pour se livrer au second. Au lieu de m’embrasser, elle me pinçait ; – vous croyez que cela ne lui coûtait pas ! – Il lui arriva même de se casser les ongles. Elle me battait pour mon bien, voyez-vous. Sa main hésita plus d’une fois ; elle dut prendre son pied.\par
Plus d’une fois aussi elle recula à l’idée de meurtrir sa chair avec la mienne ; elle prit un bâton, un balai, quelque chose qui l’empêchait d’être en contact avec la peau de son enfant, son enfant adoré.\par
Je sentais si bien l’excellence des raisons et l’héroïsme des sentiments qui guidaient ma mère, que je m’accusais devant Dieu de ma désobéissance, et je disais bien vite deux ou trois prières pour m’en disculper. Malheureusement, j’avais très peu de temps à moi, et mes \emph{mea culpa} restaient en l’air parce qu’Ernest, Charles ou Barnabé, un Vincent ou un Fabre, m’appelait pour une glissade, une promenade ou une bourrade, à propos de bottes ou de marmelade ; il y avait toujours quelque tonneau, quelque baquet, quelque querelle ou quelque pot à vider pour aider la boutique ou l’échoppe, le travail ou la rigolade.\par
\bigbreak
\noindent Nous allions au second faire enrager la femme du plâtrier.\par
La plâtrière était une grande blonde, à l’air très doux, fort propre, – un peu languissante ; – elle nous laissait nous engouffrer quelquefois dans sa chambre au milieu de nos jeux, quand son mari n’était pas là ; mais, dès qu’elle l’entendait, il fallait descendre ; elle fermait sa porte et ne reparaissait que pour montrer une figure plus lasse et des hanches plus languissantes encore. Elle parlait toujours à madame Vincent d’avoir un enfant, « qu’elle avait peur que ce ne fût pas encore pour cette fois, que cela désespérait son mari ».\par
Si un des Fabre, celui de dix-huit ans, ou celui de vingt-trois, passait à ce moment, elle se taisait ; mais lui, en manière de farce, jetait un mot qui la faisait rougir jusqu’à la racine de ses cheveux pâles : elle essayait de sourire tout de même, mais elle semblait doucement gênée.\par
« Vous avez du plâtre ici (il montrait une place blanche) et de l’édredon là – (il enlevait une petite plume sur l’épaule, et hochait la tête en rigolant).\par
— Ce M. Fabre !…\par
— Mais dame ! dit-il un jour, on ne les trouve pas sous les choux. »\par
J’étais là, quand il lâcha ce : « On ne les trouve pas sous les choux. »\par
\bigbreak
\noindent Le mot m’entra dans l’oreille comme une alène et s’y attacha comme de la poix.\par
M’a-t-on égaré ?\par
\bigbreak
\noindent Ma mère est revenue. L’affaire d’héritage s’est arrangée, je ne sais trop comment. Je suis retombé sous le fouet et je ne suis plus libre que les jours où elle est absente par hasard.\par
Mais le mardi gras, la femme d’un collègue est venue la prendre à l’improviste pour la consulter sur une toilette, – elle a tant de goût ! – et en même temps pour passer la journée. Ma mère n’a pas eu le temps de m’enfermer. Je suis mon maître, un mardi gras !\par
Ce jour-là, c’est la coutume que dans chaque rue on élève une pyramide de charbon, un bûcher en forme de meule, comme un gros bonnet de coton noir avec une mèche à laquelle on met le feu le matin.\par
On avait dit que ceux de la rue à côté devaient venir démolir notre édifice ; il y avait haine depuis longtemps entre les deux rues. Un polisson, le fils de l’aubergiste du Lion-d’Or, propose de faire sentinelle avec des pierres et une fronde dans la poche ; on a l’ordre de lancer la fronde si l’ennemi s’avance en masse et de loin, de cogner avec la pierre dans sa main si l’on est surpris et saisi.\par
\bigbreak
\noindent Je suis de garde un des premiers.\par
Voilà que je crois reconnaître le petit Somonat, un de la rue Marescaut, qui passe son nez derrière la porte de l’église…\par
Il me semble qu’il fait des signes ; ils vont arriver en masse ; je serai débordé, tourné. – Que dira le fils de l’aubergiste, et toute ma rue ? Oserai-je y repasser, si je ne me défends pas en héros ?\par
Mon parti est pris : j’ai mon tas de pierres, je charge ma fronde et je la fais claquer, en lançant au hasard du côté des Marescauts une mitraille de cailloux, qui sifflent dans l’air et dont j’entends le bruit contre les portes de bois, dans les volets fermés ! Je fouille à l’aventure comme on fouille avec le canon. – Je me figure que je suis au siège d’Arbelles ou à Mazagran. – Si j’avais un drapeau tricolore, je le planterais. – Cette histoire d’Arbelles, nous l’avons traduite hier dans \emph{Quinte-Curce}. Celle de Mazagran est toute fraîche. On ne parle que de cela et du capitaine Lelièvre.\par
Ah ! l’on parlera de moi aussi, – nom de nom !\par
Je bombarde de pierres tout un quartier, au risque de tuer les gens et d’interrompre l’existence normale d’une ville.\par
On sort des maisons et l’on regarde – pas trop – car je manie toujours ma fronde, mais je commence à me demander comment finira le siège.\par
J’ai entendu des carreaux tomber, j’ai vu un caillou entrer dans une chambre ; j’ai peut-être tué quelqu’un. On ne riposte pas ! Je me suis donc trompé ; on n’attaquait point. – Je vais être pris, jugé, mon père perdra sa place.\par
Que faire ?\par
J’ai entendu dire que pour les cessations de feu on arborait le drapeau blanc ; j’ai mon mouchoir, – il est bleu. – Se retirer ? Je le puis peut-être, la place est déserte, en filant à gauche…\par
Je prends ma course.\par
\bigbreak
\noindent Qu’ai-je donc ? Je suis tombé. On m’entoure. J’ai le bras cassé.\par
M. Dropal, le médecin passe, on l’arrête. Que va-t-il dire ?\par
Si par hasard ce n’était rien, que deviendrais-je ?\par
Comment oser rentrer devant ma mère. Et les lapidés, que me feront-ils ?\par
Le médecin hoche la tête avec un « ah ! » qui est triste. Je fais l’évanoui pour mieux l’entendre.\par
« C’est grave, c’est grave ! »\par
Dieu soit loué ! Qu’on aille vite dire à ma mère que c’est grave, pour qu’elle ne pense pas à me gronder et à me rosser !\par
C’était grave ; je ne pouvais pas dire un mot. Plus de chance que je ne méritais : on dit que j’ai la langue coupée ! Comme c’est commode ! pas d’explication à donner ; je serai malade pendant longtemps probablement, et tout sera apaisé quand je serai guéri.\par
\bigbreak
\noindent Je restai longtemps sans pouvoir parler, mais je ne parlai point dès que je le pus.\par
Je voyais bien qu’à mesure que je guérissais, ma mère faisait des additions.\par
« Déjà pour deux francs de diachylum\footnote{Sorte de cataplasme.} ! »\par
Brave femme qui voulait l’économie dans son ménage, et n’oubliait jamais les lois d’ordre, qui sont seules le salut des familles, et sans lesquelles on finit par l’hôpital et l’échafaud.\par
Moi, je me désolais à l’idée que j’allais guérir !\par
J’appréhendais le moment où je serais à point pour être corrigé, quoique je n’eusse pas besoin d’une roulée pour n’avoir pas envie de recommencer ; je ne me sentais pas la moindre inclination pour un nouveau siège, une nouvelle chute, un flot si terrible d’émotions. J’aurais voulu que ma mère le sût, que mon père le comprît, et l’on ne m’aurait peut-être pas frappé.\par
On ne me frappa pas – on fit pire.\par
On savait que je m’amusais chez les Fabre, on me punit par là.\par
\bigbreak
\noindent Au surplus, il y avait longtemps que ma mère était jalouse et honteuse ; elle souffrait de me voir traîner dans un monde de cordonniers, et depuis quelques semaines elle nourrissait le projet de m’en détacher.\par
Seulement elle était bavarde, la mère Vingtras, et on l’écoutait chez les Fabre. Avec leur bonhomie, ils croyaient peut-être qu’elle leur était supérieure, cette dame à chapeau ; en tout cas, ils lui prêtaient une oreille complaisante, et l’on écartait la poix et la colle avec politesse quand elle venait me chercher.\par
Elle voulait que son Jacques ne frayât plus avec les savetiers, mais elle ne voulait pas perdre un auditoire.\par
Mon aventure de mardi gras lui permit de basculer la situation, de ménager la chèvre et le chou.\par
Elle m’infligea comme punition de ne plus y retourner ; elle ne se brouilla point pourtant.\par
« Il faut punir Jacques, n’est-ce pas ? Il faut le punir, mais il a déjà assez souffert, le pauvre enfant.\par
— Oh oui, dit la mère Fabre qui pensait qu’une approbation – même de savetière –, ferait pencher la balance du côté du pardon.\par
— Aussi je ne veux pas le battre. »\par
J’entendais la conversation, non pas que je l’écoutasse, mais j’étais derrière la porte ; ma mère le savait et voulait peut-être que je l’entendisse. C’était la première sortie : j’étais encore assez faible, mal recousu, nourri depuis quinze jours de bouillon un peu pâle ; ma mère savait que trop de suc fait plus de mal que de bien, et qu’on grise les veines avec du jus de vache comme avec du jus de raisin – car c’était de la vache. – « C’est plus tendre, disait-elle, la vache pour les enfants, le bœuf pour les grandes personnes. »\par
J’étais donc soutenu seulement par un peu de vache détrempée ; j’avais encore le détraquement de la chute, et ma tête me semblait vide comme un globe : il me restait peu de sang ; ce qui en restait fit un tour, monta vers les joues creuses, et je les sentais qui brûlaient.\par
« On ne voulait pas me battre ! »\par
On voulait faire plus.\par
« Je ne veux pas le battre, reprit ma mère, mais comme je sais qu’il se plaît bien avec vos fils je l’empêcherai de les voir ; ce sera une bonne correction. »\par
Les Fabre ne répondaient rien, – les pauvres gens ne se croyaient pas le droit de discuter les résolutions de la femme d’un professeur de collège, et ils étaient au contraire tout confus de l’honneur qu’on faisait à leurs gamins, en ayant l’air de dire qu’ils étaient la compagnie que Jacques, qui apprenait le latin, préférait.\par
Je compris leur silence, et je compris aussi que ma mère avait deviné où il fallait me frapper, ce qui faisait mal à mon âme. J’ai quelquefois pleuré étant petit ; on a rencontré, on rencontrera des larmes sur plus d’une page, mais je ne sais pourquoi je me souviens avec une particulière amertume du chagrin que j’eus ce jour-là. Il me sembla que ma mère commettait une cruauté, était méchante.\par
Tout malade encore, presque estropié, enfermé depuis des semaines dans une chambre avec la souffrance et la fièvre, j’avais besoin de causer à des enfants comme moi, de leur demander des nouvelles, et de leur raconter mon histoire.\par
Ils avaient eu l’air bon comme tout, en venant à moi dans l’escalier, et m’avaient dit avec affection : « Comme tu es pâle !… » Il y avait dans leur voix de l’émotion, presque de l’amitié. Braves petits garçons, saine nichée de savetiers, marmaille au bon cœur ! Je les aimais bien. Ma mère aurait mieux faire de me battre et de me laisser les revoir quand mon bras fut guéri.
\section[{11. Le lycée}]{11. \\
Le lycée}\renewcommand{\leftmark}{11. \\
Le lycée}

\noindent Mon père était donc professeur de septième, professeur élémentaire, comme on disait alors.\par
J’étais dans sa classe.\par
Jamais je n’ai senti une infection pareille. Cette classe était près des latrines, et ces latrines étaient les latrines des petits !\par
Pendant une année j’ai avalé cet air empesté. On m’avait mis près de la porte parce que c’était la plus mauvaise place, et en ma qualité de fils de professeur, je devais être à l’avant-garde, au poste du sacrifice, au lieu du danger…\par
À côté de moi, un petit bonhomme qui est devenu un haut personnage, un grand préfet, et qui à cette époque-là était un affreux garnement, fort drôle du reste, et pas mauvais compagnon.\par
Il faut bien qu’il ait été vraiment un bon garçon, pour que je ne lui aie pas gardé rancune de deux ou trois brûlées que mon père m’administra, parce qu’on avait entendu de notre côté un bruit comique, ou qu’il était parti d’entre nos souliers une fusée d’encre. C’était mon voisin qui s’en payait.\par
Chaque fois que je le voyais préparer une farce, je tremblais ; car s’il ne se dénonçait pas lui-même par quelque imprudence, et si sa culpabilité ne sautait pas aux yeux, c’était moi qui la gobais ; c’est-à-dire que mon père descendait tranquillement de sa chaire et venait me tirer les oreilles, et me donner un ou deux coups de pied, quelquefois trois.\par
Il fallait qu’il prouvât qu’il ne favorisait pas son fils, qu’il n’avait pas de préférence. Il me favorisait de roulées magistrales et il m’accordait la préférence pour les coups de pied au derrière.\par
Souffrait-il d’être obligé de taper ainsi sur son rejeton ?\par
Peut-être bien, mais mon voisin, le farceur, était fils d’une autorité. – L’accabler de pensums, lui tirer les oreilles, c’était se mettre mal avec la maman, une grande coquette qui arrivait au parloir avec une longue robe de soie qui criait, et des gants à trois boutons, frais comme du beurre.\par
Pour se mettre à l’aise, mon père feignait de croire que j’étais le coupable, quand il savait bien que c’était l’autre.\par
Je n’en voulais pas à mon père, ma foi non ! je croyais, je sentais que ma peau lui était utile pour son commerce, son genre d’exercice, sa situation, – et j’offrais ma peau. – Vas-y, papa !\par
Je tenais tant bien que mal ma place (empoisonnée) dans ce milieu de moutards malins, tout disposés à faire souffrir le fils du professeur de la haine qu’ils portaient naturellement à son père.\par
Ces roulées publiques me rendaient service ; on ne me regardait pas comme un ennemi, on m’aurait plaint plutôt, si les enfants savaient plaindre !\par
Mon apparence d’insensibilité d’ailleurs ne portait pas à la pitié ; je me garais des horions tant bien que mal et pour la forme ; mais quand c’était fini, on ne voyait pas trace de peur ou de douleur sur ma figure. Je n’étais de la sorte ni un \emph{patiras} ni un pestiféré ; on ne me fuyait pas, on me traitait comme un camarade moins chançard qu’un autre et meilleur que beaucoup, puisque jamais je ne répondais : « Ça n’est pas moi. » Puis j’étais fort, les luttes avec Pierrouni m’avaient aguerri, j’avais du \emph{moignon}, comme on disait en raidissant son bras et faisant gonfler son bout de biceps. Je m’étais battu, – \emph{j’y avais fait} avec Rosée qui était le plus fort de la cour des petits. On appelait cela \emph{y faire}. « Veux-tu \emph{y faire}, en sortant de classe ? »\par
Cela voulait dire qu’à dix heures cinq ou à quatre heures cinq, on se proposait de se flanquer une trépignée dans la cour du Coq-Rouge, une auberge où il y avait un coin dans lequel on pouvait se battre sans être vu.\par
J’avais infligé à Rosée quelques atouts qui avaient fait du bruit – sur son nez et au collège. – Songez donc ! j’avais l’autorisation de mon père.\par
Il avait eu vent de la querelle – pour une plume volée – et vent de la provocation.\par
Rosée ne tenait par aucun fil à l’autorité. Il y avait plus ; son oncle, conseiller municipal, avait eu maille à partir avec l’administration. Je pouvais \emph{y faire}.\par
Et à chaque coup de poing que je lui portais, à ce malheureux, je me figurais que je semais une graine, que je plantais une espérance dans le champ de l’avancement paternel.\par
Grâce à cette bonne aventure, j’échappai au plus épouvantable des dangers, celui d’être – comme fils de professeur – persécuté, isolé, cogné. J’en ai vu d’autres si malheureux !\par
Si cependant mon père m’avait défendu de me battre ; si Rosée eût été le fils du maire ; s’il avait fallu, au contraire, être battu ?…\par
On doit faire ce que les parents ordonnent ; puis c’est leur pain qui est sur le tapis. Laisse-toi moquer et frapper, souffre et pleure, pauvre enfant, fils du professeur…\par
\bigbreak
\noindent Puis les principes !\par
« Que deviendrait une société, disait M. Beliben, une société qui… que… Il faut des principes… J’ai encore besoin d’un haricot… »\par
J’eus la chance de tomber sur Rosée.\par
Où qu’il soit dans le monde, s’il est encore vivant, que son nez reçoive mes sincères remerciements :\par


\begin{verse}
Calice à narines, sang de mon sauveur,\\
Salutaris nasus, encore un baiser !\\
\end{verse}

\noindent … J’ai été puni un jour : c’est, je crois, pour avoir roulé sous la poussée d’un grand, entre les jambes d’un petit pion qui passait par là, et qui est tombé derrière par-dessus tête ! Il s’est fait une bosse affreuse, et il a cassé une fiole qui était dans sa poche de côté ; c’est une topette de cognac dont il boit – en cachette, à petits coups, en tournant les yeux. On l’a vu : il semblait faire une prière, et il se frottait délicieusement l’estomac. – Je suis cause de la topette cassée, de la bosse qui gonfle… Le pion s’est fâché.\par
Il m’a mis aux arrêts ; – il m’a enfermé lui-même dans une étude vide, a tourné la clef, et me voilà seul entre les murailles sales, devant une carte de géographie qui a la jaunisse, et un grand tableau noir où il y a des ronds blancs et la binette du censeur.\par
Je vais d’un pupitre à l’autre : ils sont vides – on doit nettoyer la place, et les élèves ont déménagé.\par
Rien, une règle, des plumes rouillées, un bout de ficelle, un petit jeu de dames, le cadavre d’un lézard, une agate perdue.\par
Dans une fente, un livre : j’en vois le dos, je m’écorche les ongles à essayer de le retirer. Enfin, avec l’aide de la règle, en cassant un pupitre, j’y arrive ; je tiens le volume et je regarde le titre : ROBINSON CRUSOÉ.\par
\bigbreak
\noindent Il est nuit.\par
Je m’en aperçois tout d’un coup. Combien y a-t-il de temps que je suis dans ce livre ? – quelle heure est-il ?\par
Je ne sais pas, mais voyons si je puis lire encore ! Je frotte mes yeux, je \emph{tends} mon regard, les lettres s’effacent, les lignes se mêlent, je saisis encore le coin d’un mot, puis plus rien.\par
J’ai le cou brisé, la nuque qui me fait mal, la poitrine creuse ; je suis resté penché sur les chapitres sans lever la tête, sans entendre rien, dévoré par la curiosité, collé aux flancs de Robinson, pris d’une émotion immense, remué jusqu’au fond de la cervelle et jusqu’au fond du cœur ; et en ce moment où la lune montre là-bas un bout de corne, je fais passer dans le ciel tous les oiseaux de l’île, et je vois se profiler la tête longue d’un peuplier comme le mât du navire de Crusoé ! Je peuple l’espace vide de mes pensées, tout comme il peuplait l’horizon de ses craintes ; debout contre cette fenêtre, je rêve à l’éternelle solitude et je me demande où je ferai pousser du pain…\par
La faim me vient : j’ai très faim.\par
Vais-je être réduit à manger ces rats que j’entends dans la cale de l’étude ? Comment faire du feu ? J’ai soif aussi. Pas de bananes ! Ah ! lui, il avait des limons frais ! Justement j’adore la limonade !\par
\bigbreak
\noindent Clic, clac ! on farfouille dans la serrure.\par
\bigbreak
\noindent Est-ce Vendredi ? Sont-ce des sauvages ?\par
C’est le petit pion qui s’est souvenu, en se levant, qu’il m’avait \emph{oublié}, et qui vient voir si j’ai été dévoré par les rats, ou si c’est moi qui les ai mangés.\par
Il a l’air un peu embarrassé, le pauvre homme ! – Il me retrouve gelé, moulu, les cheveux secs, la main fiévreuse ; il s’excuse de son mieux et m’entraîne dans sa chambre, où il me dit d’allumer un bon feu et de me réchauffer.\par
Il a du thon mariné dans une timbale « et peut-être bien une goutte de je ne sais quoi, par là dans un coin, qu’un ami a laissée il y a deux mois ».\par
C’est une topette d’eau-de-vie, son péché mignon, sa marotte humide, son dada jaune.\par
Il est forcé de repartir, de rejoindre sa division. Il me laisse seul, seul avec du thon, – poisson d’Océan, – la goutte – salut du matelot – et du feu, – phare des naufragés.\par
Je me rejette dans le livre que j’avais caché entre ma chemise et ma peau, et je le dévore – avec un peu de thon, des larmes de cognac – devant la flamme de la cheminée.\par
Il me semble que je suis dans une cabine ou une cabane, et qu’il y a dix ans que j’ai quitté le collège ; j’ai peut-être les cheveux gris, en tout cas le teint hâlé. – Que sont devenus mes vieux parents ? Ils sont morts sans avoir eu la joie d’embrasser leur enfant perdu ? (C’était l’occasion pourtant, puisqu’ils ne l’embrassaient jamais auparavant.) Ô ma mère ! ma mère !\par
Je dis : « ô ma mère ! » sans y penser beaucoup, c’est pour faire comme dans les livres.\par
Et j’ajoute : « Quand vous reverrai-je ? Vous revoir et mourir ! »\par
Je la reverrai, \emph{si Dieu le veut.}\par
Mais quand je reparaîtrai devant elle, comment serai-je reçu ? Me reconnaîtra-t-elle ?\par
Si elle allait ne pas me reconnaître !\par
N’être pas reconnu par celle qui vous a entouré de sa sollicitude depuis le berceau, enveloppé de sa tendresse, une mère enfin !\par
Qui remplace une mère ?\par
Mon Dieu ! une trique remplacerait assez bien la mienne !\par
Ne pas me reconnaître ! mais elle sait bien qu’il me manque derrière l’oreille une mèche de cheveux, puisque c’est elle qui me l’a arrachée un jour. Ne pas me reconnaître ; mais j’ai toujours la cicatrice de la blessure que je me suis faite en tombant, et pour laquelle on m’a empêché de voir les Fabre. Toutes les traces de sa tutelle, de sa sollicitude, se lisent en raies blanches, en petites places bleues. Elle me reconnaîtra ; il me sera donné d’être encore aimé, battu, fouetté, pas gâté !\par
Il ne faut pas gâter les enfants.\par
\bigbreak
\noindent Elle m’a reconnu ! merci, mon Dieu ! Elle m’a reconnu et s’est écriée :\par
« Te voilà donc ! s’il t’arrive de me faire encore t’attendre jusqu’à deux heures du matin, à brûler la bougie, à tenir la porte ouverte, c’est moi qui te corrigerai ! Et il bâille encore ! devant sa mère !\par
— J’ai sommeil.\par
— On aurait sommeil à moins !\par
— J’ai froid.\par
— On va faire du feu exprès pour lui, – brûler un fagot de bois !\par
— Mais c’est M. Doizy qui…\par
— C’est M. Doizy qui t’a oublié, n’est-ce pas ! Si tu ne l’avais pas fait tomber, il n’aurait pas eu à te punir, et il ne t’aurait pas oublié. Il voudrait encore s’excuser, voyez-vous ! Tiens ! voilà ce qui me reste d’une bougie que j’ai commencée hier. Tout ça pour veiller en se demandant ce qu’était devenu monsieur ! Allons, ne faisons pas le gelé, – n’ayons pas l’air d’avoir la fièvre… Veux-tu bien ne pas claquer des dents comme cela ! Je voudrais que tu fusses bien malade une bonne fois, ça te guérirait peut-être… »\par
\bigbreak
\noindent Je ne croyais pas être tant dans mon tort : en effet, c’est ma faute ; mais je ne puis pas m’empêcher de claquer des dents, j’ai les mains qui me brûlent, et des frissons qui me passent dans le dos. J’ai attrapé froid cette nuit sur ces bancs, le crâne contre le pupitre ; cette lecture aussi m’a remué…\par
Oh ! je voudrais dormir ! je vais faire un somme sur la chaise.\par
« Ôte-toi de là, me dit ma mère en retirant la chaise. On ne dort pas à midi. Qu’est-ce que c’est que ces habitudes maintenant ?\par
— Ce ne sont pas des habitudes. Je me sens fatigué, parce que je n’ai pas reposé dans mon lit.\par
— Tu trouveras ton lit ce soir, si toutefois tu ne t’amuses pas à \emph{vagabonder}.\par
— Je n’ai pas vagabondé…\par
— Comment ça s’appelle-t-il, coucher dehors ? Il va donner tort à sa mère à présent ! Allons, prends tes livres. Sais-tu tes leçons pour ce soir ? »\par
\bigbreak
\noindent Oh ! l’île déserte, les bêtes féroces, les pluies éternelles, les tremblements de terre, la peau de bête, le parasol, le pas du sauvage, tous les naufrages, toutes les tempêtes, des cannibales, – mais pas les leçons pour ce soir !\par
Je grelottai tout le jour. Mais je n’étais plus seul ; j’avais pour amis Crusoé et Vendredi. À partir de ce moment, il y eut dans mon imagination un coin bleu, dans la prose de ma vie d’enfant battu la poésie des rêves, et mon cœur mit à la voile pour les pays où l’on souffre, où l’on travaille, mais où l’on est libre.\par
Que de fois j’ai lu et relu ce \emph{Robinson} !\par
Je m’occupai de savoir à qui il appartenait ; il était à un élève de quatrième qui en cachait bien d’autres dans son pupitre ; il avait le \emph{Robinson suisse}, les \emph{Contes} du Chanoine Schmidt, la \emph{Vie de Cartouche}, avec des gravures.\par
Ici se place un acte de ma vie que je pourrais cacher. Mais non ! Je livre aujourd’hui, aujourd’hui seulement, mon secret, comme un mourant fait appeler le procureur général et lui confie l’histoire d’un crime. Il m’est pénible de faire cette confession, mais je le dois à l’honneur de ma famille, au respect de la vérité, à la Banque de France, à moi-même.\par
J’ai été \emph{faussaire} ! La peur du bagne, la crainte de désespérer des parents qui m’adoraient, on le sait, mirent sur mon front de faussaire un masque impénétrable et que nulle main n’a réussi à arracher.\par
Je me dénonce moi-même, et je vais dire dans quelle circonstance je commis ce faux, comment je fus amené à cette honte, et avec quel cynisme j’entrai dans la voie du déshonneur.\par
Des gravures ! la \emph{Vie de Cartouche}, les \emph{Contes} du Chanoine Schmidt, les aventures du \emph{Robinson suisse} !… un de mes camarades – treize ans et les cheveux rouges – était là qui les possédait…\par
Il mit à s’en dessaisir des conditions infâmes ; je les acceptai… Je me rappelle même que je n’hésitai pas.\par
Voici quelles furent les bases de cet odieux marché.\par
On donnait au collège de Saint-Étienne, comme partout, des exemptions. Mon père avait le droit d’en distribuer ailleurs que dans sa classe, parce qu’il faisait tous les quinze jours une surveillance dans quelque étude ; il allait dans chacune à tour de rôle, et il pouvait infliger des punitions ou délivrer des récompenses. Le garçon qui avait les livres à gravures consentit à me les prêter, si je voulais lui procurer des exemptions.\par
Mes cheveux ne se dressèrent pas sur ma tête.\par
« Tu sais faire le paraphe de ton père ? »\par
Mes mains ne me tombèrent pas des bras, ma langue ne se sécha pas dans ma bouche.\par
« Fais-moi une exemption de deux cents vers et je te prête la \emph{Vie de Cartouche}. »\par
Mon cœur battait à se rompre.\par
« Je te la donne ! Je ne te la \emph{prête} pas, je te la \emph{donne}… »\par
Le coup était porté, l’abîme creusé ; je jetai mon honneur par-dessus les moulins, je dis adieu à la vie de société, je me réfugiai dans le faussariat.\par
\bigbreak
\noindent J’ai ainsi fourni d’exemptions pendant un temps que je n’ose mesurer, j’ai bourré de signatures contrefaites ce garçon, qui avait, il est vrai, conçu le premier l’idée de cette criminelle combinaison, mais dont je me fis, tête baissée, l’infernal complice.\par
À ce prix-là, j’eus des livres, – tous ceux qu’il avait lui-même ; – il recevait beaucoup d’argent de sa famille et pouvait même entretenir des grenouilles derrière des dictionnaires. J’aurais pu avoir des grenouilles aussi – il m’en a offert – mais si j’étais capable de déshonorer le nom de mon père pour pouvoir lire, parce que j’avais la passion des voyages et des aventures, et si je n’avais pu résister à cette tentation-là, je m’étais juré de résister aux autres, et je ne touchai jamais la queue d’une grenouille, qu’on me croie sur parole ! Je ne ferai pas des moitiés d’aveux.\par
Et n’est-ce point assez d’avoir trompé la confiance publique, imité une signature honorable et honorée, pendant deux ans ! Cela dura deux ans. Nous nous arrêtâmes, las du crime ou parce que cela ne servait plus à rien ; j’ai oublié, et nul ne sut jamais que nous avions été des faussaires. Je le fus et je ne m’en portai pas plus mal. On pourrait croire que le sentiment du crime enfièvre, que le remords pâlit ; il est des criminels, malheureusement, sur qui rien ne mord et que leur infamie n’empêche pas de jouer à la toupie et de mettre insouciamment des queues de papier au derrière des hannetons.\par
\bigbreak
\noindent Ce fut mon cas : beaucoup de queues de papier, force toupies. C’est peut-être un remède, et je n’ai jamais eu le teint si frais, l’air si ouvert, que pendant cette période du faussariat.\par
Ce n’est qu’aujourd’hui que la honte me prend et que je me confesse en rougissant. On commence par contrefaire des exemptions, on finit par contrefaire des billets. Je n’ai jamais pensé aux billets : c’est peut-être que j’avais autre chose à faire, que je suis paresseux, ou que je n’avais pas d’encre chez moi ; mais si la contrefaçon des exemptions mène au bagne, je devrais y être.\par
\bigbreak
\noindent Et qui dit que je n’irai pas ?
\section[{12. Frottage – Gourmandise – Propreté}]{12. \\
Frottage – Gourmandise – Propreté}\renewcommand{\leftmark}{12. \\
Frottage – Gourmandise – Propreté}

\noindent On me charge des soins du ménage. « Un homme doit savoir tout faire. »\par
Ce n’est pas grand embarras : quelques assiettes à laver, un coup de balai à donner, du plumeau et du torchon ; mais j’ai la main malheureuse, je casse de temps en temps une écuelle, un verre.\par
Ma mère crie que je l’ai fait exprès, et que nous serons bientôt sur la paille, si ce \emph{brise-tout} ne se corrige pas.\par
Une fois, je me suis coupé le doigt – jusqu’à l’os.\par
« Et encore il se coupe ! » fait-elle avec fureur.\par
Le malheur est qu’elle a une méthode… comme Descartes, dont M. Beliben parlait quelquefois : il faudrait que je fisse des bouquets avec des épluchures.\par
\bigbreak
\noindent « Pas pour deux liards d’idée. »\par
Et, prenant l’arrosoir et le balai, elle fait des dessins sur le plancher avec l’eau ou la poussière, en se balançant un peu, minaudière et souriante.\par
Ah ! je n’ai pas cette grâce, certainement !\par
Quelquefois, c’est le coup de la vigueur : elle prend une peau avec du tripoli ou une brosse à gros poils, et elle attaque un luisant de cuivre ou un coin de meuble.\par
Elle fait : « Han ! » comme un mitron ; elle geint à faire pousser des pains sur le parquet ! J’en ai la sueur dans le dos !\par
Mais je suis vigoureux, j’ai du moignon, et je lui prends le torchon des mains pour continuer la lutte. Je me jette sur le meuble ou je me précipite contre la rampe, et je mange le bois, je dévore le vernis.\par
« Jacques, Jacques ! tu es donc fou ! »\par
En effet, l’enthousiasme me monte au cerveau, j’ai la monomanie flottante…\par
« Jacques, veux-tu bien finir ! Il nous démolirait la maison, ce brutal, si on le laissait faire ! »\par
Je suis fort embarrassé : – ou l’on m’accuse de paresse, parce que je n’appuie pas assez, ou l’on m’appelle brutal, parce que j’appuie trop.\par
Je n’ai pas deux liards d’idée. C’est vrai, je le sens. Pas même capable de faire la vaisselle avec grâce ! Que deviendrai-je plus tard ? Je ne mangerai que de la charcuterie, – du lard sur du pain et du jambon dans le papier. J’irai dîner à la campagne pour laisser les restes dans l’herbe.\par
(Serais-je poète ? J’aime à dîner dans la prairie !)\par
C’est que je n’aurai pas à laver d’assiettes, et Dieu ne m’obligera pas à enlever les crottes des petits oiseaux.\par
Le plus terrible, dans cette histoire de vaisselle, c’est qu’on me met un tablier comme à une bonne. Mon père reçoit quelquefois des visites de parents, de mères d’élèves, et l’on m’aperçoit à travers une porte, frottant, essuyant et lavant, dans mon costume de Cendrillon. On me reconnaît et on ne sait à quoi s’en tenir, on ne sait pas si je suis un garçon ou une fille.\par
\bigbreak
\noindent Je maudis l’oignon…\par
Tous les mardis et vendredis, on mange du hachis aux oignons, et pendant sept ans je n’ai pas pu manger de hachis aux oignons sans être malade.\par
J’ai le dégoût de ce légume.\par
Comme un riche ! mon Dieu, oui ! – Espèce de petit orgueilleux, je me permettais de ne pas aimer ceci, cela, de rechigner quand on me donnait quelque chose qui ne me plaisait pas. Je \emph{m’écoutais}, je me sentais surtout, et l’odeur de l’oignon me soulevait le cœur, – ce que j’appelais mon cœur, comprenons-nous bien ; car je ne sais pas si les pauvres ont le droit d’avoir un cœur.\par
« Il faut \emph{se forcer}, criait ma mère. Tu le fais exprès, ajoutait-elle comme toujours. »\par
\bigbreak
\noindent C’était le grand mot. « Tu le fais exprès ! »\par
Elle fut courageuse heureusement : elle tint bon, et au bout de cinq ans, quand j’entrai en troisième, je pouvais manger du hachis aux oignons. Elle m’avait montré par là qu’on vient à bout de tout, que la volonté est la grande maîtresse.\par
Dès que je pus manger du hachis aux oignons sans être malade, elle n’en fit plus – à quoi bon ? c’était aussi cher qu’autre chose et ça empoisonnait. Il suffisait que sa méthode eût triomphé, – et plus tard, dans la vie, quand une difficulté se levait devant moi, elle disait :\par
« Jacques, souviens-toi du hachis aux oignons. Pendant cinq ans tu l’as vomi et au bout de cinq ans tu pouvais le garder. Souviens-toi, Jacques ! »\par
Et je me souvenais trop.\par
\bigbreak
\noindent J’aimais les poireaux.\par
Que voulez-vous ? – Je haïssais l’oignon, j’aimais les poireaux. On me les arrachait de la bouche, comme on arrache un pistolet des mains d’un criminel, comme on enlève la coupe de poison à un malheureux qui veut se suicider.\par
« Pourquoi ne pourrais-je pas en manger ? demandai-je en pleurant.\par
— Parce que tu les aimes », répondait cette femme pleine de bon sens, et qui ne voulait pas que son fils eût de passions.\par
Tu mangeras de l’oignon, parce qu’il te fait mal, tu ne mangeras pas de poireaux, parce que tu les adores.\par
« Aimes-tu les lentilles ?\par
— Je ne sais pas… »\par
Il était dangereux de s’engager, et je ne me prononçais plus qu’après réflexion, en ayant tout balancé.\par
Jacques, tu mens !\par
Tu dis que ta mère t’oblige à ne pas manger ce que tu aimes.\par
Tu aimes le gigot, Jacques.\par
Est-ce que ta mère t’en prive ?\par
Ta mère en fait cuire un le dimanche. – On t’en donne.\par
Elle en reprend du froid le lundi. – T’en refuse-t-on ?\par
On le fait revenir aux oignons le mardi – le jour des oignons, c’est sacré – tu en as deux portions au lieu d’une.\par
Et le mercredi, Jacques ! qui est-ce qui se sacrifie, le mercredi, pour son fils ? Le jeudi, qui est-ce qui laisse tout le gigot à son enfant ? Qui ? parle !\par
C’est ta mère – comme le pélican blanc ! Tu le finis, le gigot – à toi l’honneur !\par
« Décrotte l’os ! ce n’est pas moi qui t’en empêcherai, va ! »\par
Entends-tu, c’est ta mère qui te crie de ne pas avoir de scrupules, d’en prendre à ta faim, elle ne veut pas borner ton appétit… « Tu es libre, il en reste encore, ne te gêne pas ! »\par
Mais Dieu se reposa le septième jour ! voilà huit fois que j’y reviens, j’ai un mouton qui bêle dans l’estomac : grâce, pitié !\par
Non, pas de grâce, pas de pitié ! Tu aimes le gigot, tu en auras.\par
« As-tu dit que tu l’aimais !\par
— Je l’ai dit, lundi…\par
— Et tu te contredis samedi ! mets du vinaigre, – allons, la dernière bouchée ! J’espère que tu t’es régalé ?… »\par
\bigbreak
\noindent C’est que c’est vrai ! On achetait un gigot au commencement du mois, quand mon père touchait ses appointements. Ils y goûtaient deux fois ; je devais finir le reste – en salade, à la sauce, en hachis, en boulettes ; on faisait tout pour masquer cette lugubre monotonie ; mais à la fin, je me sentais devenir brebis, j’avais des bêlements et je pétaradais quand on faisait « prou, prou ».\par
\bigbreak
\noindent Le bain ! – Ma mère en avait fait un supplice.\par
Heureusement elle ne m’emmenait avec elle, pour me récurer à fond, que tous les trois mois.\par
Elle me frottait à outrance, me faisait avaler, par tous les pores, de la soude et du suif, que pleurait un savon de Marseille à deux sous le morceau, qui empestait comme une fabrique de chandelles. Elle m’en fourrait partout, les yeux m’en piquaient pendant une semaine, et ma bouche en bavait…\par
J’ai bien détesté la propreté, grâce à ce savon de Marseille !\par
\bigbreak
\noindent On me nettoyait hebdomadairement à la maison.\par
Tous les dimanches matin, j’avais l’air d’un veau. On m’avait fourbi le samedi ; le dimanche on me passait à la détrempe ; ma mère me jetait des seaux d’eau, en me poursuivant comme Galatée, et je devais comme Galatée – fuir pour être attrapé, mon beau Jacques ! Je me vois encore dans le miroir de l’armoire, pudique dans mon impudeur, courant sur le carreau qu’on lavait du même coup, nu comme un amour, cul-de-lampe léger, ange du décrotté.\par
Il me manquait un citron entre les dents et du persil dans les narines, comme aux têtes de veau. J’avais leur reflet bleuâtre, fade et mollasse ; mais j’étais propre, par exemple !\par
\bigbreak
\noindent Et les oreilles ! ah ! les oreilles ! On tortillait un bout de serviette et on l’y entrait jusqu’au fond, comme on enfonce un foret, comme on plante un tire-bouchon…\par
Le petit tortillon était enfoncé si vigoureusement que j’en avais les amygdales qui se gonflaient ; le tympan en saignait, j’étais sourd pour dix minutes, on aurait pu me mettre une pancarte.\par
La propreté avant tout, mon garçon !\par
Être propre et se tenir droit, tout est là.\par
\bigbreak
\noindent Je suis propre comme une casserole rétamée. Oui, mais je ne me tiens pas droit.\par
C’est-à-dire que pendant que j’apprends mes leçons, je m’endors souvent, et je me cache la tête dans les bras, le dos en rond.\par
Ma mère veut que je me tienne droit.\par
« Personne n’a encore été bossu dans notre famille, ce n’est pas toi qui vas commencer, j’espère ! »\par
Elle dit cela d’un ton de menace, et si j’avais l’intention d’être bossu, elle m’en ôterait du coup l’envie.
\section[{13. L’argent}]{13. \\
L’argent}\renewcommand{\leftmark}{13. \\
L’argent}

\noindent « M’man ! J’ai mal.\par
— Ce sont les vers, mon enfant !\par
— Je sens bien que j’ai mal.\par
— Douillet, va ! Ah ! si tu avais dix mille livres de rentes !… Quand tu as mal au ventre, fais comme faisait mon père, fais la culbute ! »\par
\bigbreak
\noindent L’argent ! – les rentes !\par
On me promet, comme à tous les gamins, des récompenses, un gros sou, si je suis sage, et chaque fois que je suis premier, une petite piécette blanche. On me la donne ?… Non, ma mère m’aime trop pour cela.\par
Elle ne me privait pourtant pas pour s’enrichir.\par
Les dix sous ne rentraient pas dans la famille, – ils allaient se coucher dans une tirelire dont la gueule me riait au nez.\par
« C’est pour toi », disait ma mère en me faisant voir la pièce et avant de la glisser dans le trou !\par
Je ne la revoyais plus !\par
« Ce sera, ajoutait-elle, pour t’acheter un homme ! »\par
C’est le remplaçant caché dans cette tirelire qui absorbe toutes les petites pièces et les gros sous que d’autres, mes copains, dépensent le dimanche et les jours de foire, en entrées aux baraques, cigares à paille, canons en cuivre.\par
Toujours sage, donnant la leçon sans pédantisme, ma mère, qui marchait avec son siècle, m’inspirait ainsi la haine des \emph{armées permanentes} et me faisait réfléchir sur \emph{l’impôt du sang}. Je me regimbais quelquefois et je citais mes camarades qui dépensaient leur argent au lieu de le garder pour acheter un homme.\par
\bigbreak
\noindent « C’est que sans doute ils sont infirmes, vois-tu ! »\par
Elle avait même une parole de tristesse et un accent de compassion à l’égard de ces pauvres enfants qui faisaient bien de se consoler en dépensant leurs sous, eux que le ciel avait tordus ou embossés sans que cela parût.\par
« Et pourquoi ! » disait-elle en se parlant à elle-même et arrivant jusqu’à l’impiété.\par
« C’est un crime de la nature, presque une injustice de Dieu. – Il t’a épargné, toi », reprenait-elle en me tapant sur le dos, pour me montrer qu’il n’y avait pas de gibbosité et qu’elle pouvait, qu’elle devait, – c’était son rôle de mère – continuer à nourrir le remplaçant dans le fond de la tirelire…\par
Et moi, défiant, ingrat, désirant monter sur les chevaux de bois, je regrettais souvent de n’être pas bossu, et je priais Dieu de commettre quelque injustice que je cacherais sous ma chemise, et qui, me sauvant du tirage au sort, me donnerait le droit de prendre ce qu’on avait mis et de ne plus mettre rien dans cette satanée tirelire.\par
\bigbreak
\noindent Les inspecteurs généraux vont arriver dans quelque temps.\par
Mon père éreinte les élèves et convoque les forts pour préparer l’inspection. Il leur distribue les rôles. Il demandera à celui-ci ce passage, à celui-là cet autre.\par
« Tribouillard, vous avez le \emph{que retranché}\footnote{Il s’agit d’une question de grammaire.}. – Caillotin, l’\emph{Histoire sainte}. Piochez \emph{les prophètes}.\par
— M’sieu, dit Caillotin, comment faut-il prononcer \emph{Ezéchiel ? »}\par
\bigbreak
\noindent Ma mère se frappe le front, comme André Chénier.\par
« Jacques, si tu es dans les trois premiers d’ici à ce que l’inspecteur vienne, je te donnerai… Regarde ! Pour toi, pour toi tout seul ; tu en feras ce qu’il te plaira. »\par
Elle m’a montré de \emph{l’or} ; c’est une pièce de vingt sous. Oh ! pourquoi me donner la soif des richesses ? Est-ce bien de la part d’une mère ?\par
Il se livré un combat en moi-même – pas très long.\par
« Pour moi tout seul ? J’achèterai ce qu’il me plaira avec ? Je les donnerai à un pauvre, si je veux ? »\par
Les donner à un pauvre ! – ma mère chancelle ; ma folie l’épouvante et pourtant elle répond à la face du ciel :\par
« Oui, elle sera à toi. J’espère bien que tu ne la donneras pas à un pauvre ! »\par
Mais c’est une révolution, alors ! Jusqu’ici je n’ai rien eu qui fût à moi, pas même ma peau.\par
Je lui fais répéter.\par

\labelblock{Minuit.}

\noindent Il s’agit de bien apprendre mon histoire pour être premier, – et je pioche, je pioche !\par
Le samedi arrive.\par
Le proviseur entre. Les élèves se lèvent ; le professeur lit :\par
« Thème grec.\par
— Premier : Jacques Vingtras. »\par
\bigbreak
\noindent « Eh bien ? dit ma mère en arrivant.\par
— Je suis premier.\par
— Ah ! c’est bien. Tu vois, quand tu travailles, comme tu peux avoir de bonnes places ! Demain je te ferai une bonne pachade. »\par
La pachade est une espèce de pâte pétrie avec des pommes de terre, un mortier jaune, sans beurre, que ma mère m’a présenté comme un plat de luxe. Mais il n’est pas question de pachade ! C’est une pièce de vingt sous que je veux. On n’en parle pas. La question est si grave que je n’ose pas l’attaquer. Ma mère fait l’affairée pour la pachade et me montre un œuf tout crotté en me disant : « J’espère qu’il est gros ! »\par
Des farces, tout cela. Et mes vingt sous, les ai-je gagnés, oui ou non ? Est-ce qu’on me les a promis ? Il faut peut-être que je les lui demande. Pourquoi donc ? Est-ce qu’elle a oublié ?\par
Je vois bien à un peu de gêne, à cette coquetterie de l’œuf, à la contrainte du sourire, je vois bien qu’elle se souvient. Elle tient peut-être à garder son rang. C’est le fils qui doit rappeler à la mère ce qu’elle a promis.\par
« Maman, et mes vingt sous ? »\par
Elle ne me répond pas de suite ; mais, venant à moi tout d’un coup, d’une voix qui n’est plus celle qu’elle avait, espiègle et charmante, en montrant le gros œuf crotté :\par
« Jacques, veux-tu faire crédit à ta mère ?… »\par
Il y a dans l’accent toute la dignité d’une vaincue qui accepte son sort d’avance, mais demande une grâce au vainqueur. Elle ne défend pas sa bourse, la voilà ! – Les vingt sous sont sur la table – mais elle prie qu’on lui laisse du temps.\par
Oui, ma mère, je vous fais crédit. Oh ! gardez, gardez ces vingt sous, soit qu’ils doivent servir à réparer une brèche, soit que vous vouliez les engager pour moi dans une entreprise, – et sans me rien dire, en ayant l’air plutôt de mendier un pardon, vous joignez mon capital au vôtre, vous m’intéressez dans les affaires, vous me faites l’associé de la maison ! Merci !\par
Et elle s’entend en affaires, ma mère ; elle sait comment on fait rapporter à l’argent ; car elle m’a raconté, bien souvent, qu’à quatre ans, elle pouvait déjà gagner sa vie.\par
Elle a commencé par acheter un pigeon avec sept sous qu’on lui avait donnés, parce qu’elle avait gardé les oies. Elle a engraissé le pigeon et l’a revendu pour acheter un agneau qui sortait du ventre de la mère.\par
Elle a revendu cet agneau et s’est procuré un veau, toujours du même âge.\par
Dès qu’il y avait dans une écurie, une étable, un chenil, quelque bête en travail, on voyait accourir ma mère qui attendait, curieuse des phénomènes de la nature, avec son argent tout prêt à déposer écus sur bonde, monnaie sous ventre.\par
Je n’ai pas sa force, moi ! J’aurais trois sous, je les entamerais et je ne penserais pas à acheter un lapereau à la mamelle pour gagner avec l’argent un veau au débarqué.\par
\bigbreak
\noindent Je crus bien une fois que j’allais avoir quarante sous à refuser au remplaçant et à donner aux chevaux de bois. Il s’agissait encore d’être \emph{premier} deux ou trois fois avant le bal du proviseur.\par
Je décrochai de nouveau la timbale.\par
J’avais bien fait mes conditions, cette fois. J’avais bien demandé : « \emph{Elle sera pour moi ? Je la garderai}. » J’avais indiqué que je ne voulais pas joindre cette somme à celle que j’avais déjà dans les affaires. On met cinq francs dans une entreprise, on n’en met pas sept.\par
« \emph{Je la garderai} ?\par
— \emph{Tu la garderas.} »\par
Ma mère ne manqua pas à sa promesse. On me remit les quarante sous ; je les serrai dans mon gousset ; mais quand je parlai d’aller sur les chevaux de bois, ma mère me rappela le contrat :\par
« Tu m’as dit que tu les \emph{garderais} ! »\par
Et elle ajouta que, si je m’avisais de changer la pièce, j’aurais affaire à elle. Comme je protestais :\par
« Tu es devenu menteur maintenant ; il ne te manquait plus que ça, mon garçon ! »\par
Je ne pouvais pas le nier ; j’étais écrasé par moi-même. Je m’étais suicidé avec ma propre langue.\par
J’en fus réduit à traîner ces quarante sous comme une plaque d’aveugle.\par
Tous les soirs, ma mère demandait à les voir.\par
Un jour je ne pus les lui montrer !…\par
J’étais allé sur la place Marengo, dans un bazar à treize, \emph{tout à} treize !\par
J’achetai une paire de bretelles à pattes. Elles étaient rose tendre !\par
À peine eus-je commis cette faute que j’en compris l’étendue. La pièce était entamée : j’avais treize sous de bretelles. Il ne restait que vingt-sept sous ! Qu’allait dire ma mère ? – Perdu pour perdu, je me dis qu’il fallait aller jusqu’au bout.\par
Jouir… – après moi, le déluge !\par
Je commençai par m’enfoncer dans une allée où je me déshabillai pour mettre mes bretelles. Après quelques tentatives inutiles, toujours dérangé et regardé de travers par des gens étonnés de me voir demi-nu sur le pas de leurs portes, je crus plus prudent, quoiqu’un peu moins noble, d’entrer dans un lieu retiré, le premier que je trouverais.\par
Il me restait vingt-sept sous, en sous, – jamais je n’avais eu une si grosse somme à ma disposition. Elle gonflait et crevait mes poches. – Patatras ! les sous roulent à terre, – même ailleurs !\par
C’est horrible.\par
Je n’ai retrouvé qu’un franc deux sous. Je perds la tête…\par
Je m’approche d’un des jeux qui sont installés place Marengo :\par
« Trois balles pour un sou ! On gagne un lapin. »\par
Je prends la carabine, j’épaule et je tire… Je tire les yeux fermés, comme un banquier se brûle la cervelle.\par
« Il a gagné le lapin ! »\par
\bigbreak
\noindent C’est un bruit qui monte, la foule me regarde, on me prend pour un Suisse ; quelqu’un dit que, dans ce pays-là, les enfants apprennent à tirer à trois ans et qu’à dix ans il y en a qui cassent des noisettes à vingt pas.\par
« Il faut lui donner le lapin ! »\par
Le marchand n’avait pas l’air de se presser en effet, mais la foule approche, avance et va faire une gibelotte avec l’homme s’il ne donne pas le lapin qui est là et qui broute.\par
Je l’ai, je l’ai ! Je le tiens par les oreilles et je l’emporte.\par
Il faut voir le monde qu’il y a ! Le lapin fait des sauts terribles. Il va m’échapper tout à l’heure.\par
Comme dans toutes les luttes, chaque côté a ses partisans. Les uns tiennent pour le lapin, les autres pour le Suisse – c’est moi, le Suisse – et je sens toute la responsabilité qui pèse sur ma tête. Quelquefois l’animal fait un bond qui épouvante les miens. Je voudrais changer de main, le prendre par la queue de temps en temps. Je n’ose pas devant cette foule.\par
Je n’ai pas le courage de tourner la tête, mais je devine que les rangs se sont grossis.\par
On marque le pas.\par
Je suis en avant, à quelques pas de la colonne, seul comme un prophète ou un chef de bande…\par
On se demande sur la route ce que nous voulons, si c’est une idée religieuse ou une pensée sociale qui me pousse.\par
Si elle est pratique, on verra ; – mais que je laisse là le lapin ! – Est-ce un drapeau ? – Il faut le dire alors.\par
Mes doigts sont crispés, les oreilles vont me rester dans la main. Le lapin fait un suprême effort…\par
Il m’échappe ! Mais il tombe en aveugle dans ma culotte – une culotte de mon père, mal retapée, large du fond, étroite des jambes. – Il y reste.\par
On s’inquiète, on demande…\par
Les foules n’aiment pas qu’on se joue d’elles. On n’escamote pas ainsi son drapeau !\par
« \emph{Le La-pin ! Le La-pin ! »} sur l’air des \emph{Lampions}.\par
Des gens se mettent aux fenêtres ; les curieux arrivent.\par
Le lapin est toujours entre chair et étoffe, je le sens.\par
Oh ! si je pouvais fuir ! Je vais essayer. Un passage est là – je l’enfile…\par
On me cherche, mais je connais les coins.\par
Où aller ? – Je tombe sur M. Laurier, l’économe. Je lui ai fait des commissions, j’ai porté des lettres à une dame. J’ai son secret, je suis prêt au chantage. – Il faut qu’il me sauve ! Je lui dis tout.\par
« Tiens, voilà tes quarante sous. Je vais te reconduire et dire que c’est moi qui t’ai gardé, et lâche-moi cette bête ! »\par
\bigbreak
\noindent Ma mère croit à notre mensonge.\par
« Bien, bien, M. Laurier, – du moment qu’il était avec vous… Savez-vous ce qu’il y a dans les rues, ce soir ? On dit que les mineurs ont voulu se révolter et ont mis le feu à un couvent. »\par
\bigbreak
\noindent Le lendemain.\par
« Mange donc, Jacques, mange ! Tu n’aimes donc plus le lapin maintenant ? »\par
Elle a acheté un lapin, ce matin, à bas prix, parce qu’il est un peu écrasé, et qu’on lui a trouvé des bouts de chemise dans les dents.\par
Où est la peau ?…\par
Je vais à la cuisine.\par
C’est \emph{lui !…}
\section[{14. Voyage au pays}]{14. \\
Voyage au pays}\renewcommand{\leftmark}{14. \\
Voyage au pays}

\noindent Jacques ira passer ses vacances au pays.\par
C’est ma mère qui m’annonce cette nouvelle.\par
« Tu vois, on te pardonne tes farces de cette année, nous t’envoyons chez ton oncle ; tu monteras à cheval, tu pêcheras des truites, tu mangeras du saucisson de campagne. Voilà trois francs pour tes frais de voyage. »\par
La vérité est que mon oncle le curé, qui \emph{va sur soixante-dix}, a parlé de me faire son héritier, et il demande à m’avoir près de lui pendant les vacances.\par
Le vieux prêtre, qui économise, a pour notaire un bonhomme qui en a touché deux mots à mon père dans une lettre qu’on a oubliée sur la table et que j’ai lue. Je suis au courant. On me laisserait une somme de… payable à ma majorité : c’est l’idée du testament.\par
\bigbreak
\noindent J’ai mon paletot sur le bras, une casquette sans visière et une gourde.\par
« Il a l’air d’un Anglais. »\par
Ce mot me remplit d’orgueil.\par
Mon père (il me gâte !) m’emmène au café pour lamper le coup de l’étrier.\par
« Allons, bois cela, ça te fera du bien. »\par
J’avale l’eau-de-vie tout d’un trait, ce qui me fait éternuer pendant cinq minutes et me mouille les yeux, comme si j’avais pleuré toute la nuit. La langue me cuit à vouloir la tremper dans le ruisseau.\par
« Sois aimable avec ton oncle. »\par
C’est la dernière recommandation de mon père.\par
« Aie bien soin de ta veste neuve. »\par
C’est le cri suprême de ma mère.\par
\bigbreak
\noindent En route, fouette, cocher !\par
\bigbreak
\noindent Les adieux ont été simples. Il faut que j’arrive au plus vite chez le grand-oncle.\par
On n’a pas fait de sentiment.\par
Et je n’attendais, moi, que le moment où les chevaux fileraient…\par
\bigbreak
\noindent J’ai passé ma nuit à savourer ma joie. J’ai bu, dormi, rêvé, j’ai pris des sirops au buffet, j’ai soulevé les vasistas, je suis descendu\emph{ aux côtes}.\par
À six heures du matin, je me suis trouvé en plein Puy, devant le café des Messageries.\par
Je laisse mon bagage au bureau, et je grimpe vers notre ancienne maison, où mademoiselle Balandreau doit m’attendre. On lui a écrit que j’arriverais, sans fixer le jour.\par
Je frappe.\par
Ah ! ce n’est pas long ! La bonne vieille fille m’arrive ébouriffée et émue ! et m’embrasse, m’embrasse – comme jamais ne m’a embrassé ma mère.\par
Elle s’occupe de me débarrasser, et elle a peur que je sois las, et que j’aie eu froid…\par
« Tu dois être fatigué. Ôte-moi ce paletot-là. Ce n’est pas possible, ce n’est pas toi ! – Comme tu es grand ! – Toute la nuit en voiture, pauvre petit, – tu dois avoir sommeil. As-tu dormi ?\par
— Pas fermé l’œil. »\par
Je mens comme un arracheur de dents, mais cela la flattera que son favori n’ait pas fermé l’œil et paraisse si frais, si fort. – C’est un grand garçon qui peut passer les nuits.\par
« Veux-tu te coucher ? – Tiens, couche-toi. – Tu ne veux pas ? – Tu vas prendre une tasse de café au moins ? – Tu sais, comme je t’en donnais en cachette de ta mère, avec du lait. – Tu l’écrémais toujours, – tu disais : “donne-moi \emph{la peau}”. »\par
Comme elle m’aime !\par
Nous faisons le café ensemble. Elle a l’air d’une sorcière, et moi d’un diablotin ; elle, avec ses \emph{coques} en l’air, tournant le moulin ; moi, dans les cendres, soufflant le feu…\par
Comme toutes les vieilles filles – qui ont une gourmandise – elle aime son café au lait à l’adoration, – et il est bon, ma foi ! J’en ai les lèvres toutes grasses et les joues toutes chaudes. C’est le même bol que celui où je trempais autrefois mon museau, en buvant des gorgées doubles parce que ma mère pouvait arriver et que ma mère ne voulait pas qu’on me gâtât en dehors d’elle ; – puis le café au lait, c’est mauvais pour les enfants, « ça donne des glaires ».\par
\bigbreak
\noindent « Mais venez donc le voir ! »\par
Elle est allée chercher les voisins, elle a ramené les commères. Il y a une petite demoiselle dans un coin.\par
« Tu ne reconnais pas mademoiselle Perrinet ? »\par
Quoi, cette petite fille qui avait toujours un pantalon de velours, ses cheveux défaits, avec qui je me battais, qui m’égratignait – j’en ai encore la marque, – elle était méchante comme la gale ; c’est elle qui est là avec une belle natte retenue par un peigne d’écaille, un nœud bleu au corsage, une petite fraise de tulle qui entoure son cou doré, une fumée brune sur les joues et la lèvre ?\par
« Embrassez-vous donc ! »\par
Je n’ose pas, elle attend. On me pousse, elle avance. Pas trop !\par
Je suis rouge, elle l’est bien un peu aussi ! Nous avions joué au petit mari et à la petite femme, dans le temps ; nous avions fait la dînette ensemble, et la grande égratignure, celle qui me reste comme un bout de fil blanc, avait été donnée, je crois, à la suite d’une scène de jalousie.\par
Je m’en souviens, elle ne l’a peut-être pas oublié.\par
« Ma malle est aux messageries. »\par
Je dis cela avec un revenez-y de vanité, il est entendu que j’irai avec un petit voisin la chercher.\par
« C’est bien lourd pour toi », dit mademoiselle Balandreau.\par
Il y a mon trousseau, quelques chemises, ma veste neuve, un paquet pour la tante Rosalie, un paquet pour le vieil oncle et une pierre pour un monsieur.\par
Ce monsieur est un personnage qui fait une collection de cailloux et a cherché partout un \emph{rognon}.\par
J’ai entendu parler de ce rognon pendant six mois, toujours avec le même étonnement ; à la fin on a trouvé une chose couleur de fer, que mon père a empaquetée avec soin et que je dois porter au collectionneur ; il est parent de je ne sais plus qui dans la haute Université, et la fortune professionnelle de M. Vingtras peut s’accrocher à ce rognon.\par
Ce mot de rognon me gêne tout de même, et quand une dame, qui se trouve là au moment où je déboucle ma malle, demande ce que c’est que ce caillou bleu, je ne lui dis pas comment on l’appelle.\par
J’emporte vite cette pierre chez le destinataire qui la tourne, retourne et la regarde comme on mire un œuf. Il me reconduit et me met cinq francs dans la main en arrivant à la porte.\par
« C’est pour toi, fait-il.\par
— Pas pour mes parents ? ai-je dit tout bouleversé.\par
— Pour toi, pour t’amuser en vacances. »\par
\bigbreak
\noindent Je viens de faire le tour de la ville, j’ai longé la rivière, j’ai cherché des endroits déserts, j’avais besoin d’être seul.\par
À la tête d’une fortune ! – Si jeune, à mon âge, sans que j’aie besoin d’en rendre compte à mes parents, avec le droit d’en disposer comme je l’entendrai, de faire des folies ou d’économiser, de mettre cet argent dans un pot ou de le jeter par les fenêtres !\par
Il y a peut-être un crime là-dessous.\par
Non, M. Buzon, le destinataire, est un honnête homme, il a une bonne figure, – même l’air un peu bête ; – j’ai entendu dire que les criminels n’ont jamais l’air bête. M. Buzon a une situation à l’abri du soupçon.\par
\bigbreak
\noindent Cependant ! – Je ne sais pas, moi, si je dois garder l’argent de ce monsieur !… Oh ! j’ai eu tort. Je suis un petit mendiant.\par
\bigbreak
\noindent « Dis, mademoiselle Balandreau, tu le lui rapporteras, je t’en prie ! tu diras que je l’ai pris sans savoir… »\par
Et je n’ai pas de cesse que je ne l’aie entraînée par sa robe jusque devant la porte du monsieur « au rognon ».\par
Je suis caché dans un coin et je regarde si elle entre.\par
Quand elle sort, elle me dit : « C’est fait », et elle m’embrasse en se frottant le nez plusieurs fois.\par
« Mais tu pleures !\par
— Cher petit ! fait-elle en ne cachant plus ses larmes et en s’essuyant les yeux. Le brave homme, il ne voulait pas reprendre la pièce. Je lui ai dit qu’il le fallait. Je pleure. Est-ce que je pleure ?… C’est de voir que tu as fait cela, toi, tout petit ! Déjà si fier… »\par
Elle s’éponge le nez et les cils.\par
Moi, j’ai envie de jeter des pierres dans les carreaux en m’en allant ; un peu plus, je lui en casserais pour ses cinq francs.\par
\bigbreak
\noindent À cheval !\par
Mon oncle m’attend demain. Quelques-uns de ses paroissiens venus pour la foire doivent repartir en bande ; ils m’emmèneront. L’un d’eux a justement acheté un cheval. Je le monterai et nous irons en caravane à Chaudeyrolles.\par
Le rendez-vous est chez Marcelin.\par
Marcelin tient une auberge dans une rue du faubourg. Il a la réputation à dix lieues à la ronde pour le vin blanc et les grillades de cochon.\par
Il y a, quand on entre, une odeur chaude de fumier et de bêtes en sueur qui avance, comme une buée, de l’écurie. Dans la salle où l’on boit, on sent le piquant du vinaigre cuit, versé sur la grillade, et qui mord les feuilles de persil.\par
Il y a aussi les émanations fortes du fromage bleu.\par
C’est vigoureux à respirer, et c’est plein de montant, plein de bruit, plein de vie.\par
On dit des bêtises en patois, et l’on se verse le vin à rasades.\par
Je joue avec une paire de vieux éperons qui rôdent sur la table, et je soupèse de gros bâtons cravatés de cuir : quelques-uns ont une histoire qu’on raconte. – Il y a après le bout de la peau d’huissier.\par
\emph{Anyn !…} Il faut partir.\par
Le bruit que font les étriers en se cognant au moment où l’on apporte les selles, le clic-clac des cuirs, le rongement du mors, j’ai encore cela dans l’oreille, avec le nom de Baptiste, le garçon d’écurie.\par
Je suis trop petit : on me plante et on raccourcit les courroies.\par
Encore, encore ! J’ai les jambes si courtes. M’y voilà ! On me met rênes en mains.\par
« Tu feras comme ceci, comme cela. As-tu monté quelquefois ?\par
— Non.\par
— Ça ne fait rien. \emph{As pas peur ! »}\par
Tout le monde est à cheval. Nous sommes cinq en me comptant. On s’occupe à peine de moi. On me trouve assez grand, on me trouve assez au courant pour me laisser seul. J’en suis si fier !\par

\labelblock{CHAUDEYROLLES}

\noindent Je suis arrivé bien moulu et bien écorché, mais j’ai fait celui qui n’est pas fatigué.\par
\bigbreak
\noindent Les premiers moments ont été tristes.\par
\bigbreak
\noindent Le cimetière est près de l’église, et il n’y a pas d’enfants pour jouer avec moi ; il souffle un vent dur qui rase la terre avec colère, parce qu’il ne trouve pas à se loger dans le feuillage des grands arbres. Je ne vois que des sapins maigres, longs comme des mâts, et la montagne apparaît là-bas, nue et pelée comme le dos décharné d’un éléphant.\par
C’est vide, vide, avec seulement des bœufs couchés, ou des chevaux plantés debout dans les prairies !\par
Il y a des chemins aux pierres grises comme des coquilles de pèlerins, et des rivières qui ont les bords rougeâtres, comme s’il y avait eu du sang ; l’herbe est sombre.\par
Mais, peu à peu, cet air cru des montagnes fouette mon sang et me fait passer des frissons sur la peau.\par
J’ouvre la bouche toute grande pour le boire, j’écarte ma chemise pour qu’il me batte la poitrine.\par
Est-ce drôle ? Je me sens, quand il m’a baigné, le regard si pur et la tête si claire !…\par
C’est que je sors du pays du charbon avec ses usines aux pieds sales, ses fourneaux au dos triste, les rouleaux de fumée, la crasse des mines, un horizon à couper au couteau, à nettoyer à coups de balai…\par
Ici le ciel est clair, et s’il monte un peu de fumée, c’est une gaieté dans l’espace, – elle monte, comme un encens, du feu de bois mort allumé là-bas par un berger, ou du feu de sarment frais sur lequel un petit vacher souffle dans cette hutte, près de ce bouquet de sapins…\par
Il y a le vivier, où toute l’eau de la montagne court en moussant, et si froide qu’elle brûle les doigts. Quelques poissons s’y jouent. On a fait un petit grillage pour empêcher qu’ils ne passent. Et je dépense des quarts d’heure à voir bouillonner cette eau, à l’écouter venir, à la regarder s’en aller, en s’écartant comme une jupe blanche sur les pierres !\par
La rivière est pleine de truites. J’y suis entré une fois jusqu’aux cuisses ; j’ai cru que j’avais les jambes coupées avec une scie de glace. C’est ma joie maintenant d’éprouver ce premier frisson. Puis j’enfonce mes mains dans tous les trous, et je les fouille. Les truites glissent entre mes doigts ; mais le père Regis est là, qui sait les prendre et les jette sur l’herbe, où elles ont l’air de lames d’argent avec des piqûres d’or et de petites taches de sang.\par
\bigbreak
\noindent Mon oncle a une vache dans son écurie ; c’est moi qui coupe son herbe à coups de faux. Comme elle siffle dans le gras du pré, cette faux, quand j’en ai aiguisé le fil contre la pierre bleue trempée dans l’eau fraîche !\par
Quelquefois je sabre un nid ou un nœud de couleuvres.\par
Je porte moi-même le fourrage à la bête, et elle me salue de la tête quand elle entend mon pas. C’est moi qui vais la conduire dans le pâturage et qui la ramène le soir. Les bonnes gens du pays me parlent comme à un personnage, et les petits bergers m’aiment comme un camarade.\par
Je suis heureux !\par
Si je restais, si je me faisais paysan ?\par
J’en parle à mon oncle, un soir qu’il avait fait servir le dîner sous le manteau de la cheminée, et qu’il avait bu de son vin pelure d’oignon.\par
« Plus tard, quand je serai mort. Tu pourras acheter un domaine, mais tu ne voudrais pas être valet de ferme ? »\par
Je n’en sais trop rien.\par
\bigbreak
\noindent Quand il pleut et qu’il n’y a pas moyen de pêcher ni d’aller chercher des groseilles sauvages là-bas, au pied de la montagne, entre les pierres galeuses, – ou bien quand le soleil brûle comme une plaque de tôle bleuie au feu et grille le pays sans ombre, – ces jours-là, je m’enferme dans la bibliothèque de mon oncle et je lis, je lis. Il y a la biographie des hommes illustres de l’abbé de Feller. Je cours aux passages qui parlent de Napoléon, et je fais tout éveillé des rêves pleins de Sainte-Hélène. Je regarde par la fenêtre la campagne déserte, l’horizon vide, et je cherche Hudson Lowe. Si je le tenais !\par
\bigbreak
\noindent Mon oncle attend les curés du voisinage pour la \emph{conférence}.\par
Ils viennent. Je les entends à table qui disent du mal du vicaire de Saint-Parlier, du curé de Solignac ; ils ne paraissent pas plus penser au bon Dieu qu’à l’an quarante !\par
Mon oncle se mêle peu aux conversations. Son âge l’en dispense ; il se fait même plus vieux qu’il n’est, contrefait le sourd et presque l’aveugle ; mais le vin a délié la langue des autres. Un gros, qui a l’air ivrogne, fait sauter les boutons de sa robe crasseuse tachée de vin et dérange son rabat jaune de café. Un maigre, à tête de serpent, ne boit que de l’eau ; mais il jette de côté et d’autre des regards qui me font peur. J’ai vu au théâtre de Saint-Étienne, une fois, le traître qui servait du poison dans les verres ; il a cet air-là.\par
Les autres mangent, boivent comme des goinfres, et quand ils ont une prière à dire, ils ont encore la bouche pleine.\par
On voit leur culotte sous leur robe sale.\par
Le crasseux, le gros, se tourne de mon côté.\par
« C’est votre neveu, monsieur le curé ? Il a bon appétit au moins, ce gaillard-là ; est-il râblé ! »\par
Et il me passe la main sur le dos, ce qui me dégoûte et me gêne.\par
\bigbreak
\noindent « Et Maclou, le protestant, qu’est-ce que vous en faites ? dit une voix.\par
— Il est maintenant au lac de Saint-Front.\par
— Avec le tas ! C’est là qu’ils ont fait leur nid.\par
— Nid de vipères », siffle la \emph{tête} de serpent.\par
Il y a donc des protestants ! J’ai lu ce qu’on en dit dans la bibliothèque de Chaudeyrolles, et les protestants qu’on a brûlés, qu’on envoie en enfer, me semblent une race de damnés.\par
Je vais un jour jusqu’au lac Saint-Front, tout seul. C’est un grand voyage. Je pense tout le long du chemin à la Saint-Barthélemy, et je vois des croix rouges sur le ciel bleu.\par
Voici le lac avec une ou deux barques dans les roseaux, des cabanes perdues dans des champs tout autour.\par
On m’a dit d’aller vers la hutte à gauche, chez Jean Robanès ; je n’ai qu’à dire que je suis le neveu du curé, on m’offrira du lait et on me montrera les protestants.\par
On m’accueille bien ; « et quant aux protestants, me dit l’homme, il y en a un qui est justement là-bas, debout dans le sillon. »\par
Il a l’air dur et triste, – maigre, jaune, le menton pointu, – et raide comme une épée.\par
Est-ce que les gendarmes ne le surveillent pas ? Lui parle-t-on ? A-t-il un boulet ? Je me rappelle bien que l’on punit tous les impies dans la Bible, et les livres de la bibliothèque les appellent des scélérats ! J’en touche un mot à mon oncle, le soir ; il me répond mal, et je commence à croire qu’il en est des protestants infâmes comme des bêtes qui parlent dans La Fontaine. Des farces tout ça !\par
\bigbreak
\noindent Il faut partir.\par
Mon oncle a une tournée à faire, et je dois d’ailleurs bientôt rentrer à Saint-Étienne pour le collège.\par
Nous partons par le chemin que j’ai pris pour venir, mais j’ai cette fois un cheval doux, on m’a caleçonné, ouaté, et je me suis suifé d’avance. D’ailleurs, j’ai monté à cheval depuis un mois, je suis aguerri, et je trouve une joie bien vive à me retourner sur la selle pour dire adieu au paysage. Je donne un coup de talon pour avoir un temps de galop, je flatte la bête comme un vieil ami…\par
Mon oncle me quitte à la Croix de la Mission. Il me parle avec bonté.\par
« Travaille bien, dit-il.\par
— Vous écrirez à papa de me faire revenir l’année prochaine.\par
— Ton père ! ce n’est pas ton père qui t’empêchera, mais peut-être ta mère ; je ne suis pas bien avec ta mère, vois-tu ! »\par
Je le sais. Dans les premiers jours de mon arrivée, j’ai entendu la servante parler dans la chambre.\par
« C’est le fils de madame Vingtras ?\par
— Oui.\par
— Celle qui disait tant de mal de vous ?\par
— C’est fini maintenant, je lui ai pardonné, – et j’aime cet enfant. »\par
\bigbreak
\noindent Il n’était pas beau, mon oncle, il avait les yeux petits, le nez gros, des poils un peu partout, mais il était bon.\par
Je savais qu’il sentait que j’étais malheureux chez nous et qu’en le quittant je perdais de la liberté et du bonheur. Il était aussi triste que moi.\par
\bigbreak
\noindent « Adieu, me dit-il en m’embrassant et en me donnant une poignée de main qui me fit encore plus de plaisir que son embrassade. Tu trouveras quelque chose au fond de ta valise, n’en dis rien à ta mère. »\par
Il me tendit encore ses vieux doigts gris, fit un mouvement de tête et partit.\par
Oh ! s’il eût été mon père, cet oncle au bon cœur !\par
Mais les prêtres ne peuvent être les pères de personne, il paraît : pourquoi donc ?\par
J’avais envoyé une lettre à mademoiselle Balandreau lui annonçant mon arrivée, une lettre qu’elle a montrée à tout le monde.\par
« Comme il écrit bien ! voyez ces majuscules ! »\par
Elle m’a préparé un lit dans un petit cabinet qui est à côté de sa chambre. C’est grand comme une carafe, mais j’ai le droit de fermer ma porte, de jeter ma casquette sur mon lit et de planter mon paletot en disant ouf ! Je fais des gestes de célibataire, je range des papiers, je fredonne…\par
\bigbreak
\noindent Qu’y a-t-il dans ma valise, dont m’a parlé mon oncle ?\par
{\itshape Dix francs !}\par
Je puis les accepter de lui…\par
Me voilà riche tout d’un coup.\par
\bigbreak
\noindent Le temps est superbe, et je descends dès neuf heures en ville, libre, et craquant du bonheur d’être libre ; je me sens gai, je me sens fort, je marche en battant la terre de mes talons et en avalant des yeux tout ce qui passe la nue dans le ciel, le soldat dans la rue ; je rôde à travers le marché, je longe la mairie, je vais au Breuil flâner, les mains derrière le dos, en chassant quelque caillou du bout de mon soulier, comme le receveur particulier qui marche devant moi et que j’imite un peu.\par
Il n’y a pas de devoirs, pas de pensums, ni père ni mère, personne, rien !\par
Il y a le tambour de ville qui s’arrête au coin du carrefour et amasse les gens ; il y a les officiers à épaulettes d’or que je frôle ; j’ai le droit d’aller à tous les rassemblements.\par
Je me fais cirer mes souliers tous les matins par Moustache. Ah ! mais !\par
Il m’a fallu seulement un mois de vacances avec la vache à conduire, les courses dans les champs, les promenades seul, pour m’ouvrir les idées et le cœur !\par
\bigbreak
\noindent Nous allons le soir au café ; on est trois ou quatre anciens camarades ; on joue sa demi-tasse, son petit verre et l’on fait brûler son eau-de-vie ! Cette fumée, cette odeur d’alcool, le bruit des billes, le saut des bouchons, les gros rires, tout cela double mes sens et il me semble qu’il m’est poussé des moustaches et que je soulèverais le billard !\par
On va en sortant au Fer-à-Cheval faire un tour – comme des rentiers ! – On s’arrête en rond aux moments intéressants, je marche quelquefois à reculons devant la bande.\par
Puis l’âge reprend le dessus.\par
« C’est toi qui l’es ! Sauterais-tu ce banc à pieds joints ? Lèverais-tu cette pierre à bras tendu ?\par
— Je parie que je renverse Michelon. »\par
Je ne sais si je suis le plus fort, mais on le croit, tant j’y mets de volonté ! J’aurais préféré vomir le sang par la bouche que lâcher la pierre ou demander grâce à Michelon.\par
Je suis \emph{mon maître} ; je fais ce que je veux et même je suis un peu le chef, celui qu’on écoute et qui a dit l’autre jour, quand un voyou nous a jeté une pierre : « Ne bougez pas, vous autres ! » – J’ai attrapé le voyou et je l’ai ramené en le tenant par la ceinture, et en le calottant jusque devant la bande. – « Demande pardon ! » Il était plus grand que moi.\par
\bigbreak
\noindent Nous avons fait une partie de bateau : personne ne sait ramer, et nous avons failli nous noyer dix fois. Ah ! nous nous sommes bien amusés !\par
\bigbreak
\noindent On m’avait voulu nommer capitaine.\par
« Des blagues ! nommez Michelon ; moi, je me couche. »\par
Et je me suis étendu dans le bateau, regardant le soleil qui me faisait cligner les yeux, et trempant mes mains dans l’eau bleue…\par
\bigbreak
\noindent Un oncle de je ne sais quelle branche court après moi dans le Martouret et ne prend que le temps d’aller avertir mademoiselle Balandreau qu’il m’emmène dans sa carriole voir sa famille ; il me renverra après-demain.\par
« Filons, mon neveu. Hue ! la Grise. »\par
C’est moi qui tiens les rênes en passant dans le faubourg. J’envoie de temps en temps un coup de fouet inutile et j’ai l’air de jurer en frappant avec le manche : « Ah ! \emph{carcan ! »}\par
Nous nous arrêtons au Cheval-Blanc pour le picotin à la Grise. Je saute de la carriole comme un clown et je donne un clic-clac en l’air comme un maquignon.\par
L’oncle de je ne sais quelle branche est fier comme tout.\par
« C’est mon neveu ! » dit-il à tout le monde dans l’hôtel.\par
Nous dînons les coudes sur la table, il me raconte (tout en mangeant des œufs au vin, puis des œufs au lard, pour finir par une salade aux œufs durs), il me raconte l’histoire de sa branche. Il a épousé ci, ça, il est issu de germain, etc.\par
« Tu verras tes cousines, elles sont jolies. »\par
\bigbreak
\noindent Oui, elles le sont, et comme elles ont l’air déluré, mâtin !\par
C’est moi qui suis \emph{la fille}, je redeviens gauche, je me sens bête. Elles parlent très bien français pour des paysannes. Elles ont été à l’école au bourg voisin.\par
« Un verre de vin ! me disent-elles.\par
— Oui, un verre de vin. »\par
Je n’en accepte que pour trinquer dans les cabarets ou dans les auberges, parce que c’est gai les verres qui se choquent, comme je ne prends de cognac que pour faire des brûlots : c’est joli les flammes bleues. Mais, ma foi, je me trouve dépassé tout d’un coup par ces cousines à l’air hardi, à la voix tintante, et je vais boire – boire du bleu et du courage.\par
« À votre santé ! » font-elles après avoir versé une goutte, une toute petite goutte au fond de leurs verres.\par
Elles ont rempli le mien jusqu’au bord.\par
Je crois que je suis un peu gris. – Gare à vous ! cousines.\par
C’est qu’en effet j’ai un toupet du diable, une audace d’enfer !\par
\bigbreak
\noindent Elles ont voulu me faire voir le verger. Va pour le verger ! et j’y entre en sautant par-dessus la barrière à pieds joints.\par
Voilà comme je suis, moi !\par
Mes cousines me regardent ébahies, je ris en revenant à elles pour leur tendre la main et les aider à enjamber. Une, deux, voyons !\par
Elles poussent de petits cris et me retombent dans les bras en mettant pied à terre ; elles s’appuient et s’accrochent, et nous allons dégringoler ! Nous dégringolons, ma foi, on perd tous l’équilibre, et nous tombons sur le gazon. Elles ont des jarretières bleues.\par
Comme il fait beau ! un soleil d’or ! de larges gouttes de sueur me tombent des tempes, et elles ont aussi des perles qui roulent sur leurs joues roses. Le bourdonnement des abeilles qui ronflent autour des ruches, derrière ces groseilliers, met une musique monotone dans l’air…\par
\bigbreak
\noindent « Qu’est-ce que vous faites donc là-bas ? » crie une voix du seuil de la maison.\par
Ce que nous faisons ?… Nous sommes heureux, heureux comme je ne l’ai jamais été, comme je ne le serai jamais. J’enfonce jusqu’aux chevilles dans les fleurs et je viens d’embrasser deux joues qui sentaient la fraise.\par
\bigbreak
\noindent Il faut rentrer, on nous appelle ! Nous revenons comme des gens sages, et ces demoiselles m’ont pris chacune par un bras ; elles s’appuient un peu en croisant les mains et me secouant le coude, chaque fois qu’elles veulent m’apprendre quelque chose, ou me demander ce que je sais.\par
On me gronde déjà, remarquez ! On prétend que je ne réponds pas ou que je réponds mal. « On ne me dira plus rien si je me moque comme ça… Voulez-vous bien ! »\par
On me donne des tapes, on me fait des reproches.\par
C’est que j’ai adopté un système pour être à l’aise : je les embrasse quand elles me posent une question que je trouve trop difficile.\par
Ah ! que j’ai bien fait de boire du vin !\par
Elles veulent me \emph{rouler.}\par
« Vous savez la géographie ?\par
— Pas trop.\par
— Vous savez bien quel est le chef-lieu de… »\par
Je l’ignore absolument, et, pour m’en tirer, j’embrasse, j’embrasse ; j’en perds mon assurance, malgré le verre de gros bleu, et si elles ne faisaient pas des petites mines pour se cacher, elles me verraient rougir comme une pivoine.\par
\bigbreak
\noindent Nous arrivons à table. Il est midi. Les sabots des garçons de ferme battent l’heure du dîner dans la cour, et tout le monde rentre, même les poules, qui viennent attendre leur grain et se pressent contre la porte. Un poussin estropié se dépêche en tirant la patte ; les abords de la maison sont vides, je vois dans les champs les charrues s’arrêter et les laboureurs s’asseoir pour manger la soupe que vient d’apporter la servante dans son tablier vert.\par
C’est le grand calme de midi et son grand silence.\par
\bigbreak
\noindent À notre table (on a servi le dîner à part pour le neveu), il y a une nappe blanche, des fruits dressés dans des soucoupes et une branche d’églantier, qui est là toute frissonnante dans l’eau, fraîche comme un panache vert avec des grelots rouges.\par
Il vient je ne sais quelle odeur de sureau. – Ah ! j’ai le cœur qui s’en va, tant cette odeur est douce !\par
\bigbreak
\noindent Après le dîner.\par
« Si nous partions faire un tour en carriole avec notre cousin ?\par
— La Grise est trop fatiguée, dit le père.\par
— C’est vrai. Où irons-nous alors ? »\par
J’offre d’aller du côté des sureaux, et nous voilà, au bout d’un moment, occupés à vider la moelle de ces sureaux et à faire des sifflets luisants comme des cuivres ; la cousine Marguerite se coupe le doigt et laisse tomber de grosses gouttes de sang sur le blanc des feuilles.\par
On arrache une herbe pour la panser, et l’on va loin des vilains arbres qui sont cause qu’on s’est coupé.\par
On va vers la mare où les canards barbotent, on va dans la grange où les \emph{fléaux} s’arrêtent quand les demoiselles et le cousin entrent ! Puis ils repartent décrivant un grand cercle, et battent en mesure les gerbes sur le plancher sonore. J’en attrape un pour essayer ; je sens tourner le battant qui part comme une fronde, et qui revient comme un marteau, qui prend de l’air et fait du vent… S’il touchait une tête, il la casserait comme du verre.\par
Au fond du clos, il y a un trou plein d’eau et de branches mortes, avec de petites grenouilles vertes qui luisent au soleil ; je fais une ligne avec un bâton que je ramasse à terre, un bout de ficelle que je trouve dans mes poches, et une épingle que fournit Marguerite. Sa sœur donne un morceau de ruban écarlate, et la pêche commence.\par
Quels cris quand la première rainette mord ! Mais il faut l’arracher de l’hameçon, personne n’ose, la grenouille s’échappe et les jeunes filles s’enfuient.\par
\bigbreak
\noindent Je les suis ! Nous passons une journée délicieuse à battre les champs, à entrer jusqu’aux genoux dans la rivière ! Je cours après elles en sautant sur les pierres, que polit le courant.\par
À un moment, le pied me glisse et je tombe dans l’eau.\par
Je sors ruisselant, et je m’en vais, le pantalon tout collé et pesant, m’étendre au soleil. Je fume comme une soupe.\par
« Si nous le tordions ? » dit une cousine, en faisant un geste de lessive.\par
Elles vont de leur côté, derrière une pierre qui les cache mal, ôter leurs bas ; elles ont les jambes trempées, quoi qu’elles en disent… et si blanches !\par
Enfin nous voilà séchés, et nous repartons joyeux.\par
Nous avons les yeux clairs, la peau brillante. Nous prenons des chemins bordés de mûres, et pleins de petites prunes violettes qui sont aigres comme du vinaigre, et que nous mangeons à poignées, – j’avale les noyaux pour faire l’homme.\par
On se fâche, on se perd ! mais on se retrouve toujours bras dessus, bras dessous, raccommodés et curieux : moi, racontant ce que je fais à Saint-Étienne, les farces de collège ; elles, disant des gaietés de pension, ceci, cela, et finissant par crier :\par
« Laquelle aimez-vous le mieux de nous deux ?\par
— Laquelle aimes-tu mieux ? » dit carrément Marguerite, qui jette le\emph{ vous} par-dessus les moulins et se plante devant moi.\par
Ne sachant que répondre, je les embrasse toutes deux. On me fouette la figure avec une fleur et l’on s’écarte pour me bombarder de prunes violettes.\par
\bigbreak
\noindent Le soir nous trouve un peu las, et nous causons sur la pierre usée devant la maison, comme de petits vieux à la porte d’une auberge.\par
Ah ! c’est Marguerite que je préfère décidément ! Elle me prend la main toujours à la fin de ses phrases, elle me dit, ébouriffant ma crinière de ses doigts : « Rejette donc tes cheveux en arrière, tu n’es pas beau comme ça ! »\par
On me conduit à ma chambre qui est près du grenier, – le grenier où l’on a, l’hiver dernier, pendu les raisins, entassé les pommes, avec des bouquets de fenouil et des touffes sèches de lavandes. Il en est resté une odeur et je laisse la porte ouverte pour qu’elle entre \emph{chez moi}, – encore un \emph{chez moi} d’un soir !\par
Je me mets à la fenêtre et regarde au loin s’éteindre les hameaux. Un rossignol froufroute dans un tas de fagots et se met à chanter. Il y a le coucou qui fait hou-hou ! dans les arbres du grand bois, et les grenouilles qui font croa-croa dans les herbages du marais.\par
J’écoute et finis par ne rien entendre.\par
Le coq me réveille en sursaut, je m’étais endormi le front dans mes mains et je me déshabille avec un frisson, pour dormir d’un sommeil sans rêve, étourdi de parfums, écrasé de bonheur.\par
Deux jours comme cela, – avec des disputes et des raccommodailles près des buissons, dans les fleurs, dans le foin ; le grand jeu du fléau, le chant doux des rivières et l’odeur du sureau !\par
\bigbreak
\noindent Il faut partir !\par
\bigbreak
\noindent « Tu m’écriras, soupire Marguerite, me disant adieu. Tiens tu garderas ce petit bouquet comme souvenir. Bonsoir !… »\par
Elle me donne son front à embrasser, rien que son front. Ces deux jours-ci, elle se laissait embrasser sur les lèvres ; elle a l’air toute sérieuse, et je la vois de loin, debout, qui agite son mouchoir, comme font les châtelaines dans les livres, quand leur fiancé s’en va ; je tâte le bouquet qu’elle a fourré dans ma poitrine et je me pique le doigt à ses épines. J’ai sucé ce doigt-là.\par
Nous le retrouverons, ce bouquet, avec des larmes dans les fleurs sèches…
\section[{15. Projets d’évasion}]{15. \\
Projets d’évasion}\renewcommand{\leftmark}{15. \\
Projets d’évasion}

\noindent J’entre en quatrième. Professeur Turfin.\par
Il a été reçu le second à l’agrégation ; il est le neveu d’un chef de division, il porte de grands faux-cols, des redingotes longues, il a la lèvre d’en bas grosse et humide, des yeux bleus de faïence, des cheveux longs et plats.\par
Il a du mépris pour les pions, du mépris pour les pauvres, maltraite les boursiers et se moque des mal vêtus.\par
Il fait rire les autres à mes dépens ; je crois qu’il veut faire rire de ma mère aussi.\par
Je le hais…\par
\bigbreak
\noindent On m’accorde des \emph{faveurs} en ma qualité de fils de professeur.\par
Externe, je suis puni comme un interne. Toujours en retenue. Je ne rentre presque jamais à la maison. On m’apporte du réfectoire un morceau de pain sec.\par
« De cette façon, on lui donne à déjeuner pour rien ; je sauve encore une ratatouille à la mère Vingtras. »\par
C’est Turfin qui parle ainsi à quelque collègue qui sourit ; il le dit assez loin de moi à demi-voix, mais il veut, je crois, que je l’entende.\par
Je me contente d’enfoncer mes mains dans mes poches, et j’ai l’air de rire ! Je pleure. Que de sanglots j’ai étouffés pendant qu’on ne me voyait pas !\par
\bigbreak
\noindent Je ne suis plus qu’une bête à pensums !\par
Des lignes, des lignes ! – des arrêts et des retenues, du cachot !\par
Je préfère le cachot à la retenue.\par
Je suis libre entre mes quatre murs, je siffle, je fais des boulettes, je dessine des bonshommes, je joue aux billes tout seul.\par
Avec des morceaux de bois et des bouts de ficelle je monte des potences auxquelles je pends Turfin, je me remets à la besogne vers le soir et je fais mon pensum.\par
On me renvoie à neuf heures à la maison.\par
Le cachot ne m’épouvante pas ; même j’éprouve un petit orgueil à revenir le soir par les cours désertes, en rencontrant au passage quelques élèves qui me regardent comme un révolté !\par
Nous nous croisons souvent avec Malatesta, qui sort d’un autre cachot. C’est le chef des \emph{chahuteurs} dans l’étude des grands.\par
Il va entrer en élémentaires.\par
C’est lui qui doit être reçu à Saint-Cyr l’an prochain. C’est le champion de Saint-Étienne ; on ne le renverrait pas pour un empire.\par
Il porte un képi à galons d’or et \emph{il prend des leçons d’armes.}\par
Malatesta me fait des signes de tête en passant et me dit : « Salut, Vingtras ! » Salut, comme en latin, « Vingtras », comme à un homme.\par
\bigbreak
\noindent C’est la retenue qui m’ennuie le plus.\par
J’y gobe encore des pensums. – Je suis si maladroit ! – C’est mon encrier que je renverse, c’est mon porte-plume qui tombe, mes papiers qui s’envolent, mon pupitre que je démanche.\par
« Vingtras, cent lignes ! »\par
Patatras ! mon paquet de livres qui dégringole et fait un tapage d’enfer !\par
« Cent lignes de plus.\par
— M’sieu !\par
— Vous répliquez ? Cinq pages de grammaire grecque. »\par
Encore ! Toujours !\par
Ils veulent me faire mourir sous le pensum, ces gens-là !\par
C’est à peine si je vois le soleil !\par
Le dimanche, comme les autres jours, j’arrive pour la grande retenue, de deux à six, dans cette salle vraiment lugubre ce jour-là, à cause du silence écrasant, du bruit mélancolique que fait un soulier qui passe, une porte qui tombe, un fredon solitaire, un cri de marchand bien loin, bien loin !\par
Nous sommes là une vingtaine.\par
Une plume grince, quelqu’un tousse, le pion fait deux ou trois tours en regardant le ciel à travers les croisées.\par
« M’sieu… sortir ! »\par
Il fait oui de la tête, et sous prétexte d’aller là-bas, je traîne un peu dans les longs corridors, je fourre le nez dans des salles vides, je jette par une fenêtre une bille, j’envoie une boulette de pain à un moineau, je lorgne l’infirmière et je tâche d’aller chiper des fruits au réfectoire, puis je reviens à cloche-pied, dans l’étude.\par
Je me replonge la tête dans ce qui me reste de papier, que je barbouille avec ce qui me reste d’encre, je pense à tout autre chose qu’à ce que j’écris – et il se trouve qu’il y a quelquefois dans mes pensums des « Turfin pignouf. Turfin crétin. »\par
\bigbreak
\noindent \emph{Mardi matin}.\par
C’était composition en version latine.\par
Je cherchais un mot, dans un dictionnaire tout petit que mon père m’a donné à la place de Quicherat.\par
Turfin croit que c’est une traduction.\par
Il s’avance et me demande le livre que je cachais tout à l’heure.\par
Je lui montre le petit dictionnaire.\par
« Ce n’est pas celui-là.\par
— Si, m’sieu !\par
— Vous copiez votre version.\par
— Ce n’est pas vrai ! »\par
Je n’ai pas fini le mot qu’il me soufflette.\par
Mon père et mère me battent, mais eux seuls dans le monde ont le droit de me frapper. Celui-là me bat parce qu’il déteste les pauvres.\par
Il me bat pour indiquer qu’il est l’ami du sous-préfet, qu’il a été reçu second à l’agrégation.\par
Oh ! si mes parents étaient comme d’autres, comme ceux de Destrême qui sont venus se plaindre parce qu’un des maîtres avait donné une petite claque à leur fils !\par
Mais mon père, au lieu de se fâcher contre Turfin, s’est tourné contre moi, parce que Turfin est son collègue, parce que Turfin est influent dans le lycée, parce qu’il pense avec raison que quelques coups de plus ou de moins ne feront pas grand-chose sur ma caboche. Non, mais ils font marque dans mon cœur.\par
J’ai eu un mouvement de colère sourd contre mon père.\par
Je n’y puis plus tenir ; il faut que je m’échappe de la maison et du collège.\par
\bigbreak
\noindent Où irai-je ? – À Toulon.\par
Je m’embarquerai comme mousse sur un navire et je ferai le tour du monde.\par
Si l’on me donne des coups de pied ou des coups de corde, ce sera un étranger qui me les donnera. Si l’on me bat trop fort, je m’enfuirai à la nage dans quelque île déserte, où l’on n’aura pas de leçon à apprendre ni du grec à traduire.\par
Il y a encore une consolation, même si l’on est attaché au grand mât ou enchaîné à fond de cale ; il y a l’espérance d’arriver à être officier à son tour, et l’on a le droit de souffleter le capitaine.\par
Turfin, lui, peut me tourmenter tant qu’il voudra, sans que je puisse me venger.\par
Mon père peut me faire pleurer et saigner pendant toute ma jeunesse ; je lui dois l’obéissance et le respect.\par
Les règles de la vie de famille lui donnent droit de vie et de mort sur moi.\par
Je suis un mauvais sujet, après tout !\par
On mérite d’avoir la tête cognée et les côtes cassées, quand, au lieu d’apprendre les verbes grecs, on regarde passer les nuages ou voler les mouches.\par
On est un fainéant et un drôle, quand on veut être cordonnier, vivre dans la poix et la colle, tirer le fil, manier le tranchet, au lieu de rêver une \emph{toge} de professeur, avec une toque et de l’hermine.\par
On est un insolent vis-à-vis de son père, quand on pense qu’avec la\emph{ toge} on est pauvre, qu’avec le tablier de cuir on est libre !\par
C’est moi qui ai tort, il a raison de me battre.\par
Je le déshonore avec mes goûts vulgaires, mes instincts d’apprenti, mes manies d’ouvrier.\par
Mes parents m’ont donné de l’éducation et je n’en veux plus !\par
Je me plais mieux avec les laboureurs et les savetiers qu’avec les agrégés ; et j’ai toujours trouvé mon oncle Joseph moins bête que M. Beliben !…\par
« Fort comme il est, et si fainéant ! » disent-ils toujours. C’est justement parce que je suis fort que je m’ennuie dans ces classes et ces études où l’on me garde tout le jour. Les jambes me démangent, la nuque me fait mal.\par
Je suis gai de nature ; j’aime à rire et j’ai la rate qui va en éclater quelquefois ! Quand je peux échapper aux pensums, éviter le séquestre, être loin du pion ou du professeur, je saute comme un gros chien, j’ai des gaietés de nègre.\par
Être nègre !\par
Oh ! comme j’ai désiré longtemps être nègre !\par
D’abord, les négresses aiment leurs petits. – J’aurais eu une mère aimante.\par
Puis quand la journée est finie, ils font des paniers pour s’amuser, ils tressent des lianes, cisèlent du coco, et ils dansent en rond !\par
Zizi, bamboula ! Dansez, Canada !\par
Ah ! oui ! j’aurais bien voulu être nègre. Je ne le suis pas, je n’ai pas de veine !\par
Faute de cela, je me ferai matelot.\par
Tout le monde s’en trouvera bien.\par
« Je les fais périr de chagrin ? » ils me l’ont assez dit, n’est-ce pas ?\par
Ils vont revivre, ressusciter.\par
Je leur laisse ma part de haricots, ma tranche de pain ; mais ils devront finir le gigot !\par
Finir le gigot ?\par
Je suis une triste nature décidément ! Je ne songe pas seulement au plaisir d’échapper à ce gigot ; mais, dévoré d’une idée de vengeance, je me dis, comme un petit jésuite, que c’est eux qui auront à le manger, rôti, revenu, en vinaigrette, à la sauce noire, en émincés et en boulettes, – comme je faisais.\par
\bigbreak
\noindent Je vais plus loin, hypocrite que je suis !\par
Je me dis qu’il faut m’exercer, me tâter, m’endurcir, et je cherche tous les prétextes possibles pour qu’on me \emph{rosse}.\par
J’en verrai de dures sur le navire. Il faut que je me \emph{rompe} d’avance, ou plutôt qu’on me \emph{rompe} au métier ; et me voilà pendant des semaines disant que j’ai cassé des écuelles, perdu des bouteilles d’encre, mangé tout le papier ! – Il faut dire que je mange toujours du papier et que je bois toujours de l’encre, je ne peux pas m’en empêcher.\par
Mon père ne se doute de rien et se laisse prendre au piège, le malheureux !…\par
Je lui use trois règles et une paire de bottes en quinze jours, il me casse les règles sur les doigts et m’enfonce ses bottes dans les reins.\par
Je lui coûte les yeux de la tête, je le ruine, cet homme !\par
Je pense qu’il me pardonnera plus tard en faveur de l’intention ; et d’ailleurs il me semble que cela ne l’ennuie pas trop.\par
Un peu fatigué seulement quand il m’a rossé trop longtemps, – il a chaud !\par
Je me traîne alors jusqu’à la fenêtre, et je la ferme pour qu’il n’attrape pas de courants d’air.\par
La nuit, je me couche dans une malle, – en chemise.\par


\begin{verse}
Je me couche en chemise !\\
Dieu puissant ! favorise\\
Cette sainte entreprise !\\
\end{verse}

\noindent Partirai-je seul ?\par
C’est bien ennuyeux ! Et puis à plusieurs on peut s’emparer d’un navire, faire le corsaire, au besoin mener les révoltes, et quand on est fatigué, fonder une colonie.\par
Qui entraînerai-je dans cette expédition ?\par
Malatesta est justement parti d’hier.\par
Sa mère est tout d’un coup tombée malade, et il est allé la voir.\par
Il adore sa mère, une mauvaise mère, cependant !\par
Elle lui envoie toujours des pastèques, des dattes et des oranges ; elle lui fait passer de l’argent en cachette du proviseur.\par
« Elle est donc bien riche, ta mère ? lui demandai-je un jour.\par
— Non, mais elle est si bonne !\par
— Tu l’aimes bien !\par
— Si je l’aime ! »\par
Il me dit cela avec une petite larme dans les yeux.\par
Lui qui doit être soldat !\par
Avoir une si mauvaise mère et l’aimer tant ! Une mère qui le console quand il est puni, qui mange peut-être moins de pain pour que son enfant ait plus d’oranges !\par
« Que fait-elle, ta mère ?\par
— Elle est charcutière à Modène. »\par
Et il n’a pas l’air de rougir !\par
Charcutière ! Tout s’explique. C’est une femme \emph{du commun}.\par
Ma mère n’aurait jamais été charcutière. Jamais !\par
\bigbreak
\noindent Ah ! elle est fière, ma mère, il faut lui accorder ça.\par
Si ce n’avait pas été pour elle, c’eût été pour son fils qu’elle n’eût pas voulu vendre du jambon.\par
Elle préférait crever la misère, conseiller à mon père d’être lâche !…\par
Elle préférait vivre d’une vie sourde, bête et vile ; mais elle était la femme d’un fonctionnaire, une dame, et son enfant dirait un jour :\par
« Mon père était dans l’Université. »\par
Ah ! cela me fera une belle jambe, et on a l’air de les estimer drôlement, ces messieurs de l’université !\par
Si elle entendait ce que j’entends, moi, non pas seulement ce que les élèves marmottent – ce n’est rien – mais ce que les parents disent, elle verrait ce qu’on pense des professeurs ! si elle savait comme ils sont méprisés par les chefs même : le proviseur, l’inspecteur, le censeur, qui, quand une mère riche se plaint, répondent :\par
« N’ayez peur : je lui laverai la tête ! »\par
Du petit cabinet où l’on m’enferme d’habitude avant de me mener au cachot, je puis saisir ce qu’on dit dans le salon du proviseur, et je n’ai pas manqué d’appliquer mes oreilles contre le mur, chaque fois que j’ai pu.\par
Un jour, un des maîtres est venu se plaindre qu’un domestique l’avait insulté. Le proviseur n’a fait ni une ni deux : il appelle le pion Souillard, qui lui sert de secrétaire : « M. Souillard, il y a M. Pichon qui se plaint de ce que Jean lui ait parlé insolemment devant les élèves ; – il faut que l’un des deux file. Je tiens à Jean ; il nettoie bien les lieux. M. Pichon est un imbécile qui n’a pas de protections, qui achète cent francs de bouquins pour faire son livre d’étymologie et qui porte des habits qui nous déshonorent.\par
« Écrivez en marge à son dossier :\par
« “PICHON. Se commet avec les domestiques – a des habitudes de saleté – sait ses classiques. Rendrait de grands services dans une autre localité.” »\par
\bigbreak
\noindent Ah ! vivent les charcutiers, nom d’une pipe !\par
Et les cordonniers aussi ! vivent les épiciers et les bouviers !\par
Vivent les nègres !…\par
Moi, plutôt que d’être professeur, je ferai tout, tout, tout !…\par
\bigbreak
\noindent Il n’y a donc pas à compter sur Malatesta, qui est à la charcuterie de Modène, et il a même laissé intacte dans son pupitre une boîte de fruits confits qu’on se partage en retenue.\par
Je cherche de tous côtés d’autres complices ; je jette sur la foule des camarades le regard creux du capitaine. Je fais des ouvertures à plusieurs : ils hésitent. Les uns disent qu’ils ne s’ennuient pas à la maison, qu’ils s’y amusent beaucoup, au contraire, que leur père rigole avec eux, que leur mère a les mêmes défauts que celle de Malatesta.\par
« On ne te bat donc pas ?\par
— Si, quelquefois, mais je suis content ces jours-là ; je suis sûr que le soir on me mènera au spectacle ou bien qu’on me donnera une pièce de dix sous. Mon père en est tout embêté, et ils se cherchent des raisons avec ma mère. – C’est toi qui en es cause. – Je te dis que c’est toi. – Tu ne lui as pas fait de mal au moins ! – J’ai bien tapé un peu fort, quel brutal je suis ! »\par
« Tu lui as fait du mal au moins », demande ma mère à mon père, à l’envers de ces parents imbéciles. « J’espère qu’il l’a senti cette fois ! »\par
Et il faut bien avouer que ma mère est logique. Si on bat les enfants, c’est pour leur bien, pour qu’ils se souviennent, au moment de faire une faute, qu’ils auront les cheveux tirés, les oreilles en sang, qu’ils souffriront, quoi !… Elle a un système, elle l’applique.\par
Elle est plus raisonnable que les parents de ce petit à qui on donne dix sous quand on lui a envoyé une taloche ; qui tapent sans savoir pourquoi, et qui regrettent d’avoir fait mal.\par
Je ne comprends pas comment mon camarade aime tant ses parents, qui sont si bêtes et ont si peu d’énergie.\par
Je suis tombé sur une mère qui a du bon sens, de la méthode.\par
\bigbreak
\noindent Je ne trouverai donc personne qui veuille s’enfuir avec moi !\par
Ricard ?\par
Ils sont neuf enfants.\par
On les fouette à outrance. – Quel bonheur !\par
Je tâte Ricard ; – quand je dis je tâte, je parle au figuré : il me défend de le tâter (il a trop mal aux côtes) – il est sale comme un peigne ; il m’explique que c’est parce qu’ils sont sales que leur mère les bat ; mais elle est diablement sale aussi, elle !\par
Elle les rosse encore parce qu’ils disent des gros mots ; ils jurent comme des charretiers ; il y a le petit de cinq ans qui crie toujours : « \emph{Crotte pour toi ! »}\par
Il n’y en a qu’un dans la famille qui soit bien sage et qui ne jure pas. C’est celui qui est en classe avec moi.\par
On le bat tout de même. Pourquoi donc ?\par
Parce qu’il ne faut pas faire de préférences dans les familles, c’est toujours d’un mauvais effet. Les autres pourraient s’en plaindre.\par
Puis, « il est là comme une oie. »\par
Il est là comme une oie. – Voilà pourquoi on le bat.\par
On fouette les autres parce qu’ils font du bruit et qu’ils jurent et sont grossiers : on le fouette, lui, parce qu’il ne dit rien et se tient tranquille.\par
« \emph{Il est là comme une oie}… »\par
Il a encore une faiblesse – (qui n’a pas les siennes !) – il pisse au lit.\par
Voilà le secret de sa misère, pourquoi il est triste, pourquoi sa mère crie toujours qu’elle va lui enlever la peau de ceci, la peau de cela !\par
Et ses parents ont l’air de croire que c’est pour s’amuser, parce qu’il y trouve du plaisir, que c’est par coquetterie ou défi, un jeu ou une menace, une fantaisie de talon rouge, un mouvement de désœuvré. Le malheureux fait pourtant ce qu’il peut, – ce qu’il fait ne sert à rien. – Il se réveille dans le crime, et on est obligé de mettre ses draps à la fenêtre tous les matins.\par
On lui procure cette honte. – Tout le monde sait sa faute ; comme on sait que le roi est aux Tuileries, quand le drapeau flotte au-dessus du château !…\par
Il en pleure de douleur, le pauvre mâtin, il se prive de tout, exprès, quand il soupe le soir, et boit avec une paille.\par
C’est en vain qu’il prie Dieu, la sainte Vierge et cherche s’il y a un saint spécialement affecté à ce genre de péché ; il retombe désespéré sous le coup de torchon de sa mère, qui a une drôle d’expression pour annoncer que la danse commence. Elle dit de sa grosse voix, et en levant le fouet :\par
« Ah ! nous allons \emph{faire pleurer le lapin} ! »\par
Allusion, sans doute (ironique et cruelle), à la faiblesse de son enfant et à l’opération que le chasseur fait subir au lapin atteint par son plomb meurtrier.\par
Je le décide. Il fera son hamac lui-même à bord du navire, et personne ne saura que le lapin a pleuré !\par
\bigbreak
\noindent Si je parlais aussi à Vidaljan ?\par
C’est le fils d’un rat-de-cave ; il reçoit, comme moi, des roulées à tout casser.\par
Encore un qui voudrait être ce que son père ne voudrait pas qu’il fût : il voudrait être escamoteur.\par
Il est venu un escamoteur au collège. Les élèves payaient vingt sous. Vidaljan a eu le malheur d’être choisi pour monter sur l’estrade et tenir le paquet de cartes ; il a vu couper le cou à la tourterelle, brûler le mouchoir ; il a frôlé Domingo, le compère.\par
« Pardon, mon ami, qu’avez-vous là dans votre poche ? »\par
Et l’on a retiré de sa poche une perruque.\par
« Vous portez donc vos économies dans vos cheveux ? »\par
Et l’on rafle sur sa tête une pièce de cinq francs.\par
« Maintenant, mon ami, je vous remercie. »\par
Il est descendu à sa place devant tout le collège, entouré, questionné, envié ; sa classe crève de jalousie.\par
Pourquoi est-ce lui qu’on a pris ? Qui l’a fait choisir ?\par
« Il a de la chance », a dit Ricard aîné, qui pense que, la nuit prochaine…\par
\bigbreak
\noindent Depuis cette soirée où il a eu son rôle, éclairé par toutes les bougies du sorcier, objet de l’attention de la foule, dévoré par les regards des \emph{grands} et des \emph{moyens}, depuis ce jour-là, la résolution de Vidaljan est prise, sa vocation est décidée : il va se mettre au travail tout de suite. Il a toujours eu un penchant pour l’escamotage !\par
C’est le plus grand chipeur du collège ; il aimait déjà à fouiller dans les pupitres, et il savait retirer un crayon de dessus l’oreille d’un camarade, sans que le camarade s’en doutât. Il savait couper une orange en huit et cacher une pièce dans le coin d’un mouchoir.\par
Il escamotait déjà la toupie, l’agate et la plume à tête de mort. Il avait une collection de petits dessins cueillis à l’aide de fausses clefs dans les boîtes des copains.\par
Non qu’il aimât les arts, mais il se plaisait à faire de la serrurerie sournoise et à passer sa main entre les fentes. Il volait les cahiers de punition et les listes de places dans la poche des maîtres. Il avait une fois subtilisé le porte-feuille d’un professeur, et les secrets de M. Boquin avaient été à la merci des moutards pendant huit jours.\par
Le pauvre Boquin en avait manqué un mariage et failli perdre sa place.\par
Vidaljan avait apporté aussi des améliorations dans la plume à \emph{pensums} : il était parvenu à ficeler quatre becs ensemble, ce qui ne s’était jamais vu encore, de l’aveu même de Gravier, qui avait été trois mois en pension à Paris, et il écrivait quatre vers de Virgile à la fois.\par
Déjà porté à l’escamotage, il eut la tête tournée par la magie blanche.\par
Il acheta les \emph{Secrets du petit Albert.} Nous le vîmes avec des gobelets et des muscades, avec des crapauds séchés et des coquilles d’œufs vides.\par
Il fabriquait de la poudre.\par
C’est ce qui me décida à m’adresser à lui, – malgré l’espèce de défiance que m’inspiraient ses habitudes.\par
Il avait, deux jours auparavant, failli être assommé par l’auteur de ses jours, qui avait appris qu’au lieu de faire ses devoirs son fils se livrait à la mécanique ; et, en retournant le lit de son enfant, la mère avait trouvé des peaux de serpents et des punaises de cuivre mêlées aux punaises de famille.\par
Je lui offris d’être mon lieutenant.\par
Il accepta. – Ricard aussi.\par
\bigbreak
\noindent Mais, au jour fixé, le drapeau flotte à la fenêtre de Ricard, et il me jette par cette fenêtre un papier, un peu humide, qui me donne de douloureux détails. Il a été criminel plus que de coutume et on l’a battu plus que jamais ; il ne peut pas se traîner.\par
Et Vidaljan ? – Il n’est pas au rendez-vous. Les élèves arrivent l’un après l’autre, la cloche sonne, on entre, il n’est pas là. Que s’est-il passé ?\par
Je vais du côté de sa maison en me cachant ; je rencontre des commères qui racontent que le quartier a failli sauter, et le fils Vidaljan avec.\par
« Il a laissé tomber une allumette sur une écuelle où il faisait de la poudre. C’est un petit vaurien qui lui avait mis ça dans la tête, le petit de cette dame qui marchande toujours, vous savez, et qui a son châle collé sur le dos comme une limande : Vingtrou, Vingtras… On doit être en train de le chercher. J’espère qu’on le fichera en prison.\par
— Mais le voilà, je le reconnais », crie une commère, qui m’aperçoit tout d’un coup dans le coin où j’étais courbé, et d’où j’essayais de filer.\par
On s’empare de moi. – On me ramène à la maison.\par
Ma mère m’en donna une volée !\par
Elle ne s’arrêta que quand j’eus promis sur tous les saints du paradis de ne plus m’échapper.\par
\bigbreak
\noindent Et Vidaljan ? – Il guérit et ne fit plus de poudre.\par
Et Ricard aîné ? – La peur qu’il eut en apprenant l’accident de Vidaljan lui fit une révolution, et il ne pissa plus au lit.\par
C’est toujours ça.
\section[{16. Un drame}]{16. \\
Un drame}\renewcommand{\leftmark}{16. \\
Un drame}

\noindent Madame Brignolin, une voisine, est devenue l’amie de la maison.\par
C’est une petite créature potelée, vive, aux yeux pleins de flamme ; elle est gaie comme tout, c’est plaisir de la voir trottiner, rigoler, coqueter, se pencher en arrière pour rire, tout en lissant ses cheveux d’un geste un peu long et qui a l’air d’une caresse ! Et elle vous a des façons de se trémousser qui paraissent singulières à mon père lui-même, car il rougit, pâlit, perd la voix et renverse les chaises.\par
Drôle de petite femme ! Elle a trois enfants.\par
Elle conduit et élève tout cela avec une activité fiévreuse, elle ne fait qu’aller, venir ; habillant l’un, savonnant l’autre, plantant une casquette sur cette binette, un bonnet sur ce bout de crâne, recousant les culottes, repassant les robes, mouchant celui-ci, nettoyant celle-là. Toujours en l’air !\par
Le soir, elle sort un peignoir frais et fait un bout de musique devant un vieux piano à queue ; à la fin de chaque morceau, elle en arrache un \emph{boum} grave du côté des notes graves et un \emph{hi} flûté du côté des notes minces. \emph{Boum, boum, hi hi…}\par
\bigbreak
\noindent « M. Vingtras, vous êtes triste comme un bonnet de nuit, c’est que vous ne vous êtes pas fait raser, voyez-vous ! Revenez demain en sortant de chez le coiffeur. Je vous embrasserai ; vous me donnerez l’étrenne de votre barbe. »\par
Et en même temps elle passe près de lui, met sa main sur sa main, le frôle avec sa jupe. Elle lui prend le bras même et lui donne sa ceinture à presser.\par
« Valsons », dit-elle.\par
Et avançant, d’un air joyeux, ses petits pieds hardis, le buste rejeté en arrière, les cheveux flottants, elle entraîne son cavalier ; un ou deux tours dans la chambre trop étroite, – et elle va retomber, en riant, sur une chaise qui crie, devant mon père qui ne dit rien.\par
Puis elle file du côté de la cuisine où l’on a entendu du bruit.\par
C’est la fillette qui est à terre ; c’est le gamin qui a cassé une cruche ; elle roule comme un tourbillon de mousseline, s’engouffre, disparaît, revient, tapageuse et folle, serrant ses deux mains à plat entre ses genoux, penchée pour mieux rire, et secouant sa jolie tête, en racontant quelque aventure salée arrivée à un de ses rejetons.\par
Elle trouve encore moyen d’effleurer et de bousculer M. Vingtras en passant.\par
\bigbreak
\noindent M. Brignolin est rarement là : c’est un savant. Il est associé dans une fabrique de produits chimiques, et il a déjà inventé un tas de choses qui font bouillir ses fourneaux et sa marmite : il est toujours dans les \emph{cornues}, et j’ai même remarqué que l’on riait quand on disait ce mot-là.\par
\bigbreak
\noindent Il y a une cousine dans la maison : mademoiselle Miolan.\par
Elle a vingt ans : douce, complaisante et pâle, pâle comme la cire, et j’entends dire tout bas qu’elle va bientôt mourir.\par
Madame Brignolin est pleine de bonté pour elle, nous l’aimons tous ; nous jouons aux cartes et aux dés sur ses genoux ; elle nous fait des cocardes avec des bouts de rubans, – elle est si habile de ses doigts maigres ! Elle a dans une poche un portefeuille à coins de nacre, la seule chose qu’elle nous empêche de toucher : « C’est là qu’est mon cœur », a-t-elle dit un jour, et l’on raconte qu’elle meurt d’un amour perdu.\par
Le jour où madame Brignolin contait cela, mon père était près d’elle. Ma mère était absente. Je tournai la tête : j’entendis un soupir, et, quand je regardai, je vis madame Brignolin qui avait les mains sur celles de mon père et les yeux dans ses yeux ! Il avait l’air gêné, lui. Elle souriait doucement, et elle lui dit :\par
« Grand bête ! »\par
Je devinai que je les embarrassais et ils jetèrent sur moi, tous les deux en même temps, un regard qui voulait dire : « Pas devant lui », ou « Pourquoi est-il là ? » Je n’ai jamais oublié ce « grand bête ! » si tendre et ce geste si doux.\par
\bigbreak
\noindent Pour mademoiselle Miolan, on a loué un bout de campagne, où l’on va passer deux ou trois heures le soir, après le collège, où l’on dépense, quand il fait beau, toute la journée du dimanche.\par
Les belles heures pour les petits Brignolin et moi !\par
Les environs de la maison de plaisance ne sont pas beaux, – c’est au bout d’un chemin désert, noir de charbon, jaune de sable, gris de poussière, qui sent le brûlé, a des odeurs de cendre, sur lequel les souliers s’écorchent et les voitures crient. Il y a une mine là-bas et deux briqueteries qui montrent leurs toits plats dans le vide des champs ; – l’herbe est maigre et roussie, elle traîne par places comme des restes de poil sur un dos de chameau ; il y a des débris de coke et de briques, rougeâtres et ternes comme des grumeaux de sang caillé ; mais nous entassons tout cela en forme de portiques et de cabanes, et nous faisons des trous dans la terre ; on y allume du feu, l’on souffle, et la flamme brille, la fumée tourne dans le vent. Cela sent le travail, rappelle Robinson ; on est seul dans cette vaste plaine – comme si l’on devait vivre sans le secours des villes : on parle comme des hommes, et comme des hommes on a l’émotion que donne toujours le silence.\par
Quand on est las de cette nature muette et vide, quand le froid de la nuit descend, quand les bruits tombent un à un comme des pierres dans un gouffre, on revient vers la petite maison qui est coiffée de rouge et chaussée de vert.\par
Il y a un jardinet, deux arbres, des carrés de pensées, un \emph{soleil}.\par
Ces pensées, je les vois encore, avec leurs prunelles d’or et leurs paupières bleues, je sens le velours de leurs feuilles, et je me rappelle qu’il y avait une touffe dont je prenais soin ; il en reste encore des pétales dans un vieux livre où je les avais mises.\par
À l’heure où la maison s’allume, nous voyons de loin la lampe qui luit comme une étoile.\par
Ces dames et mon père improvisent un souper de fruits, avec du lait et du pain noir. On est allé chercher tout cela dans le fond du village. – Quel calme ! J’en ai des larmes de félicité dans les yeux.\par
\bigbreak
\noindent Le dimanche, c’est un brouhaha ! Nous portons les provisions. Madame Brignolin met un tablier blanc, ma mère retrousse sa robe, et mon père aide à éplucher les légumes. – On nous jette, à nous, quelques carottes crues à grignoter, et nous aidons pour la cuisine, nous faisons tourner le poulet devant le feu de braise (en arrêtant en route les larmes de jus) : nous embrouillons tout, nous troublons tout, nous cassons tout, personne ne s’en plaint.\par
C’est un bruit de casseroles et d’assiettes, puis un bruit de mâchoires, puis un bruit de bouchons ! – Au dessert, on goûte au vin blanc mousseux.\par
On trinque, on retrinque.\par
C’est toujours à la santé de madame Vingtras qu’on boit d’abord !\par
Elle répond toute rouge de joie : son sang de paysanne coule plus libre dans cette atmosphère de campagne, avec ces petites odeurs de cabaret et ces vues de fermes dans le lointain !\par
À peine elle pense à mon pantalon que je dois retrousser, à mes chaussures neuves qui ont des boulets de boue. Madame Brignolin, d’ailleurs, l’en empêche.\par
« Il faut que tout le monde s’amuse ! » dit-elle en lui fermant la bouche et en la tirant par le bras pour l’entraîner à la promenade ou au jardin.\par
C’est mon père qui paraît heureux !\par
Il joue comme un enfant ; c’est lui qui fait le \emph{pot} aux quatre coins, qui pousse la balançoire quand on est las de jouer, il chante (il a un filet de voix). Madame Brignolin lance après lui des chansons du Midi.\par
Ma mère – paysanne – dit : « Ça, c’est des airs de freluquets », et elle entonne en auvergnat :\par

\begin{verse}
Digue d’Janette,\\
Te vole marigua\\
Laya !\\
Vole prendre un homme !\\
Que sabe trabailla,\\
Laya !\\!
\end{verse}
\noindent « \emph{Laya ! »} reprend madame Brignolin en esquissant à son tour une pose de danse – rien qu’un geste, la tête renversée, le buste pliant et puis tout d’un coup un ramassis de jupes, un rejeté de hanche !\par
Elle tape du pied, fait claquer ses doigts, et elle a l’air enfin de s’évanouir avec les lèvres entrouvertes, par où passe un souffle qui soulève sa poitrine ; elle est restée un moment sans rire, mais elle repart bien vite dans un accès de gaieté qui mêle la cachucha et la bourrée, l’espagnol et l’auvergnat,\par

\begin{verse}
La Madona et la fouchtra,\\
Laya !\\
\end{verse}
\noindent « Qu’est-ce que cela veut dire ? » demande M. Brignolin, un positif, qui vient de temps en temps pour le malheur des sauces.\par
Il essaye des jus concentrés basés sur la chimie, qui sentent le savant et gâtent le dîner.\par
On joue, – il embrouille le jeu, – ne devine jamais !\par
Il\emph{ l’}est toujours.\par
« C’est lui qui\emph{ l’est ! »}\par
Mme Brignolin dit cela d’une drôle de façon et presque toujours en regardant mon père ; puis elle ajoute en secouant son mari :\par
« Allons, tu n’es bon qu’à donner le bras ; prends le bras de Mme Vingtras. – M. Vingtras, voulez-vous me donner le vôtre ? – Jacques, toi tu seras avec Mlle Miolan. »\par
\bigbreak
\noindent Pauvre fille ! tandis que nous jouons et faisons tapage, elle est souvent prise d’un serrement de cœur ou d’une quinte de toux qui empourpre ses joues pâles, puis la laisse retomber sur l’oreiller qui rembourre sa chaise longue ; – elle sourit tout de même et elle se fâche quand nous voulons nous taire à cause d’elle.\par
« Non, non, amusez-vous, je vous en prie. Cela me fait plaisir, cela me fait du bien, amusez-vous. »\par
Sa voix s’arrête, mais son geste continue et nous dit :\par
« Amusez-vous ! »\par

\labelblock{CHÔMAGE}

\noindent La vie change tout d’un coup.\par
J’ai été jusqu’ici le tambour sur lequel ma mère a battu des \emph{rrra} et des \emph{fla}, elle a essayé sur moi des roulées et des étoffes, elle m’a travaillé dans tous les sens, pincé, balafré, tamponné, bourré, souffleté, frotté, cardé et tanné, sans que je sois devenu idiot, contrefait, bossu ou bancal, sans qu’il m’ait poussé des oignons dans l’estomac ni de la laine de mouton sur le dos – après tant de gigots pourtant !\par
À un moment, son affection se détourne. Elle se relâche de sa surveillance.\par
On n’entendait jadis que pif-paf, v’li-v’lan, et allez donc ! – On m’appelait bandit, \emph{sapré} gredin ! – \emph{Sapré} pour sacré ; – elle disait\emph{ bouffre} pour \emph{bougre}.\par
Depuis treize ans, je n’avais pas pu me trouver devant elle cinq minutes – non, pas cinq minutes, sans la pousser à bout, sans exaspérer son amour.\par
Qu’est devenu ce mouvement, ce bruit, le train-train des calottes ?\par
Je ne détestais pas qu’on m’appelât bandit, gredin ; j’y étais fait, – même cela me flattait un peu.\par
Bandit ! – comme dans le roman à gravures. – Puis je sentais bien que cela faisait plaisir à ma mère de me faire du mal ; qu’elle avait besoin de mouvement et pouvait se payer de la gymnastique sans aller au gymnase, où il aurait fallu qu’elle mît un petit pantalon et une petite blouse. – Je ne la voyais pas bien en petite blouse et en petit pantalon.\par
Avec moi, elle tirait au mur ; elle faisait envoler le pigeon, elle gagnait le lapin, elle amenait le grenadier.\par
\bigbreak
\noindent Je vis donc depuis quelque temps, sans rien qui me rafraîchisse ou me réchauffe, comme la gerbe qui moisit dans un coin, au lieu de palpiter sous le fléau, comme l’oie qui, clouée par les pattes, gonfle devant le feu.\par
Je n’ai plus à me lever pour aller – cible résignée – vers ma mère ; je puis rester assis tout le temps !\par
Ce chômage m’inquiète.\par
Rester assis, c’est bien, – mais quand on retournera aux habitudes passées, quand l’heure du fouet sonnera de nouveau, où en serai-je ? Les délices de Capoue m’auront perdu : je n’aurai plus la cuirasse de l’habitude, le caleçon de l’exercice, le grain du cuir battu !\par
\bigbreak
\noindent Que se passe-t-il donc ?\par
Je ne comprends guère, mais il me semble que madame Brignolin est pour quelque chose dans cette tristesse noire de la maison, dans cette colère blanche de ma mère.\par
Ma mère reste de longues soirées sans rien dire, les yeux fixes et les lèvres pincées. Elle se cache derrière la fenêtre et soulève le rideau, elle a l’air de guetter une proie.\par
\bigbreak
\noindent « Vous ne voyez plus madame Brignolin ? lui demande un jour une voisine.\par
— Si, si !\par
— Il y a un peu de froid ?\par
— Non, non !… nous allons même à la campagne ensemble, dimanche prochain. »\par
En effet, j’ai entendu parler d’une partie qui est comme une réconciliation après quelques semaines de froideur ; j’ai aussi distingué quelques mots que ma mère a prononcés tout bas : « N’avoir l’air de rien, les laisser seuls, venir à pas de loup… »\par
On se fait de nouveau des amitiés, on se voit le jeudi et l’on combine tout pour le dimanche.\par
\bigbreak
\noindent J’avais justement gobé une \emph{retenue} !\par
J’avais laissé tomber un morceau de charbon en pleine classe – du charbon ramassé près de la maison de campagne. J’avais entendu M. Brignolin dire qu’il y avait du diamant dans les éclats de mine ; et depuis ce jour-là, je ramassais tous les morceaux qui avaient une veine luisante, un point jaune.\par
Le professeur crut à une farce, – me voilà pincé ! forcé de rester en ville ce dimanche-là, pour aller à une heure faire ma retenue – dans l’étude des internes, au lycée même.\par
Adieu la maison de campagne !\par
Je les vis partir avec les paniers de provisions.\par
Les dames avaient mis ce jour-là des robes neuves.\par
Madame Brignolin était charmante ; un peu décolletée, avec une écharpe à raies bleues, des bottines prunelles, et elle sentait bon – mais bon !\par
Ma mère étrennait un châle vert qui criait comme un damné à côté de la robe de mousseline fraîche à pois roses, qui faisait brouillard autour de madame Brignolin.\par
\bigbreak
\noindent On m’avait tracé mon programme. Je devais déjeuner avec des haricots à l’huile, aller en retenue – puis me rendre chez l’économe, M. Laurier, qui me ferait dîner à sa table.\par
« C’est plus que tu ne mérites », m’avait dit ma mère.\par
Cette perspective était assez flatteuse pour que le regret de ne point aller à la maison de campagne ne fût pas trop grand ; et j’acceptai mon sort de bon cœur.\par
Je mangeai les haricots à l’huile, – j’allai jouer aux billes avec des petits ramoneurs que je connaissais. – J’arrivai à la retenue en retard et couvert de suie, – je trouvai moyen, sous prétexte de besoins urgents, d’aller flâner dans le gymnase, où je décrochai un trapèze et faillis me casser les reins ; je bâclai mon pensum, bus un peu d’encre, et six heures arrivèrent.\par
La retenue était finie, on nous lâcha, je montai chez M. Laurier.\par
« Te voilà, gamin ?\par
— Oui, m’sieu.\par
— Toujours en retenue, donc !\par
— Non, m’sieu !\par
— Tu as faim ?\par
— Oui, m’sieu !\par
— Tu veux manger ?\par
— Non, m’sieu ! »\par
Je croyais plus poli de dire \emph{non} : ma mère m’avait bien recommandé de ne pas accepter tout de suite, ça ne se faisait pas dans le monde. On ne va pas se jeter sur l’invitation comme un goulu, « tu entends » ; et elle prêchait d’exemple. Nous avions dîné quelquefois chez des parents d’élèves.\par
« Voulez-vous de la soupe, madame ?\par
— Non, si, comme cela, très peu…\par
— Vous n’aimez pas le potage ?\par
— Oh ! si, je l’aime bien, mais je n’ai pas faim…\par
— Diable ! pas faim, déjà ! »\par
\bigbreak
\noindent « Tu dois toujours en laisser un peu dans le fond. » Encore une recommandation qu’elle m’avait faite.\par
En laisser un peu dans le fond.\par
C’est ce que je fis pour le potage, au grand étonnement de l’économe, qui avait déjà trouvé que j’étais très bête en disant que j’avais faim, mais que je ne voulais pas manger.\par
Mais moi, je sais qu’on doit obéir à sa mère – elle connaît les belles manières, ma mère, – j’en laisse dans le fond, et je me fais prier.\par
L’économe m’offre du poisson. – Ah ! mais non !\par
Je ne mange pas du poisson comme cela du premier coup, comme un paysan.\par
« Tu veux de la carpe ?\par
— Non, M’sieu !\par
— Tu ne l’aimes pas ?\par
— Si, M’sieu ! »\par
Ma mère m’avait bien recommandé de tout aimer chez les autres ; on avait l’air de faire fi des gens qui vous invitent, si on n’aimait pas ce qu’ils vous servaient.\par
« Tu l’aimes ? eh bien ! »\par
L’économe me jette de la carpe comme à un niais, qui y goûtera s’il veut, qui la laissera s’il ne veut pas.\par
Je mange ma carpe – difficilement.\par
Ma mère m’avait dit encore : « Il faut se tenir écarté de la table ; il ne faut pas avoir l’air d’être chez soi, de prendre ses aises. » Je m’arrangeais le plus mal possible, – ma chaise à une lieue de mon assiette ; je faillis tomber deux ou trois fois.\par
\bigbreak
\noindent J’ai fini mon pain !\par
Ma mère m’a dit qu’il ne fallait jamais « demander », les enfants doivent attendre qu’on les serve.\par
J’attends ! mais M. Laurier ne s’occupe plus de moi – il m’a lâché, et il mange, la tête dans un journal.\par
Je fais des petits bruits de fourchette, et je heurte mes dents comme une tête mécanique. Ce cliquetis à la Galopeau, à la Fattet, le décide enfin à jeter un regard, à couler un œil par-dessous \emph{Le Censeur de Lyon}, mais il voit encore de la carpe dans mon assiette, avec beaucoup de sauce. J’ai le cœur qui se soulève, de manger cela sans pain, mais je n’ose pas en demander !\par
Du pain, du pain !\par
J’ai les mains comme un allumeur de réverbères, je n’ose pas m’essuyer trop souvent à la serviette. « On a l’air d’avoir les doigts trop sales, m’a dit ma mère, et cela ferait mauvais effet de voir une serviette toute tachée quand on desservira la table. »\par
Je m’essuie sur mon pantalon par derrière, – geste qui déconcerte l’économe quand il le surprend du coin de l’œil. – Il ne sait que penser !\par
« Ça te démange ?\par
— Non, m’sieu !\par
— Pourquoi te grattes-tu ?\par
— Je ne sais pas. »\par
Cette insouciance, ces réponses de rêveur et ce fatalisme mystique finissent, je le vois bien, par lui inspirer une insurmontable répulsion.\par
« Tu as fini ton poisson ?\par
— Oui, m’sieu ! »\par
M. Laurier m’ôte mon assiette et m’en glisse une autre avec du ris de veau et de la sauce aux champignons.\par
« Mange, voyons, ne te gêne pas, mange à ta faim. »\par
Ah ! puisque le maître de la maison me le recommande ! et je me jette sur le ris de veau.\par
Pas de pain ! pas de pain !\par
Le veau et le poisson se rencontrent dans mon estomac sur une mer de sauce et se livrent un combat acharné.\par
Il me semble que j’ai un navire dans l’intérieur, un navire de beurre qui fond, et j’ai la bouche comme si j’avais mangé un pot de pommade à six sous la livre !\par
Le dîner est fini : il était temps ! M. Laurier me renvoie, non sans mettre son binocle pour regarder les dessins dont j’ai tigré mon pantalon bleu ; le repas finit en queue de léopard.\par
\emph{7 heures et demie}.\par
Je suis étendu tout habillé sur mon lit ; un bout de lune perce les vitres ; pas un bruit !\par
J’ai la tête qui me brûle, et il me semble qu’on m’a cassé le crâne d’un côté.\par
Je me souviens de tout : du pain qui manquait, du poisson qui nageait, du veau qui tétait…\par
Ça ne fait rien ; je puis me rendre cette justice, que j’ai au moins conservé les belles manières. J’ai souffert, mais je suis resté loin de la table, je n’ai pas eu l’air de mendier mon pain ; j’ai été fidèle aux leçons de ma mère.\par
\bigbreak
\noindent \emph{9 heures}.\par
Deux heures de sommeil ; le mal de tête est parti. Si je voyais un veau dans la chambre, je sauterais par la fenêtre ; mais ce n’est pas probable, et je rêvasse en me déshabillant.\par
\bigbreak
\noindent \emph{10 heures}.\par
J’avais allumé la chandelle, et je lisais ; mais la chandelle va finir, il n’en reste plus qu’un bout pour mes parents quand ils rentreront.\par
Je monte dans ma soupente. Je couche dans une soupente à laquelle on arrive par une petite échelle ; on y étouffe en été, on y gèle en hiver ; mais j’y suis libre, tout seul, et je l’aime, ce cabinet suspendu, où je peux m’isoler, dont les murs de bois ont entendu tous les murmures de mes colères et de mes douleurs.\par
\bigbreak
\noindent \emph{Minuit}.\par
Je m’étais assoupi ! – Je me suis réveillé brusquement !\par
Un bruit confus, des cris déchirants, – un surtout qui m’entre au cœur et me le fend comme un coup de couteau. C’est la voix de ma mère…\par
Je saute au bas de l’échelle, en chemise ; l’échelle n’était pas accrochée et je tombe avec fracas. Je me suis presque fendu le genou sur le carreau.\par
C’est dans l’escalier que le drame se passe ; entre ma mère qui est renversée sur la rampe, les yeux hagards, et mon père qui la tire à lui, pâle, échevelée.\par
Je me jette en pleurant au milieu d’eux. Qu’y a-t-il ?\par
Je veux crier.\par
« Non, non ! fait mon père en me fermant la bouche, non ! » – Il me brise presque les dents sous son poing. – « Non, non ! » – Il y a autant de colère que de terreur dans sa voix.\par
Je me penche sur ma mère évanouie ; j’inonde sa face de mes larmes. C’est bon, il parait, des larmes d’enfant qui tombent sur les fronts des mères ! La mienne ouvre tout d’un coup les yeux, et me reconnaît, elle dit : « Jacques ! Jacques ! » – Elle prend ma main dans sa main, et elle la presse. C’est la première fois de sa vie.\par
Je ne connaissais que le calus de ses doigts, l’acier de ses yeux et le vinaigre de sa voix ; en ce moment, elle eut une minute d’abandon, un accès de tendresse, une faiblesse d’âme, elle laissa aller doucement sa main et son cœur.\par
Je sentis à ce mouvement de bonté que lui arrachait l’effroi dans cet instant suprême, je sentis que tous les gestes bons auraient eu raison de moi dans la vie.\par
« Retourne te coucher », m’a dit mon père.\par
J’y retourne glacé, j’ai attrapé froid sur les dalles de l’escalier, puis dans la grande chambre, avec les fenêtres ouvertes pour que la malade eût de l’air !\par
\bigbreak
\noindent Qu’est-il donc arrivé ?\par
Mon cœur aussi a son orage, et je ne puis assembler deux pensées, réfléchir dans ma fièvre ! Les heures tombent une à une.\par
Je regarde mourir la nuit, arriver le matin ; une espèce de fumée blanche monte à l’horizon.\par
J’ai vu, comme un assassin, passer seules en face de moi les heures sombres ; j’ai tenu les yeux ouverts tandis que les autres enfants dorment ; j’ai suivi dans le ciel la lune ronde et sans regard comme une tête de fou ; j’ai entendu mon cœur d’innocent qui battait au-dessus de cette chambre silencieuse. Il a passé un courant de vieillesse sur ma vie, il a neigé sur moi. Je sens qu’il est tombé du malheur sur nos têtes !\par
\bigbreak
\noindent Qu’est-il arrivé ? Je voudrais le savoir.\par
J’ai connu souvent des situations douloureuses ; mais je n’ai jamais tremblé comme je tremblais ce jour-là, quand je me demandais comment on allait m’accueillir, de quel œil me regarderait mon père qui avait dit si pâle : « Non, non, n’appelle pas ! »\par
J’avais peur qu’ils eussent honte devant moi.\par
Je cherchais quel visage il fallait qu’eût leur fils, quels mots je devais dire, s’il ne serait pas bon d’aller les embrasser. – Mais par qui commencer ?\par
Et je frissonnais de tous mes membres… chose bizarre, – plus effrayé d’être gauche, d’avancer, ou de pleurer à faux, qu’effrayé du drame inconnu dont je ne savais pas le secret.\par
C’est ainsi quand on n’est point sûr du cœur des siens et qu’on craint de les irriter par les explosions de sa tendresse ; instinctivement, on sent qu’il ne faut pas à ces douleurs un accueil cruel, le cœur ne saurait l’oublier et il garderait, noire ou rouge, une tache ou une plaie, une tristesse ou une colère.\par
Aussi on hésite, on recule !\par
Ne rien dire ? – mais ils peuvent vous accuser d’être méchant, puisque vous ne semblez pas ému de leur douleur ! – Parler ? Mais ils vous en voudront de ce que vous avez souligné leur faute ou leur crime, de ce que vous avez, le matin, réveillé par vos larmes, – vos \emph{simagrées} – des fantômes qui devaient mourir avec le dernier cri, le premier soleil !\par
Et je ne savais que faire !\par
\bigbreak
\noindent Il y avait longtemps que c’était le matin. – Mon père se levait d’ordinaire à sept heures afin d’être prêt pour la classe de huit heures. Je me levais aussi.\par
Je fis comme toujours ; je m’habillai, mais lentement, et ne mis pas mes souliers ; j’attendis assis sur mon lit.\par
Il ne venait aucun bruit de leur chambre ; un silence de mort.\par
Enfin, au quart avant huit heures mon père m’appela.\par
Il ne parut point étonné de me trouver tout prêt ; à travers la porte il me demanda du papier et de l’encre ; écrivit une lettre au censeur et une autre à un médecin, et me chargea de les porter.\par
« Tu reviendras dès que tu les auras remises.\par
— Je n’irai pas en classe ?\par
— Non, il faut soigner ta mère malade. Si le censeur te demande ce qu’elle a, tu lui diras qu’elle a été prise de frayeur dans la campagne, et qu’elle est au lit avec la fièvre… »\par
Il disait cela sans paraître trop ému, avec un peu de vulgarité dans la tournure, – il traînait ses pantoufles sur le parquet et rajustait son pantalon.\par
Que s’était-il passé ?\par
Je ne l’ai jamais bien su. À des cris qui échappèrent dans les orages, à des éclats de querelles que mes oreilles recueillirent, je crus comprendre que ma mère s’était mise en embuscade et avait surpris madame Brignolin causant bas avec mon père au détour du jardin, dans ce dimanche de malheur !\par
Il s’en était suivi une scène de jalousie et de bataille, il paraît, et qui s’était continuée jusqu’au milieu de la nuit, jusqu’à l’heure où je les avais vus revenir.\par
Je ne pouvais questionner personne ; d’ailleurs, le souvenir seul de ce moment m’obsédait comme un mal, et je le chassais au lieu d’essayer de le savoir !\par
Savoir quoi ? Ce qui était fait était fait !\par
\bigbreak
\noindent Je suis peut-être le plus atteint, moi, l’innocent, le jeune, l’enfant !\par
Mon père, depuis ce jour-là (est-ce la fièvre ou le remords, la honte ou le regret ?), mon père a changé pour moi. Il avait jusqu’ici vécu en dehors du foyer, par la raison ou sous le prétexte qu’il avait à donner des répétitions au collège et à assister à quelques conférences que faisait le professeur de rhétorique, pour les maîtres qui n’étaient pas agrégés.\par
Il reste à la maison, maintenant, quatre fois sur six ; il y reste, le sourcil froncé, le regard dur, les lèvres serrées, morne et pâle, et un rien le fait éclater et devenir cruel.\par
Il parle à ma mère d’une voix blanche, qui soupire ou siffle ; on sent qu’il cherche à paraître bon et qu’il souffre ; il lui montre une politesse qui fait mal et une tendresse fausse qui fait pitié.\par
Il a le cœur ulcéré, je le vois.\par
Oh ! la maison est horrible ! et l’on marche à pas lents, et l’on parle à voix basse.\par
Je vis dans ce silence et je respire cet air chargé de tristesse.\par
Quelquefois, je trouble cette paix de mes cris.\par
Mon père a besoin de rejeter sur quelqu’un sa peine et il fait passer sur moi son chagrin, sa colère. Ma mère m’a lâché, mon père m’empoigne.\par
Il me sangle à coups de cravache, il me rosse à coups de canne sous le moindre prétexte, sans que je m’y attende : bien souvent, je le jure, sans que je le mérite.\par
J’ai gardé longtemps un bout de jonc qu’on me cassa sur les côtes et auquel j’avais machinalement emmanché une lame, je m’étais dit que si jamais je me tuais, je me tuerais avec cela. – Et j’ai eu l’idée de me tuer une fois !\par
Voici à quelle occasion.\par
\bigbreak
\noindent Mon père rentre brusque et pâle, et me prenant par le bras qu’il faillit casser :\par
« Gredin ! dit-il entre ses dents, je vais te laisser pour mort sur le carreau ! »\par
J’entrevis un supplice – et justement, j’étais à peine guéri d’une dernière correction qui m’avait rompu les membres.\par
Il prétendit que chez le proviseur, au moment où l’on traitait la question des boursiers et des non payants, quand on était arrivé à mon nom, le proviseur, s’avançant, lui avait dit :\par
« M. Vingtras, votre fils pourrait tenir dans la classe un autre rang que celui qu’il tient, s’il travaillait. Nous vous conseillons de vous occuper de lui… entendez-vous ?\par
— C’est toi, misérable, qui me fais avoir des reproches du proviseur ? » et il se jeta sur moi avec fureur.\par
Ce furent de véritables souffrances, – mais mon chagrin était bien plus grand que mon mal !\par
Quoi ! j’étais pour quelque chose dans son avenir, je serais cause qu’on le déplacerait par disgrâce, ou peut-être qu’on le destituerait ! Je me donnai sur la poitrine, en \emph{mea culpa}, des coups plus forts que ceux de ses poings fermés, et le me serais peut-être tué, tant j’étais désespéré, si je n’avais pensé à réparer le mal que mon père m’accusait d’avoir fait.\par
Je me mis à travailler bien fort, bien fort ; on ne me punissait plus au collège, mais à la maison on me battait tout de même.\par
J’aurais été un ange qu’on m’aurait rossé aussi bien en m’arrachant les plumes des ailes car j’avais résolu de me raidir contre le supplice, et comme je dévorais mes larmes et cachais mes douleurs, la fureur de mon père allait jusqu’à l’écume.\par
Deux ou trois fois, je dus pousser des cris comme en poussent ceux qu’on tue en leur arrachant l’âme : il en fut épouvanté lui-même ! mais il recommençait toujours, tant il avait la pensée malade, l’esprit noir. – Il croyait vraiment que j’étais un gredin, je le pense. – Il voyait tout à travers le dégoût ou la fureur !\par
Quelquefois, c’est plus affreux encore, – ma mère intervient ; – et elle qui m’a calotté à outrance, accuse mon père de barbarie !\par
« Tu ne toucheras pas cet enfant ! »\par
\bigbreak
\noindent De temps en temps ils se raccommodent et me battent tous deux à la fois ! Les raccommodements durent peu.\par
Je suis bien malheureux, mais j’ai toujours à cœur le reproche sanglant de mon père, et je me dis que je dois expier ma faute, en courbant la tête sous les coups et en \emph{bûchant} pour que sa situation universitaire, déjà compromise, ne souffre pas encore de ma paresse !\par
Je fais tout ce que je peux ; je me couche quelquefois à minuit, et même ma mère, qui jadis m’accusait de dormir trop tôt, m’accuse maintenant de brûler trop de chandelle : « Et pour quoi faire ? Des singeries, tout ça. »\par
Mon père prétend que je lis des romans en cachette, on ne me sait pas gré du mal que je me donne, et c’est à peine si l’on paraît content de ce que j’ai de bonnes places, car j’ai repris la tête et je suis le premier de la classe.\par
Pour arriver à cela, quelles heures ennuyeuses j’ai passées !\par
Ce \emph{Gradus ad Parnassum}\footnote{Dictionnaire utilisé pour écrire des vers latins.} où je cherche les épithètes de qualité, et les brèves et les longues, ce sale bouquin me fait horreur !\par
Mon \emph{Alexandre}\footnote{Dictionnaire de grec.} a les coins mangés ; c’est moi qui les ai mordus de rage et j’ai de son cuir dans l’estomac.\par
Tout ce latin, ce grec, me paraît baroque et barbare ; je m’en bourre, je l’avale comme de la boue.\par
Je ne cause pas, je ne bavarde plus ; on m’aimait davantage avant, et j’entends qu’on dit par derrière :\par
« C’est parce que son père lui donne des danses. »\par
On dit aussi :\par
« Ne trouvez-vous pas qu’il est devenu sournois et qu’il a l’air sainte-nitouche ? »\par
\bigbreak
\noindent J’ai été premier en je ne sais plus quoi, et le premier porte les compositions au proviseur ; mais il est en conversation particulière avec quelqu’un et l’on me dit d’attendre dans le cabinet voisin. – celui d’où l’on entend tout.\par
On parlait de nous.\par
« Nous ne disons rien de l’affaire Vingtras, c’est entendu ?\par
— Non rien ; ce serait lui faire du tort pour toute sa vie dans l’Université, et puis, vous savez, j’aurais été à sa place, avec une femme comme celle qu’il a…\par
— Il est de fait ! et toujours à vous parler des cochons qu’elle a gardés, des bourrées qu’elle a dansées. – Youp, la, la ! – tandis que madame Brignolin, eh ! eh !\par
— Plus bas, dit le proviseur, si ma femme entendait ! »\par
J’eus peur dans mon cabinet. Je me les figurais allant à la porte, l’entrouvrant pour voir s’il y avait des oreilles.\par
C’était le proviseur et l’inspecteur d’académie : j’avais reconnu leur voix. Ils reprirent :\par
« Je me suis contenté de lui donner un avertissement une fois. J’ai pris le prétexte de son fils.\par
— Qu’est-ce que c’est que ce garçon-là ?\par
— Un pauvre petit malheureux qu’on habille comme un singe, qu’on bat comme un tapis, pas bête, bon cœur. Il a plu beaucoup à l’inspecteur, la dernière fois… Je l’ai donc pris pour prétexte. “Occupez-vous plus de votre fils” ; cela voulait dire : “Restez un peu plus avec votre femme”, – et il a tenu compte de l’observation. »\par
\bigbreak
\noindent Je restai rêveur toute la journée du lendemain…\par
Mon père s’en fâcha, et me bousculant avec un geste de colère :\par
« Vas-tu retomber dans tes rêvasseries, fainéant ? L’inspecteur doit arriver dans quelque temps, il ne s’agit pas de me faire honte, comme l’an passé, et de nous faire souffrir tous de ta paresse ! »\par
Quelle honte ? quelle paresse ?\par
Mon père m’avait menti.
\section[{17. Souvenirs}]{17. \\
Souvenirs}\renewcommand{\leftmark}{17. \\
Souvenirs}

\noindent M. Laurier, l’économe, qui a passé dans un collège de première classe du côté de l’Ouest, a entendu dire qu’une place est vacante à Nantes. La chaire d’un professeur de grammaire est vide. Il s’est démené pour que mon père l’obtînt.\par
\bigbreak
\noindent La nomination arrive.\par
\bigbreak
\noindent Nous allons quitter Saint-Étienne. Je viens de ranger les cahiers d’agrégation de mon père : les thèmes grecs ici, les versions latines par-là ; il y en a des tas.\par
Mes parents vont faire leurs adieux.\par
Ils sortent, je les vois qui descendent la rue sans se parler.\par
Instinctivement, près du passage Kléber, ils se détournent et prennent la gauche du chemin, pour éviter la maison où madame Brignolin demeure…\par
\bigbreak
\noindent J’enfile du regard cette rue qui d’un côté mène au collège, de l’autre à la place Marengo ; qui me rappelle le plaisir, la peine, les longues heures d’ennui et les minutes de bonheur.\par
Ah ! j’ai grandi maintenant ; je ne suis plus l’enfant qui arrivait du Puy tout craintif et tout simple. Je n’avais lu que le catéchisme et je croyais aux revenants. Je n’avais peur que de ce que je ne voyais pas, du bon Dieu, du diable ; j’ai peur aujourd’hui de ce que je vois ; peur des maîtres méchants, des mères jalouses et des pères désespérés. J’ai touché la vie de mes doigts pleins d’encre. J’ai eu à pleurer sous des coups injustes et à rire des sottises et des mensonges que les grandes personnes disaient.\par
Je n’ai plus l’innocence d’autrefois. Je doute de la bonté du ciel et des commandements de l’Église. Je sais que les mères promettent et ne tiennent pas toujours.\par
À l’instant, en rôdant dans cet appartement où traînent les meubles comme les décors d’un drame qu’on démonte, j’ai vu les débris de la tirelire où ma mère mettait l’argent pour m’acheter un homme et qu’elle vient de casser.\par
Est-ce le silence, l’effet de la tristesse qui m’envahira toujours plus tard, quand j’aurai quitté un lieu où j’ai vécu, même un coin de prison ?\par
Est-ce l’odeur qui monte de toutes ces choses entassées ? Je l’ignore ; mais tous mes souvenirs se ramassent au moment de partir.\par
\bigbreak
\noindent Voici, dans ce coin, un bout de ruban bleu.\par
C’était à ma cousine Marianne. On l’avait fait venir de Farreyrolles sous prétexte qu’elle était née avec des manières de dame, et qu’un séjour de quelque temps dans notre famille ne pouvait manquer de lui donner le vernis et la tournure qu’on gagne dans la compagnie des gens d’éducation et de goût.\par
\bigbreak
\noindent Pauvre cousine Marianne !\par
On en fit une domestique, qu’on maltraitait tout comme moi, – moins les coups.\par
Nous étions ensemble dans la cuisine, – je faisais le \emph{gros} – un homme doit savoir tout faire. Je grattais le fond des chaudrons, elle en faisait reluire le ventre. Pour les assiettes, c’est moi qui raclais le ventre, c’est elle qui essuyait le fond : c’était la consigne. Ma mère avait fait remarquer avec conviction que ce qui est sale dans les chaudrons, c’est le dessous ; que ce qui est sale dans les assiettes, c’est le dessus. Et voilà pourquoi je faisais le\emph{ gros}.\par
On l’a obligée aussi à garder son petit bonnet de campagne. Elle en était toute fière à Farreyrolles et savait que les gars disaient qu’elle le portait bien. Mais elle sentait qu’à Saint-Étienne cela faisait rire. On détournait la tête, on la regardait avec curiosité.\par
Ma mère de dire :\par
« C’est que je l’aime comme mon fils, voyez-vous ! Je ne fais pas de différence entre eux deux. » Et elle ajoutait : « Jacques pourrait presque s’en fâcher. »\par
Oui, je me fâche, et je voudrais qu’on fît une différence ; c’est bien assez qu’on m’ait ennuyé comme on l’a fait, sans qu’on l’ennuie aussi.\par
M. Laurier lui-même a fait observer que ce n’était point de mise à la ville ; ma mère a répondu :\par
« Croyez-vous donc que je rougisse de mon origine ? Voulez-vous que j’aie l’air d’être honteuse de mes sœurs et de ne pas oser sortir avec ma nièce parce qu’elle a un bonnet de campagne ?… Ah ! vous me connaissez mal, M. Laurier. »\par
\bigbreak
\noindent Un jour cependant elle crut avoir assez brisé la volonté de sa nièce et assez prouvé qu’elle ne rougissait pas de son origine ; elle supprima la coiffe ; mais elle \emph{dicta} un bonnet, coupa elle-même une robe.\par
« Je ne sortirai jamais habillée comme ça, dit Marianne le jour où on les essaya.\par
— Tu entends par là que ta tante n’a pas de goût, que ta tante est une bête, qui ne sait pas comment on s’habille, qui souillonne ce qu’elle touche. Ah ! je souillonne ?…\par
— Je n’ai pas dit ça, ma tante.\par
— Et hypocrite avec ça ! – Oui va-t’en dire partout que je souillonne les robes de mes nièces. – Tu ajouteras peut-être aussi que je les laisse mourir de faim ! »\par
Une pause.\par
Tout d’un coup se tournant vers moi, d’une voix qui était vraiment celle du sang, dans laquelle on sentait mourir la tante et ressusciter la mère :\par
« Jacques, fit-elle, mon fils, viens embrasser ta mère… »\par
Tant d’amour, de tendresse, cette explosion, ce cœur qui tout d’un coup battait au-dessus du sein qui m’avait porté, tout cela me troubla beaucoup et je m’avançai comme si j’avais marché dans de la colle.\par
« Tu ne viens pas embrasser ta mère ! » s’écria-t-elle attristée de ce retard en levant les mains au ciel.\par
Je pressai le pas, – elle m’attira par les cheveux et elle me donna un baiser à ressort qui me rejeta contre le mur où mon crâne enfonça un clou !\par
Oh ! ces mères ! quand la tendresse les prend ! Ça ne fait rien, le clou m’a fait une mâchure.\par
Ces mères qu’on croit cruelles et qui ont besoin tout d’un coup d’embrasser leur petit !\par
Quel coup ! j’ai mal pourtant ! et je me frotte l’occiput.\par
« Jacques ! veux-tu ne pas te gratter comme ça ! Ah ! tu sais, j’ai regardé le fond du grand chaudron, tout à l’heure : – tu appelles ça nettoyer, mon garçon, tu te trompes. Il y a deux jours qu’on n’y a pas touché, je parie !\par
— Ce matin, maman !\par
— Ce matin ! tu oses !…\par
— Je t’assure.\par
— Allons, c’est moi qui ai tort, c’est ta mère qui ment.\par
— Non ! m’man.\par
— Viens que je te gifle ! »\par
\bigbreak
\noindent Chère Marianne, depuis ce jour-là, elle fut bien malheureuse. Elle écrivit à sa mère qui l’aimait bien, et lui demanda de retourner tout de suite au village.\par
Mais à la lettre qui vint de Farreyrolles, ma mère répliqua :\par
« Veux-tu donner raison à ta fille contre moi ? Crois-tu ta sœur une menteuse ? Crois-tu, comme elle l’a dit, que je souillonne ! Crois-tu ?… – Si tu le crois, – c’est bien ! »\par
C’est moi qui mis les virgules et les pluriels.\par
On n’osa pas reprendre Marianne tout de suite, et elle resta un mois encore.\par
Elle souffrit beaucoup pendant ce mois-là, mais moi, comme je fus heureux !\par
Elle était blonde, avec de grands yeux bleus toujours humides, un peu froids, qui avaient l’air de baigner dans l’eau. – Ses cheveux étaient presque couleur de chanvre, et ses joues étaient saupoudrées de rousseurs ; mais la peau du cou était blanche, tendre et fine comme du lait caillé.\par
Je l’ai revue, longtemps après, dans le fond d’un couvent, à travers une grille : elle s’était faite religieuse.\par
« Si j’étais restée plus longtemps à Saint-Étienne, murmura-t-elle en baissant les paupières, je ne serais peut-être jamais venue ici.\par
— Le regrettez-vous ? »\par
Elle éloigna du guichet sa tête pâle encadrée dans la grande coiffe blanche des sœurs de Charité et ne répondit rien, mais je crus voir deux larmes tomber de ses yeux clairs, et il me sembla reconnaître un geste de regret et de tendresse…\par
Elle disparut dans le silence du couloir muet qu’ornait un Christ d’ivoire taché de sang.\par
\bigbreak
\noindent Voilà le pupitre noir devant lequel je m’asseyais, qui était si haut ; il fallait mettre des livres sur ma chaise.\par
Quelles soirées tristes et maussades j’ai passées là, et quelles mauvaises matinées de dimanche, quand on exigeai que j’eusse fait dix vers ou appris trois pages avant de mettre ma chemise blanche et mes beaux habits !\par
Mon père m’a souvent cogné la tête contre l’angle, quand je regardais le ciel par la fenêtre au lieu de regarder dans les livres. Je ne l’entendais pas venir, tant j’étais perdu dans mon rêve, et il m’appelait « fainéant », en me frottant le nez contre le bois.\par
C’est sensible, le nez ! On ne sait pas comme c’est sensible.\par
J’avais fait un jour une entaille dans ce pupitre. Il m’en est resté une cicatrice à la figure, d’un coup de règle qu’il me donna pour me punir.\par
\bigbreak
\noindent Voilà, plein de vieille vaisselle, un panier rongé !\par
C’était là que dormait Myrza, la petite chienne que l’ancien censeur, envoyé en disgrâce, nous avait donnée pour en avoir soin. Il n’avait pas d’argent pour l’emmener avec lui ; puis il ne savait pas si, dans le trou où on l’enterrait, il aurait seulement du pain pour sa femme et son enfant.\par
Myrza mourut en faisant ses petits, et l’on m’a appelé imbécile, grand niais, quand, devant la petite bête morte, j’éclatai en sanglots, sans oser toucher son corps froid et descendre le panier en bas comme un cercueil !\par
J’avais demandé qu’on attendît le soir pour aller l’enterrer. Un camarade m’avait promis un coin de son jardin.\par
Il me fallut la prendre et l’emporter devant ma mère, qui ricanait. Bousculé par mon père, je faillis rouler avec elle dans l’escalier. Arrivé en bas, je détournai la tête pour vider le panier sur le tas d’ordures, devant la porte de cette maison maudite. Je l’entendis tomber avec un bruit mou, et je me sauvai en criant :\par
« Mais puisqu’on pouvait l’enterrer ! » C’était une idée d’enfant, qu’elle n’eût point la tête entaillée par la pelle du boueux ou qu’elle ne vidât pas ses entrailles sous les roues d’un camion ! Je la vis longtemps ainsi, guillotinée et éventrée, au lieu d’avoir une petite place sous la terre où j’aurais su qu’il y avait un être qui m’avait aimé, qui me léchait les mains quand elles étaient bleues et gonflées, et regardait d’un œil où je croyais voir des larmes son jeune maître qui essuyait les siennes…
\section[{18. Le départ}]{18. \\
Le départ}\renewcommand{\leftmark}{18. \\
Le départ}

\noindent Quelle joie de partir, d’aller loin !\par
Puis, Nantes, c’est la mer ! – Je verrai les grands vaisseaux, les officiers de marine, la vigie, les hommes de quart, je pourrai contempler des tempêtes !\par
J’entrevois déjà le phare, le clignotement de son œil sanglant et j’entends le canon d’alarme lancer son soupir de bronze dans les désespoirs des naufrages.\par
J’ai lu \emph{la France maritime}, ses récits d’abordages, ses histoires de radeau, ses prises de baleine, et, n’ayant pu être marin, par la catastrophe Vidaljan, je me suis rejeté dans les livres, où tourbillonnent les oiseaux de l’Océan.\par
J’ai déjà fait des narrations de sinistres comme si j’en avais été un des héros, et je crois même que les phrases que je viens d’écrire sont des réminiscences de bouquins que j’ai lus, ou des compositions que j’ai esquissées dans le silence du cachot.\par
Désespoirs des naufrages, soupirs de bronze, tourbillonnage des oiseaux ; il me semble bien que c’est de Fulgence Girard, mon \emph{tempêtard} favori. Je me répète ces grands mots comme un perroquet enchaîné au grand mât ; mais au fond de moi-même il y a l’espérance du galérien qui pense s’évader cette fois.\par
À Nantes, je pourrai m’échapper quand je voudrai.\par
En face de \emph{la grande tasse} ! on se laisse glisser et l’on est dans l’Océan.\par
Je n’appartiens plus à mon père ; je me cache dans la sainte-barbe, je me fourre dans la gueule d’un canon, et quand on s’aperçoit de ma disparition, je suis en pleine mer.\par
Le capitaine a juré, sacré – mille sabords du diable ! – en me voyant sortir de ma cachette et m’offrir comme novice, mais il ne peut pas me jeter par-dessus bord ; je suis de l’équipage !\par
\bigbreak
\noindent Le voyage actuel, en attendant l’évasion par eau salée, est déjà plein de poésie.\par
Nous avons d’abord la diligence, – l’impériale, – puis nous entrons dans une gare !\par
Les machines renâclent comme des ânes, ou beuglent comme des bœufs, et jettent du feu par les naseaux. Il y a des coups de sifflet qui fendent l’âme !\par

\labelblock{ORLÉANS}

\noindent Nous arrivons à Orléans la nuit.\par
Les malles sont laissées à la gare.\par
« Mais il y a des choses qu’il faut garder avec soi », dit ma mère. Et elle a gardé beaucoup de choses ; on les entasse sur moi, j’ai l’air d’une boutique de marchand de paniers, et je marche avec difficulté.\par
Il s’écroule toujours quelque boîte qu’on ramasse aux clartés de la lune.\par
On ne se décide à rien : on est porté, par l’heure et le calme immense, à une espèce de recueillement très fatigant pour moi qui ai tout sur le dos.\par
Il y a bien eu des facteurs et des garçons d’hôtel qui, à la gare, ont voulu nous emmener au Lion-d’Or, au Cheval-Blanc, au Coq-Hardi. – « À deux pas, monsieur ! – Voici l’omnibus de l’hôtel ! »\par
Aller à l’hôtel, au Cheval-Blanc, au Lion-d’Or, mon cœur en battait d’émoi ; mais mes parents ne sont pas des fous qui vont se livrer comme cela au premier venu et suivre un étranger dans une ville qu’ils ne connaissent pas.\par
Ma mère sait juger son monde, elle a voulu trouver une figure qui lui convînt, et elle rôde, tirant mon père comme un aveugle, hasardant des regards et lançant des questions qui se perdent dans l’obscurité et le brouhaha.\par
Elle a si bien fait, qu’à un moment on s’est trouvé seuls comme un paquet d’orphelins.\par
\bigbreak
\noindent On éteint les lumières. – Il n’est plus resté qu’un réverbère à l’huile devant la grande porte, comme une veilleuse ; et voilà comment nous errons, muets et sans espoir, sur une place à laquelle nous sommes arrivés en nous traînant, ma mère disant à mon père : « C’est ta faute ! » mon père répondant : « C’est trop fort ; est-ce que ce n’est pas toi !\par
— Ah ! par exemple ! »\par
Nous avons hélé des isolés qui passaient par là ; nous avons même cru voir une chaise à porteurs, mais nos cris se sont perdus dans l’espace.\par
La lune est dans son plein – toutes mes nuits qui \emph{datent} l’ont eue jusqu’ici pour témoin.\par
Elle inonde la place de ses rayons, et nous tachons l’espace de notre ombre. C’est même curieux.\par
Je parais énorme avec mon échafaudage biblique, et quand mon père ou ma mère courent après un colis qui est tombé, les ombres s’allongent et se cognent sur le pavé. – Mon père a un nez !\par
Je ne puis pas rire ; – si je riais, je laisserais encore échapper quelque chose ; – puis je n’ai pas grande envie de rire.\par
\bigbreak
\noindent « Quelqu’un là-bas ! »\par
\bigbreak
\noindent Je me tourne comme une paysanne qui porte un seau, comme un jongleur qui attend une boule ; j’ai la tête qui m’entre dans la poitrine, les bras qui me tombent des épaules, j’ai l’air d’un télescope qu’on ferme.\par
« Quelqu’un !\par
— C’est une femme ! Je te dis que c’est une femme !\par
— Sur quoi est-elle montée ?\par
— Sur quoi ?\par
— Oui, sur quoi ? – (Ma mère est aigre, très aigre.)\par
— Hé ! la bonne femme ! »\par
Rien ne bouge que mes colis qui ont failli s’écrouler.\par

\astertri

\noindent « Mes amis, nous nous sommes tous trompés… »\par
La voix de mon père a un accent religieux, des notes graves ; on dirait qu’une larme vient d’en mouiller les cordes.\par
« Tous trompés, reprend-il avec le ton du plus sincère repentir.\par
« Ce que nous avons devant nous n’est pas un homme, n’est pas une femme, c’est la PUCELLE D’ORLÉANS. »\par
Il s’arrête un moment :\par
« Jacques, c’est la \emph{Pucelle ! »}\par
J’ai entendu parler d’elle en classe : la vierge de Domrémy, la bergère de Vaucouleurs !\par
« C’est la Pucelle, Jacques ! »\par
\bigbreak
\noindent Je sens qu’il faut être ému, je ne le suis pas. J’ai trop de paniers, aussi !\par
Ma mère a pris dans le ménage le rôle ingrat ; elle a voulu être mère de famille, selon la Bible, et elle n’a guère eu que le temps de fouetter son enfant et de lui faire des polonaises ; elle connaît de réputation Jeanne d’Arc, mais elle ignore le nom chaste que lui a donné l’Histoire.\par
« Quand tu auras fini de dire des saletés à cet enfant ! »\par
Les bras lui tombent en voyant que mon père me dit des mots qui ne doivent pas se dire, pendant que je porte des bagages à deux heures de la nuit, dans une ville de province, que nous ne connaissons pas…\par
« C’est Jeanne d’Arc, reprend ce père accusé d’être léger devant son enfant, celle qui a sauvé la France !\par
— Oui, répond ma mère d’un air distrait, et elle ajoute d’un air content : on peut s’asseoir contre. »\par
\bigbreak
\noindent Nous avons passé la nuit là ; – c’était un peu dur, mais on avait le dos appuyé.\par
Un sergent de ville qui nous a vus s’est approché.\par
Le sergent de ville nous a pris pour une famille de pèlerins fanatiques, qui étaient venus tomber d’épuisement – avec beaucoup de bagages, par exemple, – aux pieds de leur sainte ; – il ne nous a pas brusqués, mais il nous a dit qu’il fallait partir. Il s’est offert à nous mener dans une auberge tenue par son beau-frère même, au bout de la rue, près du marché.\par
« Tu n’as pas faim ? demande mon père à ma mère pendant le chemin.\par
— Pourquoi aurais-je faim ? »\par
Il faut dire que mon père, dans la soirée, avait parlé de dîner au buffet de Vierzon, de peur de manger trop tard si on ne prenait pas cette précaution. Ma mère s’y était opposée et elle n’entendait pas qu’on eût l’air de jeter un reproche sur sa décision en lui demandant si elle avait faim.\par
Mon père ne souffle mot. – Le sergent de ville coule vers ma mère un regard de terreur.\par
\bigbreak
\noindent Nous sommes dans l’auberge.\par
Elle s’éveillait ; un garçon d’écurie rôdait avec une lanterne, on attelait la carriole d’un paysan. Le sergent de ville appelle son beau-frère, en tapant contre une cloison.\par
Un grognement.\par
« On y va, on y va ! »\par
À travers les fentes, on voit passer une lumière et l’on entend l’homme qui s’habille en bâillant, ses bretelles qui claquent et ses souliers qui traînent.\par
« Ces personnes demandent à coucher et un morceau sur le pouce. »\par
Morceau sur le pouce est dit le visage tourné vers mon père. Il se souvient de ce : « Pourquoi aurais-je faim ? » de ma mère.\par
Mais elle intervient.\par
« Coucher seulement, fit-elle ; nous souperons en nous réveillant.\par
— Comme vous voudrez », fait l’aubergiste, à qui il importe peu de vendre ses fricots le matin ou la nuit, et qui préfère même, une fois les voyageurs couchés, se recoucher aussi.\par
J’entends les boyaux de mon père qui grognent comme un tonnerre sous une voûte : les miens hurlent ; – c’est un échange de borborygmes ; ma mère ne peut empêcher, elle aussi, des glouglous et des bâillements ; mais elle a dit, à la station, qu’il ne fallait pas dîner et l’on ne mangera pas avant demain. On ne \emph{man-ge-ra} pas.\par
Elle a pourtant crié à mon père :\par
« Mange, si tu veux, toi ! »\par
Mon père a simplement branlé la tête ; il a ouvert la bouche comme une carpe, et il a murmuré :\par
« Non, non, demain. »\par
Il sait ce que cela signifie !\par
Cela signifie : Je ne veux pas que tu prennes une miette,\par
que tu grattes un radis, que tu effleures une andouille, que tu respires un fromage ! Mon père va se coucher ; ma mère le suit. On met une paillasse pour moi dans un coin. Je tombe de fatigue et je m’endors ; mes parents en font autant. Mais nous nous réveillons tous les trois, par moments, au bruit que font nos intestins.\par
Ma mère est du concert comme les autres, – mais elle ne cédera pas. – C’est une femme de tête, ma mère. Ah ! je l’admire vraiment ! Quelle volonté ! Quelle différence avec moi ! Si j’avais faim, moi, je le dirais, et même je becquèterais… s’il y avait de quoi !\par
Nature vulgaire, poule mouillée, avorton !\par
Regarde donc ta mère, qui, pour être fidèle à sa parole, s’en tenir à ce qu’elle a dit, passe la nuit à se serrer le ventre, et attend le matin pour casser une croûte. Elle fera encore celle qui mange par habitude, sans appétit, tu verras. – Tu as pour mère une Romaine, Jacques ! tu ne tiens pas d’elle, – surtout par le nez, car tu l’as en pied de marmite.\par
\bigbreak
\noindent Nous avons déjeuné, – ma mère, du bout des dents : mais je l’ai vue qui dévorait, dans un coin, un foie de veau qu’elle avait demandé à la cuisine, et qu’on lui avait enfoui dans du pain ; – elle mordait là-dedans !\par
Mon père a mangé à en éclater, – il en a les oreilles bleues.\par
Il ne s’est pas rebiffé cette nuit, parce qu’il a les mains liées et qu’il a commis au moment du départ une grande imprudence. Il a confié à ma mère tout l’argent.\par
Ma mère avait dit, sans avoir l’air de rien :\par
« Mes poches sont plus grandes que les tiennes, l’argent y tiendra mieux ; c’est moi qui payerai en route. »\par
Mon père n’a pas compris tout de suite l’étendue de son malheur, la gravité de la faute ; mais au premier relais il a senti la blessure. Il ne lui restait plus rien, pas une pièce d’un franc, pas une pièce de deux sous. Il avait vidé sa monnaie dans les mains des gens à pourboires, porteurs du roulage ou facteurs des messageries, et il n’avait pas même de quoi rendre un verre de groseille.\par
Il mourait de soif.\par
« Donne-moi de l’argent.\par
— Tu veux de l’argent ?…\par
— Oui, Jacques a soif… »\par
Ma mère se tourne vers moi.\par
« Tu as soif ? »\par
Ma foi ! Je veux bien soutenir mon père, quand c’est possible ; mais pourquoi, quand il a soif, dit-il que c’est moi ? Je ne réponds rien à la question de ma mère, dont les yeux vont avec une ironie froide de son fils à son époux.\par
« Il peut attendre, bien sûr, dit-elle en se replongeant dans son coin, et ne paraissant pas plus se soucier de mon père que s’il n’existait pas. »\par
\bigbreak
\noindent Cela a duré trois jours, les demandes d’argent et les refus de versement !\par
Mon père s’est fâché ; – il y a même eu scandale, d’abord sur le pas d’une auberge, puis dans un wagon ; et ma mère a eu le dessus : mon père a demandé grâce.\par
C’est qu’elle est courageuse et franche. – Elle dit souvent : « Je suis franche comme l’or. »\par
Et, comme elle est franche, elle reproche tout haut à mon père, devant les hôteliers, devant les voyageurs, d’être un homme sans cœur, un époux sans conduite.\par
Elle conte son histoire, elle dit les noms tout haut.\par
« C’est le regret de quitter ta Brignoline qui te talonne. – Ah ! ah ! – On veut\emph{ s’empiffrer} pour oublier… Monsieur veut peut-être l’argent pour lâcher sa femme et son fils et retourner chez sa maîtresse. »\par
\bigbreak
\noindent Mon père qui a demandé cinq malheureux francs ! Ce n’est pas avec cela !\par
Il est sur des épines, tâche de couper les phrases, de morceler les mots, de détruire l’effet ; mais ma mère est si franche !\par
« Tu ne me feras pas taire, je pense ! Tu n’as pas besoin de me pousser le coude : ce que je dis est vrai, tu le sais bien… Heureusement qu’il y a du monde ; tu ne me frapperas pas devant le monde, peut-être ?… »\par

\labelblock{SUR LE BATEAU}

\noindent Le bateau nous affranchit, – ma mère se trouve malade heureusement.\par
Elle est restée trop longtemps sans manger, elle a avalé le foie de veau trop vite, – elle n’a pas fermé l’œil de la nuit. – Enfin la migraine la prend et l’endort.\par
Mon père reste près d’elle, le temps moral nécessaire pour être sûr qu’elle repose, qu’elle est en plein sommeil, et qu’elle n’a plus la force de fondre sur lui.\par
Il monte sur le pont…\par

\labelblock{UNE RECONNAISSANCE}

\noindent « Chanlaire !\par
— Vingtras ! »\par
Chanlaire est un ancien pion du Puy, qui possède à Nantes un oncle avec lequel il était brouillé pendant le pionnage, mais avec lequel il s’est raccommodé, et chez qui il retourne après un voyage à Paris dans l’intérêt de la maison.\par
Il est heureux, gagne de l’argent.\par
« Quelle rencontre !\par
— Nous allons faire la noce, – votre femme n’est pas avec vous ? »\par
Il pose cette question, comme on manifeste un espoir, et il semble un peu désappointé quand mon père répond, d’un air triste :\par
« En bas, – et d’un air plus gai : malade.\par
— Ce ne sera rien.\par
— Non, – non, – non.\par
— Ça n’empêche pas de décoiffer une bouteille de bourgogne, au contraire… »\par
Se tournant vers moi :\par
« Savez-vous qu’il a grandi, votre gamin ? Quelle tignasse et quels yeux ! – Garçon ! »\par
\bigbreak
\noindent Il y avait des sous-officiers qui allaient en congé, et avaient aussi rencontré des camarades.\par
La table de la cabine est couverte de bouteilles de vin et de cruches de bière.\par
De la gaieté, des rires comme je n’en ai jamais entendu de si francs ! On joue aux cartes, on allume des punchs, on boit des bishofs ; il y a une odeur de citron.\par
Voilà qu’on chante, maintenant !\par
Un fourrier entonne un air de garnison, – tous au refrain !\par
Je m’en mêle, et ma voix criarde se mêle à leurs voix mâles : j’ai bu un petit coup, il faut le dire, dans le verre de mon père, qui a les pommettes roses, les yeux brillants.\par
Il a conté bravement à Chanlaire, – après la troisième tournée, – qu’il a le gousset vide.\par
C’est la bourgeoise qui a le sac !\par
« Voulez-vous vingt francs ? vous me les rendrez à Nantes, nous nous y reverrons, j’espère, et, nous y ferons de bonnes parties… Mais, je dis cela devant le moutard…\par
— Il n’y a pas de danger. »\par
Non, père, il n’y a pas de danger. Ah ! comme il a l’air jeune ! et je ne l’ai jamais vu rire de si bon cœur.\par
Il me parle comme à un grand garçon.\par
« Allons, Jacques, une goutte ! »\par
Puis une idée lui vient :\par
« Si nous cassions une croûte ? Ces pieds de cochon me disent quelque chose ; j’ai envie de leur répondre deux mots. »\par
C’est un langage hardi pour un professeur de septième ; mais le proviseur de Saint-Étienne est loin ; le proviseur de Nantes n’est pas encore là, et les pieds de cochon tendent leurs orteils odorants.\par
Oh ! j’ai encore le goût de la sauce Sainte-Menehould, avec son parfum de ravigote, et le fumet du vin blanc qui l’arrosa !\par
\bigbreak
\noindent On me donne un couvert, comme aux autres, et on me laisse me servir et me verser moi-même. C’est la première fois que je suis camarade avec mon père, et que nous trinquons comme deux amis.\par
Je m’essuie à la serviette, – tant pis ! – je mets ma chaise commodément, – encore tant pis ! – J’ai de mauvaises manières, je suis à mon aise ! on ne me parle ni de mes coudes ni de mes jambes, j’en fais ce que je veux. C’est un quart d’heure de bonheur indicible ! Je ne l’ai pas encore connu ; ma jeunesse s’éveille, ma mère dort.\par
\bigbreak
\noindent … Ma jeunesse s’éteint, ma mère est éveillée !\par
Elle apparaît comme un spectre dans la cabine, – elle était dans celle du fond, nous sommes dans celle du devant, – elle vient droit à nous, et va commencer une scène.\par
Mais bah ! le tapage couvre sa voix. – Les garçons vont et viennent, le cuisinier passe avec ses plats, les sous-officiers rôdent avec des bouteilles sur le cœur ; il y a une farce qui part, une chanson qui éclate, un vacarme, un tohu-bohu ! Sa fureur fait long feu.\par
« Seule de femme », elle est d’avance sûre d’être vaincue ; puis, elle a vu de l’argent dans la main de mon père, qui paye les pieds de cochon.\par
« Oui, nous avons de l’argent, dit mon père guilleret et narquois, et il crie :\par
— Une autre bouteille de ce jaune-là !\par
— Je n’ai pas soif.\par
— Mais, moi, j’ai soif. – Jacques a soif aussi. As-tu soif ? »\par
C’est la riposte joyeuse au trait de la veille ; il y met de la malice, pas de méchanceté, le vin l’a rendu bon.\par
« Et vous, madame ? » fait-il en tendant un verre et la bouteille.\par
Il n’y a pas moyen de se fâcher. Ma mère ne s’y frotte pas et sent que le terrain lui manque. Elle dit sans trop de mauvaise humeur :\par
« Je monte sur le pont. Tu me rejoindras quand tu auras fini. Jacques, viens avec moi.\par
— Non, il reste avec nous ! Nous allons jouer une partie de dominos, il fera le \emph{troisième}. »\par
Faire le troisième, à côté des sous-officiers, sur la même table ; écarter les bouteilles pour placer mon jeu, avec les garçons qui me demandent pardon quand ils me heurtent en passant ! Je ne me tiens pas d’orgueil, et c’est moi, moi le fouetté, le battu, le\emph{ sanglé}, qui suis là, écartant les jambes, ôtant ma cravate, pouvant rire tout haut et salir mes manches !\par
\bigbreak
\noindent La partie de dominos est finie.\par
« Jacques, va dire à ta mère que nous montons. »\par
Nous l’avions oubliée, et j’en ai, dès que le coup de feu de la première émotion est passée, j’en ai un peu de remords. Ma mère m’accueille d’un regard dur et d’un mot menaçant ; mon remords s’en va. Il me semble qu’elle aurait dû deviner que je pensais en ce moment à elle ; qu’il y avait un sentiment tendre qui surnageait au-dessus de mon explosion de gaieté, et je lui en veux de son accueil.\par
« Quand nous serons arrivés, tu me payeras tout ça. »\par
Payer quoi ? un moment de bonheur ? Ai-je donc fait du mal ? J’ai trempé le bout de mes lèvres dans des verres où il y avait de la mousse, et où je voyais danser le soleil. Il faudrait payer cela. – Oh ! je ne le payerai jamais trop cher, et quand je serai arrivé vous pourrez me battre…\par
\bigbreak
\noindent C’est mon jour de chance !\par
Une dame est venue s’asseoir près de nous et la conversation s’est engagée. Mme Vingtras est toujours aux anges quand une femme bien mise lui fait l’honneur de causer avec elle.\par
On parle, et les enfants, qui viennent de temps en temps rire à leur mère, m’entraînent dans leurs jeux.\par
« Jacques, reste là.\par
— Laissez-les s’amuser ensemble, dit avec un air de bonté l’interlocutrice élégante.\par
— Vous n’avez pas peur qu’ils se noient ? »\par
C’est tout ce que ma mère trouve à dire, mais elle est flattée que son fils soit admis dans un jeu d’enfants de riches, et si je me noie, tant pis !\par
Je crois vraiment qu’elle a peur que je me noie ! Quand nous approchons d’un feu, elle a peur que je me brûle. Un jour, un ballon partait dans la cour du collège, elle a crié : « Il va t’emporter ! »\par
Mais elle ne sait donc pas que chaque fois qu’elle a soufflé ou tapé sur ma curiosité, mes envies ont enflé comme ma peau sous le fouet.\par
C’est plus fort que moi. Je me dis que je ne dois pas être plus poltron que les autres, et je cherche toutes les occasions de m’amuser comme mes camarades s’amusent ; ils ne se noient pas, ils ne se brûlent pas, les ballons ne les emportent pas. Et je n’ai jamais raté un \emph{filage} ; je me suis empressé de manquer la classe aussi souvent que j’ai pu, pour filer en bateau sur le Furens, ou près de la forge, dans la grande usine, dont le père de Terrasson est le contremaître.\par
Je suis monté sur le grand arbre du \emph{Clos Pélissier}, et je suis allé jusqu’au bout de la grande branche.\par
Je me rappelle tout cela en ce moment ; j’ai le cerveau un peu émoustillé. Je me figure que je tiens une balance. Si on m’empêche d’aller sur le bord de l’eau, de m’approcher des briqueteries ou des ballons, je ne dirai rien, – je ne veux pas que ma mère ait peur ; – mais, à la première occasion, je me rattraperai, j’entrerai dans la rivière jusqu’à la ceinture, et je mettrai mon pied au-dessus des \emph{coulées} de fer fondu.\par
C’est bien décidé. En attendant, ce soir, comme ma mère m’a laissé libre, je ferai tout pour ne pas me noyer.\par
Si elle m’avait défendu de jouer, je n’aurais pas pu m’empêcher de me pencher sur la roue, de chercher à prendre de l’écume dans le creux de la main…\par
Nous courons d’un bout du bateau à l’autre ; nous hélons le mécanicien, nous tourmentons l’homme du gouvernail, nous touchons aux cordages, nous tâtons le cabestan, nous essayons de soulever l’ancre…\par
La journée fuit, le soir arrive.\par
\bigbreak
\noindent Nous nous laissons prendre comme des hommes par la mélancolie du crépuscule ; les joues froides, avec un frisson dans le cou, nos grands cheveux secoués par le vent, nous regardons le sillon que creuse le bateau dans sa marche, nous fixons les premières étoiles qui tremblent au ciel, et nous suivons dans l’eau moirée les traînées de lune.\par
La machine fait \emph{poum, poum} !\par
\bigbreak
\noindent C’est la cloche qui parle à présent ; nous approchons du pont.\par
Nous voici à Tours : on relâche ici.\par
M. Chanlaire connaît un hôtel, pas cher. Nous irons tous, si l’on veut. C’est entendu. Et, dix minutes après le débarquement, nous arrivons au \emph{Grand-Cerf}.\par
\bigbreak
\noindent Nous dînons à la table d’hôte.\par
Il y a des commis voyageurs, une Anglaise, un prêtre : tout le monde fait honneur à la cuisine, qui sent bon, et une certaine moutarde de Dijon a un succès qui profite à la cave. Son piquant donne soif.\par
J’ouvre des yeux énormes, j’écarte les narines et je dresse les oreilles. Quel luxe ! Combien de réchauds d’argent ! Dix plats ! On bavarde, on dévore.\par
« Passez-moi le civet. – Voulez-vous du saumon ? »\par
Il me semble que je suis à un repas des \emph{Mille et une Nuits}.\par
Je suis profondément étonné de voir que tout le monde foule aux pieds les préceptes que m’a inculqués ma mère sur la façon de se tenir en société. Le curé lui-même a les coudes sur la nappe et sa chaise tout près de la table, comme j’étais, moi aussi, ce matin, dans la cabine, en face du pied de cochon grillé et du petit vin jaune.\par
Ma mère est à côté de la dame de Paris, qui nous a placés à sa droite, ses fils et moi.\par
Je suis presque libre, je tombe sur les plats. Ma mère ne s’en plaint pas, et même elle se fâche à un moment parce que je refuse de quelque chose.\par
« Comme si on voulait le faire mourir de faim ! C’est bien à prix fixe, n’est-ce pas ? demande-t-elle à M. Chanlaire.\par
— Oui, deux francs par tête.\par
— Jacques, crie-t-elle aussitôt, mange de tout ! »\par
C’est jeté comme un cri des croisades, comme une devise de combat : « Mange de tout ! »\par
Cela s’entend par-dessus le bruit des cuillers et des fourchettes, et fait rire tout un coin de table.\par
Elle ne peut s’empêcher de s’occuper de moi, de la place où elle est, et veille toujours sur son enfant.\par
« Jacques, on ne fait pas des tartines de moutarde. – Jacques, tu sais bien que je ne veux pas qu’on suce ses doigts. – Veux-tu bien ne pas faire ce bruit en te mouchant ! – Jacques, tu ne sais pas manger les croupions ! »\par
Je la vois en ce moment qui ramasse en cachette et glisse dans sa poche des provisions qui traînent. On la remarque.\par
J’en deviens rouge.\par
« Jacques, veux-tu bien ne pas rougir comme cela ! »\par
Ah ! elle m’a gâté mon plaisir… Je m’aperçois parfaitement que les voisins se moquent d’elle, et les maîtres de l’hôtel la regardent de travers. Puis j’aurais voulu avoir l’air d’un homme, en redemander aux garçons : « Passez-moi ce plat-là ! » m’essuyer la bouche avec une serviette, en me renversant en arrière, et dire en finissant : « En voilà encore un que les Prussiens n’auront pas. »\par
\bigbreak
\noindent M. Chanlaire se lève :\par
« Mesdames, messieurs et gamins, j’offre du champagne.\par
— Jacques, tu boiras dans mon verre, dit ma mère, du ton dont elle dirait : « On ne m’enlèvera pas mon fils. »\par
— Non, il boira dans le sien, et c’est lui qui aura l’étrenne de cette bouteille, dit M. Chanlaire en pressant le bouchon, qui part comme une balle ; les enfants les premiers ! »\par
Il remplit mon verre, qui déborde, et dit :\par
« Vide-moi ça ! »\par
Ma mère me lance des yeux terribles, et tape de petits coups sur la table, qui veulent dire : regarde-moi donc !\par
Je n’ose la regarder ni boire.\par
« Tu es là comme un empoté, voyons ! »\par
Empoté ! M. Chanlaire dit cela tout haut ; j’en ai le cœur qui se fend, la main qui tremble et je renverse la moitié du champagne sur une robe d’à côté.\par
« Nigaud ! » dit l’inondée…\par
Empoté ! Nigaud ! C’est ma mère qui est cause que j’ai été si bête.\par
Elle me sermonne encore après, en renchérissant sur les autres.\par
Je vais me coucher gonflé et piteux.\par
« Par ici, votre chambre », dit le garçon.\par
Au moment où je suis au bout du corridor, disant adieu à la dame de Paris et à ses fils, qui m’ont fait tout le soir des amitiés, ma mère m’appelle :\par
« Jacques, LES CABINETS SONT EN BAS ! »\par
\bigbreak
\noindent Il y a l’accent du commandement dans la voix – de la sollicitude aussi – elle prend des précautions auxquelles son enfant, avec l’imprudence de son âge, ne songe pas.\par
Mes camarades sourient, leur mère rougit, la mienne salue.\par
Aujourd’hui encore dans mes rêves, dans un salon quelquefois, au milieu de femmes décolletées, à table, dans un bal, j’entends, comme Jeanne d’Arc, une voix : « Jacques ! les cabinets sont en bas ! »\par
\bigbreak
\noindent Le lendemain matin nous reprenons le bateau.\par
La dame de Paris est encore avec ma mère et je suis avec ses fils.\par
Ils sont plus remuants que moi et ne s’arrêtent pas au milieu du pont, les lèvres entrouvertes et le nez frémissant, pour respirer et boire le petit vent qui passe : brise du matin qui secoue les feuilles sur les cimes des arbres et les dentelles au cou des voyageuses. Le ciel est clair, les maisons sont blanches, la rivière bleue ; sur la rive, il y a des jardins pleins de roses et j’aperçois le fond de la ville qui dégringole tout joyeux !\par
\bigbreak
\noindent Là-bas, un pont sur lequel trottinent des paysannes qui rient et un vieillard qui va lentement, avec un chapeau à grandes ailes et des cheveux gris, sans barbe, une redingote comme en ont les prêtres, l’air jésuite aussi.\par
« C’est lui ! c’est lui ! »\par
Quelqu’un a donné un nom à cet homme qui passe et on l’a reconnu.\par
« C’est le chantre des\emph{ Gueux}, Jacques, c’est Béranger\footnote{Auteur de chansons très populaire.}. »\par
Mon père me dit cela, comme il m’a dit : c’est la Pucelle !\par
Il a ôté son chapeau, je crois, et il a pris un air grave, comme s’il faisait sa prière. Il est plein de respect pour les gloires, mon père, et il s’enrhumerait pour les saluer. Il n’a pas encore réussi à m’inspirer cette vénération, et tandis qu’on regarde Béranger sur le pont, je regarde au loin, dans un champ, des oiseaux qui font des cercles autour d’un grand arbre, puis s’abattent et plongent dans l’argent des trembles et dans l’or des osiers.\par
Dans ma géographie, j’ai vu qu’on appelait ce pays le jardin de la France.\par
Jardin de la France ! oui, et je l’aurais appelé comme ça, moi gamin ! C’est bien l’impression que j’en ai gardée ; – ces parfums, ce calme, ces rives semées de maisons fraîches, et qui ourlent de vert et rose le ruban bleu de la Loire !…\par
Il se tache de noir, ce ruban ; il prend une couleur glauque, tout d’un coup, et il semble qu’il roule du sable sale, ou de la boue. C’est la mer qui approche, et vomit la marée ; la Loire va finir, et l’Océan commence.\par
Nous arrivons, voici la prairie de Mauves ! – Je suis resté tout le jour sous l’impression calme du matin. – J’ai peu joué avec mes petits camarades, qui s’étonnaient de mon silence.\par
L’espace m’a toujours rendu silencieux.\par
Nous sommes près du pont en fil de fer, je lis au loin \emph{Hôtel de la Fleur}. – C’est Nantes.\par

\labelblock{NANTES}

\noindent Ma mère a tanné M. Chanlaire pour lui demander où nous ferions bien d’aller en débarquant, et elle s’y est prise si bien, qu’il l’a envoyée au diable, – tout bas, – et qu’il s’esquive aussitôt qu’on arrive. Il jette son adresse à mon père, sa valise à un portefaix, et le voilà loin.\par
La dame de Paris s’en va de son côté. Nous nous serrons la main avec ses enfants, et voilà M. Vingtras, professeur de sixième au collège de Nantes, debout, sur le pavé de la ville, avec ses malles, sa femme et son garçon.\par
\bigbreak
\noindent Notre spécialité est d’encombrer de notre présence et de gêner de nos bagages la vie des cités où nous pénétrons. Pour le moment, nous avons l’air de vouloir demeurer sur le versant du quai et l’on croit que nous allons allumer du feu et faire la soupe. Nous sommes un obstacle au commerce, les déchargements se font mal. – À nous trois, nous tenons plus de place qu’il n’est permis dans un port marchand, et déjà il se forme des rassemblements autour de notre colonie.\par
Ma mère a \emph{entrepris} mon père.\par
« Tu ne pouvais pas demander à M. Chanlaire ?…\par
— Puisque c’est toi qui t’en étais chargée…\par
— Moi ! »\par
Elle a la note aiguë et qui fait retourner les passants. On s’attroupe. Un portefaix s’approche. « Combien ! dit ma mère, pour emporter ça ?\par
— Trois francs.\par
— Trois francs !\par
— Pas un sou de moins.\par
— Je vais en trouver un, moi, laisse faire, qui ne demandera pas trois francs », dit ma mère, confiant ses paquets, ses châles et une boîte à mon père et allant à un malheureux en guenilles qui traînait par là.\par
Il a à peine le temps de répondre que le portefaix arrive, montre sa médaille, fond dans le tas, accable le déguenillé de coups et la famille Vingtras d’injures.\par
Dans la bagarre, les boîtes s’écroulent et roulent vers la rivière.\par
« Jacques, Jacques ! »\par
Je cours après un colis, ma mère en poursuit un autre ; elle pousse des cris, le déguenillé aussi ; les gendarmes arrivent vers mon père. Je remonte pour le secourir ; on nous cerne. Voilà notre entrée à Nantes.\par
\bigbreak
\noindent Ouf ! ! !\par
Nous sommes installés, ce n’est pas sans peine.\par
Nous avons passé huit jours dans une auberge dont le propriétaire s’appelait Houdebine, je m’en souviens, je ne l’oublierai \emph{jamais.}\par
Nous avons eu naturellement des discussions avec lui, et ma mère a trouvé moyen de mettre la maison sens dessus dessous : histoires de corridors, disputes d’escalier, \emph{piques} avec des femmes de voyageurs. On a discuté sur la note ; la bonne a réclamé un pourboire. On nous a chassés ; nous nous sommes trouvés de nouveau à midi sur le pavé, M. Vingtras, son épouse et son rejeton.\par
Heureusement, M. Chanlaire est arrivé au moment où nous montions la garde autour des malles. Moi, j’avais les paquets pour pouvoir me mettre en route, comme une division sac au dos, dès qu’on saurait où se diriger.\par
Nous étions déjà connus dans le quartier, qui avait remarqué nos querelles avec les portefaix. Ce nouveau déballage en pleine rue, cet entassement de caisses qui, une fois de plus, interrompait le mouvement des affaires dans la ville, ma tournure, les cris de ma mère, l’embarras de mon père, tout avait fait sensation et, après avoir inspiré la curiosité, commençait à inspirer la défiance.\par
Que j’aurais donc voulu être sur un navire, pendant une bataille navale, la hache d’abordage à la main, sous les boulets, loin des bagages !\par
Nous étions dans la rue, – ma mère d’un côté, moi de l’autre, mon père en éclaireur morne, – quand M. Chanlaire vint par hasard ; il est notre providence décidément.\par
Il nous mena comme une bande de prisonniers dans un logement qu’il connaissait : je crois que des agents nous suivirent. Ils se demandaient ce que voulait cette famille.\par
Mon père n’avait pas voulu dire qui il était, l’auberge étant indigne de sa situation, et il planait du mystère sur nos têtes.\par
\bigbreak
\noindent Mon père est entré en fonctions le lendemain même de notre emménagement, et il a fait peur aux élèves, tout de suite : cela lui garantit la tranquillité dans sa classe pour toujours et des leçons particulières en quantité. – Il a l’air si chien, – on prendra des répétitions !\par
Tout va bien. – Voyons maintenant la ville.\par
Toutes mes illusions sur l’Océan, envolées ; tous mes rêves de tempêtes tombés dans l’eau douce, car c’était de l’eau douce !\par
Point de vaisseaux avec les canons qui tendent la gueule ni d’officiers en chapeau de commandement ; point de salves d’artillerie ni de manœuvres de guerre ; pas de faces de corsaires ni de soute aux poudres ; point de répétition de branle-bas ; pas d’exercice d’abordage ; des odeurs de goudron, point de parfums de mer. J’eus une espérance : on me parla de \emph{têtes de mort} entassées sur un trois-mâts ; c’étaient des fromages de Hollande.\par
Comme la vie de marin me paraît bête !\par
Il y a une petite buvette en bas de notre maison ; j’y vais chercher du vin en chopine pour notre dîner et j’y coudoie des matelots. Ils ne parlent jamais de combats, ils ne savent pas nager, ils ne plongent donc pas du haut du grand mât « dans la vague écumante », ils ne luttent pas « contre la fureur des flots… ». Non, s’ils tombaient à l’eau, ils se noieraient. Il n’y a pas cinq matelots sur dix capables de traverser la Loire. Ah bien ! merci !\par
Il faut dire que nous demeurons au haut de la ville et que les grands vaisseaux sont au bas, sur la Fosse ; mais je ne fais pas grande différence entre les navires marchands et les bateaux. Vu cette absence de canons et d’uniformes, je confonds le matelot et le marinier dans un même mépris ; j’enveloppe dans mon dédain, je confonds dans ma désillusion le loup de mer et l’ameneur de fromages.\par

\labelblock{MON PROFESSEUR}

\noindent J’ai pour professeur un petit homme à lunettes cerclées d’argent, au nez et à la voix pointus, avec un brin de moustache, des bouts de jambes un peu cagneuses, – elles ne l’empêcheront pas de faire son chemin, – insinuant, fouilleur, chafoin, furet, belette, taupe : il arrive de Paris, où il a été reçu, comme Turfin, un des premiers à l’agrégation ; il y a laissé des protecteurs que son esprit de gringalet amuse ; il en a rapporté une femme amusante, jolie, et qui doit trouver tous ces provinciaux bien sots.\par
M. Larbeau, c’est son nom, se fiche un peu de ses élèves, – il est caressant avec les fils des influents, qu’il ménage et auprès de qui il a conquis une popularité parce qu’il les traite comme de grands garçons, mais il n’est pas \emph{rosse} pour les autres. Pourvu qu’on rie de ce qu’il dit ! – il fait des calembours et propose quelquefois des charades ; on l’appelle le Parisien.\par
Je crois qu’il me trouve un peu \emph{couenne}, – parce que ses blagues ne m’amusent pas ; puis, il a entendu dire par un camarade qui prend des répétitions avec lui, que j’ai voulu être cordonnier et que maintenant j’aimerais être forgeron. Je lui semble commun ; ma mère d’ailleurs lui paraît vulgaire et mon père lui fait l’effet d’un pauvre diable. Mais il ne me tourmente pas, il a l’air de me croire, même quand je dis que j’ai \emph{oublié} mes devoirs, ou que je me suis \emph{trompé} de leçon.\par
À la fin de l’année, aux compositions de prix, il nous lit des romans de Walter Scott.\par
\bigbreak
\noindent Arrive la distribution solennelle ; – je n’ai rien – ou j’ai quelque chose, – il me semble bien que je remportai une ou deux couronnes et que je fus embrassé sur l’estrade par un homme qui empoisonnait. – Toujours donc !\par
Mais je n’avais pas la foi et je me moquais d’avoir des prix ou de n’en pas avoir, du moment que mon père ne me tourmentait point.\par

\labelblock{LA MAISON}

\noindent Nous demeurons dans une vieille maison replâtrée, repeinte, mais qui sent le vieux, et quand il fait chaud il s’en dégage une odeur de térébenthine et de fonte qui me cuit comme une pomme de terre à l’étouffée : pas d’air, point d’horizon !\par
Je passe là, les dimanches surtout, des heures pénibles. Pas de bruit, que celui des cloches, et ma tristesse d’ailleurs, même en semaine, est plus lourde dans ce pays, sous ce ciel clair, que sous le ciel fumeux de Saint-Étienne.\par
J’aimais le bruit des chariots, le voisinage des forgerons, le feu des brasiers, et il y avait une chronique des malheurs de la mine et des colères des mineurs.\par
Ici, dans le quartier que nous habitons du moins, il n’y a pas d’usines à étincelles et d’hommes à œil de feu, comme presque tous ceux qui travaillent le fer et vivent devant les fournaises.\par
Il y a des paysans aux cheveux longs et rares, tristes et laids : ils vont muets derrière leurs chariots à travers la ville et ont l’air terne et morne des sourds. Pas de gestes robustes, point l’allure large, la voix forte ! La lèvre est mince ou le nez est pointu, l’œil est creux et la tempe en front de serpent, – ils ne ressemblent pas, comme les paysans de la Haute-Loire, à des bœufs, – ils ne sentent pas l’herbe, mais la vase ; ils n’ont pas la grosse veste couleur de vache, ils portent une camisole d’un blanc sale, comme un surplis crotté. Je leur trouve l’air dévot, dur et faux, à ces fils de la Vendée, à ces hommes de Bretagne.\par
Le cours Saint-Pierre me paraît si vide – avec ses quelques vieux qui viennent s’asseoir sur les bancs ! Il y a aussi les ombres qui glissent comme des insectes noirs du côté de l’église…\par
Je me sens des envies de pleurer !\par
\bigbreak
\noindent On ne me bat plus. C’est peut-être pour ça. J’étais habitué à la souffrance ou à la colère, – je vivais toujours avec un peu de fièvre.\par
On ne me bat plus. Le proviseur n’est pas de cette école. Il a entendu parler d’un de ses professeurs qui appliquait la même méthode que mon père sur les reins de son fils ; – il l’a fait venir.\par
« Vous irez rosser vos enfants ailleurs, si cela vous tient trop, a-t-il dit ; mais si j’apprends que vous continuez ici, je demande votre changement et j’appuie pour votre disgrâce. »\par
La nouvelle est arrivée aux oreilles de mon père et a protégé les miennes.\par
Ma mère a fait connaissance de la femme d’un professeur qui est bossue.\par
On va se promener tous les soirs quand il fait beau.\par
J’ai l’air d’un prisonnier qu’on sort un peu. Je marche devant avec ordre de ne pas m’écarter, de ne pas courir, et je ne puis même pas me baisser pour ramasser une branche ou un caillou, – cela ferait éclater mon pantalon.\par
Il est arrivé qu’une de mes culottes a craqué un jour, et madame Boireau, qui n’y voit pas clair, a cependant été très offusquée. On m’a défendu de me baisser jusqu’à ce qu’on m’ait fait une culotte large.\par
On me l’a faite, il n’y a plus de danger, – j’y flâne à l’aise, – j’ai l’air d’un canard dont le derrière pousse.\par
Je vois bien qu’on me regarde et les mariniers m’entourent, mais ils me respectent comme l’inconnu ! Les camarades qui me connaissent me font des niches, tirent cela en passant comme la queue d’un chien, – on y met du sel aussi, – on m’appelle Circé.\par

\labelblock{COSTUMES ET TRAHISONS POLITIQUES}

\noindent Le supplice à propos de ma toilette recommence. Beaucoup de personnes me croient légitimiste. – J’ai une cravate qui fait trois fois le tour de mon cou, comme en portaient les incroyables, comme en avaient les royalistes sous la Restauration. – Cependant les espérances que ce parti a pu concevoir à mon propos ne tardent pas à s’évanouir. Ma mère a trouvé à côté d’un collier de chien, dans le fond d’une malle, un col en crin, et je le mets. On crie au « bonapartisme » cette fois ! C’est le signe de ralliement des brigands de la Loire, la cravate des duellistes du café Lemblin.\par
Suis-je venu pour chercher querelle aux membres du club blanc, qui est justement là sur la place ? On se perd en conjectures, mais l’étonnement devient bien autre, quand un dimanche on me voit apparaître sur le cours, vêtu comme la \emph{meilleure des républiques}.\par
J’ai une redingote marron, un parapluie vert et un chapeau gris.\par
C’est mon costume de demi-saison. Ma mère voit que je grandis et elle a voulu m’habiller comme un homme des classes moyennes, qui a de l’étoffe, ne vise pas au freluquet et a pourtant son cachet à lui. J’ai du cachet, – mais je suis modeste et je préférerais vivre dans l’obscurité, ne pas donner aux partis des espérances étouffées le lendemain, – avec cela que j’étouffe aussi ! cette redingote est si lourde et les manches sont si longues que je ne puis pas me moucher.\par
Légitimiste aujourd’hui, bonapartiste demain, constitutionnel après-demain, c’est ainsi qu’on pervertit les consciences et qu’on démoralise les masses !\par
Puis les camarades sont toujours là, – on m’appelle Louis-Philippe. C’est même dangereux par ce temps de régicide.\par
Les jours de \emph{classe moyenne}, quand je suis en \emph{bourgeois citoyen}, je rentre brisé.\par

\labelblock{NOS BONNES}

\noindent Nous avons une bonne, – il paraît que mon père gagne de l’argent.\par
Il donne la répétition en\emph{ tas} ; il prend six ou sept élèves qui lui valent chacun vingt-cinq francs et il leur dit pendant une heure des choses qu’ils n’écoutent pas ; à la fin du mois, il envoie sa note, – et il se fait avec cette distribution de participes, entre les deux classes, une assez jolie somme par trimestre.\par
Les répétés ont moins de pensums et flânent pendant ces va-et-vient dans les corridors. C’est pendant ce temps-là que s’écrivent ou se dessinent sur les murs et sur les tableaux des farces contre les professeurs ou les pions, – le nez de celui-ci, les cornes de celui-là, avec des vers de haulte graisse au fusain. On en met de raides, et la femme du censeur est gênée quand elle passe.\par
Nous la regardons à travers des trous, des fentes : elle est bien jolie, bien fraîche ; elle a épousé le censeur parce qu’il avait quelques sous, puis qu’il sera proviseur un jour. – C’est ce que j’ai entendu marmotter à ma mère qui ajoute aussi qu’elle s’habille mal.\par
« Si c’est ça, la mode de Paris, j’aime encore mieux celle de \emph{cheux nous.} »\par
Cela est lancé à la paysanne, d’un ton bon enfant, avec un petit rire qui a sa portée. Moi, je n’aime pas mieux celle de chez nous !\par
Bien désintéressé dans la question, – puisque j’étonne même les tailleurs du pays et que je ne suis vêtu à aucune mode connue depuis l’antiquité jusqu’à nos jours ! mannequin inconscient d’une politique que je ne comprends pas, caméléon sans le vouloir, – je puis apporter mon témoignage, il a son poids.\par
Eh bien, je préfère l’écharpe rose que la femme du censeur entortille autour de sa taille souple, au châle jaunâtre dont ma mère est maintenant si fière. Je préfère le chapeau de la Parisienne, à petites fleurs tremblotantes, avec deux ou trois marguerites aux yeux d’or, à la coiffure que porte celle qui m’a donné ou fait donner le sein, – je ne me rappelle plus, – où il y a un petit melon et un oiseau qui a un trop gros ventre.\par
\bigbreak
\noindent On est donc heureux à la maison.\par
Ça m’ennuie que l’on ait pris une bonne ! car j’étais occupé au moins, quand j’allais chercher de l’eau, quand je montais du bois, lorsque je déplaçais les gros meubles. J’aimais à donner des coups de marteau, des coups d’épaule et des coups de scie. Je me sentais fort et je m’exerçais à porter des armoires sur le dos et des seaux pleins à bras tendus. Je ne dois plus toucher à rien et si je suis pressé, je ne puis même pas décrotter mes souliers.\par
« Il y a de la boue autour !\par
— C’est l’affaire de la bonne, cela !\par
— Avec la grosse brosse seulement ?\par
— Nous avons une bonne, ce n’est pas pour qu’elle reste à bâiller toute la journée. »\par
Elle n’a pas le temps de bâiller, la pauvre fille ! Oh ! ma mère a l’œil !\par
Ce n’est pourtant pas son enfant, ni sa nièce ! Pourquoi donc lui montrer les mêmes égards qu’à moi ? Elle fait pour les étrangers ce qu’elle faisait pour Jacques. Elle n’établit pas de différence entre sa domestique et son fils. Ah ! je commence à croire qu’elle ne m’a jamais aimé !\par
La pauvre fille ne peut plus y tenir. On la nourrit bien, cependant. Ma mère lui donne tout ce dont nous n’avons pas voulu.\par
« Ce n’est pas moi qui épargnerais le manger à une bonne ! »\par
Et elle met sur un rebord d’assiette les nerfs, les peaux, le suif cuit.\par
« C’est bon pour son tempérament, ces choses-là. Et les boulettes froides, voilà qui fortifie ! »\par
Pauvre Jeanneton ! Si elle n’était pas soignée si bien, comme elle dépérirait ! Car même avec ce régime, elle se porte mal, elle n’est pas grasse, tant s’en faut !\par
Je crois m’apercevoir que Jeanneton n’est pas folle de ma mère et queue s’applique à la contrarier.\par
« Voulez-vous un verre de cidre, Jeanneton ?\par
— Merci, madame.\par
— Merci oui, ou merci non.\par
— Non, madame.\par
— Vous n’aimez pas le cidre ? »\par
Jeanneton balbutie.\par
« Comme vous voudrez, ma fille ! » Et ma mère ajoute d’un air dépité : « Je mets le verre là, vous le prendrez tout à l’heure si vous voulez ; vous le laisserez s’éventer, si cela vous amuse. »\par
Le cidre ne s’éventera pas, il y a bon temps qu’il l’est. Il y a deux jours qu’il traîne dans une bouteille que mon père a repoussée parce qu’elle sentait l’aigre et qu’on a oublié de boucher. – Il est tombé un \emph{cafard} dedans. Mais ma mère l’a retiré tout à l’heure, avec grand soin, comme elle aurait fait pour elle, et c’est parce qu’elle a senti le cidre qu’elle s’est décidée à l’offrir à Jeanneton.\par
« Le cidre neuf, le cidre frais a un acide qui est mauvais pour les femmes faibles… Rappelle-toi cela, mon enfant. »\par
Je me le rappellerai. Si jamais j’ai les poumons faibles, je prendrai du cidre comme celui-là, \emph{qui n’a pas d’acide}, qui sent l’aigre et le moisi. Faudra-t-il mettre un cafard dedans ?\par
Ma mère m’avait vu regarder ce cafard en réfléchissant.\par
« C’est signe que le cidre est bon. S’il était mauvais, il n’y serait pas allé. Les insectes ont leur\emph{ jugeote} aussi. »\par
Ah ! les malins !\par
Encore une observation dont je tiendrai compte. Quand il y a des insectes dans quelque chose, c’est bon. Et moi qui ne voulais pas manger de fromage parce qu’il y avait des vers et qui aimais mieux qu’il n’y eût pas de mouches dans l’huile !\par
Jeanneton est partie en refusant encore un verre de vin que ma mère lui offrait en signe d’adieu.\par
« Jacques, m’avait-elle dit, va chercher la bouteille qui était pour faire du vinaigre, tu sais, qui avait des\emph{ fleurs.} »\par
Jeanneton a refusé.\par
On remplace Jeanneton par Margoton.\par
Mais la maison est connue maintenant pour les distributions de nerfs, de peaux et de suif cuit. Margoton fait ses conditions en entrant.\par
« Moi, je n’ai pas les poumons faibles, dit-elle, et elle se donne un coup de poing dans l’estomac, un gros estomac qui danse dans sa robe d’indienne ; je n’ai pas les poumons faibles et j’aime la viande ; je veux manger chaud. »\par
\bigbreak
\noindent Margoton joue gros jeu.\par
Mais Margoton vient de la part de la femme du proviseur, et l’estomac de Margoton est protégé comme les reins du petit Vingtras. L’autorité veille dans le corsage de la bonne comme dans la culotte de l’enfant. On ne destituerait pas publiquement M. Vingtras parce qu’il flanquerait en passant une roulée à son rejeton, ou parce qu’il étoufferait sa bonne avec des chicots de boulettes ou de gras de mouton ; mais il fera bien tout de même de ne pas déplaire au grand chef à propos de son môme et de sa domestique.\par
Ah ! quelle faute on a commise en s’adressant à la femme du proviseur, par genre, pour avoir l’air de demander avis !\par
On n’ose pas renvoyer la grosse recommandée, malgré les prétentions qu’elle affiche, et elle entre en place.\par
\bigbreak
\noindent Ma mère a toujours la main sur le gigot et un pied dans la tombe, à propos de cette bonne.\par
Elle n’est pas forte et ça la fatigue de couper. Couper une tranche pour son mari, pour son enfant, c’est son devoir d’épouse, c’est son rôle de mère ; elle n’y faillira pas !\par
Mais quand il faut servir Margoton !…\par
« Vous avez encore faim ?\par
— Oui, madame.\par
— Comme cela ?\par
— Encore un petit morceau, si vous voulez. »\par
Ma mère en mourra ; je le vois bien, je le vois aux sons douloureux qu’elle étrangle quand elle reprend le couteau, à l’expression de ses yeux quand elle ajoute du jus, et elle est si lasse au dessert, qu’elle est forcée de mettre les cerises dans l’assiette de la bonne, une par une, comme avec un déchirement.\par
Marguerite en demande toujours.\par
\bigbreak
\noindent Mais ma mère renaît à vue d’œil. Mon Dieu ! mon Dieu ! soyez béni !\par
Elle renaît, redevient espiègle, reprend des couleurs. Elle est entrée un jour dans le cabinet de mon père, toute joyeuse.\par
« Antoine ! – et elle lui a parlé à l’oreille.\par
— Tu es sûre ? » a répondu mon père avec stupeur et en dérangeant son bonnet grec.\par
Elle se contente de hocher la tête en souriant.\par
« Il ne s’agit plus que de les surprendre… »\par
Elle enlève le bonnet grec et dépose d’un geste à la fois langoureux et hardi, sur le front d’Antoine, son époux, mon père, un baiser furtif.\par
On a surpris quelque chose ce matin, je ne sais pas quoi, mais ma mère a mis son châle jaune et son beau chapeau – celui au petit melon et à l’oiseau au gros ventre. Elle va chez la femme du proviseur.\par
Elle en revient en se frottant les mains et en balançant joyeusement la tête : à en faire tomber l’oiseau et le melon.\par
Dix minutes après, je vois Margoton qui fait ses paquets et à qui on règle son compte. Elle a laissé de la viande dans son assiette : qu’y a-t-il ?\par
Les larmes lui sortent des yeux comme des gouttes de bouillon.\par
« Madame, c’était pour le bon motif !\par
— Pour le bon motif !… dans une cave !… »\par
Qu’est-ce que c’est que le bon motif ? On ne m’en dit rien, mais quelques jours après, ma mère parlant à mon père cause de Margoton.\par
« Heureusement nous avons eu cette occasion de la renvoyer sans que le proviseur se fâche. Si elle n’avait pas eu ce routier pour amant ! »\par
Je ne comprends pas.\par
Il est décidé qu’on ne prendra plus de bonnes qu’on nourrira : ça fatigue trop ma mère !\par
Je vois arriver un matin une grosse fille, rouge, mais rouges avec des taches de rousseur, courte et ronde, – une boule. Des yeux qui sortent de la tête, et de l’estomac qui crève sa robe ! Il nous vient beaucoup d’estomac à la maison.\par
Elle doit venir faire la vaisselle, l’ouvrage sale, et accompagner ma mère au marché pour porter les provisions. Ma mère veut même qu’elle sorte avec moi, pour montrer que nous avons toujours une bonne, qu’il y a une domestique attachée à ma personne. J’obéis, en allant en peu en avant ou en arrière de Pétronille ; c’est son nom. Elle a malheureusement la manie de parler et elle s’accroche à moi ; on nous voit ensemble.\par
On nous voit, et il arrive qu’un matin, en entrant au collège, on m’appelle \emph{suçon}. Sur les murs des classes, je vois le portrait de mon père avec \emph{suçon} au bas et l’on ne nous nomme plus que les Suçons.\par
Voici pourquoi :\par
Pétronille occupe ses heures de loisir à vendre des sucres d’orge dans les rues, et les élèves la connaissent bien. On s’est demandé, en me rencontrant avec elle, quel lien mystérieux nous reliait, et le bruit se répand que nous fabriquons les sucres d’orge la nuit, que mon père a ajouté cette branche d’industrie au professorat.\par
On dit même qu’ils sont moins bons depuis qu’il est associé à Pétronille.\par
\bigbreak
\noindent Comme je m’ennuie ! – Je trouve mal qu’on ne me permette pas de rester à la maison et qu’on me force à sortir pour marcher, sans avoir le droit de ramasser des fleurs. On m’en fait ramasser quelquefois, mais c’est comme si je m’appelais \emph{Munito}, – comme si les fleurs étaient des dominos, que j’ai à aller chercher sur un coup d’œil ; qu’il faut prendre comme ceci, puis placer comme cela. Hé ! Munito !\par
Je me pique dans les orties, je m’enfonce les épines sous la peau, c’est une corvée, un embêtement ! J’en arrive à haïr les jardins, à détester les bouquets, à confondre les fleurs nobles et les fleurs comiques, les roses et les gratte-culs.\par
\bigbreak
\noindent Je dois faire de très grands pas, c’est plus \emph{homme}, puis ça use moins les souliers. Je fais de grands pas et j’ai toujours l’air d’aller relever une sentinelle, de rejoindre un guidon, d’être à la revue. Je passe dans la vie avec la raideur d’un soldat et la rapidité d’une ombre chinoise.\par
Et toujours une petite queue d’étoffe par derrière !\par
\bigbreak
\noindent Je voudrais être en cellule, être attaché au pied d’une table, à l’anneau d’un mur ; mais ne pas aller me promener avec ma famille, le soir.\par
\bigbreak
\noindent J’ai marché ce matin, pieds nus, sur un \emph{chose} de bouteille. (Ma mère dit que je grandis et que je dois me préparer à aller dans le monde ; elle me demande pour cela de châtier mon langage, et elle veut que je dise désormais : \emph{chose} de bouteille, et quand j’écris je dois remplacer \emph{chose} par un trait.)\par
J’ai marché sur un \emph{chose} de bouteille et je me suis entré du verre dans la plante des pieds. Ah ! quel mal cela m’a fait ! le médecin a eu peur en voyant la plaie.\par
« Vous devez souffrir beaucoup, mon enfant ? »\par
Oui, je souffre, mais à ce moment le vent a entrouvert ma fenêtre ; j’ai aperçu dans le fond le coin du faubourg, le bout de banlieue, le bord de campagne triste où l’on m’emmène tous les soirs. Je n’irai plus de quelque temps. J’ai le pied coupé. Quelle chance !\par
Et je regarde avec bonheur ma blessure qui est laide et profonde.\par

\labelblock{MON ENTRÉE DANS LE MONDE}

\noindent Ma mère ne se contente pas de me recommander la chasteté pour les mots, elle veut que je joigne l’élégance à la pudeur.\par
Elle a eu l’idée de me faire donner des leçons de « \emph{comme il faut} ».\par
Il y a M. Soubasson qui est maître de danse, de chausson et professeur de « \emph{maintien} ».\par
C’est un ancien soldat, qui boit beaucoup, qui bat sa femme, mais qui nage comme un poisson et a une médaille de sauvetage. Il a retiré de l’eau l’inspecteur d’académie qui allait se noyer. On lui a donné cette\emph{ chaire} de chausson et de danse au lycée en manière de récompense et de gagne-pain. Il y a adjoint son cours de \emph{maintien}, qui est très suivi, parce que M. Soubasson a la vue basse, l’oreille dure, aime à \emph{téter}, et qu’en lui portant aux lèvres un biberon plein de \emph{tord-boyaux}, on est libre de faire ce qu’on veut dans son cours.\par
Dieu sait ce qu’on n’y fait pas !\par
Mais moi, j’ai des leçons particulières en dehors du lycée. M. Soubasson vient à la maison. Il amène son fils, que mon père saupoudre d’un peu de latin, et en échange M. Soubasson me donne des répétitions de maintien.\par
Ma mère y assiste.\par
« Glissez le pied, une, deux, trois, – la révérence ! – souriez !\par
— Tu entends, Jacques, souris donc ! mais tu ne souris pas ! »\par
Je ne souris pas ? Mais je n’en ai pas envie.\par
Il faut essayer tout de même, et je fais la bouche en \emph{chose} de poule.\par
Ma mère, elle, minaude devant la glace, essaye, cherche, travaille et trouve enfin un sourire qu’elle me présente comme une grimace.\par
« Tiens, comme cela ! »\par
Je dois aussi tenir le petit doigt en l’air, ça me fatigue !\par
« Attention à l’auriculaire », dit toujours M. Soubasson, qui s’est fait indiquer les noms scientifiques des doigts de la main, et qui trouve que le latin est une bien belle chose, vu que c’est toujours avec ce petit doigt qu’il se fouille l’oreille. Il se la fouille même un peu trop à mon idée.\par
\bigbreak
\noindent Ce que ma mère me dit de choses blessantes pendant la leçon de maintien, ce que je la fais souffrir dans ses goûts d’élégance, cette femme, à quel point je suis commun et j’ai l’air d’un paysan, non, ce n’est pas possible de le dire ! Je ne puis pas arriver à glisser mon pied ni même à tenir mon petit doigt en l’air !\par
« Je te croyais fort », dit ma mère, qui sait que je pose un peu pour le \emph{moignon} et qui veut me blesser dans mon orgueil.\par
Je ne suis pas fort, il paraît, puisque au bout de dix minutes, l’auriculaire retombe énervé, demandant grâce, crispé comme une queue de rat empoisonné ! Rien que d’y penser, il se tord encore aujourd’hui et j’en ai la chair de poule.\par
Au bout de deux mois, c’est à peine si je suis en état de faire une révérence à trois glissades ; en tout cas, je suis incapable de parler en même temps. Si je parlais, il me semble que je dirais : \emph{j’avons, jarnigué, moussu le maire}, parce que je salue comme les villageois dans les pièces. Il me prend des envies, quand je répète avec ma mère, de l’appeler « Nanette » et de lui crier que je m’appelle « Jobin », ce qui est faux, on le sait, et ce qui est mal, je le sens bien !\par
\bigbreak
\noindent Il faut pourtant que tout ce temps-là n’ait pas été perdu, que je mette en pratique, tôt ou tard, mes leçons d’élégance et que je fasse plus ou moins honneur à M. Soubasson, à ma mère.\par
« Jacques, nous irons samedi voir la femme du proviseur. Prépare ton maintien. »\par
J’en serre l’auriculaire avec frénésie, je fais et refais des révérences, j’en sue le jour, j’en rêve la nuit !\par
Le samedi arrive, nous allons chez le proviseur en cérémonie.\par
« Pan, pan !\par
— Entrez ! »\par
Ma mère passe la première, je ne vois pas comment elle s’en tire, j’ai un brouillard devant les yeux.\par
\bigbreak
\noindent C’est mon tour !\par
Mais il me faut de la place, je fais machinalement signe qu’on s’écarte. La compagnie stupéfaite se retire comme devant un faiseur de tours. On se demande ce que c’est ; vais-je tirer une baguette, suis-je un sorcier ? Vais-je faire le saut de carpe ? On attend. J’entre dans le cercle et je commence :\par
Une – je glisse.\par
Deux – je recule.\par
Trois – je reviens, et je fends le tapis comme avec un couteau.\par
C’est un clou de mon soulier.\par
Ma mère était derrière modestement et n’a rien vu. Elle me souffle :\par
« Le sourire, maintenant ! »\par
Je souris.\par
« Et il rit, encore ! » murmure indignée la femme du proviseur.\par
Oui, et je continue à éventrer le tapis.\par
« C’est trop fort ! »\par
On se rapproche, on m’enveloppe, je suis fait prisonnier.\par
Ma mère demande grâce.\par
Moi, j’ai perdu la tête et je crie : « Nanette ! Nanette ! »\par
« Mon avancement est fichu pour cinq ans », dit mon père le soir en se couchant.\par
On renvoie M. Soubasson le lendemain, comme un malotru, et nous en faisons tous trois une maladie. Je retourne aux mauvaises manières ; je n’en suis pas fâché pour mon petit doigt qui se détend, reprend sa forme accoutumée. Je préfère avoir de mauvaises manières et n’avoir pas l’auriculaire comme une queue de rat empoisonné.\par
\bigbreak
\noindent J’ai une \emph{veine} dans mon malheur.\par
Ma blessure au pied était mal guérie. Elle se rouvre de temps en temps et je mens un peu d’ailleurs pour avoir le droit de ne pas sortir, sous prétexte que je ne puis marcher. Je la gratte même et je la gratterais encore davantage, mais ça me chatouille.\par
Ce\emph{ chose} de bouteille (je vous obéirai, ma mère) m’a rendu un fier service. Je reste à la maison et je ne rôde plus dans les chemins vides, bordés d’arbres, auxquels je ne puis pas grimper, ourlés d’herbe sur laquelle je ne puis pas me rouler, et dans la poussière desquels je traîne, comme un insecte estropié dans la boue.\par
Je reste devant une table où il y a des livres que j’ai l’air de lire, tandis que je fais des rêves qu’on ne devine point.\par
Mon père travaille de l’autre côté et ne me gêne pas, excepté quand il se mouche avec trop de fracas. Il a bien, bien soin de son nez.\par
Je n’ai pas besoin de bûcher beaucoup pour le collège, je suis souvent le premier et je n’ai qu’à faire claquer les feuilles du dictionnaire pour que mon père croie que je cherche des mots, tandis que je cours après des souvenirs de Farreyrolles, du Puy, de Saint-Étienne…\par
Je trouve une drôle de joie à regarder dans ce passé.\par
On nous donne quelquefois un paysage à traiter en \emph{narration}. J’y mets mes souvenirs.\par
« Vous avez fait de mauvais devoirs cette semaine », me dit le professeur, qui n’y retrouve ni du Virgile ni de l’Horace, si ce sont des vers ; ni des guenilles de Cicéron, si c’est du latin ; ni du Thomas ni du Marmontel, si c’est du français.\par
Mais je vais arriver à être le dernier un de ces matins !\par
Je me sens grandir, j’oublie les \emph{anciens}. Je songe plus à ce que je deviendrai qu’à ce qu’est devenu tel empereur romain. Ma\emph{ facilité}, mon imagination s’évanouissent, se meurent, sont mortes ! ! ! (Bossuet, \emph{Oraisons funèbres}.)\par
\bigbreak
\noindent Un M. David, qui est président de l’\emph{Académie poétique} de Nantes, donne de grandes soirées. Il invite les professeurs et leurs femmes à venir danser chez lui.\par
C’est dans un grand salon nu, où il y a le buste de Socrate sur la cheminée. Une jeune dame le regarde et dit :\par
« C’est donc si vilain que ça, un philosophe ? »\par
Ma mère vient avec mon père, \emph{naturellement}, et même on m’a amené au commencement.\par
Notre arrivée est annoncée avec plaisir et est accueillie avec faveur.\par
Mon père est, comme toujours, sec, maigre, le nez en corne, le front comme un toit sur des yeux gris : on dirait deux prunelles de chat sous une gouttière. Il a l’air peu commode.\par
Ma mère !… hum !… ma mère !… Elle a une robe raisin avec une ceinture jaune ; aux poignets, des nœuds jaunes aussi, un peu bouffants, comme des nœuds de paille à la queue troussée d’un cheval. Rien que ça comme toilette.\emph{ Être simple}, c’est sa devise.\par
Une fois seulement, elle a ajouté l’oiseau de son chapeau – en broche, le bec en bas, le \emph{chose} en l’air. Une fantaisie, un essai, comme la Metternich mit une couleuvre en bracelet.\par
« Qu’est-ce que cet oiseau fait là ? » demande-t-on.\par
Il y en avait qui auraient préféré le bec en l’air, le \emph{chose} en bas.\par
Ma mère faisait la mignonne, agaçant le bec de la bête comme s’il était vivant.\par
« Ti… ti… le joli petit oiseau, c’est mon \emph{toiseau ! »}\par
Mon père a obtenu qu’elle laissât l’oiseau sur le chapeau, – le joli toiseau !\par
Mais pour les nœuds, comme il avait voulu y toucher une fois :\par
« Antoine, avait répondu ma mère, suis-je une honnête femme ? Oui ou non ! Tu hésites, tu ne dis rien ! Ton silence devient une injure !…\par
— Ma chère amie !\par
— Tu me crois honnête, n’est-ce pas ?… Jamais tu n’as pu soupçonner que Jacques, notre enfant, provenait d’une source impure, était un fruit gâté, avec un ver dedans ?…\par
« Avec un ver dedans ? reprend-elle. Eh bien, aie confiance. Ta femme a un soupçon de coquetterie, peut-être, – nous sommes filles d’Ève, que veux-tu ? Mais aie confiance, Antoine. Si j’allais trop loin, – je suis ignorante, moi ! – tu aurais le droit de me faire des reproches. Mais, non !… Et ne prends pas pour les hommages d’une flamme coupable les politesses qu’on fait à un brin de toilette et de bon goût. »\par
Elle tape sur sa jupe et taquine un des nœuds jaunes, puis donne un petit coup sec sur la main de mon père :\par
« Vilain jaloux ! »\par
\bigbreak
\noindent On danse.\par
« Vous ne dansez pas, Mme Vingtras ?\par
— Nous sommes trop \emph{vieux}, dit mon père avec un sourire et en saluant.\par
— Trop \emph{vieux} ! C’est pour moi que tu as dit cela ? » fait ma mère.\par
La scène se passe dans un coin où elle a acculé Antoine, derrière un rideau.\par
« Ce ne peut être que pour moi, puisque ce monsieur est plus jeune que sa femme. Antoine, écoute-moi…\par
— Parle moins haut.\par
— Je parlerai sur le ton qu’il me plaît. »\par
Elle élève encore plus la voix.\par
« Oh ! tu ne me feras pas taire ! Non. Si tu veux m’insulter, je n’ai pas envie de l’être, entends-tu. Trop \emph{vieux} ! (Elle le toise des pieds à la tête.) Trop \emph{vieux} ! parce que je n’ai pas l’âge de la Brignoline, n’est-ce pas ? »\par
Je suis sur des épines et je fais un peu de bruit avec mes pieds, un peu de bruit avec ma bouche. Pour couvrir leurs voix, j’imite dans mon coin des instruments à vent, – au risque d’être calomnié !\par
Enfin, on s’apaise derrière le rideau.\par
\bigbreak
\noindent Je ne m’amuse pas aux soirées du proviseur ; on me trouve trop triste. – Je suis habillé à neuf. Seulement on a choisi une drôle d’étoffe ; j’ai l’air d’être dans un bas de laine ; c’est terne, \emph{à côtes}, mais si terne !\par
Comme ça déteint, je fais des taches aux habits des autres.\par
On s’écarte de moi. Ma mère elle-même ne me parle que de loin, comme à un étranger presque ! – Oh ! mon Dieu !\par
\bigbreak
\noindent « Je dan-se-rai », a-t-elle dit ; et elle danse.\par
Elle embrouille le quadrille, marche sur quelques pieds, mais, bah ! elle sauve tout par de petites plaisanteries et des petits airs ; – une véritable écolière, je vous dis !\par
Au galop final une idée lui vient, celle de faire partager à son enfant les joies de Terpsichore, et s’éloignant du galop une seconde, elle me saisit et m’attire dans le tourbillon. Le galop est fini que je saute encore et elle a l’air d’un Savoyard qui fait danser une marionnette. – Ça me fait si mal sous les bras !\par
Depuis quelque temps elle est rêveuse.\par
« Ta mère a quelque idée en tête », fait mon père du ton d’un homme qui prévoit un malheur.\par
Elle s’enferme toute seule et on entend des bruits, des petits cris, des tressaillements de plancher ; on l’a surprise à travers la porte qui faisait des grâces devant un miroir, en s’appuyant le front.\par
Soirée chez M. David. La femme du professeur d’histoire, qui est d’origine espagnole, esquisse un fandango assez leste, eh ! eh ! quoique revu et corrigé comme les morceaux choisis par l’archevêque de Tours.\par
La femme du professeur d’allemand, une Alsacienne, chante un \emph{titi la itou, la itou la la}, en valsant une valse du pays.\par
C’est fini. Elle se repose sur la banquette et le cercle où l’on vient de danser est vide.\par
On entend un petit cri.\par

\begin{quoteblock}
\noindent Eh ! youp ! eh ! youp !\end{quoteblock}

\noindent Mon père, qui est en face de moi, a l’air frappé d’un coup de sang et je vais \emph{voler dans ses bras.}\par

\begin{quoteblock}
\noindent Eh ! youp ! eh ! youp ! la Catarina ! eh ! youp !\end{quoteblock}

\noindent En même temps une apparition traverse le salon et tourne sur le parquet.\par
L’apparition chante :\par


\begin{verse}
Ché la bourra, la la !\\
Oui, la bourra, fouchtra !\\
\end{verse}

\noindent Et la voix devenant énergique, presque biblique, dit tout d’un coup :\par
« \emph{Anyn}, mon homme ! »\par
Cet homme, c’est \emph{Antoine} qui au premier \emph{youp ! youp} ! avait pressenti le danger, – c’est mon père qui est entraîné comme je le fus le jour des marionnettes.\par
« \emph{Anyn}, mon homme, \emph{Anyn ! »}\par
Et ma mère le plante devant elle, en le gourmandant de sa \emph{mollèche} – à la \emph{chtupéfacchion} de l’assistance, qui n’a pas été prévenue.\par
« Eh ! chante ! chante donque ! »\par
J’ai peur qu’on \emph{chonge} à moi aussi, et je disparais dans les cabinets. Toute la soirée, je répondis :\par
« \emph{Il y a quelqu’un !… »}\par
La nuit me trouva harassé, vide !\par
Je sortis enfin quand la dernière lampe fut éteinte, et je revins au logis, où l’on ne pensait pas à moi.\par
Ma mère seule avec mon père murmurait à son oreille :\par
« Eh bien ! Est-ce que la bourrée ne vaut pas le fandango ? »\par
Et elle ajouta d’une voix un peu tremblante :\par
« Dis-moi \emph{cha ! »}\par
C’était la mutinerie dans la fierté, l’espièglerie dans le bonheur !\par
\bigbreak
\noindent Tout se gâte.\par
Mon père – Antoine – n’a plus voulu aller dans le monde avec ma mère.\par
La soirée de la bourrée lui a complètement tourné la tête, elle s’est grisée avec son succès ; restant dans la veine trouvée, s’entêtant à suivre ce filon, elle parle\emph{ charabia} tout le temps, elle appelle les gens\emph{ mouchu} et\emph{ monchieu.}\par
Mon père à la fin lui interdit formellement l’auvergnat.\par
Elle répond avec amertume :\par
« Ah ! c’est bien la peine d’avoir reçu de l’éducation pour être jaloux d’une femme qui n’a pour elle que son \emph{esprit naturel} ! Mon pauvre ami, avec ta latinasserie et ta grécaillerie, tu en es réduit à défendre à ta femme, qui est de la campagne, de\emph{ t’éclipser ! »}\par
Les querelles s’enveniment.\par
« Tu sais, Antoine, je t’ai fait assez de sacrifices, n’en demande pas trop ! Tu as voulu que je ne dise plus \emph{estatue}, je l’ai fait. Tu as voulu que je ne dise plus \emph{ormoire}, je ne l’ai plus dit, mais ne me pousse pas à bout, vois-tu, ou je recommence. »\par
Elle continue :\par
« Et d’abord ma mère disait \emph{estatue}… elle était aussi respectable que la tienne, sache-le bien ! »\par
Mon père se trouve menacé de tous côtés, entre \emph{estatue} et \emph{mouchu.}\par
Il met les pieds dans le plat et défend l’un et l’autre.\par
Ma mère se venge en l’injuriant ; elle cherche des mots qui le blessent : \emph{es}cargot – \emph{es}pectacle ! \emph{es}tomac – \emph{es}quelette ! Ces diphtongues entrent profondément dans le cœur de mon père. Le samedi suivant, il s’habille sans mot dire et va en soirée sans elle.\par
\bigbreak
\noindent Le samedi d’après, même jeu, mais à minuit ma mère vient me réveiller.\par
« Lève-toi, tu vas aller attendre ton père à la porte de chez M. David, et quand il sortira tu crieras : \emph{La la, fouchtra} ! J’arriverai, tu nous laisseras. »\par
\bigbreak
\noindent J’ai crié :\emph{ La la, fouchtra} ! J’ai eu tort.\par
Elle lui fait une scène devant tout le monde, tout haut, disant qu’il laisse mourir sa famille de faim pour courir les bals.\par
« Il a un bien gros derrière pour un enfant qui meurt de faim, dit quelqu’un.\par
— Oui, répète ma mère, il nous laisse mourir de faim. »\par
Nous avons mangé une grosse soupe à dîner, puis des andouilles : pour finir, il y a eu du lapin. Moi, je ne meurs pas de faim ; elle a beaucoup mangé aussi.\par
Ma mère crie toujours.\par
« Mon enfant n’a pas une chemise à se mettre sur le dos, voyez comme il est mis ! »\par
Je ne suis pas en noir aujourd’hui, je suis en habit gris, pantalon gris ; je ressemble à un infirmier.\par
Le monde s’amasse, mon père veut glisser sous une voiture, s’égare entre les jambes des chevaux. Il faut le tirer de là-dessous.\par
Il reparaît enfin ; son chapeau de soirée est écrasé et a l’air d’un accordéon. Ma mère lui prend le bras comme ferait un sergent de ville.\par
« Viens, mon enfant, ajoute-t-elle, en me parlant avec des larmes. Viens, dis-lui que tu es son fils ! »\par
Il le sait bien ; est-ce qu’il ne m’a pas reconnu ? Est-ce que je suis changé depuis sept heures ?\par
Tout le long du chemin, je tâche de trouver à la porte des modistes ou des tailleurs une glace, pour voir quelle figure j’ai depuis que je meurs de faim.\par

\labelblock{TU, VOUS}

\noindent La maison est redevenue morne presque autant que jadis, du temps de Mme Brignolin, quand c’était si triste. Mon père ne va plus en soirée, il va je ne sais où.\par
Ma mère, un soir, m’a ordonné de le suivre en me cachant. Mais mon père est arrivé au même moment.\par
Je me tenais devant elle, tout craintif, tout honteux, me disant tout bas : Est-ce que c’est bien d’espionner son père ?\par
« Voulez-vous donc faire un policier de votre fils ? a-t-il dit. J’ai entendu ce que vous lui recommandiez. »\par
Ce \emph{vous} la fit pâlir. Jamais elle ne m’en reparla depuis.\par
Elle essaye de rattraper par quelque bout le terrain qu’elle perd, on le sent à l’accent, on le voit au geste.\par
« C’est que, dit-elle, ce n’est pas gai d’être éveillé tous les soirs quand\emph{ tu} rentres…\par
— Je ne \emph{vous} réveillerai plus », répond mon père.\par
Le soir de ce jour-là, mon père alla chercher un matelas et un pliant dans le grenier.\par
\bigbreak
\noindent On n’entendit plus de bruit dans la maison. Nous vivions chacun dans notre coin, et l’on se parlait à peine.\par
Les femmes de ménage au bout de huit jours partaient, disant qu’on jaunissait dans cette baraque.\par
« Comme c’est triste là-dedans ! » C’était le proverbe du quartier.\par
\bigbreak
\noindent Il y a longtemps que cela dure. Ma mère m’oblige à lui tenir compagnie le soir, et je lui lis des choses saintes, dans sa chambre, à la lueur d’une mauvaise chandelle, près d’un feu sans flamme.\par
Il n’est question que d’enfer et de douleur. – C’est toujours des désolations dans ces livres d’église.\par
\bigbreak
\noindent Une scène !\par
Mon père, en retournant une vieille malle, a découvert quelque chose de lourd, de sonnant. C’est un bas plein jusqu’à la cheville de pièces de cent sous.\par
Il est en train de s’étonner, quand ma mère entre comme une furie et se jette sur le bas pour le lui arracher.\par
« C’est à moi, cet argent-là. Je l’ai économisé sur ma toilette. »\par
Mon père ne lâche pas, ma mère crie :\par
« Jacques, aide-moi ! »\par
Moi, je ne sais que crier et dire en allant de l’un à l’autre :\par
« Papa ! Maman ! »\par
Mon père reste maître du sac et l’enferme dans son armoire.\par
\bigbreak
\noindent Ils se sont raccommodés !\par
Ma mère est tout simplement allée trouver mon père et lui a dit :\par
« Je ne puis plus vivre comme cela, j’aime mieux partir, – retourner chez ma sœur, emmener mon enfant. »\par
\bigbreak
\noindent Mais elle ne veut pas s’en aller, et elle finit par le dire tout haut, par l’avouer à Antoine, à qui elle confesse qu’elle a eu tort – et lui demande d’oublier.\par
Il en a assez lui aussi, sans doute, et il ne se défend que pour la forme, il se fait un peu tirer l’oreille ; il est flatté qu’on lui demande grâce ; c’est le fond de sa nature, qu’on s’agenouille devant lui ; et maintenant qu’il est sûr d’être le maître, qu’elle a lâché pied, il préfère s’évader de la gêne où le mettait tant de tristesse et de silence.\par
« Faut-il reporter le pliant et le matelas au grenier, dis, papa ? »\par
J’ai regret de ce que j’ai dit, je les vois embarrassés.\par
« Jacques, répond mon père, tu peux aller jouer avec le petit du premier. »
\section[{19. Louisette}]{19. \\
Louisette}\renewcommand{\leftmark}{19. \\
Louisette}

\noindent M. Bergougnard a été le camarade de classe de mon père.\par
C’est un homme osseux, blême, toujours vêtu sévèrement.\par
Il était le premier en dissertation, mon père n’était que le second, mais mon père redevenait le \emph{preu} en vers latins. Ils ont gardé l’un pour l’autre une admiration profonde, comme deux hommes d’État, qui se sont combattus, mais ont pu s’apprécier.\par
Ils ont tous les deux la conviction qu’ils sont nés pour les grandes choses, mais que les nécessités de la vie les ont tenus éloignés du champ de bataille.\par
Ils se sont partagé le domaine.\par
« Toi, tu es l’Imagination, dit Bergougnard, une imagination brûlante… »\par
Mon père se rengorge et se donne un mal du diable pour se mettre un éclair dans les yeux ; il jette un regard un peu trouble dans l’espace – et se dépeigne en cachette.\par
« Tu es l’Imagination folle… »\par
Mon père joue l’égarement et fait des grimaces terribles.\par
« Moi, reprend Bergougnard, je suis la Raison froide, glacée, implacable. » Et il met sa canne toute droite entre ses jambes.\par
Il ajuste en même temps, sur un nez jaunâtre, piqué de noir comme un dé, il ajuste une paire de lunettes blanches qui ressemblent à des lentilles solaires, et m’effraient pour mon habit un peu sec.\par
On croit qu’elles vont faire des trous. Je me demande même quelquefois si elles ne lui ont pas cuit les yeux, qui ont l’air d’une grosse tache noire, là-dessous.\par
« Je suis la Raison froide, glacée, implacable… »\par
Il y tient. Il dit cela presque en grinçant des dents, comme s’il écrasait un dilemme et en mâchait les cornes.\par
Il a été dans l’Université aussi, ça se voit bien ; mais il en est sorti pour épouser une veuve, – qui crut se marier à un grand homme et lui apporta des petites rentes, avec lesquelles il put travailler à son grand livre \emph{De la Raison chez les Grecs.}\par
Il y travaille depuis trois ans ; toujours en ayant l’air de grincer des dents ; il tord les arguments comme du linge, il veut raisonner serré, lui, il ne veut pas d’une logique lâche, – ce qui le constipe, il paraît, et lui donne de grands maux de tête.\par
« Le cerveau, vois-tu, dit-il à mon père, en se tapant le front avec l’index…\par
— Pas le cerveau », dit le médecin, qui croit à une affection du gros intestin ; si bien qu’il ne sait pas au juste si M. Bergougnard est philosophe parce qu’il est constipé, ou s’il est constipé parce qu’il est philosophe.\par
On en parle ; il s’élève quelques petites discussions très aigres à ce propos dans les cafés. Le cerveau a ses partisans.\par
Ma mère s’était d’abord prononcée avec violence.\par
Mon père, un certain jour, avait eu l’idée de prendre M. Bergougnard comme orateur et de le dépêcher à elle, solennel, les dents menaçantes, venant, avec l’arme de la raison, essayer de la convaincre qu’elle s’écartait quelquefois, vis-à-vis de son mari, des lois du respect tel que les anciens et les modernes l’ont compris, en lui faisant des scènes dont on n’avait pas l’équivalent dans les grands classiques.\par
« Je viens vous poser un dilemme.\par
— Vous feriez mieux de vous mettre des sinapismes quelque part. »\par
Il était parti, et il ne serait jamais revenu si ma mère n’avait surmonté ses répugnances à cause de moi.\par
Elle mit sa réponse un peu verte sur le compte d’une gaieté de paysanne qui aime à \emph{rire un brin}, et elle qui ne faisait jamais d’excuses, en avait fait pour que M. Bergougnard revînt – dans mon intérêt – par amour pour son fils.\par
C’est pour son Jacques qu’elle s’abaissait jusqu’à l’excuse, et faisait encore asseoir près d’elle, – autant que s’asseoir se pouvait, – cette statue vivante de la constipation.\par
Pour moi, oui ! – parce que M. Bergougnard m’apprenait, me montrait dans les textes, me prouvait, livre en main, que les philosophes de la vieille Grèce et de Rome battaient leurs fils à tour de bras ; il rossait les siens au nom de Sparte et de Rome, – Sparte les jours de gifles, et Rome les jours de fessées.\par
Ma mère, malgré son antipathie, par amour pour son Jacques, s’était rejetée dans les bras horriblement secs de M. Bergougnard, qui avait les entrailles embarrassées, comme homme, mais qui n’en avait pas comme philosophe, et qui mouillait des chemises à graver les principes de la philosophie sur le \emph{chose} de ses enfants, – comme on cloue une enseigne, comme on plante un drapeau.\par
Ma mère avait deviné que je n’avais pas la foi cutanée.\par
« Demande à M. Bergougnard ! vois M. Bergougnard, regarde les côtes du petit Bergougnard ! »\par
En effet, après avoir mis quatre ou cinq fois le nez dans le ménage de M. Bergougnard, je trouvais ma situation délicieuse à côté de celles dans lesquelles les petits Bergougnard étaient placés journellement : tantôt la tête entre les jambes de leur père, qui, du même coup, les étranglait un peu et les fouettait commodément ; tantôt de face, enlevés par les cheveux et époussetés à coups de canne, mais à fond, – jusqu’à ce qu’il n’y eût plus de cheveux ou de poussière.\par
On entendait quelquefois des cris terribles sortir de là-dedans.\par
Des hommes du pays montraient la villa Bergougnard à des illustrations :\par
« C’est là que demeure le philosophe, disaient-ils en étendant les bras vers la villa, – c’est là que M. Bergougnard écrit : \emph{De la Raison chez les Grecs}… C’est la maison du sage. »\par
Tout d’un coup ses fils apparaissaient à la fenêtre en se tordant comme des singes et en rugissant comme des chacals.\par
\bigbreak
\noindent Oui, les coups qu’on me donne sont des caresses à côté de ceux que M. Bergougnard distribue à sa famille.\par
M. Bergougnard ne se contente pas de battre son fils pour son bien, – le bien de Bonaventure ou de Barnabé, – et pour son plaisir à lui Bergougnard.\par
Il n’est pas égoïste et personnel, – il est dévoué à une cause, c’est à l’humanité qu’il s’adresse, en relevant d’une main la chemise de Bonaventure, en faisant signe de l’autre aux savants qu’il va exercer son système.\par
Il donne une fessée comme il tire un coup de canon, et il est content quand Bonaventure pousse des cris à faire peur à une locomotive.\par
Il aurait apporté aux rostres le derrière saignant de son fils ; en Turquie, il l’eût planté comme une tête au bout d’une pique, et enfoncé à la grille devant le palais.\par
\bigbreak
\noindent Je ne suis qu’un isolé, un déclassé, un inutile, – je ne sers à rien, – on me bat, je ne sais pas pourquoi, tandis que Bonaventure est un exemple et entre \emph{à reculons}, mais profondément dans la philosophie.\par
Je ne plains pas Bonaventure.\par
Bonaventure est très laid, très bête, très méchant. Il bat les petits comme son père le bat, il les fait pleurer et il rit. Il a coupé une fois la queue d’un chat avec un rasoir et on la voyait dégoutter comme un bâton de cire à la bougie ; il faisait mine de cacheter les lettres avec les gouttes de sang.\par
Une autre fois, il a plumé un oiseau vivant.\par
Son père était bien content.\par
« Bonaventure aime à se rendre compte, Bonaventure aime la science… »\par
Depuis qu’il a coupé la queue du chat, depuis qu’il a plumé l’oiseau, je le déteste. Je le laisserais écraser à coups de pierre comme un crapaud. Est-ce que je suis cruel aussi ?\par
L’autre jour il tordait le poignet d’un mioche ; je l’ai bourré de coups de pied et tapé le nez contre le mur.\par
\bigbreak
\noindent Mais sa petite sœur ! – ô mon Dieu !\par
\bigbreak
\noindent Elle était restée chez une tante, au pays. La tante est morte, on a renvoyé l’enfant. Pauvre innocente, chère malheureuse !\par
Mon cœur a reçu bien des blessures, j’ai versé bien des larmes ! J’ai cru que j’allais mourir de tristesse plus d’une fois, mais jamais je n’ai eu devant l’amour, la défaite, la mort, des affres de douleur, comme au temps où l’on tua Louisette devant moi.\par
Cette enfant, qu’avait-elle donc fait ? On avait raison de me battre, moi, parce que, quand on me battait, je ne pleurais pas, – je riais quelquefois même parce que je trouvais ma mère si drôle quand elle était bien en colère, – j’avais des os durs, du \emph{moignon}, j’étais un homme.\par
Je ne criais pas, pourvu qu’on ne me cassât pas les membres, – parce que j’aurais besoin de gagner ma vie.\par
« Papa, je suis un pauvre, ne m’estropie pas ! »\par
Mais la mignonne qu’on battait, et qui demandait pardon, en joignant ses menottes, en tombant à genoux, se roulant de terreur devant son père qui la frappait encore… toujours !…\par
« Mal, mal ! Papa, papa ! »\par
Elle criait comme j’avais entendu une folle de quatre-vingts ans crier en s’arrachant les cheveux, un jour qu’elle croyait voir quelqu’un dans le ciel qui voulait la tuer !\par
Le cri de cette folle m’était resté dans l’oreille, la voix de Louisette, folle de peur aussi, ressemblait à cela !\par
« Pardon, pardon ! »\par
J’entendais encore un coup ; à la fin je n’entendais plus rien, qu’un bruit étouffé, un râle.\par
Une fois je crus que sa gorge s’était cassée, que sa pauvre petite poitrine s’était crevée, et j’entrai dans la maison.\par
Elle était à terre, son visage tout blanc, le sanglot ne pouvant plus sortir, dans une convulsion de terreur, devant son père froid, blême, et qui ne s’était arrêté que parce qu’il avait peur, cette fois, de l’achever.\par
\bigbreak
\noindent On la tua tout de même. Elle mourut de douleur à dix ans…………………\par
De douleur !… comme une personne que le chagrin tue.\par
Et aussi du mal que font les coups !\par
On lui faisait si mal ! et elle demandait grâce en vain.\par
Dès que son père approchait d’elle, son brin de raison tremblait dans sa tête d’ange…………………\par
Et on ne l’a pas guillotiné, ce père-là ! on ne lui a pas appliqué la peine du talion à cet assassin de son enfant, on n’a pas supplicié ce lâche, on ne l’a pas enterré vivant à côté de la morte !\par
« Veux-tu bien ne pas pleurer », lui disait-il, parce qu’il avait peur que les voisins entendissent, et il la cognait pour qu’elle se tût : ce qui doublait sa terreur et la faisait pleurer davantage.\par
\bigbreak
\noindent Elle était gentille, toute gaie, toute contente, si rose, quand elle arriva.\par
Au bout de quelque temps, elle n’avait plus de couleurs déjà, et elle avait des frissons comme un chien qu’on bat, quand elle entendait rentrer son père.\par
Je l’avais embrassée en caressant ses joues rondes et tièdes ! aux Messageries, où nous avions accompagné M. Bergougnard, pour la recevoir comme un bouquet.\par
Dans les derniers temps (ah ! ce ne fut pas long, heureusement pour elle !) elle était blanche comme la cire ; je vis bien qu’elle savait que toute petite encore elle allait mourir, – son sourire avait l’air d’une grimace. – Elle paraissait si vieille, Louisette, quand elle mourut à dix ans, – de douleur, vous dis-je !\par
Ma mère vit mon chagrin le jour de l’enterrement.\par
« Tu ne pleurerais pas tant, si c’était moi qui étais morte ? »\par
Ils m’ont déjà dit ça quand le chien est crevé.\par
« Tu ne pleurerais pas tant. »\par
Je ne dis rien.\par
« Jacques ! quand ta mère te parle, elle entend que tu lui répondes… – Veux-tu répondre ? »\par
Je n’écoute seulement pas ce qu’ils disent, je songe à l’enfant morte, qu’ils ont vu martyriser comme moi, et qu’ils ont laissé battre, au lieu d’empêcher M. Bergougnard de lui faire mal ; ils lui disaient à elle qu’elle ne devait pas être méchante, faire de la peine à son papa !\par
Louisette, méchante ! cette miette d’enfant, avec cette voix tendre et ce regard mouillé !\par
Voilà que mes yeux s’emplissent d’eau, et j’embrasse je ne sais quoi, un bout de fichu, je crois, que j’ai pris au cou de la pauvre assassinée.\par
« Veux-tu lâcher cette saleté ! »\par

\astertri

\noindent Ma mère se précipite sur moi. Je serre le fichu contre ma poitrine ; elle se cramponne à mes poignets avec rage.\par
« Veux-tu le donner !\par
— C’était à Louisette…\par
— Tu ne veux pas ? – Antoine, vas-tu me laisser traiter ainsi par ton fils ? »\par
Mon père m’ordonne de lâcher le fichu.\par
« Non, je ne le donnerai pas !\par
— Jacques ! » crie mon père furieux.\par
Je ne bouge pas.\par
« Jacques ! »\par
Et il me tord les bras. Ils me volent ce bout de soie que j’avais de Louisette.\par
« Il y a encore une saleté dans un coin que je vais faire disparaître aussi », dit ma mère.\par
C’est le bouquet que me donna ma cousine.\par
Elle l’a trouvé au fond d’un tiroir, en fouillant un jour.\par
Elle va le chercher, l’arrache et le \emph{tue}. Oui, il me sembla qu’on \emph{tuait} quelque chose en déchirant ce bouquet fané…\par
J’allai m’enfermer dans un cabinet noir pour les maudire tout bas ; je pensais à Bergougnard et à ma mère, à Louisette et à la cousine…\par
Assassins ! assassins !\par
Cela sortait de ma poitrine comme un sanglot, et je le répétai longtemps dans un frisson nerveux…\par
Je me réveillai, la nuit, croyant que Louisette était là, assise avec son drap de morte, sur mon lit. Il y avait son bras grêle qui sortait, avec des marques de coups !…
\section[{20. Mes humanités}]{20. \\
Mes humanités}\renewcommand{\leftmark}{20. \\
Mes humanités}

\noindent Comme mon professeur de cette année est \emph{serin} !\par
Il sort de l’École normale, il est jeune, un peu chauve, porte des pantalons à sous-pieds et fait une traduction de Pindare. Il dit \emph{arakné} pour araignée, et quand je me baisse pour rentrer mes lacets dans mes souliers, il me crie : « Ne portez pas vos extrémités digitales à vos cothurnes. » De beaux \emph{cothurnes}, vrai, avec des caillots de crottes et des dorures de fumier.\par
Je vais toujours rôder dans une écurie, qui est près de chez nous, et où je connais des palefreniers, avant d’entrer en classe, et je n’ai pas seulement du crottin aux pieds, j’en dois avoir aussi dans mes livres.\par
Il dit \emph{cothurnes} et \emph{arakné} avec un bout de sourire, pour qu’on ne se moque pas trop de lui, mais il y croit au fond, cela se voit, il aime ces allusions antiques, \emph{je le sais} (imité de Bossuet).\par
\bigbreak
\noindent Il m’aime, parce que je trousse bien le vers latin.\par
« Quelle imagination il a, et quelle facilité ! Minerve est sa marraine !\par
— Tante Agnès, dit ma mère.\par
— Tantagnès, Tantagnétos, Tantagnététon.\par
— Vous dites, fait Mme Vingtras, qui semble effrayée par une de ces consonances, et a rougi du génitif pluriel !\par
— Quelle imagination ! » répète le professeur pour se sauver.\par
Et je laisse dire que je suis intelligent, que j’ai \emph{des moyens.}\par
JE N’EN AI PAS !\par
\bigbreak
\noindent On nous a donné l’autre jour comme sujet – « Thémistocle haranguant les Grecs ». Je n’ai rien trouvé, rien, rien !\par
« J’espère que voilà un beau sujet, hé ! » a dit le professeur en se passant la langue sur les lèvres, – une langue jaune, des lèvres crottées.\par
C’est un beau sujet certainement, et, bien sûr, dans les petits collèges, on n’en donne pas de comme ça ; il n’y a que dans les collèges royaux, et quand on a des élèves comme moi.\par
Qu’est-ce que je vais donc bien dire ?\par
« Mettez-vous à la place de Thémistocle. »\par
Ils me disent toujours qu’il faut se mettre à la place de celui-ci, de celui-là, – avec le nez coupé comme Zopyre ? avec le poignet rôti comme Scévola ?\par
C’est toujours des généraux, des rois, des reines !\par
Mais j’ai quatorze ans, je ne sais pas ce qu’il faut faire dire à Annibal, à Caracalla, ni à Torquatus, non plus !\par
Non, je ne le sais pas !\par
Je cherche aux adverbes, et aux adjectifs du \emph{Gradus}, et je ne fais que copier ce que je trouve dans l’\emph{Alexandre.}\par
Mon père l’ignore, je n’ai pas osé l’avouer.\par
Mais lui, lui-même ! (Oh ! je vends un secret de famille !) j’ai vu que ses exercices à lui, pour l’agrégation, étaient faits aussi de pièces et de morceaux. – Sommes-nous une famille de crétins ?…\par
Quelquefois il compose un discours où il faut faire parler une femme. – Les plaintes d’Agrippine, Aspasie à Socrate, Julie à Ovide.\par
Je le vois qui se gratte le front, et il touche sa barbe avec horreur ; – il est Agrippinus, Aspasios, il n’est pas Aspasie, il n’est pas Agrippine, – il se tord les poils et les mord, désespéré !\par
Je sens toute l’infériorité de ma nature, et j’en souffre beaucoup.\par
Je souffre de me voir accablé d’éloges que je ne mérite pas, on me prend pour un fort, je ne suis qu’un simple filou. Je vole à droite, à gauche, je ramasse des \emph{rejets} au coin des livres. Je suis même malhonnête quelquefois. J’ai besoin d’une épithète ; peu m’importe de sacrifier la vérité ! Je prends dans le dictionnaire le mot qui fait l’affaire, quand même il dirait le contraire de ce que je voulais dire. Je perds la notion juste ! Il me faut mon spondée ou mon dactyle, tant pis ! – la \emph{qualité} n’est rien, c’est la \emph{quantité} qui est tout.\par
\bigbreak
\noindent Il faut toujours être près du Janicule avec eux.\par
Je ne puis cependant pas me figurer que je suis un Latin.\par
Je ne puis pas !\par
Ce n’est pas dans les latrines de Vitellius que je vais, quand je sors de la classe. Je n’ai pas été en Grèce non plus ! Ce ne sont pas les lauriers de Miltiade qui me gênent, c’est l’oignon qui me fait du mal. Je me vante, dans mes narrations, de blessures que j’ai reçues par devant, \emph{adverso pectore ;} j’en ai bien reçu quelques-unes par derrière.\par
« Vous peindrez la vie romaine comme ci, comme ça… »\par
Je ne sais pas comment on vivait, moi ! Je fais la vaisselle, je reçois des coups, j’ai des bretelles, je m’ennuie pas mal ; mais je ne connais pas d’autre consul que mon père, qui a une grosse cravate, des bottes ressemelées, et en fait de vieille femme (\emph{anus}), la mère Gratteloux qui fait le ménage des gens du second.\par
Et l’on continue à dire que j’ai de la facilité.\par
\bigbreak
\noindent C’est trop d’hypocrisie. Oh ! le remords m’étouffe !…\par
Il y a M. Jaluzot, le professeur d’histoire, que tout le monde aime au collège. On dit qu’il est riche \emph{de chez lui}, et qu’il a son franc parler. C’est un bon garçon.\par
Je me jette à ses pieds et je lui dis tout.\par
« M’sieu Jaluzot !\par
— Quoi donc, mon enfant ?\par
— M’sieu Jaluzot ! »\par
Je baigne ses mains de mes larmes.\par
« J’ai, m’sieu, que je suis un filou ! »\par
Il croit que j’ai volé une bourse et commence à rentrer sa chaîne.\par
Enfin, j’avoue mes vols dans Alexandre, et tout ce que j’ai réavalé de \emph{rejets}, je dis où je prends le derrière de mes vers latins.\par
« Relevez-vous, mon enfant ! Avoir ramassé ces épluchures et fait vos compositions avec ? Vous n’êtes au collège que pour cela, pour mâcher et remâcher ce qui a été mâché par les autres.\par
— Je ne me mets jamais à la place de Thémistocle ! »\par
C’est l’aveu qui me coûte le plus.\par
M. Jaluzot me répond par un éclat de rire, comme s’il se moquait de Thémistocle. On voit bien qu’il a de la fortune.\par
\bigbreak
\noindent Pour la \emph{narration française}, je réussis aussi par le retapage et le ressemelage, par le mensonge et le vol.\par
Je dis dans ces narrations qu’il n’y a rien comme la patrie et la liberté pour élever l’âme.\par
Je ne sais pas ce que c’est que la liberté, moi, ni ce que c’est que la patrie. J’ai été toujours fouetté, giflé, – voilà pour la liberté ; – pour la patrie, je ne connais que notre appartement où je m’embête, et les champs où je me plais, mais où je ne vais pas.\par
Je me moque de la Grèce et de l’Italie, du Tibre et de l’Eurotas. J’aime mieux le ruisseau de Farreyrolles, la bouse des vaches, le crottin des chevaux, et ramasser des pissenlits pour faire de la salade.\par

\labelblock{RÉCITATION CLASSIQUE ET DÉBIT}

\noindent « Plus fort, mon enfant ! »\par
C’est ma mère qui parle, elle a bien de la douceur aujourd’hui ! « Plus fort » est dit comme par une sœur d’hôpital à un malade dont on tient le front brûlant ; « plus fort ! là ! du courage ! c’est bien ! »\par
Je retombe exténué sur un fauteuil, les bras pendants et mous comme un lapin mort ; j’ai même, comme le lapin assassiné, une goutte de sang au bout du museau : puis, tout autour, la peau est rougeâtre et lisse comme une pelure d’oignon, lisse, lisse !… Si j’avais quelques petits poils qui faisaient les fous, ils sont partis, noyés, tant il m’a passé d’eau dans les narines depuis ce matin !\par
\bigbreak
\noindent C’est qu’aujourd’hui on compose en \emph{récitation classique et débit}, et ma mère veut que j’aie le prix.\par
Pour cela, il faut non seulement savoir, mais \emph{bien dire} ; et un nez vigoureusement clarifié permet d’avoir la voix claire.\par
On m’a clarifié le nez.\par
Ma mère l’a pris et mis dans l’eau ; il est resté là longtemps, longtemps ! oh ! les minutes étaient des siècles !\par
Enfin elle l’a retiré bien proprement et m’a dit :\par
« Renifle, mon enfant ! renifle ! »\par
Je ne pouvais plus.\par
« Fais un effort, Jacques ! »\par
Je l’ai fait.\par
\bigbreak
\noindent Seringue molle, mon nez a tiré et craché l’eau pendant une demi-heure, peut-être plus, et il me semble qu’on m’a vidé et que ma tête tient à mon cou comme un ballon rose à un fil ; le vent la balance. J’y porte la main. « Où est-elle ? – Ah ! la voilà ! »\par
Il n’y a que le nez qui compte ; il me cuit comme tout et il flambe comme un bouchon de carafe.\par
Je m’y attache, je le prends par le bout, moi-même, et je me conduis comme cela, sans me brusquer, jusqu’à mon pupitre, où je repasse ma leçon.\par
Quelquefois le but est manqué, mon nez dégoutte dans tous les sens, il en tombe des perles d’eau comme d’un torchon pendu, et je dis : « Baban. »\par
BABAN, pour appeler celle qui m’a donné le jour !\par
\emph{Oh ! baban, ba bère} ! pour dire : « Maman, ma mère. »\par
En classe, quand je récite le premier chant de l’\emph{Iliade}, je dis : \emph{Benin, aeïde ! – atchiou ! theia Beleiadeo, – atchiou} !\par
Je traîne dans le ridicule le vieil \emph{Hobère} ! Atchoum ! Atchoum ! Zim, mala ya, boum, boum !\par
\bigbreak
\noindent Quelquefois le rhume ne vient pas, et je parle simplement comme un trombone qui a un trou, – où j’ai le nez. Je représente bien l’homme tel qu’un philosophe l’a dépeint, un tube percé par les deux bouts.\par
\bigbreak
\noindent Rien de meilleur pour une tête d’enfant, dit le proviseur parlant de l’exercice de purification nasale dont ma mère lui a parlé. Rien de meilleur pour en faire une pâte, oui.\par
Je suis malgré ou \emph{balgré} tout, – avec ou sans \emph{atchiou, atchoum}, – d’une force \emph{énorbe} en récitation. Ma mémoire prend ça comme mon nez prend l’eau, et je renifle des chants entiers de l’\emph{Iliade} et des chœurs d’Eschyle, du Virgile et du Bossuet, – mais ça part comme c’est venu. J’oublie le Bossuet comme on oublie l’aloès bienfaisant.\par

\labelblock{LES MATHÉMATIQUES}

\noindent « Il a une imagination de feu, cet enfant. »\par
C’est acquis. Je suis un petit volcan (dont la bouche sent souvent le chou : on en mange tant à la maison !).\par
\bigbreak
\noindent « Une imagination de feu, je vous dis ! ah ! ce n’est pas lui qui sera fort en mathématiques ! »\par
\bigbreak
\noindent On a l’air d’établir qu’être fort en mathématiques c’est bon pour ceux qui n’ont rien \emph{là.}\par
Est-ce qu’à Rome, à Athènes, à Sparte, il est question de chiffres, une minute ! Justement je n’aime pas faire des soustractions avec des zéros, et je ne comprends rien à la preuve de la division, rien, rien !\par
Mon père en rit, le professeur de lettres aussi.\par
Je suis toujours dans les six derniers.\par
Mais un beau jour, une nouvelle se répand.\par
Grand étonnement. Rumeur dans la cour, sous les arcades.\par
J’ai été premier en géométrie.\par
Le professeur de lettres me fait un peu la mine. Suis-je un volcan – ou n’en suis-je pas un ?…\par
Le coup est tellement inattendu qu’on se demande si je n’ai pas pillé, copié, \emph{truqué}, et l’on m’appelle au tableau pour voir si je m’en tirerai la craie à la main.\par
Je m’en tire, et j’ajoute même à la leçon. Je me tourne vers mes camarades et je leur explique le problème en faisant des gestes, en prenant des livres, en ramassant des bouts de bois ; je roule des cornets, je bâtis des figures et je ne m’arrête que quand le professeur me dit d’un air blessé :\par
« Est-ce que vous avez bientôt fini votre manège ? Est-ce vous qui faites le cours, ou moi ? »\par
Je remonte à ma place au milieu d’un murmure d’admiration.\par
À la fin de la classe, on m’interroge :\par
« Comment as-tu donc fait ! Quand as-tu appris ? » Comment j’ai appris ?\par
\bigbreak
\noindent Il y a dans une petite rue une maison bien triste avec quelques carreaux cassés qu’on a emplâtrés de papier ; une cage noire pend à la fenêtre du second, au-dessus d’un pot de fleurs qui grelotte au vent.\par
Là demeure un pauvre, un Italien proscrit.\par
La première fois que je le vis, je frissonnai ; j’étais ému. Tout le passé de mes versions allait m’apparaître en chair et en os, représenté par un homme qui s’était baigné dans le Tibre : Tacite, Tite-Live, le cheval de César, la chèvre de Septimus, la torche de Néron !…\par
Mais comme ce logement est triste !\par
Une petite lampe qui brûle sur une table chargée de vieux livres, un chien qui me regarde en faisant les yeux blancs, et un homme à cheveux gris, avec de grosses lunettes, qui raccommode une culotte en guenilles.\par
C’était le Romain.\par
« Je viens de la part de mon père, M. Vingtras… »\par
Je lui remis une lettre qu’on m’avait chargé de porter. Il lut, je le suivais des yeux.\par
Quoi ! il venait de Rome ? Il était du pays des gladiateurs, ce vieux tout gris, qui avait l’air d’un hibou dans une échoppe de savetier et qui mettait un fond à son pantalon.\par
C’était son \emph{vexillum}\footnote{Drapeau.} à lui, et cette aiguille était son épée ? Où donc son casque et son bouclier ? Il a un tricot de laine…\par
En regardant, je vis qu’il lui manquait trois doigts à la main ; c’était laid, ces bouts d’os ronds, et les autres doigts qui restaient avaient l’air de deux cornes.\par
Il trembla un peu en refermant la lettre.\par
« Vous remercierez bien votre père », dit-il.\par
Il me sembla qu’il avait une tache brillante, une goutte d’eau dans les yeux.\par
Il pleurait, – mais est-ce que les Romains pleuraient ?\par
Je commençais à croire qu’on s’était trompé ou qu’il avait menti ; il me tendait un petit livre.\par
« C’est moi qui l’ai fait, dit-il. Aimez-vous les mathématiques ?… »\par
Il vit que non à mon air.\par
« Non ! – Eh bien ! mon livre vous plaira peut-être tout de même. Tenez, il y a une boîte avec. »\par
Il me conduisit jusqu’à la porte, tenant toujours sa culotte, et relevant ses lunettes avec ses bouts de doigts je l’entendis qui disait à son chien :\par
« C’est une leçon de quarante sous ; tu auras de la pâtée ; moi, j’aurai du pain. »\par
Il avait été adressé à mon père, par hasard, et mon père lui avait trouvé une répétition ; c’était l’objet de la lettre.\par
« Aimez-vous les mathématiques ? »\par
Il ne voyait donc pas tout de suite que j’étais un \emph{volcan} ? Est-ce qu’il les aimait, lui ? Est-ce que c’était une âme de teneur de livres, ce descendant de Romulus ? Il n’avait vraiment rien du civis et du \emph{commilito}\footnote{\emph{Civis} : citoyen, \emph{commilito} : compagnon d’armes.}, avec son pantalon et ses lunettes !\par
Qu’y avait-il dans sa boîte ?\par
Des plâtres en tranches.\par
Et dans ce livre ? Des mots de géométrie.\par
\bigbreak
\noindent Le lendemain, un dimanche, au lieu d’aller chez un camarade, comme mon père me l’avait permis, je passai ma journée avec ce livre et ces plâtres.\par
C’est le samedi suivant que j’étais premier.\par
J’allai tout joyeux en faire part à cet homme, qui me raconta son histoire.\par
Il avait failli mourir sous les coups des agents du roi de Naples, qui étaient venus pour l’arrêter comme conspirateur, et contre lesquels il s’était défendu pour sauver des papiers qui compromettaient d’autres gens. C’est là qu’il avait eu les doigts hachés. Il avait pu se traîner dans un coin ; on l’avait ramassé, sauvé, et il était passé en France.\par
« Conspirateur ! Vous étiez conspirateur ?\par
— J’étais maçon, heureusement. J’ai profité de ce que je savais de mon métier pour faire ces modèles de géométrie. À propos : vous avez compris mon système, il paraît.\par
— Il n’y a qu’à regarder et à toucher. Tenez, voulez-vous que je vous explique ? »\par
Prenant les plâtres que je trouvais sous la main, je refis ma démonstration.\par
« C’est ça ! c’est ça ! disait-il en hochant la tête. On veut enseigner aux enfants ce que c’est qu’un cône, comment on le coupe, le volume de la sphère, et on leur montre des lignes, des lignes ! Donnez-leur le cône en bois, la figure en plâtre, apprenez-leur cela, comme on découpe une orange ! – De la théologie, tout leur vieux système ! Toujours le bon Dieu ! le bon Dieu !\par
— Qu’est-ce que vous dites du bon Dieu ?\par
— Rien, rien. »\par
Il eut l’air de sortir d’une colère, et il me reparla de la géométrie avec des fils et du plâtre.
\section[{21. Madame Devinol}]{21. \\
Madame Devinol}\renewcommand{\leftmark}{21. \\
Madame Devinol}

\noindent « M. Vingtras, quand Jacques sera premier, je l’emmènerai au théâtre avec moi.\par
« Voulez-vous ? »\par
C’est Mme Devinol qui demande cela. Elle a un fils dans la classe de mon père, qui est un cancre et un \emph{bouzinier}. Si M. Devinol n’était pas un personnage influent, riche, on aurait mis le moutard à la porte depuis longtemps.\par
Mais sa mère est distinguée, un peu trop brune peut-être : les yeux si noirs, les dents si blanches ! Elle vous éclaire en vous regardant. Elle vous serre les mains quand elle les prend. C’est doux, c’est bon.\par
« Pourquoi deviens-tu rouge ? » me demanda-t-elle brusquement.\par
Je balbutie et elle me tape sur la joue en disant :\par
« Voyez-vous ce grand garçon !… Oui, je l’emmènerai au théâtre chaque fois qu’il sera premier. »\par
Cela flatte mon père qu’on me voie dans la société d’une si importante personne, mais cela étonne beaucoup ma mère.\par
« Vous n’avez pas peur qu’il vous fasse honte ?\par
— Honte ! – Mais savez-vous qu’il a de la tournure, votre fils, un petit mulâtre, et qui marche comme un soldat !\par
— Il a un bien gros ventre ! dit ma mère. On ne le dirait pas… mais Jacques a beaucoup de ventre. »\par
Moi, du ventre ! Je fais des signes de protestation.\par
« Oui, oui, c’est comme ça ; peut-être moins maintenant, mais tu as eu le \emph{carreau}, mon enfant. (Se tournant vers madame Devinol.) Je dissimule ça par la toilette. »\par
\bigbreak
\noindent Madame Devinol sourit en me regardant.\par
« Moi, il me plaît comme il est. Veux-tu prendre ton chapeau, mon ami, et m’accompagner ?\par
Quel chapeau ? Le gris ? Celui des \emph{classes moyennes}, qui me fait ressembler à Louis-Philippe ?\par
Ma mère consent à me laisser sortir avec ma casquette.\par
J’ai par hasard un habit assez propre, gagné à la loterie. Il y avait une tombola. Une maison de confection avait offert un costume ; ma mère avait pris un numéro au nom de son enfant.\par
Le numéro est sorti.\par
« Tu le vois, mon fils, la vertu est toujours récompensée.\par
— Et ceux qui n’ont pas gagné ?\par
— Les desseins de Dieu sont impénétrables. Ce n’est pas tout laine, par exemple. »\par
\bigbreak
\noindent Madame Devinol m’emmène.\par
« Donne-moi ton bras, pas un petit bout de rien du tout… Comme ça, là ; très bien ! Je puis m’appuyer sur toi ; tu es fort. »\par
Je ne sais pas comment je n’éclate pas brusquement, d’un côté ou d’un autre, tant je gonfle et raidis mes muscles pour qu’elle sente la vigueur du biceps.\par
« Et maintenant, dis-moi, il y a donc une histoire sur ce chapeau gris ? Et puis, tu as eu le \emph{carreau ;} tu as bien des choses à me conter ! »\par
Je perds contenance, je rougis, je pâlis. Ah ! bah ! tant pis ! Je lui conte tout.\par
Elle rit, elle rit à pleine bouche, et elle se trémousse en disant :\par
« Vrai, la\emph{ polonaise}, le gigot ! »\par
Et ce sont des \emph{ah ! ah} ! sonores et gais comme des grelots d’argent.\par
Je lui narre mes malheurs.\par
J’ai jeté mon chapeau gris par-dessus les moulins, et je lui ai dévidé mon chapelet avec un peu de verve ; je crois même que je l’ai tutoyée à un moment ; je croyais parler à un camarade.\par
« Ça ne fait rien, va, reprend-elle en s’apercevant de ma peur. Je te tutoie bien, moi. Vous voulez bien qu’on vous tutoie, monsieur ? C’est que je pourrais être ta maman, sais-tu ? »\par
Fichtre ! comme j’aurais préféré ça !\par
« Je suis une vieille… Me trouves-tu bien vieille, dis ? »\par
Elle me regarde avec des yeux comme des étoiles.\par
« Non, non !\par
— Tu me trouves jolie ou laide ? Tu n’oses pas me répondre ? C’est que tu me trouves laide alors, trop laide pour m’embrasser…\par
— Non… oh ! non !..\par
— Eh bien ! embrasse-moi donc, alors… »\par
\bigbreak
\noindent Elle me mène au spectacle chaque fois que je suis premier, comme c’est convenu.\par
Il y a un mois que nous nous connaissons.\par
« Tu aimes à venir avec moi ? me demanda-t-elle un jour.\par
— Oui, madame ; moi, j’aime bien le théâtre, je me plais beaucoup à la comédie. »\par
Une fois, à Saint-Étienne, on m’avait mené voir \emph{les Pilules du Diable} ; j’étais sorti fou, et je n’avais fait que parler, pendant deux mois, de Seringuinos et de Babylas. C’était des drames, maintenant ; quelquefois de l’opéra. Il n’y avait plus tant de décors ! Mais comme je prenais tout de même à cœur la misère des orphelins, les malheurs du grand rôle ! Et \emph{les Huguenots}, avec la bénédiction des poignards ! La \emph{Favorite}, quand mademoiselle Masson chantait :\par
« Ô mon Fernand ! »\par
Elle dénouait ses cheveux, tordait ses bras :\par

\begin{quoteblock}
\noindent Ô mon Fernand, tous les biens de la terre !\end{quoteblock}

\noindent Elle disait cela avec son âme, et comme si elle était une de ces chrétiennes dont on nous racontait le martyre au collège, mais ce n’était pas le ciel qu’elle priait, c’était un grand brun, qui avait une moustache noire, des bottes molles.\par
Ce n’était donc pas pour le bon Dieu seulement qu’on soupirait fort et qu’on tournait les yeux !\par


\begin{verse}
Oh ! viens dans une autre patrie !\\
Viens cacher ton bonheur…\\
\end{verse}

\noindent Mes jambes tremblaient, et mon col se mouillait sur ma nuque ; – la mère Vingtras disait que ces soirées, c’était la mort du linge.\par
Même avant que le rideau fût levé, je me sentais grandi et pris d’émotion.\par
J’ouvrais les narines toutes larges pour humer l’odeur de gaz et d’oranges, de pommades et de bouquets, qui rendait l’air lourd et vous étouffait un peu. Comme j’aimais cette impression chaude, ces parfums, ce demi-silence !… ce froufrou de soie aux \emph{premières}, ce bruit de sabots au \emph{paradis} ! Les dames décolletées se penchaient nonchalamment sur le devant des loges ; les voyous jetaient des lazzis et lançaient des programmes. Les riches mangeaient des glaces ; les pauvres croquaient des pommes ; il y avait de la lumière à foison !\par
J’étais dans une île enchantée ; et devant ces femmes qui tournaient la traîne de leurs robes, comme des sirènes dans nos livres de mythologie tournaient leur queue, je pensais à Circé et à Hélène.\par
Il y avait le gémissement du trombone, le pleur du violon, le \emph{pchhh} des cymbales, en notes sourdes comme des chuchotements de voleur, quand les musiciens entraient un à un à l’orchestre et essayaient leurs instruments.\par
\bigbreak
\noindent Lorsque Mlle Masson était en scène, j’oubliais que Mme Devinol était là.\par
Elle s’en apercevait bien.\par
« Tu l’aimes plus que moi, n’est-ce pas ?\par
— Non !… oui !… je l’aime bien. »\par
\bigbreak
\noindent Mme Devinol était venue me prendre un peu plus tôt, certain jour, pour faire un tour, et nous flânions près du théâtre.\par
Nous croisons une dame en chemin.\par
« La reconnais-tu ?\par
— Qui ?\par
— Cette femme, là-bas, qui passe près du café, avec un mantelet de soie. »\par
Je regarde.\par
« Mlle Masson ? »\par
Je ne suis pas encore bien sûr.\par
« Oui, \emph{mon Fernand} », fit Mme Devinol en riant…\par
Quelle désillusion ! Elle avait presque la figure d’un homme, puis trop de choses au cou : un fichu, une dentelle, un boa, – je ne sais quoi aussi en poil ou en laine, qui pendait à sa ceinture, trop gros, et elle relevait mal sa jupe.\par
« Eh bien ! » me dit Mme Devinol.\par
À ce moment même, le directeur du théâtre passa et salua l’actrice qu’il vit la première, Mme Devinol ensuite.\par
Elles répondirent à son salut : l’actrice comme tout le monde, Mme Devinol avec une inclination de tête, et un jeu de paupières qui lui donnèrent une petite mine de religieuse, mais si jolie, et un air fier, mais si fier !\par
Le directeur disparu, elle s’appuya de nouveau sur mon bras.\par
« Eh bien ! l’aimes-tu toujours mieux que moi ?\par
— Oh ! non ! par exemple !\par
— Il dit cela de si bon cœur ! grand gamin, va ! On me préfère alors ? »\par
\bigbreak
\noindent Quand je suis dans sa baignoire, elle me fait asseoir près d’elle, tout près.\par
« Encore plus près. Je te fais donc peur ? »\par
Un peu.\par
\bigbreak
\noindent Comme je bûche mes compositions maintenant !\par
De temps en temps je rate mon affaire tout de même. Je ne suis pas premier.\par
Oh ! une fois ! en vers latins !\par
On nous avait donné à raconter la mort d’un perroquet. J’ai dit tout ce qu’on pouvait dire quand on a à parler d’un malheur comme celui-là : que jamais je ne m’en consolerais, que Caron en voyant passer la cage – cercueil aujourd’hui, – en laisserait tomber sa rame, que d’ailleurs j’allais l’ensevelir moi-même ! – \emph{triste ministerium}, – et que nous verserions des fleurs. \emph{Manibus date lilia plenis}.\footnote{« Donnez des lys à pleines mains. » (Virgile, Énéide, VI, 863.)}\par
Dans un vers ingénieux, je m’étais écrié : « Maintenant, hélas ! vous pouvez planter du persil sur la tombe ! »\par
Le professeur a rendu hommage à ce dernier trait, mais je ne dois passer qu’après Bresslair, dont l’émotion s’est encore montrée plus vive, la douleur plus vraie. Il a eu l’idée, comme dans les cantiques, de mettre un refrain qui revient :\par
Psittacus interiit ! Jam fugit psittacus, eheu !\par
\emph{Eheu}, quatre fois répété ! Je ne puis pas crier à l’injustice. Oh ! c’est bien !\par
Je ne suis que second, et je n’irai pas au théâtre. C’est à s’arracher les cheveux : et je m’en arrache. Je les mets même de côté. Qui sait ?\par
Ils sont gras comme tout, par exemple ! Car je me pommade, maintenant. J’ai soin de moi. Je me rase aussi. Je voudrais avoir de la barbe.\par
\bigbreak
\noindent Mon père cache ses rasoirs. J’ai pris un couteau que je fourre sous mon matelas, parce qu’il a le fil tout mince et tout bleu. Je l’ai usé à force de frotter sur la machine.\par
Le matin, au lever du soleil, je le tire de sa retraite, et je me glisse, comme un assassin… dans un lieu retiré.\par
Je ne suis pas dérangé. Il est trop tôt !\par
Je puis m’asseoir.\par
J’accroche un miroir contre le mur, je fouette mon savon, je fais tous mes petits préparatifs, et je commence.\par
Je racle, je racle, et je fais sortir de ma peau une espèce de jus verdâtre, comme si on battait un vieux bas.\par
\bigbreak
\noindent J’attrape des entailles terribles.\par
Elles sont souvent horizontales – ce qui fait beaucoup réfléchir le professeur d’histoire naturelle, qui demeure au second, et qui me prend la tête quand il a le temps.\par
« Ou cet enfant se penche de côté exprès, pour que le chat puisse l’égratigner, ce qui n’est pas dans la nature humaine… »\par
Il s’arrête pensif et m’interroge.\par
« Te penches-tu pour qu’il t’égratigne ?\par
— Quelquefois. (Je dis ça pour me ficher de lui.)\par
— Pas toujours ?\par
— Non, m’sieu.\par
— Pas toujours ! – C’est donc les mœurs du chat qui changent… Après avoir été donné, pendant des siècles, de haut en bas, le coup de patte est donné maintenant de droite à gauche… Bizarrerie du grand Cosmos ! métamorphose curieuse de l’animalisme ! »\par
Il s’éloigne en branlant la tête.\par
\bigbreak
\noindent Nous étions au théâtre. Mme Devinol me dit :\par
« Tu as l’air tout drôle aujourd’hui. Qu’as-tu donc ? Tu es fâché ?… »\par
Fâché ! elle croit que je puis être fâché contre elle, moi qui ai quinze ans, des lacets de cuir, qui ai un pensum à faire pour demain, moi l’indécrottable !\par
Je ne suis pas fâché. Mais je me suis, hier, presque coupé le bout de nez en me rasant, et j’ai une petite place rose comme une bague.\par
Je dirai tout de même : « Je suis fâché ! »\par
C’est commode comme tout. J’ai un prétexte pour lui tourner le dos et cacher mon nez.\par
\bigbreak
\noindent Je m’arrangeai pour n’être pas premier, tant que la cicatrice fit anneau, et pour n’être pas là quand elle venait à la maison. Enfin, il ne resta qu’une petite place blanche d’un côté. Je pus lui parler de profil.\par
Quelles soirées !\par
Nous revenons du théâtre ensemble et tout seuls quelquefois. Son mari ne s’occupe point d’elle. Il est toujours au \emph{Café des acteurs}, où l’on fait la partie après le spectacle. C’est un joueur. Elle prend mon bras la première, et elle le presse. Elle languit contre moi. Je sens depuis son épaule jusqu’à ses hanches. Il y a toujours une de ses mains qui me touche la main ; le bout de ses doigts traîne sur mon poignet entre ma manche et mon gant.\par
Arrivés à sa porte, nous revenons sur nos pas, et nous recommençons ce manège jusqu’à ce qu’elle se dégage elle-même d’un geste lent et sans me lâcher.\par
« Tu me retiens toujours si longtemps… »\par
Moi ! Mais je ne l’ai jamais retenue, j’ai même été si étonné le premier jour où, au lieu de rentrer, elle a voulu se promener encore et rôder en chatte sur le trottoir, où sonnaient ses bottines ! Elle relevait sa robe et je voyais le chevreau qui moulait sa cheville, en se fronçant quand elle posait son petit pied ; elle avait un bas blanc, d’un blanc doré comme de la laine, un peu gras comme de la chair.\par
Elle s’arrêta deux ou trois fois.\par
« Est-ce que je n’ai pas perdu mon médaillon ? »\par
Elle cherchait dans son cou mat, et elle dut défaire un bouton.\par
« Tu ne le vois pas ? dit-elle. – Oh ! il aura glissé ! »\par
Ses doigts tournaient dans sa collerette, comme les miens dans ma cravate quand elle serre trop.\par
« Aide-moi… »\par
Au même moment le médaillon jaillit et brilla sous la lune.\par
On aurait dit qu’elle en était furieuse.\par
« Tu as perdu quelque chose aussi, fit-elle, d’une voix un peu sèche, en voyant que je me baissais.\par
— Non, je lace mes souliers. »\par
Je lace toujours mes souliers parce que les lacets sont trop gros et les œillets trop petits, puis il y a une boutonnière qui a crevé.\par
\bigbreak
\noindent « Jacques, si tu es premier pour le second samedi du mois, je t’emmènerai à Aigues-la-Jolie. Je dirai à mon mari que je vais chez la nourrice de Joséphine, et nous partirons pour la campagne tous les deux \emph{en garçons}. Nous mangerons des pommes vertes dans le verger, et puis des truffes dans un restaurant. »\par
Des truffes ? Ah ! j’ai besoin de lacer mes souliers !\par
J’ai entendu parler des truffes une fois par un ami de mon père, devant ma mère qui a rougi.\par
\bigbreak
\noindent Je suis premier, parbleu !\par
J’ai accouché d’une poésie latine qui a soulevé l’admiration.\par
« Ne croirait-on pas entendre le gallinacé ? » a dit le professeur.\par
Il s’agissait encore d’un oiseau, – d’un coq.\par
Et j’avais fait un vers qui commençait :\par
\emph{Caro, cara, canens…} (harmonie imitative.)\par
Nous irons donc à la campagne, comme c’est convenu.\par
\bigbreak
\noindent Nous nous trouverons dans la cour de l’auberge où est la diligence pour Aigues. Le conducteur achève d’habiller les chevaux.\par
Je m’étais caché au coin de la rue pour \emph{la} voir venir, et je ne suis arrivé qu’après elle ; j’avais peur de rester là tout seul. Si l’on m’avait demandé : « Qui attendez-vous ? »\par
Elle m’a dit qu’il faudrait l’appeler « ma tante » devant le monde. Elle m’a dit cela hier, et elle me le répète aujourd’hui, en montant dans la voiture.\par
Il arrive une goutte d’eau, comme un crachat, sur la vitre du coucou.\par
Le ciel devient sombre – un coup de tonnerre au loin, – la pluie à torrents.\par
Un voyageur de l’impériale demande si on peut lui donner asile. On n’ose lui refuser, mais chacun se fait gros pour ne pas l’avoir à son côté.\par
Ma \emph{tante} seule se fait mince et montre qu’il y a de la place à sa gauche, de son côté.\par
Elle est bonne et se sacrifie ; elle appuie à droite, elle est presque assise sur moi, qui en ai la chair de poule…\par
À chaque coup de tonnerre, elle fait un saut et paraît avoir bien peur. Je crains qu’elle ne voie la petite cicatrice qui fait anneau, et je ne sais où mettre mon nez. Mais comme c’est doux, cette femme à moitié dans mes bras, et dont le souffle me fait chaud dans le dos !…\par
\bigbreak
\noindent Nous sommes arrivés ; il pleut toujours. Elle se retrousse, sous le porche, pendant qu’on dételle la diligence dont la bâche ruisselle, et que j’étire mes jambes moulues. « Il n’y a pas moyen d’avoir une voiture ?\par
— Une voiture, pour aller aux Aigues, avec des chemins larges d’un pied, et des ornières comme des cavernes ! Vous plaisantez, ma petite dame !\par
— Dis donc, Jacques ! Qu’allons-nous devenir ? »\par
Elle me regarde, et elle rit.\par
« S’il y avait une chambre où s’abriter en regardant l’orage.\par
— Nous en avons une, dit l’aubergiste.\par
— Ah ! »\par

\labelblock{DANS LA CHAMBRE}

\noindent « Je me sens toute mouillée, sais-tu… »\par
Comment ! le temps d’aller de la voiture sous le porche !\par
« Toute mouillée. – J’ai de l’eau plein le cou. Ça me roule dans la poitrine. Oh ! c’est froid… Il faut que j’ôte ma guimpe… Tu permets ! Je vous fais peur, monsieur ? »\par

\astertri

\noindent Des cris, une explosion de cris ! On m’appelle…\par
« Vingtras ! Vingtras ! »\par
Ils sont dix à demander Vingtras.\par
C’est la seconde étude qui est venue en promenade de ce côté et qui s’est précipitée dans l’auberge. Je vois cela à travers le rideau. Mme Devinol saute sur la porte et la ferme à clef ; puis elle se ravise. « Non, sors plutôt ; va, va vite ! » Je cherche mon chapeau, qui n’y est pas.\par
« Avez-vous vu mon chapeau ?\par
— Sors donc, que je referme !\par
— Oui, oui ; mais qu’est-ce que je dirai ?\par
— Tu diras ce que tu voudras, IMBÉCILE. »\par
\bigbreak
\noindent Voici ce qui s’était passé. En entrant dans l’auberge on avait remarqué sur une table un pardessus bizarre, c’était le mien, et mon chapeau à gros poils. On m’avait reconnu !…\par

\labelblock{ÉPILOGUE}

\noindent Je suis forcé de quitter la ville. On a jasé de mon aventure.\par
Le proviseur conseille à mon père de m’éloigner.\par
« Si vous voulez, mon beau-frère le prendra à Paris, à prix réduit, comme il est fort, dit le professeur de seconde. Voulez-vous que je lui écrive ?\par
— Oui, mon Dieu, oui », dit mon père, qui a envie d’aller faire un tour à Paris ; et c’est une occasion.\par
On fixe le chiffre. Je me jette dans les bras de ma mère ; je m’en arrache, et en route !\par
Nous courons sur Paris.
\section[{22. La pension Legnagna}]{22. \\
La pension Legnagna}\renewcommand{\leftmark}{22. \\
La pension Legnagna}

\noindent Je suis à Paris.\par
J’y suis arrivé avec une fluxion. Legnagna, le maître de pension, m’a accueilli avec étonnement. Il a dit à sa femme : « Ce n’est pas un élève, c’est une vessie. »\par
Enfin, cela n’empêche pas d’avoir des prix aux concours.\par
« Vous travaillez bien, n’est-ce pas ? »\par
Et moi dont la lèvre tient toute la joue, je réponds :\par
« Boui, boui. »\par
Il m’a trouvé moins fort qu’il ne pensait. Je mets \emph{du mien} dans mes devoirs.\par
« Il ne faut pas mettre du \emph{vôtre}, je vous dis : il faut imiter les Anciens. »\par
Il me parle haut, me fait sentir que je paye moins que les camarades.\par
Il y a fait allusion dès le second jour. Il y avait des épinards. Je n’aime pas les épinards, et voilà que je laisse le plat.\par
Il passait.\par
« Vous n’aimez pas ça ?\par
— Non, monsieur !\par
— Vous mangiez peut-être des ortolans chez vous ? Il vous faut sans doute des perdrix rouges ?\par
— Non ; j’aime mieux le lard ! »\par
Il a ricané en haussant les épaules et s’en est allé en murmurant : « Paysan ! »\par
\bigbreak
\noindent Il donne des soirées, le dimanche ; on m’invite.\par
Je dis toujours : « Sacré mâtin ! » C’est une habitude ; elle me suit jusque dans son salon.\par
« \emph{Mossieu} Vingtras, me crie-t-il d’un bout de la table à l’autre, où avez-vous été élevé ? Est-ce que vous avez gardé les vaches ?\par
— Oui, monsieur, avec ma cousine. »\par
Il en perd la tête et devient tout rouge.\par
« Croyez-vous, madame ! » dit-il à une voisine.\par
Et se tournant vers moi :\par
« Allez au dortoir ! »\par
\bigbreak
\noindent Je suis dans la classe des grands, qui se fichent de moi tant soit peu, mais sans que ça me gêne ; qui ont l’air de faire les malins, et que je trouve bêtes, mais bêtes !… Il y a une gloire, un prix de concours ; il est maigre, vert, a comme la danse de Saint-Guy, se gratte toujours les oreilles, et cherche constamment à s’attraper le bout du nez avec le petit bout de sa langue.\par
Il y a une demi-gloire, – Anatoly.\par
Il est pour les bons rapports entre les élèves et les maîtres ; il voudrait qu’on s’entendît bien, – pourquoi donc ?\par
J’ai l’air \emph{mastoc ;} on me trouve lourd quand je joue aux barres, on me blague comme provincial. Anatoly me protège.\par
« Il se fera, ne l’embêtez pas ! Dans un mois il sera comme nous ; dans deux, vous verrez ! »\par
Oh ! on ne m’embête pas beaucoup ! Je suis solide, et je n’ai pas mes parents pour me rendre timide, honteux, gauche. Ça m’est à peu près égal qu’on me blague, je ne suis pas ébloui par les copains.\par
Ah ! je me faisais une autre idée de ces forts en latin ! Je trouvais la province plus gaie, moi !\par
Ils parlent toujours, mais toujours de la même chose, – de celui-ci qui a eu un prix, de celui-là qui a failli l’avoir ; il y a eu un barbarisme commis par Gerbidon, un solécisme par…\par
« Chez Labadens, tu sais, le petit qui devait avoir le prix de version grecque, il n’est pas venu parce que son père était mort le matin. Labadens a été le chercher en lui promettant qu’il le ramènerait en voiture à l’enterrement. Il n’a pas voulu et a continué à pleurer. »\par
Ils ont l’air de trouver ce petit stupide.\par
La pension mène à Bonaparte.\par
Le mardi, on a le droit de rester pour fignoler sa composition, et je reste jusqu’à ce que le professeur ait eu le temps de tourner le coin ; alors je m’échappe aussi. J’ai devant moi une grande heure, au bout de laquelle j’irai porter chez son concierge la copie qu’on me croit en train de finir.\par
Je flâne dans les rues pleines de femmes en cheveux ; elles sont si gaies et si jolies avec leurs grands sarraux d’atelier ! Je les suis des yeux, je les écoute fredonner, et je les regarde à travers les vitres déjeuner à côté de ciseleurs en blouses blanches et d’imprimeurs en bonnets de papier. C’est tout ce que je regarde.\par
Je n’ai pas envie de voir les monuments, quoiqu’il n’y ait plus de bagages pour m’en empêcher ; je trouve que toutes les pierres se ressemblent, et je n’aime que ce qui marche et qui reluit.\par
Je ne connais donc rien de Paris, rien que les alentours du faubourg Saint-Honoré, le chemin du lycée Bonaparte, la rue Miromesnil, la rue Verte, place Beauvau ; j’y rencontre beaucoup de domestiques en gilet rouge et de femmes de chambre, en coiffe, dont les rubans volent à la brise.\par
\bigbreak
\noindent Le dimanche, nous allons en promenade.\par
Le plus souvent, c’est aux Tuileries, dans l’allée du Sanglier.\par
Ce \emph{Sanglier} ! je le déteste, il m’agace avec son groin de pierre.\par
\bigbreak
\noindent Je m’ennuie moins cependant, à partir du jour où M. Chaillu devient notre pion.\par
Il n’a pas la foi, lui ; il nous laisse nous éparpiller le dimanche, à condition qu’à six heures nous soyons là.\par
Nous, nous filons sur les Hollandais, au Palais-Royal. C’est le café des saint-cyriens et des volailles. On appelle \emph{volailles} ceux qui se destinent aux écoles à uniforme et en ont un déjà, à bande orange, à collet saumon, avec des képis à visières dures, à galons d’or ou d’argent.\par
Quoique \emph{des lettres}, je suis bien avec les volailles, surtout avec les Lauriol. Malheureusement, je n’ai que des semaines de vingt sous, et je suis forcé d’y regarder à deux fois avant de trinquer.\par
Un jour j’ai eu une fière peur. Nous avions joué et j’avais perdu un franc cinquante. À partir de la première partie, je voulais me lever ; je n’ai pas osé.\par
« Allons, allons, reste là ! »\par
Sueur dans le dos, frissons sur le crâne.\par
Je joue mal, et je laisse voir mes dominos. Tout est fini, j’ai la \emph{culotte} !…\par
Par bonheur on se battit. Il s’éleva une querelle entre une volaille jaune et une volaille rouge, entre des nouveaux et des anciens de Saint-Cyr, et les carafons se mirent à voler.\par
Ce fut une mêlée, je m’y jetai à corps perdu.\par
Je comptais sur quelque coup qui me mettrait en pièces.\par
Pas de chance ! Je donne beaucoup et ne reçois rien.\par
Je n’en fus pas moins sauvé tout de même.\par
On nous jeta à la porte, tout un lot, pour débarrasser la place, et je partis vers le Sanglier, devant trente sous aux Hollandais ; mais j’avais jusqu’à l’autre dimanche.\par
Je vendis un discours latin à la composition du mardi, – vingt sous comptant.\par
Je faisais ce commerce quelquefois, je procurais ainsi une bonne place à quelqu’un qui attendait un oncle, ou qui voulait épater pour sa fête, ou qui avait un intérêt quelconque à être \emph{dans les dix}, quoi !\par
Je retournai aux Hollandais, mes trente sous dans le creux de la main. On ne voulut pas mon argent. C’est la caisse de Saint-Cyr ou une souscription des volailles qui avait réglé la \emph{casse} et les consommations.\par
J’eus de l’argent devant moi, et en plus une réputation de friand du coup de poing.\par
N’importe, je reviens toujours pensif de cet estaminet de riches ! Et la nuit, dans mon lit d’écolier, je me demande ce que je deviendrai, moi que l’on destine à une école dans laquelle j’ai peur d’entrer, moi qui n’ai pas, comme ces volailles, ma volonté, mon but, et qui n’aurai pas de fortune.\par
\bigbreak
\noindent Ma vie des dimanches change tout d’un coup.\par
Il y avait au collège de Nantes un élève modèle nommé Matoussaint.\par
Matoussaint vient rester à Paris. Mon père lui a donné une lettre qui l’autorise à me faire sortir le dimanche.\par
Matoussaint n’est libre qu’à deux heures. C’est bien assez de la demi-journée, – nous ne savons que faire jusqu’à cinq heures ; nous ne voulons pas aller au café pour ne pas dépenser notre argent. Il m’a apporté vingt francs de la part de ma mère ; mais je les ménage.\par
Nous tuons mal l’après-midi. – C’est ennuyeux, je trouve, de se promener quand tous les autres se promènent aussi, et qu’on a tous l’air bête. Ah ! si c’était comme en semaine ! On verrait grouiller le monde. Aujourd’hui, on ne fait pas de bruit ; on glisse comme des prêtres.\par
Il faudrait aller à Meudon. Là on rit, on s’amuse.\par
Mais c’est\emph{ dix sous}, de Paris à Meudon ! Attendons qu’on ait fait fortune !\par
« Ça fait du bien de marcher par ce froid-là », dit Matoussaint, – qui veut me faire croire qu’il s’amuse, mais qui grelotte comme un lustre qu’on époussette.\par
J’aimerais mieux me porter plus mal et avoir plus chaud.\par
Les dimanches de pluie, nous allons dans les musées.\par
« On apprend toujours quelque chose, » dit Matoussaint, en entrant dans les galeries.\par
« On apprend quoi ?\par
— Tu contemples les tableaux, les marbres !\par
— Et après ? »\par
Matoussaint m’appelle positif, et me dit avec amertume :\par
« Toi qui as fait de si beaux vers latins ! »\par
C’est vrai, tout de même !\par
Matoussaint me voit ébranlé et continue\par
« Tu renies tes dieux, tu craches sur ta lyre !\par
— Messieurs, crie le gardien en habit vert, en étendant sa baguette et nous montrant du son, si vous voulez cracher, c’est dans le coin. »\par
Cinq heures arrivent enfin. Je ne suis pas fou des chefs-d’œuvre et des monuments, décidément.\par
\bigbreak
\noindent C’est à cinq heures que Lemaître nous rejoint. Lemaître est \emph{calicot} et Matoussaint le tient en petite estime ; il ne comprend que les professions nobles. Cependant, comme Lemaître connaît des \emph{douillards} et des\emph{ rigolos}, il l’accueille à bras ouverts.\par
Il arrive et l’on va prendre l’absinthe à la Rotonde, ou à la \emph{Pissote}, où l’on espère rencontrer Grassot. « Oh ! voici Sainville ! – Non ! Si ! »\par
L’absinthe une fois sirotée dans le demi-jour de six heures, nous filons du côté du Palais-Royal, où l’on doit trouver les amis chez Tavernier. Ils se mettent toujours dans la grande salle, à la table du coin.\par
Nous dînons à trente-deux sous.\par
Les calicots, camarades de Lemaître, sont avec leurs petites amies, bien chaussées, toutes gentilles, et qui rient, qui rient, à propos de tout et de rien…\par
\bigbreak
\noindent Et comme c’est bon ce qu’on mange !\par
\emph{Purée Crécy, côtelettes Soubise, sauce Montmorency}. À la bonne heure ! Voilà comment on apprend l’histoire !\par
Ça vous a un goût relevé, piquant, ces plats et ces sauces !\par
M. Radigon, le loustic de la bande, n’est pas pour toutes ces blagues-là.\par
« Garçon, un pied de cochon grillé… Pour faire des pieds de cochon, prenez vos pieds, grattez-les. »\par
On rit. Moi, je ne dis rien, j’écoute.\par
« Votre ami est muet, M. Matoussaint ? »\par
Je fais une grimace et pousse un son, pour établir que je n’appartiens pas aux disciples de l’abbé de l’Épée. On me discute au coin de la table.\par
« Une tête – des yeux. – Mais il a l’air trop \emph{couenne ! »}\par
Je me rattrape par les tours de force. J’abaisse les poignets, j’écrase les doigts, je soulève la soupière avec les dents, je reste quatre-vingts secondes sans respirer, à la grande peur des gens d’à-côté, qui voient mes veines se gonfler ; les yeux me sortent de la tête.\par
« Je n’aime pas qu’on fasse ça près de moi quand je mange », dit un voisin.\par
Radigon lui-même en a assez.\par
« Ah ! c’est qu’il nous embête à la fin, avec sa respiration ! »\par
\bigbreak
\noindent Après le dîner, il faut que je parte.\par
Les autres élèves de la pension ont jusqu’à minuit. Legnagna – par méchanceté, – exige que je sois là à huit heures.\par
Je quitte la \emph{société} et je redescends du côté du faubourg Saint-Honoré.\par
Il me reste un quart d’heure à assassiner avant de regagner le bahut, mais j’aurais l’air de n’avoir pas su où dépenser mon temps si je reparaissais avant l’heure.\par
J’aimerais mieux être rentré. Je ne crains pas la solitude de ce dortoir où j’entends revenir un à un les camarades. Je puis penser, causer avec moi, ce sont mes seuls moments de grand silence. Je ne suis pas distrait par le bruit de la foule où ma timidité m’isole, je ne suis pas troublé par les bruits de dictionnaires ni les récits de grand concours.\par
Je me souviens de ceci, de cela, – d’une promenade à Vourzac, d’une moisson au grand soleil ! – et dans le calme de cette pension qui s’endort, la tête tournée vers la fenêtre d’où j’aperçois le champ du ciel, je rêve non à l’avenir, mais au passé.\par
\bigbreak
\noindent On m’appelle un jour chez Legnagna.\par
Il me délivre un paquet que ma mère m’envoie ; il a l’air furieux.\par
« Vous emporterez cela aussi », me dit-il.\par
Il me glisse en même temps un pot et me reconduit vers la porte.\par
Je n’y comprends rien, je déplie le paquet. J’y trouve une lettre :\par
\bigbreak
\noindent « \emph{Mon cher fils},\par
« \emph{Je t’envoie un pantalon neuf pour ta fête, c’est ton père qui l’a taillé sur un de ses vieux, c’est moi qui l’ai cousu. Nous avons voulu te donner cette preuve de notre amour. Nous y ajoutons un habit bleu à boutons d’or. Par le même courrier, j’envoie à M. Legnagna un bocal de cornichons pour le disposer en ta faveur.}\par
« \emph{Travaille bien, mon enfant, et relève tes basques quand tu t’assieds.} »\par
\bigbreak
\noindent Il y avait un mot de mon père aussi.\par
Je lui avais écrit que Legnagna essayait de m’humilier, que je voudrais quitter la pension, vu que je souffrais d’être ainsi blessé tous les jours.\par
Mon père m’a répondu une lettre qui m’a tout troublé. Fait-il le comédien ? Est-il bon au fond ?\par
\bigbreak
\noindent « \emph{Prends courage, mon ami ! Je ne veux pas te dire que c’est de ta faute si tu es à Paris… Aie de la patience, travaille bien, paye avec tes prix ta pension, puis tu pourras lui dire ses vérités.} »\par
\bigbreak
\noindent Pas une allusion au passé, rien ? Pas un reproche ; presque de la bonté, un peu de tristesse !… Je lui aurais sauté au cou s’il avait été là.\par
Je ferai comme il l’a dit : j’attendrai et j’essayerai d’avoir des prix.\par
Et cependant comme ce latin et ce grec sont ennuyeux ! Et qu’est-ce que cela me fait à moi les barbarismes et les solécismes !\par
Et toujours, toujours le grand concours !\par
Le professeur s’appelle D***.\par
Il a une petite bouche pincée, il marche comme un canard, il a l’air de glousser quand il rit, et sa perruque est luisante comme de la plume. Il a eu pour la troisième fois le prix d’honneur au concours général ; l’an passé, on l’a décoré, il a une crête rouge. Il parle un peu comme un incroyable, il prononce : « Cicé-on, discou-e, \emph{Alma pa-ens}. »\par
Il est le professeur de latin, il a un français à lui.\par
Quand des élèves ont manqué la classe pour aller au café ou au bain et qu’il aperçoit des bancs vides, il dit :\par
« Je vois ici beaucoup d’élèves qui n’y sont pas. »\par
Le professeur de français s’appelle N***. c’est le frère d’un académicien qui a deux morales au lieu d’une : abondance de bien ne nuit pas.\par
Il est long, maigre et rouge, a une redingote à la prêtre, des lunettes de carnaval, une voix cassée, flûtée, sifflante.\par
De cette voix-là, il lit des tirades d’\emph{Iphigénie} ou d’\emph{Esther}, et quand c’est fini, il joint les mains, regarde le plafond plein d’araignées et crie : « À genoux ! à genoux ! devant le divin Racine ! »\par
Il y a un nouveau qui, une fois, s’est mis à genoux pour tout de bon.\par
Et d’un geste de dédain, chassant le bouquin qu’il a devant lui, le professeur continue :\par
« Il ne reste plus qu’à fermer les autres livres. »\par
Je ne demande pas mieux.\par
« Et à s’avouer impuissant. »\par
C’est son affaire.\par
\bigbreak
\noindent J’ai commencé par avoir de bonnes places en discours français, mais je dégringole vite.\par
De second, je tombe à dixième, à quinzième !\par
Ayant à parler de paysans qui, pour fêter leur roi, trinquent ensemble, j’avais dit une fois :\par
\emph{Et tous réunis, ils burent un} BON \emph{verre de vin}.\par
« UN BON ! – Ce garçon-là n’a rien de fleuri, rien, rien ; je ne serais pas étonné qu’il fût méchant. UN BON ! Quand notre langue est si fertile en tours heureux, pour exprimer l’opération accomplie par ceux qui portent à leurs lèvres le jus de Bacchus, le nectar des Dieux ! Et que ne se souvenait-il de l’image à la fois modeste et hardie de Boileau :\par
\emph{Boire un verre de vin} qui rit dans la fougère ! »\par
C’est que je n’ai jamais compris ce vers-là, moi ! Boire un verre qui se tient les côtes dans l’herbe, sous la coudrette !\par
Je suis sec, plus sec encore qu’il ne croit, car il y a un tas de choses que je ne comprends pas davantage.\par
« Bien peu là-dedans », fait le professeur en mettant un doigt sur son cœur.\par
\bigbreak
\noindent Il s’arrête un moment :\par
« Mais rien là-dedans, bien sûr », ajoute-t-il en se frappant le front, et secouant la tête d’un air de compassion profonde. « Il a une fois réussi parce qu’il avait lu Pierrot, – mais allez, c’est un garçon qui aimera toujours mieux écrire « fusil », qu’\emph{arme qui vomit la mort}. »\par
C’est que ça me vient comme cela à moi ! nous parlons comme cela à la maison ; – on parle comme cela dans celles où j’allais. – Nous fréquentions du monde si pauvre !\par
Je me rejette sur le vers latin, et le vers latin me réussit.\par
\bigbreak
\noindent Il était temps.\par
Je sentais le moment où ce misérable Legnagna, dans son dépit de me voir sans succès, me porterait trop de coups sourds. Je lui aurais, un beau matin, cassé les reins.\par
J’avais même songé une fois à filer pour tout de bon ; non pas pour aller flâner aux Champs-Élysées ou devant les saltimbanques, comme je faisais quand je manquais la classe ; mais pour lâcher la pension du coup, et me plonger, comme un évadé du bagne, dans les profondeurs de Paris.\par
Qu’aurais-je fait ? Je l’ignore.\par
Mais je me suis demandé souvent s’il n’aurait pas autant valu que je m’échappasse ce jour-là, et qu’il fût décidé tout de suite que ma vie serait une série de combats ? Peut-être bien.\par
Ma résolution était presque prise. C’est Anatoly le Pacifique qui la changea, parce qu’il crut bon d’avertir Legnagna.\par
Celui-ci me fit venir et me dit qu’il savait ce que je voulais faire. Il ajouta qu’il avait prévenu le commissaire, et que si je m’échappais, j’appartenais aux gendarmes. Ce mot me fit peur.\par
C’est sur ces entrefaites que je composai une pièce en distiques, qui fut, paraît-il, une révélation. J’aurais le prix si je m’en tirais comme cela au concours.\par
Le prix au concours, je voudrais bien. Ce serait pour payer ma dette, et en sortant de la Sorbonne, en pleine cour, je prendrais les oreilles de Legnagna et je ferais un nœud avec.\par
Le jour du concours arrive.\par
Nous nous levons de grand matin. On nous a donné un \emph{filet} qui est un des trophées de la maison, et l’on y met du vin, du poulet froid. Legnagna me tend la main. Je ne puis pas lui refuser la mienne, mais je la tends mal, et ce geste de fausse amitié est pire que l’hostilité et le silence.\par
« Distinguez-vous… »\par
Il rit d’un rire lâche.\par
Nous partons, Anatoly et moi ; il fait un petit froid piquant.\par
\bigbreak
\noindent Nous arrivons presque en retard.\par
Je n’avais jamais vu Paris par le soleil frais du matin, vide et calme, et je me suis arrêté cinq minutes sur le pont, à regarder le ciel blanc et à écouter couler l’eau. Elle battait l’arche du pont.\par
Il y avait sur le bord de la Seine un homme en chapeau qui lavait son mouchoir. Il était à genoux comme une blanchisseuse ; il se releva, tordit le bout du linge et l’étala une seconde au vent. Je le suivais des yeux. Puis il le plia avec soin et le mit à sécher sous sa redingote, qu’il entrouvrit et reboutonna d’un geste de voleur.\par
Il ramassa quelque chose que j’avais remarqué par terre. C’était un livre comme un dictionnaire.\par
Anatoly me tira par les basques, il fallait partir ; mais j’eus le temps de voir une face pâle, tout d’un coup au-dessus des marches.\par
Je l’ai encore devant les yeux, et toute la journée elle fut entre moi et le papier blanc. Je ferais mieux de dire qu’elle a été devant moi toute ma vie.\par
\bigbreak
\noindent C’est que dans la face de ce laveur de guenille, plus blanc que son mouchoir mal lavé, j’avais lu sa vie.\par
Ce livre me disait qu’il avait été écolier aussi, lauréat peut-être. Je m’étais rappelé tout d’un coup toute l’existence de mon père, les proviseurs bêtes, les élèves cruels, l’inspecteur lâche, et le professeur toujours humilié, malheureux ! menacé de disgrâce !\par
« Je parierais que ce pauvre que je viens de voir sous le pont est bachelier », dis-je à Anatoly.\par
Je ne me trompais pas.\par
Au moment même où l’on nous appelait pour entrer à la Sorbonne, \emph{un Charlemagne} avait crié, montrant une ombre noire qui montait la rue :\par
« Tiens, l’ancien répétiteur de Jauffret ! »\par
C’était la face pâle, l’homme au mouchoir, le pauvre au livre.\par
\bigbreak
\noindent On dicte la composition.\par
Vais-je la faire ? À quoi bon !\par
Pour être répétiteur comme cet homme, puis devenir laveur de mouchoir sous les ponts ? Quelle est son histoire à cet être qui obsède ma pensée ?\par
Je ne sais. Il a peut-être giflé un censeur, pas même giflé, blagué seulement.\par
Il a peut-être écrit un article dans \emph{l’Argus de Dijon} ou \emph{le Petit homme gris} d’Issingeaux, et pour cette raison on l’a destitué.\par
Pas ce métier-là, non, non !\par
Il faut cependant que je me conduise honnêtement, il faut que je fasse ce que je puis.\par
\bigbreak
\noindent Je ne trouve rien, rien, – j’ai du dégoût, comme une fois où j’avais, tout petit, mangé trop de mélasse.\par
Voilà enfin quarante alexandrins de\emph{ tournés.} C’est ma copie.\par
« Tu as fini ? me dit mon voisin.\par
— Oui.\par
— Moi aussi. Veux-tu que nous fassions cuire des petites saucisses ? »\par
Il tire un petit fourneau à esprit-de-vin et le cache entre les dictionnaires, puis il sort un bout de poêle.\par
« Ça va crier, prends garde ! »\par
Le professeur qui surveillait était Deschanel ; c’était un garçon d’esprit, – il entendait cuire les saucisses. – On avait le droit de manger cru dans la longue séance, – il pensa qu’on pouvait manger cuit. Tans pis pour celui qui tenait la casserole au lieu du dictionnaire dans la bataille !\par
« Le café, maintenant. J’aime bien mon café, et toi ? » Celui de Charlemagne fit le café.\par
Il manquait la goutte. On vendit des morceaux de composition, des tranches de copie à des \emph{bouche-trou} de Stanislas et de Rollin qui avaient des faux cols droits, des rondins de drap fin, et de l’argent dans leurs goussets. Nous eûmes une bonne rincette et une petite \emph{consolation}. Pour finir, je me chargeai spécialement du \emph{brûlot}.\par
\bigbreak
\noindent « Ton brouillon ? » fit Anatoly le Pacifique, dès que je rentrai à la pension.\par
Legnagna arriva, et ils l’épluchèrent ensemble.\par
Je sais que ma composition est ratée, et maintenant que le souvenir de la face pâle est moins vif et que les fumées de notre banquet sont évanouies, je me sens chagrin, j’éprouve comme des remords.\par
Legnagna ne me dit pas un mot. Il me jette un regard de haine.\par
\bigbreak
\noindent Le résultat est connu. – Je n’ai rien !\par
Mais Anatoly n’a rien non plus, la classe n’a rien, le collège n’a pas grand-chose. C’est un désastre pour le lycée.\par
Les bûcheurs et les malins n’ont pas fait mieux que moi ; ma conscience est plus calme.\par
La distribution des prix arrive. J’y assiste obscur et inglorieux ! \emph{Fractis occumbam inglorius armis} !\footnote{« Je tombe sans gloire, les armes brisées. »}\par
Et chacun s’en va…\par
\bigbreak
\noindent Moi, je reste.\par
J’attends une lettre de mon père, et des instructions. Rien ne vient. On me laisse ici à la merci de Legnagna, qui me hait.\par
Nous sommes quatre dans la pension.\par
Un qui n’a pas de parents et dont le tuteur envoie la pension, un créole des Antilles qui ne sort que par hasard, et un petit Japonais qui ne sort jamais.\par
Ils payent cher, ceux-là ; moi, je suis engagé au rabais, et je devais avoir des prix. Je n’ai rien eu, et je mange beaucoup.\par
\bigbreak
\noindent J’ai écrit. Si mes parents ne viennent pas demain, si je n’ai pas de réponse, je quitte la maison et je pars.\par
Legnagna me laissera filer, par économie, sans aller chez le commissaire, cette fois.\par
Oh ! ces lettres attendues ! ce facteur guetté ! mes supplications dont mon père et ma mère se rient !\par
J’ai presque pleuré dans mes phrases, en demandant qu’on vînt me chercher, parce que Legnagna me larde de reproches éternels.\par
« C’était bien assez de me nourrir pendant l’année, il faut qu’il me nourrisse encore pendant les vacances ! »\par
\bigbreak
\noindent Un jour une scène éclate ; mon père est en jeu. Legnagna arrive échevelé.\par
« Quoi ! me dit-il en écumant, je viens d’apprendre que monsieur votre père gagne de l’argent, \emph{s’est fait huit mille}, cette année ; je viens d’apprendre que j’ai été sa dupe, que je vous ai fait payer comme à un gueux, quand vous pouviez payer comme un riche. C’est de la malhonnêteté cela, monsieur, entendez-vous ? »\par
Il frappe du pied, marche vers moi…\par
Oh ! non, halte-là ! Gare dessous, Legnagna !\par
Il devine et s’échappe en déchargeant sa colère contre la porte avec laquelle il soufflette le mur.\par
Une fois parti, le bruit de ses injures tombé, je réfléchis à ce qu’il vient de dire, et je lui donne raison.\par
Oh ! mon père ! vous pouviez m’éviter ces humiliations !\par
Est-ce bien vrai que vous n’êtes pas un pauvre ?\par
C’est vrai. – Celui qui a averti Legnagna est son beau-frère lui-même, arrivé de Nantes la veille.\par
Après la scène, Legnagna est venu à moi dans la cour.\par
« Je n’aurais rien dit, fait-il, si votre père vous avait retiré à la fin des classes, mais voilà huit jours qu’on vous laisse ici sans nouvelles ; cela a l’air d’une moquerie, vous comprenez ! »\par
Je balbutie et ne trouve rien à répondre ; je pense comme lui.\par
« Mon père payera ces huit jours.\par
— Il le peut. Votre père a plus gagné que moi cette année, et il n’avait pas besoin de venir demander une remise de trois cent francs sur votre pension. »\par
C’est pour trois cent francs que j’ai tant souffert !
\section[{23. Madame Vingtras à Paris}]{23. \\
Madame Vingtras à Paris}\renewcommand{\leftmark}{23. \\
Madame Vingtras à Paris}

\noindent « Jacques ! »\par
C’est ma mère ! Elle s’avance et, mécaniquement, me prend la tête. Le petit Japonais rit, le créole bâille, – il bâille toujours.\par
Ma tête a été prise de côté, et ma mère a toutes les peines du monde à trouver une place convenable pour m’embrasser.\par
\bigbreak
\noindent On nous a fait entrer dans une chambre où l’on voit à peine clair, c’est le soir, et la bougie que le concierge apporte ne jette qu’une faible lumière.\par
« Comme tu as grandi ! comme tu es devenu fort ! »\par
C’est son premier mot. Elle ne me laisse pas le temps de parler ; elle me tourne, retourne, et vire sur ses petites jambes.\par
« Embrasse-moi donc comme il faut ; va, ne sois pas méchant pour ta mère. »\par
C’est dit d’assez bon cœur. Elle crie toujours :\par
« Tu as si bonne tournure ! Je t’ai apporté un habit à la française ; je te ferai faire des bottes. Mais fais-toi donc voir : de la moustache ! tu as des moustaches ! »\par
Elle n’y peut plus tenir de joie, d’orgueil. Elle lève les mains au ciel et va tomber à genoux.\par
« C’est que tu es beau garçon, sais-tu ! »\par
Elle me dévisage encore.\par
« Tout le portrait de sa mère ! »\par
Je ne crois pas. J’ai la tête taillée comme à coups de serpe, les pommettes qui avancent et les mâchoires aussi, des dents aiguës comme celles d’un chien. J’ai du chien. J’ai aussi de la toupie, le teint jaune comme du buis.\par
Quant à mes yeux, prétendait Mme Allard, la lingère, qui me demanda une fois si je la trouvais potelée, je ne pouvais pas cacher que j’étais Auvergnat ; ils ressemblaient à deux morceaux de charbon neuf.\par
« Tu as l’air sérieux, sais-tu ? »\par
Peut-être bien. Cette année-là a été la plus dure. J’ai été humilié pour de bon, sans gaieté pour faire balance.\par
J’ai aussi un dégoût au cœur. Ma désillusion de Paris a été profonde.\par
Je vois l’horizon bête, la vie plate, l’avenir laid. Je suis dans la grande Babylone ! Ce n’est que cela, Babylone !\par
Les gens y sont si petits ! Je n’ai entendu que parler latin !\par
Dimanche et semaine, j’ai été à la merci de ce Legnagna, qui est né faible, envieux, capon, et que l’insuccès a encore aigri.\par
Ces dix derniers jours m’ont pesé comme un supplice.\par
\bigbreak
\noindent « Pourquoi ne m’écrivais-tu pas ?\par
— Je m’attendais à partir d’un jour à l’autre », dit ma mère.\par
C’était pour épargner un timbre. Je lui parle des reproches de pauvreté qu’on me faisait, des humiliations que j’ai bues.\par
« C’est lui qui parle de notre pauvreté ! Quand il aura gagné ce qu’a gagné ton père cette année, il pourra dire quelque chose…\par
— Mais alors, si mon père a gagné de l’argent, pourquoi ne pas lui avoir payé ma pension au prix des autres, quand je vous ai écrit qu’il m’insultait et que j’étais si malheureux ?\par
— Des insultes, des insultes ? – Eh bien, après ? Est-ce que tu t’en portes plus mal, dis, mon garçon ? Nous aurons toujours épargné trois cents francs, et tu seras bien content de les trouver après notre mort. Il y a trois cents francs et plus, tiens là-dedans… Ce n’est pas lui qui les aura ! »\par
Elle rit et tape sur sa poche.\par
« Il faut faire comme ça dans le monde, vois-tu ; maintenant que tu es grand, tu dois le savoir. Crois-tu par hasard qu’il t’a pris pour tes beaux yeux et pour nous faire la charité ? Non, on t’a pris comme une bonne vache, tu ne vêles pas comme ils veulent, tu n’as pas des prix à leur grand concours. Il fallait choisir mieux : qu’ils te tâtent avant que tu commences. Je vais lui dire son affaire, moi, attends un peu, va ! »\par
\bigbreak
\noindent Je souffre de la voir se fâcher ainsi. Cet homme que je croyais haïr, voilà qu’il me fait de la peine !\par
Tout en m’annonçant ses intentions de le \emph{sabouler} d’importance, ma mère dit :\par
« Fais tes paquets ! »\par
Nous étions déjà dans le corridor, – le concierge y était aussi.\par
« Madame, rien ne peut sortir de la maison.\par
— Les affaires de mon fils ! – Je n’aurais pas le droit de prendre son linge ? Les chaussettes de mon enfant !… C’est votre \emph{Gnagnagna} qui a dit ça ?\par
— Non. C’est le propriétaire, à qui M. Legnagna doit, et qui a donné la consigne. Il y a le boulanger aussi qui a une note, puis le boucher…\par
Triste homme, oui, triste homme ! Il bousculait les pauvres, car il n’y avait pas que moi qu’il traitât mal. Tous ceux qui étaient abandonnés ou à prix réduit recevaient ses crachats, et les petits même recevaient des coups.\par
Il est bête, – on parle de lui comme d’un type, entre pensions. On emploie son nom pour dire cuistre, bêta et un peu cafard.\par
\bigbreak
\noindent Le raisonnement que vient de me tenir ma mère, l’argument de la vache, m’a ôté des scrupules, m’a frappé.\par
Cette vache… c’est vrai ! Ils ne m’ont pas pris pour mes beaux yeux, bien sûr !\par
« Non, va, tu peux être tranquille », a repris ma mère, qui lisait mes réflexions dans mon silence et mon regard.\par
\bigbreak
\noindent Je le plains tout de même, ce malheureux. J’obtiens de ma mère qu’elle ne fasse pas de scène, et nous obtenons du propriétaire qu’il laisse sortir mon trousseau.\par
On quitte la pension, je ne sais comment. On prend un fiacre pour aller rejoindre les malles que ma mère a laissées au bureau de la diligence.\par
Elle murmure toujours des injures contre Legnagna ; ce sont des ricanements, des cris : elle le blague et le bouscule de la voix, du geste, comme s’il était là :\par
« Voulez-vous bien vous taire ! Ah ! si vous m’aviez dit ce que vous lui avez dit ! (Se tournant vers moi.) Tu n’as pas eu de cœur de t’être laissé traiter ainsi ! Ah ! tu n’es pas le fils de ta mère ! »\par
Suis-je un enfant du hasard ? Ai-je été fouetté par erreur pendant treize ans ? Parlez, vous que j’ai appelée jusqu’ici \emph{genitrix}, ma mère, dont j’ai été le \emph{cara soboles}, parlez !\par
\bigbreak
\noindent « Et où allons-nous, maintenant ? »\par
Ma mère me pose cette question quand nous sommes déjà empilés dans la voiture. Le cocher attend.\par
« Nous n’allons pas coucher dans le fiacre, n’est-ce pas ? Voilà un an que tu es à Paris, et tu ne sais pas encore où mener ta mère, tu ne connais pas un endroit où descendre ? »\par
Je connais la Sorbonne ? – Le Sanglier ? – Est-ce qu’on lui ferait un lit aux Hollandais ?\par
« Allons, c’est moi qui vais te conduire ! Ah ! les enfants. »\par
Elle me pousse vers la portière.\par
« Appelle le cocher ?\par
— Cocher ! »\par
Il arrête et se penche.\par
« Connaissez-vous l’Écu-de-France ?\par
— C’est à Dijon, ça, ma bourgeoise !\par
— Dans toutes les villes, il y a un hôtel qui s’appelle l’Écu-de-France.\par
— Connais pas ici ! »\par
Relevant son châle sur ses épaules, prenant son sac de voyage d’une main, elle empoigne la portière de l’autre et saute à terre.\par
« Je ne resterai pas une minute de plus dans cette voiture.\par
— Comme vous voudrez, mes enfants ; j’aime pas trimbaler du monde qui est si \emph{chose} que ça ! Payez l’heure, et voilà vos malles. »\par
Nous payons, – et l’histoire d’Orléans, de la place de la Pucelle, de Nantes et du quai recommence. Nous sommes debout devant des colis et des cartons à chapeau qui s’écroulent. Ma mère ne peut pas entrer dans une ville sans embarrasser la voie !…\par
Elle me donne des coups de parapluie.\par
« Mais remue-toi donc ! »\par
Je remue ce que je peux, il faut que je veille aux cartons, je n’ai pas grand-chose de libre sur moi, tout est pris, il me reste un doigt.\par
« Arrête une autre voiture. »\par
Je fais signe à un nouvel automédon\footnote{Conducteur, cocher.}, mais l’équilibre a des lois fatales qu’il ne faut pas violer, et ce signe me perd ! La montagne de bagages s’écroule. – Ma mère pousse un cri ! Les voitures s’arrêtent, des sergents de ville accourent, – toujours ! toujours ! Quelle spécialité !\par
\bigbreak
\noindent Que serions-nous devenus sans des philanthropes qui passaient par là ?\par
Ils ne nous demandèrent rien qui pût attenter à nos convictions politiques ou religieuses ! Non, rien. Ils nous aidèrent de leurs conseils, sans exiger ni transaction de conscience ni lâcheté. Ce n’est pas les jésuites qui auraient fait ça !\par
Ils nous conseillèrent d’aller en face, « Juste en face, où il y a un écriteau », et ils nous apprirent que les \emph{chambres meublées} étaient pour les gens qui n’en avaient pas.\par
« Tu ne le savais donc pas, Jacques ! dit ma mère. C’est les vers latins qui l’auront rendu comme ça ! ou peut-être un coup. Tu n’es pas tombé sur la tête, dis ?\par
— Non, sur le derrière seulement. »\par
Ma mère paraît un peu plus tranquille.\par
\bigbreak
\noindent Nous sommes installés : une chambre et un cabinet.\par
Des cris dans la chambre de ma mère…\par
« Jacques, Jacques !\par
— Me voilà. »\par
À peine j’ai le temps de passer mon pantalon, mais j’ai tout le mal du monde pour le garder.\par
Elle l’a attrapé par le fond et elle m’attire à elle, à rebours.\par
« Es-tu mon fils ? »\par
Je commence à être sérieusement inquiet. Elle me l’a déjà demandé une fois.\par
Je vois, éparpillées sur la table, deux culottes et deux vestes que j’ai portées toute cette année.\par
Elle me fait tourner brusquement et me fixe comme si elle soupçonnait toujours que je lui ai présenté un étranger à ma place.\par
Enfin, presque sûre que je ne me suis pas trompé, avertie d’ailleurs par la voix du sang, elle laisse échapper sa douleur.\par
« Jacques, dit-elle, Jacques, sont-ce là les culottes, sont-ce là les vestes, est-ce l’habit bleu barbeau que je t’ai envoyés ? Je sais comme un habit est tout de suite sale avec toi, je le sais, mais je ne puis pas croire que tu aies mangé la couleur pour t’amuser, et puis ce que je t’ai envoyé était plus large ! Il y avait une ressource dans le fond, du flottant, de l’air, de la place ! Ici, rien ! rien ! »\par
« Jacques, nous l’avons cousu ensemble, ton père et moi ! Je te l’ai écrit, tu le savais ! – Qu’ont-ils fait de mon fils ? »\par
C’est la troisième fois qu’elle a l’air d’être inquiète ! Je me tâte.\par
« Mais explique-toi, imbécile ! »\par
Oh non, elle m’a bien reconnu.\par
J’explique l’histoire des vêtements.\par
J’avais usé les habits que je portais en arrivant. Ceux qu’on m’avait envoyés, taillés par mon père, cousus par ma mère, étaient trop larges ; il aurait pu tenir quelqu’un avec moi dedans. Je ne connaissais personne.\par
Je suis tombé sur Rajoux qui était deux fois gros comme moi, et qui avait, lui, des habits trop petits.\par
Il m’a demandé si je voulais changer, que j’avais une si drôle de tournure avec ces fonds trop abondants. Ça inquiétait beaucoup de gens de me voir marcher avec difficulté ! Que ne disait-on pas ?\par
Nous avons signé le marché un jour au dortoir ; il m’a donné ses frusques, j’ai pris les siennes, et j’ai pu jouer aux barres de nouveau.\par
\bigbreak
\noindent Ma mère se taisait. J’attendais, accablé ; enfin elle sortit de son silence.\par
« Ah ! ce n’est pas du mauvais drap !… Mais il ne devait rien y connaître, ton Rajoux, tu aurais pu demander quelque chose en retour, un gilet de flanelle, un bout de caleçon. Ah ! si ç’avait été moi ! va ! Oui, le drap est bon. Seulement nous n’avons pas de pièce (examinant un fond rayé) ; pour ce fond là, je ne vois que le tapis de ma chambre. Je pourrai arranger cette doublure avec mes vieux rideaux. »\par
Diable !\par
« Tu ne peux pas faire des conquêtes avec ça, par exemple. Et moi j’aime bien un homme qui a un peu de coquetterie dans sa toilette, – une redingote verte, un pantalon à carreaux… Oh ! je ne voudrais pas qu’on en abuse ! Plaire, mais non pas se lancer dans le vice ; parce qu’on est bien mis, ne pas rouler dans la vie dorée, non ! mais, tu diras ce que tu voudras, un brin d’originalité ne fait pas mal, et je ne t’en aurais pas voulu, si on s’était retourné pour te regarder à mon bras dans la rue. Qui est-ce qui se retournera pour te regarder ? personne ! Tu passeras inaperçu. Enfin, si tu es modeste !… (il y a un peu d’ironie et de désappointement dans l’accent), mais c’est du bon, je ne dis pas que ce n’est pas du bon. »\par
\bigbreak
\noindent « Où me mènes-tu dîner ? »\par
Elle dit ça presque comme Mlle Herminie le disait à Radigon, en me câlinant.\par
Il me va et me touche, cet air bon enfant, et je lui parle tout de suite de Tavernier, à trente-deux sous.\par
« Je voudrais aller une fois aux Frères-Provençaux ou chez Véfour ? – pour une fois, on n’en meurt pas, va ; puis ton père a fait une si bonne année ! »\par
J’ai eu toutes les peines du monde à éviter Véfour. Elle était disposée à ne pas lésiner ; s’il fallait dix francs, on les mettrait ! « Ah ! tant pis ! on fait la noce ! »\par
Dix francs, fichtre ! – j’entrevis la note montant à un louis, ma mère les appelant voleurs. « Je sais le prix de la viande, moi ! Vous ne m’apprendrez pas ce que c’est qu’un rognon. Vingt sous pour un fromage ! »\par
Je mentis un peu, je dis qu’il y avait des amis qui y avaient dîné, et qu’ils m’avaient juré que les côtelettes coûtaient trente sous.\par
« On s’est moqué de toi, mon garçon ! Ah ! tu ne t’es pas plus déluré que ça dans ton Paris ! Tu ne me feras pas croire qu’on demande trente sous pour une côtelette. Mais avec trente sous on peut avoir un petit cochon dans nos pays !\par
— Ce n’est pas si bon qu’on le croit ! (je hasarde cela timidement.)\par
— Si c’est mauvais, je leur savonnerai la tête pour leurs dix francs, sois tranquille ! »\par
Je ne l’étais pas, et je reprends :\par
« Essayons de Tavernier d’abord, crois-moi. »\par
Nous allons chez Tavernier.\par
\bigbreak
\noindent Elle a commencé par dire en entrant : « C’est trop beau ici pour qu’ils donnent bon ; tout ça, c’est du flafla, vois-tu ? »\par
Elle parlait tout haut, comme chez elle, et j’étais tout honteux en voyant la dame \emph{du comptoir des desserts} qui l’entendait.\par
Pour trouver une place, nous avons fait trois fois le tour de la salle. On commence à dire que nous passons bien souvent ! Enfin ma mère paraît fixée.\par
« Nous serons bien ici… – non, de ce côté-là… – Va-t’en voir si nous ne pourrions pas nous mettre près de la fenêtre, au fond. »\par
Je traverse le restaurant, rouge jusqu’aux oreilles.\par
Nous interrompons la circulation des garçons de salle et la délivrance des menus. Il m’arrive deux ou trois fois de m’opposer absolument au passage d’une sole et d’un œuf sur le plat. Le garçon prenait à gauche, moi aussi ! – À droite : il me trouvait encore ! Il allait droit – halte-là !\par
Des paris s’engagent dans le fond.\par
— Passera, passera pas !\par
Ma mère disait : C’est mon fils !\par
« \emph{Je vous en félicite, madame ! »}\par
Je parviens à la rejoindre ; le garçon m’a filé sous le bras, aux applaudissements des spectateurs. Ceux qui ont perdu à cause de moi règlent leurs paris en louchant de mon côté, en me regardant d’un air courroucé.\par
Nous sommes plus forts à deux ; ma mère ne veut plus me quitter.\par
« Restons ensemble ! » dit-elle.\par
Nous nous portons sur un point stratégique qui nous paraît le plus sûr, et nous tenons conseil.\par
On nous regarde beaucoup.\par
« Tu as faim ? mon pauvre enfant ! »\par
Pourquoi m’appelle-t-elle son pauvre enfant devant tout ce monde-là ?\par
Une scie s’organise.\par
« \emph{Va rincer l’pau…}\par
— \emph{Consoler l’pau…}\par
— \emph{Remplir l’pau… vre enfant.} »\par
Mais on est allé avertir le patron, qui mettait du vin en bouteilles. Il arrive avec sa serviette qui frémit sous son bras.\par
« Êtes-vous venus pour dîner ? Voyons ! »\par
Je réponds « non », audacieusement.\par
Étonnement de cet homme, – murmure de la foule.\par
J’ai dit non, parce qu’il avait l’air si furieux !\par
« Vous n’êtes pas venus pour dîner ? Pour quoi faire donc ?\par
— Monsieur, je m’appelle Mme Vingtras, j’arrive de Nantes. – Il s’appelle Jacques, lui ! »\par
On crie bravo dans la salle. – \emph{Écoutez ! écoutez ! laissez parler l’orateur.}\par
Mes oreilles tintent. Je n’entends plus. Je distingue seulement que le patron dit : Il faut en finir !\par
On vint à bout de nous ; on nous accula dans un coin.\par
J’avouai à la fin que nous étions venus pour dîner.\par
On nous servit en se tenant sur la défensive.\par
« Je connais ça, disait un des garçons, un vieux ; ce sont des frimes, ils font les ânes pour avoir du foin, tout à l’heure, ils pisseront à l’anglaise. »\par
« J’aime autant un autre restaurant, et toi ? demande ma mère.\par
— Moi aussi, oh ! oui, moi aussi. Je déteste la chanson : \emph{Rincer l’pau…, vider l’pau…} Nous irons chez Bessay, il est à deux pas justement, et ce n’est que vingt-deux sous. »\par
Ma mère s’installe chez Bessay.\par
« Qu’allez-vous me donner, monsieur le garçon ?\par
— Maman, on ne dit pas \emph{monsieur} le garçon ?\par
— Ah ! tu es devenu impoli, maintenant ! Il ne faut pas être si fier avec les gens, on ne sait pas ce qu’on peut devenir, mon enfant ! »\par
Le garçon n’a pas répondu à la question polie de ma mère, il est occupé avec un client, à qui il dit :\par
« Nous avons une tête de veau, n’est-ce pas ? »\par
Le monsieur fait signe que oui, il ne nie pas, il a bien une tête de veau.\par
\bigbreak
\noindent Le garçon revient à nous.\par
« Voyons, que nous conseillez-vous ? dit ma mère.\par
— Je vous recommande le fricandeau.\par
— Je ne suis pas venue à Paris pour manger ce que je puis manger chez moi, – non. – Que mangeriez-vous, vous-même ? Dites-nous ça ? »\par
Elle compte qu’il lui parlera comme un ami. « Là, voyons, qu’y a-t-il de bon ?… De quel pays êtes-vous ? » Il propose un plat, elle a l’air d’accepter, mais non, non, elle a réfléchi…\par
« Jacques, rappelle-le !\par
— Garçon ? »\par
Je dis ça timidement, comme on sonne à la porte d’un dentiste. J’espère qu’il ne m’entendra pas.\par
« Tu ne vois donc pas qu’il s’en va : cours après lui, cours donc ! »\par
Je rattrape le garçon qui, un pied en l’air, la tête en bas, crie d’une voix de stentor dans l’escalier :\par
« ET MES TRIPES ? »\par
Il se retourne brusquement :\par
« Qu’y a-t-il ?\par
— Ce n’est pas un rôti qu’il faut.\par
— Qu’est-ce qu’il faut, alors ! »\par
Ma mère, du fond de la salle :\par
« Une bonne côtelette, pas très grasse ; si elle est grasse, il n’en faut pas ; avec une assiette bien chaude, s’il vous plaît ! »\par
\bigbreak
\noindent « La côtelette… enlevons !\par
— Je vous ai dit : pas grasse !\par
— Ce n’est pas gras, ça, madame !\par
— Voyons, mon ami, si vous êtes franc… »\par
Le garçon a disparu.\par
Ma mère tourne et retourne la côtelette du bout de sa fourchette ; elle finit par accoucher de cette proposition :\par
« Jacques, va t’informer à la cuisine si on veut te la changer.\par
— Maman !\par
— Si on ne peut pas avoir ce qu’on aime, avec son argent ! Ne dirait-on pas que nous demandons la charité, maintenant ! (d’une voix tendre) : Tu voudrais donc que je mange quelque chose qui me ferait du mal ? Va prier qu’on la change, va, mon ami. »\par
Je ne sais où me fourrer ; on ne voit que moi, on n’entend que nous ; je trouve un biais, et d’un air espiègle et boudeur (je crois même que je mords mon petit doigt) :\par
« Moi qui aime tant le gras !\par
— Tu l’aimes donc, maintenant ? Qu’est-ce que je te disais, quand j’étais forcée de te fouetter pour que tu en manges, – que tu en serais fou un jour ? – Tiens, mon enfant, régale-toi. »\par
\bigbreak
\noindent Je déteste toujours le gras, mais je ne vois que ce moyen pour ne pas reporter la côtelette, puis je pourrai peut-être escamoter ce gras-là. En effet, j’arrive à en fourrer un morceau dans mon gousset, et un autre dans ma poche de derrière.\par
\bigbreak
\noindent Mais un soir ma mère me prend à part ; elle a à me parler sérieusement :\par
« Ce n’est pas tout ça, mon garçon, il faut savoir ce que nous allons faire maintenant. Voilà une semaine que nous courons les théâtres, que nous nous gobergeons dans les restaurants, et nous n’avons rien décidé pour ton avenir. »\par
Chaque fois que ma mère va être solennelle, il me passe des sueurs dans le dos. Elle a été bonne femme pendant sept jours ; le huitième, elle me fait remarquer qu’elle se saigne aux quatre veines, que j’en prends bien à mon aise. « On voit bien que ce n’est pas toi qui gagnes l’argent. Le restaurant, ce n’est que vingt-deux sous pour un, mais pour deux, c’est quarante-quatre sous, sans compter le garçon. Tu as voulu qu’on lui donnât trois sous ! Je les ai donnés, c’est bien, quand deux auraient suffi parfaitement ; si c’était moi, je ne donnerais rien, pas ça ! »\par
Elle a une façon de souligner les plaisirs qu’elle m’offre qui les gâte un peu.\par
Quand nous sommes allés au Palais-Royal, par exemple, il faut que je rie pendant deux jours – pour bien montrer que ça n’a pas été de l’argent perdu. – Si je ne me tords pas les côtes, elle dit : « C’était bien la peine de dépenser quatre francs ! »\par
Je ris autant que je puis ! Dès qu’elle tourne la tête, je me repose un peu, mais ça fatigue tout de même.\par
\bigbreak
\noindent Elle m’a mené voir l’Hippodrome – nous sommes revenus à pied. Elle aime marcher, moi pas. J’ai l’air mélancolique.\par
« Monsieur fait le triste, maintenant ! Tu ne faisais pas le triste quand tu jouais au mirliflore dans une bonne \emph{seconde} et que tu regardais les écuyères. »\par
Au mirliflore ? ? ?\par
« Allons ! Que va-t-on faire de toi ?\par
— Je n’en sais rien !\par
— As-tu une idée ?\par
— Non.\par
— Il faut finir tes classes. »\par
Je n’en vois pas la nécessité.\par
Ma mère devine le fond de ma pensée.\par
« Je parie, – oui, je parie ! – qu’il consentirait à ce que les sacrifices qu’on a faits pour lui soient perdus. Il accepterait de quitter le collège, tenez ! Il laisserait ses études en plan !… »\par
Pour ce que ça m’amuse et pour ce que ça me servira !… (c’est en dedans toujours que je fais ces réflexions).\par
« Mais répondras-tu, crie ma mère, me répondras-tu ?\par
— À quoi voulez-vous que je réponde ?\par
— Que comptes-tu faire ? As-tu une idée, quelque chose en tête ? »\par
Je ne réponds pas, mais tout bas je me dis :\par
Oui, j’ai une idée et quelque chose en tête ! J’ai l’idée que le temps passé sur ce latin, ce grec – ces blagues ! est du temps perdu ; j’ai en tête que j’avais raison étant tout petit, quand je voulais apprendre un état ! J’ai hâte de gagner mon pain et de me suffire !\par
Je suis las des douleurs que j’ai eues et las aussi des plaisirs qu’on me donne. J’aime mieux ne pas recevoir d’éducation et ne pas recevoir d’insultes. Je ne veux pas aller au théâtre le lundi, pour que le mardi on me reproche de m’y avoir conduit ; je sens que je serai malheureux toujours avec vous, tant que vous pourrez me dire que je vous coûte un sou !…\par
Voilà ce que je pense, ma mère.\par
J’ai à vous dire autre chose encore ; – malgré moi, je me souviens des jours, où, tout enfant, j’ai souffert de votre colère. Il me passe parfois des bouffées de rancune, et je ne serai content, voulez-vous le savoir, que le jour où je serai loin de vous !…\par
Ces pensées-là, à un moment, m’échappent tout haut !\par
Ma mère en est devenue pâle.\par
« Oui, je veux entrer dans une usine, je veux être d’un atelier, je porterai les caisses, je mettrai les volets, je balayerai la place, mais j’apprendrai un métier. J’aurai cinq francs par jour quand je le saurai. Je vous rendrai alors l’argent du Palais-Royal, et les trois sous du garçon…\par
— Tu veux désespérer ton père, malheureux !\par
— Laissez-moi donc avec vos désespoirs ! Ce que je veux, c’est ne pas prendre sa profession, un métier de chien savant ! Je ne veux pas devenir bête comme N***, bête comme D***. J’aime mieux une veste comme mon oncle Joseph, ma paye le samedi, et le droit d’aller où je veux le dimanche. »\par

\astertri

\noindent « Et tu voudrais ne plus nous voir, tu dis ? »\par
Elle a oublié toutes les autres colères qui blessent son orgueil, dérangent ses plans, déconcertent sa vie, pour ne se rappeler qu’une phrase, celle où j’ai crié que je ne les aimais pas, et ne voulais plus les voir !\par
Son air de tristesse m’a tout ému ; je lui prends les mains.\par
« Tu pleures ? »\par
Elle n’a pu retenir un sanglot, et avec un geste si chagrin, comme j’en ai vu dans les tableaux d’église, elle a laissé tomber sa tête dans ses mains…\par
Quand elle releva son visage, je ne la reconnaissais plus : il y avait sur ce masque de paysanne toute la poésie de la douleur ; elle était blanche comme une grande dame, avec des larmes comme des perles dans les yeux.\par
« Pardon ! »\par
Elle me prit la main. Je demandai pardon encore une fois.\par
« Je n’ai pas à te pardonner… J’ai à te demander seulement, vois-tu, de ne plus me dire de ces mots durs. »\par
Elle baissa la voix et murmura :\par
« Surtout si je les ai mérités, mon enfant…\par
— Non, non, dis-je à travers mes pleurs.\par
— Peut-être, fit-elle. Je veux être seule ce soir ; tu peux sortir… Laisse-moi. Laisse-moi. »\par
Elle me fit donner la clef – « pour qu’il puisse rester jusqu’à minuit », avait-elle dit à M. Molay, le propriétaire.\par
Je pris le premier chemin qui s’ouvrit devant moi, je me perdis dans une rue déserte, et je pensai, tout le soir, aux paroles touchantes qui venaient d’effacer tant de paroles dures et de gestes cruels…\par
\bigbreak
\noindent « Jacques ? est-ce que tu veux nous accorder cette grâce d’aller encore au collège ?\par
— Oui, mère. »\par
Je ne l’appelai plus que « mère » à partir de ce jour jusqu’à sa mort.\par
« Ah ! tu me fais plaisir ! Merci, mon enfant ! Vois-tu ! J’aurais tant souffert de voir qu’après avoir fait toutes tes classes tu t’arrêtais avant la fin. C’est pour ton père que ça me faisait de la peine. Tu le contenteras, tu seras bachelier, et puis après… Après, tu feras ce que tu voudras… puisque tu serais malheureux de faire ce que nous voulons… »\par
\bigbreak
\noindent Il a été décidé, le lendemain du jour où elle avait pleuré, que l’on ne parlerait plus de l’École normale, et que je préparerais simplement mon baccalauréat.\par
J’ai accepté, heureux d’essuyer avec cette promesse et de laver avec ce sacrifice les yeux de la pauvre femme !\par
Elle ne me parle plus comme jadis.\par
Elle est si grave et a si peur de me blesser !\par
« Je t’ai fait bien souffrir avec mes ridicules, n’est-ce pas ? »\par
Elle ajoute avec émotion :\par
« C’est toi qui me gronderas maintenant. Tu auras la bourse, d’abord. Ne dis pas non, j’y tiens, je le veux. Puis je suis une vieille femme, tu dois t’ennuyer d’être avec moi tout le temps. Je puis très bien rester à causer avec Mme Molay. Elle me mènera voir les belles choses aussi bien que toi. Je veux que tu aies tes soirées, au moins. Revois tes amis, tes camarades ; va chez Matoussaint. »\par
\bigbreak
\noindent J’ai rejoint Matoussaint dans une chambre du quartier latin, où il demeure avec un homme qui a dix ans de plus que lui, qui est jacobin et qui écrit dans un journal républicain. Il fait une histoire de la Convention.\par
Matoussaint écrit sous sa dictée.\par
Ils étaient en train de causer gravement. On m’a fait bon accueil, mais on a continué la conversation.\par
Leurs phrases font un bruit d’éperons :\par
« Un journaliste doit être doublé d’un soldat. » – « Il faut une épée près de la plume. » – « Être prêt à verser dans son écritoire des gouttes de sang. » – « Il y a des heures dans la vie des peuples. »\par
Matoussaint et son ami le journaliste, comme nous l’appelons, m’ont prêté des volumes que j’ai emportés jeudi. Le dimanche suivant, je n’étais plus le même.\par
\bigbreak
\noindent J’étais entré dans l’histoire de la Révolution.\par
On venait d’ouvrir devant moi un livre où il était question de la misère et de la faim, où je voyais passer des figures qui me rappelaient mon oncle Joseph ou l’oncle Chadenas, des menuisiers avec leurs compas écartés comme une arme, et des paysans dont les fourches avaient du sang au bout des dents.\par
Il y avait des femmes qui marchaient sur Versailles, en criant que Mme \emph{Veto} affamait le peuple ; et la pique à laquelle était embrochée la miche de pain noir – un drapeau – trouait les pages et me crevait les yeux.\par
C’était de voir qu’ils étaient des simples comme mes grands-parents, et qu’ils avaient les mains couturées comme mes oncles ; c’était de voir les femmes qui ressemblaient aux pauvresses à qui nous donnions un sou dans la rue, et d’apercevoir avec elles des enfants qu’elles traînaient par le poignet ; c’était de les entendre parler comme tout le monde, comme le père Fabre, comme la mère Vincent, comme moi ; c’était cela qui me faisait quelque chose et me remuait de la plante des pieds à la racine des cheveux.\par
Ce n’était plus du latin, cette fois. Ils disaient : « Nous avons faim ! Nous voulons êtres libres ! »\par
J’avais mangé du pain trop amer chez nous, j’avais été trop martyr à la maison pour que le bruit de ces cris ne me surprît pas le cœur.\par
Puis je déchirais, en idée, les habits si mal bâtis que j’avais toujours portés et qui avaient toujours fait rire ; je les remplaçais par l’uniforme des \emph{bleus}, je me glissais dans les haillons de Sambre-et-Meuse.\par
On n’était plus fouetté par sa mère, ni par son père, on était fusillé par l’ennemi, et l’on mourait comme Barra. \emph{Vive le peuple} !\par
C’étaient des gens en tablier de cuir, en veste d’ouvrier et en culottes rapiécées, qui étaient le peuple dans ces livres qu’on venait de me donner à lire, et je n’aimais que ces gens-là, parce que, seuls, les pauvres avaient été bons pour moi, quand j’étais petit.\par
\bigbreak
\noindent Je me rappelais maintenant des mots que j’avais entendus dans les veillées, les chansons que j’avais entendues dans les champs, les noms de Robespierre ou de \emph{Buonaparte} au bout de refrains en patois ; et un vieux, tout vieux, avec des cheveux blancs, qui vivait seul au bout du village, et qu’on appelait le fou. Il mettait quelquefois sur ses cheveux blancs un bonnet rouge et regardait les cendres d’un œil fixe.\par
Je me rappelais celui qu’on appelait le\emph{ sans-culotte} et qui ne \emph{tolérait} pas les prêtres. Il était sorti de la maison le jour où sa femme, avant de mourir, avait demandé\emph{ le bon Dieu.}\par
Je me souvenais aussi des gestes qu’on avait faits devant moi, en tapant sur la crosse d’un fusil, ou en allongeant le canon, avec un regard de colère, du côté du château.\par
Et tout mon sang de fils de paysanne, de neveu d’ouvriers, bondissait dans mes veines de savant malgré moi !\par
Il me prenait des envies d’écrire à l’oncle Joseph et à l’oncle Chadenas… « Soyez sûrs que je ne vous ai pas oubliés, que j’aurais mieux aimé être avec vous, à la charrue ou à l’étable, qu’être dans la maison au latin. Mais si vous marchez contre les \emph{aristocrates}, appelez-moi ! »\par
\bigbreak
\noindent « Tu as l’air tout exalté depuis quelque temps », dit ma mère.\par
C’est vrai ; – j’ai sauté d’un monde mort dans un monde vivant. – Cette histoire que je dévore, ce n’est pas l’histoire des dieux, des rois, des saints, – c’est l’histoire de Pierre et de Jean, de Mathurine et de Florimond, l’histoire de mon pays, l’histoire de mon village ; il y a des pleurs de pauvre, du sang de révolté, de la douleur des miens dans ces annales-là, qui ont été écrites avec une encre qui est à peine séchée.\par
\bigbreak
\noindent Comme je profite avec passion de la liberté que me laisse ma mère ! J’arrive tous les jours rue Jacob pour mettre le cœur dans les livres qui sont là, ou pour entendre le journaliste parler du drapeau républicain engagé sur les ponts, et défendu par les brigades au cri de : « \emph{Vive la nation} ! – \emph{À bas les rois} ! – \emph{La liberté ou la mort ! »}\par
Être libre ? Je ne sais pas ce que c’est, mais je sais ce que c’est d’être victime ; je le sais, tout jeune que je suis.\par
\bigbreak
\noindent Nous nous imaginons quelquefois avec Matoussaint que nous sommes en campagne, et chacun fait ses rêves.\par
Il voudrait, lui, le chapeau de Saint-Just aux armées, les épaulettes d’or et la grande ceinture tricolore.\par
Moi, je me vois sergent, je dis : \emph{Allons-y ! Eh ! mes enfants} !\par
On est tous du même pays, autour du même feu du bivouac, et l’on parle de la Haute-Loire.\par
Je rêve l’épaulette de laine, le baudrier en ficelle.\par
Je voudrais être du bataillon de la Moselle. Avec des paysans et des ouvriers. L’oncle Joseph serait capitaine et l’oncle Chadenas, lieutenant.\par
Nous retournerions faire de la menuiserie, ou moissonner les champs « après la victoire ».\par
\bigbreak

\labelblock{Rue Coq-Héron.}

\noindent Le journaliste nous mène un soir à l’imprimerie, dans le rez-de-chaussée où le journal se tire ; il est l’ami d’un des ouvriers.\par
La machine roule, avale les feuilles et les vomit, les courroies ronflent. Il y a une odeur de résine et d’encre fraîche.\par
C’est aussi bon que l’odeur du fumier. Ça sent aussi chaud que dans une étable. Les travailleurs sont en manches de chemise, en bonnet de papier. Il y a des commandements comme sur un navire en détresse. Le margeur, comme un mousse, regarde le conducteur, qui surveille comme un capitaine.\par
Un rouleau de la machine s’est cassé. – Ohé ! – oh !\par
On arrête, – et, cinq minutes après, la bête de bois et de fer se remet à souffler.\par
\bigbreak
\noindent J’ai trouvé l’état qui me convient…\par
J’aurai, moi aussi, le bourgeron bleu et le bonnet de papier gris, j’appuierai sur cette roue, je brusquerai ces rouleaux, je respirerai ce parfum, – c’est grisant, vrai ! comme du gros vin.\par
Compositeur ? Non. – Imprimeur, à la bonne heure !\par
Le beau métier, où l’on entend vivre et gémir une machine, où tout le monde à un moment est ému comme dans une bataille.\par
Il faut être fort, – de grands gestes. Il y a du fer, du bruit, j’aime ça. On gagne sa vie, et l’on lit le premier le journal.\par
\bigbreak
\noindent Je n’en parle pas ; je garde pour moi mon projet. Je sens que c’est une force d’être muet, quand ce que l’on veut est ce que les autres ne veulent pas. Je ne dirai rien, mais quelle joie !\par
Il y a un peu de vanité cruelle dans cette joie-là.\par
Je pense que je vais être si supérieur aux camarades qui mènent la vie de bohème ! – il n’y a pas à dire – parce qu’ils n’ont pas d’ouvrage sûr ; tandis que moi, je me ferai mes cinq francs par jour vaille que vaille, en ne fatiguant que mes bras.\par
Je ne dépendrai de personne, et la nuit je lirai, le dimanche j’écrirai. – Je serai d’une société secrète, si je veux. – J’aurai mangé quand j’irai, et je pourrai encore donner quelque chose pour les prisonniers politiques ou pour acheter des armes…\par

\labelblock{Vivre en travaillant, mourir en combattant !}

\noindent « Jacques, j’ai reçu une lettre de ton père, qui décide que nous retournerons à Nantes pour que tu prépares ton baccalauréat avec lui. »\par
Je n’y pensais plus. J’étais dans la révolution jusqu’au cou, et j’aimais Paris maintenant. Cette imprimerie !… Puis nous avions été manger des\emph{ ordinaires} dans des crèmeries, où il venait des ouvriers qui avaient appartenu aux \emph{Saisons} et qui avaient été mêlés à des émeutes.\par
La blouse et la redingote s’asseyaient à la même table et l’on trinquait.\par
Le dimanche, nous allions dans une goguette, \emph{la Lyre chansonnière} ou \emph{les Enfants du Luth} : je ne me rappelle plus bien.\par
Je m’ennuyais un peu quand on chantait des gaudrioles ; mais on disait tout à coup : « C’est Festeau, c’est Gille. » Et il me semblait entendre dans le lointain la batterie sourde d’un tambour républicain ; puis la batterie était plus claire, Gille entonnait, et cette musique tirait à pleines volées sur mon cœur.\par
Je ne sais pas cependant si je ne préfère pas aux chansons qui parlent de ceux qui vont se battre et mourir, les chansons de batteur de blé ou de forgeron, qu’un grand mécanicien, qui a l’air doux comme un agneau, mais fort comme un bœuf, chante à pleine voix. Il parle de la poésie de l’atelier, – le grondement et le brasier, – il parle de la ménagère qui dit : « Courage, mon homme, – travaille, – c’est pour le moutard. »\par
À ce moment, le chanteur baisse la voix. « Fermez la fenêtre », dit quelqu’un. Et l’on salue au refrain :\par

\labelblock{Le drapeau que le peuple avait à Saint-Merry !}

\noindent Il y a de la révolte au coin des vers. – Moi, j’en mets du moins, moi qui, hier, ai ouvert l’\emph{Histoire de dix ans}, qui n’en suis plus à 93. J’en suis à Lyon et au drapeau noir. Les tisseurs se fâchent, et ils crient : \emph{Du pain ou du plomb} !\par
« Jacques, c’est lundi que nous partirons pour Nantes. »\par
Un coup de couteau ne me ferait pas plus de mal.\par
Il y a un mois, je serais parti content, et j’aurais peut-être craché sur Paris en passant la barrière, tant j’avais été étouffé là-dedans, tant j’avais eu de désillusions en voyant mes camarades et mes maîtres.\par
Mais depuis un mois, il y a eu les larmes de ma mère et, au lendemain de cette scène, la liberté pleine ; de temps en temps quarante sous, pour souper d’un peu de cochon avec des amis, et, le dimanche, dîner d’un bœuf braisé à Ramponneau.\par
J’ai été mêlé à la foule, j’ai entendu rire en mauvais français, mais de bon cœur. J’ai entendu parler du peuple et des citoyens : on disait \emph{Liberté} et non pas \emph{Libertas.}\par
\bigbreak
\noindent Il a toujours été question de pauvreté autour de moi ; mon père a été humilié parce qu’il était pauvre, je l’ai été aussi, et voilà qu’au lieu des discours de Caton, de Cicéron, des gens en \emph{o, onis, us, i, orum}, je vois qu’on se réunit sur la place publique pour discuter la misère, et demander du travail ou la mort.\par
« Hé ! Jean-Marie, puisqu’il n’y a pas de miche à la maison, vaut-il pas mieux \emph{passer le goût du pain ? »}\par
\bigbreak
\noindent Retourner là-bas ?\par
À qui parlerai-je de République et de révolte ?\par
Est-ce qu’on s’est jamais soulevé à Nantes ? Ce serait autre chose à Lyon !\par
Oh ! si je n’avais promis à ma mère ! – si elle n’avait pas pleuré !\par
Si elle n’avait pas pleuré, j’aurais dit : « Je ne veux pas partir. » Le puritain m’aurait placé comme garçon de bureau comme homme de peine, dans un des journaux. Il y a justement (c’était une chance !), il y a une place au \emph{National} ; on donne trente francs par mois pour \emph{tenir la copie}, pour lire à l’homme qui corrige. J’aurais vécu avec ces trente francs-là. Ma besogne faite, je descendais dans l’imprimerie sentir l’encre et le papier, et je demandais aux ouvriers de m’apprendre l’état.\par
Si j’en parlais à ma mère ?\par
\bigbreak
\noindent Je lui en parle.\par
« Tu m’avais dit, cependant…\par
— C’est vrai, oui. »\par
Je vais dire adieu au journaliste et à Matoussaint. Le journaliste me donne du courage.\par
« Vous reviendrez, mon cher.\par
— Écrivez-moi, au moins !\par
— Oui. Même, dit-il en souriant, si c’est pour vous appeler à l’assaut de l’Élysée.\par
— Surtout dans ce cas, citoyen ! »
\section[{24. Le retour}]{24. \\
Le retour}\renewcommand{\leftmark}{24. \\
Le retour}

\noindent Ah ! que la route est triste !\par
Ma mère voit bien ma douleur et essaye de me consoler, ce qui m’irrite, et je suis forcé de me retenir pour ne pas là brusquer. Je m’en veux de paraître accablé : je n’ai donc pas de courage !\par
Non, je n’en ai pas ; les noms de stations criés à la gare m’entrent dans la poitrine comme des coups de corne.\par
Beaugency ! Amboise ! Ancenis !\par
On signale un château, une ruine ; mais c’est tout près de Nantes, cela !\par
« Jeune homme, nous n’en sommes pas à plus de cinq lieues.\par
— Oh ! mon Dieu !\par
— Nous y sommes. »\par
\bigbreak
\noindent Comme les rues paraissent désertes ! Sur le quai où nous demeurons, il y a deux ou trois personnes qui passent, – pas plus. Je reconnais un ancien capitaine sur le banc où je le voyais jadis en allant en classe, puis un nègre en guenilles qui avait des enfants à qui l’on faisait la charité.\par
Quel silence ! on dirait qu’on est dans une campagne.\par
Je lève les yeux vers la fenêtre de notre appartement.\par
Mon père est là, maigre, l’air chagrin, immobile.\par
Il me repoussait quand j’étais petit et qu’on me jetait dans ses bras pour un baiser.\par
Aussi, chaque fois qu’il y a la solennité d’un départ ou d’une \emph{retrouvée}, est-ce un embarras pour nous deux !\par
Il m’offre à embrasser, cette fois, une face pâle, un front de pierre.\par
Je n’ose pas.\par
Ma mère nous pousse un peu, j’avance le cou, il tend le sien. Mes cheveux l’aveuglent et sa barbe me pique ; nous nous grattons d’un air de rancune tous les deux.\par
On monte les escaliers sans dire un mot.\par
Mon père arrive par derrière ; on dirait une \emph{exécution} à la Tour de Londres.\par
Si l’on exécutait tout de suite, – mais non – mon père \emph{prend des temps} de solennité.\par
C’est le latin. – C’est le souvenir des pères qui assassinent leurs fils dans l’histoire : Caton, Brutus. Il ne pense pas à m’assassiner, mais au fond, je suis sûr qu’il se trouve lâche, et il voudrait que son fils, que\emph{ Bruticule} lui en sût gré ; et chaque fois que je fais un geste, ou que je dis un mot un peu vif, il fronce les sourcils, serre les lèvres (ça doit le fatiguer beaucoup, ce digne homme !) et il semble me dire : « Tu oublies donc que tu ne vis que par charité, et que je pourrais te donner un coup de hache, te livrer au licteur ? »\par
Il reste antique jusqu’à ce que le nez lui chatouille ; ou qu’il ne puisse plus y tenir.\par
Il s’épuise à la fin, à force de vouloir paraître amer, et il est forcé de se desserrer la mâchoire de temps en temps.\par
\bigbreak
\noindent Jamais il n’a été si Brutus qu’aujourd’hui.\par
Il a rejeté le gland de son bonnet grec, comme s’il y avait de la faiblesse dedans, et il se tient dans le fauteuil comme si c’était une chaise curule.\par
« Vous êtes mon fils, je suis votre père. »\par
— Oh ! oui, tu peux en être sûr, Antoine ! a l’air de dire ma mère.\par
— Il y avait à Rome une loi (m’écoutez-vous, mon fils ?) qui donnait au père déshonoré, dans la personne d’un des siens, le droit de faire mourir ce… ce… ce \emph{sien}… \emph{suum}. »\par
Il s’embrouille.\par

\labelblock{PHILOSOPHIE}

\noindent « Tu feras ta philosophie jusqu’à Pâques, et à Pâques tu te présenteras au baccalauréat. »\par
Telle est la décision adoptée.\par
On me regarde un peu quand je reparais dans la cour des classes. On m’entoure, et l’on me dévisage. Un garçon qui revient de Paris… jugez !…\par
\bigbreak
\noindent Le professeur est un jeune homme qui, sorti le premier de l’École normale, a été reçu à l’agrégation le premier ; qui arrive toujours le premier au cours, et qui se présente toujours le premier à l’économat pour toucher ses appointements. Il loge au premier, dans une maison au fond d’une rue lugubre. Au théâtre, il va aux premières, et au premier rang.\par
C’est sa mère qui a fait cette combinaison.\par
« Je veux que tu sois partout, partout, le \emph{premier.} »\par
\bigbreak
\noindent Ce professeur me traite assez bien. Il compte sur moi pour faire le péripatéticien chez lui, dans son jardin.\par
Il avait du monde autrefois, à qui il faisait tirer de l’eau pour arroser son potager ; il n’a plus personne.\par
Il pense que moi, fils de collègue – qui suis d’Éleusis aussi, – j’ai l’étoffe d’un disciple et d’un tireur d’eau.\par
Je ne sais comment il a été nommé à ce poste-là.\par
Je trouvais mes professeurs de rhétorique ennuyeux à Paris, mais l’on m’assurait qu’il y avait parmi les professeurs de philosophie des gens qui raisonnaient, qui pensaient, qui avaient la tête pleine.\par
Une fois même, il y en avait un qui était venu serrer la main du \emph{journaliste}, quoique ce journaliste fût républicain.\par
J’avais grande idée de ces chercheurs de vertu.\par
Mais celui-ci est vraiment comique !\par

\labelblock{EN CLASSE}

\noindent « M. Vingtras, quelles sont les preuves de l’existence de Dieu ? »\par
Je me gratte l’oreille.\par
« Vous ne savez pas ? »\par
Il paraît étonné, il a l’air de dire : « Vous qui arrivez de Paris, voyons !\par
— Gineston, les preuves de l’existence de Dieu ?\par
— M’sieu, je ne sais pas, il manque des pages dans mon livre.\par
— Badigeot ?\par
— M’sieu, il y a le \emph{consensus omnium} !\par
— Ce qui veut dire ?… (Le professeur prend les poses de Socrate accouchant son génie.)\par
— Ce qui veut dire… – Pitou, souffle-moi donc !\par
— Ce qui veut dire (reprend le professeur aidant le malade) que tout le monde est d’accord pour reconnaître un Dieu ?\par
— Oui, m’sieu.\par
— Ne sentez-vous pas qu’il y a un être au-dessus de nous ? »\par
Badigeot regarde attentivement le plafond ! Rafoin y a lancé le matin un petit bonhomme en papier qui pend à un fil au bout d’une boulette de pain mâché.\par
« Oui, m’sieu, il y a un bonhomme là-haut.\par
— Bonhomme, bonhomme (dit le professeur qui est myope et n’a pas vu ce qui pend au plafond), mais c’est aussi le Dieu de la Bible. Sa droite est terrible ! »\par
Le mot ne lui a pas déplu, cependant.\par
« J’aime cette familiarité, tout de même », disait-il en sortant de la classe. « Il y a un\emph{ bonhomme} là-haut !… Ce cri d’un enfant pour désigner Dieu ! »\par
Il en a parlé en haut lieu.\par
« Qu’en dites-vous, monsieur le proviseur ? N’est-ce pas l’enfant qui ne sait rien, parlant comme le vieillard qui sait tout ? – Oui, il y a un \emph{bonhomme} là-haut ! »\par
\bigbreak
\noindent À la classe suivante il s’adresse de nouveau à Badigeot et commence en lui rappelant le mot :\par
« Il y a un bonhomme là-haut ? »\par
— Non, m’sieu, il n’y est plus. Il tenait mal et il est tombé. »\par

\labelblock{MON ÂME}

\noindent Le professeur m’a mis aux\emph{ facultés de l’âme.}\par
Les autres n’y sont pas encore, il fait cela pour moi.\par
Ce n’est qu’après Pâques qu’on sait comment l’âme est faite dans ce collège-ci.\par
Il y a sept facultés de l’âme.\par
« Comptez sur vos doigts, c’est plus facile », me dit le maître.\par
\bigbreak
\noindent On annonce à Nantes l’arrivée d’un professeur de Faculté célèbre, M. Chalmat. Chalmat lui-même est dans nos murs !\par
Il a connu mon père à Paris, au moment de l’agrégation.\par
Ils dînaient à côté l’un de l’autre, dans un restaurant à prix fixe. M. Chalmat sortit le premier, oubliant un manuscrit, que mon père prit. Il y avait l’adresse, et il put rapporter le paquet à son propriétaire désespéré.\par
« Quand vous aurez besoin de moi, dit le philosophe, je suis là. »\par
Il était là, en chair et en os, par hasard, et par hasard aussi il y avait un appartement meublé dans notre maison, ce qui fit de lui notre voisin.\par
M. Chalmat dormait sur le même carré que nous.\par
Il dormait peu, et la nuit il parlait tout haut. Je l’entendais qui disait : « Il y en a HUIT, HUIT ! Oui, il y en a HUIT. »\par
\bigbreak
\noindent Il voulut me faire un cadeau.\par
Il nous prit à part, mon père et moi ; il nous parla à cœur ouvert.\par
« Mes amis, dit-il (il m’honorait moi-même de ce nom), je désire vous payer du service que vous m’avez rendu jadis, en sauvant mon manuscrit. Je n’ai pas de fortune, mais je vous donnerai ce que j’ai, le résultat de vingt ans de réflexions et de travail ! »\par
Mon père semble dire : « c’est trop ».\par
« Non, non ! Écoutez-moi bien. »\par
Nous retenons notre souffle, on aurait entendu voler une mouche.\par
« On vous dit qu’il y a sept facultés de l’âme ? \emph{Il y en a huit ! »}\par
On me trompait donc ? on me volait d’une ? Pourquoi ? Que signifie ?\par
« Oui, oui, c’est comme ça », et M. Chalmat me montrait ses cinq doigts de la main droite et trois autres couchés dans la main gauche.\par
Il a ajouté avec bonté :\par
« Servez-vous de la découverte, je vous y autorise ; on l’ignore encore, dans deux mois seulement ce sera dans mes livres. »\par

\labelblock{Rennes, lundi.}

\noindent Je suis arrivé ce matin. Demain, la version. Mon père voulait me suivre à Rennes, mais il est forcé de rester avec ses pensionnaires.\par

\labelblock{Mardi.}

\noindent Je suis le second en version.\par
J’ai \emph{fait} encore trop près du texte, sans cela j’aurais été le premier.\par
\bigbreak
\noindent Cette après-midi, l’examen.\par
Je repasse, je repasse, comme si je pouvais avaler le Manuel en trois bouchées.\par
« Monsieur Vingtras ! »\par
C’est mon tour.\par
On tire les boules.\par
« Traduisez-moi ceci, traduisez-moi cela. »\par
Je traduis comme un ange.\par
« On voit, dit publiquement le doyen, non seulement que vous avez été bercé sur les genoux d’une tête universitaire, mais encore que vous vous êtes abreuvé aux grandes sources, que vous avez passé par cette belle école de Paris, à laquelle nous avons tous appartenu. (Se ravisant.) Ah ! non, pas tous ; il y a notre collègue M. Gendrel. »\par
M. Gendrel est le professeur de philosophie. Il est licencié de \emph{province}, docteur ès lettres de \emph{province} ; il n’a pas bu aux fortes sources comme eux, comme moi, et, comme c’est un \emph{cafard}, à ce qu’on dit, le doyen le pique chaque fois qu’il le peut. Il m’a pris pour prétexte à l’instant.\par
M. Gendrel est jaune, jaune comme un coing, avec des lunettes comme celles de Bergougnard.\par
\bigbreak
\noindent Je passe par le professeur de mathématiques avant d’arriver à lui.\par
Je ne sais pas grand-chose de ce qu’on me demande, mais 1’éloge qu’on vient de m’adresser publiquement engage le professeur à être indulgent.\par
« Qu’est-ce que le pendule compensateur ?\par
— C’est un pendule qui compense.\par
— Bien, très bien ! »\par
Se penchant à l’oreille du doyen :\par
« Il est intelligent. »\par
Se retournant vers moi :\par
« Et la machine pneumatique, quel est son usage ?\par
— La machine pneumatique ?…\par
— Oh ! je ne vous demande pas grands détails. C’est pour faire le vide, n’est-ce pas ? Et si on met des oiseaux dedans, ils meurent. Bien, très bien ! »\par
Il reprend :\par
« Vous avez en géométrie la section d’un cône ? »\par
Oui, mais il me faut un chapeau pour faire une bonne démonstration, comme avec les plâtres du vieil Italien, et je la fais à la bonne franquette.\par
Prenant un chapeau qui me tombe sous la main, et d’où je retire un vieux mouchoir, je coupe mon cône.\par
On rit dans la salle parce que la coiffe est très grasse et le mouchoir très sale ; les examinateurs me regardent avec un sourire de bonne humeur.\par
Le professeur de mathématiques, qui décidément veut faire sa cour au doyen (il doit épouser sa fille), me parle à son tour :\par
« Monsieur, on voit que vous préférez Virgile à Pythagore ; mais comme le disait si bien monsieur le doyen tout à l’heure, vous avez bu aux grandes sources séquanaises, et Pythagore même en a profité. »\par
Murmure flatteur.\par
Encore un coup à Gendrel !\par
C’est à lui que j’ai affaire maintenant.\par
Il me fixe : ses lunettes flamboient comme des pièces de cent sous toutes neuves.\par
Il lui prend l’envie de se moucher.\par
Il cherche son mouchoir, c’est lui que j’ai retiré tout à l’heure et remis dans la coiffe si grasse.\par
C’était le chapeau de Gendrel.\par
Je suis perdu !\par
Il m’en veut pour les allusions que le doyen a lancées contre lui sous mon couvert ; il m’en veut pour la coiffe et le mouchoir.\par
Il ne me laisse pas le temps de me reconnaître.\par
« Monsieur, vous avez à nous parler des facultés de l’âme. »\par
(D’une voix ferme) : « Combien y en a-t-il ? »\par
Il a l’air d’un juge d’instruction qui veut faire avouer à un assassin, ou d’un cavalier qui enfonce un carré avec le poitrail de son cheval.\par
« Je vous ai demandé, monsieur, combien il y a de facultés de l’âme ? »\par
Moi, abasourdi : « Il y en a HUIT. »\par

\astertri

\noindent Stupeur dans l’auditoire, agitation au banc des examinateurs !\par
Il y a un revirement général, comme il s’en produit quelquefois dans les foules, et l’on entend : \emph{huit, huit, huit.}\par
Pi – houit !…\par
\bigbreak
\noindent J’attends l’opinion de Gendrel. Il me regarde bien en face.\par
« Vous dites qu’il y a huit facultés de l’âme ? Vous ne faites pas honneur à la \emph{source des hautes études} à laquelle monsieur le doyen vous félicitait si généreusement de vous être abreuvé, tout à l’heure. Dans le collège de Paris où vous étiez, il y en avait peut-être huit, monsieur. Nous n’en avons que \emph{sept en province}. »\par
Les examinateurs, qui lui en veulent, ne peuvent cependant accepter ma \emph{théorie des huit} publiquement, et je vais porter la peine d’avoir lancé à un examen une franchise qui avait besoin de volumes et d’hommes célèbres pour la faire accepter.\par
Le doyen rentre et dit sèchement : « Monsieur Vingtras est appelé à se présenter à une autre session. »\par
La foule se retire en se demandant qui je suis, ce que je veux, et où l’on en arriverait si l’on jouait ainsi avec l’âme ; je renverse les bases sur lesquelles repose la conscience humaine.\par
Je n’y tiens pas du tout, moi ! C’est la faute à M. Chalmat, qui m’a dit qu’il y en a huit. Je ne suis pas un instrument aux mains d’une secte ou d’une faction.\par
J’ai dit ce qu’il m’a dit !\par
Il n’y a donc que sept facultés de l’âme : j’en perds une, – je m’en fiche, – mais je serai forcé de me représenter devant la Faculté de Rennes, – et je ne m’en fiche pas. Je suis bien triste…\par
\bigbreak
\noindent Mon père me reçoit, les lèvres serrées, le front plissé, l’œil cave.\par
C’est qu’il n’est pas seulement blessé dans ma personne ! Il l’est dans son propre orgueil !\par
Un élève qui lui en veut a retourné le poignard dans la plaie.\par
Le soir du même jour où l’on apprit que j’étais refusé, on lisait sur notre porte :\par


\begin{verse}
À LA BOULE NOIRE\\
AUBERGE DES RETOQUÉS\\
AGRÉGATION ET BACCALAURÉAT\\
(On porte tout de même des participes en ville)\\
\end{verse}

\noindent \emph{On porte tout de même des participes en ville} ! c’est-à-dire qu’on donne des répétitions tout de même et qu’on demande vingt-cinq francs par mois, tout comme si on avait été reçu d’emblée, comme si on avait passé des agrégations du premier coup, et comme si le fils de la maison avait jonglé avec des \emph{blanches !…}\par
\bigbreak
\noindent « Jacques, il vaut mieux que tu ne te mettes pas à table avec nous. »\par
Ma pauvre mère ne vit plus. Elle assiste chaque jour à des scènes pénibles.\par
Mon père me reproche le pain que je mange.\par
On m’apporte des provisions dans ma chambre, comme à un homme qui se cache.\par
« Oh ! je ne veux plus de cette vie ! Je veux repartir pour Paris.\par
— Dans ces habits ? » dit ma mère en regardant mes hardes.\par
Je serai donc toujours écrasé par mon costume !\par
Ah ! je partirai tout de même !\par
Mon père a eu vent de ce propos.\par
« S’il part, dis-lui que je le ferai arrêter par les gendarmes. »\par
Legnagna m’avait déjà menacé d’eux…\par
Vous voulez faire de moi un gibier de prison, mon père ?\par
Il a donc le droit de me faire prendre, il a le droit de me traiter comme un voleur, il est maître de moi comme d’un chien…\par
« Jusqu’à ta majorité, mon garçon ! »\par
Il a dit cela avec emportement, en tapant sur un livre qui s’appelle le Code ; je le retrouve le soir dans un coin, ce vieux livre. Je le lis en cachette, à la lueur du réverbère qui éclaire ma chambre.\par
« \emph{Peut être enfermé, sur l’ordre de ses parents}, etc. »\par
\bigbreak
\noindent Me faire arrêter ? – Pourquoi ?\par
Parce que je ne veux pas qu’il dise que je ne gagne pas la pâtée que je mange, – parce que je ne veux pas qu’il s’amuse à me frapper, moi qui pourrais le casser en deux, – parce que je veux avoir un état, et que ça l’humilie de penser que lui, qui a tant lutté pour avoir une \emph{toge} roussie, il aura un fils qui aura une cotte, un bourgeron !\par
Il me fera mettre les menottes peut-être et ordonnera aux gendarmes de serrer dur si je résiste. Et cela, parce que je ne veux pas être professeur comme lui.\par
Je comprends. C’est que j’insulte toute sa vie en déclarant que je veux retourner au métier comme nos grands-parents ! Dire que je désire entrer en atelier, c’est dire qu’il a eu tort de lâcher la charrue et l’écurie.\par
Il me ferait donc conduire de brigade en brigade ; si ce n’est pas ce soir, ce sera demain, ou dans un mois. Jusqu’à vingt et un ans, il le peut.\par
On a pensé à moi pour une leçon.\par
Mes succès de collège m’ont fait une réputation ; et puis quelques personnes, devinant peut-être le drame muet qui se joue chez nous, veulent me montrer de l’amitié.\par
L’une de ces personnes s’adresse à ma mère ; c’est une dame qui veut que j’apprenne un peu de latin à son fils. Ma mère a répondu :\par
« Madame, je serais bien contente s’il pouvait gagner un peu d’argent, parce qu’il se disputerait moins avec son père. Ils sont bons tous deux, dit-elle, mais ils se chamaillent toujours. – Il faudrait, par exemple, que vous parliez à M. Vingtras pour qu’il achète une culotte à Jacques, si vous ne voulez pas (esquissant un sourire) qu’il aille chez vous tout nu – sauf votre respect. Je vous dis ça comme une paysanne ; c’est que je suis partie de bas. – J’ai gardé les vaches, voyez-vous ! »\par
J’entends cela de la chambre où je suis. Pauvre mère !\par
\bigbreak
\noindent La personne qui venait chercher la leçon s’en va, ayant peur de recevoir une carafe à la tête, quelque bouteille égarée de son chemin, – si mon père rentrait et que nous nous prissions aux cheveux. Puis elle ne se sent pas le courage de parlementer pour ma culotte. En un mot, on a gardé des animaux dans notre famille, et elle vient chercher un professeur et non pas un berger.\par
Ma mère attend une réponse. (On doit lui écrire.)\par
« Je lui ai pourtant dit ce qu’il fallait dire, fait-elle en croisant les bras ; oh ! ces riches, ces riches !… »\par
Ah ! cette paysanne !\par
\bigbreak
\noindent Ma réputation de fort en thème me fait retrouver pourtant une leçon ; mais mon père, afin de m’humilier, ne me laisse pas même prendre dans sa garde-robe une culotte neuve. Mes habits ne tiennent pas.\par
Je suis forcé de m’asseoir de côté.\par
Je tremblai si fort un jour où l’on me dit :\par
« Donnez donc votre leçon dans le jardin, M. Vingtras, et ôtez votre paletot. Il fait si chaud ! Vous suez à grosses gouttes.\par
— Oh ! non, au contraire, merci. »\par
Je ruisselle.\par
« Il a l’air timide, un peu inquiet, votre fils, dit-on à ma mère, qu’on n’attendait pas, mais qui est venue un jour pour demander si l’on était content de moi et pour parler en ma faveur.\par
— Ne vous y fiez pas ! et si vous avez des demoiselles qui ont de beaux yeux, ne les laissez pas trop courir quand il est là. Il y a déjà eu des histoires ! Il est parisien pour ça, allez ! et avant même d’aller à Paris, il avait (elle fait des cornes sur son front avec les doigts), oui, oui, comme je vous dis !… »\par
On me chasse le lendemain.\par
Mais j’étais engagé pour un mois, et l’on me paye le mois entier. « Cinquante francs. »\par
\bigbreak
\noindent Avec cet argent-là, je vais me commander des habits. Ma mère intervient. « Je te les ferai moi-même, nous achèterons du drap.\par
— Oh ! non, par exemple, non !\par
— Mon fils ne m’aime plus, conte-t-elle, le soir, à une voisine qui a sa confiance. – S’il me laissait choisir le drap encore ! »\par
J’achète un costume tout fait.\par
Ma mère me suit en cachette et pendant que je traite elle demande à parler en particulier au patron de l’établissement et lui explique mon histoire. « Donnez-lui du solide, murmure-t-elle, les larmes aux yeux ! »\par
\bigbreak
\noindent Je vois un peu plus de monde, maintenant que je suis propre. Ma mère me prie de l’accompagner chez des gens qu’elle connaît.\par
Elle en est si contente et si fière !\par
Mais au milieu d’une conversation elle dit tout à coup :\par
« Comme ça fronce ! Et comme on voit qu’il n’y a qu’une demi-doublure ! Si tu te tenais comme ça au moins, ça cacherait ! » (et elle me tire mon gilet pour le faire aller, elle tripote ma cravate).\par
Claquant la langue tristement, elle ajoute :\par
« Tu peux te vanter d’avoir choisi du salissant ! Et il n’a seulement pas demandé des morceaux ! »\par
\bigbreak
\noindent Mon père sent que je suis ulcéré, et un jour où il me voyait pâlir, il eut peur de mon désespoir.\par
« Ton fils a voulu s’empoisonner », dit-il à ma mère.\par
Il en est à croire cela.\par
La pauvre femme reste muette, glacée.\par
Il est d’ailleurs las, lui-même, de la vie que nous menons sous le même toit. La maison a l’air d’une maison maudite.\par
\bigbreak
\noindent « Dis-lui de m’écrire ce qu’il compte faire. »\par
C’est le dernier mot qu’il adresse à ma mère, après cette soûleur du suicide.\par
\bigbreak
\noindent C’est affreux de prendre cette grande feuille de papier vide pour écrire à, son père. Il faut mettre \emph{vous}.\par
Je dis \emph{vous} pour la première fois.\par
Je ne vois pas bien avec la chandelle.\par
« Mère, donne-moi donc une bougie.\par
— Ça n’éclaire pas mieux, va, c’est un peu plus propre, mais ça éclaire moins bien, et c’est beaucoup plus cher, vois-tu ! »\par
J’écris à mon père ! Je rature, et je rature !\par
Tout en écrivant, il m’est venu de la sensibilité, j’ai peur de paraître faible.\par
Je recommence ; c’est difficile et douloureux.\par
Ah ! ma foi, non ! et je déchire encore…\par
Je vais mettre deux lignes seulement, – pas deux lignes, – quatre mots. Ça m’évitera ce « \emph{vous »}, et ce que je veux dire y sera tout de même. J’écris simplement ceci :\par

\begin{quoteblock}
\noindent Je veux être ouvrier.\end{quoteblock}

\noindent « Ton père est furieux », me glisse à l’oreille ma mère, qui vient de remettre le bout de papier.\par
\bigbreak
\noindent Il me rencontre dans un corridor :\par
« Tu te f… de moi, dis… ? »\par
Il lève la main, et j’ai cru qu’il allait m’écraser.\par
L’abîme est creusé, – il va arriver un malheur.
\section[{25. La délivrance}]{25. \\
La délivrance}\renewcommand{\leftmark}{25. \\
La délivrance}

\noindent Le malheur est arrivé !\par
Je sors quelquefois, le soir – bien rarement. Que dirais-je aux gens que je rencontrerais ? Je n’ai pas le sou pour aller au café où les collégiens vont. Je ne veux pas me laisser offrir et ne pas payer : je suis trop pauvre pour cela. C’est quand j’ai de l’argent dans ma poche que j’accepte, parce que je sens que l’on ne me fait pas l’aumône et qu’à mon tour je puis régaler.\par
Mais il y a longtemps que je n’ai plus rien – même un sou.\par
J’avais fait un peu d’argent avec mes livres de prix. \emph{La Poésie au seizième siècle}, par Sainte-Beuve, un Bossuet, et les œuvres de M. Victor Cousin.\par
Ma mère trouvant cinq francs dans ma poche m’avait demandé où je les avais pris. Elle avait l’air de croire que c’était le produit d’un vol ou d’un assassinat. « Il se sera laisser entraîner par les mauvais conseils. Ce sont les mauvais conseils qui perdent les jeunes gens. »\par
Qui me donnerait des conseils ? – Des copains ? Je suis plus vieux qu’eux, même s’ils ont mon âge. On ne les a pas battus tant que moi. Ils n’ont pas connu Legnagna et la maison muette. – Des vieux ? les collègues de mon père ? Ils ont bien assez à faire de nouer les deux bouts, et puis ils ne savent que ce qui se passait chez les anciens, et n’ont pas le temps, – à cause des répétitions, – de juger ce qui se passe autour d’eux.\par
J’avais dit à ma mère d’où venaient ces cinq francs.\par
Elle avait levé les mains au ciel.\par
« Tu as vendu tes livres de prix, Jacques !… »\par
Pourquoi pas ? Si quelque chose est à moi, c’est bien ces bouquins, il me semble ! Je les aurais gardés, si j’avais trouvé dedans ce que coûte le pain et comment on le gagne. Je n’y ai trouvé que des choses de l’autre monde ! – tandis qu’avec l’argent, j’ai pu acheter une cravate qui n’était pas ridicule et aller aussi prendre un gloria aux Mille-Colonnes. J’y lis la \emph{feuille} de Paris, qui sent encore l’imprimerie, quand le facteur l’apporte.\par
Mais je me suis trouvé un soir face à face avec mon père qui passait. Il m’a insulté, d’un mot, d’un geste.\par
« Te voilà, fainéant ? »\par
Et il a continué son chemin.\par
Fainéant ? – Ah ! j’avais envie de courir après lui et de lui demander pourquoi il m’avait jeté entre les dents, et sans me regarder en face, ce mot qui me faisait mal !\par
Fainéant ! – Parce que, dans le silence glacial de la maison, ce travail de bachot et cet acharnement sur les morts m’ennuient, parce que je trouve les batailles des Romains moins dures que les miennes, et que je me sens plus triste que Coriolan ! Oh ! il ne faut pas qu’il m’appelle fainéant !\par
Fainéant !\par
Si mon père était un autre homme, j’irais à lui, et je lui dirais :\par
« Je te jure que je vais travailler, bien travailler, mais n’aie plus vis-à-vis de moi cette attitude cruelle ! »\par
Il me renverrait comme un menteur. J’ai bien vu cela, quand j’étais plus jeune.\par
Deux ou trois fois, quand il allait m’humilier ou me battre, je lui promis, s’il ne le faisait point, de tenir n’importe quelle parole il voudrait. Il avait fait fi de mes engagements, et je lui en avais voulu, tout enfant que je fusse, de si peu croire au courage de son fils.\par
Aujourd’hui encore il me rirait au nez et il croirait que je caponne !\par
Allons ! je vivrai à côté de lui comme à côté d’un garde-chiourme, et je travaillerai tout de même ! C’est dit.\par
Mais, le lendemain soir, ma mère venait m’annoncer, tout effrayée, que mon père ne voulait plus que je restasse dehors et que je courusse les cafés comme un vagabond. Il fallait être rentré à huit heures, ou sinon je coucherais dans la rue.\par
\bigbreak
\noindent J’y ai couché.\par
C’est long, une nuit à assassiner, et vers deux heures du matin il a plu. J’étais trempé jusqu’aux os, j’avais les pieds glacés, et je me cachais sous les auvents des portes. J’avais peur aussi des sergents de ville ! J’ai tourné, tourné, autour de la maison. À dix heures, elle avait été fermée, suivant la menace. J’avais trouvé le verrou mis.\par
Demain encore, je le trouverai tiré si mon père a autant de courage que moi.\par
Je ne tiens pas à rôder dans les rues. J’aimerais mieux être dans ma chambre, mais on a l’air de me \emph{menacer}. Je ne veux pas paraître avoir peur, et je grelotte, et mes dents claquent.\par
Comme c’est froid, quand le soleil se lève !\par
\bigbreak
\noindent Je ne suis rentré que quand mon père devait être au collège, à huit heures et demie du matin.\par
Il n’était pas sorti. C’est la première fois, depuis la scène sanglante avec ma mère, qu’il a manqué la classe.\par
M’avait-il vu et m’attendait-il ? Était-il malade de fureur ?\par
La porte était à peine poussée qu’il s’est jeté sur moi. Il était blanc comme un mort.\par
« Gredin, dit-il, je vais te casser les bras et les jambes ! »\par

\labelblock{Dans la maison, une heure après.}

\noindent « Qu’y a-t-il ?\par
— Il y a le fils Vingtras, qui a voulu assassiner son père ! »\par
Je n’ai pas essayé d’assassiner mon père. C’est lui qui m’aurait volontiers estropié ; il répétait :\par
« Je te casserai les bras et les jambes. »\par
« Eh bien, non ! Vous ne casserez les bras et les jambes à personne. Oh ! je ne vous frapperai pas ! Mais vous ne me toucherez point. C’est trop tard ; je suis trop grand. »\par

\begin{quoteblock}
\noindent BAS LES MAINS ! OU GARE À VOUS !\end{quoteblock}


\labelblock{Minuit.}

\noindent Mon père me fera arrêter, bien sûr.\par
La prison demain, comme un criminel.\par
Ma vie sera une vie de bataille. C’est le sort de celles qui commencent comme cela. Je le sens bien.\par
Je ne resterais en prison qu’une semaine, pas plus, que je serais tout de même montré au doigt pour longtemps dans cette province.\par
L’idée m’est presque venue d’en finir.\par
Si je me tuais cette nuit, pourtant, ce serait mon père qui m’aurait assassiné !\par
Et qu’ai-je fait de mal ? des fautes de quantité et de grammaire, voilà tout. Puis j’ai, sur un faux renseignement, dit qu’il y avait huit facultés de l’âme quand il n’y en a que sept. – Voilà pourquoi je me pendrais à cette fenêtre ?\par
Je n’ai pas un reproche à m’adresser.\par
Je n’ai pas même une bille \emph{chipée} sur la conscience. Une fois mon père me donna trente sous pour acheter un cahier qui en coûtait vingt-neuf ; je gardai le sou. C’est mon seul vol. Je n’ai jamais \emph{rapporté}, oh ! non ! ni \emph{cané}\textbf{\footnote{Fuit.}} quand il fallait se battre.\par
Si c’était à Paris, encore ! En sortant de prison, on me serrerait la main tout de même. Ici, point !\par
Eh bien ! \emph{je ferai mon temps} ici, et j’irai à Paris après ; et quand je serai là, je ne cacherai pas que j’ai été en prison, je le crierai ! Je défendrai les DROITS DE L’ENFANT, comme d’autres les DROITS DE L’HOMME.\par
Je demanderai si les pères ont liberté de vie et de mort sur le corps et l’âme de leur fils ; si M. Vingtras a le droit de me martyriser parce que j’ai eu peur d’un métier de misère, et si M. Bergougnard peut encore crever la poitrine d’une Louisette.\par
Paris ! oh ! Je l’aime !\par
J’entrevois l’imprimerie et le journal, la liberté de se défendre, la sympathie aux révoltés.\par
L’idée de Paris me sauva de la corde ce jour-là. Je tourmentais déjà ma cravate.\par
\bigbreak
\noindent Encore des cris, des cris ! C’est deux jours après.\par
Ma mère, éperdue, entre dans ma chambre.\par
« Jacques, viens, viens ! »\par
On était en train d’insulter mon père. Il avait, quelques jours auparavant, frappé un de ses élèves, et voilà que dans la maison où la veille il avait failli me tuer, les parents de l’enfant calotté venaient exiger une réparation. On voulait que M. Vingtras fît des excuses, demandât pardon ; et comme M. Vingtras balbutiait, on lui mettait le poing sous le nez.\par
Ils étaient deux, le père et le frère aîné, un vieux et un jeune.\par
« Qu’y a-t-il ?\par
— Il y a, disait le jeune, que votre père s’est permis de gifler mon frère. S’il n’était pas si décati, c’est moi qui le giflerais.\par
— Malheureux ! »\par
Je l’ai pris à bras-le-corps. Ah ! il ne pèse pas lourd ! et le vieux non plus. Par la porte, allons ! Un peu plus, ils étaient en morceaux.\par
Ils amassaient du monde dans la rue.\par
« Viens donc, me crie le frère aîné écumant.\par
— Eh ! je viens ! »\par
\bigbreak
\noindent On nous a séparés à grand-peine. Il a dix-huit ans, c’est un saint-cyrien, il est courageux, mais je le \emph{règle}. Je le tiens comme j’ai vu l’oncle Chadenas tenir des cochons. Je ne veux pas lui faire de mal, maintenant qu’il est à terre. Seulement il bouge encore. On me tire par les cheveux.\par
On me l’a à peine ôté des mains qu’il me jette une carte par-dessus la foule. « Si c’était devant une épée, tu ferais moins le fier. C’est l’épée qui est mon arme, à moi », et il gesticule, et il en conte !…\par
L’imbécile !\par
« Hé, Massion, veux-tu aller lui dire que s’il ne se tait pas, je vais le \emph{casser} de nouveau, mais que s’il se tait, je me battrai à l’épée avec lui. »\par

\labelblock{Prairie de Mauves, 7 heures du matin.}

\noindent Ça s’est arrangé sans que chez nous on en sût rien. Tout le collège en parle, par exemple, mais mon père est au lit avec la fièvre, – le médecin a même ordonné qu’on le laissât reposer, – ce qui me donne ma liberté.\par
J’ai trouvé des témoins : tous ceux de mes anciens condisciples qui ont un brin de moustache et veulent entrer à Saint-Cyr ou à la Navale s’offrent pour la chose.\par
« Vous êtes bien jeune, dit quelqu’un mêlé aux pourparlers.\par
— J’ai dix-huit ans. »\par
Je mens de deux ans, voilà tout.\par
On se demande tout bas si au dernier moment je ne \emph{fouinerai} pas devant Saint-Cyr.\par
Ils ne savent pas que la vie m’embête, qu’un duel est comme un paletot neuf non choisi par ma mère, que c’est la première fois que je fais acte d’homme. C’est que j’en ai envie ; nom d’un tonnerre ! Si le saint-cyrien ne voulait plus, je l’y forcerais.\par
Je suis ému tout de même ! Je vais peut-être avoir l’air si gauche ? Mais je me ferai tuer tout de suite si l’on rit.\par
\bigbreak
\noindent Nous sommes sur le terrain.\par
« Avancez, messieurs ! »\par
Les témoins sont plus inquiets que nous, et puis ils ont peur de rater le cérémonial.\par
L’autre ne vient donc pas ?… Il a engagé le fer, puis a fait un bond en arrière et il me laisse là.\par
J’ai l’air d’un chien qui a perdu son maître.\par
Il ne vient pas, j’avance.\par
Cri du médecin !\par
« Quoi donc ?\par
— Vous êtes blessé.\par
— Moi ?\par
— Vous avez la cuisse pleine de sang. »\par
Je ne sens rien.\par
« Recommençons, recommençons ça ! »\par
Et croyant que c’est le grand genre de bondir en arrière comme a fait l’autre, je bondis.\par
« Mais c’est un saltimbanque ! » dit le chirurgien.\par
Enfin on m’amène à lui. Je ne sais pas encore pourquoi.\par
« Le gras de la cuisse traversé !\par
— Vous croyez ?\par
— Et quinze jours sans marcher ! »\par
Oh ! je n’ai pas grand endroit où aller !\par
Je suis donc blessé, il paraît. En effet, ça saigne.\par
Le saint-cyrien me serre la main et me dit : « Je regrette… »\par
Moi, je ne regrette rien. C’est un quart d’heure de passé, et j’ai vu que ça ne me faisait pas plus qu’un cautère sur une jambe de bois.\par
J’avais laissé un mot à ma mère le matin : « Je suis chez un camarade. »\par
Elle a même fait cette remarque :\par
« C’est mal pendant que son père est malade. »\par
\bigbreak
\noindent Je suis revenu en voiture. Il a fallu de l’argent pour cette voiture ; je n’en avais pas. En arrivant, j’ai dû demander trente sous à ma mère qui m’a cru fou.\par
« Il prend des voitures, maintenant ! »\par
L’escalier est noir.\par
J’ai monté en me tenant la jambe, sans rien dire, et, sous prétexte de migraine (on croit que j’ai bu), je suis allé me fourrer dans mon lit.\par
Mais une voisine, – à peine étais-je dans les draps, lui a conté toute l’histoire. Ma mère lâche le chevet de son époux pour le mien.\par
« Jacques, tu as \emph{été en duel} !\par
— Et mon père, comment va-t-il ? »\par
\bigbreak
\noindent Il est dans la chambre à côté de la mienne depuis ce matin. Le médecin a fait observer qu’il y avait plus d’air. Ma mère retourne à lui.\par
Je ne comprends pas bien ce qu’ils disent, mais on parle de moi, elle raconte l’histoire. Je saisis des bribes.\par
Un bruit qui se faisait dans l’escalier s’éteint et j’entends tout.\par
C’est mon père qui parle avec émotion :\par
« Oui, quand il sera guéri, il partira.\par
— Pour Paris ?\par
— Pour Paris.\par
— Il n’est pas blessé grièvement, n’est-ce pas ? Ce n’est rien, au moins ?\par
— Je t’ai dit que non. »\par
Un silence.\par
« C’est pour moi qu’il s’est battu… Après la scène de la veille !… »\par
Il semble que sa voix tremble.\par
« Oui, oui… il vaut mieux que nous nous séparions. De loin, nous ne nous querellerons pas. De près, il me haïrait !… Il me hait peut-être déjà ! Mais c’est plus fort que moi ! Ce professorat a fait de moi une vieille bête qui a besoin d’avoir l’air méchant, et qui le devient, à force de faire le croquemitaine et les yeux creux… Ça vous tanne le cœur… On est cruel… J’ai été cruel.\par
— Comme moi, dit ma mère… Mais je le lui ai dit un jour à Paris, je lui ai presque demandé pardon, et si tu avais vu comme il a pleuré !\par
— Toi, tu as su lui dire, moi je ne saurais pas. J’aurais peur de \emph{blesser la discipline}. Je craindrais que les élèves, je veux dire que mon fils ne rie de moi. J’ai été pion, et il m’en reste dans le sang. Je lui parlerai toujours comme à un écolier, et je le confondrai avec les gamins qu’il faut que je punisse pour qu’ils me craignent et qu’ils n’attachent pas des rats au collet de mon habit… Il vaut mieux qu’il parte.\par
— Tu l’embrasseras avant de partir.\par
— Non. Tu l’embrasseras pour moi. Je suis sûr que j’aurais encore l’air \emph{chien} sans le vouloir. C’est le professorat, je te dis !… Tu l’embrasseras… et tu lui diras, en cachette, que je l’aime bien… Moi, je n’ose pas. »\par
\bigbreak
\noindent « Madame, madame !\par
— Quoi donc !\par
— Il y a les agents en bas !\par
— Les agents ! »\par
Il y a, en effet, des étrangers dans l’escalier, et j’entends parler.\par
« Nous venons pour emmener votre fils.\par
— Parce qu’il s’est battu ? »\par
Elle remonte vers mon père.\par
« Plus bas, plus bas, mon amie, c’est moi qui avais écrit pour qu’on se tînt prêt à l’arrêter, depuis huit jours déjà !… J’avais signé, après cette scène… Oh ! j’ai honte… Il n’entend pas, dis, au moins, à travers la cloison ? »\par

\astertri

\noindent J’entends.\par
Quel bonheur que j’aie été blessé et que je sois couché dans ce lit ! Je n’aurais jamais su qu’il m’aimait.\par
Ah ! je crois qu’on eût mieux fait de m’aimer tout haut ! Il me semble qu’il me restera toujours, de ma vie d’enfant, des trous de mélancolie et des plaies sensibles dans le cœur !\par
\bigbreak
\noindent Mais aussi j’entre dans la vie d’homme, prêt à la lutte, plein de force, bien honnête. J’ai le sang pur et les yeux clairs, pour voir le fond des âmes ; ils sont comme cela, ai-je lu quelque part, ceux qui ont un peu pleuré.\par
Il ne s’agit plus de pleurer ! il faut \emph{vivre}.\par
Sans métier, sans argent, c’est dur ; mais on verra. Je suis mon maître à partir d’aujourd’hui. Mon père avait le droit de frapper… Mais malheur maintenant, malheur à qui me touche ! – Ah ! oui ! malheur à celui-là !\par
Je me parle ainsi, la cuisse tendue dans mon lit de blessé.\par
Huit jours après, le chirurgien vient, défait le bandage et dit :\par
« Grâce à mon pansement, – un nouveau système, – vous êtes guéri ; vous pouvez vous lever aujourd’hui et vous pourrez sortir demain. »\par
Ma mère remercie Dieu.\par
« Oh ! j’ai eu si peur !… S’il avait fallu te couper la jambe ! – je vais t’apprendre une nouvelle maintenant… »\par
Elle me conte tout ce que je sais, ce que j’ai entendu à travers la cloison.\par
\bigbreak
\noindent « Tu vas me quitter ! » dit-elle en sanglotant.\par
Je veux me lever tout de suite pour ramasser un peu mes livres, faire ma petite malle, et je lui demande mes habits.\par
Ce sont ceux du duel.\par
Ma mère les apporte. Elle aperçoit mon pantalon avec un trou et taché de sang.\par
« Je ne sais pas si le sang s’en ira… la couleur partira avec, bien sûr… »\par
Elle donne encore un coup de brosse, passe un petit linge mouillé, fait ce qu’il faut, – elle a toujours eu si soin de ma toilette ! – mais finit par dire en hochant la tête :\par
« Tu vois, ça ne s’en va pas… Une autre fois, Jacques, mets au moins ton vieux pantalon ! »
 


% at least one empty page at end (for booklet couv)
\ifbooklet
  \pagestyle{empty}
  \clearpage
  % 2 empty pages maybe needed for 4e cover
  \ifnum\modulo{\value{page}}{4}=0 \hbox{}\newpage\hbox{}\newpage\fi
  \ifnum\modulo{\value{page}}{4}=1 \hbox{}\newpage\hbox{}\newpage\fi


  \hbox{}\newpage
  \ifodd\value{page}\hbox{}\newpage\fi
  {\centering\color{rubric}\bfseries\noindent\large
    Hurlus ? Qu’est-ce.\par
    \bigskip
  }
  \noindent Des bouquinistes électroniques, pour du texte libre à participation libre,
  téléchargeable gratuitement sur \href{https://hurlus.fr}{\dotuline{hurlus.fr}}.\par
  \bigskip
  \noindent Cette brochure a été produite par des éditeurs bénévoles.
  Elle n’est pas faîte pour être possédée, mais pour être lue, et puis donnée.
  Que circule le texte !
  En page de garde, on peut ajouter une date, un lieu, un nom ; pour suivre le voyage des idées.
  \par

  Ce texte a été choisi parce qu’une personne l’a aimé,
  ou haï, elle a en tous cas pensé qu’il partipait à la formation de notre présent ;
  sans le souci de plaire, vendre, ou militer pour une cause.
  \par

  L’édition électronique est soigneuse, tant sur la technique
  que sur l’établissement du texte ; mais sans aucune prétention scolaire, au contraire.
  Le but est de s’adresser à tous, sans distinction de science ou de diplôme.
  Au plus direct ! (possible)
  \par

  Cet exemplaire en papier a été tiré sur une imprimante personnelle
   ou une photocopieuse. Tout le monde peut le faire.
  Il suffit de
  télécharger un fichier sur \href{https://hurlus.fr}{\dotuline{hurlus.fr}},
  d’imprimer, et agrafer ; puis de lire et donner.\par

  \bigskip

  \noindent PS : Les hurlus furent aussi des rebelles protestants qui cassaient les statues dans les églises catholiques. En 1566 démarra la révolte des gueux dans le pays de Lille. L’insurrection enflamma la région jusqu’à Anvers où les gueux de mer bloquèrent les bateaux espagnols.
  Ce fut une rare guerre de libération dont naquit un pays toujours libre : les Pays-Bas.
  En plat pays francophone, par contre, restèrent des bandes de huguenots, les hurlus, progressivement réprimés par la très catholique Espagne.
  Cette mémoire d’une défaite est éteinte, rallumons-la. Sortons les livres du culte universitaire, cherchons les idoles de l’époque, pour les briser.
\fi

\ifdev % autotext in dev mode
\fontname\font — \textsc{Les règles du jeu}\par
(\hyperref[utopie]{\underline{Lien}})\par
\noindent \initialiv{A}{lors là}\blindtext\par
\noindent \initialiv{À}{ la bonheur des dames}\blindtext\par
\noindent \initialiv{É}{tonnez-le}\blindtext\par
\noindent \initialiv{Q}{ualitativement}\blindtext\par
\noindent \initialiv{V}{aloriser}\blindtext\par
\Blindtext
\phantomsection
\label{utopie}
\Blinddocument
\fi
\end{document}
