%%%%%%%%%%%%%%%%%%%%%%%%%%%%%%%%%
% LaTeX model https://hurlus.fr %
%%%%%%%%%%%%%%%%%%%%%%%%%%%%%%%%%

% Needed before document class
\RequirePackage{pdftexcmds} % needed for tests expressions
\RequirePackage{fix-cm} % correct units

% Define mode
\def\mode{a4}

\newif\ifaiv % a4
\newif\ifav % a5
\newif\ifbooklet % booklet
\newif\ifcover % cover for booklet

\ifnum \strcmp{\mode}{cover}=0
  \covertrue
\else\ifnum \strcmp{\mode}{booklet}=0
  \booklettrue
\else\ifnum \strcmp{\mode}{a5}=0
  \avtrue
\else
  \aivtrue
\fi\fi\fi

\ifbooklet % do not enclose with {}
  \documentclass[french,twoside]{book} % ,notitlepage
  \usepackage[%
    papersize={105mm, 297mm},
    inner=12mm,
    outer=12mm,
    top=20mm,
    bottom=15mm,
    marginparsep=0pt,
  ]{geometry}
  \usepackage[fontsize=9.5pt]{scrextend} % for Roboto
\else\ifav
  \documentclass[french,twoside]{book} % ,notitlepage
  \usepackage[%
    a5paper,
    inner=25mm,
    outer=15mm,
    top=15mm,
    bottom=15mm,
    marginparsep=0pt,
  ]{geometry}
  \usepackage[fontsize=12pt]{scrextend}
\else% A4 2 cols
  \documentclass[twocolumn]{report}
  \usepackage[%
    a4paper,
    inner=15mm,
    outer=10mm,
    top=25mm,
    bottom=18mm,
    marginparsep=0pt,
  ]{geometry}
  \setlength{\columnsep}{20mm}
  \usepackage[fontsize=9.5pt]{scrextend}
\fi\fi

%%%%%%%%%%%%%%
% Alignments %
%%%%%%%%%%%%%%
% before teinte macros

\setlength{\arrayrulewidth}{0.2pt}
\setlength{\columnseprule}{\arrayrulewidth} % twocol
\setlength{\parskip}{0pt} % classical para with no margin
\setlength{\parindent}{1.5em}

%%%%%%%%%%
% Colors %
%%%%%%%%%%
% before Teinte macros

\usepackage[dvipsnames]{xcolor}
\definecolor{rubric}{HTML}{800000} % the tonic 0c71c3
\def\columnseprulecolor{\color{rubric}}
\colorlet{borderline}{rubric!30!} % definecolor need exact code
\definecolor{shadecolor}{gray}{0.95}
\definecolor{bghi}{gray}{0.5}

%%%%%%%%%%%%%%%%%
% Teinte macros %
%%%%%%%%%%%%%%%%%
%%%%%%%%%%%%%%%%%%%%%%%%%%%%%%%%%%%%%%%%%%%%%%%%%%%
% <TEI> generic (LaTeX names generated by Teinte) %
%%%%%%%%%%%%%%%%%%%%%%%%%%%%%%%%%%%%%%%%%%%%%%%%%%%
% This template is inserted in a specific design
% It is XeLaTeX and otf fonts

\makeatletter % <@@@


\usepackage{blindtext} % generate text for testing
\usepackage[strict]{changepage} % for modulo 4
\usepackage{contour} % rounding words
\usepackage[nodayofweek]{datetime}
% \usepackage{DejaVuSans} % seems buggy for sffont font for symbols
\usepackage{enumitem} % <list>
\usepackage{etoolbox} % patch commands
\usepackage{fancyvrb}
\usepackage{fancyhdr}
\usepackage{float}
\usepackage{fontspec} % XeLaTeX mandatory for fonts
\usepackage{footnote} % used to capture notes in minipage (ex: quote)
\usepackage{framed} % bordering correct with footnote hack
\usepackage{graphicx}
\usepackage{lettrine} % drop caps
\usepackage{lipsum} % generate text for testing
\usepackage[framemethod=tikz,]{mdframed} % maybe used for frame with footnotes inside
\usepackage{pdftexcmds} % needed for tests expressions
\usepackage{polyglossia} % non-break space french punct, bug Warning: "Failed to patch part"
\usepackage[%
  indentfirst=false,
  vskip=1em,
  noorphanfirst=true,
  noorphanafter=true,
  leftmargin=\parindent,
  rightmargin=0pt,
]{quoting}
\usepackage{ragged2e}
\usepackage{setspace} % \setstretch for <quote>
\usepackage{tabularx} % <table>
\usepackage[explicit]{titlesec} % wear titles, !NO implicit
\usepackage{tikz} % ornaments
\usepackage{tocloft} % styling tocs
\usepackage[fit]{truncate} % used im runing titles
\usepackage{unicode-math}
\usepackage[normalem]{ulem} % breakable \uline, normalem is absolutely necessary to keep \emph
\usepackage{verse} % <l>
\usepackage{xcolor} % named colors
\usepackage{xparse} % @ifundefined
\XeTeXdefaultencoding "iso-8859-1" % bad encoding of xstring
\usepackage{xstring} % string tests
\XeTeXdefaultencoding "utf-8"
\PassOptionsToPackage{hyphens}{url} % before hyperref, which load url package

% TOTEST
% \usepackage{hypcap} % links in caption ?
% \usepackage{marginnote}
% TESTED
% \usepackage{background} % doesn’t work with xetek
% \usepackage{bookmark} % prefers the hyperref hack \phantomsection
% \usepackage[color, leftbars]{changebar} % 2 cols doc, impossible to keep bar left
% \usepackage[utf8x]{inputenc} % inputenc package ignored with utf8 based engines
% \usepackage[sfdefault,medium]{inter} % no small caps
% \usepackage{firamath} % choose firasans instead, firamath unavailable in Ubuntu 21-04
% \usepackage{flushend} % bad for last notes, supposed flush end of columns
% \usepackage[stable]{footmisc} % BAD for complex notes https://texfaq.org/FAQ-ftnsect
% \usepackage{helvet} % not for XeLaTeX
% \usepackage{multicol} % not compatible with too much packages (longtable, framed, memoir…)
% \usepackage[default,oldstyle,scale=0.95]{opensans} % no small caps
% \usepackage{sectsty} % \chapterfont OBSOLETE
% \usepackage{soul} % \ul for underline, OBSOLETE with XeTeX
% \usepackage[breakable]{tcolorbox} % text styling gone, footnote hack not kept with breakable


% Metadata inserted by a program, from the TEI source, for title page and runing heads
\title{\textbf{ Essai sur l'inégalité des races humaines }}
\date{1855}
\author{Gobineau, Arthur de}
\def\elbibl{Gobineau, Arthur de. 1855. \emph{Essai sur l'inégalité des races humaines}}
\def\elsource{\{source\}}

% Default metas
\newcommand{\colorprovide}[2]{\@ifundefinedcolor{#1}{\colorlet{#1}{#2}}{}}
\colorprovide{rubric}{red}
\colorprovide{silver}{lightgray}
\@ifundefined{syms}{\newfontfamily\syms{DejaVu Sans}}{}
\newif\ifdev
\@ifundefined{elbibl}{% No meta defined, maybe dev mode
  \newcommand{\elbibl}{Titre court ?}
  \newcommand{\elbook}{Titre du livre source ?}
  \newcommand{\elabstract}{Résumé\par}
  \newcommand{\elurl}{http://oeuvres.github.io/elbook/2}
  \author{Éric Lœchien}
  \title{Un titre de test assez long pour vérifier le comportement d’une maquette}
  \date{1566}
  \devtrue
}{}
\let\eltitle\@title
\let\elauthor\@author
\let\eldate\@date


\defaultfontfeatures{
  % Mapping=tex-text, % no effect seen
  Scale=MatchLowercase,
  Ligatures={TeX,Common},
}


% generic typo commands
\newcommand{\astermono}{\medskip\centerline{\color{rubric}\large\selectfont{\syms ✻}}\medskip\par}%
\newcommand{\astertri}{\medskip\par\centerline{\color{rubric}\large\selectfont{\syms ✻\,✻\,✻}}\medskip\par}%
\newcommand{\asterism}{\bigskip\par\noindent\parbox{\linewidth}{\centering\color{rubric}\large{\syms ✻}\\{\syms ✻}\hskip 0.75em{\syms ✻}}\bigskip\par}%

% lists
\newlength{\listmod}
\setlength{\listmod}{\parindent}
\setlist{
  itemindent=!,
  listparindent=\listmod,
  labelsep=0.2\listmod,
  parsep=0pt,
  % topsep=0.2em, % default topsep is best
}
\setlist[itemize]{
  label=—,
  leftmargin=0pt,
  labelindent=1.2em,
  labelwidth=0pt,
}
\setlist[enumerate]{
  label={\bf\color{rubric}\arabic*.},
  labelindent=0.8\listmod,
  leftmargin=\listmod,
  labelwidth=0pt,
}
\newlist{listalpha}{enumerate}{1}
\setlist[listalpha]{
  label={\bf\color{rubric}\alph*.},
  leftmargin=0pt,
  labelindent=0.8\listmod,
  labelwidth=0pt,
}
\newcommand{\listhead}[1]{\hspace{-1\listmod}\emph{#1}}

\renewcommand{\hrulefill}{%
  \leavevmode\leaders\hrule height 0.2pt\hfill\kern\z@}

% General typo
\DeclareTextFontCommand{\textlarge}{\large}
\DeclareTextFontCommand{\textsmall}{\small}

% commands, inlines
\newcommand{\anchor}[1]{\Hy@raisedlink{\hypertarget{#1}{}}} % link to top of an anchor (not baseline)
\newcommand\abbr[1]{#1}
\newcommand{\autour}[1]{\tikz[baseline=(X.base)]\node [draw=rubric,thin,rectangle,inner sep=1.5pt, rounded corners=3pt] (X) {\color{rubric}#1};}
\newcommand\corr[1]{#1}
\newcommand{\ed}[1]{ {\color{silver}\sffamily\footnotesize (#1)} } % <milestone ed="1688"/>
\newcommand\expan[1]{#1}
\newcommand\foreign[1]{\emph{#1}}
\newcommand\gap[1]{#1}
\renewcommand{\LettrineFontHook}{\color{rubric}}
\newcommand{\initial}[2]{\lettrine[lines=2, loversize=0.3, lhang=0.3]{#1}{#2}}
\newcommand{\initialiv}[2]{%
  \let\oldLFH\LettrineFontHook
  % \renewcommand{\LettrineFontHook}{\color{rubric}\ttfamily}
  \IfSubStr{QJ’}{#1}{
    \lettrine[lines=4, lhang=0.2, loversize=-0.1, lraise=0.2]{\smash{#1}}{#2}
  }{\IfSubStr{É}{#1}{
    \lettrine[lines=4, lhang=0.2, loversize=-0, lraise=0]{\smash{#1}}{#2}
  }{\IfSubStr{ÀÂ}{#1}{
    \lettrine[lines=4, lhang=0.2, loversize=-0, lraise=0, slope=0.6em]{\smash{#1}}{#2}
  }{\IfSubStr{A}{#1}{
    \lettrine[lines=4, lhang=0.2, loversize=0.2, slope=0.6em]{\smash{#1}}{#2}
  }{\IfSubStr{V}{#1}{
    \lettrine[lines=4, lhang=0.2, loversize=0.2, slope=-0.5em]{\smash{#1}}{#2}
  }{
    \lettrine[lines=4, lhang=0.2, loversize=0.2]{\smash{#1}}{#2}
  }}}}}
  \let\LettrineFontHook\oldLFH
}
\newcommand{\labelchar}[1]{\textbf{\color{rubric} #1}}
\newcommand{\milestone}[1]{\autour{\footnotesize\color{rubric} #1}} % <milestone n="4"/>
\newcommand\name[1]{#1}
\newcommand\orig[1]{#1}
\newcommand\orgName[1]{#1}
\newcommand\persName[1]{#1}
\newcommand\placeName[1]{#1}
\newcommand{\pn}[1]{\IfSubStr{-—–¶}{#1}% <p n="3"/>
  {\noindent{\bfseries\color{rubric}   ¶  }}
  {{\footnotesize\autour{ #1}  }}}
\newcommand\reg{}
% \newcommand\ref{} % already defined
\newcommand\sic[1]{#1}
\newcommand\surname[1]{\textsc{#1}}
\newcommand\term[1]{\textbf{#1}}

\def\mednobreak{\ifdim\lastskip<\medskipamount
  \removelastskip\nopagebreak\medskip\fi}
\def\bignobreak{\ifdim\lastskip<\bigskipamount
  \removelastskip\nopagebreak\bigskip\fi}

% commands, blocks
\newcommand{\byline}[1]{\bigskip{\RaggedLeft{#1}\par}\bigskip}
\newcommand{\bibl}[1]{{\RaggedLeft{#1}\par\bigskip}}
\newcommand{\biblitem}[1]{{\noindent\hangindent=\parindent   #1\par}}
\newcommand{\dateline}[1]{\medskip{\RaggedLeft{#1}\par}\bigskip}
\newcommand{\labelblock}[1]{\medbreak{\noindent\color{rubric}\bfseries #1}\par\mednobreak}
\newcommand{\salute}[1]{\bigbreak{#1}\par\medbreak}
\newcommand{\signed}[1]{\bigbreak\filbreak{\raggedleft #1\par}\medskip}

% environments for blocks (some may become commands)
\newenvironment{borderbox}{}{} % framing content
\newenvironment{citbibl}{\ifvmode\hfill\fi}{\ifvmode\par\fi }
\newenvironment{docAuthor}{\ifvmode\vskip4pt\fontsize{16pt}{18pt}\selectfont\fi\itshape}{\ifvmode\par\fi }
\newenvironment{docDate}{}{\ifvmode\par\fi }
\newenvironment{docImprint}{\vskip6pt}{\ifvmode\par\fi }
\newenvironment{docTitle}{\vskip6pt\bfseries\fontsize{18pt}{22pt}\selectfont}{\par }
\newenvironment{msHead}{\vskip6pt}{\par}
\newenvironment{msItem}{\vskip6pt}{\par}
\newenvironment{titlePart}{}{\par }


% environments for block containers
\newenvironment{argument}{\itshape\parindent0pt}{\vskip1.5em}
\newenvironment{biblfree}{}{\ifvmode\par\fi }
\newenvironment{bibitemlist}[1]{%
  \list{\@biblabel{\@arabic\c@enumiv}}%
  {%
    \settowidth\labelwidth{\@biblabel{#1}}%
    \leftmargin\labelwidth
    \advance\leftmargin\labelsep
    \@openbib@code
    \usecounter{enumiv}%
    \let\p@enumiv\@empty
    \renewcommand\theenumiv{\@arabic\c@enumiv}%
  }
  \sloppy
  \clubpenalty4000
  \@clubpenalty \clubpenalty
  \widowpenalty4000%
  \sfcode`\.\@m
}%
{\def\@noitemerr
  {\@latex@warning{Empty `bibitemlist' environment}}%
\endlist}
\newenvironment{quoteblock}% may be used for ornaments
  {\begin{quoting}}
  {\end{quoting}}

% table () is preceded and finished by custom command
\newcommand{\tableopen}[1]{%
  \ifnum\strcmp{#1}{wide}=0{%
    \begin{center}
  }
  \else\ifnum\strcmp{#1}{long}=0{%
    \begin{center}
  }
  \else{%
    \begin{center}
  }
  \fi\fi
}
\newcommand{\tableclose}[1]{%
  \ifnum\strcmp{#1}{wide}=0{%
    \end{center}
  }
  \else\ifnum\strcmp{#1}{long}=0{%
    \end{center}
  }
  \else{%
    \end{center}
  }
  \fi\fi
}


% text structure
\newcommand\chapteropen{} % before chapter title
\newcommand\chaptercont{} % after title, argument, epigraph…
\newcommand\chapterclose{} % maybe useful for multicol settings
\setcounter{secnumdepth}{-2} % no counters for hierarchy titles
\setcounter{tocdepth}{5} % deep toc
\markright{\@title} % ???
\markboth{\@title}{\@author} % ???
\renewcommand\tableofcontents{\@starttoc{toc}}
% toclof format
% \renewcommand{\@tocrmarg}{0.1em} % Useless command?
% \renewcommand{\@pnumwidth}{0.5em} % {1.75em}
\renewcommand{\@cftmaketoctitle}{}
\setlength{\cftbeforesecskip}{\z@ \@plus.2\p@}
\renewcommand{\cftchapfont}{}
\renewcommand{\cftchapdotsep}{\cftdotsep}
\renewcommand{\cftchapleader}{\normalfont\cftdotfill{\cftchapdotsep}}
\renewcommand{\cftchappagefont}{\bfseries}
\setlength{\cftbeforechapskip}{0em \@plus\p@}
% \renewcommand{\cftsecfont}{\small\relax}
\renewcommand{\cftsecpagefont}{\normalfont}
% \renewcommand{\cftsubsecfont}{\small\relax}
\renewcommand{\cftsecdotsep}{\cftdotsep}
\renewcommand{\cftsecpagefont}{\normalfont}
\renewcommand{\cftsecleader}{\normalfont\cftdotfill{\cftsecdotsep}}
\setlength{\cftsecindent}{1em}
\setlength{\cftsubsecindent}{2em}
\setlength{\cftsubsubsecindent}{3em}
\setlength{\cftchapnumwidth}{1em}
\setlength{\cftsecnumwidth}{1em}
\setlength{\cftsubsecnumwidth}{1em}
\setlength{\cftsubsubsecnumwidth}{1em}

% footnotes
\newif\ifheading
\newcommand*{\fnmarkscale}{\ifheading 0.70 \else 1 \fi}
\renewcommand\footnoterule{\vspace*{0.3cm}\hrule height \arrayrulewidth width 3cm \vspace*{0.3cm}}
\setlength\footnotesep{1.5\footnotesep} % footnote separator
\renewcommand\@makefntext[1]{\parindent 1.5em \noindent \hb@xt@1.8em{\hss{\normalfont\@thefnmark . }}#1} % no superscipt in foot
\patchcmd{\@footnotetext}{\footnotesize}{\footnotesize\sffamily}{}{} % before scrextend, hyperref


%   see https://tex.stackexchange.com/a/34449/5049
\def\truncdiv#1#2{((#1-(#2-1)/2)/#2)}
\def\moduloop#1#2{(#1-\truncdiv{#1}{#2}*#2)}
\def\modulo#1#2{\number\numexpr\moduloop{#1}{#2}\relax}

% orphans and widows
\clubpenalty=9996
\widowpenalty=9999
\brokenpenalty=4991
\predisplaypenalty=10000
\postdisplaypenalty=1549
\displaywidowpenalty=1602
\hyphenpenalty=400
% Copied from Rahtz but not understood
\def\@pnumwidth{1.55em}
\def\@tocrmarg {2.55em}
\def\@dotsep{4.5}
\emergencystretch 3em
\hbadness=4000
\pretolerance=750
\tolerance=2000
\vbadness=4000
\def\Gin@extensions{.pdf,.png,.jpg,.mps,.tif}
% \renewcommand{\@cite}[1]{#1} % biblio

\usepackage{hyperref} % supposed to be the last one, :o) except for the ones to follow
\urlstyle{same} % after hyperref
\hypersetup{
  % pdftex, % no effect
  pdftitle={\elbibl},
  % pdfauthor={Your name here},
  % pdfsubject={Your subject here},
  % pdfkeywords={keyword1, keyword2},
  bookmarksnumbered=true,
  bookmarksopen=true,
  bookmarksopenlevel=1,
  pdfstartview=Fit,
  breaklinks=true, % avoid long links
  pdfpagemode=UseOutlines,    % pdf toc
  hyperfootnotes=true,
  colorlinks=false,
  pdfborder=0 0 0,
  % pdfpagelayout=TwoPageRight,
  % linktocpage=true, % NO, toc, link only on page no
}

\makeatother % /@@@>
%%%%%%%%%%%%%%
% </TEI> end %
%%%%%%%%%%%%%%


%%%%%%%%%%%%%
% footnotes %
%%%%%%%%%%%%%
\renewcommand{\thefootnote}{\bfseries\textcolor{rubric}{\arabic{footnote}}} % color for footnote marks

%%%%%%%%%
% Fonts %
%%%%%%%%%
\usepackage[]{roboto} % SmallCaps, Regular is a bit bold
% \linespread{0.90} % too compact, keep font natural
\newfontfamily\fontrun[]{Roboto Condensed Light} % condensed runing heads
\ifav
  \setmainfont[
    ItalicFont={Roboto Light Italic},
  ]{Roboto}
\else\ifbooklet
  \setmainfont[
    ItalicFont={Roboto Light Italic},
  ]{Roboto}
\else
\setmainfont[
  ItalicFont={Roboto Italic},
]{Roboto Light}
\fi\fi
\renewcommand{\LettrineFontHook}{\bfseries\color{rubric}}
% \renewenvironment{labelblock}{\begin{center}\bfseries\color{rubric}}{\end{center}}

%%%%%%%%
% MISC %
%%%%%%%%

\setdefaultlanguage[frenchpart=false]{french} % bug on part


\newenvironment{quotebar}{%
    \def\FrameCommand{{\color{rubric!10!}\vrule width 0.5em} \hspace{0.9em}}%
    \def\OuterFrameSep{\itemsep} % séparateur vertical
    \MakeFramed {\advance\hsize-\width \FrameRestore}
  }%
  {%
    \endMakeFramed
  }
\renewenvironment{quoteblock}% may be used for ornaments
  {%
    \savenotes
    \setstretch{0.9}
    \normalfont
    \begin{quotebar}
  }
  {%
    \end{quotebar}
    \spewnotes
  }


\renewcommand{\headrulewidth}{\arrayrulewidth}
\renewcommand{\headrule}{{\color{rubric}\hrule}}

% delicate tuning, image has produce line-height problems in title on 2 lines
\titleformat{name=\chapter} % command
  [display] % shape
  {\vspace{1.5em}\centering} % format
  {} % label
  {0pt} % separator between n
  {}
[{\color{rubric}\huge\textbf{#1}}\bigskip] % after code
% \titlespacing{command}{left spacing}{before spacing}{after spacing}[right]
\titlespacing*{\chapter}{0pt}{-2em}{0pt}[0pt]

\titleformat{name=\section}
  [block]{}{}{}{}
  [\vbox{\color{rubric}\large\raggedleft\textbf{#1}}]
\titlespacing{\section}{0pt}{0pt plus 4pt minus 2pt}{\baselineskip}

\titleformat{name=\subsection}
  [block]
  {}
  {} % \thesection
  {} % separator \arrayrulewidth
  {}
[\vbox{\large\textbf{#1}}]
% \titlespacing{\subsection}{0pt}{0pt plus 4pt minus 2pt}{\baselineskip}

\ifaiv
  \fancypagestyle{main}{%
    \fancyhf{}
    \setlength{\headheight}{1.5em}
    \fancyhead{} % reset head
    \fancyfoot{} % reset foot
    \fancyhead[L]{\truncate{0.45\headwidth}{\fontrun\elbibl}} % book ref
    \fancyhead[R]{\truncate{0.45\headwidth}{ \fontrun\nouppercase\leftmark}} % Chapter title
    \fancyhead[C]{\thepage}
  }
  \fancypagestyle{plain}{% apply to chapter
    \fancyhf{}% clear all header and footer fields
    \setlength{\headheight}{1.5em}
    \fancyhead[L]{\truncate{0.9\headwidth}{\fontrun\elbibl}}
    \fancyhead[R]{\thepage}
  }
\else
  \fancypagestyle{main}{%
    \fancyhf{}
    \setlength{\headheight}{1.5em}
    \fancyhead{} % reset head
    \fancyfoot{} % reset foot
    \fancyhead[RE]{\truncate{0.9\headwidth}{\fontrun\elbibl}} % book ref
    \fancyhead[LO]{\truncate{0.9\headwidth}{\fontrun\nouppercase\leftmark}} % Chapter title, \nouppercase needed
    \fancyhead[RO,LE]{\thepage}
  }
  \fancypagestyle{plain}{% apply to chapter
    \fancyhf{}% clear all header and footer fields
    \setlength{\headheight}{1.5em}
    \fancyhead[L]{\truncate{0.9\headwidth}{\fontrun\elbibl}}
    \fancyhead[R]{\thepage}
  }
\fi

\ifav % a5 only
  \titleclass{\section}{top}
\fi

\newcommand\chapo{{%
  \vspace*{-3em}
  \centering % no vskip ()
  {\Large\addfontfeature{LetterSpace=25}\bfseries{\elauthor}}\par
  \smallskip
  {\large\eldate}\par
  \bigskip
  {\Large\selectfont{\eltitle}}\par
  \bigskip
  {\color{rubric}\hline\par}
  \bigskip
  {\Large TEXTE LIBRE À PARTICPATION LIBRE\par}
  \centerline{\small\color{rubric} {hurlus.fr, tiré le \today}}\par
  \bigskip
}}

\newcommand\cover{{%
  \thispagestyle{empty}
  \centering
  {\LARGE\bfseries{\elauthor}}\par
  \bigskip
  {\Large\eldate}\par
  \bigskip
  \bigskip
  {\LARGE\selectfont{\eltitle}}\par
  \vfill\null
  {\color{rubric}\setlength{\arrayrulewidth}{2pt}\hline\par}
  \vfill\null
  {\Large TEXTE LIBRE À PARTICPATION LIBRE\par}
  \centerline{{\href{https://hurlus.fr}{\dotuline{hurlus.fr}}, tiré le \today}}\par
}}

\begin{document}
\pagestyle{empty}
\ifbooklet{
  \cover\newpage
  \thispagestyle{empty}\hbox{}\newpage
  \cover\newpage\noindent Les voyages de la brochure\par
  \bigskip
  \begin{tabularx}{\textwidth}{l|X|X}
    \textbf{Date} & \textbf{Lieu}& \textbf{Nom/pseudo} \\ \hline
    \rule{0pt}{25cm} &  &   \\
  \end{tabularx}
  \newpage
  \addtocounter{page}{-4}
}\fi

\thispagestyle{empty}
\ifaiv
  \twocolumn[\chapo]
\else
  \chapo
\fi
{\it\elabstract}
\bigskip
\makeatletter\@starttoc{toc}\makeatother % toc without new page
\bigskip

\pagestyle{main} % after style

  
\chapteropen
\chapter[{Dédicace de la première édition (1854)}]{Dédicace de la première édition (1854)}\renewcommand{\leftmark}{Dédicace de la première édition (1854)}


\salute{À SA MAJESTÉ \\
GEORGES V. \\
ROI DE HANOVRE}

\salute{SIRE,}

\chaptercont
\noindent J’ai l’honneur d’offrir ici à VOTRE MAJESTÉ le fruit de longues méditations et d’études favorites, souvent interrompues, toujours reprises.\par
Les événements considérables, révolutions, guerres sanglantes, renversements de lois, qui, depuis trop d’années, ont agi sur les États européens, tournent aisément les imaginations vers l’examen des faits politiques. Tandis que le vulgaire n’en considère que les résultats immédiats et n’admire ou ne réprouve que l’étincelle électrique dont ils frappent les intérêts, les penseurs plus graves cherchent à découvrir les causes cachées de si terribles ébranlements, et, descendant la lampe à la main dans les sentiers obscurs de la philosophie et de l’histoire, ils vont demander à l’analyse du cœur humain ou à l’examen attentif des annales le mot d’une énigme qui trouble si fort et les existences et les consciences.\par
Comme chacun, j’ai ressenti ce que l’agitation des époques modernes inspire de soucieuse curiosité. Mais, en appliquant à en comprendre les mobiles toutes les forces de mon intelligence, j’ai vu l’horizon de mes étonnements, déjà si vaste, s’agrandir encore. Quittant, peu à peu, je l’avoue, l’observation de l’ère actuelle pour celle des périodes précédentes, puis du passé tout entier, j’ai réuni ces fragments divers dans un ensemble immense, et, conduit par l’analogie, je me suis tourné, presque malgré moi, vers la divination de l’avenir le plus lointain. Ce n’a plus été seulement les causes directes de nos tourmentes soi-disant réformatrices qu’il m’a semblé désirable de connaître : j’ai aspiré à découvrir les raisons plus hautes de cette identité des maladies sociales que la connaissance la plus imparfaite des chroniques humaines suffit à faire remarquer dans toutes les nations qui furent jamais, qui sont, comme, selon toute vraisemblance, dans celles qui seront un jour.\par
Je crus, d’ailleurs, apercevoir, pour de tels travaux des facilités particulières à l’époque présente. Si, par ses agitations, elle pousse à la pratique d’une sorte de chimie historique, elle en facilite aussi les labeurs. Le brouillard épais, les ténèbres profondes qui nous cachaient, depuis une date immémoriale, les débuts des civilisations différen­tes de la nôtre, se lèvent et se dissolvent aujourd’hui au soleil de la science. Une merveilleuse épuration des méthodes analytiques, après avoir, sous les mains de Niebuhr, fait apparaître une Rome ignorée de Tite-Live, nous découvre et nous explique aussi les vérités mêlées aux récits fabuleux de l’enfance hellénique. Vers un autre point du monde, les peuples germains, longtemps méconnus, se montrent à nous aussi grands, aussi majestueux que les écrivains du Bas-Empire nous les avaient dits barbares. L’Égypte ouvre ses hypogées, traduit ses hiéroglyphes, confesse l’âge de ses pyramides. L’Assyrie dévoile et ses palais et leurs inscriptions sans fin, naguère encore évanouies sous leurs propres décombres. L’Iran de Zoroastre n’a su rien cacher aux puissantes investigations de Burnouf, et l’Inde primitive nous raconte, dans les Védas, des faits bien proches du lendemain de la création. De l’ensemble de ces conquêtes, déjà si importantes en elles-mêmes, résulte encore une compréhension plus juste et plus large d’Hérodote, d’Homère et surtout des premiers chapitres du Livre saint, cet abîme d’assertions dont on n’admire jamais assez la richesse et la rectitude lorsqu’on l’aborde avec un esprit suffisamment pourvu de lumières.\par
Tant de découvertes inattendues ou inespérées ne se placent pas, sans doute, au-dessus des atteintes de toute critique. Elles sont loin de présenter, sans lacunes, les listes des dynasties, l’enchaînement régulier des règnes et des faits. Cependant, au milieu de leurs résultats incomplets, il en est d’admirables, pour les travaux qui m’occupent, il en est de plus fructueux que ne sauraient l’être les tables chronologiques les mieux suivies. Ce que j’y recueille avec joie, c’est la révélation des usages, des mœurs, jusqu’aux portraits, jusqu’aux costumes des nations disparues. On connaît désormais l’état de leurs arts. On aperçoit toute leur vie, physique et morale, publique et privée, et il nous est devenu possible de reconstruire, au moyen des matériaux les plus authentiques, ce qui fait la personnalité des races et le principal critérium de leur valeur.\par
Devant un tel amoncellement de richesses toutes neuves ou tout nouvelle­ment comprises, personne n’est plus autorisé à prétendre expliquer le jeu compliqué des rapports sociaux, les motifs des élévations et des décadences nationales avec l’unique secours des considérations abstraites et purement hypothétiques qu’une philosophie sceptique peut fournir. Puisque les faits positifs abondent désormais, qu’ils surgissent de partout, se relèvent de tous les sépulcres, et se dressent sous la main de qui veut les interroger, il n’est plus loisible d’aller, avec les théoriciens révolutionnaires, amasser des nuages pour en former des hommes fantastiques et se donner le plaisir de faire mouvoir artificiellement des chimères dans des milieux politiques qui leur ressemblent. La réalité, trop notoire, trop pressante, interdit de tels jeux, souvent impies, toujours néfastes. Pour décider sainement des caractères de l’humanité, le tribunal de l’histoire est devenu le seul compétent. C’est d’ailleurs, j’en conviens, un arbitre sévère, un juge bien redoutable à évoquer à des époques aussi tristes que celle-ci.\par
Non pas que le passé soit lui-même immaculé. Il contient tout, et, à ce titre, on en obtient l’aveu de bien des fautes et l’on y découvre plus d’une honteuse défaillance. Les hommes d’aujourd’hui seraient même en droit de faire, devant lui, trophée de quelques mérites qui lui manquent. Mais, si, pour repousser leurs accusations, il vient soudain à évoquer les ombres grandioses des périodes héroïques, que diront-ils ? S’il leur reproche d’avoir compromis la foi religieuse, la fidélité politique, le culte du devoir, que répondre ? S’il leur affirme qu’ils ne sont plus aptes qu’à poursuivre le défrichement de connaissances dont les principes ont été reconnus et exposés par lui ; s’il ajoute que l’antique vertu est devenue un objet de risée ; que l’énergie a passé de l’homme à la vapeur ; que la poésie s’est éteinte, que ses grands interprètes ne vivent plus ; que ce qu’on nomme des intérêts se ravale aux considérations les plus mesquines ; qu’alléguer ?\par
Rien, sinon que toutes les belles choses, tombées dans le silence, ne sont pas mortes et qu’elles dorment ; que tous les âges ont vu des périodes de transition, époques où la souffrance lutte avec la vie et d’où celle-ci se détache, à la fin, victorieuse et resplendissante, et que, puisque la Chaldée trop vieillie fut remplacée jadis par la Perse jeune et vigoureuse, la Grèce décrépite par Rome virile et la domination abâtardie d’Augustule par les royaumes des nobles princes teutoniques, de même les races modernes obtiendront leur rajeunissement.\par
C’est là ce que j’ai moi-même espéré un instant, un bien court instant, et j’aurais voulu répondre immédiatement à l’Histoire pour confondre ses accusations et ses sombres pronostics, si je n’avais été frappé de cette considération accablante, que je me hâtais trop d’avancer une proposition dénuée de preuves. Je voulus en chercher, et ainsi j’étais ramené sans cesse, par ma sympathie pour les manifestations de l’humanité vivante, à approfondir davantage les secrets de l’humanité morte.\par
C’est alors que, d’inductions en inductions, j’ai dû me pénétrer de cette évidence, que la question ethnique domine tous les autres problèmes de l’histoire, en tient la clef, et que l’inégalité des races dont le concours forme une nation, suffit à expliquer tout l’enchaînement des destinées des peuples. Il n’est personne, d’ailleurs, qui n’ait été frappé de quelque pressentiment d’une vérité si éclatante. Chacun a pu observer que certains groupes humains, en s’abattant sur un pays, y ont transformé jadis, par une action subite, et les habitudes et la vie, et que, là où, avant leur arrivée, régnait la torpeur, ils se sont montrés habiles à faire jaillir une activité inconnue. C’est ainsi, pour en citer un exemple, qu’une puissance nouvelle fut préparée à la Grande-Bretagne par l’invasion anglo-saxonne, au gré d’un arrêt de la Providence qui, en conduisant dans cette île quelques-uns des peuples gouvernés par le glaive des illustres ancêtres de VOTRE MAJESTÉ, se réservait, comme le remarquait, un jour, avec profondeur, une Auguste Personne, de rendre aux deux branches de la même nation cette même maison souveraine, qui puise ses droits glorieux aux sources lointaines de la plus héroïque origine.\par
Après avoir reconnu qu’il est des races fortes et qu’il en est de faibles, je me suis attaché à observer de préférence les premières, à démêler leurs aptitudes, et surtout à remonter la chaîne de leurs généalogies. En suivant cette méthode, j’ai fini par me convaincre que tout ce qu’il y a de grand, de noble, de fécond sur la terre, en fait de créations humaines, la science, l’art, la civilisation, ramène l’observateur vers un point unique, n’est issu que d’un même germe, n’a résulté que d’une seule pensée, n’appartient qu’à une seule famille dont les différentes branches ont régné dans toutes les contrées policées de l’Univers.\par
L’exposition de cette synthèse se trouve dans ce livre, dont je viens déposer l’hommage au pied du trône de VOTRE MAJESTÉ. Il ne m’appartenait pas, et je n’y ai pas songé, de quitter les régions élevées et pures de la discussion scientifique pour descendre sur le terrain de la polémique contemporaine. je n’ai cherché à éclaircir ni l’avenir de demain, ni celui même des années qui vont suivre. Les périodes que je trace sont amples et larges. Je débute avec les premiers peuples qui furent jadis, pour chercher jusqu’à ceux qui ne sont pas encore. Je ne calcule que par séries de siècles. Je fais, en un mot, de la géologie morale. Je parle rarement de l’homme, plus rarement encore du citoyen ou du sujet, souvent, toujours des différentes fractions ethniques, car il ne s’agit pour moi, sur les cimes où je me suis placé, ni des nationalités fortuites, ni même de l’existence des États, mais des races, des sociétés et des civilisations diverses,\par
En osant tracer ici ces considérations, je me sens enhardi, SIRE, par la protection que l’esprit vaste et élevé de VOTRE MAJESTÉ accorde aux efforts de l’intelligence et par l’intérêt plus particulier dont Elle honore les travaux de l’érudition historique. Je ne saurais perdre jamais le souvenir des précieux enseignements qu’il m’a été donné de recueillir de la bouche de VOTRE MAJESTÉ, et j’oserai ajouter que je ne sais qu’admirer davantage des connaissances si brillantes, si solides, dont le Souverain du Hanovre possède les moissons les plus variées, ou du généreux sentiment et des nobles aspirations qui les fécondent et assurent à ses peuples un règne si prospère.\par
Plein d’une reconnaissance inaltérable pour les bontés de VOTRE MAJESTÉ, je La prie de daigner accueillir\par
L’expression du profond respect avec lequel j’ai l’honneur d’être,\par
Sire,\par


\signed{De VOTRE MAJESTÉ, \\
Le très humble et très obéissant serviteur, \\
A. de GOBINEAU.}
\chapterclose


\chapteropen
\chapter[{Avant-propos, de la deuxième édition}]{Avant-propos \\
de la deuxième édition}\renewcommand{\leftmark}{Avant-propos \\
de la deuxième édition}


\chaptercont
\noindent Ce livre a été publié pour la première fois en 1853 (tome I et tome II) les deux derniers volumes (tome III et tome IV) sont de 1855. L’édition actuelle n’y a pas changé une ligne, non pas que, dans l’intervalle, des travaux considérables n’aient déterminé bien des progrès de détail. Mais aucune des vérités que j’ai émises n’a été ébranlée, et j’ai trouvé nécessaire de maintenir la vérité telle que je l’ai trouvée. Jadis, on n’avait sur les Races humaines que des doutes très timides. On sentait vaguement qu’il fallait fouiller de ce côté si l’on voulait mettre à découvert la base encore inaperçue de l’histoire et on pressentait que dans cet ordre de notions si peu dégrossies, sous ces mystères si obscurs, devaient se rencontrer à de certaines profondeurs les vastes substructions sur lesquelles se sont graduellement élevées les assises, puis les murs, bref tous les développements sociaux des multitudes si variées dont l’ensemble compose la marqueterie de nos peuples. Mais on ne voyait pas la marche à suivre pour rien conclure.\par
Depuis la seconde moitié du dernier siècle, on raisonnait sur les annales générales et on prétendait, pourtant, à ramener tous ces phénomènes dont ils présentent les séries, à des lois fixes. Cette nouvelle manière de tout classer, de tout expliquer, de louer, de condamner, au moyen de formules abstraites dont on s’efforçait de démontrer la rigueur, conduisait naturellement à soupçonner, sous l’éclosion des faits, une force dont on n’avait encore jamais reconnu la nature. La prospérité ou l’infortune d’une nation, sa grandeur et sa décadence, on s’était longtemps contenté de les faire résulter des vertus et des vices éclatant sur le point spécial qu’on examinait. Un peuple honnête devait être nécessaire­ment un peuple illustre, et, au rebours, une société qui pratiquait trop librement le recrutement actif des consciences relâchées, amenait sans merci la ruine de Suse, d’Athènes, de Rome, tout comme une situation analogue avait attiré le châtiment final sur les cités décriées de la Mer Morte.\par
En faisant tourner de pareilles clefs, on avait cru ouvrir tous les mystères ; mais, en réalité, tout restait clos. Les vertus utiles aux grandes agglomérations doivent avoir un caractère bien particulier d’égoïsme collectif qui ne les rend pas pareilles à ce qu’on appelle vertu chez les particuliers. Le bandit spartiate, l’usurier romain ont été des personnages publics d’une rare efficacité, bien qu’à en juger au point de vue moral, et Lysandre et Caton fussent d’assez méchantes gens ; il fallut en convenir après réflexion et, en conséquence, si on s’avisait de louer la vertu chez un peuple et de dénoncer avec indignation le vice chez un autre, on se vit obligé de reconnaître et d’avouer tout haut qu’il ne s’agissait pas là de mérites et de démérites intéressant la conscience chrétienne, mais bien de certaines aptitudes, de certaines puissances actives de l’âme et même du corps, déterminant ou paralysant le développement de la vie dans les nations, ce qui conduisit à se demander pourquoi l’une de celles-ci pouvait ce que l’autre ne pouvait pas, et ainsi on se trouva induit à avouer que c’était un fait résultant de la race.\par
Pendant quelque temps on se contenta de cette déclaration à laquelle on ne savait comment donner la précision nécessaire. C’était un mot creux, c’était une phrase, et aucune époque ne s’est jamais payée de phrases et n’en a eu le goût comme celle d’à présent. Une sorte d’obscurité translucide qui émane ordinairement des mots inexpliqués était projetée ici par les études physiologiques et suffisait, ou, du moins, on voulut quelque temps encore s’en contenter. D’ailleurs, on avait un peu peur de ce qui allait suivre. On sentait que si la valeur intrinsèque d’un peuple dérive de son origine, il fallait restreindre, peut-être supprimer tout ce qu’on appelle Égalité et, en outre, un peuple grand ou misérable ne serait donc ni à louer, ni à blâmer. Il en serait comme de la valeur relative de l’or et du cuivre. On reculait devant de tels aveux.\par
Fallait-il admettre, en ces jours de passion enfantine pour l’égalité, qu’une hiérar­chie si peu démocratique existât parmi les fils d’Adam ? combien de dogmes, aussi bien philosophiques que religieux, se déclaraient prêts à réclamer !\par
Tandis qu’on hésitait, on marchait pourtant ; les découvertes s’accumulaient et leurs voix se haussaient et exigeaient qu’on parlât raison. La géographie racontait ce qui s’étalait à sa vue ; les collections regorgeaient de nouveaux types humains. L’histoire antique mieux étudiée, les secrets asiatiques plus révélés, les traditions américaines devenues accessibles comme elles ne l’étaient pas auparavant, tout proclamait l’importance de la race. Il fallait se décider à entrer dans la question telle qu’elle est.\par
 Sur ces entrefaites, se présenta un physiologiste, M. Pritchard, historien médiocre, théologien plus médiocre encore, qui, voulant surtout prouver que toutes les races se valaient, soutint qu’on avait tort d’avoir peur et se donna peur à lui-même. Il se proposa non pas de savoir et de dire la vérité des choses, mais de rassurer la philanthropie. Dans cette intention, il cousu les uns aux autres un certain nombre de faits isolés, observés plus ou moins bien et qui ne demandaient pas mieux que de prouver l’aptitude innée du nègre de Mozambique, et du Malais des îles Mariannes à devenir de fort grands personnages pour peu que l’occasion s’en présentât. M. Pritchard fut néanmoins grandement à estimer par cela seul qu’il toucha réellement à la difficulté. Ce fut, il est vrai, par le petit côté, mais ce fut pourtant et on ne saurait trop lui en savoir gré.\par
J’écrivis alors le livre dont je présente ici la seconde édition. Depuis qu’il a paru, des discussions nombreuses ont eu lieu à son sujet. Les principes en ont été moins combattus que les applications et surtout que les conclusions. Les partisans du progrès illimité ne lui ont pas été favorables. Le savant Ewald émettait l’avis que c’était une inspiration des catholiques extrêmes ; l’école positiviste l’a déclaré dangereux. Cependant des écrivains qui ne sont ni catholiques ni positivistes, mais qui possèdent aujourd’hui une grande réputation, en ont fait entrer incognito, sans l’avouer, les principes et même des parties entières dans leurs œuvres et, en somme, Fallmereyer n’a pas eu tort de dire qu’on s’en servait plus souvent et plus largement qu’on n’était disposé à en convenir.\par
Une des idées maîtresses de cet ouvrage, c’est la grande influence des mélanges ethniques, autrement dit des mariages entre les races diverses. Ce fut la première fois qu’on posa cette observation et qu’en en faisant ressortir les résultats au point de vue social, on présenta cet axiome que tant valait le mélange obtenu, tant valait la variété humaine produit de ce mélange et que les progrès et les reculs des sociétés ne sont autre chose que les effets de ce rapprochement. De là fut tirée la théorie de la sélection devenue si célèbre entre les mains de Darwin et plus encore de ses élèves. Il en est résulté, entre autres, le système de Buckle, et par l’écart considérable que les opinions de ce philosophe présentent avec les miennes, on peut mesurer l’éloignement relatif des routes que savent se frayer deux pensées hostiles parties d’un point commun. Buckle a été interrompu dans son travail par la mort, mais la saveur démocratique de ses sentiments lui a assuré, dans ces temps-ci, un succès que la rigueur de ses déductions ne justifie pas plus que la solidité de ses connaissances.\par
Darwin et Buckle ont créé ainsi les dérivations principales du ruisseau que j’ai ouvert. Beaucoup d’autres ont simplement donné comme des vérités trouvées par eux-mêmes ce qu’ils copiaient chez moi en y mêlant tant bien que mal les idées aujourd’hui de mode.\par
Je laisse donc mon livre tel que je l’ai fait et je n’y changerai absolument rien. C’est l’exposé d’un système, c’est l’expression d’une vérité qui m’est aussi claire et aussi indubitable aujourd’hui qu’elle me l’était au temps où je l’ai professée pour la première fois. Les progrès des connaissances historiques ne m’ont fait changer d’opinion en aucune sorte ni dans aucune mesure. Mes convictions d’autrefois sont celles d’aujourd’hui, qui n’ont incliné ni à droite ni à gauche, mais qui sont restées telles qu’elles avaient poussé dès le premier moment où je les ai connues. Les acquisitions survenues dans le domaine des faits ne leur nuisent pas. Les détails se sont multipliés, j’en suis aise. Ils n’ont rien altéré des constatations acquises. Je suis satisfait que les témoignages fournis par l’expérience aient encore plus démontré la réalité de l’inégalité des Races.\par
J’avoue que j’aurais pu être tenté de joindre ma protestation à tant d’autres qui s’élèvent contre le darwinisme. Heureusement, je n’ai pu oublier que mon livre n’est pas une œuvre de polémique. Son but est de professer une vérité et non de faire la guerre aux erreurs. Je dois donc résister à une tentation belliqueuse. C’est pourquoi je me garderai également de disputer contre ce prétendu approfondissement de l’érudition qui, sous le nom d’études préhistoiques, ne laisse pas que d’avoir fait dans le monde un bruit assez sonore. Se dispenser de connaître et surtout d’examiner les documents les plus anciens de tous les peuples, c’est comme une règle, toujours facile, de ce prétendu genre de travaux. C’est une manière de se supposer libre de tous renseignements ; on déclare ainsi la table rase, et l’on se trouve parfaitement autorisé à l’encombrer à son choix de telles hypothèses qui peuvent convenir et que l’on peut mettre oit l’on suppose le vide. Alors, on dispose tout à son gré et, au moyen d’une phraséologie spéciale, en supputant les temps, par âges de pierre, de bronze, de fer, en substituant le vague géologique à des approximations de chronologie qui ne seraient pas assez surpre­nantes, on parvient à se mettre l’esprit dans un état de surexcitation aiguë, qui permet de tout imaginer et de tout trouver admissible. Alors au milieu des incohérences les plus fantasques, on ouvre tout à coup, dans tous les coins du globe terrestre, des trous, des caves, des cavernes de l’aspect le plus sauvage, et on en fait sortir des amoncelle­ments épouvantables de crânes et de tibias fossiles, de détritus comestibles, d’écailles d’huîtres et d’ossements de tous les animaux possibles et impossibles, taillés, gravés, éraflés, polis et non polis, de haches, de têtes de flèches, d’outils sans noms ; et le tout s’écroulant sur les imaginations troublées, aux fanfares retentissantes d’une pédanterie sans pareille, les ahurit d’une manière si irrésistible que les adeptes peuvent sans scrupule, avec sir John Lubbock et M. Evans, héros de ces rudes labeurs, assigner à toutes ces belles choses une antiquité, tantôt de cent mille années, tantôt une autre de cinq cent mille, et ce sont des différences d’avis dont on ne s’explique pas le moins du monde le motif.\par
Il faut savoir respecter les congrès préhistoriques et leurs amusements. Le goût en passera quand de pareils excès auront été poussés encore un peu plus loin, et que les esprits rebutés réduiront simplement à rien toutes ces folies. À dater de cette réforme indispensable on enlèvera enfin les haches de silex et les couteaux d’obsidienne aux mains des anthropoïdes de M. le professeur Haeckel, gens qui en font un si mauvais usage.\par
Ces rêveries, dis-je, passeront d’elles-mêmes. On les voit déjà passer. L’ethnologie a besoin de jeter ses gourmes avant de se trouver sage. Il fut un temps, et il n’est pas loin, où les préjugés contre les mariages consanguins étaient devenus tels qu’il fut question de leur donner la consécration de la loi. Épouser une cousine germaine équivalait à frapper à l’avance tous ses enfants de surdité et d’autres affections hérédi­taires. Personne ne semblait réfléchir que les générations qui ont précédé la nôtre, fort adonnées aux mariages consanguins, n’ont rien connu des conséquences morbides qu’on prétend leur attribuer ; que les Séleucides, les Ptolémées, les Incas, époux de leurs sœurs, étaient, les uns et les autres, de très bonne santé et d’intelligence fort acceptable, sans parler de leur beauté, généralement hors ligne. Des faits si concluants, si irréfutables, ne pouvaient convaincre personne, parce qu’on prétendait utiliser, bon gré mal gré, les fantaisies d’un libéralisme qui, n’aimant pas l’exclusivité chapitrale, était contraire à toute pureté du sang, et l’on voulait autant que possible célébrer l’union du nègre et du blanc d’où provient le mulâtre. Ce qu’il fallait démontrer dangereux, inadmissible, c’était une race qui ne s’unissait et ne se perpétuait qu’avec elle-même. Quand on eut suffisamment déraisonné, les expériences tout à fait concluantes du docteur Broca ont rejeté pour toujours un paradoxe que les fantasmagories du même genre iront rejoindre quand leur fin sera arrivée.\par
Encore une fois, je laisse ces pages telles que je les ai écrites à l’époque où la doctrine qu’elles contiennent sortait de mon esprit, comme un oiseau met la tête hors du nid et cherche sa route dans l’espace où il n’y a pas de limites. Ma théorie a été ce qu’elle était, avec ses faiblesses et sa force, son exactitude et sa part d’erreurs, pareille à toutes les divinations de l’homme. Elle a pris son essor, elle le continue. Je n’essaierai ni de raccourcir, ni d’allonger ses ailes, ni moins encore de rectifier son vol. Qui me prouverait qu’aujourd’hui je le dirigerais mieux et surtout que j’atteindrais plus haut dans les parages de la vérité ? Ce que je pensais exact, je le pense toujours tel et n’ai, par conséquent, aucun motif d’y rien changer.\par
Aussi bien ce livre est la base de tout ce que j’ai pu faire et ferai par la suite. Je l’ai, en quelque sorte, commencé dès mon enfance. C’est l’expression des instincts apportés par moi en naissant. J’ai été avide, dès le premier jour où j’ai réfléchi, et j’ai réfléchi de bonne heure, de me rendre compte de ma propre nature, parce que fortement saisi par cette maxime : « Connais-toi toi-même », je n’ai pas estimé que je pusse me connaître, sans savoir ce qu’était le milieu dans lequel je venais vivre et qui, en partie, m’attirait à lui par la sympathie la plus passionnée et la plus tendre, en partie me dégoûtait et me remplissait de haine, de mépris et d’horreur. J’ai donc fait mon possible pour pénétrer de mon mieux dans l’analyse de ce qu’on appelle, d’une façon un peu plus générale qu’il ne faudrait, l’espèce humaine, et c’est cette étude qui m’a appris ce que je raconte ici.\par
Peu à peu est sortie, pour moi, de cette théorie, l’observation plus détaillée et plus minutieuse des lois que j’avais posées. J’ai comparé les races entre elles. J’en ai choisi une au milieu de ce que je voyais de meilleur et j’ai écrit l’Histoire des Perses,\emph{ pour montrer par l’exemple de la nation aryane la plus isolée de toutes ses congénères, combien sont impuissantes, pour changer ou brider le génie d’une race, les différences de climat, de voisinage et les circonstances des temps.}\par
 C’est après avoir mis fin à cette seconde partie de ma tâche que j’ai pu aborder les difficultés de la troisième, cause et but de mon intérêt J’ai fait l’histoire d’une famille, de ses facultés reçues dès soit origine, de ses aptitudes, de ses défauts, des fluctuations qui ont agi sur ses destinées, et j’ai écrit l’histoire d’Ottar Jarl, pirate norvégien, et de sa descendance, C’est ainsi qu’après avoir enlevé l’enveloppe verte, épineuse, épaisse de la noix, puis l’écorce ligneuse, j’ai mis à découvert le noyau. Le chemin que j’ai parcouru ne mène pas à un de ces promontoires escarpés où la terre s’arrête, mais bien à une de ces étroites prairies, où la route restant ouverte, l’individu hérite des résultats suprêmes de la race, de ses instincts bons ou mauvais, forts ou faibles, et se développe librement dans sa personnalité.\par
Aujourd’hui on aime les grandes unités, les vastes amas où les entités isolées disparaissent. C’est ce qu’on suppose être le produit de la science À chaque époque, celle-ci voudrait dévorer une vérité qui la gêne. Il ne faut pas s’en effrayer. Jupiter échappe toujours à la voracité de Saturne, et l’époux et le fils de Rhée, dieux, l’un comme l’autre, règnent, sans pouvoir s’entredétruire, sur la majesté de l’univers.
\chapterclose


\chapteropen
\chapter[{I. considérations préliminaires définitions, recherche et exposition des lois naturelles qui régissent le monde social.}]{I. \\
considérations préliminaires définitions, recherche et exposition des lois naturelles qui régissent le monde social.}\renewcommand{\leftmark}{I. \\
considérations préliminaires définitions, recherche et exposition des lois naturelles qui régissent le monde social.}


\chaptercont
\section[{I.1. La condition mortelle, des civilisations et des sociétés, résulte d’une cause générale et commune.}]{I.1. \\
La condition mortelle \\
des civilisations et des sociétés \\
résulte d’une cause générale et commune.}
\noindent La chute des civilisations est le plus frappant et en même temps le plus obscur de tous les phénomènes de l’histoire. En effrayant l’esprit, ce malheur réserve quelque chose de si mystérieux et de si grandiose, que le penseur ne se lasse pas de le considérer, de l’étudier, de tourner autour de son secret. Sans nul doute, la naissance et la formation des peuples proposent à l’examen des observations très remarquables : le développement successif des sociétés, leurs succès, leurs conquêtes, leurs triomphes, ont de quoi frapper bien vivement l’imagination et l’attacher ; mais tous ces faits, si grands qu’on les suppose, paraissent s’expliquer aisément ; on les accepte comme les simples consé­quences des dons intellectuels de l’homme ; une fois ces dons reconnus, on ne s’étonne pas de leurs résultats ; ils expliquent, par le fait seul de leur existence, les grandes choses dont ils sont la source. Ainsi, pas de difficultés, pas d’hésitations de ce côté. Mais quand, après un temps de force et de gloire, on s’aperçoit que toutes les sociétés humaines ont leur déclin et leur chute, toutes, dis-je, et non pas telle ou telle ; quand on remarque avec quelle taciturnité terrible le globe nous montre, épars sur sa surface, les débris des civilisations qui ont précédé la nôtre, et non seulement des civilisations connues, mais encore de plusieurs autres dont on ne sait que les noms, et de quelques-unes qui, gisant en squelettes de pierre au fond de forêts presque contemporaines du monde \footnote{M. A. de Humboldt, \emph{Examen critique de l’histoire de la géographie du nouveau continent.} Paris, in-8-.}, ne nous ont pas même transmis cette ombre de souvenir ; lorsque l’esprit, faisant un retour sur nos États modernes, se rend compte de leur jeunesse extrême, s’avoue qu’ils ont commencé d’hier et que certains d’entre eux sont déjà caducs : alors on reconnaît, non sans une certaine épouvante philosophique, avec combien de rigueur la parole des prophètes sur l’instabilité des choses s’applique aux civilisations comme aux peuples, aux peuples comme aux États, aux États comme aux individus, et l’on est contraint de constater que toute agglomération humaine, même protégée par la complication la plus ingénieuse de liens sociaux, contracte, au jour même où elle se forme, et caché parmi les éléments de sa vie, le principe d’une mort inévitable.\par
Mais quel est ce principe ? Est-il uniforme ainsi que le résultat qu’il amène, et toutes les civilisations périssent-elles par une cause identique ?\par
Au premier aspect, on est tenté de répondre négativement ; car on a vu tomber bien des empires, l’Assyrie, l’Égypte, la Grèce, Rome, dans des conflits de circonstances qui ne se ressemblaient pas. Toutefois, en creusant plus loin que l’écorce, on trouve bientôt, dans cette nécessité même de finir qui pèse impérieusement sur toutes les sociétés sans exception, l’existence irrécusable, bien que latente, d’une cause générale, et, partant de ce principe certain de mort naturelle indépendant de tous les cas de mort violente, on s’aperçoit que toutes les civilisations, après avoir duré quelque peu, accusent à l’observation des troubles intimes, difficiles à définir, mais non moins difficiles à nier, qui portent dans tous les lieux et dans tous les temps un caractère analogue ; enfin, en relevant une différence évidente entre la ruine des États et celle des civilisations, en voyant la même espèce de culture tantôt persister dans un pays sous une domination étrangère, braver les événements les plus calamiteux, et tantôt, au contraire, en présence de malheurs médiocres, disparaître ou se transformer, on s’arrête de plus en plus à cette idée, que le principe de mort, visible au fond de toutes les sociétés, est non seulement adhérent à leur vie, mais encore uniforme et le même pour toutes.\par
J’ai consacré les études dont je donne ici les résultats à l’examen de ce grand fait.\par
C’est nous modernes, nous les premiers, qui savons que toute agglomération d’hommes et le mode de culture intellectuelle qui en résulte doivent périt. Les époques précédentes ne le croyaient pas. Dans l’antiquité asiatique, l’esprit religieux, ému com­me d’une apparition anormale par le spectacle des grandes catastrophes politiques, les attribuait à la colère céleste frappant les péchés d’une nation ; c’était là, pensait-on, un châtiment propre à amener au repentir les coupables encore impunis. Les juifs, interprétant mal le sens de la Promesse, supposaient que leur empire ne finirait jamais. Rome, au moment même où elle commençait à sombrer, ne doutait pas de l’éternité du sien \footnote{Amédée Thierry, \emph{La Gaule sous l’administration romaine}, t. I, p. 244.}. Mais, pour avoir vu davantage, les générations actuelles savent beaucoup plus aussi ; et, de même que personne ne doute de la condition universellement mortelle des hommes, parce que tous les hommes qui nous ont précédés sont morts, de même nous croyons fermement que les peuples ont des jours comptés, bien que plus nombreux ; car aucun de ceux qui régnèrent avant nous ne poursuit à nos côtés sa carrière. Il y a donc, pour l’éclaircissement de notre sujet, peu de choses à prendre dans la sagesse antique, hormis une seule remarque fonda­mentale, la reconnaissance du doigt divin dans la conduite de ce monde, base solide et première dont il ne faut pas se départir, l’acceptant avec toute l’étendue que lui assigne l’Église catholique. Il est incontestable que nulle civilisation ne s’éteint sans que Dieu le veuille, et appliquer à la condition mortelle de toutes les sociétés l’axiome sacré dont les anciens sanctuaires se servaient pour expliquer quelques destructions remarquables, considérées par eux, mais à tort, comme des faits isolés, c’est proclamer une vérité de premier ordre, qui doit dominer la recherche des vérités terrestres. Ajouter que toutes les sociétés périssent parce qu’elles sont coupables, j’y consens aisément ; ce n’est encore qu’établir un juste parallélisme avec la condition des individus, en trouvant dans le péché le germe de la destruction. Sous ce rapport, rien ne s’oppose, à raisonner même suivant les simples lumières de l’esprit, à ce que les sociétés suivent le sort des êtres qui les composent, et, coupables par eux, finissent comme eux ; mais, ces deux vérités admises et pesées, je le répète, la sagesse antique ne nous offre aucun secours.\par
Elle ne nous dit rien de précis sur les voies que suit la volonté divine pour amener la mort des peuples ; elle est, au contraire, portée à considérer ces voies comme essentiellement mystérieuses. Saisie d’une pieuse terreur à l’aspect des ruines, elle admet trop aisément que les États qui s’écroulent ne peuvent être ainsi frappés, ébranlés, engloutis, si ce n’est à l’aide de prodiges. Qu’un fait miraculeux se soit produit dans certaines occurrences, en tant que les livres saints l’affirment, je me soumets sans peine à le croire ; mais là où les témoignages sacrés ne se prononcent pas d’une manière formelle, et c’est le plus grand nombre des cas, on peut légitimement considérer l’opinion des anciens temps comme incomplète, insuffisamment éclairée, et reconnaître, contrairement au côté où elle penche, que, puisque la sévérité céleste s’exerce sur nos sociétés constamment et par suite d’une décision antérieure à l’établissement du premier peuple, l’arrêt s’exécute d’une manière prévue, normale et en vertu de prescrip­tions définitivement inscrites au code de l’univers, à côté des autres lois qui, dans leur imperturbable régularité, gouvernent la nature animée tout comme le monde inorganique.\par
Si l’on est en droit de reprocher justement à la philosophie sacrée des premiers temps de s’être, dans son défaut d’expérience, bornée, pour expliquer un mystère, à l’exposition d’une vérité théologique indubitable, mais qui elle-même est un autre mystère, et de n’avoir pas poussé ses recherches jusqu’à l’observation des faits tombant sous le domaine de la raison, du moins ne peut-on pas l’accuser d’avoir méconnu la grandeur du problème en cherchant des solutions au ras de terre. Pour bien dire, elle s’est contentée de poser noblement la question, et, si elle ne l’a point résolue ni même éclaircie, du moins n’en a-t-elle pas fait un thème d’erreurs. C’est en cela qu’elle se place bien au-dessus des travaux fournis par les écoles rationalistes.\par
Les beaux esprits d’Athènes et de Rome ont établi cette doctrine acceptée jusqu’à nos jours, que les États, les peuples, les civilisations ne périssent que par le luxe, la mollesse, la mauvaise administration, la corruption des mœurs, le fanatisme. Toutes ces causes, soit réunies, soit isolées, furent déclarées responsables de la fin des sociétés ; et la conséquence nécessaire de cette opinion, c’est que là où elles n’agissent point, aucune force dissolvante ne doit exister non plus. Le résultat final, c’est d’établir que les sociétés ne meurent que de mort violente, plus heureuses en cela que les hommes, et que, sauf à éluder les causes de destruction que je viens d’énumérer, on peut parfaitement se figurer une nationalité aussi durable que le globe lui-même. En inventant cette thèse, les anciens n’en apercevaient nullement la portée ; ils n’y voyaient autre chose qu’un moyen d’étayer la doctrine morale, seul but, comme on sait, de leur système historique. Dans les récits des événements, ils se préoccupaient si fort de relever avant tout l’influence heureuse de la vertu, les déplorables effets du crime et du vice, que tout ce qui sortait de ce cadre moral leur important médiocrement, restait le plus souvent inaperçu ou négligé. Cette méthode était fausse, mesquine, et trop souvent même marchait contre l’intention de ses auteurs, car elle appliquait, suivant les besoins du moment, le nom de vertu et de vice d’une façon arbitraire ; mais, jusqu’à un certain point, le sévère et louable sentiment qui en faisait la base lui sert d’excuse, et, si le génie de Plutarque et celui de Tacite n’ont tiré de cette théorie que des romans et des libelles, ce sont de sublimes romans et des libelles généreux.\par
Je voudrais pouvoir me montrer aussi indulgent pour l’application qu’en ont faite les auteurs du dix-huitième siècle ; mais il y a entre leurs maîtres et eux une trop grande différence : les premiers étaient dévoués jusqu’à l’exagération au maintien de l’établissement social ; les seconds furent avides de nouveautés et acharnés à détruire : les uns s’efforçaient de faire fructifier noblement leur mensonge ; les autres en ont tiré d’épouvantables conséquences, en y sachant trouver des armes contre tous les principes de gouvernement, auxquels tour à tour venait s’appliquer le reproche de tyrannie, de fanatisme, de corruption. Pour empêcher les sociétés de périr, la façon voltairienne consiste à détruire la religion, la loi, l’industrie, le commerce, sous prétexte que la religion, c’est le fanatisme ; la loi, le despotisme ; l’industrie et le commerce, le luxe et la corruption. À coup sûr, le règne de tant d’abus, c’est le mauvais gouvernement.\par
Mon but n’est pas le moins du monde d’entamer une polémique ; je n’ai voulu que faire remarquer combien l’idée commune à Thucydide et à l’abbé Raynal produit des résultats divergents ; pour être conservatrice chez l’un, cyniquement agressive chez l’autre, c’est partout une erreur. Il n’est pas vrai que les causes auxquelles sont buées les chutes des nations en soient nécessairement coupables, et, tout en reconnaissant volontiers qu’elles peuvent se faire voir au moment de la mort d’un peuple, je nie qu’elles aient assez de force, qu’elles soient douées d’une énergie assez sûrement destructive pour déterminer à elles seules la catastrophe irrémédiable.
\section[{I.2. Le fanatisme, le luxe, les mauvaises mœurs et l’irréligion n’amènent pas nécessairement la chute des sociétés.}]{I.2. \\
Le fanatisme, le luxe, les mauvaises mœurs et l’irréligion n’amènent pas nécessairement la chute des sociétés.}
\noindent Il est nécessaire de bien expliquer d’abord ce que j’entends par une société. Ce n’est pas le cercle plus ou moins étendu dans lequel s’exerce, sous une forme ou sous une autre, une souveraineté distincte. La république d’Athènes n’est pas une société, non plus que le royaume de Magadha, l’empire du Pont ou le califat d’Égypte au temps des Fatimites. Ce sont des fragments de société qui se transforment sans doute, se rapprochent ou se subdivisent sous la pression des lois naturelles que je cherche, mais dont l’existence ou la mort ne constitue pas l’existence ou la mort d’une société. Leur formation n’est qu’un phénomène le plus souvent transitoire, et qui n’a qu’une action bornée ou même indirecte sur la civilisation au milieu de laquelle elle éclôt. Ce que j’entends par société, c’est une réunion, plus ou moins parfaite au point de vue politique, mais complète au point de vue social, d’hommes vivant sous la direction d’idées semblables et avec des instincts identiques. Ainsi l’Égypte, l’Assyrie, la Grèce, l’Inde, la Chine, ont été ou sont encore le théâtre où des sociétés distinctes ont déroulé leurs destinées, abstraction faite des perturbations survenues dans leurs constitutions politiques. Comme je ne parlerai des fractions que lorsque mon raisonnement pourra s’appliquer à l’ensemble, j’emploierai le mot de \emph{nation} ou celui de \emph{peuple} dans le sens général ou restreint, sans que nulle amphibologie puisse en résulter. Cette définition faite, je reviens à l’examen de la question, et je vais démontrer que le fanatisme, le luxe, les mauvaises mœurs et l’irréligion ne sont pas des instruments de mort certaine pour les peuples.\par
Tous ces faits se sont rencontrés, quelquefois isolément, quelquefois simulta­nément et avec une très grande intensité, chez des nations qui ne s’en portaient que mieux, ou qui, tout au moins, n’en allaient pas plus mal.\par
C’était pour la plus grande gloire du fanatisme que l’empire américain des Aztèques semblait surtout exister. Je n’imagine rien de plus fanatique qu’un état social qui, comme celui-là, reposait sur une base religieuse, incessamment arrosé du sang des boucheries humaines \footnote{Prescott, \emph{History of the conquest of Mejico}. In-8°, Paris, 1844.}. On a nié récemment \footnote{C. F. Weber, \emph{M. A. Lucani Pharsalia}. In-8°. Leipzig, 1828, t. I, p. 122-123, note.}, et peut-être avec quelque apparence de raison, que les anciens peuples européens aient jamais pratiqué le meurtre religieux sur des victimes considérées comme innocentes, les prisonniers de guerre ou les naufragés n’étant pas compris dans cette catégorie ; mais, pour les Mexicains, toutes victimes leur étaient bonnes. Avec cette férocité qu’un physiologiste moderne reconnaît être le caractère général des races du nouveau monde \footnote{Prichard, \emph{Histoire naturelle de l’homme} (trad. de M. Roulin. In-8°. Paris, 1843). – Le D\textsuperscript{r} Martius est encore plus explicite. Voir \emph{Martius und Spix, Reise in Brasilien.} In-4°. Munich, t. I, p. 379-380.}, ils massacraient impitoyablement sur leurs autels des concitoyens, et sans hésitation comme sans choix, ce qui ne les empêchait pas d’être un peuple puissant, industrieux, riche, et qui certainement aurait encore longtemps duré, régné, égorgé, si le génie de Fernand Cortez et le courage de ses compagnons n’étaient venus mettre fin à la monstrueuse existence d’un tel empire. Le fanatisme ne fait donc pas mourir les États.\par
Le luxe et la mollesse ne sont pas des coupables plus avérés ; leurs effets se font sentir dans les hautes classes, et je doute que chez les Grecs, chez les Perses, chez les Romains, la mollesse et le luxe, pour avoir d’autres formes, aient eu plus d’intensité qu’on ne leur en voit aujourd’hui en France, en Allemagne, en Angleterre, en Russie, en Russie surtout et chez nos voisins d’outre-Manche ; et précisément ces deux derniers pays semblent doués d’une vitalité toute particulière parmi les États de l’Europe moderne. Et au moyen âge, les Vénitiens, les Génois, les Pisans, pour accumuler dans leurs magasins, étaler dans leurs Palais, promener dans leurs vaisseaux, sur toutes les mers, les trésors du monde entier, n’en étaient certainement pas plus faibles. La mollesse et le luxe ne sont donc pas pour un peuple des causes nécessaires d’affaiblis­sement et de mort.\par
La corruption des mœurs elle-même, le plus horrible des fléaux, ne joue pas inévitablement un rôle destructeur. Il faudrait, pour que cela fût, que la prospérité d’une nation, sa puissance et sa prépondérance se montrassent développées en raison directe de la pureté de ses coutumes ; et c’est ce qui n’est pas. On est assez généralement revenu de la fantaisie si bizarre qui attribuait tant de vertus aux premiers Romains \footnote{Balzac, \emph{Lettre à madame la duchesse de Montausier.}}. On ne voit rien de bien édifiant, et on a raison, dans ces patriciens de l’ancienne roche qui traitaient leurs femmes en esclaves, leurs enfants comme du bétail, et leurs créanciers comme des bêtes fauves ; et, s’il restait à une si mauvaise cause des défenseurs qui voulussent arguer d’une prétendue variation dans le niveau moral aux diverses époques, il ne serait pas bien difficile de repousser l’argument et d’en démontrer le peu de solidité. Dans tous les temps, l’abus de la force a excité une indignation égale ; si les rois ne furent pas chassés pour le viol de Lucrèce, si le tribunat ne fut pas établi pour l’attentat d’Appius, du moins les causes plus profondes de ces deux grandes révolutions, en s’armant de tels prétextes, témoignaient assez des dispositions contemporaines de la morale publique. Non, ce n’est pas dans la vertu plus grande qu’il faut chercher la cause de la vigueur des premiers temps chez tous les peuples ; depuis le commencement des époques historiques, il n’est pas d’agrégation humaine, fût-elle aussi petite qu’on voudra se la figurer, chez qui toutes les tendances répréhensibles ne se soient trahies ; et cependant, ployant sous cet odieux bagage, les États ne s’en maintiennent pas moins, et souvent, au contraire, semblent redevables de leur splendeur à d’abominables institutions. Les Spartiates n’ont vécu et gagné l’admiration que par les effets d’une législation de bandits. Les Phéniciens ont-ils dû leur perte à la corruption qui les rongeait et qu’ils allaient semant partout ? Non ; tout au contraire, c’est cette corruption qui a été l’instrument principal de leur puissance et de leur gloire ; depuis le jour où, sur les rivages des îles grecques \footnote{\emph{Odyssée}, XV.}, ils allaient, trafiquants fripons, hôtes scélérats, séduisant les femmes pour en faire marchandise, et volant çà et là les denrées qu’ils couraient vendre, leur réputation fut, à coup sûr, bien et justement flétrissante ; ils n’en ont pas moins grandi et tenu dans les annales du monde un rang dont leur rapacité et leur mauvaise foi n’ont nullement contribué à les faire descendre.\par
Loin de découvrir dans les sociétés jeunes une supériorité de morale, je ne doute pas que les nations en vieillissant, et par conséquent en approchant de leur chute, ne présentent aux yeux du censeur un état beaucoup plus satisfaisant. Les usages s’adoucissent, les hommes s’accordent davantage, chacun trouve à vivre plus aisément, les droits réciproques ont eu le temps de se mieux définir et comprendre ; si bien que les théories sur le juste et l’injuste ont acquis peu à peu un plus haut degré de délicatesse. Il serait difficile de démontrer qu’au temps où les Grecs ont jeté bas l’empire de Darius, comme à l’époque où les Goths sont entrés dans Rome, il n’y avait pas à Athènes, à Babylone et dans la grande ville impériale beaucoup plus d’honnêtes gens qu’aux jours glorieux d’Harmodius, de Cyrus le Grand et de Publicola.\par
Sans remonter à ces époques éloignées, nous pouvons en juger par nous-mêmes. Un des points du globe où le siècle est le plus avancé, et présente un plus parfait contraste avec l’âge naïf, c’est bien certainement Paris ; et cependant grand nombre de personnes religieuses et savantes avouent que dans aucun lieu, dans aucun temps, on ne trouverait autant de vertus efficaces, de solide piété, de douce régularité, de finesse de conscience, qu’il s’en rencontre aujourd’hui dans cette grande ville. L’idéal que l’on s’y fait du bien est tout aussi élevé qu’il pouvait l’être dans l’âme des plus illustres modèles du dix-septième siècle, et encore a-t-il dépouillé cette amertume, cette sorte de roideur et de sauvagerie, oserais-je dire cette pédanterie, dont alors il n’était pas toujours exempt ; de sorte que, pour contre-balancer les épouvantables écarts de l’esprit moderne, on trouve, sur les lieux mêmes où cet esprit a établi le principal siège de sa puissance, des contrastes frappants, dont les siècles passés n’ont pas eu, à un aussi haut degré que nous, le consolant spectacle.\par
Je ne vois pas même que les grands hommes manquent aux périodes de corruption et de décadence, je dis les grands hommes les mieux caractérisés par l’énergie du caractère et les fortes vertus. Si je cherche dans le catalogue des empereurs romains, la plupart d’ailleurs supérieurs à leurs sujets par le mérite comme par le rang, je relève des noms comme ceux de Trajan, d’Antonin le Pieux, de Septime Sévère, de Jovien ; et au-dessous du trône, dans la foule même, j’admire tous les grands docteurs, les grands martyrs, les apôtres de la primitive Église, sans compter les vertueux païens. J’ajoute que les esprits actifs, fermes, valeureux, remplissaient les camps et les municipes de façon à faire douter qu’au temps de Cincinnatus, et proportion gardée, Rome ait possé­dé autant d’hommes éminents dans tous les genres d’activité. L’examen des faits est complètement concluant.\par
Ainsi, gens de vertu, gens d’énergie, gens de talent, loin de faire défaut aux périodes de décadence et de vieillesse des sociétés, s’y rencontrent au contraire avec plus d’abondance peut-être qu’au sein des empires qui viennent de naître, et, en outre, le niveau commun de la moralité y est supérieur. Il n’est donc pas généralement vrai de prétendre que, dans les États qui tombent, la corruption des mœurs soit plus intense que dans ceux qui naissent ; que cette même corruption détruise les peuples est également sujet à contestation, puisque certains États, loin de mourir de leur perversité, en ont vécu ; mais on peut aller même au delà, et démontrer que l’abaissement moral n’est pas nécessairement mortel, car, parmi les maladies qui affectent les sociétés, il a cet avantage de pouvoir se guérir, et quelquefois assez vite.\par
En effet, les mœurs particulières d’un peuple présentent de très fréquentes ondulations suivant les périodes que l’histoire de ce peuple traverse. Pour ne s’adresser qu’à nous, Français, constatons que les Gallo-Romains des cinquième et sixième siècles, race soumise, valaient certainement mieux que leurs héroïques vainqueurs, à tous les points de vue que la morale embrasse ; ils n’étaient même pas toujours, individuelle­ment pris, leurs inférieurs en courage et en vertu militaire \footnote{Augustin Thierry, \emph{Récits des temps mérovingiens.} Voir, entre autres, l’histoire de Mummolus.}. Il semblerait que, dans les âges qui suivirent, lorsque les deux races eurent commencé à se mêler, tout s’empira, et que, vers le huitième et le neuvième siècle, le territoire national ne présentait pas un tableau dont nous ayons à tirer grande vanité. Mais aux onzième, douzième et treizième siècles, le spectacle s’était totalement transformé, et, tandis que la société avait réussi à amalgamer ses éléments les plus discords, l’état des mœurs était générale­ment digne de respect ; il n’y avait pas, dans les notions de ce temps, de ces ambages qui éloignent du bien celui qui veut y parvenir. Le quatorzième et le quinzième siècles furent de déplorables moments de perversité et de conflits ; le brigandage prédomina ; ce fut de mille façons, et dans le sens le plus étendu et le plus rigoureux du mot, une période de décadence ; on eût dit qu’en face des débauches, des massacres, des tyrannies, de l’affaiblissement complet de tout sentiment honnête dans les nobles qui volaient leurs vilains, dans les bourgeois qui vendaient la patrie à l’Angleterre, dans un clergé sans régularité, dans tous les ordres enfin, la société entière allait s’écrouler, et sous ses ruines engloutir et cacher tant de hontes. La société ne s’écroula pas, elle continua de vivre, elle s’ingénia, elle combattit, elle sortit de peine. Le seizième siècle, malgré ses folies sanglantes, conséquences adoucies de l’âge précédent, fut beaucoup plus honorable que son prédécesseur ; et, pour l’humanité, la Saint-Barthélemy n’est pas ignominieuse comme le massacre des Armagnacs. Enfin, de ce temps à demi corrigé, la société française passa aux lumières vives et pures de l’âge des Fénelon, des Bossuet et des Montausier. Ainsi, jusqu’à Louis XIV, notre histoire présente des successions rapides du bien au mal, et la vitalité propre à la nation reste en dehors de l’état de ses mœurs. J’ai tracé en courant les plus grandes différences ; celles de détail abondent ; il faudrait bien des pages pour les relever ; mais, à ne parler que de ce que nous avons presque vu de nos yeux, ne sait on pas que tous les dix ans, depuis 1787, le niveau de la moralité a énormément varié ? Je conclus que, la corruption des mœurs étant, en défini­tive, un fait transitoire et flottant, qui tantôt s’empire et tantôt s’améliore, on ne saurait la considérer comme une cause nécessaire et déterminante de ruine pour les États.\par
Ici je me trouve amené à examiner un argument d’espèce contemporaine qu’il n’entrait pas dans les idées du dix-huitième siècle de faire valoir ; mais, comme il s’enchaîne à merveille avec la décadence des mœurs, je ne crois pas pouvoir en parler plus à propos. Plusieurs personnes sont portées à penser que la fin d’une société est imminente quand les idées religieuses tendent à s’affaiblir et à disparaître. On observe une sorte de corrélation à Athènes et à Rome entre la profession publique des doctrines de Zénon et d’Épicure, l’abandon des cultes nationaux qui s’en est suivi, dit-on, et la fin des deux républiques. On néglige d’ailleurs de remarquer que ces deux exemples sont à peu près les seuls que l’on puisse citer d’un pareil synchronisme ; que l’empire des Perses était fort dévot au culte des mages lorsqu’il est tombé ; que Tyr, Carthage, la Judée, les monarchies aztèque et péruvienne ont été frappées de mort en embrassant leurs autels avec beaucoup d’amour, et que par conséquent il est impossible de prétendre que tous les peuples qui voient se détruire leur nationalité expient par ce fait un abandon du culte de leurs pères. Mais ce n’est pas tout : dans les deux seuls exemples que l’on me paraisse fondé à invoquer, le fait que l’on relève a beaucoup plus d’apparence que de fond, et je nie tout à fait qu’à Rome comme à Athènes, le culte antique ait jamais été délaissé, jusqu’au jour où il fut remplacé dans toutes les consciences par le triomphe complet du christianisme ; en d’autres termes, je crois qu’en matière de foi religieuse, il n’y a jamais eu chez aucun peuple du monde une véritable solution de continuité ; que, lorsque la forme ou la nature intime de la croyance a changé, le Teutatès gaulois a saisi le Jupiter romain, et le Jupiter le christianisme, absolument comme, en droit, le mort saisit le vif, sans transition d’incré­dulité ; et dès lors, s’il ne s’est jamais trouvé une nation dont on fût en droit de dire qu’elle était sans foi, on est mal fondé à mettre en avant que le manque de foi détruit les États.\par
Je vois bien sur quoi le raisonnement s’appuie. On dira que c’est un fait notoire qu’un peu avant le temps de Périclès, à Athènes, et chez les Romains vers l’époque des Scipions, l’usage se répandit, dans les classes élevées, de raisonner sur les choses religieuses d’abord, puis d’en douter, puis décidément de n’y plus croire et de tirer vanité de l’athéisme. De proche en proche, cette habitude gagna, et il ne resta plus, ajoute-t-on, personne, ayant quelques prétentions à un jugement sain, qui ne défiât les augures de s’entre-regarder sans rire.\par
Cette opinion, dans un peu de vrai, mêle aussi beaucoup de faux. Qu’Aspasie, à la fin de ses petits soupers, et Lélius, auprès de ses amis, se fissent gloire de bafouer les dogmes sacrés de leur pays, il n’y a, à le soutenir, rien que de très exact ; mais pourtant, à ces deux époques, les plus brillantes de l’histoire de la Grèce et de Rome, on ne se serait pas permis de professer trop publiquement de pareilles idées. Les imprudences de sa maîtresse faillirent coûter cher à Périclès lui-même ; on se souvient des larmes qu’il versa en plein tribunal, et qui, seules, n’auraient pas réussi à faire absoudre la belle incrédule. On n’a pas oublié non plus le langage officiel des poètes du temps, et comme Aristophane avec Sophocle, après Eschyle, s’établissait le vengeur impitoyable des divinités outragées. C’est que la nation tout entière croyait à ses dieux, regardait Socrate comme un novateur coupable, et voulait voir juger et con­damner Anaxagore. Mais, plus tard ?... Plus tard les théories philosophiques et impies réussirent-elles à pénétrer dans les masses populaires ? Jamais, dans aucun temps, à aucun jour, elles n’y parvin­rent. Le scepticisme resta une habitude des gens élégants, et ne dépassa pas leur sphère. On va objecter qu’il est bien inutile de parler de ce que pensaient des petits bourgeois, des populations villageoises, des esclaves, tous sans influence dans la conduite de l’État, et dont les idées n’avaient pas d’action sur la politique. La preuve qu’elles en avaient, c’est que, jusqu’au dernier soupir du paganisme, il fallut leur conserver leurs temples et leurs chapelles ; il fallut payer leurs hiérophantes ; il fallut que les hommes les plus éminents, les plus éclairés, les plus fermes dans la négation religieuse, non seulement s’honorassent publiquement de porter la robe sacerdotale, mais remplissent eux-mêmes, eux, accoutumés à tourner les feuillets du livre de Lucrèce, \emph{manu diurna, manu nocturna}, les emplois les plus répugnants du culte, et non seulement s’en acquittassent aux jours de cérémonie, mais encore employassent leurs rares loisirs, des loisirs disputés péniblement aux plus terribles jeux de la politique, à écrire des traités d’aruspicine. Je parle ici du grand Jules \footnote{ \noindent César, démocrate et sceptique, savait mettre son langage en désaccord avec ses opinions lorsque la circonstance le requérait. Rien de curieux comme l’oraison funèbre qu’il prononça pour sa tante : « L’origine maternelle de ma tante Julia, dit-il, remonte aux rois ; la paternelle se rattache aux dieux immortels ; car les rois Marciens, dont fut le nom de sa mère, étaient issus d’Ancus Marcius, et c’est de Vénus que viennent les Jules, race à laquelle appartient notre famille. Ainsi, dans ce sang, il y avait tout à la fois la sainteté des rois, les plus puissants des hommes, et l’adorable majesté (\emph{cerimonia}) des dieux, qui tiennent les rois eux-mêmes en leur pouvoir. » (Suétone, \emph{Julius}, 5.)\par
 On n’est pas plus monarchique ; mais aussi, pour un athée, on n’est pas plus religieux.
}. Eh quoi ! tous les empereurs après lui furent et durent être des souverains pontifes, Constantin encore ; et, tandis qu’il avait des raisons bien plus fortes que tous ses prédécesseurs pour repousser une charge si odieuse à son honneur de prince chrétien, il dut, contraint par l’opinion publique, évidemment bien puissante, quoiqu’à la veille de s’éteindre, il dut compter encore avec l’antique religion nationale. Ainsi, ce n’était pas la foi des petits bourgeois, des populations villageoises, des esclaves, qui était peu de chose, c’était l’opinion des gens éclairés. Cette dernière avait beau s’insurger, au nom de la raison et du bon sens, contre les absurdités du paganisme ; les masses populaires ne voulaient pas, ne pouvaient pas renoncer à une croyance avant qu’on leur en eût fourni une autre, donnant là une grande démonstration de cette vérité, que c’est le positif et non le négatif qui est d’emploi dans les affaires de ce monde ; et la pression de ce sentiment général fut si forte qu’au troisième siècle il y eut, dans les hautes classes, une réaction religieuse, réaction solide, sérieuse, et qui dura jusqu’au passage définitif du monde aux bras de l’Église ; de sorte que le règne du philosophisme aurait atteint son apogée sous les Antonins, et commencé son déclin peu après leur mort. Mais ce n’est pas le lieu de débattre cette question, d’ailleurs intéressante pour l’histoire des idées ; qu’il me suffise d’établir que la rénovation gagna de plus en plus, et d’en faire ressortir la cause la plus apparente.\par
Plus le monde romain alla vieillissant, plus le rôle des armées fut considérable. Depuis l’empereur, qui sortait inévitablement des rangs de la milice, jusqu’au dernier officier de son prétoire, jusqu’au plus mince gouverneur de district, tous les fonction­naires avaient commencé par tourner sous le cep du centurion. Tous sortaient donc de ces masses populaires dont j’ai déjà signalé l’indomptable piété, et, en arrivant aux splendeurs d’un rang élevé, trouvaient pour leur déplaire, les choquer, les blesser, l’antique éclat des classes municipales, de ces sénateurs des villes, qui les regardaient volontiers comme des parvenus, et les auraient raillés de grand cœur, n’eût été la crainte. Il y avait ainsi hostilité entre les maîtres réels de l’État et les familles jadis supérieures. Les chefs de l’armée étaient croyants et fanatiques, témoin Maximin, Galère, cent autres ; les sénateurs et les décurions faisaient encore leurs délices de la littérature sceptique ; mais comme on vivait, en définitive, à la cour, donc parmi les militaires, on était contraint d’adopter un langage et des opinions officielles qui ne fussent pas dangereuses. Tout devint, peu à peu, dévot dans l’empire, et ce fut par dévotion que les philosophes eux-mêmes, conduits par Évhémère, se mirent à inventer des systèmes pour concilier les théories rationalistes avec le culte de l’État, méthode dont l’empereur Julien fut le plus puissant coryphée. Il n’y a pas lieu de louer beaucoup cette renaissance de la piété païenne, puisqu’elle causa la plupart des persécutions qui ont atteint nos martyrs. Les populations, offensées dans leur culte par les sectes athées, avaient patienté aussi longtemps que les hautes classes les avaient dominées ; mais, aussitôt que la démocratie impériale eut réduit ces mêmes classes au rôle le plus humble, les gens d’en bas se voulurent venger d’elles, et, se trompant de victimes, égorgèrent les chrétiens, qu’ils appelaient impies et prenaient pour des philosophes. Quelle différence entre les époques ! Le païen vraiment sceptique, c’est ce roi Agrippa qui, par curiosité, veut entendre saint Paul \footnote{\emph{Act. Apost}. XXVI, 24, 28, 31}. Il l’écoute, discute avec lui, le tient pour un fou, mais ne songe pas à le punir de penser autrement qu’il ne fait lui-même. C’est l’historien Tacite, plein de mépris pour les nouveaux religionnaires, mais blâmant Néron de ses cruautés envers eux ; Agrippa et Tacite étaient des incrédules. Dioclétien était un politique conduit par les clameurs des gouvernés ; Décius, Aurélien étaient des fanatiques comme leurs peuples.\par
Et combien de peine n’éprouva-t-on pas encore, lorsque le gouvernement romain eut définitivement embrassé la cause du christianisme, à conduire les populations dans le giron de la foi ! En Grèce, de terribles résistances éclatèrent, aussi bien dans la chaire des écoles que dans les bourgs et les villages et partout les évêques éprouvèrent tant de difficultés à triompher des petites divinités topiques, que, sur bien des points, la victoire fut moins l’œuvre de la conversion et de la persuasion que de l’adresse, de la patience et du temps. Le génie des hommes apostoliques, réduit à user de fraudes pieuses, substitua aux divinités des bois, des prés, des fontaines, les saints, les martyrs et les vierges. Ainsi les hommages continuèrent, pendant quelque temps s’adressèrent mal, et finirent par trouver la bonne voie. Que dis-je ? Est-ce vraiment certain ? Est-il avéré que, sur quelques points de la France même, il ne se trouve pas telle paroisse où quelques superstitions aussi tenaces que bizarres, n’inquiètent pas encore la sollicitude des curés ? Dans la catholique Bretagne, au siècle dernier, un évêque luttait contre des populations obstinées dans le culte d’une idole de pierre. En vain on jetait à l’eau le grossier simulacre, ses adorateurs entêtés savaient l’en retirer, et il fallut l’intervention d’une compagnie d’infanterie pour le mettre en pièces. Voilà quelle fut et quelle est la longévité du paganisme. Je conclus qu’on est mal fonde à soutenir que Rome et Athènes se soient trouvées un seul jour sans religion.\par
Puisque donc il n’est jamais arrivé, ni dans les temps anciens, ni dans les temps modernes, qu’une nation abandonnât son culte avant d’être bien et dûment pourvue d’un autre, il est impossible de prétendre que la ruine des peuples soit la conséquence de leur irréligion.\par
Après avoir refusé une puissance nécessairement destructive au fanatisme, au luxe, à la corruption des mœurs, et la réalité politique à l’irréligion, il me reste à traiter de l’influence d’un mauvais gouvernement ; ce sujet vaut bien qu’on lui ouvre un chapitre à part.
\section[{I.3. Le mérite relatif des gouvernements n’a pas d’influence sur la longévité des peuples.}]{I.3. \\
Le mérite relatif des gouvernements n’a pas d’influence sur la longévité des peuples.}
\noindent Je comprends quelle difficulté je soulève. Oser seulement l’aborder semblera à beaucoup de lecteurs une sorte de paradoxe. On est convaincu, et l’on fait très bien de l’être, que les bonnes lois, la bonne administration, influent d’une manière directe et puissante sur la santé d’une nation ; mais on l’est si fort, que l’on attribue à ces lois, à cette administration, le fait même de la durée d’une agrégation sociale, et c’est ici qu’on a tort.\par
On aurait raison, sans doute, si les peuples ne pouvaient vivre que dans l’état de bien-être ; mais nous savons bien qu’ils subsistent pendant longtemps, tout comme l’individu, en portant dans leurs flancs des affections désorganisatrices, dont les ravages éclatent souvent avec force au dehors. Si les nations devaient toujours mourir de leurs maladies, il n’en est pas qui dépasseraient les premières années de formation ; car c’est précisément alors que l’on peut leur trouver la pire administration, les plus mauvaises lois et le plus mal observées ; mais elles ont précisément ce point de dissemblance avec l’organisme humain, que, tandis que celui-ci redoute, surtout dans l’enfance, une série de fléaux à l’attaque desquels on sait d’avance qu’il ne résisterait pas, la société ne reconnaît pas de tels maux, et des preuves surabondantes sont fournies par l’histoire, qu’elle échappe sans cesse aux plus redoutables, aux plus longues, aux plus dévasta­trices invasions des souffrances politiques, dont les lois mal conçues et l’administration oppressive ou négligente sont les extrêmes \footnote{On comprend assez qu’il ne s’agit pas ici de l’existence politique d’un centre de souveraineté, mais de la vie d’une société entière, de la perpétuité d’une civilisation. C’est ici le lieu d’appliquer la distinction indiquée plus haut.}.\par
Essayons d’abord de préciser ce que c’est qu’un mauvais gouvernement.\par
Les variétés de ce mal paraissent assez nombreuses ; il serait même impossible de les compter toutes ; elles se multiplient à l’infini suivant la constitution des peuples, les lieux, les temps. Toutefois, en les groupant sous quatre catégories principales, peu de variétés échapperont.\par
Un gouvernement est mauvais lorsqu’il est imposé par l’influence étrangère. Athènes a connu ce gouvernement sous les Trente Tyrans ; elle s’en est débarrassée, et l’esprit national, loin de mourir chez elle dans le cours de cette oppression, ne fit que s’y retremper.\par
Un gouvernement est mauvais lorsque la conquête pure et simple en est la base. La France, au quatorzième siècle, a, dans sa presque totalité, subi le joug de l’Angleterre. Elle en est sortie plus forte et plus brillante. La Chine a été couverte et prise par les hordes mongoles ; elle a fini par les rejeter hors de ses limites, après leur avoir fait subir un singulier travail d’énervement. Depuis cette époque, elle est retombée sous un autre joug ; mais, bien que les Mantchoux comptent déjà un règne plus que séculaire, ils sont à la veille d’éprouver le même sort que les Mongols, après avoir passé par une semblable préparation affaiblissante.\par
Un gouvernement est surtout mauvais lorsque le principe dont il est sorti, se laissant vicier, cesse d’être sain et vigoureux comme il était d’abord. Ce fut le sort de la monarchie espagnole. Fondée sur l’esprit militaire et la liberté communale, elle commença à s’abaisser, vers la fin du règne de Philippe II, par l’oubli de ses origines. Il est impossible d’imaginer un pays où les bonnes maximes fussent plus tombées en oubli, où le pouvoir parût plus faible et plus déconsidéré, où l’organisation religieuse elle-même donnât plus de prise à la critique. L’agriculture et l’industrie, frappées comme tout le reste, étaient quasi ensevelies dans le marasme national. L’Espagne est-elle morte ? Non. Ce pays, dont plusieurs désespéraient, a donné à l’Europe l’exemple glorieux d’une résistance obstinée à la fortune de nos armes, et c’est peut-être celui de tous les États modernes dont la nationalité se montre en ce moment la plus vivace.\par
Un gouvernement est encore bien mauvais lorsque, par la nature de ses institutions, il autorise un antagonisme, soit entre le pouvoir suprême et la masse de la nation, soit entre les différentes classes. Ainsi l’on a vu, au moyen âge, des rois d’Angleterre et de France aux prises avec leurs grands vassaux, les paysans en lutte avec leurs seigneurs ; ainsi, en Allemagne, les premiers effets de la liberté de penser ont amené les guerres civiles des hussites, des anabaptistes et de tant d’autres sectaires ; et, à une époque un peu plus éloignée, l’Italie souffrit tellement par le partage d’une autorité tiraillée entre l’empereur, le pape, les nobles et les communes, que les masses, ne sachant à qui obéir, finirent souvent par ne plus obéir à personne. La société italienne est-elle morte alors ? Non. Sa civilisation ne fut jamais plus brillante, son industrie plus productive, son influence au dehors plus incontestée.\par
Et je veux bien croire que parfois, au milieu de ces orages, un pouvoir sage et régulier, semblable à un rayon de soleil, se fit jour quelque temps pour le plus grand bien des peuples ; mais c’était une fortune courte, et, de même que la situation contraire ne donnait pas la mort, l’exception, pas davantage, ne donnait la vie. Pour parvenir à un tel résultat, il s’en manqua de tout que les époques prospères aient été fréquentes et de durée assez longue. Et si les règnes judicieux furent alors clairsemés, il en fut en tout temps de même. Pour les meilleurs même, que de contestations et que d’ombres aux plus heureux tableaux ! Tous les auteurs regardent-ils également le temps du roi Guillaume d’Orange comme une ère de prospérité pour l’Angleterre ? Tous admirent-ils Louis XIV, le Grand, sans nulle réserve ? Au contraire. Les détracteurs ne manquent pas, et les reproches savent où se prendre ; c’est cependant, à peu près, ce que nos voisins et nous avons, soit de mieux ordonné, soit de plus fécond, dans le passé. Les bons gouvernements se distribuent d’une manière si parcimonieuse au milieu du cours des temps, et, lorsqu’ils se produisent, sont tellement contestables encore ; cette science de la politique, la plus haute, la plus épineuse de toutes, est si dispropor­tionnée à la faiblesse de l’homme, qu’on ne peut pas prétendre, en bonne foi, que, pour être mal conduits, les peuples périssent. Grâce au ciel, ils ont de quoi s’habituer de bonne heure à ce mal, qui, même dans sa plus grande intensité, est préférable, de mille façons, à l’anarchie ; et C’est un fait avéré, et que la plus mince étude de l’histoire suffira à démontrer, que le gouvernement, si mauvais soit-il, entre les mains duquel un peuple expire, est souvent meilleur que telle des administrations qui le précédèrent.
\section[{I.4. De ce qu’on doit entendre par le mot dégénération ; du mélange des principes ethniques, et comment les sociétés se forment et se défont.}]{I.4. \\
De ce qu’on doit entendre par le mot dégénération ; du mélange des principes ethniques, et comment les sociétés se forment et se défont.}
\noindent Pour peu que l’esprit des pages précédentes ait été compris, on n’en aura pas conclu que je ne donnais aucune importance aux maladies du corps social, et que le mauvais gouvernement, le fanatisme, l’irréligion, ne constituaient, à mes yeux, que des accidents sans portée. Ma pensée est certainement tout autre. Je reconnais, avec l’opinion générale, qu’il y a bien lieu de gémir lorsque la société souffre du développement de ces tristes fléaux, et que tous les soins, toutes les peines, tous les efforts que l’on peut appliquer à y porter remède, ne sauraient être perdus ; ce que j’affirme seulement, c’est que si ces malheureux éléments de désorganisation ne sont pas entés sur un principe destructeur plus vigoureux, s’ils ne sont pas les conséquences d’un mal caché plus terrible, on peut rester assuré que leurs coups ne seront pas mortels, et qu’après une période de souffrance plus ou moins longue, la ,société sortira de leurs filets peut-être rajeunie, peut-être plus forte.\par
Les exemples allégués me semblent concluants ; on pourrait en grossir le nombre à l’infini ; et c’est pour cette raison sans doute que le sentiment commun a fini par sentir l’instinct de la vérité. Il a entrevu qu’en définitive il ne fallait pas donner aux fléaux secondaires une importance disproportionnée, et qu’il convenait de chercher ailleurs et plus profondément les raisons d’exister ou de mourir qui dominent les peuples. Indépendamment donc des circonstances de bien-être ou de malaise, on a commencé à envisager la constitution des sociétés en elle-même, et on s’est montré disposé à admettre que nulle cause extérieure n’avait sur elle une prise mortelle, tant qu’un principe destructif né d’elle-même et dans son sein, inhérent, attaché à ses entrailles, n’était pas puissamment développé, et qu’au contraire, aussitôt que ce fait destructeur existait, le peuple, chez lequel il fallait le constater, ne pouvait manquer de mourir, fût-il le mieux gouverné des peuples, absolument comme un cheval épuisé s’abat sur une route unie.\par
En prenant la question sous ce point de vue, on faisait un grand pas, il faut le reconnaître, et on se plaçait sur un terrain, dans tous les cas, beaucoup plus philosophique que le premier. En effet, Bichat n’a pas cherché à découvrir le grand mystère de l’existence en étudiant les dehors ; il a tout demandé à l’intérieur du sujet humain. En faisant de même, on s’attachait au seul vrai moyen d’arriver à des découvertes. Malheureusement cette bonne pensée, n’étant que le résultat de l’instinct, ne poussa pas très loin sa logique, et on la vit se briser sur la première difficulté. On s’était écrié : Oui, réellement, c’est dans le sein même d’un corps social qu’existe la cause de sa dissolution ; mais quelle est cette cause ? La \emph{dégénération}, fut-il répliqué ; les nations meurent lorsqu’elles sont composées d’éléments \emph{dégénérés.} La réponse était fort bonne, étymologiquement et de toute manière ; il ne s’agissait plus que de définir ce qu’il faut entendre par ces mots : \emph{nation dégénérée.} C’est là qu’on fit naufrage : on expliqua un \emph{peuple dégénéré} par un peuple qui, mal gouverné, abusant de ses richesses, fanatique ou irréligieux, a perdu les vertus caractéristiques de ses premiers pères. Triste chute ! Ainsi une nation périt sous les fléaux sociaux parce qu’elle est dégénérée, et elle est dégénérée parce qu’elle périt. Cet argument circulaire ne prouve que l’enfance de l’art en matière d’anatomie sociale. Je veux bien que les peuples périssent parce qu’ils sont dégénérés, et non pour autre cause ; c’est par ce malheur qu’ils sont rendus définitive­ment incapables de souffrir le choc des désastres ambiants, et qu’alors, ne pouvant plus supporter les coups de la fortune adverse, ni se relever après les avoir subis, ils donnent le spectacle de leurs illustres agonies ; s’ils meurent, c’est qu’ils n’ont plus pour traverser les dangers de la vie la même vigueur que possédaient leurs ancêtres, c’est, en un mot enfin, qu’ils sont \emph{dégénérés.} L’expression, encore une fois, est fort bonne ; mais il faut l’expliquer un peu mieux et lui donner un sens. Comment et pourquoi la vigueur se perd-elle ? Voilà ce qu’il faut dire. Comment dégénère-t-on ? C’est là ce qu’il s’agit d’exposer. jusqu’ici on s’est contenté du mot, on n’a pas dévoilé la chose. C’est ce pas de plus que je vais essayer de faire.\par
Je pense donc que le mot \emph{dégénéré}, s’appliquant à un peuple, doit signifier et signifie que ce peuple n’a plus la valeur intrinsèque qu’autrefois il possédait, parce qu’il n’a plus dans ses veines le même sang, dont des alliages successifs ont graduellement modifié la valeur ; autrement dit, qu’avec le même nom, il n’a pas conservé la même race que ses fondateurs ; enfin, que l’homme de la décadence, celui qu’on appelle l’homme \emph{dégénéré}, est un produit différent, au point de vue ethnique, du héros des grandes époques. Je veux bien qu’il possède quelque chose de son essence ; mais, plus il dégénère, plus ce quelque chose s’atténue. Les éléments hétérogènes qui prédominent désormais en lui composent une nationalité toute nouvelle et bien malencontreuse dans son originalité ; il n’appartient à ceux qu’il dit encore être ses pères, qu’en ligne très collatérale. Il mourra définitivement, et sa civilisation avec lui, le jour où l’élément ethnique primordial se trouvera tellement subdivisé et noyé dans des apports de races étrangères, que la virtualité de cet élément n’exercera plus désormais d’action suffisante. Elle ne disparaîtra pas, sans doute, d’une manière absolue ; mais, dans la pratique, elle sera tellement combattue, tellement affaiblie, que sa force deviendra de moins en moins sensible, et c’est à ce moment que la dégénération pourra être considérée comme complète, et que tous ses effets apparaîtront.\par
Si je parviens à démontrer ce théorème, j’ai donné un sens au mot de dégénération. En montrant comment l’essence d’une nation s’altère graduelle­ment, je déplace la responsabilité de la décadence ; je la rends, en quelque sorte, moins honteuse ; car elle ne pèse plus sur des fils, mais sur des neveux, puis sur des cousins, puis sur des alliés de moins en moins proches ; et lorsque je fais toucher au doigt que les grands peuples, au moment de leur mort, n’ont qu’une bien faible, bien impondérable partie du sang des fondateurs dont ils ont hérité, j’ai suffisamment expliqué comment il se peut faire que les civilisations finissent, puisqu’elles ne restent pas dans les mêmes mains. Mais là, en même temps, je touche à un problème encore bien plus hardi que celui dont j’ai tenté l’éclaircissement dans les chapitres qui précèdent, puisque la question que j’aborde est celle-ci :\par
Y a-t-il entre les races humaines des différences de valeur intrinsèque réelle­ment sérieuses, et ces différences sont-elles possibles à apprécier ?\par
Sans tarder davantage, j’entame la série des considérations relatives au premier point ; le second sera résolu par la discussion même.\par
Pour faire comprendre ma pensée d’une manière plus claire et plus saisis­sable, je commence par comparer une nation, toute nation, au corps humain, à l’égard duquel les physiologistes professent cette opinion, qu’il se renouvelle constamment, dans toutes ses parties constituantes, que le travail de transformation qui se fait en lui est incessant, et qu’au bout de certaines périodes, il renferme bien peu de ce qui en était d’abord partie intégrante, de telle sorte que le vieillard n’a rien de l’homme fait, l’homme fait rien de l’adolescent, l’adolescent rien de l’enfant, et que l’individualité matérielle n’est pas autrement maintenue que par des formes internes et externes qui se sont succédé les unes aux autres en se copiant à peu près. Une différence que j’admettrai pourtant entre le corps humain et les nations, c’est que, dans ces dernières, il est très peu question de la conservation des formes, qui se détruisent et disparaissent avec infiniment de rapidité. je prends un peuple, ou, pour mieux dire, une tribu, au moment où, cédant à un instinct de vitalité prononcé, elle se donne des lois et commence à jouer un rôle en ce monde. Par cela même que ses besoins, que ses forces s’accroissent, elle se trouve en contact inévitable avec d’autres familles, et, par la guerre ou par la paix, réussit à se les incorporer.\par
Il n’est pas donné à toutes les familles humaines de se hausser à ce premier degré, passage nécessaire qu’une tribu doit franchir pour parvenir un jour à l’état de nation. Si un certain nombre de races, qui même ne sont pas cotées très haut sur l’échelle civilisatrice, l’ont pourtant traversé, on ne peut pas dire avec vérité que ce soit là une règle générale ; il semblerait, au contraire, que l’espèce humaine éprouve une assez grande difficulté à s’élever au-dessus de l’organisation parcellaire, et que c’est seulement pour des groupes spécialement doués qu’a lieu le passage à une situation plus complexe. J’invoquerai, en témoignage, l’état actuel d’un grand nombre de groupes répandus dans toutes les parties du monde. Ces tribus grossières, surtout celles des nègres pélagiens de la Polynésie, les Samoyèdes et autres familles du monde boréal et la plus grande partie des nègres africains, n’ont, jamais pu sortir de cette impuissance, et vivent juxtaposées les unes aux autres et en rapports de complète indépendance. Les plus forts massacrent les plus faibles, les plus faibles cherchent à mettre une distance aussi grande que possible entre eux et les plus forts ; là se borne toute la politique de ces embryons de sociétés qui se perpétuent depuis le commencement de l’espèce humaine, dans un état si imparfait, sans avoir jamais pu mieux faire. On objectera que ces misérables hordes forment la moindre partie de la population du globe ; sans doute, mais il faut tenir compte de toutes leurs pareilles qui ont existé et disparu. Le nombre en est incalculable, et il compose certainement la grande majorité des races pures dans les variétés jaune et noire.\par
Si donc il faut admettre que, pour un nombre très important d’humains, il a été impossible et l’est à jamais de faire même le premier pas vers la civilisation ; si, en outre, nous considérons que ces peuplades se trouvent dispersées sur la face entière du monde, dans les conditions de lieux et de climats les plus diverses, habitant indifférem­ment les pays glacés, tempérés, torrides, le bord des mers, des lacs et des rivières, le fond des bois, les prairies herbeuses, ou les déserts arides, nous sommes induits à conclure qu’une partie de l’humanité est, en elle-même, atteinte d’impuissance à se civiliser jamais, même au premier degré, puisqu’elle est inhabile à vaincre les répugnan­ces naturelles que l’homme, comme les animaux, éprouve pour le croisement.\par
Nous laissons donc ces tribus insociables de côté, et nous continuons la marche ascendante avec celles qui comprennent que, soit par la guerre, soit par la paix, si elles veulent augmenter leur puissance et leur bien-être, c’est une absolue nécessité que de forcer leurs voisins d’entrer dans leur cercle d’existence. La guerre est bien incontesta­blement le plus simple des deux moyens. La guerre se fait donc ; mais, la campagne finie, quand les passions destructives sont satisfaites, il reste des prisonniers, ces prisonniers deviennent des esclaves, ces esclaves travaillent ; voilà des rangs, voilà une industrie voilà une tribu devenue peuplade. C’est un degré supérieur qui, à son tour, n’est pas nécessairement franchi par les agrégations d’hommes qui ont su s’y élever ; beaucoup s’en contentent et y croupissent.\par
Mais certaines autres, de beaucoup plus imaginatives et plus énergiques, compren­nent quelque chose de mieux que le simple maraudage ; elles font la conquête d’une vaste terre, et prennent en propriété, non plus les habitants seulement, mais le sol avec eux. Une véritable nation est dès lors formée. Souvent alors, pendant un temps, les deux races continuent à vivre côte à côte sans se mêler ; et cependant, comme elles sont devenues indispensables l’une à l’autre, que la communauté de travaux et d’intérêts s’est à la longue établie, que les rancunes de la conquête et son orgueil s’émoussent, que, tandis que ceux qui sont dessous tendent naturellement à monter au niveau de leurs maîtres, les maîtres rencontrent aussi mille motifs de tolérer et quelquefois de servir cette tendance, le mélange du sang finit par s’opérer, et les hommes des deux origines, cessant de se rattacher à des tribus distinctes, se confondent de plus en plus.\par
L’esprit d’isolement est toutefois tellement inhérent à l’espèce humaine que, même dans cet état de croisement avancé, il y a encore résistance à un croisement ultérieur. Il est des peuples dont nous savons d’une manière très positive que leur origine est multiple, et qui pourtant conservent avec une force extraordinaire l’esprit de clan. Nous le savons pour les Arabes, qui font plus que de sortir de différents rameaux de la souche sémitique ; ils appartiennent, tout à la fois, à ce qu’on nomme la famille de Sem et à celle de Cham, sans parler d’autres parentés locales infinies. Malgré cette diversité de sources, leur attachement à la séparation par tribu forme un des traits les plus frappants de leur caractère national et de leur histoire politique ; si bien qu’on a cru pouvoir attribuer, en grande partie, leur expulsion de l’Espagne, non seulement au fractionnement de leur puissance dans ce pays, mais encore et surtout au morcellement plus intime que la distinction continue, et par suite la rivalité des familles, perpétuait au sein des petites monarchies de Valence, de Tolède, de Cordoue et de Grenade \footnote{Cet attachement des nations arabes à l’isolement ethnique se manifeste quelquefois d’une manière bien bizarre. Un voyageur (M. Fulgence Fresnel, si je ne me trompe) raconte qu’à Djiddah, où les mœurs sont très relâchées, la même Bédouine qui ne refuse rien à la plus légère séduction d’argent, se trouverait déshonorée, si elle épousait en légitime mariage soit le Turk, soit l’Européen auquel elle se prête en le méprisant.}. Pour la plupart des peuples on peut faire la même remarque, en ajoutant que là où la séparation par tribu s’est effacée, celle par nation la remplace, agissant avec une énergie presque semblable, et telle que la communauté de religion ne suffit pas à la paralyser. Elle existe entre les Arabes et les Turks comme entre les Persans et les Juifs, les Parsis et les Hindous, les Nestoriens Syriens et les Kurdes ; on la retrouve également dans la Turquie d’Europe ; on suit sa trace en Hongrie, entre les Madjars, les Saxons, les Valaques, les Croates, et je puis affirmer, pour l’avoir vu, que dans certaines parties de la France, ce pays où les races sont mélangées plus que partout ailleurs peut-être, il est des populations qui, de village à village, répugnent encore aujourd’hui à contracter alliance.\par
Je me crois en droit de conclure, d’après ces exemples qui embrassent tous les pays et tous les siècles, même notre pays et notre temps, que l’humanité éprouve, dans toutes ses branches, une répulsion secrète pour les croisements ; que, chez plusieurs de ces rameaux, cette répulsion est invincible ; que, chez d’autres, elle n’est domptée que dans une certaine mesure ; que ceux, enfin, qui secouent le plus complètement le joug de cette idée ne peuvent cependant s’en débarrasser de telle façon qu’il ne leur en reste au moins quelques traces : ces derniers forment ce qui est civilisable dans notre espèce.\par
 Ainsi le genre humain se trouve soumis à deux lois, l’une de répulsion, l’autre d’attraction, agissant, à différents degrés, sur ses races diverses ; deux lois, dont la première n’est respectée, que par celles de ces races qui ne doivent jamais s’élever au-dessus des perfectionnements tout à fait élémentaires de la vie de tribu, tandis que la seconde, au contraire, règne avec d’autant plus d’empire, que les familles ethniques sur lesquelles elle s’exerce sont plus susceptibles de développements.\par
Mais c’est ici qu’il faut surtout être précis. Je viens de prendre un peuple à l’état de famille, d’embryon ; je l’ai doué de l’aptitude nécessaire pour passer à l’état de nation ; il y est ; l’histoire ne m’apprend pas quels étaient les éléments constitutifs du groupe originaire ; tout ce que je sais, c’est que ces éléments le rendaient apte aux transforma­tions que je lui ai fait subir ; maintenant agrandi, deux possibilités sont seules présentes pour lui ; entre deux destinées, l’une ou l’autre est inévitable : ou il sera conquérant, ou il sera conquis.\par
Je le suppose conquérant ; je lui fais la plus belle part : il domine, gouverne et civilise tout à la fois ; il n’ira pas, dans les provinces qu’il parcourt, semer inutilement le meurtre et l’incendie ; les monuments, les institutions, les mœurs, lui seront également sacrés ; ce qu’il changera, ce qu’il trouvera bon et utile de modifier, sera remplacé par des créations supérieures ; la faiblesse deviendra force dans ses mains ; il se comportera de telle façon que, suivant le mot de l’Écriture, il sera grand devant les hommes.\par
Je ne sais si le lecteur y a déjà pensé, mais, dans le tableau que je trace, et qui n’est autre, à certains égards, que celui présenté par les Hindous, les Égyptiens, les Perses, les Macédoniens, deux faits me paraissent bien saillants. Le premier, c’est qu’une nation, sans force et sans puissance, se trouve tout à coup, par le fait d’être tombée aux mains de maîtres vigoureux, appelée au partage d’une nouvelle et meilleure destinée, ainsi qu’il arriva aux Saxons de l’Angleterre, lorsque les Normands les eurent soumis ; la seconde, c’est qu’un peuple d’élite, un peuple souverain, armé, comme tel, d’une propension marquée à se mêler à un autre sang, se trouve désormais en contact intime avec une race dont l’infériorité n’est pas seulement démontrée par la défaite, mais encore par le défaut des qualités visibles chez les vainqueurs. Voilà donc, à dater précisément du jour où la conquête est accomplie et où la fusion commence, une modification sensible dans la constitution du sang des maîtres. Si la nouveauté devait s’arrêter là, on se trouverait, au bout d’un laps de temps d’autant plus considérable que les nations superposées auraient été originaire­ment plus nombreuses, avoir en face une race nouvelle, moins puissante, à coup sûr, que le meilleur de ses ancêtres, forte encore cependant, et faisant preuve de qualités spéciales résultant du mélange même, et inconnues aux deux familles génératrices. Mais il n’en va pas ainsi d’ordinaire, et l’alliage n’est pas longtemps borné à la double race nationale seulement.\par
L’empire que je viens d’imaginer est puissant ; il agit sur ses voisins. Je suppose de nouvelles conquêtes ; c’est encore un nouveau sang qui, chaque fois, vient se mêler au courant. Désormais, à mesure que la nation grandit, soit par les armes, soit par les traités, son caractère ethnique s’altère de plus en plus. Elle est riche, commerçante, civilisée ; les besoins et les plaisirs des autres peuples trouvent chez elle, dans ses capitales, dans ses grandes villes, dans ses ports, d’amples satisfactions, et les mille attraits qu’elle possède fixent au milieu d’elle le séjour de nombreux étrangers. Peu de temps se passe, et une distinction de castes peut, à bon droit, succéder à la distinction primitive par nations.\par
Je veux que le peuple sur lequel je raisonne soit confirmé dans ses idées de séparation par les prescriptions religieuses les plus formelles, et qu’une pénalité redoutable veille à l’entour pour épouvanter les délinquants. Parce que ce peuple est civilisé, ses mœurs sont douces et tolérantes, même au mépris de sa foi ; ses oracles auront beau parler, il naîtra des gens décastés : il faudra créer tous les jours de nouvelles distinctions, inventer de nouvelles classifications, multiplier les rangs, rendre presque impossible de se reconnaître au milieu de subdivisions variant à l’infini, changeant de province à province, de canton à canton, de village à village  ; faire enfin ce qui a lieu dans les pays hindous. Mais il n’est guère que le brahmane qui ses ait montré autant de ténacité dans ses idées séparatrices ; les peuples civilisés par lui, en dehors de son sein, n’ont jamais adopté, ou du moins ont rejeté depuis longtemps, des entraves gênantes. Dans tous les États avancés en culture intellectuelle, on ne s’est pas même arrêté un instant aux ressources désespérées que le désir de concilier les prescriptions du code de Manou avec le courant irrésistible des choses inspira aux législateurs de l’Aryavarta, Partout ailleurs, les castes, lorsqu’il y en a eu réellement, ont cessé d’exister au moment où le pouvoir de faire fortune, de s’illustrer par des découvertes utiles ou des talents agréables, a été acquis à tout le monde, sans distinc­tion d’origine. Mais aussi, à dater du même jour, la nation primitivement conquérante, agissante, civilisatrice, a commencé à disparaître : son sang était immergé dans celui de tous les affluents qu’elle avait détournés vers elle.\par
Le plus souvent, en outre, les peuples dominateurs ont commencé par être infini­ment moins nombreux que leurs vaincus, et il semble, d’autre part, que certaines races qui servent de base à la population de contrées fort étendues, soient singulièrement prolifiques ; je citerai les Celtes, les Slaves. Raison de plus pour que les races maîtresses disparaissent rapidement. Encore un autre motif, c’est que leur activité plus grande, le rôle plus direct qu’elles jouent dans les affaires de leur État, les exposent particulièrement aux funestes résultats des batailles, des proscriptions et des révoltes. Ainsi, tandis que, d’une part, elles amassent autour d’elles, par le fait même de leur génie civilisateur, des éléments divers où elles doivent s’absorber, elles sont encore victimes d’une cause première, leur petit nombre originel, et d’une foule de causes secondes, qui toutes concourent à les détruire.\par
Il est assez évident de soi que la disparition de la race victorieuse est soumise, suivant les différents milieux, à des conditions de temps variant à l’infini. Toutefois elle s’achève partout, et partout elle est aussi parfaite que de besoin, longtemps avant la fin de la civilisation qu’elle est censée animer, de sorte qu’un peuple marche, vit, fonc­tionne, souvent même grandit après que le mobile générateur de sa vie et de sa gloire a cessé d’être. Croit-on trouver là une contradiction avec ce qui précède ? Nullement ; car, tandis que l’influence du sang civilisateur va s’épuisant par la division, la force de propulsion jadis imprimée aux masses soumises ou annexées subsiste encore ; les institutions que le défunt maître avait inventées, les lois qu’il avait formulées, les mœurs dont il avait fourni le type se sont maintenues après lui. Sans doute, mœurs, lois, institutions, ne survivent que fort oublieuses de leur antique esprit, défigurées tous les jours davantage, caduques et perdant leur sève ; mais, tant qu’il en reste une ombre, l’édifice se soutient, le corps semble avoir une âme, le cadavre marche. Quand le dernier effort de cette impulsion antique est achevé, tout est dit ; rien ne reste, la civilisation est morte.\par
Je me crois maintenant pourvu de tout le nécessaire pour résoudre le problème de la vie et de la mort des nations, et je dis qu’un peuple ne mourrait jamais en demeurant éternellement composé des mêmes éléments nationaux. Si l’empire de Darius avait encore pu mettre en ligne, à la bataille d’Arbelles, des Perses, des Arians véritables ; si les Romains du Bas Empire avaient eu un sénat et une milice formés d’éléments ethniques semblables à ceux qui existaient au temps des Fabius, leurs dominations n’auraient pas pris fin, et, tant qu’ils auraient conservé la même intégrité de sang, Perses et Romains auraient vécu et régné. On objectera qu’ils auraient néanmoins, à la longue, vu venir à eux des vainqueurs plus irrésistibles qu’eux-mêmes et qu’ils auraient succom­bé sous des assauts bien combinés, sous une longue pression, ou, plus simplement, sous le hasard d’une bataille perdue. Les États, en effet, auraient pu prendre fin de cette manière, non pas la civilisation, ni le corps social. L’invasion et la défaite n’auraient constitué que la triste mais temporaire traversée d’assez mauvais jours. Les exemples à fournir sont en grand nombre.\par
Dans les temps modernes, les Chinois ont été conquis à deux reprises toujours ils ont forcé leurs vainqueurs à s’assimiler à eux ; ils leur ont imposé le respect de leurs mœurs ; ils leur ont beaucoup donné, et n’en ont presque rien reçu. Une fois ils ont expulsé les premiers envahisseurs, et, dans un temps donné, ils en feront autant des seconds.\par
Les Anglais sont les maîtres de l’Inde, et pourtant leur action morale sur leurs sujets est presque absolument nulle. Ils subissent eux-mêmes, en bien des manières, l’influence de la civilisation locale, et ne peuvent réussir à faire pénétrer leurs idées dans les esprits d’une foule qui redoute ses dominateurs, ne plie que physiquement devant eux, et maintient ses notions debout en face des leurs. C’est que la race hindoue est devenue étrangère à celle qui la maîtrise aujourd’hui, et sa civilisation échappe à la loi du plus fort. Les formes extérieures, les royaumes, les empires ont pu varier, et varieront encore, sans que le fond sur lequel de telles constructions reposent, dont elles ne sont qu’émanées, soit altéré essentiellement avec elles ; et Haïderabad, Lahore, Dehli cessant d’être des capitales, la société hindoue n’en subsistera pas moins. Un moment viendra où, de façon ou d’autre, l’Inde recommencera à vivre publiquement d’après ses lois propres, comme elle le fait tacitement, et, soit par sa race actuelle, soit par des métis, reprendra la plénitude de sa personnalité politique.\par
Le hasard des conquêtes ne saurait trancher la vie d’un peuple. Tout au plus, il en suspend pour un temps les manifestations, et, en quelque sorte, les honneurs extérieurs. Tant que le sang de ce peuple et ses institutions conservent encore, dans une mesure suffisante, l’empreinte de la race initiatrice, ce peuple existe ; et, soit qu’il ait affaire, comme les Chinois, à des conquérants qui ne sont que matériellement plus énergiques que lui ; soit, comme les Hindous, qu’il soutienne une lutte de patience, bien autrement ardue, contre une nation de tous points supérieure, telle qu’on voit les Anglais, son avenir certain doit le consoler ; il sera libre un jour. Au contraire, ce peuple, comme les Grecs, comme les Romains du Bas-Empire, a-t-il absolument épuisé son principe ethnique et les conséquences qui en découlaient, le moment de sa défaite sera celui de sa mort : il a usé les temps que le ciel lui avait d’avance concédés, car il a complètement changé de race, donc de nature, et par conséquent il est dégénéré.\par
En vertu de cette observation, on doit considérer comme résolue la question, souvent agitée, de savoir ce qui serait advenu, si les Carthaginois, au lieu de succomber devant la fortune de Rome, étaient devenus maîtres de l’Italie. En tant qu’appartenant à la souche phénicienne, souche inférieure en vertus politiques aux races d’où sortaient les soldats de Scipion, l’issue contraire de la bataille de Zama ne pouvait rien changer à leur sort. Heureux un jour, le lendemain les aurait vus tomber devant une revanche ; ou bien encore, absorbés dans l’élément italien par la victoire, comme ils le furent par la défaite, le résultat final aurait été identiquement le même. Le destin des civilisations ne va pas au hasard, il ne dépend pas d’un coup de dé ; le glaive ne tue que des hommes ; et les nations les plus belliqueuses, les plus redoutables, les plus triomphantes, quand elles n’ont eu dans le cœur, dans la tête et dans la main, que bravoure, science stratégique et succès guerriers, sans autre instinct supé­rieur, n’ont jamais obtenu une plus belle fin que d’apprendre de leurs vaincus, et de l’apprendre mal, comment on vit dans la paix. Les Celtes, les hordes nomades de l’Asie, ont des annales pour ne rien raconter de plus.\par
Après avoir assigné un sens au mot \emph{dégénération}, et avoir traité, avec ce secours, le problème de la vitalité des peuples, il faut prouver maintenant ce que j’ai dû, pour la clarté de la discussion, avancer \emph{a priori} : qu’il existe des différences sensibles dans la valeur relative des races humaines. Les conséquences d’une pareille démonstration sont considérables ; leur portée va loin. Avant de les aborder, on ne saurait les étayer d’un ensemble trop complet de faits et de raisons capables de soutenir un aussi grand édifice. La première question que j’ai résolue n’était que le propylée du temple.
\section[{I.5. Les inégalités ethniques ne sont pas le résultat des institutions.}]{I.5. \\
Les inégalités ethniques ne sont pas le résultat des institutions.}
\noindent L’idée d’une inégalité native, originelle, tranchée et permanente entre les diverses races, est, dans le monde, une des opinions le plus anciennement répandues et adoptées ; et, vu l’isolement primitif des tribus, des peuplades, et ce \emph{retirement} vers elles-mêmes que toutes ont pratiqué à une époque plus ou moins lointaine, et d’où un grand nombre n’est jamais sorti, on n’a pas lieu d’en être étonné. À l’exception de ce qui s’est passé dans nos temps les plus modernes, cette notion a servi de base à presque toutes les théories gouvernementales. Pas de peuple, grand ou petit, qui n’ait débuté par en faire sa première maxime d’État. Le système des castes, des noblesses, celui des aristocraties, tant qu’on les fonde sur les prérogatives de la naissance, n’ont pas d’autre origine ; et le droit d’aînesse, en supposant la préexcellence du fils premier-né et de ses descendants, n’en est aussi qu’un dérivé. Avec cette doctrine concordent la répulsion pour l’étranger et la supériorité que chaque nation s’adjuge à l’égard de ses voisines. Ce n’est qu’à mesure que les groupes se mêlent et se fusionnent, que, désormais agrandis, civilisés et se considérant sous un jour plus bienveillant par suite de l’utilité dont ils se sont les uns aux autres, l’on voit chez eux cette maxime absolue de l’inégalité, et d’abord de l’hostilité des races, battue en brèche et discutée. Puis, quand le plus grand nombre des citoyens de l’État sent couler dans ses veines un sang mélangé, ce plus grand nombre, transformant en vérité universelle et absolue ce qui n’est réel que pour lui, se sent appelé à affirmer que tous les hommes sont égaux. Une louable répugnance pour l’oppression, la légitime horreur de l’abus de la force, jettent alors, dans toutes les intelligences, un assez mauvais vernis sur le souvenir des races jadis dominantes et qui n’ont jamais manqué, car tel est le train du monde, de légitimer, jusqu’à un certain point, beaucoup d’accusations. De la déclamation contre la tyrannie, on passe à la négation des causes naturelles de la supériorité qu’on insulte ; on la déclare non seulement perverse, mais encore usurpatrice ; on nie, et bien à tort, que certaines aptitudes soient nécessairement, fatalement, l’héritage exclusif de telles ou telles descendances ; enfin, plus un peuple est composé d’éléments hétérogènes, plus il se complaît à proclamer que les facultés les plus diverses sont possédées ou peuvent l’être au même degré par toutes les fractions de l’espèce humaine sans exclusion. Cette théorie, à peu près soutenable pour ce qui les concerne, les raisonneurs métis l’appliquent à l’ensemble des générations qui ont paru, paraissent et paraîtront sur la terre, et ils finissent un jour par résumer leurs sentiments en ces mots, qui, comme l’outre d’Éole, renferment tant de tempêtes : « Tous les hommes sont frères ! »\par
Voilà l’axiome politique. Veut-on l’axiome scientifique ? « Tous les hommes, disent les défenseurs de l’égalité humaine, sont pourvus d’instruments intellectuels pareils, de même nature, de même valeur, de même portée. » Ce ne sont pas les paroles expresses, peut-être, mais du moins c’est le sens. Ainsi, le cervelet du Huron contient en germe un esprit tout à fait semblable à celui de l’Anglais et du Français ! Pourquoi donc, dans le cours des siècles, n’a-t-il découvert ni l’imprimerie ni la vapeur ? Je serais en droit de lui demander, à ce Huron, s’il est égal à nos compatriotes, d’où il vient que les guerriers de sa tribu n’ont pas fourni de César ni de Charlemagne, et par quelle inexplicable négligence ses chanteurs et ses sorciers ne sont jamais devenus ni des Homères ni des Hippocrates ? À cette difficulté on répond, d’ordinaire, en mettant en avant l’influence souveraine des milieux. Suivant cette doctrine, une île ne verra point, en fait de prodiges sociaux, ce que connaîtra un continent ; au nord, on ne sera pas ce qu’on est au midi ; les bois ne permettront pas les développements que favorisera la plaine découverte ; que sais-je ? L’humidité d’un marais fera pousser une civilisation que la sécheresse du Sahara aurait infailliblement étouffée. Quelque ingénieuses que soient ces petites hypothèses, elles ont contre elles la voix des faits. Malgré le vent, la pluie, le froid, le chaud, la stérilité, la plantureuse abondance, partout le monde a vu fleurir tour à tour, et sur les mêmes sols, la barbarie et la civilisation. Le fellah abruti se calcine au même soleil qui brûlait le puissant prêtre de Memphis ; le savant professeur de Berlin enseigne sous le même ciel inclément qui vit jadis les misères du sauvage finnois.\par
Le plus curieux, c’est que l’opinion égalitaire, admise par la masse des esprits, d’où elle a découlé dans nos institutions et dans nos mœurs n’a pas trouvé assez de force pour détrôner l’évidence, et que les gens les plus convaincus de sa vérité font tous les jours acte d’hommage au sentiment contraire. Personne ne se refuse à constater, à chaque instant, de graves différences entre les nations, et le langage usuel même les confesse avec la plus naïve inconséquence. On ne fait, en cela, qu’imiter ce qui s’est pratiqué à des époques non moins persuadées que nous, et pour les mêmes causes, de l’égalité absolue des races.\par
Chaque nation a toujours su, à côté du dogme libéral de la fraternité, maintenir, auprès des noms des autres peuples, des qualifications et des épithètes qui indiquaient des dissemblances. Le Romain d’Italie appelait le Romain de la Grèce, \emph{Graeculus}, et lui attribuait le monopole de la loquacité vaniteuse et du manque de courage. Il se moquait du colon de Carthage, et prétendait le reconnaître entre mille à son esprit processif et à sa mauvaise foi. Les Alexandrins passaient pour spirituels, insolents et séditieux. Au moyen âge, les monarques anglo-normands taxaient leurs sujets gallois de légèreté et d’inconsistance d’esprit. Aujourd’hui qui n’a pas entendu relever les traits distinctifs de l’Allemand, de l’Espagnol, de l’Anglais et du Russe ? Je n’ai pas à me prononcer sur l’exactitude des jugements. Je note seulement qu’ils existent, et que l’opinion courante les adopte, Ainsi donc, si, d’une part, les familles humaines sont dites égales, et que, de l’autre, les unes soient frivoles, les autres posées ; celles-ci âpres au gain, celles-là à la dépense ; quelques-unes énergiquement amoureuses des combats, plusieurs économes de leurs peines et de leurs vies, il tombe sous le sens que ces nations si différentes doivent avoir des destinées bien diverses, bien dissemblables, tranchons le mot, bien inégales. Les plus fortes joueront dans la tragédie du monde les personnages des rois et des maîtres. Les plus faibles se contenteront des bas emplois.\par
Je ne crois pas qu’on ait fait de nos jours le rapprochement entre les idées générale­ment admises sur l’existence d’un caractère spécial pour chaque peuple et la conviction non moins répandue que tous les peuples sont égaux. Cependant cette contradiction frappe bien fort ; elle est flagrante, et d’autant plus grave que les partisans de la démocratie ne sont pas les derniers à célébrer la supériorité des Saxons de l’Amérique du Nord sur toutes les nations du même continent. Ils attribuent, à la vérité, les hautes prérogatives de leurs favoris à la seule influence de la forme gouvernementale. Toute­fois ils ne nient pas, que je sache, la disposition particulière et native des compatriotes de Penn et de Washington à établir dans tous les lieux de leur séjour des institutions libérales, et, ce qui est plus, à les savoir conserver. Cette force de persistance n’est-elle pas, je le demande, une bien grande prérogative départie à cette branche de la famille humaine, prérogative d’autant plus précieuse que la plupart des groupes qui ont peuplé jadis ou peuplent encore l’univers semblent en être privés ?\par
Je n’ai pas la prétention de jouir sans combat de la vue de cette inconséquence. C’est ici, sans doute, que les partisans de l’égalité objecteront bien haut la puissance des institutions et des mœurs ; c’est ici qu’ils diront, encore une fois, combien l’essence du gouvernement par sa seule et propre vertu, combien le fait du despotisme ou de la liberté, influent puissamment sur le mérite et le développement d’une nation : mais c’est ici que moi, de même, je contesterai la force de l’argument.\par
Les institutions politiques n’ont à choisir qu’entre deux origines : ou bien elles dérivent de la nation qui doit vivre sous leur règle, ou bien, inventées chez un peuple influent, elles sont appliquées par lui à des États tombés dans sa sphère d’action.\par
Avec la première hypothèse il n’y a pas de difficulté. Le peuple évidemment a calculé ses institutions sur ses instincts et sur ses besoins ; il s’est gardé de rien statuer qui pût gêner les uns ou les autres ; et si, par mégarde ou maladresse, il l’a fait, bientôt le malaise qui en résulte l’amène à corriger ses lois et à les mettre dans une concordance plus parfaite avec leur but. Dans tout pays autonome, on peut dire que la loi émane toujours du peuple ; non pas qu’il ait constamment la faculté de la promulguer directement, mais parce que, pour être bonne, il faut qu’elle soit modelée sur ses vues, et telle que, bien informé, il l’aurait imaginée lui-même. Si quelque très sage législateur semble, au premier abord, l’unique source de la loi, qu’on y regarde de bien près, et l’on se convaincra aussitôt que, par l’effet de sa sagesse même, le vénérable maître se borne à rendre ses oracles sous la dictée de sa nation. Judicieux comme Lycurgue, il n’ordonnera rien que le Dorien de Sparte ne puisse admettre, et, théoricien comme Dracon, il créera un code qui bientôt sera ou modifié ou abrogé par l’Ionien d’Athènes, incapable, comme tous les enfants d’Adam, de conserver longtemps une législation étrangère à ses vraies et naturelles tendances. L’intervention d’un génie supérieur dans cette grande affaire d’une invention de lois n’est jamais qu’une manifestation spéciale de la volonté éclairée d’un peuple, ou, si ce n’est que le produit isolé des rêveries d’un individu, nul peuple ne saurait s’en accommoder longtemps. On ne peut donc admettre que les institutions ainsi trouvées et façonnées par les races fassent les races ce qu’on les voit être. Ce sont des effets, et non des causes. Leur influence est grande évidemment : elles conservent le génie national, elles lui frayent des chemins, elles lui indiquent son but, et même, jusqu’à un certain point, échauffent ses instincts, et lui mettent à la main les meilleurs instruments d’action ; mais elles ne créent pas leur créateur, et, pouvant servir puissamment ses succès en l’aidant à développer ses qualités innées, elles ne sauraient jamais qu’échouer misérablement quand elles préten­dent trop agrandir le cercle ou le changer. En un mot, elles ne peuvent pas l’impossible.\par
Les institutions fausses et leurs effets ont cependant joué un grand rôle dans le monde. Quand Charles 1\textsuperscript{er}, fâcheusement conseillé par le comte de Strafford, voulait plier les Anglais au gouvernement absolu, le roi et son ministre marchaient sur le terrain fangeux et sanglant des théories. Quand les calvinistes rêvaient chez nous une administration tout à la fois aristocratique et républicaine, et travaillaient à l’implanter par les armes, ils se mettaient également à côté du vrai.\par
Quand le régent prétendit donner gain de cause aux courtisans vaincus en 1652, et essayer du gouvernement d’intrigue qu’avaient souhaité le coadjuteur et ses amis \footnote{M. le comte de Saint-Priest, dans un excellent article de la \emph{Revue des Deux Mondes}, a très justement démontré que le parti écrasé par le cardinal de Richelieu n’avait rien de commun avec la féodalité ni avec les grands systèmes aristocratiques. MM. de Montmorency, de Cinq-Mars, de Marillac, ne cherchaient à bouleverser l’État que pour obtenir des honneurs et des faveurs. Le grand cardinal est tout à fait innocent du meurtre de la noblesse française, qu’on lui a tant reproché.}, ses efforts ne plurent à personne, et blessèrent également noblesse, clergé, parlement et tiers état. Quelques traitants seuls se réjouirent. Mais, lorsque Ferdinand le Catholique institua contre les Maures d’Espagne ses terribles et nécessaires moyens de destruc­tion ; lorsque Napoléon rétablit en France la religion, flatta l’esprit militaire, organisa le pouvoir d’une manière à la fois protectrice et restrictive, l’un et l’autre de ces potentats avaient bien écouté et bien compris le génie de leurs sujets, et ils bâtissaient sur le terrain pratique. En un mot, les fausses institutions, très belles souvent sur le papier, sont celles qui, n’étant pas conformes aux qualités et aux travers nationaux, ne conviennent pas à un État, bien que pouvant faire fortune dans le pays voisin. Elles ne créent que le désordre et l’anarchie, fussent-elles empruntées à la législation des anges. Les autres, tout au rebours, qu’à tel ou tel point de vue, et même d’une manière absolue, le théoricien et le moraliste peuvent blâmer, sont bonnes pour les raisons contraires. Les Spartiates étaient petits de nombre, grands de cœur, ambitieux et violents : de fausses lois n’en auraient tiré que de pâles coquins ; Lycurgue en fit d’héroïques brigands.\par
Qu’on n’en doute pas. Comme la nation est née avant la loi, la loi tient d’elle et porte son empreinte avant de lui donner la sienne. Les modifications que le temps amène dans les institutions en sont encore une bien grande preuve.\par
Il a été dit plus haut qu’à mesure que les peuples se civilisaient, s’agrandissaient, devenaient plus puissants, leur sang se mélangeait et leurs instincts subissaient des altérations graduelles. En prenant ainsi des aptitudes différentes, il leur devient impossible de s’accommoder des lois convenables pour leurs devanciers. Aux généra­tions nouvelles, les mœurs le sont également et les tendances de même, et des modifications profondes dans les institutions ne tardent pas à suivre. On voit ces modifications devenir plus fréquentes et plus profondes, à mesure que la race change davantage, tandis qu’elles restaient plus rares et plus graduées, tant que les populations elles-mêmes étaient plus proches parentes des premiers inspirateurs de l’État. En Angleterre, celui de tous les pays de l’Europe où les modifications du sang ont été les plus lentes et jusqu’ici les moins variées, on voit encore les institutions du quatorzième et du quinzième siècle subsister dans les bases de l’édifice social. On y retrouve, presque dans sa vigueur ancienne, l’organisation communale des Plantagenêts et des Tudors, la même façon de mêler la noblesse au gouvernement et de composer cette noblesse, le même respect pour l’antiquité des familles uni au même goût pour les parvenus de mérite \footnote{Macaulay, \emph{History of England}. In-8°. Paris, 1849, t. I}. Mais cependant, comme, depuis Jacques 1\textsuperscript{er}, et surtout depuis l’Union de la reine Anne, le sang anglais a tendu de plus en plus à se mélanger avec celui d’Écosse et d’Irlande, que d’autres nations ont aussi contribué, bien qu’impercepti­blement, à altérer la pureté de la descendance, il en résulte que les innovations, tout en restant toujours assez fidèles à l’esprit primitif de la constitution, sont devenues, de nos jours, plus fréquentes qu’autrefois.\par
En France, les mariages ethniques ont été bien autrement nombreux et variés. Il est même arrivé que, par de brusques revirements, le pouvoir a passé d’une race à une autre. Aussi y a-t-il eu, dans la vie sociale, plutôt des changements que des modifica­tions, et ces changements ont été d’autant plus graves que les groupes qui se succédaient au pouvoir étaient plus différents. Tant que le nord de la France est resté prépondérant dans la politique du pays, la féodalité, ou, pour mieux dire, ses restes informes, se sont défendus avec assez d’avantage, et l’esprit municipal a tenu bon avec eux. Après l’expulsion des Anglais, au quinzième siècle, les provinces du centre, bien moins germaniques que les contrées d’outre-Loire, et qui, venant de restaurer l’indépen­dance nationale sous la conduite de Charles VII, voyaient naturellement leur sang gallo-romain prédominer dans les conseils et dans les camps, firent régner le goût de la vie militaire, des conquêtes extérieures, bien particulier à la race celtique, et l’amour de l’autorité, infus dans le sang romain. Pendant le seizième siècle, elles préparèrent largement le terrain sur lequel les compagnons aquitains de Henri IV, moins celtiques et plus romains encore, vinrent, en 1599, placer une autre et plus grosse pierre du pouvoir absolu. Puis, Paris ayant, à la fin, acquis la domination par suite de la concen­tration que le génie méridional avait favorisée, Paris, dont la population est assurément un résumé des spécimens ethniques les plus variés, n’eut plus de motif pour comprendre, aimer ni respecter aucune tradition, aucune tendance spéciale, et cette grande capitale, cette tour de Babel, rompant avec le passé, soit de la Flandre, soit du Poitou, soit du Languedoc, attira la France dans les expérimentations multipliées des doctrines les plus étrangères à ses coutumes anciennes.\par
On ne peut donc admettre que les institutions fassent les peuples ce qu’on les voit, quand ce sont les peuples qui les ont inventées. Mais en est-il de même dans la seconde hypothèse, c’est-à-dire lorsqu’une nation reçoit son code de mains étrangères pourvues de la puissance nécessaire pour le lui faire accepter, bon gré mal gré ?\par
Il y a des exemples de pareilles tentatives. Je n’en trouverai pas, à la vérité, qui aient été exécutées sur une grande échelle par les gouvernements vraiment politiques de l’antiquité ou des temps modernes ; leur sagesse ne s’est jamais appliquée à transformer le fond même de grandes multitudes. Les Romains étaient trop habiles pour se livrer à d’aussi dangereuses expériences. Alexandre, avant eux, ne les avait pas essayées ; et convaincus, par l’instinct ou la raison, de l’inanité de pareils efforts, les successeurs d’Auguste se contentèrent, comme le vainqueur de Darius, de régner sur une vaste mosaïque de peuples qui tous conservaient leurs habitudes, leurs mœurs, leurs lois, leurs procédés propres d’administration et de gouvernement, et qui, pour la plupart, tant que du moins ils restèrent par la race assez identiques à eux-mêmes, n’acceptèrent, en commun avec leurs co-sujets, que des prescriptions de fiscalité ou de précaution militaire.\par
Toutefois il est une circonstance qu’il ne faut pas négliger. Plusieurs des peuples asservis aux Romains avaient, dans leurs codes, des points tellement en désaccord avec les sentiments de leurs maîtres, qu’il était impossible à ces derniers d’en tolérer l’existence : témoins les sacrifices humains des druides, qu’en effet poursuivirent les défenses les plus sévères. Eh bien, les Romains, avec toute leur puissance, ne réussi­rent jamais complètement à extirper des rites aussi barbares. Dans la Narbonnaise, la victoire fut facile : la population gallique avait été presque entièrement remplacée par des colons romains ; mais, dans le centre, chez les tribus plus intactes, la résistance s’obstina, et, dans la presqu’île bretonne, où, au quatrième siècle, une colonie rapporta d’Angleterre les vieilles mœurs avec le vieux sang, les peuplades persistèrent, par patriotisme, par attachement à leurs traditions, à égorger des hommes sur leurs autels aussi souvent qu’elles l’osèrent. La surveillance la plus active ne réussissait pas à leur arracher des mains le couteau et le flambeau sacrés. Toutes les révoltes commençaient par la restauration de ce terrible trait du culte national, et le christianisme, vainqueur encore indigné d’un polythéisme sans morale, vint, chez les Armoricains, se heurter avec épouvante contre des superstitions plus repoussantes encore. Il ne parvint à les détruire qu’après des efforts bien longs, puisqu’au dix-septième siècle, le massacre des naufragés et l’exercice du droit de bris subsistaient dans toutes les paroisses maritimes où le sang kimrique s’était conservé pur. C’est que ces coutumes barbares répondaient aux instincts et aux sentiments indomptables d’une race qui, n’ayant pas été suffisam­ment mélangée, n’avait pas eu jusqu’alors de raisons déterminantes pour changer d’avis.\par
Ce fait est digne de réflexion ; mais les temps modernes présentent surtout des exemples d’institutions imposées et non subies. Un caractère remarquable de la civilisation européenne, c’est son intolérance, conséquence de la conscience qu’elle a de sa valeur et de sa force. Elle se trouve dans le monde, soit en face de barbaries décidées, soit à côté d’autres civilisations. Elle traite les unes et les autres avec un dédain presque égal, et, voyant dans tout ce qui n’est pas elle des obstacles à ses conquêtes, elle est fort disposée à exiger des peuples une complète transformation. Toutefois les Espagnols, les Anglais et les Hollandais, et nous aussi quelquefois, nous n’avons pas osé nous abandonner trop complètement aux impulsions du génie novateur, là où nous avions des masses un peu considérables devant nous, imitant ainsi la discrétion forcée des conquérants de l’antiquité. L’Orient et l’Afrique, soit septentrionale, soit occiden­tale, sont des témoins irréfragables que les nations les plus éclairées ne parviennent pas à donner à des peuples conquis des institutions antipathiques à leur nature. J’ai déjà rappelé que l’Inde anglaise continue son mode de vie séculaire sous les lois qu’elle s’est jadis données. Les Javanais, bien que très soumis, sont fort éloignés de se sentir entraînés vers des institutions approchant de celles de la Néerlande. Ils continuent à vivre en face de leurs maîtres comme ils vivaient libres, et, depuis le seizième siècle, où l’action européenne dans le monde oriental a commencé, on ne s’aperçoit pas qu’elle ait le moindrement influé sur les mœurs des tributaires les mieux domptés.\par
Mais tous les peuples vaincus ne sont pas assez forts par le nombre pour que le maître européen soit disposé à se contraindre. Il en est sur lesquels on a pesé avec toute la puissance du sabre pour aider à celle de la persuasion. On a résolument voulu changer leur mode d’existence, leur donner des institutions que nous savons bonnes et utiles. A-t-on réussi ?\par
L’Amérique nous offre à ce sujet le champ d’expériences le plus riche. Dans tout le sud, où la puissance espagnole a régné sans contrainte, à quoi a-t-elle abouti ? À déraciner les anciens empires, sans doute, non pas à éclairer les populations ; elle n’a pas créé des hommes semblables à leurs précepteurs.\par
Dans le nord, avec des procédés différents, les résultats ont été aussi négatifs ; que dis-je ? ils ont été plus nuls quant à la bienfaisante influence, plus calamiteux au point de vue de l’humanité, car, du moins, les Indiens espagnols multiplient d’une manière remarquable \footnote{M. Al. de Humboldt, \emph{Examen critique de l’histoire de la géogr. du N. C.}, t. II, p. 129-130.} ; ils ont même transformé le sang de leurs vainqueurs, qui ainsi sont descendus à leur niveau, tandis que les hommes à peaux rouges des États-Unis, saisis par l’énergie anglo-saxonne, sont morts du contact. Le peu qui en reste encore disparaît chaque jour, et disparaît tout aussi incivilisé, tout aussi incivilisable que ses pères.\par
Dans l’Océanie, les observations concluent de même : les peuplades aborigènes vont partout s’éteignant. On réussit quelquefois à leur arracher leurs armes, à les empêcher de nuire ; on ne les change pas. Partout où l’Européen est le maître, elles ne s’entre-mangent plus, elles se gorgent d’eau-de-vie, et cet abrutissement nouveau est tout ce que notre esprit initiateur réussit à leur faire aimer. Enfin il est au monde deux gouvernements formés par des peuples étrangers à nos races sur des modèles fournis par nous : l’un fonctionne aux Îles Sandwich, l’autre à Saint-Domingue. L’appréciation de ces deux États achèvera de démontrer l’impuissance de toutes tentatives pour donner à un peuple des institutions qui ne lui sont pas suggérées par son propre génie.\par
Aux îles Sandwich, le système représentatif brille de tout son éclat. On y trouve une chambre haute, une chambre basse, un ministère qui gouverne, un roi qui règne ; rien n’y manque. Mais tout cela n’est que décoration. Le rouage indispensable de la machine, celui qui la met en branle, c’est le corps des missionnaires protestants. Sans eux, roi, pairs et députés, ignorant la route à suivre, cesseraient bientôt de fonctionner. Aux missionnaires seuls revient l’honneur de trouver les idées, de les présenter, de les faire accepter, soit par le crédit dont ils jouissent sur leurs néophytes, soit, au besoin, par la menace. Je doute cependant que, si les missionnaires n’avaient pour instruments de leur volonté que le roi et les chambres, ils ne se vissent obligés, après avoir lutté quelque temps contre l’inaptitude de leurs écoliers, de prendre dans le maniement des affaires une part très grande, très directe, et par conséquent trop apparente. Ils ont paré à cet inconvénient au moyen d’un ministère qui est tout simplement composé d’hommes de race européenne. Ainsi, les affaires se traitent et se décident, en fait, entre la mission protestante et ses agents  ; le reste n’est là que pour la montre.\par
Quant au toi Kamehameha III, c’est, paraît-il, un prince de mérite. Il a, pour son compte, renoncé à se tatouer la figure, et, bien que n’ayant pas encore converti tous ses courtisans, il éprouve déjà la juste satisfaction de ne les plus voir tracer sur leurs fronts et leurs joues que d’assez légers dessins. Le gros de la nation, nobles de campagne et gens du peuple, persiste sur ce point, comme sur les autres, dans les vieilles idées. Toutefois des causes très nombreuses amènent chaque jour aux îles Sandwich un surcroît de population européenne. Le voisinage de la Californie fait du royaume hawaïen un point très intéressant pour la clairvoyante énergie de nos nations. Les baleiniers déserteurs et les matelots réfractaires de la marine militaire n’y sont plus les seuls colons de race blanche : des marchands, des spéculateurs, des aventuriers de toute espèce, accourent, y bâtissent des maisons et s’y fixent. La race indigène, envahie, va peu à peu se mélanger et disparaître. Je ne sais si le gouvernement représentatif et indépendant ne fera pas bientôt place à une simple administration déléguée, relevant de quelque grande puissance étrangère  ; ce dont je ne doute pas, c’est que les institutions importées finiront par s’établir solidement dans ce pays, et le jour de leur triomphe verra, synchronisme nécessaire, la ruine totale des naturels.\par
 À Saint-Domingue, l’indépendance est complète. Là, point de missionnaires exerçant une autorité voilée et absolue ; point de ministère étranger fonctionnant avec l’esprit européen : tout est abandonné aux inspirations de la population elle-même. Cette population, dans la partie espagnole, est composée de mulâtres. Je n’en parlerai pas. Ces gens paraissent imiter, tant bien que mal, ce que notre civilisation a de plus facile : ils tendent comme tous les métis, à se fondre dans la branche de leur généalogie qui leur fait le plus d’honneur ; ils sont donc susceptibles, jusqu’à un certain point, de mettre en pratique nos usages. Ce n’est pas chez eux qu’il faut étudier la question absolue. Passons donc les montagnes qui séparent la république dominicaine de l’État d’Haïti.\par
Nous nous trouvons là en face d’une société dont les institutions sont non seulement pareilles aux nôtres, mais encore dérivent des maximes les plus récentes de notre sagesse politique. Tout ce que, depuis soixante ans, le libéralisme le plus raffiné a fait proclamer dans les assemblées délibérantes de l’Europe, tout ce que les penseurs les plus amis de l’indépendance et de la dignité de l’homme ont pu écrire, toutes les déclarations de droits et de principes, ont trouvé leur écho sur les rives de l’Artibonite. Rien d’africain n’a survécu dans les lois écrites ; les souvenirs de la terre chamitique ont officiellement disparu des esprits ; jamais le langage officiel n’en a montre la trace ; les institutions, je le répète, sont complètement européennes. Voyons maintenant com­ment elles s’adaptent avec les mœurs.\par
Quel contraste ! Les mœurs ? on les voit aussi dépravées, aussi brutales, aussi féroces que dans le Dahomey ou le pays des Fellatahs. Le même amour barbare de la parure s’allie à la même indifférence pour le mérite de la forme ; le beau réside dans la couleur, et, pourvu qu’un vêtement soit d’un rouge éclatant et garni de faux or, le goût ne s’occupe guère des solutions de continuité de l’étoffe ; et, quant à la propreté, personne ne s’en soucie. Veut-on, dans ce pays-là, s’approcher d’un haut fonction­naire ? on est introduit près d’un grand nègre étendu à la renverse sur un banc de bois, la tête enveloppée d’un mauvais mouchoir déchiré et couverte d’un chapeau à cornes largement galonné d’or. Un sabre immense pend à côté de cet amas de membres ; l’habit brodé n’est pas accompagné d’un gilet ; le général a des pantoufles. L’interrogez-vous, cherchez-vous à pénétrer dans son esprit pour y apprécier la nature des idées qui l’occupent ? vous trouvez l’intelligence la plus inculte unie à l’orgueil le plus sauvage, qui n’a d’égal qu’une aussi profonde et incurable nonchalance. Si cet homme ouvre la bouche, il va vous débiter tous les lieux communs dont les journaux nous ont fatigués depuis un demi-siècle. Ce barbare les sait par cœur ; il a d’autres intérêts, des instincts très différents ; il n’a pas d’autres notions acquises. Il parle comme le baron d’Holbach, raisonne comme M. de Grimm, et, au fond, il n’a de sérieux souci que de mâcher du tabac, boire de l’alcool, éventrer ses ennemis et se concilier les sorciers. Le reste du temps, il dort.\par
L’État est partagé en deux fractions, que ne séparent pas des incompatibilités de doctrines, mais de peaux : les mulâtres se tiennent d’un côté, les nègres de l’autre. Aux mulâtres appartient, sans aucun doute, plus d’intelligence, un esprit plus ouvert à la conception. Je l’ai déjà fait remarquer pour les Dominicains : le sang européen a modifié la nature africaine, et ces hommes pourraient, fondus dans une masse blanche, et avec de bons modèles constamment sous les yeux, devenir ailleurs des citoyens utiles. Par malheur la suprématie du nombre et de la force appartient, pour le moment, aux nègres. Ceux-là, bien que leurs grands-pères, tout au plus, aient connu la terre d’Afrique, en subissent encore l’influence entière ; leur suprême joie, c’est la paresse ; leur suprême raison, c’est le meurtre. Entre les deux partis qui divisent l’île, la haine la plus intense n’a jamais cessé de régner. L’histoire d’Haïti, de la démocratique Haïti, n’est qu’une longue relation de massacres : massacres des mulâtres par les nègres, lorsque ceux-ci sont les plus forts, des nègres par les mulâtres, quand le pouvoir est aux mains de ces derniers. Les institutions, pour philanthropiques qu’elles se donnent, n’y peuvent rien ; elles dorment impuissantes sur le papier où l’on les a écrites ; ce qui règne sans frein, c’est le véritable esprit des populations. Conformément à une loi naturelle indiquée plus haut, la variété noire, appartenant à ces tribus humaines qui ne sont pas aptes à se civiliser, nourrit l’horreur la plus profonde pour toutes les autres races ; aussi voit-on les nègres d’Haïti repousser énergiquement les blancs et leur défendre l’entrée de leur territoire ; ils voudraient de même exclure les mulâtres, et visent à leur extermination. La haine de l’étranger est le principal mobile de la politique locale. Puis, en conséquence de la paresse organique de l’espèce, l’agriculture est annulée, l’industrie n’existe pas même de nom, le commerce se réduit de jour en jour, la misère, dans ses déplorables progrès, empêche la population de se reproduire, tandis que les guerres continuelles, les révoltes, les exécutions militaires, réussissent constamment à la diminuer. Le résultat inévitable et peu éloigné d’une telle situation sera de rendre désert un pays dont la fertilité et les ressources naturelles ont jadis enrichi des générations de plan­teurs, et d’abandonner aux chèvres sauvages les plaines fécondes, les magnifiques vallées, les mornes grandioses de la reine des Antilles \footnote{La colonie de Saint-Domingue, avant son émancipation, était un des lieux de la terre où la richesse et l’élégance des mœurs avaient poussé le plus loin leurs raffinements. Ce que la Havane est devenue en fait d’activité commerciale, Saint-Domingue le montrait avec surcroît. Les esclaves affranchis y ont mis bon ordre.}.\par
Je suppose le cas où les populations de ce malheureux pays auraient pu agir conformément à l’esprit des races dont elles sont issues, où, ne se trouvant pas sous le protectorat inévitable et l’impulsion de doctrines étrangères, elles auraient formé leur société tout à fait librement et en suivant leurs seuls instincts. Alors, il se serait fait, plus ou moins spontanément, mais jamais sans quelques violences, une séparation entre les gens des deux couleurs.\par
Les mulâtres auraient habité les bords de la mer, afin de se tenir toujours avec les Européens dans des rapports qu’ils recherchent. Sous la direction de ceux-ci, on les aurait vus marchands, courtiers surtout, avocats, médecins, resserrer des liens qui les flattent, se mélanger de plus en plus, s’améliorer graduellement, perdre, dans des proportions données, le caractère avec le sang africain.\par
Les nègres se seraient retirés dans l’intérieur, et ils y auraient formé de petites sociétés analogues à celles que créaient jadis les esclaves marrons à Saint-Domingue même, à la Martinique, à la Jamaïque et surtout à Cuba, dont le territoire étendu et les forêts profondes offrent des abris plus sûrs. Là, au milieu des productions si variées et si brillantes de la végétation antillienne, le noir américain, abondamment pourvu des moyens d’existence que prodigue, à si peu de frais, une terre opulente, serait revenu en toute liberté à cette organisation despotiquement patriarcale si naturelle à ceux de ses congénères que les vainqueurs musulmans de l’Afrique n’ont pas encore contraints. L’amour de l’isolement aurait été tout à la fois la cause et le résultat de ces institutions. Des tribus se formant seraient, au bout de peu de temps, devenues étrangères et hostiles les unes aux autres. Des guerres locales auraient été le seul événement politique des différents cantons, et l’île, sauvage, médiocrement peuplée, fort mal cultivée, aurait cependant conservé une double population, maintenant condamnée à disparaître, par suite de la funeste influence de lois et d’institutions sans rapports avec la structure de l’intelligence des nègres, avec leurs intérêts, avec leurs besoins.\par
Ces exemples de Saint-Domingue et des îles Sandwich sont assez concluants. Je ne puis cependant résister au désir de toucher encore, avant de quitter définitivement ce sujet, à un autre fait analogue et dont le caractère particulier prête une bien grande force à mon opinion. J’ai appelé en témoignage un État où les institutions, imposées par des prédicateurs protestants, ne sont qu’un calque assez puéril de l’organisation britanni­que ; ensuite j’ai parlé d’un gouvernement matériellement libre, mais intellectuellement lié à des théories européennes, et qui a dû mettre en pratique l’application de ces théories, d’où la mort s’ensuit pour les malheureuses populations haïtiennes. Voici maintenant un exemple d’une tout autre nature, qui m’est offert par les tentatives des pères jésuites pour civiliser les indigènes du Paraguay \footnote{Voir, à ce sujet, Prichard, d’Orbigny, A. de Hurnboldt, etc.}.\par
Ces missionnaires, par l’élévation de leur intelligence et la beauté de leur courage, ont excité l’admiration universelle ; et les ennemis les plus déclarés de leur ordre n’ont pas cru pouvoir leur refuser un ample tribut d’éloges. En effet, si des institutions issues d’un esprit étranger à une nation ont eu jamais quelques chances de succès, c’étaient assurément celles-là, fondées sur la puissance du sentiment religieux et appuyées de ce qu’un génie d’observation, aussi juste que fin, avait pu trouver d’idées d’appropriation. Les Pères s’étaient persuadés, opinion du reste fort répandue, que la barbarie est à la vie des peuples ce que l’enfance est à celle des individus, et que plus une nation se montre sauvage et inculte, plus elle est jeune.\par
Pour mener leurs néophytes à l’adolescence, ils les traitèrent donc comme des enfants, et leur firent un gouvernement despotique aussi ferme dans ses vues et volontés, que doux et affectueux dans ses formes. Les peuplades américaines ont, en général, des tendances républicaines, et la monarchie ou l’aristocratie, rares chez elles, ne s’y montrent jamais que très limitées. Les dispositions natives des Guaranis, auxquelles les jésuites venaient s’adresser, ne contrastaient pas, sur ce point, avec celles des autres indigènes. Toutefois, par une circonstance heureuse, ces peuples témoi­gnaient d’une intelligence relativement développée, d’un peu moins de férocité peut-être que certains de leurs voisins, et de quelque facilité à concevoir des besoins nouveaux. Cent vingt mille âmes environ furent réunies dans les villages des missions sous la conduite des Pères. Tout ce que l’expérience, l’étude journalière, la vive charité, appre­naient aux jésuites, portait profit ; on faisait d’incessants efforts pour hâter le succès sans le compromettre. Malgré tant de soins, on sentait cependant que ce n’était pas trop du pouvoir absolu pour contraindre les néophytes à persister dans la bonne voie, et l’on pouvait se convaincre, en maintes occasions, du peu de solidité réelle de l’édifice.\par
Quand les mesures du comte d’Aranda vinrent enlever au Paraguay ses pieux et habiles civilisateurs, on en reçut la plus triste et la plus complète démonstration. Les Guaranis, privés de leurs guides spirituels, refusèrent toute confiance aux chefs laïques envoyés par la couronne d’Espagne. Ils ne montrèrent aucune attache à leurs nouvelles institutions. Le goût de la vie sauvage les reprit, et aujourd’hui, à l’exception de trente-sept petits villages qui végètent encore sur les bords du Parana, du Paraguay et de l’Uruguay, villages qui contiennent certainement un noyau de population métisse, tout le reste est retourné aux forêts et y vit dans un état aussi sauvage que le sont à l’occident les tribus de même souche, Guaranis et Cirionos. Les fuyards ont repris, je ne dis pas leurs vieilles coutumes dans toute leur pureté, mais du moins des coutumes à peine rajeunies et qui en découlent directement, et cela parce qu’il n’est donné à aucune race humaine d’être infidèle à ses instincts, ni d’abandonner le sentier sur lequel Dieu l’a mise. On peut croire que, si les jésuites avaient continué à régir leurs missions du Paraguay, leurs efforts, servis par le temps, auraient amené des succès meilleurs je l’admets ; mais à cette condition unique, toujours la même, que des groupes de population européenne seraient venus peu à peu, sous la protection de leur dictature, s’établir dans le pays, se seraient mêlés avec les natifs, auraient d’abord modifié, puis complètement changé le sang, et, à ces conditions, il se serait formé dans ces contrées un État portant peut-être un nom aborigène, se glorifiant peut-être de descendre d’ancê­tres autochtones, mais par le fait, mais dans la vérité, aussi européen que les institutions qui l’auraient régi.\par
Voilà ce que j’avais à dire sur les rapports des institutions avec les races.
\section[{I.4. Dans le progrès ou la stagnation, les peuples sont indépendants des lieux qu’ils habitent.}]{I.4. \\
Dans le progrès ou la stagnation, les peuples sont indépendants des lieux qu’ils habitent.}
\noindent Il est impossible de ne pas tenir quelque compte de l’influence accordée par plusieurs savants aux climats, à la nature du soi, à la disposition topographique sur le développement des peuples ; et, bien qu’à propos de la doctrine des milieux, j’y aie touché en passant, ce serait laisser une véritable lacune que de ne pas en parler à fond.\par
On est généralement porté à croire qu’une nation établie sous un ciel tempéré, non pas assez brûlant pour énerver les hommes, non pas assez froid pour rendre le sol improductif, au bord de grands fleuves, routes larges et mobiles, dans des plaines et des vallées propres à plusieurs genres de culture, au pied de montagnes dont le sein opulent est gorgé de métaux, que cette nation, ainsi aidée par la nature, sera bien promptement amenée à quitter la barbarie, et, sans faute, se civilisera. D’autre part, et par une conséquence de ce raisonnement, on admet sans peine que des tribus brûlées par le soleil ou engourdies sur les glaces éternelles, n’ayant d’autre territoire que des rochers stériles, seront beaucoup plus exposées à rester dans l’état de barbarie. Alors il va sans dire que, dans cette hypothèse, l’humanité ne serait perfectible qu’à l’aide du secours de la nature matérielle, et que toute sa valeur et sa grandeur existeraient en germe hors d’elle-même. Pour assez spécieuse, au premier aspect, que semble cette opinion, elle ne concorde sur aucun point avec les réalités nombreuses que l’obser­vation procure.\par
Nuls pays certainement ne sont plus fertiles, nuls climats plus doux que ceux des différentes contrées de l’Amérique. Les grands fleuves y abondent, les golfes, les baies, les havres y sont vastes, profonds, magnifiques, multipliés ; les métaux précieux s’y trouvent à fleur de terre ; la nature végétale y prodigue presque spontanément les moyens d’existence les plus abondants et les plus variés, tandis que la faune, riche en espèces alimentaires, présente des ressources plus substantielles encore. Et pourtant la plus grande partie de ces heureuses contrées est parcourue, depuis des séries de siècles, par des peuplades restées étrangères à la plus médiocre exploitation de tant de trésors.\par
Plusieurs ont été sur la voie de mieux faire. Une maigre culture, un travail barbare du minerai, sont des faits qu’on observe dans plus d’un endroit. Quelques arts utiles, exercés avec une sorte de talent, surprennent encore le voyageur. Mais tout cela, en définitive, est très humble et ne forme pas un ensemble, un faisceau dont une civilisation quelconque soit jamais sortie. Certainement il a existé, à des époques fort lointaines, dans la contrée étendue entre le lac Érié et le golfe du Mexique, depuis le Missouri jusqu’aux Montagnes Rocheuses, une nation qui a laissé des traces remar­quables de sa présence. Les restes de constructions, les inscriptions gravées sur des rochers, les tumulus \footnote{La construction très particulière de ces tumulus, et les nombreux ustensiles et instruments qu’ils recèlent, occupent beaucoup, en ce moment, la perspicacité et le talent des antiquaires américains. J’aurai occasion, dans le quatrième volume de cet ouvrage, d’exprimer une opinion sur la valeur de ces reliques, au point de vue de la civilisation ; pour le moment, je me bornerai à en dire que leur excessive antiquité est impossible à révoquer en doute. M. Squier est parfaitement fondé à en trouver une preuve dans ce fait seul, que les squelettes découverts dans les tumulus tombent en poussière au moindre contact de l’air, bien que les conditions, quant à la qualité du sol, soient des meilleures, tandis que les corps enterrés sous les cromlechs bretons, et qui ont au moins 1 800 ans de sépulture, sont parfaitement solides. On peut donc concevoir aisément qu’entre ces très anciens possesseurs du sol de l’Amérique et les tribus Lenni-Lénapés et autres, il n’y ait pas de rapports. Avant de clore cette note, je ne puis me dispenser de louer l’industrieuse habileté que déploient les savants américains dans l’étude des antiquités de leur grand continent. Fort embarrassés par l’excessive fragilité des crânes exhumés, ils ont imaginé, après plusieurs autres essais infructueux, de couler dans les cadavres, avec des précautions inouïes, une préparation bitumineuse qui, en se solidifiant aussitôt, préserve les ossements de la dissolution. Il paraît que ce procédé, fort délicat à employer et qui demande autant d’adresse que de promptitude, obtient généralement un entier succès.}, les momies indiquent une culture intellectuelle avancée. Mais rien ne prouve qu’entre cette mystérieuse nation et les peuplades errant aujourd’hui sur ses tombes, il y ait une parenté bien proche. Dans tous les cas, si, par suite d’un lien naturel quelconque, ou d’une initiation d’esclaves, les aborigènes actuels tiennent des anciens maîtres du pays la première notion de ces arts qu’ils pratiquent à l’état élémentaire, on ne pourrait qu’être frappé davantage de l’impossibilité où ils se sont trouvés de perfectionner ce qu’on leur avait appris, et je verrais là un motif de plus pour rester persuadé que le premier peuple venu, placé dans les circonstances géographiques les plus favorables, n’est pas destiné par cela même à se civiliser.\par
Au contraire, il y a, entre l’aptitude d’un climat et d’un pays à servir les besoins de l’homme et le fait même de la civilisation, une indépendance complète. L’Inde est une contrée qu’il a fallu fertiliser, l’Égypte de même. Voilà deux centres bien célèbres de la culture et du perfectionnement humains. La Chine, à côté de la fécondité de certaines de ses parties, a présenté, dans d’autres, des difficultés très laborieuses à vaincre. Les premiers événements y sont des combats contre les fleuves ; les premiers bienfaits des antiques empereurs consistent en ouvertures de canaux, en dessèchements de marais. Dans la contrée mésopotamique de l’Euphrate et du Tigre, théâtre de la splendeur des premiers États assyriens, territoire sanctifié par la majesté des plus sacrés souvenirs, dans ces régions où le froment, dit-on, croît spontanément, le sol est cependant si peu productif par lui-même, que de vastes et courageux travaux d’irrigation ont pu seuls le rendre propre à nourrit les hommes. Maintenant que les canaux sont détruits, comblés ou encombrés, la stérilité a repris ses droits. Je suis donc très porté à croire que la nature n’avait pas autant favorisé ces régions qu’on le pense d’ordinaire. Toutefois je ne discuterai pas sur ce point. J’admets que la Chine, l’Égypte, l’Inde et l’Assyrie aient été des lieux complètement appropriés à l’établissement de grands empires et au dévelop­pement de puissantes civilisations ; j’accorde que ces lieux aient réuni les meilleures conditions de prospérité. On l’avouera aussi ces conditions étaient de telle nature, que, pour en profiter, il était indispensable d’avoir atteint préalablement, par d’autres voies, un haut degré de perfectionnement social. Ainsi, pour que le commerce pût s’emparer des grands cours d’eau, il fallait que l’industrie, ou pour le moins l’agriculture, existassent déjà, et l’attrait sur les peuples voisins n’aurait pas eu lieu avant que des villes et des marchés ne fussent bâtis et enrichis de longue main. Les grands avantages départis à la Chine, à l’Inde et à l’Assyrie supposent donc, chez les peuples qui en ont tiré bon parti, une véritable vocation intellectuelle et même une civilisation antérieure au jour où l’exploitation de ces avantages put commencer. Mais quittons les régions spécialement favorisées, et regardons ailleurs.\par
Lorsque les Phéniciens, dans leur migration, vinrent de Tylos, ou de quelque autre endroit du sud-est que l’on voudra, que trouvèrent-ils dans le canton de Syrie où ils se fixèrent ? Une côte aride, rocailleuse, serrée étroitement entre la mer et des chaînes de rochers qui semblaient devoir rester à tout jamais stériles. Un territoire si misérable contraignait la nation à ne jamais s’étendre, car, de tous côtés, elle se trouvait enserrée dans une ceinture de montagnes. Et cependant ce lieu, qui devait être une prison, devint, grâce au génie industrieux du peuple qui l’habita, un nid de temples et de palais. Les Phéniciens, condamnés pour toujours à n’être que de grossiers ichtyophages, ou tout au plus de misérables pirates, furent pirates à la vérité, mais grandement, et, de plus, marchands hardis et habiles, spéculateurs audacieux et heureux. Bon ! dira quelque contradicteur, nécessité est mère d’invention ; si les fondateurs de Tyr et de Sidon avaient habité les plaines de Damas, contents des produits de l’agriculture, ils n’auraient peut-être jamais été un peuple illustre. La misère les a aiguillonnés, la misère a éveillé leur génie.\par
Et pourquoi donc n’éveille-t-elle pas celui de tant de tribus africaines, américaines, océaniennes, placées dans des circonstances analogues ? Pourquoi voyons-nous les Kabyles du Maroc, race ancienne et qui a eu, bien certainement, tout le temps néces­saire pour la réflexion, et, chose plus surprenante encore, toutes les incitations possibles à la simple imitation, n’avoir jamais conçu une idée plus féconde, pour adoucir son sort malheureux, que le pur et simple brigandage maritime ? Pourquoi, dans cet archipel des Indes, qui semble créé pour le commerce, dans ces îles océaniennes, qui peuvent si aisément communiquer l’une avec l’autre, les relations pacifiquement fruc­tueuses sont-elles presque absolument dans les mains des races étrangères, chinoise, malaise et arabe ? et là où des peuples à demi indigènes, où des nations métisses ont pu s’en emparer, pourquoi l’activité diminue-t-elle ? Pourquoi la circulation n’a-t-elle lieu que d’après des données de plus en plus élémentaires ? C’est qu’en vérité, pour qu’un État commercial s’établisse sur une côte ou sur une île quelconque, il faut quelque chose de plus que la mer ouverte, que les excitations nées de la stérilité du sol, que même les leçons de l’expérience d’autrui : il faut, dans l’esprit du naturel de cette côte ou de cette île, l’aptitude spéciale qui seule l’amènera à profiter des instruments de travail et de succès placés à sa portée.\par
Mais je ne me bornerai pas à montrer qu’une situation géographique, déclarée convenable parce qu’elle est fertile, ou, précisément encore, parce qu’elle ne l’est pas, ne donne pas aux nations leur valeur sociale : il faut encore bien établir que cette valeur sociale est tout à fait indépendante des circonstances matérielles environnantes. Je citerai les Arméniens, renfermés dans leurs montagnes, dans ces mêmes montagnes où tant d’autres peuples vivent et meurent barbares de génération en génération, parve­nant, dès une antiquité très reculée, à une civilisation assez haute. Ces régions pourtant étaient presque closes, sans fertilité remarquable, sans communication avec la mer.\par
Les Juifs se trouvaient dans une position analogue, entourés de tribus parlant des dialectes d’une langue patente de la leur, et dont la plupart leur tenaient d’assez près par le sang ; ils devancèrent pourtant tous ces groupes. On les vit guerriers, agricul­teurs, commerçants ; on les vit, sous ce gouvernement singulièrement compliqué, où la monarchie, la théocratie, le pouvoir patriarcal des chefs de famille et la puissance démocratique du peuple, représentée par les assemblées et les prophètes, s’équili­braient d’une manière bien bizarre, traverser de longs siècles de prospérité et de gloire, et vaincre, par un système d’émigration des plus intelligents, les difficultés qu’oppo­saient à leur expansion les limites étroites de leur domaine. Et qu’était-ce encore que ce domaine ? Les voyageurs modernes savent au prix de quels efforts savants les agronomes israélites en entretenaient la factice fécondité. Depuis que cette race choisie n’habite plus ses montagnes et ses plaines, le puits où buvaient les troupeaux de Jacob est comblé par les sables, la vigne de Naboth a été envahie par le désert, tout comme l’emplacement du palais d’Achab par les ronces. Et dans ce misérable coin du monde, que furent les Juifs ? Je le répète, un peuple habile en tout ce qu’il entreprit, un peuple libre, un peuple fort, un peuple intelligent, et qui, avant de perdre bravement, les armes à la main, le titre de nation indépendante, avait fourni au monde presque autant de docteurs que de marchands \footnote{Salvador, \emph{Histoire des juifs}. In-8°. Paris.}.\par
Les Grecs, les Grecs eux-mêmes, étaient loin d’avoir à se louer en tout des circons­tances géographiques. Leur pays n’était, en bien des parties, qu’une terre misérable. Si l’Arcadie fut un pays aimé des pasteurs, si la Béotie se déclara chère à Cérès et à Triptolème, l’Arcadie et la Béotie jouent un rôle bien mince dans l’histoire hellénique. La riche Corinthe elle-même, la ville favorite de Plutus et de Vénus Mélanis, ne brille ici qu’au second rang. À qui revient la gloire ? à Athènes, dont une poussière blanchâtre couvrait la campagne et les maigres oliviers ; à Athènes, qui, pour commerce principal, vendait des statues et des livres ; puis à Sparte, enterrée dans une vallée étroite, au fond des entassements de rocs où la victoire allait la chercher.\par
Et Rome, dans le pauvre canton du Latium où la mirent ses fondateurs, au bord de ce petit Tibre, qui venait déboucher sur une côte presque inconnue, que jamais vaisseau phénicien ou grec ne touchait que par hasard, est-ce par sa disposition topographique qu’elle est devenue la maîtresse du monde ? Mais, aussitôt que le monde obéit aux enseignes romaines, la politique trouva sa métropole mal placée, et la ville éternelle commença la longue série de ses affronts. Les premiers empereurs, ayant surtout les yeux tournés vers la Grèce, y résidèrent presque toujours. Tibère, en Italie, se tenait à Captée, entre les deux moitiés de son univers. Ses successeurs allaient à Antioche. Quelques-uns, préoccupés des affaires gauloises, montèrent jusqu’à Trèves. Enfin un décret final enleva à Rome le titre même de capitale pour le donner à Milan. Que si les Romains ont fait parler d’eux dans le monde, c’est bien certainement malgré la position du district d’où sortaient leurs premières armées, et non pas à cause de cette position.\par
En descendant aux temps modernes, la multitude des faits dont je puis m’étayer m’embarrasse. Je vois la prospérité quitter tout à fait les côtes méditerranéennes, preuve sans réplique qu’elle ne leur était pas attachée. Les grandes cités commerçantes du moyen âge naissent là où nul théoricien des époques précédentes n’auraient été les bâtir. Novogorod s’élève dans un pays glacé ; Brême sur une côte presque aussi froide. Les villes hanséatiques du centre de l’Allemagne se fondent au milieu de pays qui s’éveillent à peine ; Venise apparaît au fond d’un golfe profond. La prépondérance politique brille dans des lieux à peine aperçus jadis. En France, c’est au nord de la Loire et presque au delà de la Seine que réside la force. Lyon, Toulouse, Narbonne, Marseille, Bordeaux, tombent du haut rang où les avait portées le choix des Romains. C’est Paris qui devient la cité importante, Paris, une bourgade trop éloignée de la mer quand il s’agit du commerce, et qui en sera trop près quand viendront les barques normandes. En Italie, des villes, jadis du dernier ordre, priment la cité des papes ; Ravenne s’éveille au fond de ses marais, Amalfi est longtemps puissante. Je note, en passant, que le hasard n’a eu aucune part à tous ces revirements, que tous s’expliquent par la présence sur le point donné d’une race victorieuse ou prépondérante. Je veux dire que ce n’était pas le lieu qui faisait la valeur de la nation, qui jamais l’a faite, qui la fera jamais : au contraire, c’était la nation qui donnait, a donné et donnera au territoire sa valeur économique, morale et politique.\par
Afin d’être aussi clair que possible, j’ajouterai cependant que ma pensée n’est pas de nier l’importance de la situation pour certaines villes, soit entrepôts, soit ports de mer, soit capitales. Les observations que l’on a faites, au sujet de Constantinople et d’Alexandrie notamment, sont incontestables. Il est certain qu’il existe sur le globe différents points qu’on peut appeler les clefs du monde, et ainsi l’on conçoit que, dans le cas du percement de l’isthme de Panama, la puissance qui posséderait la ville encore à construire sur ce canal hypothétique aurait un grand rôle à jouer dans les affaires de l’univers. Mais ce rôle, une nation le joue bien, le joue mal, ou même ne le joue pas du tout, suivant ce qu’elle vaut. Agrandissez Chagres, et faites que les deux mers s’unissent sous ses murs ; puis soyez libre de peupler la ville d’une colonie à votre gré : le choix auquel vous vous arrêterez déterminera l’avenir de la cité nouvelle. Que la race soit vraiment digne de la haute fortune à laquelle elle aura été appelée, si l’emplacement de Chagres n’est pas précisément le plus propre à développer tous les avantages de l’union des deux Océans, cette population le quittera et ira ailleurs déployer en toute liberté les splendeurs de son sort \footnote{ \noindent Voici, sur le sujet débattu dans ce chapitre, l’opinion, un peu durement exprimée, d’un savant historien et philologue :\par
 « Un assez grand nombre d’écrivains s’est laissé persuader que le pays faisait le peuple ; que « les Bavarois ou les Saxons avaient été prédestinés par la nature de leur sol à devenir ce « qu’ils sont aujourd’hui ; que le christianisme protestant ne convenait pas aux régions du « sud ; que le catholicisme n’allait pas à celles du nord, et autres choses semblables. Des « hommes qui interprètent l’histoire d’après leurs maigres connaissances, ou même leurs « cœurs étroits et leurs esprits myopes, voudraient bien aussi établir que la nation qui fait « l’objet de nos récits (les juifs) a possédé telle ou telle qualité, bien ou mal comprise, pour « avoir habité la Palestine et non pas l’Inde ou la Grèce. Mais si ces grands docteurs, habiles « à tout prouver, voulaient réfléchir que le sol de la terre sainte a porté dans son espace « resserré les religions et les idées des peuples les plus différents, et qu’entre ces peuples si « variés et leurs héritiers actuels, il existe encore des nuances à l’infini, bien que la contrée « soit restée la même, ils verraient alors combien peu le territoire matériel a d’influence sur « le caractère et la civilisation d’un peuple. »\par
 
\bibl{(Ewald, \emph{Geschichte des Volkes Israël}, t. I, p. 259)}
}.
\section[{I.7. Le christianisme ne crée pas et ne transforme pas l’aptitude civilisatrice.}]{I.7. \\
Le christianisme ne crée pas et ne transforme pas l’aptitude civilisatrice.}
\noindent Après les objections tirées des institutions, des climats, il en vient une qu’à vrai dire, j’aurais dû placer avant toutes les autres, non pas que je la juge plus forte, mais pour la révérence naturellement inspirée par le fait sur lequel elle s’appuie. En adoptant comme justes les conclusions qui précèdent, deux affirmations deviennent de plus en plus évidentes : c’est, d’abord, que la plupart des races humaines sont inaptes à se civiliser jamais, à moins qu’elles ne se mélangent ; c’est, ensuite, que non seulement ces races ne possèdent pas le ressort intérieur déclaré nécessaire pour les pousser en avant sur l’échelle du perfectionnement, mais encore que tout agent extérieur est impuissant à féconder leur stérilité organique, bien que cet agent puisse être d’ailleurs très énergique. Ici l’on demandera, sans doute, si le christianisme doit briller en vain pour des nations entières ? s’il est des peuples condamnés à ne jamais le connaître ?\par
Certains auteurs ont répondu affirmativement. Se mettant sans scrupule en contradiction avec la promesse évangélique, ils ont nié le caractère le plus spécial de la loi nouvelle, qui est précisément d’être accessible à l’universalité des hommes. Une telle opinion reproduisait la formule étroite des Hébreux. C’était y rentrer par une porte un peu plus large que celle de l’ancienne Alliance ; néanmoins c’était y rentrer. Je ne sens nulle disposition à suivre les partisans de cette idée condamnée par l’Église, et n’éprouve pas la moindre difficulté à reconnaître pleinement que toutes les races humaines sont douées d’une égale capacité à entrer dans le sein de la communion chrétienne. Sur ce point-là, pas d’empêchement originel, pas d’entraves dans la nature des races ; leurs inégalités n’y font rien. Les religions ne sont pas, comme on a voulu le prétendre, parquées par zones sur la surface du globe avec leurs sectateurs. Il n’est pas vrai que, de tel degré du méridien à tel autre, le christianisme doive dominer, tandis qu’à dater de telle limite, l’islamisme prendra l’empire pour le garder jusqu’à la frontière infranchissable où il devra le remettre au bouddhisme ou au brahmanisme, tandis que les chamanistes, les fétichistes se partageront ce qui restera du monde.\par
Les chrétiens sont répandus dans toutes les latitudes et sous tous les climats. La statistique, imparfaite sans doute, mais probable en ses données, nous les montre en grand nombre, Mongols errant dans les plaines de la haute Asie, sauvages chassant sur les plateaux des Cordillères, Esquimaux pêchant dans les glaces du pôle arctique, enfin Chinois et japonais mourant sous le fouet des persécuteurs. L’observation ne permet plus sur cette question le plus léger doute. Mais la même observation ne permet pas non plus de confondre, comme on le fait journellement, le christianisme, l’aptitude universelle des hommes à en reconnaître les vérités, à en pratiquer les préceptes, avec la faculté, toute différente, d’un tout autre ordre, d’une tout autre nature, qui porte telle famille humaine, à l’exclusion de telles autres, à comprendre les nécessités purement terrestres du perfectionnement social, et à savoir en préparer et en traverser les phases, pour s’élever à l’état que nous appelons civilisation, état dont les degrés marquent les rapports d’inégalité des races entre elles.\par
On a prétendu, à tort bien certainement, dans le dernier siècle, que la doctrine du renoncement, qui constitue une partie capitale du christianisme, était, de sa nature, très opposée au développement social, et que des gens dont le suprême mérite doit être de ne rien estimer ici-bas, et d’avoir toujours les yeux fixés et les désirs tendus vers la Jérusalem céleste, ne sont guère propres à faire progresser les intérêts de ce monde. L’imperfection humaine se charge de rétorquer l’argument. Il n’a jamais été sérieusement à craindre que l’humanité renonçât aux choses du siècle, et, si expresses que fussent à cet égard les recommandations et les conseils, on peut dire que, luttant contre un courant reconnu irrésistible, on demandait beaucoup à cette seule fin d’obtenir un peu. En outre, les préceptes chrétiens sont un grand véhicule social, en ce sens qu’ils adoucissent les mœurs, facilitent les rapports par la charité, condamnent toute violence, forcent d’en appeler à la seule puissance du raisonnement, et réclament ainsi pour l’âme une plénitude d’autorité qui, dans mille applications, tourne au bénéfice bien entendu de la chair. Puis, par la nature toute métaphysique et intellectuelle de ses dogmes, la religion appelle l’esprit à s’élever, tandis que, par la pureté de sa morale, elle tend à le détacher d’une foule de faiblesses et de vices corrosifs, dangereux pour le progrès des intérêts matériels. Contrairement donc aux philosophes du dix-huitième siècle, on est fondé à accorder au christianisme l’épithète de civilisateur : mais il y faut de la mesure, et cette donnée trop amplifiée conduirait à des erreurs profondes.\par
Le christianisme est civilisateur en tant qu’il rend l’homme plus réfléchi et plus doux ; toutefois il ne l’est qu’indirectement, car cette douceur et ce développement de l’intelligence, il n’a pas pour but de les appliquer aux choses périssables, et partout on le voit se contenter de l’état social où il trouve ses néophytes, quelque imparfait que soit cet état. Pourvu qu’il en puisse élaguer ce qui nuit à la santé de l’âme, le reste ne lui importe en rien. Il laisse les Chinois avec leurs robes, les Esquimaux avec leurs fourrures, les premiers mangeant du riz, les seconds du lard de baleine, absolument comme il les a trouvés, et il n’attache aucune importance à ce qu’ils adoptent un autre genre d’existence. Si l’état de ces gens comporte une amélioration conséquente à lui-même, le christianisme tendra certainement à l’amener ; mais il ne changera pas du tout au tout les habitudes qu’il aura d’abord rencontrées et ne forcera pas le passage d’une civilisation à une autre, car il n’en a adopté aucune ; il se sert de toutes, et est au-dessus de toutes. Les faits et les preuves abondent : je vais en parler ; mais, auparavant, qu’il me soit permis de le confesser, je n’ai jamais compris cette doctrine toute moderne qui consiste à identifier tellement la loi du Christ avec les intérêts de ce monde, qu’on en fasse sortir un prétendu ordre de choses appelé la \emph{civilisation chrétienne.}\par
Il y a indubitablement une civilisation païenne, une civilisation brahmanique, bouddhique, judaïque. Il a existé, il existe des sociétés dont la religion est la base, a donné la forme, composé les lois, réglé les devoirs civils, marqué les limites, indiqué les hostilités ; des sociétés qui ne subsistent que sur les prescriptions plus ou moins larges d’une formule théocratique, et qu’on ne peut pas imaginer vivantes sans leur foi et leurs rites, comme les rites et la foi ne sont pas possibles non plus sans le peuple qu’ils ont formé. Toute l’antiquité a plus ou moins vécu sur cette règle. La tolérance légale, invention de la politique romaine, et le vaste système d’assimilation et de fusion des cultes, œuvre d’une théologie de décadence, furent, pour le paganisme, les fruits des époques dernières. Mais, tant qu’il fut jeune et fort, autant de villes, autant de Jupiters, de Mercures, de Vénus différents, et le dieu, jaloux, bien autrement que celui des Juifs et plus exclusif encore, ne reconnaissait, dans ce monde et dans l’autre, que ses concitoyens. Ainsi chaque civilisation de ce genre se forme et grandit sous l’égide d’une divinité, d’une religion particulière. Le culte et l’État s’y sont unis d’une façon si étroite et si inséparable, qu’ils se trouvent également responsables du mal et du bien. Que l’on reconnaisse donc à Carthage les traces politiques du culte de l’Hercule tyrien, je crois qu’avec vérité l’on pourra confondre l’action de la doctrine prêchée par les prêtres avec la politique des suffètes et la direction du développement social. Je ne doute pas non plus que l’Anubis à tête de chien, l’Isis Neith et les Ibis n’aient appris aux hommes de la vallée du Nil tout ce qu’ils ont su et pratiqué ; mais la plus grande nouveauté que le christianisme ait apportée dans le monde, c’est précisément d’agir d’une manière tout opposée aux religions précédentes. Elles avaient leurs peuples, il n’eut pas le sien : il ne choisit personne, il s’adressa à tout le monde, et non seulement aux riches comme aux pauvres, mais tout d’abord il reçut de l’Esprit-Saint la langue de chacun \footnote{Act. Apost., II, 4, 8, 9, 10, 11.}, afin de parler à chacun l’idiome de son pays et d’annoncer la foi avec les idées et au moyen des images les plus compréhensibles pour chaque nation. Il ne venait pas changer l’extérieur de l’homme, le monde matériel, il venait apprendre à le mépriser. Il ne prétendait toucher qu’à l’être intérieur. Un livre apocryphe, vénérable par son antiquité, a dit : « Que le fort ne tire point vanité de sa force, ni le riche de ses richesses ; mais celui qui veut être glorifié se glorifie dans le Seigneur \footnote{\emph{Évangiles apocryphes. Histoire de joseph le Charpentier}, chap. I. In-12. Paris, 1849.}. » Force, richesse, puissance mondaine, moyens de l’acquérir, tout cela ne compte pas pour notre loi. Aucune civilisation, de quelque genre qu’elle soit, n’appela jamais son amour ni n’excita ses dédains, et c’est pour cette rare impartialité, et uniquement par les effets qui en devaient sortir, que cette loi put s’appeler avec raison \emph{catholique}, universelle, car elle n’appartient en propre à aucune civilisation, elle n’est venue préconiser exclusivement aucune forme d’existence terrestre, elle n’en repousse aucune et veut les épurer toutes.\par
Les preuves de cette indifférence pour les formes extérieures de la vie sociale, pour la vie sociale elle-même, remplissent les livres canoniques d’abord, puis les écrits des Pères, puis les relations des missionnaires, depuis l’époque la plus reculée jusqu’au jour présent. Pourvu que, dans un homme quelconque, la croyance pénètre, et que, dans les actions de sa vie, cette créature tende à ne rien faire qui puisse transgresser les prescriptions religieuses, tout le reste est indifférent aux yeux de la foi. Qu’importent, dans un converti, la forme de sa maison, la coupe et la matière de ses vêtements, les règles de son gouvernement, la mesure de despotisme ou de liberté qui anime ses institutions publiques ? Pêcheur, chasseur, laboureur, navigateur, guerrier, qu’importe ? Est-il, dans ces modes divers de l’existence matérielle, rien qui puisse empêcher l’homme, je dis l’homme de quelque race qu’il soit issu, Anglais, Turc, Sibérien, Américain, Hottentot, rien qui puisse l’empêcher d’ouvrir les yeux à la lumière chrétienne ? Absolument quoi que ce soit ; et, ce résultat une fois obtenu, tout le reste compte peu. Le sauvage Galla est susceptible de devenir, en restant Galla, un croyant aussi parfait, un élu aussi pur que le plus saint prélat d’Europe. Voilà la supériorité saillante du christianisme, ce qui lui donne son principal caractère de \emph{grâce.} Il ne faut pas le lui ôter simplement pour complaire à une idée favorite de notre temps et de nos pays, qui est de chercher partout, même dans les choses les plus saintes, un côté matériellement utile.\par
Depuis dix-huit cents ans qu’existe l’Église, elle a converti bien des nations, et chez toutes elle a laissé régner, sans l’attaquer jamais, l’état politique qu’elle avait trouvé. Son début, vis-à-vis du monde antique, fut de protester qu’elle ne voulait toucher en rien à la forme extérieure de la société. On lui a même reproché, à l’occasion, un excès de tolérance à cet égard. J’en veux pour preuve l’affaire des jésuites dans la question des cérémonies chinoises. Ce qu’on ne voit pas, c’est qu’elle ait jamais fourni au monde un type unique de civilisation auquel elle ait prétendu que ses croyants dussent se rattacher. Elle s’accommode de tout, même de la hutte la plus grossière, et là où il se rencontre un sauvage assez stupide pour ne pas vouloir comprendre l’utilité d’un abri, il se trouve également un missionnaire assez dévoué pour s’asseoir à côté de lui sur la roche dure, et ne penser qu’à faire pénétrer dans son âme les notions essentielles du salut. Le christianisme n’est donc pas civilisateur comme nous l’entendons d’ordinaire ; il peut donc être adopté par les races les plus diverses sans heurter leurs aptitudes spéciales, ni leur demander rien qui dépasse la limite de leurs facultés.\par
Je viens de dire plus haut qu’il élevait l’âme par la sublimité de ses dogmes, et qu’il agrandissait l’esprit par leur subtilité. Oui, dans la mesure où l’âme et l’esprit auxquels il s’adresse sont susceptibles de s’élever et de s’agrandir. Sa mission n’est pas de répandre le don du génie ni de fournir des idées à qui en manque. Ni le génie ni les idées ne sont nécessaires pour le salut. Le christianisme a déclaré, au contraire, qu’il préférait aux forts les petits et les humbles. Il ne donne que ce qu’il veut qu’on lui rende. Il féconde, il ne crée pas ; il soutient, il appuie, il n’enlève pas ; il prend l’homme comme il est, et seulement l’aide à marcher : si l’homme est boiteux, il ne lui demande pas de courir. Ainsi, j’ouvrirai la vie des saints : y trouverai-je surtout des savants ? Non, certes. La foule des bienheureux dont l’Église honore le nom et la mémoire se compose surtout d’individualités précieuses par leurs vertus ou leur dévouement, mais qui, pleines de génie dans les choses du ciel, en manquaient pour celles de la terre; et quand on me montre sainte Rose de Lima vénérée comme saint Bernard, sainte Zite implorée comme sainte Thérèse, et tous les saints anglo-saxons, la plupart des moines irlandais, et les solitaires grossiers de la Thébaïde d’Égypte, et ces légions de martyrs qui, du sein de la populace terrestre, ont dû à un éclair de courage et de dévouement de briller éternelle­ment dans la gloire, respectés à l’égal des plus habiles défenseurs du dogme, des plus savants panégyristes de la foi, je me trouve autorisé à répéter que le christianisme n’est pas civilisateur dans le sens étroit et mondain que nous devons attacher à ce mot, et que, puisqu’il ne demande à chaque homme que ce que chacun a reçu, il ne demande aussi à chaque race que ce dont elle est capable, et ne se charge pas de lui assigner, dans l’assemblée politique des peuples de l’univers, un rang plus élevé que celui où ses facultés lui donnent le droit de s’asseoir. Par conséquent, je n’admets pas du tout l’argument égalitaire qui confond la possibilité d’adopter la foi chrétienne avec l’aptitude à un développement intellectuel indéfini. Je vois la plus grande partie des tribus de l’Amérique méridionale amenées depuis des siècles au giron de l’Église, et cependant toujours sauvages, toujours inintelligentes de la civilisation européenne qui se pratique sous leurs yeux. Je ne suis pas surpris que, dans le nord du nouveau continent, les Cherokees aient été en grande partie convertis par des ministres méthodistes ; mais je le serais beaucoup si cette peuplade venait jamais à former, en restant pure, bien entendu, un des États de la confédération américaine, et à exercer quelque influence dans le congrès. Je trouve encore tout naturel que les luthériens danois et les Moraves aient ouvert les yeux des Esquimaux à la lumière religieuse ; mais je ne le trouve pas moins que leurs néophytes soient restés d’ailleurs absolument dans le même état social où ils végétaient auparavant. Enfin, pour terminer, c’est, à mes yeux, un fait simple et naturel que de savoir les Lapons suédois dans l’état de barbarie de leurs ancêtres, bien que, depuis des siècles, les doctrines salutaires de l’Évangile leur aient été apportées. Je crois sincèrement que tous ces peuples pourront produire, ont produit peut-être déjà, des personnes remarquables par leur piété et la pureté de leurs mœurs, mais je ne m’attends pas à en voir sortir jamais de savants théologiens, des militaires intelligents, des mathématiciens habiles, des artistes de mérite, en un mot cette élite d’esprits raffinés dont le nombre et la succession perpétuelle font la force et la fécondité des races dominatrices, bien plus encore que la rare apparition de ces génies hors ligne qui ne sont suivis par les peuples, dans les voies où ils s’engagent, que si ces peuples sont eux-mêmes conformés de manière à pouvoir les comprendre et avancer sous leur conduite. Il est donc nécessaire et juste de désintéresser entièrement le christianisme dans la question. Si toutes les races sont également capables de le reconnaître et de goûter ses bienfaits, il ne s’est pas donné la mission de les rendre pareilles entre elles : son royaume, on peut le dire hardiment, dans le sens dont il s’agit ici, n’est pas de ce monde.\par
Malgré ce qui précède, je crains que quelques personnes, trop accoutumées, par une participation naturelle aux idées du temps, à juger les mérites du christianisme à travers les préjugés de notre époque, n’aient quelque peine à se détacher de notions inexactes, et, tout en acceptant en gros les observations que je viens d’exposer, ne se sentent portées à donner à l’action indirecte de la religion sur les mœurs, et des mœurs sur les institutions, et des institutions sur l’ensemble de l’ordre social, une puissance déterminante que je conclus à ne pas lui reconnaître. Ces contradicteurs penseront que, ne fût-ce que par l’influence personnelle des propagateurs de la foi, il y a, dans leur seule fréquentation, de quoi modifier sensiblement la situation politique des convertis et leurs notions de bien-être matériel. Ils diront, par exemple, que ces apôtres, sortis presque constamment, bien que non pas nécessairement, d’une nation plus avancée que celle à laquelle ils apportent la foi, vont se trouver portés d’eux-mêmes, et comme par instinct, à réformer les habitudes purement humaines de leurs néophytes, en même temps qu’ils redresseront leurs voies morales. Ont-ils affaire à des sauvages, à des peuples réduits, par leur ignorance, à supporter de grandes misères ? ils s’efforceront de leur apprendre les arts utiles et de leur montrer comment on échappe à la famine par des travaux de campagne, dont ils voudront leur fournir les instruments. Puis ces missionnaires, allant plus loin encore, leur apprendront à construire de meilleurs abris, à élever du bétail, à diriger le cours des eaux, soit pour aménager les irrigations, soit pour prévenir les inondations. De proche en proche, ils en viendront à leur donner assez de goût des choses purement intellectuelles pour leur apprendre à se servir d’un alphabet, et peut-être encore, comme cela est arrivé chez les Cherokees \footnote{Prichard, \emph{Histoire naturelle de 1’homme}, t. II, p. 120.}, à en inventer un eux-mêmes. Enfin, s’ils obtiennent des succès vraiment hors ligne, ils amèneront leur peuplade bien élevée à imiter de si près les mœurs qu’ils lui auront prêchées, que désormais, complètement façonnée à l’exploitation des terres, elle possédera, comme ces mêmes Cherokees dont je parle, et comme les Creeks de la rive sud de l’Arkansas, des troupeaux bien entretenus et même de nombreux esclaves noirs pour travailler aux plantations.\par
J’ai choisi exprès les deux peuples sauvages que l’on cite comme les plus avancés ; et, loin de me rendre à l’avis des égalitaires, je n’imagine pas, en observant ces exemples, qu’il puisse s’en trouver de plus frappants de l’incapacité générale des races à entrer dans une voie que leur nature propre n’a pas suffi à leur faire trouver.\par
Voilà deux peuplades, restes isolés de nombreuses nations détruites ou expulsées par les blancs, et d’ailleurs deux peuplades qui se trouvent naturellement hors de pair avec les autres, puisqu’on les dit descendues de la race alléghanienne, à laquelle sont attribués les grands vestiges d’anciens monuments découverts au nord du Mississipi \footnote{Id., ibid., t. II, p. 119 et pass.}. Il y a là déjà, dans l’esprit de ceux qui prétendent constater l’égalité entre les Cherokees et les races européennes, une grande déviation à l’ensemble de leur système, puisque le premier mot de leur démonstration consiste à établir que les nations alléghaniennes ne se rapprochent des Anglo-Saxons que parce qu’elles sont supérieures elles-mêmes aux autres races de l’Amérique septentrionale. En outre, qu’est-il arrivé à ces deux tribus d’élite ? Le gouvernement américain leur a pris les territoires sur lesquels elles vivaient anciennement, et, au moyen d’un traité de transplantation, il les a fait émigrer l’une et l’autre sur un terrain choisi, où il leur a marqué à chacune leur place. Là, sous la surveillance du ministère de la guerre et sous la conduite des missionnaires protestants, ces indigènes ont dû embrasser, bon gré mal gré, le genre de vie qu’ils pratiquent aujourd’hui. L’auteur où je puise ces détails, et qui les tire lui-même du grand ouvrage de M. Gallatin \footnote{Gallatin, \emph{Synopsis of the Indian tribes of North-America}.}, assure que le nombre des Cherokees va augmentant. Il allègue pour preuve qu’au temps où Adair les visita, le nombre de leurs guerriers était estimé à 2 300, et qu’aujourd’hui le chiffre total de leur population est porté à 15 000 âmes, y compris, à la vérité, 1 200 nègres esclaves, devenus leur propriété ; et, comme il ajoute aussi que leurs écoles sont, ainsi que leurs églises, dirigées par les missionnaires ; que ces missionnaires, en leur qualité de protestants, sont mariés, sinon tous, au moins pour la plupart, ont des enfants ou des domestiques de race blanche, et probablement aussi une sorte d’état-major de commis et d’employés européens de tous métiers, il devient très difficile d’apprécier si réellement il y a eu accroissement dans le nombre des indigènes, tandis qu’il est très facile de constater la pression vigoureuse que la race européenne exerce ici sur ses élèves \footnote{Je n’ai pas voulu taquiner M. Prichard sur la valeur de ses assertions, et je les discute sans les contredire. J’aurais pu cependant me borner à les nier complètement, et j’aurais eu pour moi l’imposante autorité de M. A. de Tocqueville, qui, dans son admirable ouvrage \emph{De la Démocratie en Amérique}, s’exprime ainsi au sujet des Cherokees : « Ce qui a « singulièrement favorisé le développement rapide des habitudes européennes chez ces « Indiens, a été la présence des métis. Participant aux lumières de son père, sans « abandonner entièrement les coutumes sauvages de sa race maternelle, le métis forme le « lien naturel entre la civilisation et la barbarie. Partout où les métis se sont multipliés, on a « vu les sauvages modifier peu à peu leur état social et changer leurs mœurs. » (\emph{De la Démocratie en Amérique}, in-12 ; Bruxelles, 1837 ; t. III, p. 142.) M. A. de Tocqueville termine en présageant que, tout métis qu’ils sont, et non aborigènes, comme l’affirme M. Prichard, les Cherokees et les Creeks n’en disparaîtront pas moins, avant peu, devant les envahissements des blancs.}.\par
Placés dans une impossibilité reconnue de faire la guerre, dépaysés, entourés de tous côtés par la puissance américaine incommensurable pour leur imagination, et, d’autre part, convertis à la religion de leurs dominateurs, et l’ayant adoptée, je pense, sincèrement ; traités avec douceur par leurs instituteurs spirituels et bien convaincus de la nécessité de travailler comme ces maîtres-là l’entendent et le leur indiquent, à moins de vouloir mourir de faim, je comprends qu’on réussisse à en faire des agriculteurs. On doit finir par leur inculquer la pratique de ces idées que tous les jours, et constamment, et sans relâche, on leur représente.\par
Ce serait ravaler bien bas l’intelligence même du dernier rameau, du plus humble rejeton de l’espèce humaine, que de se déclarer surpris, lorsque nous voyons qu’avec certains procédés de patience, et en mettant habilement en jeu la gourmandise et l’abstinence, on parvient à apprendre à des animaux ce que leur instinct ne les portait pas le moins du monde à savoir. Quand les foires de village ne sont remplies que de bêtes savantes auxquelles on fait exécuter les tours les plus bizarres, faudrait-il se récrier de ce que les hommes soumis à une éducation rigoureuse, et éloignés de tout moyen de s’y soustraire comme de s’en distraire, parviennent à remplir celles des fonctions de la vie civilisée qu’en définitive, dans l’état sauvage, ils pourraient encore comprendre, même avec la volonté de ne pas les pratiquer ? Ce serait mettre ces hommes au-dessous, bien au-dessous du chien qui joue aux cartes et du cheval gastro­nome ! À force de vouloir tirer à soi tous les faits pour les transformer en arguments démonstratifs de l’intelligence de certains groupes humains, on finit par se montrer par trop facile à satisfaire, et par ressentir des enthousiasmes peu flatteurs pour ceux-là mêmes qui les excitent.\par
Je sais que des hommes très érudits, très savants, ont donné lieu à ces réhabilita­tions un peu grossières, en prétendant qu’entre certaines races humaines et les grandes espèces de singes il n’y avait que des nuances pour toute séparation. Comme je repousse sans réserve une telle injure, il m’est également permis de ne pas tenir compte de l’exagération par laquelle on y répond. Sans doute, à mes yeux, les races humaines sont inégales ; mais je ne crois d’aucune qu’elle ait la brute à côté d’elle et semblable à elle. La dernière tribu, la plus grossière variété, le sous-genre le plus misérable de notre espèce est au moins susceptible d’imitation, et je ne doute pas qu’en prenant un sujet quelconque parmi les plus hideux Boschimens, on ne puisse obtenir, non pas de ce sujet même, s’il est déjà adulte, mais de son fils, à tout le moins de son petit-fils, assez de conception pour apprendre et exercer un état, voire même un état qui demande un certain degré d’étude. En conclura-t-on que la nation à laquelle appartient cet individu pourra être civilisée à notre manière ? C’est raisonner légèrement et conclure vite. Il y a loin entre la pratique des métiers et des arts, produits d’une civilisation avancée, et cette civilisation elle-même. Et d’ailleurs les missionnaires protestants, chaînon indis­pensable qui rattache la tribu sauvage à convertir au centre initiateur, est-on bien certain qu’ils soient suffisants pour la tâche qu’on leur impose ? Sont-ils donc les dépositaires d’une science sociale bien complète ? J’en doute ; et si la communication venait soudain à se rompre entre le gouvernement américain et les mandataires spiri­tuels qu’il entretient chez les Cherokees, le voyageur, au bout de quelques années, retrouverait dans les fermes des indigènes des institutions bien inattendues, bien nouvelles, résultat du mélange de quelques blancs avec ces peaux rouges, et il ne reconnaîtrait plus qu’un bien pâle reflet de ce qui s’enseigne à New York.\par
On parle souvent de nègres qui ont appris la musique, de nègres qui sont commis dans des maisons de banque, de nègres qui savent lire, écrire, compter, danser, parler comme des blancs; et l’on admire, et l’on conclut que ces gens-là sont propres à tout ! Et à côté de ces admirations et de ces conclusions hâtives, les mêmes personnes s’étonneront du contraste que présente la civilisation des nations slaves avec la nôtre. Elles diront que les peuples russe, polonais, serbe, cependant bien autrement parents à nous que les nègres, ne sont civilisés qu’à la surface ; elles prétendront que, seules, les hautes classes s’y trouvent en possession de nos idées, grâce encore à ces incessants mouvements de fusion avec les familles anglaise, française, allemande ; et elles feront remarquer une invincible inaptitude des masses à se confondre dans le mouvement du monde occidental, bien que ces masses soient chrétiennes depuis tant de siècles, et que plusieurs même l’aient été avant nous ! Il y a donc une grande différence entre l’imita­tion et la conviction. L’imitation n’indique pas nécessairement une rupture sérieuse avec les tendances héréditaires, et l’on n’est vraiment entré dans le sein d’une civilisa­tion que lorsqu’on se trouve en état d’y progresser soi-même, par soi-même et sans guide \footnote{ \noindent Carus, en raisonnant sur les listes de nègres remarquables données primitivement par Blumenbach et qu’on peut enrichir, fait très bien remarquer qu’il n’y a jamais eu ni politique, ni littérature, ni conception supérieure de l’art chez les peuples noirs ; que lorsque des individus de cette variété se sont signalés d’une manière quelconque, ce n’a jamais été que sous l’influence des blancs, et qu’il n’est pas un seul d’entre eux que l’on puisse comparer, je ne dirai pas à un de nos hommes de génie, mais aux héros des peuples jaunes, à Confucius, par exemple.\par
 Carus, \emph{Ueber die ungleiche Befæhigung der Menscheitsstæmmen zur geistigen Entwickelung}, p. 24-25.
}. Au lieu de nous vanter l’habileté des sauvages, de quelque partie du monde que ce soit, à guider la charrue quand on le leur a enseigné, ou à épeler ou lire quand on le leur a appris, qu’on nous montre, sur un des points de la terre en contact séculaire avec les Européens, et il en est certainement beaucoup, un seul lieu où les idées, les institutions, les mœurs d’une de nos nations aient été si bien adoptées avec nos doctrines religieuses, que tout y progresse par un mouvement aussi propre, aussi franc, aussi naturel qu’on le voit dans nos États ; un seul lieu où l’imprimerie produise des effets analogues à ce qui est chez nous, où nos sciences se perfectionnent, où des applications nouvelles de nos découvertes s’essayent, où nos philosophies enfantent d’autres philosophies, des systèmes politiques, une littérature, des arts, des livres, des statues et des tableaux !\par
Non ! je ne suis pas si exigeant, si exclusif. Je ne demande plus qu’avec notre foi un peuple embrasse tout ce qui fait notre individualité ; je supporte qu’il la repousse ; j’admets qu’il en choisisse une toute différente. Eh bien ! que je le voie du moins, au moment où il ouvre les yeux aux clartés de l’Évangile, comprendre subitement combien sa marche terrestre est aussi embarrassée et misérable que l’était naguère sa vie spirituelle ; que je le voie se créer à lui-même un nouvel ordre social à sa guise, rassem­blant des idées jusqu’alors restées infécondes, admettant des notions étrangères qu’il transforme. Je l’attends à l’œuvre ; je lui demande seulement de s’y mettre. Aucun ne commence. Aucun n’a jamais essayé. On ne m’indiquera pas, en compulsant tous les registres de l’histoire, une seule nation venue à la civilisation européenne par suite de l’adoption du christianisme, pas une seule que le même grand fait ait portée à se civiliser d’elle-même lorsqu’elle ne l’était pas déjà.\par
Mais, en revanche, je découvrirai dans les vastes régions de l’Asie méridionale et dans certaines parties de l’Europe, des États formés de plusieurs masses superposées de religionnaires différents. Les hostilités des races se maintiendront inébranlablement à côté, au milieu des hostilités des cultes, et l’on distinguera le Patan devenu chrétien de l’Hindou converti, avec autant de facilité que l’on peut séparer aujourd’hui le Russe d’Orenbourg des tribus nomades christianisées au milieu desquelles il vit. Encore une fois, le christianisme n’est pas civilisateur, et il a grandement raison de ne pas l’être.
\section[{I.8. Définition du mot civilisation ; le développement social résulte d’une double source.}]{I.8. \\
Définition du mot civilisation ; le développement social résulte d’une double source.}
\noindent Ici trouvera sa place une digression indispensable. Je me sers à chaque instant d’un mot qui enferme dans sa signification un ensemble d’idées important à définir. Je parle souvent de la civilisation, et, à bon droit sans doute, car c’est par l’existence relative ou l’absence absolue de cette grande particularité que je puis seulement graduer le mérite respectif des races. Je parle de la civilisation européenne, et je la distingue de civilisa­tions que je dis être différentes. Je ne dois pas laisser subsister le moindre vague, et d’autant moins que je ne me trouve pas d’accord avec l’écrivain célèbre qui, en France, s’est spécialement occupé de fixer le caractère et la portée de l’expression que j’emploie.\par
M. Guizot, si j’ose me permettre de combattre sa grande autorité, débute, dans son livre sur la \emph{Civilisation en Europe}, par une confusion de mots d’où découlent d’assez graves erreurs positives. Il énonce cette pensée que la civilisation est un \emph{fait.}\par
Ou le mot \emph{fait} doit être entendu ici dans un sens beaucoup moins précis et positif que le commun usage ne l’exige, dans un sens large et un peu flottant, j’oserais presque dire élastique et qui ne lui a jamais appartenu, ou bien, il ne convient pas pour caractériser la notion comprise dans le mot \emph{civilisation}. La civilisation n’est pas un fait, \emph{c’est une série, un enchaînement de faits} plus ou moins logiquement unis les uns aux autres, et engendrés par un concours d’idées souvent assez multiples ; idées et faits se fécondant sans cesse. Un roulement incessant est quelquefois la conséquence des premiers principes ; quelquefois aussi cette conséquence est la stagnation ; dans tous les cas, la civilisation n’est pas un fait, c’est un faisceau de faits et d’idées, c’est un \emph{état} dans lequel une société humaine se trouve placée, un \emph{milieu} dans lequel elle a réussi à se mettre, qu’elle a créé, qui émane d’elle, et qui à son tour réagit sur elle.\par
Cet \emph{état} a un grand caractère de généralité qu’un\emph{ fait} ne possède jamais ; il se prête à beaucoup de variations qu’un \emph{fait} ne saurait pas subir sans disparaître, et, entre autres, il est complètement indépendant des formes gouvernementales, se développant aussi bien sous le despotisme que sous le régime de la liberté, et ne cessant pas même d’exister lorsque des commotions civiles modifient ou même transforment absolument les conditions de la vie politique.\par
Ce n’est pas à dire cependant qu’il faille estimer peu de chose les formes gouvernementales. Leur choix est intimement lié à la prospérité du corps social : faux, il l’entrave ou la détruit ; judicieux, il la sert et la développe. Seulement, il ne s’agit pas ici de prospérité ; la question est plus grave : il s’agit de l’existence même des peuples et de la civilisation, phénomène intimement lié à certaines conditions élémentaires, indépendantes de l’état politique, et qui puisent leur raison d’être, les motifs de leur direction, de leur expansion, de leur fécondité ou de leur faiblesse, tout enfin ce qui les constitue, dans des racines bien autrement profondes. Il va donc sans dire que, devant des considérations aussi capitales, les questions de conformation politique, de pros­périté ou de misère se trouvent rejetées à la seconde place ; car, partout et toujours, ce qui prend la première, c’est cette question fameuse d’Hamlet : \emph{être ou ne pas être.} Pour les peuples aussi bien que pour les individus, elle plane au-dessus de tout. Comme M. Guizot ne paraît pas s’être mis en face de cette vérité, la civilisation est pour lui, non pas un \emph{état}, non pas un \emph{milieu}, mais un \emph{fait} ; et le principe générateur dont il le tire est un autre fait d’un caractère exclusivement politique.\par
Ouvrons le livre de l’éloquent et illustre professeur : nous y trouvons un faisceau d’hypothèses choisies pour mettre la pensée dominante en relief. Après avoir indiqué un certain nombre de situations dans lesquelles peuvent se trouver les sociétés, l’auteur se demande « si l’instinct général y reconnaîtrait « l’état d’un peuple qui se civilise ; si c’est là le sens que le genre humain « attache naturellement au mot \emph{civilisation} \footnote{M. Guizot, \emph{Histoire de la civilisation en Europe}, p. 11 et passim.}. »\par
La première hypothèse est celle-ci : « Voici un peuple dont la vie extérieure est « douce, commode : il paye peu d’impôt, il ne souffre point ; la justice lui est bien « rendue dans les relations privées ; en un mot, l’existence matérielle et « morale de  ce peuple est tenue avec grand soin dans un état « d’engourdissement, d’inertie, je ne veux pas dire d’oppression, parce qu’il « n’en a pas le sentiment, mais de compression. Ceci n’est pas sans exemple. « Il y a un grand nombre de petites républiques aristocratiques, où les sujets « ont été ainsi traités comme des troupeaux, bien tenus et matériellement « heureux, mais sans activité intellectuelle et morale. Est-ce là la « civilisation ? Est-ce là un peuple qui se civilise ? »\par
Je ne sais pas si c’est là un peuple qui se civilise, mais certainement ce peut être un peuple très civilisé, sans quoi il faudrait repousser parmi les hordes sauvages ou barbares toutes ces républiques aristocratiques de l’antiquité et des temps modernes qui se trouvent, ainsi que M. Guizot le remarque lui-même, comprises dans les limites de son hypothèse ; et l’instinct public, le sens général, ne peuvent manquer d’être blessés d’une méthode qui rejette les Phéniciens, les Carthaginois, les Lacédémoniens, du sanctuaire de la civilisation, pour en faire de même ensuite des Vénitiens, des Génois, des Pisans, de toutes les villes libres impériales de l’Allemagne, en un mot, de toutes les municipalités puissantes des derniers siècles. Outre que cette conclusion paraît en elle-même trop violemment paradoxale pour que le sentiment commun auquel il est fait appel soit disposé à l’admettre, elle me semble affronter encore une difficulté plus grande. Ces petits États aristocratiques auxquels, en vertu de leur forme de gouvernement, M. Guizot refuse l’aptitude à la civilisation, ne se sont jamais trouvés, pour la plupart en possession d’une culture spéciale et qui n’appartînt qu’à eux. Tout puissants qu’on en ait vu plusieurs, ils se confondaient, sous ce rapport, avec des peuples différemment gouvernés, mais de race très parente, et ne faisaient que partici­per à un ensemble de civilisation, Ainsi, les Carthaginois et les Phéniciens, éloignés les uns des autres, n’en étaient pas moins unis dans un mode de culture semblable et qui avait son type en Assyrie. Les républiques italiennes s’unissaient dans le mouvement d’idées et d’opinions dominant au sein des monarchies voisines. Les villes impériales souabes et thuringiennes, fort indépendantes au point de vue politique, étaient tout à fait annexées au progrès ou à la décadence générale de la race allemande. Il résulte de ces observations que M. Guizot, en distribuant ainsi aux peuples des numéros de mérite calculés sur le degré et la forme de leurs libertés, crée dans les races des disjonctions injustifiables et des différences qui n’existent pas. Une discussion poussée trop loin ne serait pas à sa place ici, et je passe rapidement ; si pourtant il y avait lieu d’entamer la controverse, ne devrait-on pas se refuser à admettre pour Pise, pour Gênes, pour Venise et les autres, une infériorité vis-à-vis de pays tels que Milan, Naples et Rome ?\par
Mais M. Guizot va lui-même au-devant de cette objection. S’il ne reconnaît pas la civilisation chez un peuple « doucement gouverné, mais retenu dans une « situation de compression », il ne l’admet pas davantage chez un autre peuple « dont l’existence matérielle est moins douce, moins commode, supportable « cependant dont, en revan­che, on n’a point négligé les besoins moraux, « intellectuels... ; dont on cultive les sentiments élevés, purs ; dont les « croyances religieuses, morales, ont atteint un certain degré de développement, « mais chez qui le principe de la liberté est étouffé ; où l’on mesure à chacun sa « part de vérité ; où l’on ne permet à personne de la chercher à lui tout seul. « C’est l’état où sont tombées la plupart des populations de l’Asie, où les « dominations théocratiques retiennent l’humanité ; c’est l’état des Hindous, par « exemple \footnote{M. Guizot, \emph{Histoire de la civilisation en Europe}, p. 11 et passim.} ».\par
 Ainsi, dans la même exclusion que les peuples aristocratiques, il faut repousser encore les Hindous, les Égyptiens, les Étrusques, les Péruviens, les Thibétains, les Japonais, et même la moderne Rome et ses territoires.\par
Je ne touche pas à deux dernières hypothèses, par la raison que, grâce aux deux premières, voilà l’état de civilisation déjà tellement restreint que, sur le globe, presque aucune nation ne se trouve plus autorisée à s’en prévaloir légitimement. Du moment que, pour posséder le droit d’y prétendre, il faut jouir d’institutions également modératrices du pouvoir et de la liberté, et dans lesquelles le développement matériel et le progrès moral se coordonnent de telle façon et non de telle autre ; où le gouverne­ment, comme la religion, se confine dans des limites tracées avec précision ; où les sujets, enfin, doivent de toute nécessité posséder des droits d’une nature définie, je m’aperçois qu’il n’y a de peuples civilisés que ceux dont les institutions politiques sont constitutionnelles et représentatives. Dès lors, je ne pourrai pas même sauver tous les peuples européens de l’injure d’être repoussés dans la barbarie, et si, de proche en proche, et mesurant toujours le degré de civilisation à la perfection d’une seule et unique forme politique, je dédaigne ceux des États constitutionnels qui usent mal de l’instrument parlementaire, pour réserver le prix exclusivement à ceux-là qui s’en servent bien, je me trouverai amené à ne considérer comme vraiment civilisée, dans le passé et dans le présent, que la seule nation anglaise.\par
Certainement je suis plein de respect et d’admiration pour le grand peuple dont la victoire, l’industrie, le commerce racontent en tous lieux la puissance et les prodiges. Mais je ne me sens pas disposé pourtant à ne respecter et à n’admirer que lui seul : il me semblerait trop humiliant et trop cruel pour l’humanité d’avouer que, depuis le commencement des siècles, elle n’a réussi à faire fleurir la civilisation que sur une petite île de l’Océan occidental, et n’a trouvé ses véritables lois que depuis le règne de Guillaume et de Marie. Cette conception, on l’avouera, peut sembler un peu étroite. Puis voyez le danger ! Si l’on veut attacher l’idée de civilisation à une forme politique, le raisonnement, l’observation, la science vont bientôt perdre toute chance de décider dans cette question, et la passion seule des partis en décidera. Il se trouvera des esprits qui, au gré de leurs préférences, refuseront intrépidement aux institutions britanniques l’honneur d’être l’idéal du perfectionnement humain : leur enthousiasme sera pour l’ordre établi à Saint-Pétersbourg ou à Vienne. Beaucoup enfin, et peut-être le plus grand nombre, entre le Rhin et les monts Pyrénées, soutiendront que, malgré quelques taches, le pays le plus policé du monde c’est encore la France. Du moment que déterminer le degré de culture devient une affaire de préférence, une question de sentiment, s’entendre est impossible. L’homme le plus noblement développé sera, pour chacun, celui-là qui pensera comme lui sur les devoirs respectifs des gouvernants et des sujets, tandis que les malheureux doués de visées différentes seront les barbares et les sauvages. Je crois que personne n’osera affronter cette logique, et l’on avouera, d’un commun accord, que le système où elle prend sa source est, à tout le moins, bien incomplet.\par
Pour moi, je ne le trouve pas supérieur, il me semble inférieur même à la définition donnée par le baron Guillaume de Humboldt : « La civilisation est « l’humanisation des peuples dans leurs institutions extérieures, dans leurs « mœurs et dans le sentiment intérieur qui s’y rapporte ».\par
Je rencontre là un défaut précisément opposé à celui que je me suis permis de relever dans la formule de M. Guizot. Le lien est trop lâche, le terrain indiqué trop large. Du moment que la civilisation s’acquiert au moyen d’un simple adoucissement des mœurs, plus d’une peuplade sauvage, et très sauvage, aura le droit de réclamer le pas sur telle nation d’Europe dont le caractère offrira tant soit peu d’âpreté. Il est dans les îles de la mer du Sud, et ailleurs, plus d’une tribu fort inoffensive, d’habitudes très douces, d’humeur très accorte, que cependant on n’a jamais songé, tout en la louant, à mettre au-dessus des Norvégiens assez durs, ni même à côté des Malais féroces qui, vêtus de brillantes étoffes fabriquées par eux-mêmes, et parcourant les flots sur des barques habilement construites de leurs propres mains, sont tout à la fois la terreur du commerce maritime et ses plus intelligents courtiers dans les parages orientaux de l’océan Indien. Cette observation ne pouvait pas échapper à un esprit aussi éminent que celui de M. Guillaume de Humboldt ; aussi, à côté de la civilisation et sur un degré supérieur, il imagine \emph{la culture}, et il déclare que, par elle, les peuples, adoucis déjà, gagnent \emph{la science et l’art.}\par
D’après cette hiérarchie, nous trouvons le monde peuplé, au second âge, d’êtres \emph{affectueux} et \emph{sympathiques}, de plus érudits, poètes et artistes, mais, par l’effet de toutes ces qualités réunies, étrangers aux grossières besognes, aux nécessités de la guerre, comme à celles du labourage et des métiers.\par
En réfléchissant au petit nombre des loisirs que l’existence perfectionnée et assurée des époques les plus heureuses donne à leurs contemporains pour se livrer aux pures occupations de l’esprit, en regardant combien est incessant le combat qu’il faut livrer à la nature et aux lois de l’univers pour seulement parvenir à subsister, on s’aperçoit vite que le philosophe berlinois a moins prétendu à dépeindre les réalités qu’à tirer du sein des abstractions certaines entités qui lui paraissaient belles et grandes, qui le sont en effet, et à les faire agir et se mouvoir dans une sphère idéale comme elles-mêmes. Les doutes qui pourraient rester à cet égard disparaissent bientôt quand on parvient au point culminant du système, consistant en un troisième et dernier degré supérieur aux deux autres. Ce point suprême est celui où se place l’homme formé, c’est-à-dire l’homme qui, dans sa nature, possède « quelque chose de plus haut, de plus « intime à la fois, c’est-à-dire une façon, de comprendre qui répand « harmonieusement sur la sensibilité et le caractère les impressions qu’elle « reçoit de l’activité intellectuelle et morale dans son ensemble ».\par
Cet enchaînement, un peu laborieux, va donc de l’homme civilisé ou adouci, humanisé, à l’homme cultivé, savant, poète et artiste, pour arriver enfin au plus haut développement où notre espèce puisse parvenir, à l’homme formé, qui, si je comprends bien à mon tour, sera représenté avec justesse par ce qu’on nous dit qu’était Gœthe dans sa sérénité olympienne. L’idée d’où sort cette théorie n’est rien autre que la profonde différence remarquée par M. Guillaume de Humboldt entre la civilisation d’un peuple et la hauteur relative du perfectionnement des grandes individualités ; différence telle que les civilisations étrangères à la nôtre ont pu, de toute évidence, posséder des hommes très supérieurs sous certains rapports à ceux que nous admirons le plus : la civilisation brahmanique, par exemple.\par
Je partage sans réserve l’avis du savant dont j’expose ici les idées. Rien n’est plus exact : notre état social européen ne produit ni les meilleurs ni les plus sublimes penseurs, ni les plus grands poètes, ni les plus habiles artistes. Néanmoins je me permets de croire, contrairement à l’opinion de l’illustre philologue, que, pour juger et définir la civilisation en général, il faut se débarrasser avec soin, ne fût-ce que pour un moment, des préventions et des jugements de détail concernant telle ou telle civilisation en particulier. Il ne faut être ni trop large, comme pour l’homme du premier degré, que je persiste à ne pas trouver civilisé, uniquement parce qu’il est adouci ; ni trop étroit, comme pour le sage du troisième. Le travail améliorateur de l’espèce humaine est ainsi trop réduit. Il n’aboutit qu’à des résultats purement isolés et typiques.\par
Le système de M. Guillaume de Humboldt fait, du reste, le plus grand honneur à la délicatesse grandiose qui était le trait dominant de cette généreuse intelligence, et on peut le comparer, dans sa nature essentiellement abstraite, à ces mondes fragiles imagi­nés par la philosophie hindoue. Nés du cerveau d’un Dieu endormi, ils s’élèvent dans l’atmosphère pareils aux bulles irisées que souffle dans le savon le chalumeau d’un enfant, et se brisent et se succèdent au gré des rêves dont s’amuse le céleste sommeil.\par
Placé par le caractère de mes recherches sur un terrain plus rudement positif, j’ai besoin d’arriver à des résultats que la pratique et l’expérience puissent palper un peu mieux. Ce que l’angle de mon rayon visuel s’efforce d’embrasser, ce n’est pas, avec M. Guizot, l’état plus ou moins prospère des sociétés ; ce n’est pas non plus, avec M. G. de Humboldt, l’élévation isolée des intelligences individuelles : c’est l’ensemble de la puissance, aussi bien matérielle que morale, développée dans les masses. Troublé, je l’avoue, par le spectacle des déviations où se sont égarés deux des hommes les plus admirés de ce siècle, j’ai besoin, pour suivre librement une route écartée de la leur, de me recorder avec moi-même et de prendre du plus haut possible les déductions indis­pensables afin d’arriver d’un pas ferme à mon but. Je prie donc le lecteur de me suivre avec patience et attention dans les méandres où je dois m’engager, et je vais m’efforcer d’éclairer de mon mieux l’obscurité naturelle de mon sujet.\par
Il n’y a pas de peuplade si abrutie chez laquelle ne se démêle un double instinct : celui des besoins matériels, et celui de la vie morale. La mesure d’intensité des uns et de l’autre donne naissance à la première et la plus sensible des différences entre les races. Nulle part, voire dans les tribus les plus grossières, les deux instincts ne se balancent à forces égales. Chez les unes, le besoin physique domine de beaucoup ; chez les autres, les tendances contemplatives l’emportent au contraire. Ainsi les basses hordes de la race jaune nous apparaissent dominées par la sensation matérielle, sans cependant être absolument privées de toute lueur portée sur les choses surhumaines. Au contraire, chez la plupart des tribus nègres du degré correspondant, les habitudes sont agissantes moins que pensives, et l’imagination y donne plus de prix aux choses qui ne se voient pas qu’à celles qui se touchent. Je n’en tirerai pas la conséquence d’une supériorité de ces dernières races sauvages sur les premières, au point de vue de la civilisation, car elles ne sont pas, l’expérience des siècles le prouve, plus susceptibles d’y atteindre les unes que les autres. Les temps ont passé et ne les ont vues rien faire pour améliorer leur sort, enfermées qu’elles sont toutes dans une égale incapacité de combiner assez d’idées avec assez de faits pour sortir de leur abaissement. Je me borne à remarquer que, dans le plus bas degré des peuplades humaines, je trouve ce double courant, diversement constitué, dont je vais avoir à suivre la marche à mesure que je monterai.\par
Au-dessus des Samoyèdes, comme des nègres Fidas et Pélagiens, il faut placer ces tribus qui ne se contentent pas tout à fait d’une cabane de branchages et de rapports sociaux basés sur la force seule, mais qui comprennent et désirent un état meilleur. Elles sont élevées d’un degré au-dessus des plus barbares. Appartiennent-elles à la série des races plus actives que pensantes, on les verra perfectionner leurs instruments de travail, leurs armes, leur parure ; avoir un gouvernement où les guerriers domineront sur les prêtres, où la science des échanges acquerra un certain développement, où l’esprit mercantile paraîtra déjà assez accusé. Les guerres, toujours cruelles, auront cependant une tendance caractérisée vers le pillage ; en un mot, le bien-être, les jouissances physi­ques, seront le but principal des individus. Je trouve la réalisation de ce tableau dans plusieurs des nations mongoles ; je la découvre encore, bien qu’avec des différences honorables, chez les Quichuas et les Aymaras du Pérou ; et j’en rencontrerai l’antithèse, c’est-à-dire plus de détachement des intérêts matériels, chez les Dahomeys de l’Afrique occidentale et chez les Cafres.\par
Maintenant je poursuis la marche ascendante. J’abandonne ces groupes dont le système social n’est pas assez vigoureux pour savoir s’imposer, avec la fusion du sang, à des multitudes bien grandes. J’arrive à celles dont le principe constitutif possède une virtualité si forte, qu’il relie et enserre tout ce qui avoisine son centre d’action, se l’incorpore et élève sur d’immenses contrées la domination incontestée d’un ensemble d’idées et de faits plus ou moins bien coordonné, en un mot ce qui peut s’appeler \emph{une civilisation.} La même différence, la même classification que j’ai fait ressortir pour les deux premiers cas, se retrouve ici tout entière, bien plus reconnaissable encore ; et même ce n’est qu’ici qu’elle porte des fruits véritables, et que ses conséquences ont de la portée. Du moment où, de l’état de peuplade, une agglomération d’hommes étend assez ses relations, son horizon, pour passer à celui de peuple, on remarque chez elle que les deux courants, matériel et intellectuel, ont augmenté de force, suivant que les groupes qui sont entrés dans son sein et qui s’y fusionnent appartiennent en plus grande quantité à l’un ou à l’autre. Ainsi, quand la faculté pensive domine, il arrive tels résultats ; quand c’est la faculté active, il s’en produit tels autres, La nation déploie des qualités de nature différente, suivant que règne celui-ci ou celui-là des deux éléments. On pourrait ici appliquer le symbolisme hindou, en représentant ce que j’ai appelé le courant intellectuel par Prakriti, principe femelle, et le courant matériel par Pouroucha, principe mâle, à condition toutefois, bien entendu, de ne comprendre sous ces mots qu’une idée de fécondation réciproque, sans mettre d’un côté un éloge et de l’autre un blâme \footnote{M. Klemm (\emph{Allgemeine Kulturgeschichte der Menschheit}, Leipzig, 1849) imagine une distinction de l’humanité en races actives et races passives. Je n’ai pas eu ce livre entre les mains, et ne puis savoir si l’idée de son auteur est en rapport avec la mienne. Il serait naturel qu’en battant les mêmes sentiers, nous fussions tombés sur la même vérité.}.\par
On remarquera, en outre, qu’aux différentes époques de la vie d’un peuple et dans une stricte dépendance avec les inévitables mélanges du sang, l’oscillation devient plus forte entre les deux principes, et il arrive que l’un l’emporte alternativement sur l’autre, Les faits qui résultent de cette mobilité sont très importants, et modifient d’une manière sensible le caractère d’une civilisation en agissant sur sa stabilité.\par
Je partagerai donc, pour les placer plus particulièrement, mais jamais absolument, qu’on s’en souvienne, sous l’action d’un des courants, tous les peuples en deux classes. À la tête de la catégorie mâle, j’inscrirai les Chinois ; et comme prototype de la classe adverse, je choisirai les Hindous.\par
À la suite des Chinois, il faudra inscrire la plupart des peuples de l’Italie ancienne, les premiers Romains de la république, les tribus germaniques. Dans le camp contraire, je vois les nations de l’Égypte, celles de l’Assyrie. Elles prennent place derrière les hommes de l’Hindoustan.\par
En suivant le cours des siècles, on s’aperçoit que presque tous les peuples ont transformé leur civilisation par suite des oscillations des deux principes. Les Chinois du nord, population d’abord presque absolument matérialiste, se sont alliés peu à peu à des tribus d’un autre sang, dans le Yunnan surtout, et ce mélange a rendu leur génie moins exclusivement utilitaire. Si ce développement est resté stationnaire, ou du moins fort lent depuis des siècles, c’est que la masse des populations mâles dépassait de beaucoup le faible appoint de sang contraire qu’elles se sont partagé.\par
Pour nos groupes européens, l’élément utilitaire qu’apportaient les meilleures des tribus germaniques s’est fortifié sans cesse dans le nord, par l’accession des Celtes et des Slaves. Mais, à mesure que les peuples blancs sont descendus davantage vers le sud, les influences mâles se sont trouvées moins en force, se sont perdues dans un élément trop féminin (il faut faire quelques exceptions comme, par exemple, pour le Piémont et le nord de l’Espagne), et cet élément féminin a triomphé.\par
Passons maintenant de l’autre côté. Nous voyons les Hindous pourvus à un haut degré du sentiment des choses supernaturelles, et plus méditatifs qu’agissants. Comme leurs plus anciennes conquêtes les ont mis surtout en contact avec des races pourvues d’une organisation de même ordre, le principe mâle n’a pu se développer suffisamment. La civilisation n’a pas pris dans ces milieux un essor utilitaire proportionné à ses succès de l’autre genre. Au contraire, Rome antique, naturellement utilitaire, n’abonde dans le sens opposé que lorsqu’une fusion complète avec les Grecs, les Africains et les Orientaux, transforme sa première nature et lui crée un tempérament tout nouveau.\par
Pour les Grecs, le travail intérieur fut encore plus comparable à celui des Hindous.\par
De l’ensemble de tels faits, je tire cette conclusion, que toute activité humaine, soit intellectuelle, soit morale, prend primitivement sa source dans l’un des deux courants, mâle ou femelle, et que c’est seulement chez les races assez abondamment pourvues d’un de ces deux éléments, sans qu’aucun soit jamais complètement dépourvu de l’autre, que l’état social peut parvenir à un degré satisfaisant de culture, et par conséquent à la civilisation.\par
Je passe maintenant à d’autres points qui sont encore dignes de remarque.
\section[{I.9. Suite de la définition du mot civilisation ; caractères différents des sociétés humaines ; notre civilisation n’est pas supérieure à celles qui ont existé avant elle.}]{I.9. \\
Suite de la définition du mot civilisation ; caractères différents des sociétés humaines ; notre civilisation n’est pas supérieure à celles qui ont existé avant elle.}
\noindent Lorsqu’une nation, appartenant à la série féminine ou masculine, possède un instinct civilisateur assez fort pour imposer sa loi à des multitudes, assez heureux surtout pour cadrer avec leurs besoins et leurs sentiments en s’emparant de leurs convictions, la culture qui doit en résulter existe de ce moment même. C’est là, pour cet instinct, le plus essentiel, le plus pratique des mérites, et ce qui seulement le rend usuel et peut lui donner la vie ; car les intérêts individuels sont, de leur nature, portés à s’isoler. L’association ne manque jamais de les léser partiellement ; ainsi, pour qu’une conviction puisse avoir lieu d’une manière intime et féconde, il faut qu’elle s’accorde dans ses vues avec la logique particulière et les sentiments du peuple qu’elle sollicite.\par
Quand une façon de comprendre le droit est acceptée par des masses, c’est qu’en réalité elle donne satisfaction, sur les points principaux, aux besoins considérés comme les plus chers. Les nations mâles voudront surtout du bien-être les nations féminines se préoccuperont davantage des exigences d’imagination mais, du moment, je le répète, que des multitudes s’enrôlent sous une bannière, ou, ce qui est plus exact ici, du moment qu’un régime particulier parvient à se faire accepter, il y a civilisation naissante.\par
 Un second caractère indélébile de cet état, c’est le besoin de la stabilité, et il découle directement de ce qui précède ; car, aussitôt que les hommes ont admis, en commun, que tel principe doit les réunir, et ont consenti à des sacrifices individuels pour faire régner ce principe, leur premier sentiment est de le respecter, pour ce qu’il leur rapporte comme pour ce qu’il leur coûte, et de le déclarer inamovible. Plus une race se maintient pure, moins sa base sociale est attaquée, parce que la logique de la race demeure la même. Cependant il s’en faut que ce besoin de stabilité ait longtemps satisfaction. Avec les mélanges de sang, viennent les modifications dans les idées nationales ; avec ces modifications, un malaise qui exige des changements corrélatifs dans l’édifice. Quelquefois ces changements amènent des progrès véritables, et surtout à l’aurore des sociétés où le principe constitutif est, en général, absolu, rigoureux, par suite de la prédominance trop complète d’une seule race. Ensuite, quand les variations se multiplient au gré de multitudes hétérogènes et sans convictions communes, l’intérêt général n’a plus toujours à s’applaudir des transformations. Toutefois, aussi longtemps que le groupe aggloméré subsiste sous la direction des impressions premières, il ne cesse pas de poursuivre, à travers l’idée du mieux-être qui l’emporte, une chimère de stabilité. Varié, inconstant, changeant à chaque heure, il se croit éternel et en marche vers une sorte de but paradisiaque. Il conserve, même en la démentant à chaque heure par ses actes, cette doctrine, que l’un des traits principaux de la civilisation, c’est d’emprunter à Dieu, en faveur des intérêts humains, quelque chose de son immutabilité; et si cette ressemblance visiblement n’existe pas, il se rassure et se console en se persuadant que demain il va y atteindre.\par
À côté de la stabilité et du concours des intérêts individuels se touchant sans se détruire, il faut placer un troisième et un quatrième caractère, l’anathème de la violence, puis la sociabilité.\par
Enfin, de la sociabilité et du besoin de se défendre moins avec le poing qu’avec la tête, naissent les perfectionnements de l’intelligence, qui, à leur tour, amènent les perfectionnements matériels, et c’est à ces deux derniers traits que l’œil reconnaît surtout un état social avancé \footnote{C’est là aussi que se trouve la source principale des faux jugements sur l’état des peuples étrangers. De ce que l’extérieur de leur civilisation ne ressemble pas à la partie correspondante de la nôtre, nous sommes souvent portés à conclure hâtivement, ou qu’ils sont barbares ou qu’ils sont nos inférieurs en mérite. Rien n’est plus superficiel, et partant ne doit être plus suspect, qu’une conclusion tirée de pareilles prémisses.}.\par
Je crois maintenant pouvoir résumer ma pensée sur la civilisation, en la définissant comme Un \emph{état de stabilité relative, où des multitudes s’efforcent de chercher pacifiquement la satisfaction de leurs besoins, et raffinent leur intelligence et leurs mœurs.}\par
Dans cette formule tous les peuples que j’ai cités jusqu’ici comme civilisés entrent les uns aussi bien que les autres. Il s’agit maintenant de savoir si, les conditions indiquées étant remplies, toutes les civilisations sont égales. C’est ce que je ne pense pas ; car, les besoins et la sociabilité de toutes les nations d’élite n’ayant pas la même intensité ni la même direction, leur intelligence et leurs mœurs prennent, dans leur qualité, des degrés très divers. De quoi l’Hindou a-t-il besoin matériellement ? de riz et de beurre pour sa nourriture, d’une toile de coton pour son vêtement. On sera tenté, sans doute, d’attribuer cette sobriété extrême aux conditions climatériques. Mais les Thibétains habitent un climat rigoureux ; cependant leur sobriété est encore très notable. Ce qui domine pour l’un et l’autre de ces peuples, c’est le développement philosophique et religieux chargé de donner un aliment aux exigences, bien autrement inquiètes, de l’âme et de l’esprit. Ainsi, là, aucun équilibre entre les deux principes mâle et femelle ; la prédominance étant du côté de la partie intellectuelle, lui donne trop de poids, et il en résulte que tous les travaux de cette civilisation sont presque uniquement portés vers un résultat au détriment de l’autre. Des monuments immenses, des montagnes de pierre, seront sculptés au prix d’efforts et de peines qui épouvantent l’imagination. Des constructions gigantesques couvriront la terre : dans quel but ? celui d’honorer les dieux, et on ne fera rien pour l’homme, à moins que ce ne soient des tombes. À côté des merveilles produites par le ciseau du sculpteur, la littérature, non moins puissante, créera d’admirables chefs-d’œuvre. Dans la théologie, dans la méta­physique, elle sera aussi ingénieuse, aussi subtile que variée, et la pensée humaine descendra, sans s’effrayer, jusqu’à d’incommensurables profondeurs. Dans la poésie lyrique, la civilisation féminine sera l’orgueil de l’humanité.\par
Mais si du domaine de la rêverie idéaliste je passe aux inventions matériellement utiles et aux sciences qui en sont la théorie génératrice, d’un sommet je tombe dans un abîme, et le jour éclatant fait place à la nuit. Les inventions utiles demeurent rares, mesquines, stériles ; le talent d’observation n’existe pour ainsi dire pas. Tandis que les Chinois trouvaient beaucoup, les Hindous n’imaginaient qu’assez peu, et n’en prenaient guère souci ; les Grecs, de même, nous transmettaient des connaissances souvent indignes d’eux, et les Romains, une fois arrivés au point culminant de leur histoire, tout en faisant plus, ne purent aller bien loin, car le mélange asiatique, dans lequel ils s’absor­baient avec une rapidité effrayante, leur refusait les qualités indispensables pour une patiente investigation des réalités. Ce qu’on peut dire d’eux toutefois, c’est que leur génie administratif, leur législation et les monuments utiles dont ils pourvoyaient le sol de leurs territoires, attestent suffisamment le caractère positif que revêtit leur pensée sociale à un certain moment, et prouve que si le midi de l’Europe n’avait pas été si promptement couvert par les colonisations incessantes de l’Asie et de l’Afrique, la science positive y aurait gagné, et l’initiative germanique aurait, par la suite, récolté moins de gloire.\par
Les vainqueurs du V\textsuperscript{e} siècle apportèrent en Europe un esprit de la même catégorie que l’esprit chinois, mais bien autrement doué. On le vit armé, dans une plus grande mesure, de facultés féminines. Il réalisa un plus heureux accord des deux mobiles. Partout où domina cette branche de peuples, les tendances utilitaires, ennoblies, sont imméconnaissables. En Angleterre, dans l’Amérique du Nord, en Hollande, en Hanovre, ces dispositions dominent les autres instincts nationaux. Il en est de même en Belgique, et encore dans le nord de la France, où tout ce qui est d’application positive a constamment trouvé des facilités merveilleuses à se faire comprendre. À mesure qu’on avance vers le sud, ces prédispositions s’affaiblissent. Ce n’est pas à l’action plus vive du soleil qu’il faut l’attribuer, car certes les Catalans, les Piémontais habitent des régions plus chaudes que les Provençaux et les habitants du bas Languedoc ; c’est à l’influence du sang.\par
La série des races féminines ou féminisées tient la plus grande place sur le globe ; cette observation s’applique à l’Europe en particulier. Qu’on en excepte la famille teutonique et une partie des Slaves, on ne trouve, dans notre partie du monde, que des groupes faiblement pourvus du sens utilitaire, et qui, ayant déjà joué leur rôle dans les époques antérieures, ne pourraient plus le recommencer. Les masses, nuancées dans leurs variétés, présentent, du Gaulois au Celtibérien, du Celtibérien au mélange sans nom des nations italiennes et romanes, une échelle descendante non pas quant à toutes les aptitudes du principe mâle, du moins quant aux principales.\par
Le mélange des tribus germaniques avec les races de l’ancien monde, cette union de groupes mâles à un si haut degré avec des races et des débris de races consommés dans les détritus d’anciennes idées, a créé notre civilisation ; la richesse, la diversité, la fécondité, dont nous faisons honneur à nos sociétés, est un résultat naturel des éléments tronqués et disparates qu’il était dans le propre de nos tribus paternelles de savoir, jusqu’à un certain point, mêler, travestir et utiliser.\par
Partout où s’étend notre mode de culture, il porte deux caractères communs : l’un, c’est d’avoir été au moins touché par le contact germanique ; l’autre, d’être chrétien. Mais, je le dis encore, ce second trait, bien que le plus apparent et celui qui d’abord saute aux yeux, parce qu’il se produit à l’extérieur de nos États, dont il semble en quelque sorte le vernis, n’est pas absolument essentiel, attendu que beaucoup de nations sont chrétiennes, et un plus grand nombre encore pourra le devenir, sans faire partie de notre cercle de civilisation. Le premier caractère est, au contraire, positif, décisif. Là où l’élément germanique n’a jamais pénétré, il n’y a pas de civilisation à notre manière.\par
Ceci m’amène naturellement à traiter cette question : Peut-on affirmer que les sociétés européennes soient entièrement civilisées ? que les idées, les faits qui se produisent à leurs surfaces, aient leur raison d’être bien profondément enracinée dans les masses, et que les conséquences de ces idées et de ces principes répondent aux instincts du plus grand nombre ? On y doit encore ajouter cette demande, qui en est le corollaire : Les dernières couches de nos populations pensent-elles et agissent-elles dans le sens de ce qu’on appelle la civilisation européenne ?\par
On a admiré avec raison l’extrême homogénéité d’idées et de vues qui, dans les États grecs de la belle époque, dirigeait le corps entier des citoyens. Sur chaque point essen­tiel, les données, souvent hostiles, partaient pourtant de la même source : on voulait plus ou moins de démocratie, plus ou moins d’oligarchie en politique ; en religion, on adorait de préférence ou la Cérès Éleusinienne ou la Minerve du Parthénon ; en matière de goût littéraire, on pouvait préférer Eschyle à Sophocle, Alcée à Pindare ; au fond, les idées sur lesquelles on disputait étaient toutes ce qu’on pourrait appeler nationales ; la discussion n’en attaquait que la mesure. À Rome, avant les guerres puniques, il en était de même, et la civilisation du pays était uniforme, incontestée. Dans sa façon de procéder, elle s’étendait du maître à l’esclave ; tout le monde y participait à des degrés divers, mais ne participait qu’à elle.\par
Depuis les guerres puniques chez les successeurs de Romulus, et chez tous les Grecs depuis Périclès et surtout depuis Philippe, ce caractère d’homogénéité tendit de plus en plus à s’altérer. Le mélange plus grand des nations amena le mélange des civilisations, et il en résulta un produit extrêmement multiple, très savant, beaucoup plus raffiné que l’antique culture, qui avait cet inconvénient capital, en Italie comme dans l’Hellade, de n’exister que pour les classes supérieures, et de laisser les couches du dessous tout à fait ignorantes de sa nature, de ses mérites et de ses voies. La civilisa­tion romaine, après les grandes guerres d’Asie, fut sans doute une manifestation puissante du génie humain ; cependant, à l’exception des rhéteurs grecs, qui en fournissaient la partie transcendantale, des jurisconsultes syriens, qui vinrent lui com­poser un système de lois athée, égalitaire et monarchique, des hommes riches, engagés dans l’administration publique ou dans les entreprises d’argent, et enfin des gens de loisir et de plaisir, elle eut ce malheur de ne jamais être que subie par les masses, attendu que les peuples d’Europe ne comprenaient rien à ses éléments asiatiques et africains, que ceux de l’Égypte n’avaient pas davantage l’intelligence de ce qu’elle leur apportait de la Gaule et de l’Espagne, et que ceux de Numidie n’appréciaient pas plus ce qui leur venait du reste du monde. De sorte qu’au-dessous de ce qu’on pourrait appeler les classes sociales, vivaient des multitudes innombrables, civilisées autrement que le monde officiel, ou n’ayant pas du tout de civilisation. C’était donc la minorité du peuple romain qui, en possession du secret, y attachait quelque prix. Voilà un exemple d’une civilisation acceptée et régnante, non plus par la conviction des peuples qu’elle couvre, mais par leur épuisement, leur faiblesse, leur abandon.\par
En Chine, un tout autre spectacle se présente. Le territoire est sans doute immense ; mais, d’un bout à l’autre de cette vaste étendue, circule, chez la race nationale (je laisse les autres à l’écart), un même esprit, une même intelligence de la civilisation possédée. Quels qu’en puissent être les principes, soit qu’on en approuve ou blâme les fins, il faut avouer que les multitudes y prennent une part démonstrative de l’intelli­gence qu’elles en ont. Et ce n’est pas que ce pays soit libre dans le sens où nous l’entendons, qu’une émulation démocratique pousse tout le monde à bien faire, afin de parvenir à la place que les lois lui garantissent. Non ; j’éloigne tout tableau idéal. Les paysans comme les bourgeois sont fort peu assurés, dans l’empire du Milieu, de sortir de leur position par la seule puissance du mérite. À cette extrémité du monde, et malgré les promesses officielles du système des examens appliqué au recrutement des emplois publics, il n’est personne qui ne se doute que les familles de fonctionnaires absorbent les places, et que les suffrages scolaires coûtent souvent plus d’argent que d’efforts de science \footnote{« Il n’y a encore que la Chine où un pauvre étudiant puisse se présenter au concours « impérial et en sortir grand personnage. C’est le côté brillant de l’organisation sociale des « Chinois, et leur théorie est incontestablement la meilleure de toutes ; malheureusement « l’application est loin d’être parfaite. Je ne parle pas ici des erreurs de jugement et de la « corruption des examinateurs, ni même de la vente des titres littéraires, expédient auquel « le gouvernement a quelquefois recours en temps de détresse financière... » (F. J. Mohl, \emph{Rapport annuel fait à la Société asiatique}, 1846, p. 49.)} ; mais les ambitions lésées, en gémissant sur les torts de cette organisation, n’en imaginent pas de meilleure, et l’ensemble de la civilisation existante est pour le peuple entier l’objet d’une imperturbable admiration.\par
Chose assez remarquable, l’instruction est en Chine très répandue, générale ; elle atteint et dépasse des classes dont on ne se figure pas aisément, chez nous, qu’elles puissent même sentir des besoins de ce genre. Le bon marché des livres, la multiplicité et le bas prix des écoles, mettent les gens qui le veulent en état de s’instruire, au moins dans une mesure suffisante. Les lois, leur esprit, leurs tendances, sont très bien connues, et même le gouvernement se pique d’ouvrir à tous l’entendement sur cette science utile. L’instinct commun a la plus profonde horreur des bouleversements politi­ques. Un juge fort compétent en cette matière, qui non seulement a habité Canton, mais y a étudié les affaires avec l’attention d’un homme intéressé à les connaître, M. John Francis Davis, commissaire de S. M. Britannique en Chine, affirme qu’il a vu là une nation dont l’histoire ne présente pas une seule tentative de révolution sociale, ni de changement dans les formes du pouvoir. À son avis, on ne peut mieux la définir qu’en la déclarant composée tout entière de conservateurs déterminés.\par
C’est là un contraste bien frappant avec la civilisation du monde romain, où les modifications gouvernementales se suivirent dans une si effrayante rapidité jusqu’à l’arrivée des nations du Nord. Sur tous les points de cette grande société on trouvait toujours et facilement des populations assez désintéressées de l’ordre existant pour se montrer prêtes à servir les plus folles tentatives. Il n’y eut rien d’inessayé pendant cette longue période de plusieurs siècles, pas de principe respecté. La propriété, la religion, la famille soulevèrent, là comme ailleurs, des doutes considérables sur leur légitimité et des masses nombreuses se trouvèrent disposées, soit au nord, soit au sud, à appliquer de force les théories des novateurs. Rien, non rien, ne reposa, dans le monde gréco-romain, sur une base solide, pas même l’unité impériale, si indispensable pourtant, ce semble, au salut commun, et ce ne furent pas seulement les armées, avec leurs nuées d’Augustes improvisés, qui se chargèrent d’ébranler constamment ce palladium de la société ; les empereurs eux-mêmes, à commencer par Dioclétien, croyaient si faiblement à la monarchie, qu’ils essayèrent volontairement le dualisme dans le pouvoir, puis se mirent à quatre pour gouverner. Je le répète, pas une insti­tution, pas un principe ne fut stable dans cette misérable société, qui ne possédait pas de meilleure raison d’être que l’impossibilité physique d’échouer d’un côté ou de l’autre, jusqu’au moment où des bras vigoureux vinrent, en la démantelant, la forcer de devenir quelque chose de défini.\par
Ainsi nous trouvons chez deux grands êtres sociaux, l’Empire Céleste et le monde romain, une parfaite opposition. À la civilisation de l’Asie orientale j’ajouterai la civilisation brahmanique, dont il faut en même temps admirer l’intensité et la diffusion. Si, en Chine, un certain niveau de connaissances atteint tout le monde, ou presque tout le monde, il en est de même parmi les Hindous : chacun, dans sa caste, est animé d’un esprit séculaire, et connaît nettement ce qu’il doit apprendre, penser et croire. Chez les bouddhistes du Thibet et des autres parties de la haute Asie, rien de plus rare que de rencontrer un paysan ne sachant pas lire. Tout le monde y a des convictions pareilles sur les sujets importants.\par
Trouvons-nous la même homogénéité dans nos nations européennes ? La question ne vaut pas la peine d’être posée. À peine l’empire gréco-romain nous offre-t-il des nuances, des couleurs aussi tranchées, non pas entre les différents peuples, mais je dis dans le sein des mêmes nationalités. Je glisserai sur ce qui concerne la Russie et une grande partie des États autrichiens ; ma démonstration y serait trop facile. Voyons l’Allemagne, ou bien l’Italie, l’Italie méridionale surtout ; l’Espagne, bien qu’à un moindre degré, présenterait un pareil tableau ; la France, de même.\par
Prenons la France : je ne dirai pas seulement que la différence des manières y frappe si bien les observateurs les plus superficiels, que l’on s’est aperçu depuis longtemps qu’entre Paris et le reste du territoire il y a un abîme, et qu’aux portes mêmes de la capitale, commence une nation tout autre que celle qui est dans les murs. Rien de plus vrai ; les gens qui se fient à l’unité politique établie chez nous pour en conclure l’unité des idées et la fusion du sang, se livrent à une grande illusion.\par
Pas une loi sociale, pas un principe générateur de la civilisation compris de la même manière dans tous nos départements. Il est inutile de faire comparaître ici le Normand, le Breton, l’Angevin, le Limousin, le Gascon, le Provençal ; tout le monde doit savoir combien ces peuples se ressemblent peu et varient dans leurs jugements. Ce qu’il faut signaler, c’est que, tandis qu’en Chine, au Thibet et dans l’Inde, les notions les plus essentielles au maintien de la civilisation sont familières à toutes les classes, il n’en est aucunement de même chez nous. La première, la plus élémentaire de nos connaissances, la plus abordable, reste un mystère fort négligé par la masse de nos populations rurales : car très généralement on n’y sait ni lire ni écrire, et on n’attache aucune importance à l’apprendre, parce qu’on n’en voit pas l’utilité, parce qu’on n’en trouve pas l’application. Sur ce point-là, je crois peu aux promesses des lois, aux beaux semblants des institutions, beaucoup à ce que j’ai vu moi-même, et aux faits constatés par de bons observateurs. Les gouvernements ont épuisé les efforts les plus louables pour tirer les paysans de leur ignorance ; non seulement les enfants trouvent, dans leurs villages, toutes facilités pour s’instruire, mais les adultes même, saisis, à l’âge de vingt ans, par la conscription, rencontrent, dans les écoles régimentaires, les meilleurs moyens d’acquérir les connaissances les plus indispensables. Malgré ces précautions, malgré cette paternelle sollicitude et ce perpétuel \emph{compelle intrare} dont, tous les jours, l’administration répète l’avis à ses agents, les classes agricoles n’apprennent rien. J’ai vu, et toutes les personnes qui ont habité la province l’ont vu comme moi, les parents n’envoyer leurs enfants à l’école qu’avec une répugnance marquée, et taxer de temps perdu les heures qui s’y passent ; les en retirer en hâte, sous le plus léger prétexte, ne jamais permettre que les premières années de force s’y prolongent ; et quand une fois l’école est quittée, le jeune homme n’a rien de plus pressé que d’oublier ce qu’il y a appris. Il s’en fait, en quelque sorte, un point d’honneur, ce en quoi il est imité par les soldats congédiés, qui, dans plus d’une partie de la France, non seulement ne veulent plus avoir su lire et écrire, mais, affectant même d’oublier le français, y parviennent souvent. J’approuverais donc, avec plus de tranquillité d’âme, tant d’efforts généreux vainement dépensés pour instruire nos populations rurales, si je n’étais convaincu que la science qu’on veut leur donner ne leur convient pas, et qu’il y a, au fond de leur nonchalance apparente, un sentiment invinciblement hostile à notre civilisation. J’en trouve une preuve dans cette résistance passive ; mais ce n’est pas la seule, et là où on parvient, avec l’aide de circonstances qui semblent favorables, à faire céder cette obstination, une autre preuve plus convaincante encore m’apparaît et me poursuit. Sur quelques points, on réussit mieux dans les tentatives d’instruction. Nos départements de l’est et nos grandes villes manufacturières comptent beaucoup d’ouvriers qui apprennent volontiers à lire et à écrire. Ils vivent dans un milieu qui leur en démontre l’utilité. Mais aussitôt que ces hommes possèdent à un degré suffisant les premiers éléments de l’instruction, qu’en font-ils pour la plupart ? Des moyens d’acquérir telles idées et tels sentiments non plus instinctivement, mais désormais activement hostiles à l’ordre social. Je ne fais une exception que pour nos populations agricoles et même ouvrières du nord-est, où les connaissances élémentaires sont beaucoup plus répandues que partout ailleurs, conservées une fois acquises, et ne portent généralement que de bons fruits. On remarquera que ces populations tiennent de beaucoup plus près que toutes les autres à la race germanique, et je ne m’étonne pas de les voir ce qu’elles sont. Ce que je dis ici de nos départements du nord-est s’applique à la Belgique et à la Néerlande.\par
Si, après avoir constaté le peu de goût pour notre civilisation, nous considérons le fond des croyances et des opinions, l’éloignement devient encore plus remarquable. Quant aux croyances, c’est encore là qu’il faut remercier la foi chrétienne de n’être pas exclusive et de n’avoir pas voulu imposer un formulaire trop étroit. Elle aurait rencontré des écueils bien dangereux. Les évêques et les curés ont à lutter, non moins aujourd’hui qu’il y a un siècle, qu’il y en a cinq, qu’il y en a quinze, contre des préventions et des tendances transmises héréditairement, et d’autant plus à redouter que, ne s’avouant presque jamais, elles ne se laissent ni combattre ni vaincre. Il n’est pas de prêtre éclairé, ayant évangélisé des villages, qui ne sache avec quelle astuce profonde le paysan, même dévot, continue à cacher, à caresser au fond de son esprit, quelque idée traditionnelle dont l’existence ne se révèle que malgré lui et dans de rares instants. Lui en parle-t-on ? il nie, n’accepte jamais la discussion et demeure inébranla­blement convaincu. Il a dans son pasteur toute confiance, toute, jusqu’à ce qu’on pourrait appeler sa religion secrète exclusivement, et de là cette taciturnité qui, dans toutes nos provinces, est le caractère le plus marqué du paysan vis-à-vis de ce qu’il appelle le bourgeois, et cette ligne de démarcation si infranchissable entre lui et les propriétaires les plus aimés de son canton. Voilà, à l’encontre de la civilisation, l’attitude de la majorité de ce peuple qui passe pour y être le plus attaché ; je serais porté à croire que si, dressant une sorte de statistique approximative, on disait qu’en France 10 millions d’âmes agissent dans notre sphère de sociabilité, et que 26 millions restent en dehors, on serait au-dessous de la vérité.\par
Et encore si nos populations rurales n’étaient que grossières et ignorantes, on pourrait se préoccuper médiocrement de cette séparation, et se consoler par l’espoir vulgaire de les conquérir peu à peu et de les fondre dans les multitudes déjà éclairées. Mais il en est de ces masses absolument comme de certains sauvages : au premier abord, on les juge irréfléchissantes et à demi brutes, parce que l’extérieur est humble et effacé ; puis à mesure qu’on pénètre, si peu que ce soit, au sein de leur vie particulière, on s’aperçoit qu’elles n’obéissent pas, dans leur isolement volontaire, à un sentiment d’impuissance. Leurs affections et leurs antipathies ne vont pas au hasard, et tout, chez elles, concorde dans un enchaînement logique d’idées fort arrêtées. En parlant tout à l’heure de la religion, j’aurais pu faire remarquer aussi quelle distance immense sépare nos doctrines morales de celles des paysans, \footnote{Une nourrice tourangelle avait mis un oiseau dans les mains de son nourrisson, enfant de trois ans, et l’excitait à lui arracher plumes et ailes. Comme les parents lui reprochaient cette leçon de méchanceté : « C’est pour le rendre fier, » répliqua-t-elle. Cette réponse de 1847 descend des maximes d’éducation en vigueur au temps de Vercingétorix.} combien ce qu’ils appelleraient \emph{délicatesse} est différent de ce que nous entendons sous ce nom ; et, enfin, avec quelle ténacité ils continuent à regarder tout ce qui n’est pas, comme eux, paysan, sous le même aspect que les hommes de la plus lointaine antiquité considéraient l’étranger. À la vérité, ils ne le tuent pas, grâce à la terreur, même singulière et mystérieuse, que leur inspirent des lois qu’ils n’ont point faites ; mais ils le haïssent franchement, s’en défient, et, quant à ce qui est de le rançonner, s’en donnent à cœur joie, lorsqu’ils le peuvent sans trop de risques. Sont-ils donc méchants ? Non, pas entre eux ; on les voit échanger de bons procédés et des complaisances. Seulement ils se regardent comme une autre espèce, espèce, à les en croire, opprimée, faible, qui doit avoir son recours à la ruse, mais qui garde aussi son orgueil très tenace, très méprisant. Dans quelques-unes de nos provinces, le laboureur s’estime de beaucoup meilleur sang et de plus vieille souche que son ancien seigneur. L’orgueil de famille, chez certains paysans, égale aujourd’hui, pour le moins, ce qu’on observait dans la noblesse du moyen âge \footnote{Il s’agissait, il y a très peu d’années, d’élire un marguillier dans une très petite et très obscure paroisse de la Bretagne française, cette partie de l’ancienne province que les vrais Bretons appellent le \emph{pays gallais.} Le conseil de fabrique, composé de paysans, délibéra pendant deux jours sans pouvoir se décider à faire un choix, attendu que le candidat présenté, fort honnête homme, très bon chrétien, riche et considéré, était pourtant \emph{étranger.} On. n’en démordait pas, et pourtant cet \emph{étranger} était né dans le pays, son père également ; mais on se souvenait encore que son grand-père, mort depuis longues années et que personne de l’assemblée n’avait connu, était venu d’ailleurs. – Une fille de cultivateur-propriétaire se mésallie quand elle épouse un tailleur, un meunier eu même un fermier à gages, fût-il plus riche qu’elle, et la malédiction paternelle punit souvent ce crime-là. Ne sont-ce pas des opinions bien chapitrales ?}.\par
Qu’on n’en doute pas, le fond de la population française n’a que peu de points communs avec sa surface ; c’est un abîme au-dessus duquel la civilisation est suspen­due, et les eaux profondes et immobiles, dormant au fond du gouffre, se montreront, quelque jour, irrésistiblement dissolvantes. Les événements les plus tragiques ont ensanglanté le pays, sans que la nation agricole y ait cherché une autre part que celle qu’on la forçait d’y prendre. Là où son intérêt personnel et direct ne s’est pas trouvé en jeu, elle a laissé passer les orages sans s’y mêler, même par la sympathie. Effrayées et scandalisées à ce spectacle, beaucoup de personnes ont prononcé que les paysans étaient essentiellement pervers ; c’est tout à la fois une injustice et une très fausse appréciation. Les paysans nous regardent presque comme des ennemis. Ils n’entendent rien à notre civilisation, ils n’y contribuent pas de leur gré, et, en tant qu’ils le peuvent, ils se croient autorisés à profiter de ses désastres. Si on les considère en dehors de cet antagonisme, quelquefois actif, le plus souvent inerte, on ne révoque plus en doute que de hautes qualités morales, quoique souvent très singulièrement appliquées, ne résident chez eux.\par
J’applique à toute l’Europe ce que je viens de dire de la France, et j’en infère que, pareil en ceci à l’empire romain, le monde moderne embrasse infiniment plus qu’il n’étreint. On ne peut donc accorder beaucoup de confiance à la durée de notre état social, et le peu d’attachement qu’il inspire, même dans des couches de population supérieures aux classes rurales, m’en paraît une démonstration patente. Notre civilisa­tion est comparable à ces îlots temporaires poussés au-dessus des mers par la puissance des volcans sous-marins. Livrés à l’action destructive des courants et abandonnés de la force qui les avait d’abord soutenus, ils fléchissent un jour, et vont engloutir leurs débris dans les domaines des flots conquérants. Triste fin, et que bien des races généreuses ont dû subir avant nous ! Il n’y a pas à détourner le mal, il est inévitable. La sagesse ne peut que prévoir, et rien davantage. La prudence la plus consommée n’est pas capable de contrarier un seul instant les lois immuables du monde.\par
Ainsi, inconnue, dédaignée ou haïe du plus grand nombre des hommes assemblés sous son ombre, notre civilisation est pourtant un des monuments les plus glorieux que le génie de l’espèce ait jamais édifié. Ce n’est pas, à la vérité, par l’invention qu’elle se signale. Cette qualité mise à part, disons qu’elle a poussé loin l’esprit compréhensif et la puissance de la conquête, qui en est une conséquence. Comprendre tout, c’est tout prendre. Si elle n’a pas créé les sciences exactes, elle leur a donné du moins leur exactitude et les a débarrassées des divagations dont, par un singulier phénomène, elles étaient peut-être encore plus mêlées que toutes les autres connaissances. Grâce à ses découvertes, elle connaît mieux le monde matériel que ne faisaient les sociétés précé­dentes. Elle a deviné une partie de ses lois principales, elle sait les exposer, les décrire et leur emprunter des forces vraiment merveilleuses pour centupler celles de l’homme. De proche en proche et par la rectitude avec laquelle elle manie l’induction, elle a reconstruit d’immenses fragments de l’histoire, dont les anciens ne s’étaient jamais doutés, et, plus elle s’éloigne des époques primitives, plus elle les voit et pénètre leurs mystères. Ce sont là de grandes supériorités, et qu’on ne saurait lui disputer sans injustice.\par
Ceci admis, est-on bien en droit d’en conclure, comme on le fait généralement avec trop de facilité, que notre civilisation ait la préexcellence sur toutes celles qui ont existé et existent en dehors d’elle ? Oui et non. Oui, parce qu’elle doit à la prodigieuse diversité des éléments qui la composent, de reposer sur un esprit puissant de compa­raison et d’analyse, qui lui rend plus facile l’appropriation de presque tout ; oui, parce que cet éclectisme favorise ses développements dans les sens les plus divers ; oui, encore, parce que, grâce aux conseils du génie germanique, trop utilitaire pour être destructeur, elle s’est fait une moralité dont les sages exigences étaient inconnues généralement jusqu’à elle. Mais, si l’on pousse cette idée de son mérite jusqu’à la déclarer supérieure absolument et sans réserve, je dis non, car précisément elle n’excelle en presque rien.\par
 Dans l’art du gouvernement, on la voit soumise, en esclave, aux oscillations inces­santes amenées par les exigences des races si tranchées qu’elle renferme. En Angleterre, en Hollande, à Naples, en Russie, les principes sont encore assez stables, parce que les populations sont plus homogènes, ou du moins appartiennent à des groupes de la même catégorie et ont des instincts similaires. Mais, partout ailleurs, surtout en France, dans l’Italie centrale, en Allemagne, où la diversité ethnique est sans bornes, les théories gouvernementales ne peuvent jamais s’élever à l’état de vérités, et la science politique est en perpétuelle expérimentation. Notre civilisation, rendue ainsi incapable de prendre une croyance ferme en elle-même, manque donc de cette stabilité qui est un des principaux caractères que j’ai dû comprendre plus haut dans la formule de définition. Comme on ne trouve pas cette triste impuissance au milieu des sociétés bouddhiques et brahmaniques, comme le Céleste Empire ne la connaît pas non plus, c’est un avantage que ces civilisations ont sur la nôtre. Là, tout le monde est d’accord quant à ce qu’il faut croire en matière politique. Sous une sage administration, quand les institutions séculaires portent de bons fruits, on se réjouit. Lorsque, entre des mains maladroites, elles nuisent au bien-être public, on les plaint comme on se plaint soi-même. Mais, en aucun temps, le respect ne cesse de les entourer. On veut quelquefois les épurer, jamais les mettre à néant ni les remplacer par d’autres. Il faudrait être aveugle pour ne pas voir là une garantie de longévité que notre civilisation est bien loin de comporter.\par
Au point de vue des arts, notre infériorité vis-à-vis de l’Inde est marquée, tout autant qu’en face de l’Égypte, de la Grèce et de l’Amérique. Ni dans le grandiose, ni dans le beau, nous n’avons rien de comparable aux chefs-d’œuvre des races antiques, et lorsque, nos jours étant consommés, les ruines de nos monuments et de nos villes couvriront la face de nos contrées, certainement le voyageur ne découvrira rien, dans les forêts et les marécages des bords de la Tamise, de la Seine et du Rhin, qui rivalise avec les somptueuses ruines de Philæ, de Ninive, du Parthénon, de Salsette, de la vallée de Tenochtitlan. Si, dans le domaine des sciences positives, les siècles futurs ont à apprendre de nous, il n’en est pas ainsi pour la poésie. L’admiration désespérée que nous avons vouée, avec tant de justice, aux merveilles intellectuelles des civilisations étrangères, en est une preuve surabondante.\par
Parlant maintenant du raffinement des mœurs, il est de toute évidence que nous y sommes primés de tous côtés. Nous le sommes par notre propre passé, où il se trouve des moments pendant lesquels le luxe, la délicatesse des habitudes et la somptuosité de la vie étaient compris d’une manière infiniment plus dispendieuse, plus exigeante et plus large que de nos jours, À la vérité, les jouissances étaient moins généralisées. Ce qu’on appelle \emph{bien-être} n’appartenait comparativement qu’à peu de monde. Je le crois : mais, s’il faut admettre, fait incontestable, que l’élégance des mœurs élève autant l’esprit des multitudes spectatrices qu’elle ennoblit l’existence des individus favorisés, et qu’elle répand sur tout le pays dans lequel elle s’exerce un vernis de grandeur et de beauté, devenu le patrimoine commun, notre civilisation, essentiellement mesquine dans ses manifestations extérieures, n’est pas comparable à ses rivales.\par
 Je terminerai ce chapitre en faisant observer que le caractère primitivement organisateur de toute civilisation est identique avec le trait le plus saillant de l’esprit de la race dominatrice ; que la civilisation s’altère, change, se transforme à mesure que cette race subit elle-même de tels effets ; que c’est dans la civilisation que se continue, pen­dant une durée plus ou moins longue, l’impulsion donnée par une race qui cependant a disparu, et, par conséquent, que le genre d’ordre établi dans une société est le fait qui accuse le mieux les aptitudes particulières et le degré d’élévation des peuples ; c’est le miroir le plus clair où ils puissent refléter leur individualité.\par
Je m’aperçois que j’ai fait une digression bien longue, et dont les ramifications se sont étendues plus loin que je ne comptais. Je ne le regrette pas trop. J’ai pu émettre, à cette occasion, certaines idées qui devaient nécessairement passer sous les yeux du lecteur. Cependant il est temps que je rentre dans le courant naturel de mes déductions. La série est encore loin d’être complète.\par
J’ai posé d’abord cette vérité, que la vie ou la mort des sociétés résultait de causes internes. J’ai dit quelles étaient ces causes. Je me suis adressé à leur nature intime pour les pouvoir reconnaître. J’ai démontré la fausseté des origines qu’on leur attribue généralement. En cherchant un signe qui pût les dénoncer constamment, et servir à constater, dans tous les cas, leur existence, j’ai trouvé l’aptitude à créer la civilisation, mise en regard de l’impossibilité de concevoir cet état. C’est de cette recherche que je sors en ce moment. Maintenant quel est le premier point dont je dois m’occuper ? C’est incontestablement, après avoir reconnu en elle-même la cause latente de la vie ou de la mort des sociétés à un signe naturel et constant, d’étudier la nature intime de cette cause. J’ai dit qu’elle dérivait du mérite relatif des races. La logique exige donc que je précise immédiatement ce que j’entends par le mot race, et c’est ce qui fera l’objet du chapitre suivant.
\section[{I.10. Certains anatomistes attribuent à l’humanité des origines multiples.}]{I.10. \\
Certains anatomistes attribuent à l’humanité des origines multiples.}
\noindent Il faut interroger, d’abord, le mot \emph{race} dans sa portée physiologique.\par
L’opinion d’un grand nombre d’observateurs, procédant de la première impression et jugeant sur les extrêmes \footnote{M. Flourens, \emph{Éloge de Blumenbach, Mémoires de l’Académie des sciences}, Paris, 1847, in-4°, p. XIII. Ce savant se prononce, avec raison, contre cette méthode.}, déclare que les familles humaines sont marquées de différences tellement radicales, tellement essentielles, qu’on ne peut faire moins que de leur refuser l’identité d’origine. À côté de la descendance adamique, les érudits ralliés à ce système supposent plusieurs autres généalogies. Pour eux l’unité primordiale n’existe pas dans l’espèce, ou, pour mieux dire, il n’y a pas une seule espèce ; il y en a trois, quatre, et davantage, d’où sont issues des générations parfaitement distinctes, qui, par leurs mélanges, ont formé des hybrides.\par
Pour appuyer cette théorie, on s’empare assez aisément de la conviction commune en plaçant sous les yeux du critique les dissemblances évidentes, claires, frappantes des groupes humains. Lorsque l’observateur se voit mettre en face d’un sujet à carnation jaunâtre, à barbe et cheveux rares, à masque large, à crâne pyramidal, aux yeux fortement obliques, à la peau des paupières si étroitement tendue vers l’angle externe que l’œil s’ouvre à peine, à la stature assez humble et aux membres lourds \footnote{Prichard, \emph{Histoire nat. de l’homme}, t. I, p. 133, 146, 162.}, cet observateur reconnaît un type bien caractérisé, bien marqué, et dont il est certainement facile de garder les principaux traits dans la mémoire.\par
 Un autre individu paraît : c’est un nègre de la côte occidentale d’Afrique, grand, d’aspect vigoureux, aux membres lourds, avec une tendance marquée à l’obésité \footnote{\emph{Id., ibid.}, t. I, p. 108, 134, 174.}. La couleur n’est plus jaunâtre, mais entièrement noire ; les cheveux ne sont plus rares et effilés, mais, au contraire, épais, grossiers, laineux et poussant avec exubérance ; la mâchoire inférieure avance en saillie, le crâne affecte cette forme que l’on a appelée \emph{prognathe}, et quant à la stature, elle n’est pas moins particulière. « Les os longs sont déjetés en dehors, le tibia et le « péroné sont, en avant, plus convexes que chez les Européens, les mollets sont « très hauts et atteignent jusqu’au jarret ; les pieds sont très plats, et le « calcanéum, au lieu d’être arqué, se continue presque en ligne droite avec les « autres os du pied, qui est remarquablement large. La main présente aussi, « dans sa disposition générale, quelque chose d’analogue \footnote{Id.,\emph{ ibid.}, passim.}. »\par
Quand l’œil s’est fixé un instant sur un individu ainsi conformé, l’esprit se rappelle involontairement la structure du singe et se sent enclin à admettre que les races nègres de l’Afrique occidentale sont sorties d’une souche qui n’a rien de commun, sinon certains rapports généraux dans les formes, avec la famille mongole.\par
Viennent ensuite des tribus dont l’aspect flatte moins encore que celui du nègre congo l’amour-propre de l’humanité. C’est un mérite particulier de l’Océanie que de fournir les spécimens à peu près les plus dégradés, les plus hideux, les plus repous­sants de ces êtres misérables, formés, en apparence, pour servir de transition entre l’homme et la brute pure et simple. Vis-à-vis de plusieurs tribus australiennes, le nègre africain, lui-même, se rehausse, prend de la valeur, semble trahir une meilleure descendance. Chez beaucoup des malheureuses populations de ce monde dernier trouvé, la grosseur de la tête, l’excessive maigreur des membres, la forme famélique du corps, présentent un aspect hideux. Les cheveux sont plats ou ondulés, plus souvent laineux, la carnation est noire, sur un fond gris \footnote{Prichard, ouvrage cité, t. II, p. 71.}.\par
Enfin, si, après avoir examiné ces types pris dans tous les coins du globe, on revient aux habitants de l’Europe, du sud et de l’ouest de l’Asie, on leur trouve une telle supériorité de beauté, de justesse dans la proportion des membres, de régularité dans les traits du visage, que, tout de suite, on est tenté d’accepter la conclusion des partisans de la multiplicité des races. Non seulement, les derniers peuples que je viens de nommer sont plus beaux que le reste de l’humanité, compendium assez triste, il faut en convenir, de bien des laideurs \footnote{C’est parce que Meiners était extrêmement frappé de cet aspect repoussant de la plus grande partie des variétés humaines, qu’il avait imaginé une classification des plus simples ; elle n’était composée que de deux catégories : la \emph{belle}, c’est-à-dire la race blanche, et la \emph{laide}, qui renfermait toutes les autres. (Meiners, \emph{Grundriss der Geschichte der Menschheit.}) On s’apercevra que je n’ai pas cru devoir passer en revue tous les systèmes ethnologiques. Je ne me suis arrêté qu’aux plus importants.} ; non seulement ces peuples ont eu la gloire de fournir les modèles admirables de la Vénus, de l’Apollon et de l’Hercule Farnèse ; mais, de plus, entre eux, une hiérarchie visible est établie de toute antiquité, et, dans cette noblesse humaine, les Européens sont les plus éminents par la beauté des formes et la vigueur du développement musculaire. Rien donc qui semble plus raisonnable que de déclarer les familles dont l’humanité se compose aussi étrangères, l’une à l’autre, que le sont, entre eux, les animaux d’espèces différentes.\par
Telle fut aussi la conclusion tirée des premières remarques, et, tant que l’on ne prononça que sur des faits généraux, il ne sembla pas que rien pût l’infirmer.\par
Camper, un des premiers, systématisa ces études. Il ne se contenta plus de décider uniquement d’après des témoignages superficiels ; il voulut asseoir ses démonstrations d’une manière mathématique, et chercha à préciser, anatomiquement, les différences caractéristiques des catégories humaines. En réussissant, il établissait une méthode stricte qui ne laissait plus de place aux doutes, et ses opinions acquéraient cette rigueur sans laquelle il n’y a point véritablement de science. Il imagina donc de prendre la face latérale de la tête osseuse, et de mesurer l’ouverture du profil au moyen de deux lignes appelées, par lui, \emph{lignes faciales.} Leur intersection formait un angle, qui, par sa plus ou moins grande ouverture, devait donner la mesure du degré d’élévation de la race. L’une de ces lignes allait de la base du nez au méat auditif ; l’autre était tangente à la saillie du front par le haut, et par en bas à la partie la plus proéminente de la mâchoire inférieure. Au moyen de l’angle ainsi formé, on établissait, non seulement pour l’homme, mais pour toutes les classes d’animaux, une échelle dont l’Européen formait le sommet ; et plus l’angle était aigu, plus les sujets s’éloignaient du type qui, dans la pensée de Camper, résumait le plus de perfection. Ainsi, les oiseaux formaient avec les poissons, le plus petit angle. Les mammifères des différentes classes l’agrandissaient. Une certaine espèce de singe montait jusqu’à 42 degrés, même jusqu’à 50. Puis venait la tête du nègre d’Afrique, qui, ainsi que celle du Kalmouk, en présentait 70. L’Européen atteignait 80, et, pour citer les paroles mêmes de l’inventeur, paroles si flatteuses pour notre congénère : « C’est, dit-il, de cette différence de « 10 degrés que dépend sa beauté plus grande, ce qu’on peut appeler sa beauté « comparative. Quant à cette beauté absolue qui nous frappe à un si haut degré « dans quelques œuvres de la statuaire antique, comme dans la tête de l’Apollon « et dans la Méduse de Sosiclès, elle résulte d’une ouverture encore plus grande « de l’angle, qui, dans ce cas, atteint jusqu’à 100 degrés \footnote{Prichard, ouvrage cité, t. I, p. 152.}. »\par
Cette méthode était séduisante par sa simplicité. Malheureusement, elle eut contre elle les faits, accident arrivé à bien des systèmes. Owen établit, par une série d’observations sans réplique, que Camper n’avait étudié la conformation de la tête osseuse des singes que sur de jeunes sujets, et que, chez les individus parvenus à l’âge adulte, la croissance des dents, l’élargissement des mâchoires et le développement de l’arcade zygomatique n’étant pas accompagnés d’un agrandissement correspondant du cerveau, les différences avec la tête humaine sont tout autres que celles dont Camper avait établi les chiffres, puisque l’angle facial de l’orang noir ou du chimpanzé le plus favorisé de la nature ne dépasse par 30 et 35 degrés au plus. De ce chiffre aux 70 degrés du nègre et du Kalmouk, il y a trop loin pour que la série imaginée par Camper demeure admissible.\par
 La phrénologie avait marié beaucoup de ses démonstrations à la théorie du savant hollandais. On aimait à reconnaître, dans la série ascendante des animaux vers l’homme, des développements correspondants dans les instincts. Cependant les faits furent encore contraires à ce point de vue. On objecta, entre autres que l’éléphant, dont l’intelligence est incontestablement supérieure à celle des orangs-outangs, présente un angle facial beaucoup plus aigu que le leur, et, parmi les singes eux-mêmes, il s’en faut que les plus intelligents, les plus susceptibles de recevoir une sorte d’éducation domestique, appartiennent aux plus grandes espèces.\par
Outre ces deux graves défauts, la méthode de Camper présentait encore un côté très attaquable. Elle ne s’appliquait pas à toutes les variétés de la race humaine. Elle laissait en dehors de ses catégories les tribus à tête pyramidale, et c’est là cependant un caractère assez frappant.\par
Blumenbach, ayant beau jeu contre son prédécesseur, proposa, à son tour, un système : c’était d’étudier la tête de l’homme par en haut. Il appela son invention, \emph{norma verticalis}, la méthode verticale. Il assurait que la comparaison de la largeur supérieure des têtes faisait ressortir les principales différences dans la configuration générale du crâne. Suivant lui, l’étude de cette partie du corps soulève tant de remar­ques, surtout quant aux points déterminant le caractère national, qu’il est impossible de soumettre toutes ces diversités à une mesure unique de lignes et d’angles, et que, pour parvenir à une classification satisfaisante, il faut considérer les têtes sous l’aspect qui peut embrasser, d’un seul coup d’œil, le plus grand nombre de variétés. Or, son idée devait présenter cet avantage. Elle se résumait ainsi : « Placer la série des « crânes que l’on veut comparer de manière à ce que les os malaires se trouvent « sur une même ligne horizontale, comme cela a lieu quand ces crânes reposent « sur la mâchoire inférieure ; puis se placer derrière en amenant l’œil « successivement au-dessus du vertex de chacun ; de ce point, en effet, on « saisira les variétés dans la forme des parties qui contribuent le plus au « caractère national, soit qu’elles consistent dans la direction des os maxillaires « et malaires, soit qu’elles dépendent de la largeur ou de l’étroitesse du contour « ovale présenté par le vertex ; soit, enfin, qu’elles se trouvent dans la « confi­guration aplatie ou bombée de l’os frontal \footnote{Prichard, ouvrage cité, t. I, p. 157.}. »\par
La conséquence de ce système fut, pour Blumenbach, une division de l’humanité en cinq grandes catégories, partagées à leur tour en un certain nombre de genres et de types.\par
Plusieurs doutes s’attachèrent à cette classification. On put lui reprocher, avec raison, comme à celle de Camper, de négliger plusieurs caractères importants, et ce fut, en partie, pour en éviter les objections principales qu’Owen proposa d’examiner les crânes non plus par leur sommet, mais par leur base. Un des résultats principaux de cette nouvelle façon de procéder était de trouver définitivement une ligne de démarca­tion si nette et si forte ,entre l’homme et l’orang, qu’il devenait à jamais impossible de retrouver entre les deux espèces le lien imaginé par Camper. En effet, le premier coup d’œil jeté sur deux crânes, l’un d’orang, l’autre d’homme, examinés par leurs bases, suffit pour faire apercevoir des différences capitales. Le diamètre antéro-postérieur est plus allongé chez l’orang que chez l’homme ; l’arcade zygomatique, au lieu de se trouver comprise dans la moitié antérieure de la base crânienne, forme, dans la région moyenne, juste un tiers de la longueur totale du diamètre ; enfin, la position du trou occipital, si intéressante par ses rapports avec le caractère général des formes de l’individu, et surtout par l’influence qu’elle exerce sur les habitudes, n’est nullement la même. Chez l’homme, elle occupe presque le milieu de la base du crâne ; chez l’orang, elle se trouve repoussée au milieu du tiers postérieur \footnote{Prichard, ouvrage cité, t. I, p. 60.}.\par
Le mérite des observations d’Owen est grand, sans doute ; je préférerais cependant le plus récent des systèmes cranioscopiques, qui en est, en même temps, le plus ingénieux, à bien des égards, celui du savant américain M. Morton, adopté par M. Carus \footnote{Carus\emph{, Ueber ungleiche Befæhigung}, etc., p. 19.}. Voici en quoi il consiste :\par
Pour démontrer la différence des races, les deux savants que je cite sont partis de cette idée, que plus les crânes sont vastes, plus, en thèse générale, les individus aux­quels appartiennent ces crânes se montrent supérieurs \footnote{Id., \emph{ibid.}, p. 20.}. La question posée est donc celle-ci : Le développement du crâne est-il égal chez toutes les catégories humaines ?\par
Pour obtenir la solution voulue, M. Morton a pris un certain nombre de têtes appartenant à des blancs, à des Mongols, à des nègres, à des Peaux-Rouges de l’Amérique du Nord, et, bouchant avec du coton toutes les ouvertures, sauf le \emph{foramen magnum}, il a rempli complètement l’intérieur de grains de poivre soigneusement séchés ; puis il a comparé les quantités ainsi contenues. Cet examen lui a fourni le tableau suivant \footnote{Ouvrage cité, p. 19.} :\par

\tableopen{}
\begin{tabularx}{\linewidth}
{|l|X|X|X|X|X|}
\hline\multicolumn{2}{l}{} & 1 & \multicolumn{2}{l}{2} & 3 & \multicolumn{2}{l}{4} \\
\hline
 &  & \multicolumn{2}{l}{ 
\begin{center}
\noindent Nombre\par
\end{center}

 
\begin{center}
\noindent des crânes\par
\end{center}

 
\begin{center}
\noindent mesurés
\end{center}

 } &  
\begin{center}
\noindent Moyenne\par
\end{center}

 
\begin{center}
\noindent du chiffre\par
\end{center}

 
\begin{center}
\noindent de capacité
\end{center}

  & \multicolumn{2}{l}{ 
\begin{center}
\noindent Maximum\par
\end{center}

 
\begin{center}
\noindent de\par
\end{center}

 
\begin{center}
\noindent capacité
\end{center}

 } &  
\begin{center}
\noindent Minimum\par
\end{center}

 
\begin{center}
\noindent de\par
\end{center}

 
\begin{center}
\noindent capacité
\end{center}

  \\
\hline
 \noindent Peuples blancs
  &  & \multicolumn{2}{l}{ 
\begin{center}
\noindent 52
\end{center}

 } &  
\begin{center}
\noindent 87
\end{center}

  & \multicolumn{2}{l}{ 
\begin{center}
\noindent 109
\end{center}

 } &  
\begin{center}
\noindent 75
\end{center}

  \\
\hline
Peuples jaunes &  \noindent Mongols\par
 Malais
  & \multicolumn{2}{l}{ 
\begin{center}
\noindent 10\par
\end{center}

 
\begin{center}
\noindent 18
\end{center}

 } &  
\begin{center}
\noindent 83\par
\end{center}

 
\begin{center}
\noindent 81
\end{center}

  & \multicolumn{2}{l}{ 
\begin{center}
\noindent 93\par
\end{center}

 
\begin{center}
\noindent 89
\end{center}

 } &  
\begin{center}
\noindent 69\par
\end{center}

 
\begin{center}
\noindent 64
\end{center}

  \\
\hline
\multicolumn{2}{l}{Peaux-Rouges} & \multicolumn{2}{l}{147} & 82 & \multicolumn{2}{l}{100} & 60 \\
\hline
\multicolumn{2}{l}{Nègres} & \multicolumn{2}{l}{29} & 78 & \multicolumn{2}{l}{94} &  
\begin{center}
\noindent 65
\end{center}

  \\
\hline
\end{tabularx}
\tableclose{}

\noindent Les résultats inscrits dans les deux premières colonnes sont certainement très curieux. En revanche, j’attache peu de prix à ceux des deux dernières ; car pour que la violente perturbation qu’elles semblent apporter dans les observations de la seconde colonne fût réelle, il faudrait, d’abord, que M. Morton eût opéré sur un nombre beau­coup plus considérable de crânes, et, ensuite, qu’il eût spécifié la position sociale des personnes auxquelles les crânes auraient appartenu. Ainsi il a pu avoir d’assez beaux sujets pour les blancs et les Peaux-Rouges : il s’est procuré là des têtes ayant appar­tenu à des hommes au-dessus du niveau tout à fait vulgaire ; tandis que, pour les noirs, il n’est pas probable qu’il ait eu à sa disposition des crânes de chefs de peuplades, et, pour les jaunes, des têtes de mandarins. C’est ce qui m’explique comment il a pu attribuer le chiffre 100 à un indigène américain, tandis que le Mongol le plus intelligent qu’il ait examiné ne dépasse pas 93, et se laisse ainsi primer par le nègre même, qui atteint 94. De tels résultats sont tout à fait incomplets, fortuits et sans valeur scientifi­que et, dans de telles questions, on ne saurait éviter avec trop de soin des jugements fondés sur l’examen des individualités. Je serais donc porté à rejeter tout à fait la seconde moitié des calculs de M. Morton.\par
Je me sens également disposé à contester un détail des autres. Ainsi, dans la secon­de colonne, entre les chiffres 87, indicatif de la capacité du crâne blanc, 83 du jaune et 78 du noir, il y a gradation claire et évidente. Mais les mesures de 83, 81 et 82, données pour les Mongols, les Malais et les Peaux-Rouges, sont des moyennes qui, évidem­ment, se confondent, et d’autant mieux que M. Carus n’hésite pas à comprendre les Mongols et les Malais dans une seule et même race, c’est-à-dire, à réunir les chiffres 83 et 81. Pourquoi, dès lors, prendre 82 pour caractéristique d’une race distincte, et créer ainsi tout à fait arbitrairement, une quatrième grande subdivision humaine ?\par
Cette anomalie soutient d’ailleurs la partie faible du système de M. Carus. Le savant saxon aime à supposer que, ainsi que l’on voit notre planète passer par les quatre états de jour, de nuit, de crépuscule du soir et de crépuscule du matin, de même, il faut qu’il y ait dans l’espèce humaine, quatre subdivisions correspondantes à ces variations de la lumière. Il aperçoit là un symbole \footnote{Carus, ouvrage cité, p. 12.}, tentation toujours bien dangereuse pour un esprit raffiné M. Carus y a cédé, comme beaucoup de ses savants compa­triotes l’eussent fait à sa place. Les peuples blancs sont les peuples du jour ; les noirs, ceux de la nuit ; les jaunes, ceux du matin ou du crépuscule d’orient ; les rouges, ceux du soir ou du crépuscule d’occident. On devine assez tous les rapprochements ingénieux qui viennent se rattacher à ce tableau. Ainsi, les nations européennes, par l’éclat de leurs sciences et la netteté de leur civilisation, ont les rapports les plus évidents avec l’état lumineux, et, tandis que les noirs dorment dans les ténèbres de l’ignorance, les Chinois vivent dans un demi-jour qui leur donne une existence sociale incomplète, cependant puissante. Pour les Peaux-Rouges, disparaissant peu à peu de ce monde, où trouver une plus belle image de leur sort que le soleil qui se couche !\par
 Malheureusement, comparaison n’est pas raison, et, pour s’être abandonné indû­ment à ce courant poétique, M. Carus a gâté quelque peu sa belle théorie. Du reste, il faut avouer encore ici ce que j’ai dit pour toutes les autres doctrines ethnologiques, celles de Camper, de Blumenbach, d’Owen : M. Carus ne parvient pas à systématiser régulièrement l’ensemble des diversités physiologiques remarquées dans les races \footnote{Il en est de légères qui sont pourtant fort caractéristiques. Je mettrais de ce nombre un certain renflement des chairs aux côtés de la lèvre inférieure qui se rencontre chez les Allemands et les Anglais, je retrouve aussi cet indice d’une origine germanique dans quelques figures de l’école flamande, dans la \emph{Madone} de Rubens du musée de Dresde, dans les \emph{Satyres} et \emph{Nymphes} de la même collection, dans une \emph{joueuse de luth de Miéris}, etc. Aucune méthode craniascopique n’est en état de relever de tels détails, qui ont cependant leur valeur dans nos races si mélangées.}.\par
Les partisans de l’unité ethnique n’ont pas manqué de s’emparer de cette impuis­sance, et de prétendre que, du moment où les observations sur la conformation de la tête osseuse semblent ne pouvoir être classées de manière à formuler un système démonstratif de la séparation originelle des types, il faut en considérer les divergences, non plus comme de grands traits radicalement distinctifs, mais comme les simples résultats de causes secondes indépendantes, tout à fait destituées du caractère spécifique.\par
C’est chanter victoire un peu vite. La difficulté de trouver une méthode n’autorise pas toujours à conclure à l’impossibilité de la découvrir. Les unitaires cependant n’ont pas admis cette réserve. Pour étayer leur opinion, ils ont fait remarquer que certaines tribus appartenant à une même race, loin de présenter le même type physique, s’en écartent, au contraire, assez notablement. Pour exemple, sans tenir compte de la quotité des éléments dans chaque mélange, ils ont cité les différentes branches de la famille métisse malayo-polynésienne, et ils ont ajouté que, si des groupes dont l’origine est commune \footnote{Prichard, ouvrage cité, t. II, p. 35.} peuvent cependant revêtir des formes crâniennes et faciales totalement différentes, il en résulte que les plus grandes diversités dans ce genre ne prouvent pas la multiplicité première des origines ; que, dès lors, si étranges que puissent paraître, à des yeux européens, les types nègres ou mongols, ce n’est pas une démonstration de cette multiplicité d’origines, et que les causes de la séparation des familles humaines devant être cherchées moins haut et moins loin, on peut considérer les déviations physiologiques comme les simples résultats de certaines causes locales agissant pendant un laps de temps plus ou moins long \footnote{ \noindent Job Ludolf, dont les données sur cette matière étaient nécessairement fort incomplètes et inférieures à celles que nous possédons aujourd’hui, n’en combat pas moins, en termes très piquants, et avec des raisons sans réplique pour ce qui concerne les nègres, l’opinion acceptée par M. Prichard. Je ne résiste pas au plaisir de citer : « De nigredine Ethiopum hic agere nostri non est instituti, plerique ardoribus solis atque zonæ torridæ id tribuant. Verum etiam intra solis orbitam populi dantur, si non plane albi, saltem non prorsus nigri. Multi extra utrumque tropicum a media mundi linea longius obsunt quam Persæ aut Syri, veluti promontorii Bonæ Spei habitantes, et tamen isti surit nigerrimi. Si Africæ tantum et Chami posteris id inspectare velis, Malabares et Ceilonii aliique remotiores Asiæ populi æque nigri excipiendi erunt. Quod si causam ad cœli solique naturam referas, non homines albi in illis regionibus renascentes non nigrescunt ? Aut qui ad occultas qualitates confugiunt, melius fecerint si sese nescire, fateantur. – Jobus Ludolfus, \emph{Commentarium ad Historiam Æthiopicam}, in-fol., Norimb., p. 56. – J’ajouterai encore un passage de M. Pickering ; ce passage est court et concluant. Parlant des séjours de la race noire, le voyageur américain s’exprime ainsi : « Excluding the northern and southern extremes with the tableland of Abyssinia, it holds all the \emph{more temperate}, and fertile parts of the Continent. » Ainsi, là où il se trouve moins de noirs purs, c’est là qu’il fait le moins chaud...\par
 Pickering, \emph{The Races of Man, and their geographical distribution}, dans l’ouvrage intitulé : \emph{United States exploring Expedition during the years 1838, 1839, 1840, 1841 and 1842, under the command of Charles Wilkes, U.} S. N. ; Philadelphia, 1848, in-4°, vol. IX.
}.\par
Poursuivis par tant d’objections bonnes et mauvaises, les partisans de la multipli­cité des races ont cherché à agrandir le cercle de leurs arguments ; et, cessant de s’en tenir à la seule étude des crânes, ils ont passé à celle de l’individu humain tout entier. Pour montrer, ce qui est vrai, que les différences n’existent pas uniquement dans l’aspect de la face et dans la construction osseuse des têtes, ils ont allégué des faits non moins graves, comme la forme du bassin, la proportion relative des membres, la couleur de la peau, la nature du système pileux.\par
Camper et d’autres anatomistes avaient reconnu, depuis longtemps, que le bassin du nègre présentait quelques particularités. Le docteur Vrolik, étendant plus loin ses recherches, a observé que, pour les Européens, les différences entre le bassin de l’homme et celui de la femme sont beaucoup moins marquées, et dans la race nègre il voit, chez les deux sexes, un caractère très saillant d’animalité. Le savant d’Amsterdam, partant de l’idée que la conformation du bassin influe nécessairement sur celle du fœtus, conclut à des différences originelles \footnote{Prichard, \emph{Histoire natur. de l’homme}, t. I, p. 168.}.\par
M. Weber est venu attaquer cette théorie ; toutefois, avec peu d’avantages. Il lui a fallu reconnaître que certaines formes de bassin se rencontraient plus fréquemment dans une race que dans une autre, et tout ce qu’il a pu faire, c’est de montrer que la règle n’est pas sans exception, et que tels sujets américains, africains, mongols, présentent des formes ordinaires aux Européens. Ce n’est pas là prouver beaucoup, d’autant que M. Weber, en parlant de ces exceptions, ne paraît pas avoir été préoccupé de l’idée que leur conformation particulière pouvait n’être que le résultat d’un mélange de sang.\par
Pour ce qui est de la dimension des membres, les adversaires de l’unité de l’espèce prétendent que l’Européen est mieux proportionné. On leur répond que la maigreur des extrémités, chez les nations qui se nourrissent particulièrement de végétaux, ou dont l’alimentation est imparfaite, n’a rien qui doive surprendre ; et cette réplique est bonne assurément. Mais lorsqu’on objecte, en outre, le développement extraordinaire du buste chez les Quichuas, les critiques, décidés à ne pas le reconnaître comme caractère spéci­fique, réfutent l’argument d’une manière moins concluante : car prétendre, ainsi qu’ils le font, que cette ampleur de la poitrine s’explique, chez les montagnards du Pérou, par l’élévation de la chaîne des Andes, ce n’est pas donner une raison bien sérieuse \footnote{Prichard, Id.\emph{, ibid.}, t. II, p. 180 et passim.}. Il est dans le monde nombre de populations de montagnes, et qui sont constituées tout différemment que les Quichuas \footnote{Ni les Suisses ni les Tyroliens, ni les Highlanders de l’Écosse, ni les Slaves des Balkans, ni les tribus de l’Hymalaya n’offrent l’aspect monstrueux des Quichuas.}.\par
 Viennent ensuite les observations sur la couleur de la peau. Les Unitaires soutien­nent que là ne peut se trouver aucun caractère spécifique : d’abord, parce que cette coloration tient à des circonstances climatériques, et n’est pas permanente, assertion plus que hardie ; ensuite, parce que la couleur se prête à l’établissement de gradations infinies, par lesquelles on passe insensiblement du blanc au jaune, du jaune au noir, sans pouvoir découvrir une ligne de démarcation suffisamment tranchée. Ce fait prouve simplement l’existence d’innombrables hybrides, observation à laquelle les Unitaires ont le tort fondamental d’être constamment inattentifs. Sur le caractère spécifique des cheveux, M. Flourens apporte sa grande autorité en faveur de l’unité originelle des races.\par
Après avoir passé rapidement en revue les arguments inconsistants, j’arrive à la véritable citadelle scientifique des Unitaires. Ils possèdent un argument d’une grande force, et je l’ai réservé pour le dernier : je veux dire la facilité avec laquelle les différents rameaux de l’espèce humaine produisent des hybrides, et la fécondité de ces mêmes hybrides.\par
Les observations des naturalistes semblent avoir démontré que, dans le monde animal ou végétal, les métis ne peuvent naître que d’espèces assez parentes, et que, même dans ce cas, leurs produits sont condamnés d’avance à la stérilité. On a observé, en outre, qu’entre les espèces rapprochées, bien que la fécondation soit possible, l’accouplement est répugnant et ne s’obtient, en général, que par la ruse ou la force ; ce qui indiquerait que, dans l’état libre, le nombre des hybrides est encore plus limité que l’intervention de l’homme n’est parvenue à le faire. On en a conclu qu’il fallait mettre au nombre des caractères spécifiques la faculté de produire des individus féconds.\par
Comme rien n’autorise à croire que l’espèce humaine soit exempte de cette règle, rien non plus, jusqu’ici, n’a pu ébranler la force de l’objection qui, plus que toutes les autres, tient en échec le système des adversaires de l’unité. On affirme, il est vrai, que, dans certaines parties de l’Océanie, les femmes indigènes, devenues mères de métis européens, ne sont plus aptes à être fécondées par leurs compatriotes. En admettant ce renseignement comme exact il serait digne de servir de point de départ à des investiga­tions plus approfondies ; mais, quant à présent, on ne saurait encore s’en servir pour infirmer les principes admis sur la génération des hybrides. Il ne prouve rien contre les déductions qu’on en tire.
\section[{I.11. Les différences ethniques sont permanentes.}]{I.11. \\
Les différences ethniques sont permanentes.}
\noindent Les Unitaires affirment que la séparation des races est apparente, et due unique­ment à des circonstances locales telles que celles dont nous éprouvons aujourd’hui l’influence, ou à des déviations accidentelles de conformation dans l’auteur d’une branche. Toute l’humanité est, pour eux, accessible aux mêmes perfectionnements ; partout le type originel commun, plus ou moins voilé, persiste avec une égale force, et le nègre, le sauvage américain, le Tongouse du nord de la Sibérie peuvent et doivent, sous l’empire d’une éducation similaire, parvenir à rivaliser avec l’Européen pour la beauté des formes. Cette théorie est inadmissible.\par
On a vu plus haut quel était le plus solide rempart scientifique des Unitaires : c’est la fécondité des croisements humains. Cette observation, qui paraît présenter jusqu’ici à la réfutation de grandes difficultés, ne sera peut-être pas toujours aussi invincible, et elle ne suffirait pas à m’arrêter si je ne la voyais appuyée par un autre argument, d’une nature bien différente, qui, je l’avoue, me touche davantage : on dit que la Genèse n’admet pas, pour notre espèce, plusieurs origines.\par
Si le texte est positif, péremptoire, clair, incontestable, il faut baisser la tête : les plus grands doutes doivent céder, la raison n’a qu’à se déclarer imparfaite et vaincue, l’origine de l’humanité est une, et tout ce qui semble démontrer le contraire n’est qu’une apparence à laquelle on ne doit pas s’arrêter. Car mieux vaut laisser l’obscurité s’épaissir sur un point d’érudition que de se hasarder contre une autorité pareille. Mais si la Bible n’est pas explicite ? Si les livres saints, consacrés à tout autre chose qu’à l’éclaircissement de questions ethniques, ont été mal compris, et que, sans leur faire violence, on puisse en extraire un autre sens, alors je n’hésiterai pas à passer outre.\par
Qu’Adam soit l’auteur de notre espèce blanche, il faut l’admettre certainement. Il est bien clair que les Écritures veulent qu’on l’entende ainsi, puisque de lui descendent des générations qui incontestablement ont été blanches. Ceci posé, rien ne prouve que, dans la pensée des premiers rédacteurs des généalogies adamites, les créatures qui n’appartenaient pas à la race blanche aient passé pour faire partie de l’espèce. Il n’est pas dit un mot des nations jaunes, et ce n’est que par une interprétation dont je réussirai, je pense, dans le livre suivant, à faire ressortir le caractère arbitraire, que l’on attribue au patriarche Cham la couleur noire. Sans doute, les traducteurs, les commen­tateurs, en affirmant qu’Adam a été l’auteur de tout ce qui porte le nom d’homme, ont fait entrer dans les familles de ses fils l’ensemble des peuples venus depuis. Suivant eux, les Japhétides sont la souche des nations européennes, les Sémites occupent l’Asie antérieure, les Chamites, dont on fait, sans bonnes raisons, je le répète, une race originairement mélanienne, occupent les régions africaines. Voilà pour une partie du globe : c’est à merveille ; et la population du reste du monde, qu’en fait-on ? Elle demeure en dehors de cette classification.\par
Je n’insiste pas, en ce moment, sur cette idée. Je ne veux pas entrer en lutte apparente, même avec de simples interprétations, du moment qu’elles sont accréditées. Je me contente d’indiquer qu’on pourrait peut-être, sans sortir des limites imposées par l’Église, en contester la valeur ; puis je me rabats à chercher si, en admettant, telle quelle, la partie fondamentale de l’opinion des Unitaires, il n’y aurait pas encore moyen d’expliquer les faits autrement qu’ils ne font, et d’examiner si les différences physiques et morales les plus essentielles ne peuvent pas exister entre les races humaines et avoir toutes leurs conséquences, indépendamment de l’unité ou de la multiplicité d’origine première ?\par
On admet l’identité ethnique pour toutes les variétés canines \footnote{M. Frédéric Cuvier, entre autres, \emph{Annales du Muséum}, t. XI, p. 458.} ; qui donc, cepen­dant, ira entreprendre la thèse difficile de constater chez tous ces animaux, sans distinction de genres, les mêmes formes, les mêmes tendances, les mêmes habitudes, les mêmes qualités ? Il en est de même pour d’autres espèces, telles que les chevaux, la race bovine, les ours, etc. Partout : identité quant à l’origine, diversité pour tout le reste, et diversité si profondément établie qu’elle ne peut se perdre que par les croisements, et même alors les types ne reviennent pas à une identité réelle de caractère. Tandis que, tant que la pureté de race se maintient, les traits spéciaux restent permanents et se reproduisent, de génération en génération, sans offrir de déviations sensibles.\par
Ce fait, qui est incontestable, a conduit à se demander si, dans les espèces animales soumises à la domesticité et en ayant contracté les habitudes, on pouvait reconnaître les formes et les instincts de la souche primitive. La question paraît devoir demeurer insoluble. Il est impossible de déterminer quelles devaient être les formes et le naturel de l’individu primitif, et de combien s’en éloignent ou s’en rapprochent les déviations placées aujourd’hui sous nos yeux, Un très grand nombre de végétaux offrent le même problème. L’homme surtout, la créature la plus intéressante à connaître dans ses origines, semble se refuser à tout déchiffrement, sous ce rapport.\par
Les différentes races n’ont pas douté que l’auteur antique de l’espèce n’eût précisé­ment leurs caractères. Sur ce point, sur celui-là seul, leurs traditions sont unanimes. Les blancs se sont fait un Adam et une Ève que Blumenbach aurait déclarés caucasiques ; et un livre, frivole en apparence, mais rempli d’observations justes et de faits exacts, \emph{les Mille et une Nuits}, raconte que certains nègres donnent pour noirs Adam et sa femme ; que, ces auteurs de l’humanité ayant été créés à l’image de Dieu, Dieu est noir aussi, et les anges de même, et que le prophète de Dieu était naturellement trop favorisé pour montrer une peau blanche à ses disciples.\par
Malheureusement, la science moderne n’a pu rien faire pour simplifier le dédale de ces opinions. Aucune hypothèse vraisemblable n’a réussi à éclairer cette obscurité, et, en toute vraisemblance, les races humaines diffèrent autant de leur générateur commun, si en effet elles en ont eu un, qu’elles le font entre elles. Reste à expliquer, sur le terrain modeste et étroit où je me confine, en admettant l’opinion des Unitaires, cette déviation du type primitif.\par
Les causes en sont fort difficiles à démêler. L’opinion des Unitaires l’attribue, je l’ai dit, à l’influence du climat, de la position topographique et des habitudes. Il est impos­sible de se ranger à un pareil avis \footnote{Les Unitaires se servent constamment, pour appuyer cette thèse, de la comparaison de l’homme avec les animaux. Je viens de me prêter à ce mode de raisonnement. Cependant, je n’en voudrais pas abuser, et je ne le saurais faire, en conscience, lorsqu’il s’agit d’expliquer les modifications des espèces au moyen de l’influence des climats ; car, sur ce point, la différence entre les animaux et l’homme est radicale, et on pourrait dire spécifique. Il y a une géographie des animaux, comme une géographie des plantes ; il n’y a pas de géographie des hommes. Il est telle latitude où tels végétaux, tels quadrupèdes, tels reptiles, tels poissons, tels mollusques peuvent vivre ; et l’homme, de toutes les variétés existe également partout. C’est plus qu’il n’en faut pour expliquer une immense diversité d’organisation. Je conçois, sans nulle difficulté, que les espèces qui ne peuvent franchir tel degré du méridien ou telle élévation du relief de la terre sans mourir, subissent avec soumission l’influence des climats et en ressentent rapidement les effets dans leurs formes et leurs instincts ; mais c’est précisément parce que l’homme échappe complètement à cet esclavage, que je refuse de comparer perpétuellement sa position, vis-à-vis des forces de la nature, à celle des animaux.}, attendu que les modifications dans la constitution des races, depuis le commencement des temps historiques, sous l’empire des circons­tances ici indiquées, ne paraissent pas avoir eu l’importance qu’il faudrait leur prêter pour expliquer suffisamment tant et de si profondes dissemblances. On va le comprendre à l’instant.\par
Je suppose que deux tribus, pareilles encore au type primitif, se trouvent habiter, l’une une contrée alpestre, située dans l’intérieur d’un continent, l’autre une île de la région maritime. La condition de l’air ambiant sera toute différente pour les deux populations, la nourriture le sera de même. Si, de plus, j’attribue des moyens d’alimen­tation abondants à l’une, précaires à l’autre ; qu’en outre, je place la première sous l’action d’un climat froid, la seconde sous celle d’un soleil tropical, il est bien certain que j’aurai accumulé les contrastes locaux les plus essentiels. Le cours du temps venant ajouter ce qu’on lui suppose de forces à l’activité naturelle des agents physiques, peu à peu les deux groupes finiront certainement par revêtir quelques caractères propres qui aideront à les distinguer. Mais, fût-ce au bout d’une série de siècles, rien d’essentiel, rien d’organique n’aura changé dans leurs conformation ; et la preuve, c’est qu’on rencontre des populations séparées par le monde entier, placées dans des conditions de climat et d’existence très disparates, dont les types offrent cependant la ressemblance la plus parfaite. Tous les ethnologistes en conviennent. On a même voulu que les Hottentots fussent une colonie chinoise, tant ils ressemblent aux habitants du Céleste Empire, supposition d’ailleurs inacceptable \footnote{C’est Barrow qui a émis cette idée, se fondant sur quelques ressemblances dans les formes de la tête et sur la carnation, en effet jaunâtre, des indigènes du Cap de Bonne-Espérance. Un voyageur dont le nom m’échappe a même corroboré cette opinion de la remarque que les Hottentots portent, en général, une coiffure qui ressemble au chapeau conique des Chinois.}. On découvre, de même, une grande similitude entre le portrait qui nous est resté des anciens Étrusques et le type des Araucans de l’Amérique méridionale. La figure, les formes corporelles des Cherokees semblent se confondre tout à fait avec celles de plusieurs populations italiennes, telles que les Calabrais. La physionomie accusée des habitants de l’Auvergne, surtout chez les femmes, est bien plus éloignée du caractère commun des nations européennes que celui de plusieurs tribus indiennes de l’Amérique du Nord. Ainsi, du moment que, sous des climats éloignés et différents, et dans des conditions de vie si peu pareilles, la nature peut produire des types qui se ressemblent, il est bien clair que ce ne sont pas les agents extérieurs aujourd’hui agissants qui imposent aux types humains leurs caractères.\par
Néanmoins, on ne saurait méconnaître que les circonstances locales peuvent au moins favoriser l’intensité plus ou moins grande de certaines nuances de carnation, la tendance à l’obésité, le développement relatif des muscles de la poitrine, l’allongement des membres inférieurs ou des bras, la mesure de la force physique. Mais, encore une fois, il n’y a rien là d’essentiel, et à juger d’après les très faibles modifications que ces causes, lorsqu’elles changent de nature, apportent dans la conformation des individus, il n’y a pas à croire non plus, et c’est encore une preuve qui a du poids, qu’elles aient exercé jamais beaucoup d’action.\par
Si nous ne savons pas quelles révolutions ont pu survenir dans l’organisation physique des peuples jusqu’à l’aurore des temps historiques, nous pouvons du moins remarquer que cette période ne comprend environ que la moitié de l’âge attribué à notre espèce ; et si donc, pendant trois ou quatre mille ans, l’obscurité est impénétrable, il nous reste trois mille autres années, jusqu’au début desquelles nous pouvons remonter pour quelques nations, et tout prouve que les races alors connues, et restées, depuis ce temps, dans un état de pureté relative, n’ont pas notablement changé d’aspect, bien que quelques-unes aient cessé d’habiter les mêmes lieux, d’être soumises, par conséquent, aux mêmes causes extérieures. Je citerai les Arabes. Comme les monuments égyptiens nous les représentent, ainsi les trouvons-nous encore, non seulement dans les déserts arides de leur pays, mais dans les contrées fertiles, souvent humides, du Malabar et de la côte de Coromandel, dans les îles de la mer des Indes, sur plusieurs points de la côte septentrionale de l’Afrique, où ils sont, à la vérité, plus mélangés que partout ailleurs ; et leur trace se rencontre encore dans quelques parties du Roussillon, du Languedoc et de la plage espagnole, bien que deux siècles, à peu près, se soient écoulés depuis leur invasion, La seule influence des milieux, si elle avait la puissance, comme on le suppose, de faire et de défaire les démarcations organiques, n’aurait pas laissé subsister une telle longévité de types. En changeant de lieux, les descendants de la souche ismaélite auraient également changé de conformation.\par
Après les Arabes, je citerai les juifs, plus remarquables encore en cette affaire, parce qu’ils ont émigré dans des climats extrêmement différents, de toute façon, de celui de la Palestine, et qu’ils n’ont pas conservé davantage leur ancien genre de vie. Leur type est pourtant resté semblable à lui-même, n’offrant que des altérations tout à fait insignifiantes, et qui n’ont suffi, sous aucune latitude, dans aucune condition de pays, à altérer le caractère général de la race. Tels on voit les belliqueux Réchabites des déserts arabes, tels nous apparaissent aussi les pacifiques Israélites portugais, français, allemands et polonais. J’ai eu l’occasion d’examiner un homme appartenant à cette dernière catégorie. La coupe de son visage trahissait parfaitement son origine. Ses yeux surtout étaient inoubliables. Cet habitant du Nord, dont les ancêtres directs vivaient, depuis plusieurs générations, dans la neige, semblait avoir été bruni, de la veille, par les rayons du soleil syrien. Ainsi, force est d’admettre que le visage du Sémite a conservé, dans ses traits principaux et vraiment caractéristiques, l’aspect qu’on lui voit sur les peintures égyptiennes exécutées il y a trois ou quatre mille ans et plus ; et cet aspect se retrouve dans les circonstances climatériques les plus multiples, les mieux tranchées, également frappant, également reconnaissable. L’identité des descendants avec les ancêtres ne s’arrête pas aux traits du visage : elle persiste, de même, dans la conforma­tion des membres et dans la nature du tempérament. Les juifs allemands sont, en général, plus petits, et présentent une structure plus grêle que les hommes de race européenne, parmi lesquels ils vivent depuis des siècles. En outre, l’âge de la nubilité est, pour eux, beaucoup plus précoce que pour leurs compatriotes d’une autre race \footnote{Müller, \emph{Handbuch der Physiologie des Menschen}, t. II, p. 639.}.\par
Voilà, du reste, une assertion diamétralement opposée au sentiment de M. Prichard. Ce physiologiste, dans son zèle à prouver l’unité de l’espèce, cherche à démontrer que l’époque de la puberté, dans les deux sexes, est la même partout et pour toutes les races \footnote{Prichard, \emph{Histoire naturelle de l’homme}, t. II, p. 249, et passim.}. Les raisons qu’il met en avant sont tirées de l’Ancien Testament pour les Juifs, et, pour les Arabes, de la loi religieuse du Coran par laquelle l’âge du mariage des femmes est fixé à 15 ans et même à 18, dans l’opinion d’Abou-Hanifah.\par
Ces deux arguments paraissent fort discutables. D’abord, les témoignages bibliques ne sont guère recevables en cette matière, puisqu’ils émettent souvent des faits en dehors de la marche habituelle des choses, et que, pour en citer un, l’enfantement de Sarah, arrivé dans son extrême vieillesse, et quand Abraham lui-même comptait 100 ans, est un événement sur lequel ne peut s’appuyer un raisonnement ordinaire \footnote{\emph{Gen}., XXI, 5.}. Passant à l’opinion et aux prescriptions de la loi musulmane, je remarque que le Coran n’a pas eu uniquement l’intention de constater l’aptitude physique avant d’autoriser le mariage : il a voulu aussi que la femme fût assez avancée d’intelligence et d’éducation pour être en état de comprendre les devoirs d’un état si sérieux. La preuve en est que le Prophète met beaucoup de soin à ordonner, à l’égard des jeunes filles, la continuation de l’enseignement religieux jusqu’à l’époque des noces. À un tel point de vue, il était tout simple que ce moment fût retardé autant que possible, et que le législateur trouvât très important de développer la raison avant de se montrer aussi hâtif, dans ses autorisations, que la nature l’était dans les siennes. Ce n’est pas tout. Contre les graves témoignages qu’invoque M. Prichard, il en est d’autres plus concluants, quoique plus légers, et qui tranchent la question en faveur de mon opinion.\par
Les poètes, attachés seulement, dans leurs récits d’amour, à montrer leurs héroïnes à la fleur de leur beauté, sans se soucier du développement moral, les poètes orientaux ont toujours fait leurs amantes bien plus jeunes que l’âge indiqué par le Coran. Zélika Leïla n’ont certes pas quatorze ans. Dans l’Inde, la différence est plus marquée encore. Sakontala serait en Europe une toute jeune fille, une enfant. Le bel âge de l’amour pour une femme de ce pays-là, c’est de neuf à douze ans. Voilà donc une opinion très générale, bien établie, bien admise dans les races indiennes, persanes et arabes, que le printemps de la vie, chez les femmes, éclôt à une époque un peu précoce pour nous. Longtemps nos écrivains ont pris l’avis, en cette matière, des anciens modèles de Rome. Ceux-ci, d’accord avec leurs instituteurs de la Grèce, acceptaient quinze ans pour le bel âge. Depuis que les idées du Nord ont influé sur notre littérature, nous n’avons plus vu dans les romans que des adolescentes de dix-huit ans, et même au delà.\par
Si, maintenant, on retourne à des arguments moins gais, on ne les trouvera pas en moindre abondance. Outre ce qui a déjà été dit, plus haut, sur les juifs allemands, on pourra relever que, dans plusieurs parties de la Suisse, le développement physique de la population est tellement tardif, que, pour les hommes, il n’est pas toujours achevé à la vingtième année. Une autre série d’observations, très facile à aborder, serait offerte par les Bohémiens ou Zingaris \footnote{D’après M. Krapff, missionnaire protestant dans l’Afrique orientale, les Wanikas se marient à douze ans avec des filles du même âge. (\emph{Zeitschrift der deutschen morgenlændischen Gesellschaft}, t. III, p. 317.) Au Paraguay, les jésuites avaient établi la coutume, qui s’est conservée, de marier leurs néophytes, à 10 ans les filles, à 13 ans les garçons. On voit, dans ce pays, des veuves et des veufs de 11 et 12 ans. (A. d’Orbigny \emph{l’Homme américain}, t. I, p. 40.) – Dans le Brésil méridional, les femmes se marient vers 10 à 11 ans. La menstruation paraît de très bonne heure et passe de même. (Martius et Spix, \emph{Reise in Brasilien}, t. I , p. 382.) On pourrait multiplier ces citations à l’infini ; je n’en ajouterai qu’une : c’est que, dans le roman d’Yo-Kiao-li, l’héroïne chinoise a 16 ans, et que son père est désolé qu’à un tel âge, elle ne soit pas encore mariée.}. Les individus de cette race présentent exactement la même précocité physique que les Hindous, leurs parents ; et sous les cieux les plus âpres, en Russie, en Moldavie, on les voit conserver, avec leurs notions et leurs habitudes anciennes, l’aspect, la forme des visages et les proportions corporelles des parias. Je ne prétends cependant pas combattre M. Prichard sur tous les points. Il est une de ses observations que j’adopte avec empressement : c’est que « la différence du climat n’a que peu ou point d’effet pour produire des diversités importantes dans les époques des changements physiques auxquels la constitution humaine est assujettie \footnote{Prichard, ouvrage cité, t. II, p. 253.} ». Cette remarque est très fondée, et je ne chercherais pas à l’infirmer, me bornant à ajouter seulement qu’elle semble contredire un peu les principes défendus par le savant physiologiste et antiquaire américain.\par
On n’aura pas manqué de s’apercevoir que la question de permanence dans les types est, ici, la clef de la discussion. S’il est démontré que les races humaines sont, chacune, enfermées dans une sorte d’individualité d’où rien ne les peut faire sortir que le mélange, alors la doctrine des Unitaires se trouve bien pressée et ne peut se soustraire à reconnaître que, du moment où les types sont si complètement héréditaires, si constants, si \emph{permanents}, en un mot, malgré les climats et le temps, l’humanité n’est pas moins complètement et inébranlablement partagée, que si les distinctions spécifi­ques prenaient leur source dans une diversité primitive d’origine.\par
Cette assertion, si importante, nous est devenue facile à soutenir désormais. On l’a vue appuyée par le témoignage des sculptures égyptiennes, au sujet des Arabes, et par l’observation des Juifs et des Zingaris. Ce serait se priver, sans nul motif, d’un précieux secours que de ne pas rappeler, en même temps, que les peintures des temples et des hypogées de la vallée du Nil attestent également la permanence du type nègre à chevelure crépue, à tête prognathe, à grosses lèvres, et que la récente découverte des bas-reliefs de Khorsabad \footnote{Botta, \emph{Monuments de Ninive} ; Paris, 1850.}, venant confirmer ce que proclamaient déjà les monuments figurés de Persépolis, établit, à son tour, d’une manière incontestable, l’identité physiologique des populations assyriennes avec telles nations qui occupent aujourd’hui le même territoire.\par
Si l’on possédait, sur un plus grand nombre de races encore vivantes, des docu­ments semblables, les résultats demeureraient les mêmes. La permanence des types n’en serait que plus démontrée. Il suffit cependant d’avoir établi le fait pour tous les cas où l’étude en est possible. C’est maintenant aux adversaires à proposer leurs objections.\par
Les ressources leur manquent, et dans la défense qu’ils essayent, ils se démentent eux-mêmes, dès le premier mot, ou se mettent en contradiction avec les réalités les plus palpables. Ainsi, ils allèguent que les Juifs ont changé de type suivant les climats, et les faits démontrent le contraire. Leur raison, c’est qu’il y a en Allemagne beaucoup d’Israélites blonds avec des yeux bleus. Pour que cette allégation ait de la valeur, au point de vue où se placent les Unitaires, il faut que le climat soit reconnu comme étant la cause unique ou du moins principale de ce phénomène, et précisément les savants de cette école assurent, d’autre part, que la couleur de la peau, des yeux et des cheveux ne dépend, en aucune façon, de la situation géographique, ni des influences du froid ou du chaud \footnote{\emph{Edinburgh Review, Ethnology or the Science of Races}, 1848, p. 444 et passim.}. Ils trouvent et signalent, avec raison, des yeux bleus et des cheveux blonds chez les Cinghalais ; ils y observent même une grande variété de teint passant du brun clair au noir. D’autre part encore, ils avouent que les Samoyèdes et les Tongouses, bien que vivant sur les bords de la mer Glaciale, sont extrêmement basanés. Le climat n’est donc pour rien dans la carnation fixe, non plus que dans la couleur des cheveux et des yeux. Il faut dès lors laisser ces marques ou comme indifférentes en elles-mêmes, ou comme annexées à la race, et puisqu’on sait d’une manière très précise que les cheveux rouges ne sont pas rares en Orient et ne l’ont jamais été, personne, non plus, ne peut être surpris d’en voir aujourd’hui à des Juifs allemands. Il n’y a là de quoi rien établir, ni la permanence des types ni le contraire.\par
Les Unitaires ne sont pas plus heureux lorsqu’ils appellent à leur aide les preuves historiques. Ils n’en fournissent que deux : l’une s’applique aux Turcs, l’autre aux Madjars. Pour les premiers, l’origine asiatique est considérée comme hors de question. On croit pouvoir en dire autant de leur étroite parenté avec les rameaux finniques des Ostiaks et des Lapons. Dès lors ils ont eu primitivement la face jaune, les pommettes saillantes, la taille petite des Mongols. Ce point établi, on se tourne vers leurs descen­dants actuels, et, voyant ceux-ci pourvus du type européen, avec la barbe épaisse et longue, les yeux coupés en amande et non plus bridés, on conclut victorieusement que les races ne sont pas permanentes, puisque les Turcs se sont ainsi transformés \footnote{\emph{Ethnology}, p. 439.}. « À la vérité, disent « les Unitaires, quelques personnes ont prétendu qu’il y avait eu des mélanges « avec les familles grecque, géorgienne et circassienne. Mais, ajoutent-ils « aussitôt, ces mélanges n’ont pu être que très partiels : tous les Turcs n’étaient « pas assez riches pour acheter leurs femmes dans le Caucase ; tous n’avaient « pas des harems peuplés d’esclaves blanches, et, d’autre part, la haine des « Grecs pour leurs conquérants et les antipathies religieuses n’ont pas favorisé « les alliances, puisque les deux peuples, bien que vivant ensemble, sont encore « aujourd’hui aussi séparés qu’au premier jour de la conquête \footnote{\emph{Ibid.}, p. 439.} ».\par
Ces raisons sont plus spécieuses que solides. On ne saurait admettre que sous bénéfice d’inventaire l’origine finnique de la race turque. Cette origine n’a été démontrée, jusqu’ici, qu’au moyen d’un seul et unique argument : la parenté des langues, J’établirai plus bas combien cet argument, lorsqu’il se présente isolé, laisse de prise à la critique et de place au doute. En supposant, toutefois, que les premiers auteurs de la nation aient appartenu au type jaune, les moyens abondent d’établir qu’ils ont eu les meilleures raisons de s’en éloigner.\par
Entre le moment où les premières hordes touraniennes descendirent vers le sud-ouest et le jour où elles s’emparèrent de la cité de Constantin, entre ces deux dates que tant de siècles séparent, il s’est passé bien des événements ; les Turcs occidentaux ont eu bien des fortunes diverses. Tour à tour, vainqueurs et vaincus, esclaves ou maîtres, ils se sont installés au milieu de nationalités très diverses. Suivant les annalistes \footnote{Hammer, \emph{Geschichte des Osmanischen Reichs}, t. I, p. 2.}, leurs ancêtres Oghouzes, descendus de l’Altaï, habitaient, au temps d’Abraham, ces steppes immenses de la haute Asie qui s’étendent du Kataï au lac Aral, de la Sibérie au Thibet, précisément l’ancien et mystérieux domaine où vivaient encore à cette époque, de nombreuses nations germaniques \footnote{Ritter, Erdkunde, Asien, t. I, p. 433 et passim., p. 1115, etc. Tassen \emph{Zeitschrift für die Kunde des Morgenlandes}, t, II, p. 65 ; Benfey \emph{Encyclopædie} de Etsch et Gruber. \emph{Indien}, p. 12. M. le baron Alexandre de Humboldt, en parlant de ce fait, le signale comme une des découvertes les plus importantes de nos temps. (\emph{Asie centrale}, t. II, p, 639.) Au point de vue des sciences historiques, rien n’est plus vrai.}. Circonstance assez singulière : aussitôt que les écrivains de l’Orient commencent à parler des peuples du Turkestan, c’est pour vanter la beauté de leur taille et de leur visage \footnote{Nouschirwan, dont le règne tombe dans la première moitié du sixième siècle de notre ère, épousa Schahrouz, fille du Khakan des Turcs. C’était la plus belle personne de son temps. (Haneberg, \emph{Zeitsch f. d. K. des Morgenl}., t. I, p. 187.) Le Schahnameh fournit beaucoup de faits du même genre.}. Toutes les hyperboles leur sont, à ce sujet, familières, et comme ces écrivains avaient, sous les yeux, pour leur servir de point de comparaison, les plus beaux types de l’ancien monde, il n’est pas très probable qu’ils se soient enthousiasmés à l’aspect de créatures aussi incontestablement laides et repoussantes que le sont d’ordinaire les individus de sang mongol. Ainsi, malgré la linguistique, peut-être mal appliquée \footnote{De même que les Scythes, peuples mongols, avaient accepté une langue ariane, il n’y aurait rien de surprenant à ce que les Oghouzes fussent une nation ariane, tout en parlant un idiome finnois ; et cette hypothèse est singulièrement appuyée par une phrase naïve du voyageur Rubruquis, envoyé par saint Louis auprès du souverain des Mongols : « Je fus « frappé, dit ce bon moine, de la ressemblance du prince \emph{avec feu M. Jean de Beaumont}, « dont le teint coloré avait la même fraîcheur. » M. le baron Alexandre de Humboldt, intéressé, à bon droit, par cette remarque, ajoute avec non moins de sens : « Cette « observation physionomique mérite quelque attention, si l’on se rappelle que la famille de « Tchinguiz était vraisemblablement de race turque non mongole. » Et poursuivant cette donnée, le judicieux érudit corrobore le résultat par ces mots : « L’absence des traits « mongols frappe aussi dans les portraits que nous possédons des Baburides, dominateurs de l’Inde. » (\emph{Asie centrale}, t. I, p. 248 et note.)}, il y aurait là quelque chose à dire. Admettons pourtant que les Oghouzes de l’Altaï aient été, comme on le suppose, un peuple fin­nois, et descendons à l’époque musulmane où les tribus turques se trouvaient établies dans la Perse et l’Asie Mineure sous différentes dénominations et dans des situations non moins variées.\par
Les Osmanlis n’existaient pas encore, et les Seldjoukis, d’où ils devaient sortir, étaient fortement mélangés déjà avec les races de l’islamisme. Les princes de cette nation, tels que Ghaïaseddin-Keïkosrew, en 1237, épousaient librement des femmes arabes. Ils faisaient mieux encore, puisque la mère d’un autre dynaste seldjouki, Aseddin, était chrétienne ; et, du moment que les chefs, en tous pays, plus jaloux que le vulgaire de garder la pureté généalogique, se montraient si dégagés de préjugés, il est, au moins, permis de supposer que les sujets n’étaient pas plus scrupuleux. Comme leurs courses perpétuelles leur donnaient tous les moyens d’enlever des esclaves sur le vaste territoire qu’ils parcouraient, nul doute que dès le XIII\textsuperscript{e} siècle l’ancien rameau oghouze, auquel appartenaient de loin les Seldjoukis du Roum, ne fût extrêmement imprégné de sang sémitique.\par
Ce fut de ce rameau que sortit Osman, fils d’Ortoghroul et père des Osmanlis. Les familles ralliées autour de sa tente étaient peu nombreuses. Son armée ne valait guère mieux qu’une bande, et si les premiers successeurs de ce Romulus errant purent réussir à l’augmenter, ce ne fut qu’en usant du procédé pratiqué par le frère de Rémus, c’est-à-dire, en ouvrant leurs tentes à tous ceux qui en souhaitèrent l’entrée.\par
Je veux supposer que la ruine de l’empire seldjouki contribua à leur envoyer des recrues de leur race. Cette race était bien altérée, on le voit, et d’ailleurs la ressource fut insuffisante, puisqu’à dater de ce moment les Turcs firent la chasse aux esclaves dans le but avoué d’épaissir leurs rangs. Au commencement du XIV\textsuperscript{e} siècle, Ourkan, conseillé par Khalil Tjendereli le Noir, instituait la milice des janissaires. D’abord, il n’y en eut que mille. Mais, sous Mahomet IV, les nouvelles milices comptaient cent quarante mille soldats, et, comme jusqu’à cette époque, on fut soigneux de ne remplir les compa­gnies que d’enfants chrétiens enlevés en Pologne, en Allemagne et en Italie, ou recrutés dans la Turquie d’Europe, puis convertis à l’islamisme, ce furent au moins cinq cent mille chefs de famille qui, dans une période de quatre siècles, vinrent infuser un sang européen dans les veines de la nation turque.\par
Là ne se bornèrent pas les adjonctions ethniques. La piraterie, pratiquée sur une si grande échelle dans tout le bassin de la Méditerranée, avait surtout pour but de recruter les harems, et, ce qui est plus concluant encore, pas de bataille n’était livrée et gagnée qui n’augmentât de même le peuple croyant. Une bonne partie des captifs mâles abjurait, et dès lors comptait parmi les Turcs. Puis les environs du champ de combat parcourus par les troupes livraient toutes les femmes que les vainqueurs pouvaient saisir. Souvent ce butin se trouva tellement abondant, qu’il ne se plaçait qu’avec peine ; on échangeait la plus belle fille pour \emph{une} botte \footnote{Hammer, ouvrage cité, t. I, p. 448.–}. En rapprochant ces observations du chiffre bien connu de la population turque, tant d’Asie que d’Europe, et qui n’a jamais dépassé 12 millions, on restera convaincu que la question de la permanence du type n’a rien absolument à emprunter, en fait d’arguments pour ou contre, à l’histoire d’un peuple aussi mélangé que les Turcs. Et cette vérité est si claire, qu’en retrouvant, ce qui arrive quelquefois, dans des individus osmanlis, quelques traits assez reconnaissables de la race jaune, ce n’est pas à une origine finnique directe qu’il faut attribuer cette rencontre ; c’est simplement aux effets d’une alliance slave ou tatare, livrant, de seconde main ce qu’elle avait reçu elle-même d’étranger. Voilà ce qu’on peut observer sur l’ethnologie des Ottomans. Je passe maintenant aux Madjars.\par
La prétention des Unitaires est fondée sur le raisonnement que voici : « Les « Madjars sont d’origine finnoise, parents des Lapons, des Samoyèdes, des « Esqui­maux, tous gens de petite taille, à faces larges et à pommettes saillantes, « à teints jaunâtres ou bruns sales. Cependant les Madjars ont une stature « élevée et bien prise, des membres longs, souples et vigoureux, des traits « pareils à ceux des nations blanches et d’une évidente beauté. Les Finnois ont « toujours été faibles, inintelligents, opprimés. Les Madjars tiennent parmi les « conquérants du monde un rang illustre. Ils ont fait des esclaves et ne l’ont pas « été ; donc... puisque les Madjars sont Finnois, et, au physique comme au « moral, diffèrent de si loin de tous les autres rameaux de leur souche « primitive, c’est qu’ils ont énormément « changé \footnote{\emph{Ethnology}, p. 439\emph{. –}} ».\par
 Le changement serait tellement extraordinaire, s’il avait eu lieu, qu’il serait inexplicable, même pour les Unitaires, en supposant, d’ailleurs, les types doués de la mobilité la plus excessive ; car la métamorphose se serait opérée entre la fin du IX\textsuperscript{e} siècle et notre époque, c’est-à-dire dans un espace de 800 ans seulement, pendant lequel on sait que les compatriotes de saint Étienne se sont assez peu mêlés aux nations au milieu desquelles ils vivent. Heureusement pour le sens commun, il n’y a pas lieu à s’étonner, puisque le raisonnement que je vais combattre, parfait d’ailleurs, pèche dans l’essentiel ; les Hongrois ne sont certainement pas des Finnois.\par
Dans une notice fort bien écrite, M. A. de Gérando \footnote{\emph{Essai historique sur l’origine des Hongrois}, Paris, in-8°, 1844.} a désormais réduit à rien les théories de Schlotzer et de ses partisans, et prouvé, par les raisons les plus solides, tirées des historiens grecs et arabes, par l’opinion des annalistes hongrois, par des faits constatés et des dates qui bravent toute critique, par des raisons philologiques enfin, la parenté des Sicules avec les Huns et l’identité primitive de la tribu transylvaine avec les derniers envahisseurs de la Pannonie. Les Hongrois sont donc des Huns.\par
Ici se produira sans doute une objection nouvelle. On dira qu’il en résulte seulement pour les Madjars une parenté différente, mais non moins intime avec la race jaune. C’est une erreur. Si la dénomination de Huns est un nom de nation, c’est aussi, historiquement parlant, un mot collectif, et qui ne désigne pas une masse homogène. Dans la foule des tribus enrôlées sous la bannière des ancêtres d’Attila, on a distingué, entre autres, de tout temps, certaines bandes appelées les Huns blancs, où l’élément germanique dominait \footnote{Il semblerait qu’il y a beaucoup à modifier, désormais, dans les opinions reçues au sujet des peuples de l’Asie centrale. Maintenant que l’on ne peut plus nier que le sang des nations jaunes s’y trouve affecté par des mélanges plus ou moins considérables avec celui de peuples blancs, fait dont on ne se doutait pas autrefois, toutes les notions anciennes se trouvent atteintes et sujettes à révision. M. Alexandre de Humboldt fait une remarque très importante, à ce sujet, en parlant des Kirghiz-Kasakes, cités par Ménandre de Byzance et par Constantin Porphyrogénète, et il montre, très justement, que, lorsque le premier de ces écrivains parle d’une concubine kirghize (mot grec), présent du chagan turc Dithouboul à l’ambassadeur Zémarch, envoyé par l’empereur Justin II, en 569, il s’agit d’une fille métisse. C’est le pendant exact des belles filles turques si vantées par les Persans et qui n’avaient pas, plus que celle-là, le type mongol. (Voir \emph{Asie centrale}, t. I, p. 237 et passim., et t, II, p. 130-131)}.\par
À la vérité, le contact avec les groupes jaunes avait altéré la pureté du sang : mais c’est aussi ce que le faciès un peu anguleux et osseux du Madjar confesse avec une remarquable sincérité. La langue est très voisine, dans ses affinités, des dialectes turcs : les Madjars sont donc des Huns blancs, et cette nation, dont on a fait improprement un peuple jaune, parce qu’elle était confondue, par des alliances volontaires ou forcées, avec cette race, se trouve ainsi composée de métis à base germanique. La langue a des racines et une terminologie tout étrangères à leur espèce dominante, absolument comme il en était pour les Scythes jaunes, qui parlaient un dialecte arian \footnote{Schaffarik, \emph{Slawische Alterthümer}, t. I, p. 279 et passim.}, et pour les Scandinaves de la Neustrie, gagnés, après quelques années de conquête, au dialecte celto-latin de leurs sujets \footnote{Aug. Thierry, \emph{Histoire de la Conquête de l’Angleterre} ; Paris, in-12, 1846 ; t. I, p. 155.}. Rien, dans tout cela, n’autorise à supposer que le temps, l’effet des climats divers et du changement d’habitudes aient, d’un Lapon ou d’un Ostiak, d’un Tongouse ou d’un Permien, fait un saint Étienne. En vertu de cette réfutation des seuls arguments présentés par les Unitaires, je conclus que la perma­nence des types chez les races est au-dessus de toute contestation, et si forte, si inébranlable, que le changement de milieu le plus complet ne peut rien pour la détruire, tant qu’il n’y a pas mélange d’une branche humaine avec quelque autre.\par
Ainsi, quelque parti qu’on veuille prendre sur l’unité ou la multiplicité des origines de l’espèce, les différentes familles sont aujourd’hui parfaitement séparées les unes des autres, puisque aucune influence extérieure ne saurait les amener à se ressembler, à s’assimiler, à se confondre.\par
Les races actuelles sont donc des branches bien distinctes d’une ou de plusieurs souches primitives perdues, que les temps historiques n’ont jamais connues, dont nous ne sommes nullement en état de nous figurer les caractères même les plus généraux ; et ces races, différant entre elles par les formes extérieures et les proportions des membres, par la structure de la tête osseuse, par la conformation interne du corps, par la nature du système pileux, par la carnation, etc., ne réussissent à perdre leurs traits principaux qu’à la suite et par la puissance des croisements.\par
Cette permanence des caractères génériques suffit pleinement à produire les effets de dissemblance radicale et d’inégalité, à leur donner la portée de lois naturelles, et à appliquer à la vie physiologique des peuples les mêmes distinctions que j’appliquerai plus tard à leur vie morale.\par
Puisque je me suis résigné, par respect pour un agent scientifique que je ne puis détruire, et, plus encore, par une interprétation religieuse que je n’oserais attaquer, à laisser de côté les doutes véhéments qui m’assiègent au sujet de la question d’unité pri­mordiale, je vais maintenant tâcher d’exposer, autant que faire se peut, par les moyens qui me restent, les causes probables de divergences physiologiques si indélébiles.\par
Personne ne sera tenté de le nier, il plane au-dessus d’une question de cette gravité une mystérieuse obscurité, grosse de causes à la fois physiques et immatérielles. Certaines raisons relevant du domaine divin, et dont l’esprit effrayé sent le voisinage sans en deviner la nature, dominent au fond des plus épaisses ténèbres du problème, et il est bien vraisemblable que les agents terrestres, auxquels on demande la clef du secret, ne sont eux-mêmes que des instruments, des ressorts inférieurs de la grande œuvre. Les origines de toutes choses, de tous les mouvements, de tous les faits, sont, non pas des infiniment petits, comme on s’amuse souvent à le dire, mais tellement immenses, au contraire, tellement vastes et démesurées vis-à-vis de notre faiblesse, que nous pouvons les soupçonner et indiquer que peut-être elles existent, sans jamais pouvoir espérer les toucher du doigt ni les révéler d’une manière sûre. De même que, dans une chaîne de fer destinée à supporter un grand poids, il arrive fréquemment que l’anneau le plus rapproché de l’objet est le plus petit, de même la cause dernière peut sembler souvent presque insignifiante, et si on s’arrête à la considérer isolément, on oublie la longue série qui la précède et la soutient, et qui, forte et puissante, prend son attache hors de la vue. Il ne faut donc pas, avec l’anecdote antique, s’émerveiller de la puissance de la feuille de rose qui fit déborder l’eau : il est plus juste de considérer que l’accident gisait au fond du liquide surabondamment renfermé dans les flancs du vase. Rendons tout respect aux causes premières, génératrices, célestes et lointaines, sans lesquelles rien n’existerait, et qui, confidentes du motif divin, ont droit à une part de la vénération rendue à leur auteur omnipotent ; cependant, abstenons-nous d’en parler ici. Il n’est pas à propos de sortir de la sphère humaine où seulement on peut espérer de rencontrer des certitudes, et il convient de se borner à saisir la chaîne, sinon par son dernier et moindre anneau, du moins par sa partie visible et tangible, sans avoir la prétention, trop difficile à soutenir, de remonter au delà de la portée du bras. Ce n’est pas de l’irrévérence ; c’est, au contraire, le sentiment sincère d’une faiblesse insur­montable.\par
L’homme est un nouveau venu dans le monde. La géologie, ne procédant que par inductions, il est vrai, toutefois avec une persistance bien remarquable, constate son absence dans toutes les formations antérieures du globe ; et, parmi les fossiles, elle ne le rencontre pas. Lorsque, pour la première fois, nos parents apparurent sur la terre déjà vieille, Dieu, suivant les livres saints, leur apprit qu’ils en seraient les maîtres, et que tout plierait sous leur autorité. Cette promesse de domination s’adressait moins aux individus qu’à leur descendance ; car ces faibles créatures semblaient pourvues de bien peu de ressources, je ne dirai pas pour dompter toute la nature, mais seulement pour résister à ses moindres forces \footnote{Lyell’s, \emph{Principles of Geology}, t. I, p. 178.}. Les cieux éthérés avaient vu, dans les périodes précédentes, sortir, du limon terrestre et des eaux profondes, des êtres bien autrement imposants que l’homme. Sans doute, la plupart des races gigantesques avaient disparu dans les révolutions terribles où le monde inorganique témoigna d’une puissance si fort éloignée de toute proportion avec celle de la nature animée. Pourtant un grand nombre de ces bêtes monstrueuses vivaient encore. Les éléphants et les rhinocéros hantaient par troupeaux tous les climats, et le mastodonte même laisse encore les traces de son existence dans les traditions américaines \footnote{Link, \emph{die Urwelt und das Alterthum}, t. I, p. 84.}.\par
Ces monstres attardés devaient suffire et au delà pour imprimer aux premiers individus de notre espèce, avec un sentiment craintif de leur infériorité, des pensées bien modestes sur leur royauté problématique. Et ce n’étaient pas les animaux seuls auxquels il £allait disputer et enlever l’empire. On pouvait, à la rigueur, les combattre, employer contre eux la ruse, à défaut de la force, et, sinon les vaincre, du moins les éviter et les fuir. Il n’en était pas de même de cette immense nature qui, de toutes parts, embrassait, enfermait les familles primitives et leur faisait sentir lourdement son effrayante domination \footnote{Link, ouvrage cité, t. I, p. 91.}. Les causes cosmiques auxquelles on doit attribuer les antiques bouleversements agissaient toujours, bien qu’affaiblies. Des cataclysmes partiels déran­geaient encore les positions relatives des terres et des océans. Tantôt le niveau des mers s’élevait et engloutissait de vastes plages ; tantôt une terrible éruption volcanique soulevait du sein des flots quelque contrée montagneuse qui venait s’annexer à un continent. Le monde était encore en travail, et Jéhovah ne l’avait pas calmé en lui disant : Tout est bien !\par
Dans cette situation, les conditions atmosphériques se ressentaient nécessairement du manque général d’équilibre. Les luttes entre la terre, l’eau, le feu, amenaient des variations rapides et tranchées d’humidité, de sécheresse, de froid, de chaud, et les exhalaisons d’un sol encore tout frémissant exerçaient sur les êtres une action irrésis­tible. Toutes ces causes enveloppant le globe d’un souffle de combats, de souffrances, de peines, redoublaient nécessairement la pression que la nature exerçait sur l’homme, et l’influence des milieux et les différences climatériques ont alors possédé, pour réagir sur nos premiers parents, une tout autre efficacité qu’aujourd’hui. Cuvier affirme dans son \emph{Discours sur les Révolutions du Globe}, que l’état actuel des forces inorganiques ne pourrait, en aucune façon, déterminer des convulsions terrestres, des soulèvements, des formations semblables à celles dont la géologie constate les effets. Ce que cette nature, si terriblement douée, exerçait alors sur elle-même de modifications devenues aujour­d’hui impossibles, elle le pouvait aussi sur l’espèce humaine, et ne le peut plus désormais. Son omnipotence s’est tellement perdue, ou du moins tellement amoindrie et rapetissée, que dans une série d’années équivalant à peu près à la moitié du temps que notre espèce a passé sur la terre, elle n’a produit aucun changement de quelque importance, encore bien moins rien de comparable à ces traits arrêtés qui ont séparé à jamais les différentes races \footnote{Cuvier,\emph{ Discours sur les Révolutions du Globe. – Voici}, également, sur ces matières, l’opinion exprimée par M. le baron Alexandre de Humboldt : « Dans les temps qui ont « précédé l’existence de la race humaine, l’action de l’intérieur du globe sur la croûte « solide, augmentant d’épaisseur, a dû modifier la température de l’atmosphère et rendre le « globe entier habitable aux productions que l’on regarde comme exclusivement « tropicales ; depuis que, par l’effet du rayonnement et du refroi­dissement, les rapports de « position de notre planète avec un corps central (le soleil) ont commencé à déterminer « presque exclusivement les climats à diverses latitudes. C’est dans ces temps primitifs « aussi que les fluides élastiques, ou forces volcaniques de l’intérieur, plus puissantes « qu’aujourd’hui, se sont fait jour à travers la croûte oxydée et peu solidifiée de la « planète. » (\emph{Asie centrale}, t. I, p. 47.)}.\par
Deux points ne sont pas douteux : c’est que les principales différences qui séparent les branches de notre espèce ont été fixées dans la première moitié de notre existence terrestre, et, ensuite que, pour concevoir un moment où, dans cette première moitié, ces séparations physiologiques aient pu s’effectuer, il faut remonter aux temps où l’influence des agents extérieurs a été plus active que nous ne la voyons être dans l’état ordinaire du monde, dans sa santé normale. Cette époque ne saurait être autre que celle qui a immédiatement entouré la création, alors qu’émue encore par les dernières catas­trophes, elle était soumise sans réserve aux influences horribles de leurs derniers tressaillements.\par
En s’en tenant à la doctrine des Unitaires, il est impossible d’assigner à la séparation des types une date postérieure.\par
Il n’y a pas à tirer parti de ces déviations fortuites qui se produisent quelquefois dans certains individus, et qui, si elles se perpétuaient, créeraient, incontestablement, des variétés très dignes d’attention. Sans parler de plusieurs affections, comme la gibbosité, on a relevé des faits curieux qui semblent, au premier abord, propres à expliquer la diversité des races. Pour n’en citer qu’un seul, M. Prichard parle, d’après M. Baker \footnote{Prichard, ouvrage cité, t. I, p. 124.}, d’un homme couvert sur tout le corps, à l’exception de la face, d’une sorte de carapace de couleur obscure, analogue à une immense verrue fort dure, insensible et calleuse, et qui, lorsqu’on l’entamait, ne donnait point de sang. À différentes époques, ce tégument singulier, ayant atteint une épaisseur de trois quarts de pouce, se détachait, tombait, et était remplacé par un autre tout pareil. Quatre fils naquirent de cet homme. Ils étaient semblables à leur père. Un seul survécut : mais M. Baker, qui le vit dans son enfance, ne dit pas s’il est parvenu à l’âge adulte. Il conclut seulement que, puisque le père avait produit de tels rejetons, une famille particulière aurait pu se former, qui aurait conservé un type spécial, et que, le temps et l’oubli aidant, on se serait cru autorisé, plus tard, à considérer cette variété d’hommes comme présentant des caractères spécifiques particuliers.\par
La conclusion est admissible. Seulement, les individus, si différents de l’espèce en général, ne se perpétuent pas. Leur postérité rentre dans la règle commune ou s’éteint bientôt. Tout ce qui dévie de l’ordre naturel et normal ne peut qu’emprunter la vie et n’est pas apte à la conserver. Sans quoi, les accidents les plus étranges auraient écarté, depuis longtemps, l’humanité des conditions physiologiques observées de tous temps chez elle. Il faut en inférer qu’une des conditions essentielles, constitutives, de ces anomalies est précisément d’être transitoires, et on ne saurait dès lors faire rentrer dans de telles catégories la chevelure du nègre, sa peau noire, la couleur jaune du Chinois, sa face large, ses yeux bridés. Ce sont autant de caractères permanents qui n’ont rien d’anormal et qui, en conséquence, ne proviennent pas d’une déviation accidentelle.\par
Résumons ici tout ce qui précède.\par
Devant les difficultés que présentent l’interprétation la plus répandue du texte biblique et l’objection tirée de la loi qui régit la génération des hybrides, il est impos­sible de se prononcer catégoriquement et d’affirmer, pour l’espèce, la multiplicité d’origines.\par
Il faut donc se contenter d’assigner des causes inférieures à ces variétés si tranchées dont la permanence est incontestablement le caractère principal, permanence qui ne peut se perdre que par l’effet des croisements. Ces causes, on peut les apercevoir dans l’énergie climatérique que possédait notre globe aux premiers temps où parut la race humaine. Il n’y a pas de doute que les conditions de force de la nature inorganique étaient, alors, tout autrement puissantes qu’on ne les a connues depuis, et il a pu s’accomplir, sous leur pression, des modifications ethniques devenues impossibles. Probablement aussi, les êtres exposés à cette action redoutable s’y prêtaient beaucoup mieux que ne le pourraient les types actuels. L’homme, étant nouvellement créé, pré­sentait des formes encore incertaines, peut-être même n’appartenait d’une manière bien tranchée ni à la variété blanche, ni à la noire, ni à la jaune. Dans ce cas, les déviations qui portèrent les caractères primitifs de l’espèce vers les variétés aujourd’hui établies, eurent beaucoup moins de chemin à faire que n’en aurait maintenant la race noire, par exemple, pour être ramenée au type blanc, ou la jaune pour être confondue avec la noire. Dans cette supposition, on devrait se représenter l’individu adamite comme également étranger à tous les groupes humains actuels ; ceux-ci auraient rayonné autour de lui et se seraient éloignés, les uns des autres, du double de la distance existant entre lui et chacun d’eux. Qu’auraient dès lors conservé les individus de toutes races du spécimen primitif ? Uniquement les caractères les plus généraux qui constituent notre espèce : la vague ressemblance de formes que les groupes les plus distants ont en commun ; la possibilité d’exprimer leurs besoins au moyen de sons articulés par la voix ; mais rien davantage. Quant au surplus des traits les plus spéciaux de ce premier type, nous les aurions tous perdus, aussi bien les peuples noirs que les peuples non noirs ; et, quoique descendus primitivement de lui, nous aurions reçu d’influences étrangères tout ce qui constitue désormais notre nature propre et distincte. Dès lors, produits tout à la fois de la race adamique primitive et des milieux cosmogoniques, les races humaines n’auraient entre elles que des rapports très faibles et presque nuls. Le témoignage persistant de cette fraternité primordiale serait la possibilité de donner naissance à des hybrides féconds, et il serait unique. Il n’y aurait rien de plus, et en même temps que les différences des milieux primordiaux auraient distribué à chaque groupe son caractère isolé, ses formes, ses traits, sa couleur d’une manière permanente, elles auraient brisé décidément l’unité primitive, demeurée à l’état de fait stérile quant à son influence sur le développement ethnique. La permanence rigoureuse, indélébile des traits et des formes, cette permanence que les plus lointains documents historiques affirment et garantissent, serait le cachet, la confirmation de cette éternelle séparation des races.
\section[{I.12. Comment les races se sont séparées physiologiquement, et quelles variétés elles ont ensuite formées par leurs mélanges. Elles sont inégales en force et en beauté.}]{I.12. \\
Comment les races se sont séparées physiologiquement, et quelles variétés elles ont ensuite formées par leurs mélanges. Elles sont inégales en force et en beauté.}
\noindent Il est bon d’éclairer complètement la question des influences cosmogoniques, puisque les arguments qui en sortent sont ceux dont je me contente ici. Le premier doute à écarter est le suivant : Comment les hommes, réunis sur un seul point par suite d’une origine commune, ont-ils pu être exposés à des actions physiques totalement diverses ? Et si leurs groupes, quand les différences de races ont commencé, étaient déjà assez nombreux pour se répandre dans des climats distincts, comment se fait-il qu’ayant à lutter contre des difficultés immenses, telles que traversées de forêts pro­fondes et de contrées marécageuses, de déserts de sable ou de neige, passages de fleuves, rencontres de lacs et d’océans, ils soient parvenus à réaliser des voyages que l’homme civilisé, avec toute sa puissance, n’accomplit encore qu’avec grand-peine ? Pour répondre à ces objections, il faut examiner quelle a pu être la première station de l’espèce.\par
C’est une notion fort ancienne, et adoptée par de grands esprits des temps modernes, tels que Georges Cuvier, que les différents systèmes de montagnes ont dû servir de points de départ à certaines catégories de races. Ainsi les blancs, et même quelques variétés africaines, qui, par la forme de la tête osseuse, se rapprochent des proportions de nos familles, auraient eu leur première résidence dans le Caucase. La race jaune serait descendue des hauteurs glacées de l’Altaï. À leur tour, les tribus de nègres prognathes auraient, sur les versants méridionaux de l’Atlas, construit leurs premières cabanes, tenté leurs premières migrations ; et, de cette façon, ce que les temps originels auraient le mieux connu, ce seraient précisément ces lieux redoutables, de difficile accès, pleins de sombres horreurs, torrents, cavernes, glaces, neiges éter­nelles, infranchissables abîmes ; tandis que toutes les terreurs de l’inconnu se seraient trouvées, pour nos plus antiques parents, dans les plaines découvertes, sur les grandes rives des fleuves, des lacs et des mers.\par
Le motif premier qui semble avoir conduit les philosophes anciens à émettre cette théorie, et les modernes à la renouveler, c’est l’idée que, pour traverser les grandes crises physiques de notre globe, l’espèce humaine a dû se rallier sur des sommets où les flots des déluges ne pouvaient l’atteindre. Mais cette application agrandie et généralisée de la tradition de l’Ararat, bien que convenant peut-être à des époques postérieures aux temps primitifs, à des temps où les populations avaient déjà couvert la face du monde, devient tout à fait inadmissible pour les temps où précisément l’espèce a dû naître dans le calme au moins relatif de la nature, et, soit dit en passant, elle est tout à fait contraire aux notions d’unité de l’espèce. De plus, les montagnes ont toujours été, dès les temps les plus reculés, l’objet d’une profonde crainte, d’un respect superstitieux. C’est là que toutes les mythologies ont placé le séjour des dieux. C’est sur la cime nuageuse de l’Olympe, c’est sur le mont Mérou que les Grecs et les Brahmes ont rêvé leurs assemblées divines ; c’est sur le haut du Caucase que Prométhée souffrait le châtiment mystérieux d’un crime plus mystérieux encore ; et, si les hommes avaient commencé par habiter ces hautes retraites, il est peu probable que leur imagination les eût ainsi relevées si fort que de les porter jusque dans le ciel. On vénère médiocrement ce que l’on a vu, connu, foulé aux pieds : il n’y aurait eu de divinités que dans les eaux et les plaines. Je suis donc induit à admettre l’idée contraire, et à supposer que les terrains découverts et plats ont été les témoins des premiers pas de l’homme. Du reste, c’est la notion biblique \footnote{\emph{Gen}. II, 8 et passim.}, et du moment où le premier séjour se trouve ainsi établi, les difficul­tés des migrations sont sensiblement diminuées ; car les terrains plats, généralement coupés par des fleuves, aboutissent à des mers, et il n’est plus besoin de se préoccuper de la traversée bien autrement difficile des forêts, des déserts et des grands marécages.\par
Il y a deux genres de migrations : les unes volontaires ; de celles-là il ne saurait être question dans les âges tout à fait génésiaques. Les autres sont imprévues et plus possi­bles et plus probables encore chez des sauvages imprudents, maladroits, que chez des nations perfectionnées. Il suffit d’une famille embarquée sur un radeau qui dérive, de quelques malheureux surpris par une irruption de la mer, cramponnés à des troncs d’arbres et saisis par les courants, pour donner la raison d’une transplantation lointaine. Plus l’homme est faible, plus il est le jouet des forces inorganiques. Moins il a d’expérience, plus il obéit en esclave à des accidents qu’il n’a pas su prévoir et qu’il ne peut éviter. On connaît des exemples frappants de la facilité avec laquelle des êtres de notre espèce peuvent être transportés, malgré eux, à des distances considérables. Ainsi l’on raconte qu’en 1696, deux pirogues d’Ancorso, montées d’une trentaine de sauvages, hommes et femmes, furent saisies par le mauvais temps, et, après avoir vogué quelque temps à la dérive, arrivèrent enfin à l’une des îles Philippines, Samal, distante de trois cents lieues du point d’où les pirogues étaient parties. Autre exemple : Quatre naturels d’Ulea, se trouvant dans un canot, furent emportés par un coup de vent, errèrent pendant huit mois en mer, et finirent par arriver à l’une des îles de Radack, à l’extrémité orientale de l’archipel des Carolines, ayant ainsi fait involontairement une traversée de 550 lieues. Ces malheureux vivaient uniquement de poisson ; ils recueillaient les gouttes de pluie avec le plus grand soin. Cette ressource venait-elle à leur manquer, ils plongeaient au fond de la mer et buvaient de cette eau, qui, dit-on, est moins salée. Il va sans dire qu’en arrivant à Radack, les navigateurs étaient dans l’état le plus déplorable ; cependant ils se remirent assez promptement, et recouvrèrent la santé \footnote{Lyell’s, \emph{Principles of Geology}, t. II, p. 119.}.\par
Ces deux citations suffisent pour rendre admissible l’idée d’une rapide diffusion de certains groupes humains dans des climats très différents, et sous l’empire des circonstances locales les plus opposées. Si, cependant, il fallait encore d’autres preuves, on pourrait parler de la facilité avec laquelle les insectes, les testacés, les plantes, se répandent partout, et certainement il n’est pas nécessaire de démontrer que ce qui arrive pour les catégories d’êtres que je viens de nommer est, à plus forte raison, moins difficile pour l’homme \footnote{M. Alexandre de Humboldt ne pense pas que cette hypothèse puisse s’appliquer à la migration des plantes. « Ce que nous savons, dit cet érudit, de l’action délétère qu’exerce « l’eau de mer dans un trajet de 500 à 600 lieues sur l’excitabilité germinative de la plupart « des grains, n’est d’ailleurs pas en faveur du système trop généralisé sur la migration des « végétaux au moyen des « courants pélagiques. » (\emph{Examen critique de l’Histoire de la géographie du nouveau continent}, t. II, p. 78.)}. Les testacés terrestres sont entraînés dans la mer par la destruction des falaises, puis emportés jusqu’à des plages lointaines au moyen des courants. Les zoophytes, attachés à la coquille des mollusques, ou laissant flotter leurs bourgeons sur la surface de l’Océan, vont, où les vents les emportent, établir de lointaines colonies ; et ces mêmes arbres d’espèces inconnues, ces mêmes poutres sculptées qui, dans le XV\textsuperscript{e} siècle, vinrent s’échouer, après tant d’autres inobservées, sur les côtes des Canaries, et servant de texte aux méditations de Christophe Colomb, contribuèrent à la découverte du nouveau monde, portaient probablement aussi, sur leurs surfaces, des œufs d’insectes, que la chaleur d’une sève nouvelle devait faire éclore bien loin du lieu de leur origine et du terrain où vivaient leurs congénères.\par
Ainsi nulle difficulté à ce que les premières familles humaines aient pu habiter promptement des climats très divers, des lieux très éloignés les uns des autres. Mais, pour que la température et les circonstances locales qui en résultent soient diverses, il n’est pas nécessaire, même dans l’état actuel du globe, que les lieux se trouvent à de longues distances. Sans parler des pays de montagnes, comme la Suisse, où, dans l’espace d’une à deux lieues de terrain, les conditions de l’atmosphère et du sol varient tellement que l’on y trouve confondues, en quelque sorte, la flore de la Laponie et celle de l’Italie méridionale ; sans rappeler que l’Isola Madre, sur le lac Majeur, nourrit des orangers en pleine terre, de grands cactus et des palmiers nains à la vue du Simplon, personne n’ignore combien la température de la Normandie est plus rude que celle de l’île de jersey. Dans un triangle étroit, et sans qu’il soit besoin de faire appel aux déductions de l’orographie, nos côtes de l’ouest présentent le spectacle le plus varié en fait d’existences végétales \footnote{M. Alexandre de Humboldt expose la loi déterminante de cette vérité lorsqu’il dit (\emph{Asie centrale}, t. III, p. 23) : « La première base de la climatologie est la connaissance précise des « inégalités de la surface d’un continent. Sans cette connaissance hypsométrique, on « attribuerait à l’élévation du sol ce qui est l’effet d’autres causes, qui influent, dans les « basses régions, dans une surface qui a une même courbure avec la surface de l’océan, sur « l’inflexion des lignes isothermes (ou d’égale chaleur d’été). » En appelant l’attention sur cette grande multiplicité d’influences qui agissent sur la température d’un point géographique indiqué, le grand érudit berlinois conduit l’esprit à concevoir sans peine que, dans des lieux très voisins, et indépendamment de l’élévation du sol, il se forme des phénomènes climatériques très divers. Ainsi, il est un point de l’Irlande, dans le nord-est de l’île, sur la côte de Glenarn, qui, contrastant avec ce qui est possible aux environs, nourrit des myrtes en pleine terre, et aussi vigoureux que ceux du Portugal, sous le parallèle de Kœnigsberg en Prusse. « Il y gèle à peine en hiver, et cependant les chaleurs de l’été ne « suffisent pas pour mûrir le raisin. Les mares et les petits lacs des îles Fœroë ne se « couvrent pas de glace pendant l’hiver, malgré leur latitude de 62°... En Angleterre, sur les « côtes du Devonshire, les myrtes, le camelia japonica, le fuchsia coccinea et le boddleya « globosa passent l’hiver sans abri en pleine terre... À Salcombe, les hivers sont tellement « doux, qu’on y a vu des orangers en espaliers portant du fruit et à peine abrités par le « moyen des estères (p. 147-148). »}.\par
Quelle ne devait pas être la valeur des contrastes, sur l’espace le plus resserré, dans les époques redoutables au lendemain desquelles se reporte la naissance de notre espèce! Un seul et même lieu devenait aisément le théâtre des plus grandes révolutions atmosphériques, lorsque la mer s’en éloignait ou s’en approchait par l’inondation ou la mise à sec des régions voisines ; lorsque des montagnes s’élevaient, tout à coup, en masses énormes, ou s’abaissaient au niveau commun du globe, de manière à laisser des plaines remplacer leurs crêtes ; lorsque, enfin, des tressaillements dans l’axe de la terre et, par suite, dans l’équilibre général et dans l’inclinaison des pôles sur l’écliptique, venaient troubler l’économie générale de la planète.\par
On doit ainsi considérer comme écartée toute objection tirée de la difficulté du changement de lieux et de température aux premiers âges du monde, et rien ne s’oppose à ce que la famille humaine ait pu, soit étendre fort loin quelques-uns de ses groupes, soit, en les conservant réunis tous dans un espace assez resserré, les voir subir des influences très multiples. C’est de cette manière que purent se former les types secondaires dont sont descendues les branches actuelles de l’espèce. Quant à l’homme de la création première, quant à l’Adamite, puisqu’il est impossible de rien savoir de ses caractères spécifiques, ni combien chacune des familles nouvelles a conservé ou perdu de sa ressemblance, laissons-le, tout à fait, en dehors de la controverse. De cette façon, nous ne remontons pas plus haut dans notre examen que les races de seconde formation.\par
Je rencontre ces races bien caractérisées au nombre de trois seulement : la blanche, la noire et la jaune \footnote{J’expliquerai en leur lieu les motifs qui me portent à ne pas compter les sauvages peaux-rouges de l’Amérique au nombre des types purs et primitifs. J’ai déjà laissé entrevoir mon opinion, à ce sujet, au chapitre X de ce volume. D’ailleurs, je ne fais ici que me rallier à l’avis de M. Flourens, qui ne reconnaît aussi que trois grandes subdivisions dans l’espèce : celles d’Europe, d’Asie et d’Afrique. Ces dénominations me semblent prêter le flanc à la critique, mais le fond est juste.}. Si je me sers de dénominations empruntées à la couleur de la peau, ce n’est pas que je trouve l’expression juste ni heureuse, car les trois catégories dont je parle n’ont pas précisément pour trait distinctif la carnation, toujours très multiple dans ses nuances, et on a vu plus haut qu’il s’y joignait des faits de conformation plus importants encore. Mais, à moins d’inventer moi-même des noms nouveaux, ce que je ne me crois pas en droit de faire, il faut bien me résoudre à choisir, dans la terminologie en usage, des désignations non pas absolument bonnes, mais moins défectueuses que les autres, et je préfère décidément celles que j’emploie ici et qui, après avertissement préalable, sont assez inoffensives, à tous ces appellatifs tirés de la géographie ou de l’histoire, qui ont jeté tant de désordre sur un terrain déjà assez embarrassé par lui-même. Ainsi, j’avertis, une fois pour toutes, que j’entends par \emph{blancs} ces hommes que l’on désigne aussi sous le nom de race caucasique, sémitique, japhétide. J’appelle \emph{noirs}, les Chamites, et \emph{jaunes}, le rameau altaïque, mongol, finnois, tatare. Tels sont les trois éléments purs et primitifs de l’humanité. Il n’y a pas plus de raisons d’admettre les vingt-huit variétés de Blumenbach que les sept de M. Prichard, l’un et l’autre classant dans leurs séries des hybrides notoires. Chacun des trois types originaux, en son particulier, ne présenta probablement jamais une unité parfaite. Les grandes causes cosmogoniques n’avaient pas seulement créé dans l’espèce des variétés tranchées : elles avaient aussi, sur les points où leur action s’était exercée, déterminé, dans le sens de chacune des trois variétés principales, l’apparition de plusieurs genres qui possédèrent, outre les caractères généraux de leur branche, des traits distinctifs particuliers. Il n’y eut pas besoin de croisements ethniques pour amener ces modifications spéciales ; elles préexistèrent à tous les alliages. C’est vainement qu’on chercherait aujourd’hui à les constater dans l’agglomération métisse qui constitue ce qu’on nomme la race blanche. Cette impossibilité doit exister aussi pour la jaune. Peut-être le type mélanien s’est-il conservé pur quelque part ; du moins, il est certainement resté plus original, et il démontre ainsi, sur le vu même, ce que nous pouvons, pour les deux autres catégories humaines, admettre, non pas d’après le témoignage de nos sens, mais d’après les inductions fournies par l’histoire.\par
Les nègres ont continué d’offrir différentes variétés originelles, telles que le type prognathe à chevelure laineuse, celui du nègre hindou du Kamaoun et du Dekkhan, celui du Pélagien de la Polynésie. Très certainement des variétés se sont formées entre ces genres au moyen de mélanges, et c’est de là que dérivent, tant pour les noirs que pour les blancs et les jaunes, ce qu’on peut appeler les types tertiaires.\par
On a relevé un fait bien digne de remarque, dont on prétend se servir aujourd’hui comme d’un critérium sûr pour reconnaître le degré de pureté ethnique d’une popu­lation. C’est la ressemblance des visages, des formes, de la constitution et, partant, des gestes et du maintien. Plus une nation serait exempte d’alliage et plus tous ses membres auraient en commun ces similitudes que j’énumère. Plus au contraire elle se serait croisée, et plus on trouverait de différences dans la physionomie, la taille, le port, l’apparence enfin des individualités. Le fait est incontestable, et le parti à en tirer est précieux ; mais ce n’est pas tout à fait celui que l’on pense.\par
La première observation qui a fait découvrir ce fait, a eu lieu sur des Polynésiens ; or, les Polynésiens ne sont pas une race pure, tant s’en faut, puisqu’ils sont issus de mélanges différemment gradués entre les noirs et les jaunes. La transmission intégrale du type dans les différents individus n’indique donc pas la pureté de la race, mais seulement ceci : que les éléments, plus ou moins nombreux, dont cette race est compo­sée, sont arrivés à se fondre parfaitement ensemble, de manière à ce que la combinaison en est, à la fin, devenue homogène, et que chaque individu de l’espèce n’ayant pas, dans les veines, d’autre sang que son voisin, il n’y a pas moyen qu’il en diffère physique­ment. De même que les frères et sœurs se ressemblent souvent, comme provenant d’éléments semblables, ainsi, lorsque deux races productrices sont parvenues à s’amalgamer si complètement qu’il n’y a plus dans la nation de groupes ayant plus de l’essence de l’une que de l’autre, il s’établit, par équilibre, une sorte de pureté fictive, un type artificiel, et tous les nouveau-nés en apportent l’empreinte.\par
De cette façon, le type tertiaire, dont j’ai défini le mode de formation, put avoir de bonne heure le cachet faussement attribué à la pureté absolue et vraie de race, c’est-à-dire la ressemblance de ses individualités, et cela fut possible dans un délai d’autant plus court que deux variétés d’un même type furent relativement peu différentes entre elles. C’est pour ce motif que, dans une famille, si le père appartient à une nation autre que celle de la mère, les enfants ressembleront soit à l’un, soit à l’autre de leurs auteurs, et auront peine à établir une identité de caractères physiques entre eux ; tandis que, si les parents sont issus tous deux d’une même souche nationale, cette identité se produira sans aucune peine.\par
Il est encore une loi à signaler avant d’aller plus loin : les croisements n’amènent pas seulement la fusion de deux variétés. Ils déterminent la création de caractères nouveaux, qui deviennent dès lors le côté le plus important par lequel on puisse envisager un sous-genre. On va en voir bientôt des exemples. Je n’ai pas besoin d’ajouter, ce qui s’entend assez de soi, que le développement de cette originalité nouvelle ne peut être complet sans cette condition que la fusion des types générateurs sera préalablement parfaite, sans quoi la race tertiaire ne pourrait passer pour véritablement fondée. On devine donc qu’il faut ici des conditions de temps d’autant plus considérables, que les deux nations fusionnées seront plus nombreuses. Jusqu’à ce que le mélange soit complet et que la ressemblance et l’identité physiologique des individualités aient été établies, il n’y a pas sous-genre nouveau, il n’y a pas développement normal d’une originalité propre, bien que composite ; il n’existe que la confusion et le désordre qui naissent toujours de la combinaison inachevée d’éléments naturellement étrangers l’un à l’autre.\par
Nous n’avons qu’une très faible connaissance historique des races tertiaires. Ce n’est qu’aux débuts les plus brumeux des chroniques humaines que nous pouvons entrevoir, sur certains points, l’espèce blanche dans cet état qui ne paraît, nulle part, avoir duré longtemps. Les penchants essentiellement civilisateurs de cette race d’élite la poussaient constamment à se mélanger avec les autres peuples. Quant aux deux types jaune et noir, là où on les trouve à cet état tertiaire, ils n’ont pas d’histoire, car ce sont des sauvages \footnote{M. Carus donne son puissant appui à la loi que j’ai établie au sujet de l’aptitude particulière des races civilisatrices à se mélanger, lorsqu’il fait ressortir la variété extrême de l’organisme humain perfectionné et la simplicité des corpuscules microscopiques qui occupent le plus bas degré de l’échelle des êtres. Il tire de cette remarque ingénieuse l’axiome suivant : « Toutes les fois qu’entre les éléments d’un tout organique, il y a la plus « grande similitude possible, leur état ne peut être considéré comme l’expression haute et « parfaite d’un développement complet. Ce n’est qu’un développement primitif et « élémentaire. » (\emph{Ueber die ungl. B. d. versch. Menschheitst f. bœb. geist. Entwick.}, p. 4.) Ailleurs, il ajoute : « La plus grande diversité, c’est-à-dire inégalité possible des parties, « jointe à l’unité la plus complète de l’ensemble, apparaît partout comme la mesure de la « plus haute perfection d’un organisme. » C’est, dans l’ordre politique, l’état d’une société où les classes gouvernantes, habilement hiérarchisées, sont strictement distinctes, ethniquement parlant, des classes populaires.}.\par
Aux races tertiaires en succèdent d’autres que j’appellerai quartenaires. Elles proviennent de l’hymen de deux grandes variétés. Les Polynésiens nés du mélange du type jaune avec le type noir \footnote{C’est probablement par suite d’une faute de typographie que M. Flourens (\emph{Éloge de Blumenbach}, p. XI) donne la race polynésienne comme « un mélange de deux autres, la \emph{caucasique} et la mongo­lique. » C’est la noire et la mongolique que le savant académicien a certainement voulu dire.}, les mulâtres, produits par les blancs et les noirs, voilà des générations qui appartiennent au type quartenaire. Inutile de faire remarquer, une fois de plus, que le nouveau type unit d’une manière plus ou moins parfaite des caractères spéciaux aux traits qui rappellent sa double descendance.\par
Du moment qu’une race quartenaire est encore modifiée par l’intervention d’un type nouveau, le mélange ne se pondère plus que difficilement, ne se combine plus que lentement et a grand-peine à se régulariser. Les caractères originels entrés dans sa com­position, déjà considérablement affaiblis, sont de plus en plus neutralisés. Ils tendent à disparaître dans une confusion qui devient le principal cachet du nouveau produit. Plus ce produit se multiplie et se croise, plus cette disposition augmente. Elle arrive à l’infini. La population où on la voit s’accomplir est trop nombreuse pour que l’équilibre ait quelque chance de s’établir avant des séries de siècles. Elle ne présente qu’un spectacle effrayant d’anarchie ethnique. Dans les individualités, on retrouve, çà et là, tel trait dominant qui rappelle d’une manière sûre que cette population a dans les veines du sang de toute provenance. Tel homme aura la chevelure du nègre, tel autre le faciès mongol ; celui-ci les yeux du Germain, celui-là la taille du Sémite, et ce seront tous des parents ! Voilà le phénomène offert par les grandes nations civilisées, et on l’observe surtout dans leurs ports de mer, leurs capitales et leurs colonies, lieux où les fusions s’accomplissent avec le plus de facilité. À Paris, à Londres, à Cadix, à Constantinople, on trouvera, sans sortir de l’enceinte des murs, et en se bornant à l’observation de la population qui se dit indigène, des caractères appartenant à toutes les branches de l’humanité. Dans les basses classes, depuis la tête prognathe du nègre jusqu’à la face triangulaire et aux yeux bridés du Chinois, on verra tout ; car, depuis la domination des Romains principalement, les races les plus lointaines et les plus disparates ont fourni leur contingent au sang des habitants de nos grandes villes. Les invasions successives, le commerce, les colonies implantées, la paix et la guerre ont contribué, à tour de rôle, à augmenter le désordre, et si l’on pouvait remonter un peu haut sur l’arbre généalogique du premier homme venu, on aurait chance d’être étonné de l’étrangeté de ses aïeux \footnote{Les caractères physiologiques des différents ancêtres se représentent dans les descendants suivant des règles fixes. Ainsi l’on observe dans l’Amérique du Sud que les produits d’un blanc et d’une négresse peuvent, à la première génération, avoir les cheveux plats et souples ; mais, invaria­blement, à la seconde, le lainage crépu apparaît. (A. d’Orbigny, \emph{l’Homme américain}, t. I, p. 143.)}.\par
Après avoir établi la différence physique des races, il reste encore à décider si ce fait est accompagné d’inégalité, soit dans la beauté des formes soit dans les mesures de la force musculaire. La question ne saurait rester longtemps douteuse.\par
J’ai déjà constaté que, de tous les groupes humains, ceux qui appartiennent aux nations européennes et à leur descendance sont les plus beaux. Pour en être pleinement convaincu, il suffit de comparer les types variés répandus sur le globe, et l’on voit que depuis la construction et le visage, en quelque sorte, rudimentaire du Pélagien et du Pécherai jusqu’à la taille élevée, aux nobles proportions de Charlemagne, jusqu’à l’intelligente régularité des traits de Napoléon, jusqu’à l’imposante majesté qui respire sur le visage royal de Louis XIV, il y a une série de gradations par laquelle les peuples qui ne sont pas du sang des blancs approchent de la beauté, mais ne l’atteignent pas.\par
Ceux qui y touchent de plus près sont nos plus proches parents : telles la famille ariane dégénérée de l’Inde et de la Perse, et les populations sémitiques les moins rabaissées par le contact noir \footnote{Il est à remarquer que les mélanges les plus heureux, au point de vue de la beauté, sont ceux qui sont formés par l’hymen des blancs et des noirs. On n’a qu’à mettre en parallèle le charme souvent puissant des mulâtresses, des capresses, des quarteronnes avec les produits des jaunes et des blancs, comme les femmes russes et hongroises. La comparaison ne tourne pas à l’avantage de ces dernières. Il n’est pas moins certain qu’un beau Radjepout est plus idéalement beau que le Slave le plus accompli.}. À mesure que toutes ces races s’éloignent trop du type blanc, leurs traits et leurs membres subissent des incorrections de formes, des défauts de proportion qui, en s’amplifiant, de plus en plus, chez celles qui nous sont devenues étrangères, finissent par produire cette excessive laideur, partage antique, caractère ineffaçable du plus grand nombre des branches humaines. On n’en est plus à écouter la doctrine reproduite par Helvétius dans son livre de l’\emph{Esprit}, et qui consiste à faire de la notion du beau une idée purement factice et variable. Que tous ceux qui pourraient conserver encore quelques scrupules à cet égard consultent l’admirable essai de M. Gioberti \footnote{Gioberti, \emph{Essai sur le Beau}, traduction de M. Bertinatti, p. 6 et 25.}, il ne leur restera rien à contester. Nulle part on n’a mieux démontré que le beau est une idée absolue et nécessaire, qui ne saurait avoir une application facultative, et c’est en vertu des principes solides établis par le philosophe piémontais que je n’hésite pas à reconnaître la race blanche pour supérieure en beauté à toutes les autres, qui, entre elles, diffèrent encore dans la mesure où elles se rapprochent ou s’éloignent du modèle qui leur est offert. Il y a donc inégalité de beauté dans les groupes humains, inégalité logique, expliquée, permanente et indélébile.\par
Y a-t-il aussi inégalité de forces ? Sans contredit, les sauvages de l’Amérique, comme les Hindous, sont de beaucoup nos inférieurs sur ce point. Les Australiens se trouvent dans le même cas. Les nègres ont également moins de vigueur musculaire \footnote{Voir, entre autres, pour les indigènes américains, Martius et Spix, \emph{Reise in Brasilien}, t. I, p. 259 ; pour les nègres, Pruner, der Neger, eine aphoristische Skizze aus der medicinischen \emph{Topographie von Cairo}, dans la \emph{Zeitsch.dl. deutsch. morgenl. Gesellsch}., t. I, p. 131 ; pour la supériorité musculaire des blancs sur toutes les autres races, Carus, \emph{Ueber die hungl. Befæhigung}, etc., p. 84.}. Tous ces peuples supportent infiniment moins les fatigues. Mais il y a lieu de distin­guer entre la force purement musculaire, celle qui n’a besoin pour vaincre que de se déployer à un seul moment donné, et cette puissance de résistance dont le caractère le plus remarquable est la durée. Cette dernière est plus typique que la première, qui rencontrerait au besoin des rivales, même dans les races les plus notoirement faibles. La pesanteur du poing, si on voulait la prendre comme unique critérium de la force, trouve chez des peuplades nègres fort abruties, chez des Nouveaux-Zélandais très débilement constitués, chez des Lascars, chez des Malais, quelques individus qui peuvent l’exercer de manière à contre-balancer les exploits de la populace anglaise ; tandis qu’à prendre les nations en masse, et en les jugeant d’après la somme de travaux qu’elles endurent sans fléchir, la palme appartient à nos peuples de race blanche.\par
Parmi ces peuples même, pour la force comme pour la beauté, l’inégalité se rencon­tre encore dans les différents groupes tout aussi bien, quoiqu’à un degré inférieur. Les Italiens sont plus beaux que les Allemands et que les Suisses, plus beaux que les Français et que les Espagnols. De même les Anglais présentent un caractère de beauté corporelle supérieur à celui des nations slaves.\par
Quant à la force du poing, les Anglais priment toutes les autres races européennes ; tandis que les Français et les Espagnols possèdent une puissance supérieure de résis­tance à la fatigue, aux privations, aux intempéries des climats les plus durs. La question a été mise hors de doute pour les Français, lors de la funeste campagne de Russie. Là où les Allemands et les troupes du Nord, habituées cependant aux rigueurs de la température, s’affaissèrent, presque en totalité, sous la neige, nos régiments, tout en payant un horrible tribut aux rigueurs de la retraite, purent cependant sauver le plus de monde. On a voulu attribuer cette prérogative à la supériorité de l’éducation morale et du sentiment guerrier. L’explication est peu satisfaisante. Les officiers allemands, qui périrent par centaines, avaient tout autant d’honneur et une conception aussi élevée du devoir que nos soldats, et ils n’en succombèrent pas moins. Concluons donc que les populations françaises possèdent certaines qualités physiques supérieures à celles de la famille allemande et qui leur permettent de braver, sans mourir, les neiges de la Russie comme les sables brûlants de l’Égypte.
\section[{I.13. Les races humaines sont intellectuellement inégales ; l’humanité n’est pas perfectible à l’infini.}]{I.13. \\
Les races humaines sont intellectuellement inégales ; l’humanité n’est pas perfectible à l’infini.}
\noindent Pour bien apprécier les différences intellectuelles des races, le premier soin doit être de constater jusqu’à quel degré de stupidité l’humanité peut descendre. Nous connais­sons déjà le plus bel effort qu’elle puisse produire : c’est la civilisation.\par
La plupart des observateurs scientifiques ont eu jusqu’ici une tendance marquée à rabaisser, au delà de la vérité, les types les plus infimes.\par
Presque tous les premiers renseignements sur une tribu sauvage la dépeignent sous des couleurs faussement horribles, et lui assignent une telle impuissance d’intelligence et de raisonnement, qu’elle tombe au niveau du singe et au-dessous de l’éléphant. Ce jugement, il est vrai, a ses contrastes. Un navigateur est-il bien reçu dans une île, croit-il trouver, chez les habitants, de la douceur et un accueil hospitalier, réussit-il à en déterminer quelques-uns à travailler, un tant soit peu, avec les matelots, aussitôt les éloges s’accumulent sur l’heureuse peuplade ; elle est déclarée bonne à tout, propre à tout, capable de tout, et quelquefois l’enthousiasme, franchissant toutes limites, jure avoir trouvé chez elle des esprits supérieurs.\par
Il faut en appeler du jugement trop favorable comme du trop sévère. Parce que certains Taïtiens auront contribué au radoubage d’un baleinier, leur nation n’est pas pour cela civilisable. Parce que tel homme de Tonga-Tabou aura montré de la bienveil­lance à des étrangers, il n’est pas nécessairement accessible à tous les progrès, et, de même, on n’est pas autorisé à ravaler jusqu’à la brute tel indigène d’une côte longtemps inconnue, parce qu’il aura reçu les premiers visiteurs à coups de flèche, ou même parce qu’on l’aura trouvé mangeant des lézards crus et des boules de terre. Ce genre de repas n’annonce pas, sans doute, une intelligence bien relevée ni des mœurs bien cultivées. Mais, qu’on en soit certain toutefois, chez le cannibale le plus répugnant, il reste une étincelle du feu divin, et la compréhension peut s’allumer chez lui au moins jusqu’à un certain degré. Pas de tribus si humbles qui ne portent, sur les choses dont elles sont entourées, des jugements quelconques, vrais ou faux, justes ou erronés, qui, par le fait seul qu’ils existent, prouvent suffisamment la persistance d’un rayon intellectuel dans toutes les branches de l’humanité. C’est par là que les sauvages les plus dégradés sont accessibles aux enseignements de la religion et qu’ils se distinguent, d’une manière toute particulière et toujours reconnaissable, des brutes les plus intelligentes.\par
Cependant, cette vie morale, placée au fond de la conscience de chaque individu de notre espèce, est-elle capable de se dilater à l’infini ? Tous les hommes ont-ils, à un degré égal, le pouvoir illimité de progresser dans leur développement intellectuel ? Autrement dit, les différentes races humaines sont-elles douées de la puissance de s’égaler les unes les autres ? Cette question est, au fond, celle de la perfectibilité indéfinie de l’espèce et de l’égalité des races entre elles. Sur les deux points, je réponds non.\par
L’idée de la perfectibilité à l’infini séduit beaucoup les modernes et ils s’appuient sur cette remarque que notre mode de civilisation possède des avantages et des mérites que nos prédécesseurs, différemment cultivés, n’avaient pas. On cite tous les faits qui distinguent nos sociétés. J’en ai parlé déjà ; je me prête volontiers à les énumérer de nouveau.\par
On assure donc que nous possédons, sur tout ce qui relève du domaine de la science, des opinions plus vraies ; que nos mœurs sont, en général, douces, et notre morale préférable à celles des Grecs et des Romains. Nous avons aussi, ajoute-t-on, au sujet de la liberté politique, des idées et des sentiments, des opinions, des croyances, des tolérances qui prouvent mieux que tout le reste notre supériorité. Il ne manque pas de théoriciens à belles espérances pour soutenir que les conséquences de nos insti­tutions doivent nous conduire tout droit à ce jardin des Hespérides, si cherché et si peu trouvé depuis que les plus anciens navigateurs en ont constaté l’absence aux îles Canaries.\par
Un examen un peu plus sérieux de l’histoire fait justice de ces hautes prétentions.\par
Nous sommes, à la vérité, plus savants que les anciens. C’est que nous avons profité de leurs découvertes. Si nous possédons plus de connaissances, c’est unique­ment parce que nous sommes leurs continuateurs, leurs élèves et leurs héritiers. S’ensuit-il que la découverte des forces de la vapeur et la solution de quelques problèmes de la mécanique nous acheminent vers l’omniscience ? Tout au plus, ces succès nous conduiront à pénétrer dans tous les secrets du monde matériel. Lorsque nous aurons achevé cette conquête, pour laquelle il y a encore à faire bien et bien des choses qui ne sont pas même commencées, ni entrevues, aurons-nous avancé d’un seul pas au delà de la pure et simple constatation des lois physiques ? Nous aurons, je le veux, beaucoup augmenté nos forces pour réagir sur la nature et la plier à nos besoins. Nous aurons encore traversé la terre de part en part, ou reconnu définitivement ce trajet impraticable. Nous aurons appris à nous diriger dans les airs, et, en nous rapprochant de quelques milliers de mètres des limites de l’air respirable, découvert et éclairci certains problèmes astronomiques ou autres ; rien de plus. Tout cela ne nous mène pas à l’infini. Et eussions-nous compté tous les systèmes planétaires qui se meuvent dans l’espace, serions-nous plus près de cet infini ? Avons-nous appris, sur les grands mystères, une chose ignorée des anciens ? Nous avons, ce me semble, changé les méthodes employées avant nous, pour tourner autour du secret. Nous n’avons pas fait un pas de plus dans ses ténèbres.\par
Puis, en admettant que nous soyons plus éclairés sur certains faits, combien, d’autre part, nous avons perdu de notions familières à nos plus lointains ancêtres ! Est-il douteux qu’au temps d’Abraham, on ne sût de l’histoire primordiale beaucoup plus que nous n’en connaissons ? Combien de choses découvertes par nous, à grand-peine, ou par hasard, ne sont en définitive que des connaissances oubliées et retrouvées ! Et comme, sur bien des points, nous sommes inférieurs à ce qu’on a été jadis ! Que pourrait-on comparer, ainsi que je le disais plus haut pour un autre objet, oui, que pourrait-on comparer, en choisissant dans nos plus splendides travaux, à ces merveilles que l’Égypte, l’Inde, la Grèce, l’Amérique nous montrent encore, attestant la magnifi­cence sans bornes de tant d’autres édifices que le poids des siècles a fait disparaître, bien moins que les ineptes ravages de l’homme ? Que sont nos arts auprès de ceux d’Athènes ? Que sont nos penseurs auprès de ceux d’Alexandrie et de l’Inde ? Que sont nos poètes auprès de Valmiki, de Kalidasa, d’Homère et de Pindare ?\par
En somme, nous faisons autrement. Nous appliquons notre esprit à d’autres buts, à d’autres recherches que les autres groupes civilisés de l’humanité ; mais, en changeant de terrain, nous n’avons pu conserver dans toute leur fertilité les terres qu’ils cultivaient déjà. Il y a donc eu abandon d’un côté, en même temps qu’il y avait conquête de l’autre. C’était une triste compensation, et, loin d’annoncer un progrès, elle n’indique qu’un déplacement. Pour qu’il y eût acquisition réelle, il faudrait qu’ayant au moins gardé dans toute leur intégrité les principales richesses des sociétés antérieures, nous eussions réussi à édifier, à côté de leurs travaux, certains grands résultats qu’elles et nous avons cherchés également ; que nos sciences et nos arts, appuyés sur leurs arts et leurs sciences, eussent trouvé quelque nouveauté profonde touchant la vie et la mort, la formation des êtres, les principes primordiaux du monde. Or, sur toutes ces questions, la science moderne n’a plus ces lueurs qui se projetaient, on a lieu de le penser, à l’aurore des temps antiques, et, de son propre cru et de ses propres efforts, elle n’est parvenue encore qu’à cet humiliant aveu : « Je cherche et ne trouve pas. » Il n’y a donc guère de progrès réels dans les conquêtes intellectuelles de l’homme. Notre critique seule est incontestablement meilleure que celle de nos devanciers. C’est un grand point ; mais \emph{critique} veut dire \emph{classement}, et non pas \emph{acquisition.}\par
Pour ce qui est de nos idées prétendues neuves sur la politique, on peut sans inconvénient prendre avec elles des libertés plus vives encore qu’avec nos sciences.\par
Cette fécondité de théories, dont nous aimons à nous faire honneur, on la retrouve tout aussi grande à Athènes après Périclès. Le moyen de s’en convaincre, c’est de relire ces comédies d’Aristophane, amplifications satiriques, dont Platon recommandait la lecture à qui voulait connaître les mœurs publiques de la ville de Minerve. On récuse la comparaison depuis que l’on s’est avisé de prétendre qu’entre notre ordre social actuel et l’état de l’antiquité grecque la servitude crée une différence fondamentale. La démago­gie n’en était que plus profonde, si l’on veut, et voilà tout. On parlait alors des esclaves sur le même ton où l’on parle aujourd’hui des ouvriers et des prolétaires, et combien n’était-il pas avancé, ce peuple athénien qui fit tant pour plaire à sa plèbe servile après le combat des Arginuses !\par
Transportons-nous à Rome. Ouvrons les lettres de Cicéron. Quel tory modéré que cet orateur romain ! quelle similitude parfaite entre sa république et nos sociétés constitutionnelles, quant au langage des partis et aux luttes parlementaires ! Là, aussi, dans les bas-fonds, s’agitait une population d’esclaves dépravés, toujours la révolte dans le cœur, quand ils ne l’avaient pas au bout des poings. Laissons cette tourbe. Nous le pouvons d’autant mieux que la loi ne lui reconnaissait pas d’existence civile, qu’elle ne comptait pas dans la politique, et n’agissait sur les décisions, aux jours d’émeute, que comme auxiliaire des perturbateurs de naissance libre.\par
Eh bien ! les esclaves rejetés dans le néant, n’avons-nous pas, sur le Forum, tout ce qui constitue un état social à la moderne ? La populace, qui demandait du pain, des jeux, des distributions gratuites et le droit de jouir ; la bourgeoisie, qui voulait et obtint le partage des emplois publics ; le patriciat, transformé successivement et reculant toujours, et toujours perdant de ses droits, jusqu’au moment où ses défenseurs mêmes acceptèrent, comme unique système de défense, de refuser toute prérogative en ne réclamant que la liberté pour tous ? Ne sont-ce pas là des ressemblances parfaites ?\par
Croit-on que dans les opinions qui s’expriment aujourd’hui, si variées qu’elles puissent être, il en existe une seule, il se trouve même une nuance qui n’ait été connue à Rome ? Je parlais tout à l’heure des lettres écrites de Tusculum : c’est la pensée d’un conservateur progressiste. Vis-à-vis de Sylla, Pompée et Cicéron étaient des libéraux. Ils ne l’étaient pas encore assez pour César. Ils l’étaient trop pour Caton. Plus tard, sous le principat, nous voyons, dans Pline le Jeune, un royaliste modéré, ami du repos quand même. Il ne veut ni de trop de liberté, ni d’excès de pouvoir, et, positif dans ses doctrines, tenant très peu aux grandeurs évanouies de l’âge des Fabius, il leur préférait la prosaïque administration de Trajan. Ce n’était pas l’avis de tout le monde. Beaucoup de gens pensaient, redoutant quelque résurrection de l’ancien Spartacus, que l’empereur ne pouvait trop faire sentir sa puissance. Quelques provinciaux, au rebours, deman­daient et obtenaient ce que nous appellerions des garanties constitutionnelles ; tandis que les opinions socialistes ne trouvaient pas de moindres interprètes que le césar gaulois C. Junius Posthumus, qui s’écriait dans ses déclamations : \foreign{Dives et pauper, inimici}, le riche et le pauvre sont des ennemis-nés.\par
Bref, tout homme ayant quelque prétention à participer aux lumières du temps soutenait avec force l’égalité du genre humain, le droit universel à posséder les biens de cette terre, la nécessité évidente de la civilisation gréco-latine, sa perfection, sa douceur, ses progrès futurs plus grands encore que ses avantages actuels, et, pour couronner le tout, son éternité. Ces idées n’étaient pas seulement la consolation et l’orgueil des païens ; c’était aussi l’espoir solide des premiers, des plus illustres Pères de l’Église, dont Tertullien se faisait l’interprète \footnote{Amédée Thierry, \emph{Histoire de la Gaule sous l’administration romaine}, t. I, p. 241.}.\par
Enfin, pour achever le tableau d’un dernier trait frappant, le plus nombreux de tous les partis était celui des indifférents, de ces gens trop faibles, trop dégoûtés, trop craintifs ou trop indécis pour saisir une vérité au milieu de toutes les théories dispa­rates qu’ils voyaient sans cesse miroiter à leurs yeux, et qui, jouissant de l’ordre quand il existait, supportant, tant bien que mal, le désordre quand il venait, admiraient, en tous temps, le progrès des jouissances matérielles inconnues à leurs pères, et, sans trop vouloir penser au reste, se consolaient en répétant à satiété :\par

\begin{center}
\noindent On travaille aujourd’hui d’un air miraculeux.\par
\end{center}

\noindent Il y aurait plus de raisons de croire à des perfectionnements dans la science politique, si nous avions inventé quelque rouage inconnu jusqu’à nous, et qui n’ait pas été auparavant pratiqué, au moins dans l’essentiel. Cette gloire nous manque. Les monarchies limitées ont été connues de tous temps. On en voit même des modèles curieux chez certaines peuplades américaines restées cependant barbares. Les républi­ques démocratiques et aristocratiques de toutes formes et pondérées suivant les méthodes les plus variées ont existé dans le nouveau monde comme dans l’ancien. Tlascala est, en ce genre, un spécimen complet tout comme Athènes, Sparte, et La Mecque avant Mahomet. Et quand même, d’ailleurs, il serait vrai que nous eussions appliqué à la science gouvernementale quelque perfectionnement secondaire de notre invention, en serait-ce assez pour justifier une prétention si grosse que celle de la perfectibilité illimitée ? Soyons modestes, comme le fut un jour le plus sage des rois : \foreign{Nil novi sub sole} \footnote{ \noindent On est quelquefois disposé à considérer le gouvernement des États-Unis d’Amérique comme une création tout à fait originale et particulière à notre époque, et ce qu’on y relève de surtout remarquable, c’est la part restreinte abandonnée dans cette société à 1’initiative et même à la simple intervention de l’autorité gouvernementale ou administrative. Si l’on veut jeter les yeux sur tous les commencements d’États fondés par la race blanche, on aura identiquement le même spectacle. Le \emph{self-government} n’est pas aujourd’hui plus triomphant à New-York, qu’il ne le fut jadis à Paris, au temps des Franks. Les Indiens, il est vrai, sont traités beaucoup plus inhumainement par les Américains que ne le furent les Gaulois par les leudes de Khlodowig. Mais il faut considérer que la distance ethnique est bien plus grande entre les républicains éclairés du nouveau monde et leurs victimes, qu’elle ne l’était entre le conquérant germain et ses vaincus.\par
 Du reste, lorsque, par la suite, j’exposerai les débuts de toutes les sociétés arianes, on verra que \emph{toutes} ont commencé par l’exagération de l’indépendance vis-à-vis du magistrat et vis-à-vis de la loi.\par
 Les inventions politiques de ce monde ne sauraient, ce me semble, sortir des deux limites tracées par deux peuples situés, l’un dans le nord-est de l’Europe, l’autre dans les pays riverains du Nil, à l’extrême sud de l’Égypte. Le gouvernement du premier de ces peuples, à Bolgari, près de Kazan, avait l’habitude de \emph{faire pendre les gens d’esprit}, comme moyen préventif. C’est au voyageur arabe Ibn Foszlan que nous devons la connaissance de ce fait. (A. de Humboldt, \emph{Asie centrale}, t. I, p. 494.)\par
 Chez l’autre nation, habitant le Fazoql, lorsque le roi ne convient plus, ses parents et ses ministres viennent le lui annoncer, et on lui fait remarquer que, puisqu’il ne plaît plus \emph{aux hommes, aux femmes, aux enfants, aux bœufs, aux ânes}, etc., le mieux qu’il puisse faire, c’est de mourir, et on l’y aide aussitôt. (Lepsius, \emph{Briefe aux Ægypten, Æthiopien und der Halbinsel des Sinai;} Berlin, 1852.)
}.\par
Voyons nos mœurs, maintenant. On les dit plus douces que celles des autres gran­des sociétés humaines : c’est encore une affirmation qui tente bien fort la critique.\par
Il est des rhétoriciens qui voudraient aujourd’hui faire disparaître du code des nations le recours à la guerre. Ils ont pris cette théorie dans Sénèque. Certains sages de l’Orient professaient aussi, à cet égard, des idées toutes conformes à celles des Frères moraves. Mais quand bien même les amis de la paix universelle réussiraient à dégoûter l’Europe de l’appel aux armes, il leur faudrait encore amener les passions humaines à se transformer pour toujours. Ni Sénèque ni les brahmanes n’ont obtenu cette victoire. Il est douteux qu’elle nous soit réservée, et pour ce qui est de notre mansuétude, regardez dans nos champs, dans nos rues, la trace sanglante qu’elle y creuse.\par
Nos principes sont purs et élevés, je le veux. La pratique y répond-elle ?\par
Attendons, pour nous vanter, que nos pays, qui depuis le commencement de la civilisation moderne ne sont pas encore restés cinquante ans sans massacres, puissent se glorifier, comme l’Italie romaine, de deux siècles de paix, qui n’ont d’ailleurs, hélas ! rien prouvé pour l’avenir \footnote{Amédée Thierry, \emph{Histoire de la Gaule sous l’administration romaine}, t. I, p. 241.} !\par
La perfectibilité humaine n’est donc pas démontrée par l’état de notre civilisation. L’homme a pu apprendre certaines choses, il en a oublié beaucoup d’autres. Il n’a pas ajouté un sens à ses sens, un membre à ses membres, une faculté à son âme. Il n’a fait que tourner d’un autre côté du cercle qui lui est dévolu, et la comparaison de ses destinées à celles de nombreuses familles d’oiseaux et d’insectes n’est pas même propre à inspirer toujours des pensées bien consolantes sur son bonheur d’ici-bas.\par
Depuis le moment où les termites, les abeilles, les fourmis noires ont été créées, elles ont trouvé spontanément le genre de vie qui leur convenait. Les termites et les fourmis, dans leurs communautés, ont d’abord découvert, pour leurs demeures, un mode de construction, et pour leurs provisions un emmagasinement, pour leurs actifs un système de soins, dont les naturalistes pensent qu’il n’admet pas de variations ni de perfectionnements \footnote{Martius und Spix, \emph{Reise in Brasilien}, t. III, p. 950 et passim.}. Du moins tel qu’il est, il a constamment suffi aux besoins des pauvres êtres qui l’emploient. De même les abeilles, avec leur gouvernement monarchi­que exposé à des renversements de souveraines, jamais à des révolutions sociales, n’ont pas, un seul jour, ignoré la manière de vivre la plus appropriée à ce que désire leur nature. Il a été loisible longtemps aux métaphysiciens d’appeler les animaux des machines, et de reporter à Dieu, \foreign{anima brutorum}, la cause de leurs mouvements. Aujourd’hui que, d’un œil un peu plus soigneux, on étudie les mœurs de ces prétendus automates, on ne s’est pas borné à abandonner cette doctrine dédaigneuse : on a reconnu à l’instinct une portée qui l’approche de la dignité de la raison.\par
Que dire lorsque, dans les royaumes des abeilles, on voit les souveraines exposées à la colère des sujettes, ce qui suppose, ou l’esprit de mutinerie chez ces dernières, ou l’inaptitude à remplir de légitimes obligations chez les reines ? Que dire, lorsqu’on voit les termites épargner leurs ennemis vaincus, puis les enchaîner et les employer à l’utilité publique en les forçant d’avoir soin des jeunes individus ?\par
Sans doute nos États, à nous, sont plus compliqués, satisfont à plus de besoins ; mais, lorsque je regarde le sauvage errant, sombre, sale, farouche, désœuvré, traînant paresseusement ses pas et le bâton pointu qui lui sert de lance sur un sol sans culture ; quand je le contemple, suivi de sa femme, unie à lui par un hymen dont une violence férocement inepte a constitué toute la cérémonie \footnote{Chez plusieurs peuplades de l’Océanie, voici comme on a conçu l’institution du mariage : l’homme remarque une fille. Elle lui convient. Il l’obtient du père moyennant quelques cadeaux, parmi lesquels une bouteille d’eau-de-vie, quand le futur a pu l’offrir, tient le rang le plus distingué. Alors le prétendu va s’embusquer au coin d’un buisson ou derrière un rocher. La fille passe sans songer à mal. Il la renverse d’un coup de bâton ; la frappe jusqu’à ce qu’elle ait perdu connaissance et l’emporte amoureusement chez lui, baignée dans son sang. Il est en règle. L’union légale est accomplie.} ; quand je vois cette femme portant son enfant, qu’elle va tuer elle-même s’il tombe malade, ou seulement s’il l’ennuie \footnote{M. d’Orbigny raconte que les mères indiennes aiment leurs enfants à l’excès, qu’elles les chérissent au point d’en être véritablement les esclaves ; que cependant, par une bizarrerie sans exemple, si l’enfant vient à les gêner un jour, elles le noient ou l’écrasent, ou l’abandonnent, sans nul regret, dans les bois. (D’Orbigny, \emph{l’Homme américain}, t. II, p. 232.)} ; que tout à coup, la faim se faisant sentir, ce misérable groupe, à la recherche d’un gibier quelconque, s’arrête charmé devant une de ces demeures d’intelligentes fourmis, donne du pied dans l’édifice, en ravit et en dévore les oeufs, puis, le repas fait, se retire tristement dans un creux de rocher, je me demande si les insectes qui viennent de périr n’ont pas été plus favorablement doués que la stupide famille du destructeur ; si l’instinct des animaux, borné à un court ensemble de besoins, ne les rend pas plus heureux que cette raison avec laquelle notre humanité s’est trouvée nue sur la terre, et plus exposée cent fois que les autres espèces aux souffrances que peuvent causer l’air, le soleil, la neige et la pluie conjurés. Pauvre humanité ! elle n’est jamais parvenue à inventer un moyen de vêtir tout le monde et de mettre tout le monde à l’abri de la soif et de la faim. Certes le moindre des sauvages en sait plus long que les animaux ; mais les animaux connaissent ce qui leur est utile, et nous l’ignorons. Ils s’y tiennent, et nous ne le pouvons garder, quand parfois nous l’avons découvert. Ils sont toujours, en temps normal, assurés, par leurs instincts, de trouver le nécessaire. Nous, nous voyons de nombreuses hordes qui, depuis le commencement des siècles, n’ont pu sortir d’un état précaire et souffreteux. En tant qu’il n’est question que du bien-être terrestre, nous n’avons de mieux que les animaux, rien de mieux qu’un horizon plus étendu à parcourir, mais fini et borné comme le leur.\par
Je n’ai pas assez insisté sur cette triste condition humaine, de toujours perdre d’un côté quand nous gagnons de l’autre ; c’est là cependant le grand fait qui nous condamne à errer dans nos domaines intellectuels, sans réussir jamais, tout limités qu’ils sont, à les posséder dans leur entier. Si cette loi fatale n’existait pas, on comprendrait qu’à un jour donné, lointain peut-être, en tous cas, probable, l’homme, se trouvant en posses­sion de toute l’expérience des âges successifs, sachant ce qu’il peut savoir, s’étant emparé de ce qu’il peut prendre, aurait enfin appris à appliquer ses richesses, vivrait au milieu de la nature, sans combat avec ses semblables non plus qu’avec la misère, et, tranquille à la fin, se reposerait, sinon à l’apogée des perfections, au moins dans un état suffisant d’abondance et de joie.\par
Une telle félicité, toute restreinte qu’elle serait, ne nous est même pas promise, puisqu’à mesure que l’homme apprend, il désapprend ; puisqu’il ne peut gagner sous le rapport intellectuel et moral sans perdre sous le rapport physique, et qu’il ne tient assez fortement aucune de ses conquêtes pour être assuré de les garder toujours.\par
Nous croyons, nous, que notre civilisation ne périra jamais, parce que nous avons l’imprimerie, la vapeur, la poudre à canon. L’imprimerie, qui n’est pas moins connue au Tonquin, dans l’empire d’Annam et au Japon \footnote{M. J. Mohl, \emph{Rapport annuel à la Société asiatique}, 1851, p. 92 : « La librairie indienne « indigène est extrêmement active, et les ouvrages qu’elle fournit n’entrent jamais dans la « librairie européenne même de l’Inde. M. Sprenger dit, dans une lettre, qu’il y a dans la « seule ville de Luknau treize établissements lithographiques uniquement occupés à « multiplier les livres pour les écoles, et il donne une liste considérable d’ouvrages dont « probablement aucun n’est parvenu en Europe. Il en est de même à Dehli, Agra, Cawnpour, « Allahabad et d’autres villes. »} que dans l’Europe actuelle, a-t-elle, par hasard, donné aux peuples de ces contrées une civilisation même passable ? Ils ont cependant des livres, beaucoup de livres, des livres qui se vendent à bien plus bas prix que les nôtres. D’où vient que ces peuples soient si abaissés, si faibles, si rapprochés du degré où l’homme civilisé, corrompu, faible et lâche, ne vaut pas, en puissance intellectuelle, tel barbare qui, l’occasion s’offrant, va l’opprimer \footnote{Les Siamois sont le peuple le plus déhonté de la terre. Ils gisent au plus bas degré de la civilisation indo-chinoise ; cependant ils savent tous lire et écrire. (Ritter. \emph{Erdkunde, Asien}, t. II, p. 1152.)} ? D’où cela vient-il ? Uniquement de ce que l’imprimerie est un moyen, et non pas un principe. Si vous l’employez à reproduire des idées saines, vigoureuses, salutaires, elle fonctionnera de la manière la plus fructueuse, et contribuera à soutenir la civilisation. Si, au contraire, les intelligences sont tellement abâtardies que personne n’apporte plus sous les presses des œuvres philosophiques, historiques, littéraires, capables de nourrir fortement le génie d’une nation ; si ces presses avilies ne servent plus qu’à multiplier les malsaines et venimeuses compositions de cerveaux énervés, les productions empoisonnées d’une théologie de sectaires, d’une politique de libellistes, d’une poésie de libertins, comment et pourquoi l’imprimerie sauverait-elle la civilisation ?\par
On suppose sans doute que, par la facilité avec laquelle elle peut répandre en grand nombre les chefs-d’œuvre de l’esprit, l’imprimerie contribue à les conserver, et même, dans les temps où la stérilité intellectuelle ne permet pas de leur donner de rivaux, de les offrir au moins aux méditations des gens honnêtes. Il en est ainsi en effet. Toutefois, pour aller chercher un livre du passé et s’en servir à sa propre amélioration, il faut déjà posséder, sans ce livre, le meilleur des biens : la force d’une âme éclairée. Dans les temps mauvais, témoins du départ des vertus publiques, on fait peu de cas des anciennes compositions, et personne ne se soucie de troubler le silence des biblio­thèques. C’est valoir beaucoup déjà que de songer à fréquenter ces lieux augustes, et à de telles époques on ne vaut rien...\par
D’ailleurs on s’exagère beaucoup la longévité assurée aux productions de l’esprit par la découverte de Gutenberg. À l’exception de quelques ouvrages reproduits pendant une certaine période, tous les livres meurent aujourd’hui, comme jadis mouraient les manuscrits. Tirées à quelques centaines d’exemplaires, les œuvres de la science surtout disparaissent avec rapidité du domaine commun. On peut encore les trouver, bien qu’avec peine, dans les grandes collections. Il en était absolument de même des riches­ses intellectuelles de l’antiquité, et, encore une fois, ce n’est pas l’érudition qui sauve un peuple arrivé à la décrépitude.\par
Cherchons ce que sont devenues ces myriades d’excellents ouvrages publiés depuis le jour où fonctionna la première presse. La plupart sont oubliés. Ceux dont on parle encore n’ont plus guère de lecteurs, et tel qui se recherchait il y a cinquante ans voit son titre même disparaître peu à peu de toutes les mémoires.\par
Pour rehausser le mérite de l’imprimerie, on a trop nié la diffusion des manuscrits. Elle était plus grande qu’on ne se l’imagine. Aux temps de l’empire romain, les moyens d’instruction étaient très répandus, les livres étaient même communs, si l’on en doit juger d’après ce nombre extraordinaire de grammairiens déguenillés qui pullulaient jusque dans les plus petites villes, sortes de gens comparables aux avocats, aux romanciers, aux journalistes de notre époque, et dont le \emph{Satyricon} de Pétrone nous raconte les mœurs dévergondées, la misère et le goût passionné des jouissances. Quand la décadence fut complète, tous ceux qui voulaient des livres en trouvaient encore. Virgile était lu partout. Les paysans, qui l’entendaient vanter, le prenaient pour un dangereux enchanteur. Les moines le copiaient. Ils copiaient aussi Pline, Dioscoride, Platon et Aristote. Ils copiaient de même Catulle et Martial. Dans le moyen âge, on peut, au grand nombre qui nous en reste après tant de guerres, de dévastations, d’incendies d’abbayes et de châteaux, deviner combien les œuvres littéraires, scientifi­ques, philosophiques, sorties de la plume des contemporains, avaient été multipliées au delà de ce qu’on pense. On s’exagère donc les mérites réels de l’imprimerie envers la science, la poésie, la moralité et la vraie civilisation, et l’on serait plus exact si, glissant modestement sur cette thèse, on s’attachait surtout à parler des services journaliers rendus par cette invention aux intérêts religieux et politiques de toutes venues, L’imprimerie, je le répète, est un merveilleux instrument ; mais, lorsque la main et la tête font défaut, l’instrument ne saurait bien fonctionner par lui-même.\par
Une longue démonstration n’est pas nécessaire pour établir que la poudre à canon ne peut non plus sauver une société en danger de mort. C’est une connaissance qui ne s’oubliera certainement pas. D’ailleurs il est douteux que les peuples sauvages qui la possèdent aujourd’hui comme nous, et s’en servent autant, la considèrent jamais à un autre point de vue que celui de la destruction.\par
Pour la vapeur et toutes les découvertes industrielles, je dirai aussi, comme de l’imprimerie, que ce sont de grands moyens ; j’ajouterai que l’on a vu quelquefois des procédés nés de découvertes scientifiques se perpétuer à l’état de routine, quand le mouvement intellectuel qui les avait fait naître s’était arrêté pour toujours, et avait laissé perdre le secret théorique d’où ces procédés émanaient. Enfin, je rappellerai que le bien-être matériel n’a jamais été qu’une annexe extérieure de la civilisation, et qu’on n’a jamais entendu dire d’une société qu’elle avait vécu uniquement parce qu’elle connaissait les moyens d’aller vite et de se bien vêtir.\par
Toutes les civilisations qui nous ont précédés ont pensé, comme nous, s’être cramponnées au rocher du temps par leurs inoubliables découvertes. Toutes ont cru à leur immortalité. Les familles des Incas, dont les palanquins parcouraient avec rapidité ces admirables chaussées de cinq ou six cents lieues de long qui unissent encore Cuzco à Quito, étaient convaincues certainement de l’éternité de leurs conquêtes. Les siècles, d’un coup d’aile, ont précipité leur empire, à côté de tant d’autres, dans le plus profond du néant. Ils avaient, eux aussi, ces souverains du Pérou, leurs sciences, leurs mécaniques, leurs puissantes machines dont nous admirons avec stupeur les œuvres sans pouvoir en deviner le secret. Ils connaissaient, eux aussi, le secret de transporter des masses énormes. Ils construisaient des forteresses où l’on entassait les uns sur les autres des blocs de pierre de trente-huit pieds de long sur dix-huit de large. Les ruines de Tihuanaco, nous montrent un tel spectacle, et ces matériaux monstrueux étaient apportés de plusieurs lieues de distance. Savons-nous comment s’y prenaient les ingénieurs de ce peuple évanoui pour résoudre un tel problème ? Nous ne le savons pas plus que les moyens appliqués à la construction des gigantesques murailles cyclo­péennes dont les débris résistent encore, sur tant de points de l’Europe méridionale, aux efforts du temps.\par
Ainsi, ne prenons pas les résultats d’une civilisation pour ses causes. Les causes se perdent, les résultats s’oublient quand disparaît l’esprit qui les avait fait éclore, ou, s’ils persistent, c’est grâce à un nouvel esprit qui va s’en emparer, et souvent leur donner une portée différente de celle qu’ils avaient d’abord. L’intelligence humaine, constam­ment vacillante, court d’un point à un autre, n’a point d’ubiquité, exalte la valeur de ce qu’elle tient, oublie ce qu’elle lâche, et, enchaînée dans le cercle qu’elle est condamnée à ne jamais franchir, ne réussit à féconder une partie de ses domaines qu’en laissant l’autre en friche, toujours à la fois supérieure et inférieure à ses ancêtres. L’humanité ne se surpasse donc jamais elle-même ; l’humanité n’est donc pas perfectible à l’infini.
\section[{I.14. Suite de la démonstration de l’inégalité intellectuelle des races. Les civilisations diverses se repoussent mutuellement. Les races métisses ont des civilisations également métisses.}]{I.14. \\
Suite de la démonstration de l’inégalité intellectuelle des races. Les civilisations diverses se repoussent mutuellement. Les races métisses ont des civilisations également métisses.}
\noindent Si les races humaines étaient égales entre elles, l’histoire nous présenterait un tableau bien touchant, bien magnifique et bien glorieux. Toutes intelligentes, toutes l’œil ouvert sur leurs intérêts véritables, toutes habiles au même degré à trouver le moyen de vaincre et de triompher, elles auraient, dès les premiers jours du monde, égayé la face du globe par une foule de civilisations simultanées et identiques également florissantes ; en même temps que les plus anciens peuples sanscrits fondaient leur empire, et, par la religion et par le glaive, couvraient l’Inde septentrionale de moissons, de villes, de palais et de temples ; en même temps que le premier empire d’Assyrie illustrait les plaines du Tigre et de l’Euphrate par ses somptueuses constructions, et que les chars et la cavalerie de Nemrod défiaient les peuples des quatre vents, on aurait vu, sur la côte africaine, parmi les tribus des nègres à tête prognathe, surgir un état social raisonné, cultivé, savant dans ses moyens, puissant dans ses résultats.\par
Les Celtes voyageurs auraient apporté au fond de l’extrême occident de l’Europe, avec quelques débris de la sagesse orientale des âges primitifs, les éléments indispen­ sables d’une grande société, et auraient certainement trouvé chez les populations ibériennes alors répandues sur la face de l’Italie, dans les îles de la Méditerranée, dans la Gaule et l’Espagne, des rivaux aussi bien renseignés qu’eux-mêmes sur les traditions anciennes, aussi experts dans les arts nécessaires et dans les inventions d’agrément.\par
L’humanité unitaire se serait promenée noblement à travers le monde, riche de son intelligence, fondant partout des sociétés similaires, et peu de temps eût suffi pour que toutes les nations, jugeant leurs besoins de la même façon, considérant la nature du même œil, lui demandant les mêmes choses, se trouvassent dans un contact étroit et pussent lier ces relations, ces échanges multiples, si nécessaires partout et si profi­tables aux progrès de la civilisation.\par
Certaines tribus, malheureusement confinées sous des climats stériles, au fond des gorges de montagnes rocheuses, sur le bord de plages glacées, dans des steppes inces­samment balayées par les vents du nord, auraient pu avoir à lutter plus longtemps que les nations favorisées contre l’ingratitude de la nature. Mais enfin ces tribus, n’ayant pas moins que les autres d’intelligence et de sagesse, n’auraient pas tardé à découvrir qu’il est des remèdes contre l’âpreté des climats. On les aurait vues déployer l’intelli­gente activité que montrent aujourd’hui les Danois, les Norwégiens, les Islandais. Elles auraient dompté le sol rebelle, contraint malgré lui de produire. Dans les régions monta­gneuses, elles auraient, comme les Suisses, exploité les avantages de la vie pastorale, ou, comme les Cachemiriens, recouru aux ressources de l’industrie, et si leur pays avait été si mauvais, sa situation géographique si défavorable que l’impossibilité d’en tirer jamais parti leur eût été bien démontrée, elles auraient réfléchi que le monde était grand, possédait bien des vallons, bien des plaines douces à leurs habitants, et, quittant leur rétive patrie, elles n’auraient pas tardé à rencontrer des terres où déployer fructueuse­ment leur intelligente activité.\par
Alors les nations d’ici-bas, également éclairées, également riches, les unes par le commerce, se multipliant dans leurs cités maritimes, les autres par l’agriculture, florissant dans leurs vastes campagnes, celles-ci par l’industrie exercée dans les lieux alpestres, celles-là par le transit, résultat heureux de leur situation mitoyenne, toutes ces nations, malgré des dissensions passagères, des guerres civiles, des séditions, malheurs inséparables de la condition humaine, auraient imaginé bientôt, entre leurs intérêts, un système de pondération quelconque. Les civilisations identiques d’origine se prêtant beaucoup, s’empruntant de même, auraient fini par se ressembler à peu près de tout point, et l’on aurait vu s’établir cette confédération universelle, rêve de tant de siècles, et que rien ne pourrait empêcher de se réaliser, si, en effet, toutes les races étaient pourvues de la même dose et de la même forme de facultés.\par
On sait de reste que ce tableau est fantastique. Les premiers peuples, dignes de ce nom, se sont agglomérés sous l’empire d’une idée d’association que les barbares, vivant plus ou moins loin d’eux, non seulement n’avaient pas eue aussi promptement, mais n’ont pas eue depuis. Ils ont émigré de leur premier domaine et ont rencontré d’autres peuplades : ces peuplades ont été domptées, elles n’ont jamais ni embrassé sciemment ni compris l’idée qui dominait dans la civilisation qu’on venait leur imposer. Bien loin de témoigner que l’intelligence de toutes les tribus humaines fût semblable, les nations civilisables ont toujours prouvé le contraire, d’abord en asseyant leur état social sur des bases complètement diverses, ensuite en montrant les unes pour les autres un éloigne­ment décidé. La force de l’exemple n’a rien éveillé chez les groupes qui ne se trouvaient pas poussés par un ressort intérieur. L’Espagne et les Gaules ont vu tour à tour les Phéniciens, les Grecs, les Carthaginois établir sur leurs côtes des villes florissantes. Ni l’Espagne ni les Gaules n’ont consenti à imiter les mœurs, les gouvernements de ces marchands célèbres, et, quand les Romains sont venus, ces vainqueurs ne sont parvenus à transformer leur nouveau domaine qu’en le saturant de colonies. Les Celtes et les Ibères ont prouvé alors que la civilisation ne s’acquiert pas sans le mélange du sang.\par
Les peuplades américaines, à quel spectacle ne leur est-il pas donné d’assister en ce moment ? Elles se trouvent placées aux côtés d’un peuple qui veut grandir de nombre pour augmenter de puissance. Elles voient sur leurs rivages passer et repasser des milliers de navires. Elles savent que la force de leurs maîtres est irrésistible. L’espoir de voir, un jour, leurs contrées natales délivrées de la présence des conquérants n’existe chez aucune d’elles. Toutes ont conscience que leur continent tout entier est désormais le patrimoine de l’Européen. Elles n’ont qu’à regarder pour se convaincre de la fécondité de ces institutions exotiques qui ne font plus dépendre la prolongation de la vie de l’abondance du gibier et de la richesse de la pêche. Elles savent, puisqu’elles achètent de l’eau-de-vie des couvertures, des fusils, que même leurs goûts grossiers trouveraient plus aisément satisfaction dans les rangs de cette société qui les appelle, qui les sollicite à venir, qui les paye et les flatte pour avoir leur concours. Elles s’y refusent, elles aiment mieux fuir de solitudes en solitudes ; elles s’enfoncent de plus en plus dans l’intérieur des terres. Elles abandonnent tout, jusqu’aux os de leurs pères. Elles mourront, elles le savent ; mais une mystérieuse horreur les maintient sous le joug de leurs invincibles répugnances, et, tout en admirant la force et la supériorité de la race blanche, leur conscience, leur nature entière, leur sang enfin, se révoltent à la seule idée d’avoir rien de commun avec elle.\par
Dans l’Amérique espagnole on croit rencontrer moins d’aversion chez les indigènes. C’est que le gouvernement métropolitain avait jadis laissé ces peuples sous l’adminis­tration de leurs caciques. Il ne cherchait pas à les civiliser. Il leur permettait de conserver leurs usages et leurs lois, et, pourvu qu’ils fussent chrétiens, il ne leur deman­dait qu’un tribut d’argent. Lui-même ne colonisait guère. La conquête une fois achevée, il s’abandonna à une tolérance indolente, et n’opprima que par boutades. C’est pourquoi les Indiens de l’Amérique espagnole sont moins malheureux et continuent à vivre, tandis que les voisins des Anglo-Saxons périront sans miséricorde.\par
Ce n’est pas seulement pour les sauvages que la civilisation est incommunicable, c’est aussi pour les peuples éclairés. La bonne volonté et la philanthropie française en font, en ce moment, l’épreuve dans l’ancienne régence d’Alger d’une manière non moins complète que les Anglais dans l’Inde et les Hollandais à Batavia. Pas d’exemples, pas de preuves plus frappantes, plus concluantes de la dissemblance et de l’inégalité des races entre elles.\par
Car si l’on raisonnait seulement d’après la barbarie de certains peuples, et que, déclarant cette barbarie originelle, on en conclût que toute espèce de culture leur est refusée, on s’exposerait à des objections sérieuses. Beaucoup de nations sauvages ont conservé des traces d’une situation meilleure que celle où nous les voyons plongées. Il est des tribus, fort brutales d’ailleurs, qui, pour la célébration des mariages, pour la répartition des héritages, pour l’administration politique, ont des règlements tradition­nels d’une complication curieuse, et dont les rites, aujourd’hui privés de sens, dérivent évidemment d’un ordre d’idées supérieur. On en cite, comme témoignage, les tribus de Peaux-Rouges errant dans les vastes solitudes que l’on suppose avoir vu jadis les établissements des Alléghaniens \footnote{Prichard, \emph{Histoire naturelle de l’homme}, t. II, p. 78.}. Il est d’autres peuples qui possèdent des procédés de fabrication dont ils ne peuvent être les inventeurs : tels les naturels des îles Mariannes. Ils les conservent sans réflexion, et les mettent en usage, pour ainsi dire, machinalement.\par
Il y a donc lieu d’y regarder de près lorsque, voyant une nation dans l’état de barbarie, on se sent porté à conclure qu’elle y a toujours été. Pour ne commettre aucune erreur, tenons compte de plusieurs circonstances.\par
Il y a des peuples qui, saisis par l’activité d’un race parente, s’y soumettent à peu près, en acceptent certaines conséquences, en retiennent certains procédés ; puis, lors­que la race dominatrice vient à disparaître, soit par expulsion, soit par immersion complète dans le sein des vaincus, ceux-ci laissent périr la culture presque entière, les principes surtout, et n’en gardent que le peu qu’ils en ont pu comprendre. Ce fait ne peut d’ailleurs arriver qu’entre des nations alliées par le sang. Ainsi ont agi les Assyriens envers les créations chaldéennes ; les Grecs syriens et égyptiens, vis-à-vis des Grecs d’Europe ; les Ibères, les Celtes, les Illyriens, à l’encontre des idées romaines. Si donc les Cherokees, les Catawhas, les Muskhogees, les Séminoles, les Natchez, etc., ont gardé une certaine empreinte de l’intelligence alléghanienne, je n’en conclurai pas qu’ils sont les descendants directs et purs de la partie initiatrice de la race, ce qui entraî­nerait la conséquence qu’une race peut avoir été civilisée et ne l’être plus : je dirai que, si quelqu’une de ces tribus tient encore ethniquement à l’ancien type dominateur, c’est par un lien indirect et très bâtard, sans quoi les Cherokees ne seraient jamais tombés dans la barbarie, et, quant aux autres peuplades moins bien douées, elles ne me repré­sentent que le fond de la population étrangère, conquise, vaincue, agglomérée de force, sur laquelle reposait jadis l’état social. Dès lors, il n’est pas étonnant que ces détritus sociaux aient conservé, sans les comprendre, des habitudes, des lois, des rites combinés par plus habile qu’eux, et dont ils n’ont jamais su la portée et le secret, n’y devinant rien de plus qu’un objet de superstitieux respect. Ce raisonnement s’applique à la perpé­tuité des débris d’arts mécaniques. Les procédés qu’on y admire peuvent provenir primitivement d’une race d’élite depuis longtemps disparue. Quelquefois aussi la source en remonte plus loin. Ainsi, pour ce qui concerne l’exploitation des mines chez les Ibères, les Aquitains et les Bretons des Îles Cassitérides, le secret de cette science était dans la haute Asie, d’où les ancêtres des populations occidentales l’avaient jadis apporté dans leur émigration.\par
Les habitants des Carolines sont les insulaires à peu près les plus intéressants de la Polynésie. Leurs métiers à tisser, leurs barques sculptées, leur goût pour la navigation et le commerce tracent entre eux et les nègres pélagiens une ligne profonde de démar­cation. L’on découvre sans peine d’où leur viennent leurs talents. Ils les doivent au sang malais infusé dans leurs veines, et comme, en même temps, ce sang est loin d’être pur, les dons ethniques n’ont pu que se conserver parmi eux sans fructifier et en se dégradant.\par
Ainsi, de ce que chez un peuple barbare il existe des traces de civilisation, il n’est pas prouvé par là que ce peuple ait jamais été civilisé. Il a vécu sous la domination d’une tribu parente et supérieure, ou bien, se trouvant dans son voisinage, il a humble­ment et faiblement profité de ses leçons. Les races aujourd’hui sauvages l’ont toujours été, et, à raisonner par analogie, on est tout à fait en droit de conclure qu’elles continueront à l’être jusqu’au jour où elles disparaîtront.\par
Ce résultat est inévitable aussitôt que deux types, entre lesquels il n’existe aucune parenté, se trouvent dans un contact actif, et je n’en connais pas de meilleure démons­tration que le sort des familles polynésiennes et américaines. Il est donc établi, par les raisonnements qui précèdent :\par
1° Que les tribus actuellement sauvages l’ont toujours été, quel que soit le milieu supérieur qu’elles aient pu traverser, et qu’elles le seront toujours ; 2° que, pour qu’une nation sauvage puisse même supporter le séjour dans un milieu civilisé, il faut que la nation qui crée ce milieu soit un rameau plus noble de la même race ; 3° que la même circonstance est encore nécessaire pour que des civilisations diverses puissent non pas se confondre, ce qui n’arrive jamais, seulement se modifier fortement l’une par l’autre, se faire de riches emprunts réciproques, donner naissance à d’autres civilisations composées de leurs éléments ; 4° que les civilisations issues de races complètement étrangères l’une à l’autre ne peuvent que se toucher à la surface, ne se pénètrent jamais et s’excluent toujours. Comme ce dernier point n’a pas été suffisamment éclairci, je vais y insister.\par
Des conflits ont mis en présence la civilisation persane avec la civilisation grecque, l’égyptienne avec la grecque et la romaine, la romaine avec la grecque ; puis la civilisa­tion moderne de l’Europe avec toutes celles qui existent aujourd’hui dans le monde, et notamment la civilisation arabe.\par
Les rapports de l’intelligence grecque avec la culture persane étaient aussi multipliés que forcés. D’abord, une grande partie de la population hellénique, et la plus riche, sinon la plus indépendante, était concentrée dans ces villes du littoral syrien, dans ces colonies de l’Asie Mineure et du Pont, qui, très promptement réunies aux États du grand roi, vécurent sous la surveillance des satrapes, en conservant, jusqu’à un certain point, leur isonomie. La Grèce continentale et libre entretenait, de son côté, des rapports très intimes avec la côte d’Asie.\par
Les civilisations des deux pays vinrent-elles à se confondre ? On sait que non. Les Grecs traitaient leurs puissants antagonistes de barbares et probablement ceux-ci le leur rendaient bien. Les mœurs politiques, la forme des gouvernements, la direction donnée aux arts, la portée et le sens intime du culte public, les mœurs privées de nations entremêlées sur tant de points demeurèrent pourtant distinctes. À Ecbatane, on ne comprenait qu’une autorité unique, héréditaire, limitée par certaines prescriptions traditionnelles, absolue dans le reste. Dans l’Hellade, le pouvoir était subdivisé en une foule de petites souverainetés. Le gouvernement, aristocratique chez les uns, démocra­tique chez les autres, monarchique chez ceux-ci, tyrannique chez ceux-là, affichait à Sparte, à Athènes, à Sicyone, en Macédoine, la plus étrange bigarrure. Chez les Perses, le culte de l’État, beaucoup plus rapproché de l’émanatisme primitif, montrait la même tendance à l’unité que le gouvernement, et surtout avait une portée morale et métaphysique qui ne manquait pas de profondeur. Chez les Grecs, le symbolisme, ne se prenant qu’aux apparences variées de la nature, se contentait de glorifier les formes. La religion abandonnait aux lois civiles le soin de commander à la conscience, et du moment qu’étaient parachevés les rites voulus, les honneurs rendus au dieu ou au héros topique, la foi avait rempli sa mission. Puis ces rites, ces honneurs, ces dieux et ces héros changeaient à chaque demi-lieue. Au cas où, dans quelques sanctuaires, comme à Olympie par exemple, ou à Dodone, on voudrait reconnaître, non plus l’adoration d’une des forces ou d’un des éléments de la nature, mais celle du principe cosmique lui-même, cette sorte d’unité ne ferait que rendre le fractionnement plus remarquable, comme n’étant pratiquée que dans des lieux isolés. D’ailleurs l’oracle Dodonéen, le Jupiter d’Olympie étaient des cultes étrangers.\par
Pour les usages, il n’est pas besoin de faire ressortir à quel point ils différaient de ceux de la Perse. C’était s’exposer au mépris public, lorsqu’on était jeune, riche, volup­tueux et cosmopolite, que de vouloir imiter les façons de vivre de rivaux bien autrement luxueux et raffinés que les Hellènes. Ainsi, jusqu’au temps d’Alexandre, c’est-à-dire, pendant la belle et grande période de la puissance grecque, pendant la période féconde et glorieuse, la Perse, malgré toute sa prépondérance, ne put convertir la Grèce à sa civilisation.\par
Avec Alexandre, ce fait reçut une confirmation singulière. En voyant l’Hellade conquérir l’empire de Darius, on crut, sans doute, un moment, que l’Asie allait devenir grecque, et d’autant mieux, que le vainqueur s’était permis, dans une nuit d’égarement, contre les monuments du pays, des actes d’une agression tellement violente qu’elle semblait témoigner d’autant de mépris que de haine. Mais l’incendiaire de Persépolis changea bientôt d’avis, et si complètement que l’on put deviner son projet de se substituer purement et simplement à la dynastie des Achéménides et de gouverner comme son prédécesseur ou comme le grand Xerxès, avec la Grèce de plus dans ses États. De cette façon, la sociabilité persane aurait absorbé celle des Hellènes.\par
 Cependant, malgré toute l’autorité d’Alexandre, rien de semblable n’arriva. Ses géné­raux, ses soldats ne s’accommodèrent pas de le voir revêtir la robe longue et flottante, ceindre la mitre, s’entourer d’eunuques et renier son pays. Il mourut. Quelques-uns de ses successeurs continuèrent son système. Ils furent pourtant forcés de le mitiger, et pourquoi encore purent-ils établir ce moyen terme qui devint l’état normal de la côte asiatique et des hellénisants d’Égypte ? Parce que leurs sujets se composèrent d’une population bigarrée de Grecs, de Syriens, d’Arabes, qui n’avait nul motif pour accepter autre chose qu’un compromis en fait de culture. Mais là où les races restèrent distinctes, point de transaction. Chaque pays garda ses mœurs nationales.\par
De même encore, jusqu’aux derniers jours de l’empire romain, la civilisation métisse qui régnait dans tout l’Orient, y compris alors la Grèce continentale, était devenue beaucoup plus asiatique que grecque, parce que les masses tenaient beaucoup plus du premier sang que du second. L’intelligence semblait, il est vrai, se piquer de formes helléniques. Il n’est cependant pas malaisé de découvrir, dans la pensée de ces temps et de ces pays, un fond oriental qui vivifie tout ce qu’a fait l’école d’Alexandrie, comme les doctrines unitaires des jurisconsultes gréco-syriens. Ainsi la proportion, quant à la quantité respective du sang, est gardée : la prépondérance appartient à la part la plus abondante.\par
Avant de terminer ce parallèle, qui s’applique au contact de toutes les civilisations, quelques mots seulement sur la situation de la culture arabe vis-à-vis de la nôtre.\par
Quant à la répulsion réciproque, il n’y a pas à en douter. Nos pères du moyen âge ont pu admirer de près les merveilles de l’État musulman, lorsqu’ils ne se refusaient pas à envoyer leurs étudiants dans les écoles de Cordoue. Cependant rien d’arabe n’est resté en Europe hors des pays qui ont gardé quelque peu de sang ismaélite, et l’Inde brahmanique ne s’est pas montrée de meilleure composition que nous. Comme nous, soumise à des maîtres mahométans, elle a résisté avec succès à leurs efforts.\par
Aujourd’hui, c’est notre tour d’agir sur les débris de la civilisation arabe. Nous les balayons, nous les détruisons : nous ne réussissons pas à les transformer, et, pourtant, cette civilisation n’est pas elle-même originale, et devrait dès lors moins résister. La nation arabe, si faible de nombre, n’a fait notoirement que s’assimiler des lambeaux des races soumises par son sabre. Ainsi les Musulmans, population extrêmement mélangée, ne possèdent pas autre chose qu’une civilisation de ce même caractère métis dont il est facile de retrouver tous les éléments. Le noyau des vainqueurs, on le sait, n’était pas, avant Mahomet, un peuple nouveau ni inconnu. Ses traditions lui étaient communes avec les familles chamites et sémites d’où il tirait son origine. Il s’était frotté aux Phéniciens comme aux Juifs. Il avait dans les veines du sang des uns et des autres, et leur avait servi de courtier pour le commerce de la mer Rouge, de la côte orientale d’Afrique et de l’Inde. Auprès des Perses et des Romains, il avait joué le même rôle. Plusieurs de ses tribus avaient pris part à la vie politique de la Perse sous les Arsacides et les fils de Sassan, tandis que tel de ses princes, comme Odénat, s’instituait César, que telle de ses filles, comme Zénobie, fille d’Amrou, souveraine de Palmyre, se couvrait d’une gloire toute romaine, et que tel de ses aventuriers, comme Philippe, put même s’élever jusqu’à revêtir la pourpre impériale. Cette nation bâtarde n’avait donc jamais cessé, dès l’antiquité la plus haute, d’entretenir des relations suivies avec les sociétés puissantes qui l’avoisinaient. Elle avait pris part à leurs travaux et, semblable à un corps moitié plongé dans l’eau, moitié exposé au soleil, elle tenait, tout à la fois, d’une culture avancée et de la barbarie.\par
Mahomet inventa la religion la plus conforme aux idées de son peuple, où l’idolâtrie trouvait de nombreux adeptes, mais où le christianisme, dépravé par les hérétiques et les judaïsants, ne faisait guère moins de prosélytes. Le thème religieux du prophète koréischite fut une combinaison telle, que l’accord entre la loi de Moïse et la foi chrétienne, ce problème si inquiétant pour les premiers catholiques et toujours assez présent à la conscience des populations orientales, s’y trouva plus balancé que dans les doctrines de l’Église. C’était déjà un appât d’une saveur séduisante, et du reste, toute nouveauté théologique avait chance de gagner des croyants parmi les Syriens et les Égyptiens. Pour couronner l’œuvre, la religion nouvelle se présentait le sabre à la main, autre garantie de succès chez des masses sans lien commun, et pénétrées du sentiment de leur impuissance.\par
C’est ainsi que l’islamisme sortit de ses déserts. Arrogant, peu inventeur, et déjà, d’avance, conquis, aux deux tiers, à la civilisation gréco-asiatique, à mesure qu’il avan­çait il trouvait, sur les deux plages de l’est et du sud de la Méditerranée, toutes ses recrues saturées d’avance de cette combinaison compliquée. Il s’en imprégna davantage, Depuis Bagdad jusqu’à Montpellier, il étendit son culte emprunté à l’Église, à la Synagogue, aux traditions défigurées de l’Hedjaz et de l’Yémen, ses lois persanes et romaines, sa science gréco-syrienne \footnote{W. de Humboldt, \emph{Ueber die Kawie-Sprache, Einleitung}, p. CCLXIII.} et égyptienne, son administration, dès le premier jour, tolérante comme il convient, lorsque rien d’unitaire ne réside dans un corps d’État. On a eu grand tort de s’étonner des rapides progrès des Musulmans dans le raffinement des mœurs. Le gros de ce peuple avait simplement changé d’habits, et on l’a méconnu quand il s’est mis à jouer le rôle d’apôtre sur la scène du monde où, depuis longtemps, on ne le remarquait plus sous ses noms anciens. Il faut tenir compte encore d’un fait capital. Dans cette agrégation de familles si diverses, chacun apportait sans doute sa quote-part à la prospérité commune. Qui, pourtant, avait donné l’impulsion, qui soutint l’élan tant qu’on le vit durer, ce qui ne fut pas long ? Uniquement, le petit noyau de tribus arabes sorties de l’intérieur de la péninsule, et qui fournirent non pas des savants, mais des fanatiques, des soldats, des vainqueurs et des maîtres.\par
La civilisation arabe ne fut pas autre chose que la civilisation gréco-syrienne, rajeunie, ravivée par le souffle d’un génie assez court, mais plus neuf, et altérée par un mélange persan de plus. Ainsi faite, disposée à beaucoup de concessions, elle ne s’accorde cependant avec aucune formule sociale sortie d’autres origines que les sien­nes ; non, pas plus que la culture grecque ne s’était accordée avec la romaine, parente si proche et qui resta renfermée tant de siècles dans les limites du même empire. C’est là ce que je voulais dire sur l’impossibilité des civilisations possédées par des groupes ethniques étrangers l’un à l’autre, de se confondre jamais.\par
Quand l’histoire établit si nettement cet irréconciliable antagonisme entre les races et leurs modes de culture, il est bien évident que la dissemblance et l’inégalité résident au fond de ces répugnances constitutives, et du moment que l’Européen ne peut pas espérer de civiliser le nègre, et qu’il ne réussit à transmettre au mulâtre qu’un fragment de ses aptitudes ; que ce mulâtre, à son tour, uni au sang des blancs, ne créera pas encore des individus parfaitement aptes à comprendre quelque chose de mieux qu’une culture métisse d’un degré plus avancé vers les idées de la race blanche, je suis autorisé à établir l’inégalité des intelligences chez les différentes races.\par
Je répète encore ici qu’il ne s’agit nullement de retomber dans une méthode malheu­reusement trop chère aux ethnologistes, et, pour le moins, ridicule. Je ne discute pas, comme eux, sur la valeur morale et intellectuelle des individus pris isolément.\par
Pour la valeur morale, je l’ai mise complètement hors de question quand j’ai constaté l’aptitude de toutes les familles humaines à reconnaître, dans un degré utile, les lumières du christianisme. Lorsqu’il s’agit du mérite intellectuel, je me refuse absolument à cette façon d’argumenter qui consiste à dire : Tout nègre est inepte \footnote{Le jugement le plus rigoureux peut-être qui ait été porté sur la variété mélanienne émane d’un des patriarches de la doctrine égalitaire. Voici comment Franklin définissait le nègre : « C’est un animal qui mange le plus possible et travaille le moins possible. »}, et ma principale raison pour m’en abstenir, c’est que je serais forcé de reconnaître, par compensation, que tout Européen est intelligent, et je me tiens à cent lieues d’un pareil paradoxe.\par
Je n’attendrai pas que les amis de l’égalité des races viennent me montrer tel passage de tel livre de missionnaire ou de navigateur, d’où il conte qu’un Yolof s’est montré charpentier vigoureux, qu’un Hottentot est devenu bon domestique, qu’un Cafre danse et joue du violon, et qu’un Bambara sait l’arithmétique.\par
J’admets, oui, j’admets, avant qu’on me le prouve, tout ce qu’on pourra raconter de merveilleux, dans ce genre, de la part des sauvages les plus abrutis. J’ai nié l’excessive stupidité, l’ineptie chronique, même chez les tribus le plus bas ravalées. Je vais même plus loin que mes adversaires, puisque je ne révoque pas en doute qu’un bon nombre de chefs nègres dépassent, par la force et l’abondance de leurs idées, par la puissance de combinaison de leur esprit, par l’intensité de leurs facultés actives, le niveau commun auquel nos paysans, voire même nos bourgeois convenablement instruits et doués, peuvent atteindre. Encore une fois, et cent fois, ce n’est pas sur le terrain étroit des individualités que je me place. Il me paraît trop indigne de la science de s’arrêter à de si futiles arguments. Si Mungo-Park ou Lander ont donné à quelque nègre un certificat d’intelligence, qui me répond qu’un autre voyageur, rencontrant le même phénix, n’aura pas fondé sur sa tête une conviction diamétralement opposée ? Laissons donc ces puérilités, et comparons, non pas les hommes, mais les groupes. C’est lorsqu’on aura bien reconnu de quoi ces derniers sont ou non capables, dans quelle limite s’exercent leurs facultés, à quelles hauteurs intellectuelles ils parviennent, et quelles autres nations les dominent depuis le commencement des temps historiques, que l’on sera, peut-être un jour, autorisé entrer dans le détail, à rechercher pourquoi les grandes individualités de telle race sont inférieures aux beaux génies de telle autre. Ensuite, comparant entre elles les puissances des hommes vulgaires de tous les types, on s’enquerra des côtés par où ces puissances s’égalent et de ceux par où elles se priment. Ce travail difficile et délicat ne pourra s’accomplir tant qu’on n’aura pas balancé de la manière la plus exacte, et, en quelque sorte, par des procédés mathématiques, la situation relative des races. Je ne sais même si jamais on obtiendra des résultats d’une clarté incontestable, et si, libre de ne plus prononcer uniquement sur des faits généraux, on se verra maître de serrer les nuances de si près que l’on puisse définir, reconnaître et classer les couches inférieures de chaque nation et les individualités passives. Dans ce cas, on prouvera sans peine que l’activité, l’énergie, l’intelligence des sujets les moins doués dans les races domina­trices, surpassent l’intelligence, l’énergie, l’activité des sujets correspondants produits par les autres groupes \footnote{Je n’hésite pas à considérer comme une marque spécifique, dénotant l’infériorité intellectuelle, le développement exagéré des instincts qui se remarque chez les races sauvages. Certains sens y acquièrent un développement qui ne s’ouvre qu’au détriment des facultés pensantes. Voir, à ce sujet, ce que dit M. Lesson des Papous, dans un mémoire inséré au 10\textsuperscript{e} volume des \emph{Annales des sciences naturelles}.}.\par
Voici donc l’humanité partagée en deux fractions très dissemblables, très inégales, ou, pour mieux dire, en une série de catégories subordonnées les unes aux autres, et où le degré d’intelligence marque le degré d’élévation.\par
Dans cette vaste hiérarchie, il est deux faits considérables agissant incessamment sur chaque série. Ces faits, causes éternelles du mouvement qui rapproche les races et tend à les confondre, sont, comme je l’ai déjà indiqué \footnote{Voir chapitre XI.} : la similitude approximative des principaux caractères physiques, et l’aptitude générale à exprimer les sensations et les idées par les modulations de la voix.\par
J’ai surabondamment parlé du premier de ces phénomènes en le renfermant dans ses limites vraies.\par
Je vais m’occuper, maintenant, du second et rechercher quels rapports existent entre la puissance ethnique et la valeur du langage : autrement dit, si les plus beaux idiomes appartiennent aux fortes races ; dans le cas contraire, comment l’anomalie peut s’expliquer.
\section[{I.15. Les langues, inégales entre elles, sont dans un rapport parfait avec le mérite relatif des races.}]{I.15. \\
Les langues, inégales entre elles, sont dans un rapport parfait avec le mérite relatif des races.}
\noindent S’il était possible que des peuples grossiers, placés au bas de l’échelle ethnique, ayant aussi peu marqué dans le développement mâle que dans l’action féminine de l’humanité, eussent cependant inventé des langages philosophiquement profonds, esthétiquement beaux et souples, riches d’expressions diverses et précises, de formes caractérisées et heureuses, également propres aux sublimités, aux grâces de la poésie, comme à la sévère précision de la politique et de la science, il est indubitable que ces peuples auraient été doués d’un génie bien inutile : celui d’inventer et de perfectionner un instrument sans emploi au milieu de facultés impuissantes.\par
Il faudrait croire alors que la nature a des caprices sans but, et avouer que certaines impasses de l’observation aboutissent non pas à l’inconnu, rencontre fréquente, non pas à l’indéchiffrable, mais tout simplement à l’absurde.\par
Le premier coup d’œil jeté sur la question semble favoriser cette solution fâcheuse. Car, en prenant les races dans leur état actuel, on est obligé de convenir que la perfec­tion des idiomes est bien loin d’être partout proportionnelle au degré de civilisation. À ne considérer que les langues de l’Europe moderne, elles sont inégales entre elles, et les plus belles, les plus riches n’appartiennent pas nécessairement aux peuples les plus avancés. Si on compare, en outre, ces langues à plusieurs de celles qui ont été répandues dans le monde, à différentes époques, on les voit sans exception rester bien en arrière.\par
Spectacle plus singulier, des groupes entiers de nations arrêtées à des degrés de culture plus que médiocre sont en possession de langages dont la valeur n’est pas niable. De sorte que le réseau des langues, composé de mailles de différents prix, sem­blerait jeté au hasard sur l’humanité la soie et l’or couvrant parfois de misérables êtres incultes et féroces ; la laine, le chanvre et le crin embarrassant des sociétés inspirées, savantes et sages. Heureusement, ce n’est là qu’une apparence et, en y appliquant la doctrine de la diversité des races, aidée du secours de l’histoire, on ne tarde pas à en avoir raison, de manière à fortifier encore les preuves données plus haut sur l’inégalité intellectuelle des types humains.\par
Les premiers philologues commirent une double erreur : la première, de supposer que, parallèlement à ce que racontent les Unitaires de l’identité d’origine de tous les groupes, toutes les langues se trouvent formées sur le même principe ; la seconde, d’assigner l’invention du langage à la pure influence des besoins matériels.\par
Pour les langues, le doute n’est même pas permis. Il y a diversité complète dans les modes de formation et, bien que les classifications proposées par la philologie puissent être encore susceptibles de révision, on ne saurait garder, une seule minute, l’idée que la famille altaïque, l’ariane, la sémitique ne procèdent pas de sources parfaitement étran­gères les unes aux autres. Tout y diffère. La lexicologie a, dans ces différents milieux linguistiques, des formes parfaitement caractérisées à part. La modulation de la voix y est spéciale : ici, se servant surtout des lèvres pour créer les sons ; là, les rendant par la contraction de la gorge ; dans un autre système, les produisant par l’émission nasale et comme du haut de la tête. La composition des parties du discours n’offre pas des marques moins distinctes, réunissant ou séparant les nuances de la pensée, et présen­tant, surtout dans les flexions des substantifs et dans la nature du verbe, les preuves les plus frappantes de la différence de logique et de sensibilité qui existe entre les catégories humaines. Que résulte-t-il de là ? C’est que, lorsque le philosophe s’efforçant de se rendre compte, par des conjectures purement abstraites, de l’origine des langages, débute dans ce travail par se mettre en présence de l’homme idéalement conçu, de l’homme dépourvu de tous caractères spéciaux de race, de \emph{l’homme} enfin, il commence par un véritable non-sens, et continue infailliblement de même. Il n’y a pas d’homme idéal, \emph{l’homme} n’existe pas, et si je suis persuadé qu’on ne le découvre nulle part, c’est surtout lorsqu’il s’agit de langage. Sur ce terrain, je connais le possesseur de la langue finnoise, celui du système arian ou des combinaisons sémitiques ; mais \emph{l’homme} absolu, je ne le connais pas. Ainsi, je ne puis pas raisonner d’après cette idée, que tel point de départ unique ait conduit l’humanité dans ses créations idiomatiques. Il y a eu plusieurs points de départ parce qu’il y avait plusieurs formes d’intelligence et de sensibilité \footnote{M. Guillaume de Humboldt, dans un de ses plus brillants opuscules, a exprimé, d’une manière admirable, la partie essentielle de cette vérité : « Partout, dit ce penseur de génie, « l’œuvre du temps s’unit dans les langages à l’œuvre de l’originalité nationale, et ce qui « caractérise les idiomes des hordes guerrières de l’Amérique et de l’Asie septentrionale, n’a « pas nécessairement appartenu aux races primitives de l’Inde et de la Grèce. Il n’est pas « possible d’attribuer une marche parfaite­ment pareille et, en quelque sorte, imposée par la « nature, au développement, soit d’une langue appartenant à une nation prise isolément, « soit d’une autre qui aura servi à plusieurs peuples. » (W. v. Humboldt’s, \emph{Ueber das entstehen der grammatischen Formen, und ibrer Einflussh auf die Ideenentwickelung}).}.\par
Passant maintenant à la seconde opinion, je ne crois pas moins à sa fausseté. Suivant cette doctrine, il n’y aurait eu développement que dans la mesure où il y aurait eu nécessité. Il en résulterait que les races mâles posséderaient un langage plus précis, plus abondant, plus riche que les races femelles, et comme, en outre, les besoins matériels s’adressent à des objets qui tombent sous les sens et se manifestent surtout par des actes, la lexicologie serait la partie principale des idiomes.\par
Le mécanisme grammatical et la syntaxe n’auraient jamais eu l’occasion de dépasser les limites des combinaisons les plus élémentaires et les plus simples. Un enchaîne­ment de sons bien ou mal liés suffit toujours pour exprimer un besoin, et le geste, commentaire facile, peut suppléer à ce que l’expression laisse d’obscur \footnote{W. de Humboldt, \emph{Ueber die Kawi-Sprache. Einl}.}, comme le savent bien les Chinois. Et ce n’est pas seulement la synthèse du langage qui serait demeurée dans l’enfance. Il aurait fallu subir un autre genre de pauvreté non moins sensible, en se passant d’harmonie, de nombre et de rythme. Qu’importe, en effet, le mérite mélodique là où il s’agit seulement d’obtenir un résultat positif ? Les langues auraient été l’assemblage irréfléchi, fortuit, de sons indifféremment appliqués.\par
Cette théorie dispose de quelques arguments. Le chinois, langue d’une race masculine, semble, d’abord, n’avoir été conçu que dans un but utilitaire. Le mot ne s’y est pas élevé au-dessus du son. Il est resté monosyllabe. Là, point de développements lexicologiques. Pas de racine donnant naissance à des familles de dérivés. Tous les mots sont racines, ils ne se modifient pas par eux-mêmes, mais entre eux, et suivant un mode très grossier de juxtaposition. Là se rencontre une simplicité grammaticale d’où il résulte une extrême uniformité dans le discours, et qui exclut, pour des intelligences habituées aux formes riches, variées, abondantes, aux intarissables combinaisons d’idio­mes plus heureux, jusqu’à l’idée même de la perfection esthétique. Il faut cependant ajouter que rien n’autorise à admettre que les Chinois eux-mêmes éprouvent cette dernière impression, et, par conséquent, puisque leur langage a un but de beauté pour ceux qui le parlent, puisqu’il est soumis à certaines règles propres à favoriser le développement mélodique des sons, s’il peut être taxé, au point de vue comparatif, d’atteindre à ces résultats moins bien que d’autres langues, on n’est pas en droit de méconnaître que, lui aussi, les poursuit. Dès lors, il y a dans les premiers éléments du chinois autre chose et plus qu’un simple amoncellement d’articulations utilitaires \footnote{Je serais porté à croire que la nature monosyllabique du chinois ne constitue pas un caractère linguistique spécifique, et, malgré ce que cette particularité offre de saillant, elle ne me paraît pas essentielle. Si cela était, le chinois serait une langue isolée et se rattacherait, tout au plus, aux idiomes qui peuvent offrir la même structure. On sait qu’il n’en est rien. Le chinois fait partie du système tatare ou finnois, qui possède des branches parfaitement polysyllabiques. Puis, dans des groupes de toute autre origine, on retrouve des spécimens de la même nature. Je n’insisterai pas trop sur l’othomi. Cet idiome mexicain, suivant du Ponceau, présente, à la vérité, les traces que je relève ici dans le chinois, et cependant, placé au milieu des dialectes américains, comme le chinois parmi les langues tatares, l’othomi n’en fait pas moins partie de leur réseau. (Voir Morton, \emph{An Inquiry into the distinctive characteristics of the aboriginal race of America}, Philadelphia, 1844; voir aussi Prescott, \emph{History of the conquest of Mejico}, t. III, p. 245.) Ce qui m’empêcherait d’attacher à ce fait toute l’importance qu’il semble comporter, c’est qu’on pourrait alléguer que les langues américaines, langues ultra-polysyllabiques, puisque, seules au monde avec l’euskara, elles poussent la faculté de combiner les sons et les idées jusqu’au polysynthétisme, seront peut-être un jour reconnues comme ne formant qu’un vaste rameau de la famille tatare, et qu’en conséquence l’argument que j’en tirerais se trouverait corroborer seulement ce que j’ai dit de la parenté du chinois avec les idiomes ambiants, parenté que ne dément, en aucune façon, la nature particulière de la langue du Céleste Empire. Je trouve donc un exemple plus concluant dans le copte, qu’on supposera difficilement allié au chinois. Là, également, toutes les syllabes sont des racines et des racines qui se modifient par de simples affixes tellement mobiles, que, même pour marquer les temps du verbe, la particule déterminante ne reste pas toujours annexée au mot. Par exemple : \emph{hôn} veut dire ordonner ; \emph{a-hôn}, il ordonna ; \emph{Moïse ordonna}, se dit : \emph{a Moyses hôn.} (Voir E. Meier’s. \emph{Hebraeisches Wurzelwœrterbuch}, in-8°; Mannheim, 1845.) Il me paraît donc que le monosyllabisme peut se présenter chez toutes les familles d’idiomes. C’est. une sorte d’infirmité déterminée par des accidents d’une nature encore inconnue, mais point un trait spécifique propre à séparer le langage qui en est revêtu du reste des langages humains, en lui constituant une individualité spéciale.}.\par
Néanmoins, je ne repousse pas l’idée d’attribuer aux races masculines une infériorité esthétique assez marquée \footnote{Gœthe a dit dans son roman de \emph{Wilhelm Meister} : « Peu d’Allemands et peut-être peu « d’hommes, dans les nations modernes, possèdent le sens d’un ensemble esthétique. Nous « ne savons louer et blâmer que par morceaux, nous ne sommes ravis que d’une façon « fragmentaire. »}, qui se reproduirait dans la construction de leurs idiomes. J’en trouve l’indice, non seulement dans le chinois et son indigence relative, mais encore dans le soin avec lequel certaines races modernes de l’Occident ont dépouillé le latin de ses plus belles facultés rythmiques, et le gothique de sa sonorité. Le faible mérite de nos langues actuelles, même des plus belles, comparées au sanscrit, au grec, au latin même, n’a pas besoin d’être démontré, et concorde parfaitement avec la médiocrité de notre civilisation et de celle du Céleste Empire, en matière d’art et de littérature. Cependant, tout en admettant que cette différence puisse servir, avec d’autres traits, à caractériser les langues des races masculines, comme il existe pourtant dans ces langues un sentiment, moindre sans doute, cependant puissant encore, de l’eurythmie, et une tendance réelle à créer et à maintenir des lois d’enchaînement entre les sons et des conditions particulières de formes et de classes pour les modifications parlées de la pensée, j’en conclus que, même au sein des idiomes des races masculines, le sentiment du beau et de la logique, l’étincelle intellectuelle se fait encore apercevoir et préside donc partout à l’origine des langages, aussi bien que le besoin matériel.\par
Je disais, tout à l’heure, que, si cette dernière cause avait pu régner seule, un fond d’articulations formées au hasard aurait suffi aux nécessités humaines, dans les premiers temps de l’existence de l’espèce. Il paraît établi que cette hypothèse n’est pas soutenable.\par
Les sons ne se sont pas appliqués fortuitement à des idées. Le choix en a été dirigé par la reconnaissance instinctive d’un certain rapport logique entre des bruits extérieurs recueillis par l’oreille de l’homme, et une idée que son gosier ou sa langue voulait rendre. Dans le dernier siècle, on avait été frappé de cette vérité. Par malheur, l’exagération étymologique, dont on usait alors, s’en empara, et l’on ne tarda pas à se heurter contre des résultats tellement absurdes, qu’une juste impopularité vint les frapper et en faire justice. Pendant longtemps, ce terrain, si follement exploité par ses premiers explora­teurs, a effrayé les bons esprits. Maintenant, on y revient, et, en profitant des sévères leçons de l’expérience pour se montrer prudent et retenu, on pourra y recueillir des observations très dignes d’être enregistrées. Sans pousser des remarques, vraies en elles-mêmes, jusqu’au domaine des chimères, on peut admettre, en effet, que le langage primitif a su, autant que possible, profiter des impressions de l’ouïe pour former quelques catégories de mots, et que, dans la création des autres, il a été guidé par le sentiment de rapports mystérieux entre certaines notions de nature abstraite et certains bruits particuliers. C’est ainsi, par exemple, que le son de l’\emph{i} semble propre à exprimer la dissolution ; celui du \emph{w}, le vague physique et moral, le vent, les vœux ; celui de l’\emph{m}, la condition de la maternité \footnote{W. de Humboldt, \emph{Ueber die Kawi-Sprache, Einl}., p .XCV.}. Cette doctrine, contenue dans de très prudentes limites, trouve assez fréquemment son application pour qu’on soit contraint de lui reconnaître quelque réalité. Mais, certes, on ne saurait en user avec trop de réserve, sous peine de s’aventurer dans des sentiers sans clarté, où le bon sens se fourvoie bientôt.\par
Ces indications, si faibles qu’elles soient, démontrent que le besoin matériel n’a pas seul présidé à la formation des langages, et que les hommes y ont mis en jeu leurs plus belles facultés. Ils n’ont pas appliqué arbitrairement les sons aux choses et aux idées. Ils n’ont procédé, en cette matière, qu’en vertu d’un ordre préétabli dons ils trouvaient en eux-mêmes la révélation. Dès lors, tel de ces premiers langages, si rude, si pauvre et si grossier qu’on se le représente, n’en contenait pas moins tous les éléments nécessai­res pour que ses rameaux futurs pussent se développer un jour dans un sens logique, raisonnable et nécessaire.\par
M. Guillaume de Humboldt a remarqué, avec sa perspicacité ordinaire, que chaque langue existe dans une grande indépendance de la volonté des hommes qui la parlent. Se nouant étroitement à leur état intellectuel, elle est, tout à fait, au-dessus de la puissance de leurs caprices, et il n’est pas en leur pouvoir de l’altérer arbitrairement, Des essais dans ce genre en fournissent de curieux témoignages.\par
Les tribus des Boschismans ont inventé un système d’altération de leur langage, destiné à le rendre inintelligible à tous ceux qui ne sont pas initiés au procédé modifica­teur. Quelques peuplades du Caucase pratiquent la même coutume. Malgré tous les efforts, le résultat obtenu ne dépasse pas la simple adjonction ou intercalation d’une syllabe subsidiaire au commencement, au milieu ou à la fin des mots. À part cet élément parasite, la langue est demeurée la même, aussi peu altérée dans le fond que dans les formes.\par
Une tentative plus complète a été relevée par M. Sylvestre de Sacy, à propos de la langue balaïbalan. Ce bizarre idiome avait été composé par les Soufis, à l’usage de leurs livres mystiques, et comme moyen d’entourer de plus de mystères les rêveries de leurs théologiens. Ils avaient inventé, au hasard, les mots qui leur paraissaient résonner le plus étrangement à l’oreille. Cependant, si cette prétendue langue n’appartenait à aucune souche, si le sens attribué aux vocables était entièrement factice, la valeur eurythmique des sons, la grammaire, la syntaxe, tout ce qui donne le caractère typique fut invinciblement le calque exact de l’arabe et du persan. Les Soufis produisirent donc un jargon sémitique et arian tout à la fois, un chiffre, et rien de plus. Les dévots confrères de Djelat-Eddin-Roumi n’avaient pas pu inventer une langue. Ce pouvoir, évidemment, n’a pas été donné à la créature \footnote{Un jargon semblable au balaïbalan est probablement cette langue nommée \emph{afnskoë} qui se parle entre les maquignons et colporteurs de la Grande-Russie, surtout dans le gouvernement de Wladimir. Il n’y a que les hommes qui s’en servent. Les racines sont étrangères au russe ; mais la grammaire est entièrement de cet idiome. (Voir Pott, \emph{Encyclopædie} Ersch und Gruber, \emph{Indogerman. Sprachstamm}, p. 110.)}.\par
J’en tire cette conséquence, que le fait du langage se trouve intimement lié à la forme de l’intelligence des races, et, dès sa première manifestation, a possédé, ne fût-ce qu’en germe, les moyens nécessaires de répercuter les traits divers de cette intelligence à ses différents degrés \footnote{Je ne résiste pas à la tentation de copier ici une admirable page de C. O. Müller où cet érudit, plein de sentiment et de tact, a précisé, d’une manière rare, la véritable nature du langage. « Notre temps, dit-il, a appris par l’étude des langues hindoues, et plus encore par celle des langues germaniques, que les idiomes obéissent à des lois aussi nécessaires que le font les êtres organiques eux-mêmes. Il a appris qu’entre les différents dialectes, qui, une fois séparés, se développent indépendamment l’un de l’autre, des rapports mystérieux continuent à subsister, au moyen desquels les sons et la liaison des sons se déterminent réciproquement. Il sait de plus, désormais, que la littérature et la science, tout en modérant et en contenant, il est vrai, le bel et riche développement de cette croissance, ne peuvent lui imposer aucune règle supérieure à celle que la nature, mère de toutes choses, lui a imposée dès le principe. Ce n’est pas que les langues, longtemps avant les époques de fantaisie et de mauvais goût, ne puissent succomber à des causes internes et externes de maladie et souffrir de profondes perturbations ; mais, aussi longtemps que la vie réside en elles, leur virtualité intime suffit à guérir leurs blessures, à réparer leurs maux, à réunir leurs membres lacérés, à rétablir une unité, une régularité suffisante, alors même que la beauté et la perfection de ces nobles plantes a déjà presque entièrement disparu. » (C. O. Müller, \emph{die Etrusker}, p. 65.)}.\par
Mais, là où l’intelligence des races a rencontré des impasses et éprouvé des lacunes, la langue en a eu aussi. C’est ce que démontrent le chinois, le sanscrit, le grec, le groupe sémitique. J’ai déjà relevé, pour le chinois, une tendance plus particulièrement utilitaire conforme à la voie où chemine l’esprit de la variété. La plantureuse abondance d’expressions philosophiques et ethnologiques du sanscrit, sa richesse et sa beauté eurythmiques sont encore parallèles au génie de la nation. Il en est de même dans le grec, tandis que le défaut de précision des idiomes parlés par les peuples sémites s’accorde parfaitement avec le naturel de ces familles.\par
Si, quittant les hauteurs un peu vaporeuses des âges reculés, nous descendons sur des collines historiques plus rapprochées de nos temps, nous assistons, cette fois, à la naissance même d’une multitude d’idiomes, et ce grand phénomène nous fait voir plus nettement encore avec quelle fidélité le génie ethnique se mire dans les langages.\par
Aussitôt qu’a lieu le mélange des peuples, les langues respectives subissent une révolution, tantôt lente, tantôt subite, toujours inévitable. Elles s’altèrent, et, au bout de peu de temps, meurent. L’idiome nouveau qui les remplace est un compromis entre les types disparus, et chaque race y apporte une part d’autant plus forte qu’elle a fourni plus d’individus à la société naissante \footnote{Pott, \emph{Encycl.} Ersch und Gruber, \emph{Indo-german. Sprachst.}, p. 74.}. C’est ainsi que, dans nos populations occidentales, depuis le XIII\textsuperscript{e} siècle, les dialectes germaniques ont dû céder, non pas devant le latin, mais devant le roman \footnote{Le mélange des idiomes, proportionnel au mélange des races dans une nation, avait déjà été observé lorsque la science philologique n’existait, pour ainsi dire, pas encore. J’en citerai le témoignage que voici : « On peut poser comme une règle constante qu’à proportion du « nombre des étrangers qui s’établiront dans un pays, les mots de la langue qu’ils parlent « entreront dans le langage de ce pays-là, et par degrés s’y naturaliseront, pour ainsi dire, « et deviendront aussi familiers aux habitants que s’ils étaient de leur cru. » (Kaempfer, \emph{Histoire du Japon}, in-fol., La Haye, 1729, liv. I\textsuperscript{er}, p. 73.)}, à mesure que renaquit la puissance gallo-romaine. Quant au celtique, il n’avait point reculé devant la civilisation italienne, c’est devant la colonisation qu’il avait fui, et encore peut-on dire avec vérité qu’il avait remporté en fin de compte, grâce au nombre de ceux qui le parlaient, plus qu’une demi-victoire puisqu’il lui avait été donné, quand la fusion des Galls, des Romains et des hommes du Nord s’était opérée définitivement, de préparer à la langue moderne sa syntaxe, d’éteindre en elle les accentuations rudes venues de la Germanie et les plus vives sonorités apportées de la Péninsule, et de faire triompher l’eurythmie assez terne qu’il possédait lui-même. Le développement graduel de notre français n’est que l’effet de ce travail latent, patient et sûr. Les causes qui ont dépouillé l’allemand moderne des formes assez éclatantes remarquées dans le gothique de l’évêque Ulphila, ne sont pas autres, non plus, que la présence d’une épaisse population kymrique sous le petit nombre d’éléments germaniques demeurés au delà du Rhin \footnote{Keferstein (\emph{Ansichten über die keltischen Alterthümer}, Halle, 1846-1851 ; \emph{Einleit.}, 1, XXXVIII) prouve que l’allemand n’est qu’une langue métisse composée de celtique et de gothique. Grimm exprime le même avis.}, après les grandes migrations qui suivirent le Ve siècle de notre ère.\par
Les mélanges de peuples présentant sur chaque point des caractères particuliers issus du quantum des éléments ethniques, les résultats linguistiques sont également nuancés. On peut poser en thèse générale qu’aucun idiome ne demeure pur après un contact intime avec un idiome différent ; que même, lorsque les principes respectifs offrent le plus de dissemblances, l’altération se fait au moins sentir dans la lexicologie ; que, si la langue parasite a quelque force, elle ne manque pas d’attaquer le mode d’eurythmie, et même les côtés les plus faibles du système grammatical, d’où il résulte que le langage est une des parties les plus délicates et les plus fragiles de l’individualité des peuples. On aura donc souvent le singulier spectacle d’une langue noble et très cultivée passant, par son union avec un idiome barbare, à une sorte de barbarie relative, se dépouillant par degrés de ses plus belles facultés, s’appauvrissant de mots, se desséchant de formes, et témoignant ainsi d’un irrésistible penchant à s’assimiler, de plus en plus, au compagnon de mérite inférieur que l’accouplement des races lui aura donné. C’est ce qui est arrivé au valaque et au rhétien, au kawi et au birman. L’un et l’autre de ces derniers idiomes sont imprégnés d’éléments sanscrits, et, malgré la noblesse de cette alliance, les juges compétents les déclarent inférieurs en mérite au delaware \footnote{W. de Humboldt, \emph{Ueber die Kawi-Sprache, Einl.}, p. XXXIV.}.\par
 Issue du tronc des Lenni-Lénapes, l’association de tribus qui parle ce dialecte vaut primitivement plus que les deux groupes jaunes remorqués par la civilisation hindoue, et si, malgré cette prérogative, elle est au-dessous d’eux, c’est que les Asiatiques en question vivent sous l’impression des inventions sociales d’une race noble, et profitent de ces mérites, tout en étant peu de chose par eux-mêmes. Le contact sanscrit a suffi pour les élever assez haut, tandis que les Lénapes, que rien de semblable n’a fécondés jamais, n’ont pu monter, en civilisation, au-dessus de la valeur qu’on leur voit. C’est ainsi, pour me servir d’une comparaison facile à apprécier, que les jeunes mulâtres élevés dans les collèges de Londres et de Paris, peuvent, tout en restant mulâtres et très mulâtres, présenter, sous certains rapports, une apparence de culture plus satisfaisante que tels habitants de l’Italie méridionale dont la valeur intime est incontestablement plus grande. Il faut donc, lorsqu’on rencontre un peuple sauvage en possession d’un idiome supérieur à celui de nations plus civilisées, distinguer soigneusement si la civilisation de ces dernières leur appartient en propre, ou si elle ne provient que d’une infiltration de sang étranger. Dans ce dernier cas, l’imperfection du langage primitif et l’abâtardissement du langage importé s’accordent parfaitement avec l’existence d’un certain degré de culture sociale \footnote{C’est cette différence de niveau qui, se marquant entre l’intelligence du conquérant et celle des peuples soumis, a donné cours, au début des nouveaux empires, à l’usage des \emph{langues sacrées.} On en a vu dans toutes les parties du monde. Les Égyptiens avaient la leur, les Incas du Pérou de même. Cette langue sacrée, objet d’un superstitieux respect, propriété exclusive des hautes classes et souvent du groupe sacerdotal, à l’exclusion de tous les autres, est toujours la preuve la plus forte que l’on puisse donner de l’existence d’une race étrangère dominant sur le sol où on la trouve.}.\par
J’ai dit ailleurs que, chaque civilisation ayant une portée particulière, il ne fallait pas s’étonner si le sens poétique et philosophique était plus développé chez les Hindous sanscrits et chez les Grecs que chez nous, tandis que l’esprit pratique, critique, érudit, distingue davantage nos sociétés. Pris en masse, nous sommes doués d’une vertu active plus énergique que les illustres dominateurs de l’Asie méridionale et de l’Hellade. En revanche, il nous faut leur céder le pas sur le terrain du beau, et il est, dès lors, naturel que nos idiomes tiennent l’humble rang de nos esprits. Un essor plus puissant vers les sphères idéales se reflète naturellement dans la parole dont les écrivains de l’Inde et de l’Ionie ont fait usage, de sorte que le langage, tout en étant, je le crois, je l’admets, un très bon critérium de l’élévation générale des races, l’est pourtant, d’une manière plus spéciale, de leur élévation esthétique, et il prend surtout ce caractère lorsqu’il s’applique à la comparaison des civilisations respectives.\par
Pour ne pas laisser ce point douteux, je me permettrai de discuter une opinion émise par M. le baron Guillaume de Humboldt, au sujet de la supériorité du mexicain sur le péruvien \footnote{M. de Humboldt, \emph{Ueber die Kawi-Sprache, Einl.}, XXXIV.}, supériorité évidente, dit-il, bien que la civilisation des Incas ait été fort au-dessus de celle des habitants de l’Anahuac.\par
Les mœurs des Péruviens se montraient, sans doute, plus douces, leurs idées religieuses aussi inoffensives qu’étaient féroces celles des sujets de Montézuma. Malgré tout cela, l’ensemble de leur état social était loin de présenter autant d’énergie, autant de variété. Tandis que leur despotisme, assez grossier, ne réalisait qu’une sorte de communisme hébétant, la civilisation aztèque avait essayé des formes de gouver­nement très raffinées. L’état militaire y était beaucoup plus vigoureux, et, bien que les deux empires ignorassent également l’usage de l’écriture, il semblerait que la poésie, l’histoire et la morale, fort cultivées au moment où apparut Cortez, auraient joué un plus grand rôle au Mexique qu’au Pérou, dont les institutions penchaient vers un épicuréisme nonchalant peu favorable aux travaux de l’intelligence. Il devient alors tout simple d’avoir à constater la supériorité du peuple le plus actif sur le peuple le plus modeste.\par
Au reste, l’opinion de M. Guillaume de Humboldt est, ici, conséquente à la manière dont il définit la civilisation. Sans renouveler la controverse, il m’était indispensable de ne pas laisser ce point dans l’ombre ; car, si deux civilisations avaient pu se développer jamais parallèlement à des langues en contradiction avec leurs mérites respectifs, il faudrait abandonner l’idée de toute solidarité entre la valeur des idiomes et celle des intelligences. Ce fait est impossible à concéder dans une mesure différente de ce que j’ai dit plus haut pour le sanscrit et le grec comparés à l’anglais, au français, à l’allemand.\par
D’ailleurs, en suivant cette voie, ce ne serait pas une médiocre difficulté que de déterminer pour les populations métisses les causes de l’état idiomatique où on les trouve. On ne possède pas toujours, sur la quotité des mélanges ou sur leur qualité, des lumières suffisantes pour pouvoir en examiner le travail organisateur. Cependant l’influence de ces causes premières persiste, et, si elle n’est pas démasquée, elle peut aisément conduire à des conclusions erronées. Précisément parce que le rapport de l’idiome à la race est assez étroit, il se conserve beaucoup plus longtemps que les peuples ne gardent leurs corps d’État. Il se fait reconnaître après que les peuples ont changé de nom. Seulement, s’altérant comme leur sang, il ne disparaît, il ne meurt qu’avec la dernière parcelle de leur nationalité \footnote{Une observation intéressante, c’est de voir, dans les langues issues d’une langue moyenne, certains dérivés se présenter sous une forme bien plus rapprochée de la racine primitive que le mot d’où, en général, on les suppose formés ou que celui qui, dans la langue la plus voisine, exprime la même idée. Ainsi FUREUR : all. \foreign{Wuth}, angl. \foreign{mad}, sanscrit \foreign{mada}; DÉSIR, comme expression de la passion : all. \foreign{Begierde}, franç. \foreign{rage}, sanscrit \foreign{raga} ; DEVOIR : all. \foreign{Pflicht}, angl. \foreign{Duty}, sanscrit \foreign{dutia.} (Voir Klaproth, \foreign{Asia polyglotta}, in-4°.) On pourrait induire de ce fait que quelques races, après avoir subi un certain nombre de mélanges, sont partiellement ramenées à une pureté plus grande, à une vigueur blanche plus prononcée que d’autres qui les ont devancées dans l’ordre des temps.}. Le grec moderne est dans ce cas ; mutilé autant que possible, dépouillé de la meilleure part de ses richesses gramma­ticales, troublé et souillé dans sa lexicologie, appauvri même, à ce qu’il semble, quant au nombre de ses sons, il n’en a pas moins conservé son empreinte originelle \footnote{La Grèce antique, qui possédait de nombreux dialectes, n’en avait cependant pas autant que celle du XVI\textsuperscript{e} siècle, lorsque Siméon Kavasila en comptait soixante et dix ; et, remarque à rattacher à ce qui va suivre, au XVIII\textsuperscript{e} siècle, on parlait le français dans toute 1’Hellade et surtout dans l’Attique. (Heilmayer, cité par Pott, \emph{Encycl. v, Erseh und Gruber, Indo-germanischer Sprachstamm}, p. 73.)}. C’est, en quelque sorte, dans l’univers intellectuel, ce qu’est, sur la terre, ce Parthénon si dégradé, qui, après avoir servi d’église aux popes, puis, devenu poudrière, avoir éclaté, en mille endroits de son fronton et de ses colonnes, sous les boulets vénitiens de Morosini, présente encore à l’admiration des siècles l’adorable modèle de la grâce sérieuse et de la majesté simple.\par
Il arrive aussi qu’une parfaite fidélité à la langue des aïeux n’est pas dans le caractère de toutes les races. C’est encore là une difficulté de plus quand on cherche à démêler, à l’aide de la philologie, soit l’origine, soit le mérite relatif des types humains. Non seulement il arrive aux idiomes de subir des altérations dont il n’est pas toujours facile de retrouver la cause ethnique ; il se rencontre encore des nations qui, pressées par le contact des langues étrangères, abandonnent la leur. C’est ce qui est advenu, après les conquêtes d’Alexandre, à la partie éclairée des populations de l’Asie occidentale, telles que les Cariens, les Cappadociens et les Arméniens, et c’est ce que j’ai signalé aussi pour nos Gaulois. Les uns et les autres ont cependant inculqué dans les langues victorieuses un principe étranger qui les a, à la fin, transfigurées à leur tour. Mais, tandis que ces peuples maintenaient encore, bien que d’une manière imparfaite, leur propre instrument intellectuel ; que d’autres, beaucoup plus tenaces, tels que les Basques, les Berbères de l’Atlas, les Ekkhilis de l’Arabie méridionale, parlent jusqu’à nos jours comme parlaient leurs plus anciens parents, il est des groupes, les juifs par exemple, qui semblent n’y avoir jamais tenu, et cette indifférence éclate dès les premiers pas de la migration des favoris de Dieu. Tharé, venant d’Ur des Chaldéens, n’avait certainement pas appris, dans le pays de sa parenté, la langue chananéenne qui devint nationale pour les enfants d’Israël. Ceux-ci s’étaient donc dépouillés de leur idiome natif pour en accepter un autre différent, et qui, subissant quelque peu, je le veux croire, l’influence des souvenirs premiers, devint, dans leur bouche, un dialecte particulier de cette langue très ancienne, mère de l’arabe le plus ancien, héritage légitime des tribus alliées, de fort près, aux Chamites noirs \footnote{Les Hébreux eux-mêmes ne nommaient pas leur langue l\emph{’hébreu} ; ils l’appelaient très justement \emph{la langue de Chanaan}, rendant ainsi hommage à la vérité. (Isaïe, 19, 18). Voir, à ce sujet, les observations de Rœdiger sur la \emph{Grammaire hébraïque} de Gésénius, 16\textsuperscript{e} édition, Leipzig, 1851, p. 7 et passim.}. Cette langue, les Juifs ne devaient pas s’y montrer plus fidèles qu’à la première. Au retour de la captivité, les bandes de Zorobabel l’avaient oubliée sur les bords des fleuves de Babylone, pendant leur séjour, pourtant bien court, de soixante et dix ans. Le patriotisme, fort contre l’exil, avait conservé sa chaleur : le reste avait été abandonné avec une bizarre facilité par ce peuple tout à la fois jaloux de lui-même et cosmopolite à l’excès. Dans Jérusalem reconstruite, la multitude reparut, parlant un jargon araméen ou chaldéen qui, d’ailleurs, n’était peut-être pas sans ressemblance avec l’idiome des pères d’Abraham.\par
Aux temps de Jésus-Christ, ce dialecte résistait avec peine à l’invasion d’un patois grec qui, de tous côtés, pénétrait l’intelligence juive. Ce n’était plus guère que sous ce nouveau costume, plus ou moins élégant, affichant plus ou moins de prétentions attiques, que les écrivains juifs d’alors produisaient leurs ouvrages. Les derniers livres canoniques de l’Ancien Testament, comme les écrits de Philon et de Josèphe, sont des œuvres hellénistiques.\par
Lorsque la destruction de la ville sainte eut dispersé la nation désormais déshéritée des bontés de l’Éternel, l’Orient ressaisit l’intelligence de ses fils. La culture hébraïque rompit avec Athènes comme avec Alexandrie, et la langue, les idées du Talmud les enseignements de l’école de Tibériade furent de nouveau sémitiques, quelquefois arabes et souvent chananéens, pour employer l’expression d’Isaïe. Je parle de la langue désormais sacrée, de celle des rabbins, de la religion, de celle dès lors considérée comme nationale. Mais pour le commerce de la vie, les Juifs usèrent des idiomes des pays où ils se trouvèrent transportés. Il est encore à noter que partout ces exilés se firent remarquer par leur accent particulier. Le langage qu’ils avaient adopté et appris dès la première enfance ne réussit jamais à assouplir leur organe vocal. Cette observation confirmerait ce que dit M. Guillaume de Humboldt d’un rapport si intime de la race avec la langue, qu’à son avis, les générations ne s’accoutument pas à bien prononcer les mots que ne savaient pas leurs ancêtres \footnote{C’est aussi le sentiment de M W. Edwards, \emph{Caractères physiques des races humaines}, p. 101 et passim.}.\par
Quoi qu’il en soit, voilà, dans les Juifs, une preuve remarquable de cette vérité, qu’on ne doit pas toujours, à première vue, établir une concordance exacte entre une race et la langue dont elle est en possession, attendu que cette langue peut ne pas lui appartenir originairement. Après les Juifs, je pourrais citer encore l’exemple des Tsiganes et de bien d’autres peuples \footnote{Il est encore un cas qui peut se présenter, c’est celui où une population parle deux langues. Dans les Grisons, presque tous les paysans de l’Engadine emploient avec une égale facilité le romanche dans leurs rapports entre compatriotes, l’allemand quand ils s’adressent à des étrangers. En Courlande, il est un district où les paysans, pour s’entretenir entre eux, se servent de l’esthonien, dialecte finnois. Avec toute autre personne, ils parlent letton. (Voir Pott, \emph{Encycl. Erseh und Gruber, Indo-germanischer Sprachstamm}, p. 104.)}.\par
On voit avec quelle prudence il convient d’user de l’affinité et même de la similitude des langues pour conclure à l’identité des races, puisque, non seulement des nations nombreuses n’emploient que des langages altérés dont les principaux éléments n’ont pas été fournis par elles, témoin la plupart des populations de l’Asie occidentale et presque toutes celles de l’Europe méridionale, mais encore que plusieurs autres en ont adopté de complètement étrangers, à la confection desquels elles n’ont presque pas contribué. Ce dernier fait est sans doute plus rare. Il se présente même comme une anomalie. Il suffit cependant qu’il puisse avoir lieu pour qu’on ait à se tenir en garde contre un genre de preuves qui souffre de telles déviations. Toutefois, puisque le fait est anormal, puisqu’il ne se rencontre pas aussi fréquemment que son opposite, c’est-à-dire la conservation séculaire d’idiomes nationaux par de très faibles groupes humains ; puisque l’on voit aussi combien les langues ressemblent au génie particulier du peuple qui les crée, et combien elles s’altèrent justement dans la mesure où le sang de ce peuple se modifie ; puisque le rôle qu’elles jouent dans la formation de leurs dérivées est proportionnel à l’influence numérique de la race qui les apporte dans le nouveau mélange, tout donne le droit de conclure qu’un peuple ne saurait avoir une langue valant mieux que lui-même, à moins de raisons spéciales. Comme on ne saurait trop insister sur ce point, je vais en faire ressortir l’évidence par une nouvelle espèce de démonstration.\par
On a vu déjà que, dans une nation d’essence composite, la civilisation n’existe pas pour toutes les couches successives. En même temps que les anciennes causes ethniques poursuivent leur travail dans le bas de l’échelle sociale, elles n’y admettent, elles n’y laissent pénétrer que faiblement, et d’une façon tout à fait transitoire, les influences du génie national dirigeant. J’appliquais naguère ce principe à la France, et je disais que, sur ses 36 millions d’habitants, il y en avait, au moins, 20 qui ne prenaient qu’une part forcée, passive, temporaire, au développement civilisateur de l’Europe moderne. Excepté la Grande-Bretagne, servie par une plus grande unité dans ses types, conséquence de son isolement insulaire, cette triste proportion est plus considérable encore sur le reste du continent. Puisqu’une fois déjà j’ai choisi la France pour exemple, je m’y tiens, et crois trouver que mon opinion sur l’état ethnique de ce pays, et celle que je viens d’exprimer à l’instant pour toutes les races en général, quant à la parfaite concordance du type et de la langue, s’y confirment l’une l’autre d’une manière frappante.\par
Nous savons peu, ou, pour mieux dire, nous ne savons pas, preuves en main, par quelles phases le celtique et le latin rustique \footnote{La route n’était pas si longue du latin rustique, \foreign{lingua rustica Romanorum, lingua romana}, du roman, en un mot, à la corruption, que de la langue élégante, dont les formes précises et cultivées présentaient plus de résistance. Il est aussi à remarquer que, chaque légionnaire étranger apportant dans les colonies de la Gaule le patois de ses provinces, l’avènement d’un dialecte général et mitoyen était hâté, non seulement par les Celtes, mais par les émigrants eux-mêmes.} ont d’abord dû passer avant de se rapprocher et de finir par se confondre. Saint Jérôme et son contemporain Sulpice Sévère nous apprennent pourtant, le premier dans ses \emph{Commentaires} sur l’Épître de saint Paul aux Galates, le second dans son \emph{Dialogue sur les mérites des moines d’Orient}, que, de leur temps, on parlait au moins deux langues vulgaires dans la Gaule : le \emph{celtique}, conservé si pur sur les bords du Rhin, que le langage des Gallo-Grecs, éloignés de la mère patrie depuis six cents ans, y ressemblait de tous points ; puis ce qu’on appelait \emph{le gaulois}, et qui, de l’avis d’un commentateur, ne pouvait être qu’un romain déjà altéré. Mais ce gaulois, différent de ce qui se parlait à Trèves, n’était pas non plus la langue de l’ouest ni celle de l’Aquitaine. Ce dialecte du IV\textsuperscript{e} siècle, probablement partagé lui-même en deux grandes divisions, ne trouve donc de place que dans le centre et le midi de la France actuelle. C’est à cette source commune qu’il faut reporter les courants, différemment latinisés, qui ont formé plus tard, avec d’autres mélanges, et dans des proportions diverses, la langue d’oïl et le roman proprement dit. Je parlerai d’abord de ce dernier.\par
Pour lui donner naissance, il ne s’agissait que de créer une altération assez facile de la terminologie latine, modifiée par un certain nombre d’idées grammaticales emprun­tées au celtique et à d’autres langues jadis inconnues dans l’ouest de l’Europe. Les colonies impériales avaient apporté bon nombre d’éléments italiens, africains, asiatiques. Les invasions bourguignonnes, et surtout les gothiques, fournirent un nouvel apport doué d’une grande vivacité d’harmonie, de sons larges et brillants. Les irruptions sarrasines en renforcèrent la puissance. De sorte que le roman, se distinguant tout à fait du gaulois, quant à son mode d’eurythmie, revêtit bientôt un cachet très spécial. Sans doute, nous ne le trouvons pas, dans la formule de serment des fils de Louis le Débonnaire, arrivé à sa perfection, comme plus tard, dans les poésies de Raimbaud de Vachères ou de Bertrand de Born. Cependant on le reconnaît déjà pour ce qu’il est, ses caractères principaux lui sont acquis, sa direction lui est nettement indiquée. C’était bien, dès lors, dans ses différents dialectes limousin, provençal, auvergnat, la langue d’une population aussi mélangée d’origine qu’il y en ait jamais eu au monde. Cette langue souple, fine, spirituelle, railleuse, pleine d’éclat, mais sans profondeur, sans philosophie, clinquant et non pas or, n’avait pu, dans aucune des mines opulentes qui lui avaient été ouvertes, que glaner à la surface. Elle était sans principes sérieux : elle devait rester un instrument d’universelle indifférence, partant, de scepticisme et de moquerie. Elle ne manqua pas à cette vocation. La race ne tenait à rien qu’aux plaisirs et aux brillantes apparences. Brave à l’excès, joyeuse avec autant d’emportement, passionnée sans sujet et vive sans conviction, elle eut un instrument tout propre à servir ses tendances, et qui d’ailleurs, objet de l’admiration du Dante, ne servit jamais, en poésie, qu’à rimer des satires, des chansons d’amour, des défis de guerre, et, en religion, à soutenir des hérésies comme celle des Albigeois, manichéisme licencieux, dénué de valeur, même littéraire, dont un auteur anglais, peu catholique, félicite la papauté d’avoir délivré le moyen âge \footnote{Macaulay, \emph{History of England}, t. I, p. 18, éd. de Paris. Les Albigeois sont l’objet d’une prédilection toute spéciale de la part des écrivains révolutionnaires, surtout en Allemagne (voir à ce sujet le poème de Lenau, \emph{die Albigenser}). Cependant les sectaires du Languedoc se recrutaient surtout dans les classes chevaleresques et chez les dignitaires ecclésiastiques. Mais leurs doctrines étaient antisociales : c’est de quoi leur faire beaucoup pardonner.}. Telle fut, jadis, la langue romane, telle on la trouve encore aujourd’hui. Elle est jolie, non pas belle, et il suffit de l’examiner pour voir combien peu elle est apte à servir une grande civilisation.\par
La langue d’oil se forma-t-elle dans des conditions semblables ? L’examen va prouver que non, et, de quelque manière que la fusion des éléments celtique, latin, germanique, se soit faite, ce qu’on ne peut parfaitement apprécier \footnote{La préface de la \emph{Chanson de Roland}, par M. Génin, contient, à ce sujet, des observations assez curieuses. (\emph{Chanson de Roland}, in-8°, Imprimerie nationale, Paris, 1851.)}, faute de monu­ments appartenant à la période de création, il est du moins certain qu’elle naissait d’un antagonisme décidé entre trois idiomes différents, et que le produit représenté par elle devait être pourvu d’un caractère et d’un fond d’énergie tout à fait étranger aux nombreux compromis, aux transactions assez molles d’où était sorti le roman. Cette langue d’oïl fut, à un moment de sa vie, assez rapprochée des principes germaniques. On y découvre, dans les restes écrits parvenus jusqu’à nous, un des meilleurs caractères des langues arianes : c’est le pouvoir, limité il est vrai, moins grand que dans le sanscrit, le grec et l’allemand, mais considérable encore, de former des mots composés. On y reconnaît, pour les noms, des flexions indiquées par des affixes, et, comme consé­quence, une facilité d’inversion perdue pour nous, et dont la langue française du XVI\textsuperscript{e} siècle, ayant imparfaitement hérité, ne jouissait qu’aux dépens de la clarté du discours. Sa lexicologie contenait également de nombreux éléments apportés par la race franque \footnote{Consulter le \emph{Fœmina}, cité par Hickes dans son \emph{Thesaurus litteraturæ septentrionalis} et par l’\emph{Histoire littéraire de France}, t. XVII, p. 633.}. Ainsi, la langue d’oïl débutait par être presque autant germanique que gauloise, et le celtique y apparaissait au second plan, comme décidant peut-être des raisons mélodiques du langage. Le plus bel éloge qu’on puisse en faire se trouve dans la réussite de l’ingénieux essai de M. Littré, qui a pu traduire littéralement et vers pour vers, en français du XIII\textsuperscript{e} siècle, le premier chant de l’\emph{Iliade}, tour de force impraticable dans notre français d’aujourd’hui \footnote{\emph{Revue des Deux Mondes}.}.\par
 Cette langue ainsi dessinée appartenait évidemment à un peuple qui faisait grandement contraste avec les habitants du sud de la Gaule. Plus profondément attaché aux idées catholiques, portant dans la politique des notions vives d’indépendance, de liberté, de dignité, et dans toutes ses institutions une recherche très caractérisée de l’utile, la littérature populaire de cette race eut pour mission de recueillir, non pas les fantaisies de l’esprit ou du cœur, les boutades d’un scepticisme universel, mais bien les annales nationales, telles qu’on les comprenait alors et qu’on les jugeait vraies. Nous devons à cette glorieuse disposition de la nation et de la langue les grandes composi­tions rimées, surtout Garin le Loherain, témoignage, renié depuis, de la prédominance du Nord. Malheureusement, comme les compilateurs de ces traditions, et même leurs premiers auteurs, avaient, avant tout, l’intention de conserver des faits historiques ou de servir des passions positives, la poésie proprement dite, l’amour de la forme et la recherche du beau ne tiennent pas toujours assez de place dans leurs grands récits. La littérature de la langue d’oïl eut, avant tout, la prétention d’être utilitaire. C’est ainsi que les races, le langage et les écrits se trouvent ici en accord parfait.\par
Mais il était naturel que l’élément germanique, beaucoup moins abondant que le fond gaulois et que la mixture romaine, perdît peu à peu du terrain dans le sang. En même temps, il en perdit dans la langue et, d’une part, le celtique, d’autre part, le latin gagnèrent à mesure qu’il se retira. Cette belle et forte langue, dont nous ne connaissons guère que l’apogée, et qui se serait encore perfectionnée en suivant sa voie, commença à déchoir et à se corrompre vers la fin du XIII\textsuperscript{e} siècle. Au XV\textsuperscript{e}, ce n’était plus qu’un patois d’où les éléments germaniques avaient complètement disparu. Ce qui restait de ce trésor dépensé, n’apparaissant désormais que comme une anomalie au milieu des progrès du celtique et du latin, n’offrait plus qu’un aspect illogique et barbare. Au XVI\textsuperscript{e} siècle, le retour des études classiques trouva le français dans ce délabrement, et voulut s’en emparer pour le perfectionner dans le sens des langues anciennes. Tel fut le but avoué des littérateurs de cette belle époque. Ils ne réussirent guère, et le XVI\textsuperscript{e} siècle, plus sage, ou s’apercevant qu’il ne pouvait maîtriser la puissance irrésistible des choses, ne s’occupa qu’à améliorer, par elle-même, une langue qui se précipitait chaque jour davantage vers les formes les plus naturelles à la race prédominante, c’est-à-dire vers celles qui avaient autrefois constitué la vie grammaticale du celtique.\par
Bien que la langue d’oïl d’abord, la française ensuite, aient dû à la simplicité plus grande des mélanges de races et d’idiomes d’où elles sont issues un plus grand caractère d’unité que le roman, elles ont eu cependant des dialectes qui ont vécu et se maintien­nent. Ce n’est pas trop d’honneur pour ces formes que de les appeler des dialectes, et non pas des patois. Leur raison d’être ne se trouve pas dans la corruption du type dominant dont elles ont toujours été au moins les contemporaines. Elle réside dans la proportion différente des éléments celtique, romain et germanique qui ont constitué ou constituent encore notre nationalité. En deçà de la Seine, le dialecte picard est, par l’eurythmie et la lexicologie, tout près du flamand, dont les affinités germaniques sont si évidentes qu’il n’est pas besoin de les relever. En cela, le flamand est resté fidèle le aux prédilections de la langue d’oïl, qui put, à un certain moment, sans cesser d’être elle-même, admettre, dans les vers d’un poème, les formes et les expressions presque pures du langage parlé à Arras \footnote{P. Pâris, \emph{Garin le Loherain}, préface.}.\par
À mesure qu’on s’avance au delà de la Seine et en deçà de la Loire, les idiomes provinciaux tiennent, de plus en plus, de la nature celtique. Dans le bourguignon, dans les dialectes du Pays de Vaud et de la Savoie, la lexicologie même, chose bien digne de remarque, en a gardé de nombreuses traces, qui ne se trouvent pas dans le français, où généralement le latin rustique domine \footnote{Il est toutefois à remarquer que l’accent vaudois et savoyard a quelque chose de méridional qui rappelle fortement la colonie d’Aventicum.}.\par
Je relevais ailleurs comment, à dater du XV\textsuperscript{e} siècle, l’influence du nord de la France avait cédé devant la prépondérance croissante des races d’outre-Loire. Il n’y a qu’à rapprocher ce que je dis ici, touchant le langage, de ce qu’alors je disais du sang, pour voir combien est serrée la relation entre l’élément physique et l’instrument phonétique de l’individualité d’une population \footnote{Pott exprime très bien comment les dialectes sont les modifications parlées qui maintiennent l’accord entre l’état de composition du sang et celui de la langue, lorsqu’il dit : « Les dialectes sont la diversité dans l’unité, les sections chromatiques de l’Un primordial « et de la lumière unicolore. » (Pott, \emph{Encycl. Erchs. und Gruber}, p. 66.) – C’est, sans doute, une phraséologie obscure ; mais ici elle indique assez ce qu’elle entend.}.\par
Je me suis un peu étendu sur un fait particulier à la France. Si l’on veut le généraliser à toute l’Europe, on ne lui trouvera guère de démentis. Partout on verra que les modifications et les changements successifs d’un idiome ne sont pas, comme on le dit communément, l’œuvre des siècles : s’il en était ainsi, l’ekkhili, le berbère, l’euskara, le bas-breton, auraient depuis longtemps disparu, et ils vivent. Modifications et changements sont amenés, avec un parallélisme bien frappant, par les révolutions survenues dans le sang des générations successives.\par
Je ne passerai pas, non plus, sous silence un détail qui doit trouver ici son explication. J’ai dit comment certains groupes ethniques pouvaient, sous l’empire d’une aptitude et de nécessités particulières, renoncer à leur idiome naturel pour en accepter un qui leur était plus ou moins étranger. J’ai cité les Juifs, j’ai cité les Parsis. Il existe encore des exemples plus singuliers de cet abandon. Nous voyons des peuples sauvages en possession de langages supérieurs à eux-mêmes, et c’est l’Amérique qui nous offre ce spectacle.\par
Ce continent a eu cette singulière destinée, que ses populations les plus actives se sont développées, pour ainsi dire, en secret. L’art de l’écriture a fait défaut à ses civilisations. Les temps historiques n’y commencent que très tard, pour rester presque toujours obscurs. Le sol du nouveau monde possède un grand nombre de tribus qui, voisines à voisines, se ressemblent peu, bien qu’appartenant toutes à des origines communes diversement combinées.\par
 M. d’Orbigny nous apprend que, dans l’Amérique centrale, le groupe qu’il appelle rameau chiquitéen, est un composé de nations comptant, pour la plus nombreuse, environ quinze mille âmes, et pour celles qui le sont moins, entre trois cents et cinquante membres, et que toutes ces nations, même les infiniment petites, possèdent des idiomes distincts. Un tel état de choses ne peut résulter que d’une immense anarchie ethnique.\par
Dans cette hypothèse, je ne m’étonne nullement de voir plusieurs d’entre ces peuplades, comme les Chiquitos, maîtresses d’une langue compliquée et, à ce qu’il semble, assez savante. Chez ces indigènes, les mots dont l’homme se sert ne sont pas toujours les mêmes que ceux dont use la femme. En tous cas, l’homme, lorsqu’il emploie les expressions de la femme, en modifie les désinences. Ceci est assurément fort raffiné. Malheureusement, à côté de ce luxe lexicologique, le système de numéra­tion se présente restreint aux nombres les plus élémentaires. Très probablement, dans une langue en apparence si travaillée, ce trait d’indigence n’est que l’effet de l’injure des siècles, servie par la barbarie des possesseurs actuels. On se rappelle involontairement, en contemplant de telles bizarreries, ces palais somptueux, merveilles de la Renaissance, que les effets des révolutions ont adjugés définitivement à de grossiers villageois. L’œil y admire encore des colonnettes délicates, des rinceaux élégants, des porches sculptés, des escaliers hardis, des arêtes imposantes, luxe inutile à la misère qui les habite ; tandis que les toits crevés laissent entrer la pluie, que les planchers s’effondrent et que la pariétaire disjoint les murs qu’elle envahit.\par
Je puis établir désormais que la philologie, dans ses rapports avec la nature particulière des races, confirme toutes les observations de la physiologie et de l’histoi­re. Seulement, ses assertions se font remarquer par une extrême délicatesse, et lorsqu’on ne peut s’appuyer que sur elles, rien de plus hasardé que de s’en contenter pour conclure. Sans doute, sans nul doute, l’état d’un langage répond à l’état intellectuel du groupe qui le parle, mais non pas toujours à sa valeur intime. Pour obtenir ce rapport, il faut considérer uniquement la race par laquelle et pour laquelle ce langage a été primitivement créé. Or l’histoire ne paraît nous adresser, à part la famille noire et quelques peuplades jaunes, qu’à des races quartenaires, tout au plus. En conséquence, elle ne nous conduit que devant des idiomes dérivés, dont on ne peut préciser nettement la loi de formation que lorsque ces idiomes appartiennent à des époques comparativement récentes. Il s’ensuit que des résultats ainsi obtenus, et qui ont besoin constamment de la confirmation historique, ne sauraient fournir une classe de preuves bien infaillibles. À mesure qu’on s’enfonce dans l’antiquité et que la lumière vacille davantage, les arguments philologiques deviennent plus hypothétiques encore. Il est fâcheux de s’y voir réduit lorsqu’on cherche à éclairer la marche d’une famille humaine et à reconnaître les éléments ethniques qui la composent. Nous savons que le sanscrit, le zend, sont des langues parentes. C’est un grand point. Quant à leur racine commune, rien ne nous est révélé. De même pour les autres langues très anciennes. De l’euskara, nous ne connaissons rien que lui-même. Comme il n’a pas, jusqu’à présent, d’analogue, nous ignorons sa généalogie, nous ignorons s’il doit être considéré comme tout à fait primitif, ou bien s’il ne faut voir en lui qu’un dérivé. Il ne saurait donc rien nous apprendre de positif sur la nature simple ou composite du groupe qui le parle.\par
En matière d’ethnologie, il est bon d’accepter avec gratitude les secours philologi­ques. Pourtant il ne faut les recevoir que sous réserve, et, autant que possible, ne rien fonder sur eux seuls \footnote{On ne doit pas perdre de vue que les précautions ici indiquées ne s’appliquent qu’à la détermination de la généalogie d’un peuple, et non pas d’une famille de peuples. Si une nation change quelquefois de langue, jamais ce fait ne s’est produit et ne pourrait se produire pour tout un faisceau de nationalités, ethniquement identiques, politiquement indépendantes. Les juifs ont abandonné leur idiome ; l’ensemble des nations sémitiques n’a jamais pu perdre ses dialectes natifs et ne saurait en avoir d’autres.}.\par
Cette règle est commandée par une nécessaire prudence. Cependant tous les faits qui viennent d’être passés en revue établissent que l’identité est originairement entière entre le mérite intellectuel d’une race et celui de sa langue naturelle et propre ; que les langues sont, par conséquent, inégales en valeur et en portée, dissemblables dans les formes et dans le fond, comme les races ; que leurs modifications ne proviennent que de mélanges avec d’autres idiomes, comme les modifications des races ; que leurs qualités et leurs mérites s’absorbent et disparaissent, absolument comme le sang des races, dans une immersion trop considérable d’éléments hétérogènes ; enfin que, lorsqu’une langue de caste supérieure se trouve chez un groupe humain indigne d’elle, elle ne manque pas de dépérir et de se mutiler. Si donc il est souvent difficile, dans un cas particulier, de conclure, de prime abord, de la valeur de la langue à celle du peuple qui s’en sert, il n’en reste pas moins incontestable qu’en principe on le peut faire. Je pose donc cet axiome général :\par
La hiérarchie des langues correspond rigoureusement à la hiérarchie des races.
\section[{I.16. Récapitulation ; caractères respectifs des trois grandes races ; effets sociaux des mélanges ; supériorité du type blanc et, dans ce type, de la famille ariane.}]{I.16. \\
Récapitulation ; caractères respectifs des trois grandes races ; effets sociaux des mélanges ; supériorité du type blanc et, dans ce type, de la famille ariane.}
\noindent J’ai montré la place réservée qu’occupe notre espèce dans le monde organique. On a pu voir que de profondes différences physiques, que des différences morales non moins accusées, la séparaient de toutes les autres classes d’êtres vivants. Ainsi mise à part, je l’ai étudiée en elle-même, et la physiologie, bien qu’incertaine dans ses voies, peu sûre dans ses ressources, et défectueuse dans ses méthodes, m’a néanmoins permis de distinguer trois grands types nettement distincts, le noir, le jaune et le blanc.\par
La variété mélanienne est la plus humble et gît au bas de l’échelle. Le caractère d’animalité empreint dans la forme de son bassin lui impose sa destinée, dès l’instant de la conception. Elle ne sortira jamais du cercle intellectuel le plus restreint. Ce n’est cependant pas une brute pure et simple, que ce nègre à front étroit et fuyant, qui porte, dans la partie moyenne de son crâne, les indices de certaines énergies grossièrement puissantes. Si ces facultés pensantes sont médiocres ou même nulles, il possède dans le désir, et par suite dans la volonté, une intensité souvent terrible. Plusieurs de ses sens sont développés avec une vigueur inconnue aux deux autres races : le goût et l’odorat principalement \footnote{« Le goût et l’odorat sont, chez le nègre, aussi puissants qu’informes. Il mange tout, et les odeurs les plus répugnantes, à notre avis, lui sont agréables. » (Pruner, ouvrage cité, t. I p. 133.)}.\par
Mais là, précisément, dans l’avidité même de ses sensations, se trouve le cachet frappant de son infériorité. Tous les aliments lui sont bons, aucun ne le dégoûte, aucun ne le repousse. Ce qu’il souhaite, c’est manger, manger avec excès, avec fureur ; il n’y a pas de répugnante charogne indigne de s’engloutir dans son estomac. Il en est de même pour les odeurs, et sa sensualité s’accommode non seulement des plus grossières, mais des plus odieuses. À ces principaux traits de caractère il joint une instabilité d’humeur, une variabilité de sentiments que rien ne peut fixer, et qui annule, pour lui, la vertu comme le vice. On dirait que l’emportement même avec lequel il poursuit l’objet qui a mis sa sensitivité en vibration et enflammé sa convoitise, est un gage du prompt apaisement de l’une et du rapide oubli de l’autre. Enfin il tient également peu à sa vie et à celle d’autrui ; il tue volontiers pour tuer, et cette machine humaine, si facile à émouvoir, est, devant la souffrance, ou d’une lâcheté qui se réfugie volontiers dans la mort, ou d’une impassibilité monstrueuse.\par
La race jaune se présente comme l’antithèse de ce type. Le crâne, au lieu d’être rejeté en arrière, se porte précisément en avant. Le front, large, osseux, souvent saillant, développé en hauteur, plombe sur un faciès triangulaire, où le nez et le menton ne montrent aucune des saillies grossières et rudes qui font remarquer le nègre. Une tendance générale à l’obésité n’est pas là un trait tout à fait spécial, pourtant il se rencontre plus fréquemment chez les tribus jaunes que dans les autres variétés. Peu de vigueur physique, des dispositions à l’apathie. Au moral, aucun de ces excès étranges, si communs chez les Mélaniens. Des désirs faibles, une volonté plutôt obstinée qu’extrême, un goût perpétuel mais tranquille pour les jouissances matérielles ; avec une rare gloutonnerie, plus de choix que les nègres dans les mets destinés à la satisfaire. En toutes choses, tendances à la médiocrité ; compréhension assez facile de ce qui n’est ni trop élevé ni trop profond ; amour de l’utile, respect de la règle, conscience des avantages d’une certaine dose de liberté. Les jaunes sont des gens pratiques dans le sens étroit du mot. Ils ne rêvent pas, ne goûtent pas les théories, inventent peu, mais sont capables d’apprécier et d’adopter ce qui sert. Leurs désirs se bornent à vivre le plus doucement et le plus commodément possible. On voit qu’ils sont supérieurs aux nègres. C’est une populace et une petite bourgeoisie que tout civilisateur désirerait choisir pour base de sa société : ce n’est cependant pas de quoi créer cette société ni lui donner du nerf, de la beauté et de l’action.\par
Viennent maintenant les peuples blancs. De l’énergie réfléchie, ou pour mieux dire, une intelligence énergique ; le sens de l’utile, mais dans une signification de ce mot beaucoup plus large, plus élevée, plus courageuse, plus idéale que chez les nations jaunes ; une persévérance qui se rend compte des obstacles et trouve, à la longue, les moyens de les écarter ; avec une plus grande puissance physique, un instinct extraordinaire de l’ordre, non plus seulement comme gage de repos et de paix, mais comme moyen indispensable de conservation, et, en même temps, un goût prononcé de la liberté, même extrême ; une hostilité déclarée contre cette organisation formaliste où s’endorment volontiers les Chinois, aussi bien que contre le despotisme hautain, seul frein suffisant aux peuples noirs.\par
 Les blancs se distinguent encore par un amour singulier de la vie. Il paraît que, sachant mieux en user, ils lui attribuent plus de prix, ils la ménagent davantage, en eux-mêmes et dans les autres. Leur cruauté, quand elle s’exerce, a la conscience de ses excès, sentiment très problématique chez les noirs. En même temps, cette vie occupée, qui leur est si précieuse, ils ont découvert des raisons de la livrer sans murmure. Le premier de ces mobiles, c’est l’honneur, qui, sous des noms à peu près pareils, a occupé une énorme place dans les idées, depuis le commencement de l’espèce. Je n’ai pas besoin d’ajouter que ce mot d’honneur et la notion civilisatrice qu’il renferme sont, également, inconnus aux jaunes et aux noirs.\par
Pour terminer le tableau, j’ajoute que l’immense supériorité des blancs, dans le domaine entier de l’intelligence, s’associe à une infériorité non moins marquée dans l’intensité des sensations. Le blanc est beaucoup moins doué que le noir et que le jaune sous le rapport sensuel. Il est ainsi moins sollicité et moins absorbé par l’action corporelle, bien que sa structure soit remarquablement plus vigoureuse \footnote{M. Martius remarque que l’Européen surpasse les hommes de couleur en intensité du fluide nerveux. (\emph{Reise in Brasilien}, t. I, p. 259.)}.\par
Tels sont les trois éléments constitutifs du genre humain, ce que j’ai appelé les types secondaires, puisque j’ai cru devoir laisser en dehors de la discussion l’individu adamite. C’est de la combinaison des variétés de chacun de ces types, se mariant entre elles, que les groupes tertiaires sont issus. Les quatrièmes formations sont nées du mariage d’un de ces types tertiaires ou d’une tribu pure avec un autre groupe ressortant d’une des deux espèces étrangères.\par
Au-dessous de ces catégories, d’autres se sont révélées et se révèlent chaque jour. Les unes très caractérisées, formant de nouvelles originalités distinctes, parce qu’elles proviennent de fusions achevées ; les autres incomplètes, désordonnées, et, on peut le dire, antisociales, parce que leurs éléments, ou trop disparates, ou trop nombreux, ou trop infimes, n’ont pas eu le temps ni la possibilité de se pénétrer d’une manière féconde. À la multitude de toutes ces races métisses si bigarrées qui composent désormais l’humanité entière, il n’y a pas à assigner d’autres bornes que la possibilité effrayante de combinaisons des nombres.\par
Il serait inexact de prétendre que tous les mélanges sont mauvais et nuisibles. Si les trois grands types, demeurant strictement séparés, ne s’étaient pas unis entre eux, sans doute la suprématie serait toujours restée aux plus belles des tribus blanches, et les variétés jaunes et noires auraient rampé éternellement aux pieds des moindres nations de cette race. C’est un état en quelque sorte idéal, puisque l’histoire ne l’a pas vu. Nous ne pouvons l’imaginer qu’en reconnaissant l’incontestable prédominance de ceux de nos groupes demeurés les plus purs.\par
Mais tout n’aurait pas été gain dans une telle situation. La supériorité relative, en persistant d’une manière plus évidente, n’aurait pas, il faut le reconnaître, été accom­pagnée de certains avantages que les mélanges ont produits, et qui, bien que ne contre-balançant pas, tant s’en faut, la somme de leurs inconvénients, n’en sont pas moins dignes d’être, quelquefois, applaudis. C’est ainsi que le génie artistique, également étranger aux trois grands types, n’a surgi qu’à la suite de l’hymen des blancs avec les nègres. C’est encore ainsi que, par la naissance de la variété malaise, il est sorti des races jaunes et noires une famille plus intelligente que sa double parenté, et que de l’alliance jaune et blanche il est issu, de même, des intermédiaires très supérieurs aux populations purement finnoises aussi bien qu’aux tribus mélaniennes.\par
Je ne le nie pas : ce sont là de bons résultats. Le monde des arts et de la noble littérature résultant des mélanges du sang, les races inférieures améliorées, ennoblies, sont autant de merveilles auxquelles il faut applaudir. Les petits ont été élevés. Malheureusement les grands, du même coup, ont été abaissés, et c’est un mal que rien ne compense ni ne répare. Puisque j’énumère tout ce qui est en faveur des mélanges ethniques, j’ajouterai encore qu’on leur doit bien des raffinements de mœurs, de croyances, surtout des adoucissements de passions et de penchants. Mais ce sont autant de bénéfices transitoires, et si je reconnais que le mulâtre, dont on peut faire un avocat, un médecin, un commerçant, vaut mieux que son grand-père nègre, entièrement inculte et propre à rien, je dois avouer aussi que les Brahmanes de l’Inde primitive, les héros de l’Iliade, ceux du Schahnameh, les guerriers scandinaves, tous fantômes si glorieux des races les plus belles, désormais disparues, offraient une image plus brillante et plus noble de l’humanité, étaient surtout des agents de civilisation et de grandeur plus actifs, plus intelligents, plus sûrs que les populations métisses, cent fois métisses, de l’époque actuelle, et cependant, déjà, ils n’étaient pas purs.\par
Quoi qu’il en soit, l’état complexe des races humaines est l’état historique, et une des principales conséquences de cette situation a été de jeter dans le désordre une grande partie des caractères primitifs de chaque type. On a vu, par suite d’hymens multipliés, les prérogatives, non seulement diminuer d’intensité comme les défauts, mais aussi se séparer, s’éparpiller et se faire souvent contraste. La race blanche possédait originairement le monopole de la beauté, de l’intelligence et de la force. À la suite de ses unions avec les autres variétés, il se rencontra des métis beaux sans être forts, forts sans être intelligents, intelligents avec beaucoup de laideur et de débilité. Il se trouva aussi que la plus grande abondance possible du sang des blancs, quand elle s’accumulait, non pas d’un seul coup, mais par couches successives, dans une nation, ne lui apportait plus ses prérogatives naturelles. Elle ne faisait souvent qu’augmenter le trouble déjà existant dans les éléments ethniques et ne semblait conserver de son excellence native qu’une plus grande puissance dans la fécondation du désordre. Cette anomalie apparente s’explique aisément, puisque chaque degré de mélange parfait produit, outre une alliance d’éléments divers, un type nouveau, un développement de facultés particulières. Aussitôt qu’à une série de créations de ce genre d’autres éléments viennent s’adjoindre encore, la difficulté d’harmoniser le tout crée l’anarchie, et plus cette anarchie augmente, plus les meilleurs, les plus riches, les plus heureux apports perdent leur mérite et, par le seul fait de leur présence, augmentent un mal qu’ils se trouvent impuissants à calmer. Si donc les mélanges sont, dans une certaine limite, favorables à la masse de l’humanité, la relèvent et l’ennoblissent, ce n’est qu’aux dépens de cette humanité même, puisqu’ils l’abaissent, l’énervent, l’humilient, l’étêtent dans ses plus nobles éléments, et quand bien même on voudrait admettre que mieux vaut transformer en hommes médiocres des myriades d’êtres infimes que de conserver des races de princes dont le sang, subdivisé, appauvri, frelaté, devient l’élément déshonoré d’une semblable métamorphose, il resterait encore ce malheur que les mélanges ne s’arrêtent pas ; que les hommes médiocres, tout à l’heure formés aux dépens de ce qui était grand, s’unissent à de nouvelles médiocrités, et que de ces mariages, de plus en plus avilis, naît une confusion qui, pareille à celle de Babel, aboutit à la plus complète impuissance, et mène les sociétés au néant auquel rien ne peut remédier.\par
C’est là ce que nous apprend l’histoire. Elle nous montre que toute civilisation découle de la race blanche, qu’aucune ne peut exister sans le concours de cette race, et qu’une société n’est grande et brillante qu’à proportion qu’elle conserve plus longtemps le noble groupe qui l’a créée et que ce groupe lui-même appartient au rameau le plus illustre de l’espèce. Pour exposer ces vérités dans un jour éclatant, il suffit d’énumérer, puis d’examiner les civilisations qui ont régné dans le monde, et la liste n’en est pas longue.\par
Du sein de ces multitudes de nations qui ont passé ou vivent encore sur la terre, dix seulement se sont élevées à l’état de sociétés complètes. Le reste, plus ou moins indépendant, gravite à l’entour comme les planètes autour de leurs soleils. Dans ces dix civilisations, s’il se trouve, soit un élément de vie étranger à l’impulsion blanche, soit un élément de mort qui ne provienne pas des races annexées aux civilisateurs, ou du fait des désordres introduits par les mélanges, il est évident que toute la théorie exposée dans ces pages est fausse. Au contraire, si les choses se trouvent telles que je les annonce, la noblesse de notre espèce reste prouvée de la manière la plus irréfragable, et il n’y a plus moyen de la contester. C’est là que se rencontrent donc, tout à la fois, la seule confirmation suffisante et le détail désirable des preuves du système. C’est là, seulement, que l’on peut suivre, avec une exactitude satisfaisante, le développement de cette affirmation fondamentale, que les peuples ne dégénèrent que par suite et en proportion des mélanges qu’ils subissent, et dans la mesure de qualité de ces mélanges ; que, quelle que soit cette mesure, le coup le plus rude dont puisse être ébranlée la vitalité d’une civilisation, c’est quand les éléments régulateurs des sociétés et les éléments développés par les faits ethniques en arrivent à ce point de multiplicité qu’il leur devient impossible de s’harmoniser, de tendre, d’une manière sensible, vers une homogénéité nécessaire, et, par conséquent, d’obtenir, avec une logique commune, ces instincts et ces intérêts communs, seules et uniques raisons d’être d’un lien social. Pas de plus grand fléau que ce désordre, car, si mauvais qu’il puisse rendre le temps présent, il prépare un avenir pire encore.\par
Pour entrer dans ces démonstrations, je vais aborder la partie historique de mon sujet. C’est une tâche vaste, j’en conviens ; cependant, elle se présente si fortement enchaînée dans toutes ses parties, et, là, si concordante, convergeant si strictement vers le même but, que, loin d’être embarrassé de sa grandeur, il me semble en tirer un puissant secours pour mieux établir la solidité des arguments que je vais moissonner. Il me faudra, sans doute, parcourir, avec les migrations blanches, une grande partie de notre globe. Mais ce sera toujours rayonner autour des régions de la haute Asie, point central d’où la race civilisatrice est primitivement descendue. J’aurai à rattacher, tour à tour, au domaine de l’histoire, des contrées qui, entrées une fois dans sa possession, ne pourront plus s’en séparer. Là, je verrai se déployer, dans toutes leurs conséquences, les lois ethniques et leur combinaison. Je constaterai avec quelle régularité inexorable et monotone elles imposent leur application. De l’ensemble de ce spectacle, à coup sûr bien imposant, de l’aspect de ce paysage animé qui embrasse, dans son cadre immense, tous les pays de la terre où l’homme s’est montré vraiment dominateur ; enfin, de ce concours de tableaux également émouvants et grandioses, je tirerai, pour établir l’inégalité des races humaines et la prééminence d’une seule sur toutes les autres, des preuves incorruptibles comme le diamant, et sur lesquelles la dent vipérine de l’idée démagogique ne pourra mordre. Je vais donc quitter, ici, la forme de la critique et du raisonnement pour prendre celle de la synthèse et de l’affirmation. Il ne me reste plus qu’à faire bien connaître le terrain sur lequel je m’établis. Ce sera court.\par
J’ai dit que les grandes civilisations humaines ne sont qu’au nombre de dix et que toutes sont issues de l’initiative de la race blanche \footnote{Je suis encore plus généreux que M. J. Mohl. Le savant professeur exprime ainsi son opinion à ce sujet : « Quand on réfléchit qu’il n’y a eu dans le monde que trois grandes impulsions civilisatrices, celle donnée par les Indiens, celle donnée par les Sémites et celle donnée par les Chinois, que l’histoire de l’esprit humain n’est que le développement et la lutte de ces trois éléments, on comprend alors de quelle importance, etc. » (\emph{Rapport annuel fait à la Société asiatique, 1851.}) On ne verra rien, du reste, dans ce que j’ai à dire qui contredise ce point de vue fort exact, mais un peu abstrait.}. Il faut mettre en tête de la liste :\par
\textbf{I.} La civilisation indienne. Elle s’est avancée dans la mer des Indes, dans le nord et à l’est du continent asiatique, au delà du Brahmapoutra. Son foyer se trouvait dans un rameau de la nation blanche des Arians.\par
\textbf{II.} Viennent ensuite les Égyptiens. Autour d’eux se rallient les Éthiopiens, les Nubiens, et quelques petits peuples habitant à l’ouest de l’oasis d’Ammon. Une colonie ariane de l’Inde, établie dans le haut de la vallée du Nil, a créé cette société.\par
\textbf{III.} Les Assyriens, auxquels se rattachent les Juifs, les Phéniciens les Lydiens les Carthaginois, les Hymiarites, ont dû leur intelligence sociale à ces grandes invasions blanches auxquelles on peut conserver le nom de descendants de Cham et de Sem. Quant aux Zoroastriens-Iraniens qui dominèrent dans l’Asie antérieure sous le nom de Mèdes, de Perses et de Bactriens, c’était un rameau de la famille ariane.\par
\textbf{IV.} Les Grecs étaient issus de la même souche ariane, et ce furent les éléments sémitiques qui la modifièrent.\par
\textbf{V.} Le pendant de ce qui arrive pour l’Égypte se rencontre en Chine. Une colonie ariane, venue de l’Inde, y apporta les lumières sociales. Seulement, au lieu de se mêler, comme sur les bords du Nil, avec des populations noires, elle se fondit dans des masses malaises et jaunes, et reçut, en outre, par le nord-ouest, d’assez nombreux apports d’éléments blancs, également arians, mais non plus hindous \footnote{Ainsi que j’ai déjà eu l’occasion d’en avertir le lecteur je me vois quelquefois contraint de poser \emph{a priori}, comme déjà démontrés, des faits qui sont discutés plus tard. Je demande pardon de cette liberté sans laquelle il me serait impossible de cheminer. Tout ce que je puis faire, c’est d’en restreindre l’usage aux cas véritablement impérieux. L’origine ariane des sociétés égyptienne et chinoise appelle la démonstration, je ne me le dissimule pas, et je ferai de mon mieux pour la donner.}.\par
\textbf{VI}. L’ancienne civilisation de la péninsule italique, d’où sortit la culture romaine, fut une marqueterie de Celtes, d’Ibères, d’Arians et de Sémites.\par
\textbf{VII.} Les races germaniques transformèrent, au V\textsuperscript{e} siècle, le génie de l’Occident. Elles étaient arianes.\par
\textbf{VIII, IX, X.} Sous ces chiffres, je classerai les trois civilisations de l’Amérique, celles des Alléghaniens, des Mexicains et des Péruviens.\par
Sur les sept premières civilisations, qui sont celles de l’ancien monde, six appar­tiennent, en partie du moins, à la race ariane, et la septième, celle d’Assyrie, doit à cette même race la renaissance iranienne, qui est restée son plus illustre monument historique. Presque tout le continent d’Europe est occupé, actuellement, par des grou­pes où existe le principe blanc, mais où les éléments non-arians sont les plus nombreux. Point de civilisation véritable chez les nations européennes, quand les rameaux arians n’ont pas dominé.\par
Dans les dix civilisations, pas une race mélanienne n’apparaît au rang des initiateurs. Les métis seuls parviennent au rang des initiés.\par
De même, point de civilisations spontanées chez les nations jaunes, et la stagnation lorsque le sang arian s’est trouvé épuisé.\par
Voilà le thème dont je vais suivre le rigoureux développement dans les annales universelles. La première partie de mon ouvrage se termine ici.
\chapterclose


\chapteropen
\chapter[{II. civilisation antique rayonnant de l’asie centrale au sud-ouest}]{II. \\
civilisation antique rayonnant de l’asie centrale au sud-ouest}\renewcommand{\leftmark}{II. \\
civilisation antique rayonnant de l’asie centrale au sud-ouest}


\chaptercont
\section[{II.1. Les Chamites.}]{II.1. \\
Les Chamites.}
\noindent Les premières traces de l’histoire certaine remontent à une époque antérieure à l’an 5000 avant la naissance de Jésus-Christ  \footnote{L’opinion de Klaproth (\emph{Asia polyglotta}) ne les reporte pas plus haut que l’an 3000 ; mais d’autres chronologistes sont plus larges dans leur estimation, entre autres M. Lepsius, dans ses travaux sur l’Égypte. Il rend l’opinion de Klaproth tout à fait inadmissible, puisqu’il fait remonter une classe entière de monuments égyptiens à l’an 4000. (Lepsius, \emph{Briefe über Ægypten, Æthiopien und der Halbinsel des Sinaï} ; Berlin, 1852). Je n’ai pas, du reste, à m’occuper d’un tel problème. Il importe peu à mon sujet. Je ne prétends ici qu’à fixer, à peu près, la pensée du lecteur.}. Vers cette date, la présence évidente des hommes commence à troubler le silence des siècles. On entend bourdonner les four­milières des nations du côté de l’Asie inférieure. Le bruit se prolonge au sud, dans la direction de la péninsule arabique et du continent africain ; tandis que, vers l’est, partant des hautes vallées ouvertes sur les versants du Bolor \footnote{J’entends désigner la chaîne qui, s’attachant à l’Hindou-Kho septentrional, remonte au nord, coupe le Thian-Chan et incline à l’ouest vers le lac Kabankoul. (Voir M. A. de Humboldt, \emph{Asie centrale}, carte.)}, il se répercute, d’échos en échos, jusque vers les régions situées sur la rive gauche de l’Indus.\par
Les populations qui appellent d’abord nos regards sont de race noire.\par
 Cette diffusion extrême de la famille mélanienne ne peut manquer de surprendre \footnote{Il résulte, des plus récentes découvertes opérées dans le centre et le sud de l’Afrique, que les populations de cette partie du monde ont été étrangement agitées et déplacées à des époques inconnues. (Voir dans la \emph{Zeitschrift für die Kunde des Morgenlandes} et dans la \emph{Zeitschrift der deutschen morgenlændischen Gesellschaft}, les travaux de Pott, d’Ewald et du missionnaire protestant Krapf.)}. Non contente du continent qui lui appartient tout entier, nous la voyons, avant la naissance d’aucune société, maîtresse et dominatrice absolue de l’Asie méridionale, et lorsque, plus tard, nous monterons vers le pôle nord, nous découvrirons encore d’anciennes peuplades du même sang, oubliées jusqu’à nos jours dans les montagnes chinoises du Kouenloun et au delà des îles du Japon. Si extraordinaire que le fait puisse paraître, telle fut pourtant, aux premiers âges, la fécondité de cette immense catégorie du genre humain \footnote{Sur les habitants noirs du Kouenloun, voir Ritter, \emph{Erdkunde, Asien} ; Lassen, \emph{Indische Alterthumskunde}, t. I, p. 391. ‑ On trouve encore d’autres noirs à cheveux crépus et laineux dans le Kamaoun, où ils s’appellent Rawats et Raieh. C’est, probablement, une branche des Thums du Népal. (Ritter, \emph{Erdkunde, Asien}, t. II, p. 1044.) ‑ Dans l’Assam, au sud du district de Queda, habitent les \emph{Samang}, sauvages à cheveux crépus, ressemblant du reste aux Papouas de la Nouvelle-Guinée (Ritter, ouvr. cité, t. III, p. 1131.) ‑ À Formose, autres nègres ressemblant aux Haraforas. (Ritter, t. III, p. 879.) ‑ Kæmfer parle d’habitants noirs dans les îles au sud du japon (p. 81.) ‑Elphinstone (\emph{Account of the kingdom ot Cabul}, p. 493) mentionne dans le Sedjistan, sur le lac Zareh, la présence d’une peuplade nègre, etc.}.\par
Soit qu’il faille la tenir pour simple ou composée \footnote{Elle comptait, certainement, plusieurs variétés, puisque la note précédente indique des nègres à cheveux crépus dans le Kamaoun, dans l’Assan, etc., tandis que la plupart des nègres asiatiques ont les cheveux plats. M. Lassen a donc eu tort de dire (\emph{Indische Alterthumskunde}, t. I, p. 390) que les nègres asiatiques n’ont pas les cheveux laineux des Africains ni le ventre saillant des Pélagiens. C’est une race très mélangée, un type tertiaire incontestable et qui tient, par tous les côtés, aux familles africaines et océaniennes.}, soit qu’on la considère dans les régions brûlantes du midi ou dans les vallées glacées du septentrion, elle ne transmet aucun vestige de civilisation, ni présente ni possible. Les mœurs de ces peuplades paraissent avoir été des plus brutalement cruelles. La guerre d’extermination, voilà pour leur politique ; l’anthropophagie, voilà pour leur morale et leur culte. Nulle part, on ne voit ni villes, ni temples, ni rien qui indique un sentiment quelconque de sociabilité. C’est la barbarie dans toute sa laideur, et l’égoïsme de la faiblesse dans toute sa férocité. L’impression qu’en reçurent les observateurs primitifs, issus d’un autre sang, que je vais bientôt introduire sur la scène, fut partout la même, mêlée de mépris, de terreur et de dégoût. Les bêtes de proie semblèrent d’une trop noble essence pour servir de point de comparaison avec ces tribus hideuses. Des singes suffirent à en représenter l’idée au physique, et quant au moral, on se crut obligé d’évoquer la ressemblance des esprits de ténèbres \footnote{Deuteron., II, 9. ‑ « Filiis Loth tradidi Ar in possessionem, 10. Enim primi fuerunt « habitatores ejus, populus magnus, et validus, et tam excelsus, ut de Enacim Stirpe, 11. « Quasi gigantes crederentur. » Et encore dans le même livre  : « 20. Terra gigantum « reputata est, et in ipsa olim habitaverunt gigantes quos Ammonitæ vocant Zomzommim, « 21. Populus magnus, et multus et proceræ lengitudinis, sicut Enacim. » (Voir, plus bas, la note sur les Chorréens.)}.\par
Tandis que le monde central était, jusque très avant dans le nord-est, inondé par de pareils essaims, la partie boréale de l’Asie, les bords de la mer Glaciale et l’Europe, presque en totalité, se trouvaient au pouvoir d’une variété toute différente \footnote{Les nègres affectionnent les généalogies qui commencent, non pas au soleil, ni à la lune, mais aux bêtes. Les Sahos, sur la mer Rouge, non loin de Massowa, se disent descendus, à la treizième génération, d’un certain Aa’saor, (mot en alphabet étranger) fils d’une lionne et habitant des montagnes. Le choix de l’animal est, cette fois, assez noble, il faut l’avouer. Les fréquents contacts avec les Arabes ont produit quelque ennoblissement de l’imagination. (Voir Ewald, \emph{Ueber die Sahosprache in Æthiopien}, dans la \emph{Zeitschrift für die Kunde der Morgenlander}. (t. II, p. 13.)}. C’était la race jaune, qui, s’échappant du grand continent d’Amérique, s’était avancée à l’est et à l’ouest sur les bords des deux océans, et se répandait, d’un côté, vers le sud, où, par son hymen avec l’espèce noire, elle donnait naissance à la populeuse famille malaye, et, de l’autre, vers l’ouest, ce qui la conduisait sur les terres européennes encore inoccupées.\par
Cette bifurcation de l’invasion jaune démontre, d’une manière évidente, que les flots des arrivants rencontraient, sur leur front, une cause puissante qui les contraignait à se diviser. Ils étaient brisés, vers les plaines de la Mantchourie, par une digue forte et compacte, et bien du temps se passa avant qu’ils pussent inonder, à leur aise, les vastes régions centrales où campent, aujourd’hui, leurs descendants. Ils ruisselaient donc, en nombreux courants, sur les flancs de l’obstacle, occupant d’abord les contrées désertes, et c’est pour ce motif que les peuples jaunes devinrent les premiers possesseurs de l’Europe.\par
Cette race a semé ses tombeaux et quelques-uns de ses instruments de chasse et de guerre dans les steppes de la Sibérie, comme dans les forêts scandinaves et les tourbières des îles Britanniques \footnote{Prichard. \emph{Histoire naturelle de l’homme} (trad. de M. Roulin), t. I, p. 259.}. À prononcer d’après la façon de ces ustensiles, on ne saurait juger la race jaune beaucoup plus favorablement que les maîtres noirs du sud. Ce n’était pas alors, sur la plus grande partie de la terre, le génie, ni même l’intelligence, qui tenait le sceptre. La violence, la plus faible des forces, possédait seule la domination.\par
Combien de temps dura cet état de choses ? En un sens, la réponse est facile : ce régime se prolonge encore partout où les espèces noire et jaune sont demeurées à l’état tertiaire. Ainsi, cette ancienne histoire n’est pas spéculative. Elle peut servir de miroir à l’état contemporain d’une notable portion du globe. Mais de dire quand la barbarie a commencé, voilà ce qui dépasse les forces de la science. Par sa nature même elle est négative, parce qu’elle reste sans action. Elle végète inaperçue, et l’on ne peut constater son existence que le jour où une force de nature contraire se présente pour la battre en brèche. Ce jour fut celui de l’apparition de la race blanche au milieu des noirs. De ce moment seul, nous pouvons entrevoir une aurore planant au-dessus du chaos humain. Tournons-nous donc vers les origines de la famille d’élite, afin d’en saisir les premiers rayonnements.\par
Cette race ne paraît pas être moins ancienne que les deux autres. Avant ses invasions, elle vivait en silence, préparant les destinées humaines et grandissant, pour la gloire de la planète, dans une partie de notre globe qui, depuis, est devenue bien obscure.\par
Il est, entre les deux mondes du nord et du sud, et, pour me servir de l’expression hindoue, entre le pays du midi, contrée de la mort, et le pays septentrional, région des richesses \footnote{Lassen, \emph{Indische Alterthumskunde}, t. I.}, une série de plateaux qui semblent isolés du reste de l’univers, d’un côté par des montagnes d’une hauteur incomparable, de l’autre par des déserts de neige et une mer de glace \footnote{A. de Humboldt, \emph{Asie centrale}, t I.}.\par
Là, un climat dur et sévère semblerait particulièrement propre à l’éducation des races fortes, s’il en avait élevé ou transformé plusieurs. Des vents glacés et violents, de courts étés, de longs hivers, en un mot, plus de maux que de biens, rien de ce que l’on dit propre à exciter, à développer, à créer le génie civilisateur : voilà l’aspect de cette terre. Mais, à côté de tant de rudesse, et comme un véritable symbole des mérites secret de toute austérité, le sol recouvre d’immenses richesses minérales. Ce pays redoutable est, par excellence, le pays des richesses et des pierres fines \footnote{A. de Humboldt, \emph{Asie centrale}, t. I, p. 389. ‑ « Les recherches des dernières années et la « conviction que l’on a obtenue de la richesse métallique que possède encore de nos jours « l’Asie boréale, jusque dans la région des plaines, nous conduit presque involontairement « aux Issédons, aux Arimaspes et à ces griffons, gardiens de l’or, auxquels Aristée de « Proconnèse et, deux cents ans après lui, Hérodote, ont donné une si grande célébrité. J’ai « visité ces vallons où, à la pente méridionale de l’Oural, on a trouvé, il n’y a que quinze « ans, à peu de pouces sous le gazon, et très rapprochées les unes des autres, des masses « arrondies d’or, d’un poids de 13, de 16 et de 24 livres. Il est assez probable que des « masses plus volumineuses encore ont existé jadis à la surface même du sol, sillonnée par « les eaux courantes. Comment donc s’étonner que cet or, analogue aux blocs erratiques, « ait été recueilli par des peuples chasseurs ou pasteurs, etc. » C’est le Hataka, le pays de l’or de la géographie mythologique des Hindous. Les trésors y sont abondants et gardés par des gnomes appelés \emph{Guhyakas} (de \emph{guh}, cacher), dans lesquels on reconnaît les Finnois, les mineurs à la taille ramassée. Nous leur verrons jouer le même rôle chez les Scandinaves. (Lassen, \emph{Ind. Alterth.}, t. II, p. 62.)}. Sur ses montagnes habitent des animaux à fourrures et à lainage précieux, et le musc, cette production si chère aux Asiatiques, devait un jour en sortir. Tant de merveilles restent pourtant inutiles quand des mains habiles ne sont pas là pour les dévoiler et leur donner leur prix.\par
Mais ce n’étaient ni l’or, ni les diamants, ni les fourrures, ni le musc, dont ces régions devaient tirer leur gloire : leur honneur incomparable, c’est d’avoir élevé la race blanche.\par
Différente, tout à la fois, et des sauvages noirs du sud et des barbares jaunes du nord, cette variété humaine, bornée, dans ses débuts, à la part du monde la plus restreinte, la moins fertile, devait évidemment conquérir le reste, s’il était dans les desseins de la Providence que ce reste fût jamais mis en valeur. Un tel effort dépassait trop absolument le pouvoir des misérables multitudes maîtresses du tout. La tâche semble d’ailleurs tellement difficile, même pour les blancs, que cinq mille années n’ont pas encore suffi à son entier accomplissement.\par
La famille prédestinée ne peut, comme ses deux servantes, qu’être très obscurément définie. Elle porta partout de grandes similitudes, qui autorisent et forcent même à la ranger, tout entière, sous une même dénomination : celle, un peu vague et très incomplète, de race blanche. Comme, en même temps, ses principales ramifications trahissent des aptitudes assez diverses et se caractérisent facilement à part, on peut juger qu’il n’y a pas d’identité complète dans les origines de l’ensemble ; et, de même que la race noire et les habitants de l’hémisphère boréal présentent, dans le sein de leurs espèces respectives, des différences bien tranchées, il est vraisemblable aussi que la physiologie des blancs offrait, dès le principe, une semblable multiplicité de types. Plus tard nous rechercherons les traces de ces divergences. Ne nous occupons ici que des caractères communs.\par
Le premier examen en met en lumière un bien important : la race blanche ne nous apparaît jamais à l’état rudimentaire où nous voyons les autres. Dès le premier moment, elle se montre relativement cultivée et en possession des principaux éléments d’un état supérieur, qui, développé, plus tard, par ses rameaux multiples, aboutira à des formes diverses de civilisation.\par
Elle vivait encore réunie dans les pays reculés de l’Asie septentrionale, qu’elle jouissait déjà des enseignements d’une cosmogonie que nous devons supposer savante, puisque les peuples modernes les plus avancés n’en ont pas d’autre, que dis-je ? n’ont que des fragments de cette science antique consacrée par la religion \footnote{Suivant Ewald, les Sémites reconnaissent, comme leur lieu commun d’origine, le haut pays du nord-est, c’est-à-dire le lieu d’où sortirent les Zoroastriens. Il existe aussi, entre les premiers peuples de l’Asie intérieure et les Arians, des traditions communes qui ont devancé la formation des systèmes idiomatiques respectifs, tels que les quatre âges du monde, les dix ancêtres primitifs, le déluge, etc. (Lassen, \emph{Indisch. Alterth}., t. I, p. 528 ; Ewald, \emph{Geschichte des Volkes Israël}, t. I, p. 304)}. Outre ces lumières sur les origines du monde, les blancs gardaient le souvenir des premiers ancêtres, tant de ceux qui avaient succédé aux Noachides, que des patriarches antérieurs à la dernière catastrophe cosmique. On serait en droit d’en induire que, sous les trois noms de Sem, de Cham et de Japhet, ils classaient non pas tous nos congénères, mais uniquement les branches de la seule race considérée par eux comme véritablement humaine, c’est-à-dire de la leur. Le mépris profond qu’on leur connut, plus tard, pour les autres espèces en serait une preuve assez forte.\par
Lorsqu’on a appliqué le nom de Cham, tantôt aux Égyptiens, tantôt aux races noires, on ne l’a fait qu’arbitrairement dans un seul pays, dans des temps relativement récents et par suite d’analogies de sons qui ne présentent rien de certain et ne suffisent pas à une étymologie sérieuse.\par
Quoi qu’il en soit, voilà ces peuples blancs, longtemps avant les temps historiques, pourvus, dans leurs différentes branches, des deux éléments principaux de toute civilisation : une religion, une histoire.\par
Quant à leurs mœurs un trait saillant en est resté : ils ne combattaient pas à pied, comme, probablement, leurs grossiers voisins du nord et de l’est. Ils s’élançaient contre leurs ennemis, montés sur des chariots de guerre, et, de cette habitude conservée, unanimement, par les Égyptiens, les Hindous, les Assyriens, les Perses, les Grecs, les Galls, on est en droit de conclure un certain raffinement dans la science militaire, qu’il eût été impossible d’atteindre sans la pratique de plusieurs arts compliqués, tels que le travail du bois, du cuir, la connaissance des métaux, et le talent de les extraire et de les fondre. Les blancs primitifs savaient, aussi, tisser des étoffes \footnote{Lassen, \emph{Indisch. Alterth}., t. I, p. 815.} pour leur habillement et vivaient réunis et sédentaires dans de grands villages \footnote{Id., \emph{ibid.}, t. I, p. 816.}, ornés de pyramides, d’obélisques et de tumulus de pierre ou de terre.\par
Ils avaient su réduire les chevaux en domesticité. Leur mode d’existence était la vie pastorale. Leurs richesses consistaient en troupeaux nombreux de taureaux et de génisses \footnote{Il semble que l’existence pastorale ait d’abord été inventée par l’espèce blanche. Ce qui l’indiquerait, c’est que plusieurs familles jaunes ont ignoré l’usage du lait, et cela dans un état de civilisation avancée. Les habitants de certaines parties de la Chine et de la Cochinchine ne traient jamais leurs vaches. Les Aztèques ne pratiquent même pas la domestication des animaux. (Voir Prescott, \emph{History of the conquest of Mejico}, t. III, p. 257 ; et A. de Humboldt, \emph{Essai politique sur la Nouvelle-Espagne}, t. III, p. 58.)}. L’étude comparée des langues, d’où jaillissent, chaque jour, tant de faits curieux et inattendus, paraît établir, d’accord avec la nature de leurs territoires, qu’ils ne s’adonnaient que peu à l’agriculture \footnote{ \noindent Les méthodes que l’on a employées pour tirer, en quelque sorte, du néant ces renseignements, que l’on pourrait appeler l’histoire antéhistorique, ne sont pas sans analogie avec les ingénieux travaux des géologues, et, trouvées par non moins de sagacité et d’acutesse d’esprit, elles conduisent à des résultats aussi précis, aussi incontestables, et tels que les annales positives sont loin de les donner toujours. Ainsi, de ce qu’on rencontre l’usage du char de guerre chez tous les peuples que j’ai énumérés, on conclut, et avec toute raison, que cette mode guerrière était pratiquée par les rameaux blancs d’où sont descendus les Égyptiens, les Hindous, les Galls. En effet, l’idée de combattre en voiture n’est pas de ces notions essentielles qui, comme celles de manger et de boire, viennent indifféremment à toutes les créatures, sans consultation ni entente préalable. D’autre part, c’est une de ces découvertes compliquées qui, une fois faites et jusqu’à ce qu’elles soient remplacées par de plus heureuses, ou entravées dans leur application par des circonstances locales, persistent dans les nations et contribuent à leur luxe comme à leur force.\par
 On a pu préciser de la même manière le genre de vie des populations blanches primitives. L’examen des langues qu’on nomine indo-germaniques a fait reconnaître dans le sanscrit, le grec, le latin, les dialectes celtiques et slaves, une parfaite identité de termes pour tout ce qui touche à la vie pastorale et aux habitudes politiques. C’est en considérant les mots de près et dans leurs racines qu’on a appris de quelles idées découlaient les notions simples ou complexes que ces mots étaient chargés de reproduire. On a trouvé que, pour nommer un bœuf, un cheval, un chariot, une arme, les blancs primitifs avaient des expressions qui sont demeurées inébranlablement attachées au lexique de la plupart des langues de la même famille. Les habitudes guerrières et pastorales avaient donc chez eux de profondes racines. En même temps, on remarquait, dans toutes ces langues, la diversité des formes employées pour tout ce qui ressort de l’agriculture, comme les noms des végétaux et des instruments aratoires. Le travail de la terre est donc une invention postérieure aux séparations de la grande famille, etc.\par
 En poursuivant le même travail étymologique, on a de même connu ce que les blancs primitifs entendaient par un \emph{Dieu} ; l’idée qu’emportaient, pour eux, le mot \emph{roi}, celui de \emph{chef}. L’étude comparée des idiomes a donné, ainsi, trois grands résultats à l’histoire  : 1° la preuve de la parenté des nations blanches les plus séparées par les distances géographiques ; 2° l’état commun dans lequel ces nations vivaient antérieurement à leurs migrations ; 3° la démonstration de leur précoce sociabilité et de ses caractères.
}.\par
Voilà donc une race en possession des vérités primordiales de la religion, douée à un haut degré de la préoccupation du passé, sentiment qui la distinguera toujours et qui n’illustrera pas moins les Arabes et les Hébreux que les Hindous, les Grecs, les Romains, les Gaulois et les Scandinaves. Habile dans les principaux arts mécaniques, ayant assez médité déjà sur l’art militaire pour en faire quelque chose de plus que les rixes élémentaires des sauvages, et souveraine de plusieurs classes d’animaux soumises à ses besoins, cette race se montre à nous, placée vis-à-vis des autres familles humaines, sur un tel degré de supériorité, qu’il nous faut, dès à présent, établir, en principe, que toute comparaison est impossible par cela seul que nous ne trouvons pas trace de barbarie dans son enfance même. Faisant preuve, à son début, d’une intelli­gence bien éveillée et forte, elle domine les autres variétés incomparablement plus nombreuses, non pas encore en vertu d’une autorité acquise sur ces rivales humiliées, puisque aucun contact notable n’a eu lieu, mais déjà de toute la hauteur de l’aptitude civilisatrice sur le néant de cette faculté.\par
Le moment d’entrer en lutte arriva vers la date indiquée plus haut. Cinq mille ans pour le moins avant notre ère, le territoire occupé par les tribus blanches fut franchi. Poussées probablement par des masses parentes qui commençaient, elles-mêmes, à s’ébranler dans le nord sous la pression des peuples jaunes, les nations de cette espèce qui se trouvaient placées le plus au sud, abandonnèrent leurs demeures antiques, traversèrent les contrées basses, connues des Orientaux sous le nom de Touran \footnote{M. A. de Humboldt fait observer que les contrées à l’est de la Caspienne subissent une dépression considérable (\emph{Asie centrale}, t. I, p. 31). Le passage est intéressant ; le voici tout entier : « Ces deux grandes masses (le monde anglo-hindou et le monde russe-sibérien) ou « divisions politiques ne communiquent, depuis des siècles, que par les basses régions de la « Bactriane, je pourrais dire par la dépression du sol qui entoure l’Aral et le bord oriental de « la Caspienne entre Balkh et Astrabad, comme entre Tachkend et l’isthme de « Troukhmènes. C’est une bande de terrains, en partie très fertile, à travers laquelle l’Oxus a « tracé son cours... C’est le chemin de Delhy, de Lahore et de Kaboul à Khiva et à « Orenbourg... La dépression du sol asiatique, sur laquelle des mesures très récentes et de la « plus haute précision ont rectifié les notions, se prolonge « sans doute aussi au delà du « rivage occidental de la Caspienne ; mais en descendant du plateau de la Perse par Tebriz « et par Erivan (plateau de 600 à 700 toises d’élévation), vers Tiflis, on rencontre la chaîne « du Caucase touchant presque au bassin des deux mers et offrant une route militaire très fréquentée, qui a 7530 pieds de hauteur. »}, et, attaquant à l’ouest les races noires qui leur barraient le passage, parurent en dehors des limites qu’elles n’avaient encore jamais touchées ni même jamais vues.\par
Cette descente primordiale des peuples blancs est celle des Chamites, et dévelop­pant, ici, ce que j’indiquais quelques pages plus haut, je réclamerai contre l’habitude, peu justifiée à mon sens, de déclarer ces multitudes primitivement noires. Rien dans les témoignages anciens, n’autorise à considérer le patriarche, auteur de leur descendance, comme souillé par la malédiction paternelle, des caractères physiques des races réprou­vées. Le châtiment de son crime ne se développa qu’avec le temps, et les stigmates vengeurs ne s’étaient pas encore réalisés à cet instant où les tribus chamites se séparèrent du reste des nations noachides.\par
Les menaces mêmes dont l’auteur de l’espèce blanche, dont le père sauvé des eaux a flétri une partie de ses enfants, confirment mon opinion. D’abord, elles ne s’adressent pas à Cham lui-même, ni à tous ses descendants. Puis, elles n’ont qu’une portée morale, et ce n’est que par une induction très forcée que l’on a pu leur attribuer des consé­quences physiologiques. « Maudit soit Chanaan, dit le texte, il sera serviteur des serviteurs de ses frères \footnote{ \noindent Genèse, ch. IX, v. 25 : « Ait : Maledictus Chanaan, servus servorum erit fratribus suis. »\par
 Jamais l’expression de Chanaan n’a indiqué un peuple nègre ni même complètement noir. Elle s’applique, historiquement, à des populations métisses inclinant, sans doute, vers l’élément mélanien, mais non pas identiques avec lui, et la Vulgate a parfaitement établi le fait en reprodui­sant rigoureusement le terme hébreu (en hébreux) et non pas (en hébreux) de sorte qu’il n’est même pas possible de se méprendre au sens du passage. D’ailleurs, si l’on veut un commentaire, il se trouve clair et précis au chap. XX, v. 5, de l’Exode, où il est dit : « Ego sum Dominus Deus tuus fortis, zelotes, \emph{visitans iniquitatem patrum in filios, in tertiam et quartam generationem eorum qui oderunt me.} » La punition des coupables dans la décadence de leur famille est trop fréquemment racontée par les livres saints pour que je ne sois pas dispensé d’en fournir ici tous les exemples.\par
 Je conclus que la Bible ne déclare pas que Cham, personnellement, sera noir, ni même esclave, mais seulement que Chanaan, c’est-à-dire un des fils de Cham, sera un jour dégradé dans son sang, dans sa noblesse, et réduit à servir ses cousins. ‑ J’ajouterai encore une dernière observation. La postérité de Cham ne s’est pas bornée au seul Chanaan. Le patriarche eut encore trois fils, outre celui-là : Chus, Mesraïm et Phuth (Gen., X, 6), et le texte ne dit nullement qu’ils aient été atteints par la malédiction. N’y a-t-il pas quelque chose de singulier dans un récit qui respecte le vrai coupable et la plus grande partie de sa postérité, pour ne faire tomber les effets vengeurs du crime que sur un seul membre de la famille, Chanaan, sur celui-là même qui se trouva en compétition territoriale et religieuse avec les enfants d’Israël ? Il s’agirait donc ici bien moins d’une question physiologique que d’une haine politique.
} ».\par
Les Chamites arrivèrent ainsi flétris d’avance dans leur destinée et dans leur sang. Pourtant, l’énergie qu’ils avaient empruntée au trésor des forces particulières à la nature blanche ne leur en permit pas moins de fonder plusieurs vastes sociétés. La première dynastie assyrienne, les patriciats des cités de Chanaan, sont les monuments princi­paux de ces âges éloignés, dont le caractère se trouve, en quelque sorte, résumé dans le nom de Nemrod \footnote{M. le colonel Rawlinson pense que Nemrod est un mot collectif, participe passif régulier d’un verbe assyrien, et signifie : ceux qui sont trouvés ou les colons, les premiers possesseurs, c’est-à-dire, ici, les premiers habitants blancs de la basse Chaldée, (Rawlinson, \emph{Report of the Royal Asiatic Society}, 1852, p. XVII.)}.\par
Ces grandes conquêtes, ces courageuses et lointaines invasions, ne pouvaient être pacifiques. Elles s’exerçaient aux dépens de peuplades de la variété la plus inepte, mais aussi la plus féroce  : de celle qui appelle davantage l’abus de la contrainte. Naturelle­ment portée à résister à ces étrangers irrésistibles qui venaient la dépouiller, elle leur opposa l’incurable sauvagerie de son essence, et les obligea à ne compter que sur l’emploi incessant de leur vigueur. Elle n’était pas à convertir, puisqu’il lui manquait l’intelligence nécessaire pour être persuadée. Il fallait donc n’en pas espérer une participation réfléchie à l’œuvre civilisatrice, et se contenter de plier ses membres à devenir les machines animées appliquées au labeur social.\par
Ainsi que je l’ai déjà annoncé, l’impression éprouvée par les Chamites blancs, à la vue de leurs hideux antagonistes, est peinte des mêmes couleurs dont les conquérants hindous ont plus tard revêtu leurs ennemis locaux, frères de ceux-là. Ce sont, pour les nouveaux venus, des êtres féroces et d’une taille gigantesque. Ce sont des monstres également redoutables par leur laideur, leur vigueur et leur méchanceté. Si la première conquête fut difficile, et par l’épaisseur des masses attaquées, et par leur résistance, soit furieuse, soit stupidement inerte, le maintien des États qu’inaugurait la victoire ne dut pas exiger moins d’énergie. La compression devint l’unique moyen de gouverne­ment. Voilà pourquoi Nemrod, dont je citais le nom tout à l’heure, fut un grand chasseur devant l’Éternel \footnote{Movers, \emph{Das Phœnizische Alterthum}, t. II, 1\textsuperscript{re} partie, p. 271.}.\par
Toutes les sociétés issues de cette première immigration révélèrent le même caractère de despotisme altier et sans bornes.\par
Mais, vivant en despotes au milieu de leurs esclaves, les Chamites donnèrent bientôt naissance à une population métisse. Dès lors, la position des anciens conqué­rants devint moins éminente, et celle des peuples vaincus moins abjecte.\par
L’omnipotence gouvernementale ne pouvait pourtant rien perdre de ses préroga­tives, trop conformes, par leur nature excessive, à l’esprit même de l’espèce noire. Aussi n’y eut-il aucune modification dans l’idée qu’on se faisait de la façon et des droits de régner. Seulement, le pouvoir, désormais, s’exerça à un autre titre que celui de la supériorité du sang. Son principe fut limité à ne plus supposer que des préexcellences de familles et non plus de peuples. L’opinion qu’on avait du caractère des dominateurs commença cette marche décroissante, qui toujours s’accomplit dans l’histoire des nations mêlées.\par
Les anciens Chamites blancs allèrent se perdant chaque jour, et finirent par disparaître. Leur descendance mulâtre, qui pouvait très bien encore porter leur nom comme un titre d’honneur, devint par degrés, un peuple saturé de noir. Ainsi le voulaient les branches génératrices les plus nombreuses de leur arbre généalogique. De ce moment, le cachet physique qui devait faire reconnaître la postérité de Chanaan et la réserver à la servitude des enfants plus pieux, était à jamais imprimé sur l’ensemble des nations formées par l’union trop intime des conquérants blancs avec leurs vaincus de race mélanienne.\par
En même temps que cette fusion matérielle s’opérait, une autre toute morale avait lieu, qui achevait de séparer, à jamais, les nouvelles populations métisses de l’antique souche noble, à laquelle elles ne devaient plus qu’une partie de leur origine. Je veux parler du rapprochement entre les langages. Les premiers Chamites avaient apporté du nord-est un dialecte de cet idiome originellement commun aux familles blanches, dont il est encore aujourd’hui si facile de reconnaître les vestiges dans les langues de nos races européennes. À mesure que les tribus immigrantes s’étaient trouvées en contact avec les multitudes noires, elles n’avaient pas pu empêcher leur langage naturel de s’altérer ; et quand elles se trouvèrent alliées de plus en plus avec les noirs, elles le perdirent tout à fait. Elles l’avaient laissé envahir par les dialectes mélaniens de façon à le défigurer.\par
À la vérité, nous ne sommes pas complètement en droit d’appliquer, péremp­toirement, aux langues de Cham les réflexions que suggère ce que nous connaissons du phénicien et du libyque. Beaucoup d’éléments, développés postérieurement par les migrations sémitiques, se sont infusés dans ces idiomes métis, et on pourrait objecter que les apports nouveaux possédèrent un autre caractère que celui des langues formées d’abord par les Chamites noirs. Je ne le crois cependant pas. Ce que nous savons du chananéen, et l’étude des dialectes berbères, paraissent révéler un système commun de langage imbu de l’essence qu’on a appelée sémitique, à un degré supérieur à ce qu’en possèdent les langues sémitiques elles-mêmes, par conséquent s’éloignant davantage des formes appartenant aux langues des peuples blancs, et conservant ainsi moins de traces de l’idiome typique de la race noble. Je ne fais pas difficulté, pour ma part, de considérer cette révolution linguistique comme une conséquence de la presque identification avec les peuples noirs, et je donnerai plus bas mes raisons.\par
Le Chamite était dégénéré  : le voilà au sein de sa société d’esclaves, entouré par elle, dominé par son esprit, tandis qu’il domine lui-même sa matière, engendrant, de ses femmes noires, des fils et des filles qui portent, de moins en moins, le cachet des antiques conquérants. Cependant, parce qu’il lui reste quelque chose du sang de ses pères, il n’est pas un sauvage, il n’est pas un barbare. Il maintient debout une organi­sation sociale qui, depuis tant de siècles qu’elle a disparu, laisse encore tomber sur l’imagination du monde l’ombre de quelque chose de monstrueux et d’insensé, mais de non moins grandiose.\par
Le monde ne saurait plus rien voir de comparable, par les effets, aux résultats du mariage des Chamites blancs avec les peuples noirs. Les éléments d’une pareille alliance n’existent nulle part, et il n’est pas étonnant que, dans la production si fréquente des hybrides des deux espèces, rien ne représente plus au physique ni au moral l’énergie de la première création Si l’élément noir a généralement assez conservé de la pureté pour montrer des qualités à peu près analogues à celles de ses plus anciens types, il n’en est pas de même du blanc. L’espèce ne se retrouve nulle part dans sa valeur primitive. Nos nations les plus dégagées d’alliages ne sont que des résultats très décomposés, très peu harmoniques, d’une série de mélanges, soit noirs et blancs comme, au midi de l’Europe, les Espagnols, les Italiens, les Provençaux ; soit jaunes et Blancs comme, dans le nord, les Anglais, les Allemands, les Russes. De sorte que les métis, produits d’un père soi-disant blanc, dont l’essence originelle est déjà si modifiée, ne saurait nullement s’élever à la valeur cliniquement possédée par les Chamites noirs.\par
Chez ces hommes, l’hymen s’était accompli entre des types également et complètement armés de leur vigueur et de leur originalité propres. Le conflit des deux natures avait pu s’accuser fortement dans leurs fruits et y portait ce caractère de vigueur, source d’excès aujourd’hui impossibles. L’observation de faits contemporains en fournit une preuve concluante : lorsqu’un Provençal ou un Italien donne le jour à un hybride mulâtre, ce rejeton est infiniment moins vigoureux que lorsqu’il est né d’un père anglais. C’est qu’en effet le type blanc de l’Anglo-Saxon, quoique loin d’être pur, n’est pas du moins affaibli d’avance par des séries d’alluvions mélaniennes comme celui des peuples du sud de l’Europe, et il peut transmettre à ses métis une plus grande part de la force primordiale. Cependant, je le répète, il s’en faut que le plus vigoureux mulâtre actuel équivaille au Chamite noir d’Assyrie, qui, la lance à la main, faisait trembler tant de nations esclaves.\par
Pour présenter de ce dernier un portrait ressemblant, je ne trouve rien de mieux que de lui appliquer le récit de la Bible sur certains autres métis plus anciens encore que lui, et dont l’histoire trop obscure et en partie mythique ne doit pas trouver place dans ces pages. Ces métis sont les êtres antédiluviens donnés comme fils des Caïnites et des anges. Ici il est indispensable de se débarrasser de l’idée agréable dont les notions chrétiennes ont revêtu le nom de ces créatures mystérieuses. L’imagination chana­néenne, origine de la notion mosaïque, ne prenait pas les choses ainsi. Les anges étaient, pour elle, comme, du reste, pour les Hébreux, des messagers de la divinité, sans doute, mais plutôt sombres que doux, plutôt animés d’une grande force matérielle que représentant une énergie purement idéale. À ce titre, on se les imaginait sous des formes monstrueuses et propres à inspirer l’épouvante, non pas la sympathie \footnote{Tels étaient, par exemple, les chérubins à tête de bœuf. Gesénius les définit ainsi : « (mot « hébreu) in Hebræcorum theologia natura quædam sublimior et cœlestis cujus formam ex « humana, bovina, leonina et aquilina (quæ tria animalia cum homine potentiæ et sapientiæ « symbola sunt), compositam sibi fingebanl. » (\emph{Lexicon manuale hebraïcum et chaldaïcum}.)}.\par
Lorsque ces créatures robustes se furent unies aux filles des Caïnites, il en naquit des géants \footnote{Gen., VI. 2, 4. : « Videntes filii dei filias hominum quod essent pulchræ, acceperunt sibi « uxores ex omnibus quas elegerant... Gigantes autem crant super terram in diebus illis. « Postquam enim ingressi sunt filii Dei ad filias hominum, illæque genuerunt, isti sunt « potentes a sæculo viri famosi. »} dont on peut juger le caractère par le morceau littéraire le plus ancien, peut-être, du monde, par cette chanson, que disait à ses femmes un des descendants du meurtrier d’Abel, parent probablement bien proche de ces redoutables métis :\par
« Entendez ma voix, femmes de Lamech ; écoutez ma parole  : De même « que j’ai tué un homme pour une blessure et un enfant pour un affront, de « même la vengeance septuple de Caïn sera pour Lamech soixante-dix-sept fois septuple! \footnote{Gen., IV, 23, 24 : « Dixitque Lamech uxoribus suis Adæ et Sellæ : Audite vocem meam, uxores Lamech, auscultate sermonem meum. – Quoniam occidi virum in vulnus meum et adolescentulum in livorem meum, ‑ septuplum ultio dabitur de Caïn ; de Lamech vero septuagies septies. » ‑ Le \emph{sel} de cette composition ne consiste pas seulement dans la rudesse du sentiment. Il y a encore là plus d’orgueil que d’esprit de vengeance. Dieu, en condamnant Caïn, n’avait cependant pas voulu le punir de mort, et il l’avait couvert de sa protection, en déclarant que celui qui le tuerait serait puni au septuple. Lamech se mettait au-dessus même de son aïeul, objet de la vénération de la famille, en promettant soixante-dix-sept fois plus de châtiment à ses agresseurs.} »\par
Voilà, je m’imagine, ce qui peint le mieux les Chamites noirs, et je me laisserais aller aisément à voir un rapport étroit de similitude entre le mélange d’où ils sont sortis et l’hymen maudit des aïeules de Noé avec cet autre type inconnu que la pensée primitive relégua, non sans quelque horreur, dans un rang surnaturel.
\section[{II.2. Les Sémites.}]{II.2. \\
Les Sémites.}
\noindent Tandis que les Chamites se répandaient fort avant dans toute l’Asie antérieure et au long des côtes arabes jusque dans l’est de l’Afrique \footnote{Il est probable que très anciennement des mélanges chamites ont atteint le sang des populations cafres, vers le méridien de Monbaz.}, d’autres tribus blanches, se pressant sur leurs pas, avaient gagné, à l’ouest, les montagnes de l’Arménie et les pentes méridionales du Caucase \footnote{\emph{Movers, das Phœniz Alterth}., t. I, 2\textsuperscript{e} partie, p. 461 ; Ewald, \emph{Gesch., des Volkes Israël}, t. I, p. 332.}.\par
Ces peuples sont ceux qu’on appelle Sémites. Leur force principale paraît s’être concentrée, dans les premiers temps, au milieu des régions montagneuses de la haute Chaldée. C’est de là que sortirent, à différentes époques, leurs masses les plus vigoureuses. C’est de là que provinrent les courants dont le mélange régénéra le mieux, et pendant le plus longtemps, le sang dénaturé des Chamites, et, dans la suite, l’espèce aussi abâtardie des plus anciens émigrants de leur propre race. Cette famille si féconde rayonna sur une très grande étendue de territoires. Elle poussa, dans la direction du sud-est, les Arméniens, les Araméens, les Élamites, les Élyméens, même nom sous différentes formes \footnote{Ewald, ouvrage cité, t. I, p. 327 et passim} ; elle couvrit de ses rejetons l’Asie Mineure. Les Lyciens, les Lydiens, les Cariens lui appartiennent. Ses colonies envahirent la Crète, d’où elles revinrent plus tard, sous le nom de Philistins, occuper les Cyclades, Théra, Mélos, Cythère et la Thrace. Elles s’étendirent sur le pourtour entier de la Propontide, dans la Troade, le long du littoral de la Grèce, arrivèrent à Malte, dans les îles Lipari, en Sicile.\par
 Pendant ce temps, d’autres Sémites, les Joktanides \footnote{Id., \emph{ibid.}, t. I, p. 337.}, envoyèrent, jusqu’à l’extrême sud de l’Arabie, des tribus appelées à jouer un rôle important dans l’histoire des anciennes sociétés. Ces Joktanides furent connus de l’Antiquité grecque et latine sous le nom d’Homérites, et ce que la civilisation de l’Éthiopie ne dut pas à l’influence égyptienne, elle l’emprunta à ces Arabes qui formèrent, non pas la partie la plus ancienne de la nation, prérogative des Chamites noirs, fils de Cush, mais certainement la plus glorieuse, quand les Arabes ismaélites, encore à naître au moment où nous parlons, furent venus se placer à leurs côtés. Ces établissements sont nombreux. Ils n’épuisent cependant pas la longue liste des possessions sémitiques. Je n’ai rien dit jusqu’à présent de leurs envahissements sur plusieurs points de l’Italie, et il faut ajouter que, maîtres de la côte nord de l’Afrique, ils finirent par occuper l’Espagne en si grand nombre, qu’à l’époque romaine on y constatait aisément leur présence.\par
Une si énorme diffusion ne s’expliquerait pas, quelle que pût être d’ailleurs la fécondité de la race, si l’on voulait revendiquer pour ces peuples une longue pureté de sang. Mais, pour bien des causes, cette prétention ne serait pas soutenable. Les Chamites, retenus par une répugnance naturelle, avaient peut-être résisté quelque temps au mélange qui confondait leur sang avec celui de leurs noirs sujets. Pour soutenir ce combat et maintenir la séparation des vainqueurs et des vaincus, les bonnes raisons ne manquaient pas, et les conséquences du laisser-aller sautaient aux yeux. Le sentiment paternel devait être médiocrement flatté en ne retrouvant plus la ressem­blance des blancs dans le rejeton mulâtre. Cependant l’entraînement sensuel avait triomphé de ce dégoût, comme il en triompha toujours, et il en était résulté une population métisse plus séduisante que les anciens aborigènes, et qui présentait, avec des tentations physiques plus fortes que celles dont les Chamites avaient été victimes, la perspective de résultats, en définitive, beaucoup moins repoussants. Puis la situation n’était pas non plus la même  : les Chamites noirs ne se trouvaient pas, vis-à-vis des arrivants, dans l’infériorité où les ancêtres de leurs mères s’étaient vus en face des anciens conquérants. Ils formaient des nations puissantes auxquelles l’action des fondateurs blancs avait infusé l’élément civilisé, donné le luxe et la richesse, prêté tous les attraits du plaisir. Non seulement les mulâtres ne pouvaient pas faire horreur, mais ils devaient, sous beaucoup de rapports, exciter et l’admiration et l’envie des Sémites, encore inhabiles aux arts de la paix.\par
En se mêlant à eux, ce n’étaient pas des esclaves que les vainqueurs acquéraient, c’étaient des compagnons bien façonnés aux raffinements d’une civilisation depuis longtemps assise. Sans doute la part apportée par les Sémites à l’association était la plus belle et la plus féconde, puisqu’elle se composait de l’énergie et de la faculté initiatrice d’un sang plus rapproché de la souche blanche ; pourtant elle était la moins brillante. Les Sémites offraient des prémices et des primeurs, des espérances et des forces. Les Chamites noirs étaient déjà en possession d’une culture qui avait donné ses fruits.\par
On sait ce que c’était  : de vastes et somptueuses cités gouvernaient les plaines assyriennes. Des villes florissantes s’élevaient sur les côtes de la Méditerranée. Sidon étendait au loin son commerce, et n’étonnait pas moins le monde par ses magnificences que Ninive et Babylone. Sichem, Damas, Ascalon \footnote{ \noindent Je me sers ici de ces noms de cités célèbres sans prétendre affirmer qu’elles aient les premières servi de métropoles aux États chamites ou même sémo-chamites. Longtemps avant ces grandes villes, la Bible et les inscriptions cunéiformes nous révèlent l’existence d’autres capitales, telles que Niffer, Warka, Sanchara (probablement la Lanchara de Bérose). La fameuse ville où résidait le roi chamite Chedarlaomer, roi d’Elam (Gen., XIV), bien que moins ancienne, florissait cependant avant Ninive. (Voit le lieut.-colonel Rawlinson, \emph{Report of the Royal Asiatic Society}, 1852, p. XV-XVI.) ‑ De même la capitale de Sennacherib était à Kar-Dunyas, et non pas à Babylone (ouvr. cité, p. XXXII), ce qui est assez remarquable à cette époque, relativement basse, puisque Sennacherib régnait en 716 av. J.-C. seulement. Cependant Babylone était bâtie depuis fort longtemps ; le lieutenant-colonel Rawlinson, s’appuyant sur le 13\textsuperscript{e} verset du 23\textsuperscript{e} chap. d’Isaïe (j’avoue ne pas comprendre très bien les motifs du célèbre antiquaire), pense que l’on peut considérer le treizième siècle avant notre ère comme l’époque de fondation de cette cité. (Ouvr. cité, p. XVII.)\par
 La raison qui me porte à m’en tenir aux notions les plus répandues c’est l’état encore imparfait des connaissances modernes sur l’histoire des États assyriens. Nul doute que les découvertes de Botta, de Layard, de Rawlinson, et celles que poursuit, en ce moment, avec tant de zèle, d’énergie et d’habileté, le consul de France à Mossoul, M. Place, n’amènent, dans ce que nous savons des peuples primitifs de l’Asie, une révolution plus considérable encore et suivie de résultats plus heureux et plus brillants que celle qui fut opérée, il y a quelques années, dans les annales de l’Italie antique par les savants travaux des Niebuhr, des O. Müller, des Aufrecht. Mais nous n’en sommes encore qu’aux débuts, et il y aurait témérité à vouloir trop user de résultats, jusqu’ici fragmentaires et souvent si inattendus, si émouvants pour l’imagination la plus froide, qu’avant de les utiliser, il faut qu’une critique sévère en ait plus que constaté la valeur. Lorsque le savant colonel Rawlinson donne, d’après deux cylindres en terre cuite, l’histoire complète des huit premières années du règne de Sennacherib avec le récit de la campagne de ce monarque contre les juifs (\emph{Outlines of Assyrian history, collection from the cuneiform inscriptions}, p. XV), c’est bien le moins que nous ne cédions pas trop facilement au charme inévitable qu’exerce sur l’esprit cette autobiographie où le roi raconte sa défaite et la met en regard du récit de la Bible. Une grande réserve ne me semble pas moins obligatoire, lorsque l’infatigable érudit nous offre une découverte plus surprenante encore. Dans des tablettes en terre cuite trouvées sur le bas Euphrate et envoyées à Londres par M. Loftus, membre de la Commission mixte pour la délimitation des frontières turco-persanes, M. Rawlinson pense avoir découvert des reconnaissances du trésor d’un prince assyrien pour un certain poids d’or ou d’argent, déposé dans les caisses publiques, reconnaissances qui auraient eu, dans les mains des particuliers, un cours légal. M. Mohl, en rendant compte de cette opinion, ajoute prudemment : « Ce « serait un premier essai de valeurs de convention dans un temps où certainement personne ne « l’aurait soupçonné, et cette supposition a quelque chose de si surprenant, qu’on ose à peine espérer « qu’elle se vérifiera. » (\emph{Rapport à la Société asiatique}, 1851, p. 46.)\par
 J’espère que personne ne me blâmera d’imiter la discrétion dont un juge si compétent me donne l’exemple. Plus on fera de progrès dans la lecture des inscriptions cunéiformes, plus on découvrira de ruines dans ces vastes provinces, dont le sol inexploré parait en être couvert, plus on accomplira de miracles, j’en suis convaincu, en faisant revivre des faits déjà morts et oubliés à l’époque des Grecs. Mais c’est précisément parce qu’il y a lieu de beaucoup attendre de l’avenir, qu’il ne faut pas le compromettre en embarrassant le présent d’assertions trop hâtives, inutilement hypothétiques et souvent erronées. Je continuerai donc à me tenir de préférence sur des terrains connus et solides, et c’est pourquoi j’invoque les noms de Ninive et de Babylone comme étant ceux qui, jusqu’ici, personnifient le mieux les splendeurs assyriennes.
}, d’autres villes encore, renfermaient des populations actives habituées à toutes les jouissances de la vie. Cette société puissante se morcelait en des myriades d’États qui tous, à un degré plus ou moins complet, mais sans exception, subissaient l’influence religieuse et morale du centre d’action placé en Assyrie \footnote{Movers, das Phœniz. Alterthum, t. II, 1\textsuperscript{re} partie, p. 265 ; Ewald, \emph{Geschichte d. V. Israël}, t. I, p. 367.}. Là était la source de la civilisation ; là se trouvaient réunis les principaux mobiles des développements, et ce fait, prouvé par des considérations multiples, me fait accepter pleinement l’assertion d’Hérodote, amenant de ce voisinage les tribus phéniciennes, bien que le fait ait été contesté récemment \footnote{Movers, t. II, 1\textsuperscript{re} partie, p. 302}. L’activité chananéenne était trop vive pour n’avoir pas puisé la naissance aux sources les plus pures de l’émigration chamite \footnote{Id. \emph{ibid.}, p. 31. ‑ L’opinion de cet auteur est victorieusement réfutée par Ewald, Taber, Michaelis, etc.}.\par
Partout dans cette société, à Babylone comme à Tyr, règne avec force le goût des monuments gigantesques, que le grand nombre des ouvriers disponibles, leur servitude et leur abjection, rendaient si faciles à élever. Jamais, nulle part, on n’eut de pareils moyens de construite des monuments énormes, si ce n’est en Égypte, dans l’Inde et en Amérique, sous l’empire de circonstances et par la force de raisons absolument semblables, Il ne suffisait pas aux orgueilleux Chamites de faire monter vers le ciel de somptueux édifices ; il leur fallait encore ériger des montagnes pour servir de base à leurs palais, à leurs temples, montagnes artificielles non moins solidement soudées au sol que les montagnes naturelles, et rivalisant avec elles par l’étendue de leurs contours et l’élévation de leurs crêtes. Les environs du lac de Van \footnote{Voir les découvertes du docteur Schultz.} montrent encore ce que furent ces prodigieux chefs-d’œuvre d’une imagination sans frein, servie par un despotisme sans pitié, obéie par la stupidité vigoureuse. Ces tumulus géants sont d’autant plus dignes d’arrêter l’attention, qu’ils nous reportent à des temps antérieurs à la séparation des Chamites blancs du reste de l’espèce. Le type en constitue le monument primordial commun à toute la race. Nous le retrouverons dans l’Inde, nous le verrons chez les Celtes. Les Slaves nous le montreront également, et ce ne sera pas sans surprise qu’après l’avoir contemplé sur les bords du Jénisséi et du fleuve Amour, nous le reconnaîtrons s’élevant au pied des montagnes alléghaniennes, et servant de base aux téocallis mexicains.\par
Nulle part, sauf en Égypte, les tumulus ne reçurent les proportions puissantes que les Assyriens surent leur donner. Accompagnements ordinaires de leurs plus vastes constructions, ceux-ci les érigèrent avec une recherche de luxe et de solidité inouïe. Comme d’autres peuples, ils n’en firent pas seulement des tombeaux ; ils ne les réduisirent pas non plus au rôle de bases pleines, ils les disposèrent en palais souterrains pour servir de refuge aux monarques et aux grands contre les ardeurs de l’été.\par
Leur besoin d’expansion artistique ne se contenta pas de l’architecture. Ils furent admirables dans la sculpture figurée et écrite. Les surfaces des rochers, les versants des montagnes devinrent des tableaux immenses où ils se plurent à sculpter des personnages gigantesques et des inscriptions qui ne l’étaient pas moins, et dont la copie embrasse des volumes \footnote{Botta, \emph{Monuments de Ninive.}}. Sur leurs murailles, des scènes historiques, des cérémonies religieuses, des détails de la vie privée, entaillèrent savamment le marbre et la pierre, et servirent le besoin d’immortalité qui tourmentait ces imaginations démesurées.\par
La splendeur de la vie privée n’était pas moindre. Un immense luxe domestique entourait toutes les existences et, pour me servir d’une expression d’économiste, les États sémo-chamites étaient remarquablement consommateurs. Des étoffes variées par la matière et le tissu, des teintures éclatantes, des broderies délicates, des coiffures recherchées, des armes dispendieuses et ornées jusqu’à l’extravagance, comme aussi les chars et les meubles, l’usage des parfums, les bains de senteur, la frisure des cheveux et de la barbe, le goût effréné des bijoux et des joyaux, bagues, pendants d’oreilles, colliers, bracelets, cannes de jonc indien ou de bois précieux, enfin, toutes les exigences, tous les caprices d’un raffinement poussé jusqu’à la mollesse la plus absolue : telles étaient les habitudes des métis assyriens \footnote{Tout ce qui concernait l’élégance et le luxe délicat, ce qui était caprice, les objets de mode et, en un mot, ce qui répondait à ce que la langue commerciale d’aujourd’hui appelle l’\emph{article Paris}, se fabriquait dans les grandes capitales mésopotamiques. Voir Heeren, \emph{Ideen über die Politik, den Verkehr und den Handel der vornehmsten Vœlker der alten Welt}, t. I, p. 810 et pass.}. N’oublions pas qu’au milieu de leur élégance, et comme un stigmate infligé par la partie la moins noble de leur sang, ils pratiquaient la barbare coutume du tatouage \footnote{\emph{Wilkinson, Customs and Manners of the ancient Egyptians}, t. I, p. 386. Les peintures égyptiennes portent témoignage de ce fait curieux, et ce qui établit complètement l’origine mélanienne de la coutume qu’elles dénoncent, c’est de voir cette même coutume répandue dans toute l’Afrique et sur la côte occidentale aussi bien qu’à l’est. Pour expliquer cette particularité, Degrandpré, surpris de voir des nègres \emph{tatoués}, dit-il, \emph{en couleur, à la manière des Indiens}, fait remarquer que les naturels traversent assez souvent toute la largeur de leur continent parallèlement à l’équateur, et que, de cette façon, en peut s’expliquer que les habitants de la Guinée pratiquent ce que les gens du Congo ont pu apprendre des navigateurs de l’Inde. (Voir Pott, \emph{Verwandtschaftliches Verhæltniss der Sprachen vom Kaffer und Kongo-Stamme untereinander} dans la \emph{Zeitschrift der deutsch. morgenl. Gesellschaft}, t. II, p. 9.) C’est une démonstration un peu pénible, à laquelle je substitue celle que voici : Comme il n’y a au monde aucun peuple se tatouant au moyen de peintures, appliquées seulement sur la peau ou pénétrant sous l’épiderme par incision, qui n’appartienne, de très près, aux espèces noire ou jaune, j’en conclus que le tatouage est une habitude propre à ces deux variétés et qu’elles l’ont fait adopter aux races blanches les plus fortement mêlées à elles. Ainsi, de même que les Chamo-Sémites et les Hindous, alliés aux noirs, se sont peints, de même les Celtes alliés aux jaunes en ont fait autant par une raison toute semblable. Il faut donc considérer les tatouages comme une marque de l’origine métisse et apporter beaucoup de soin à les étudier au point de vue ethnologique. C’est ce qu’ont très bien compris les savants américains. Les formes et les caractères des dessins tracés dans une tribu du nouveau continent ou de la Polynésie, sur le visage ou le corps des guerriers, ont souvent servi à faire reconnaître la descendance, en révélant des rapports avec une autre peuplade souvent fort lointaine. Il m’a été donné, à moi-même, de remarquer le fait dans la belle collection de plâtres de M. de Froberville. Ces empreintes reproduisent des têtes de nègres de la côte orientale d’Afrique. Sur le front de plusieurs de ces spécimens, on retrouve une série de points longitudinaux relevés en saillie par un gonflement artificiel des chairs, ornement de la nature la plus bizarre, mais tout à fait identique à ce que l’on voit pratiquer à plusieurs groupes pélagiens de l’Océanie. Le savant ethnologiste, dont l’obligeance m’a mis à même de faire cette observation, n’hésite pas à y découvrir la preuve d’une identité primitive d’origine entre les deux familles barbares que sépare une mer immense.}.\par
Pour satisfaire à leurs besoins, sans cesse renaissants, sans cesse augmentant, le commerce allait fouiller tous les coins du monde, y quêter le tribut de chaque rareté. Les vastes territoires de l’Asie inférieure et supérieure demandaient sans relâche, réclamaient toujours de nouvelles acquisitions. Rien n’était pour eux ni trop beau ni trop cher. Ils se trouvaient, par l’accumulation de leurs richesses, en situation de tout vouloir, de tout apprécier et de tout payer.\par
Mais à côté de tant de magnificence matérielle, mêlée à l’activité artistique et la favorisant, de terribles indices, des plaies hideuses révélaient les maladies dégradantes que l’infusion du sang noir avait fait naître et développait d’une façon terrible. L’antique beauté des idées religieuses avait été graduellement souillée par les besoins superstitieux des mulâtres. À la simplicité de l’ancienne théologie avait succédé un émanatisme grossier, hideux dans ses symboles, se plaisant à représenter les attributs divins et les forces de la nature sous des images monstrueuses, défigurant les idées saines, les notions pures, sous un tel amas de mystères, de réserves, d’exclusions et d’indéchiffrables mythes, qu’il était devenu impossible à la vérité, refusée ainsi systé­matiquement au plus grand nombre, de ne pas finir, avec le temps, par devenir inabordable, même au plus petit. Ce n’est pas que je ne comprenne les répugnances que durent éprouver les Chamites blancs à commettre la majesté des doctrines de leurs pères avec l’abjecte superstition de la tourbe noire, et de ce sentiment on peut faire dériver le premier principe de leur amour du secret. Puis ils ne manquèrent pas non plus de comprendre bientôt toute la puissance que le silence donnait à leurs pontificats sur des multitudes plus portées à redouter la réserve hautaine du dogme et ses menaces qu’à en rechercher les côtés sympathiques et les promesses. D’autre part, je conçois aussi que le sang des esclaves, ayant, un jour, abâtardi les maîtres, inspira bientôt à ces derniers ce même esprit de superstition contre lequel le culte s’était d’abord mis en garde.\par
Ce qui primitivement avait été pudeur, puis moyen politique, finit par devenir croyance sincère, et, les gouvernants étant tombés au niveau des sujets, tout le monde crut à la laideur, admira et adora la difformité, lèpre victorieuse, invinciblement unie désormais aux doctrines et aux représentations figurées.\par
Et ce n’est pas en vain que le culte se déshonore chez un peuple. Bientôt la morale de ce peuple, suivant avec fidélité la triste route dans laquelle s’engage la foi, ne s’avilit pas moins que son guide. Il est impossible, à la créature humaine qui se prosterne devant un tronc de bois ou un morceau de pierre laidement contourné, de ne pas perdre la notion du bien après celle du beau. Les Chamites noirs avaient eu, d’ailleurs, tant de bonnes raisons pour se pervertir ! Leurs gouvernements les mettaient si directement sur la voie, qu’ils ne pouvaient y manquer. Tant que la puissance souveraine était restée entre les mains de la race blanche, l’oppression des sujets avait peut-être tourné au profit de l’amélioration des mœurs. Depuis que le sang noir avait tout souillé de ses superstitions brutales, de sa férocité innée, de son avidité pour les jouissances matérielles, l’exercice du pouvoir avait profité particulièrement à la satisfaction des instincts les moins nobles, et la servitude générale, sans devenir plus douce, s’était trouvée beaucoup plus dégradante. Tous les vices s’étaient donné rendez-vous dans les pays assyriens.\par
À côté des raffinements de luxe, que j’énumérais tout à l’heure, les sacrifices humains, ce genre d’hommage à la divinité, que la race blanche n’a jamais pratiqué que par emprunt aux habitudes des autres espèces, et que la moindre infusion nouvelle de son propre sang lui a fait aussitôt maudire, les sacrifices humains déshonoraient les temples des cités les plus riches et les plus civilisées. À Ninive, à Tyr, et plus tard à Carthage, ces infamies furent d’institution politique, et ne cessèrent jamais de s’accom­plir avec le cérémonial le plus imposant. On les jugeait nécessaires à la prospérité de l’État.\par
Les mères donnaient leurs enfants pour être éventrés sur les autels. Elles s’enorgueillissaient à voir leurs nourrissons gémir et se débattre dans les flammes du foyer de Baal. Chez les dévots, l’amour de la mutilation était l’indice le plus estimé du zèle. Se couper un membre, s’arracher les organes de la virilité, c’était faire œuvre pie. Imiter, de plein gré, sur sa personne les atrocités que la justice civile exerçait envers les coupables, s’abattre le nez et les oreilles, et se consacrer tout sanglant, dans cet équipage, au Melkart Tyrien ou au Bel de Ninive, c’était mériter les faveurs de ces abominables fétiches.\par
Voilà le côté féroce ; passons au dépravé. Les turpitudes que, bien des siècles après, Pétrone décrivait dans Rome, devenue asiatique, et celles dont le célèbre roman d’Apulée, d’après les fables milésiennes, faisait matière à badinage, avaient droit de cité chez tous les peuples assyriens. La prostitution, recommandée, honorée et pratiquée dans les sanctuaires, s’était propagée au sein des mœurs publiques, et les lois de plus d’une grande ville en avaient fait un devoir religieux et un moyen naturel et avouable de s’acquérir une dot. La polygamie, pourtant bien jalouse et terrible dans ses soupçons et ses vengeances, ne s’armait d’aucune délicatesse à cet égard. Le succès vénal de la fiancée ne jetait sur le front de l’épouse l’ombre d’aucun opprobre.\par
Lorsque les Sémites, descendus de leurs montagnes, étaient apparus, 2.000 ans avant Jésus-Christ \footnote{Je donne ici la date indiquée par Movers (\emph{Das Phœnizische Alterthum}, t. II, 1\textsuperscript{re}, partie, p. 259). Lassen (\emph{Indische Alterthumskunde}, t. I, p. 752) fait mention d’une dynastie existant à cette époque, mais ne se prononce pas sur son origine ethnique. Le colonel Rawlinson (\emph{Outlines of Assyrian history}, p. XV) ne connaît pas d’empire sémitique avant le treizième siècle qui a précédé notre ère. C’est alors qu’il trouve dans les inscriptions la mention d’un roi nommé honorifiquement Derceto, ou Sémiramis, mais dont il n’a pu encore déchiffrer le nom véritable. Il pense que Ninive a été construite sous ce monarque. M. Rawlinson me paraît ici prendre la quatrième dynastie de Lassen (\emph{Ind. Alterth}., I, p. 752) et de Movers (loc. cit.) pour la première. Dans tous les cas, sa date est trop basse et ne concorde pas avec la chronologie biblique.}, au milieu de la société chamite et l’avaient même, dans la basse Chaldée \footnote{Les inscriptions cunéiformes et la Genèse s’accordent à signaler l’établissement primitif d’un État sémite dans la basse Chaldée, ou dans le pays voisin, la Susiane. Longtemps, le lieu d’origine de leur race, c’est-à-dire la haute Chaldée, la région des montagnes, fut pour les souverains sémites de l’Assyrie un point dangereux d’où sortaient des compétiteurs qu’il fallait mater d’avance, et je crois facilement à l’assertion de M. Rawlinson, qui remarque qu’un des plus illustres conquérants de la dynastie que je persiste à considérer comme la quatrième, monarque dont le nom paraît devoir se lire Amak-bar-bethkira, dirigea l’effort de ses armes vers les sources du Tigre et de l’Euphrate, en Arménie et dans toute la contrée septentrionale avoisinante. (\emph{Outlines of Assyrian history}, p. XXIII.)}, soumise à une dynastie issue de leur sang, les nouveaux principes blancs jetés au milieu des masses avaient dû régénérer et régénérèrent, en effet, les nations dans lesquelles ils furent infusés. Mais leur rôle ne fut pas complètement actif. C’était chez des métis et des lâches qu’ils arrivaient, non pas chez des barbares. Ils auraient pu tout détruire, s’il leur avait plu d’agir en maîtres brutaux. Beaucoup de choses regrettables auraient péri : ils firent mieux. Ils usèrent de l’admirable instinct qui jamais n’a abandonné l’espèce, et, donnant de loin un exemple que, plus tard, les Germains n’ont pas manqué de suivre, ils s’imposèrent l’obligation d’étayer la société vieillie et mourante à laquelle venait s’associer la jeunesse de leur sang. Pour y parvenir, ils se mirent à l’école de leurs vaincus et apprirent ce que l’expérience de la civilisation avait à leur enseigner. À en juger par l’événement, leurs succès ne laissèrent rien à souhaiter. Leur règne fut plein d’éclat et leur gloire si brillante, que les collecteurs grecs d’anti­quités asiatiques leur ont fait l’honneur de la fondation de l’empire d’Assyrie, dont ils n’étaient que les restaurateurs. Erreur bien honorable pour eux et qui donne, tout à la fois, la mesure de leur goût pour la civilisation et de la vaste étendue de leurs travaux.\par
Dans la société chamite, aux destinées de laquelle ils se trouvèrent dès lors présider, ils apparaissent dans des fonctions bien multipliées. Soldats, matelots, ouvriers, pas­teurs, rois, continuateurs des gouvernements auxquels ils se substituaient, ils acceptèrent la politique assyrienne en ce qu’elle avait d’essentiel. Ils furent ainsi amenés à consacrer une part de leur attention aux intérêts du commerce.\par
Si l’Asie antérieure était le grand marché du monde occidental et son point principal de consommation, la côte de la Méditerranée se présentait comme l’entrepôt naturel des denrées tirées des continents d’Afrique et d’Europe, et le pays de Chanaan, où se concentrait l’activité intellectuelle et mercantile des Chamites maritimes, devenait un point très intéressant pour les gouvernements et les peuples assyriens. Les Sémites babyloniens et ninivites l’avaient compris à merveille. Tous leurs efforts tendaient donc à dominer, soit directement, soit par voie d’influence, sur ces peuples habiles. Ceux-ci, de leur côté, s’étaient toujours efforcés de maintenir leur indépendance politique vis-à-vis des dynasties anciennes auxquelles la victoire avait substitué le nouveau rameau blanc. Pour modifier cet état de choses, les conquérants chaldéens engagèrent une suite de négociations et de guerres le plus souvent heureuses, qui ont rendu célèbre le génie de leur race, sous le nom caractéristique et dédoublé par l’histoire des reines Sémiramis \footnote{Les Assyriens ont occupé trois fois la Phénicie  : la première fois, 2,000 ans avant J.-C. ; la seconde, vers le milieu du treizième siècle ; la troisième, en 750. (Movers, \emph{Das Phœn. Alterth}, t. II, 1\textsuperscript{re} partie, p. 259.)}.\par
Toutefois, parce que les Sémites se trouvaient mêlés à des populations civilisées, leur action sur les villes chananéennes ne s’exerça pas uniquement par la force des armes et la politique. Doués d’une grande activité, ils agirent individuellement autant que par nations, et ils pénétrèrent en très grand nombre et pacifiquement dans les campagnes de la Palestine, aussi bien que dans les murs de Sidon et de Tyr, en qualité de soldats mercenaires, d’ouvriers, de marins. Ce mode paisible d’infiltration n’eut pas de moins grands résultats que la conquête, pour l’unité de la civilisation asiatique et l’avenir des États phéniciens \footnote{C’est ainsi qu’il faut comprendre l’histoire mythique de Sémiramis, personnification d’une invasion chaldéenne. Avant d’être reine, elle avait commencé par être servante. (Movers, \emph{Das Phœnizische Alterthum}, t. II, 1\textsuperscript{re} partie, p. 261.)}.\par
La Genèse nous a conservé une relation aussi curieuse qu’animée de la façon dont s’accomplissaient les déplacements paisibles de certaines tribus ou, pour mieux dire, de simples familles sémitiques. Il est une de celles-ci que le Livre saint prend au milieu des montagnes chaldéennes, promène de provinces en provinces, et dont il nous fait voir les misères, les travaux, les succès jusque dans les moindres détails. Ce serait manquer à notre sujet que de ne pas utiliser des renseignements si précieux.\par
La Genèse, donc, nous apprend qu’un homme de la race de Sem, de la branche arménienne d’Arphaxad, de la nation si prolifique de Hebr, vivait dans la haute Chaldée, au pays montagneux d’Ur ; que cet homme conçut un jour la pensée de quitter son pays pour aller habiter la terre de Chanaan \footnote{Gen., XI, 10 : « Sem... genuit Arphaxad... 12. Arphaxad ... genuit... Sale... 14. Sale genuit Hebr... 16. Hebr genuit Phaleg... 18. Phaleg... genuit Reu ... 20. Reu genuit Sarug... 22. Sarug... genuit... Nachor... 24. Nachor... genuit Thare. »}. Le Livre saint ne nous dit pas quelles raisons puissantes avaient dicté la résolution du Sémite. Ces raisons étaient graves, sans doute, puisque le fils de l’émigrant défendit plus tard à sa race de se rapatrier jamais, bien qu’en même temps il commandât à son héritier de choisir une épouse dans le pays de sa parenté \footnote{Gen., XXIV, 6 : « Cave, ne quando, reducas filium meum illuc. »}.\par
Tharé (c’est le nom du voyageur), ayant pris le parti du départ, réunit ceux des siens qui devaient l’accompagner, et se mit en chemin avec eux. Les parents dont il s’entourait étaient Abram, son fils aîné ; Saraï, sa fille d’un autre lit, femme d’Abram \footnote{Gen., XX, 12 : « Alia autem et vere soror mea est, filia patris mei, et non filia matris meæ, « et duxi eam in uxorem. »}, et Loth, son petit-fils, dont le père, Aran, était mort quelques années en çà \footnote{Gen., XI, 31 : « Tulit itaque Thare Abram filium suum, et Loth filium Aran, filium filii sui, « et Saraï nurum suam, uxorem Abram, filii sui, et eduxit eus de Ur Chaldæorum ut irent in « terram Chanaan... » ‑ 28 : « Mortuusque est Aran ante Thare, patrem suum, in terra nativitatis suæ in Ur « Chaldæorum. »}. À ce groupe de maîtres se joignaient des esclaves, en bien petit nombre, car la famille était pauvre, et quelques chameaux et chamelles, des ânes, des vaches, des brebis, des chèvres.\par
Le motif pour lequel Tharé avait choisi le Chanaan comme terme de son voyage est facile à deviner. Il était berger comme ses pères, et ne s’expatriait pas avec l’intention de changer d’état \footnote{Gen., XLVI, 3... : « Responderunt : Pastores ovium sumus servi tui, et nos, et patres « nostri. »}. Ce qu’il allait chercher, c’était une terre neuve, abondante en pâturages, et où la population fût assez clairsemée pour qu’il y pût à son aise promener ses troupeaux et les multiplier. Tharé appartenait donc à la classe la moins aventureuse de ses concitoyens.\par
Il était d’ailleurs très vieux lorsqu’il quitta la haute Chaldée. À 70 ans, il avait eu son fils Abram, et, au moment du départ, ce fils était marié. Si Tharé nourrissait l’espoir de conduire bien loin sa caravane, cet espoir fut déçu. Le vieillard expira à Haran, avant d’avoir pu sortir de la Mésopotamie \footnote{Gen., XI, 32 : « Et facti sunt dies Thare ducentorum quinque annorum et mortuus est in « Haran. »}. Les siens marchaient d’ailleurs fort lentement et comme gens préoccupés, avant tout, de laisser paître leurs troupeaux et de ne pas les fatiguer. Lorsque les tentes étaient plantées en un lieu favorable, elles y restaient jusqu’à ce que les puits fussent à sec et les prés tondus.\par
Abram, devenu le chef de l’émigration, avait vieilli sous la tutelle de son père. Il avait 75 ans quand la mort de ce dernier l’émancipa, et il devenait chef à un moment où il n’avait pas à se plaindre de l’être. Le nombre des esclaves s’était augmenté comme aussi celui des troupeaux \footnote{Gen., XII, 5 : «Tulit... universam substantiam, quam possederant, et animas, quas fecerant « in Haran. »}. Ce qui ne laissait pas que d’avoir aussi quelque importance, une fois sorti des pays assyriens et entré dans la terre quasi-déserte de Chanaan, le pasteur sémite n’aperçut autour de son campement que des nations trop faibles pour l’inquiéter.\par
Des tribus de nègres aborigènes, des peuplades chamitiques, un petit nombre de groupes sémitiques, émigrant comme lui, quoique beaucoup plus anciennement arrivés dans la contrée, c’était tout, et le fils de Tharé qui, dans le pays d’Ur, n’avait compté, selon toute vraisemblance, que pour un très mince personnage, se trouva être, dans cette nouvelle patrie, un grand propriétaire, un homme considérable, presque un roi \footnote{Gen., XXIII, 6 : « Audi nos, domine, princeps Dei es apud nos. »}. Il en arrive ainsi, d’ordinaire, à ceux qui, abandonnant à propos une terre ingrate, portent dans un pays neuf du courage, de l’énergie et la résolution de s’agrandir.\par
Aucune de ces qualités ne manquait à Abram. Il ne forma pas d’abord un établis­sement fixe. Dieu lui avait promis de le rendre un jour maître de la contrée et d’y établir les générations sorties de ses reins. Il voulut connaître son empire. Il le parcourut tout entier. Il contracta des alliances utiles avec plusieurs des nomades qui l’exploitaient comme lui \footnote{Gen., XIV, 13 : « Nunciavit Abram Hebræo qui habitabat in convalle Mambre Amorrhæi, « fratris Eschol et fratris Aner ; hi enim pepigerant fœdus cum Abram. » ‑ « XXI, 27... « Percusseruntque ambo (cum Abimelech) fœdus. »}. Il descendit même en Égypte ; bref, quand il approcha du terme de sa carrière, il était puissant, il était riche. Il avait gagné beaucoup d’or et d’esclaves, beaucoup de troupeaux. Il était surtout devenu l’homme du pays, et il pouvait le juger ainsi que les peuples qui l’habitaient.\par
Ce jugement était sévère. Il avait bien connu les mœurs brutales et abominables des Chamites. Ce qui était arrivé à Sodome et Gomorrhe lui avait paru hautement mérité par les crimes des deux villes où Dieu lui avait prouvé qu’il ne se trouvait pas dix honnêtes gens \footnote{Gen., XVIII, 32 : « Et dixit (Deus) : Non delebo propter decem. »}. Il ne voulut pas que sa descendance fût souillée, dans le seul rameau qui lui tînt à cœur par une parenté avec des races si perverties, et il commanda à son intendant d’aller quérir, dans le pays natal de sa tribu, une femme de sa parenté, une fille de Bathuel, fils de Melcha et de Nachor \footnote{Gen., XIV, 24... : « Filia sum Bathuelis, filii Nachor, quem peperit ei Melcha. »}, par conséquent sa petite-nièce. Jadis on lui avait fait savoir la naissance de cette enfant \footnote{Gen., XXII, 20 : « His ira gestis, nunciatum est Abrahæ, quod Melcha quoque genuisset filios Nachor fratri suo. »}. Ainsi, à ces époques primitives, l’émigration ne rompait pas tous les liens entre les Sémites absents de leurs montagnes et les membres de leurs familles qui avaient continué d’y habiter. Les nouvelles traversaient les plaines et les rivières, volaient de la maison chaldéenne à la tente errante du Chanaan, et circulaient à travers de vastes contrées morcelées entre tant de souverainetés diverses. C’est un exemple et une preuve de l’activité de vie et de la communauté d’idées et de sentiments qui embrassaient le monde chamo-sémitique.\par
Je ne veux pas pousser plus avant les détails de cette histoire : on les connaît assez. On sait que les Sémites abrahamides finirent par se fixer à demeure dans le pays de la Promesse. Ce que je veux seulement ajouter, c’est que les scènes du premier établisse­ment, comme celles du départ et des hésitations qui précédèrent, rappellent d’une manière frappante ce que montrent, de nos jours, tant de familles irlandaises ou allemandes sur la terre d’Amérique. Quand un chef intelligent les conduit et dirige leurs travaux, elles réussissent comme les enfants du patriarche. Lorsqu’elles sont mal inspirées, elles échouent et disparaissent comme tant de groupes sémitiques dont la Bible nous laisse par éclairs entrevoir les désastres. C’est la même situation ; les mêmes sentiments s’y montrent dans des circonstances toujours analogues. On y voit persister au fond des cœurs cette touchante partialité à l’égard de la patrie lointaine, vers laquelle, pour rien au monde, on ne voudrait cependant rétrograder. C’est une joie semblable d’en recevoir des nouvelles, le même orgueil attaché à la parenté qu’on y conserve ; en un mot, tout est pareil.\par
J’ai montré une famille de pasteurs assez obscurs, assez humbles. Ce n’était pas là ce qui faisait surtout l’importance des émigrations sémitiques isolées dans les États assyriens ou chananéens. Ces bergers vivaient trop pour eux-mêmes et n’étaient pas d’une utilité assez directe aux populations visitées par eux. Il est donc tout simple que ceux de leurs frères qui avaient embrassé le métier des armes et se montraient experts dans cette utile profession fussent plus recherchés et plus remarqués.\par
Un des traits principaux de la dégradation des Chamites, et la cause la plus apparente de leur chute dans le gouvernement des États assyriens, ce fut l’oubli du courage guerrier et l’habitude de ne plus prendre part aux travaux militaires. Cette honte, profonde à Babylone et à Ninive, ne l’était guère moins à Tyr et à Sidon. Là, les vertus militaires étaient négligées et méprisées par ces marchands, trop absorbés dans l’idée de s’enrichir. Leur civilisation avait déjà trouvé les raisonnements dont les patriciens italiens du moyen âge se servirent plus tard pour déconsidérer la profession du soldat \footnote{Ewald, \emph{Gesch. d. V. Israël}, I, 294. Les Carthaginois ne se montrèrent pas plus militaires que les Tyriens. Ils employaient des stipendiés.}.\par
Des troupes d’aventuriers sémites s’offrirent en foule à combler la lacune que les idées et les mœurs tendaient à rendre, chaque jour, plus profonde. Ils furent acceptés avec empressement. Sous les noms de Cariens, de Pisidiens, de Ciliciens, de Lydiens, de Philistins, coiffés de casques de métal, sur le front desquels leur coquetterie martiale inventa de faire flotter des panaches, vêtus de tuniques courtes et serrées, cuirasses, le bras passé dans un bouclier rond, ceints d’une épée qui dépassait la mesure ordinaire des glaives asiatiques et portant en main des javelots, ils furent chargés de la garde des capitales et devinrent les défenseurs des flottes \footnote{ \noindent Ewald, ouvrage cité, t. I, p. 293 et pass. Ces troupes mercenaires jouèrent un très grand rôle dans tous les États chamites et sémites d’Asie et d’Afrique. Les Égyptiens mêmes en enrôlaient. Au temps d’Abraham, les petites principautés de la Palestine se confiaient sur elles de leur défense. Phicol, que la Genèse appelle le \emph{chef de l’armée d’Abimélech} (mot hébreu) Gen., XXI, 22), était probablement un condottiere de cette espèce.\par
 Plus tard, la garde de David fut aussi composée de Philistins. Tout cela prouve combien les mœurs générales étaient peu militaires
}. Leurs mérites étaient moins grands toutefois que l’énervement de ceux qui les payaient \footnote{Ewald, Id. \emph{ibid.}, t. I, p. 294.}. La très haute noblesse phéni­cienne était la seule partie de la nation qui, quelque peu fidèle aux souvenirs de ses pères, les grands chasseurs de l’Éternel, eût gardé l’habitude de porter les armes. Elle aimait encore à suspendre ses boucliers, richement peints et dorés, aux sommets des grandes tours et à embellir ses villes de cette parure brillante qui au dire des témoi­gnages, les faisait resplendir de loin comme des étoiles \footnote{Isaïe.}. Le reste du peuple travaillait. Il jouissait des produits de son industrie et de son commerce. Quand la politique réclamait quelque coup de vigueur, une colonisation, une émigration, les rois et les conseils aristocratiques, après avoir enlevé l’écume de leurs populations par une presse forcée, lui donnaient pour gardes et pour soutiens des Sémites ; tandis que quelques rejetons des Chamites noirs, se mettant à la tête de ce mélange, tantôt commandaient temporairement, tantôt allaient, au delà des mers, former le noyau d’un nouveau patriciat local et créer un État modelé sur les habitudes politiques et religieuses de la mère patrie.\par
De cette façon, les bandes sémites pénétraient partout où les Chamites avaient de l’action. Elles ne se séparaient pas, pour ainsi dire, de leurs vaincus, et le cercle de ces derniers, leur milieu, leur puissance étaient également les leurs. Les blancs de la seconde alluvion semblaient, en un mot, n’avoir pas d’autre mission à remplir que de prolonger autant que possible, par l’adjonction de leur sang, demeuré plus pur, l’antique établis­sement de la première invasion blanche dans le sud-ouest.\par
On dut croire longtemps que cette source régénératrice était inépuisable. Tandis que, vers le temps de la première émigration des Sémites, quelques-unes des nations arianes, autres tribus blanches, s’établissaient dans la Sogdiane et le Pendjab actuel, il arrivait que deux rameaux étaient détachés de celles-ci. Les peuples arians-helléniques et arians-zoroastriens, cherchant une issue pour gagner l’ouest, pressaient avec force sur les Sémites, et les contraignaient d’abandonner leurs vallées montagneuses pour se jeter dans les plaines et descendre vers le midi. Là se trouvaient les plus considérables des États fondés par les Chamites noirs.\par
Il est difficile de savoir d’une manière exacte si la résistance opposée aux envahis­seurs helléniques fut bien vigoureuse dans son malheur. Il ne le semble pas. Les Sémites, supérieurs aux Chamites noirs, n’étaient cependant pas de taille à lutter contre les nouveaux venus. Moins pénétrés par les alliages mélaniens que les descendants de Nemrod, ils étaient cependant infectés dans une grande mesure, puisqu’ils avaient abandonné la langue des blancs pour accepter le système issu de l’hymen de ses débris avec les dialectes des noirs, système qui nous est connu sous le nom très discutable de sémitique.\par
La philologie actuelle divise les langues sémitiques en quatre groupes principaux \footnote{Gesenius, \emph{Geschichte der hebraeischen Sprache und Schrift}, p. 4} : le premier contient le phénicien, le punique et le libyque, dont les dialectes berbères sont des dérivés \footnote{Les nations berbères et amazighs, d’origine sémitique, s’étendent très avant au sud, dans le Sahara africain, et, dans l’ouest, jusqu’aux îles Canaries. Les Guanches étaient des Berbères. Les invasions sémitiques se sont répétées sur le littoral occidental de l’Afrique pendant mille ans au moins. (Movers,\emph{ Das Phœnizische Alterthum}, t. II, 2\textsuperscript{e} partie, p. 363 et pass.)} ; le second renferme l’hébreu et ses variations \footnote{Gesénius, \emph{Hebraeische Grammatik}, l6\textsuperscript{e} édition, 1851, p. 12. On n’a que peu d’indices de l’existence de dialectes hébraïques. Les Ephraïmites donnaient au \emph{Schin} la prononciation du \emph{Sin} ou du \emph{Samech}. Il paraît aussi, suivant Néhémie, qu’il y avait un langage particulier à Asdod.} ; le troisième, les branches araméennes ; le quatrième, l’arabe, le gheez et l’amharique.\par
À considérer le groupe sémitique dans son ensemble et en faisant abstraction des mots importés par des mélanges ethniques postérieurs avec des nations blanches, on ne peut pas affirmer qu’il y ait eu séparation radicale entre ce groupe et ce qu’on nomme les langues indo-germaniques, qui sont celles de l’espèce d’où sont sortis, incontesta­blement, les pères des Chamites et de leurs continuateurs.\par
Le système sémitique présente, dans son organisme, des lacunes remarquables. Il semblerait que, lorsqu’il s’est formé, ses premiers développements ont rencontré autour d’eux, dans les langues qu’ils venaient remplacer, de puissantes antipathies dont ils n’ont pas pu complètement triompher. Ils ont détruit les obstacles sans pouvoir ferti­liser leurs restes, de sorte que les langues sémitiques sont des langues incomplètes \footnote{Gesenius les définit ainsi : 1° Parmi les consonnes, beaucoup de gutturales ; les voyelles ne jouent qu’un rôle très subordonné ; 2° la plupart des racines, trilittères ; 3° dans le verbe, deux temps seulement ; une régularité singulière quant à la formation des modes ; 4° dans le nom, deux genres, sans plus ; des désignations de cas d’une extrême simplicité ; 5° dans le pronom, tous les cas obliques déterminés par des affixes ; 6° presque aucun composé ni dans le verbe ni dans le nom (excepté dans les noms propres) ; 7° dans la syntaxe, une simple juxtaposition des membres de la phrase, sans grande coordination périodique. (\emph{Hebraeische Grammatik}, t. I, p. 3.)}.\par
Ce n’est pas uniquement par ce qui leur fait défaut qu’on peut constater en elles ce caractère, c’est aussi par ce qu’elles possèdent. Un de leurs traits principaux, c’est la richesse des combinaisons verbales. Dans l’arabe ancien, les formes existent pour quinze conjugaisons dans lesquelles un verbe idéal peut passer. Mais ce verbe, comme je le dis, est idéal, et aucun des verbes réels n’est apte à profiter de la facilité de flexion ni de la multiplicité de nuances qui lui sont offertes par la théorie grammaticale \footnote{Sylvestre de Sacy, \emph{Grammaire arabe}, 2\textsuperscript{e} édition, t. I, p. 125 et passim. ‑ Ce savant philologue, contrairement à l’avis de plusieurs grammairiens nationaux, trouve l’emploi des dernières formes si rare, qu’il réduit le nombre total à treize, en y comprenant la conjugaison radicale du primitif trilittère.}. Il y a certainement, au fond de la nature de ces langues, quelque chose d’inconnu qui s’y oppose. Il s’ensuit que tous les verbes sont défectueux et que les irrégularités et les exceptions abondent, Or, comme on l’a bien démontré, toute langue a le complément de ce qui lui manque dans l’opulence plus logique de quelque autre à laquelle elle a fait ses emprunts imparfaits \footnote{M. Prisse d’Avennes a récemment fait une très heureuse application de ce principe, dans son examen de la grammaire persane de M. Chodzko. Voir \emph{Revue orientale}.}.\par
Le complément du système sémitique paraît se rencontrer dans les langues africaines. Là, on est frappé de retrouver tout entier l’appareil des formes verbales, si saillant dans les idiomes sémitiques, avec cette grave différence que rien n’y est stérile ; tous les verbes passent, sans difficulté, par toutes les conjugaisons \footnote{Pott, Verwandtschaftliches \emph{Verhæltniss der Sprachen vom Kafferund Kongo-Stamme}, p. 11, p. 25. « Noch erwæhne ich hier behuf allgemeinerer Charakterisirungs gegenwærtiger « Idiome ihre Ueberfülle an dem, was die semitische Grammatik unter \emph{Conjugationen} « versteht ; ich meine die Menge besonderer Verbal-formen, welche eigentümliche « Begriffsabschattungen und Nebenbezeichnungen des im jedesmaligen Verbum liegenden « Grundgedankens abgeben und darstellen. Diese Conjugationen entshehen aber, in der « Regel, durch Zusætze hinten an der Wurzel. » Et page 138 : « Es giebt gar keine « Wurzelverba, die nicht æhnlicher Modificationen faehig wären ; und vermittelst gewisser « Partikeln oder Zusætze zeigt ein jeder dieser Verba, und alle daraus abgeleiteten, an, ob « die Handlung, die sie ausdrücken, selten oder haüfig ist ; ob sich Schwierigkeit, « Leichtigkeit, Uebermaas oder andere Unterschiede dabei finden. »}. D’autre part, on n’y trouve plus de ces racines dont la parenté visible avec l’indo-germanique trouble singulièrement les idées de ceux qui veulent faire du groupe sémitique un système entièrement original, absolument isolé des langues de notre espèce \footnote{Ce qui n’est pas l’opinion de M. Rawlinson. Voir journal of the R. A. Society, t. XIX art. 1, p. XXIII, la note sur le pronom \emph{kaga} de l’inscription de Bi-Soutoun et le rapprochement qu’en fait le savant colonel avec le mot pouschtou \emph{haga} et le latin \emph{hic}. ‑ Voir encore, pour les affinités indo-germaniques de l’assyrien, le travail de Rawlinson, précité, p. XCV. Il n’est plus douteux désormais que la plus ancienne classe d’inscriptions cunéiformes recouvre une langue sémitique. MM. Westergaard et de Saulcy, feu M. Burnouf, ont mis le fait hors de question. Et à ce propos, qu’il me soit permis de déposer ici l’expression des profonds regrets que la perte prématurée de M. Burnouf inspire à tous les amis de la science. Homme rare, d’une érudition inouïe, d’une sagacité qui tenait du prodige, d’une prudence merveilleuse, l’Angleterre et l’Allemagne nous l’enviaient justement. Il avait fait, sur les écritures assyriennes, des travaux préparatoires qu’il n’a pas eu le temps de terminer et dont le fruit est ainsi perdu pour nous. Peut-être se passera-t-il bien du temps avant que la place éminente de ce grand esprit soit occupée de nouveau.}. Pour les idiomes nègres, pas de trace, pas de soupçon possible d’une alliance quelconque avec les langues de l’Inde et de l’Europe ; au contraire, alliance intime, parenté visible avec celles de l’Assyrie, de la Judée, du Chanaan et de la Libye.\par
Je parle ici des langues de l’Afrique orientale. On était déjà bien d’avis que le gheez et l’amharique, parlés en Abyssinie, sont franchement sémitiques, et, d’un commun accord, on les rattachait, purement et simplement, à la souche arabe \footnote{Ewald, \emph{Zeitschrift für die Kunde des Morgenlandes, Ueber die Saho-Sprache in Æthiopien}, t. V, p. 410.}, Mais voilà que la liste s’allonge, et dans les nouveaux rameaux linguistiques qu’il faut, bon gré mal gré, rattacher au nom de Sem, il se manifeste des caractères spéciaux qui forcent de les constituer à part de l’idiome des Cushites, des Joktanides et des Ismaélites. En première ligne se présentent le tögr-jana et le tögray ; puis la langue du Gouraghé au sud-ouest, l’adari dans le Harar, le gafat à l’ouest du lac Tzana, l’ilmorma, en usage chez plusieurs tribus gallas, l’afar et ses deux dialectes ; le saho \footnote{Les Sahos habitent non loin de Mossawa, ou mieux Massowa (alphabet étranger) sur la mer Rouge. Jusqu’à d’Abbadie, on les avait toujours confondus tantôt avec les Gallas, tantôt avec les Danakils. (Ewald, \emph{Ueber die Saho-Sprache}, t. v, p. 412.)}, le ssomal, le sechuana et le wanika \footnote{Ewald, loc. cit., p. 422, pense que le saho s’est \emph{séparé} des autres langues sémitiques dans une antiquité incommensurable. Il se sert de ce mot séparé, parce qu’il part de la supposition que le foyer sémitique est en Asie. Cependant, frappé du monde d’idées que soulève l’examen des langues noires, il s’écrie : « Quelles clartés nouvelles nous sont présentées par l’existence de pareilles « langues sur le continent africain, au point de vue de l’histoire primitive des peuples et des idiomes « sémitiques ! » M. Ewald ne se trompe pas, c’est toute une révélation.}. Toutes ces langues présentent des caractères nettement sémitiques. Il faut leur adjoindre encore le suahili, qui ouvre à son tour un autre coin de l’horizon.\par
C’est une langue cafre, et le peuple qui en parle les dialectes, jadis borné, dans l’opinion des Européens, aux territoires les plus méridionaux de l’Afrique, s’étend main­tenant, pour nous, 5° plus au nord, jusque par delà Monbaz \footnote{Pott, ouvr. cité, t. II, p. 8.}. Il atteint l’Abyssinie, confesse, lui noir et non pas nègre, une communauté fondamentale d’idiome avec des tribus purement nègres, telles que les Suahilis proprement dits, les Makouas et les Monjous. Enfin, les Gallas parlent tous des dialectes qui se rapprochent du cafre \footnote{Pott, ouvr. cité, \emph{loc. cit}.}.\par
Ces observations ne s’arrêtent pas là. On est en droit d’y ajouter ce dernier mot, de la plus haute importance : tout le continent d’Afrique, du sud au nord et de l’est à l’ouest, ne connaît qu’une seule langue, ne parle que des dialectes d’une même origine. Dans le Congo comme dans la Cafrerie et l’Angola, sur tout le pourtour des côtes, on retrouve les mêmes formes et les mêmes racines \footnote{Cette opinion, basée sur les travaux des missionnaires et des voyageurs, et en particulier ceux de d’Abbadie et de Krapf, trouve de vigoureux propagateurs dans M. de la Gabelentz, \emph{Zeitschrift d. m. Gesellsch}., t. I, p. 238 ; M. Ewald, dans son beau mémoire sur la langue saho ; M. Krapf, directement, dans un essai intitulé  : \emph{Von der afrikanischen Ostküste} (même recueil, t. III, p. 311), et M. Pott, dont l’autorité est si grande en un pareil sujet. Ritter et Carus partagent le même avis (Erdkunde ; \emph{Ueber ungleiche Befæhigung der Menschbeitsstæmme}, p. 34.)}. La Nigritie, qui n’a pas encore été étudiée, et le patois des Hottentots, restent, provisoirement, en dehors de cette affirmation, mais ne la réfutent pas.\par
Maintenant, récapitulons. 1° Tout ce qu’on connaît des langues de l’Afrique, tant de celles qui appartiennent aux nations noires que de celles qui sont parlées par les tribus nègres, se rapporte à un même système ; 2° ce système présente les caractères princi­paux du groupe sémitique dans un plus grand état de perfection que dans ce groupe même ; 3° plusieurs des langues qui en ressortent sont classées hardiment, par ceux qui les étudient, dans le groupe sémitique.\par
En faut-il davantage pour reconnaître que ce groupe, tant dans ses formes que dans ses lacunes, puise ses raisons d’exister au fond des éléments ethniques qui le compo­sent, c’est-à-dire dans les effets d’une origine blanche absorbée au sein d’une proportion infiniment forte d’éléments mélaniens ?\par
Il n’est pas nécessaire, pour comprendre ainsi la genèse des langues de l’Asie antérieure, de supposer que les populations sémitiques se soient préalablement noyées dans le sang des noirs. Le fait, incontestable pour les Chamites, ne l’est pas pour leurs associés.\par
À la manière dont ceux-ci se sont mêlés aux sociétés antérieures, tantôt s’abattant victorieux sur les États du centre, tantôt se glissant, en serviteurs utiles et intelligents, dans les communautés maritimes, il est fort à croire qu’ils firent comme les enfants d’Abraham : ils apprirent les langues du pays où ils venaient aussi bien gagner leur vie que régner \footnote{À cette époque, l’araméen était déjà distinct de la langue de Chanaan. (Gen., XXXI, 47) : « Quem (tumulum) vocavit Laban Tumulum testis, et Jacob, Acervum testimonii, uterque juxta proprietatem lingum suit. » Les mots araméens sont (en araméen) les mots hébreux (en hébreu).}. L’exemple donné par le rameau hébreu a très bien pu être suivi par toutes les branches de la famille, et je ne répugne pas davantage à croire que les dialectes formés postérieurement par celle-ci n’aient eu précisément pour caractère typique de créer, ou au moins d’agrandir des lacunes. Je les signalais tout à l’heure dans l’organisme des langues sémitiques. Ceci n’est d’ailleurs pas une synthèse. Les Sémites les moins mélangés de sang chamite, tels que les Hébreux, ont possédé un idiome plus imparfait que les Arabes. Les alliances multipliées de ces derniers avec les peuplades environ­nantes avaient sans cesse replongé la langue dans ses origines mélaniennes. Toutefois, l’arabe est encore loin d’atteindre à l’idéal noir, comme l’essence de ceux qui le possèdent est loin d’être identique avec le sang africain.\par
Quant aux Chamites, il en fut différemment : il fallut, de toute nécessité, que, pour donner naissance au système linguistique qu’ils adoptèrent et transmirent aux Sémites, ils s’abandonnassent sans réserve à l’élément noir. Ils durent posséder le système sémitique beaucoup plus purement, et je ne serais pas surpris si, malgré la rencontre de racines indo-germaniques dans les inscriptions de Bi-Soutoun, on était amené à reconnaître un jour que la langue de quelques-unes de ces annales du plus lointain passé se rapproche plus du type nègre que l’arabe, et, à plus forte raison, que l’hébreu et l’araméen.\par
Je viens de montrer comment il y avait plusieurs degrés vers la perfection sémitique. On part de l’araméen, la plus défectueuse des langues de cette famille, pour arriver au noir pur. Je ferai voir plus tard comment on sort de ce système, avec les peuples les moins atteints par le mélange noir, pour remonter par degrés vers les langues de la famille blanche. Toutefois, laissons ce sujet pour un moment : c’est assez d’avoir établi la situation ethnique des conquérants sémites. Plus respectés que les Assyriens primitifs par la lèpre mélanienne, ils étaient métis comme eux. Ils ne se trouvaient en état de triompher que de nations malades, et nous les verrons succomber toujours quand ils auront affaire à des hommes d’extraction plus noble.\par
Mais, vers l’an 2000 avant Jésus-Christ, ces hommes d’énergie supérieure, les Arians zoroastriens, commençaient à poindre à l’horizon oriental. Ils s’occupaient uniquement de s’assurer les demeures conquises par eux dans la Médie. De leur côté, les Arians hellènes ne cherchaient qu’à se faire place dans leur migration vers l’Europe. Les Sémites avaient ainsi de longs siècles de prédominance et de triomphes assurés sur les gens civilisés du sud-ouest.\par
Chaque fois qu’un mouvement des Arians hellènes les forçait de céder quelque part de leur ancien territoire, la défaite se résolvait pour eux en une victoire fructueuse, car elle s’opérait aux dépens des colons de la riche Babylonie. C’est ainsi que ces bandes de vaincus fugitifs, ensevelissant la honte de leur déroute dans les ténèbres des pays situés vers le Caucase et la Caspienne, frappaient le monde d’admiration à la vue des faciles lauriers que recueillait leur fuite.\par
Les invasions sémitiques constituent donc des œuvres reprises à plusieurs fois. Le détail n’en importe pas ici. Il suffit de rappeler que la première émigration s’empara des États situés dans la basse Chaldée. Une autre expédition, celle des Joktanides, se prolongea jusqu’en Arabie \footnote{Ewald, \emph{Geschichte des Volkes Israël}, t. I, p. 337. ‑ L’arrivée des Joktanides et la fondation de leurs principaux États dans l’Arabie méridionale sont antérieures à l’époque d’Abraham.}. Une autre, d’autres encore, peuplèrent de nouveaux maî­tres les contrées maritimes de l’Asie supérieure. Le sang noir combattait souvent avec succès, chez les plus mélangés de ces peuples, les tendances sédentaires de l’espèce ; et, non seulement des déplacements très considérables avaient lieu dans les masses, mais quelquefois aussi des tribus peu nombreuses, cédant à des considérations de toute nature, abandonnaient leurs résidences pour gagner une autre patrie.\par
Les Sémites étaient déjà en pleine possession de tout l’univers chamite, où les chefs sociaux qui n’étaient pas directement vaincus subissaient pourtant leur influence, quand parut au milieu de leurs établissements un peuple destiné à de grandes épreuves et à de grandes gloires : je veux parler du rameau de la nation hébraïque, que j’ai déjà amené hors des montagnes arméniennes, et qui, sous la conduite d’Abraham, et bientôt avec le nom d’Israël, avait poursuivi sa marche jusqu’en Égypte pour revenir ensuite dans le pays de Chanaan. Lorsque avec le père des patriarches la nation traversa ce pays, il était peu peuplé. Quand Josué y reparut, le sol était largement occupé et bien cultivé par de nombreux Sémites \footnote{Movers, \emph{das Phœnizische Alterthum}, t. II, 1\textsuperscript{re} partie, p. 63-70. ‑ Entre Abraham et Moïse, la Palestine avait été le théâtre de mouvements de population considérables, D’ailleurs de nombreuses nations abrahamides, non israélites, s’y étaient établies, telles que les enfants de Cétura, les fils d’Ismaël, ceux d’Ésaü, ceux de Loth, etc.}.\par
La naissance d’Abraham est fixée par l’exégèse à l’an 2017, postérieurement aux premières attaques des nations helléniques contre les peuples des montagnes, par conséquent non loin de l’époque des victoires de ces derniers sur les Chamites, et de l’élévation de la nouvelle dynastie assyrienne. Abraham appartenait à une nation d’où les Joktanides étaient déjà issus, et dont les branches, restées dans la mère patrie, y formèrent, plus tard, différents États sous les noms de Péleg, de Réhou, de Saroudj, de Nachor et autres \footnote{Ewald, \emph{G. d. V. Israël}, t. I, p. 338.}. Le fils de Tharé devint lui-même le fondateur vénéré de plusieurs peuples, dont les plus célèbres ont été les enfants de Jacob, puis les Arabes occiden­taux, qui, sous le nom d’Ismaélites, partageant avec les Joktanides hébreux et les Chamites kuschites la domination de la péninsule, agirent, dans la suite, avec le plus de force sur les destinées du monde, soit lorsqu’ils donnèrent de nouvelles dynasties aux Assyriens, soit lorsque, avec Mahomet, ils dirigèrent la dernière renaissance de la race sémitique.\par
Avant de suivre plus avant les destinées ethniques du peuple d’Israël, et maintenant que j’ai trouvé dans la date de la naissance de son patriarche un point chronologique assuré qui peut servir à fixer la pensée, j’épuiserai ce qui me reste à dire sur les autres nations chamo-sémites les plus apparentes.\par
Il ne faut pas perdre de vue que le nombre des États indépendants compris dans la société d’alors était innombrable. Toutefois, je ne puis parler que de ceux qui ont laissé les traces les plus profondes de leur existence et de leurs actes. Attachons-nous d’abord aux Phéniciens.
\section[{II.3. Les Chananéens maritimes.}]{II.3. \\
Les Chananéens maritimes.}
\noindent Au temps d’Abraham, la civilisation chamite était dans tout l’éclat de son perfec­tionnement et de ses vices \footnote{Ewald, \emph{G. d. V. Israël}, t. I, p. 262}. Un de ses territoires les plus remarquables était la Palestine \footnote{Même ouvrage, p. 278.}, où les villes de Chanaan florissaient, grâce à leur commerce alimenté par des colonies innombrables déjà. Ce qui pouvait manquer, en population, à toutes ces villes était amplement compensé par cette circonstance heureuse, que nul concurrent ne leur disputait encore les immenses profits de leurs manufactures d’étoffes, de leurs teintu­reries, de leur navigation et de leur transit \footnote{Je ne mentionne pas les ports de Gaza et d’Ascalon, parce qu’ils ne furent fondés qu’après l’émigration de Crète, déterminée par les conquêtes de l’Hellène Minos, 1548 avant J.-C. Du reste, les Assyriens, fidèles à leur système de s’affranchir du monopole phénicien, s’emparèrent très promptement de ces deux cités et leur donnèrent beaucoup de puissance. (Ewald, ouvrage cité, t. I, p. 294 et 367 ; Gesenius,\emph{ Geschichte der hebraeischen Sprache}, p. 14.)}.\par
Toutes les ressources de richesses que je viens d’énumérer restaient concentrées entre les mains de leurs créateurs. Mais, comme pour prouver combien c’est une faible marque de la force vitale des nations qu’un commerce productif, les Phéniciens, déchus de l’antique énergie qui les avait amenés jadis des bords de la mer Persique aux rives de la Méditerranée, n’avaient conservé aucune indépendance politique réelle \footnote{Movers, das \emph{Phœnizische Alterthum}, t. II-I, p. 298 et 378. La politique assyrienne faisait trembler les États chananéens ; quand il n’y avait pas domination directe, l’influence restait énorme et, se mêlant aux querelles des partis, appuyant le faible pour ruiner le fort, suscitait des querelles incessantes et rendait la paix encore plus redoutable que la guerre. M. Movers décrit très bien le jeu de ces antiques combinaisons, et prouve que le but principal des hommes d’État d’Assyrie touchait aux questions commerciales.}. Ils se gouvernaient, le plus souvent, il est vrai, par leurs propres lois et dans leurs formes aristocratiques anciennes. Mais, en fait, la puissance assyrienne avait annulé leur indépendance. Ils recevaient et respectaient les ordres venus des contrées de l’Euphrate \footnote{Movers, \emph{das Phœnizische Alterthum}, t. II-I, p. 259 et 271, et passim.}. Lorsque, dans quelques mouvements intérieurs, ils essayaient de secouer ce joug, leur unique ressource était de se tourner vers l’Égypte et de substituer l’influence de Memphis à celle de Ninive. De véritable isonomie, il n’en était plus question.\par
Outre la prépondérance des deux grands empires entre lesquels les villes chana­néennes se trouvaient resserrées, un motif d’une autre nature forçait les Phéniciens aux plus constants ménagements envers ces puissants voisins. Les territoires de l’Assyrie et de l’Égypte, mais surtout de l’Assyrie, étaient les grands débouchés du commerce de Sidon et de Tyr. À la vérité, les Chananéens allaient, sur d’autres points encore, porter les étoffes de pourpre, les verreries, les parfums et les denrées de toutes sortes, dont leurs magasins regorgeaient. Mais quand la proue élevée de leurs navires noirs et longs venait toucher la grève encore si jeune des côtes grecques ou les rivages de l’Italie, de l’Afrique, de l’Espagne, l’équipage ne faisait là que d’assez maigres profits. La longue barque était tirée à terre par les rameurs noirs, aux tuniques rouges, courtes et serrées. Les populations aborigènes entouraient, la convoitise et l’étonnement peints sur le visage, ces navigateurs arrogants qui commençaient par disposer autour de leur navire les groupes prudemment armés de leurs mercenaires sémites ; puis on étalait devant les rois et les chefs, accourus de tous les points de la contrée, ce que contenaient les flancs du vaisseau. Autant que possible, on cherchait à obtenir en échange des métaux précieux. C’était ce qu’on demandait à l’Espagne, riche en ce genre. Avec les Grecs, on traitait surtout pour des troupeaux, pour des bois principalement, comme en Afrique pour des esclaves. Quand l’occasion s’y prêtait et que le marchand se jugeait le plus fort, sans scrupule il se jetait, avec son monde, sur les belles filles, vierges royales ou servantes, sur les enfants, sur les jeunes garçons, sur les hommes faits, et rapportait joyeusement dans les marchés de sa patrie les fruits abondants de ce commerce sans foi qui, dès la plus haute antiquité, a rendu célèbres l’avidité, la lâcheté et la perfidie des Chamites et de leurs alliés. On comprend, de reste, quelle aversion dangereuse devaient inspirer ces marchands sur les côtes, où ils ne s’étaient pas encore assuré, par des établissements fixes, la haute main et la domination absolue. En somme, ce qu’ils faisaient par tous ces pays, c’était une exploitation des richesses locales. Donnant peu pour obtenir ou extorquer, ou arracher, beaucoup, leurs opérations se bornaient à un commerce de troc, et leurs plus beaux produits, comme leurs plus précieuses denrées, ne trouvaient pas là de placement. La grande importance de l’Occident ne consistait donc nullement pour eux dans ce qu’ils y apportaient, mais bien dans ce qu’ils en tiraient, au meilleur marché possible. Nos régions fournissaient la matière première, que Tyr, Sidon, les autres cités chananéennes travaillaient, façonnaient ou faisaient valoir ailleurs, chez les Égyptiens et dans les contrées mésopotamiques. cananéen\par
Ce n’était pas seulement en Europe et en Afrique que les Phéniciens allaient chercher les éléments de leurs spéculations. Par des relations très antiques avec les Arabes kouschites et les enfants de Joktan, ils prenaient part au commerce des parfums, des épices, de l’ivoire et de l’ébène, provenant de l’Yémen ou de lieux beaucoup plus éloignés, tels que la côte orientale d’Afrique, de l’Inde, ou même de l’extrême Orient \footnote{Le Mahabharata ne connaît pas les noms de Babylone ni de la Chaldée. Cependant il y avait eu, de tout temps, un grand commerce entre les Arians hindous et le monde occidental par l’intermédiaire des Phéniciens, soit avant, soit après que ceux-ci eurent quitté Tylos et Aradus dans le golfe Persique. (Lassen, \emph{Indische Alterthumskunde}, t. I, p. 858 et passim.) Je parlerai ailleurs des vases de porcelaine chinoise trouvés dans des tombeaux, égyptiens des plus anciennes dynasties.}. Pourtant n’ayant pas là, comme pour les produits de l’Europe, un monopole absolu, leur attention restait fixée de préférence sur les pays occidentaux, et c’était entre ces terres accaparées et les deux grands centres de la civilisation contemporaine qu’ils jouaient, dans toute sa plénitude, le rôle avantageux de facteurs uniques.\par
Leur existence et leur prospérité se trouvaient ainsi liées d’une manière étroite aux destinées de Ninive et de Thèbes. Quand ces pays souffraient, aussitôt la consom­mation était en baisse, et immédiatement le coup portait sur l’industrie et le commerce chananéens. Si les rois de la Mésopotamie croyaient avoir à se plaindre des États marchands de la Phénicie, ou bien s’ils voulaient, dans une querelle, les amener à composition sans tirer l’épée, quelques mesures fiscales dirigées contre l’introduction des denrées de l’Occident dans les pays assyriens ou dans les provinces égyptiennes nuisaient beaucoup plus aux patriciens de Tyr, les atteignaient plus profondément et plus sensiblement dans leur existence et, par là, dans leur tranquillité intérieure, que si l’on avait envoyé contre eux d’innombrables armées de cavaliers et de chars. Voilà donc, dans la plus lointaine antiquité, les Phéniciens, si fiers de leur activité mercantile, si dépravés, si abaissés par les vices un peu ignobles, compagnons inséparables de ce genre de mérite, réduits à ne posséder que l’ombre de l’indépendance et vivant serviteurs humiliés de leurs puissants acheteurs.\par
Le gouvernement des villes de la côte avait jadis commencé par être sévèrement théocratique. C’était l’usage de la race de Cham. En effet, les premiers vainqueurs blancs s’étaient montrés au milieu des populations noires avec l’appareil d’une telle supériorité d’intelligence, de volonté et de force, que ces masses superstitieuses ne purent dépeindre mieux la sensation d’admiration et d’épouvante qu’elles en éprou­vèrent qu’en les déclarant dieux. C’est par suite d’une idée toute semblable que les peuples de l’Amérique, aux temps de la découverte, demandaient aux Espagnols s’ils ne venaient pas du ciel, s’ils n’étaient pas des dieux, et, malgré les réponses négatives dictées aux conquérants par la foi chrétienne, leurs vaincus persistaient à les soup­çonner véhémentement de cacher leur qualité. C’est de même encore que, de nos jours, les tribus de l’Afrique orientale ne dépeignent pas autrement l’état dans lequel ils voient les Européens qu’en disant : ce sont des dieux \footnote{Les nègres donnent même ce titre aux Mahalaselys, tribu cafre, qui paraît mériter cet honneur par la possession de vêtements d’étoffe et de maisons pourvues d’escaliers. (Prichard, \emph{Histoire naturelle de l’homme}, t. II, p. 21.)}.\par
Les Chamites blancs, médiocrement retenus par les délicatesses de conscience des temps modernes, n’avaient vraisemblablement eu aucune peine à se résoudre aux adorations. Mais lorsque le sang se mêla, et qu’à la race pure succédèrent partout les mulâtres, le noir découvrit des traces nombreuses d’humanité dans le maître que sa fille ou sa sœur avait mis au monde, Le nouvel hybride, toutefois, était puissant et hautain. Il tenait aux anciens vainqueurs par sa généalogie, et si le règne des divinités finit, celui de leurs prêtres commença. Le despotisme, pour changer de forme, n’en fut pas moins aveuglément vénéré. Les Chananéens conservaient dans leur histoire \footnote{Les annales chamites paraissent avoir été conservées avec beaucoup de soin par les intéressés. M. d’Ewald considère le XIV\textsuperscript{e} chapitre de la Genèse et d’autres fragments du même livre comme des emprunts faits à ces histoires. (Ewald, \emph{Geschichte des Volkes Israël}, t. I, p. 71.) ‑ À son avis, ces travaux des peuples chananéens auraient, en outre, servi de base à la partie cosmogonique et généalogique de la Genèse, rédigée par un lévite au temps de Salomon. (Ouvr. cité, p. 87 et passim.)} l’exposé très complet de ce double état de choses. Ils avaient été gouvernés par Melkart et Baal, et plus tard par les pontifes de ces êtres surhumains \footnote{On verra, lorsqu’il s’agira des nations arianes, tous les motifs qui existent d’assimiler les dieux d’Assyrie aux antiques héros blancs. Il ne paraît pas douteux à M. Rawlinson que le dieu-poisson et la déesse Derceto, représentés sur les sculptures de Khorsabad et de Bi-Soutoun, n’aient été les images des patriarches échappés au dernier déluge.}.\par
Quand les Sémites arrivèrent, la révolution fit un pas en avant. Les Sémites étaient, au fond, plus proches parents des dieux que les dynasties hiératiques des Chamites noirs. Ils avaient quitté plus récemment la souche commune, et leur sang, bien qu’assez altéré, l’était moins que celui des métis dont ils venaient partager les richesses et soutenir l’existence politique, chaque jour plus débile. Toutefois, les prêtres phéniciens ne seraient pas tombés d’accord de cette supériorité de noblesse, et l’auraient-ils voulu qu’ils ne l’auraient pas pu, car l’essence noire prédominait tellement dans leurs veines, qu’ils avaient oublié le Dieu de leurs dieux et l’origine réelle de ces derniers. Ils se considéraient, avec eux, comme autochtones \footnote{Movers, \emph{das Phœnizische Alterth}., t. II-I, p. 15. ‑ C’est là ce qui porte M. Movers à combattre le témoignage d’Hérodote, et à soutenir que les Phéniciens n’étaient pas des émigrants de Tylos.}. C’est dire qu’ils avaient adopté les superstitions grossières des ancêtres de leurs mères. Pour ces gens dégénérés, point de migration blanche de Tylos sur la côte méditerranéenne. Melkart et son peuple étaient sortis du limon sur lequel s’élevaient leurs demeures. Dans d’autres pays et dans d’autres temps, les Hindous, les Grecs, les Italiens et d’autres nations empruntèrent la même erreur aux mêmes sources.\par
Mais les faits vont à leurs conséquences, sans se soucier du concours des opinions. Les Sémites ne purent, sans doute, devenir des dieux puisqu’ils n’avaient pas le sang pur et que, prépondérants, ils ne l’étaient pas assez pour agir sur les imaginations au degré nécessaire à l’apothéose. Les Chamites noirs surent également leur refuser l’entrée des sacerdoces réservés depuis tant de siècles aux mêmes familles. Alors les Sémites humilièrent la théocratie et, plus haut qu’elle, placèrent le gouvernement et le pouvoir du sabre. Après une lutte assez vive, de sacerdotal, monarchique et absolu, le gouvernement des villes phéniciennes devint aristocratique, républicain et absolu, ne gardant ainsi de la triade de forces qu’il remplaçait que la dernière.\par
Il ne détruisit pas complètement les deux autres, fidèle en cela au rôle réformateur, modificateur, plutôt que révolutionnaire, imposé à ses actes par son origine, si voisine de celle des Chamites noirs, et dès lors respectueuse pour le fond de leurs œuvres. Parmi les grandeurs de son aristocratie, il fit une place des plus honorables aux pontificats. Il leur assigna dans l’État le second rang, et continua à en laisser les honneurs aux nobles familles chamites qui jusqu’alors les avaient possédés. La royauté ne fut pas traitée si bien. Peut-être, d’ailleurs, les Chamites noirs eux-mêmes n’en avaient-ils jamais que médiocrement développé la puissance, comme on est tenté de le croire pour les États assyriens.\par
Soit qu’on acceptât désormais, dans le gouvernement des villes phéniciennes, un chef unique, ou bien, combinaison plus fréquente, que la couronne dédoublée se parta­geât entre deux rois intentionnellement choisis dans deux maisons rivales, l’autorité de ces chefs suprêmes devint entièrement limitée, surveillée, contrainte, et on ne leur accorda guère, avec plénitude, que des prérogatives sans effet et des splendeurs sans liberté. Il est permis de croire que les Sémites étendirent à toutes les contrées où ils dominèrent cette jalouse surveillance de la puissance monarchique, et qu’à Ninive comme à Babylone, les titulaires de l’empire ne furent, sous leur inspiration, que les représentants sans initiative des prêtres et des nobles.\par
Telle fut l’organisation sortie de la fusion des Chamites noirs de la Phénicie avec les Sémites. Les rois, autrement dit les suffètes, vivaient dans des palais somptueux. Rien ne semblait ni trop beau ni trop bon pour rehausser la magnificence dont les vrais maîtres de l’État se plaisaient à en orner la double tête. Des multitudes d’esclaves des deux sexes, splendidement vêtus, étaient aux ordres de ces mortels accablés sous l’étalage des jouissances. Des eunuques par troupeaux gardaient l’entrée de leurs jardins et de leurs gynécées. Des femmes de tous les pays leur étaient amenées par les navires voyageurs. Ils mangeaient dans l’or, ils se couronnaient de diamants et de perles, d’améthystes, de rubis, de topazes, et la pourpre, si, exaltée par l’imagination antique, était la couleur respectueusement réservée à tous leurs vêtements. En dehors de cette vie somptueuse et des formes de vénération que la loi commandait d’y ajouter, il n’y avait rien. Les suffètes donnaient leur avis sur les affaires publiques comme les autres nobles, rien de plus ; ou s’ils allaient au delà, c’était par l’usage d’une influence personnelle qui avait été disputée avant d’être subie ; car l’action légale et régulière, et même la puissance exécutive, se concentraient entre les mains des chefs des grandes maisons \footnote{Movers, \emph{Das Phœnizische Allerthum}, t. II, 1\textsuperscript{re} part.}.\par
Pour ces derniers, collectivement, l’autorité n’avait pas de bornes. Du moment qu’un accord conclu entre eux avait pris le caractère impératif qui constitue la loi, tout devait plier devant cette loi, dont les législateurs eux-mêmes étaient les premières victimes. Nulle part et jamais cette abstraction ne ménageait les situations personnelles. Une rigueur inflexible en introduisait les redoutables effets jusque dans l’intérieur des familles, tyrannisait les rapports les plus intimes des époux, planait sur la tête du père, despote de ses enfants, mettait la contrainte entre l’individu et sa conscience. Dans l’État tout entier, depuis le dernier matelot, le plus infime ouvrier, jusqu’au grand prêtre du Dieu le plus révéré, jusqu’au noble le plus arrogant, la loi étendait le niveau terrible révélé par cette courte sentence  : Autant d’hommes, autant d’esclaves !\par
C’est ainsi que les Sémites, unis à la postérité de Cham, avaient compris et prati­quaient la science du gouvernement. J’insiste d’autant plus sur cette sévère conception, que nous la verrons, avec le sang sémitique, pénétrer dans les constitutions de presque tous les peuples de l’antiquité, et toucher même aux temps modernes, où elle ne recule, provisoirement, que devant les notions plus équitables et plus saines de la race germanique.\par
N’oublions pas d’analyser les inspirations qui avaient présidé à cette organisation rigoureuse. En ce qu’elles avaient de brutal et d’odieux, leur source, évidemment, trempait dans la nature noire, amie de l’absolu, facile à l’esclavage, s’attroupant volon­tiers dans une idée abstraite à qui elle ne demande pas de se laisser comprendre, mais de se faire craindre et obéir. Au contraire, dans les éléments d’une nature plus élevée, qu’on ne peut y méconnaître, dans cet essai de pondération entre la royauté, le sacerdoce et la noblesse armée, dans cet amour de la règle et de la légalité, on retrouve les instincts bien marqués que nous constaterons partout chez les peuples de race blanche.\par
Les villes chananéennes attiraient à elles de nombreuses troupes de Sémites, appar­tenant à tous les rameaux de la race, et par conséquent différemment mélangées. Les hommes qui arrivaient d’Assyrie apportaient, du mélange chamite particulier auquel ils avaient touché, un sang tout autre que celui du Sémite qui, venu de la basse Égypte ou du sud de l’Arabie, avait été longtemps en contact avec le nègre à chevelure laineuse. Le Chaldéen du nord, celui des montagnes de l’Arménie \footnote{L’homme venu du pays d’Arpaxad (Gen., X-22). ‑ Tous les peuples sortis de Sem, à la première génération, sont dénommés dans l’ordre de leur position géographique, en commençant par le sud et en finissant par le nord-ouest : Elam, au delà du Tigre, près du golfe Persique ; Assur, l’Assyrie, remontant le Tigre, vers le nord ; Arpaxad, l’Arménie, inclinant à l’ouest ; Lud, la Lydie ; Aram redescend vers le sud avec le cours de l’Euphrate. (Ewald, \emph{Geschichte des Volkes Israël}, t. I.)}, l’Hébreu, enfin, dans les alliages subis par sa race, avait eu plus de participation à l’essence blanche. Cet autre, qui descendait des régions voisines du Caucase, pouvait déjà, directement ou indirec­tement, apporter dans ses veines un ressouvenir de l’espèce jaune. Telles bandes sorties de la Phrygie avaient pour mères des femmes grecques.\par
Autant de nouvelles émigrations, autant d’éléments ethniques nouveaux qui venaient s’accoster dans les cités phéniciennes. Outre ces différents rapports de la famille sémitique, il y avait encore des Chamites du Pays, des Chamites fournis par les grands États de l’est, et encore des Arabes cuschites et des Égyptiens et des nègres purs. En somme, les deux familles blanche et noire, et quelque peu même l’espèce jaune, se combinaient de mille manières différentes au milieu de Chanaan, s’y renou­velaient sans cesse et y abondaient constamment, de manière à y former des variétés et des types jusque-là inconnus.\par
Un tel concours avait lieu parce que la Phénicie offrait de l’occupation à tout ce monde. Les travaux de ses ports, de ses fabriques, de ses caravanes, demandaient beaucoup de bras. Tyr et Sidon, outre qu’elles étaient de grandes villes maritimes et commerciales à la façon de Londres et de Hambourg, étaient en même temps de grands centres industriels comme Liverpool et Birmingham ; devenues les déversoirs des populations de l’Asie antérieure, elles les occupaient toutes et en reportaient le trop-plein sur le vaste cercle de leurs colonies. Elles y envoyaient de la sorte, par des immi­grations constantes, des forces fraîches et un surcroît de leur propre vie. N’admirons pas trop cette activité prodigieuse. Tous ces avantages d’une population sans cesse augmentée avaient leurs revers fâcheux : ils commencèrent par altérer la constitution politique de façon à l’améliorer ; ils finirent par déterminer sa ruine totale.\par
On a vu par quelles transformations ethniques le règne des dieux avait pris fin pour être remplacé par celui des prêtres, qui, à leur tour, avaient cédé le pas à une organi­sation compliquée et savante, destinée à donner accès dans la sphère du pouvoir aux chefs et aux puissants des villes. À la suite de cette réforme, la distinction des races était tombée dans le néant. Il n’y avait plus eu que celle des familles. Devant la mutabilité perpétuelle et rapide des éléments ethniques, cet état aristocratique, dernier mot, terme extrême du sentiment révolutionnaire chez les premiers arrivants sémites, se trouva un jour ne plus suffire aux exigences des générations qui s’élevaient, et les idées démocratiques commencèrent à poindre.\par
Elles s’appuyèrent d’abord sur les rois. Ceux-ci prêtèrent volontiers l’oreille à des principes dont la première application devait être d’humilier les patriciats. Elles s’adressèrent ensuite aux troupeaux d’ouvriers employés dans les manufactures, et en firent le nerf de la faction qu’elles réunissaient. Comme agents actifs des intrigues et des conspirations, on recruta largement dans une classe d’hommes particulière, troupe habituée au luxe, touchant, au moins des yeux, aux grandes séductions de la puissance, mais sans droits, sans autre considération que celle de la faveur, méprisée surtout par les nobles, et dès lors les favorisant peu ; j’entends les esclaves royaux, les eunuques des palais, les favoris ou ceux qui tendaient à le devenir. Telle tut la composition du parti qui poussa à la destruction de l’ordre aristocratique.\par
Les adversaires de ce parti possédaient bien des ressources pour se défendre. Contre les désirs et les velléités des rois, ils avaient l’impuissance légale, la dépendance de ces magistrats sans autorité. Ils s’attachaient à en resserrer les nœuds. Aux masses turbulentes des ouvriers et des matelots, ils présentaient les épées et les dards de cette multitude de troupes mercenaires, surtout cariennes et philistines, qui formaient les garnisons des villes et dont eux seuls exerçaient le commandement. Enfin, aux ruses et aux menées des esclaves royaux, ils opposaient une longue habitude des affaires une méfiance suffisamment aiguisée de la nature humaine, une sagesse pratique bien supérieure aux roueries de leurs rivaux ; en un mot, contre les intrigues des uns, la force brutale des autres, l’ambition ardente des plus grands, les convoitises grossières des plus petits, ils pouvaient user de cette immense ressource d’être les maîtres, arme qui ne se brise pas aisément dans le poing des forts.\par
Certes ils auraient gardé leur empire comme le garderait toute aristocratie, à perpétuité, si la victoire n’avait pu résulter que de l’énergie des assaillants ; mais c’était de leur affaiblissement qu’elle devait éclore. La défaite n’était à prévoir que du mélange de leur sang.\par
La révolution ne triompha que lorsqu’il lui fut né des auxiliaires à l’intérieur des palais dont elle s’évertuait à briser les portes.\par
Dans des États où le commerce donne la richesse et la richesse l’influence, les mésalliances, pour user d’un terme technique, sont toujours difficiles à éviter. Le matelot d’hier est le riche armateur de demain, et ses filles pénètrent, à la manière de la pluie d’or, dans le sein des plus orgueilleuses familles. Le sang des patriciens de la Phénicie était d’ailleurs si mélangé déjà, qu’on avait certainement peu de soin de le garantir contre de séduisantes modifications. La polygamie, si chère aux peuples noirs ou demi-noirs, rend aussi, sous ce rapport, toutes les précautions inutiles. L’homogé­néité avait donc cessé d’exister parmi les races souveraines de la côte de Chanaan, et la démocratie trouva moyen de faire parmi celles-ci des prosélytes. Plus d’un noble commença à goûter des doctrines mortelles à sa caste.\par
L’aristocratie, s’apercevant de cette plaie ouverte dans ses flancs, se défendit au moyen de la déportation. Quand les séditions étaient sur le point d’éclater, ou quand une émeute était vaincue, on saisissait les coupables ; le gouvernement les embarquait de force avec des troupes cariennes, chargées de les surveiller, et les envoyait soit en Libye, soit en Espagne, soit au delà des colonnes d’Hercule, dans des lieux si éloignés, qu’on a prétendu retrouver la trace de ces colonisations jusqu’au Sénégal.\par
Les nobles apostats, mêlés à la tourbe, devaient, dans cet exil éternel, former à leur tour le patriciat des nouvelles colonies, et on n’a pas entendu dire que, malgré leur libéralisme, ils aient jamais désobéi à ce dernier ordre de la mère patrie.\par
Un jour arriva pourtant où la noblesse dut succomber. On connaît la date de cette défaite définitive ; on sait la forme qu’elle revêtit ; on peut en désigner la cause déterminante. La date, c’est l’an 829 avant J.-C. ; la forme, c’est l’émigration aristocra­tique qui fonda Carthage \footnote{Movers, \emph{das Phœnizische Atterthum}, t. II, 1\textsuperscript{re} partie, p. 352 et passim.} ; la cause déterminante est indiquée par l’extrême mélange où en étaient arrivées les populations sous l’action d’un élément nouveau qui, depuis un siècle environ, fomentait d’une manière irrésistible l’anarchie des éléments ethniques.\par
Les peuples hellènes avaient pris un développement considérable. Ils avaient commencé, de leur côté, à créer des colonies, et ces ramifications de leur puissance, s’étendant sur la côte de l’Asie Mineure, n’avaient pas tardé à envoyer en Chanaan de très nombreuses immigrations \footnote{Id. \emph{ibid.}, p. 369.}. Les nouveaux venus, bien autrement intelligents et alertes que les Sémites, bien autrement vigoureux de corps et d’esprit, apportèrent un précieux concours de forces à l’idée démocratique, et hâtèrent par leur présence la maturité de la révolution. Sidon avait succombé la première sous les efforts démagogi­ques. La populace victorieuse avait chassé les nobles, qui étaient allés fonder à Aradus une nouvelle cité, où le commerce et la prospérité s’étaient réfugiés, au détriment de l’ancienne ville, demeurée complètement ruinée \footnote{Movers, \emph{loc. cit.}}. Tyr eut bientôt un sort pareil.\par
Les patriciens, craignant à la fois les séditieux des fabriques, le bas peuple, les esclaves royaux et le roi ; avertis du destin qui les menaçait par l’assassinat du plus grand d’entre eux, le pontife de Melkart, et ne jugeant pas pouvoir maintenir davantage leur autorité, ni sauver leur vie devant une génération issue de mélanges trop multiples, prirent le parti de s’expatrier. La flotte leur appartenait, les navires étaient gardés par leurs troupes. Ils se résignèrent, ils s’éloignèrent avec leurs trésors, et surtout avec leur science gouvernementale et administrative, leur longue et traditionnelle pratique du négoce, et ils s’en allèrent porter leurs destins sur un point de la côte d’Afrique qui fait face à la Sicile.\par
Ainsi s’accomplit un acte héroïque qu’on n’a guère revu depuis. À deux reprises pourtant, dans les temps modernes, il fut question de le renouveler. Le sénat de Venise, dans la guerre de Chiozza, délibéra s’il ne devait pas s’embarquer pour le Péloponèse avec toute sa nation, et il n’y a pas de trop longues années qu’une éventualité semblable fut prévue et discutée dans le parlement anglais.\par
Carthage n’eut point d’enfance \footnote{Movers, t. II, 1\textsuperscript{re}, partie, p. 367 et passim.}. Les maîtres qui la gouvernaient étaient sûrs d’avance de leur volonté. Ils avaient pour but précis ce que la Tyr ancienne leur avait appris à estimer et à poursuivre. Ils étaient entourés de populations presque entière­ment noires, et partant inférieures aux métis qui venaient trôner au milieu d’elles. Ils n’éprouvèrent aucune peine à se faire obéir. Leur gouvernement, remontant le cours des siècles, reprit, en face des sujets, toute la dureté et l’inflexibilité chamitiques ; et comme la cité de Didon ne reçut jamais, pour toute immigration blanche, que les nobles tyriens ou chananéens, victimes, ainsi que ses fondateurs, des catastrophes démagogiques, elle appesantit son joug tant qu’il lui plut. Jusqu’au moment de sa ruine, elle ne fit pas la moindre concession à ses peuples. Lorsqu’ils osèrent en appeler aux armes, elle sut les châtier sans faiblir jamais. C’est que son autorité était fondée sur une différence ethnique qui n’eut pas le temps de composer et de disparaître.\par
L’anarchie tyrienne était devenue complète après le départ des nobles qui, seuls, avaient encore possédé une ombre de l’ancienne valeur de la race, surtout de son homogénéité relative. Quand les rois et le bas peuple se trouvèrent seuls à agir, la diversité des origines se jeta au travers de la place publique pour empêcher toute réorganisation sérieuse. L’esprit chamitique, la multiplicité des branches sémitiques, la nature grecque, tout parla haut, tout parla fort. Il fut impossible de s’entendre, et l’on s’aperçut que, loin de prétendre à retrouver jamais un système de gouvernement logique et fermement dessiné, il faudrait s’estimer très heureux quand on pourrait obtenir une paix temporaire au moyen de compromis passagers. Après la fondation de Carthage, Tyr ne créa pas de colonies nouvelles. Les anciennes, désertant sa cause, se rallièrent, l’une après l’autre, à la cité patricienne, qui devint ainsi leur capitale : rien de plus logique. Elles ne déplacèrent pas leur obéissance : le sol métropolitain fut seul changé. La race dominatrice resta la même, et si bien la même, que désormais ce fut elle qui colonisa. À la fin du VIII\textsuperscript{e} siècle, elle posséda des établissements en Sardaigne : elle-même n’avait pas encore cent années d’existence. Cinquante ans plus tard, elle s’emparait des Baléares. Dans le VI\textsuperscript{e} siècle, elle faisait réoccuper par des colons libyens toutes les cités autrefois phéniciennes de l’Occident, trop peu peuplées à son gré \footnote{Movers, t. II, 2\textsuperscript{e} partie, p. 629.}. Or, dans les nouveaux venus, le sang noir dominait encore plus que sur la côte de Chanaan, d’où étaient venus leurs prédécesseurs : aussi, lorsque, peu de temps avant J.-C., Strabon écrivait que la plus grande partie de l’Espagne était au pouvoir des Phéniciens, que trois cents villes du littoral de la Méditerranée, pour le moins, n’avaient pas d’autres habitants, cela signifiait que ces populations étaient formées d’une base noire assez épaisse sur laquelle étaient venus se superposer, dans une proportion moindre, des éléments tirés des races blanche et jaune ramenées encore par des alluvions carthaginoises vers le naturel mélanien.\par
Ce fut de son patriciat chamite que la patrie d’Annibal reçut sa grande prépon­dérance sur tous les peuples plus noirs. Tyr, privée de cette force et livrée à une complète incohérence de race, s’enfonça dans l’anarchie à pas de géant.\par
Peu de temps après le départ de ses nobles, elle tomba, pour toujours, dans la servitude étrangère, d’abord assyrienne, puis persane, puis macédonienne. Elle ne fut plus à jamais qu’une ville sujette. Pendant le petit nombre d’années qui lui restèrent encore pour exercer son isonomie, soixante-dix-neuf ans seulement après la fondation de Carthage, elle se rendit célèbre par son esprit séditieux, ses révolutions constantes et sanglantes. Les ouvriers de ses fabriques se portèrent, à plusieurs reprises, à des violences inouïes, massacrant les riches, s’emparant de leurs femmes et de leurs filles et s’établissant en maîtres dans les demeures des victimes au milieu de richesses usurpées \footnote{Movers, t. II, 1\textsuperscript{re} partie, p. 366.}. Bref, Tyr devint l’horreur de tout le Chanaan, dont elle avait été la gloire, et elle inspira à toutes les contrées environnantes une haine et une indignation si fortes et de si longue haleine que, lorsque Alexandre vint mettre le siège devant ses murailles, toutes les villes du voisinage s’empressèrent de fournir des vaisseaux pour la réduire. Suivant une tradition locale, on applaudit unanimement en Syrie, quand le conquérant condamna les vaincus à être mis en croix. C’était le supplice légal des esclaves révoltés : les Tyriens n’étaient pas autre chose.\par
Tel fut, en Phénicie, le résultat du mélange immodéré, désordonné des races, mélange trop compliqué pour avoir eu le temps de devenir une fusion, et qui, n’arrivant qu’à juxtaposer les instincts divers, les notions multiples, les antipathies des types différents, favorisait, créait et éternisait des hostilités mortelles.\par
Je ne puis m’empêcher de traiter ici épisodiquement une question curieuse, un vrai problème historique. C’est l’attitude humble et soumise des colonies phéniciennes vis-à-vis de leurs métropoles : Tyr d’abord, Carthage ensuite. L’obéissance et le respect furent tels que, pendant une longue suite de siècles, on ne cite pas un seul exemple de proclamation d’indépendance dans ces colonies, qui cependant n’avaient pas toujours été formées des meilleurs éléments.\par
On connaît leur mode de fondation. C’étaient d’abord de simples campements temporaires, fortifiés sommairement pour défendre les navires contre les déprédations des indigènes. Lorsque le lieu prenait de l’importance par la nature des échanges, ou que les Chananéens trouvaient plus fructueux d’exploiter eux-mêmes la contrée, le campement devenait bourg ou ville. La politique de la métropole multipliait ces cités, en prenant grand soin de les maintenir dans un état de petitesse qui les empêchât de songer à aller seules. On pensait aussi que les répandre sur une plus grande étendue de pays augmentait le profit des spéculations. Rarement plusieurs émissions d’émigrants furent dirigées vers un même point, et de là vient que Cadix, au temps de sa plus grande splendeur et quand le monde était plein du bruit de son opulence, n’avait pourtant qu’une étendue des plus modestes et une population permanente très restreinte \footnote{Strabon, livre III ‑ La ville de cette époque, avec une population que le grand géographe ne pouvait comparer qu’à celle de Rome, n’occupait encore que l’île. Elle avait cependant été agrandie par Balbus.}.\par
Toutes ces bourgades étaient strictement isolées les unes des autres. Une complète indépendance réciproque était le droit inné qu’on leur apprenait à maintenir, avec une jalousie fort agréable à l’esprit centralisateur de la capitale. Libres, elles étaient sans force vis-à-vis de leurs gouvernants lointains, et, ne pouvant se passer de protection, elles adhéraient avec ferveur à la puissante patrie d’où leur venait et qui leur conservait l’existence. Une autre raison très forte de ce développement, c’est que ces colonies fondées en vue du commerce n’avaient toutes qu’un grand débouché, l’Asie, et on n’arrivait en Asie qu’en passant par le Chanaan. Pour parvenir aux marchés de Babylone et de Ninive, pour pénétrer en Égypte, il fallait l’aveu des cités phéniciennes et les factoreries se trouvaient ainsi contraintes de confondre en une seule et même idée la soumission politique et le désir de vendre. Se brouiller avec la mère patrie, ce n’était autre chose que se fermer les portes du monde, et voir bientôt richesses et profits passer à quelque bourgade rivale plus soumise, et dès lors plus heureuse.\par
L’histoire de Carthage montre bien toute la puissance de cette nécessité. Malgré les haines qui semblaient devoir creuser un abîme entre la métropole démagogique et sa fière colonie, Carthage ne voulut pas rompre le lien d’une certaine dépendance. Des rapports longs et bienveillants ne cessèrent d’exister que lorsque Tyr ne compta plus comme entrepôt, et ce ne fut qu’après sa ruine et quand les cités grecques se furent substituées à son activité commerciale, que Carthage affecta la suprématie. Elle rallia alors sous son empire les autres fondations, et devint chef déclaré du peuple chananéen, dont elle avait conservé orgueilleusement le nom, jadis si glorieux. C’est ainsi que ses populations s’appelèrent de tout temps \emph{Chanani} \footnote{Les Phéniciens donnaient à leur pays le nom de \emph{Chna} ou terre de Chanaan par excellence ; mais cette prétention n’était pas reconnue par les autres nations même de la famille, qui n’attribuaient pas d’appellation collective à l’ensemble des États de la côte syrienne (Movers, t. II, 1\textsuperscript{re} partie, p. 65.) ‑ Outre les Phéniciens, la race de Chanaan compte de nombreux rameaux. Voici l’énumération qu’en donne la Genèse, X, 15 : « Chanaan autem « genuit Sidonem, primogenitum suum, Hethæum, 16 : et Zebusæum et Amorrhæum, « Gergesæum, 17 : Hevæum et Aracæum, Sinæm, 18 : et Aradium, Samaræum et Amathæm... »}, bien que le sol de la Palestine ne leur ait jamais appartenu \footnote{Encore au temps de saint Augustin, le bas peuple de la Carthage romaine se donnait le nom de \emph{Chanani}. (Gesenius, \emph{Hebræische Grammatik} p. 16.)}. Ce que les Carthaginois ménageaient si fort dans les Tyriens, avec lesquels ils n’avaient pu vivre, c’était moins le foyer du culte national que le libre passage des marchandises vers l’Asie. Voici maintenant un second fait qui redouble l’évidence des déductions à tirer du premier.\par
Quand les rois perses se furent emparés de la Phénicie et de l’Égypte, ils prétendirent considérer Carthage comme conquise \emph{ipso facto} et légitimement unie au sort de son ancienne capitale. Ils envoyèrent donc des hérauts aux patriciens du lac Tritonide pour leur donner certains ordres et leur faire certaines défenses. Carthage alors était fort puissante ; elle avait peu sujet de craindre les armées du grand roi, d’abord à cause de ses énormes ressources, puis parce qu’elle était bien loin du centre de la monarchie persane. Pourtant elle obéit et s’humilia. C’est qu’il fallait à tout prix conserver la bienveillance d’une dynastie qui pouvait fermer à son gré les ports orientaux de la Méditerranée. Les Carthaginois, politiques positifs, se déterminèrent, en cette occasion, par des motifs analogues à ceux qui, aux XVII\textsuperscript{e} et XVIII\textsuperscript{e} siècles, portèrent plusieurs nations européennes, désireuses de conserver leurs relations avec le Japon et la Chine, à subir des humiliations assez dures pour la conscience chrétienne. Devant une telle résignation de la part de Carthage, et lorsqu’on en pèse les causes, on s’explique que les colonies phéniciennes aient toujours montré un esprit bien éloigné de toute velléité de révolte.\par
Du reste, on se tromperait fort si l’on croyait que ces colonies se soient jamais préoccupées de la pensée de civiliser les nations au milieu desquelles elles se fondaient \footnote{Rien de plus ridicule que le sens philanthropique attribué par quelques modernes au mythe de l’Hercule tyrien. Le héros sémite et ses compagnons se donnaient des torts et ne redressaient pas ceux des autres.}. Animées uniquement d’idées mercantiles, nous savons par Homère quelle aversion elles inspiraient aux populations antiques de l’Hellade. En Espagne et sur les côtes de la Gaule, elles ne donnèrent pas une meilleure opinion d’elles. Là où les Chananéens se trouvaient en face de populations faibles, ils poussaient la compression jusqu’à l’atrocité, et réduisaient à l’état de bêtes de somme les indigènes employés aux travaux des mines. S’ils rencontraient plus de résistance, ils employaient plus d’astuce. Mais le résultat était le même. Partout les populations locales n’étaient pour eux que des instruments dont ils abusaient, ou des adversaires qu’ils exterminaient. L’hostilité fut permanente entre les aborigènes de tous les pays et ces marchands féroces. C’était encore là une raison qui forçait les colonies, toujours isolées, faibles et mal avec leurs voisins, de rester fidèles à la métropole, et ce fut aussi un grand levier dans la main de Rome pour renverser la puissance carthaginoise. La politique de la cité italienne, comparée à celle de sa rivale, parut humaine et conquit par là des sympathies, et finalement la victoire. Je ne veux pas ici adresser aux consuls et aux préteurs un éloge peu mérité. Il y avait grand moyen de se montrer cruel et oppressif en l’étant moins que la race chananéenne. Cette nation de mulâtres, phénicienne ou carthaginoise, n’eut jamais la moindre idée de justice ni le moindre désir d’organiser, je ne dirai pas d’une manière équitable, seulement tolérable, les peuples soumis à son empire. Elle resta fidèle aux principes reçus par les Sémites de la descendance de Nemrod, et puisés par celle-ci dans le sang des noirs.\par
L’histoire des colonies phéniciennes, si elle fait honneur à l’habileté des organi­sateurs, doit, en somme, ce qu’elle eut de particulièrement heureux pour les métropoles à des circonstances toutes particulières, et qui n’ont jamais pu se renouveler depuis. Les colonies des Grecs furent moins fidèles ; celles des peuples modernes, également : c’est que les unes et les autres avaient le monde ouvert, et n’étaient pas contraintes de traverser la mère patrie pour parvenir à des marchés où elles pussent débiter leurs productions.\par
Il ne me reste plus rien à dire sur la branche la plus vivace de la famille chana­néenne. Elle fournit, par ses mérites et ses vices, la première certitude que l’histoire présente à l’ethnologie  : l’élément noir y domina. De là, amour effréné des jouissances matérielles, superstitions profondes, dispositions pour les arts, immoralité, férocité.\par
Le type blanc s’y montra en force moindre. Son caractère mâle tendit à s’effacer devant les éléments féminins qui l’absorbaient. Il apporta, dans ce vaste hymen, l’esprit utilitaire et conquérant, le goût d’une organisation stable et cette tendance naturelle à la régularité politique qui dit son mot et joue son rôle dans l’institution du despotisme légal, rôle contrarié sans doute, cependant efficace, Pour achever le tableau, la sura­bondance de types inconciliables, issus des proportions diverses entre les mélanges, enfanta le désordre chronique, et amena la paralysie sociale et cet état d’abaissement grégaire où chaque jour a dominé davantage la puissance de l’essence mélanienne. C’est dans cette situation que croupirent désormais les races formées par les alliages chananéens.\par
Retournons aux autres branches des familles de Cham et de Sem.
\section[{II.4. Les Assyriens ; les Hébreux ; les Choréens.}]{II.4. \\
Les Assyriens ; les Hébreux ; les Choréens.}
\noindent Le sentiment unanime de l’antiquité n’a jamais cessé d’attribuer aux peuples de la région mésopotamique cette supériorité marquée sur toutes les autres nations issues de Cham et de Sem, dont j’ai déjà touché quelques mots. Les Phéniciens étaient habiles ; les Carthaginois le furent à leur tour. Les États juifs, arabes, lydiens, phrygiens eurent leur éclat et leur gloire. Rien de mieux : en somme, ces planètes n’étaient que les satellites de la grande contrée où s’élaboraient leurs destinées. L’Assyrie dominait tout, sans conteste.\par
D’où pouvait provenir une telle supériorité ? La philologie va répondre strictement.\par
J’ai montré que le système des langues sémitiques était une extension imparfaite de celui des langues noires. C’est là seulement que se trouve l’idéal de ce mode d’idiome. Il est altéré dans l’arabe, plus incomplet encore dans l’hébreu, et je ne me suis pas avancé, dans la progression descendante, au delà de l’araméen, où la décadence des principes constitutifs est plus prononcée encore. On se trouve là comme un homme qui, s’enfonçant dans un passage souterrain, perd la lumière à mesure qu’il avance. En continuant de marcher, on reverra la clarté, mais ce sera par un autre côté de la caverne, et sa lueur sera différente.\par
L’araméen n’offre encore qu’une désertion négative de l’esprit mélanien. Il ne dévoile pas des formes nettement étrangères à ce système. En regardant un peu plus loin, géographiquement parlant, se présente bientôt l’arménien ancien, et là, sans aucun doute, s’aperçoivent des nouveautés. On met la main sur une originalité qui frappe. On la regarde, on l’étudie : c’est l’élément indo-germanique. Il n’y a pas à en douter. Bien limité encore, faible peut-être, toutefois vivant et imméconnaissable.\par
Je poursuis ma route. À côté des Arméniens sont les Mèdes. J’écoute leur langue. Je constate encore et des sons et des formes sémitiques. Les uns et les autres sont plus effacés que dans l’arménien, et l’indo-germanique y occupe une plus grande place \footnote{Un érudit d’une réputation aussi grande que méritée, M. de Saulcy, a émis une théorie nouvelle au sujet du médique, dans lequel il découvre des éléments appartenant aux langues turques. En adoptant cette très intéressante hypothèse, il deviendrait indispensable sans doute d’ajouter une partie constitutive de plus au médique. Mais les rapports existant aussi dans le sein de cet idiome, entre l’indo-germanique et le sémitique, et que je signale, n’en seraient pas troublés. (Voir F. de Saulcy, \emph{Recherches analytiques sur les inscriptions cunéiformes du système médique}, Paris, 1850.)}. Aussitôt que j’entre sur les territoires placés au nord de la Médie, je passe au zend. J’y trouve encore du sémitique, cette fois à l’état tout à fait subordonné. Si, par un pas de côté, je tombais vers le sud, le pehlvi, toujours indo-germanique, me ramènerait cependant vers une plus grande abondance d’éléments empruntés à Sem. Je l’évite, je pousse toujours plus avant dans le nord-est, et les premiers parages hindous m’offrent aussitôt le meilleur type connu des langues de l’espèce blanche, en me présentant le sanscrit \footnote{Klaproth, \emph{Asie polyglotta}, p, 65 ; voir aussi, au sujet du médique, Rœdiger et Pott, \emph{Kurdische Studien}, dans la \emph{Zeitschrift für die Kunde des Morgenlandes}, t. III, p. 12-13.}.\par
Je tire de ces faits cette conséquence que, plus je descends au midi, plus je trouve d’alliage sémitique, et qu’à proportion où je m’élève vers le nord, je rencontre les éléments blancs dans un meilleur état de pureté et avec une abondance incomparable. Or les États assyriens étaient, de toutes les fondations chamo-sémites les plus reculées dans cette direction. Ils étaient sans cesse atteints par des immigrations, latentes ou déclarées, descendues des montagnes du nord-est. C’est donc là qu’était la cause de leur longue, de leur séculaire prépondérance.\par
Avec quelle rapidité les invasions se succédaient, on l’a vu. La dynastie sémite-chaldéenne, qui avait mis fin à la domination exclusive des Chamites vers l’an 2000, fut renversée, deux cents ans après environ, par de nouvelles bandes sorties des montagnes.\par
À celles-ci, l’histoire donne le nom de médiques. On aurait lieu d’être un peu surpris de rencontrer des nations indo-germaniques si avant dans le sud-ouest, à une époque encore bien reculée, si, persistant dans l’ancienne classification, on prétendait tirer une rigoureuse ligne de démarcation entre les peuples blancs, des différentes origines, et séparer nettement les Sémites des nations dont les principales branches ont peuplé l’Inde et plus tard l’Europe. Nous venons de voir que la vérité philologique repousse cette méthode de classifications strictes. Nous sommes complètement en droit d’admettre les Mèdes comme fondateurs d’une très ancienne dynastie assyrienne, et de considérer ces Mèdes, soit, avec Movers, comme des Sémites-Chaldéens \footnote{Movers, \emph{Das Phœnizische Alterthum}, t. II, 1\textsuperscript{re} partie, p. 420.}, soit avec Ewald, comme des peuples arians ou indo-germains, suivant la face sous laquelle il nous plaît le mieux d’envisager la question \footnote{Ewald, \emph{Geschichte des Volkes Israël}, t. I, p. 334.}. Servant de transition aux deux races, ils tiennent de l’une et de l’autre. Ce sont indifféremment, à parler géographie, les derniers des Sémites ou les premiers des Arians, comme on voudra.\par
Je ne doute pas que, sous le rapport des qualités qui tiennent à la race, ces Mèdes de première invasion ne fussent supérieurs aux Sémites plus mêlés aux noirs dont ils étaient les parents. J’en veux pour témoignage leur religion, qui était le magisme. Il faut l’induire du nom du second roi de leur dynastie, Zaratuschtra \footnote{Lassen, \emph{Indische Alterthumskunde}, t. I, p. 753}. Non pas que je sois tenté de confondre ce monarque avec le législateur religieux : celui-là vivait à une époque beaucoup plus ancienne ; mais l’apparition du nom de ce prophète, porté par un souverain, est une garantie de l’existence de ses dogmes au milieu de la nation. Les Mèdes n’étaient donc pas dégradés par les monstruosités des cultes chamitiques, et, avec des notions religieuses plus saines, ils gardaient certainement plus de vigueur militaire et plus de facultés gouvernementales.\par
Il n’était cependant pas possible que leur domination se maintînt indéfiniment. Les raisons qui leur imposaient une prompte décadence sont de différent ordre.\par
La nation médique n’a jamais été très nombreuse, nous aurons l’occasion de le démontrer plus tard, et si, au VIII\textsuperscript{e} siècle avant Jésus-Christ, elle a repris sur les États assyriens une autorité perdue depuis l’an 2234 avant notre ère, c’est qu’alors elle fut puissamment aidée par l’abâtardissement final des races chamo-sémitiques, par l’absence complète de tout concurrent à l’empire et par l’alliance de plusieurs nations arianes, qui, à l’époque de sa première invasion, n’avaient pas encore paru dans les régions du sud-ouest qu’elles occupèrent plus tard, entre autres les tribus persiques.\par
De sorte que les Mèdes formaient une sorte d’avant-garde de la famine ariane. Ils n’étaient pas nombreux par eux-mêmes, ils n’étaient pas appuyés par les autres peuples, leurs parents ; et non seulement ils ne l’étaient pas, parce que ceux-ci n’étaient pas encore descendus, à leurs côtés, vers les contrées méridionales, mais parce que, dans ces époques reculées et après le départ des Arians Hellènes (dont les migrations jetaient constamment des essaims de Sémites sur le monde assyrien et chananéen) une civilisation imposante exerçait un immense empire sur le gros des peuples arians zoroastriens, dans les régions situées entre la Caspienne et l’Hindoukoh, et, plus particulièrement, dans la Bactriane. Là régnait une populeuse cité, Balk, \emph{la mère des villes}, pour me servir de l’expression emphatique employée par les traditions iranien­nes lorsqu’elles veulent peindre d’un même trait et la puissance et l’incroyable antiquité de l’ancienne métropole du magisme.\par
Il s’était formé sur ce point un centre de vie qui, concentrant toute l’attention et toute la sympathie des nations zoroastriennes, les détournait d’entrer dans le courant assyrien. Ce qui leur restait d’activité, en dehors de cette sphère, se reportait d’ailleurs tout entier du côté de l’est, vers les régions de l’Inde, vers les pays du Pendjab, où des relations étroites de parenté, des souvenirs importants, d’anciennes habitudes, la similitude de langage, et même des haines religieuses et l’esprit de controverse, qui en est la suite naturelle, reportaient leur pensée.\par
Les Mèdes, dans leurs entreprises sur l’Asie antérieure, se trouvaient ainsi réduits à la modicité de leurs seules ressources, situation d’autant plus faible que des compé­titeurs ambitieux, des bandes de Sémites descendant du nord, se succédaient sans cesse pour ébranler leur domination.\par
À égalité de nombre, ces Sémites ne les valaient pas. Mais leurs flots épais, se multipliant, les astreignaient à des efforts qui ne pouvaient pas être toujours heureux, et d’autant moins que les mérites allaient, en définitive, s’égalisant, et même quelque chose de plus, à mesure que les années passaient sur les maîtres du trône.\par
Ceux-ci résidaient dans les villes d’Assyrie, soutenus, sans doute, de loin, par leur nation, cependant séparés d’elle et vivant loin d’elle, perdus dans la foule chamo-sémitique. Leur sang s’altéra, comme s’était altéré celui des Chamites blancs et celui des premiers Chaldéens. Les incursions sémitiques, d’abord rembarrées avec vigueur, ne trouvèrent plus, un jour, la même résistance. Ce jour-là, elles firent brèche et la domi­nation médique fut si bien renversée que l’épée des vainqueurs commanda même au gros du peuple, découragé et accablé par les multitudes qui vinrent fondre sur lui.\par
Les États assyriens avaient recommencé à décliner sous les derniers souverains mèdes. Ils reprirent leur éclat, leur omnipotence dans toute l’Asie antérieure, avec le nouvel apport de sang frais et choisi qui vint, sinon relever leurs races nationales, du moins les gouverner sans conteste. C’est, par cette série incessante de régénérations que l’Assyrie se maintenait toujours à la tête des contrées chamo-sémitiques.\par
La nouvelle invasion donna naissance, pour le pays-roi, à de grandes extensions territoriales \footnote{Lassen, \emph{Indische Alterthumskunde}, t. I, p. 858 et pass. ‑ Movers, \emph{Das Phœnizische Alterthum}, t. II, 1\textsuperscript{re} partie, p. 272 et pass.}.\par
Après avoir asservi le pays des Mèdes, les conquérants sémites firent des inva­sions au nord et à l’est. Ils ravagèrent une partie de la Bactriane et pénétrèrent jusqu’aux premiers confins de l’Inde. La Phénicie, autrefois conquise, le fut de nouveau, et les idées, les notions, les sciences, les mœurs assyriennes se répandirent plus que jamais, et poussèrent plus avant leurs racines. Les grandes entreprises, les grandes créations se succédèrent rapidement. Tandis que de puissants monarques babyloniens fondaient dans l’est, aux environs de la ville actuelle de Kandahar, cette cité de Kophen, dont les ruines ont été retrouvées par le colonel Rawlinson \footnote{Movers, \emph{Das Phœnizische Alterthum}, t. II, 1\textsuperscript{re} partie, p. 265.}, Mabudj s’élevait sur l’Euphrate, Damas et Gadara plus à l’ouest \footnote{Damas fut possédé, quelque temps après Abraham, par une émigration de Sémites venus d’Arménie. Ewald, \emph{Geschichte des Volkes Israël}, t. I, p. 367. Plus tard, une autre invasion de la même provenance renversa la dynastie nationale des Ben-Hadad, et la remplaça par une famille qui porta le titre de Derketade, ibid., p. 274. ‑ Dans les temps grecs et romains, les Damascènes, par une prétention qui se rencontre rarement chez les peuples comme chez les individus, niaient l’extrême antiquité de leur ville, et prétendaient pour elle à l’honneur d’avoir été fondée par Abraham.}. Les civilisateurs sémites passaient l’Halys, et organisaient sur la côte de la Troade, dans les pays lydiens, des souverainetés qui, plus tard indépendantes, se firent gloire à jamais de leur avoir dû la naissance \footnote{Les Sandonides de Lydie se vantaient d’une origine assyrienne. (Ewald, \emph{Geschichte des Volkes Israël}, t. I, p. 329.)}.\par
Il est inutile de suivre le mouvement de ces dynasties assyriennes, qui retinrent pendant tant de siècles le gouvernement de l’Asie antérieure dans des mains régénéra­trices. Tant que les contrées voisines de l’Arménie et adossées au Caucase fournirent des populations plus blanches que celles qui habitaient les plaines méridionales, les forces des États assyriens se renouvelèrent toujours à propos. Une dynastie d’Arabes Ismaélites interrompit seule (de 1520 à 1274 av. J.-C.) le cours de la puissance chaldéenne. Une race dégénérée fut ainsi remplacée par des Sémites du sud, moins corrompus que l’élément chamitique, si prompt à pourrir tous les apports de sang noble dans les pays mésopotamiques. Mais aussitôt que des Chaldéens, plus purs que la famille ismaélite, se montrèrent de nouveau, celle-ci descendit du trône pour le leur céder.\par
On le voit : dans les sphères élevées du pouvoir, là où s’élaborent les idées civilisa­trices, il n’est plus question, il ne doit plus jamais être tenu compte des Chamites noirs. Leurs masses se sont tout à fait humiliées sous les couches successives de Sémites. Elles font nombre dans l’État, et ne jouent plus de personnage actif. Mais un rôle si humble en apparence n’en est pas moins terrible et décisif. C’est le fond stagnant où tous les conquérants viennent, après peu de générations, s’abattre et s’engloutir. D’abord, de ce terrain corrompu sur lequel marchent triomphalement les vainqueurs, la boue ne leur monte que jusqu’à la cheville. Bientôt les pieds s’enfoncent, et l’immersion dépasse la tête. Physiologiquement comme moralement, elle est complète. Au temps d’Agamemnon, ce qui frappa le plus les Grecs dans les Assyriens venus au secours de Priam, ce fut la couleur de Memnon, le fils de l’Aurore. À ces peuples orientaux les rapsodes appliquaient sans hésitation le nom significatif d’Éthiopiens \footnote{Movers, t. II, 1\textsuperscript{re} partie, p. 277. Les Éthiopiens, (en grec), des Grecs, sont les enfants de Kouch. Ce sont des Arabes ce mot (en arabe) indique la couleur noire des visages, comme celui de (en grec) indique la carnation cuivrée, rougeâtre, des Chananéens.}.\par
Après la destruction de Troie, les mêmes motifs commerciaux qui avaient engagé les Assyriens à favoriser l’établissement des villes maritimes dans le pays des Philistins et au nord de l’Asie Mineure \footnote{Movers, t. II, 1\textsuperscript{re} partie, p. 411. Cette alliance naturelle entre les Assyriens et les Grecs, concurrents des Phéniciens, est très bien caractérisée par ce qui se passait à Chypre. Il y eut là, de bonne heure, une double population ; l’une sémitique, l’autre grecque. Les Chypriotes grecs tenaient pour les Assyriens, les Sémites pour Tyr. (Movers, t. II, 1\textsuperscript{re} partie, 387.)}, les portèrent également à pardonner aux Grecs la destruction d’une ville, leur tributaire, et à protéger l’Ionie. Leur but était de mettre fin au monopole des cités phéniciennes, et en conséquence, les Troyens une fois tombés sans remède, leurs vainqueurs furent admis à les remplacer. Les Grecs asiatiques devinrent ainsi les facteurs préférés du commerce de Ninive et de Babylone. C’est la première preuve que nous ayons encore rencontrée de cette vérité si souvent répétée par l’histoire, que, si l’identité de race crée entre les peuples l’identité de destinée, elle ne détermine nullement l’identité d’intérêts, et par suite l’affection mutuelle.\par
Tant que les Phéniciens furent seuls à exploiter les régions occidentales du monde, ils vendirent trop cher leurs denrées aux Assyriens, qui n’eurent pas de cesse jusqu’à ce que, leur ayant suscité des concurrents, d’abord dans les Troyens, puis dans les Grecs, ils eussent réussi à obtenir à meilleur compte les produits que réclamait leur consommation \footnote{Movers, \emph{das Phœnizische Alterthum}, t. II, 1\textsuperscript{re} partie, p. 411.}.\par
Ainsi, dans toute l’Asie antérieure on vivait sous la direction des Assyriens. Si l’on devait réussir, on réussissait par eux, et tout ce qui essayait de sortir de leur ombre restait faible et languissant. Encore cette indépendance funeste n’était-elle jamais que relative, même chez les tribus nomades du désert. Pas une nation, grande ou petite, qui n’éprouvât l’action des populations et du pouvoir de la Mésopotamie. Cependant, parmi celles qui s’en ressentaient le moins, les fils d’Israël semblent se présenter en première ligne. Ils se disaient jaloux de leur individualité plus que toute autre tribu sémite. Ils désiraient passer pour purs dans leur descendance. Ils affectaient de s’isoler de tout ce qui les entourait. À ce titre seul, ils mériteraient d’occuper dans ces pages une place réservée, si les grandes idées que leur nom réveille ne la leur avaient pas assurée d’avance.\par
Les fils d’Abraham ont changé plusieurs fois de nom. Ils ont commencé par s’appeler Hébreux. Mais ce titre, qu’ils partageaient avec tant d’autres peuples, était trop vaste, trop général. Ils y substituèrent celui de fils d’Israël. Plus tard, Juda ayant dominé en éclat et en gloire tous les souvenirs de leurs patriarches, ils devinrent les Juifs. Enfin, après la prise de Jérusalem par Titus, ce goût de l’archaïsme, cette passion des origines, triste aveu de l’impuissance présente qui ne manque jamais de saisir les peuples vieillards, sentiment naturel et touchant, leur fit reprendre le nom d’Hébreux.\par
Cette nation, malgré ce qu’elle a pu prétendre, ne posséda jamais, non plus que les Phéniciens, une civilisation qui lui fût propre. Elle se borna à suivre les exemples venus de la Mésopotamie, en les mélangeant de quelque peu de goût égyptien. Les mœurs des Israélites, dans leur plus beau moment, au temps de David et de Salomon \footnote{Ewald, \emph{Geschichte des Volkes Israël}, t. I, p. 87.}, furent tout à fait tyriennes, et partant ninivites. On sait avec quelle peine et même quels succès mélangés, les efforts de leurs prêtres tendirent constamment à les tenir loin des plus horribles abus de l’émanatisme oriental.\par
Si les fils d’Abraham avaient pu garder, après leur descente des montagnes chaldéennes, la pureté relative de race qu’ils apportaient avec eux, il n’y a pas de doute qu’ils eussent conservé et étendu cette prépondérance qu’avec le père de leurs patriarches, on leur vit exercer sur les populations chananéennes plus civilisées, plus riches, mais moins énergiques, parce qu’elles étaient plus noires. Par malheur, en dépit de prescriptions fondamentales, malgré les défenses successives de la loi, malgré même les exemples terribles de réprobation que rappellent les noms des Ismaélites, des Édomites, descendants illégitimes et rejetés de la souche abrahamide, il s’en fallut de tout que les Hébreux ne s’alliassent que dans leur parenté \footnote{D’ailleurs la famille même du fils de Tharé ne se composait pas que de personnes issue de la même souche. Lorsqu’il forma alliance avec le Seigneur et qu’il eut circoncis tous les mâles de sa maison, ceux-ci devinrent tous hébreux, bien que le texte dise expressément qu’il y avait parmi eux des esclaves achetés à prix d’argent et des étrangers (Gen., XVII, 27) : « Et omnes viri domus illius, « tam vernaculi, quam \emph{emptitii} et alienigenæ, pariter circumcisi sunt. » On doit conclure aussi des paroles expresses du livre saint que la nationalité israélite résultait beaucoup moins de la descendance que du fait de la circoncision. Voici les paroles expresses (Gen., XVII, 11) : « Et « circumcidetis carnem præputii vestri ut sit in signum fœderis inter me et vos... » (121 : « Omne masculinum in generationibus vestris ; tam « vernaculus quam emptitius circumcidetur... » Et (XXXIV, 15) : « Sed in hoc valebimus « fœderari, si volueritis esse similes nostri et circumcidatur in vobis omne masculini « sexus. » (13) : « Tunc dabimus mutuo filias vestras ac nostras  : et habitabimus vobiscum, « erimusque unus populus. » D’après un tel système, il était impossible que la pureté des races se maintînt, quels que fassent les efforts que l’on pouvait faire d’ailleurs dans ce but.}. Dès leurs premiers temps, la politique les contraignit d’accepter l’alliance de plusieurs nations réprouvées, de résider au milieu d’elles, de mêler leurs tentes et leurs troupeaux aux troupeaux et aux tentes de l’étranger, et les jeunes gens des deux familles se rencontraient aux citernes. Les Kénaens, fraction d’Amalek, et bien d’autres, furent fondus de la sorte dans le peuple des douze tribus \footnote{Gen., XV, 19 ; Sam., 1, 15, 6 ; Ewald, \emph{Geschichte des Volkes} Israël, t. I, p. 298 et passim.}.\par
Puis les patriarches avaient été des premiers à violer la loi. Les généalogies mosaïques nous enseignent bien que Sara était la demi-sœur de son mari, et par conséquent d’un sang pur \footnote{Gen., XX, 12 : « Alias autem et vere soror mea est, filia patris mei ; et non filia matris meæ et duxi eam in uxorem.}. Mais si Jacob épousa Lia et Rachel, ses cousines, et en eut huit de ses fils, ses quatre autres enfants, qui ne sont pas moins comptés parmi les véritables pères d’Israël, naquirent des deux servantes Bala et Zelpha \footnote{Gen., XXIX, 3-13.}. L’exemple donné fut suivi par ses rejetons \footnote{Je ne citerai, de tous les passages qui l’établissent, que celui qui a rapport à la descendance de Joseph. C’était le fils favori d’Israël, l’homme pur par excellence ; il avait cependant épousé une Égyptienne. ‑ Gen., XLIV, 20 : « Natique sunt joseph filii in terra Ægypti, quos « genuit ei Aseneth, filia Putiphare sacerdotis Heliopoleos : Manasses et Ephraim. »}.\par
Dans les époques suivantes, on trouve d’autres alliances ethniques, et, quand on arrive à l’époque monarchique, il est impossible de les énumérer, tant elles sont devenues communes.\par
Le royaume de David, s’étendant jusqu’à l’Euphrate, embrassait bien des popula­tions diverses. Il ne pouvait même être question d’y maintenir la pureté ethnique. Le mélange pénétra donc par tous les pores, dans les membres d’Israël. Il est vrai que le principe resta ; que plus tard Zorobabel exerça des sévérités approuvées contre les hommes mariés aux filles des nations. Mais l’intégrité du sang d’Abraham n’en avait pas moins disparu, et les Juifs étaient aussi souillés de l’alliage mélanien que les Chamites et les Sémites au milieu desquels ils vivaient. Ils avaient adopté leur langue \footnote{Isaïe appelle l’hébreu, \emph{langue de Chanaan} (XXXIV, 11, 13).}. Ils avaient pris leurs coutumes ; leurs annales étaient en partie celles de leurs voisins, Philistins, Édomites, Amalécites, Amorrhéens. Trop souvent, ils porteront l’imitation des mœurs jusqu’à l’apostasie religieuse \footnote{Ewald, t. I, p. 71.}. Hébreux et gentils étaient taillés, en vérité, sur un seul et même modèle. Enfin, je donne ceci, tout à la fois, comme une preuve et comme une conséquence  : ni au temps de Josué, ni sous David ou Salomon, ni quand les Machabées régnèrent, les Juifs ne parvinrent à exercer sur les peuples de leur entourage, sur tant de petites nations parentes, pourtant si faibles, une supériorité quelque peu durable. Ils furent comme les Ismaélites, comme les Philistins. Ils eurent des jours, rien que quelques jours de puissance, et l’égalité d’ailleurs fut complète avec leurs rivaux.\par
J’ai déjà expliqué pourquoi les Israélites, les fils d’Ismaël, ceux d’Edom, et d’Amalek, composés des mêmes éléments fondamentaux noirs, chamites et sémites, que les Phéniciens et les Assyriens, sont constamment demeurés au plus bas degré de la civi­lisation typique de la race, laissant aux peuples de la Mésopotamie le rôle inspirateur et dirigeant. C’est que les éléments d’origine blanche se renouvelaient périodiquement chez ces derniers, et jamais chez eux. Ils ne réussirent donc point à faire des conquêtes stables, et, lorsqu’ils se trouvèrent avoir le loisir et le goût de perfectionner leurs mœurs, ils ne purent que tout emprunter à la culture assyrienne, sans lui rendre jamais rien, la pratiquant un peu, j’imagine, comme les provinciaux font des modes de Paris. Les Tyriens, tout grands marchands qu’ils étaient, n’étaient pas plus inspirés. Ils ne comprenaient que d’une façon incomplète ce que leur enseignait Ninive. Salomon, à son tour, lorsqu’il voulait bâtir son temple, faisant venir de Tyr architectes, sculpteurs et brodeurs, n’obtenait pas le dernier mot des talents de son époque. Il est vraisemblable que, dans les magnificences qui éblouirent si fort Jérusalem, l’œil d’un homme de goût venu de Ninive, n’aurait démêlé qu’une copie faite de seconde main des belles choses qu’il avait contemplées en original dans les grandes métropoles mésopotamiques, où l’Occident, l’Orient, l’Inde et la Chine même, au dire d’Isaïe \footnote{Isaïe, XLIX, 12. Lassen, \emph{Indische Alterthumskunde}, t. I, p. 857.}, envoyaient, sans se lasser, tout ce qu’il y avait de plus accompli dans tous les genres.\par
Rien de plus simple. Les petits peuples dont je parle en ce moment étaient des Sémites trop chamitisés pour jouer un autre rôle que celui de satellites dans un système de culture qui d’ailleurs, étant celui de leur race, leur convenait et n’avait besoin pour leur sembler parfait que de subir des modifications locales. Ce furent précisément ces modifications locales qui, réduisant les splendeurs ninivites au degré voulu par des nations obscures et pauvres, créait l’amoindrissement de la civilisation. Transporté à Babylone, le Phénicien, l’Hébreu, l’Arabe, s’y mettaient aisément de pair avec le reste des populations, sauf peut-être les Sémites du nord les plus récemment arrivés, et devenaient habiles à secouer les liens que leur imposait la médiocrité de leurs milieux nationaux ; mais c’était là de l’imitation, rien de plus. En ces groupes fractionnaires ne résidait pas l’excellence du type \footnote{Movers, \emph{Das Phœnizische Alterthum}, t. II, 1\textsuperscript{re} partie, p. 302.}.\par
 Je ne quitterai pas les Israélites sans avoir touché quelques mots de certaines tribus qui vécurent longtemps parmi eux, dans les districts situés ou nord du Jourdain. Cette population mystérieuse paraît n’avoir été autre que les débris restés purs de quelques-unes des familles mélaniennes, de ces noirs jadis seuls maîtres de l’Asie antérieure avant la venue des Chamites blancs. La description que les livres saints nous font de ces hommes misérables est précise, caractéristique, terrible par l’idée de dégradation profonde qu’elle éveille.\par
Ils n’habitaient plus, au temps de Job, que dans le district montagneux de Séir ou Edom, au sud du Jourdain. Abraham les y avait déjà connus. Ésaü, ce ne fut vraisem­blablement pas sa moindre faute, habita parmi eux \footnote{Gen. XXXVI, 8 : « Habitavitque Esau in monte Seir... »}, et, conséquence naturelle dans ces temps-là, il prit, au nombre de ses épouses, une de leurs femmes, Oolibama, fille d’Ana, fille de Sébéon, de sorte que les fils qu’il en eut, Jehus, Jhelon et Coré, se trouvèrent liés très directement par leur mère à la race noire.\par
Les Septante appellent ces peuplades les Chorréens ; la Vulgate les nomme moins justement Horréens, et il en est fait mention en plusieurs endroits des Écritures \footnote{(En hébreu) trou, caverne.}. Ils vivaient au milieu des rochers et se blottissaient dans des cavernes. Leur nom même signifie \emph{troglodytes} \footnote{Tantôt la Vulgate dit \emph{Horræi} (\emph{Gen}., XXXIV, 20, 21 et 29), et tantôt \emph{Horrhæi} (Deutéron., II, 12).}. Leurs tribus avaient des communautés indépendantes. Toute l’année, errant au hasard, ils allaient volant ce qu’ils trouvaient, assassinant quand ils pouvaient. Leur taille était très élevée. Misérables à l’excès, les voyageurs les redou­taient pour leur férocité. Mais toute description pâlit en face des versets de Job, où M. d’Ewald \footnote{ \noindent Ewald, Geschichte des Volkes Israël, t. I, p. 273.\par
 Les Chorréens avaient occupé, à des époques plus anciennes, les deux rives du Jourdain jusqu’à l’Euphrate vers le nord-est et au sud jusqu’à la met Rouge.\par
 Il est d’ailleurs assez fréquemment question de ces peuplades noires dans la Genèse, le Deutéronome et les Paralipomènes, partout, enfin, où paraissent des aborigènes. Elles ne sont pas connues que sous un seul nom. Appelées Chorréens dans la Genèse, le Deutéronome les nomme aussi Emim (en hébreux) dont le singulier est (en hébreu) qui signifie terreur. Les Emim seraient donc les terreurs, les gens dont l’aspect épouvante (Deutér., II, 10 et 11). On trouve encore une tribu particulière, anciennement établie sur le territoire d’Ar, assigné depuis aux Ammonites. Ces derniers les nommaient les Zomzommin (en hébreu)Le texte décrit ainsi leur pays et eux-mêmes. (Deutér., II, 20) : « Terra gigantum « reputata est et in ipsa olim habitaverunt gigantes, quos Animonitæ vocant Zomzommim, « 21. Populus magnus et multus et proceræ longitudinis, sicut Enacim, quos delevit « Dominus a facie eorum... » Gesenius rapporte la racine de ce nom de peuple au quadrilatère inusité : (en hébreu) (murmuravit, fremuit). Enfin les Chorréens, les Emim, les Zomzommim, ces hommes de terreur et de bruit, sont toujours comparés aux Enacim, les bommes aux longs cous, les géants par excellence. Ces derniers, avant l’arrivée des Israélites, habitaient les environs d’Hébron. En partie exterminés, ce qui en survécut se réfugia dans les villes des Philistins, où on en rencontrait encore à une époque assez basse. Il n’est pas douteux que le célèbre champion qui combattit contre le berger David, Goliath (dont le nom signifie l’exilé, le réfugié), appartenait à cette famille proscrite.
} reconnaît leur portrait. Voici le passage : « Ils se moquent de moi, ceux-là même dont je n’aurais pas daigné mettre « les pères avec les chiens de mon troupeau...\par
« De disette et de faim, ils se tenaient à l’écart, fuyant dans les lieux arides, « ténébreux, désolés et déserts.\par
« Ils coupaient des herbes sauvages auprès des arbrisseaux et la racine des « genévriers pour se chauffer.\par
« Ils étaient chassés d’entre les autres hommes, et l’on criait après eux comme « après un larron.\par
« Ils habitaient dans les creux des torrents, dans les trous de la terre* et des « rochers.\par
« Ils faisaient du bruit entre les arbrisseaux, et ils s’attroupaient entre* les « chardons.\par
« Ce sont des hommes de néant et sans nom qui ont été abaissés plus* bas que « la terre. » (Job, XXX, I, 3-8).\par
Les noms de ces sauvages sont sémitiques, s’il faut absolument employer l’expression abusive consacrée ; mais, à parler d’une manière plus exacte, les langues noires en réclament la propriété directe. Quant aux êtres qui portaient ces noms, peut-on rien imaginer de plus dégradé ? Ne croit-on pas lire, dans les paroles du saint homme, une description exacte du Boschisman et du Pélagien ? En réalité, la parenté qui unissait l’antique Chorréen à ces nègres abrutis est intime. On reconnaît dans ces trois branches de l’espèce mélanienne, non pas le type même des nègres, mais un degré d’avilissement auquel cette branche de l’humanité peut seule tomber. Je veux bien admettre que l’oppression exercée par les Chamites sur ces misérables êtres, comme celle des Cafres sur les Hottentots et des Malais sur les Pélagiens, puisse être considérée comme la cause immédiate de leur avilissement. Qu’on en soit certain cependant, une telle excuse, trouvée par la philanthropie moderne à l’abrutissement et à ses opprobres, n’eut jamais besoin d’être invoquée pour les populations de notre famille. Certes les victimes n’y manquèrent pas plus que chez les noirs et les jaunes. Les peuples vaincus, les peuples vexés, tyrannisés, ruinés, s’y sont rencontrés et s’y rencontreront en foule. Mais, tant qu’une goutte active du sang des blancs persiste dans une nation, l’abaissement, quelquefois individuel, ne devient jamais général. On citera, oui, l’on citera des multitudes réduites à une condition abjecte, et l’on dira que le malheur seul a pu les y conduire. On verra ces misérables habiter les buissons, dévorer tout crus des lézards et des serpents, vaguer nus sur les grèves, perdre quelquefois la majeure partie des mots nécessaires pour former une langue, et les perdre avec la somme des idées ou des besoins que ces mots représentaient, et le missionnaire ne trouvera d’autre solution à ce triste problème que les cruautés d’un vainqueur despotique et le manque de nourriture. C’est une erreur. Qu’on y regarde mieux. Les peuples ravalés à cet infime niveau seront toujours des nègres et des Finnois, et, sur aucune page de l’histoire, les plus malheureux des blancs ne verront leur souvenir aussi honteusement consacré. Ainsi les annales primitives ne peuvent nous faire découvrir nos ancêtres blancs à l’état sauvage ; au contraire, elles nous les montrent doués de l’aptitude et des éléments civilisateurs, et voici de plus un nouveau principe qui se pose, et dont l’enchaînement des siècles nous apportera en foule d’incessantes démons­trations : jamais ces glorieux ancêtres n’ont pu être amenés par les malheurs les plus accablants à ce point déshonorant d’où ils n’étaient pas venus. C’est là, ce me semble, une grande preuve de leur supériorité absolue sur le reste de l’espèce humaine.\par
Les Chorréens cessèrent de résister et disparurent. Dépossédés du peu qui leur restait par leurs parents, fils d’Ésaü, enfants d’Oolibama, Édomites \footnote{Deutéron., II, 12 « In Seir autem prius habitaverunt Horrhæi quibus expulsis atque deletis, « habitaverunt filii Esaü, sicut fecit Israël in terra possessionis suæ, quam dedit illi « Dominus. »} ils s’éteignirent devant la civilisation, comme s’éteignent aujourd’hui les aborigènes de l’Amérique septentrionale. Ils ne jouèrent aucun rôle politique. Leurs expéditions ne furent que des brigandages. On sait par l’histoire de Goliath qu’ils n’avaient plus d’autre rôle que de servir les haines de leurs spoliateurs contre les Israélites.\par
Quant aux Juifs, ils restèrent fidèles à l’influence ninivite tant que les Sémites la dirigèrent. Plus tard, lorsque le sceptre eut passé dans les mains des Arians Zoroastriens, comme les rapports de race n’existaient plus entre les dominateurs de la Mésopotamie et les nations du sud-ouest, il put y avoir obéissance politique : il n’y eut plus communion d’idées. Mais ces considérations seraient ici prématurées. Avant de descendre aux époques où elles doivent trouver leur place, il me reste beaucoup de faits à examiner, parmi lesquels ceux qui ont trait à l’Égypte réclament immédiatement l’attention.
\section[{II.5. Les Égyptiens, les Éthiopiens.}]{II.5. \\
Les Égyptiens, les Éthiopiens.}
\noindent Jusqu’à présent il n’a encore été question que d’une seule civilisation, sortie du mélange de la race blanche des Chamites et des Sémites avec les noirs, et que j’ai appelée assyrienne. Elle acquit une influence non seulement longue, non seulement durable, mais éternelle, et ce n’est pas trop que de la considérer, même de nos jours, comme beaucoup plus importante par ses conséquences que toutes celles qui ont éclairé le monde, sauf la dernière.\par
Toutefois, à l’idée de la suprématie de domination, il serait inexact de joindre celle d’antériorité d’existence. Les plaines de l’Asie inférieure n’ont pas vu fleurir des États réguliers avant tout autre pays de la terre. Il sera question plus tard de l’antiquité extrême des établissements hindous ; pour le moment, je vais parler des gouvernements égyptiens, dont la fondation est probablement à peu près synchronique à celle des pays ninivites. La première question à débattre, c’est l’origine de la partie civilisatrice de la nation habitant la vallée du Nil.\par
La physiologie interrogée répond avec une précision très satisfaisante les statues et les peintures les plus anciennes accusent d’une manière irréfragable la présence du type blanc \footnote{Wilkinson, \emph{Customs and manners of the ancient Egyptians}, t. I, p. 3. – Cet auteur croit les Égyptiens d’origine asiatique. Il cite le passage de Pline (VI, 34) qui, d’après Juba, remarque que les riverains du Nil, de Syène à Méroé, étaient Arabes. Lepsius (\emph{Briefe aus Ægypten, Æthyopien}, etc.; Berlin, 1852) affirme le même fait pour toute la vallée du Nil jusqu’à Khartoum, peut-être même pour les populations plus méridionales encore, le long du Nil Bleu, p. 220.} On a souvent cité avec raison, pour la beauté et la noblesse des traits, la tête de la statue connue au Musée britannique sous le nom de Jeune Memnon \footnote{A. W. v. Schlegel, \emph{Vorrede zur Darstellung der Ægyptischen Mythologie}, von Prichard, übers. von Z. Haymann (Bonn, 1837), p. XIII.}. De même, dans d’autres monuments figurés, dont la fondation remonte précisément aux époques les plus lointaines, les prêtres, les rois, les chefs militaires appartiennent, sinon à la race blanche parfaitement pure, du moins à une variété qui ne s’en est pas encore écartée beaucoup \footnote{Lepsius (\emph{ouvrage cité}, p. 220) dit que les peintures exécutées dans les hypogées de l’ancien empire représentent les Égyptiennes avec la couleur jaune. Sous la XVIII\textsuperscript{e} dynastie, elles sont rougeâtres.}. Cependant, l’élargissement de la face, la grandeur des oreilles, le relief des pommettes, l’épaisseur des lèvres sont autant de caractères fréquents dans les représentations des hypogées et des temples, et qui, variés à l’extrême et gradués de cent manières, ne permettent pas de révoquer en doute l’infusion assez forte du sang des noirs des deux variétés, à cheveux plats et crépus \footnote{Parmi les nations nègres représentées et nommées sur les monuments, les Toreses, les Tarcao, les Éthiopiens ou Kush, présentent un type très prognathe et laineux, (Wilkinson, \emph{ouvrage cité}, t. I, p. 387-388.)}. Il n’y a rien à opposer, en cette matière, au témoignage des constructions de Médinet-Abou. Ainsi l’on peut admettre que la population égyptienne avait à combiner les éléments que voici : des noirs à cheveux plats, des nègres à tête laineuse, plus une immigration blanche, qui donnait la vie à tout ce mélange.\par
La difficulté est de décider à quel rameau de la famille noble appartenait ce dernier terme de l’alliage. Blumenbach, citant la tête d’un Rhamsès, le compare au type hindou. Cette observation, toute juste qu’elle est, ne saurait malheureusement suffire à fonder un jugement arrêté, car l’extrême variété que présentent les types égyptiens des différentes époques hésite beaucoup, comme il est facile de le concevoir, entre les données mélaniennes et les traits des blancs. Partout, en effet, même dans la tête attribuée à Rhamsès, des traits encore fort beaux et très voisins du type blanc sont cependant assez altérés déjà, par les effets des mélanges, pour offrir un commencement de dégradation qui déroute les idées et empêche la conviction de se fixer. Outre cette raison décisive, on ne doit jamais oublier non plus que les apparences physionomiques ne fournissent souvent que des raisons bien imparfaites, quand il s’agit de décider sur des nuances \footnote{C’est une vérité qui a frappé M. Shaffarik dans ses \emph{Slawische Alterthümer} (t. I, p. 24).}. Si donc la physiologie suffit à nous apprendre que le sang des blancs coulait dans les veines des Égyptiens, elle ne peut nous dire à quel rameau était emprunté ce sang, s’il était chamite ou arian. Elle fait assez pour nous, toutefois, en nous affirmant le fait en gros et en renversant de fond en comble l’opinion de De Guignes, d’après laquelle les ancêtres de Sésostris auraient été une colonie chinoise, hypothèse écartée aujourd’hui de toute discussion.\par
L’histoire, plus explicite que la physiologie, épouvante cependant par l’éloignement excessif dans lequel elle semble vouloir se reporter et cacher les origines de la nation égyptienne \footnote{M. Lepsius, d’accord avec M. Bunsen, s’exprime ainsi au sujet de la chronologie égyptienne : « Lorsqu’il s’agit des monuments, des sculptures et des inscriptions de la 5\textsuperscript{e} « dynastie, nous sommes transportés à une époque de florissante civilisation qui a devancé « l’ère chrétienne de \emph{quatre mille ans.} On ne saurait trop se rappeler à soi-même et redite aux a« utres cette date jusqu’ici jugée si incroyable. Plus la critique sera sollicitée sur ce « point et obligée à des recherches de plus en plus sévères, mieux cela vaudra pour la question. » (\emph{Briefe aus Ægypten}, etc., p. 36.)}. Après tant de siècles de recherches et d’efforts, on n’a pu réussir à s’entendre encore sur la chronologie des rois, sur la composition des dynasties, et encore bien moins sur les synchronismes qui unissent les faits arrivés dans la vallée du Nil aux événements accomplis ailleurs. Ce coin des annales humaines n’a jamais cessé d’être un des terrains les plus mouvants, les plus variables de la science, et à chaque instant une découverte ou seulement une théorie le déplace. Il n’y a pas à choisir ici entre les opinions brillantes de M. le chevalier Bunsen et l’allure plus modeste de sir Gardiner Wilkinson. Je me garderais de vouloir exclure les unes pour me confier uniquement à l’autre. Il se peut que la publication de la dernière partie, encore inconnue, de l’\emph{Ægyptens Stelle in der Welt-Geschichte}, élève les assertions du savant diplomate prussien à la hauteur d’une démonstration irréfragable. En attendant ce grand résultat, et malgré la tendance que je pourrais avoir à adopter avec empressement une doctrine qui se relie si bien aux opinions de ce livre, le plus prudent est, sans nul doute, de s’en tenir, pour le principal, à la manière de voir de l’auteur anglais.\par
Suivant ce dernier, il faudrait placer le moment le plus éclatant de la civilisation, des arts et de la puissance militaire de l’Égypte, à l’époque strictement historique entre le règne d’Osirtasen, roi de la 18\textsuperscript{e} dynastie, et celui du Diospolite de la 19\textsuperscript{e}, Rhamsès III, le Mi-A-Moun des monuments, c’est-à-dire entre l’année 1740 et l’année 1355 avant J.-C. \footnote{Il s’agit ici de la période postérieure à l’expulsion des Hyksos, et que l’on appelle le \emph{nouvel empire.} L’âge des pyramides est plus reculé, comme on le verra ailleurs. M. Champollion-Figeac place à l’année 2200 avant J.-C. l’avènement de la 12\textsuperscript{e} dynastie. (\emph{Égypte ancienne}, Paris, 1840.)}. Toutefois, cette splendeur n’était pas à son début. L’époque où furent construites les pyramides remonte plus haut, et c’est sur ces mystérieux témoignages que M. Bunsen a surtout fait porter ses essais de déchiffrement les plus ingénieux. Calculons, avec la méthode d’explication la plus ordinairement appliquée au récit d’Eratosthènes, que les pyramides situées au nord de Memphis, généralement tenues pour les plus anciennes, ont été construites vers l’an 2120 avant J.-C. par Suphis et son frère Sensuphis. Ainsi, en 2120 avant J.-C., l’Égypte aurait présenté déjà un état de civili­sation fort avancé et capable d’entreprendre et de conduire à bonne fin les travaux les plus étonnants accomplis jamais par la main de l’homme. L’émigration blanche avait donc eu lieu avant cette époque, puisque chaque groupe de pyramides appartient à un âge différent, et que chaque pyramide, en particulier, a dû coûter assez d’efforts pour qu’une seule génération ne pût entreprendre la construction de plusieurs \footnote{Un roi, en montant sur le trône, commençait l’érection de la pyramide qui devait un jour lui servir de tombe. Il la faisait de taille médiocre, afin d’avoir le temps de l’achever s’il survivait à la première construction, il la couvrait d’un revêtement de pierre qui la faisait croître en épaisseur et en hauteur. Ce travail achevé, il en entreprenait un tout semblable, et continuait ainsi jusqu’à la fin de ses jours. Lui mort, le revêtement commencé était seul achevé ; mais le successeur, se mettant à travailler pour son propre compte, n’en ajoutait pas d’autres. (Lepsius, \emph{Briefe aus Ægypten}, p. 42.)}.\par
Veut-on supposer qu’un rameau chamite se soit avancé jusque dans les régions du Nil, entre Syène et la mer, et y ait fondé la civilisation égyptienne ? Cette hypothèse se renverse d’elle-même. Pourquoi ces Chamites, après avoir établi un État considérable, auraient-ils rompu ensuite toute relation avec les autres peuples de leur race, en se confinant loin de la route suivie par ces derniers, par eux-mêmes, dans les migrations vers l’Afrique, loin de la Méditerranée, loin du Delta, pour inventer là, dans l’isolement, une civilisation tout égoïste, hostile sur mille points à celle des Chamites noirs ? Comment auraient-ils adopté une langue si remarquablement différente des idiomes de leurs congénères ? On ne voit pas à ces objections de réponse raisonnable. Les Égyptiens ne sont donc pas des Chamites, et il faut se tourner d’un autre côté.\par
L’ancienne langue égyptienne se compose de trois parties. L’une appartient aux langues noires. L’autre, provenant du contact de ces langues noires avec l’idiome des Chamites et des Sémites, produit ce mélange que l’on dénomme d’après la seconde de ces races. Enfin se présente une troisième partie, très mystérieuse, très originale, sans doute, mais qui, sur plusieurs points, paraît trahir des affinités arianes et une certaine parenté avec le sanscrit \footnote{ \noindent M. le baron d’Eckstein ne convient pas de ce fait très fort et trop affirmé par M. de Bohlen. Cependant il reconnaît, de la manière la plus explicite, l’origine hindoue. Voici ses expressions mêmes : « Quoique le copte soit aux antipodes du sanscrit, mille raisons me semblent toutefois « conspirer pour retrouver dans le bassin de l’Indus le siège de la primitive civilisation transportée « dans la vallée du Nil. » (Recherches \emph{historiques sur l’humanité primitive}, p. 76.)\par
 M. Wilkinson partage cet avis et considère les Égyptiens comme une colonie hindoue (t. I, p. 3).
}. Ce fait important, s’il était solidement établi, pourrait être considéré comme terminant la discussion, et pouvant servir à tracer l’itinéraire des colons blancs de l’Égypte, depuis le Pendjab jusqu’à l’embouchure de l’Indus, et de là dans la vallée supérieure du Nil. Malheureusement, bien qu’indiqué, il n’est pas clair et ne peut servir que d’indice \footnote{Il ne faut pas perdre de vue que le copte ou langue démotique, le seul secours que nous ayons pour traduire les inscriptions hiéroglyphiques, n’est qu’un dialecte, une dégénération, une sorte de mutilation de la langue sacrée, et il faudrait savoir si les traces sanscrites ne sont pas plus abon­dantes dans ce plus ancien idiome. – Voir Brugsch, \emph{Zeitschrift der deutschen morgenlændischen Gesellschaft}, t. III, p. 266.}. Cependant il n’est pas impossible de lui trouver des étais.\par
On a considéré longtemps les contrées basses de l’Égypte comme ayant fait partie primitive du pays de Misr. C’était une opinion erronée. Les lieux où la civilisation égyptienne établit ses plus anciennes splendeurs, sont tout à fait au-dessus du Delta. En dehors de la côte arabique, parce que le caractère stérile du sol n’y permettait pas de vastes établissements, la colonisation antique ne s’en écarte cependant pas trop et ne cherche pas encore à gagner les rives de la Méditerranée. C’est que, probablement, elle ne voulait pas rompre toute relation avec l’ancienne patrie. Malgré les sables, malgré les rocs qui bordent le golfe par où l’immigration avait pu se faire, des ports de commerce existaient sur ces rivages, entre autres, Philotéras \footnote{Wilkinson, t. I, p. 225 et pass.}, tous reliés au centre fertile où se mouvaient principalement les populations, au moyen de stations établies dans le désert, Wadi-Djasous, par exemple, dont on sait que les puits furent réparés par Amounm-Gori (1686 avant J.-C., suivant Wilkinson ; à une date plus ancienne, au dire de M. le chevalier Bunsen), et lorsque les Égyptiens ne possédaient rien du côté de la Palestine. Il y a même lieu de croire que les mines d’émeraudes de Djebel-Zabara étaient déjà exploitées avant cette époque. Dans les tombeaux des Pharaons de la 18\textsuperscript{e} dynastie, le lapis-lazuli et d’autres pierres précieuses, originaires de l’Inde, se rencontrent en abondance. Je ne parle pas ici des vases de porcelaine, venus indubitablement de la Chine, et découverts dans des hypogées dont la date de fondation est inconnue. Cette dernière circonstance suffit, à elle seule, pour donner le droit d’attribuer ces monuments et leur contenu à une époque très reculée \footnote{Id. \emph{Ibid.}, p. 231.}.\par
De ce que les Égyptiens étaient établis dans le centre de la vallée du Nil, je conclus qu’ils n’appartenaient pas aux nations chamites et sémites, dont la route vers l’Afrique occidentale était, au contraire, la rive méditerranéenne. De ce qu’ils portent, dans toutes les représentations figurées, le caractère évidemment caucasien, je conclus que la partie civilisatrice de la nation avait une origine blanche. Des traces arianes qui se trouvent dans leur langue, je conclus aussi, dès à présent, leur identité primitive avec la famille sanscrite. À mesure que nous allons avancer dans l’examen du peuple d’Isis, de nombreux détails vont confirmer, l’un après l’autre, ces prémisses.\par
J’ai montré qu’aux époques historiques les plus lointaines, les Égyptiens n’avaient que peu ou point de rapports avec les peuples chamites ou sémites et les contrées habitées par ces peuples ; tandis qu’au contraire, ils paraissent avoir entretenu des relations suivies avec les nations maritimes du sud-est. Leur activité se tournait si naturellement de ce côté, les transactions qui en résultaient avaient un tel degré d’importance, qu’au temps de Salomon le commerce entre les deux pays dépassait, pour un seul voyage d’importation, la valeur de 80 millions de nos francs \footnote{Id. \emph{ibid}., p. 225 et pass.}.\par
Tout en constatant l’origine sanscrite du noyau civilisateur de la race, il ne faudrait pas nier que, dès une époque très ancienne, cette race ne se soit fortement imprégnée du sang des noirs et mêlée aussi à de nombreux essaims chamites et à des fils de Sem. J’ai cité, sur ce point, l’autorité de Juba, qui reconnaît aux riverains du Nil, de Syène à Méroé, une provenance arabe \footnote{La Genèse trouve des Sémites parmi les fils de Mesraïm, fils de Cham : « At vero Mesraïm « genuit Ludin et Anamim, et Laabim Nephtuïm et Phetrusim et Chasluim ; de quibus « egressi sunt Philistiim, et Caphtorim (X, 13, 14). »}. Malgré cette descendance multiple, les Égyptiens se croyaient et se disaient autochtones. Ils l’étaient en effet, en tant qu’héritiers, par le sang des aborigènes mélaniens. Cependant, si l’on veut s’attacher à la partie la plus noble de leur généalogie, on se refusera à partager leur opinion, et, persistant à les considérer comme des immigrants, non pas tant du nord et de l’est que du sud-est, on relèvera dans la constitution de leurs mœurs les traces très apparentes de la filiation que l’ignorance leur faisait renier.\par
À la religion féroce des nations assyriennes les Égyptiens opposaient les magni­ficences d’un culte, sinon plus idéal, au moins plus humain, et qui, aptes avoir aboli au temps de l’ancien empire, sous les premiers successeurs de Menès \footnote{M. de Bohlen a trouvé entre le fondateur de la royauté égyptienne et le législateur mythique de l’Inde, Manou, un grand rapport de noms.}, l’usage nègre des massacres hiératiques, n’avait jamais osé tenter de le faire renaître.\par
Les principes généraux de l’art religieux pratiqués à Thèbes et à Memphis ne craignaient certainement pas de produire le laid, mais ils ne cherchaient pas trop l’horrible, et bien que l’image de Typhon et d’autres encore soient assez repoussantes, la divinité égyptienne affectionne les formes grotesques plutôt que les contorsions de la bête sauvage, ou les grimaces du cannibale. Ces déviations de goût, mêlées à un véritable caractère de grandeur et commandées évidemment par la quantité noire infusée dans la race, étaient dominées par la valeur spéciale de la partie blanche, qui, supérieure autant qu’on en doit juger, d’après ce fait même, à l’affluent chamo-sémite, se montrait plus douce, et forçait l’élément noir à abonder dans le ridicule, en abandonnant l’atroce.\par
Il y aurait pourtant exagération à trop louer les populations riveraines du Nil. Si, au point de vue de la moralité, on doit féliciter une société d’être plus ridicule que méchante, à celui de la force, il faut l’en plaindre. Les nations assyriennes eurent le coupable malheur d’abâtardir leurs consciences aux pieds des monstrueuses images d’Astarté, de Baal, de Melkart, de ces idoles horribles trouvées dans le sol de la Sardaigne comme sous le seuil des portes de Khorsabad ; mais les gens de Thèbes et de Memphis furent, de leur côté, assez ravalés, par leur alliance avec la race aborigène, pour prostituer leur adoration à ce qu’ont de plus humble et le règne végétal et la nature animale. Ne parlons pas ici de la \foreign{cobra di capello}, dont le culte symbolique, commun aux populations de l’Inde et de l’Égypte, n’était peut-être qu’une importation de la mère patrie \footnote{Schlegel, \emph{Préface à la Mythologie Égyptienne} de Prichard, p. XV. – Une différence ave les Hindous que M. de Schlegel trouve radicale, c’est la circoncision. Les Hindous ne connaissaient pas cet usage pratiqué en Égypte et dans lequel on voit, à tort, une coutume judaïque. Comme le tatouage, c’est une idée originairement nègre et tout à fait conforme aux notions de cette espèce. Le but hygiénique, par lequel on cherche à la justifier ou à l’expliquer aujourd’hui, me semble peu admissible, soit que la circoncision ait lieu sur les hommes seulement ou sur les hommes et les femmes sans distinction, comme on le voit dans plusieurs tribus africaines. Je ne reconnais dans l’origine de cette coutume que le désir de créer une marque distinctive, ou, peut-être même, uniquement un simple dérivé du goût natif pour la mutilation, que, suivant les temps et les lieux, les populations qui l’ont adopté ont expliqué à leur guise. Chez les Ekkhilis, la circoncision se pratique sur les adultes et d’une manière atroce. L’opérateur arrache la peau du prépuce, en présence des parents et de la fiancée de la victime. La moindre marque de douleur est considérée comme déshonorante. Souvent le tétanos emporte le malade au bout de quelques jours.}. Laissons aussi en dehors les crocodiles et tout ce qui peut se faire craindre, culte éternel de qui a du sang des noirs dans les veines. L’infatuation pour des êtres inoffensifs, comme le bouc, le chat, le scarabée ; pour des légumes qui n’offraient rien que de très vulgaire dans leurs formes et dans leurs mérites : voilà ce qui est particulier à l’Égypte, de sorte que l’influence nègre, tout en s’y montrant apprivoisée, ne s’y faisait pas moins sentir que dans le Chanaan et sur les terres de Ninive. L’absurde régnait seul ; il n’en était que plus complet et l’action mélanienne, si naturellement puissante, ne différait d’intensité et de forme qu’au gré de la valeur particulière à l’influence blanche, qui la dirigeait encore en se laissant obscurcir par elle. De là les différences des deux nationalités assyrienne et égyptienne.\par
Je ne confonds pas, tout à fait, le culte d’Apis, ni surtout le respect profond dont la vache et le taureau étaient l’objet, avec le culte des végétaux. L’adoration, en tant qu’hommage rendu à la Divinité, est un témoignage de respect un peu excessif, sans doute ; et quand on le donne à la chose créée, le sentiment d’où naît cette erreur peut fort bien se rapporter à la même source que les autres apothéoses condamnables \footnote{Le lecteur a déjà remarqué peut-être que les nations modernes sont les seules qui aient su tracer une barrière exacte entre le respect et l’adoration. Soit qu’il provienne de la crainte ou de l’amour, le respect des peuples mélangés fortement de noir ou de jaune va facilement à l’extrême. Chez les uns, il crée la divinisation pure et simple ; chez les autres, le culte superstitieux des ancêtres.}. Mais, au fond de la sympathie égyptienne pour la race bovine, il y a quelque chose d’étranger au pur et simple fétichisme. On doit sans scrupule le rattacher aux antiques habitudes pastorales de la race blanche, et, comme à la vénération rendue à la cobra di capello, lui assigner une origine hindoue. C’est une folie dont la source n’est pas grossière.\par
Je ferais la même réserve pour d’autres similitudes très frappantes, telles que le personnage de Typhon, l’amour du lotus et, avant tout, la physionomie particulière de la cosmogonie qui se rapproche tout à fait des idées brahmaniques. À la vérité, il est quelquefois dangereux d’ajouter une foi trop explicite aux conclusions tirées de comparaisons semblables. Les idées peuvent souvent voyager à demi mortes et venir se régénérer sur un terrain propre à les faire réussir, après avoir passé par bien des milieux. Ainsi se trouveraient déçues les espérances que l’on aurait pu concevoir de leur présence à deux points extrêmes, pour constater une identité de race chez leurs possesseurs différents. Cette fois, cependant, il est difficile de se tenir en méfiance. L’hypothèse la plus défavorable à la communication directe entre les Hindous et les Égyptiens serait de supposer que les notions théologiques des premiers seraient passées du territoire sacré dans la Gédrosie, de là chez les diverses tribus arabes, pour tomber enfin chez les seconds. Or, les Gédrosiens étaient de misérables barbares, détritus immondes des tribus noires \footnote{À une époque assez basse, les Arians ont poussé jusque chez ces peuplades. Ils n’ont fait que passer et n’ont laissé aucune trace de leur séjour. (Lassen. \emph{Indisch. Alterth.}, t. I, p. 533.)}. Les Arabes s’adonnaient entièrement aux notions des Chamites, et on ne trouve pas trace, parmi eux, de celles dont il s’agit. Ces dernières venaient donc directement de l’Inde, sans transmission intermédiaire. C’est un grand argument de plus en faveur de l’origine ariane du peuple des Pharaons.\par
Je ne considérerai pas tout à fait comme aussi concluante une particularité qui, au premier aspect, frappe cependant beaucoup. C’est l’existence, dans les deux pays, du régime des castes. Cette institution semble porter en elle un tel cachet d’originalité, qu’elle donne toutes les tentations possibles de la considérer comme ne pouvant être que le résultat d’une source unique, et de conclure de sa présence chez plusieurs peuples à leur identité originelle. Mais, en y réfléchissant un peu, on n’a pas de peine à se convaincre que l’organisation généalogique des fonctions sociales n’est qu’une conséquence directe de l’idée d’inégalité des races entre elles, et que partout où il y a eu des vainqueurs et des vaincus, principalement quand ces deux pôles de l’État ont été visiblement séparés par des barrières physiologiques, le désir est né chez les forts de conserver le pouvoir à leurs descendants, en les contraignant de garder pur, autant que possible, ce même sang dont ils regardaient les vertus comme l’unique cause de leur domination. Presque tous les rameaux de la race blanche ont essayé, un moment, l’ébauche de ce système exclusif, et s’ils ne l’ont pas généralement poussé aussi loin que les gardiens des Védas et les sectateurs d’Osiris, c’est que les populations au milieu desquelles ils se trouvaient leur étaient déjà parentes de trop près quand ils se sont avisés de se rendre inaccessibles. Sous ce rapport, toutes les sociétés blanches s’y sont prises trop tard ; les Égyptiens, comme les autres, et même les Brahmanes. Leur prétention ne pouvait naître qu’après expérience faite des inconvénients à éviter. Elle ne constituait, dès lors, qu’un effort plus ou moins impuissant.\par
Ainsi, l’existence des castes ne suppose pas en elle-même l’identité des peuples, puisqu’elle existe chez les Germains, chez les Étrusques, chez les Romains comme à Thèbes, tout comme à Videha. Cependant on pourrait répondre que, si l’idée séparatiste doit se produire partout où deux races inégales sont en présence, il n’en est pas de même des applications variées qui en ont été faites, et on insistera sur cette grande ressemblance dans les systèmes de l’Égypte et de l’Inde : la contrainte perpé­tuelle des lignées au métier de leurs ancêtres. C’est là, en effet, le rapport. Il y a aussi la dissemblance, et la voici : en Égypte, pourvu qu’un fils remplît les mêmes fonctions que son père, la loi était satisfaite ; la mère pouvait sortir de toute descendance, sauf d’une famille de bergers. Cette exception contre les gardiens de troupeaux, corollaire forcé de cette autre qui leur fermait l’entrée des sanctuaires, confirme très bien la tolérance de la règle. Du reste, les exemples abondent. Des rois épousent des négresses, témoin Aménoph 1\textsuperscript{er}. Des rois sont mulâtres comme Aménoph II, et la société, fidèle à la lettre de l’institution, ne paraît nullement avoir pris soin d’en observer, ni même d’en comprendre l’esprit.\par
Enfin, voici deux preuves dernières, et ce sont certainement les plus fortes.\par
Les annales égyptiennes donnent la date de l’institution des castes et en font honneur à un de leurs premiers rois, le troisième de la 3\textsuperscript{e} dynastie, le Sésonchosis du scoliaste des Argonautiques, le Sésostris d’Aristote.\par
Second argument : l’antiquité si haute à laquelle il faudrait reporter l’époque où les émigrants arians quittèrent les bouches de l’Indus pour se diriger vers l’ouest, rend inadmissible l’origine sanscrite de la loi, attendu qu’alors elle n’existait certainement pas dans le pays même auquel se rattache, à son sujet, une sorte de réputation classique.\par
Je viens de prouver que je ne cherche pas à renforcer mon opinion d’un argument que je juge fragile. Maintenant j’ajouterai qu’en me prononçant contre toutes les conclusions directes à tirer de l’existence simultanée des castes dans l’Inde et en Égypte, je ne prétends nullement affirmer que certaines inductions collatérales ne s’en puissent extraire, qui ne laissent pas que de corroborer d’une manière fort utile le principe de la communauté d’origine : telle est la vénération égale pour les ministres du culte, leur longue domination et la dépendance dans laquelle ils ont su retenir la caste militaire, même quand celle-ci a porté la couronne, triomphe que le sacerdoce chamite n’a pas su remporter, et qui fit également la gloire, la force des civilisations de l’Indus et du Nil. C’est que la race ariane est surtout religieuse. Il faut encore observer l’intervention constante des prêtres dans les habitudes et les actes les plus intimes du foyer domestique \footnote{Schlegel, \emph{ouvrage cité}, p. XXIV.}. En Égypte, ainsi que dans l’Inde, on voit les hommes des temples réglementer tout, jusqu’au choix des aliments, et établir, à ce sujet, une discipline à peu près pareille. Bref, et bien que le nombre des castes ne corresponde pas, la hiérarchie en est assez semblable sur les deux territoires \footnote{Wilkinson, t. I, p. 237 et pass. Il n’y avait, en Égypte, de caste réellement impure que la subdivision des porchers. Suivant Hérodote, on comptait sept classes ; suivant Diodore, trois ou cinq. Strabon en nomme trois ; Platon, dans le Timée, six, avec des subdivisions de métiers, d’arts, etc.} C’est là tout ce qu’il peut être utile de remarquer sur des faits, en apparence secondaires, mais qui ont cet avantage de se laisser très bien rapprocher, fragments séparés d’une primitive unité sinon d’institu­tions, du moins d’instincts, en même temps que de sang.\par
Les plus anciens monuments de la civilisation égyptienne se trouvent dans les parties haute et moyenne du pays \footnote{Une des capitales de l’ancien empire, c’est Thèbes, \emph{Tapou}. Elle fut fondée par Sesortesen 1\textsuperscript{er}, premier roi de la dynastie thébaine, la 12\textsuperscript{e} de Manéthon, 2,300 ans av. J.-C. (Lepsius, \emph{Briefe aus Ægypten}, p. 272.)}. Négligeant le nord et le nord-est, les premières dynasties ont laissé des traces d’une prédilection évidente pour la direction contraire, et leurs communications avec l’Inde ont dû nécessairement multiplier leurs rapports avec les contrées situées sur cette toute, telles que la région des Arabes Kuschites, la côte orientale de l’Afrique et, peut-être, quelques-unes des grandes îles de l’Océan \footnote{Rosellini a trouvé le nom de Sesortesen (M. de Bunsen, Orsitasen 1\textsuperscript{er} de Wilkinson), sur une stèle en Nubie, près de Wadi-Halfa. Ce même prince avait également envahi la presqu’île du Sinaï. (Bunsen, t. II, p. 307. Voir aussi Lepsius, \emph{Briefe aus Ægypten}, etc., p. 336 et pass.) –L’exploitation des mines de cuivre du Sinaï a commencé sous l’ancien empire. C’est alors qu’elle eut le plus d’importance.}.\par
Cependant rien n’indique sur tous ces points, excepté la presqu’île du Sinaï, une action régulièrement dominatrice, et il n’en est pas de même si l’on se tourne vers le sud et vers l’ouest africain \footnote{Movers, t. II, 1\textsuperscript{re} partie, p. 301.}. Là, les Égyptiens apparaissent comme des maîtres. Aussi le théâtre principal de l’ancienne civilisation égyptienne laisse-t-il le Nil descendre jusqu’à la mer sans s’étendre avec son cours inférieur ; tandis qu’il le remonte au delà de Méroé et le quitte même pour s’avancer dans la région occidentale, sous les palmiers de l’oasis d’Ammon.\par
Les anciens se rendaient compte de cette situation lorsqu’ils attribuaient la dénomination géographique de \emph{Kousch} \footnote{Wilkinson, t. I, p. 4. Movers, t. II, 1\textsuperscript{re} partie, 282. Ce nom s’appliquait aussi au Nedj et à l’Yémen. Il s’étendait encore à la partie de l’Asie la plus voisine. L’Écriture sainte fait de Nemrod un Kuschite.}, tant à la haute Égypte et à une partie de l’Égypte moyenne qu’à l’Abyssinie, à la Nubie et aux districts de l’Yémen habités par les descendants des Chamites noirs. Faute de s’être placé à ce point de vue, on s’est beaucoup inquiété de la véritable valeur de ce nom, et trop souvent on s’est épuisé sur la tâche impossible de lui créer une signification topographique positive. Il en est de ce mot comme de tant d’autres : Inde, Syrie, Éthiopie, Illyrie, appellations vagues qui ont sans cesse varié suivant les temps et les mouvements de la politique. Le mieux qu’on puisse faire, c’est de ne pas chercher à leur attribuer une rectitude scientifique que leur bon usage ne comporte pas. Je ne ferai donc nul effort pour préciser les frontières de ce pays de Kousch, en tant que l’Éthiopie est ainsi désignée, et, considérant que, parmi les territoires qu’il embrasse, l’Égypte, incontestablement, prend le pas sur tous les autres, et les rallie autour de ses provinces supérieures dans une civilisation commune, je profiterai de ce que le mot existe, pour faire observer qu’il pourrait être employé très justement à dénommer et le foyer et les conquêtes de cette antique culture, si exclusi­vement tournée vers le sud, et étrangère aux rivages de la Méditerranée.\par
Les pyramides sont les restes imposants de cette gloire primitive. Elles furent construites par les premières dynasties qui, s’étendant depuis Ménès jusqu’à l’époque d’Abraham et un peu au-dessous, se sont, jusqu’à présent, si bien prêtées à la discus­sion et si peu à la certitude \footnote{Parmi les pyramides les plus anciennes, plusieurs sont construites en briques crues, ce qui les identifie presque avec les tumulus des peuples blancs primitifs. (Wilkinson, t. I, p. 50.)}. Tout ce qu’il est utile d’en remarquer ici, c’est que là, comme en Assyrie, le gouvernement commence par être exercé par les dieux, des dieux passent aux prêtres, des prêtres tombent aux chefs militaires \footnote{Les plus anciens noms, dans les ovales, sont précédés du titre de prêtre au lieu de celui de roi. (Wilkinson, t. I, p. 19.)}. C’est l’idée nègre qui reparaît dans la même forme et suscitée par des circonstances toutes semblables. Les dieux, ce sont les blancs, les prêtres, les mulâtres de la caste hiératique. Les rois, ce sont les chefs armés, autorisés par la communauté d’origine blanche à prétendre au partage de l’empire, c’est-à-dire à s’emparer du gouvernement des corps en laissant celui des âmes à leurs rivaux. On peut supposer que la lutte fut longue et bien soutenue, que les pontifes ne se laissèrent pas aisément arracher la couronne ni chasser du trône, car la royauté militaire eut tous les caractères, non d’une victoire, mais d’un compromis. Le souverain pouvait appartenir indifféremment à l’une ou l’autre caste, celle des pontifes ou celle des guerriers. C’est la concession. La restriction la suit : si le souverain était de la seconde catégorie, il lui fallait, avant que d’entrer en jouissance des droits royaux, se faire admettre parmi les desservants des temples et s’instruire dans les sciences du sanctuaire \footnote{Wilkinson, t. I, p. 246.}. Une fois devenu hiérophante de forme et de fait, et seulement alors, le soldat heureux pouvait s’appeler roi, et, pendant tout le reste de sa vie, témoignant d’un respect sans bornes pour la religion et le sacerdoce, il devait, dans sa conduite privée et ses habitudes les plus intimes, ne s’écarter jamais des règles dont les prêtres étaient les auteurs et les gardiens. Jusqu’au fond du retrait le plus particulier de l’existence royale, les rivaux du maître avaient les yeux fixés. Quand il s’agissait d’affaires publiques, la dépendance était plus étroite encore. Rien ne s’exécutait sans la participation de l’hiérophante : membre du conseil souverain, sa voix avait le poids des oracles, et comme si tous ces liens de servitude eussent paru trop faibles encore pour sauvegarder cette part si énorme de pouvoir, les rois savaient qu’après leur mort ils auraient à subir un jugement, non pas de la part de leurs peuples, mais de la part de leurs prêtres ; et chez une nation qui avait sur l’existence d’au delà du tombeau des idées si particulières, on peut aisément s’imaginer quelle terreur entretenait dans l’esprit du despote le plus audacieux l’idée d’un procès qui, suscité à son cadavre impuissant, pouvait le priver du bonheur le plus désirable au gré des idées nationales, une sépulture magnifique et les derniers honneurs. Ces juges futurs étaient donc constamment redoutables, et ce n’était pas trop de prudence que de les ménager pendant toute la vie \footnote{Wilkinson, t. I, p. 250.}.\par
L’existence d’un roi d’Égypte ainsi enchaînée, surveillée, contrariée sur les points les plus importants comme dans les détails les plus futiles, aurait été intolérable, si quelque dédommagement ne lui avait été offert. Les droits religieux mis à part, le monarque était tout-puissant, et ce que le respect a de plus raffiné lui était constam­ment offert par les peuples à genoux. Il n’était pas Dieu, sans doute, et on ne l’adorait pas de son vivant ; mais on le vénérait en tant qu’arbitre absolu de la vie et de la mort, et aussi comme personnage sacré, car il était pontife lui-même. À peine les plus grands de l’État étaient-ils assez nobles pour le servir dans les plus humbles emplois. C’était à ses fils que revenait l’honneur de courir derrière son char, dans la poussière, en portant ses parasols.\par
Ces mœurs n’étaient pas sans rapport avec ce qui se passa en Assyrie. Le caractère absolu du pouvoir, et l’abjection qu’il imposait aux sujets, se rencontraient aussi très complètement à Ninive. Pourtant l’esclavage des rois vis-à-vis des prêtres ne paraît pas y avoir existé, et si l’on se tourne vers un autre rameau des Sémo-Chamites noirs, si l’on regarde à Tyr, on y trouve bien un roi esclave ; mais c’est une aristocratie qui le domine, et le pontife de Melkart, apparaissant dans les rangs des patriciens comme une force, n’y représente pas la force unique ou dominante.\par
À considérer similitudes et dissemblances au point de vue ethnique, les similitudes se montrent dans l’abaissement des sujets et dans l’énormité du pouvoir. La prérogative exercée sur des êtres brutaux est complète en Égypte comme en Assyrie, comme à Tyr. La raison en est que, dans tous les pays où l’élément noir se trouva ou se trouve soumis au pouvoir des blancs, l’autorité emprunte un caractère constant d’atrocité, d’une part, à la nécessité de se faire obéir d’êtres inintelligents, et, d’autre part, à l’idée même que ces êtres se font des droits illimités de la puissance à leur soumission.\par
Pour les dissemblances, leur source est en ceci que le rameau civilisateur de l’Égypte était supérieur en mérite aux branches de Cham et de Sem. Dès lors, les Sanscrits Égyptiens avaient pu apporter, dans le pays de leur conquête, une organisation assez différente et certainement plus morale ; car ce n’est pas un point à controverser que, partout où le despotisme est le seul gouvernement possible, l’autorité sacerdotale, même poussée à l’extrême, a toujours les résultats les plus salutaires, parce que, du moins, est-elle toujours plus trempée d’intelligence.\par
Après les rois et les prêtres de l’Égypte, il ne faut pas oublier les nobles, qui, pareils aux Kchattryas de l’Inde, avaient seuls le droit de porter les armes et l’emploi de défendre le pays. En supposant qu’ils s’en soient acquittés avec distinction, ils paraissent avoir mis non moins d’énergie à opprimer leurs inférieurs : je viens de l’indiquer tout à l’heure, et il n’est pas mal à propos d’y revenir. Le bas peuple de l’Égypte était aussi malheureux que possible, et son existence, à peine garantie par les lois, se trouvait constamment exposée aux violences des hautes classes. On le contrai­gnait à un travail sans relâche ; l’agriculture dévorait et ses sueurs et sa santé ; logé dans de misérables cabanes, il y mourait de fatigue et de maladie sans que personne s’en préoccupât, et des admirables moissons qu’il produisait, de fruits merveilleux qu’il faisait croître, rien ne lui appartenait. À peine lui en était-il accordé une part insuffi­sante à sa nourriture. Tel est le témoignage porté sur l’état des basses classes en Égypte par les écrivains de l’antiquité grecque \footnote{Hérodote, 11, 47.}. À la vérité, on peut citer également, dans un sens contraire, les lamentations des Israélites fatigués de manger la manne du désert. Ces nomades regrettèrent alors les oignons de la captivité. Mais aussi incrimine-t-on avec justice les murmures de la nation coupable, comme provenant d’un excès incon­cevable de bassesse et d’abattement. Ceux qui proféraient ces blasphèmes oubliaient qu’ils n’avaient quitté le pays de Misr que pour fuir une oppression devenue exorbitante, qui n’était, à peu de chose près, que le régime ordinaire du peuple indigène. Mais celui-ci était impuissant à imiter les enfants d’Israël dans leur Exode, et, né d’une race infiniment moins noble, il sentait aussi beaucoup moins sa misère. La fuite des Israélites, envisagée à ce point de vue, n’est pas un des moindres exemples de la résolution avec laquelle le génie des peuples alliés de près à la famille blanche sait éviter de descendre jusqu’à un trop profond degré d’avilissement.\par
Ainsi le régime politique imposé à la population inférieure était au moins aussi dur en Égypte que dans les pays chamites et sémites, quant à l’intensité de l’esclavage et à la nullité des droits des sujets. Pourtant, au fond il était moins sanguinaire parce que la religion, clémente et douce, ne réclamait pas les homicides horreurs où se complaisaient les dieux de Chanaan, de Babylone et de Ninive \footnote{Le sort des prisonniers semble avoir été moins dur. M. Wilkinson l’affirme. On ne les voit pas, comme sur les monuments ninivites, traînés par les vainqueurs au moyen d’un anneau passé dans la lèvre inférieure. Ils étaient vendus et devenaient esclaves. (Wilkinson, t. I, p. 403 et passim.)}. Sous ce rapport, le paysan, l’ouvrier, l’esclave égyptiens étaient moins à plaindre que la tourbe asiatique ; sous ce rapport seul, et si ces misérables ne devaient pas craindre de tomber jamais sous le couteau saint du sacrificateur, ils rampaient toute leur vie aux pieds des hautes castes.\par
On les employait, eux aussi, comme des bêtes de somme, pour exécuter ces gigantesques travaux que tous les siècles admireront. C’étaient eux qui charriaient les blocs destinés à l’érection des statues et des obélisques monolithes. C’était cette population noire ou presque noire dont la foule mourait en creusant les canaux, tandis que les castes plus blanches imaginaient, ordonnaient et surveillaient l’ouvrage, et, lorsqu’il était achevé, en recueillaient justement la gloire. Que l’humanité gémisse d’un si terrible spectacle, c’est à propos ; mais, après un tribut suffisant d’indignation et de regrets, on apprécie les terribles raisons qui forçaient les masses populaires de l’Égypte et de l’Assyrie à s’accommoder patiemment d’un joug aussi durement imposé : il y avait chez la plèbe de ces pays nécessité ethnique invincible de subir les caprices de tous les maîtres, à cette condition cependant que ces maîtres conserveraient le talisman qui leur assurait l’obéissance, c’est-à-dire, assez du sang des blancs pour justifier leurs droits à la domination.\par
Cette condition fut certainement remplie dans les belles périodes de la puissance égyptienne. Aux plus illustres moments de l’empire d’Assyrie, les trônes de Babylone et de Ninive ne voyaient pas défiler sous les yeux des rois de plus nobles profils que ceux dont on admire encore la majesté sur les sculptures de Beni-Hassan \footnote{Le type de l’Égypte était fixé sous la troisième dynastie qui, suivant M. Bunsen commença quatre-vingt-dix ans après la première. (Bunsen, \emph{Ægyptens Stelle in der Weltgeschichte}, t. III, p. 7.)}.\par
Mais il est bien évident que cette pureté, d’ailleurs relative, ne pouvait pas durer indéfiniment. Les castes n’étaient pas organisées de manière à la conserver d’une manière suffisante. Aussi n’est-il pas douteux que, si la civilisation égyptienne n’avait eu d’autre raison d’exister que la seule influence du type hindou auquel elle devait la vie, elle n’aurait pas eu la longévité qu’on peut lui attribuer, et longtemps avant Rhamsès III, qui termine l’ère de plus grande splendeur, longtemps avant le XIII\textsuperscript{e} siècle avant J.-C., la décadence aurait commencé.\par
Ce qui soutint cette civilisation, ce fut le sang de ses ennemis asiatiques, chamites et sémites, qui, à plusieurs reprises et de différentes façons, vinrent quelque peu la régénérer. Sans se prononcer d’une manière rigoureuse sur la nationalité des Hyksos, on ne peut douter qu’ils n’appartinssent à une race alliée à l’espèce blanche \footnote{Dans les hypogées de Beni-Hassan on voit des peintures représentant des combats de gladiateurs d’une carnation très claire, avec les yeux bleus, la barbe et les cheveux rougeâtres. M. Lepsius considère ces figures comme étant les images d’hommes de race sémitique, probablement ancêtres des Hyksos (Lepsius, \emph{Reise in Ægypten}, etc., p. 98.). – Avant de renverser l’ancien empire et de forcer les dynasties égyptiennes à chercher un refuge en Éthiopie, les Hyksos avaient commencé par s’établir pacifiquement dans le pays, et très probablement ils s’étaient mêlés à la population indigène. – Je remarquerai, en passant que, d’après le témoignage des monuments que je cite, les contrées de l’Asie antérieure possédaient, dans l’âge des Pharaons, certains groupes de populations beaucoup plus blanches qu’aujourd’hui. Elles ne faisaient, pour ainsi dire, que de descendre des montagnes du nord et n’avaient encore contracté qu’un nombre limité d’alliances avec l’espèce mélanienne.}. Au point de vue politique, leur arrivée fut un malheur, mais un malheur qui rafraîchit pourtant le sang national et en raviva l’essence. Les guerres avec les peuples asiatiques, soutenues longtemps à égalité, bien qu’il soit prudent de douter beaucoup de ces conquêtes étendues jusqu’à la mer Caspienne, dont l’Asie n’offre de traces ni dans son histoire ni dans ses monuments, ces guerres des Sésostris, des Rhamsès et autres princes heureux, firent affluer, dans les nomes de l’intérieur, les captifs de Chanaan, d’Assyrie et d’Arabie, et leur sang, bien que mêlé lui-même, tempéra quelque peu la sauvagerie du sang des noirs, que les basses classes, et surtout le voisinage et le contact intime avec les tribus abyssines et nubiennes, versaient incessamment dans les veines de la nation.\par
Puis, il faut tenir compte de ce double courant chamite et sémite qui, pendant tant de siècles, longea l’Égypte moyenne et la pénétra. Ce fut par cette voie que les hordes à demi blanches s’étendirent sur la côte occidentale de l’Afrique, et la population qui s’y forma apporta plus tard à l’État des successeurs de Ménès une race mêlée, dans laquelle le sang hindou n’existait pas, et qui tirait tout son mérite des mélanges multipliés avec les groupes civilisateurs de l’Asie inférieure.\par
De ces alluvions successives de principes blancs naquirent les nations qui défen­dirent la civilisation kouschite d’une disparition trop prématurée, et en même temps, comme ces alluvions ne furent jamais fort riches, l’esprit égyptien put se tenir toujours à distance des notions démocratiques finalement triomphantes à Tyr et à Sidon, parce que sa populace ne s’éleva jamais à une telle amélioration de sang, qu’elle pût concevoir la pensée ambitieuse et acquérir la faculté de devenir l’égale de ses maîtres. Toutes les révolutions se passèrent entre les castes supérieures. L’organisation hiératique et royale ne se vit pas attaquée. Si quelquefois des dynasties mélaniennes, comme celle dont Tirhakah fut le héros \footnote{Wilkinson, t. I, p. 140. – Les deux prédécesseurs de Tirhakah, Éthiopiens comme lui, étaient Sabakoph et Shebek. Tirhakah, d’ailleurs, rendit hommage au génie égyptien en retournant, de lui-même, en Éthiopie (Lepsius, p. 275). Espèce de Mantchou, il n’avait jamais régné, aussi bien que ses prédécesseurs de même sang, qu’à la façon antique du pays.}, parurent à la tête du gouvernement d’un nome, leur triomphe fut court : ce ne fut qu’une élévation profitable à certains chefs, élévation résultant des jeux fortuits de la politique, et qui n’inspira jamais à ceux qu’elle glorifiait la tentation d’user de leur omnipotence pour établir cette égalité de droits cherchée par les groupes, en effet à peu près égaux, qui se querellaient dans les rues et sur les places des villes de la Phénicie. C’est ainsi que se précisent les causes de la stabilité égyptienne.\par
Cette stabilité devint de très bonne heure de la stagnation, parce que l’Égypte ne grandit réellement que tant que persista la suprématie du rameau hindou qui l’avait fondée : ce que les autres races blanches lui procurèrent de secours suffit pour prolonger sa civilisation, et non pour la développer.\par
Néanmoins, même dans la décadence, et bien que l’art égyptien des temps postérieurs à la 19\textsuperscript{e} dynastie, c’est-à-dire à Ménéphthah (1480 avant J.-C.), ne pré­sente plus qu’à de lointains intervalles des monuments dignes de rivaliser par la beauté de l’exécution, et jamais plus par le grandiose, avec ceux des âges précédents \footnote{Wilkinson, t, I, p. 22, 85 et passim, 165 et passim, 206 et passim, W. v. Humboldt, \emph{Ueber die Kawi-Sprache}, t. I, p. 60.}, néanmoins, dis-je, l’Égypte resta toujours tellement au-dessus des pays situés au sud et au sud-ouest de son territoire, qu’elle ne cessa pas d’être pour eux le foyer d’où émanait leur vie.\par
Cette prérogative civilisatrice fut loin cependant d’être absolue, et, pour ne pas errer, il est nécessaire de remarquer que la civilisation de l’Abyssinie provenait de deux sources. L’une, sans doute, était bien égyptienne et se montra toujours la plus abondante et la plus féconde ; mais l’autre exerçait une action qui vaut aussi la peine d’être signalée. Elle était due à une émigration très antique des Chamites noirs d’abord, les Arabes Cuschites, puis de Sémites, les Arabes Himyarites, qui passèrent, les uns et les autres, le détroit de Bab-el-Mandeb et allèrent porter aux populations d’Afrique une part de ce qu’elles possédaient elles-mêmes de culture assyrienne. À en juger d’après la situation qu’occupaient sur la côte sud de l’Arabie ces nations, et le commerce étendu auquel elles prenaient part avec l’Inde, commerce qui paraît avoir déterminé sur leur côte la fondation d’une ville sanscrite \footnote{Cette ville s’appelait Nagara. (Lassen, \emph{Indisch Alterth.}, t. I, p. 748.)}, il est assez probable que leurs propres idées devaient avoir reçu une certaine teinte ariane, proportionnée au mélange ethnique qui avait pu se faire de la part de ces marchands avec la famille hindoue. Quoi qu’il en soit, et en étendant autant que possible la somme de leurs richesses civilisatrices, nous avons, dans l’exemple des Phéniciens, la mesure du degré de développement auquel atteignaient ces populations annexes de la race d’Assyrie, mesure qui ne dépassait pas de beaucoup l’aptitude à comprendre et à accepter ce que les rameaux plus blancs, c’est-à-dire les nations de la Mésopotamie, avaient la puissance exclusive de créer et de développer. Les Phéniciens, tout habiles qu’ils fussent, ne s’élevaient pas au-dessus de cet humble rang, et quand on considère pourtant que leur sang fut sans cesse renouvelé et amélioré par des émigrations au moins à demi blanches, qui, bien certainement, faisaient défaut aux Himyarites, en tant que le mélange de ceux-ci avec les Hindous ne pût être ni bien intime ni bien fécond, on est amené à conclure que la civilisation des Arabes extrêmes, bien qu’assyrienne, n’était pas comparable en mérite et en éclat au reflet dont jouissaient les cités chananéennes \footnote{Ce sera peut-être un jour la gloire la plus solide et la plus réelle de notre époque que ces admirables découvertes qui viennent aujourd’hui transformer et enrichir, de toutes parts, le domaine autrefois si sec et si restreint de l’histoire primordiale. Des ruines considérables et des inscriptions sans nombre ont été découvertes dans l’Arabie méridionale. Les annales himyarites sortent du néant où elles étaient presque entièrement ensevelies, et, avant peu, ce qu’on saura de cette antiquité, non seulement lointaine, mais plus étrangère pour nous que celle de Ninive et même de Thèbes, parce qu’elle fut plus absolument locale et tournée vers l’Inde dans ce qu’elle eut d’expansion au dehors, n’aura pas moins d’intérêt dans l’ensemble des chroniques humaines que toutes les conquêtes du même genre dont la science s’enrichit par ailleurs.}.\par
Suivant cette proportion décroissante, les émigrants qui passèrent le détroit de Bab-el-Mandeb et vinrent s’établir en Éthiopie, n’y apportèrent qu’une civilisation fragmentaire, et les races noires de Nubie et d’Abyssinie n’auraient pu être bien sérieusement ni bien longtemps affectées, soit dans leur type physique, soit dans leur valeur morale, si le voisinage de l’Égypte n’avait pas suppléé un jour, plus largement que de coutume, à la pauvreté des dons ordinaires provenant des civilisations de Misr et d’Arabie.\par
Je ne veux pas dire ici que l’Abyssinie et les contrées environnantes soient devenues le théâtre d’une société très avancée. Non seulement la culture de ce pays ne fut jamais originale, non seulement elle se borna toujours à la simple et lointaine imitation de ce qui se faisait, soit dans les villes arabes de la côte, soit dans l’Inde ariane et dans les capitales égyptiennes, Thèbes, Memphis, et plus tard Alexandrie, mais encore l’imitation ne se montra ni complète ni étendue.\par
Je sais que je prononce là des paroles très irrévérencieuses et qui ne peuvent manquer d’indigner les panégyristes de l’espèce nègre, car on n’ignore pas que, l’esprit de parti s’en mêlant, les flatteurs de cette fraction de l’humanité se sont mis en humeur de lui conquérir des titres de gloire, et n’ont pas hésité à présenter la civilisation abyssine comme typique, sortie uniquement de l’intellect de leurs favoris et antérieure à toute autre culture. De là, pris d’un noble élan que rien n’arrête, ils ont fait ruisseler cette prétendue civilisation noire sur toute l’Égypte, et l’ont encore tirée vers l’Asie. À la vérité, la physiologie, la linguistique, l’histoire, les monuments, le sens commun, réclament unanimement contre cette façon de représenter le passé. Mais les inventeurs de ce beau système ne se laissent pas aisément étonner. Embarrassés de peu de science, armés de beaucoup d’audace, il est vraisemblable qu’ils continueront leur route et ne cesseront pas de proposer Axoum pour la capitale du monde. Ce sont là des excentricités dont je ne fais mention que pour établir qu’elles ne valent pas la peine d’être discutées \footnote{Wilkinson, t. I, p. 4. – Ce savant se prononce sans hésitation contre le système chéri des négrophiles. M. Lepsius n’est pas moins péremptoire. En parlant de la pyramide d’Assur, il prononce l’arrêt suivant : « Le plus important résultat de notre examen, exécuté moitié à la « clarté de la lune, moitié à celle des torches, ne fut pas précisément de la nature la plus « réjouissante. J’acquis la conviction irréfragable (\emph{unabweissliche}) que, dans ce monument, « le plus célèbre de tous ceux de l’ancienne Éthiopie, je n’avais sous les yeux que des débris « d’un art relativement très moderne. » (\emph{Briefe aus Ægypten}, etc., p. 147.) Et quelques lignes plus bas : « Ce serait vainement, « désormais, que l’on prétendrait appuyer sur le « témoignage d’anciens monuments les hypothèses concernant une Méroé glorieuse et « antique, dont les habitants auraient été les prédécesseurs et les maîtres des Égyptiens « dans la civilisation. » (Ouvr. cité, p. 184.) M. Lepsius ne pense pas que les constructions éthiopiennes les plus anciennes dépassent le règne de Tirhakah, prince qui avait fait son éducation royale en Égypte et qui florissait au VII\textsuperscript{e} siècle avant J.-C. seulement.}.\par
La réalité scientifique, pour qui ne veut pas rire, est que la civilisation abyssine procède des deux sources que je viens d’indiquer, égyptienne et arabe, et que la première surtout domina de beaucoup sur la seconde dans l’âge antique. Il sera toujours difficile d’établir à quelle époque eurent lieu les premières émigrations des Cuschites d’Asie et des Himyarites. Une opinion qui date de notre XVII\textsuperscript{e} siècle, et dont Scaliger fut l’auteur, ne faisait remonter qu’à l’époque de Justinien l’invasion des Joktanides dans ce pays d’Afrique. Job Ludolf la réfute très bien et lui préfère avec raison le sentiment de Conringius. Sans citer tous ses motifs, je lui ferai deux emprunts : l’un, d’un argument qui fixe du moins l’esprit sur la très haute antiquité de l’émigration himyarite \footnote{J. Ludolf, \emph{Comm. ad. Histor. Æthiopic.}, p. 61.}, et l’autre, d’une phrase dans laquelle il caractérise l’ancienne langue éthio­pienne, et sur laquelle il est bon de ne pas laisser régner une obscurité qui pourrait faire supposer une apparente contradiction avec ce que j’ai avancé de la prédominance de l’élément égyptien dans la civilisation abyssine.\par
D’abord, le premier point : Ludolf retourne très adroitement les raisonnements de Scaliger au sujet du silence des historiens grecs sur l’émigration himyarite en Abyssinie. Il prouve que ce silence n’a pas eu d’autre cause que l’oubli accumulé par une longue suite de siècles sur un fait trop fréquent dans l’histoire des âges reculés pour que les observateurs d’alors aient songé à lui reconnaître de l’importance. Au temps où les Grecs ont commencé à s’occuper de l’ethnologie des nations qui, pour eux, avoisinaient le bout du monde, ces événements étaient déjà trop loin pour que leurs renseignements, toujours assez incomplets sur les annales étrangères, pussent percer jusque-là. Le silence des voyageurs hellènes ne signifie absolument rien, et n’infirme pas les raisons tirées de l’antique communauté de culte, de la ressemblance physique, et enfin de l’affinité des langues, tous arguments que Ludolf fait très bien valoir. C’est de ce point qu’il faut surtout parler, et il constitue mon second emprunt.\par
Cette affinité entre l’arabe et l’ancienne langue éthiopienne, ou le gheez, ne crée pas un rapport de descendance ; c’est simplement une conséquence de la nature des deux idiomes qui les classe l’un et l’autre dans un même groupe \footnote{Prichard, \emph{Histoire naturelle de l’homme} (traduction allemande de Wagner, avec annotations), t. I, p. 324.}. Si le gheez se range dans la famille sémitique, ce n’est pas qu’il ait emprunté ce caractère à l’arabe. La population indigène purement noire du pays lui fournissait la base la plus large, l’étoffe la plus riche de ce système. Elle en possédait les éléments, les principes, les causes déter­minantes bien plus parfaitement encore que les Himyarites, puisque ceux-ci avaient laissé altérer la pureté de l’idiome noir par les souvenirs arians restés avec la partie blanche de leur origine ; et pour jeter dans la langue de l’Éthiopie civilisée ces traces de l’action étrangère, il n’était même pas rigoureusement nécessaire que l’intervention des Sémites fût mise en jeu. On se souvient que ces mêmes éléments sémitiques se trouvent aussi dans l’ancien égyptien \footnote{M. T. Benfey a réuni un grand nombre d’arguments et de faits tant lexicologiques que gramma­ticaux, pour mettre cette dernière vérité en lumière. Voir son livre intitulé : \emph{Ueber das Verhæltniss der ægyptischen Sprache zum semitischen Sprachstamme}, in-8°; Leipzig, 184.}. Ainsi, sans nier que les Himyarites aient apporté à la langue de l’Éthiopie des marques de leur origine blanche, on doit pourtant remarquer que de tels restes ont pu également provenir de l’importation égyptienne et, en tout cas, en ont profité pour augmenter de force. De plus, certains éléments, non seulement arians, mais plus particulièrement sanscrits, déposés dans l’ancien égyptien, ayant passé de là dans le gheez, donnent à cette langue cette triplicité de source existant dans l’idiome des civilisateurs. Ainsi, la langue nationale représente très bien les origines ethniques : beaucoup plus chargée d’éléments sémitiques, c’est-à-dire noirs, que l’arabe et l’égyptien surtout, elle eut aussi moins de traces sanscrites que ce dernier.\par
Sous les 18\textsuperscript{e} et 19\textsuperscript{e} dynasties (de 1575 à 1180 avant J.-C.), les Abyssins étaient soumis aux Pharaons et payaient tribut \footnote{Wilkinson, t. I, p. 387 et passim.}. Les monuments nous les montrent apportant aux intendants royaux les richesses et les curiosités de leur pays. Ces hommes fortement marqués de l’empreinte nègre sont couverts de tuniques de mousseline transparente fournies par les manufactures de l’Inde ou des villes d’Arabie et d’Égypte. Ce vêtement court et n’allant qu’aux genoux est retenu par une ceinture de cuir ouvré, richement dorée et peinte \footnote{Id., \emph{ibid}.}. Une peau de léopard attachée aux épaules fait manteau ; des colliers tombent sur la poitrine, des bracelets serrent les poignets, de grandes boucles de métal se balancent aux oreilles, et la tête est chargée de plumes d’autruche. Bien que cette magnificence barbare ne fût pas conforme au goût égyptien, elle en tenait, et l’imitation se fait sentir dans toutes les parties importantes du costume, telles que la tunique et la ceinture. La peau de léopard était empruntée d’ailleurs aux nègres par plusieurs hiérophantes.\par
La nature du tribut n’indique pas un peuple avancé. Ce sont des produits bruts, pour la plupart, des animaux rares, du bétail, et surtout des esclaves. Les troupes four­nies aussi comme auxiliaires n’avaient pas l’organisation savante des corps égyptiens ou sémites, et combattaient irrégulièrement. Rien donc, à ce moment, n’indiquait un grand développement, même dans la simple imitation de ce que les vainqueurs, les maîtres, pratiquaient le plus communément.\par
Il faut descendre jusqu’à une époque plus basse pour trouver, avec plus de raffi­nement, la cause ethnique des innovations à laquelle j’ai déjà fait allusion.\par
Au temps de Psammatik (664 avant J.-C.), ce prince, le premier d’une dynastie saïte, la 26\textsuperscript{e} de Manéthon, ayant mécontenté l’armée nationale par son goût pour les mercenaires ioniens-grecs et cariens-sémites, une grande émigration militaire eut lieu vers l’Abyssinie, et 240.000 soldats, abandonnant femmes et enfants, s’enfoncèrent dans le sud pour ne plus en revenir \footnote{Hérodote, II, 30}. C’est de là que date l’ère brillante de l’Abyssinie et nous pouvons maintenant parler de monuments dans cette région, où l’on en chercherait vainement d’antérieurs qui aient été vraiment nationaux \footnote{Suivant M. Lepsius, les dynasties chassées par les Hyksos se réfugièrent sur la limite de l’Éthiopie et y ont laissé quelques monuments. (\emph{Briefe aus Ægypten}, etc., p. 267.)}.\par
Deux cent quarante mille chefs de famille égyptiens, appartenant à la caste militaire, fort mélangés, sans doute, de sang noir, et, probablement, ayant reçu un certain apport de race blanche par les intermédiaires chamites et sémites, un tel groupe venant s’ajouter à ce que l’Abyssinie possédait déjà de facultés de la race supérieure, pouvait déterminer dans l’ensemble du mouvement national une activité propre à la séparer davantage de la stagnation de la race noire \footnote{À Abou-Simbel, sur la jambe gauche d’un des quatre colosses de Rhamsès, le second en allant vers le sud, on trouve une inscription grecque et plusieurs inscriptions chananéennes commémoratives de la poursuite faite des guerriers fugitifs par les soldats grecs et cariens à la solde de Psammatik. – Lepsius, \emph{Briefe aus Ægypten}, p. 261.}. Mais il eût été bien surprenant et tout à fait inexplicable qu’une civilisation originale, ou seulement une copie faite de main de maître, sortît de ce mélange où, en définitive, le noir continuait à dominer. Les monu­ments ne présentèrent que des imitations médiocres de ce qui se voyait à Thèbes, à Memphis et ailleurs. Rien, pas un indice, pas une trace, ne montre une création personnelle des Abyssins, et leur plus grande gloire, ce qui a rendu leur nom illustre, c’est, il faut bien l’avouer, le mérite, en lui-même assez pâle, d’avoir été le dernier des peuples situés en Afrique chez lequel les recherches les plus minutieuses aient pu faire découvrir les vestiges d’une véritable culture politique et intellectuelle.\par
Dans les temps de l’empire romain, le commerce du monde s’étant beaucoup étendu, les Abyssins y jouèrent un rôle derrière les Himyarites. Le génie de l’Égypte ancienne était alors tout à fait éteint. Des colons hellénisés pénétrèrent jusque dans la Nubie, et l’élément sémite, apporté par eux, commença à l’emporter sur le souvenir des Pharaons. Le gheez eut une écriture empruntée à l’Arabie. Cependant, malgré tout, les naturels du pays donnèrent un si petit éclat à leur action, on les connaissait si mal et si peu, leur influence était si lointaine, si effacée, qu’ils restèrent constamment, même pour les géographes les plus savants et les plus perspicaces, à l’état de demi-énigmes.\par
L’avènement du christianisme ne haussa pas le degré de leur culture. À la vérité, persistant encore quelque temps dans leurs habitudes de tout recevoir de l’Égypte, et touchés par le zèle apostolique des premiers missionnaires, ils embrassèrent assez généralement la foi. Ils avaient déjà dû au voisinage des tribus arabes avec lesquelles quelques invasions, exécutées sous l’empereur Justin \footnote{Ludolf, \emph{Comm. ad Hist.Æthiop.}, p. 61.– C. T. Johannsen, \emph{Historia Jemanæ}, Bonn, 1828, p. 80 : « Ait deinde Hamza, Maaditis eum sororis filium Alharithsum b. Amru præfecisse, « Meccam et Medinam expugnasse, tum ad Jemanam reversum judaismum cum populo suo « amplexum, Judæos in jemanam vocasse, atque jemanenses et Rebiitas fœdere « conjunxisse. »}, avaient resserré leurs liens antiques, l’adoption de certaines idées juives fort remarquées, plus tard, et qui s’accordaient assez naturellement avec la portion sémitique de leur sang \footnote{Prichard, \emph{Naturgeschichte d. M.} G., t. I, p. 324.}.\par
Le christianisme apporté par les Pères du désert, ces terribles anachorètes rompus aux plus rudes austérités, aux macérations les plus effrayantes, voire enclins aux mutilations les plus énergiques, était de nature à frapper les imaginations de ces peuples. Ils auraient été très probablement insensibles aux douces et sublimes vertus d’un saint Hilaire de Poitiers. Les pénitences d’un saint Antoine ou d’une sainte Marie Égyptienne exerçaient sur eux une autorité illimitée, et c’est ainsi que le catholicisme, si admirable dans sa diversité, si universel dans ses pouvoirs, si complet dans ses déductions, n’était pas moins armé pour ouvrir les cœurs de ces compagnons de la gazelle, de l’hippopotame et du tigre, qu’il ne le fut plus tard pour aller, avec Adam de Brême, parler raison aux Scandinaves et les convaincre. Les Abyssins, déjà plus qu’à demi déserteurs de la civilisation égyptienne depuis l’affaiblissement des provinces hautes de l’ancien empire des Pharaons, et plus tournés du côté de l’Yémen, restèrent pendant des siècles dans une sorte de situation intermédiaire entre la barbarie complète et un état social un peu meilleur ; et, pour continuer la transformation dont ils étaient devenus susceptibles, il fallut un nouvel apport de sang sémitique. L’irruption qui le fournit eut lieu 600 ans après J.-C. : ce fut celle des Arabes musulmans.\par
J’insiste peu sur les quelques conquêtes opérées à différentes reprises par les Abyssins dans la péninsule arabique. Il n’y a rien d’extraordinaire à ce que, de deux populations vivant en face l’une de l’autre, la moins noble ait quelquefois des succès passagers. L’Abyssinie ne tira jamais assez d’avantages de ses victoires dans l’Yémen pour y former un établissement durable. Seulement, le supplément de sang noir qu’elle y apporta ne contribua pas peu à hâter la submersion du mérite des Himyarites \footnote{Johannsen, \emph{Historia Jemanæ}, p. 89 et passim. – La domination des Abyssins dans l’Yémen fut d’une très courte durée, elle commença en 529 de notre ère et finit en 589. (\emph{Ibid.}, p. 100.)}.\par
Les rapports des populations arabes avec l’Éthiopie, au temps de l’islamisme, eurent un sens ethnique tout contraire. Dirigés, et en grande partie exécutés par des Ismaélites, au lieu d’abâtardir l’espèce dans la péninsule, ils la renouvelèrent chez les hommes d’Afrique. Ni la Grèce ni Rome, malgré la gloire de leur nom et la majesté de leurs exemples, n’avaient eu le pouvoir d’entraîner les Abyssins dans le sein de leurs civilisations. Les Sémites de Mahomet opérèrent cette conversion et obtinrent, non pas tant des apostasies religieuses, qui ne furent jamais très complètes, que de nombreuses désertions de l’ancienne forme sociale. Le sang des nouveaux venus et celui des anciens habitants se mêla abondamment. Sans peine les esprits se reconnurent et s’entendirent, ils eurent la même logique, ils comprirent les faits de la même façon. Le sang hindou s’était assez tari pour n’avoir plus rien à prétendre dans la domination. Le costume, les mœurs, les principes de gouvernement et le goût littéraire des Arabes envahirent sur les souvenirs du passé ; mais l’œuvre ne fut pas complète. La civilisation musulmane proprement dite ne pénétra jamais bien. Dans sa plus belle expression, elle avait pour raison d’être une combinaison ethnique trop différente de celle des populations abyssines. Ces dernières se bornèrent simplement à épeler la portion sémitique de la culture musulmane, et jusqu’à nos jours, chrétiennes ou mahométanes, elles n’ont pas eu autre chose, elles n’ont pas eu davantage et n’ont pas cessé d’être la fin, le terme extrême, l’application frontière de cette civilisation gréco-sémitique, comme dans l’antiquité la plus lointaine, où j’ai hâte de retourner, elles n’avaient été également que l’écho du perfectionnement égyptien, soutenu par un souvenir d’Assyrie transmis de main en main jusqu’à elle. Les splendeurs fantastiques de la cour du Prêtre-Jean, si l’on veut qu’il ait été le grand Négu, n’ont existé que dans l’imagination des voyageurs romanesques du temps passé.\par
Pour la première fois, nos recherches viennent de trouver dans l’Éthiopie un de ces pays annexes d’une grande civilisation étrangère, ne la possédant que d’une manière incomplète et absolument comme le disque lunaire fait pour la clarté du soleil. L’Abyssinie est à l’ancienne Égypte ce que l’empire d’Annam est à la Chine, et le Thibet à la Chine et à l’Inde \footnote{Et aussi Tombouctou au Maroc. (Voir \emph{Journal asiatique}, 1\textsuperscript{er} janvier 1853; \emph{Lettre à M. Defrémery, sur Ahmed Baba, le Tombouctien}, par M. A. Cherbonneau.)}. Ces sortes de sociétés imitatrices ou mixtes offrent les points où se rattache l’esprit de système pour remonter à l’encontre de tous les faits présentés par l’histoire. C’est là qu’on aime à défigurer les vestiges à peine apparents d’une importation certaine, et à leur prêter la valeur d’inspirations primordiales. C’est là surtout qu’on a trouvé des armes pour défendre cette théorie moderne qui veut que les peuples sauvages ne soient que des peuples dégénérés, doctrine parallèle à cette autre, que tous les hommes sont de grands génies désarmés par les circonstances.\par
Cette opinion, partout où on l’applique, chez les indigènes des deux Amériques, chez les Polynésiens comme chez les Abyssins, est un abus de langage ou une erreur profonde. Bien loin de pouvoir attribuer à la pression des faits extérieurs l’engour­dissement fatal qui a toujours pesé, avec plus ou moins de force, sur les nations cultivées de l’Afrique orientale, il faut se persuader que c’est là une infirmité étroitement inhérente à leur nature ; que jamais ces nations n’ont été civilisées parfai­tement, intimement ; que leurs éléments ethniques les plus nombreux ont toujours été radicalement inaptes à se perfectionner ; que les faibles effets de fertilité importés par des filons de sang meilleur étaient trop peu considérables pour pouvoir durer longtemps ; que leur groupe a rempli le simple rôle d’imitateurs inintelligents et temporaires des peuples formés d’éléments plus généreux. Cependant, même dans cette nation abyssine et surtout là, puisque c’est au point extrême, l’heureuse énergie du sang des blancs réclame encore l’admiration. Certes, ce qui, après tant de siècles, en reste aujourd’hui dans les veines de ces populations est subdivisé bien à l’infini. D’ailleurs, avant de leur parvenir, combien de souillures hétérogènes ne s’y étaient pas attachées chez les Himyarites, chez les Égyptiens, chez les Arabes musulmans ? Toutefois, là où le sang noir a pu contracter cette illustre alliance, il en conserve les précieux effets pendant des temps incalculables. Si l’Abyssin se classe tout au dernier degré des hommes riverains de la civilisation, il marche, en même temps, le premier des peuples noirs. Il a secoué ce que l’espèce mélanienne a de plus abaissé. Les traits de son visage se sont anoblis, sa taille s’est développée ; il échappe à cette loi des races simples de ne présenter que des déviations légères d’un type national immobile, et dans la variété des physionomies nubiennes on retrouve même, d’une manière surprenante, les traces, honorables en ce cas, de l’origine métisse. Pour la valeur intellectuelle, bien que médiocre et désormais inféconde, elle présente du moins une réelle supériorité sur celle de plusieurs tribus de Gallas, oppresseurs du pays, plus véritables noirs et plus véritables barbares dans toute la portée de l’expression.
\section[{II.6. Les Égyptiens n’ont pas été conquérants ; pourquoi leur civilisation resta stationnaire.}]{II.6. \\
Les Égyptiens n’ont pas été conquérants ; pourquoi leur civilisation resta stationnaire.}
\noindent Il n’y a pas à s’occuper des oasis de l’ouest, et en particulier de l’oasis d’Ammon. La culture égyptienne y régna seule, et probablement même ne fut-elle jamais possédée que par les familles sacerdotales groupées autour des sanctuaires. Le reste de la population ne pratiqua guère que l’obéissance. Ne nous occupons donc plus que de l’Égypte proprement dite, où cette question, la seule importante, reste à résoudre presque en entier : la grandeur de la civilisation égyptienne a-t-elle correspondu exacte­ment à la plus ou moins grande concentration du sang de la race blanche dans les groupes habitants du pays ? En d’autres termes, cette civilisation, sortie d’une migration hindoue et modifiée par des mélanges chamites et sémites, alla-t-elle toujours en décroissant à mesure que le fond noir, existant sous les trois éléments vitaux, prit graduellement le dessus ?\par
Avant Ménès, premier roi de la première dynastie humaine, l’Égypte était déjà civilisée et possédait au moins deux villes considérables, Thèbes et This. Le nouveau monarque réunit sous sa domination plusieurs petits États jusque-là séparés. La langue avait déjà revêtu son caractère propre. Ainsi l’invasion hindoue et son alliance avec des Chamites remontent au delà de cette très antique période, qui en fut le couronnement. jusque-là point d’histoire. Les souffrances, les dangers et les fatigues du premier établissement forment, comme chez les Assyriens, l’âge des dieux, l’époque héroïque.\par
Cette situation n’est pas particulière à l’Égypte : dans tous les États qui commencent on la retrouve.\par
Tant que durent les difficiles travaux de l’arrivée, tant que la colonisation demeure incertaine, que le climat n’est pas encore assaini, ni la nourriture assurée, ni l’aborigène dompté, que les vainqueurs eux-mêmes, dispersés dans les marais fangeux, sont trop absorbés par les assauts auxquels chaque individualité doit faire tête, les faits arrivent sans qu’on les recueille ; on n’a d’autre souci que la préservation, si ce n’est la conquête.\par
Cette période a une fin. Aussitôt que le labeur porte réellement ses premiers fruits, que l’homme commence à jouir de cette sécurité relative vers laquelle le portent tous ses instincts, et qu’un gouvernement régulier, organe du sentiment général, est enfin assis ; à ce moment, l’histoire commence, et la nation se connaît véritablement elle-même. C’est ce qui s’est passé, sous nos yeux, à plusieurs reprises, dans les deux Amériques, depuis la découverte du XV\textsuperscript{e} siècle.\par
La conséquence de cette observation est que les temps véritablement antéhis­toriques ont peu de valeur, soit parce qu’ils appartiennent aux races incivilisables, soit parce qu’ils constituent, pour les sociétés blanches, des époques de gestation où rien n’est complet ni coordonné, et ne peut confier un ensemble de faits logiques à la mémoire des siècles.\par
Dès les premières dynasties égyptiennes, la civilisation marcha si rapidement que l’écriture hiéroglyphique fut trouvée ; elle ne fut pas perfectionnée du même coup. Rien n’autorise à supposer que le caractère figuratif ait été immédiatement transformé, de manière à se simplifier, et, en même temps, à s’idéaliser sous une forme purement graphique \footnote{Brugsch, \emph{Zeitschrift d. deutsch Morgenl. Geselisch.}, t. III, p. 266 et passim.}.\par
La bonne critique attache de nos jours, et très justement, une haute idée de supériorité civilisatrice à la possession d’un moyen de fixer la pensée, et le mérite est d’autant plus grand que le moyen est moins compliqué. Rien ne dénote chez un peuple plus de profondeur de réflexion, plus de justesse de déduction, plus de puissance d’application aux nécessités de la vie, qu’un alphabet réduit à des éléments aussi simples que possible. À ce titre, les Égyptiens sont loin de pouvoir se réclamer de leur invention pour occuper une des places d’honneur. Leur découverte, toujours ténébreuse, toujours laborieuse à mettre en œuvre, les rejette sur les bas degrés de l’échelle des nations cultivées. Derrière eux, il n’est que les Péruviens nouant leurs cordelettes teintes, leurs quipos, et les Mexicains peignant leurs dessins énigmatiques. Au-dessus d’eux se placent les Chinois eux-mêmes ; car, du moins, ces derniers ont franchement passé du système figuratif à une expression conventionnelle des sons, opération, sans doute, imparfaite encore, mais qui, pourtant, a permis, à ceux qui s’en sont contentés, de rallier les éléments de l’écriture sous un nombre de clefs assez restreint. Du reste, combien cet effort, plus habile que celui des hommes de Thèbes, est-il encore inférieur aux intelligentes combinaisons des alphabets sémitiques, et même aux écritures cunéiformes, moins parfaites, sans doute, que celles-ci qui, à leur tour, doivent céder la palme à la belle réforme de l’alphabet grec, dernier terme du bien en ce genre, et que le système sanscrit, si beau cependant, n’égale pas ! Et pourquoi ne l’égale-t-il pas ? C’est uniquement parce que nulle race, autant que les familles occidentales, n’a été douée, tout à la fois, de cette puissance d’abstraction qui, unie au vif sentiment de l’utile, est la vraie source de l’alphabet.\par
Ainsi donc, tout en considérant l’écriture hiéroglyphique comme un titre solide de la nation égyptienne à prendre place parmi les peuples civilisés, on ne peut méconnaître que la nature de cette conception, parvenue même à ses perfectionnements derniers, ne classe ses inventeurs au-dessous des peuples assyriens. Ce n’est pas tout : dans le fait de cette idée stérilisée, il y a encore quelque chose à remarquer. Si les peuples noirs de l’Égypte n’avaient été gouvernés, dès avant le temps de Ménès, par des initiateurs blancs, ce premier pas de la découverte de l’écriture hiéroglyphique n’aurait certaine­ment pas été fait. Mais, d’autre part, si l’inaptitude de l’espèce noire n’avait pas, à son tour, dominé la tendance naturelle des Arians à tout perfectionner, l’écriture hiéroglyphique et, après elle, les arts de l’Égypte n’auraient pas été frappés de cette immobilité, qui n’est pas un des caractères les moins spéciaux de la civilisation du Nil.\par
Tant que le pays ne fut soumis qu’à des dynasties nationales, tant qu’il fut dirigé, éclairé par des idées nées sur son sol et issues de sa race, ses arts purent se modifier dans les parties ; ils ne changèrent jamais dans l’ensemble. Aucune innovation puissante ne les bouleversa. Plus rudes peut-être sous la 2\textsuperscript{e} et la 3\textsuperscript{e} dynastie, ils n’obtinrent, sous les 18\textsuperscript{e} et 19\textsuperscript{e}, que l’adoucissement de cette rudesse, et sous la 29\textsuperscript{e}, qui précéda Cambyse, la décadence ne s’exprime que par la perversion des formes, et non par l’introduction de principes jusque-là inconnus. Le génie local vieillit et ne changea pas. Élevé, porté au sublime tant que l’élément blanc exerça la prépondérance, stationnaire aussi longtemps que cet élément illustre put se maintenir sur le terrain civilisateur, décroissant toutes les fois que le génie noir prit accidentellement le dessus, il ne se releva jamais. Les victoires de l’influence néfaste étaient trop constamment soutenues par le fond mélanien sur lequel reposait l’édifice \footnote{Wilkinson, t. I, p. 85 et passim, p. 206; Lepsius, 276.}. On a de tous temps été frappé de cette mystérieuse somnolence. Les Grecs et les Romains s’en étonnèrent comme nous, et puisqu’il n’est rien qui demeure sans une explication, telle quelle, on crut bien dire en accusant les prêtres d’avoir produit le mal.\par
Le sacerdoce égyptien fut dominateur, sans nul doute, ami du repos, ennemi des innovations comme toutes les aristocraties. Mais quoi ! les sociétés chamites, sémites, hindoues eurent aussi des pontificats vigoureusement organisés et jouissant d’une vaste influence. D’où vient que, dans ces contrées, la civilisation ait remué, marché, traversé des phases multiples ; que les arts aient progressé, que l’écriture ait changé de formes et soit arrivée à sa perfection ? C’est que, simplement, dans ces différents lieux, la puissance des pontificats, tout immense qu’elle pût être, ne fut rien devant l’action exercée par les couches successives du sang des blancs, source intarissable de vie et de puissance. Les hommes des sanctuaires eux-mêmes, pénétrés du besoin d’expansion qui échauffait leur poitrine, n’étaient pas les derniers à trouver et à créer. C’est rabaisser la valeur et la force des éternels principes de l’existence sociale que d’y supposer des obstacles infranchissables dans le fait essentiellement mobile et transitoire des institutions.\par
Quand, par ces inventions de la convenance humaine, la civilisation se trouve gênée dans sa marche, elle, qui les a créées uniquement pour en tirer profit, est parfaitement armée pour les défaire, et l’on peut hardiment décider que, lorsqu’un régime dure, c’est qu’il convient à ceux qui le supportent et ne le changent pas. La société égyptienne, n’ayant reçu dans son sein que bien peu de nouveaux affluents blancs, n’eut pas lieu de renoncer à ce que, primitivement, elle avait trouvé bon et complet, et qui continua à lui paraître tel. Les Éthiopiens, les nègres, auteurs des plus anciennes et plus nombreuses invasions, n’étaient pas gens à transformer l’ordre de l’empire. Après l’avoir pillé, ils n’avaient que deux alternatives : ou se retirer, ou obéir aux règles établies avant leur venue. Les rapports mutuels des éléments ethniques de l’Égypte n’ayant été modifiés, jusqu’à la conquête de Cambyse, que par l’inondation croissante de la race noire, il n’y a rien d’étonnant à ce que tout mouvement ait commencé par se ralentir, puis se soit arrêté, et que les arts, l’écriture, l’ensemble entier de la civilisation, se soient, jusqu’au septième siècle avant J.-C., développés dans un sens unique, sans abandonner aucune des conventions qui avaient d’abord servi d’étais, et qui finirent, suivant la règle, par constituer la partie la plus saillante de l’originalité nationale.\par
On a la preuve que, dès la seconde dynastie, l’influence des vaincus de race noire se faisait déjà sentir dans les institutions, et, si l’on se représente l’oppression résolue des maîtres et leur mépris systématique des populations, on ne doutera pas que, pour obtenir ainsi créance, il fallait que les idées des sujets s’exprimassent par la bouche de puissants intéressés, d’hommes placés de manière à exercer les prérogatives domina­trices de la race blanche, tout en partageant jusqu’à un certain point les sentiments de la noire. Ces hommes ne pouvaient être autres que des mulâtres. Le fait dont il s’agit ici est celui que Jules Africain rapporte dans les termes qui suivent, au règne de Kaïechos, second roi de la dynastie thinite : « Depuis ce monarque, dit l’abréviateur, on établit en loi que les bœufs Apis à Memphis, et Mnévis à Héliopolis, et le bouc Mendésien étaient des dieux. »\par
Je regrette de ne pas trouver, sous la plume savante de M. le chevalier Bunsen, la traduction suffisamment exacte de cette phrase plus pleine de sens qu’il ne lui en attribue \footnote{Voici le texte et la traduction de M. de Bunsen : (Phrase en langue étrangère) Kaiechos... Unter ihm wurde die gœttliche Verehrung der Stiere, des Apis in Memphis und des Mnævis in Heliopolis, so wie des mendesischen Bockes eingeführt. (Bunsen, II, p. 103.)}. Jules Africain ne dit pas, ainsi qu’on pourrait l’induire des expressions dont se sert le savant diplomate prussien, que le culte des animaux sacrés fut, pour \emph{la} \emph{première fois}, introduit, mais bien qu’il fût officiellement reconnu, étant déjà ancien. Quant à ce dernier point, je m’en rapporte aux nègres pour n’avoir pas manqué, dès l’origine de leur espèce, de calculer la religion sur le pied de l’animalité. Si donc cette adoration de tous les temps avait besoin d’être consacrée par un décret pour devenir légale, c’est que, jusque-là, elle n’avait pu rallier les sympathies de la partie dominante de la société, et comme cette partie dominante était d’origine blanche, il fallut, pour que se fît une révolution aussi grave contre toutes les notions arianes du vrai, du sage et du beau, que le sens moral et intellectuel de la nation eût déjà subi une dégradation fâcheuse. C’était la conséquence des innovations survenues dans la nature du sang. De blanche, la société active était devenue métisse et, s’abaissant de plus en plus dans le noir, s’était, chemin faisant, associée à l’idée qu’un bœuf et un bouc méritaient des autels.\par
On peut être tenté de reprocher à ceci une sorte de contradiction. Je semble donner toutes les raisons et rassembler toutes les causes d’une décadence sans miséricorde dans les mains même du premier roi Ménès et, pourtant, l’Égypte n’a fait que commencer sous lui de longs siècles d’illustration \footnote{Il ne saurait être inutile de rappeler ici quelle fut la prospérité à laquelle parvinrent les États de la vallée du Nil. On sait que, dans sa plus grande étendue, cette contrée n’a pas 50 milles allemands de largeur, et qu’en longueur, depuis la mer Méditerranée jusqu’à Syène, elle en comporte environ 120. Dans cet espace étroit, Hérodote place 20,000 villes et villages, à l’époque d’Amasis. Diodore en compte 18,000. La France actuelle, douze fois plus grande, n’en a que 39,000. La population de Thèbes, au temps d’Homère, peut se calculer à 2,800,000 habitants, et quand je songe à celle que, dans les époques postérieures, atteignit Syracuse, beaucoup moins riche et moins puissante, je ne partage nullement la surprise et l’incrédulité de M. de Bohlen. (\emph{Das alte Indien}, t. I, p. 32 et passim.)}. En y regardant de près, la difficulté apparente s’évanouit. On a vu déjà, dans les États assyriens, avec quelle lenteur s’opère la fusion ethnique étendue sur un grand ensemble. C’est un véritable combat entre ses éléments et, outre cette lutte générale dont l’issue est très facile à préciser, il y a sur mille points particuliers des luttes partielles où l’influence à laquelle est assurée, par la raison de quantité, la victoire définitive, n’en subit pas moins des défaites momen­tanées, d’autant plus multipliées que cette influence se trouve aux prises avec un compétiteur, en lui-même, bien autrement doué et puissant. De même que sa victoire sera la fin de tout, de même aussi, tant que la vie, importée par le principe étranger, se manifeste, la puissance dont l’inertie est le caractère reçoit échecs sur échecs. Tout ce qu’elle peut, c’est de tracer le cercle d’où son adversaire finit par ne pouvoir sortir, et qui, se rétrécissant de plus en plus, l’étouffera un jour. Ainsi en advint-il de l’élément blanc qui dirigeait les destinées de la nation égyptienne, au milieu et contrairement aux tendances d’une masse trop considérable de principes mélaniens. Aussitôt que ces principes commencèrent assez notablement à se trouver mêlés à lui, ils imposèrent à ses découvertes, à ses inventions, une limite qu’il ne put jamais leur faire franchir. Ils bridaient son génie et ne lui permirent que les œuvres de patience et d’application. Ils voulurent bien le laisser toujours édifier ces prodigieuses pyramides dont il avait apporté, du voisinage des monts Oural et Altaï, l’inspiration et le modèle. Ils voulurent bien encore que les principaux perfectionnements trouvés aux premiers temps de l’établissement (car, là, tout ce qui était vraiment de génie datait de la plus haute antiquité) continuassent à être appliqués ; mais, graduellement, le mérite de l’exécution grandissait aux dépens de la conception, et, au bout d’une période qu’en l’étendant autant que possible, on ne peut guère agrandir au delà de sept à huit siècles, la décadence commença. Après Rhamsès III, vers le milieu du treizième siècle avant J.-C. \footnote{D’après la chronologie de Wilkinson, qui reconnaît ce prince dans le Rhamsès Amoun-Maï des monuments, roi diospolite de la 19\textsuperscript{e} dynastie, et qui le fait régner en 1235 avant J.-C. (Wilkinson, t. I, p. 83.) – M. Lepsius reporte ce Rhamsès beaucoup plus haut et le place dans la 20\textsuperscript{e} dynastie, au 15, siècle avant notre ère. (\emph{Briefe aus Ægypten}, p. 274.)}, ce fut fini de toute la grandeur égyptienne. On ne vécut plus que sur les indications, chaque jour s’effaçant, des errements anciens \footnote{Sous Osirtasen I\textsuperscript{er} (1740 av. J.-C., suivant le calcul de Wilkinson), les monuments sont magnifiques. Les sculptures de Beni-Hassan appartiennent à cette époque, la plus brillante pour les arts. (Wilkinson, t. I, p. 22.) C’est le commencement du nouvel empire. Il ne s’agit déjà plus des constructions les plus colossales ; ainsi, bien que l’art soit dans tout son beau, il a déjà dépassé sa période de croissance. L’Osirtasen I\textsuperscript{er} de Wilkinson est le même que le Sesortesen de M. le chevalier Bunsen (t. II, p. 306.)}.\par
Il est impossible que les plus fervents admirateurs de l’ancienne Égypte n’aient pas été frappés d’une remarque qui forme un singulier contraste avec l’auréole dont l’imagination entoure ce pays. Cette remarque ne laisse pas que de jeter une ombre fâcheuse sur la place qu’il occupe parmi les splendeurs du monde : c’est l’isolement à peu près entier dans lequel il a vécu vis-à-vis des États civilisés de son temps. Je parle, bien entendu, de l’ancien empire, et surtout, comme pour les Assyriens, je ne fais pas descendre au-dessous du septième siècle avant J.-C. le texte de mes considérations actuelles \footnote{M. Lepsius remarque que, pendant toute la durée de l’ancien empire, la civilisation fut essentiellement pacifique ; il ajoute que les Grecs ne soupçonnèrent même jamais l’existence de cette période de gloire et de puissance antérieure à la domination des Hyksos. (Lepsius, \emph{Briefe aus Ægypten}, etc.) Le nouvel empire, dont l’établissement fut déterminé par l’expulsion des Hyksos, commença 1700 ans avant notre ère, et Amosis en fut le premier roi. (Lepsius, p. 272.)}.\par
À la vérité, le grand nom de Sésostris plane sur toute l’histoire de l’Égypte primitive, et notre esprit, s’étant accoutumé à enchaîner derrière le char de ce vainqueur des populations innombrables, se laisse aller aisément à promener avec lui les drapeaux égyptiens du fond de la Nubie aux colonnes d’Hercule, des colonnes d’Hercule à l’extrémité sud de l’Arabie, du détroit de Bab-el-Mandeb à la mer Caspienne, et à les faire rentrer à Memphis, entourés encore des Thraces et de ces fabuleux Pélasges dont le héros égyptien est censé avoir dompté les patries. C’est un spectacle grandiose, mais la réalité en soulève des objections.\par
Pour commencer, la personnalité du conquérant n’est pas elle-même bien claire. On ne s’est jamais accordé ni sur l’âge qui l’a vu fleurir, ni même sur son nom véritable. Il a vécu longtemps avant Minos, dit un auteur grec ; tandis qu’un autre le repousse impitoyablement jusque dans les nuages des époques mythologiques. Celui-ci l’appelle Sésostris ; celui-là Sesoosis ; un dernier veut le reconnaître dans un Rhamsès, mais dans lequel ? Les chronologistes modernes, héritiers embarrassés de toutes ces contradic­tions, se divisent, à leur tour, pour faire de ce personnage mystérieux un Osirtasen ou un Sésortesen, ou encore un Rhamsès II ou un Rhamsès III. Un des arguments les plus solides au moyen desquels on pensait pouvoir appuyer l’opinion favorite touchant l’étendue des conquêtes de ce mystérieux personnage, c’était l’existence de stèles victorieuses dressées par lui sur plusieurs points de ses marches. On en a, en effet, trouvé, qui doivent être attribuées à des souverains du Nil, et dans la Nubie près de Wadi Halfah, et dans la presqu’île du Sinaï\footnote{Bunsen, t. II p. 307; Lepsius, p. 336 et passim ; Movers, das Phœniz. Alterth., t. II, l\textsuperscript{re} partie, p. 301.} . Mais un autre monument, d’autant plus célèbre qu’Hérodote le mentionne, monument existant encore près de Beyrouth, a été positivement reconnu, de nos jours, pour le gage de victoire d’un triomphateur assyrien \footnote{Movers, t. II, 1\textsuperscript{re} partie, p. 281. Cet historien attribue la stèle en question à Memnon, et la fait contemporaine de la guerre de Troie.}. D’ailleurs, rien d’égyptien ne s’est jamais rencontré au-dessus de la Palestine.\par
Avec toute la réserve que je dois apporter à me présenter dans ce débat, j’avoue que des différentes façons dont on a voulu prouver les conquêtes des Pharaons en Asie, aucune ne m’a jamais semblé satisfaisante \footnote{M. de Bunsen porte un jugement bien vrai et bien concluant sur les prétendues expansions de la puissance égyptienne du côté de l’Asie. Voici en quels termes il s’exprime : « Il nous « paraît hasardé de déclarer asiatiques les noms des peuples indiqués sur ces monuments (le tombeau de Neropt à Beni-Hassan) comme septentrionaux, toutes les fois que des « contrées connues, telles que le Chanana et le Naharaïm (Chanaan et la Mésopotamie) ne « sont pas indiquées, et de prétendre chercher parmi ces noms de nouvelles listes de « nations, dans l’Iran et le Touran. Est-ce donc le sud que la Libye septentrionale, la « Cyrénaique, la Syrtique, la Numidie, la Gétulie, en un mot, toute la côte nord de « l’Afrique ? Est-ce même un pays de nègres (nahao) ? Ou bien les Égyptiens n’avaient-ils « à penser qu’aux pays septentrionaux de l’Asie, à la Palestine, à la Syrie, où ils ne « pouvaient exécuter que des courses ? En revanche, ils se seraient tenus isolés de tout « contact avec les pays du nord de l’Afrique ! » (\emph{Ægypten’s Stelle in der Welt-Geschichte}, t. II, p. 311.)}. Elles reposent sur des allégations trop vagues ; elles font courir trop loin les vainqueurs et leur livrent trop de terres pour ne pas éveiller la méfiance \footnote{Deux causes me paraissent surtout induire les égyptologues à céder à leur enthousiaste admiration pour le peuple illustre dont ils étudient l’histoire et dont un penchant bien naturel les porte à exagérer les mérites. L’une, c’est l’expression \emph{peuples septentrionaux}, inscrite dans les hiéroglyphes commémoratifs des expéditions guerrières et qui reporte aisément la pensée vers le nord-est ; l’autre, c’est la rencontre de certaines appellations ethniques ou géographiques que l’on trouve moyen de rapprocher des noms de plusieurs peuples asiatiques connus. Il est tout simple, sans doute, que lorsque les monuments parlent du \emph{Kanana}, du \emph{Lemanon} et d’\emph{Ascalon}, on reconnaisse des contrées du littoral de Syrie. (Wilkinson, t. I, p. 386.) Mais lorsque, dans les \emph{Kheta}, on veut reconnaître les Gètes, c’est absolument comme si dans les Gallas d’Abyssinie on prétendait retrouver des Gallas celtiques, et d’autant plus que les Gètes ou (en grec) des Grecs étaient des peuples barbares, tandis que les Kheta sont représentés, sur les monuments égyptiens, comme une nation très civilisée. Les peintures de Médinet-Abou nous les montrent vêtus de longues robes de couleurs brillantes tombant jusqu’à la cheville, avec la barbe épaisse et les yeux droits. Ce ne sont donc pas, dans tous les cas, des hommes de race jaune. Ils combattent en fort belle ordonnance, les soldats armés d’épées au premier rang, les piquiers au second. Le Memnonium de Thèbes représente aussi leurs forteresses entourées d’un double fossé. (Wilkinson, t. I, 384.) Aussi, bien que le nom de \emph{Kheta on Sheta} ait un certain rapport de son avec celui de \emph{Gêtes}, il n’y a pas là de quoi justifier une identification de nations qui certainement étaient fort dissemblables. Même chose des \emph{Tokhari.} Les peintures égyptiennes leur attribuent un profil régulier, un nez légèrement aquilin, une coiffure un peu semblable à la mitre persane. On les voit cheminer dans des espèces de charrettes avec leurs femmes et leurs enfants. C’en est assez pour que M. Wilkinson les confonde avec les \emph{Tokhari} connus des Grecs, les \emph{Tokkhara} du Mahabharata, habitants de la Sogdiane et de la Bactriane, sur le Iaxarte supérieur et le Zariaspe. M. Lassen partage cette opinion (\emph{Indisch. Alterth.}, t. I, p. 852). M. le lieutenant-colonel Rawlinson me paraît mieux inspiré lorsque, trouvant sur un cylindre assyrien la mention d’une expédition de Sennachérib contre les \emph{Tokhari} qui habitent la vallée de Salbura, il se refuse à conduire les troupes de son héros chaldéen jusque vers l’Oxus, et se borne à chercher ces fameux Tokhari dans le sud de l’Asie Mineure (\emph{Report of the R. A. S.}, p. XXXVIII). Je crois que la véritable histoire ne saurait que gagner à se tenir fort en garde contre des extensions indéfinies de prétendues conquêtes qui ne se justifient que d’après des preuves aussi fragiles que des ressemblances de noms et quelques vagues ressemblances physiologiques.}.\par
 Puis elles se heurtent contre une très grave difficulté : l’ignorance complète où l’on trouve les prétendus vaincus de leur malheur. Je ne vois, à l’exception de quelques petits États de Syrie, pas un moment dans l’histoire unie, suivie, compacte des nations assyriennes jusqu’au VII\textsuperscript{e} siècle, où l’on puisse introduire d’autres conquérants que les différentes couches de Sémites et quelques Arians, et quant à reporter bien haut la douteuse omnipotence d’un nébuleux Sésostris, la tâche n’en devient que plus scabreuse. À ces époques indéterminées, témoins, il est vrai, de la plus belle efflores­cence de Thèbes et de Memphis, les principaux efforts du pays se portaient vers le sud \footnote{Les premières conquêtes en Éthiopie remontent, suivant M. Lepsius, à l’ancien empire, et eurent pour auteur Sesortesen III, roi de la 12\textsuperscript{e} dynastie, qui fonda les remparts de Semleh et devint, plus tard, divinité topique. (\emph{Briefe aus Ægypten}, p. 259.) – M. Bunsen envoie Sesortesen II non seulement dans la presqu’île du Sinaï, mais sur toute la côte septentrionale de l’Afrique jusque vis-à-vis l’Espagne ; il le ramène ensuite en Asie et en Europe jusqu’à la Thrace. C’est beaucoup. (Bunsen, ouvrage cité, t. II, p. 306 et passim.)}, vers l’Afrique intérieure, un peu vers l’est, tandis que le Delta servait de passage à des peuples de races diverses longeant les plages de l’Afrique septentrionale.\par
Outre les expéditions dans la Nubie et les contrées sinaïtiques, il faut tenir compte également des immenses travaux de canalisation et de défrichement, tels que le dessèchement du Fayoum, la mise en rapport de ce bassin, et les vastes constructions dont les différents groupes de pyramides sont les dispendieux résultats. Toutes ces œuvres pacifiques des premières dynasties n’indiquent pas un peuple qui ait eu ni beaucoup de goût ni beaucoup de loisir pour des expéditions lointaines, que rien, pas même la raison de voisinage, ne rendait attrayantes, encore bien moins nécessaires \footnote{Bunsen, t. II, p, 214 et passim.}.\par
Cependant, faisons céder un moment toutes ces objections si fortes. Réduisons-les au silence, et adoptons Sésostris, et ses conquêtes pour ce qu’on nous les donne. Il restera incontesté que ces invasions ont été tout à fait temporaires, n’en déplaise à la fondation vaguement indiquée de cités soi-disant nombreuses, et tout à fait inconnues dans l’Asie Mineure, et à la colonisation de la Colchide, occupée par des peuples noirs, des Éthiopiens, disaient les Grecs, c’est-à-dire des hommes qui, de même que l’Éthiopien Memnon, peuvent fort bien n’avoir été que des Assyriens.\par
Tous les récits qui font des monarques de Memphis autant d’incarnations anté­rieures de Tamerlan, outre qu’ils sont contraires à l’humeur pacifique et à la molle langueur des adorateurs de Phtah, à leur goût pour les occupations rurales, à leur religiosité casanière, se montrent trop incohérents pour ne pas reposer sur des confusions infinies d’idées, de dates, de faits et de peuples \footnote{Movers, \emph{das Phœn. Alterth.}, t. II, 1\textsuperscript{re} partie, p. 298.}. Jusqu’au dix-septième siècle avant J.-C. l’influence égyptienne, et toujours l’Afrique exceptée, n’avait que très peu d’action ; elle exerçait un faible prestige, elle était à peine connue \footnote{La Phénicie en tenait seule quelque compte ; les petites nations hébraïques ou chananéennes montraient une prédilection presque absolue pour les idées assyriennes. Je l’ai expliqué plus haut du reste : ces petits États-frontières étaient soumis à beaucoup de ménagements, en même temps qu’à beaucoup de séductions, et il n’y a rien d’extraordinaire à ce que, dans le voisinage immédiat de l’Égypte, il se trouve quelques traces de l’influence de ce pays. En tout cas, on aurait tort de trop facilement en accepter l’idée. Plus d’une coutume supposée égyptienne est tout aussi facile à revendiquer pour d’autres origines. La forme des chars est identique à Memphis et à Khorsabad (Wilkinson, t. I, p. 346 ; Botta, \emph{Monuments de Ninive}) ; la construction des places de guerre se ressemblait extrêmement (\emph{loc. cit}.), etc., etc.}. Des travaux de défense du genre de ceux que les rois avaient fait construire sur les frontières orientales pour fermer le passage aux sables et surtout aux étrangers \footnote{Bunsen, t. II, p. 320.}, sont toujours l’œuvre d’un peuple qui, en se garantissant des invasions, limite lui-même son terrain. Les Égyptiens étaient donc volontairement séparés des nations orientales. Sans que tous rapports guerriers ou pacifiques fussent détruits, il n’en résultait pas un échange durable des idées, et par conséquent la civilisation resta confinée au sol qui l’avait vue naître, et ne porta point ses merveilles à l’est ni au nord, ni même dans l’ouest africain \footnote{Au VIII\textsuperscript{e} siècle avant J.-C., les Égyptiens n’avaient pas même de marine, bien qu’à cette époque ils eussent englobé le Delta dans leur empire. Les peuples chananéens, sémites ou grecs étaient les seuls navigateurs qui auraient pu animer le commerce de leur pays ; ils attachaient une importance si secondaire à cet avantage, que, pour se défendre des insultes des pirates, ils n’avaient pas hésité à fermer l’entrée du Nil par des barrages qui la rendaient impraticable à tous les navires. (Movers, das \emph{Phœnizich Alterth.}, t. II, 1\textsuperscript{re} partie, p. 370.) – En somme, les guerres des Égyptiens du côté de l’Asie ont toujours eu un caractère plutôt défensif qu’agressif, et l’influence même que les Pharaons s’efforçaient de gagner dans les cités phéniciennes avait plutôt pour but de neutraliser l’action des gouvernements assyriens que de poursuivre des résultats positifs. (Movers, \emph{ibid}, p. 298, 299, 415 et passim.)}.\par
Quelle différence avec la culture assyrienne ! Celle-ci embrassa dans son vol immense un si vaste tour de pays, qu’il dépasse l’essor où purent s’emporter, dans des temps postérieurs, la Grèce d’abord, Rome ensuite. Elle domina l’Asie moyenne, découvrit l’Afrique, découvrit l’Europe, sema profondément dans tous ces lieux ses mérites et ses vices, s’implanta partout, de la manière la plus durable, et, vis-à-vis d’elle, le perfectionnement égyptien, demeuré à peu près local, se trouva dans une situation semblable à ce que la Chine a été depuis pour le reste du monde.\par
Bien simple est la raison de ce phénomène, si on veut la chercher dans les causes ethniques. De la civilisation assyrienne, produit des Chamites blancs mêlés aux peuples noirs, puis de différentes branches des Sémites ajoutées au tout, il résulta la naissance de masses épaisses qui, se poussant et se pénétrant de mille manières, allèrent porter en cent endroits divers, entre le golfe Persique et le détroit de Gibraltar, les nations composites nées de leur fécondation incessante. Au contraire, la civilisation égyptienne ne put jamais se rajeunir dans son élément créateur qui fut toujours sur la défensive et toujours perdit du terrain. Issue d’un rameau d’Arians-Hindous mêlé à des races noires et à quelque peu de Chamites et de Sémites, elle revêtit un caractère particulier qui, dès ses premiers temps, était parfaitement fixé et se développa longtemps dans un sens propre avant d’être attaqué par des éléments étrangers. Elle était mûre déjà lorsque des invasions ou introductions de Sémites vinrent se super­ poser à elle \footnote{J’entends parler ici des Hyksos qui renversèrent l’ancien empire.}. Ces courants auraient pu la transformer, s’ils avaient été considérables. Ils restèrent faibles, et l’organisation des castes, tout imparfaite qu’elle était, suffit longtemps à les neutraliser.\par
Tandis qu’en Assyrie les émigrants du nord pénétraient et se montraient rois, prêtres, nobles, tout, ils rencontraient sur le sol de l’Égypte une législation jalouse qui commençait par leur fermer l’entrée du territoire à titre d’êtres impurs, et lorsque, malgré cette défense, maintenue jusqu’au temps de Psammatik (664 av. J.-C.), les intrus parvenaient à se glisser à côté des maîtres du pays, décastés et haïs, ce n’était que lentement qu’ils se fondaient dans cette société rébarbative. Ils y réussissaient cependant, je le crois ; mais pour quel résultat ? Pour imiter l’œuvre du sang hellénique en Phénicie. Comme lui, ils contribuaient, unis à l’action noire, à hâter la dissolution d’une race que, plus nombreux et arrivés plus tôt, ils auraient fait vivre et se régénérer. Si, dès les premières années où régna Ménès, au mélange arian, chamite et noir, une forte dose de sang sémitique avait pu s’ajouter, l’Égypte aurait été profondément révolutionnée et agitée. Elle ne serait pas restée isolée dans le monde, et elle se serait trouvée en communication directe et intime avec les États assyriens.\par
Pour en faire juger, il suffit de décomposer les deux groupes de nations :\par

\tableopen{}
\begin{tabularx}{\linewidth}
{|l|X|}
\hlineASSYRIENS \\
ÉLÉMENT NOIR FONDAMENTAL & ÉGYPTIENS \\
ÉLÉMENT NOIR FONDAMENTAL \\
\hline
 \noindent \emph{Chamites}, en quantité suffisamment grande pour être fécondante.\par
 \emph{Sémites}, de plusieurs couches, singulièrement fécondants.\par
 \emph{Noirs}, toujours dissolvants.\par
 \emph{Grecs}, en quantité dissolvante.
  &  \noindent \emph{Arians}, dominants sur l’élément chamite.\par
 \emph{Chamites}, en quantité fécondante.\par
 \emph{Noirs}, nombreux et dissolvants.\par
 \emph{Sémites}, en quantité dissolvante.
  \\
\hline
\end{tabularx}
\tableclose{}

\noindent On peut tirer encore une autre vérité de ce tableau : c’est que, le sang chamite tendant à s’épuiser chez les deux peuples, les ressemblances également tendaient à disparaître avec cet élément qui, seul, les avait fondées et aurait été en état de les maintenir, puisque l’action sémitique s’exerçait dans les deux sociétés en sens inverse. En Égypte, elle ne pénétrait qu’en quantité dissolvante ; en Assyrie, elle se répandait avec profusion, débordait de là sur l’Afrique, l’Europe, et devenait, entre mille nations, le lien d’une alliance dont la terre des Pharaons allait être exclue, réduite qu’elle se voyait à sa fusion noire et ariane ; les vertus s’en épuisaient chaque jour, sans que rien vînt les relever. L’Égypte ne fut admirable que dans la plus haute antiquité. Alors, c’est vraiment le sol des miracles. Mais quoi ! ses qualités et ses forces sont concentrées sur un point trop étroit. Les rangs de sa population initiatrice ne peuvent se recruter nulle part. La décadence commence de bonne heure, et rien ne l’arrête plus, tandis que la civilisation assyrienne vivra bien longtemps, subira bien des transformations, et, plus immorale, plus tourmentée que sa contemporaine, aura joué un bien plus important personnage.\par
C’est ce dont on sera convaincu lorsque, après avoir considéré la situation de l’Égypte au VII\textsuperscript{e} siècle, situation déjà bien humble et désespérée, on la verra réduite à un tel degré d’impuissance, que, sur son propre domaine, dans ses propres affaires, elle ne jouera plus de rôle, laissera le pouvoir et l’influence aux mains des conquérants et des colons étrangers, et en arrivera à ce point d’être si oubliée, que le nom d’Égyptien indiquera bien moins un des descendants de la race antique qu’un fils des nouveaux habitants sémites, grecs ou romains. Cette nouveauté le cédera encore en singularité à celle-ci : l’Égypte, ce ne sera plus, comme autrefois, la haute partie du pays, le voisinage des Pyramides, la terre classique, Memphis, Thèbes : ce sera plutôt Alexandrie, ce rivage abandonné, dans l’époque de gloire, au trajet des invasions sémitiques. Ainsi Ninive, victorieuse de sa rivale, aura à la fois dépouillé du nom national et les hommes et le sol. Malgré le mur d’Héliopolis, la terre de Misr sera devenue la proie inerte des sables et des Sémites, parce qu’aucun élément arian nouveau n’aura sauvé sa population du malheur de s’engloutir dans la prépondérance enfin décidée de ses principes mélaniens.
\section[{II.7. Rapport ethnique entre les nations assyriennes et l’Égypte. Les arts et la poésie lyrique sont produits par le mélange des blancs avec les peuples noirs.}]{II.7. \\
Rapport ethnique entre les nations assyriennes et l’Égypte. Les arts et la poésie lyrique sont produits par le mélange des blancs avec les peuples noirs.}
\noindent Toute la civilisation primordiale du monde se résume, pour les Occidentaux, dans ces deux noms illustres : Ninive et Memphis. Tyr et Carthage, Axoum et les cités des Himyarites ne sont que des colonies intellectuelles de ces deux points royaux. En essayant de caractériser les civilisations qu’ils représentent, j’ai touché quelques-uns de leurs points de contact. Mais j’ai réservé jusqu’ici l’étude des principaux rapports communs, et au moment où leur déclin va commencer, avec des fortunes diverses, où le rôle de l’un va cesser, le rôle de l’autre s’agrandir encore dans des mains étrangères, en changeant de nom, de forme et de portée ; en ce moment, où je vais me voir forcé, dans un sujet très grave, d’imiter la méthode des poètes chevaleresques, de passer des bords de l’Euphrate et du Nil aux montagnes de la Médie et de la Perse, et de m’enfoncer dans les steppes de la haute Asie, pour y quérir les nouveaux peuples qui vont transfigurer le monde politique et les civilisations, je ne puis tarder davantage à préciser et à définir les causes de la ressemblance générale de l’Égypte et de l’Assyrie,\par
Les groupes blancs qui avaient créé la civilisation dans l’une et dans l’autre n’appartenaient pas à une même variété de l’espèce, sans quoi il serait impossible d’expliquer leurs différences profondes. En dehors de l’esprit civilisateur qu’ils possé­daient également, des traits particuliers les marquaient, et imprimèrent comme un cachet de propriété sur leurs créations respectives. Les fonds, étant également noirs, ne pouvaient amener de dissemblances ; et quand bien même on voudrait trouver des diversités entre leurs populations mélaniennes, en ne découvrant que des noirs à cheveux plats dans les pays assyriens, des nègres à chevelure crépue en Égypte, outre que rien n’autorise cette supposition, rien n’a jamais indiqué non plus qu’entre les rameaux de la race noire les différences ethniques impliquent une plus ou moins grande dose d’aptitude civilisatrice. Loin de là partout où l’on étudie les effets des mélanges, on s’aperçoit qu’un fond noir, malgré les variétés qu’il peut présenter, crée les similitu­des entre les sociétés en ne leur fournissant que ces aptitudes négatives bien évidemment étrangères aux facultés de l’espèce blanche. Force est donc d’admettre, devant la nullité civilisatrice des noirs, que la source des différences réside dans la race blanche ; que, par conséquent, il y a entre les blancs des variétés ; et si nous en envisageons maintenant le premier exemple dans l’Assyrie et en Égypte, à voir l’esprit plus régularisateur, plus doux, plus pacifique, plus positif surtout, du faible rameau arian établi dans la vallée du Nil, nous sommes enclins à donner à l’ensemble de la famille une véritable supériorité sur les branches de Cham et de Sem. Plus l’histoire déroulera ses pages, plus nous serons confirmés dans cette première impression.\par
Revenant aux peuples noirs, je me demande quelles sont les marques de leur nature, les marques semblables qu’ils ont portées dans les deux civilisations d’Assyrie et d’Égypte. La réponse est évidente. Elle ressort de faits qui prennent la conviction par les yeux.\par
Nul doute que ce ne soit ce goût frappant des choses de l’imagination, cette passion véhémente de tout ce qui pouvait mettre en jeu les partie de l’intelligence les plus faciles à enflammer, cette dévotion à tout ce qui tombe sous les sens, et, finalement, ce dévouement à un matérialisme qui, pour être orné, paré, ennobli, n’en était que plus entier. Voilà ce qui unit les deux civilisations primordiales de l’Occident. L’on rencontre, dans l’une comme dans l’autre, les conséquences d’une pareille entente. Chez toutes deux, les grands monuments, chez toutes deux, les arts de la représentation de l’homme et des animaux, la peinture, la sculpture prodiguées dans les temples et les palais, et évidemment chéries par les populations. On y remarque encore l’amour égal des ajustements magnifiques, des harems somptueux, les femmes confiées aux eunuques, la passion du repos, le croissant dégoût de la guerre et de ses travaux, et enfin les mêmes doctrines de gouvernement : un despotisme tantôt hiératique, tantôt royal, tantôt nobiliaire, toujours sans limites, l’orgueil délirant dans les hautes classes, l’abjection effrénée dans les basses. Les arts et la poésie devaient être et furent, en effet, l’expression la plus apparente, la plus réelle, la plus constante de ces époques et de ces lieux.\par
Dans la poésie règne l’abandon complet de l’âme aux influences extérieures. J’en veux, pour preuve, ramassée ou hasard, cette espèce de lamentation phénicienne à la mémoire de Southoul, fille de Kabirchis, gravée à Eryx sur son tombeau :\par
« Les montagnes d’Eryx gémissent. C’est partout le son des cithares et les « chants, et la plainte des harpes dans l’assemblée de la maison de Mécamosch.\par
 « Son peuple a-t-il encore sa pareille ? Sa magnificence était comme un « torrent de feu.\par
« Plus que la neige brillait l’éclat de son regard... Ta poitrine voilée était comme « le cœur de la neige.\par
« Telle qu’une fleur fanée, notre âme est flétrie par ta perte ; elle est brisée par « le gémissement des chants funéraires.\par
« Sur notre poitrine coulent nos larmes \footnote{Blau, \emph{Zeitschrift der deutsch. Morgenl. Geselisch}, t. III, p. 448.}. »\par
Voilà le style lapidaire des Sémites.\par
Tout dans cette poésie est brûlant, tout vise à emporter les sens, tout est extérieur. De telles strophes n’ont pas pour but d’éveiller l’esprit et de le transporter dans un monde idéal. Si, en les écoutant, on ne pleure, si l’on ne crie, si l’on ne déchire ses habits, si l’on ne couvre son visage de cendres, elles ont manqué leur but. C’est là le souffle qui a passé depuis dans la poésie arabe, lyrisme sans bornes, espèce d’intoxica­tion qui touche à la folie et nage quelquefois dans le sublime.\par
Lorsqu’il s’agit de peindre dans un style de feu, avec des expressions d’une énergie furieuse et vagabonde, des sensations effrénées, les fils de Cham et ceux de Sem ont su trouver des rapprochements d’images, des violences d’expression qui, dans leurs incohérences, en quelque sorte volcaniques, laissent de bien loin derrière elles tout ce qu’a pu suggérer aux chanteurs des autres nations l’enthousiasme ou le désespoir.\par
La poésie des Pharaons a laissé moins de traces que celle des Assyriens, dont tous les éléments nécessaires se retrouvent soit dans la Bible, soit dans les compilations arabes du Kitab-Alaghani, du Hamasa et des Moallakats. Mais Plutarque nous parle des chansons des Égyptiens, et il semblerait que le naturel assez régulier de la nation ait inspiré à ses poètes des accents sinon plus raisonnables, du moins un peu plus tièdes. Au reste, pour l’Égypte comme pour l’Assyrie, la poésie n’avait que deux formes ou lyrique, ou didactique, froidement et faiblement historique, et, dans ce dernier cas, ne poursuivant d’autre but que d’enfermer des faits dans une forme cadencée et commode pour la mémoire. Ni en Égypte, ni en Assyrie, on ne trouve ces beaux et grands poèmes qui ont besoin pour se produire de facultés bien supérieures à celles d’où peut jaillir l’effusion lyrique. Nous verrons que la poésie épique est le privilège de la famille ariane ; encore n’a-t elle tout son feu, tout son éclat, que chez les nations de cette branche qui ont été atteintes par le mélange mélanien.\par
À côté de cette littérature si libérale pour la sensation, et si stérile pour la réflexion, se placent la peinture et la sculpture. Ce serait une faute que d’en parler en les séparant ; car si la sculpture était assez perfectionnée pour qu’on pût l’étudier et l’admirer à part, il n’en était pas de même de sa sœur, simple annexe de la figuration en relief, et qui, dénuée du clair-obscur comme de la perspective, et ne procédant que par teintes plates, se rencontre quelquefois isolée dans les hypogées, mais ne sert alors qu’à l’ornementation, ou bien laisse regretter l’absence de la sculpture qu’elle devrait recouvrir. Une peinture plate ne peut valoir que pour une abréviation.\par
D’ailleurs, comme il est fort douteux que la sculpture se soit jamais passée du complément des couleurs, et que les artistes assyriens ou égyptiens aient consenti à présenter aux regards exigeants de leurs spectateurs matérialistes des œuvres habillées uniquement des teintes de la pierre, du marbre, du porphyre ou du basalte ; Séparer les deux arts ou élever la peinture à un rang d’égalité avec la sculpture, c’est se méprendre sur l’esprit de ces antiquités. Il faut, à Ninive et à Thèbes, ne se figurer les statues, les hauts, les bas et les demi-reliefs, que dorés et peints des plus riches couleurs.\par
Avec quelle exubérance la sensualité assyrienne et égyptienne s’empressait de se ruer vers toutes les manifestations séduisantes de la matière ! À ces imaginations surexcitées et voulant toujours l’être davantage, l’art devait arriver non par la réflexion, mais par les yeux, et lorsqu’il avait touché juste, il en était récompensé par de prodigieux enthousiasmes et une domination presque incroyable. Les voyageurs qui parcourent aujourd’hui l’Orient remarquent, avec surprise, l’impression profonde, et quelque peu folle, produite sur les populations par les représentations figurées, et il n’est pas un penseur qui ne reconnaisse, avec la Bible et le Coran, l’utilité spiritualiste de la prohibition jetée sur l’imitation des formes humaines chez des peuples si singulièrement enclins à outrepasser les bornes d’une légitime admiration, et à faire des arts du dessin la plus puissante des machines démoralisatrices.\par
De telles dispositions excessives sont, tout à la fois, favorables et contraires aux arts. Elles sont favorables, parce que, sans la sympathie et l’excitation des masses, il n’y a pas de création possible. Elles nuisent, elles empoisonnent, elles tuent l’inspira­tion, parce que, l’égarant dans une ivresse trop violente, elles l’écartent de la recherche de la beauté, abstraction qui doit se poursuivre en dehors et au-dessus du gigantesque des formes et de la magie des couleurs.\par
L’histoire de l’art a beaucoup à apprendre encore, et on pourrait dire qu’à chacune de ses conquêtes elle aperçoit de nouvelles lacunes. Toutefois, depuis Winckelmann, elle a fait des découvertes qui ont changé ses doctrines à plusieurs reprises. Elle a renoncé à attribuer à l’Égypte les origines de la perfection grecque. Mieux renseignée, elle les cherche désormais dans la libre allure des productions assyriennes. La compa­raison des statues éginétiques avec les bas-reliefs de Khorsabad ne peut manquer de faire naître entre ces deux manifestations de l’art l’idée d’une très étroite parenté.\par
Rien de plus glorieux pour la civilisation de Ninive que de s’être avancée si loin sur la route qui devait aboutir à Phidias. Cependant ce n’était pas à ce résultat que tendait l’art assyrien. Ce qu’il voulait, c’était la splendeur, le grandiose, le gigantesque, le sublime, et non pas le beau. Je m’arrête devant ces sculptures de Khorsabad, et qu’y vois-je ? Bien certainement la production d’un ciseau habile et libre. La part faite à la convention est relativement petite, si l’on compare ces grandes œuvres à ce qui se voit dans le temple-palais de Karnak et sur les murailles du Memnonium. Toutefois, les attitudes sont forcées, les muscles saillants, leur exagération systématique. L’idée de la force oppressive ressort de tous ces membres fabuleusement vigoureux, orgueilleuse­ment tendus. Dans le buste, dans les jambes, dans les bras, le désir qui animait l’artiste, de peindre le mouvement et la vie, est poussé au delà de toutes mesures. Mais la tête ? la tête, que dit-elle ? que dit le visage, ce champ de la beauté, de la conception idéale, de l’élévation de la pensée, de la divinisation de l’esprit ? La tête, le visage, sont nuls, sont glacés. Aucune expression ne se peint sur ces traits impassibles. Comme les combat­tants du temple de Minerve, ils ne disent rien ; les corps luttent, mais les visages ne souffrent ni ne triomphent. C’est que là il n’était pas question de l’âme, il ne s’agissait que du corps. C’était le fait et non la pensée qu’on recherchait ; et la preuve que ce fut bien l’unique cause de l’éternel temps d’arrêt où mourut l’art assyrien, c’est que, pour tout ce qui n’est pas intellectuel, pour tout ce qui s’adresse uniquement à la sensation, la perfection a été atteinte. Lorsque l’on examine les détails d’ornementation de Khorsabad, ces grecques élégantes, ces briques émaillées de fleurs et d’arabesques délicieuses, on convient bien vite avec soi-même que le génie hellénique n’a eu là qu’à copier, et n’a rien trouvé à ajouter è la perfection de ce goût, non plus qu’à la fraîcheur gracieuse et correcte de ces inventions.\par
Comme l’idéalisation morale est nulle dans l’art assyrien, celui-ci ne pouvait, malgré ses grandes qualités, éviter mille énormités monstrueuses qui l’accompagnèrent sans cesse et qui furent son tombeau. C’est ainsi que les Kabires et les Telchines sémites fabriquèrent, pour l’édification de la Grèce, leur demi-compatriote, ces idoles mécani­ques, remuant les bras et les jambes, imitées depuis par Dédale, et bientôt méprisées par le sens droit d’une nation trop mâle pour se plaire à de telles futilités. Quant aux populations féminines de Cham et de Sem, je suis bien persuadé qu’elles ne s’en lassèrent jamais ; l’absurde ne pouvait exister pour elles dans des tendances à imiter, d’aussi près que possible, ce que la nature présente de matériellement vrai.\par
Qu’on pense au Baal de Malte avec sa perruque et sa barbe blondes, rougeâtres ou dorées ; que l’on se rappelle ces pierres informes, habillées de vêtements splendides et saluées du nom de divinités dans les temples de Syrie, et que de là on passe à la laideur systématique et repoussante des poupées hiératiques de l’Armeria de Turin, il n’y a rien, dans toutes ces aberrations, que de très conforme aux penchants de la race chamite et de son alliée. Elles voulaient, l’une et l’autre, du frappant, du terrible, et, à défaut de gigantesque, elles se jetaient dans l’effroyable et frottaient leurs sensations même au dégoûtant. C’était une annexe naturelle du culte rendu aux animaux.\par
Ces considérations s’appliquent également à l’Égypte, avec cette seule différence que, dans cette société plus méthodique, le vilain et le difforme ne se développèrent pas avec la même abondance de liberté sauvage où s’abandonnaient Ninive et Carthage. Ces tendances revêtirent les formes immobiles de la nationalité qui les introduisait, du reste, bien volontiers, dans son panthéon.\par
Ainsi, les civilisations de l’Euphrate et du Nil sont également caractérisées par la prédominance victorieuse de l’imagination sur la raison, et de la sensualité sur le spiritualisme. La poésie lyrique et le style des arts du dessin furent les expressions intellectuelles de cette situation. Si l’on remarque, en outre, que jamais la puissance des arts ne fut plus grande, puisqu’elle atteignit et dépassa les bornes que partout ailleurs le sens commun réussit à lui imposer et que, dans ces dangereuses divagations, elle envahit de beaucoup sur le domaine théologique, moral, politique et social, on se demandera quelle fut la cause, l’origine première de cette loi exorbitante des sociétés primitives.\par
Le problème est, je crois, résolu déjà pour le lecteur. Il est bon, cependant, de regarder si, dans d’autres lieux et dans d’autres temps, rien de semblable ne s’est représenté. L’Inde mise à part, et encore l’Inde d’une époque postérieure à sa véritable civilisation ariane, non, rien de semblable n’a jamais existé. Jamais l’imagination humaine ne s’est ainsi trouvée libre de tout frein et n’a éprouvé, avec tant de soif et tant de faim de la matière, de si indomptables penchants à la dépravation ; le fait est donc, sans contestation, particulier à l’Assyrie et à l’Égypte. Ceci fixé, considérons encore, avant de conclure, une autre face de la question.\par
Si l’on admet, avec les Grecs et les juges les plus compétents en cette matière, que l’exaltation et l’enthousiasme sont la vie du génie des arts, que ce génie, même lorsqu’il est complet, confine à la folie, ce ne sera dans aucun sentiment organisateur et sage de notre nature que nous irons en chercher la cause créatrice, mais bien au fond des soulèvements des sens, dans ces ambitieuses poussées qui les portent à marier l’esprit et les apparences, afin d’en tirer quelque chose qui plaise mieux que la réalité. Or, nous avons vu que, pour les deux civilisations primitives, ce qui organisa, disciplina, inventa des lois, gouverna à l’aide de ces lois, en un mot, fit œuvre de raison, ce fut l’élément blanc, chamite, arian et sémite. Dès lors se présente cette conclusion toute rigoureuse, que la source d’où les arts ont jailli est étrangère aux instincts civilisateurs. Elle est cachée dans le sang des noirs. Cette universelle puissance de l’imagination, que nous voyons envelopper et pénétrer les civilisations primordiales, n’a pas d’autre cause que l’influence toujours croissante du principe mélanien.\par
Si cette assertion est fondée, voici ce qui doit arriver : la puissance des arts sur les masses se trouvera toujours être en raison directe de la quantité de sang noir que celles-ci pourront contenir. L’exubérance de l’imagination sera d’autant plus forte que l’élément mélanien occupera plus de place dans la composition ethnique des peuples. Le principe se confirme par l’expérience : maintenons en tête du catalogue les Assyriens et les Égyptiens.\par
Nous mettrons à leurs côtés la civilisation hindoue, postérieure à Sakya-Mouni ;\par
Puis viendront les Grecs ;\par
À un degré inférieur, les Italiens du moyen âge ;\par
Plus bas, les Espagnols ;\par
Plus bas encore, les Français des temps modernes ;\par
Et enfin, après ceux-ci, tirant une ligne, nous n’admettrons plus rien que des inspi­rations indirectes et des produits d’une imitation savante, non avenues pour les masses populaires.\par
C’est, dira-t-on, une bien belle couronne que je pose sur la tête difforme du nègre, et un bien grand honneur à lui faire que de grouper autour de lui le chœur harmonieux des Muses. L’honneur n’est pas si grand. Je n’ai pas dit que toutes les Piérides fussent là réunies, il y manque les plus nobles, celles qui s’appuient sur la réflexion, celles qui veulent la beauté préférablement à la passion. En outre, que faut-il pour construire une lyre ? un fragment d’écaille et des morceaux de bois ; et je ne sache pas que personne ait rapporté à la traînante tortue, au cyprès, voire aux entrailles du porc ou au laiton de la mine, le mérite des chants du musicien : et cependant, sans tous ces ingrédients nécessaires, quelle musique harmonieuse, quels chants inspirés ?\par
Certainement l’élément noir est indispensable pour développer le génie artistique dans une race, parce que nous avons vu quelle profusion de feu, de flammes, d’étin­celles, d’entraînement, d’irréflexion réside dans son essence, et combien l’imagination, ce reflet de la sensualité, et toutes les appétitions vers la matière le rendent propre à subir les impressions que produisent les arts, dans un degré d’intensité tout à fait inconnu aux autres familles humaines. C’est mon point de départ, et s’il n’y avait rien à ajouter, certainement le nègre apparaîtrait comme le poète lyrique, le musicien, le sculpteur par excellence. Mais tout n’est pas dit, et ce qui reste modifie considérablement la face de la question. Oui, encore, le nègre est la créature humaine la plus énergiquement saisie par l’émotion artistique, mais à cette condition indispensable que son intelligence en aura pénétré le sens et compris la portée. Que si vous lui montrez la Junon de Polyclète, il est douteux qu’il l’admire. Il ne sait ce que c’est que Junon, et cette représentation de marbre destinée à rendre certaines idées transcendantales du beau qui lui sont bien plus inconnues encore, le laissera aussi froid que l’exposition d’un problème d’algèbre. De même, qu’on lui traduise des vers de l’Odyssée, et notamment la rencontre d’Ulysse avec Nausicaa, le sublime de l’inspiration réfléchie : il dormira. Il faut chez tous les êtres, pour que la sympathie éclate, qu’au préalable l’intelligence ait compris, et là est le difficile avec le nègre, dont l’esprit est obtus, incapable de s’élever au-dessus du plus humble niveau, du moment qu’il faut réfléchir, apprendre, comparer, tirer des conséquences. La sensitivité artistique de cet être, en elle-même puissante au delà de toute expression, restera donc nécessairement bornée aux plus misérables emplois. Elle s’enflammera et elle se passionnera, mais pour quoi ? Pour des images ridicules grossièrement coloriées. Elle frémira d’adoration devant un tronc de bois hideux, plus émue d’ailleurs, plus possédée mille fois, par ce spectacle dégradant, que l’âme choisie de Périclès ne le fut jamais aux pieds du Jupiter Olympien. C’est que le nègre peut relever sa pensée jusqu’à l’image ridicule, jusqu’au morceau de bois hideux, et qu’en face du vrai beau cette pensée est sourde, muette et aveugle de naissance. Il n’y a donc pas là d’entraînement possible pour elle. Aussi, parmi tous les arts que la créature mélanienne préfère, la musique tient la première place, en tant qu’elle caresse son oreille par une succession de sons, et qu’elle ne demande rien à la partie pensante de son cerveau. Le nègre l’aime beaucoup, il en jouit avec excès ; pourtant, combien il reste étranger à ces conventions délicates par lesquelles l’imagination européenne a appris à ennoblir les sensations !\par
Dans l’air charmant de Paolino du \emph{Mariage secret} :\par
Pria che spunfi in ciel’ l’aurora, etc…\\
\noindent la sensualité du blanc éclairé, dirigée par la science et la réflexion, va, dès les premières mesures, se faire, comme on dit, un tableau. La magie des sons évoque autour de lui un horizon fantastique où les premières lueurs de l’aube jonchent un ciel déjà bleu ! L’heureux auditeur sent la fraîche chaleur d’une matinée printanière se répandre et le pénétrer dans cette atmosphère idéale où le ravissement le transporte. Les fleurs s’ouvrent, secouent la rosée, répandent discrètement leurs parfums au-dessus du gazon humide parsemé déjà de leurs pétales. La porte du jardin s’ouvre, et, sous les clématites et les pampres dont elle est demi cachée, paraissent, appuyés l’un sur l’autre, les deux amants qui vont s’enfuir. Rêve délicieux ! les sens y soulèvent doucement l’esprit et le bercent dans les sphères idéales où le goût et la mémoire lui offrent la part la plus exquise de son délicat plaisir.\par
Le nègre ne voit rien de tout cela. Il n’en saisit pas la moindre part et cependant, qu’on réussisse à éveiller ses instincts : l’enthousiasme, l’émotion, seront bien autrement intenses que notre ravissement contenu et notre satisfaction d’honnêtes gens.\par
Il me semble voir un Bambara assistant à l’exécution d’un des airs qui lui plaisent. Son visage s’enflamme, ses yeux brillent. Il rit, et sa large bouche montre, étincelantes au milieu de sa face ténébreuse, ses dents blanches et aiguës. La jouissance vient, l’Africain se cramponne à son siège : on dirait qu’en s’y pelotonnant, en ramenant ses membres les uns sous les autres, il cherche, par la diminution d’étendue de sa surface, à concentrer davantage dans sa poitrine et dans sa tête les crispations tumultueuses du bien-être furieux qu’il éprouve. Des sons inarticulés font effort pour sortir de sa gorge, que comprime la passion ; de grosses larmes roulent sur ses joues proéminentes ; encore un moment, il va crier : la musique cesse, il est accablé de fatigue \footnote{Le mot\emph{ ku-teta} signifie en cafre \emph{parler}, et en suahili, \emph{se battre}, parce que l’expression violente et criarde des Africains ressemble à une querelle. (Krapf, \emph{Von der afrikanischen Ostküste}, dans la \emph{Zeitschrift der deutsch. morgenl. Gesellschaft}, t. III, p. 317.)}.\par
Dans nos habitudes raffinées, nous nous sommes fait de l’art quelque chose de si intimement lié avec ce que les méditations de l’esprit et les suggestions de la science ont de plus sublime, que ce n’est que par abstraction, et avec un certain effort, que nous pouvons en étendre la notion jusqu’à la danse. Pour le nègre, au contraire, la danse est, avec la musique, l’objet de la plus irrésistible passion. C’est parce que la sensualité est pour presque tout, sinon tout, dans la danse. Aussi tenait-elle une bien grande place dans l’existence publique et privée des Assyriens et des Égyptiens ; et là où le monde antique de Rome la rencontrait encore plus curieuse et plus enivrante que partout ailleurs, c’est encore là que nous, modernes, nous allons la chercher, chez les populations sémitiques de l’Espagne, et principalement à Cadix.\par
Ainsi le nègre possède au plus haut degré la faculté sensuelle sans laquelle il n’y a pas d’art possible ; et, d’autre part, l’absence des aptitudes intellectuelles le rend complètement impropre à la culture de l’art, même à l’appréciation de ce que cette noble application de l’intelligence des humains peut produire d’élevé. Pour mettre ses facultés en valeur, il faut qu’il s’allie à une race différemment douée. Dans cet hymen, l’espèce mélanienne apparaît comme personnalité féminine, et bien que ses branches diverses présentent, sur ce point, du plus ou du moins, toujours, dans cette alliance avec l’élément blanc, le principe mâle est représenté par ce dernier. Le produit qui en résulte ne réunit pas les qualités entières des deux races. Il a de plus cette dualité même qui explique la fécondation ultérieure. Moins véhément dans la sensualité que les individualités absolues du principe féminin, moins complet dans la puissance intellectuelle que celles du principe mâle, il jouit d’une combinaison des deux forces qui lui permet la création artistique, interdite à l’une et à l’autre des souches associées. Il va sans dire que cet être que j’invente est abstrait, tout idéal. On ne voit que rarement, et par l’effet de circonstances très multiples, des entités dans lesquelles ces principes générateurs se reproduisent et s’affrontent à forces convenablement pondérées. En tout cas, et si on peut croire à de telles combinaisons chez des hommes isolés, il n’y faut pas penser une minute pour les nations, et il n’est question ici que de ces dernières. Les éléments ethniques sont en constante oscillation dans les masses. Il est tellement difficile de saisir les moments où ils se trouvent à peu près en équilibre ; ces moments sont si rapides, si impossibles à prévoir, qu’il vaut mieux n’en pas parler et ne raisonner que sur ceux où tel élément, l’emportant manifestement sur l’autre, préside un peu plus longuement aux destinées nationales.\par
Les deux civilisations primordiales fortement imbues de germes mélaniens, en même temps que dirigées et inspirées par la puissance propre à la race blanche, ont dû à la prédominance de plus en plus déclarée de l’élément noir l’exaltation qui les caractérisa : la sensualité fut donc leur cachet principal et commun.\par
L’Égypte, peu ou point régénérée, se montra moins longtemps agissante que les nations chamites noires, si heureusement renouvelées par le sang sémitique. Le pays avait pourtant dans son mobile arian quelque chose d’évidemment supérieur ; mais la marée montante du sang mélanien, sans détruire absolument les prérogatives de ce sang, les domina, et, donnant à la nation cette immobilité qu’on lui reproche, ne lui permit de sortir de l’immense que pour tomber dans le grotesque.\par
La société assyrienne reçut, de la série d’invasions blanches qui la renouvelèrent, plus d’indépendance dans ses inspirations artistiques. Elle y gagna aussi, il faut l’avouer, une splendeur plus éclatante ; car si rien, dans le genre sublime, ne dépasse la majesté des pyramides et de certains temples palais de la haute Égypte, ces merveilleux monuments n’offrent pas de représentations humaines qui, pour la fermeté de l’exécution, la science des formes, puissent être comparées aux superbes bas-reliefs de Khorsabad. Quant à la partie d’ornementation des édifices ninivites, comme les mosaïques, les briques émaillées, j’en ai déjà dit tout ce que le jugement le moins favorable serait contraint de reconnaître : que les Grecs eux-mêmes n’ont su que copier ces inventions, et n’en ont dépassé jamais le goût sûr et exquis.\par
Malheureusement le principe mélanien était trop fort et devait l’emporter. Les belles sculptures assyriennes, qu’il faut rejeter dans une antiquité antérieure au septième siècle avant J.-C., ne marquèrent qu’une période assez courte. Après la date que j’indique, la décadence fut profonde, et le culte de la laideur, si cher à l’incapacité des noirs, ce culte toujours triomphant, toujours pratiqué, même à côté des chefs-d’œuvre les plus frappants, finit par l’emporter tout à fait.\par
D’où il résulte que, pour assurer aux arts une véritable victoire, il fallait obtenir un mélange du sang des noirs avec celui des blancs, dans lequel le dernier entrât pour une proportion plus forte que les meilleurs temps de Memphis et de Ninive n’avaient pu l’obtenir, et formât ainsi une race douée d’infiniment d’imagination et de sensibilité unies à beaucoup d’intelligence. Ce mélange fut combiné plus tard lorsque les Grecs méridionaux apparurent dans l’histoire du monde.
\chapterclose


\chapteropen
\chapter[{III. Civilisation rayonnant de l’asie centrale vers le sud et sud-est.}]{III. \\
Civilisation rayonnant de l’asie centrale vers le sud et sud-est.}\renewcommand{\leftmark}{III. \\
Civilisation rayonnant de l’asie centrale vers le sud et sud-est.}


\chaptercont
\section[{III.1. Les Arians ; les brahmanes et leur système social.}]{III.1. \\
Les Arians ; les brahmanes et leur système social.}
\noindent Je suis parvenu à l’époque où Babylone fut prise d’assaut par les Mèdes. L’empire assyrien va changer tout à la fois de forme et de valeur. Les fils de Cham et de Sem cesseront à jamais d’être au premier rang des nations. Au lieu de diriger et de conduite les États, ils en formeront désormais le fond corrupteur. Un peuple arian paraît sur la scène, et, se laissant mieux apercevoir et juger que le rameau de même race enveloppé dans les alliages égyptiens, il nous invite à considérer de près, et avec l’attention qu’elle mérite, cette illustre famille humaine, la plus noble, sans contredit, de l’extraction blanche.\par
Ce serait s’exposer à mettre cette vérité dans un jour incomplet, que de présenter les Mèdes, sans avoir préalablement étudié et connu tout le groupe dont ils ne sont qu’une faible fraction. Je ne puis donc commencer par eux. Je m’attacherai d’abord aux branches les plus puissantes de leur parenté. À cet effet, je vais m’enfoncer dans les régions situées à l’orient de l’Indus, où se sont développés d’abord les plus consi­dérables essaims des peuples arians.\par
Mais ces premiers pas, détournés de la partie de l’histoire que j’ai d’abord examinée, m’entraîneront au delà des régions hindoues ; car la civilisation brahmanique, à peu près étrangère à l’occident du monde, a puissamment vivifié la région orientale, et, rencontrant là des races que l’Assyrie et l’Égypte n’ont qu’entrevues, elle s’est trouvée en contact intime avec les hordes jaunes. L’étude de ces rapports et de leurs résultats est de première importance. Nous verrons, avec ce secours, si la supériorité de la race blanche pourra s’établir vis-à-vis des Mongols comme vis-à-vis des noirs, dans quelle mesure l’histoire la démontre, et par suite l’état respectif des deux races inférieures et de leurs dérivées.\par
Il est difficile de trouver des synchronismes entre les émigrations primordiales des Chamites et celles des Arians ; il ne l’est pas moins de se soustraire au besoin d’en chercher. La descente des Hindous dans le Pendjab est un fait si reculé au delà de toutes les limites de l’histoire positive, la philologie lui assigne une date si ancienne, que cet événement paraît toucher aux époques antérieures à l’an 4000 avant J.-C. Chamites et Arians auraient ainsi quitté, à peu près à la même heure et sous le coup des mêmes nécessités, les demeures primordiales de la famille blanche, pour descendre dans le sud, les uns vers l’ouest, les autres vers l’orient.\par
Les Arians, plus heureux que les Chamites, ont gardé, pendant une longue série de siècles, avec leur langue nationale, annexe sacrée de l’idiome blanc primitif, un type physique qui ne les exposa pas, tant il resta particulier, à être confondus parmi les populations noires. Pour expliquer ce double phénomène, il faut admettre que, devant leurs pas, les races aborigènes se retiraient, dispersées ou détruites par des incursions d’avant-garde, ou bien qu’elles étaient très clairsemées dans les vallées hautes du Kachemyr, premier pays hindou envahi par les conquérants. Du reste, il n’y a pas à douter que la population première de ces contrées n’appartînt au type noir \footnote{Lassen, \emph{Indisch. Atterth.}, t. I, p. 853; voir la note 1 p. 229 de ce volume. L’Himalaya contient de nombreux débris de populations noires ou mulâtres qui sont certainement aborigènes.}. Les tribus mélaniennes que l’on rencontre encore aujourd’hui dans le Kamaoun en portent témoignage. Elles sont formées des descendants des fugitifs qui, n’ayant pas suivi leurs congénères lors du grand reflux vers les monts Vyndhia et le Dekkhan \footnote{D’après Ritter, les peuples sanscrits ont repoussé jusqu’à Lanka (Ceylan) les nègres et les métis jaunes et noirs (Malais), qui s’étendaient primitivement dans le nord. (Ritter, \emph{Erdkunde, Asien}, t. I, p. 435.)}, se sont jetés au milieu des gorges alpestres, asile sûr, puisqu’ils y conservent leur individualité depuis des séries d’années incalculables.\par
Avant de mettre le pied plus avant sur le sol de l’Inde, saisissons tout l’ensemble de la famille ariane primitive, à ce moment où son mouvement de marche vers le sud est déjà prononcé, mais où, toutefois, si elle a commencé à envahir la vallée de Kachemyr par ses têtes de colonnes, le gros de ses nations n’a pas encore dépassé la Sogdiane.\par
Déjà les Arians sont détachés des nations celtiques, acheminées vers le nord-ouest et contournant la mer Caspienne par le haut ; tandis que les Slaves, très peu différents de ce dernier et vaste amas de peuples, suivent vers l’Europe une route plus septentrionale encore.\par
Les Arians donc, longtemps avant d’arriver dans l’Inde, n’avaient plus rien de commun avec les nations qui allaient devenir européennes. Ils formaient une immense multitude tout à fait distincte du reste de l’espèce blanche, et qui a besoin d’être désignée, ainsi que je le fais, par un nom spécial. Par malheur, des savants de premier ordre n’ont pas apprécié cette nécessité. Absorbés par la philologie, ils ont donné un peu légèrement, à l’ensemble des langues de la race, le nom fort inexact d’indo-germanique, sans s’arrêter à cette considération, pourtant très sérieuse, que, de tous les peuples qui possèdent ces idiomes, un seul est allé dans l’Inde, tandis que les autres n’en ont jamais approché. Le besoin, d’ailleurs impérieux, des classifications a été de tout temps la source principale des erreurs scientifiques. Les langues de la race blanche ne sont pas plus hindoues que celtiques \footnote{Si l’on voulait absolument appliquer aux groupes de langues des noms de nations, il serait plus raisonnable pourtant de qualifier le rameau arian d’\emph{hindou-celtique}. On aurait du moins ainsi la désignation des deux extrêmes géographiques, et on indiquerait les deux faces les plus différentes du système ; mais, pour mille causes, cette dénomination serait encore détestable.}, et je les vois beaucoup moins germaniques que grecques. Le plus tôt on renoncera à ces dénominations géographiques sera le mieux.\par
Le nom d’Arian possède cet avantage précieux d’avoir été choisi par les tribus mêmes auxquelles il s’applique, et de les suivre partout indépendamment des lieux qu’elles habitent ou ont pu habiter. Ce nom est le plus beau qu’une race puisse adopter : il signifie honorable \footnote{Lassen, \emph{Indisch. Alterth}., t. I, p. 6 ; Burnouf, \emph{Commentaire sur le Yaçna}, t. I, p. 461, note.} ; ainsi, les nations arianes étaient des nations \emph{d’hommes honorables}, d’hommes dignes d’estime et de respect, et probablement, par extension, d’hommes qui, lorsqu’on ne leur rendait pas ce qui leur était dû, savaient le prendre. Si cette interprétation n’est pas strictement dans le mot, on verra qu’elle se trouve dans les faits.\par
Les peuples blancs qui s’appliquèrent cette dénomination en comprenaient la portée hautaine et pompeuse. Ils s’y attachèrent avec force, et ne la laissèrent que tardivement disparaître sous les qualifications particulières que chacun d’eux se donna par la suite. Les Hindous appelèrent le pays sacré, l’Inde légale, \emph{Arya-varta}, la terre des hommes honorables \footnote{Le \emph{Manava-Dharma-Sastra}, traduction de Haughton, partage le territoire national, en dehors duquel un çoudra, pressé par la faim, a seul le droit d’habiter, en plusieurs catégories. Voici sa classification (t. XI, chap. II, § 17) : « Between the two divine rivers « Saraswati and Drishadwati, lies the tract of land, which the sages have named « Brahmaverta, because it was frequented by « Gods. » (C’est le territoire primitivement habité par les Arians purs de tout mélange noir ou jaune.) Viennent maintenant les §§ 21 et 22, qui s’expriment ainsi : « That country which lies between Himawat and Vindhya, to « the east of Vinasana and to the west of Prayaga, is celebrated by the title of « Medhyadesa, or the central region. » §22 : « As far as the eastern, and as far as the « western Oceans between the two mountains just mentioned, lies the tract which the wise « have named Aryaverta, or inhabited by respectable men. »}. Plus tard, quand ils furent divisés en castes, le nom d’Arya resta au gros de la nation, aux Vaycias, la dernière catégorie des vrais Hindous, deux fois nés, lecteurs des Védas.\par
Le nom primitif, réclamé par les Arians Iraniens, auxquels appartenaient les Mèdes, fut (alphabet étranger). Une autre branche de cette famille, les Perses, avaient également commencé par s’appeler (alphabet étranger), et quand ils y renoncèrent pour l’ensemble de la nation, ils conservèrent la racine de ce mot dans la plupart de leurs noms d’hommes, tels qu’Arta-xerxès, Ario-barzane, Arta-baze, et les prêtèrent ainsi faits aux Scythes-Mongols convertis à leur langage, et qui trouvèrent plus tard à en renouveler l’usage dans l’emploi qu’en faisaient de leur côté les Arians Sarmates \footnote{Lassen, \emph{Indisch. Alterth}., t. I, p. 6.}.\par
Dans leurs idées cosmogoniques, les Iraniens regardaient comme le pays le premier créé une région qu’ils appelaient \emph{Airyanem-Vaëgo}, et ils la plaçaient bien loin dans le nord-est, vers les sources de l’Oxus et du Yaxartes \footnote{ \noindent \emph{Ibid}., 526. On trouve, aux époques historiques, un grand nombre de noms de peuples arians dans ce pays, que les Orientaux appellent le Touran, et que, jusqu’ici, on a faussement considéré comme habité par des hordes jaunes exclusivement. Ainsi, on y voit, avec Pline, les Ariacæ, les Antariani, les Aramæi, qui rappellent si fort le mot zend aïryaman. (Burnouf, \emph{Commentaire sur le Yaçna}, t. I, p. CV-CVI).\par
 Burnouf remarque aussi que des dénominations de lieux évidemment arianes sont celles où l’on trouve les mots : Açp, cheval, arvat ou aurvat, eau, pati, maître. Ptolémée en cite dans la Scythie et même dans la Sérique, Açpabota, Açpacara, Açparatah.
}. Ils se rappelaient que là l’été ne durait que deux mois de l’année, et que, pendant dix autres mois, l’hiver y sévissait avec une rigueur extrême. Ainsi, pour eux, le pays des hommes honorables était resté l’ancienne patrie ; tandis que les Hindous des temps postérieurs, attachés au nom et oubliant la chose, transportèrent la désignation et en firent don à leur patrie nouvelle.\par
Cette racine \emph{ar} suivit partout les rameaux divers de la race et les préoccupa constamment. Les Grecs la montrent, bien conservée et en bon lieu, dans le mot (en grec) qui personnifie l’être honorable par excellence, le dieu des batailles, le héros parfait ; dans cet autre mot, (en grec), qui indique d’abord la réunion des qualités nécessaires à un homme véritable, la bravoure, la fermeté, la sagesse, et qui, plus tard, voulut dire la vertu. On le trouve encore dans cette expression de (en grec), qui se rapporte à l’action d’honorer les puissances surhumaines ; enfin, il ne serait pas trop hardi, peut-être, ni contraire à toute bonne étymologie de voir l’appellation générique de la famille ariane attachée à une de ses plus glorieuses descendances, en rapprochant les mots \emph{arya, ayrianem}, de (en grec), et de (en grec). Les Grecs, en se séparant à une époque antique du faisceau commun, n’auraient point abjuré son nom ni dans leurs habitudes de pensée, le fait est incontestable, ni même dans leur dénomination nationale.\par
On pourrait pousser beaucoup plus loin cette recherche, et l’on trouverait cette racine \emph{ar, ir} ou\emph{ er}, conservée jusque dans le mot allemand moderne \emph{Ehre}, qui semble prouver qu’un sentiment d’orgueil fondé sur le mérite moral a toujours occupé une grande place dans les pensées de la plus belle des races humaines \footnote{La même racine se trouve dans le pa-zend \emph{hir} ou \emph{ir}, qui signifie maître, dans le latin \emph{herus} et dans l’allemand \emph{Herr}. (Burnouf, \emph{op. cit.}, t. I, p. 460.)}.\par
D’après des témoignages aussi nombreux, on trouvera peut-être à propos de rendre un jour, au réseau de peuples dont il s’agit, le nom général et très mérité qu’il s’était appliqué à lui-même et de renoncer à ces appellations de Japhétides, de Caucasiens et d’Indo-Germains, dont on ne saurait trop signaler les inconvénients. En attendant cette restitution bien désirable pour la clarté des généalogies humaines, je me permettrai de la devancer, et je formerai une classe particulière de tous les peuples blancs qui, ayant inscrit cette qualification soit sur des monuments de pierre, soit dans leurs lois, soit dans leurs livres, ne permettent pas qu’on la leur enlève. Partant de ce principe, je crois pouvoir dénommer cette race spéciale d’après les parties qui la constituent au moment où, déjà séparée du reste de l’espèce, elle s’avance vers le sud.\par
On y compte les multitudes qui vont envahir l’Inde et celles qui, s’engageant sur la route où ont marché les Sémites, gagneront les rivages inférieurs de la mer Caspienne, et de là, passant dans l’Asie Mineure et dans la Grèce, en différentes émissions s’y nommeront les Hellènes. On y reconnaît encore ces colonnes nombreuses dont quelques-unes, descendant au sud-ouest, pénétreront jusqu’au golfe Persique, tandis que les autres, demeurant pendant des siècles aux environs de l’Imaüs, réservent les Sarmates au monde européen. Hindous, Grecs, Iraniens, Sarmates, ne forment ainsi qu’une seule race distincte des autres branches de l’espèce et supérieure à toutes \footnote{Lassen, \emph{Indisch., Alterth.}, t. I, p. 516. – J’ajouterai à l’avis de M. Lassen celui d’un grand partisan de l’unité physique et morale de l’espèce humaine. Voici l’aveu qui échappe à M. Prichard : « Diese Eindringlinge (die indo-Europæer) scheinen ihnen (den Allophylen) « überall an geistigen Gaben überlegen gewesen zu seyn. Einige indo-europæische Nationen « haben wirklich viele charakteristische Kennzeichen von Barbarei und Wildheit « zurückbehalten oder bekommen ; aber mit diesen verbanden sie alle, unzweifelhafte « Zeichen von frühzeitiger inteIlectueller Entwickelung, besonders eine hœhere Kultur der « Sprache. » (Prichard, \emph{Naturgeschichte des menschlichen Geschlechts}, t. III, 1\textsuperscript{re} partie, p. 11.)}.\par
Pour la conformation physique, il n’y a pas de doute : c’était la plus belle dont on ait jamais entendu parler \footnote{Lassen, p. 404.}. La noblesse de ses traits, la vigueur et la majesté de sa stature élancée, sa force musculaire, nous sont attestées par des témoignages qui, pour être postérieurs à l’époque où elle était réunie, n’en ont pas moins un poids irrésistible \footnote{Lassen, p. 404 et 854.}. Ils établissent tous, sur les points différents où on les recueille, une grande identité de traits généraux, et ne laissent apercevoir les déviations locales que comme des conséquences d’alliages postérieurs \footnote{C’est ainsi que M. Lassen remarque fort bien que le climat ne saurait être rendu responsable du degré de coloration des populations hindoues, attendu que les Malabares sont plus bruns que les Kandys de Ceylan, et les gens du Guzarate que ceux du Karnatik (t. I p. 407.)}. Dans l’Inde, les croisements eurent lieu avec des races noires ; dans l’Iran, avec des Chamites, des Sémites et des noirs ; en Grèce, avec des peuples blancs qu’il ne s’agit pas de déterminer ici et des Sémites. Mais le fond du type demeura partout le même, et il est peu contestable que la souche qui, même dégénérée de sa beauté primordiale, fournissait des types comme ceux des Kachemyriens actuels et comme la plupart des Brahmanes du nord, comme ceux dont la représentation a été figurée sous les premiers successeurs de Cyrus, dans les cons­tructions de Nakschi-Roustam et de Persépolis ; enfin, que les hommes dont l’aspect physique a inspiré les sculpteurs de l’Apollon Pythien, du Jupiter d’Athènes, de la Vénus de Milo, formaient la plus belle espèce d’hommes dont la vue ait pu réjouir les astres et la terre.\par
La carnation des Arians était blanche et rosée : tels apparurent les plus anciens Grecs et les Perses ; tels se montrèrent aussi les Hindous primitifs. Parmi les couleurs des cheveux et de la barbe, le blond dominait, et l’on ne peut oublier la prédilection que lui portaient les Hellènes : ils ne se figuraient pas autrement leurs plus nobles divinités. Tous les critiques ont vu, dans ce caprice d’une époque où les cheveux blonds étaient devenus bien rates à Athènes et sur les quais de l’Eurotas, un ressouvenir des âges primitifs de la race hellénique. Aujourd’hui encore, cette nuance n’est pas absolument perdue dans l’Inde, et notamment au nord, c’est-à-dire dans la partie où la race ariane a le mieux conservé et renouvelé sa pureté. Dans le Kattiwar, on trouve fréquemment des cheveux rougeâtres et des yeux bleus.\par
L’idée de la beauté est restée pour les Hindous attachée à celle de la blancheur, et rien ne le prouve mieux que les descriptions d’enfants prédestinés, si fréquentes dans les légendes bouddhiques \footnote{Burnouf, \emph{Introduction à l’histoire du bouddhisme indien}, t. I, p. 237, 314.}. Ces pieux récits montrent la divine créature, aux premiers jours de son berceau, avec le teint blanc, la peau de couleur d’or. Sa tête doit avoir la forme d’un parasol (c’est-à-dire, être ronde et éloignée de la configuration pyramidale chez les noirs). Ses bras sont longs, son front large, ses sourcils réunis, son nez proéminent.\par
Comme cette description, postérieure au VII\textsuperscript{e} siècle av. J.-C., s’applique à une race dont les meilleures branches étaient assez mélangées, on ne peut se montrer surpris d’y voir des exigences un peu anormales, telles que la couleur d’or souhaitée pour la peau du corps et les sourcils réunis. Quant au teint blanc, aux bras longs, au front large, à la tête ronde, au nez proéminent, ce sont autant de traits qui révèlent la présence de l’espèce blanche et qui, ayant continué à être caractéristiques des hautes castes, autori­sent à penser que la race ariane, dans son ensemble, les possédait également.\par
Cette variété humaine, ainsi entourée d’une suprême beauté de corps, n’était pas moins supérieure d’esprit \footnote{Lassen, \emph{Indisch. Alterth.}, t. I, p. 854.}. Elle avait à dépenser une somme inépuisable de vivacité et d’énergie, et la nature du gouvernement qu’elle s’était donné coïncide parfaitement avec les besoins d’un naturel si actif.\par
Les Arians, divisés en tribus ou petits peuples concentrés dans de grands villages \footnote{Ces villages étaient appelés \emph{pour} chez les Hindous, (en grec) chez les Grecs.}, mettaient, à leur tête, des chefs dont le pouvoir très limité n’avait rien de commun avec l’omnipotence absolue exercée par les souverains chez les peuples noirs ou chez les nations jaunes \footnote{Lassen, \emph{Indisch. Alterth.}, t. I, p. 807.}. Le nom sanscrit le plus ancien pour rendre l’idée d’un roi, d’un directeur de la communauté politique, c’est \emph{viç pati} ; le zend \emph{viç païtis} l’a parfaitement conservé, et le lithuanien \emph{wiespati} indique aujourd’hui encore un seigneur terrien \footnote{On suit très bien, dans les langues arianes, les deux parties de ce mot composé : \emph{viç}, qui signifie \emph{maison}, devient, par extension, une collection de maisons, et se retrouve dans le \emph{vicus} latin et son dérivé \emph{ci-vis}, l’habitant du \emph{vicus. Pati}, le chef, en sanscrit, c’est dans l’arménien \emph{pod}, dans le slave \emph{pod}, dans le letton \emph{patin}, dans le polonais \emph{pan}, dans le gothique \emph{faths.} (Burnouf, \emph{Comment. sur le Yaçna}, t. I, p. 461 ; Schaffarik, \emph{Stawische Alterthümer}, t. I, p. 283.)}. La signification en est tout entière dans le (mots grecs) si fréquent chez Homère et Hésiode, et, comme la monarchie grecque de l’époque héroïque, tout à fait conforme à celle des Iraniens avant Cyrus, ne montre, dans les souverains, qu’une autorité des plus limitées ; comme les épopées du Ramayana et du Mahabharata ne connaissent également que la royauté élective conférée par les habitants des villes, les brahmanes et même les rois alliés, tout nous porte à conclure qu’un pouvoir émanant, d’une façon si complète, de la volonté générale, ne devait être qu’une délégation assez faible, peut-être même précaire, tout à fait dans le goût de l’organisation germanique antérieure à l’espèce de réforme qu’en fit chez nous Khlodowig \footnote{Le \emph{Manava-Dharma-Sastra} (traduction de Haughton ; Londres, 1825, in-4°, t. II) est beaucoup plus dévoué à l’idée de la monarchie absolue que les grands poèmes ; cependant il n’a pas encore, sur ce sujet, les notions des Asiatiques modernes. Après avoir dit magnifiquement (chap. VII, t. VIII, 1) : « A King, even though a child, must not be treated « lightly, from an idea that he is a mere mortal : no ; he is a powerful divinity, who appears « in a human shape, » verset qui, par parenthèse, pourrait bien avoir été dicté par un esprit d’opposition à des doctrines différentes et antérieures, le législateur ajoute (p. 37) : « Let the « king, having risen at early dawn, respectfully attend to brahmens, learned in the three « Vedas, and in the sciences of ethicks ; and by their decision let him abide ; » et § 54 : « The king must appoint seven or eight ministers, who must be sworn by touching a « sacred image and the like ; men whose ancestors were servants of kings ; who are versed « in the holy book ; who are personally braves ; who are skilled in the use of weapons et « whose lineage is noble. » § 56 : « Let him perpetually consult with those ministers on « peace and war, on his forces, on his revenues, on the protection of his people, and on the « means of bestowing aptly the wealth which he has acquired. » § 57 : « Having « ascertained the several opinions of his « counsellors, first apart and then collectively, let him do what is most beneficial for him in public « affairs. » § 58 : « To one learned « Brahmen, distinguished among them all, let the king impart his momenteous counsel, « relating to six principal articles. » § 59 : « To him, with full confidence, let « him intrust « all transactions ; and, with him, having taken his final resolution, let him begin all his « measures. »}.\par
Ces rois des Arians, siégeant dans leurs villages, parmi des troupeaux de bœufs, de vaches et de chevaux, juges nécessaires des contestations violentes qui accidentent, à tout moment, la vie des nations pastorales, étaient entourés d’hommes plus belliqueux encore que bergers.\par
Lorsque j’ai parlé, lorsque je parle de la nation ariane, de la famille ariane, je n’entends pas dire que les différents peuples qui la formaient vécussent entre eux dans des sentiments d’affectueuse parenté \footnote{Ce serait nier l’affirmation positive des hymnes védiques. (Lassen, \emph{Indisch. Alterthüm.}, t. 1, p. 734.)}. Le contraire est incontestable : leur état le plus ordinaire paraît avoir été l’hostilité flagrante et approuvée, et ces hommes honorables ne voyaient rien de si digne d’admiration qu’un guerrier monté sur un chariot, courant, aidé de son écuyer, épuiser ses flèches contre une tribu voisine \footnote{Dans le Zend-Avesta, l’homme de guerre se nomme \emph{ratbâestâo}, celui qui est sur le chariot.}. Cet écuyer, toujours présent dans les sculptures égyptiennes, assyriennes, perses, dans les poèmes grecs ou sanscrits, dans le Schah-nameh, dans les chants scandinaves et les épopées chevale­resques du moyen âge, fut aussi dans l’Inde une figure militaire d’une grande importance.\par
 Les Arians guerroyaient donc entre eux \footnote{Lassen, \emph{Indisch. Alterth.}, t. I, p. 617.}, et comme ils n’étaient pas nomades \footnote{Lassen, \emph{ibid}., p. 816. – Bien que pasteurs par excellence, ils n’étaient pas absolu ment étrangers non plus aux travaux de l’agriculture, et je serais tenté de croire que, si, dans leur première partie, ils ne s’y adonnèrent pas davantage, c’est que le sol et le climat ne leur permettaient pas d’en tirer des avantages suffisants.}, comme ils restaient le plus longtemps possible dans la patrie qu’ils avaient adoptée, et que leur vaillante audace en avait partout fini promptement avec la résistance des indigènes, leurs expéditions les plus fréquentes, leurs campagnes les plus longues, leurs désastres les plus complets, comme aussi leurs plus beaux triomphes, n’avaient qu’eux-mêmes pour acteurs. La vertu, c’était donc l’héroïsme du combattant, et, avant toute autre considération, la bonté, c’était la bravoure, notion que l’on retrouve, bien loin de ces temps, dans les poésies italiennes où le \emph{buon Rinaldo} est aussi \emph{il gran virtuoso} de l’Arioste. Les récompenses les plus éclatantes étaient assurées aux plus énergiques champions. On les nommait \emph{çoura}, les célestes \footnote{\emph{Ibid.}, p. 734}, parce que, s’ils tombaient dans la bataille, ils allaient habiter le Svarga, palais splendide où les recevait Indra, le roi des dieux, et cet honneur était si grand, si au-dessus de tout ce que pouvait réserver l’autre vie, que, ni par les riches sacrifices, ni par l’étendue et la profondeur du savoir, ni par aucun moyen humain, il n’était donné à personne d’occuper au ciel la même place que les çouras. La mort reçue en combattant, tout mérite s’éclipsait devant celui-là. Mais la prérogative des guerroyeurs intrépides ne s’arrêtait même pas à ce point suprême. Il pouvait leur arriver, non pas seulement d’aller habiter, hôtes vénérés, la demeure éthérée des dieux : ils étaient en passe de détrôner les dieux mêmes, et, au sein de sa puissance, Indra, menacé sans cesse de se voir arracher le sceptre par un mortel indomptable, tremblait toujours \footnote{Lassen, \emph{Indisch. Alterth.}, t. I.}.\par
On trouvera entre ces idées et celles de la mythologie scandinave des rapports frappants. Ce ne sont pas des rapports, c’est une identité parfaite qu’il faut constater ici entre les opinions de ces deux tribus de la famille blanche, si éloignées par les siècles et par les lieux. D’ailleurs, cette orgueilleuse conception des relations de l’homme avec les êtres surnaturels se rencontre dans les mêmes proportions grandioses chez les Grecs de l’époque héroïque. Prométhée, enlevant le feu divin, se montre plus rusé et plus prévoyant que Jupiter ; Hercule arrache par la force Cerbère à l’Érèbe ; Thésée fait trembler Pluton sur son trône ; Ajax blesse Vénus ; et Mercure, tout dieu qu’il est, n’ose se commettre avec l’indomptable courage des compagnons de Ménélas.\par
Le Schah-nameh montre également ses champions aux prises avec les personnages infernaux, qui succombent sous la vigueur de leurs adversaires.\par
Le sentiment sur lequel se base, chez tous les peuples blancs, cette exagération fanfaronne est incontestablement une idée très franche de l’excellence de la race, de sa puissance et de sa dignité. Je ne suis pas étonné de voir les nègres reconnaître si aisément la divinité des conquérants venus du nord, quand ceux-ci supposent, de bonne foi, la puissance surnaturelle communicable à leur égard, et croient pouvoir, en certains cas, et au prix de certains exploits guerriers ou moraux, s’élever aux lieu et place d’où les dieux les contemplent, les encouragent et les redoutent. C’est une observation qui peut se faire aisément, dans l’existence commune, que les gens sincères sont pris aisément pour ce qu’ils se donnent. À plus forte raison devait-il en être ainsi quand l’homme noir d’Assyrie et d’Égypte, dépouillé et tremblant, entendait son souverain affirmer que, s’il n’était pas encore dieu, il ne tarderait pas à le devenir. Le voyant gouverner, régir, instituer des lois, défricher des forêts, dessécher des marais, fonder des villes, en un mot, accomplir cette œuvre civilisatrice dont lui-même se reconnaissait incapable, l’homme noir disait aux siens : « Il se trompe : il ne va pas devenir dieu, il l’est déjà. » Et ils l’adoraient.\par
À ce sentiment exagéré de sa dignité on pourrait croire que le cœur de l’homme blanc associait quelque penchant à l’impiété. On serait dans l’erreur ; car précisément le blanc est religieux par excellence \footnote{Lassen, \emph{Indisch. Alterth.}, t. I, p. 755.}. Les idées théologiques le préoccupent à un très haut degré. Déjà on a vu avec quel soin il conservait les anciens souvenirs cosmogoniques, dont la tribu sémite des Hébreux abrahamides posséda, moitié par son propre fonds, moitié par transmission chamitique, les fragments les plus nombreux. La nation ariane, de son côté, prêtait son témoignage à quelques-unes des vérités de la Genèse \footnote{Voici les notions cosmogoniques conservées par une des hymnes du Rigvéda : « Alors il n’y « avait ni être ni non-être. Pas d’univers, pas d’atmosphère, ni rien au-dessus ; rien, nulle « part, pour le bien de qui que ce fût, enveloppant ou enveloppé. La mort n’était pas, ni « non plus l’immortalité, ni la distinction du jour et de la nuit. Mais CELA palpitait sans « respirer, seul avec le rapport à lui-même contenu en lui. Il n’y avait rien de plus. Tout « était voilé d’obscurité et plongé dans l’eau indiscernable. Mais cette masse ainsi voilée « fut manifestée par la force de la contemplation. Le désir (\emph{kama}, l’amour) naquit d’abord « dans son essence, et ce fut la semence originelle, créatrice, que les sages, qui la « reconnaissaient dans leur propre cœur, par la méditation, distinguent, au sein du néant, « comme étant le lien de l’Existence. » – Lassen, \emph{Indisch. Atterth.}, t. I, p. 774 C’est plus profond et plus vigoureusement analysé que le langage d’Hésiode et que les chants celtiques ; mais ce n’est pas différent.}. D’ailleurs, ce qu’elle cherchait surtout dans la religion, c’étaient les idées métaphysi­ques, les prescriptions morales. Le culte en lui-même était des plus simples.\par
Également simple se montrait, à cette époque reculée, l’organisation du Panthéon. Quelque peu de dieux présidés par Indra dirigeaient plutôt qu’ils ne dominaient le monde \footnote{Un dieu antérieur à Indra paraît avoir été \emph{Vourounas}, ou\emph{ Vouranas} ; il est devenu, depuis, chez les Hindous primitifs, \emph{Varouna}, et chez les plus anciens Grecs, \emph{Ouranos} ; « c’est physiquement le ciel qui couvre la terre. » – Eckstein, \emph{Recherches historiques sur l’humanité primitive}, p. 1-2.}. Les fiers Arians avaient mis le ciel en république.\par
Cependant ces dieux qui avaient l’honneur de dominer sur des hommes si hautains leur devaient certainement d’être dignes d’hommages. Contrairement à ce qui arriva plus tard dans l’Inde, et tout à fait en accord avec ce qu’on vit dans la Perse, et surtout dans la Grèce, ces dieux furent d’une irréprochable beauté \footnote{Lassen, \emph{Indisch. Alterth.}, t. I, p. 771.}. Le Peuple arian voulut les avoir à son image. Comme il ne connaissait rien de supérieur à lui sur la terre, il prétendit que rien ne fût autrement parfait que lui dans le ciel ; mais il fallait aux êtres surhumains qui conduisaient le monde une prérogative distincte. L’Arian la choisit dans ce qui est encore plus beau que la forme humaine à sa perfection, dans la source de la beauté et qui semble aussi l’être de la vie : il la choisit dans la lumière et dériva le nom des êtres suprêmes de la racine \emph{dou}, qui veut dire \emph{éclairer} ; il leur créa donc une nature lumineuse \footnote{Lassen, \emph{ouvr. cité}, t. I, p. 755. – Un autre étymologiste fait dériver le mot \emph{dou} de \emph{dhâ}, poser, créer. (Windischmann, \emph{Jenaïsche Litteratur-Zeitung}, juillet 1834, cité par Burnouf, \emph{Comment. sur le Yaçna}, t. I, p. 357.)}. L’idée parut bonne à toute la race, et la racine choisie porta partout une majestueuse unité dans les idées religieuses des peuples blancs. Ce fut le \emph{Dévas} des Hindous ; le (mot grec), le (mot grec) des Hellènes ; le \emph{Diewas} des Lithuaniens, le \emph{Duz} gallique \footnote{Schaffarik, \emph{Slawische Alterth.}, t. I, p. 58.} ; le \emph{Dia} des Celtes d’Irlande ; le \emph{Tyr} de l’Edda ; le \emph{Zio} du haut-allemand ; la \emph{Dewana} slave ; la \emph{Diana} latine. Partout enfin où pénétra la race blanche, et où elle domina, se retrouve ce vocable sacré, au moins à l’origine des tribus. Il s’oppose, dans les régions où existent des points de contact avec les éléments noirs, à l’\emph{Al} des aborigènes mélaniens \footnote{Ewald, \emph{Gesch. des Volkes Israël}, t. I, p. 69. En Abyssinie, on ne se sert pas de cette expression. On dit \emph{egzie} et \emph{amlak}, qui signifient simplement \emph{seigneur}, et qui ont probablement fait disparaître le mot primitif par suite d’une idée analogue à celle qui fait substituer aux Juifs le mot d’Adonaï à celui de Jéhovah, lorsqu’ils le rencontrent dans la lecture de la Bible. – Ewald, \emph{Ueber die Saho-Sprache}, dans la \emph{Zeitschrift der d. morgent. Gesellsch.}, t. V, p. 419.} Ce dernier représente la superstition, l’autre la pensée ; l’un est l’œuvre de l’imagination en délire et courant à l’absurde, l’autre sort de la raison. Quand le \emph{Deus} et l’\emph{Al} se sont mêlés, ce qui a eu lieu par malheur trop souvent, il est arrivé, dans la doctrine religieuse, des confusions analogues à celles qui résultaient, pour l’organisation sociale, des mélanges de la race noire avec la blanche. L’erreur a été d’autant plus monstrueuse et dégradante, qu’\emph{Al} l’emportait davantage dans cette union. Au contraire, le \emph{Deus} a-t-il eu le dessus ? L’erreur s’est montrée moins vile, et, dans le charme que lui prêtèrent des arts admirables et une philosophie savante, l’esprit de l’homme, s’il ne s’endormit pas sans danger, le put du moins sans honte. Le \emph{Deus} est donc l’expression et l’objet de la plus haute vénération chez la race ariane. Exceptons-en la famille iranienne pour des causes tout à fait particulières, dont l’exposition viendra en son temps \footnote{Un autre nom, donné par la race ariane à la Divinité, est le mot \emph{Gott}, en gothique \emph{Gouth}, qui se rapporte au grec (mot grec), et au sanscrit \emph{Goûddhah.} Ce mot veut dire \emph{le Caché}. – V. Windischmam, \emph{Fortschritt der Sprachen-Kunde}, p. 20, et Eckstein, \emph{Recherches historiques sur l’humanité primitive}. – Burnouf incline à voir la racine de ce mot dans le sanscrit \foreign{quaddhâta}, l’Incréé. (\emph{Comment. sur le Yaçna}, t. I, p. 554.)}\par
Ce fut à l’époque où les peuples arians touchaient déjà à la Sogdiane que le départ des nations helléniques rendit la confédération moins nombreuse. Les Hellènes se trouvaient en face de la route qui devait les mener à leurs destinées ; s’ils avaient accompagné plus bas la descente des autres tribus, ils n’auraient pas eu l’idée de remonter ensuite vers le nord-ouest. Marchant directement à l’ouest, ils auraient pris le rôle que remplirent plus tard les Iraniens. Ils n’auraient créé ni Sicyone, ni Argos, ni Athènes, ni Sparte, ni Corinthe. Ainsi je conclus qu’ils partirent à ce moment.\par
Je doute que cet événement soit résulté des causes qui avaient décidé l’émigration primitive des populations blanches. Le contre-coup en était déjà épuisé, car si les envahisseurs jaunes avaient poursuivi les fugitifs, on aurait vu tous les peuples blancs, arians, celtes et slaves, pour échapper à leurs atteintes, se précipiter également vers le sud et inonder cette partie du monde. Il n’en fut pas ainsi. À la même époque, à peu près, où les Arians descendaient vers la Sogdiane, les Celtes et les Slaves gravitaient dans le nord-ouest et trouvaient des routes, sinon libres, du moins assez faiblement défendues pour que le passage restât praticable. Il faut donc reconnaître que la pression qui déterminait les Hellènes à gagner vers l’ouest ne venait pas des régions supérieures : elle était causée par les congénères arians.\par
Ces nations, toutes également braves, étaient en froissement continuel. Les consé­quences de cette situation violente amenaient la destruction des villages, le boulever­sement des États et l’obligation pour les peuplades vaincues de subir le joug ou de s’enfuir. Les Hellènes, s’étant trouvés les plus faibles, prirent ce dernier parti, et, faisant leurs adieux à la contrée qu’ils ne pouvaient plus défendre contre des frères turbulents, ils montèrent sur leurs chariots, et, l’arc à la main, s’engagèrent dans les montagnes de l’ouest. Ces montagnes étaient occupées par les Sémites, qui en avaient chassé ou, du moins, asservi les Chamites, auxquels avait plus anciennement appartenu l’honneur d’en dompter les aborigènes noirs. Les Sémites, battus par les Hellènes, ne résistèrent pas à ces vaillants exilés et se renversèrent sur la Mésopotamie, et plus les Hellènes avançaient, poussés par les nations iraniennes, plus ils forçaient de populations sémitiques à se déplacer pour leur donner passage, et plus ils augmen­taient l’inondation de l’ancien monde assyrien par cette race mêlée. Nous avons déjà assisté à ce spectacle. Laissons les émigrants continuer leur voyage. On sait dans quels illustres lieux ce récit les retrouvera.\par
Après cette séparation, deux groupes considérables forment encore la famille ariane, les nations hindoues et les Zoroastriens. Gagnant du terrain et se considérant comme un seul peuple, ces tribus arrivèrent à la contrée du Pendjab. Elles s’y établirent dans les pâturages arrosés par le Sindh, ses cinq affluents et un septième cours d’eau difficile à reconnaître, mais qui est ou la Yamouna ou la Sarasvati \footnote{Lassen, \emph{Zeitschrift der Deutsch. Morgenl. Gesellschaft}, t. II, p. 200.} Ce vaste paysage et ses beautés étaient restés profondément gravés dans la mémoire des Zoroastriens Iraniens longtemps après qu’ils l’avaient quitté pour ne plus le revoir. Le Pendjab était, à leur sens, l’Inde entière : ils n’en avaient pas vu davantage. Leurs connaissances sur ce point dirigèrent celles de toutes les nations occidentales, et le Zend-Avesta, se réglant plus tard sur ce que les ancêtres avaient raconté, donnait à l’Inde la qualification de septuple.\par
Cette région, objet de tant de souvenirs, fut ainsi témoin du nouveau dédoublement de la famille ariane, et les clartés déjà plus vives de l’histoire \footnote{C’est ici que commence véritablement l’existence des peuples hindous. La philologie va les chercher avec raison dans leur berceau ethnique, au delà des montagnes du nord ; mais leurs annales, mal instruites, les déclarent autochtones. Il est à croire que, dans les temps védiques, le brahmanisme n’avait pas encore imité les Chananéens, les Grecs et les peuplades d’Italie, en admettant comme sienne la tradition de la race inférieure qu’il avait subjuguée. – Lassen, \emph{Indisch. Alterth.}, t. I, p. 511.} permettent de démêler assez bien les circonstances du débat qui en fut l’origine. Je vais raconter la plus ancienne des guerres de religion.\par
 Le genre de piété particulier à la race blanche se révèle d’autant mieux dans sa portée raisonnante, qu’on est en situation de le mieux examiner. Après en avoir constaté des lueurs pâles, mais bien reconnaissables, chez les descendants métis des Chamites, après en avoir retrouvé de précieux fragments chez les familles sémitiques, on a vu plus à plein l’antique simplicité des croyances et l’importance souveraine qui leur était attribuée chez les Arians réunis dans leur première station avant l’exode des Hellènes. À ce moment le culte était simple. Il semblerait que tout, dans l’organisation sociale, fût tourné vers le côté pratique et jugé de ce point de vue. Ainsi, de même que le chef de la communauté, le juge du grand village, le viç-pati n’était qu’un magistrat électif entouré, pour tout prestige, du renom que lui donnaient sa bravoure, sa sagesse et le nombre de ses serviteurs et de ses troupeaux ; de même que les guerriers, pères de famille, ne voyaient dans leurs filles que des aides utiles au labeur pastoral, chargées du soin de traire les chamelles, les vaches et les chèvres, et ne leur donnaient pas d’autre nom que celui de leur emploi ; ainsi, encore, s’ils honoraient les nécessités du culte, ils n’imaginaient pas que les fonctions dussent en être remplies par des personnages spéciaux, et chacun était son propre pontife, et se jugeait les mains assez pures, le front assez haut, le cœur assez noble, l’intelligence assez éclairée, pour s’adresser sans intermédiaire à la majesté des dieux immortels \footnote{Lassen, \emph{Indisch. Alterth.}, t. I, p. 795.}.\par
Mais soit que, dans la période qui s’écoula entre le départ des Grecs et l’occupation du Pendjab, la famille ariane, s’étant trouvée en long contact avec les nations abori­gènes, eût déjà perdu de sa pureté et compliqué son essence physique et morale de l’adjonction d’une pensée et d’un sang étrangers ; soit que les modifications survenues ne fussent que le développement naturel du génie progressif des Arians, toujours est-il que les anciennes notions sur la nature du pontificat se modifièrent insensiblement, et qu’un moment vint où les guerriers ne se crurent plus le droit ni la science de vaquer aux fonctions sacerdotales : des prêtres furent institués.\par
Ces nouveaux guides des consciences devinrent sur-le-champ les conseillers des rois et les modérateurs des peuples. On les appelait \emph{purohitas}. La simplicité du culte s’altéra entre leurs mains ; elle se compliqua, et l’art des sacrifices devint une science pleine d’obscurités dangereuses pour les profanes. On redouta dès lors de commettre, dans l’acte de l’adoration, des erreurs de forme qui pouvaient offenser les dieux, et, afin d’éviter ce danger, on ne se risqua plus à agir soi-même : on eut recours au seul purohita. Il est probable qu’à la pratique de la théologie et des fonctions liturgiques cet homme spécial joignit, de bonne heure, des connaissances en médecine et en chirurgie ; qu’il se livra à la composition des hymnes sacrés, et qu’il se rendit triplement vénérable aux yeux des rois, des guerriers, des populations tout entières par les mérites qui éclataient en sa personne au point de vue de la religion, de la morale et de la science \footnote{Lassen, \emph{loc. cit}. Il est ici question de l’époque où furent composés les hymnes les plus anciens des Védas.}.\par
Tandis que le pontife se créait ainsi des fonctions sublimes et bien propres à lui concilier l’admiration et les sympathies, les hommes libres n’étaient pas sans gagner quelque chose à la perte de plusieurs de leurs anciens droits, et, tout ainsi que le purohita, en s’emparant exclusivement d’une partie de l’activité sociale, en savait extraire des merveilles que les générations antérieures n’avaient pas soupçonnées, de même le chef de famille, vacant tout entier aux soins terrestres, se perfectionnait dans les arts matériels de la vie, dans la science du gouvernement, dans celle de la guerre et dans l’aptitude aux conquêtes.\par
L’ambition la plus inquiète n’avait pas le temps de réfléchir à la valeur de ce qu’elle avait cédé, et d’ailleurs les conseils du purohita, non moins que ses secours, lorsque le guerrier était vaincu, ou blessé, ou malade, non moins que ses chants et ses récits, quand il était de loisir, contribuaient à l’impressionner en faveur de l’influence qu’il avait laissé nette, qu’il laissait croître à ses côtés, et à l’étourdir sur les dangers dont, pour l’avenir, elle pouvait menacer sa puissance et sa liberté.\par
D’ailleurs, le purohita n’était pas un être qui pût sembler redoutable. Il vivait isolé auprès des chefs assez riches ou généreux pour entretenir sa vie simple et pacifique. Il ne portait pas les armes ; il n’était pas d’une race ennemie. Sorti de la famille même du viç-pati ou de sa tribu, il était le fils, le frère, le cousin des guerriers \footnote{Lassen, \emph{ouvr. cité}, t. I, p. 812.}. Il communiquait sa science à des disciples qui pouvaient le quitter à leur gré et reprendre l’arc et la flèche. C’était donc insensiblement et par des voies inconnues, même à ceux qui les suivaient, que le brahmanisme jetait ainsi les fondements d’une autorité qui allait devenir exorbitante.\par
Un des premiers pas que fit le sacerdoce dans le maniement direct des affaires temporelles, témoigne d’un grand perfectionnement politique et moral chez ces contemporains d’une époque que les érudits allemands appellent, avec une poétique justesse, la \emph{grise antériorité des temps} \footnote{Die graue Vorzeit.}\footnote{Lassen, \emph{ouvr. cité}, t. I, p. 812. La consécration royale, dont il est si fort question dans le Ramayana, a encore été pratiquée dans les temps modernes. W. v. Schlegel, \emph{Indische Bibliothek}, t. I, p. 430.}. Les viç-pati comprirent qu’il serait bon de ne plus être pour leurs administrés, qui, insensiblement, devenaient leurs sujets, les produits irréguliers de la ruse ou de la violence heureuse. On voulut qu’une consé­cration supérieure à l’élection populaire investît les pasteurs des peuples de droits particuliers au respect, et on imagina de faire dépendre la légitimité de leur caractère d’une espèce de sacre administré par les purohitas . Dès lors l’importance des rois s’accrut sans doute, car ils étaient devenus participants à la nature des choses saintes, même sans avoir encore détrôné un dieu. Mais le pouvoir mondain du sacerdoce fut également fondé, et l’on devine maintenant ce qu’il va devenir entre les mains d’hommes éclairés, pacifiques, d’une redoutable énergie dans le bien, et qui, sachant que, pour une nation dévouée, corps et âme, à l’admiration de la bravoure, aucun prétexte, si sacré fût-il, ne pouvait couvrir le soupçon d’être lâche, commençaient déjà à pratiquer des doctrines austères d’abstinences intrépides et de renoncements obstinés. Cet esprit de pénitence devait aboutir, un jour, à des mutilations effrénées, à des supplices absurdes, également révoltants pour le cœur et pour la raison. Les purohitas n’en étaient pas là encore. Prêtres d’une nation blanche, ils ne songeaient même pas à de pareilles énormités.\par
La puissance sacerdotale était désormais assise sur des bases solides. Le pouvoir séculier, fier d’en obtenir sa consécration et de s’appuyer sur elle, servait volontiers ses développements. Bientôt il put s’apercevoir que ce qui se demande se refuse aussi. Tous les rois ne furent pas également bien reçus des maîtres des sacrifices, et il suffit de quelques rencontres où la fermeté de ceux-ci se trouva d’accord avec les sentiments des peuples, il suffit que certains d’entre eux périssent martyrs de leur résistance aux vœux d’un usurpateur, pour que l’opinion publique, frappée de reconnaissance et d’admiration, fit aux purohitas réunis un pont vers les plus hautes entreprises.\par
Ils acceptèrent le rôle éminent qui leur était attribué. Cependant je ne crois ni à la prédominance des calculs égoïstes dans la politique d’une classe entière, ni aux grands résultats amenés par de petites causes. Quand une révolution durable se produit au sein des sociétés, c’est que les passions des triomphateurs ont pour rebondir un sol plus ferme que des intérêts personnels, sans quoi elles rasent la terre et ne montent à rien. Le fait d’où le sacerdoce arian s’avisa de faire jaillir ses destinées, loin d’être misérable ou ridicule, devait, au contraire, lui gagner les sympathies intimes du génie de la race, et l’observation qu’en firent les prêtres de cette époque antique accuse, chez eux, une rare aptitude à la science du gouvernement, en même temps qu’un esprit subtil, savant, combinateur et logique jusqu’à la rage.\par
Voici ce dont s’aperçurent ces philosophes, et ce qu’ensuite imagina leur pré­voyance. Ils considérèrent que les nations arianes se trouvaient entourées de peuplades noires dont les multitudes s’étendaient à tous les coins de l’horizon et dépassaient de beaucoup par le nombre les tribus de race blanche établies sur le territoire des Sept-Fleuves, et déjà descendues jusqu’à l’embouchure de l’Indus. Ils virent, en outre, qu’au milieu des Arians vivaient, soumises et paisibles, d’autres populations aborigènes qui ne laissaient pas que de former encore une masse considérable, et qui avaient déjà commencé à se mêler à certaines familles, probablement les plus pauvres, les moins illustres, les moins fières de la nation conquérante. Ils remarquèrent sans peine combien les mulâtres étaient inférieurs en beauté, en intelligence, en courage à leurs parents blancs ; et surtout ils eurent à réfléchir aux conséquences que pouvait amener, pour la domination des Arians, une influence exercée par les individualités métisses sur les populations noires soumises ou indépendantes. Peut-être avaient-ils sous les yeux l’expérience de quelques accessions fortuites de sang mêlé à la dignité royale.\par
Guidés par le désir de conserver le souverain pouvoir à la race blanche, ils imaginè­rent un état social hiérarchisé suivant le degré d’élévation d’intelligence. Ils prétendirent confier aux plus sages et aux plus habiles la conduite suprême du gouvernement. À ceux dont l’esprit était moins élevé, mais le bras vigoureux, le cœur avide d’émotions guerrières, l’imagination sensible aux excitations de l’honneur, ils remirent le soin de défendre la chose publique. Aux hommes d’humeur douce, curieux de travaux paisibles, peu disposés aux fatigues de la guerre, ils se piquèrent de trouver un emploi convenable en les conviant à nourrir l’État par l’agriculture, à l’enrichir par le commerce et l’industrie. Puis, du grand nombre de ceux dont le cerveau n’était éclairé que de lueurs incomplètes, de tous ceux qui n’avaient pas l’âme prête à subir, sans faiblesse, le choc du danger, des gens trop pauvres pour vivre libres, ils composèrent un amalgame sur lequel ils jetèrent le niveau d’une égale infériorité, et décidèrent que cette classe humble gagnerait sa subsistance en remplissant ces fonctions pénibles ou même humiliantes qui sont cependant nécessaires dans les sociétés établies.\par
Le problème avait trouvé sa solution idéale, et personne ne peut refuser son approbation à un corps social ainsi organisé qu’il est gouverné par la raison et servi par l’inintelligence. La grande difficulté, c’est de faire passer un projet abstrait de cette espèce dans le moule d’une réalisation pratique. Tous les théoriciens du monde occidental y ont échoué : les purohitas crurent avoir trouvé le sûr moyen d’y réussir.\par
Partant de cette observation établie, pour eux, sur des preuves irréfragables, que toute supériorité était du côté des Arians, toute faiblesse, toute incapacité du côté des noirs, ils admirent, comme conséquence logique, que la proportion de valeur intrinsè­que chez tous les hommes était en raison directe de la pureté du sang, et ils fondèrent leurs catégories sur ce principe.\par
Ces catégories, ils les appelèrent \emph{varna}, qui signifiait \emph{couleur}, et qui, depuis lors, a pris la signification de \emph{caste} \footnote{Lassen, \emph{ouvr. cité}, t. I, p. 514. En kawi, \emph{varna} a gardé son sens primitif et n’a pas acquis le sens dérivé. – Voir W. v. Humboldt, \emph{Ueber die Kawi-Sprache}, t. I, p. 83.}.\par
Pour former la première \emph{caste}, ils réunirent les familles des purohitas en qui éclatait quelque mérite, telles que celles des Gautama, des Bhrigou, des Atri \footnote{Lassen, \emph{ouvr. cité}, p. 804.}, célèbres par leurs chants liturgiques, transmis héréditairement comme une propriété précieuse. Ils supposèrent que le sang de ces familles recommandables était plus arian, plus pur que celui de toutes les autres.\par
À cette classe, à cette \emph{varna}, à cette \emph{couleur} blanche par excellence, ils attribuèrent non pas d’abord le droit de gouverner, résultat définitif qui ne pouvait être que l’œuvre du temps, mais du moins le principe de ce droit et tout ce qui pouvait y conduire, c’est-à-dire le monopole des fonctions sacerdotales, la consécration royale qu’ils possédaient déjà, la propriété des chants religieux, le pouvoir de les composer, de les interpréter et d’en communiquer la science ; enfin ils se déclarèrent, eux-mêmes, per­sonnages sacrés, inviolables ; ils se refusèrent aux emplois militaires, s’entourèrent d’un loisir nécessaire, et se vouèrent à la méditation, à l’étude, à toutes les sciences de l’esprit, ce qui n’excluait ni l’aptitude ni la science politique \footnote{Lassen, \emph{Indisch. Alterthüm.}, t. I, p. 804 et pass. – Burnouf, \emph{Introduction à l’hist. du bouddhisme indien}, t. I, p. 141. Le trait essentiel des brahmanes est de pouvoir lire les mantrâs. – Lassen, \emph{ouvr. cité}, p. 806. L’aumône, jadis facultative, est aujourd’hui obligatoire à l’égard des brahmanes. Le bien qui est fait à un homme de caste ordinaire acquiert un mérite simple ; à un membre de la caste sacerdotale, un mérite double ; à un étudiant des Védas, le mérite se multiplie par cent mille, et si c’est d’un ascète qu’il s’agit, alors il devient incommensurable.}.\par
Immédiatement au-dessous d’eux, ils placèrent la catégorie des rois alors existants avec leurs familles. En exclure aucun, c’eût été donner un démenti à la valeur de la consécration, et, en même temps, créer à l’organisation naissante des hostilités trop redoutables. À côté des rois, ils placèrent les guerriers les plus éminents, tous les hommes distingués par leur influence et leurs richesses, et ils supposèrent, plus ou moins justement, que cette classe, cette \emph{varna}, cette \emph{couleur}, était déjà moins franchement blanche que la leur, avait déjà contracté un certain mélange avec le sang aborigène, ou bien que, égale en pureté, tout aussi fidèle à la souche ariane, elle ne méritait néanmoins que le second rang, par la supériorité de la vocation intellectuelle et religieuse sur la vigueur physique. C’était une race grande, noble, illustre, que celle qui pouvait accepter une telle doctrine. Aux membres de la caste militaire, les purohitas donnèrent le nom de \emph{kschattryas ou hommes forts.} Ils leur firent un devoir religieux de l’exercice des armes, de la science stratégique, et, tout en leur concédant le gouver­nement des peuples, sous la réserve de la consécration religieuse, ils s’appuyèrent sur le sentiment public, imbu des doctrines libres de la race, pour leur refuser la puissance absolue \footnote{Rien d’admirable comme les prescriptions que le \foreign{Manava-Dharma-Sastra} (traduction de Haughton, Londres, 1825, in-4°, t. II) adresse à la caste militaire et compile probablement de règlements plus anciens. Je ne puis résister au plaisir de traduire cette page, animée du plus pur esprit chevaleresque. Chap. XII, § 88 : « Ne jamais quitter le combat, protéger le « peuple et honorer les prêtres, tel est le suprême devoir des rois, celui qui assure, « félicité. » § 89 : « Ces maîtres du monde, qui, ardents à s’entre-défaire, déploient leur « vigueur dans la bataille sans jamais tourner le « visage, montent, après leur mort, « directement au ciel. » § 90 : « Que nul homme, en combattant, ne frappe son ennemi « avec des armes pointues emmanchées de bois, ni avec des flèches méchamment « barbelées, ni avec des traits empoisonnés, ni avec des dards de feu. » § 91 : « Que, « monté sur un char ou chevauchant un coursier, il n’attaque pas un ennemi à pied, ni un « homme efféminé, ni celui qui demande la vie à mains jointes, ni celui dont la chevelure « dénouée couvre la vue, ni celui qui, épuisé de fatigue, s’est assis sur la terre, ni celui qui « dit : je suis ton captif. » § 92 : « Ni celui qui dort, ni celui qui a perdu sa cotte de mailles, « ni celui qui est nu ; ni celui qui est désarmé, ni celui qui est spectateur et non acteur dans « le combat, ni celui qui est aux prises avec un autre. » § 93 : « Ayant toujours présent à « l’esprit le devoir des Arians, des hommes honorables, qu’il ne tue jamais quelqu’un qui a « rompu son arme, ni celui qui pleure pour un chagrin particulier, ni celui qui a été blessé « grièvement, ni celui qui a peur, ni celui qui tourne le dos. » § 98 : « Telle est la loi antique « et irréprochable des guerriers. De cette loi, nul roi ne doit jamais se départir, quand il « attaque ses ennemis dans la bataille. »}.\par
Ils déclarèrent que chaque \emph{varna} conférait à ses membres des privilèges inaliéna­bles, devant lesquels la volonté royale expirait. Il était défendu au souverain d’empiéter sur les droits des prêtres. Il ne lui était pas moins interdit d’attenter à ceux des kschattryas ou des castes inférieures \footnote{Manava-Dbarma-Sastra, chap. VII, § 123 : « Since the servants of the king, whom he has « appointed guardians of districts, are generally knaves, who seize what belongs to other « men, from such knaves let him defend his people. » Cet article fut inspiré, selon toute vraisemblance, par la féodalité des kschattryas.}. Le monarque fut entouré d’un certain nombre de ministres ou de conseillers, sans le concours desquels il ne pouvait agir et qui appartenaient aussi bien à la classe des purohitas qu’à celle des guerriers \footnote{Lassen, \emph{ouvr. cité}, t. I, p. 805.}.\par
 Les constituants firent plus. Au nom des lois religieuses, ils prescrivirent aux rois une certaine conduite dans la vie intérieure. Ils réglèrent jusqu’à la nourriture et proscri­virent, de la manière la plus énergique, et sous des peines temporelles et spirituelles, toute infraction à leurs mandements. Leur chef-d’œuvre, à mon avis, à l’encontre des kschattryas et de la caste qui va suivre, est d’avoir su se départir de la rigueur des classifications pour ne pas monopoliser absolument les choses de l’intelligence dans le sein de leur confrérie. Ils comprirent, sans doute, que l’instruction ne peut être refusée à qui est capable de l’acquérir, de même qu’on la permet sans résultat aux intelligences mal créées pour la recevoir ; puis, que si le savoir est une force et exerce un prestige, c’est à la condition d’avoir des spectateurs qui se peuvent faire, par eux-mêmes, une idée juste de son mérite, et qui, pour être en état d’en apprécier la valeur, doivent au moins avoir approché les lèvres de sa coupe.\par
Loin donc de défendre l’instruction aux kschattryas, les purohitas la leur recommandèrent, leur permirent la lecture des livres sacrés, les engagèrent à se les faire expliquer, et les virent avec complaisance s’adonner aux connaissances laïques, telles que la poésie, l’histoire et l’astronomie. Ils formaient ainsi, autour d’eux, une classe militaire intelligente autant que brave, et qui, si elle pouvait un jour trouver, dans l’éveil de ses idées, des excitations à combattre les progrès du sacerdoce, n’y rencontrait pas moins de motifs d’en être séduite, d’y sourire et de les favoriser au nom de cette sympathie instinctive que l’esprit inspire à l’esprit et le talent au talent. Toutefois, il ne faut pas se le dissimuler : quelles que fussent les dispositions intimes des kschattryas, l’intérêt général de leur caste et la nature des choses en faisaient pour les novateurs religieux une terrible pierre d’achoppement, et un danger devait tôt ou tard se montrer de ce côté-là.\par
Il n’en était pas de même de la \emph{varna} qui venait après la caste guerrière. Ce fut celle des \emph{vayçias}, supposés moins blancs que les deux catégories sociales supérieures, et qui, probablement aussi, étaient moins riches et moins influents dans la société. Toutefois, leur parenté avec les deux hautes castes étant encore évidente et indiscutable, le nouveau système les considéra comme des hommes d’élite, des hommes deux fois nés (\emph{dvidja}), expression consacrée pour représenter l’excellence de la race vis-à-vis des populations aborigènes \footnote{Lassen, \emph{ouvr. cité}, t. I, p. 818.}, et on en forma le peuple, le gros de la nation proprement dite, au-dessus duquel étaient les prêtres et les soldats, et ce fut pour cette raison que le nom d’Arians, abandonné par les kschattryas, comme par les purohitas, plus fiers, les uns de leur titre de \emph{forts}, les autres de la qualification nouvellement prise de \emph{brahmanes}, resta le partage de la troisième caste.\par
La loi de Manou, postérieure, du reste, dans sa forme actuelle, à l’époque en question, établit, d’après des autorités plus anciennes qu’elle-même, le cercle d’action où devait s’écouler l’existence des vayçias. On leur confia le soin du bétail. Le raffine­ment déjà considérable des mœurs ne permettait plus aux hautes classes de s’en occuper, comme avaient fait les ancêtres. Les vayçias firent le négoce, prêtèrent de l’argent à intérêt et cultivèrent la terre \footnote{Id. \emph{ibid.}, p. 817.}. Appelés à concentrer ainsi dans leurs mains les plus grandes richesses, on leur commanda l’aumône et les sacrifices aux dieux. À eux aussi on permit de lire ou de se faire lire les Védas \footnote{Manava-Dharma-Sastra, chap. X, § 1 : « Let the three twiceborn classes, remaining firm in « their several duties, carefully read the Veda; but a brahman must explain it to them, not a « man of the other two classes : this is an established rule. » – Chap. X, § 79 : « The means « of subsistence peculiar to... the vaisya (are), merchandize, attending on cattle and « agriculture; but, with a view to the next life ; the duties... are almsgiving, reading, « sacrificing. »}, et, afin d’assurer à leur caractère pacifique la tranquille jouissance des humbles, prosaïques mais fructueux avantages qui leur étaient concédés, il fut sévèrement interdit aux brahmanes, comme aux kschattryas d’empiéter sur leurs attributions, de se mêler à leurs travaux et d’obtenir soit un épi de blé, soit un objet fabriqué, autrement que par leur intermédiaire. Ainsi, dès l’antiquité la plus haute, la civilisation ariane de l’Inde asseyait ses travaux sur l’existence d’une nombreuse bourgeoisie, fortement organisée et défendue, dans l’exercice de droits consi­dérables, par toute la puissance des prescriptions religieuses \footnote{L’importance de cette caste et l’influence extralégale qu’elle était capable d’exercer n’échappèrent pas du tout aux législateurs de l’Inde. Je lis dans le \emph{Manava-Dharma-Sastra}, ch. VIII, § 418 : « With vigilant care should the king exert himself in compelling merchant and mechanicks « to perform their respective duties ; \emph{for, when such men swerve from their duty, they throw this} « \emph{world in confusion.} »}. On remarquera encore que, non moins que les kschattryas, cette classe était admise aux études intellectuelles, et que ses habitudes, plus paisibles, plus casanières que celles des guerriers, tendaient à l’en faire profiter davantage.\par
Avec ces trois hautes castes, la société hindoue, dans son idéal, était complète. En dehors de leur cercle, plus d’Arians, plus d’hommes deux fois nés. Cependant, il fallait tenir compte des aborigènes, qui, soumis depuis plus ou moins longtemps et peut-être un peu apparentés au sang des vainqueurs, vivaient obscurément au bas de l’échelle sociale. On ne pouvait repousser absolument ces hommes attachés à leurs vainqueurs et ne recevant que d’eux leur subsistance, sans se jeter, avec une barbare imprudence, dans des périls inutiles. D’ailleurs, par ce qui se passa ensuite, il est fort probable que les brahmanes avaient déjà senti combien il serait contraire à leurs véritables intérêts de rompre avec ces multitudes noires qui, si elles ne leur rendaient pas les honneurs délicats et raisonnés des autres castes, les entouraient d’une admiration plus aveugle et les servaient avec un fanatisme plus dévoué. L’esprit mélanien se retrouvait là bien entier. Le brahmane, prêtre pour les kschattryas et les vayçias, était dieu pour la foule noire. On ne se brouille pas de gaieté de cœur avec de si chauds amis, et surtout quand il n’est pas besoin de faire beaucoup pour se les conserver.\par
Les brahmanes composèrent une quatrième caste de toute cette population de manœuvres, d’ouvriers, de paysans et de vagabonds. Ce fut celle des \emph{çoudras} ou des \emph{dazas}, des \emph{serviteurs}, qui reçut le monopole de tous les emplois serviles. Il fut rigoureusement défendu de les maltraiter, et on les soumit à un état de tutelle éternelle, mais avec l’obligation, pour les hautes classes, de les régir doucement et de les garder de la famine et des autres effets de la misère. La lecture des livres sacrés leur fut interdite ; ils ne furent pas considérés comme purs, et rien de plus juste, car ils n’étaient pas Arians \footnote{Lassen, \emph{Indisch., Alterth.}, t. I, p. 817 et pass.}.\par
 Après avoir ainsi distribué leurs catégories, les inventeurs du système des castes en fondèrent la perpétuité, en décrétant que chaque situation serait héréditaire, qu’on ne ferait partie d’une varna qu’à la condition d’être né de père et de mère y appartenant l’un et l’autre \footnote{Burnouf, \emph{Introduct. à l’histoire du bouddh. indien}, t. I, p. 155. –\emph{ Manava-Dharma-Sastra}, chap. X, § 5 : « In all classes they, and they only, who are born, in a direct order, of wives « equal in classes and virgins at the time of marriage, are to be considered as the same in « class with their father. »}. Ce ne fut pas encore assez. De même que les rois ne pouvaient gouverner sans avoir obtenu la consécration brahmanique, de même nul ne fut admis à la jouissance des privilèges de sa caste avant d’avoir accompli, avec l’assentiment sacerdotal, les cérémonies particulières de l’accession \footnote{Manava-Dharma-Sastra, chap. II, § 26: « With auspicious acts prescribed by the veda, « must ceremonies over conception and so forth, be duty performed, which purify the « bodies of the three classes in this life, and qualify them for the next. » Ainsi ce n’était pas seulement pour le bonheur de cette vie qu’il était nécessaire de se pourvoir de la consécration de sa caste, c’était encore pour assurer le sort ultérieur dans l’autre. Puis les cérémonies commençaient dès le moment présumé de la conception. C’était, à proprement parler, celles qui constituaient l’Hindou, indépendamment de l’idée de caste. Cette seconde condition était remplie d’une manière plus complète quelques années après. Chap. II, p. 37 : « Should a brahman, or his father for him, be « desirous of his advancement in sacred « knowledge; a cshatriya, of extending his power ; or a vaisya of engaging in mercantile « business ; the investiture may be made in the fifth, sixth or eighth year respectively. »}.\par
Les gens oublieux de ces formalités obligées étaient exclus de la société hindoue \footnote{Manava-Dharma-Sastra, ch II, § 38 : « The ceremony of the investiture hallowed by the « gayatri must not be delayed, in the case of a priest, beyond the sixteenth year, not in that « of a soldier, beyond the twenty second ; nor in that of a merchant, beyond the twenty « fourth. » § 39 : « After that, all youths of these three classes, who have not been invested « at the proper time, become vratyas, or outcasts, degraded from the gayatri, and contemned « by the virtuous.}. Impurs, fussent-ils nés brahmanes de père et de mère, on les appelait \emph{vratyas} \footnote{Lassen, \emph{Indische Alterth.}, t. I, p. 821. \emph{Vrâta} signifie une horde vivant de pillage et formée de gens de toute origine.} :\emph{ brigands, pillards, assassins}, et il est bien probable que, pour vivre, ces rebuts de la loi étaient souvent contraints de s’armer contre, elle. Ils formèrent la base de tribus nombreuses qui devinrent étrangères à la nationalité hindoue.\par
Telle est la classification sur laquelle les successeurs des purohitas imaginèrent de construire leur état social. Avant d’en juger les conséquences et le succès, avant, surtout, de nous arrêter devant la subtilité, les ressources inouïes, l’énergie soutenue, l’irrésistible patience employées par les brahmanes pour défendre leur ouvrage, il est indispensable de l’envisager à un point de vue général.\par
Au point de vue ethnographique, le système avait pour premier et grand tort de reposer sur une fiction. Les brahmanes n’étaient pas et ne pouvaient être les plus authentiques Arians, à l’exclusion de telles familles de kschattryas et de vayçias dont la pureté n’était peut-être pas contestable, mais qui, par la position qu’elles occupaient dans la société, la mesure de leurs ressources, se voyaient forcément désignées pour tenir tel rang et non tel autre. Je suppose, d’autre part, que les illustres races des Gautama et des Atri aient compté dans leur arbre généalogique plusieurs aïeules issues de pères guerriers à une époque où ces alliances étaient légales, et que, de plus, ces aïeules aient eu, dans leur sang, une quantité plus ou moins grande d’alliage mélanien : voilà les Gautama, voilà les Atri reconnus métis. En sont-ils moins possesseurs des hymnes sacrés composés par leurs ancêtres ? Ne remplissent-ils pas auprès de rois puissants les fonctions de sacerdoces révérés ? Puissants ! ne le sont-ils pas eux-mêmes ? Ils comptent parmi les coryphées du nouveau parti, et il ne faut pas s’attendre à ce que, faisant un retour sur leur propre extraction, dont peut-être, d’ailleurs, ils ignorent le vice, ils s’excluent volontairement de la caste suprême.\par
Toutefois, s’il s’agissait de n’examiner les choses qu’à travers les notions hindoues, on pourrait répondre qu’aussitôt que, par des mariages exclusifs, les races spéciales des brahmanes, des kschattryas, des vayçias eurent été fixées, la gradation, d’abord supposée, quant à la pureté relative, devint bientôt réelle ; que les brahmanes se trouvèrent être plus blancs que les kschattryas, ceux-ci que les hommes de la troisième classe qui, à leur tour, dominèrent, en ce point, ceux de la quatrième, presque entièrement noirs. En admettant cette façon de raisonner, il n’en est pas moins vrai que les brahmanes eux-mêmes n’étaient plus des blancs parfaits et sans mélange. En face du reste de l’espèce, vis-à-vis des Celtes, vis-à-vis des Slaves, et plus encore des autres membres de la famille ariane, les Iraniens et les Sarmates, ils avaient adopté, dès lors, une nationalité spéciale et étaient devenus distincts de la souche commune. Supérieurs en illustration au reste des tribus blanches contemporaines, ils étaient inférieurs au type primitif et n’en possédaient plus l’énergie ancienne.\par
Plusieurs des facultés de la race noire avaient commencé à déteindre sur eux. On ne leur reconnaît plus cette rectitude de jugement, cette froideur de raison, patrimoine de l’espèce blanche, dans sa pureté, et l’on s’aperçoit, à la grandeur même des plans de leur société, que l’imagination tenait désormais une grande place dans leurs calculs et exerçait une influence dominante sur la combinaison de leurs idées. Comme élan d’intelligence, ouverture de vue, envergure de génie, ils avaient gagné. Ils avaient gagné par l’adoucissement de leurs premiers instincts, devenus moins rêches et plus souples. Mais en tant que métis, je ne leur trouve plus qu’un diminutif des vertus souveraines, et si les brahmanes se présentent ainsi déchus, à plus forte raison les kschattryas et, à un degré plus grand encore, les vayçias étaient ce qu’on peut appeler dégénérés des mérites fondamentaux. Nous avons observé en Égypte que le premier effet, et le plus général, de l’immixtion du sang noir est d’efféminer le naturel. Cette mollesse ne fait pas des êtres dénués de courage ; cependant elle altère et passionne la vigueur calme, et on pourrait dire compacte, apanage du plus excellent des types. Les Chamites ne tombent sous l’observation qu’à un moment où ils ont trop perdu les caractères spéciaux de leur origine paternelle, et l’on ne saurait baser sur eux une démonstration exacte. Néanmoins, dans la langueur mêlée de férocité où nous les avons vus plongés, on reconnaît un point où sont arrivées aujourd’hui les classes ethniquement correspon­dantes de la nation hindoue. On est donc en droit de supposer que, dans leurs commencements, les Chamites ont eu aussi une période comparable à celle de la caste brahmanique à ses débuts. Pour les Sémites, dont on découvre mieux le principe, un tel rapprochement ne laisse rien à désirer. Ainsi toutes les expériences envisagées jusqu’ici donnent ce résultat identique : le mélange avec l’espèce noire, lorsqu’il est léger, développe l’intelligence chez la race blanche, en tant qu’il la tourne vers l’imagination, la rend plus artiste, lui prête des ailes plus vastes ; en même temps, il désarme sa raison, diminue l’intensité de ses facultés pratiques, porte un coup irrémédiable à son activité et à sa force physique, et enlève aussi, presque toujours, au groupe issu de cet hymen le pouvoir et le droit, sinon de briller beaucoup plus que l’espèce blanche et de penser plus profondément, du moins de lutter avec elle de patience, de fermeté et de sagacité. Je conclus que les brahmanes, s’étant engagés, avant la formation des castes, dans quelques mélanges mélaniens, étaient ainsi préparés pour la défaite, quand viendrait le jour de lutter avec des races demeurées plus blanches.\par
Ces réserves faites, si l’on consent à ne plus envisager les nations hindoues qu’en elles-mêmes, l’admiration pour les législateurs doit être sans réserve. En face des castes normales et des populations décastées qui les entourent, ils paraissent vraiment sublimes. Il ne sera que trop facile de reconnaître plus tard combien, avec le cours des temps et la perversion inévitable des types sans cesse grandissant malgré tous les efforts, les brahmanes ont dégénéré ; mais jamais les voyageurs, les administrateurs anglais, les érudits qui ont consacré leurs veilles à l’étude de la grande péninsule asiatique, n’ont hésité à reconnaître que, au sein de la société hindoue, la caste des brahmanes conserve une supériorité imperturbable sur tout ce qui vit autour d’elle. Aujourd’hui, souillée par les alliages qui faisaient tant d’horreur à ses premiers pères, elle montre cependant, au milieu de son peuple, un degré de pureté physique dont rien n’approche. C’est chez elle que l’on retrouve encore le goût de l’étude, la vénération des monuments écrits, la science de la langue sacrée ; et le mérite de ses membres comme théologiens et grammairiens est assez véritable pour que les Colebrooke, les Wilson et d’autres indianistes justement admirés aient à se féliciter d’avoir recouru à leurs lumières. Le gouvernement britannique leur a même confié une partie importante de l’enseignement au collège de Fort-William. Ce reflet de l’ancienne gloire est bien terne, sans doute. Ce n’est qu’un écho, et cet écho va de plus en plus s’affaiblissant, à mesure qu’augmente la désorganisation sociale dans l’Inde. Pourtant le système hiérarchique inventé par les antiques purohitas est resté debout tout entier. On peut l’étudier bien complet dans toutes ses parties, et pour être amené à lui rendre, sans nul regret, l’honneur qui lui est dû, il suffit de calculer à peu près depuis combien de temps il dure.\par
L’ère de Kali remonte à l’an 3102 avant J.-C., et on ne la fait commencer pourtant qu’après les grandes guerres héroïques des Kouravas et des Pandavas \footnote{Lassen, \emph{Indisch. Alterth.}, t. I, p. 507 et pass.}. Or, à cette époque, si le brahmanisme n’avait pas encore atteint tous ses développements, il existait dans ses points principaux. Le plan des castes était, sinon rigoureusement fermé, du moins tracé, et la période des purohitas dépassée depuis longtemps. Malheureusement le chiffre de 3102 ans a quelque chose de si énorme \footnote{Si l’on admet un jour, couramment, les dates extraordinaires de l’histoire égyptienne, il faudra bien s’accommoder de calculs plus lointains encore pour les faits brahmaniques.} que je ne veux pas trop presser la conviction sur ce point, et je me tourne d’un autre côté.\par
 L’ère kachemyrienne commence un peu plus modestement, 2,448 ans avant J.-C. On la dit également postérieure à la grande guerre héroïque ; par conséquent, elle laisse un intervalle de 654 ans entre son début et l’ère de Kali.\par
Tout incertaines que soient ces deux dates, si l’on en veut chercher de plus récentes, on n’en trouve pas, et à mesure que l’on avance, la clarté historique, devenant plus intense, ne permet pas de douter qu’on ne s’éloigne de l’objet cherché. Ainsi, après une lacune, à la vérité assez longue, au XIV\textsuperscript{e} siècle avant J.-C., on trouve le brahmanisme parfaitement assis et organisé, les écrits liturgiques fixés et le calendrier védique établi ; il est donc impossible de descendre plus bas.\par
Nous avons trouvé l’ère de Kali trop exagérée : n’en parlons pas. Diminuons le nombre des années qu’elle réclame et rabattons-nous à l’ère kachemyrienne. On ne peut descendre davantage sans rendre toute chronologie égyptienne impossible. À mon sens même, c’est beaucoup trop concéder au doute. Mais, pour ce dont il est question ici, je m’en contente. Ne considérons même pas que le brahmanisme existait visiblement longtemps avant cette époque et concluons que, de l’an 2448 avant J.-C. à l’an du Seigneur 1852, il s’est écoulé 4300 ans, que l’organisation brahmanique vit toujours, qu’elle est aujourd’hui dans un état comparable à la situation des Égyptiens sous les Ptolémées du III\textsuperscript{e} siècle avant notre ère, et à celle de la première civilisation assyrienne à différentes époques, entre autres au VII\textsuperscript{e} siècle. Ainsi, en se montrant généreux envers la civilisation égyptienne, en lui accordant, ce que je ne fais pas pour celle des brahmanes, toute la période antérieure à la migration et toute celle de ses débuts avant Ménès, elle aura duré depuis l’an 2448 jusqu’à l’an 300 avant J.-C., c’est-à-dire 2148 ans. Quant à la civilisation assyrienne, en reculant son point de départ aussi haut que l’on voudra, comme on ne peut le faire antérieur de beaucoup de siècles à l’ère kachemyrienne, il s’ensuit qu’il n’en faut pas même parler : elle s’arrête trop loin du but.\par
L’organisation égyptienne reste le seul terme de comparaison, et elle est en arrière, sur le type d’où elle a tiré sa vie, de 2152 ans. Je n’ai pas besoin de confesser tout ce qu’il y a d’arbitraire dans ce calcul : on s’en aperçoit de reste. Seulement, il ne faut pas oublier que cet arbitraire a pour effet de rabaisser d’une manière énorme le chiffre des années de l’existence brahmanique ; que j’y suppose bien bénévolement l’organisation des castes contemporaines de l’ère de Kachemyr ; qu’avec une facilité non moins exagérée j’admets, contre toute vraisemblance, un synchronisme parfait entre les premiers développements du brahmanisme et la naissance de la civilisation dans la vallée du Nil, et enfin que je reporte au III\textsuperscript{e} siècle avant J.-C. (époque où les véritables Égyptiens ne comptaient, pour ainsi dire, plus) la comparaison que j’en fais avec les brahmanes actuels, ce qui procure peu d’honneur à ces derniers. J’ai cru, toutefois, devoir cet hommage au siècle où naquit Manéthon. Ainsi, il est bien entendu qu’en ne faisant vivre la société hindoue que 2500 ans de plus que celle d’Assyrie, et 2000 ans de plus que celle d’Égypte, je la calomnie, je rabaisse sa longévité d’un bon nombre de siècles. Toutefois je persiste, parce que les chiffres incomplets qui me sont là entre les mains me permettent encore d’établir le raisonnement qui suit :\par
 Trois sociétés étant données, elles se perpétuent dans la mesure où se maintient le principe blanc qui fait également leur base.\par
La société assyrienne, incessamment renouvelée au moyen d’affluents médiocre­ment purs, a déployé une extrême intensité de vie, a témoigné d’une activité en quelque sorte convulsive. Puis, assaillie par trop d’éléments mélaniens et livrée à des luttes ethniques perpétuelles, la lumière qu’elle projetait a été perpétuellement syncopée, a sans cesse changé de direction, de formes et de couleurs, jusqu’au jour où la race ariane-médique est venue lui donner une nouvelle nature. Voilà le sort d’une société très mélangée : c’est d’abord l’agitation extrême, ensuite la torpeur morbide, enfin la mort.\par
L’Égypte offre un terme moyen, parce que l’organisation de ce pays se tenait dans les demi-mesures. Le système des castes n’y exerçait qu’une influence ethnique très restreinte, car il était incomplètement appliqué, les alliances hétérogènes étant restées possibles. Probablement, le noyau arian s’était senti trop faible pour commander absolument et il s’était rabattu à des transactions avec l’espèce noire. Il reçut le juste loyer de cette modération. Plus vivace que l’organisation assyrienne, surtout plus logique, plus compact, moins fragile et moins variable, il eut une existence effacée, mêlée à moins d’affaires, moins influente sur l’histoire générale, mais plus honorable et plus longue de beaucoup.\par
Voici maintenant le troisième terme de l’observation : c’est l’Inde. Point de compro­mis avoué avec la race étrangère, une pureté supérieure ; les brahmanes en jouissent d’abord, les kschattryas ensuite. Les vayçias et même les çoudras conservent la nationalité première d’une manière relative. Chaque caste équilibre, vis-à-vis de l’autre, sa valeur ethnique particulière. Les degrés se consolident et se maintiennent. La société élargit ses bases, et, pareille aux végétaux de ce climat torride, pousse, de toutes parts, la plus luxuriante végétation. Quand la science européenne ne connaissait que la lisière du monde oriental, son admiration pour la civilisation antique faisait des Phéniciens et des hommes de l’Égypte et de l’Assyrie autant de personnages d’une nature titanique. Elle leur attribuait la possession de toutes les gloires du passé. En considérant les pyramides, on s’étonnait qu’il eût pu exister des créatures capables de si vastes travaux. Mais depuis que nos pas se sont risqués plus loin et que, sur les rives du Gange, nous voyons ce que l’Inde a été dans les temps antiques, pendant des séries infinies de siècles, notre enthousiasme se déplace, passe le Nil, passe l’Euphrate, et va se prendre aux merveilles accomplies entre l’Indus et le cours inférieur du Brahmapoutra. C’est là que le génie humain a vraiment créé, dans tous les genres, des prodiges qui étonnent l’esprit. C’est là que la philosophie et la poésie ont leur apogée, et que la vigoureuse et intelligente bourgeoisie des vayçias a longtemps attiré et absorbé tout ce que le monde ancien possédait de richesses en or, en argent, en matières précieuses. Le résultat général de l’organisation brahmanique fut supérieur encore aux détails de l’œuvre. Il en sortit une société presque immortelle par rapport à la durée de toutes les autres. Elle avait deux périls à redouter, et seulement deux : l’attaque d’une nation plus purement blanche qu’elle-même, la difficulté de maintenir ses lois contre les mélanges ethniques.\par
Le premier péril a éclaté plusieurs fois, et jusqu’à présent, si l’étranger s’est trouvé constamment assez fort pour subjuguer la société hindoue, il s’est, non moins cons­tamment, reconnu impuissant à la dissoudre. Aussitôt que la cause de sa supériorité momentanée a cessé, c’est-à-dire qu’il a laissé entamer la pureté de son sang, il n’a pas tardé à disparaître et à laisser libre sa majestueuse esclave.\par
Le second danger s’est réalisé aussi. Il était, d’ailleurs, en germe dans l’organisation primitive. Le secret ne s’est pas trouvé de l’étouffer ni même d’arrêter sa croissance, causée par des alliages qui, pour être rares et souvent inaperçus, n’en sont pas moins certains et ne se montrent que trop dans l’abâtardissement graduel des hautes castes de l’Inde. Toutefois, si le régime des castes n’est pas parvenu à paralyser entièrement les exigences de la nature, il les a beaucoup réduites. Les progrès du mal ne se sont accomplis qu’avec une extrême lenteur, et comme la supériorité des brahmanes et des kschattryas sur les populations hindoues n’a pas cessé, jusqu’à nos jours, d’être un fait incontestable, on ne saurait prévoir, avant un avenir très nébuleux, la fin définitive de cette société. C’est une grande démonstration de plus acquise à la supériorité du type blanc et aux effets vivifiants de la séparation des races.
\section[{III.2. Développements du brahmanisme.}]{III.2. \\
Développements du brahmanisme.}
\noindent Dans le tableau du régime inventé par les purohitas, et qui devint le brahmanisme, je n’ai encore indiqué que le système en lui-même, sans l’avoir montré aux prises avec les difficultés d’application, et j’ai choisi pour le dépeindre, non pas le moment où il commença à se former, se développant petit à petit, se complétant par des actes additionnels, mais l’époque de son apogée. Si j’ai voulu le représenter ainsi, dans sa plus haute taille, et des pieds à la tête, c’est afin qu’après avoir décrit l’enfance, je n’eusse pas à expliquer la maturité. Maintenant, pour voir le système à l’œuvre, rentrons dans le domaine de l’histoire.\par
La puissance des purohitas s’était établie sur deux fortes colonnes : la piété intelligente de la race ariane, d’une part ; de l’autre, le dévouement, moins noble mais plus fanatique, des métis et des aborigènes soumis. Cette puissance reposait sur les vayçias, toujours enclins à chercher un appui contre la prépondérance des guerriers, et sur les çoudras, pénétrés d’un sentiment nègre de terreur et d’admiration superstitieuse pour des hommes honorés de communications journalières avec la Divinité. Sans ce double appui, les purohitas n’auraient pu raisonnablement songer à attaquer l’esprit d’indépendance si cher à leur race, ou, l’ayant osé, n’auraient pas réussi. Se sachant soutenus, ils furent audacieux. Tout aussitôt, comme ils devaient s’y attendre, une vive résistance éclata dans une fraction nombreuse des Arians. Ce fut certainement à la suite des combats et des grands désastres amenés par cette nouveauté religieuse que les nations zoroastriennes, faisant scission avec la famille hindoue, sortirent du Pendjab et des contrées avoisinantes, et s’éloignèrent vers l’ouest, rompant à jamais avec des frères dont l’organisation politique ne leur convenait plus. Si l’on s’enquiert des causes de cette scission, si l’on demande pourquoi ce qui agréait aux uns écartait les autres, la réponse sans doute est difficile. Cependant je doute peu que les Zoroastriens, étant restés plus au nord et à l’arrière-garde des Arians hindous, n’aient conservé, avec une plus grande pureté ethnique, de bonnes raisons de se refuser à l’établissement d’une hiérarchie de naissance, factice à leur point de vue, et, donc, sans utilité, sans popularité chez eux. S’ils n’avaient pas dans leurs rangs des çoudras noirs, ni de vayçias câpres, ni de kschattryas mulâtres ; s’ils étaient tous blancs, tous forts, tous égaux, aucun motif raisonnable n’existait pour qu’ils acceptassent, à la tête du corps social, des brahmanes moralement souverains. Il est, dans tous les cas, certain que le nouveau système leur inspira une aversion qui ne se dissimulait point. On trouve les traces de cette haine dans la réforme dont un très ancien Zoroastre, Zerduscht ou Zeretoschtro, fut le promoteur ; car les dissidents ne conservèrent pas plus que les Hindous l’ancien culte arian. Ils prétendaient peut-être le ramener à une formule plus exacte. Tout porte, en effet, dans le magisme un caractère protestant, et c’est là que se voit la colère contre le brahmanisme \footnote{Il y a dans le Zend-Avesta des restes de croyances brahmaniques qui ne se retrouvent pas dans la croyance actuelle des Parsis. Burnouf, \emph{Comment. sur le Yaçna}, t. I, p. 342.}. Dans le langage sacré des nations zoroastriennes, le Dieu des Hindous, le \emph{Deva}, devint le \emph{Diw}, le mauvais esprit \footnote{Le nom d’Indra est également donné par les Zoroastriens à un mauvais génie. – Lassen, \emph{ouvr. cité}, t. I, p. 516.}, et le mot \emph{maaniou} reçut la signification de \emph{céleste} quand sa racine, pour les nations brahmaniques, conservait celle de \emph{fureur} et de \emph{haine} \footnote{Lassen, \emph{ouvr. cité}, t. I, p. 525.}. Ce serait ici le cas d’appliquer le 101\textsuperscript{e} vers du premier livre de Lucrèce.\par
La séparation eut donc lieu, et les deux peuples, poursuivant leur vie à part, n’eurent plus de rapports que l’arc à la main. Néanmoins, tout en se rendant, sans mesure, aversion pour aversion, insulte pour insulte, ils se souvinrent toujours de leur origine commune et ne renièrent pas leur parenté.\par
Je noterai ici, en passant, que ce fut, selon toute vraisemblance, à peu de temps de cette séparation, que commença à se former le dialecte prâcrit et que la langue ariane proprement dite, si jamais elle exista sous une forme plus concrète qu’un faisceau de dialectes, acheva de disparaître. Le sanscrit domina longtemps encore à l’état d’idiome parlé et préexcellent, ce qui n’empêcha pas les dérivations de se multiplier et de tendre à refouler, à la longue, la langue sainte dans le mutisme éloquent des livres.\par
Heureux les brahmanes, si le départ des nations zoroastriennes avait pu les délivrer de toute opposition ! Mais ils n’avaient encore lutté qu’avec un seul ennemi, et beaucoup d’opposants devaient s’efforcer de briser leur œuvre. Ils n’avaient expéri­menté qu’une seule forme de protestation : d’autres plus redoutables allaient se révéler.\par
Les Arians n’avaient pas cessé de graviter vers le sud et vers l’est, et ce mouvement, qui a duré jusqu’au XVIII\textsuperscript{e} siècle de notre ère, et qui, peut-être même, se poursuit encore obscurément tant le brahmanisme est vivace, était suivi et, en partie, causé par la pression septentrionale d’autres populations qui arrivaient de l’ancienne patrie. Le Mahabharata raconte la grande histoire de cette tardive migration \footnote{Lassen, \emph{ouvr. cité}, t. 1, 626 et pass.}. Ces nouveaux venus, sous la conduite des fils de Pandou, paraissent avoir suivi la route de leurs prédécesseurs et être venus dans l’Inde par la Sogdiane, où ils fondèrent une ville qui, du nom de leur patriarche, s’appelait \emph{Panda} \footnote{\emph{Ibid}, p. 652.}, Quant à la race à laquelle appartenaient ces envahisseurs, le doute n’est pas permis. Le mot qui les désigne veut dire \emph{un homme blanc} \footnote{\emph{Ibid}., p. 664.}\footnote{\emph{Ibid}, p. 822.}. Les brahmanes reconnaissent, sans difficulté, ces ennemis pour des rejetons de la famille humaine, source de la nation hindoue. Ils avouent même la parenté de ces intrus avec la race royale orthodoxe des Kouravas. Leurs femmes étaient grandes et blondes, et jouissaient de cette liberté qui, chez les Teutons, bizarrerie à demi condam­née des Romains, n’était que la continuation des primitives coutumes de la famille blanche .\par
Ces Pandavas mangeaient toutes sortes de viandes, c’est-à-dire, se nourrissaient de bœufs et de vaches, suprême abomination pour les Arians hindous. Sur ce point, les réformés zoroastriens conservaient l’ancienne doctrine, et c’est une nouvelle et forte preuve rétrospective qu’un mode particulier de civilisation et une déviation commune dans les idées religieuses, avaient réuni longtemps les deux rameaux en dehors des idées primordiales de la race. Les Pandavas, irrespectueux pour les animaux sacrés, ne connaissaient pas davantage la hiérarchie des castes. Leurs prêtres n’étaient pas des brahmanes, pas même les purohitas de l’ancien temps. À ces différents titres, ils paraissaient, aux yeux des Hindous, frappés d’impureté et leur contact compromettait gravement la civilisation brahmanique.\par
Comme on les reçut fort mal (ils ne s’attendaient pas, sans doute, à un autre accueil), une guerre s’engagea, qui eut pour théâtre tout le nord, le sud, l’est de la péninsule jusqu’à Videha et Viçala, et pour acteurs toutes les populations, tant arianes qu’aborigènes \footnote{\emph{Ibid}, p. 713.}. La querelle fut d’autant plus longue que les envahisseurs avaient des alliés naturels dans beaucoup de nations arianes de l’Himalaya, hostiles au régime brahmanique. Ils en trouvaient dans plusieurs peuples métis, plus intéressés encore à le repousser, et, s’il était possible, à l’abattre : conquérants et pillards, les pillards de toute couleur devenaient leurs amis \footnote{Ibid, p. 689. – Les Pandavas paraissent avoir dû surtout leur victoire à des renforts venus des régions septentrionales, tels que les Kulindas, établis à l’est vers les sources du Gange. Le Mahabharata les considère comme une race pure, mais très en dehors de la culture hindoue.}.\par
L’intérêt incline évidemment du côté des Kouravas, qui défendaient la civilisation. Pourtant, après bien du temps et des peines, après avoir longtemps repoussé leurs antagonistes, les Kouravas finirent par succomber. Le Pendjab et de vastes contrées aux alentours restèrent acquis aux envahisseurs plus blancs, et, par conséquent, plus énergiques que les nations brahmaniques, et la civilisation hindoue, forcée de céder, s’enfonça davantage dans le sud-est. Mais elle était tenace en raison de l’immobilité de ses races. Elle n’eut qu’à attendre, et sa revanche sur les descendants des Pandavas fut éclatante. Ceux-ci, vivant libres de toute restriction sacrée, se mêlèrent rapidement aux indigènes. Leur mérite ethnique se dégrada. Les brahmanes reprirent le dessus. Ils enlacèrent les fils dégénérés de Pandou dans leur sphère d’action, leur imposèrent idées et dogmes, et, les forçant de s’organiser sur les modèles donnés par eux, couronnèrent la victoire en leur fournissant une caste sacerdotale qui ne fut pas triée parmi ce qu’il y avait de mieux. Aussi remarque-t-on, dans le Kachemyr, que les hommes de la classe suprême sont plus bruns aujourd’hui que le reste de la population. C’est que leurs ancêtres viennent du sud \footnote{Les populations du Kachemyr et du Pendjab ont eu des contacts de toute espèce avec les peuples jaunes, tout aussi bien qu’avec les tribus noires ou mulâtres. Dans les temps plus modernes, elles ont été envahies par les Grecs Bactriens et les Saces, puis par les Arabes, les Afghans, les Baloukis. F. Lassen, \emph{Zeitschrift für die Kunde des Morgenlandes}, t III, p. 208 : \emph{Indisch. Alterth.}, t. I, p. 404. Il résulte d’un tel état de choses que le pays hindou qui vit le premier dominer les tribus arianes est aujourd’hui un de ceux où ces dernières ont subi le plus de mélanges. Dans les temps épiques, les Dârâdas du Pendjab étaient déjà comptés parmi les peuples réprouvés. – Lassen,\emph{ loc. cit.}, p. 544.}.\par
Les rapports entre les castes ne furent pas, dans le nord, pareils à ce qu’ils étaient dans le sud. Les brahmanes ne s’y montrèrent pas intellectuellement supérieurs au reste des nationaux, ceux-ci n’obéirent jamais aisément à leur sacerdoce \footnote{C’est ainsi que la fameuse classification que faisaient les écrivains grecs des nations hindoues en trois classes : les \emph{pêcheurs}, les \emph{agriculteurs} et les \emph{montagnards}, ne peut, de toute évidence, s’appliquer qu’à des groupes fort peu arianisés et habitant les confins occidentaux.}, et le mépris profond des vrais Hindous, des qualifications injurieuses, et, mieux que tout, une infériorité morale très marquée punirent à jamais les descendants des Pandavas de la perturbation qu’ils avaient apportée un moment dans l’œuvre brahmanique. On peut donc observer ici ce phénomène, que ce fut moins de la pureté de la race que de l’homogénéité des éléments ethniques que résulta la victoire des brahmanes sur les descendants des Pandavas. Chez les premiers, tous les instincts étaient classes et agissaient, sans se nuire, dans des sphères spéciales ; chez les seconds, le mélange illimité du sang les brouillait à l’infini. Nous avons déjà vu l’analogue de cette situation dans la dernière période de l’histoire tyrienne.\par
À dater de ce moment, de nombreuses nations arianes se trouvèrent encore à peu près retranchées de la nationalité hindoue, et réduites à un degré inférieur de dignité et d’estime. Il faut placer, dans cette catégorie, les tribus blanches, vivant entre la Sarasvati et l’Hindou-koh, et plusieurs des riverains de l’Indus, c’est-à-dire celles-là mêmes qui, aux yeux de l’antiquité grecque ou romaine, représentaient les populations de l’Inde \footnote{ \noindent « Quant aux Pandits (Cachemyriens), tous bramines de caste, ils sont d’une ignorance « grossière, et il n’y a pas un de nos serviteurs hindous qui ne se regarde comme de « meilleure caste qu’eux. Ils mangent de tout, excepté du bœuf, et boivent de l’arak ; il n’y a « dans l’Inde que les gens des castes infâmes qui le fassent. »\par
 (\emph{Correspondance de V. Jacquemont. –} Lettre du 22 avril 1831.)
}. Au-dessous de ces peuplades dédaignées, il y en avait un très grand nombre d’impures, puis venaient les aborigènes \footnote{Les populations attaquées par Alexandre étaient à demi arianes, mais considérées comme vratyas par les vrais Hindous. Tels étaient les Mali (Malavas) et les sujets de Porus (Pourou). Les Malavas étaient comptés au nombre des Bahlikas, avec les Ksudrakas (Oxydraques). Leurs brahmanes étaient considérés comme peu réguliers, et le Manava-Dharma-Sastra les accuse de négliger l’enseignement religieux. – Lassen, \emph{Indisch. Alterth.}, t. I, p. 197; A W. V. Schlegel, \emph{Indische Bibliothek}, t. I, p. 169 et pass. – Si les Grecs ne connaissaient les Hindous que par approximation, ceux-ci n’étaient pas moins ignorants à leur égard. Dans les temps les plus anciens, les hommes d’au delà du Sindh avaient appelé les populations de l’ouest, Chamites et Sémites, avec lesquelles ils avaient des relations commerciales, \emph{Javana}, mot très difficile à expliquer, car s’il paraît désigner généralement des nations occidentales, il s’applique aussi à des tribus du nord, voire même du sud. \emph{Jawa} signifie \emph{courir, faire invasion.} (W. de Humboldt, \emph{Ueber die Kawi-Sprache}, t. I, p. 65 et pass.; Burnouf, \emph{Nouveau journal asiatique}, t. X, p. 238.) Plus tard, \emph{javana} désigna particulièrement les Arabes. La Bible, s’emparant de cette expression, l’applique aux habitants sémites de Chypre et de Rhodes, et même aux Turdétains d’Espagne, et les nomme \emph{Javanim.} (Movers, \emph{das Phœnizische Allerthum}, t. II, 1\textsuperscript{re} partie, p. 270.) Enfin on trouve, dans une inscription de Darius, \emph{Jouna} devenu la dénomination des Grecs insulaires, et, comme l’usage de ce mot chez les Hellènes est postérieur à Homère, il est à croire que les colons de la côte l’ont reçu des Perses, et, après l’avoir adopté pour eux-mêmes, l’ont transmis aux populations continentales. (Lassen, \emph{Indisch. Alterth.}, t. I, p. 730.) Ce n’est que très tard que les Hindous ont sciemment reconnu les Grecs dans les javanas et l’époque n’en est pas antérieure au V\textsuperscript{e} siècle avant notre ère. Le Mahabharata, dans ses derniers livres, dénomme ainsi les Macédoniens-Bactriens, et les vante comme faisant partie d’un peuple brave et savant. (Lassen, \emph{ibid.}, p. 862, et \emph{Zeitschrift für d. K. des Morgenl.}, II, p. 215.)}.\par
Ainsi, pour les brahmanes, terribles logiciens, l’humanité politique se divisait en trois grandes fractions : la nation hindoue proprement dite, avec ses trois castes sacrées et sa caste supplémentaire, que l’on pourrait appeler de tolérance – sacrifice que la conviction faisait à la nécessité – puis les nations arianes, nommées vratyas, trop ouvertement mêlées au sang indigène, qui avaient adopté tard la règle sacrée et ne la suivaient pas rigoureusement, ou bien, pis est, s’étaient obstinées à la repousser. Dans ce cas, l’appellation de vratya, voleur, pillard, ne suffisait pas à l’aversion indignée du véritable Hindou, et de pareilles gens étaient qualifiés de \emph{dasyou}, terme qui emporte un sens à peu près semblable avec le superlatif. Cette injure agréait d’autant mieux à la rancune acrimonieuse de ceux qui l’employaient, qu’elle se rapproche étymologi­quement du zend \emph{dandyou, dakyou, dakhou} \footnote{Lassen, \emph{Zeitschrift für K. des Morgenl.}, t. II, p. 49.}, dont usaient les Zoroastriens du sud pour désigner les provinces de leurs États. Rien de plus semblable (charité à part) au rebut du genre humain qu’un hérétique, et réciproquement.\par
Enfin, en troisième lieu et même au-dessous de ces \emph{dasyous} si détestés, venaient les nations aborigènes. Nulle part on n’imaginera de plus complets sauvages, et, par malheur, c’est que leur nombre était exorbitant. Pour juger de leur valeur morale, il faut voir aujourd’hui ce que sont leurs descendants les plus purs, soit dans le Dekkhan, soit dans les monts Vyndhias et dans les forêts centrales de la péninsule, où ils vont errant par bandes. Regardons-les vivant, après tant de siècles, comme faisaient leurs aïeux au temps où Rama vint combattre les insulaires de Ceylan, alors leurs congénères. Je ne prétends pas les énumérer tous, ce n’est pas mon affaire ; j’indiquerai seulement quelques noms.\par
Les Kad-Erili-Garou, parlent le tamoul. Ils vont entièrement nus, dorment sous des grottes et des buissons, vivent de racines, de fruits et d’animaux qu’ils attrapent.\par
Ne sont-ce pas là les fils d’Anak, les Chorréens de l’Écriture \footnote{Lassen, \emph{Indiscb. Alterth.}, t. I, 364. – Une tribu qui rappelle encore mieux les fils d’Anak est celle qui habitait jadis au delà de la rive sud de la Yamouna, dans le désert de Dandaka, jusqu’à la Gadaouri. C’étaient des géants féroces, toujours enclins à attaquer les ermitages des ascètes brahmaniques. (\emph{Ouvr. cité}, p. 524 et passim.)}?\par
 Les Katodis campent sous les arbres, mangent les reptiles crus, et, quand ils l’osent, se couchent sur les fumiers des villages hindous.\par
Les Kauhirs ne savent même pas se défendre contre les attaques des bêtes féroces. Ils fuient ou sont dévorés, et se laissent faire \footnote{Lassen, \emph{Ibid.}, p. 372.}.\par
Les Kandas, très adonnés aux sacrifices humains, égorgent les enfants hindous qu’ils volent, ou même en achètent des plus misérables parias, leurs semblables à beaucoup d’égards. En voilà assez \footnote{\emph{Ibid}., p. 377.}.\par
Les brahmanes donnaient à tous les peuples de cette triste catégorie le nom général de \emph{Mlekkhas} \footnote{\emph{Mlekkba} veut dire \emph{faible}. (Benfey, \emph{Encycl. Ersch u. Gruber, Indien}, p. 7.)}\emph{, sauvages}, ou de \emph{Barbaras.} Ce dernier nom est incrusté dans toutes les langues de l’espèce blanche. Il témoigne assez de la supériorité que cette famille s’adjuge sur le reste de l’espèce humaine \footnote{Barbara, varvara indique un homme qui a les cheveux crépus ; papoua a la même signification. (Benfey, loc. cit.) Comme le mot barbare est en usage dans toutes les langues de notre société, il en faut conclure que les premiers peuples non blancs connus des Arians furent des noirs, ce qui est d’accord avec ce qui a été remarqué de l’énorme diffusion de cette race vers le nord. (Lassen, Indisch Alterth., t. I, p. 855.) Plusieurs nations, non blanches, métisses ou noires portent aujourd’hui ce nom. Ainsi les Barbaras, sur la côte occidentale de l’Indus (Lassen, Zeitschrift für die Kunde des Morgenlandes, t. III, p. 215) ; les Barabras, sur le cours supérieur du Nil ; les Berbers d’Afrique, etc. (Meïer, Hebraisches Wurzelwœrterbuch, 1845.)}.\par
À considérer le nombre immense des aborigènes, les politiques de l’Inde compre­naient cependant que les renier ne les paralysait pas, et qu’il fallait, mettant de côté toute répugnance, les rallier par un appât quelconque à la civilisation ariane. Mais le moyen ? Que restait-il à leur offrir qui pût les tenter ? Tous les bonheurs de ce monde étaient distribués. Les brahmanes imaginèrent pourtant de les leur proposer, même les plus hauts, même ceux que les premiers Arians se faisaient fort de conquérir par la vigueur de leurs bras, j’entends le caractère divin, avec cette seule réserve, que tant de magnifiques perspectives ne devaient s’ouvrir qu’après la mort, que dis-je ? après une longue série d’existences. Le dogme de la métempsycose une fois admis, rien de plus plausible, et comme le Mlekkha voyait, sous ses yeux, toutes les classes de la société hindoue agir en vertu de cette croyance, il avait déjà, dans la bonne foi de ses convertisseurs, une forte raison de se laisser convaincre.\par
Le brahmane véritablement pénitent, mortifié, vertueux, se flattait hautement de prendre place, après sa mort, dans une catégorie d’êtres supérieurs à l’humanité. Le kschattrya renaissait brahmane avec la même espérance au deuxième degré, le vayçia reparaissait kschattrya, le çoudra, vayçia \footnote{Les fautes, les crimes produisaient le même effet en sens contraire : « As the son of a Sudra « may thus attain the rank of a Brahman, and as the son of a Brahman may sink to a level « with Sudras, even so must it be with him who springs from a Chsatriya ; even so with him, « who was born of a Vaisya. » (\emph{Manava-Dharma-Sastra}, chap. X, § 65.)}. Pourquoi l’indigène ne serait-il pas devenu çoudra, et ainsi de suite ? D’ailleurs, il arriva que ce dernier rang lui fut conféré même de son vivant. Quand une nation se soumettait en masse, et qu’il fallait l’incorporer à un État hindou, on était contraint, malgré le dogme, de l’organiser, et le moins qu’on pût faire pour elle, c’était encore de l’admettre immédiatement dans la dernière des castes régulières \footnote{Les temps les plus anciens offrent des exemples de cette politique tolérante. Ainsi les Angas, les Poundras, les Bangas, les Souhmas et les Kalingas, populations aborigènes du sud-est, s’étant converties, furent d’abord déclarées çoudras en masse. Puis le roi des Angas, Lomâpâda, ayant obtenu la main de la fille du souverain arian d’Ayodhya, ses descendants furent considérés comme fils de brahmanis et de kschattryas. (Lassen, \emph{Indische Alterthumskunde}, t. I, p. 559.)}.\par
Des ressources politiques comme ce système de promesses réalisables moyennant résurrection ne peuvent s’improviser. Elles n’ont de valeur que lorsque la bonne foi de ceux qui les emploient est intacte. Dans ce cas elles deviennent irrésistibles, et l’exemple de l’Inde le prouve.\par
Il y eut ainsi, vis-à-vis des Aborigènes, deux sortes de conquêtes. L’une, la moins fructueuse, fut opérée par les kschattryas. Ces guerriers, formant une armée régulière quadruple, disent les poèmes, c’est-à-dire composée d’infanterie, de cavalerie, de chars armés et d’éléphants, et généralement appuyée d’un corps auxiliaire d’indigènes, se mettaient en campagne et allaient attaquer l’ennemi. Après la victoire, la loi civile et religieuse interdisait aux militaires de procéder à l’incorporation des populations impures. Les kschattyras se contentaient d’enlever le pouvoir au chef promoteur de la querelle, et lui substituaient un de ses parents ; après quoi ils se retiraient en emportant le butin et des promesses précaires de soumission et d’alliance \footnote{Lassen, \emph{Indische Alterth.}, t. I, p. 535. – Il est douteux que la campagne de Rama contre les Raksasas, démons noirs du sud, ait déterminé l’établissement des Arians à Lanka ou Ceylan. Le vainqueur, après avoir détrôné Ravana, donna l’empire à un des frères de ce géant et s’en retourna vers le nord. – \emph{Ramayana}}. Les brahmanes procé­daient tout autrement, et leur manière constitue seule la véritable prise de possession du pays et les conquêtes sérieuses \footnote{Lassen,\emph{ ouvr cité}, t. I, p. 578}.\par
Ils s’avançaient par petits groupes au delà du territoire sacré de l’Aryavarta ou Brahmavarta. Une fois dans ces forêts épaisses, dans ces marécages incultes où la nature des tropiques fait croître en abondance les arbres, les fruits, les fleurs, place les oiseaux aux riches plumages et aux chants variés, les gazelles par troupeaux, mais aussi les tigres et les reptiles les plus redoutables, ils construisaient des ermitages isolés où les aborigènes les voyaient s’appliquant incessamment à la prière, à la méditation, à l’enseignement. Le sauvage pouvait les tuer sans peine. À demi nus, assis à la porte de leurs cabanes de branchages, seuls le plus souvent, tout au plus assistés de quelques disciples aussi désarmés qu’eux-mêmes, le massacre ne présentait ni les difficultés ni les enivrements de la lutte. Cependant des milliers de victimes tombèrent \footnote{D’après les légendes brahmaniques et les poèmes, les ascètes avaient affaire à des anthropo­phages.(Lassen, \emph{Indische Alterth}., t I, p. 535.)}. Mais, pour un ermite égorgé dix accouraient, se disputant le sanctuaire désormais sanctifié, et les vénérables colonies, étendant de plus en plus leurs ramifications, conquéraient irrésis­tiblement le sol. Leurs fondateurs ne s’emparaient pas moins de l’imagination de leurs farouches meurtriers. Ceux-ci, frappés de surprise ou d’une superstitieuse épouvante, voulaient enfin savoir ce qu’étaient ces mystérieux personnages si indifférents à la souffrance et à la mort, et quelle tâche étrange ils accomplissaient. Et voilà alors ce que les anachorètes leur apprenaient. « Nous sommes les plus augustes des hommes, et nul ici-bas ne « nous est comparable. Ce n’est pas sans l’avoir mérité que nous possédons « cette dignité suprême. Dans nos existences antérieures, on nous vit aussi « misérables que vous-mêmes. À force de vertus et de degrés en degrés, nous « voici au point où les rois même rampent à nos pieds. Toujours poussés d’une « unique ambition, aspirant à des grandeurs sans limites, nous travaillons à « devenir dieux. Nos pénitences, nos austérités, notre présence ici, n’ont pas « d’autre but. Tuez-nous : nous aurons réussi. Écoutez-nous, croyez, humiliez-« vous, servez, et vous deviendrez ce que nous sommes \footnote{Manava-Dharma-Sastra, chap. X, § 62 : Desertion of life, without reward, for the « sake of « preserving a priest or a cow, a woman or a child, may cause the beatitude of those base-born « tribes. »}. »\par
Les sauvages écoutaient, croyaient et servaient. L’Aryavarta gagnait une province. Les anachorètes devenaient la souche d’un rameau brahmanique local. Une colonie de kschattryas accourait pour gouverner et garder le nouveau territoire. Bien souvent, presque toujours, une tolérance nécessaire souffrit que les rois du pays prissent rang dans la caste militaire. Des vayçias se formèrent également, et, je le crois, sans un trop grand respect pour la pureté du sang. D’un district de l’Inde à l’autre, le reproche de manquer de pureté n’a jamais cessé de courir et d’atteindre même les brahmanes \footnote{« Of two telingas bramines, who came from the vicinity of Hyderabad, one was derived of « intermixture with the white race. This man stated that his cast intermarried with the « bramins of the Dekkan ; but not with those of Bengal or Guzerat. All the Mahrattas « bramins I meet with appeared to be of unmixed white descent ; but one of them said that « the telinga bramins were highly respected, while the Pendjaub, Guzerat, Cutche and « Cashmere bramins were regarded as impure. » (Pickering, p. 181.)}. Il est incontestable que ce reproche est fondé, et l’on en peut donner des preuves éclatantes. Ainsi, dans les temps épiques, Lomâpâda, le roi indigène des Angas convertis, épouse Çanta, fille du roi arian d’Ayodhya \footnote{De même aux termes du Ramayana, une des femmes du roi héroïque Dasaratha appartient à la nation kêkaya. Ce peuple, à la vérité, était arian ; mais habitant au delà de la Sarasvati, hors des limites du territoire sacré, il était considéré comme réfractaire ou vratya.}. Ainsi encore, au XVIII\textsuperscript{e} siècle, lors des colonisa­tions hindoues opérées chez les peuples jaunes, à l’est de la Kali, dans le Népaul et le Boutan, on a vu les brahmanes se mêler aux filles du pays et installer leur progéniture métisse comme caste militaire \footnote{Lassen, \emph{ouvr. cité}., t. I, p. 443 et 449.}.\par
Procédant de cette manière, au nom de leur principe ; rendant ce principe indis­pensable à l’organisation sociale, cependant le faisant plier, malheureusement pour l’avenir, très judicieusement pour le présent, devant les difficultés trop grandes, les ascètes brahmaniques formaient une corporation d’autant plus nombreuse que la vie de ses membres était généralement sobre et toujours éloignée des travaux de la guerre. Leur système s’implantait profondément dans la société qui leur devait la vie. Tout se pré­sentait bien : seulement, si grands que fussent les obstacles déjà surmontés, il en allait surgir de plus redoutables encore.\par
Les kschattryas s’apercevaient que si, dans cette organisation sociale, le rôle le plus brillant leur était assigné, la puissance que leur laissait le sacerdoce avait plus de fleurs que de fruits. À peu près réduits à la situation de satellites effacés, il leur devenait difficile d’avoir une idée, une volonté, un plan différent de celui qu’avaient arrêté, sans eux, les brahmanes, et, tout rois qu’on les disait, ils se sentaient tellement enlacés par les prêtres, que leur prestige, vis-à-vis des peuples, devenait secondaire. Ce n’était pas non plus, pour leur avenir, un symptôme peu menaçant que de voir les brahmanes se poser, dans l’État, en médiateurs éternels entre les souverains et leurs bourgeois, leurs peuples, peut-être même leurs guerriers, tandis qu’au moyen d’une énergique patience, d’un indomptable détachement des joies humaines, ces mêmes brahmanes se faisaient les pères, les augmentateurs de l’Aryavarta, par les conversions en masse que leurs courageux missionnaires opéraient dans les nations aborigènes. Un tel tableau devait cesser, tôt ou tard, d’être considéré d’un œil placide par les princes, et les brahmanes paraissent ne pas avoir assez ménagé, même d’après les données de leur propre système, les méfiances et l’ambition des hommes qu’ils avaient le plus à craindre.\par
Ce n’est pas qu’ils n’aient usé de quelques ménagements. De même qu’ils avaient fait plier la rigueur de leur système jusqu’au point d’admettre des chefs aborigènes à la dignité de kschattryas, ils avaient fait preuve d’une tolérance plus difficile encore à l’égard des Arians de cette caste, en permettant à plusieurs, que signalaient la sainteté, la science et des pénitences extraordinaires, de s’élever au rang de brahmane. L’épisode de Visvamitra, dans le Ramayana, n’a pas d’autre signification \footnote{Burnouf, \emph{Introduction à l’histoire du bouddhisme indien}, t. I, p. 891.}. On citerait encore la consécration d’un autre guerrier de la race des Kouravas. Mais de telles concessions ne pouvaient être que rares, et il faut avouer qu’en échange ils se réservaient la faculté d’épouser des filles de kschattryas et de devenir rois à leur tour. Gendres des souverains, ils admettaient encore que les rejetons de leurs alliances suivaient une loi de décroissance, et se trouvaient exclus de la caste sacerdotale. Mais, du chef de leur mère, les prérogatives de la tribu militaire leur revenaient pleinement, et la dignité royale du même coup. Il y a, sur ce sujet, une anecdote que j’intercalerai ici, bien qu’elle interrompe, ou peut-être parce qu’elle interrompt des considérations un peu longues et assez arides.\par
Il existait, dans des temps très anciens, à Tchampa, un brahmane. Ce brahmane eut une fille, et il demanda aux astrologues quel avenir était réservé à l’objet de son inquiète tendresse. Ceux-ci, ayant consulté les astres, reconnurent, à l’unanimité, que la petite brahmani serait un jour mère de deux enfants, dont l’un deviendrait un saint illustre et l’autre un grand souverain. Le père fut transporté de joie à cette nouvelle, et aussitôt que la jeune fille se trouva nubile, remarquant avec orgueil comme elle était douée d’une beauté parfaite, il voulut concourir à l’accomplissement du destin, peut-être le hâter, et il s’en alla offrir son enfant à Bandusara, roi de Pataliputhra, monarque renommé pour ses richesses et sa puissance.\par
Le don fut accepté, et la nouvelle épouse conduite dans le gynécée royal. Ses grâces y firent trop de sensation. Les autres épouses du kschattrya la jugèrent tellement dangereuse, qu’elles appréhendèrent d’être remplacées dans le cœur du roi, et se mirent à chercher une ruse qui, tout aussi bien qu’une violence impossible, les pût débarrasser de leurs craintes, en écartant leur rivale. La belle brahmani était, comme je l’ai dit, fort jeune, et, probablement, sans beaucoup de malice. Les conjurées surent lui persuader que, pour plaire à son mari, il lui fallait apprendre à le raser, à le parfumer et à lui couper les cheveux. Elle avait tout le désir imaginable d’être une épouse soumise : elle obéit donc promptement à ces perfides conseils, de sorte que la première fois que Bandusara la fit appeler, elle se présenta devant lui une aiguière d’une main et portant, dans l’autre, tout l’appareil de la profession qu’elle venait d’apprendre.\par
Le monarque, qui, sans doute, se perdait un peu dans le nombre de ses femmes et avait en tête des préoccupations de toute nature, oublia les tendres mouvements dont il était agité un moment auparavant, tendit le cou et se laissa parer. Il fut ravi de l’adresse et de la grâce de sa servante, et tellement que le lendemain il la demanda encore. Nouvelle cérémonie, nouvel enchantement, et, cette fois, voulant, en prince généreux, reconnaître le plaisir qu’il recevait, il demanda à la jeune fille comment il pourrait la récompenser.\par
La belle brahmani indiqua naïvement un moyen sans lequel les promesses des astrologues ne pouvaient, en effet, s’accomplir. Mais le roi se récria bien fort. Il remontra cependant avec bonté, à la belle postulante, que, puisqu’elle était de la caste des barbiers, sa prétention était insoutenable, et qu’il ne commettrait certainement pas une action aussi énorme que celle dont elle le sollicitait. Aussitôt, explication ; l’épouse méconnue revendique, avec le juste sentiment de la dignité blessée, sa qualité de brahmani, raconte pourquoi et dans quelle louable intention elle remplit les fonctions serviles qui scandalisent le roi tout en lui agréant. La vérité se fait jour, la beauté triomphe, l’intrigue s’évanouit, et l’astrologie s’honore d’un succès de plus, à la grande satisfaction du vieux brahmane \footnote{Bournouf, \emph{Introduction à l’histoire du bouddhisme indien}, t. I, p. 149.}.\par
Ainsi, dans l’organisation antique de l’Inde, l’union de deux castes était, pour le moins, tolérée, et, en mille circonstances, les brahmanes devaient se trouver en concur­rence directe avec les kschattryas pour l’exercice matériel de la souveraine puissance \footnote{La \emph{Manava-Dharma-Sastra} (chap. III) stipule, évidemment, une loi de tolérance que le système rigoureux n’admettait pas (§ 12) : « For the first marriage of the twice born classes, « a woman of the same class is recommended ; but for such as are impelled by inclination to « marry again, women in the direct order of the classes are to be preferred. » – § 13 : « A « Sudra-Woman only must be wife of a Sudra ; she and a Vaicya, of a Vaicya ; they two and « a Kshatriya of a Kshatriya ; those two and a Brahmany of a Brahman. » – § 14 : « A « woman of the servile class is not mentioned, even in the recital of any ancient story, as the « first wife of a Brahman or of a Kshatriya, athough in the greatest difficulty to find a « suitable match. » – Aujourd’hui, routes ces atténuations, en effet illogiques, ont été supprimées ; les alliances d’une caste à l’autre sont sévèrement interdites, et le \emph{Madana-Ratna-Pradipa} dit expressément : « These marriage of twice born men with damsels not of « the same class... these \emph{parts of ancient law} were abrogated by wise legislators. » Malheureusement, la défense est venue quand le mal s’était déjà beaucoup développé. Elle n’est cependant pas inutile.}. Comment faire ? Appliquer le principe de séparation dans sa rigueur entière, n’était-ce pas blesser tout le monde ? Il y fallait des ménagements. D’autre part, si l’on en gardait trop, le système même était en péril. On essaya de recourir, pour éviter le double écueil, à la logique et à la subtilité si admirables de la politique brahmanique.\par
Il fut établi que, dans la règle, le fils d’un kschattrya et d’une brahmani ne pourrait être ni roi ni prêtre. Participant, tout à la fois, des deux natures, il serait le barde et l’écuyer des rois. En tant que brahmane dégénéré, il pourrait être savant dans l’histoire, connaître les poésies profanes, en composer lui-même, les réciter à son maître et aux kschattryas rassemblés. Pourtant il n’aurait pas le caractère sacerdotal, il ne connaîtrait pas les hymnes liturgiques, et l’étude directe des sciences sacrées serait interdite à son intelligence. Comme kschattrya incomplet, il aurait le droit de porter les armes, de monter à cheval, de diriger un char, de combattre, mais en sous-ordre, et sans espoir de commander jamais lui-même à des guerriers. Une grande vertu lui fut réservée : ce fut l’abnégation. Accomplir des exploits pour son prince et s’oublier en chantant les traits de valeur des plus braves, tel fut son lot ; on l’appelait le soûta. Aucune figure héroïque des épopées hindoues n’a plus de douceur, de grâce, de tendresse et de mélancolie. C’est le dévouement d’une femme dans le cœur indomptable d’un héros \footnote{Lassen, \emph{ouvr. cité}, t. I, p. 480.–- Le soutâ est le véritable prototype de l’écuyer de la chevalerie errante, du GandoIin ou Gwendofin d’Amadis.}.\par
Une fois le principe admis, les applications en devenaient constantes, et, en dehors des quatre castes légales, le nombre des associations parasites allait devenir incom­mensurable \footnote{Lassen, \emph{ibid.}, p. 196.}. Il le devint tellement, les combinaisons se croisant formèrent un réseau si inextricable, que l’on peut considérer aujourd’hui, dans l’Inde, les castes primitives comme presque étouffées sous les ramifications prodigieuses auxquelles elles ont donné naissance, et sous les greffes perpétuelles que ces ramifications supplémentaires ont causées à leur tour. D’une brahmani et d’un kschattrya nous avons vu naître les bardes-écuyers ; d’une brahmani et d’un vayçia sortirent les ambastas, qui prirent le monopole de la médecine, et ainsi de suite. Quant aux noms imposés à ces subdivi­sions, les uns indiquent les fonctions spéciales qu’on leur attribuait, les autres sont simplement des dénominations de peuples indigènes étendues à des catégories qui, sans doute, avaient mérité de les prendre, en se mêlant à leurs véritables propriétaires \footnote{La loi cherchait cependant à retenir, tout en cédant ; ainsi elle n’est à peu près clémente que pour les unions contractées entre les castes rapprochées l’une de l’autre, et voici ce qu’elle dit, par exemple, du produit d’un guerrier avec une femme de la classe servile : « From a « Kshatrya with a wife or the Sudra class, springs a creature, called Ugra, with a nature « partly warlike and partly servile, ferocious in his manners, cruel in his acts. » (Manava-Dharma-Sastra, chap. X, § 9.) – Ce passage suffirait seul à prouver l’importance que les brahmanes apportaient à conserver le sang arian en vue des qualité morales qu’ils lui reconnaissaient.}.\par
Cet ordre apparent, tout ingénieux qu’il fût, devenait, en définitive, du désordre, et bien que les compromis dont il résultait eussent été inséparables des débuts du système, il n’était pas douteux que, si l’on voulait empêcher le système lui-même de périr sous l’exubérance de ces concessions néfastes, il ne fallait pas louvoyer plus longtemps, et qu’un remède vigoureux devait, quoi qu’il pût arriver, cautériser au plus vite la plaie ouverte aux flancs de l’état social. Ce fut d’après ce principe que le brahmanisme inventa la catégorie des tchandalas, qui vint compléter d’une manière terrible la hiérarchie des castes impures.\par
Les dénominations insultantes et les rigueurs n’avaient pas été ménagées aux Arians réfractaires ni aux aborigènes insoumis. Mais on peut dire que l’expulsion, et même la mort, furent peu de chose auprès de la condition immonde à laquelle les quatre castes légales eurent à savoir que seraient désormais condamnés les malheureux issus de leurs mélanges par des hymens défendus. L’approche de ces tristes êtres fut à elle seule une honte, une souillure dont le kschattrya pouvait, à son gré, se laver en immolant ceux qui s’en rendaient coupables. On leur refusait l’entrée des villes et des villages. Qui les apercevait pouvait lancer les chiens sur eux. Une fontaine où on les avait vus boire était condamnée. S’établissaient-ils en un lieu quelconque, on avait le droit de détruire leur asile. Enfin, il ne s’est jamais trouvé sur la terre de monstres détestés contre lesquels une théorie sociale, une abstraction politique, se soit plu à imaginer de si épouvantables effets d’anathème. Ce n’étaient pas les malheureux tchandalas que l’on considérait au moment où l’on fulminait des menaces si atroces : c’étaient leurs futurs parents qu’il s’agissait d’effrayer. Aussi faut-il le reconnaître, si la caste éprouvée a senti, en quelques occasions, s’appesantir sur elle le bras sanguinaire de la loi, ces occasions ont été rares. La théorie lutta ici vainement contre la douceur des mœurs hindoues. Les tchandalas furent méprisés, détestés ; pourtant ils vécurent. Ils possédèrent des villages qu’on aurait eu le droit d’incendier, et qu’on n’incendia point. On ne prit même pas tant de soin de fuir leur contact, qu’on ne tolérât leur présence dans les villes. On les laissa s’emparer de plusieurs branches d’industrie, et nous avons vu tout à l’heure la brahmani de Tchampa prise pour une tchandala par le roi son mari, parce qu’elle remplissait un office concédé à cette tribu, et cependant favorablement accueillie chez un monarque même. Dans l’Inde moderne, des fonctions réputées impures, comme celles de boucher par exemple, rapportent de gros bénéfices aux tchandalas qui s’en mêlent. Plusieurs se sont enrichis par le commerce des blés. D’autres jouent un rôle important dans les fonctions d’interprètes. En montant au plus haut de l’échelle sociale, on trouve des tchandalas riches, heureux et, indépendamment de l’idée de caste, considérés et respectés. Telle dynastie hindoue est bien connue pour appartenir à la caste impure, ce qui ne l’empêche pas d’avoir pour conseillers des brahmanes qui se prosternent devant elle. Il est vrai qu’un pareil état de choses n’a pu être amené que par les bouleversements survenus depuis les invasions étrangères. Quant à la tolérance pratique et à la douceur des mœurs opposées à la fureur théorique de la loi, elle est de tous les temps \footnote{Le comte E. de Warren, \emph{l’Inde anglaise en 1843. –} Dans les époques antiques, on a vu déjà des hommes qui, sans être de la caste guerrière, pouvaient devenir souverains. Le plus ancien empire établi dans le sud fut celui du Pândja, dont Madhûra était la capitale. Il avait été fondé par un vayçia venu du nord, postérieurement à l’époque des guerres de Rama. (Lassen, \emph{Indische Alterthumskunde}, t. I, p. 536.)}.\par
 J’ajouterai seulement que, de tous les temps aussi, les tchandalas, s’ils eurent quelque chose d’arian dans leur origine, comme on ne peut en douter, n’ont rien eu de plus pressé que de le perdre. Ils ont usé de la vaste latitude de déshonneur où on les abandonnait, pour s’allier et se croiser, sans fin, avec les indigènes. Aussi sont-ils, en général, les plus noirs des Hindous, et quant à leur dégradation morale, à leur lâche perversité, elle n’a pas de limites \footnote{C’est à ce dernier trait que les brahmanes prétendent reconnaître surtout les castes impures : « Him, who was born of a sinful mother, and consequently in a low class, but is not openly « known, who, though worthless in truth, bears the semblance of a worthy man, let people « discover by his acts. – Want of virtuous dignity, harshness of speech, cruelty, and habitual « neglect of prescribed « duties, betray in this world the son of a criminal mother. » (\emph{Manava-Dharma-Sastra}, chap. X, §§ 57 et 58.)}.\par
L’invention de cette terrible caste eut certainement de grands résultats, et je ne doute pas qu’elle n’ait été assez puissante pour maintenir dans la société hindoue la classification qui en formait la base, et mettre un grand obstacle à la naissance de nouvelles castes, au moins au sein des provinces déjà réunies à l’Aryavarta. Quant à celles qui le furent ensuite, les sources des catégories ne doivent pas non plus être recherchées trop strictement.\par
Là comme ailleurs, alors comme auparavant, les brahmanes firent ce qu’ils purent. Il leur suffit d’avoir une apparence pour commencer, et de n’établir leurs règles qu’une fois l’organisation assise. Je ne répéterai pas ici ce que j’ai dit pour le Boutan et le Népaul. Ce qui arriva dans ces contrées se produisit dans bien d’autres. Toutefois, il ne faut pas perdre de vue que, quel que fût le degré dans lequel la pureté du sang arian se compromit en tel ou tel lieu, cette pureté restait toujours plus grande dans les veines des brahmanes d’abord, des kschattryas ensuite, que dans celles des autres castes locales, et de là cette supériorité incontestable qui, même aujourd’hui, après tant de bouleversements, n’a pas encore fait défaut à la tête de la société brahmanique. Puis, si la valeur ethnique de l’ensemble perdait de son élévation, le désordre des éléments n’y était que passager. L’amalgame des races se faisait plus promptement au sein de chaque caste en se trouvant limité à un petit nombre de principes, et la civilisation haussait ou baissait, mais ne se transformait pas, car la confusion des instincts faisait assez promptement place dans chaque catégorie à une unité véritable, bien que de mérite souvent très pâle. En d’autres termes, autant de castes, autant de races métisses, mais closes et facilement équilibrées.\par
La catégorie des tchandalas répondait à une nécessité implacable de l’institution, qui devait surtout paraître odieuse aux familles militaires. Tant de lois, tant de restrictions arrêtaient les kschattryas dans l’exercice de leurs droits guerriers et royaux, les humiliaient dans leur indépendance personnelle, les gênaient dans l’effervescence de leurs passions, en leur défendant l’abord des filles et des femmes de leurs sujets. Après de longues hésitations, ils voulurent secouer le joug, et, portant la main à leurs armes, déclarèrent la guerre aux prêtres, aux ermites, aux ascètes, aux philosophes dont l’œuvre avait épuisé leur patience. C’est ainsi qu’après avoir triomphé des hérétiques zoroas­triens et autres, après avoir vaincu la féroce inintelligence des indigènes, après avoir surmonté des difficultés de toute nature pour creuser au courant de chaque caste un lit contenu entre les digues de la loi et le contraindre à n’empiéter pas sur le lit des voisins, les brahmanes voyaient venir maintenant la guerre civile, et la guerre de l’espèce la plus dangereuse, puisqu’elle avait lieu entre l’homme armé et celui qui ne l’était pas \footnote{Lassen, \emph{ouvr. cité}, t. I, p. 719-720.}.\par
L’histoire du Malabar nous a conservé la date, sinon de la lutte en elle-même, du moins d’un de ses épisodes qui fut certainement parmi les principaux. Les annales de ce pays racontent qu’une grande querelle s’émut entre les kschattryas et les sages dans le nord de l’Inde, que tous les guerriers furent exterminés, et que les vainqueurs, conduits par Paraçou Rama, célèbre brahmane qu’il ne faut pas confondre avec le héros du Ramayana, vinrent, après leurs triomphes, s’établir sur la côte méridionale, et y consti­tuèrent un État républicain. La date de cet événement, qui fournit le commencement de l’ère malabare, est l’an 1176 av. J.-C. \footnote{Lassen, \emph{ouvr. cité}, t. I, p. 537.}.\par
Dans ce récit, il entre un peu de forfanterie. Généralement l’usage des plus forts n’est pas d’abandonner le champ de bataille, et surtout quand le vaincu est anéanti. Il est donc vraisemblable que, tout au rebours de ce que prétend leur chronique, les brahmanes furent battus et forcés de s’expatrier, et qu’en haine de la caste royale dont ils avaient dû subir l’insulte, ils adoptèrent la forme gouvernementale qui ne reconnaît pas l’unité du souverain.\par
Cette défaite ne fut, d’ailleurs, qu’un épisode de la guerre, et il y eut plus d’une rencontre où les brahmanes n’obtinrent pas l’avantage. Tout indique aussi que leurs adversaires, Arians presque autant qu’eux, ne se montrèrent pas dénués d’habileté, et qu’ils ne mirent pas dans la puissance de leurs épées une confiance tellement absolue, qu’ils n’aient cru nécessaire d’aiguiser encore des armes moins matérielles. Les kschattryas se placèrent très adroitement au sein même des ressources de l’ennemi, dans la citadelle théologique, soit afin d’émousser l’influence des brahmanes sur les vayçias, les çoudras et les indigènes, soit pour calmer leur propre conscience et éviter à leur entreprise un caractère d’impiété qui l’aurait rendue promptement odieuse à l’esprit profondément religieux de la nation.\par
On a vu que, pendant le séjour dans la Sogdiane et plus tard, l’ensemble des tribus zoroastriennes et hindoues professait un culte assez simple. S’il était plus chargé d’erreurs que celui des époques tout à fait primordiales de la race blanche, il était moins compliqué cependant que les notions religieuses des purohitas qui commencèrent le travail du brahmanisme. À mesure que la société hindoue gagnait de l’âge et qu’en conséquence le sang noir des aborigènes de l’ouest et du sud et le type jaune de l’est et du nord s’infiltraient davantage dans son sein, les besoins religieux auxquels il fallait répondre variaient et devenaient exigeants. Pour satisfaire l’élément noir, Ninive et l’Égypte nous ont appris déjà les concessions indispensables. C’était le commencement de la mort des nations arianes. Celles-ci avaient continué à être purement abstraites et morales, et bien que l’anthropomorphisme fût peut-être au fond des idées, il ne s’était pas encore manifesté. On disait que les dieux étaient beaux, beaux à la manière des héros arians. On n’avait pas songé à les portraire.\par
Quand les deux éléments noir et jaune eurent la parole, il fallut changer de système, il fallut que les dieux eux-mêmes sortissent du monde idéal dans lequel les Arians avaient trouvé du plaisir à laisser planer leurs sublimes essences. Quelles que pussent être les différences capitales existant, d’ailleurs, entre le type noir et le type jaune, sans avoir besoin de tenir compte, non plus, de ce fait que ce fut le premier qui parla d’abord et fut toujours écouté, tout ce qui était aborigène se réunit, non seulement pour vouloir voir et toucher les dieux qu’on lui vantait tant, mais aussi pour qu’ils lui apparussent plutôt terribles, farouches, bizarres et différents de l’homme, que beaux, doux, bénins, et ne se plaçant au-dessus de la créature humaine que par la perfection plus grande des formes de celle-ci. Cette doctrine eût été trop métaphysique au sens de la tourbe. Il est bien permis de croire aussi que l’inexpérience primitive des artistes la rendait plus difficile à réaliser. On voulut donc des idoles très laides et d’un aspect épouvantable. Voilà le côté de dépravation.\par
On a dit quelquefois, pour trouver une explication à ces bizarreries repoussantes des images païennes de l’Inde, de l’Assyrie et de l’Égypte, à ces obscénités hideuses où les imaginations des peuples orientaux se sont toujours complu, que la faute en revenait à une métaphysique abstruse, qui ne regardait pas tant à présenter aux yeux des monstruosités qu’à leur proposer des symboles propres à donner pâture aux considérations transcendantales. L’explication me paraît plus spécieuse que solide. Je trouve même qu’elle prête, bien gratuitement, un goût pervers aux esprits élevés qui, pour vouloir pénétrer les plus subtils mystères, ne sont cependant pas, \emph{ipso facto}, dans la nécessité absolue de rudoyer et d’avilir leurs sensations physiques. N’est-il pas moyen de recourir à des symboles qui ne soient pas répugnants ? Les puissances de la nature, les forces variées de la Divinité, ses attributs nombreux ne sauraient-ils être exprimés que par des comparaisons révoltantes ? Lorsque l’hellénisme a voulu produire la statue mystique de la triple Hécate, lui a-t-il donné trois têtes, six bras, six jambes, a-t-il contourné ses visages dans d’abominables contractions ? L’a-t-il assise sur un Cerbère immonde ? Lui a-t-il disposé sur la poitrine un collier de têtes et dans les mains des instruments de supplice souillés des marques d’un emploi récent ? Quand, à son tour, la foi chrétienne a représenté la Divinité triple et une, s’est-elle jetée dans les horreurs ? Pour montrer un saint Pierre, ouvrant à la fois le monde d’en haut et celui d’en bas, a-t-elle pris son recours à la caricature ? Nullement. L’hellénisme et la pensée catholique ont su parfaitement se dispenser d’en appeler à la laideur dans des sujets qui cependant n’étaient pas moins métaphysiques que les dogmes hindous, assyriens, égyptiens, les plus compliqués. Ainsi, ce n’est pas à la nature de l’idée abstraite en elle-même qu’il faut s’en prendre quand les images sont odieuses : c’est à la disposition des yeux, des esprits, des imaginations auxquelles doivent s’adresser les représentations figurées. Or, l’homme noir et l’homme jaune ne pouvaient bien comprendre que le laid : c’est pour eux que le laid fut inventé et resta toujours rigoureusement nécessaire.\par
 En même temps que chez les Hindous il fallait produire ainsi les personnifications théologiques, il était de même nécessaire de les multiplier afin, en les dédoublant, de leur faire présenter un sens plus clair et plus facile à saisir. Les dieux peu nombreux des âges primordiaux, Indra et ses compagnons, ne suffirent plus à rendre les séries d’idées qu’une civilisation de plus en plus vaste enfantait à profusion. Pour en citer un exemple, la notion de la richesse étant devenue plus familière à des masses qui avaient appris à en apprécier les causes et les effets, on mit ce puissant mobile social sous la garde d’un maître céleste, et on inventa \emph{Kouvéra}, déesse faite de manière à satisfaire pleinement le goût des noirs \footnote{Lassen, \emph{Indische Alterthumskunde}, t. I, p. 771. – Du reste, l’esprit brahmanique lutta longtemps avant d’en venir à l’anthropomorphisme, et c’est ainsi que M. de Schlegel paraît avoir eu toute raison de dire que les monuments hindous ne peuvent rivaliser d’antiquité avec ceux de l’Égypte. Il n’est pas autant dans le vrai, quand il ajoute : « Et ceux de la Nubie. » (A. W. V. Schlegel, \emph{Vorrede zur Darstellung der ægyptischen Mythologie} von Prichard, übersetzt von Haymann. Bonn, 1837), p. XIII.)}.\par
Dans cette multiplication des dieux il n’y avait cependant pas que de la grossièreté. À mesure que l’esprit brahmanique lui-même se raffinait, il faisait effort et cherchait à ressaisir l’antique vérité échappée jadis à la race ariane, et, en même temps qu’il créait des dieux inférieurs pour satisfaire les aborigènes ralliés, ou encore qu’il tolérait d’abord et acceptait ensuite des cultes autochtones, il montait de son côté. Il cherchait par en haut, et, imaginant des puissances, des entités célestes supérieures à Indra, à Agni, il découvrait Brahma, lui donnait le caractère le plus sublime que jamais philosophie humaine ait pu combiner, et, dans le monde de création sur-éthérée où son instinct des belles choses concevait un si grand être, il ne laissait pénétrer que peu d’idées qui en fussent indignes.\par
Brahma resta longtemps pour la foule un dieu inconnu. On ne le figura que très tard. Négligé des castes inférieures, qui ne le comprenaient ni ne s’en souciaient, il était par excellence le dieu particulier des ascètes, celui dont ils se réclamaient, qui faisait l’objet de leurs plus hautes études, et qu’ils n’avaient nulle pensée de détrôner jamais. Après avoir passé par toute la série des existences supérieures, après avoir été dieux eux-mêmes, tout ce qu’ils espéraient, c’était d’aller se confondre dans son sein et se reposer, un temps, des fatigues de la vie, lourde à porter pour eux, même dans les délices de l’existence céleste.\par
Si le dieu supérieur des brahmanes planait trop au-dessus de la compréhension étroite des classes inférieures et peut-être des vayçias eux-mêmes, il était cependant accessible au sens élevé des kschattryas, qui, restés participants de la science védique, avaient, sans doute, une piété moins active que leurs contemplatifs adversaires, mais possédaient assez de science avec assez de netteté d’esprit, pour ne pas heurter de front une notion dont ils appréciaient très bien la valeur. Ils prirent un biais, et, les théologiens militaires aidant, ou quelque brahmane déserteur, ils transformèrent la nature subalterne d’un dieu kschattrya jusque-là peu remarqué, Vischnou \footnote{Lassen \emph{Indische Alterth.}, t. I, p. 781.}, et, lui dressant un trône métaphysique, l’élevèrent aussi haut que le maître céleste de leurs ennemis. Placé alors en face et sur le même plan que Brahma, l’autel guerrier valut celui du rival et les guerriers n’eurent pas à s’humilier sous une supériorité de doctrine.\par
Un tel coup, bien médité sans doute, et longtemps réfléchi, car il accuse par les développements qui lui furent nécessaires la longueur et l’acharnement d’une lutte obstinée, menaçait le pouvoir des brahmanes, et, avec lui, la société hindoue, d’une ruine complète. D’un côté, aurait été Vischnou avec ses kschattryas libres et armés ; de l’autre, Brahma, égalé par un dieu nouveau, avec ses prêtres pacifiques, et les classes impuissantes des vayçias et des çoudras. Les aborigènes auraient été mis en demeure de choisir entre deux systèmes, dont le premier leur eût offert, avec une religion tout aussi complète que l’ancienne, une délivrance absolue de la tyrannie des castes et la perspective, pour le dernier des hommes, de parvenir à tout, pendant le cours même de la vie actuelle, sans avoir à attendre une seconde naissance. L’autre régime n’avait rien de nouveau à dire ; situation toujours défavorable quand il s’agit de plaider devant les masses ; et, de même qu’il ne pouvait pas accuser ses rivaux d’impiété, puisqu’ils reconnaissaient le même panthéon que lui, sauf un dieu supérieur différent, il ne pouvait non plus se poser, comme il l’avait fait jusqu’alors, en défenseur des droits des faibles, en libéral, comme on dirait aujourd’hui ; car le libéralisme était évidemment du côté de ceux qui promettaient tout aux plus humbles, et voulaient même leur accorder le rang suprême à l’occasion. Or, si les brahmanes perdaient la fidélité de leur monde noir, quels soldats auraient-ils à opposer au tranchant des épées royales, eux qui ne pouvaient payer de leur personne ?\par
Comment la difficulté fut traitée, c’est ce qu’il est impossible de saisir. Ce sont choses si vieilles, qu’on les devine plutôt qu’on ne les aperçoit au milieu des décombres mutilés de l’histoire. Il est toutefois évident que, dans les deux sommes de fautes que deux partis politiques belligérants ne manquent jamais de commettre, le chiffre le plus petit revient aux brahmanes. Ils eurent aussi le mérite de ne pas s’obstiner sur des détails, et de sauver le fond en sacrifiant beaucoup du reste. À la suite de longues discussions, prêtres et guerriers se raccommodèrent, et, s’il faut en juger sur l’événement, voici quels furent les termes du traité.\par
Brahma partagea le rang suprême avec Vischnou. De longues années après, d’autres révolutions dont je n’ai pas à parler, car elles n’ont pas un caractère directement ethnique, leur adjoignirent Siva \footnote{Au jugement de Lassen, cette divinité est originairement empruntée à quelque culte des aborigènes noirs. Dans le sud, on l’adore sous la forme du Linga, et un brahmane n’accepte jamais d’emploi dans les temples où elle se trouve. (\emph{Indische Alterth.}, t. I, .p. 783 et passim.)}, et, plus tard encore, une certaine doctrine philoso­phique, ayant fondu ces trois individualités divines en une trinité pourvue du caractère de la création, de la conservation et de la destruction, ramena, par ce détour, la théologie brahmanique à la primitive conception d’un dieu unique enveloppant l’univers \footnote{\emph{Ibid.}, t. I, p. 784.}.\par
Les brahmanes renoncèrent à occuper jamais le rang suprême, et les kschattryas le conservèrent comme un droit imprescriptible de leur naissance.\par
 Moyennant quoi, le régime des castes fut maintenu dans sa rigueur entière, et toute infraction conduisit résolument le fruit du crime à l’impureté des basses castes.\par
La société hindoue, scellée sur les bases choisies par les brahmanes, venait encore de passer heureusement une des crises les plus périlleuses, qu’elle pût subir. Elle avait acquis bien des forces, elle était homogène et n’avait qu’à poursuivre sa route : c’est ce qu’elle fit avec autant de suite que de succès. Elle colonisa, vers le sud, la plus grande partie des territoires fertiles, elle refoula les récalcitrants dans les déserts et les marais, sur les cimes glacées de l’Himalaya, au fond des monts Vyndhias. Elle occupa le Dekkhan, elle s’empara de Ceylan, et y porta sa culture avec ses colonies. Tout porte à croire qu’elle s’avança, dès lors, jusqu’aux îles lointaines de Java et de Bali \footnote{W. de Humboldt, \emph{Ueber die Kawi-Sprache}.} ; elle s’instilla aux bords inférieurs du Gange, et osa pénétrer le long du cours malsain du Brahmapoutra, au milieu des populations jaunes que, dès longtemps, elle avait connues sur quelques points du nord, de l’est, et dans les îles du sud \footnote{Les Arians n’ont jamais possédé dans l’Inde un territoire compact. Sur plusieurs points, des populations complètement aborigènes interrompent encore et isolent leurs établissements. Le Dekkhan est presque absolument privé de leurs colonisations. (Lassen, \emph{Indische Alterth.}, t. I, p. 391.)}.\par
Pendant que s’accomplissaient de tels travaux, d’autant plus difficiles que les régions étaient plus vastes, les distances plus longues, les difficultés naturelles bien autrement accumulées qu’en Égypte, un immense commerce maritime allait de toutes parts, en Chine, entre autres, et cela, d’après un calcul très vraisemblable, 1 400 ans avant J.-C., porter les magnifiques produits du sol, des mines et des manufactures, et rapporter ce que le Céleste Empire et les autres lieux civilisés du monde possédaient de plus excellent. Les marchands hindous fréquentaient de même Babylone \footnote{Le vayçia naviguait beaucoup. Une légende bouddhique cite un marchand qui avait fait sept voyages sur mer. (Burnouf, \emph{Introduction à l’histoire du bouddhisme indien}, t. I, p. 196.) - Les Hindous pouvaient ainsi se mettre en communication avec les Chaldéens, qui avaient eux-mêmes une marine (Isaïe, XLIII, 14) et une colonie à Gerrha sur la côte occidentale du golfe Persique, où se faisait un grand commerce avec l’Inde. Les Phéniciens, avant et après leur départ de Tylos, y prenaient part. – L’Ophir des livres saints était sur la côte de Malabar (Lassen, \emph{Indische Alterth.}, t. I, p. 539), et, comme les noms hébraïques des marchandises qui en provenaient sont sanscrits et non dekkhaniens, il s’ensuit que les hautes castes du pays étaient arianes au temps où les vaisseaux de Salomon les visitaient. (\emph{Ibid.}) Il faut aussi remarquer ici que les plus anciennes colonisations arianes, dans le sud de l’Inde, eurent lieu sur les côtes de la mer, ce qui indique clairement que leurs fondateurs étaient, en même temps, des navigateurs. (\emph{Ouvrage cité}, p. 537). Il est très probable qu’arrivés de bonne heure aux embouchures de l’Indus, ils y établirent leurs premiers empires, tels que celui de Pôtâla. (\emph{Ibid}., p. 543.)}. Sur la côte de l’Yémen, leur séjour était, pour ainsi dire, permanent. Aussi les brillants États de leur péninsule regorgeaient de trésors, de magnificences et de plaisirs, résultats d’une civilisation développée sous des règles strictes à la vérité, mais que le caractère national rendait douces et paternelles. C’est, du moins, le sentiment qu’on éprouve à la lecture des grandes épopées historiques et des légendes religieuses fournies par le bouddhisme.\par
La civilisation ne se bornait pas à ces brillants effets externes. Fille de la science théologique, elle avait puisé à cette source le génie des plus grandes choses, et on peut dire d’elle ce que les alchimistes du moyen âge pensaient du grand œuvre, dont le moindre mérite était de faire de l’or. Avec tous ses prodiges, avec tous ses travaux, avec ses revers si noblement supportés, ses victoires si sagement mises à profit, la civilisa­tion hindoue considérait comme la moindre partie d’elle-même ce qu’elle accomplissait de positif et de visible, et, à ses yeux, ses seuls triomphes dignes d’estime commen­çaient au delà du tombeau.\par
Là était le grand point de l’institution brahmanique. En établissant les catégories dans lesquelles elle divisait l’humanité, elle se faisait fort de se servir de chacune pour perfectionner l’homme, et l’envoyer, à travers le redoutable passage dont l’agonie est la porte, soit à une destinée supérieure, s’il avait bien vécu, soit, dans le cas contraire, à un état dont l’infériorité donnait du temps au repentir. Et quelle n’est pas la puissance de cette conception sur l’esprit du croyant, puisque aujourd’hui même l’Hindou des castes les plus viles, soutenu, presque enorgueilli par l’espérance de renaître à un rang meilleur, méprise le maître européen qui le paye, ou le musulman qui le frappe, avec autant d’amertume et de sincérité que peut le faire un kschattrya ?\par
La mort et le jugement d’outre-tombe sont donc les grands points de la vie d’un Hindou, et on peut dire, à l’indifférence avec laquelle il porte communément l’existence présente, qu’il n’existe que pour mourir. Il y a là des similitudes évidentes avec cet esprit sépulcral de l’Égypte, tout porté vers la vie future, la devinant et, en quelque façon, l’arrangeant à l’avance. Le parallèle est facile, ou mieux, les deux ordres d’idées se coupent à angle droit et partent d’un sommet commun. Ce dédain de l’existence, cette foi solide et délibérée dans les promesses religieuses, donnent à l’histoire d’une nation une logique, une fermeté, une indépendance, une sublimité que rien n’égale. Quand l’homme vit à la fois, par la pensée, dans les deux mondes, et, en embrassant de l’œil et de l’esprit ce que les horizons du tombeau ont de plus sombre pour l’incrédule, les illumine d’éclatantes espérances, il est peu retenu par les craintes ordinaires aux sociétés rationalistes, et, dans la poursuite des affaires d’ici-bas, il ne compte plus parmi les obstacles la crainte d’un trépas qui n’est qu’un passage d’habitude. Le plus illustre moment des civilisations humaines est celui où la vie n’est pas encore cotée si haut qu’on ne place, avant le besoin de la conserver, bien d’autres soucis plus utiles aux individus. D’où dépend cette disposition heureuse ? Nous la verrons toujours et partout corrélative à la plus ou moins grande abondance de sang arian dans les veines d’un peuple.\par
La théologie et les recherches métaphysiques furent donc le pivot de la société hindoue. De là sortirent, sans s’en détacher jamais, les sciences politiques, les sciences sociales. Le brahmanisme ne fit pas deux parts spéciales de la conscience du citoyen et de celle du croyant. La théorie chinoise et européenne de la séparation de l’Église et de l’État ne fut jamais admissible pour lui. Sans religion, point de société brahmanique. Pas un seul acte de la vie privée ne s’en isolait. Elle était tout, pénétrait partout, vivifiait tout et d’une manière bien puissante, puisqu’elle relevait le tchandala lui-même, tout en l’abaissant, et donnait même à ce misérable un motif d’orgueil et des inférieurs à mépriser.\par
 Sous l’égide de la science et de la foi, la poésie des soutas avait aussi trouvé d’illustres imitateurs dans les ermitages sacrés. Les anachorètes, descendus des hauteurs inouïes de leurs méditations, protégeaient les poètes profanes, les excitaient et savaient même les devancer. Valmiki, l’auteur du Ramayana, fut un ascète vénéré. Les deux rapsodes auxquels il confia le soin d’apprendre et de répéter ses vers, étaient des kschattryas, Cuso et Lavo, fils de Rama lui-même. Les cours des rois du pays accueil­laient avec feu les jouissances intellectuelles, une partie des brahmanes se consacra bientôt au seul emploi de leur en procurer \footnote{Burnouf, \emph{ouvr. cité}, t. I, p. 141.}. Les poèmes, les élégies, les récits de toute nature vinrent se placer auprès des élucubrations volumineuses des sciences austères \footnote{La critique littéraire a existé de très bonne heure dans l’Inde. Vers le XI\textsuperscript{e} siècle avant notre ère, les hymnes védiques de l’Atharvan furent réunies et mises en ordre. Au VI\textsuperscript{e} siècle parurent les grammairiens, qui étudièrent et classèrent le langage de toutes les nations habitant le territoire sacré ou ses frontières. Ce travail philologique et les résultats qu’il consacre sont du plus précieux secours pour l’ethnologie. À cette même époque, le langage des Védas fut si parfaitement fixé, que l’on ne trouve, ni dans les manuscrits ni dans les citations, la moindre variante. (Lassen, \emph{Indische Alterth.}, t. I, p. 739 et 756 et passim.)}. Sur une scène illustrée par les génies les plus magnifiques, le drame et la comédie représentèrent, avec éclat, les mœurs des temps présents et les actions les plus grandioses des époques passées. Certes, le grand nom de Kalidasa mérite de briller à l’égal des plus illustres mémoires dont s’enorgueillissent les fastes littéraires \footnote{Les Hindous n’ont pas eu la même manière que nous d’envisager l’histoire, de sorte que, bien que nous ayant conservé les souvenirs les plus remarquables des faits, des caractères et des habitudes de leurs plus anciens ancêtres, ils ne nous fournissent pas d’ouvrage vraiment méthodique à ce sujet. M. Jules Mohl a très bien constaté et apprécié cette particularité remarquable : « On sait, dit cet admirable juge des choses asiatiques, que l’Inde n’a pas « produit d’historien, ni même de chroniqueur. La littérature sanscrite ne manque pas pour « cela de données historiques ; elle est plus riche, peut-être, que toute autre littérature en « renseignements sur l’histoire morale de la nation, sur l’origine et le développement de ses « idées et de ses institutions, enfin sur tout ce qui forme le cœur, comme le noyau de « l’histoire de ce que les chroniqueurs de la plupart des peuples négligent pour se contenter « de l’écorce. Mais, comme dit Albirouni : « Ils ont toujours négligé de rédiger les « chroniques des règnes de leurs rois. » De sorte que nous ne savons jamais exactement « quand leurs dynasties commencent et quand elles finissent, ni sur quels pays elles ont « régné. Leurs généalogies sont en mauvais ordre et leur chronologie est nulle. » (\emph{Rapport annuel fait à la Société asiatique}, 1849, p. 26-27.)}. À côté de cet homme illustre, plusieurs encore créaient ces chefs-d’œuvre recueillis en partie par le savant Wilson, dans son \emph{Théâtre indien}, et, bref, l’amour des plaisirs intellec­tuels, d’une part, et celui des profits qu’il rapportait, de l’autre, avaient fini par créer, dans ce monde antique, le métier d’homme de lettres, comme nous le voyons pratiquer sous nos yeux depuis trente ans environ, non pas tout à fait dans la même forme quant aux productions, mais sans la moindre différence quant à l’esprit \footnote{C’est probablement à l’école de ces littérateurs que se formaient les poètes du genre de celui qui a écrit le \emph{Hásyarnavah} (\emph{l’Océan des plaisanteries}). C’est une comédie très mordante dirigée contre les rois, les hommes de cour et les prêtres. Les uns sont traités de fainéants inutiles et les autres d’hypocrites. (W, v. Schlegel, \emph{Indische Bibliothek}, t. III, p. 161.)}. Je n’en veux pour démonstration qu’une courte anecdote que je citerai, afin d’ouvrir aussi une échappée de vue sur le côté familier de cette grande civilisation.\par
Un brahmane faisait le métier que je dis, et, soit qu’il y gagnât peu, ou peut-être qu’il dépensât trop, il se trouvait à court d’argent. Sa femme lui conseilla d’aller se mettre sur le passage du rajah et, aussitôt qu’il le verrait sortir de son palais, de s’avancer hardiment et de lui réciter quelque chose qui lui pût être agréable.\par
Le poète trouva l’idée ingénieuse, et, suivant le conseil de la brahmani, il rencontra le roi au moment où celui-ci allait faire sa promenade, assis sur le dos de son éléphant. L’auteur vénal ne se piquait pas d’un grand respect. « Qui des deux louerai-je ? se dit-il. Cet éléphant est cher et agréable au peuple ; laissons là le roi, je vais chanter l’éléphant \footnote{Burnouf\emph{, ouvr. cité}, t. I, p. 140.}. »\par
Voilà le laisser-aller de ce qu’on nomme aujourd’hui la vie d’artiste ou de journaliste, avec cette différence que le danger n’en était pas grand au milieu des barrières qui encadraient tous les chemins. Je ne répondrais pas cependant que ces façons d’indé­pendance, séduisant quelques esprits, n’aient contribué à amener la dernière grande insurrection et une des plus dangereuses, à coup sûr, que le brahmanisme ait eu à subir. Je veux parler de la naissance des doctrines bouddhiques et de l’application politique qu’elles essayèrent.
\section[{III.3. Le bouddhisme, sa défaite ; l’Inde actuelle.}]{III.3. \\
Le bouddhisme, sa défaite ; l’Inde actuelle.}
\noindent On était arrivé à une époque qui, suivant le comput cinghalais, concorderait avec le VII\textsuperscript{e} siècle avant J.-C. \footnote{Burnouf, \emph{ouvr. cité}, p. 287.}, et suivant d’autres calculs bouddhiques dressés pour le nord de l’Inde, descendrait jusqu’à l’an 543 avant notre ère \footnote{Lassen, \emph{Indische Alterth}., t. I, p. 356 et 711. – C’est à l’époque de Cyrus. Vers le même temps, Scylax exécuta son périple de la mer Érythrée, et rapporta dans l’occident les premières notions sur les pays hindous que recueillirent Hécatée et Hérodote par l’intermédiaire des Perses. – L’Inde était, à ce moment, à l’apogée de sa civilisation et de sa puissance. (Burnouf\emph{, ouvr. cité}, t. I, p. 131 .)}. Depuis quelque temps déjà, des idées très dangereuses s’étaient glissées dans cette branche de la science hindoue qui porte le nom de philosophie sankhya. Deux brahmanes, Patandjali et Kapila, avaient enseigné que les œuvres ordonnées par les Védas étaient inutiles de soi au perfection­nement des créatures, et que, pour arriver aux existences supérieures, il suffisait de la pratique d’un ascétisme individuel et arbitraire. Par cette doctrine, on était mis en droit, sans inconvénient pour l’avenir du tombeau, de mépriser tout ce que le brahmanisme recommandait et de faire ce qu’il prohibait \footnote{Burnouf, \emph{op. cit}., p. 152 et passim. et 211.}.\par
Une telle théorie pouvait renverser la société. Cependant, comme elle ne se présentait que sous une forme purement scientifique et ne se communiquait que dans les écoles, elle resta matière à discussion pour les érudits et ne descendit pas dans la politique. Mais, soit que les idées qui lui avaient donné naissance fussent quelque chose de plus que la découverte accidentelle d’un esprit chercheur, ou bien que des hommes très pratiques en aient eu connaissance, il se trouva qu’un jeune prince, de la plus illustre origine, appartenant à une branche de la race solaire, Sakya, fils de Çuddodhana, roi de Kapilavastu, entreprit d’initier les populations à ce que cette doctrine avait de libéral.\par
Il se mit à enseigner, comme Kapila, que les oeuvres védiques étaient sans valeur ; il ajouta que ce n’était ni par les lectures liturgiques, ni par les austérités et les supplices, ni par le respect des classifications, qu’il était possible de s’affranchir des entraves de l’existence actuelle ; que, pour cela, il ne fallait avoir recours qu’à l’observance des lois morales, dans lesquelles on était d’autant plus parfait qu’on s’occupait moins de soi et plus d’autrui. Comme vertus supérieures et d’une efficacité incomparable, il proclama la libéralité, la continence, la science, l’énergie, la patience et la miséricorde. Il acceptait, du reste, en fait de théologie et de cosmogonie, tout ce que le brahmanisme savait, hors un dernier point, sur lequel il avait la prétention de promettre beaucoup plus que la loi régulière. Il affirmait pouvoir conduire les hommes, non seulement dans le sein de Brahma, d’où, après un temps, l’ancienne théologie enseignait que, par suite de l’épuise­ment des mérites, il fallait sortir pour recommencer la série des existences terrestres, mais dans l’essence du Bouddha parfait, où l’on trouvait le nirwana, c’est-à-dire le complet et éternel néant. Ainsi le brahmanisme était un panthéisme très compliqué, et le bouddhisme le compliquait encore en le faisant poursuivre sa route jusqu’à l’abîme de la négation \footnote{Lassen, \emph{Indische Alterth}., t. I, p. 831 ; Burnouf, \emph{Introduction à l’hist. du bouddhisme indien}, t. I, p. 152 et passim.}.\par
Maintenant, comment Sakya produisait-il ses idées et cherchait-il à les répandre ? Il commença par renoncer au trône ; il se couvrit d’une robe de grosse toile commune et jaune, composée de haillons qu’il avait recueillis lui-même dans les bourriers, dans les cimetières, et cousus de sa main ; il prit un bâton et une écuelle, et désormais ne mangea plus que ce que l’aumône voulut lui donner. Il s’arrêtait sur les places publiques des villes et des villages et prêchait sa doctrine morale \footnote{Burnouf, Introd. à l’hist. du bouddh. indien, t. I, p. 194.}. Se trouvait-il là des brahmanes, il faisait avec eux assaut de science et de subtilité, et les assistants écou­taient, pendant des heures entières, une polémique qu’enflammait la conviction égale des antagonistes. Bientôt il eut des disciples. Il en recruta beaucoup dans la caste militaire, peut-être plus encore dans celle des vayçias, alors bien puissante et bien honorée, comme fort riche. Quelques brahmanes vinrent aussi à lui. Ce fut surtout dans le bas peuple qu’il enrôla ses plus nombreux prosélytes \footnote{Un de ses principaux arguments à l’adresse des hommes des basses castes était de leur dire que, dans leurs existences antérieures, ils avaient fait partie des plus hautes, et que, par le seul fait qu’ils l’écoutaient, ils étaient dignes d’y rentrer. (Burnouf, \emph{ouvr. cité}, t. I, p. 196.)}. Du moment qu’il avait repoussé les prescriptions des Védas, les séparations des castes n’existaient plus pour lui et il déclarait ne reconnaître d’autre supériorité que celle de la vertu \footnote{\emph{Ouvrage cité}, t. I, p. 211.}.\par
Un de ses premiers disciples et des plus dévoués, Ananda, son cousin, kschattrya d’une grande famille, revenant un jour d’une longue course dans les campagnes, accablé de fatigue et de chaleur, s’approche d’un puits où il voit une jeune fille occupée à tirer de l’eau. Il exprime le désir d’en avoir. Celle-ci s’excuse, en lui faisant observer qu’en lui rendant ce service elle le souillerait, étant de la tribu matanghi, de la caste des tchandalas. « Je ne te demande, ma sœur, lui répond Ananda, ni ta caste ni ta famille, mais seulement de l’eau, si tu peux m’en donner \footnote{Burnouf, \emph{Introd. à l’hist}., etc., t. I, p. 205.}. »\par
Il prit la cruche et but, et, pour porter de la liberté de ses idées un témoignage plus éclatant encore, quelque temps après il épousa la tchandala. Que des novateurs de cette force exerçassent de la puissance sur l’imagination du bas peuple, on le conçoit aisément. Les prédications de Sakya convertirent un nombre infini de personnes, et, après sa mort, des disciples ardents, poursuivant son œuvre de tous côtés, en étendirent les succès bien au delà des bornes de l’Inde, où des rois se firent bouddhistes avec toute leur maison et leur cour.\par
Cependant l’organisation brahmanique était tellement puissante, que la réforme n’osa pas, dans la pratique, se montrer aussi hostile ni aussi téméraire que dans la théorie. On niait bien, en principe, et souvent même en action, la nécessité religieuse des castes. En politique, on n’avait pu trouver le moyen de s’y soustraire. Qu’Ananda épousât une fille impure, c’était de quoi se faire applaudir de ses amis, mais non pas empêcher ses enfants d’être impurs à leur tour. En tant que bouddhistes, ils pouvaient devenir des bouddhas parfaits et être en grande vénération dans leur secte ; en tant que citoyens, ils n’avaient que justement les droits et la position assignés à leur naissance. Aussi, malgré le grand ébranlement dogmatique, la société menacée n’était pas sérieusement entamée \footnote{Les éléments révolutionnaires ne manquaient pas absolument dans ce monde hindou où les classes moyennes, les chefs de métiers, les marchands, les chefs de marins, avaient acquis une importance extraordinaire. Mais l’édifice était si bien cimenté, qu’il pouvait résister à tout. – Voir Burnouf, \emph{ouvr. cité}, t. I, p. 163, où il est fait mention d’une légende bouddhique qui met bien en relief la puissance de la bourgeoise vayçia à l’époque où se forma le bouddhisme. Je remarquerai ici que, pour ces temps de l’histoire hindoue, les légendes des bouddhas ont le même genre d’intérêt historique que, chez nous, les vies des saints, lorsqu’il s’agit des âges de la domination mérovingienne. Ces productions, d’une piété également vive, bien que différemment appliquée, se ressemblent de très près. Elles racontent les mœurs, les usages du temps où le vénérable personnage dont elles s’occupent a vécu, et ont, les unes et les autres, celles des Arians-Franks, comme celles des Arians-Hindous, la même prédilection pour la partie philosophique de l’histoire, unie au même dédain de la chronologie.}.\par
Cette situation se prolongea d’une manière qui prouve, à elle seule, la vigueur extraordinaire de l’organisation brahmanique. Deux cents ans après la mort de Sakya, et dans un royaume gouverné par le roi bouddhiste Pyadassi, les édits ne manquaient jamais de donner le pas aux brahmanes sur leurs rivaux \footnote{Burnouf, \emph{Introduct.} à \emph{l’hist.}, etc., t. I, p. 395, note.}, et la guerre véritable, la guerre d’intolérance, la persécution ne commença qu’avec le V\textsuperscript{e} siècle de notre ère \footnote{\emph{Ibid}., p. 586.}. Ainsi le bouddhisme avait pu vivre pendant près de huit cents ans, à tout le moins, côte à côte avec l’antique régulateur du sol, sans parvenir à se rendre assez fort pour l’inquiéter et le faire courir aux armes.\par
 Ce n’était pas faute de bonne volonté. Les conversions dans les basses classes avaient toujours été en augmentant. À l’appel d’une doctrine qui, prétendant ne tenir compte que de la valeur morale des hommes, leur disait : « Par ce seul fait que vous m’accueillez, je vous relève de votre abaissement en ce monde », tout ce qui ne voulait ou ne pouvait obtenir naturellement un rang social était fortement tenté d’accourir. Puis, dans les brahmanes il y avait des hommes sans science, sans considération ; dans les kschattryas, des guerriers qui ne savaient pas se battre ; dans les vayçias, des dissipateurs regrettant leur fortune, et trop paresseux ou trop nuls pour s’en refaire une autre par le travail \footnote{Quand les brahmanes reprochaient à Sakya de s’entourer de gens appartenant aux castes impures ou de personnes de mauvaise vie, Sakya répondait : « Ma loi est une loi de grâce pour tous. » (Burnouf, \emph{ouvr. cité}, t. I, p. 198.) – Cette loi de grâce devint très promptement une sorte de religiosité facile qui recrutait des partisans dans les classes supérieures, parmi les hommes dégoûtés de toutes les restrictions que le régime brahmanique inflige à ses fidèles, par suite de cette idée qu’on ne peut se faire pardonner les fautes de l’existence actuelle et se rendre dignes de passer dans un rang supérieur, qu’au prix des plus redoutables austérités. Ainsi, un jeune ascète, après de longues abstinences au fond d’une forêt, se donne en pâture a une tigresse, qui vient de mettre bas, en s’écriant : « Comme il « est vrai que je n’abandonne la vie ni pour la royauté, ni pour les jouissances du plaisir, ni « pour le rang de sakya, ni pour celui de monarque souverain, mais bien pour arriver à « l’état suprême de bouddha parfaitement accompli ! » (Burnouf, \emph{ibid.}, p. 159 et passim.) – Les bouddhistes prenaient les choses d’une façon plus commode. Ils condamnaient ces rigueurs personnelles comme inutiles, et leur substituaient le simple repentir et l’aveu de la faute, ce qui, du reste, les fit arriver très promptement à instituer la confession. (\emph{Ibid}., p. 299.)}. Toutes ces accessions donnaient du relief à la secte en la répandant dans les hautes classes, et il était, en somme, aussi flatteur que facile de se glorifier de vertus intimes et inaperçues, de débiter des discours de morale, et aussitôt d’être tenu pour saint et quitte du reste \footnote{Burnouf, \emph{Introd. à l’hist}., etc., t. I, p. 196, 277.}.\par
Les couvents se multiplièrent. Des religieux et des religieuses remplirent ces asiles appelés \emph{viharas}, et les arts, que l’antique civilisation avait formés et élevés, prêtèrent leur concours à la glorification de la nouvelle secte \footnote{\emph{Ibid}., p. 287.}. Les cavernes de Magatanie, de Baug, sur la route d’Oudjeïn, les grottes d’Eléphanta sont des temples bouddhiques. Il en est d’aussi extraordinaires par la vaste étendue des proportions que par le fini précieux des détails. Tout le panthéon brahmanique, doublé de la nouvelle mythologie qui vint s’enter sur ses rameaux, de tous les bouddhas, de tous les boddhisatvas et autres inventions d’une imagination d’autant plus féconde qu’elle plongeait davantage dans les classes noires, tout ce que la pensée humaine, ivre de raffinements et complètement déroutée par l’abus de la réflexion, a jamais pu imaginer d’extravagant en fait de formes, vint trôner sous ces splendides asiles \footnote{Burnouf, \emph{Introduction à l’hist.}, etc., t. I, p. 337. – Le bouddhisme hindou est aujourd’hui tellement dégénéré dans les provinces lointaines où il végète encore, que les religieux se marient, usage diamétralement opposé à l’esprit de la foi fondamentale. Ces religieux mariés se nomment au Népaul \emph{vadira âtchâryas.} (\emph{Ibid}.)}. Il était temps, pour peu que les brahmanes voulussent sauver leur société, de se mettre à l’œuvre. La lutte s’engagea, et, si l’on compare le temps du combat à celui de la patience, l’un fut aussi long que l’autre. La guerre commencée au V\textsuperscript{e} siècle se termine au XIV\textsuperscript{e} \footnote{Burnouf, \emph{Introd. à l’hist.}, etc., t. I, p. 586.}.\par
 Autant qu’on peut en juger, le bouddhisme mérita d’être vaincu, parce qu’il recula devant ses conséquences. Sensible, de bonne heure, au reproche, évidemment très mérité, de démentir ses prétentions à la perfection morale en se recrutant de tous les gens perdus, il s’était laissé persuader d’admettre des motifs d’exclusion physiques et moraux. Par là, il n’était déjà plus la religion universelle, et se fermait les accessions les plus nombreuses, si elles n’étaient pas les plus honorables. En outre, comme il n’avait pas pu détruire, de prime abord, les castes, et qu’il avait été obligé de les reconnaître de fait, tout en les niant en théorie, il avait dû, dans son propre sein, compter avec elles \footnote{Ibid., p. 144. – Il fit plus que de les admettre en pratique. Il se montra faible au point de donner un démenti à sa prétention d’être une loi de grâce pour tous, en avouant que les boddhissatvas ne pouvaient s’incarner que dans des familles de brahmanes ou de kschattryas. (Ibid.)}. Les rois kschattryas et fiers de l’être bien que bouddhistes, les brahmanes convertis et qui n’avaient rien à gagner, les uns et les autres, à la nouvelle foi, si ce n’est la dignité de bouddha et l’anéantissement parfait, devaient, tôt ou tard, soit par eux, soit par leurs descendants, éprouver, en mille circonstances, des tentations violentes de rompre avec la tourbe qui s’égalait à eux, et de reprendre la plénitude de leurs anciens honneurs.\par
De cent façons le bouddhisme perdit du terrain ; au XI\textsuperscript{e} siècle, il disparut tout à fait du sol de l’Inde. Il se réfugia dans des colonies, comme Ceylan ou Java, que la culture brahmanique avait sans doute formées, mais où, par l’infériorité ethnique des prêtres et des guerriers, la lutte put continuer indécise et même se terminer à l’avantage des hérétiques. Le culte dissident trouva encore un asile dans le nord-est de l’Inde, où cependant, comme au Népaul, on le voit aujourd’hui, dégénéré et sans forces, reculer devant le brahmanisme. En somme, il ne fut vraiment à l’aise que là où il ne rencontra pas de castes, en Chine, dans l’Annam, au Thibet, dans l’Asie centrale. Il s’y déploya à son aise, et, contrairement à l’avis de quelques critiques superficiels, il faut avouer que l’examen ne lui est pas favorable et montre d’une manière éclatante le peu que réussit à produire, pour les hommes et pour les sociétés, une doctrine politique et religieuse qui se pique d’être basée uniquement sur la morale et la raison.\par
Bientôt l’expérience démontre combien cette prétention est vaine et creuse. Comme le bouddhisme, la doctrine incomplète veut réparer sa faute en se donnant, après coup, des fondements. Il est trop tard, elle ne crée qu’absurdités. Procédant à l’inverse de ce qui se voit dans les véritables philosophies, au lieu de faire que la loi morale découle de l’ontologie, c’est, au contraire, l’ontologie qui découle de la loi morale \footnote{M. Burnouf se sert très habilement de la postériorité de l’ontologie dans le bouddhisme pour établir l’âge de ce système religieux (\emph{Ouvr. cité}, t. I, p. 132.)}. De là, encore plus de non-sens, s’il est possible, que dans le brahmanisme dégénéré, qui en contient tant. De là, une théologie sans âme, toute factice, et les niaiseries du cylindre de prières, qui, placardé de manuscrits d’oraisons et mis en rotation perpétuelle par une force hydraulique, est censé envoyer au ciel l’esprit pieux contenu sous les lettres, et en réjouir les intelligences suprêmes \footnote{Voir les détails nombreux sur ce cylindre, très en usage chez les Mongols, dans les \emph{Souvenirs d’un voyage dans la Tartarie, le Thibet et la Chine, pendant les années} 1844, 1845 et 1846 (Paris, 1850), par M. Huc, prêtre missionnaire de la congrégation de Saint-Lazare. – Voir aussi, dans le même ouvrage, ce qui a rapport à la réforme moderne du bouddhisme lamaïque, appelée réforme de Tsong-Kaba, et qui date du XVII\textsuperscript{e} siècle. L’esprit hindou, dont il restait peu, a été presque absolument expulsé par ces innovations.}. À quel point d’avilissement tombe bientôt une théorie rationaliste qui s’aventure hors des écoles et va entreprendre la conduite des peuples ! Le bouddhisme le montre pleinement, et l’on peut dire que les multitudes immenses dont il dirige les consciences appartiennent aux classes les plus viles de la Chine et des pays circonvoisins. Telle fut sa fin, tel est son sort actuel.\par
Le brahmanisme ne fit pas que profiter des infirmités et des fautes de son ennemi. Il eut aussi des bénéfices d’habileté, et il suivit, en ces circonstances, la même politique dont il avait déjà usé avec succès lors de la révolte des kschattryas. Il sut pardonner et accorder les concessions indispensables. Il ne voulut pas violenter les consciences ou les humilier. Il imagina, au moyen d’un syncrétisme accommodant, de faire du bouddha Sakya-mouni une incarnation de Vischnou. De cette façon, il permettait à ceux qui voulaient revenir à lui de toujours vénérer leur idole, et leur épargnait ce que les conversions ont de plus amer, le mépris de ce que l’on a adoré. Puis, peu à peu, son panthéon accueillit beaucoup de divinités bouddhiques, avec cette seule réserve, que ces dernières venues n’occupèrent que des rangs inférieurs. Enfin il manœuvra de telle sorte qu’aujourd’hui le bouddhisme est aussi bien non avenu dans l’Inde que s’il n’y avait jamais existé. Les monuments sortis des mains de cette secte passent, dans l’opinion générale, pour l’œuvre de son rival heureux \footnote{Burnouf, \emph{ouvr. cité}, t. I, p. 339. – Bouddha, considéré comme une incarnation de Vischnou, est une idée qui ne remonte pas plus haut que l’an 1005 de l’ère de Vikramâditya, 943 de la nôtre.}. L’opinion publique ne les dispute pas au vainqueur, tellement que l’adversaire est mort, sa dépouille est restée aux brahmanes, et le retour des esprits est aussi complet que possible. Que dire de la puissance, de la patience et de l’habileté d’une école qui, après une campagne de près de deux mille ans, sinon plus, remporta une victoire semblable ? Pour moi, je l’avoue, je ne vois rien d’aussi extraordinaire dans l’histoire, et je ne sache rien, non plus, qui fasse autant d’honneur à l’autorité de l’esprit humain.\par
Que doit-on ici admirer davantage ? Est-ce la ténacité avec laquelle le brahmanisme se conserva, pendant cet énorme laps de temps, parfaitement pareil à lui-même dans ses dogmes essentiels et dans ce que son système politique avait de plus vital, sans jamais transiger sur ces deux terrains ? Est-ce, au contraire, sa condescendance à rendre hommage à la partie honorifique des idées de son adversaire et à désintéresser l’amour-propre au moment suprême de la défaite ? Je n’oserais en décider. Le brahmanisme montra, pendant cette longue contestation, ce double genre d’habileté, loué jadis avec tant de raison dans l’aristocratie anglaise, de savoir maintenir le passé en s’accom­modant aux exigences du présent. Bref, il fut animé d’un véritable esprit de gouvernement, et il en reçut la récompense par le salut de la société qui était son œuvre.\par
Son triomphe, il le dut surtout à ce bonheur d’avoir été compact, ce qui manquait au bouddhisme. L’excellence du sang arian était aussi beaucoup plus de son côté que de celui de ses adversaires qui, recrutés principalement dans les basses castes et moins strictement attachés aux lois de séparation dont ils niaient la valeur religieuse, offraient, au point de vue ethnique, des qualités très inférieures. Le brahmanisme représentait, dans l’Inde, la juste suprématie du principe blanc, bien que très altéré, et les boud­dhistes essayaient, au contraire, une protestation des rangs inférieurs. Cette révolte ne pouvait réussir tant que le type arian, malgré ses souillures, conservait encore, au moyen de son isolement, la majeure partie de ses vertus spéciales. Il ne s’ensuit pas, il est vrai, que la longue résistance des bouddhistes n’ait pas eu des résultats : loin de là. Je ne doute pas que la rentrée au sein brahmanique de nombreuses tribus de la caste sacerdotale et de kschattryas médiocrement fidèles, pendant tant de siècles, aux prescriptions ethniques, n’ait considérablement développé les germes fâcheux qui existaient déjà. Cependant la nature ariane était assez forte, et l’est encore aujourd’hui, pour maintenir debout son organisation au milieu des plus terribles épreuves que jamais peuple ait traversées.\par
Dès l’an 1001 de notre ère, l’Inde avait cessé d’être ce pays fermé aux nations occidentales, dont le plus grand des conquérants, Alexandre lui-même, n’avait pu que soupçonner les merveilles chez les peuples impurs, chez les nations vratyas de l’ouest qu’il avait combattues \footnote{Lassen, \emph{Indische Alterth.}, t. I, p. 353.}. Le fils de Philippe n’avait pas touché au territoire sacré. Un prince musulman de race mélangée, beaucoup plus blanche que ne l’était devenu l’alliage d’où sortent maintenant les brahmanes et les kschattryas, Mahmoud le Gnaznévide, à la tête d’armées qu’animait le fanatisme musulman, promena le fer et le feu sur la péninsule, détruisit les temples, persécuta les prêtres, massacra les guerriers, s’en prit aux livres et commença, sur une vaste échelle, une persécution qui, dès lors, n’a jamais complètement cessé. S’il est difficile à toute civilisation de se tenir debout contre les assauts intérieurs que les passions humaines lui livrent constamment, qu’est-ce donc lorsqu’elle est, non seulement attaquée, mais possédée par des étrangers qui ne l’épargnent pas et n’ont pas de plus cher souci que d’amener sa perte ? Est-il, dans l’histoire, un exemple de résistance heureuse et longue à cette terrible conspiration ? Je n’en connais qu’un seul, et c’est dans l’Inde que je le trouve, Depuis le rude sultan de Ghizni, on peut affirmer que la société brahmanique n’a pas joui d’un moment de tranquillité et, au milieu de ces attaques constantes, elle a gardé la force d’expulser le bouddhisme. Après les Persans de Mahmoud sont venus les Turcs, les Mongols, les Afghans, les Tatares, les Arabes, les Abyssins, puis de nouveau les Persans de Nadir-Schah, les Portugais, les Anglais, les Français. Au nord, à l’ouest, au sud, des routes d’invasions incessantes se sont ouvertes, des nuées disparates de populations étrangères sont venues couvrir les provinces. Contraintes par le sabre, des nations entières ont fait défection à la religion nationale. Les Kachemyriens sont devenus musulmans ; les Syndhis aussi, encore d’autres groupes du Malabar et de la côte de Coromandel. Partout les apôtres de Mahomet, favorisés par les princes de la conquête, ont prodigué, et non sans succès, des prédications redoutées. Le brahmanisme n’a pas un instant renoncé au combat, et l’on sait, au contraire, que dans l’est, dans les montagnes du nord, notamment depuis la conquête du Népaul par les Gorkhas au XV\textsuperscript{e} siècle, il poursuit encore son prosélytisme, et qu’il réussit \footnote{Ritter, \emph{Erdkunde, Asien}, t. III, p. 111 et passim.}. L’infusion du sang demi-arian, dans le Pendjab, a produit la religion égalitaire de Nanek. Le brahmanisme s’est dédommagé de cette perte en rendant de plus en plus imparfaite la foi musulmane qui habite avec lui.\par
Miné depuis un siècle par l’action européenne, on sait avec quelle imperturbable confiance il a jusqu’ici résisté, et je ne crois pas qu’il existe un homme, ayant vécu dans l’Inde, qui se laisse aller à croire que ce pays puisse jamais subir une transformation et devenir civilisé à notre manière. Plusieurs des observateurs qui l’ont le plus pratiqué et le mieux connu ont témoigné que, dans leur conviction, ce moment-là n’arriverait pas.\par
Pourtant le brahmanisme est en décadence complète ; ses grands hommes ont disparu ; les absurdes ou féroces superstitions, les niaiseries théologiques de la partie noire de son culte, ont pris le dessus d’une manière effrayante sur ce que son antique philosophie présentait de si élevé, de si noblement ardu. Le type nègre et le principe jaune ont creusé leur chemin dans ses populations d’élite, et, sur plusieurs points, il est difficile, même impossible, de distinguer les brahmanes de telles individualités apparte­nant aux basses castes. En tout cas, jamais la nature pervertie de cette race dégénérée ne pourra prévaloir contre la force supérieure des nations blanches venues de l’occident de l’Europe.\par
Mais s’il arrivait que, par suite de circonstances étrangères aux événements de la politique locale, la domination anglaise cessât dans ces vastes contrées et que, rendues à elles-mêmes, il leur fallût se reconstituer, sans doute après un temps plus ou moins long, le brahmanisme, seul ordre social qui offre encore, dans ce pays, quelque solidité, quelques doctrines inébranlables, finirait par prévaloir.\par
Dans le premier moment, la force matérielle résidant plutôt chez les Rohillas de l’ouest et chez les Sykhes du nord, l’honneur de fournir les souverains reviendrait à ces tribus. Néanmoins, la civilisation musulmane est trop dégradée, trop intimement unie aux types les plus vils de la population pour fournir une longue carrière. Quelques nations de cette croyance échappent, peut-être, à ce dur jugement ; mais il tombe en plein sur le plus grand nombre. Le brahmanisme est patient dans ses conquêtes. Il userait, par les coups même qu’il saurait supporter sans mourir, le tranchant du sabre ébréché de ses ennemis, et, d’abord relevé avec triomphe chez les Mahrattes et les Radjapoutes, il ne tarderait pas à se retrouver maître de la plus grande partie du terrain qu’il a perdu depuis tant de siècles. D’ailleurs il n’est pas inflexible aux transactions, et, s’il consentait, dans un traité définitif, à recevoir au rang de deux premières castes les belliqueux convertis des races arianisées du nord et cette classe remuante et active des métis anglo-hindous, ne contre-balancerait-il pas, dans son sein même, la longue infusion des types inférieurs, et ne pourrait-il ainsi renaître à quelque médiocre puis­sance ? Il se passerait probablement quelque chose de ce genre. Toutefois, je l’avoue, le désordre ethnique en serait plus compliqué, et l’unité majestueuse de la civilisation primitive ne renaîtrait pas.\par
Ce ne sont là que les applications rigoureuses des principes posés jusqu’ici et des expériences que j’ai relevées et indiquées. Si, quittant ces hypothèses, on veut laisser l’avenir, et se borner à résumer les enseignements qu’au point de vue des races on peut tirer de l’histoire de l’Inde, voici les faits, tout à fait incontestables, qui en ressortent.\par
Nous devons considérer la famille ariane comme la plus noble, la plus intelligente, la plus énergique de l’espèce blanche. En Égypte, où nous l’avons aperçue d’abord, sur la terre hindoue, où nous venons de l’observer, nous lui avons reconnu de hautes facultés philosophiques, un grand sentiment de moralité, de la douceur dans ses institutions, de l’énergie à les maintenir ; en somme, une supériorité marquée sur les aborigènes, soit de la vallée du Nil, soit des bords de l’Indus, du Gange et du Brahmapoutra.\par
En Égypte, pourtant, nous n’avons réussi à la considérer que déjà, et dès la plus haute antiquité, violemment combattue et paralysée par des immixtions trop considé­rables de sang noir, et, à mesure que les temps ont marché, cette immixtion, prenant plus de forces, a fini par absorber les énergies du principe auquel la civilisation égyptienne devait la vie. Dans l’Inde, il n’en a pas été de même. Le torrent arian, précipité du haut de la vallée de Kachemyr sur la péninsule cisgangétique, était des plus considérables. Il eut beau être dédoublé par la désertion des Zoroastriens, il resta toujours puissant, et le régime des castes fut, malgré sa décomposition lente, malgré ses déviations répétées, une cause décisive, qui conserva aux deux hautes classes de la société hindoue les vertus et les avantages de l’autorité. Puis, si des infiltrations illégales de sang étranger eurent lieu, par l’influence des révolutions, dans les veines des brahmanes et des kschattryas, toutes ne furent pas nuisibles de la même façon, toutes ne produisirent pas de mauvaises conséquences semblables. Ce qui provint des tribus arianes ou demi-arianes du nord renforça la vigueur de l’ancien principe blanc, et nous avons remarqué que l’invasion des Pandavas avait fait une trouée bien profonde dans l’Aryavarta. L’influence de cette immigration y fut donc désorganisatrice, et non pas énervante. Puis, au pourtour entier de cette même frontière montagneuse, d’autres populations blanches paraissaient incessamment sur les crêtes, et descendant jusque dans l’Inde, à différentes époques, elles ont également apporté quelque ressouvenir des mérites de l’espèce.\par
Quant aux mélanges nuisibles, la famille hindoue n’a pas autant à gémir des parentés jaunes qu’elle s’est données que des noires, et bien que, sans nul doute, elle n’ait pas vu sortir de ces mélanges des descendances aussi robustes que lorsqu’elle ne produisait qu’avec elle-même, elle possède cependant, de ce côté, des lignées qui ne sont pas absolument dénuées de valeur, et qui, mêlant à la culture hindoue, dont elles ont adopté les principales règles, certaines idées chinoises, prêtent, au besoin, quelque secours à la civilisation brahmanique. Tels sont les Mahrattes, tels encore, les Birmans.\par
En somme, la force de l’Inde contre les invasions étrangères, la force qui persiste tout en cédant reste cantonnée dans le nord-ouest, le nord et l’ouest, c’est-à-dire chez les peuples d’origine ariane plus ou moins pure : Syndhis, Rohillas, montagnards de l’Hindou-koh, Sykhes, Radjapoutes, Gorkhas du Népaul ; puis viennent les Mahrattes, enfin les Birmans que j’ai nommés plus haut. Dans ce camp de réserve, la suprématie appartient, incontestablement, aux descendances les plus arianisées du nord et du nord-ouest. Et quelle singulière persistance ethnique, quelle conscience vive et puissante toute famille alliée à la race ariane a de son mérite ! J’en trouverais une marque singulière dans l’existence curieuse d’une religion bien étrange répandue chez quelques peuplades misérables, habitantes des pics septentrionaux. Là, des tribus encore fidèles à l’ancienne histoire sont cernées de tous côtés par des jaunes qui, maîtres des vallées basses, les ont repoussées sur les hauteurs neigeuses et dans les gorges alpestres, et ces peuples, nos derniers et malheureux parents, adorent, avant tout, un ancien héros appelé Bhim-Sem. Ce dieu, fils de Pandou, est la personnification de la race blanche dans la dernière grande migration qu’elle ait opérée de ce côté du monde \footnote{Ritter, \emph{Erdkunde, Asien}, t. III p. 115.}.\par
Il reste le sud de l’Inde, la partie qui s’étend vers Calcutta, le long du Gange, les vastes provinces du centre et le Dekkhan. Dans ces régions, les tribus de sauvages noirs sont nombreuses, les forêts immenses, impénétrables, et l’usage des dialectes dérivés du sanscrit cesse presque complètement. Un amas de langues, plus ou moins ennoblies par des emprunts à l’idiome sacré, le tamoul, le malabare et cent autres se partagent les populations. Une bigarrure infinie de carnations étonne d’abord l’Européen, qui, dans l’aspect physique des hommes, ne découvre aucune trace d’unité, pas même chez les hautes castes. Ces contrées sont celles où le mélange avec les aborigènes est le plus avancé. Elles sont aussi les moins recommandables, à tous égards. Des multitudes molles, sans énergie, sans courage, plus bassement supers­titieuses que partout ailleurs, semblent mortes, et ce n’est qu’être juste envers elles que de les déclarer incapables de se laisser galvaniser, un seul instant, par un désir d’indépendance. Elles n’ont jamais été que soumises et sujettes, et le brahmanisme n’en a reçu nul secours, car la proportion de sang des noirs, répandue au sein de cette masse, dépasse trop ce que l’on voit dans le nord, d’où les tribus arianes n’ont jamais poussé jusque-là, soit par terre, soit par mer, que des colonies insuffisantes \footnote{Lassen, \emph{Indische Alterth}., t. I, p. 391.}.\par
Cependant ces contrées méridionales de l’Inde possèdent, aujourd’hui, un nouvel élément ethnique d’une grande valeur, auquel j’ai déjà fait allusion plus haut. Ce sont les métis, nés de pères européens et de mères indigènes et croisés de nouveau avec des Européens et des natifs. Cette classe, qui va, chaque jour s’augmentant, montre des qualités si spéciales, une intelligence si vive, que l’attention des savants et des politiques s’est déjà éveillée à son sujet, et l’on a vu, dans son existence, la cause future des révolutions de l’Inde.\par
Il est de fait qu’elle mérite l’intérêt. Du côté des mères, l’origine n’est pas brillante : ce ne sont guère que les plus basses classes qui fournissent des sujets aux plaisirs des conquérants. Si quelques femmes appartiennent à un rang social un peu moins rabaissé, ce sont des musulmanes, et cette circonstance ne garantit aucune supériorité de sang. Toutefois, comme l’origine de ces Hindoues a cessé d’être absolument identique avec l’espèce noire et qu’elle a déjà été relevée par l’accession d’un principe blanc, si faible qu’on veuille le supposer, il y a profit, et l’on doit établir une immense distance entre le produit d’une femme bengali de basse caste et celui d’une négresse yolof ou bambara.\par
Du côté du père, il peut exister de grandes différences dans l’intensité du principe blanc transmis à l’enfant. Suivant que cet homme est anglais, irlandais, français, italien ou espagnol, les variations sont notables. Comme, le plus souvent, le sang anglais domine, comme il est celui qui, en Europe, a conservé le plus d’affinités avec l’essence ariane, les métis sont généralement beaux ou intelligents. Je m’unis donc à l’opinion qui attache de l’importance pour l’avenir de l’Inde au développement de cette population nouvelle, et, en m’abstenant de penser qu’elle soit jamais en état de mettre la main au collet de ses maîtres et de s’attaquer au radieux génie de la Grande-Bretagne, je ne crois pas inadmissible qu’après les dominateurs européens le sol de l’Inde ne la voie saisir le sceptre. À la vérité, cette race composite est exposée au même danger sous lequel ont succombé presque toutes les nations musulmanes, j’entends la continuité des mélanges et l’abâtardissement qui en est la conséquence. Le brahmanisme seul possède le secret de contrarier le progrès d’un tel fléau.\par
Après avoir ainsi classé les groupes hindous et indiqué les points d’où l’étincelle vivante, encore que très affaiblie, jaillira à l’occasion, je ne saurais m’empêcher de revenir sur la longévité si extraordinaire d’une civilisation qui fonctionnait avant les âges héroïques de la Grèce, et qui, sauf les modifications voulues par les variations ethni­ques, a gardé, jusqu’à nos jours, les mêmes principes, a toujours cheminé dans les mêmes voies, parce que la race dirigeante est demeurée suffisamment compacte. Ce colosse merveilleux de génie, de force, de beauté, a, depuis Hérodote, offert au monde occidental l’image d’une de ces prêtresses qui, bien que couvertes d’une robe épaisse et d’un voile discret, parvenaient cependant, par la majesté de leur attitude, à convaincre tous les regards qu’elles étaient belles. On ne la voyait pas, on n’apercevait que les grands plis de ses vêtements, on n’avait jamais dépassé la zone occupée par les peuples qu’elle-même renonçait comme siens. Plus tard, les conquêtes des musulmans, à demi connues en Europe, et leurs découvertes, dont les résultats n’arrivaient que défigurés, augmentèrent graduellement l’admiration pour ce pays mystérieux, bien que la connaissance en restât fort imparfaite.\par
Mais, depuis une vingtaine d’années que la philologie, la philosophie, la statistique, ont commencé l’inventaire de la société et de la nature hindoues, sans presque avoir l’espérance de le compléter de bien longtemps, tant la matière est riche et abondante, il est arrivé le contraire de ce que révèle l’expérience commune : moins une chose est connue, plus on l’admire ; ici, à mesure qu’on connaît et qu’on apprécie mieux, on admire davantage. Habitués à l’existence bornée de nos civilisations, nous répétions, imperturbablement, les paroles du psautier sur la fragilité des choses humaines, et lorsque le rideau immense qui cachait l’activité de l’existence asiatique a été soulevé, et que l’Inde et la Chine ont apparu clairement à nos regards, avec leurs constitutions inébranlables, nous n’avons su comment prendre cette découverte si humiliante pour notre sagesse et notre force.\par
Quelle honte, en effet, pour des systèmes qui se sont proclamés chacun à leur tour et se proclament encore sans rivaux ! Quelle leçon pour la pensée grecque, romaine, pour la nôtre, que de voir un pays qui, battu par huit cents ans de pillage et de massa­ cres, de spoliations et de misères, compte plus de cent quarante millions d’habitants, et, probablement, avant ses malheurs, en nourrissait plus du double ; pays qui n’a jamais cessé d’entourer de son affection sans bornes et de sa conviction dévouée les idées religieuses, sociales et politiques auxquelles il doit la vie, et qui, dans leur abaissement, lui conservent le caractère indélébile de sa nationalité ! Quelle leçon, dis-je, pour les États de l’Occident, condamnés par l’instabilité de leurs croyances à changer incessamment de formes et de direction, pareils aux dunes mobiles de certains rivages de la mer du Nord !\par
Il y aurait pourtant injustice à blâmer trop les uns comme à trop louer les autres. La longévité de l’Inde n’est que le bénéfice d’une loi naturelle qui n’a pu trouver que rare­ment à s’appliquer en bien. Avec une race dominante éternellement la même, ce pays a possédé des principes éternellement semblables ; tandis que, partout ailleurs, les groupes, se mêlant sans frein et sans choix, se succédant avec rapidité, n’ont pas réussi à faire vivre leurs institutions, parce qu’ils disparaissaient eux-mêmes rapidement devant des successeurs pourvus d’instincts nouveaux.\par
Mais je viens de le dire : l’Inde n’a pas été le seul pays où se soit réalisé le phénomène que j’admire : il faut citer encore la Chine. Recherchons si les mêmes causes y ont amené les mêmes effets. Cette étude se lie d’autant mieux à celle qui finit ici, qu’entre le Céleste Empire et les pays hindous s’étendent de vastes régions, comme le Thibet, où des institutions mixtes portent le caractère des deux sociétés d’où elles émanent. Mais, avant de nous informer si cette dualité est vraiment le résultat d’un double principe ethnique, il faut, de toute nécessité, connaître la source de la culture sociale en Chine, et nous rendre compte du rang que cette contrée a droit d’occuper parmi les nations civilisées du monde.
\section[{III.4. La race jaune.}]{III.4. \\
La race jaune.}
\noindent À mesure que les tribus hindoues se sont plus avancées vers l’est, et qu’après avoir longé les monts Vyndhias, elles ont dépassé le Gange et le Brahmapoutra pour pénétrer dans le pays des Birmans, nous les avons vues se mettre en contact avec des variétés humaines que l’occident de l’Asie ne nous avait pas encore fait connaître. Ces variétés, non moins multipliées dans leurs nuances physiques et morales que les différences déjà constatées chez l’espèce nègre, nous sont une nouvelle raison d’admet­tre, par analogie, que la race blanche eut aussi, comme les deux autres, ses séparations propres, et que non seulement il exista des inégalités entre elle et les hommes noirs et ceux de la nouvelle catégorie que j’aborde, mais encore que, dans son propre sein, la même loi exerça son influence, et qu’une diversité pareille distingua ses tribus et les disposa par étages.\par
Une nouvelle famille, très bigarrée de formes, de physionomie et de couleur, très spéciale dans ses qualités intellectuelles, se présente à nous aussitôt que nous sortons du Bengale en marchant vers l’est, et comme des affinités évidentes réunissent à cette avant-garde de vastes populations marquées de son cachet, il nous faut adopter, pour tout cet ensemble, un nom unique, et, malgré les différences qui le fractionnent, lui attribuer une dénomination commune. Nous nous trouvons en face des peuples jaunes, troisième élément constitutif de la population du monde.\par
Tout l’empire de la Chine, la Sibérie, l’Europe entière, à l’exception, peut-être, de ses extrémités les plus méridionales, tels sont les vastes territoires dont le groupe jaune se montre possesseur aussitôt que des émigrants blancs mettent le pied dans les contrées situées à l’ouest, au nord ou à l’est des plateaux glacés de l’Asie centrale.\par
Cette race est généralement petite, certaines même de ses tribus ne dépassent pas les proportions réduites des nains. La structure des membres, la puissance des muscles sont loin d’égaler ce que l’on voit chez les blancs. Les formes du corps sont ramassées, trapues, sans beauté ni grâce, avec quelque chose de grotesque et souvent de hideux. Dans la physionomie, la nature a économisé le dessin et les lignes. Sa libéralité s’est bornée à l’essentiel : un nez, une bouche, de petits yeux sont jetés dans des faces larges et plates, et semblent tracés avec une négligence et un dédain tout à fait rudimentaires. Évidemment, le Créateur n’a voulu faire qu’une ébauche. Les cheveux sont rares chez la plupart des peuplades. On les voit cependant, et comme par réaction, effroyablement abondants chez quelques-unes et descendant jusque dans le dos ; pour toutes, noirs, roides, droits et grossiers comme des crins. Voilà l’aspect physique de la race jaune \footnote{M. Pickering ajoute, à tous ces caractères, un autre trait qui lui semble tout à fait spécifique : c’est l’aspect féminin que le défaut de barbe donne aux peuples jaunes. En revanche, il ne considère pas l’obliquité de l’œil comme essentielle. Je crois qu’ici il ne tient pas assez de compte des immixtions noires qui souvent, et à dose même très légère, ont pu suffire pour faire disparaître cette particularité. (\emph{United-States exploring Expedition during the years 1838, 1839, 1843, 1841 and 1842, under the command of Charles Wilkes, U. S. N. ; vol. IX : The Races of man and their geographical distribution}, by Charles Pickering, M. D. ; Philadelphia, 1848, in-4°.) – M. Pickering pense que la race jaune couvre actuellement deux cinquièmes de la surface du globe. Il comprend évidemment, dans cette classification, beaucoup de populations hybrides.}.\par
Quant à ses qualités intellectuelles, elles ne sont pas moins particulières, et font une opposition si tranchée aux aptitudes de l’espèce noire, qu’ayant donné à cette dernière le titre de féminine, j’applique à l’autre celui de mâle, par excellence. Un défaut absolu d’imagination, une tendance unique à la satisfaction des besoins naturels, beaucoup de ténacité et de suite appliqué à des idées terre à terre ou ridicules, quelque instinct de la liberté individuelle, manifesté, dans le plus grand nombre des tribus, par l’attachement à la vie nomade, et, chez les peuples les plus civilisés, par le respect de la vie domestique ; peu ou point d’activité, pas de curiosité d’esprit, pas de ces goûts passionnés de parure, si remarquables chez les nègres : voilà les traits principaux que toutes les branches de la famille mongole possèdent, en commun, à des degrés différents. De là, leur orgueil profondément convaincu et leur médiocrité non moins caractéristique, ne sentant rien que l’aiguillon matériel, et ayant trouvé dès longtemps le moyen d’y satisfaire. Tout ce qui se fait en dehors du cercle étroit qu’elles connaissent leur paraît insensé, inepte, et ne leur inspire que pitié. Les peuples jaunes sont beaucoup plus contents d’eux-mêmes que les nègres, dont la grossière imagination, constamment en feu, rêve à tout autre chose qu’au moment présent et aux faits existants.\par
Mais, il faut aussi en convenir, cette tendance générale et unique vers les choses humblement positives, et la fixité de vues, conséquence de l’absence d’imagination, donnent aux peuples jaunes plus d’aptitude à une sociabilité grossière que les nègres n’en possèdent. Les plus ineptes esprits, n’ayant, pendant des siècles, qu’une seule pensée dont rien ne les distrait, celle de se nourrir, de se vêtir et de se loger, finissent par obtenir, dans ce genre, des résultats plus complets que des gens qui, naturellement non moins stupides, sont encore dérangés sans cesse, des réflexions qui pourraient leur venir, par des fusées d’imagination. Aussi les peuples jaunes sont-ils devenus assez habiles dans quelques métiers, et ce n’est pas sans surprise qu’on les voit, dès l’antiquité la plus haute, laisser, comme marque irréfragable de leur présence dans une contrée, des traces d’assez grands travaux de mines. C’est là, pour ainsi dire, le rôle antique et national de la race jaune \footnote{Ritter, \emph{Erdkunde, Asien}, t. I, p. 337.}. Les nains sont des forgerons, sont des orfèvres, et de ce qu’ils ont possédé une telle science et l’ont conservée à travers les siècles jusqu’à nos jours (car, à l’est des Tongouses orientaux et sur les bords de la mer d’Ochotsk, les Doutcheris et d’autres peuplades ne sont pas des forgerons moins adroits que les Permiens des chants scandinaves), il faut conclure que, de tout temps, les Finnois se sont trouvés, au moins, propres à former la partie passive de certaines civilisations \footnote{Lassen, \emph{Zeitschrift für d. K .d. Morgenl.}, t. II, p. 62 ; Ritter, \emph{Erdkunde, Asien}, t. II.}.\par
D’où venaient ces peuples ? Du grand continent d’Amérique. C’est la réponse de la physiologie comme de la linguistique ; c’est aussi ce qu’on doit conclure de cette observation, que, dès les époques les plus anciennes, avant même ce que nous nom­mons les âges primitifs, des masses considérables de populations jaunes s’étaient accumulées dans l’extrême nord de la Sibérie, et de là avaient prolongé leurs campe­ments et leurs hordes jusque très avant dans le monde occidental, donnant sur leurs premiers ancêtres des renseignements fort peu honorables.\par
Elles prétendaient descendre des singes, et s’en montraient très satisfaites. Il n’est dès lors pas étonnant que l’épopée hindoue, ayant à dépeindre les auxiliaires aborigènes de l’héroïque époux de Sita dans sa campagne contre Ceylan, nous dise tout simple­ment que ces auxiliaires étaient une armée de singes. Peut-être, en effet, Rama, voulant combattre les peuples noirs du sud du Dekkhan, eut-il recours à quelques tribus jaunes campées sur les contreforts méridionaux de l’Himalaya.\par
Quoi qu’il en puisse être, ces nations étaient fort nombreuses, et quelques déduc­tions bien claires de points déjà connus vont l’établir à l’instant.\par
Ce n’est pas un fait nécessaire à prouver, car il l’est surabondamment, que les nations blanches ont toujours été sédentaires, et, comme telles, n’ont jamais quitté leurs demeures que par contrainte. Or, le plus ancien séjour connu de ces nations étant le haut plateau de l’Asie centrale, si elles l’ont abandonné, c’est qu’on les en a chassées. Je comprends bien que certaines branches, parties seules, isolément, pourraient être considérées comme ayant été victimes de leurs congénères, et battues, violentées par des parents. Je l’admettrai pour les tribus helléniques et pour les zoroastriennes ; mais je ne saurais étendre ce raisonnement à la totalité des migrations blanches. La race entière n’a pas dû s’expulser de chez elle dans tout son ensemble, et cependant on la voit se déplacer, pour ainsi dire, en masse et presque en même temps, avant l’an 5000. À cette époque et dans les siècles qui en sont le plus rapprochés, les Chamites, les Sémites, les Arians, les Celtes et les Slaves désertent également leurs domaines primitifs. L’espèce blanche s’échappe de tous côtés, s’en va de toutes parts, et certes dans une telle dissolution, qui finit par laisser ses plaines natales aux mains des jaunes, il est difficile de voir autre chose que le résultat d’une pression des plus violentes opérée par ces sauvages sur son faisceau primordial.\par
D’un autre côté, l’infériorité physique et morale des multitudes conquérantes est si claire et si constatée, que leur invasion et la victoire finale qui en démontre la force, ne peuvent avoir leur source ailleurs que dans le très grand nombre des individus agglomérés dans ces bandes. Il n’est, dès lors, pas douteux que la Sibérie regorgeait de populations finnoises, et c’est aussi ce que va démontrer bientôt un ordre de preuves qui, cette fois, appartient à l’histoire. Pour le moment, poursuivant le rayon de clarté que la comparaison de la vigueur relative des races jette sur les événements de ces temps obscurs, je ferai remarquer encore que, si l’on admet la victoire des nations jaunes sur les blanches et la dispersion de ces dernières, il faudra aussi s’accommoder de l’alternative suivante :\par
Ou bien le territoire des nations blanches s’étendait beaucoup vers le nord et très peu vers l’est, atteignant au moins, dans la première direction, l’Oural moyen, et, dans l’autre, ne dépassant pas le Kouen-loun, ce qui semblerait impliquer un certain développement vers les steppes du nord-ouest ;\par
Ou bien ces peuples, ramassés sur les crêtes du Mouztagh, dans les plaines élevées qui suivent immédiatement, et dans les trois Thibets, n’existaient qu’en nombre très faible et dans une proportion compatible avec l’étendue médiocre de ces territoires et les ressources alimentaires fort réduites, presque nulles, qu’ils peuvent offrir.\par
Je vais d’abord expliquer comment je me vois contraint de tracer ces limites ; ensuite j’établirai par quelle raison il faut repousser la seconde hypothèse et s’attacher fortement à la première.\par
J’ai dit que la race jaune se montrait en possession primordiale de la Chine, et, en outre, que le type noir à tête prognathe et laineuse, l’espèce pélagienne, remontait jusqu’au Kouen-loun, d’une part, et, de l’autre côté, jusqu’à Formose \footnote{Ce sont les habitants de l’intérieur de l’île qui sont complètement noirs. Les hommes des côtes appartiennent à l’espèce malaise et ont beaucoup de rapports avec les Haraforas. (Ritter, t. III, p. 879.) – Le nombre des tribus nègres est assez considérable dans l’Inde transgangétique. On peut citer entre autres les Samangs, retirés dans la partie méridionale du district de Queda, au pays de Siam. C’est une race petite, à cheveux crépus, sans demeures fixes et se nourrissant de reptiles crus et de vers. (Ritter, \emph{loc. cit}., p. 1131.) – Ce géographe avoue ne pouvoir s’expliquer l’extrême diffusion de la famille mélanienne en Asie. Le fait serait, en effet, incompréhensible, s’il fallait le considérer comme postérieur aux temps historiques ; mais il devient très simple quand on admet qu’il s’est opéré à une époque tout à fait primordiale, où les immigrants nègres trouvaient le pays désert.}, au japon et par delà. Aujourd’hui même des populations de ce genre habitent ces pays reculés.\par
Voir le nègre établi si avant dans l’intérieur de l’Asie a déjà été pour nous la grande preuve de l’alliance, en quelque sorte, originelle des Chamites et des Sémites avec ces peuples d’essence inférieure ; j’ai dit originelle, parce que l’alliance fut évidemment contractée avant la descente des envahisseurs dans les pays mésopotamiques de l’Euphrate et du Tigre.\par
Maintenant, en nous transportant des plaines de la Babylonie à celles de la Chine, nous trouverons un spécimen des résultats gradués du mélange des deux espèces noire et jaune dans ces métis qui habitent le Yun-nan, et que Marco-Polo appelle les Zerdendam. En allant plus loin, nous rencontrerons encore cette autre famille, non moins marquée des caractères de l’alliage, qui couvre la province chinoise du Fo-kien, et enfin nous tomberons au milieu des nuances innombrables de ces groupes cantonnés dans les provinces méridionales du Céleste Empire, dans l’Inde transgangétique, dans les archipels de la mer des Indes, depuis Madagascar jusqu’à la Polynésie, et depuis la Polynésie jusqu’aux rives occidentales de l’Amérique, atteignant l’île de Pâques \footnote{ \noindent Ritter, \emph{Erdkunde, Asien}, t. II, p. 1046.\par
 Pickering, p. 135. Cet excellent observateur n’hésite pas à déclarer qu’à ses yeux les Ovahs de Madagascar sont des Malais imméconnaissables.
}.\par
Ainsi la race noire a embrassé tout le sud de l’ancien monde et envahi fortement sur le nord, tandis que la jaune, se rencontrant avec elle à l’orient de l’Asie, y contractait un hymen fécond dont les rejetons occupent tous les amas d’îles prolongés dans la direction du pôle austral. Si l’on réfléchit que le centre, le foyer de l’espèce mélanienne est l’Afrique, et que c’est de là que s’est opérée sa diffusion principale, et, en outre, que la race jaune, en même temps que ses métis possédaient les îles, allait aussi se reproduisant au nord et à l’est de l’Asie et dans toute l’Europe, on en conclura que la famille blanche, pour ne pas se perdre et disparaître au milieu des variétés inférieures, devait unir à la puissance de son génie et de son courage la garantie du nombre, bien qu’à un moindre degré, sans doute, que ses adversaires.\par
Nous ne pouvons même essayer le dénombrement des masses chamites et sémites qui descendirent, par les passages de l’Arménie, dans les régions du sud et de l’ouest. Mais, du moins, considérons le nombre énorme des mélanges qui s’en firent avec la race noire, jusque par delà les plaines de l’Éthiopie, et, au nord, sur toute la côte d’Afrique, au delà de l’Atlas, tendant vers le Sénégal ; regardons les produits de ces hymens peuplant l’Espagne, la basse Italie, les îles grecques, et nous serons en situation de nous persuader que l’espèce blanche ne se limitait pas à quelques tribus. Nous en devons décider ainsi d’autant plus sûrement, qu’aux multitudes que je viens d’énumérer il convient d’ajouter encore les nations arianes de toutes les branches méridionales, et les Celtes, et les Slaves, et les Sarmates, et d’autres peuples sans célébrité, mais nullement sans influence, qui restèrent au milieu des jaunes.\par
La race blanche était donc aussi fort prolifique, et puisque les deux espèces noire et finnoise ne lui permettaient pas de dépasser le Mouztagh et l’Altaï à l’est, l’Oural à l’ouest, resserrée dans de telles limites, elle s’étendait, au nord, jusque vers le cours moyen de l’Amour, le lac Baïkal et l’Obi.\par
 Les conséquences de cette disposition géographique sont considérables et vont, tout à l’heure, trouver leurs applications.\par
J’ai constaté les facultés pratiques de la race jaune. Toutefois, en lui reconnaissant des aptitudes supérieures à celles de la noire pour les basses fonctions d’une société cultivée, je lui ai refusé la capacité d’occuper un rang glorieux sur l’échelle de la civilisation, et cela parce que son intelligence, bornée autrement, ne l’est pas moins étroitement que celle des nègres, et parce que son instinct de l’utile est trop peu exigeant.\par
Il faut relâcher quelque chose de la sévérité de ce jugement lorsqu’il s’agit, non plus de l’espèce jaune, non plus du type noir, mais du métis des deux familles, le Malais. Que l’on prenne, en effet, un Mongol, un habitant de Tonga-Tabou et un nègre pélagien ou hottentot, l’habitant de Tonga-Tabou, tout inculte qu’il soit, montrera certainement un type supérieur.\par
Il semblerait que les défauts des deux races se sont balancés et modérés dans le produit commun, et que, plus d’imagination relevant l’esprit, tandis qu’un sentiment moins faux de la réalité restreignait l’imagination, il en est résulté plus d’aptitude à comparer, à saisir, à conclure. Le type physique a éprouvé aussi d’heureuses modifica­tions. Les cheveux du Malais sont durs et revêches, à la vérité ; mais, enclins à se crêper, ils ne le font pas ; le nez est plus formé que chez les Kalmouks. Pour quelques insulaires, à Tahiti par exemple, il devient presque semblable au nez droit de la race blanche. L’œil n’est plus toujours relevé à l’angle externe. Si les pommettes restent saillantes, c’est que ce trait est commun aux deux races génératrices. Les Malais sont, du reste, on ne peut plus différents entre eux. Suivant que le sang noir ou jaune domine dans la formation d’une tribu, les caractères physiques et moraux s’en ressentent. Les alliages postérieurs ont augmenté cette extrême variabilité de types. En somme, deux signes, nettement distinctifs, demeurent à toutes ces familles, comme un présent de leur double origine : plus intelligentes que le nègre et l’homme jaune, elles ont gardé de l’un l’implacable férocité, de l’autre l’insensibilité glaciale \footnote{Aux témoignages sur lesquels je me suis déjà appuyé, je joins celui de Ritter, confirmé par Finlayson et sir Stamford Raffles : « Les Malais, suivant le grand géographe allemand, sont de taille moyenne et plutôt petits. Ils ont une carnation plus claire que les peuples d’au-delà du Gange. Le tissu de la peau est, chez eux, doux et brillant. Leur disposition à engraisser est remarquable. La musculature est molle, lâche, quelquefois très volumineuse, généralement sans élasticité. Les hanches sont très fortes, ce qui leur donne une apparence lourde. Les visages sont larges et plats, les pommettes saillantes. Les yeux sont espacés et très petits, quelquefois droits, le plus souvent relevés à l’angle externe. L’occiput est resserré ; les cheveux, épais, grossiers, tendant à se crêper, sont plantés très bas et restreignent le front. Le trou occipital est souvent très en arrière. Les bras, très longs, rappellent ceux du singe. » (Ritter, III, p. 1145.) – À ces détails j’en ajouterai encore un que je dois à l’intéressante observation d’un voyageur : « Lorsque les matelots malais employés sur les navires européens montent aux cordages, ils se cramponnent non seulement par les mains, mais encore par les orteils, qu’ils ont très gros et très vigoureux. Un homme de race blanche n’en pourrait faire autant. »}.\par
J’ai achevé ce qu’il y avait à dire sur les peuples qui figurent dans l’histoire de l’Asie orientale, il est maintenant à propos de passer à l’examen de leur civilisation. Le plus haut degré s’en rencontre en Chine. C’est là qu’est, tout à la fois, le point de départ de leur culture et sa plus originale expression : c’est donc là qu’il convient de l’étudier.
\section[{III.5. Les Chinois.}]{III.5. \\
Les Chinois.}
\noindent Je me trouve, d’abord, en dissentiment avec une idée assez généralement répandue. On incline à considérer la civilisation chinoise comme la plus ancienne du monde, et je n’en aperçois l’avènement qu’à une époque inférieure à l’aurore du brahmanisme, inférieure à la fondation des premiers empires chamites, sémites et égyptiens. Voici mes raisons. Il va sans dire que l’on ne discute plus les affirmations chronologiques et historiques des Tao-sse. Pour ces sectaires, les cycles de 300 000 années ne coûtent absolument rien. Comme ces périodes un peu longues forment le milieu où agissent des souverains à têtes de dragons, et dont les corps sont contournés en serpents mons­trueux, ce qu’il y a de mieux à faire, c’est d’en abandonner l’examen à la philosophie, qui pourra y glaner quelque peu, mais d’en écarter, avec grand soin, l’étude des faits positifs \footnote{Nu-oua, sœur de Fou-hi, et qui lui succéda, était un esprit. Elle avait ramassé, dans un marais, un peu de terre \emph{jaune}, et, en s’aidant d’une corde, elle en fabriqua le premier homme. (Le père Gaubil, \emph{Chronologie chinoise}, in-4°, p. 7.)}.\par
La date la plus rationnelle où se placent les lettrés du Céleste Empire pour juger de leur état antique, c’est le règne de Tsin-chi-hoang-ti, qui, pour couper court aux conspi­rations féodales et sauver la cause unitaire dont il était le promoteur, voulut étouffer les anciennes idées, fit brûler la plupart des livres, et ne consentit à sauver que les annales de la dynastie princière de Tsin, dont lui-même descendait. Cet événement arriva 207 ans avant J.-C.\par
 Depuis cette époque, les faits sont bien détaillés, suivant la méthode chinoise. Je n’en goûte pas moins l’observation d’un savant missionnaire, qui voudrait voir dans ces lourdes compilations un peu plus de critique européenne \footnote{Id. \emph{ibid.}}. Quoi qu’il en soit, à dater de ce moment, tout s’enchaîne tant bien que mal. Quand on veut remonter au-delà, il n’en est pas longtemps de même. Tant qu’on reste dans les temps rapprochés de Tsin-chi-hoang-ti, la clarté continue en s’affaiblissant. On remonte ainsi, de proche en proche, jusqu’à l’empereur Yaô. Ce prince régna cent et un ans, et son avènement est placé à l’an 2357 avant J.-C. Par delà cette époque, les dates, déjà fort conjecturales, sont remplacées par une complète incertitude \footnote{Suivant M. Lassen, il ne faut pas demander d’histoire positive aux Chinois avant l’année 782 qui précéda notre ère. Toutefois, ce même savant confesse que l’avènement de la première dynastie humaine peut être reporté, avec une grande vraisemblance, à l’année 2205 av. J.-C. (\emph{Indische Alterthumskunde}, t. I, p. 751.) – Nous voilà loin des dates extraordinaires des annales hindoues, égyptiennes et assyriennes.}. Les lettrés ont prétendu que cette fâcheuse interruption d’une chronique dont les matériaux, suivant eux, pourraient remonter aux premiers jours du monde, n’est que la conséquence de ce fameux incendie des livres, déploré de père en fils, et devenu un des beaux sujets d’amplification que la rhétorique chinoise ait à commandement. Mais, à mon gré, ce malheur ne suffit pas pour expliquer le désordre des premières annales. Tous les peuples de l’ancien monde ont eu leurs livres brûlés, tous ont perdu la chaîne systématique de leurs dynasties en tant que les livres primitifs devaient en être les dépositaires, et cependant tous ces peuples ont conservé assez de débris de leur histoire pour que, sous le souffle vivifiant de la critique, le passé se relève, se remue, ressuscite, et, se dévoilant peu à peu, nous montre une physionomie à coup sûr bien ancienne, bien différente des temps dont nous avons la tradition. Chez les Chinois, rien de semblable. Aussitôt que les temps positifs cessent, le crépuscule s’évanouit, et de suite on arrive, non pas aux temps mythologiques, comme partout ailleurs, mais à des chronologies inconciliables, à des absurdités de l’espèce la plus plate, dont le moindre défaut est de ne rien contenir de vivant.\par
Puis, à côté de cette nullité prétentieuse de l’histoire écrite, une absence complète et bien significative de monuments. Ceci appartient au caractère de la civilisation chinoise. Les lettrés sont grands amateurs d’antiquités, et les antiquités manquent ; les plus anciennes ne remontent pas au delà du VIII\textsuperscript{e} siècle après J.-C. \footnote{Gaubil, \emph{Traité de la chronologie chinoise}.}. De sorte que, dans ce pays stable par excellence, les souvenirs figurés, statues, vases, instruments, n’ont rien qui puisse être comparé, pour l’ancienneté, avec ce que notre Occident si remué, si tourmenté, si ravagé et transformé tant de fois, peut cependant étaler avec une orgueilleuse abondance. La Chine n’a matériellement rien conservé \footnote{Il faut excepter de ce jugement certains travaux de colonisation et de dessèchement sur les rives du Hoang-ho, qui paraissent remonter à des temps fort reculés. Ce ne sont pas là, à proprement parler, des monuments. C’est un tracé cent fois fait et refait depuis sa création.} qui nous reporte même de loin, à ces époques extravagantes ou quelques savants du dernier siècle se réjouissaient de voir l’histoire s’enfoncer en narguant les témoignages mosaïques.\par
 Laissons donc de côté les concordances impossibles des différents systèmes suivis par les lettrés pour fixer les époques antérieures à Tsin-chi-hoang-ti, et ne recueillons que les faits appuyés de l’assentiment des autres peuples, ou portant avec eux une suffisante certitude.\par
Les Chinois nous disent que le premier homme fut Pon-kou. Le \emph{premier homme}, disent-ils ; mais ils entourent cet être primordial de telles circonstances qu’évidemment il n’était pas seul dans le lieu où ils le font apparaître. Il était entouré de créatures inférieures à lui, et ici on se demande s’il n’avait pas affaire à ces fils de singes, ces hommes jaunes dont la singulière vanité se complaisait à réclamer une si brutale origine.\par
Le doute se change bientôt en certitude. Les historiens indigènes affirment qu’à l’arrivée des Chinois, les Miao \footnote{Gaubil, \emph{Traité de la chronologie chinoise}.} occupaient déjà la contrée, et que ces peuples étaient étrangers aux plus simples notions de sociabilité. Ils vivaient dans des trous, dans des grottes, buvaient le sang des animaux qu’ils attrapaient à la course, ou bien, à défaut de chair crue, mangeaient de l’herbe et des fruits sauvages. Quant à la forme de leur gouvernement, elle ne démentait pas tant de barbarie. Les Miao se battaient à coups de branches d’arbres, et le plus vigoureux restait le maître jusqu’à ce qu’il en vînt un plus fort que lui. On ne rendait aucun, honneur aux morts. On se contentait de les empaqueter dans des branches et des herbages, on les liait au milieu de ces espèces de fagots, et on les cachait sous des buissons \footnote{Gaubil, \emph{ouvr. cité}, p. 2, 80, 109 ; Ritter, \emph{Erdkunde, Asien}, t. III, p. 758 ; Lassen, \emph{Indische Alterth.}, t. I, p. 454.}.\par
Je remarquerai, en passant, que voilà bien, dans une réalité historique, l’homme primitif de la philosophie de Rousseau et de ses partisans ; l’homme qui, n’ayant que des égaux, ne peut aussi fonder qu’une autorité transitoire dont une massue est la légitimité, genre de droit assez souvent frappé de défaveur devant des esprits un peu libres et fiers. Malheureusement pour l’idée révolutionnaire, si cette théorie rencontre une preuve chez les Miao et chez les noirs, elle n’a pas encore réussi à la découvrir chez les blancs, où nous ne pouvons apercevoir une aurore privée des clartés de l’intelligence.\par
Pan-Kou, au milieu de ces fils de singes \footnote{Les Miao ne manquaient pas de se donner cette généalogie. (Ritter, \emph{Erdkunde, Asien}, t. II, p. 273.)}, fut donc regardé, et j’ose le dire, avec pleine raison, comme le premier homme. La légende chinoise ne nous fait pas assister à sa naissance. Elle ne nous le montre pas créature, mais bien créateur, car elle déclare expressément qu’il commença à régler les rapports de l’humanité. D’où venait-il, puisque, à la différence de l’Adam de la Genèse, de l’autochtone, phénicien et athénien, il ne sortait pas du limon ? Sur ce point la légende se tait ; cependant, si elle ne sait pas nous apprendre où il est né, elle nous indique, du moins, où il est mort et où il fut enterré : c’est, dit-elle, dans la province méridionale de Honan \footnote{Gaubil, \emph{ouvr. cité.}}.\par
 Cette circonstance n’est pas à négliger, et il faut la rapprocher, sans retard, d’un renseignement très clairement articulé par le Manava-Dharma-Sastra. Ce code religieux des Hindous, compilé à une époque postérieure à la rédaction des grands poèmes, mais sur des documents incontestablement fort anciens, déclare, d’une manière positive, que le Maha-Tsin, le grand pays de la Chine, fut conquis par des tribus des kschattryas réfractaires qui, après avoir passé le Gange et erré pendant quelque temps dans le Bengale, traversèrent les montagnes de l’est et se répandirent dans le sud du Céleste Empire, dont ils civilisèrent les peuples \footnote{Ritter, \emph{Erdkunde, Asien}, t. III, p. 716 ; \emph{Manava-Dharma-Sastra}, ch. X, § 43, p. 346 : « The « following races of Kshattryas, by their omission of holy rites and by seeing no brahmens, « have gradually sunk among men, to the lowest of the four classes. – 44 : Paunidracas, « Odras and Draviras ; Cambojas, Vavanas and Sacas ; Paradas, Pahlavà, CHINAS, « Ciratas, Deradas and Chasas. – 45 : All those tribes of men who sprang from the mouth, « the arm, the thigh and the foot of Brahma, but who became out casts by having « neglected their duties, are called Dasyus, or plunderers, whether they speak the language « of Mlechchas or that of Aryas. »}.\par
Ce renseignement acquiert beaucoup plus de poids encore venant des brahmanes que s’il émanait d’une autre source. On n’a pas la moindre raison de supposer que la gloire d’avoir civilisé un territoire différent du leur, par une branche de leur nation, ait eu de quoi tenter leur vanité et égarer leur bonne foi. Du moment qu’on sortait de l’organisation voulue chez eux, on leur devenait odieux, on était coupable à tous les chefs et renié ; et, de même qu’ils avaient oublié leurs liens de parenté avec tant de nations blanches, ils en auraient fait autant de ceux-là, si la séparation s’était opérée à une époque relativement basse et dans un temps où, la civilisation de l’Inde étant déjà fixée, il n’y avait plus moyen de ne pas apercevoir un fait aussi considérable que le départ et la colonisation séparatiste d’un nombre important de tribus appartenant à la seconde caste de l’État. Ainsi, rien n’infirme, tout appuie, au contraire, le témoignage des lois de Manou, et il en résulte que la Chine, à une époque postérieure aux premiers temps héroïques de l’Inde, a été civilisée par une nation immigrante de la race hindoue, kschattrya, ariane, blanche, et, par conséquent, que Pan-Kou, ce premier homme que, tout d’abord, on est surpris de voir défini en législateur par la légende chinoise, était ou l’un des chefs, ou le chef, ou la personnification d’un peuple blanc venant opérer en Chine, dans le Honan, les mêmes merveilles qu’un rameau également hindou avait, antérieurement, préparées dans la vallée supérieure du Nil \footnote{M. Biot raconte, d’après les documents chinois, que le pays fut civilisé, entre le XXX\textsuperscript{e} siècle et le XXVII\textsuperscript{e} avant notre ère, par une colonisation d’étrangers venant du nord-ouest et désignés généralement, dans les textes, sous le nom de \emph{peuple aux cheveux noirs.} Cette nation conquérante est aussi appelée les \emph{cent familles.} Ce qui résulte principalement de cette tradition, c’est que les Chinois avouent que leurs civilisateurs n’étaient pas autochtones. (\emph{Tcheou-li ou Rites des Tcheou, traduit pour la première fois}, par feu Edouard Biot ; Paris, Imprimerie nationale, 1851, in-fol., \emph{Avertiss.}, p. 2, et \emph{Introduct.}, p. V.)}.\par
Dès lors s’expliquent aisément les relations très anciennes de l’Inde avec la Chine, et l’on n’a plus besoin, pour les commenter, de recourir à l’hypothèse aventurée d’une navigation toujours difficile. La vallée du Brahmapoutra et celle qui, longeant le cours de l’Irawaddy, enferme les plaines et les nombreux passages du pays des Birmans, offraient aux vratyas du Ho-nan des chemins déjà bien connus, puisqu’il avait jadis fallu les suivre pour quitter l’Aryavarta.\par
Ainsi, en Chine, comme en Égypte, à l’autre extrémité du monde asiatique, comme dans toutes les régions que nous avons déjà parcourues jusqu’ici, voilà un rameau blanc chargé par la Providence d’inventer une civilisation. Il serait inutile de chercher à se rendre compte du nombre de ces Arians réfractaires qui, dès leur arrivée dans le Ho-nan, étaient probablement mélangés et déchus de leur pureté primitive. Quelle que fût leur multitude, petite ou grande, leur tâche civilisatrice n’en était pas moins possible. Ils avaient, par suite de leur alliance, des moyens d’agir sur les masses jaunes. Puis, ils n’étaient pas les seuls rejetons de la race illustre adressés vers ces contrées lointaines, et ils devaient s’y associer d’anciens parents aptes à concourir, à aider à leur œuvre.\par
Aujourd’hui, dans les hautes vallées qui bordent le grand Thibet du côté du Boutan, on rencontre, tout aussi bien que sur les crêtes neigeuses, des contrées situées plus à l’ouest, des tribus très faibles, très clairsemées, pour la plupart étrangement mêlées, à la vérité, qui cependant accusent une descendance ariane \footnote{Tel est l’État alpestre de Gwalior, près du Ladakh et du Gherwal. (Ritter, \emph{Erdkunde, Asien}, t. III.) – Telles sont encore certaines populations du Thibet oriental, où l’on retrouve, avec certains caractères physiques de l’espèce blanche, des mœurs qu’on peut dire tout à fait contraires aux habitudes des nations jaunes : le régime féodal et un grand esprit de liberté belliqueuse. (Hue, \emph{Souvenirs d’un voyage dans la Tartarie, le Thibet et la Chine}, t. II, p. 467, et passim et 482.)}. Perdues, comme elles le sont, au milieu des débris noirs et jaunes de toute provenance, on est en droit de comparer ces peuplades à tels morceaux de quartz qui, entraînés par les eaux, contiennent de l’or et viennent de fort loin. Peut-être les orages ethniques, les catastrophes des races les ont-elles portées là où leur espèce elle-même n’avait jamais apparu. Je ne me servirai donc pas de ces détritus par trop altérés, et je me borne à constater leur existence \footnote{Ritter, \emph{Erdkunde, Asien}, t. II.}.\par
Mais, beaucoup plus avant dans le nord, nous apercevons, à une époque assez récente, vers l’an 177 avant J.-C., de nombreuses nations blanches à cheveux blonds ou rouges, à yeux bleus, cantonnées sur les frontières occidentales de la Chine. Les écri­vains du Céleste Empire, à qui l’on doit la connaissance de ce fait, nomment cinq de ces nations. Remarquons d’abord la position géographique qu’elles occupaient à l’époque où elles nous sont révélées.\par
Les deux plus célèbres sont les Yue-tchi et les Ou-soun. Ces deux peuples habitaient au nord du Hoang-ho, sur la limite du désert de Gobi \footnote{Ritter, \emph{Erdkunde, Asien}, t. I, p. 433 et passim.}.\par
Venaient ensuite, à l’est des Ou-soun, les Khou-te \footnote{Ritter identifie cette nation avec les Goths, et M. le baron A. de Humboldt accepte cette opinion. (\emph{Asie centrale}, t. II, p. 130.) Elle ne me paraît cependant s’appuyer que sur une vague ressemblance de syllabes. – Les Ou-soun, vivant au nord-ouest de la Chine, sont signalés par Vensse-kou, le commentateur des Annales de la dynastie des Han, traduit par M. Stanislas Julien, comme étant un peuple blond « à barbe rousse et à yeux bleus. » Ils étaient au nombre de 120,000 familles. (A. de Humboldt, \emph{Asie centrale}, t. I, p. 393.)}.\par
 Plus haut, au nord des Ou-soun, à l’ouest du Baïkal, étaient les Tingling \footnote{Ritter, \emph{loc. cit.}}.\par
Les Kian-kouans, ou Ha-kas, succédaient à ces derniers et dépassaient le Yénisseï \footnote{Les Ha-kas étaient de très haute taille. Ils avaient les cheveux rouges, le visage blanc, les yeux verts ou bleus. Ils se mêlèrent avec les soldats chinois de Li-ling, 97 ans avant J.-C. (Ritter, t. I, p. 1115.)}.\par
Enfin, plus au sud, dans la contrée actuelle du Kaschgar, au delà du Thian-chan, s’étendaient les Chou-le ou Kin-tcha, que suivaient les Yan-Thsai, Sarmates-Alains, dont le territoire allait jusqu’à la met Caspienne \footnote{Ibid. Les Chinois désignaient ces nations arianes, dont les traits différaient si fort des leurs, comme « ayant de longs visages de cheval. » (Asie centrale, t. II, p. 64.)}.\par
De cette façon, à une époque relativement rapprochée de nous, puisque c’est au II\textsuperscript{e} siècle avant notre ère, et après tant de grandes migrations de la race blanche qui auraient dû épuiser l’espèce, il en restait encore, dans l’Asie centrale, des branches assez nombreuses et assez puissantes pour enserrer le Thibet et le nord de la Chine, de sorte que non seulement le Céleste Empire possédait, au sein des provinces du sud, des nations arianes-hindoues immigrantes à l’époque où commence son histoire, mais, de plus, il est bien difficile de ne pas admettre que les antiques peuples blancs du nord et de l’ouest, fuyant la grande irruption de leurs ennemis jaunes, n’aient pas été souvent rejetés sur la Chine et forcés de s’unir à ses populations originelles \footnote{Le Chou-king, dont on fait remonter la composition à plus de 2000 ans avant J.-C., atteste que la population de la Chine admettait les mélanges. Ainsi, je lis dans la 1\textsuperscript{re} partie, chap. II, § 20 : « Kao-Yao. Les étrangers excitent des troubles. » Et chap. III, § 6 : « Si vous êtes « appliqués aux affaires, les étrangers viendront se soumettre à vous avec obéissance. »}. Ce n’eût été, dans l’est de l’Asie, que la répétition de ce qui s’était fait au sud-ouest par les Chamites, les enfants de Sem et les Arians hellènes et zoroastriens. En tout cas, il est hors de doute que ces populations blanches des frontières orientales se montraient, à une époque très ancienne, beaucoup plus compactes qu’elles ne le pouvaient être aux débuts de notre ère. Cela suffit pour démontrer la vraisemblance, la nécessité même de fréquentes invasions et partant de fréquents mélanges \footnote{Les alliages anciens ne furent pas les seuls qui introduisirent le sang de l’espèce blanche dans les masses chinoises. Il y en eut, à des époques très rapprochées de nous, qui ont sensiblement modifié certaines populations du Céleste Empire. En 1286, Koubilai régnait et introduisait un grand nombre d’immigrants hindous et malais dans le Fo-kien. Aussi la population de cette province, comme celle du Kouang-toung, diffère-t-elle assez notablement de celle des autres contrées de la Chine. Elle est plus novatrice, plus portée vers les idées étrangères. Elle fournit le plus de monde à cette énorme émigration, qui n’est pas moindre de 3 millions d’hommes, et qui couvre aujourd’hui la Cochinchine, le Tonkin, les îles de la Sonde, Manille, Java, s’étendant chez les Birmans, à Siam, à l’île du Prince de Galles, en Australie, en Amérique. (Ritter, t. II, p. 783 et passim.) – Il vint aussi en Chine, antérieurement, sous la dynastie des Thangs, qui commença en 618 et finit en 907, de nombreux musulmans qui se sont mêlés à la population jaune et que l’on nomme aujourd’hui Hoeï-hoeï. Leur physionomie est devenue tout à fait chinoise, mais leur esprit, non. Ils sont plus énergiques que les masses qui les entourent, dont ils se font craindre et respecter (Huc, \emph{Souvenirs d’un voyage dans la Tartarie, le Thibet et la Chine}, t. II, p. 75.) – Enfin, d’autres Sémites, des juifs, ont aussi pénétré en Chine à une époque inconnue de la dynastie Tcheou (de 1122 av. notre ère à 255 après J.-C.) Ils ont exercé jadis une très grande influence et ont revêtu les premières charges de l’État. Aujourd’hui ils sont fort déchus, et beaucoup d’entre eux se sont faits musulmans. (Gaubil, \emph{Chronologie chinoise}, p. 264 et passim.)\emph{ –} Ces mélanges de sang ont eu pour conséquence des modifications importantes dans le langage. Les dialectes du sud diffèrent beaucoup du haut chinois, et l’homme du Fo-kien, du Kuang-toung ou du Yun-nan a autant de peine à comprendre le pékinois qu’un habitant de Berlin le suédois ou le hollandais. (K. F. Neumann, \emph{die Sinologen und Ihre Werke, Zeitschrift der deutschen morgenlændischen Gesellschaft}, t. I, p. 104.)}.\par
Je ne doute pas toutefois que l’influence des kschattryas du sud n’ait été d’abord dominante. L’histoire l’établit suffisamment. C’est au sud que la civilisation jeta ses premières racines, c’est de là qu’elle s’étendit dans tous les sens \footnote{Ritter, \emph{Erdkunde, Asien}, t. III, p. 714.)}.\par
On ne s’attend pas sans doute à trouver, dans des kschattryas réfractaires, des propagateurs de la doctrine brahmanique. En effet, le premier point qu’ils devaient rayer de leurs codes, c’était la supériorité d’une caste sur toutes les autres, et, pour être logiques, l’organisation même des castes. D’ailleurs, comme les Égyptiens, ils avaient quitté le gros des nations arianes à une époque où peut-être le brahmanisme lui-même n’avait pas encore complètement développé ses principes. On ne trouve donc rien en Chine qui se rattache directement au système social des Hindous ; cependant, si les rapports positifs font défaut, il n’en est pas de même des négatifs. On en rencontre de cette espèce qui donnent lieu à des rapprochements assez curieux.\par
Quand, pour cause de dissentiments théologiques, les nations zoroastriennes se séparèrent de leurs parents, elles leur témoignèrent une haine qui se manifesta par l’attribution du nom vénéré des dieux brahmaniques aux mauvais esprits et par d’autres violences de même sorte. Les kschattryas de la Chine, déjà mêlés au sang des jaunes, paraissent avoir considéré les choses sous un aspect plutôt mâle que féminin, plutôt politique que religieux, et, de ce point de vue, ils ont fait une opposition tout aussi vive que les Zoroastriens. C’est en se mettant au rebours des idées les plus naturelles qu’ils ont manifesté leur horreur contre la hiérarchie brahmanique.\par
Ils n’ont pas voulu admettre de différence de rangs, ni de situations pures ou impures résultant de la naissance. Ils ont substitué à la doctrine de leurs adversaires l’égalité absolue. Cependant, comme ils étaient poursuivis, malgré eux et en vertu de leur origine blanche, par l’idée indestructible d’une inégalité annexée à la race, ils conçurent la pensée singulière d’anoblir les pères par leurs enfants, au lieu de rester fidèles à l’antique notion de l’illustration des enfants par la gloire des pères. Impossible de voir dans cette institution, qui relève, suivant le mérite d’un homme, un certain nombre des générations ascendantes, un système emprunté aux peuples jaunes. Il ne se trouve nulle part chez eux, que là où la civilisation chinoise l’a importé. En outre, cette bizarrerie répugne à toute idée réfléchie, et, même en se mettant au point de vue chinois, elle est encore absurde. La noblesse est une prérogative honorable pour qui la possède. Si l’on veut la faire adhérer uniquement au mérite, il n’est pas besoin de lui créer un rang à part dans l’État en la forçant de monter ou de descendre autour de la personne qui en jouit. Si, au contraire, on se préoccupe de lui créer une suite, une conséquence étendue à la famille de l’homme favorisé, ce n’est pas à ses aïeux qu’il faut l’appliquer, puisqu’ils n’en peuvent jouir. Autre raison très forte : il n’y a aucune espèce d’avantage, pour celui qui reçoit une telle récompense, à en parer ses ancêtres, dans un pays où tous les ancêtres sans distinction, étant l’objet d’un culte officiel et national, sont assez respectés et même adorés. Un titre de noblesse rétrospectif n’ajoute donc que peu de chose aux honneurs dont ils jouissent. Ne cherchons pas, en conséquence, dans l’idée chinoise ce qu’elle a l’air de donner, mais bien une opposition aux doctrines brahmaniques, dont les kschattryas immigrants avaient horreur et qu’ils voulaient combattre. Le fait est d’autant plus incontestable, qu’à côté de cette noblesse fictive les Chinois n’ont pu empêcher la formation d’une autre, qui est très réelle et qui se fonde, comme partout ailleurs, sur les prérogatives de la descendance. Cette aristocratie est composée des fils, petits-fils et agnats des maisons impériales, de ceux de Confucius, de ceux de Meng-tseu, et encore de plusieurs autres personnages vénérés. À la vérité, cette classe fort nombreuse ne possède que des privilèges honorifiques ; cependant elle a, par cela seul qu’on la reconnaît, quelque chose d’inviolable, et prouve très bien que le système à rebours placé à ses côtés est une invention artificielle tout à fait contraire aux suggestions naturelles de l’esprit humain, et résultant d’une cause spéciale.\par
Cet acte de haine pour les institutions brahmaniques me semble intéressant à relever. Mis en regard de la scission zoroastrienne et des autres événements insur­rectionnels accomplis sur le sol même de l’Inde, il prouve toute la résistance que rencontra l’organisation hindoue et les répulsions irréconciliables qu’elle souleva. Le triomphe des brahmanes en est plus grand.\par
Je reviens à la Chine. Si l’on doit signaler comme une institution anti-brahmanique, et, par conséquent, comme un souvenir haineux pour la mère patrie, la création de la noblesse rétroactive, il n’est pas possible d’assigner la même origine à la forme patriarcale choisie par le gouvernement de l’empire du Milieu. Dans une conjoncture aussi grave que le choix d’une formule politique, comme il s’agit de satisfaire, non pas à des théories de personnes, ni à des idées acquises, mais à ce que les besoins des races, qui, combinées ensemble, forment l’État, réclament le plus impérieusement, il faut que ce soit la raison publique qui juge et décide, admette ou retienne en dernier ressort ce qu’on lui propose, et l’erreur ne dure jamais qu’un temps. À la Chine, la formule gou­vernementale n’ayant reçu, dans le cours des siècles, que des modifications partielles sans être jamais atteinte dans son essence, elle doit être considérée comme conforme à ce que voulait le génie national.\par
Le législateur prit pour type de l’autorité le droit du père de famille. Il établit comme un axiome inébranlable que ce principe était la force du corps social, et que, l’homme pouvant tout sur les enfants mis au monde, nourris et élevés par lui, de même le prince avait pleine autorité sur ses sujets, que, comme des enfants, il surveille, garde et défend dans leurs intérêts et dans leurs vies. Cette notion, en elle-même, et si on l’envisage d’une certaine façon, n’est pas, à proprement parler, chinoise. Elle appartient très bien à la race ariane, et, précisément, parce que, dans cette race, chaque individu isolé possédait une importance qu’il ne paraît jamais avoir eue dans les multitudes inertes des peuples jaune et noir, l’autorité de l’homme complet, du père de famille, sur ses membres, c’est-à-dire sur les personnes groupées autour de son foyer, devait être le type du gouvernement.\par
Où l’idée s’altère aussitôt que le sang arian se mêle à d’autres espèces qu’à des blancs, c’est dans les conséquences diverses tirées de ce premier principe. – Oui, disait l’Arian hindou, ou sarmate, ou grec, ou perse, ou mède, et même le Celte, oui, l’autorité paternelle est le type du gouvernement politique ; mais c’est cependant par une fiction que l’on rapproche ces deux faits. Un chef d’État n’est pas un père : il n’en a ni les affections ni les intérêts. Tandis qu’un chef de famille ne veut que très difficilement, et par une sorte de renversement des lois naturelles, le mal de sa progéniture, il se peut fort bien faire que, sans même être coupable, le prince dirige les tendances de la communauté d’une façon trop nuisible aux besoins particuliers de chacun, et, dès lors, la valeur de l’homme arian, sa dignité est compromise ; elle n’existe plus ; l’Arian n’est plus lui-même : ce n’est plus un homme.\par
Voilà le raisonnement par lequel le guerrier de race blanche arrêtait tout court le développement de la théorie patriarcale, et, en conséquence, nous avons vu les premiers rois des États hindous n’être que des magistrats électifs, pères de leurs sujets dans un sens très restreint et avec une autorité fort surveillée. Plus tard, le rajah prit des forces. Cette modification dans la nature de sa puissance ne se réalisa que lorsqu’il commanda bien moins à des Arians qu’à des métis, qu’à des noirs, et il eut d’autant moins la main libre qu’il voulut faire agir son sceptre sur des sujets plus blancs. Le sentiment politique de la race ariane ne répugne donc pas absolument à la fiction patriarcale : seulement, il la commente d’une façon précautionneuse.\par
Ce n’est pas, du reste, chez les seuls Arians hindous que nous avons déjà observé l’organisation des pouvoirs publics. Les États de l’Asie antérieure et la civilisation du Nil nous ont offert également l’application de la formule patriarcale. Les modifications qui y furent apportées à l’idée primitive se montrent non seulement très différentes de ce qu’on voit en Chine, elles le sont beaucoup aussi de ce qui s’observa dans l’Inde. Beaucoup moins libérale que dans ce dernier pays, la notion du gouvernement paternel était commentée par des populations étrangères aux sentiments raisonnables et élevés de la race dominante. Elle ne put être l’expression d’un despotisme paisible comme en Chine, parce qu’il s’agissait de dompter des multitudes mal disposées pour comprendre l’utile, et ne se courbant que devant la force brutale. La puissance fut donc, en Assyrie, terrible, impitoyable, armée du glaive, et se piqua surtout de se faire obéir. Elle n’admit pas la discussion et ne se laissa pas limiter. L’Égypte ne parut pas aussi rude. Le sang arian maintint là une ombre de ses prétentions, et les castes, moins parfaites que dans l’Inde, s’entourèrent pourtant, surtout les castes sacerdotales, de certaines immunités, de certains respects qui, ne valant pas ceux de l’Aryavarta, gardaient encore quelque reflet des nobles exigences de l’espèce blanche. Quant à la population noire, elle fut constamment traitée par les Pharaons comme la tourbe qui lui était parente l’était sur l’Euphrate, le Tigre, et aux bords de la Méditerranée.\par
 La formule patriarcale, s’adressant à des nègres, n’eut donc affaire qu’à des vaincus insensibles à tout autre argument qu’à ceux de la violence, elle devint lourdement, absolument despotique, sans pitié, sans limite, sans relâche, sans restriction, si ce n’est la révolte sanguinaire.\par
En Chine, la seconde partie de la formule fut bien différente. À coup sûr, la famille ariane qui l’apportait n’avait pas lieu de se dessaisir des droits et des devoirs du conquérant civilisateur pour proclamer sa conclusion propre. Ce n’était pas plus possible que tentant ; mais la conclusion noire ne fut pas adoptée non plus, par cette raison que les populations indigènes avaient un autre naturel et des tendances bien spéciales.\par
Le mélange malais, c’est-à-dire le produit du sang noir mêlé au type jaune, était l’élément que les kschattryas immigrants avaient à dompter, à assujettir, à civiliser, en se mêlant à lui. Il est à croire que, dans cet âge, la fusion des deux races inférieures était loin d’être aussi complète qu’on le voit aujourd’hui, et que, sur bien des points du midi de la Chine, où les civilisateurs hindous opéraient, des tribus, des fragments de tribus ou même des individualités de chaque espèce demeuraient encore à peu près pures et tenaient en échec le type opposé. Cependant il ressortait de ce mélange imparfait des besoins, des sentiments, en bloc très analogues à ceux qui ont pu se produire plus tard comme résultats d’une fusion achevée, et les blancs se voyaient là aux prises avec des nécessités d’un ordre tout différent de celles auxquelles leurs congénères vainqueurs dans l’Asie occidentale avaient été forcés de se plier.\par
La race malaise, je l’ai déjà définie : sans être susceptible de grands élans d’imagina­tion, elle n’est pas hors d’état de comprendre les avantages d’une organisation régulière et coordonnée. Elle a des goûts de bien-être, comme l’espèce jaune tout entière, et de bien-être exclusivement matériel. Elle est patiente, apathique, et subit aisément la loi, s’arrangeant, sans difficulté, de façon à en tirer les avantages qu’un état social comporte, et à en subir la pression sans trop d’humeur.\par
Avec des gens animés de pareilles dispositions, il n’y avait pas lieu à ce despotisme violent et brutal qu’amenèrent la stupidité des noirs et l’avilissement graduel des Chamites, devenus trop près parents de leurs sujets et participant à leurs incapacités. Au contraire, en Chine, quand les mélanges eurent commencé à énerver l’esprit arian, il se trouva que ce noble élément, à mesure qu’en se subdivisant il se répandait dans les masses, relevait d’autant les dispositions natives des peuples. Il ne leur donnait pas, assurément, sa souplesse, son énergie généreuse, son goût de la liberté. Toutefois, il confirmait leur amour instinctif de la règle, de l’ordre, leur antipathie pour les abus d’imagination. Qu’un souverain d’Assyrie se plongeât dans des cruautés exorbitantes, que, pareil à ce Zohak ninivite dont la tradition persane raconte les horreurs, il nourrît de la chair et du sang de ses sujets les serpents bourgeonnants sur son corps, le peuple en souffrait, sans doute ; mais comme les têtes s’exaltaient devant de tels tableaux ! Comme, au fond, le Sémite comprenait bien l’exagération passionnée des actes de la toute-puissance et comme la férocité la plus dépravée en grandissait encore à ses yeux l’image gigantesque ! Un prince doux et tranquille risquait, chez lui, de devenir un objet de dédain.\par
Les Chinois ne concevaient pas ainsi les choses. Esprits très prosaïques, l’excès leur faisait horreur, le sentiment public s’en révoltait, et le monarque qui s’en rendait coupable perdait aussitôt tout prestige et détruisait tout respect pour son autorité.\par
Il arriva donc, en ce pays, que le principe du gouvernement fut le patriarcat, parce que les civilisateurs étaient Arians, que son application fut le pouvoir absolu, parce que les Arians agissaient en vainqueurs et en maîtres au milieu de populations infé­rieures ; mais que, dans la pratique, l’absolutisme du souverain ne se manifesta ni par des traits d’orgueil surhumain, ni par des actes de despotisme repoussant, et se renferma entre des limites généralement étroites, parce que le sens malais n’appelait pas de trop grosses démonstrations d’arrogance, et que l’esprit arian, en se mêlant à lui, y trouvait un fond disposé à comprendre de mieux en mieux que le salut d’un État est dans l’observance des lois, aussi bien sur les hauteurs sociales que dans les bas-fonds.\par
Voilà le gouvernement de l’empire du Milieu organisé. Le roi est le père de ses sujets, il a droit à leur soumission entière, il devient pour eux le mandataire de la Divinité, et on ne l’approche qu’à genoux. Ce qu’il veut, il le peut théoriquement ; mais, dans la pratique, s’il veut une énormité, il a bien de la peine à l’accomplir. La nation se montre irritée, les mandarins font entendre des représentations, les ministres, proster­nés aux pieds du trône impérial, gémissent tout haut des aberrations du père commun, et le père commun, au milieu de ce \emph{tolle} général, reste le maître de pousser sa fantaisie jusqu’au bout, à la seule condition de rompre avec ce qu’on lui a appris, dès l’enfance, à tenir pour sacré et inviolable. Il se voit isolé et n’ignore pas que, s’il continue dans la route où il s’engage, l’insurrection est au bout.\par
Les annales chinoises sont éloquentes sur ce sujet. Dans les premières dynasties, ce qu’on raconte des méfaits des empereurs réprouvés aurait paru bien véniel aux historiens d’Assyrie, de Tyr ou de Chanaan. J’en veux donner un exemple.\par
L’empereur Yeou-wang, de la dynastie de Tcheou, qui monta sur le trône 781 ans avant J.-C., régna trois ans sans qu’on eût aucun reproche grave à lui faire. La troisième année, il devint amoureux d’une fille nommée Pao-sse, et s’abandonna sans réserve à la fougue de ce sentiment. Pao-sse lui donna un fils, qu’il nomma Pe-fou, et qu’il voulut instituer prince héritier à la place de l’aîné, Y-kieou. Pour y parvenir, il exila l’impéra­trice et son fils, ce qui mit le comble au mécontentement déjà éveillé par une conduite qui n’était pas conforme aux rites. De tous côtés l’opposition éclata.\par
Les grands de l’empire firent assaut d’observations respectueuses auprès de l’empe­reur. On demanda, de toutes parts, l’éloignement de Pao-sse, on l’accusa d’épuiser l’État par ses dépenses, de détourner le souverain de ses devoirs. Des satires violentes couraient de toutes parts, répétées par les populations. De leur côté, les parents de l’impératrice s’étaient réfugiés, avec elle, chez les Tartares, et on s’attendait à une invasion de ces terribles voisins, crainte qui n’augmentait pas peu la fureur générale. L’empereur aimait éperdument Pao-sse et ne cédait pas.\par
Toutefois, comme à son tour il redoutait, non sans raison, l’alliance des mécontents avec les hordes de la frontière, il réunit des troupes, les plaça dans des positions convenables, et ordonna qu’en cas d’alarme on allumât des feux et battît du tambour, auquel signal tous les généraux auraient à accourir, avec leur monde, pour tenir tête à l’ennemi.\par
Pao-sse était d’un caractère très sérieux. L’empereur se consumait perpétuellement en efforts pour attirer sur ses lèvres un sourire. C’était grand hasard quand il y réussissait, et rien ne lui était plus agréable. Un jour, une panique soudaine se répandit partout, les gardiens des signaux crurent que les cavaliers tartares avaient franchi les limites et approchaient ; ils mirent promptement le feu aux bûchers qu’on avait préparés, et aussitôt tous les tambours de battre. À ce bruit, princes et généraux, rassemblant leurs troupes, accoururent ; on ne voyait que gens en armes, se hâtant deçà et delà et demandant où était l’ennemi, que personne ne voyait, puisqu’il n’existait pas et que l’alerte était fausse.\par
Il paraît que les visages animés des chefs et leurs attitudes belliqueuses parurent souverainement ridicules à la sérieuse Pao-sse, car elle se mit à rire. Ce que voyant, l’empereur se déclara au comble de la joie. Il n’en fut pas de même des graves plastrons de tant de bonne humeur. Ils se retirèrent profondément blessés, et la fin de l’histoire est que, lorsque les Tartares parurent pour de bon, personne ne vint au signal, l’empereur fut pris et tué, Pao-sse enlevée, son fils dégradé, et tout rentra dans l’ordre sous la domination d’Y-kieou, qui prit la couronne sous le nom de Ping-wang \footnote{Gaubil, \emph{Traité de la chronologie chinoise}, p. 111.}.\par
En voilà assez pour montrer combien, en fait, l’autorité absolue des empereurs était limitée par l’opinion publique et par les mœurs ; et c’est ainsi que l’on a toujours vu, en Chine, la tyrannie n’apparaître que comme un accident constamment détesté, réprimé, et qui ne se perpétue guère, parce que le naturel de la race gouvernée ne s’y prête pas. L’empereur est, sans doute, le maître des États du Milieu, voire, par une fiction plus hardie, du monde entier, et tout ce qui se refuse à son obéissance est, par cela même, réputé barbare et en dehors de toute civilisation. Mais, tandis que la chancellerie chinoise s’épuise en formules de respect lorsqu’elle s’adresse au Fils du ciel, l’usage ne permet pas à celui-ci de s’exprimer, sur son propre compte, d’une manière aussi pompeuse. Son langage affecte une extrême modestie : le prince se représente comme au-dessous, par son petit mérite et sa vertu médiocre, des sublimes fonctions que son auguste père a confiées à son insuffisance. Il conserve toute la phraséologie douce et affectueuse du langage domestique, et ne manque pas une occasion de protester de son ardent amour pour le bien de ses chers enfants : ce sont ses sujets \footnote{J. F. Davis, \emph{The Chinese}, p. 178.}.\par
 L’autorité est donc, de fait, assez bornée, car je n’ai pas besoin de dire que, dans cet empire, dont les principes gouvernementaux n’ont jamais varié, quant à l’essentiel, ce qui était considéré comme bon autrefois est devenu, pour cela seul, meilleur aujourd’hui. La tradition est toute-puissante \footnote{« En Chine, l’empire n’a pas passé d’un peuple à l’autre, et les traditions sont restées « nécessai­rement plus familières et ont pénétré plus profondément dans les esprits que chez « nous. » (Jules Mohl, \emph{Rapport tait à la Société asiatique}, 1851, p. 85.)}, et c’est déjà une tyrannie, dans un empereur, que de s’éloigner, pour le moindre détail, de l’usage suivi par les ancêtres. Bref, le Fils du ciel peut tout, à condition de ne rien vouloir que de déjà connu et approuvé.\par
Il était naturel que la civilisation chinoise, s’appuyant, à son début, sur des peuples malais, et plus tard sur des agglomérations de races jaunes, mélangées de quelques Arians, fût invinciblement dirigée vers l’utilité matérielle \footnote{J’ai mentionné plus haut que des infiltrations blanches, assez importantes, avaient gagné la Chine, à différentes époques. Cependant l’avantage du nombre reste toujours à la race jaune, d’abord parce que le fond primitif lui appartient, ensuite parce que des immigrations mongoles se sont effectuées, de tout temps, qui ont augmenté la force de la masse nationale. C’est ainsi qu’une invasion de Tartares, considérée comme la première, avait lieu en 1531 avant J.-C. (Gaubil, \emph{Chronologie chinoise}, p. 28.) – C’est encore ainsi que de la Sibérie venait, en 398 de notre ère, la dynastie des Weï je n’insiste pas trop sur ce dernier fait, que pourrait bien recouvrir une immixtion de métis blancs et jaunes. (A. de Humboldt, \emph{Asie centrale}, t. I, p. 27.}. Tandis que, dans les grandes civilisations du monde antique occidental, l’administration proprement dite et la police n’étaient que des objets fort secondaires et à peine ébauchés, ce fut, en Chine, la grande affaire du pouvoir, et on rejeta tout à fait sur l’arrière-plan les deux questions qui ailleurs l’emportaient : la guerre et les relations diplomatiques.\par
On admit en principe éternel que, pour que l’État se maintînt dans une situation normale, il fallait que les vivres s’y trouvassent abondamment, que chacun pût se vêtir, se nourrir et se loger ; que l’agriculture reçût des encouragements perpétuels, non moins que l’industrie ; et, comme moyen suprême d’arriver à ces fins, il fallait par-dessus tout une tranquillité solide et profonde, et des précautions minutieuses contre tout ce qui était capable d’émouvoir les populations ou de troubler l’ordre. Si la race noire avait exercé quelque action influente dans l’empire, il n’est pas douteux que nul de ces préceptes n’eût tenu longtemps. Les peuples jaunes, au contraire, gagnant chaque jour du terrain, et comprenant l’utilité de cet ordre de choses, ne trouvaient rien en eux qui n’appréciât vivement le bonheur matériel dans lequel on voulait les ensevelir. Les théories philosophiques et les opinions religieuses, ces brandons ordinaires de l’incendie des États, restèrent à jamais sans force devant l’inertie nationale, qui, bien repue de riz et avec son habit de coton sur le dos, ne se soucia pas d’affronter le bâton des hommes de police pour la plus grande gloire d’une abstraction \footnote{W. v. Schlegel, \emph{Indische Bibliothek}, t. II, p. 214 : « L’idée du bonheur est représentée en « Chine, à ce que l’on m’assure, par un plat de riz bouilli et une bouche ouverte ; celle du « gouvernement, par une canne de bambou et par un second caractère qui signifie \emph{agiter} « \emph{l’air.} »}.\par
Le gouvernement chinois laissa prêcher tout, affirmer tout, enseigner les absurdités les plus monstrueuses, à la condition que rien, dans les nouveautés les plus hardies, ne tendrait à un résultat social quelconque. Aussitôt que cette barrière menaçait d’être franchie, l’administration agissait sans pitié et réprimait les innovations avec une sévérité inouïe, confirmée par les dispositions constantes de l’opinion publique \footnote{La vigilance de la police chinoise est incomparable. on sait toutes les inquiétudes que les Russes et les Anglais inspirent au cabinet impérial dans le sud-ouest. Le voyageur Burnes donne un exemple des précautions qui sont prises : le signalement et même le portrait de tout étranger suspect est envoyé aux villes du haut Turkestan avec l’ordre de tuer l’original, s’il est saisi au delà de la frontière. Moorcroft avait été si bien représenté sur les murs de Yarkend, et sa physionomie anglaise si parfaitement saisie, que c’était à faire reculer le plus audacieux de ses compatriotes qui aurait pu se voir exposé aux suites d’une confrontation. (Burnes, \emph{Travels}, t. II, p. 233.)}.\par
Dans l’Inde, le brahmanisme avait installé, lui aussi, une administration bien supérieure à ce que les États chamites, sémites ou égyptiens possédèrent jamais. Cependant, cette administration n’occupait pas le premier rang dans l’État, où les préoccupations créatrices de l’intelligence réclamaient la meilleure part de l’attention. Il ne faut donc pas s’étonner si le génie hindou, dans sa liberté, dans sa fierté, dans son, goût pour les grandes choses et dans ses théories surhumaines, ne regardait, en définitive, les intérêts matériels que comme un point secondaire. Il était, d’ailleurs, sensiblement encouragé dans une telle opinion par les suggestions de l’alliage noir. À la Chine, l’apogée fut donc atteint en matière d’organisation matérielle, et, en tenant compte de la différence des races, qui nécessite des procédés différents, il me semble qu’on peut admettre que, sous ce rapport, le Céleste Empire obtint des résultats beaucoup plus parfaits et surtout plus continus qu’on ne le voit dans les pays de l’Europe moderne, depuis que les gouvernements se sont particulièrement appliqués à cette branche de la politique. En tout cas, l’empire romain n’y est pas comparable.\par
Cependant, il faut aussi en convenir, c’est un spectacle sans beauté et sans dignité. Si cette multitude jaune est paisible et soumise, c’est à la condition de rester, à tout jamais, privée des sentiments étrangers à la plus humble notion de l’utilité physique. Sa religion est un résumé de pratiques et de maximes qui rappellent fort bien ce que les moralistes genevois et leurs livres d’éducation se plaisent à recommander comme le \emph{nec plus ultra} du bien : l’économie, la retenue, la prudence, l’art de gagner et de ne jamais perdre. La politesse chinoise n’est qu’une application de ces principes. C’est, pour me servir du mot anglais, un \emph{cant} perpétuel, qui n’a nullement pour raison d’être, comme la courtoisie de notre moyen âge, cette noble bienveillance de l’homme libre envers ses égaux, cette déférence pleine de gravité envers les supérieurs, cette affectueuse condescendance envers les inférieurs ; ce n’est qu’un devoir social, qui, prenant sa source dans l’égoïsme le plus grossier, se traduit par une abjecte prosternation devant les supérieurs, un ridicule combat de cérémonies avec les égaux et une arrogance avec les inférieurs qui s’augmente dans la proportion où décroît le rang de ceux-ci. La politesse est ainsi plutôt une invention formaliste, pour tenir chacun à sa place, qu’une inspiration du cœur. Les cérémonies que chacun doit faire, dans les actes les plus ordinaires de la vie, sont réglées par des lois tout aussi obligatoires et aussi rigoureuses que celles qui portent sur des sujets en apparence plus essentiels.\par
 La littérature est une grande affaire pour le Chinois. Loin de se rendre, comme partout ailleurs, un moyen de perfectionnement, elle est devenue, au contraire, un agent puissant de stagnation. Le gouvernement se montre grand ami des lumières ; il faut seulement savoir comment lui et l’opinion publique l’entendent. Dans les 300 millions d’âmes, attribués généralement à l’empire du Milieu, qui, suivant la juste expression de M. Ritter, compose à lui seul un monde, il est très peu d’hommes, même dans les plus basses classes, qui ne sachent lire et écrire suffisamment pour les besoins ordinaires de la vie, et l’administration a soin que cette instruction soit aussi générale que possible. La sollicitude du pouvoir va encore au delà. Il veut que chaque sujet connaisse les lois ; on prend toutes les mesures nécessaires pour qu’il en soit ainsi. Les textes sont mis à la portée de tout le monde, et, de plus, des lectures publiques s’exécutent aux jours de nouvelle lune, afin de bien inculquer aux sujets les prescriptions essentielles, telles que les devoirs des enfants envers leurs parents et, partant, des citoyens envers l’empereur et les magistrats. De cette façon, le peuple chinois est, très certainement, ce qu’on appelle, de nos jours, plus avancé que nos Européens. Dans l’antiquité asiatique, grecque et romaine, la pensée d’une comparaison ne peut pas même se présenter.\par
Ainsi, instruit dans le plus indispensable, le bas peuple comprend que la première chose pour arriver aux fonctions publiques, c’est de se rendre capable de subir les examens. Voilà encore un puissant encouragement à apprendre \footnote{« Le principe de l’admission aux fonctions administratives, c’est le choix au village, la « promotion au district. Sans ces principes fondamentaux, il serait difficile de chercher à « gouverner l’empire. » (Tcheou-li, \emph{Commentaire Weï-kiao}, sur le § 36 du livre XI, t. I, p. 261.)}. On apprend donc. Et quoi ? On apprend ce qui est utile, et là est l’infranchissable point d’arrêt. Ce qui est utile, c’est ce qui a toujours été su et pratiqué, ce qui ne peut donner matière à discussion. Il faut apprendre, mais ce que les générations précédentes ont su avant vous, et comme elles l’ont su : toute prétention à créer du nouveau, dans ce sens, conduirait l’étudiant à se voir repousser de l’examen, et, s’il s’obstinait, à un procès de trahison où personne ne lui ferait grâce. Aussi n’est-il personne qui se risque à de tels hasards, et, dans ce champ de l’éducation et de la science chinoises, si constamment, si exemplairement labouré, il n’y a pas la moindre chance qu’une idée inconnue lève jamais la tête. Elle serait arrachée sur l’heure avec indignation \footnote{L’amour du médiocre est de principe. Voici la maxime : « Le ministre de Chine Kao-yao fit « connaître les punitions différentes et dit : « Le peuple est uni dans le juste milieu. Ainsi, « c’est par les châtiments que l’on instruit les hommes à garder le juste milieu. » Il n’est pas « d’étudiant qui ne tienne pour dûment prévenu et n’évite d’avoir plus d’esprit qu’il ne « convient. » (\emph{Tcheou-li}, t. I, p. 197.)}.\par
Dans la littérature proprement dite, le bout-rimé et toutes les distractions ingénieusement puériles qui y ressemblent, sont tenues en grand honneur. Des élégies assez douces, des descriptions de la nature plus minutieuses que pittoresques, bien que non sans grâce, voilà le meilleur. Le réellement bon, c’est le roman. Ces peuples sans imagination ont beaucoup d’esprit d’observation et de finesse, et telle production issue de ces deux qualités rappelle chez eux, et peut-être en les dépassant, les œuvres anglaises destinées à peindre la vie du grand monde. Là s’arrête le vol de la muse chinoise. Le drame est mal conçu et assez plat. L’ode à la façon de Pindare n’a jamais passé par l’esprit de cette nation rassise. Quand le poète chinois se bat les flancs pour échauffer sa verve, il se jette à plein corps dans les nuages, fait intervenir les dragons de toute couleur, s’essouffle, et ne saisit rien que le ridicule.\par
La philosophie, et surtout la philosophie morale, objet d’une grande prédilection, ne consiste qu’en maximes usuelles, dont l’observance parfaite serait assurément fort méritoire, mais qui, par la manière puérilement obscure et sèchement didactique dont elles sont exposées et déduites, ne constituent pas une branche de connaissances très dignes d’admiration \footnote{Il n’y a pas de philosophie possible là où les rites ont réglé d’avance jusqu’aux plus petits détails de la vie, et où tous les intérêts matériels conspirent également à étouffer la pensée. M. Ritter remarque très bien que la Chine s’est arrangée de façon à former un monde à elle seule et que la nature servait cette pensée. De tous côtés, le pays est peu accessible. Le gouvernement n’a pas voulu changer cette situation en créant des routes. À part le voisinage de Pékin, deux chemins entre le Kuang-toung et le Kiang-si, les passages du Thibet et quelques voies impériales en très petit nombre, les moyens de communication font absolument défaut, et non seulement la politique ne veut pas de rapports avec les autres pays de la terre, mais elle s’oppose même, avec une persistante énergie, à toutes relations suivies entre les provinces. (Ritter, \emph{ouvr. cité}, p. 727 et passim.)}. Les gros ouvrages scientifiques donnent lieu à plus d’éloges.\par
À la vérité, ces compilations verbeuses manquent de critique. L’esprit de la race jaune n’est ni assez profond, ni assez sagace pour saisir cette qualité réservée à l’espèce blanche. Toutefois, on peut encore beaucoup apprendre et recueillir dans les documents historiques \footnote{Ce jugement n’est pas absolu, il comporte des exceptions, et on en doit faire une notable, par exemple, en faveur de Matouan-lin.}. Ce qui a trait aux sciences naturelles est quelquefois précieux, surtout par l’exactitude de l’observation et la patience des artistes à reproduire les plantes et les animaux connus. Mais il ne faut pas s’attendre à des théories générales. Quand la fantaisie vague d’en créer passe par l’esprit des lettrés, ils tombent aussitôt au-dessous de la niaiserie. On ne les verra pas, comme les Hindous ou les peuples sémitiques, inventer des fables qui, dans leur incohérence, sont du moins grandioses ou séduisantes. Non : leur conception restera uniquement lourde et pédantesque. Ils vous conteront gravement, comme un fait incontestable, la transformation du crapaud en tel ou tel animal. Il n’y a rien à dire de leur astronomie. Elle peut fournir quelques lueurs aux travaux difficiles des chronologistes, sans que sa valeur intrinsèque, corrélative à celle des instruments qu’elle emploie, cesse d’être très médiocre. Les Chinois l’ont reconnu eux-mêmes par leur estime pour les missionnaires jésuites. Ils les chargeaient de redresser leurs observations et de travailler même à leurs almanachs.\par
 En somme, ils aiment la science dans sa partie d’application immédiate \footnote{Ainsi, ils entendent bien la littérature utilitaire. Ils ont de bons routiers (une Encyclopédie agricole), d’où l’on a déjà extrait et traduit d’excellents renseignements sur la culture du mûrier et l’élève des vers à soie. (J. Mohl, \emph{Rapport fait à la Société asiatique de Paris}, 1851, p. 83.) – M. le baron A. de Humboldt a pu louer avec vérité, au sujet de la géographie et de l’histoire, les documents chinois, « dont les surprenantes richesses embrassent une immense étendue du continent (\emph{Asie centrale, introduction}, t. I, p. XXXIII) », et il dit encore très bien : «  Dans les grandes « monarchies, en Chine comme dans l’empire persan, « divisées en satrapies, on a senti de bonne heure le besoin d’ouvrages descriptifs, de ces « tableaux statistiques détaillés pour lesquels, en Europe, les peuples de l’antiquité les plus « spirituels et les plus lettrés ont montré si peu de penchant. Un gouvernement « pédantesquement réglé dans les moindres détails de son administration, embrassant tant « de tribus de races diverses, nécessitait, en même temps, de nombreux bureaux « d’interprètes. Il existait, dès l’an 1407, des collèges établis dans les grandes villes des « frontières, où l’on enseignait huit à dix langues à la fois. C’est ainsi que la vaste étendue « de l’empire et les exigences d’un gouvernement despotique et central favorisaient « simultanément la géographie et la littérature linguistique. » (\emph{Asie centrale}, t. I, p. 29.)}. Pour ce qui est grand, sublime, fécond, d’une part, ils ne peuvent y atteindre, de l’autre, ils le redoutent et l’excluent avec soin. Des savants très appréciés à Pékin auraient été Trissotin et ses amis.\par

\begin{verse}
Pour avoir eu, trente ans, des yeux et des oreilles ;\\
Pour avoir employé neuf à dix mille veilles\\
À savoir ce qu’ont dit les autres avant eux.\\
\end{verse}
\noindent Le sarcasme de Molière ne serait pas compris dans un pays où la littérature est tombée en enfance aux mains d’une race dont l’esprit arian s’est complètement noyé dans les éléments jaunes, race composite, pourvue de certains mérites qui ne renferment pas ceux de l’invention et de la hardiesse.\par
En fait d’art, il y a moins à approuver encore. Je parlais, tout à l’heure, de l’exactitude des peintres de fleurs et de plantes. On connaît, en Europe, la délicatesse de leur pinceau. Dans le portrait, ils obtiennent aussi des succès honorables, et, assez habiles à saisir le caractère des physionomies, ils peuvent lutter avec les plats chefs-d’œuvre du daguerréotype. Puis, c’est là tout. Les grandes peintures sont bizarres, sans génie, sans énergie, sans goût. La sculpture se borne à des représentations mons­trueuses et communes. Les vases ont les formes qu’on leur connaît. Cherchant le bizarre et l’inattendu, leurs bronzes sont conçus dans le même sentiment que leurs porcelaines. Pour l’architecture, ils préfèrent à tout ces pagodes à huit étages dont l’invention ne vient pas complètement d’eux, ayant quelque chose d’hindou dans l’ensemble ; mais les détails leur en appartiennent, et, si l’œil qui ne les a pas encore observées peut être séduit par la nouveauté, il se dégoûte bientôt de cette uniformité excentrique. Dans ces constructions, rien n’est solide, rien n’est en état de braver les siècles. Les Chinois sont trop prudents et trop bons calculateurs pour employer à la construction d’un édifice plus de capitaux qu’il n’est besoin. Leurs travaux les plus remarquables ressortent tous du principe d’utilité : tels les innombrables canaux dont l’empire est traversé, les digues, les levées pour prévenir les inondations, surtout celles du Hoang-ho. Nous retrouvons là le Chinois sur son véritable terrain. Répétons-le donc une dernière fois : les populations du Céleste Empire sont exclusivement utilitaires ; elles le sont tellement, qu’elles ont pu admettre, sans danger, deux institutions qui paraissent peu compatibles avec tout gouvernement régulier : les assemblées populai­res réunies spontanément pour blâmer ou approuver la conduite des magistrats et l’indépendance de la presse \footnote{Davis, \emph{the Chinese}, p. 99 : « The people sometimes hold public meetings by advertisement, « for the express purpose of addressing the magistrate and this without being punished. The « influence of public opinion seems indicated by this practice ; together with that frequent « custom of placarding and lampooning (though of course anonymously) obnoxious « officers. Honours are rendered to a just magistrate, and addresses presented to him on his « departure by the people ; testimonies which are highly valued... It may be added, that « there is no established censorship of the press in China, nor any limitations but those « which the interests of social peace and order seem to render necessary. If these are « endangered, the process of the government is of course more « summary than even an « information filed by the attorney general. » – Le système chinois me semble s’accorder encore avec une autre idée adoptée par les écoles libérales d’Europe : c’est la \emph{sécularisation} du système militaire. Ils ne connaissent que la garde nationale ou la landwehr. Je ne parle pas ici des Mantchous, mais seulement des véritables indigènes de l’empire. Les Mantchous, étant tous soldats de naissance, sont censés plus habiles sur le maniement des armes. (Davis, p. 105.)}. On ne prohibe, en Chine, ni la libre réunion, ni la diffusion des idées \footnote{On consulte le peuple en des occasions fort graves, par exemple, en matière de justice criminelle. Ainsi, Je lis dans le commentaire de Tching-khang-tching, sur le 26\textsuperscript{e} § du livre XXXV du \emph{Tcheou-li} : « Si le peuple dit : Tuez ! le sous-préposé aux brigands tue. Si le « peuple dit : Faites grâce ! alors, il fait grâce. » Et un autre commentateur, Wang-tchao-yu, ajoute : « Lorsque le peuple pense qu’on doit exécuter le coupable, on applique sans « incertitude les peines supérieures... Lorsque le peuple pense qu’il faut gracier, on « n’accorde pas la grâce pleine et entière. Seulement on applique les peines inférieures, « qui sont moindres que les premières. » (\emph{Tcheou-li}, t. I. p. 323.)}. Il va sans dire, toutefois, que lorsque l’abus se montre, ou, pour mieux dire, que si l’abus se montrait, la répression serait aussi prompte qu’implacable, et aurait lieu sous la direction des lois contre la trahison.\par
On en conviendra : quelle solidité, quelle force n’a pas une organisation sociale qui peut permettre de telles déviations à son principe et qui n’a jamais vu sortir de sa tolérance le moindre inconvénient !\par
L’administration chinoise a atteint, dans la sphère des intérêts matériels, à des résultats auxquels nulle autre nation antique ou moderne n’est jamais parvenue \footnote{Le commentaire de Tching-khang-tching sur le 9\textsuperscript{e} verset du livre VII du \emph{Tcheou-li} donne une excellente formule de la cité chinoise. La voici : « Un royaume est constitué par « l’établissement du marché et du palais dans la capitale. L’empereur établit le palais ; « l’impératrice établit le marché. C’est le symbole de la concordance parfaite des deux « principes mâle et femelle qui président au mouvement et au repos. » (\emph{Tcheou-li}, t. I, p. 145.)} ; instruction populaire partout propagée, bien-être des sujets, liberté entière dans la sphère permise, développements industriels et agricoles des plus complets, production aux prix les plus médiocres, et qui rendraient toute concurrence européenne difficile avec les denrées de consommation ordinaire, comme le coton, la soie, la poterie. Tels sont les résultats incontestables dont le système chinois peut se vanter \footnote{Vers l’an 1070 (de notre ère), le Premier ministre de l’empereur Chin-tsong, nommé « Wang-« ngan-tchi, introduisit des changements dans les droits des marchés et \emph{institua un} « \emph{nouveau système d’avances en grains faites aux cultivateurs}. » Voilà des idées tout à fait analogues à celles que, depuis soixante ans seulement, on déclare, en Europe, dominer, en importance, toutes les autres notions politiques. (Voir \emph{Tcheou-li}, t. I, introd., p. XXII.)}.\par
Il est impossible ici de se défendre de la réflexion que, si les doctrines de ces écoles que nous appelons socialistes venaient jamais à s’appliquer et à réussir dans les États de l’Europe, le \emph{nec plus ultra} du bien serait d’obtenir ce que les Chinois sont parvenus à immobiliser chez eux. Il est certain, dans tous les cas, et il faut le reconnaître à la gloire de la logique, que les chefs de ces écoles n’ont pas le moins du monde repoussé la condition première et indispensable du succès de leurs idées, qui est le despotisme. Ils ont très bien admis, comme les politiques du Céleste Empire, qu’on ne force pas les nations à suivre une règle précise et exacte, si la loi n’est pas armée, en tout temps, d’une complète et spontanée initiative de répression. Pour introniser leur régime, ils ne se refuseraient pas à tyranniser. Le triomphe serait à ce prix, et une fois la doctrine établie, l’universalité des hommes aurait la nourriture, le logement, l’instruction pratique assurés. Il ne serait plus besoin de s’occuper des questions posées sur la circulation du capital, l’organisation du crédit, le droit au travail et autres détails \footnote{« C’est un système étonnant (l’organisation chinoise), reposant sur une idée unique, celle de « l’État chargé de pourvoir à tout ce qui peut contribuer au bien public et subordonnant « l’action de chacun à ce but suprême. Tcheou-kong a dépassé, dans son organisation, tout « ce que les États modernes les plus centralisés et les plus bureaucratiques ont essayé, et il « s’est rapproché en beaucoup de choses de ce que tentent certaines théories socialistes de « notre temps... » (J. Mohl, \emph{Rapport fait à la Société asiatique}, 1851, p. 89.)}.\par
Il y a, sans doute, quelque chose, en Chine, qui semble répugner aux allures des théories socialistes. Bien que démocratique dans sa source, puisqu’il sort des concours et des examens publics, le mandarinat est entouré de bien des prérogatives et d’un éclat gênant pour les idées égalitaires. De même, le chef de l’État, qui, en principe, n’est pas nécessairement issu d’une maison régnante (car, dans les temps anciens, règle toujours présente, plus d’un empereur n’a été proclamé que pour son mérite), ce souverain, choisi parmi les fils de son prédécesseur et sans égard à l’ordre de naissance, est trop vénéré et placé trop haut au-dessus de la foule. Ce sont là, en apparence, autant d’oppositions aux idées sur lesquelles bâtissent les phalanstériens et leurs émules.\par
Cependant, si l’on consent à y réfléchir, on verra que ces distinctions ne sont que des résultats auxquels M. Fourrier et Proudhon, chefs d’État, seraient eux-mêmes amenés bientôt. Dans des pays où le bien-être matériel est tout et où, pour le conserver, il convient de retenir la foule entre les limites d’une organisation stricte, la loi, immuable comme Dieu (car si elle ne l’était pas, le bien-être public serait sans cesse exposé aux plus graves revirements), doit finir, un jour ou l’autre, par participer aux respects rendus à l’intelligence suprême. Ce n’est plus de la soumission qu’il faut à une loi si préservatrice, si nécessaire, si inviolable, c’est de l’adoration, et on ne saurait aller trop loin dans cette voie. Il est donc naturel que les puissances qu’elle institue pour répandre ses bienfaits et veiller à son salut, participent du culte qu’on lui accorde ; et comme ces puissances sont bien armées de toute sa rigueur, il est inévitable qu’elles sauront se faire rendre ce qu’elles ne seront pas les dernières à juger leur être dû.\par
J’avoue que tant de bienfaits, conséquences de tant de conditions, ne me paraissent pas séduisants. Sacrifier sur la huche du boulanger, sur le seuil d’une demeure confortable, sur le banc d’une école primaire, ce que la science a de transcendantal, la poésie de sublime, les arts de magnifique, jeter là tout sentiment de dignité humaine. abdiquer son individualité dans ce qu’elle a de plus précieux : le droit d’apprendre et de savoir, de communiquer à autrui ce qui n’était pas su auparavant, c’est trop, c’est trop donner aux appétits de la matière. Je serais bien effrayé de voir un tel genre de bonheur menacer nous ou nos descendants, si je n’étais rassuré par la conviction que nos générations actuelles ne sont pas encore capables de se plier à de pareilles jouissances au prix de pareils sacrifices. Nous pouvons bien inventer des alcorans de toutes sortes ; mais cette féconde variabilité, à laquelle je suis loin d’applaudir, a les revers de ses défauts. Nous ne sommes pas gens capables de mettre en pratique tout ce que nous imaginons. À nos plus hautes folies d’autres succèdent, qui les font négliger. Les Chinois s’estimeront encore les premiers administrateurs du monde, qu’oublieux de toutes propositions de les imiter, nous aurons passé à quelque nouvelle phase de nos histoires, hélas ! si bariolées !\par
Les annales du Céleste Empire sont uniformes. La race blanche, auteur premier de la civilisation chinoise, ne s’est jamais renouvelée d’une manière suffisante pour faire dévier de leurs instincts naturels des populations immenses. Les adjonctions qui se sont accomplies, à différentes époques, ont généralement appartenu à un même élément, à l’espèce jaune. Elles n’ont apporté presque rien de nouveau, elles n’ont fait que contribuer à étendre les principes blancs en les délayant dans des masses d’autre nature et de plus en plus fortes. Quant à elles-mêmes, trouvant une civilisation conforme à leurs instincts, elles l’ont embrassée volontiers et ont toujours fini par se perdre au sein de l’océan social, où leur présence n’a, cependant, pas laissé que de déterminer plusieurs perturbations légères, qu’il n’est pas impossible de démêler et de constater. Je vais l’essayer en reprenant les choses de plus haut.\par
Lorsque les Arians commencèrent à civiliser les mélanges noirs et jaunes, autrement dit malais, qu’ils trouvèrent en possession des provinces du sud, ils leur portèrent, ai-je dit, le gouvernement patriarcal, forme susceptible de différentes applications, restric­tives ou extensives. Nous avons vu que cette forme, appliquée aux noirs, dégénère rapidement en despotisme dur et exalté, et que, chez les Malais, et surtout chez les peuples plus purement jaunes, si le despotisme est entier, il est, au moins, tempéré dans son action et forcé de s’interdire les excès inutiles, faute d’imagination chez les sujets pour en être plus effrayés qu’irrités, pour les comprendre et les tolérer. Ainsi s’explique la constitution particulière de la royauté en Chine.\par
Mais un rapport général de la première constitution politique de ce pays avec les organisations spéciales de tous les rameaux blancs, rapport curieux que je n’ai pas encore fait ressortir, c’est l’institution fragmentaire de l’autorité et sa dissémination en un grand nombre de souverainetés plus ou moins unies par le lien commun d’un pouvoir suprême. Cette sorte d’éparpillement de forces, nous l’avons vue en Assyrie, où les Chamites, puis les Sémites, fondèrent tant d’États isolés sous la suzeraineté, reconnue ou contestée, suivant les temps, de Babylone et de Ninive ; dissémination si extrême, qu’après les revers des descendants de Salomon il se créa trente-deux États distincts dans les seuls débris des conquêtes de David, du côté de l’Euphrate \footnote{Movers, \emph{das Phœnizische Alterthum}, t. II, 1\textsuperscript{re} partie, p. 374. – I, \emph{Rois}, 20, 24, 25.}. En Égypte, avant Ménès, le pays était également divisé entre plusieurs princes, et il en fut de même du côté de l’Inde, où le caractère arian s’était toujours mieux conservé. Une complète réunion territoriale de la contrée n’eut jamais lieu sous aucun prince brahmanique.\par
En Chine, il en alla autrement, et c’est une nouvelle preuve de la répugnance du génie arian pour l’unité dont, suivant l’expression romaine, l’action se résume dans ces deux mots : \emph{reges et greges.}\par
Les Arians, vainqueurs orgueilleux dont on ne fait pas facilement des sujets, voulurent, toutes les fois qu’ils se trouvèrent maîtres des races inférieures, ne pas laisser aux mains d’un seul d’entre eux les jouissances du commandement. En Chine, donc, comme dans toutes les autres colonisations de la famille, la souveraineté du territoire fut fractionnée, et sous la suzeraineté précaire d’un empereur une féodalité, jalouse de ses droits \footnote{« Sous les trois premières races, l’empire était entièrement composé de principautés, de fiefs « et d’apanages héréditaires. Les hommes qui en étaient investis avaient sur leurs « subordonnés une autorité plus grande que celle des pères sur leurs fils, des chefs de « famille sur leurs propriétés... Chaque chef gouvernait son fief comme sa propriété « héréditaire. » (\emph{Ma-touan-lin}, cité par M. E. Biot, voir le \emph{Tcheou-li}, t. I, \emph{Introduct}., p. XXVII.)}, s’installa et se maintint depuis l’invasion des Kschattryas jusqu’au règne de Tsin-chi-hoang-ti, l’an 246 avant J.-C., autrement dit, aussi long­temps que la race blanche conserva assez de virtualité pour garder ses aptitudes principales \footnote{Les Chinois, qui forment aujourd’hui une grande démocratie impériale, ne jouissaient pas du principe de l’égalité au XXII\textsuperscript{e} siècle avant notre ère, dans l’époque féodale. Le peuple était en servage complet, il n’était pas apte à posséder des biens immeubles. Les Tcheou l’admirent au partage des bas emplois jusqu’au grade de préfet. Plus anciennement, il n’avait pas le droit d’acquérir l’instruction. (\emph{Tcheou-li}, t. I, \emph{Introduct.}, p. LV, et pass.) – Ainsi les Chinois, comme tous les autres peuples, n’ont eu l’égalité politique qu’à la suite de la disparition des grandes races.}. Mais, aussitôt que sa fusion avec les familles malaise et jaune fut assez prononcée pour qu’il ne restât pas de groupes même à demi blancs, et que la masse de la nation chinoise se trouva élevée de tout ce dont ces groupes jusque-là dominateurs avaient été diminués pour être rabaissés et confondus avec elle, le système féodal, la domination hiérarchisée, le grand nombre des petites royautés et des indépendances de personnes, n’eurent plus nulle raison d’exister, et le niveau impérial passa sur toutes les têtes, sans distinction.\par
Ce fut de ce moment que la Chine se constitua dans sa forme actuelle \footnote{Et c’est seulement de ce moment-là que date la philosophie politique nationale. Confucius, et plus tard Meng-tseu, furent également centralisateurs et impérialistes. Le système féodal ne leur est pas moins odieux qu’aux écoles politiques de l’Europe actuelle. (Gaubil, \emph{Chronologie chinoise}, p. 90.)\emph{ –} Les moyens qu’employa Tsin-chi-hoang-ti pour abattre les familles seigneuriales furent des plus énergiques. On commença par brûler les livres : c’étaient les archives du droit souverain des nobles et les annales de leur gloire. On abolit les alphabets particuliers des provinces. On désarma toute la nation. On abrogea les noms des anciennes circonscriptions territoriales, et l’on partagea le pays en trente-six départements administrés par des mandarins que l’on eut soin de changer fréquemment de postes. On força cent vingt mille familles à venir résider dans la capitale, avec défense de s’en éloigner sans permission, etc., etc. (Gaubil, \emph{Chronologie chinoise}, p. 61.)}. Cependant la révolution de Tsin-chi-hoang-ti ne faisait qu’abolir la dernière trace apparente de la race blanche, et l’unité du pays n’ajoutait rien à ses formes gouvernementales, qui restaient patriarcales comme ci-devant. Il n’y avait de plus que cette nouveauté, grande d’ailleurs en elle-même, que la dernière trace de l’indépendance, de la dignité person­nelle, comprises à la manière ariane, avait disparu à jamais devant les envahissements définitifs de l’espèce jaune \footnote{Il se passa alors un fait absolument semblable à celui qui eut lieu, chez nous, en 1789, lorsque l’esprit novateur considéra comme de première nécessité la destruction des anciennes subdivisions territoriales. En Chine, on abolit les circonscriptions qui pouvaient rappeler des idées de nationalités ou de souverainetés. On créa des provinces et des arrondissements purement administratifs. Je remarque toutefois une différence assez sérieuse. Les départements chinois furent très étendus et les nôtres très petits. Matouan-lin prétend que la méthode de son pays n’a pas été sans inconvénient, en rendant plus difficiles la surveillance et la bonne gestion des magistrats impériaux. D’autre part, notre système a soulevé bien des critiques. (Le \emph{Tcheou-li}, t. i, \emph{Introduct}., XXVIII.)}.\par
Autre point encore. Nous avons d’abord vu la race malaise recevant dans le Yun-nan les premières leçons des Arians en s’alliant avec eux ; puis, par les conquêtes et les adjonctions de toute nature, la famille jaune s’augmenta rapidement et finit par ne pas moins neutraliser, dans le plus grand nombre des provinces de l’empire, les métis mélaniens, qu’elle ne transformait, en la divisant, la vertu de l’espèce blanche. Il en résulta pendant quelque temps un défaut d’équilibre manifesté par l’apparition de quelques coutumes tout à fait barbares.\par
Ainsi, dans le nord, des princes défunts furent souvent enterrés avec leurs femmes et leurs soldats, usages certainement empruntés à l’espèce finnoise \footnote{Gaubil, \emph{Chronologie chinoise}, p. 46 et pass.}. On admit aussi que c’était une grâce impériale que d’envoyer un sabre à un mandarin disgracié pour qu’il pût se mettre à mort lui-même \footnote{\emph{Ibid}, p. 51.}. Ces traces de dureté sauvage ne tinrent pas. Elles disparurent devant les institutions restées de la race blanche et ce qui survivait encore de son esprit. À mesure que de nouvelles tribus jaunes se fondaient dans le peuple chinois, elles en prenaient les mœurs et les idées. Puis, comme ces idées se trouvaient désormais partagées par une plus grande masse, elles allaient diminuant de force, elles s’émoussaient, la faculté de grandir et de se développer leur était ravie, et la stagnation s’étendait irrésistiblement.\par
Au XIII\textsuperscript{e} siècle de notre ère, une terrible catastrophe ébranla le monde asiatique. Un prince mongol, Témoutchin, réunit sous ses étendards un nombre immense de tribus de la haute Asie, et, entre autres conquêtes, commença celle de la Chine, terminée par Koubilaï. Les Mongols, se trouvant les maîtres, accoururent de toutes parts, et l’on se demande pourquoi, au, lieu de fonder des institutions inventées par eux, ils s’empressè­rent de reconnaître pour bonnes les inspirations des mandarins ; pourquoi ils se mirent sous la direction de ces vaincus, se conformèrent de leur mieux aux idées du pays, se piquèrent de se civiliser à la façon chinoise, et finirent, au bout de quelques siècles, après avoir ainsi côtoyé plutôt qu’embrassé l’empire, par se faire chasser honteusement.\par
Voici ce que je réponds. Les tribus mongoles, tatares et autres qui formaient les armées de Djinghiz-khan, appartenaient, en presque totalité, à la race jaune. Cependant comme, dans une antiquité assez lointaine, les principales branches de la coalition, c’est-à-dire les mongoles et les tatares, avaient été pénétrées par des éléments blancs, tels que ceux venus des Hakas, il en était résulté un long état de civilisation relative vis-à-vis des rameaux purement jaunes de ces nations, et, comme conséquence de cette supériorité, la faculté, sous des circonstances spéciales, de réunir ces rameaux autour d’un même étendard et de les faire concourir quelque temps vers un seul but. Sans la présence et la conjonction heureuse des principes blancs répandus dans des multitudes jaunes, il est complètement impossible de se rendre compte de la formation des grandes armées envahissantes qui, à différentes époques, sont sorties de l’Asie centrale avec les Huns, les Mongols de Djinghiz-khan, les Tatares de Timour, toutes multitudes coalisées et nullement homogènes.\par
Si, dans ces agglomérations, les tribus dominantes possédaient leur initiative, en vertu d’une réunion fortuite d’éléments blancs jusque-là trop disséminés pour agir, et qui, en quelque sorte, galvanisaient leur entourage, la richesse de ces éléments n’était pourtant pas suffisante pour douer les masses qu’ils entraînaient d’une bien grande aptitude civilisatrice, ni même pour maintenir, dans l’élite de ces masses, la puissance de mouvement qui les avait élevées à la vie de conquêtes. Qu’on se figure donc ces triomphateurs jaunes animés, je dirai presque enivrés par le concours accidentel de quelques immixtions blanches en dissolution dans leur sein, exerçant dès lors une supériorité relative sur leurs congénères plus absolument jaunes. Ces triomphateurs ne sont pas cependant assez rehaussés pour fonder une civilisation propre. Ils ne feront pas comme les peuples germaniques, qui, débutant par adopter la civilisation romaine, l’ont transformée bientôt en une autre culture tout originale. Ils n’ont pas la valeur d’aller jusque-là. Seulement, ils possèdent un instinct assez fin qui leur fait comprendre les mérites de l’ordre social, et, capables ainsi du premier pas, ils se tournent respectueusement vers l’organisation qui régit des peuples jaunes comme eux-mêmes.\par
Cependant, s’il y a parenté, affinité entre les nations demi-barbares de l’Asie centrale et les Chinois, il n’y a pas identité. Chez ces derniers, le mélange blanc et surtout malais se fait sentir avec beaucoup plus de force, et, par conséquent, l’aptitude civilisatrice est bien autrement active. Au sein des autres, il y a un goût, une partialité pour la civilisation chinoise, toutefois moins pour ce qu’elle a conservé d’arian que pour ce qui est corrélatif, en elle, au génie ethnique des Mongols. Ceux-ci sont donc toujours des barbares aux yeux de leurs vaincus, et plus ils font d’efforts afin de retenir les leçons des Chinois, plus ils se font mépriser. Se sentant ainsi isolés au milieu de plusieurs centaines de millions de sujets dédaigneux, ils n’osent pas se séparer, ils se concentrent sur des points de ralliement, ils ne renoncent pas, ils n’osent pas renoncer à l’usage des armes, et comme cependant la manie d’imitation qui les travaille les a poussés en plein dans la mollesse chinoise, un jour vient où, sans racines dans le pays, bien que nés de ses femmes, un coup d’épaule suffit pour les pousser dehors. Voilà l’histoire des Mongols. Ce sera également celle des Mantchous.\par
Afin d’apprécier la vérité de ce que j’avance, touchant le goût des dominateurs jaunes de l’Asie centrale pour la civilisation chinoise, il suffit de considérer ces nomades dans leurs conquêtes, autres que celles du Céleste Empire. En général, on a beaucoup exagéré leur sauvagerie. Ainsi, les Huns, les Hioung-niou des Chinois \footnote{Ritter identifie les Hioung-niou, les Thou-kieou, les Ouïgours et les Hoei-he. De tous ces peuples, il fait des nations turques. Cette opinion, peut-être fondée quant à certaines tribus, me paraît fort critiquable pour l’ensemble. (\emph{Erdkunde, Asien}, t. I, p. 437.)}, étaient loin d’être ces cavaliers stupides que les terreurs de l’Occident ont rêvés. Placés assurément à un degré social peu élevé, ils n’en avaient pas moins des institutions politiques assez habiles, une organisation militaire raisonnée, de grandes villes de tentes, des marchands opulents, et même des monuments religieux. On pourrait en dire autant de plusieurs autres nations finnoises, telles que les Kirghizes, race plus remarquable que toutes les autres, parce qu’elle fut plus mêlée encore d’éléments blancs \footnote{Ritter, \emph{Erdkunde, Asien}, t. I, p. 744, p. 1114 et pass. ; t. II, p. 116. Schaffarik, \emph{Slawiche Alterthümer}, t. I, p. 68. – Les langues turques, mongoles, tongouses et mantchoues contiennent un grand nombre de racines indo-germaniques. (Ritter, t. 1, p. 436.)}. Cependant ces peuples qui savaient apprécier le mérite d’un gouvernement pacifique et des mœurs sédentaires, montrèrent constamment des sentiments très hostiles à toute civilisation quand ils se trouvèrent en contact avec des rameaux appartenant à des variétés humaines différentes de l’espèce jaune. Dans l’Inde, jamais Tatare n’a fait mine d’éprouver la moindre propension pour l’organisation brahmanique. Avec une facilité qui accuse le peu d’aptitude dogmatique de ces esprits utilitaires, les hordes de Tamerlan s’empressèrent, en général, d’adopter l’islamisme. Les vit-on conformer aussi leurs mœurs à celles des populations sémitiques qui leur communiquaient la foi ? En aucune façon. Ces conquérants ne changèrent ni de mœurs, ni de costumes, ni de langue. Ils restèrent isolés, cherchèrent très peu à faire passer dans leur idiome les chefs-d’œuvre d’une littérature brillante plus que solide, et qui devait leur sembler déraisonnable. Ils campèrent en maîtres, et en maîtres indifférents, sur le sol de leurs esclaves. Combien ce dédain est éloigné du respect sympathique que ces mêmes tribus jaunes laissaient éclater lorsqu’elles s’approchaient des frontières de la civilisation chinoise !\par
J’ai donné les raisons ethniques qui me paraissaient empêcher les Montchous, comme elles ont empêché les Mongols, de fonder un empire définitif en Chine. S’il y avait identité parfaite entre les deux races, les Mantchous, qui n’ont rien apporté à la somme des idées du pays, recevraient les notions existantes, ne craindraient pas de se débander et de se confondre avec les différentes classes de cette société, et il n’y aurait plus qu’un seul peuple. Mais, comme ce sont des maîtres qui ne donnent rien et qui ne prennent que dans une certaine mesure ; comme ce sont des chefs qui, en réalité, sont inférieurs, cette situation présente une inconséquence choquante et qui ne se terminera que par l’expulsion de la dynastie.\par
On peut se demander ce qui arriverait, si une invasion blanche venait remplacer le gouvernement actuel et réaliser le hardi projet de lord Clive.\par
Ce grand homme pensait n’avoir besoin que d’une armée de trente mille hommes pour soumettre tout l’empire du Milieu, et on est porté à croire son calcul exact, à voir la lâcheté chronique de ces pauvres gens, qui ne veulent pas qu’on les arrache à la douce fermentation digestive dont ils font leur unique affaire. Supposons donc la conquête tentée et achevée. Dans quelle position se seraient trouvés ces trente mille hommes ? Suivant lord Clive, leur rôle aurait dû se borner à garnisonner les villes. Comme le succès se serait accompli dans un simple but d’exploitation, les troupes auraient occupé les principaux ports, peut-être auraient poussé des expéditions dans l’intérieur du pays pour maintenir la soumission, assurer la libre circulation des marchandises et la rentrée des impôts ; rien de plus.\par
Un pareil état de choses, tout convenable qu’il peut être, ne saurait jamais se prolonger longtemps. Trente mille hommes pour en dominer trois cents millions, c’est trop peu, surtout quand ces trois cents millions sont aussi compacts de sentiments et d’instincts, de besoins et de répugnances. L’audacieux général aurait fini par augmenter ses forces et les aurait portées à un chiffre mieux proportionné à l’immensité de l’océan populaire dont sa volonté aurait voulu contenir les orages. Ici je commence une sorte d’utopie.\par
Si je continue à supposer lord Clive simple et fidèle représentant de la mère patrie, il apparaît toujours, malgré l’augmentation indéfinie de son armée, fort isolé, fort menacé, et, un jour, lui-même ou ses descendants seront expulsés de ces provinces qui reçoivent tous les vainqueurs en intrus. Mais changeons d’hypothèse : laissons-nous aller au soupçon qui fit repousser, dit-on, par les directeurs de la Compagnie des Indes, les somptueuses propositions du gouverneur général. Imaginons que lord Clive, sujet peu loyal de la couronne d’Angleterre, veut régner pour son compte, repousse l’allégeance de la métropole et s’installe, véritable empereur de la Chine, au milieu des populations soumises par son épée. Alors les choses peuvent se passer bien différemment que dans le premier cas.\par
Si ses soldats sont tous de race européenne ou si un grand nombre de cipayes hindous ou musulmans sont mêlés aux Anglais, l’élément immigrant s’en ressentira, de toute nécessité, dans la mesure de sa vigueur. À la première génération, le chef et l’armée étrangère, fort exposés à être mis dehors, auront encore entière leur énergie de race pour se défendre et sauront traverser, sans trop d’encombre, ces moments dangereux. Ils s’occuperont à faire entrer de force leurs notions nouvelles dans le gouvernement et dans l’administration. Européens, ils s’indigneront de la médiocrité prétentieuse de tout le système, de la pédanterie creuse de la science locale, de la lâcheté créée par de mauvaises institutions militaires. Ils feront au rebours des Mantchous, qui se sont pâmés d’admiration devant de si belles choses. Ils y mettront courageusement la hache et renouvelleront, sous de nouvelles formes, la proscription littéraire de Tsin-chi-hoang-ti.\par
À la seconde génération, ils seront beaucoup plus forts au point de vue du nombre. Un rang serré de métis, nés des femmes indigènes, leur aura créé un heureux intermédiaire avec les populations. Ces métis, instruits, d’une part, dans la pensée de leurs pères, et, de l’autre, dominés par le sentiment des compatriotes de leurs mères, adouciront ce que l’importation intellectuelle avait de trop européen, et l’accommo­deront mieux aux notions locales. Bientôt, de génération en génération, l’élément étranger ira se dispersant dans les masses en les modifiant, et l’ancien établissement chinois, cruellement ébranlé, sinon renversé, ne se rétablira plus ; car le sang arian des kschattryas est épuisé depuis longtemps, et si son œuvre était interrompue, elle ne pourrait plus être reprise.\par
D’un autre côté, les graves perturbations infusées dans le sang chinois ne conduiraient certainement pas, je viens de le dire, à une civilisation à l’européenne. Pour transformer trois cents millions d’âmes, toutes nos nations réunies auraient à peine assez de sang à donner, et les métis, d’ailleurs, ne reproduisent jamais ce qu’étaient leurs pères. Il faut donc conclure :\par
1° Qu’en Chine, des conquêtes provenant de la race jaune et ne pouvant ainsi qu’humilier la force des vainqueurs devant l’organisation des vaincus, n’ont jamais rien changé et ne changeront jamais rien à l’état séculaire du pays ;\par
2° Qu’une conquête des blancs, dans de certaines conditions, aurait bien la puissance de modifier et même de renverser pour toujours l’état actuel de la civilisation chinoise, mais seulement par le moyen des métis.\par
Encore cette thèse, qui peut être théoriquement posée, rencontrerait-elle, en prati­que, de très graves difficultés, résultant du chiffre énorme des populations agglomérées, circonstance qui rendrait fort difficile, à la plus nombreuse émigration, d’entamer sérieusement leurs rangs.\par
Ainsi, la nation chinoise semble devoir garder encore ses institutions pendant des temps incalculables. Elle sera facilement vaincue, aisément dominée ; mais transformée, je n’en vois guère le moyen.\par
Elle doit cette immutabilité gouvernementale, cette persistance inouïe dans ses formes d’administration, à ce seul fait que toujours la même race a dominé sur son sol depuis qu’elle a été lancée dans les voies sociales par des Arians, et qu’aucune idée étrangère n’a paru avec une escorte assez forte pour détourner son cours.\par
Comme démonstration de la toute-puissance du principe ethnique dans les destinées des peuples, l’exemple de là Chine est aussi frappant que celui de l’Inde. Ce pays, grâce à la faveur des circonstances, a obtenu, sans trop de peine et sans nulle exagération de ses institutions politiques, au contraire, en adoucissant ce que son absolutisme avait en germe de trop extrême, le résultat que les brahmanes, avec toute leur énergie, tous leurs efforts, n’ont cependant qu’imparfaitement touché. Ces derniers, pour sauvegarder leurs règles, ont dû étayer, par des moyens factices, la conservation de leur race. L’invention des castes a été d’une maintenue toujours laborieuse, souvent illusoire, et a eu cet inconvénient, de rejeter hors de la famille hindoue beaucoup de gens qui ont servi plus tard les invasions étrangères et augmenté le désordre extrasocial. Toutefois, le brahmanisme a atteint à peu près son but, et il faut ajouter que ce but, incomplètement touché, est beaucoup plus élevé que celui au pied duquel rampe la population chinoise. Celle-ci n’a été favorisée de plus de calme et de paix, dans son interminable vie, que parce que, dans les conflits des races diverses qui l’ont assaillie depuis 4000 ans, elle n’a jamais eu affaire qu’à des populations étrangères trop peu nombreuses pour entamer l’épaisseur de ses masses somnolentes. Elle est donc restée plus homogène que la famille hindoue, et dès lors plus tranquille et plus stable, mais aussi plus inerte.\par
En somme, la Chine et l’Inde sont les deux colonnes, les deux grandes preuves vivantes de cette vérité, que les races ne se modifient, par elles-mêmes, que dans les détails ; qu’elles ne sont pas aptes à se transformer, et qu’elles ne s’écartent jamais de la voie particulière ouverte à chacune d’elles, dût le voyage durer autant que le monde.
\section[{III.6. Les origines de la race blanche.}]{III.6. \\
Les origines de la race blanche.}
\noindent De même qu’on a vu, à côté des civilisations assyrienne et égyptienne, des sociétés de mérite secondaire se former à l’aide d’emprunts faits à la race civilisatrice, de même l’Inde et la Chine sont entourées d’une pléiade d’États, dont les uns sont formés sur le norme hindou, dont les autres s’efforcent d’approcher, d’aussi près que possible, l’idéal chinois, tandis que les derniers se balancent entre les deux systèmes.\par
Dans la première catégorie, on doit placer Ceylan et, très anciennement, Java, aujourd’hui musulmane \footnote{Le commencement de l’ère javanaise de Aje-Saka reporte les souvenirs au temps de Sâliwâhana, et répond à l’année 78 après J.-C. Ce fut une époque de civilisation brahmanique, nais non pas de première civilisation de ce genre. Ce ne fut que le renouvellement et comme un rajeunissement d’une domination hindoue beaucoup plus ancienne qui avait vu l’île occupée par des nègres pélagiens fort abrutis. Le Fo-koue-ki raconte que les navigateurs chinois trouvèrent ces aborigènes horriblement laids et sales, avec les cheveux semblables au « gazon naissant. » Ils se nourrissaient de vermine. La loi brahmanique de Java a conservé le souvenir de cet état de choses par la défense formelle qu’elle adresse aux personnes d’un rang élevé de ne manger ni chiens, ni rats, ni couleuvres, ni lézards, ni chenilles. Il semblerait que le brahmanisme n’a jamais pu s’établir à l’état pur dans l’île. Le bouddhisme ne fut pas plus heureux. Au commencement du XVII\textsuperscript{e} siècle de notre ère, les javanais adoptèrent l’islamisme. (W. v. Humboldt, \emph{Ueber die Kawi-Sprache}, t. I, p. 10, 11, 15, 18, 43, 49, 208.)}, plusieurs des îles de l’archipel, comme Bali \footnote{Les coutumes et la religion brahmaniques se sont, jusqu’ici, conservées à Bali pures de tout mélange mahométan ou européen. C’est, au jugement de Raffes, l’image vivante de ce qu’était Java avant sa conversion par les musulmans. (W. v. Humboldt, \emph{Ueber die Kawi-Sprache}, t. I, p. 111.)}, Sumatra, puis d’autres.\par
Dans la seconde, il faut mettre le japon, la Corée, le Laos au dernier rang.\par
 La troisième comprend, avec des modifications infinies dans la mesure où est acceptée chacune des deux civilisations contendantes, le Népaul, le Boutan, les deux Thibets, le royaume de Ladakh, les États de l’Inde transgangétique et une partie de l’archipel de la mer des Indes, de telle sorte que, d’île en île, de groupe en groupe, les populations malaises ont fait circuler jusqu’à la Polynésie des inventions chinoises ou hindoues, qui vont s’effaçant davantage à mesure que le mélange avec le sang de l’une des deux races initiatrices diminue.\par
Nous avons vu Ninive rayonner sur Tyr, et, par Tyr, sur Carthage, inspirer les Himyarites, les enfants d’Israël, et perdre d’autant plus son action sur ces pays, que l’identité des races était plus troublée entre eux et elle. Pareillement nous avons vu l’Égypte envoyer la civilisation à l’Afrique intérieure. Les sociétés secondaires de l’Asie présentent, avec le même spectacle, l’observation rigoureuse des mêmes lois.\par
À Ceylan, à Java, à Bali, des émigrations brahmaniques très anciennes apportèrent le genre de culture particulier à l’Inde et le système des castes. Ces colonisations, de plus en plus restreintes, à mesure que les rivages du Dekkhan s’éloignaient, s’échelonnèrent aussi en mérite. Les plus lointaines, où le sang hindou était en moindre abondance, furent aussi les plus imparfaites \footnote{Guillaume de Humboldt, \emph{Ueber die Kawi-Sprache.}}.\par
Longtemps avant l’arrivée des Arians, des invasions de peuples jaunes étaient venues modifier le sang des aborigènes noirs, et les métis malais, en plusieurs lieux, avaient même commencé déjà à se substituer aux tribus purement mélaniennes. Ce fut une raison déterminante pour que les sociétés dérivées, formées plus tard sous l’influence des métis blancs, ne ressemblassent pas, malgré tous les efforts des initia­teurs, à celle des pays où la race noire pure servait de base. Le naturel malais, plus froid, plus raisonneur, plus apathique, s’accommoda mal de la séparation des castes, et aussitôt qu’apparut le bouddhisme, cette religion grossière réussit vite à s’implanter au milieu des multitudes à demi jaunes. Quels succès ne devait-elle pas obtenir auprès de celles dont les éléments étaient plus libres encore de principes mélaniens. Ceylan et java restèrent longtemps les citadelles de la foi de Bouddha. Comme le principe arian hindou existait dans ces deux îles, le culte de Sakya y resta assez noble. Il construisit de beaux monuments à Java, témoins ceux de Boro-Budor, de Madjapahit, de Brambanan, et, ne s’écartant pas trop, ne dégénérant pas d’une manière complète des données intellectuelles qui font la gloire de l’Inde, il donna naissance à une littérature remarquable, où se trouvaient mêlées les idées brahmaniques et celles du nouveau système religieux. Plus tard, Ceylan et Java reçurent des colonisations arabes. L’islamisme y fit de grands progrès, et le sang malais, ainsi modifié et relevé par les immigrations brahmaniques, bouddhiques et sémitiques, ne rentra jamais dans l’humilité des autres peuples de sa race.\par
Au japon, les apparences sont chinoises, et un grand nombre d’institutions ont été apportées par plusieurs colonies venues originairement, et à différentes époques, du Céleste Empire. Il y existe aussi des éléments ethniques tout différents et qui amènent des divergences sensibles. Ainsi, l’État est encore féodal, l’humeur des nobles hérédi­ taires est restée belliqueuse. Le double gouvernement laïque et ecclésiastique ne se fait pas obéir sans peine. La politique soupçonneuse de la Chine, à l’égard des étrangers, a été adoptée par le Koubo, qui prend grand soin d’isoler ses sujets du contact de l’Europe. Il paraît que l’état des esprits lui donne raison, et que, taillés sur un tout autre modèle que ceux de la Chine, ses administrés, doués d’une façon dangereuse, sont âpres aux nouveautés. Le Japon semble donc entraîné dans le sens de la civilisation chinoise par les résultats des nombreuses immigrations jaunes, et en même temps il y résiste par l’effet de principes ethniques qui n’appartiennent pas au sang finnois. En effet, il existe certainement dans la population japonaise une forte dose d’alliage noir, et peut-être même quelques éléments blancs dans les hautes classes de la société \footnote{Kaempfer, \emph{Histoire du Japon. –} Ce voyageur, d’ailleurs judicieux, sacrifie, comme il était de mode de son temps, à la manie de faire venir d’Assyrie tous les peuples, et il trace ainsi, d’une manière assez curieuse, l’itinéraire de ses japonais : « Mais, pour finir ce chapitre, il « résulte que, peu de temps après le déluge, lorsque la confusion des langues à Babel « força les Babyloniens d’abandonner le désir qu’ils avaient de bâtir une tour d’une hauteur « extraordinaire et les obligea de se disperser par toute la terre ; lorsque les Grecs, les « Goths et les Esclavons passèrent en Europe, d’autres en Asie et en Afrique, d’autres en « Amérique, qu’alors, dis-je, les japonais partirent aussi ; que, \emph{selon toutes les apparences}, « après avoir voyagé plusieurs années et souffert plusieurs incommodités, lis « rencontrèrent cette partie éloignée du monde ; que, trouvant sa situation, sa fertilité fort « à leur gré, ils résolurent de la choisir pour le lieu de leur demeure, etc., etc. (p. 83.) »}. De sorte que, les premiers faits de l’histoire de cette contrée ne remontant pas bien haut, seulement 660 ans avant J.-C., le Japon serait à peu près aujourd’hui dans la situation où la Chine se trouva sous la direction des descendants des kschattryas réfractaires, jusqu’à l’empereur Tsin-chi-hoang-ti. Ce qui confirmerait l’idée que des colonies de race blanche ont civilisé primitivement la population malaise qui fait le fond de ce pays, c’est qu’on y retrouve exactement, aux débuts de l’histoire, les mêmes récits mythiques qu’en Assyrie, en Égypte et même à la Chine, quoique d’une manière plus marquée encore. Les premiers souverains antérieurs à l’époque positive sont des dieux, puis des demi-dieux. Je m’explique le développement d’imagination poétique accusé par la nature de cette tradition, développement qui serait incompréhensible chez un peuple jaune pur, par une certaine prédominance d’éléments mélaniens. Cette opinion n’est pas une hypothèse. On a vu plus haut que Kaempfer constate la présence des noirs dans une île au nord du Japon, peu de siècles avant son voyage, et, au sud du même point, il invoque le témoignage des annales écrites pour établir le même fait \footnote{Kaempfer, \emph{Histoire du Japon}, p. 81 et pass.}. Ainsi s’expliqueraient les particularités physiologiques et morales qui créent l’originalité japonaise \footnote{M. Pickering, jugeant sur ses observations personnelles, tient les japonais pour identiques de race avec les Malais polynésiens (p. 117). – Il n’est pas impossible qu’avant toute invasion hindoue à Java, les japonais n’y aient eu des établissements. Un des noms anciens de l’île est Cha-po. On y connaît deux districts appelés, l’un Ja-pan et l’autre Ji-pang. On sait, d’ailleurs, qu’à une époque très lointaine, les Japonais ont navigué dans tout l’archipel. (W. v. Humboldt. \emph{Ueber die Kawi-Sprache}, t. I, p. 19 ; Crawfurd, \emph{Arcbipelago}, t. III, p. 465.)}.\par
Il n’y a pas, du reste, à s’y tromper : ce coin du monde si peu connu, beaucoup plus mystérieux que son prototype chinois, recèle la solution des questions ethnogra­phiques les plus hautes. Quand il sera permis de l’aborder, de l’étudier en paix, d’y comparer les races, de faire rayonner les observations sur les archipels qui le touchent au nord, on trouvera, sur ce sol, bien des secours décisifs pour l’éclaircissement de ce que les origines américaines présentent de plus ardu.\par
La Corée est, de même que le Japon, une copie de la Chine, moins intéressante toutefois. Comme le sang arian n’est arrivé dans ces parages reculés que par commu­nication très indirecte, il n’y a produit que des efforts d’imitation bien maladroits. Le Laos, je l’ai déjà fait entrevoir, est encore au-dessous, et, encore plus bas, se place la population de l’archipel Lieou-kieou \footnote{M. Jurien de la Gravière a fait justice de l’espèce d’Arcadie que les voyageurs anglais avaient installée dans ces îles. (\emph{Revue des Deux-Mondes}, 1852.)}.\par
Les contrées où les deux principes, hindou et chinois, se partagent les sympathies des populations, sont également étrangères à la plus belle conquête des civilisations qu’elles vénèrent, la stabilité. Rien de plus mouvant, de plus variable, que les idées, les doctrines, les mœurs de ces territoires. Cette mobilité n’a rien à reprocher à la nôtre. Dans les terres transgangétiques, les peuples sont malais, et leurs nationalités se brouillent en nuances imperceptibles autant qu’innombrables, suivant que les éléments jaunes ou noirs dominent. Lorsqu’une invasion de l’est donne la prépondérance aux premiers, l’esprit brahmanique recule, et c’est la situation des derniers siècles, dans bien des provinces, où des ruines imposantes et de pompeuses inscriptions en caractères dévanagaris proclament encore l’antique domination de la race sanscrite, ou, du moins, des bouddhistes chassés par elle.\par
Quelquefois aussi le principe blanc reprend le dessus. Ainsi, ses missions poursui­vent, en ce moment, de véritables succès dans l’Assam \footnote{La civilisation de ce pays affecte des formes brahmaniques. Les rois ont la prétention de descendre des dieux de l’Inde ; mais ils ne font pas dater leurs annales plus haut que l’ère des Vikramaditya (deux siècles av. J.-C.). Il y a eu des immigrations de kschattryas assez récentes, puis le brahmanisme fut étouffé pendant quelque temps pour être rétabli au XVII\textsuperscript{e} siècle. (Ritter, \emph{Erdkunde, Asien}, t. III, p. 298 et pass.)}, les États annamitiques \footnote{Les Siamois sont, à coup sûr, le peuple le plus avili de la terre, parmi les nations relativement civilisées ; et ce qui est assez remarquable, c’est qu’ils savent tous lire et écrire. (Ritter, \emph{Erdkunde, Asien}, t. III, p. 1152.) Ceci semblerait fort contraire à l’avis des économistes anglais et français, qui ont, d’un commun accord, adopté ce genre de connaissances pour le criterium le plus irréfragable de la moralité et de l’intelligence d’un peuple.}, chez les Birmans \footnote{Le brahmanisme s’étend jusqu’au Tonkin ; il y est, à la vérité, très défiguré. (Ritter, \emph{ibid.}, p. 956.)}. Au Népaul, des invasions modernes ont également donné de la puissance au brahmanisme, mais quel brahmanisme ! Aussi imparfait que la race jaune a pu le rendre.\par
Au nord, vers le centre des chaînes de l’Hymalaya, dans ce dédale de montagnes où les deux Thibets ont établi les sanctuaires du bouddhisme lamaïque, commencent les imitations inadmissibles des doctrines de Sakya qui atteignent, en s’altérant, jusqu’aux rivages de la mer Glaciale, presque jusqu’au détroit de Behring.\par
Des invasions arianes, de différentes époques, ont laissé, au fond de ces montagnes, de nombreuses tribus mêlées de près au sang jaune. C’est là qu’il faut chercher la source de la civilisation thibétaine et la cause de l’éclat qu’elle a jeté. L’influence chinoise est venue, de bonne heure, combattre sur ce terrain le génie de la famille hindoue, et, soutenue par la majorité des éléments ethniques, elle a naturellement beaucoup gagné de terrain et en gagne chaque jour davantage.\par
La culture hindoue est en perte visible autour de Hlassa \footnote{Ritter, \emph{Erdkunde, Asien}, t. III, p. 238, 273 et pass., 744. Les idées religieuses du Thibet portent témoignage de l’extrême mélange de la race. On y remarque des notions hindoues, des traces de l’ancien culte idolâtrique du pays, puis des inspirations chinoises, enfin, s’il faut en croire un missionnaire moderne, M. Huc, des traces probables de catholicisme importées au XVI\textsuperscript{e} siècle par des moines européens et acceptées dans la réforme de Tsong-Kaba. (\emph{Souvenirs d’un voyage dans la Tartarie, le Thibet et la Chine}, t. I.) – Au X\textsuperscript{e} siècle, une grande invasion de Kalmoucks et de Dzoungars avait presque anéanti le bouddhisme. (Ritter, \emph{Erdkunde, Asien}, t. III, p. 242.) – Depuis cette époque, et particulièrement sous le règne réparateur de Srong-dzan-gambo, il y a eu quelques immigrations de religieux venus du nord de l’Inde, c’est-à-dire du Bouran et du Népaul. (Ritter, \emph{ibid.}, p. 278.) Mais, désormais, c’est le sens chinois qui domine et progresse chaque jour davantage. La double origine de la civilisation actuelle du Thibet est très bien symbolisée par l’histoire du mariage de Srong-dzan-gambo. Ce monarque épousa deux femmes, l’une que les chroniques appellent Dara-Nipol, la Blanche, et qui était fille du souverain du Népaul ; l’autre, nommée Dara-wen-tching, la Verte, qui venait du palais impérial de Péking. Hlassa fut fondée sous l’influence de ces deux reines, et l’architecture des monuments de cette ville est tout à la fois chinoise et hindoue. (Ritter, \emph{ibid.}, p. 238.)}.\par
Plus haut, vers le nord, elle cesse bientôt d’apparaître, lorsque s’ouvrent les steppes parcourues par les grandes nations nomades de l’Asie centrale. La contrefaçon des idées chinoises règne seule, dans ces froides régions, avec un bouddhisme réformé, à peu près complètement dépouillé d’idées hindoues.\par
Je ne saurais trop le répéter : on s’est représenté comme beaucoup plus barbares qu’ils ne le sont, et surtout qu’ils ne l’étaient, ces puissants amas d’hommes qui ont influé si fort, sous Attila, sous Djen-ghiz-khan, à l’époque de Timour le Boiteux, sur les destinées du monde, même du monde occidental. Mais, en revendiquant plus de justice pour les cavaliers jaunes des grandes invasions, je conviens que leur culture manquait d’originalité et que les constructeurs étrangers de tous ces temples, de tous ces palais, dont les ruines couvrent les steppes mongoles, demeurant isolés au milieu des guerriers qui leur demandaient et leur payaient l’emploi de leurs talents, venaient généralement de la Chine. Cette réserve faite, je puis dire qu’aucun peuple n’a poussé plus loin que les Kirghizes l’amour de l’imprimerie et de ses productions. Des princes, sans grande renommée et d’une puissance médiocre, Ablaï, entre autres, ont semé le désert de monastères bouddhiques, aujourd’hui en décombres. Plusieurs de ces monuments offraient, jusque dans le siècle dernier, où l’académicien Müller les visita \footnote{Ce savant avait une manière, toute particulière à lui, d’explorer les contrées sur lesquelles devait s’escrimer son érudition. Il s’établissait de son mieux dans une ville ou dans un village, et s’entourait de tout le confortable disponible. Puis il envoyait à la découverte un caporal et trente Cosaques, et consignait gravement dans ses notes les observations que ces doctes militaires lui rapportaient. (Ritter, \emph{ibid.}, p. 734.)}, le spectacle de leurs grandes salles dévastées depuis des années, à moitié démantelées et sans toits ni fenêtres, pourtant toutes remplies encore de milliers de volumes. Les livres tombés sur le sol, par suite de la rupture des tablettes moisies qui les suppor­taient jadis, fournissaient des bourres pour les fusils et du papier pour coller les fenêtres, à toutes les tribus nomades et aux Cosaques des environs \footnote{Ritter, t. I, p. 744 et pass.}.\par
D’où avaient pu provenir cette persévérance, cette bonne volonté pour la civilisation, chez les multitudes belliqueuses du XVI\textsuperscript{e} siècle, menant une existence des plus dures, des plus hérissées de privations, sur une terre improductive ? Je l’ai dit plus haut : d’un mélange antique de ces races avec quelques rameaux blancs perdus \footnote{Les langues turques et mongoles. le tongouse et son dérivé, le mandchou, portent des marques de ce fait si considérable. Tous ces idiomes contiennent un grand nombre de racines indo-germaniques. (Ritter, \emph{Erdkunde, Asien}, t. I, p. 436.) – Au point de vue physiologique, on observe encore que les yeux bleus ou verdâtres, les cheveux blonds ou rouges se rencontrent fréquemment chez certaines populations actuelles de la Mongolie. (\emph{Ibid.})}.\par
C’est maintenant l’occasion de toucher un problème qui va prendre, tout à l’heure, les proportions les plus imposantes et faire presque reculer l’audace de l’esprit.\par
J’ai cité, dans le chapitre précédent, les noms de six nations blanches connues des Chinois pour avoir résidé, à une époque relativement récente, sur leurs frontières du nord-ouest et de l’est. Par ces mots, \emph{relativement récente}, j’indique le II\textsuperscript{e} siècle avant notre ère.\par
Ces nations ont toutes eu des destinées ultérieures qui sont connues.\par
Deux d’entre elles, les Yue-tchi et les Ou-soun, habitant sur la rive gauche du Hoang-ho, contre la lisière du désert de Gobi, furent attaquées par les Huns, Hioung-niou, peuple de race turque, venu du nord-est. Obligées de céder au nombre, et séparées dans leurs retraites, elles allèrent se fixer, les Yue-tchi, un peu plus bas vers le sud-ouest, et les Ou-soun, assez loin dans la même direction, sur le versant septentrional du Thian-chan \footnote{Ritter, t. I, p. 431 et pass.}.\par
La redoutable progression des masses ennemies ne les laissa pas longtemps jouir en paix de leur patrie improvisée. Au bout de douze ans les Yue-tchi furent accablés de nouveau. Ils traversèrent le Thian-chan, longèrent le nouveau pays des Ou-soun et vinrent s’abattre au sud, sur le Sihoun, dans la Sogdiane. Là se trouvait une nation blanche comme eux, appelée les Szou par les Chinois, et que les historiens grecs nomment les Gètes ou Indo-Scythes. Ce sont les Khétas du Mahabharata, les Ghats actuels du Pendjab, les Utsavaran-Kétas du Kachemyr occidental. Ces Gètes, attaqués par les Yue-tchi, leur cédèrent la place, et reculèrent sur la monarchie métisse et dégénérée des Bactriens-Macédoniens. L’ayant renversée, ils fondèrent, au milieu de ses débris, un empire qui ne laissa pas que de devenir assez important.\par
Pendant ce temps, les Ou-soun avaient résisté avec bonheur aux assauts des hordes hunniques. Ils s’étaient étendus sur les rives de la rivière Yli, et y avaient établi un État considérable. Comme chez les Arians primitifs, leurs mœurs étaient pastorales et guerrières, leurs chefs portaient ce titre que la transcription chinoise fait prononcer \emph{kouen-mi} ou\emph{ houen-mo}, et dans lequel on retrouve aisément la racine du mot germanique \emph{kunig} \footnote{Ritter, \emph{Erdkunde, Asien}, t. I, p. 433-434.}. Les demeures des Ou-soun étaient sédentaires.\par
La prospérité de cette nation courageuse s’éleva rapidement. L’an 107 avant J.-C., c’est-à-dire 170 ans après la migration, l’établissement de ce peuple offrait assez de solidité pour que la politique chinoise crût devoir s’en faire un appui contre les Huns. Une alliance étroite fut formée entre l’empereur et le kouen-mi des Ou-soun, et une princesse vint, du royaume du Milieu, partager la puissance du souverain blanc et porter le titre de \emph{kouen-ti} (queen) \footnote{Ritter, \emph{Erdkunde, Asien}, t. I, p. 433-434.}.\par
Mais l’esprit d’indépendance personnelle et de fractionnement, propre à la race ariane, décida trop tôt du sort d’une monarchie qui, exposée à d’incessantes attaques, aurait eu besoin d’être fortement unie pour y faire tête. Sous le petit-fils de la reine chinoise, la nation se partagea en deux branches, régies par des chefs différents, et, à la suite de cette scission malencontreuse, la partie du nord se vit bientôt accablée par des barbares jaunes, appelés les Sian-pi, qui, accourant en grand nombre, chassèrent les habitants. D’abord les fugitifs se retirèrent vers l’ouest et le nord. Après être restés dans leur asile pendant quatre cents ans, ils furent de nouveau expulsés et dispersés. Une fraction chercha un refuge au delà du Jaxartes, sur les terres de la Transoxiane ; le reste gagna vers l’Irtisch et se retira dans la steppe des Kirghizes, où, en 619 de notre ère, étant tombé sous la sujétion des Turcs, il s’allia à ses vainqueurs et disparut \footnote{Ritter, \emph{loc. cit.}}.\par
Pour l’autre branche des Ou-soun, elle fut absorbée par les envahisseurs, et se mêla à eux comme l’eau d’un lac à celle du grand fleuve qui la traverse.\par
À côté des Ou-soun et des Yue-tchi, quand ils habitaient sur le Hoang-ho, vivaient d’autres peuples blancs. Les Ting-ling occupaient le pays à l’occident du lac Baïkal ; les Khou-te tenaient les plaines à l’ouest des Ou-soun ; les Chou-le s’étendaient vers la contrée plus méridionale où est aujourd’hui Kaschgar ; les Kian-kouan ou Ha-kas montaient vers le Jénisseï où, plus tard, ils se sont fondus avec les Kirghizes. Enfin, les Yan-thsaï, Alains-Sarmates, touchaient à l’extrémité septentrionale de la mer Caspienne \footnote{Ritter, t. I, p. 1110 et 1114. – Les Kirghizes ont absorbé, à la fois, les Ting-ling et les Ha-kas.}.\par
On n’a pas perdu de vue qu’il s’agit ici de l’an 177 ou 200 avant J.-C. On a remarqué aussi que tous ceux des peuples blancs que je viens de nommer, quand ils ont pu se maintenir, ont fondé des sociétés : tels les Szou ou Khétas, les Ou-soun et les Yan-thsaï ou Alains. Je passe à une nouvelle considération qui se déduit de ce qui précède.\par
Puisque la race noire occupait, dans les temps primordiaux, et avant la descente des nations blanches, la partie australe du monde, ayant pour frontières, en Asie, tout au moins la partie inférieure de la mer Caspienne d’une part, de l’autre les montagnes du Kouen-loun, vers le 36° degré de latitude nord, et les îles du Japon sous le 4° à peu près ; que la race jaune, à la même époque, antérieurement à toute apparition des peuples blancs dans le sud, se trouvait avancée au moins jusqu’au Kouen-loun, et, dans la Chine méridionale, jusqu’au rivage de la mer Glaciale, tandis que, dans les pays de l’Europe, elle allait jusqu’en Italie et en Espagne, ce qui suppose l’occupation préalable du nord \footnote{Les invasions dans l’ouest étaient extrêmement facilitées à la race jaune par la configuration du terrain. M. le baron A. de Humboldt remarque que, depuis les rives de l’Obi, par le 78° de longitude, jusqu’aux bruyères du Lunebourg, de la Westphalie et du Brabant, le pays offre exactement le même aspect, triste et monotone. (\emph{Asie centrale}, t. I, p. 55.)} ; puisque, enfin, la race blanche, en apparaissant sur les crêtes de l’Imaüs et se laissant voir sur les limites du Touran, envahissait des terres qui lui étaient toutes nouvelles ; pour toutes ces raisons, il est bien évident, bien incontestable, bien positif que les premiers domaines de cette race blanche doivent être cherchés sur les plateaux du centre de l’Asie, vérité déjà admise, mais de plus, qu’on peut les délimiter d’une manière exacte. Au sud, ces territoires ont leur frontière depuis le lac Aral jusqu’au cours supérieur du Hoang-ho, jusqu’au Khou-khou-noor. À l’ouest, la limite court de la mer Caspienne aux monts Ourals. À l’est, elle remonte brusquement en dehors du Kouen-loun vers l’Altaï. La délimitation au nord semble plus difficile ; cependant nous allons, tout à l’heure, la chercher et la trouver.\par
La race blanche était très nombreuse, le fait n’est pas contestable \footnote{Les territoires sibériens qu’elle occupait étaient assez vastes pour la contenir, car ils ne mesurent pas moins de 300,000 lieues carrées. (Humboldt, \emph{Asie centrale}, t. I, p. 176.) Les ressources que présentaient ces pays pour la nourriture de masses considérables étaient également très suffisantes. Les plaines de la Mongolie actuelle, appelées par les Chinois la Terre des Herbes, offraient des pâturages immenses aux nombreux troupeaux d’une famille humaine essentiellement pastorale. Le seigle et l’orge réussissent très avant dans le nord. À Kaschgar, à Khoten, à Aksou, à Koutché, dans le parallèle de la Sardaigne, on cultive le coton et les vers à soie. Plus au nord, à Yarkand, à Hami, à Kharachar. les grenades et les raisins arrivent à maturité. (\emph{Asie centrale}, t. III, p. 20.) – « Au delà du Jenisséï, à l’est du « méridien de Sayansk, et surtout au delà du lac Baïkal, la Sibérie même prend un « caractère montueux et agréablement pittoresque. » (\emph{Ibid.}, p. 23.)}. J’en ai donné ailleurs les preuves principales. Elle était, de plus, sédentaire et, de plus, malgré les émissions considérables de peuples qu’elle avait faites au dehors de ses frontières, plusieurs de ses nations restèrent encore dans le nord-ouest de la Chine, longtemps après que la race jaune eut réussi à rompre la résistance du tronc principal, à le briser, à le disperser et à s’avancer à sa place dans l’Asie australe. Or, la position qu’occupent, au II\textsuperscript{e} siècle avant notre ère, les Yue-tchi et les Ou-soun, sur la rive gauche du Hoang-ho, en tirant vers le Gobi supérieur, c’est-à-dire sur la route directe des invasions jaunes, vers le centre de la Chine, a de quoi surprendre, et l’on pourrait la considérer comme forcée, comme étant le résultat violent de certains chocs qui auraient repoussé les deux rameaux blancs d’un territoire plus ancien et plus naturellement placé, si la position relative des six autres nations que j’ai aussi nommées, n’indiquait pas que tous ces membres de la grande famille dispersée se trouvaient réellement chez eux et formaient le jalonnement des anciennes possessions de leur race, au temps de la réunion. Ainsi, il y avait eu extension primitive des peuples blancs au delà du lac Khou-khou-noor vers l’est, tandis qu’au nord ces mêmes peuples touchaient encore, à une époque assez basse, au lac Baïkal et au cours supérieur du Jénisseï. Maintenant que toutes les limites sont précisées, il y a lieu de chercher si le sol qu’elles embrassent ne renferme plus aucun débris matériel, aucune trace, qui puissent se rapporter à nos premiers parents. Je sais bien que je demande ici des antiquités presque hyperboliques. Cependant la tâche n’est pas chimérique en présence des découvertes curieuses et entourées de tant de mystères qui eurent l’honneur, au dernier siècle, d’attirer l’attention de l’empereur Pierre le Grand, et de donner, en sa personne, une preuve de plus de cette espèce de divination qui appartient au génie.\par
Les Cosaques, conquérants de la Sibérie à la fin du XVI\textsuperscript{e} siècle, avaient trouvé des traînées de tumulus soit de terre, soit de pierres, qui, au milieu de steppes complè­tement désertes, accompagnaient le cours des rivières. Dans l’Oural moyen, on en rencontrait aussi. Le plus grand nombre était de grandeur médiocre. Quelques-uns, magnifiquement construits en blocs de serpentin et de jaspe, affectaient la forme pyramidale et mesuraient jusqu’à cinq cents pieds de tour à la base \footnote{Ritter, \emph{Erdkunde, Asien}, t. II, p. 332 et pass., p. 336.}.\par
Dans le voisinage de ces sépultures, on remarquait, en outre, des restes étendus de circonvallations, des remparts massifs, et, ce qui est encore aujourd’hui d’une grande utilité pour les Russes, d’innombrables travaux de mines sur tous les points riches en or, en argent et en cuivre \footnote{La limite des tombeaux et des mines tchoudes s’arrête vers le nord, au 58° ; et, du côté du sud, elle descend jusqu’au 45°. L’extension de l’est à l’ouest va depuis l’Amour moyen jusque sur le Volga, jusqu’au pied oriental de l’Oural. (Ritter, \emph{ibid.}, p. 337.)}.\par
Les Cosaques et les administrateurs impériaux du XVII\textsuperscript{e} siècle auraient fait peu d’attention à ces restes d’antiquités inconnues, sauf, peut-être, les ouvertures de mines, si une circonstance intéressante ne les avait captivés. Les Kirghizes étaient dans l’habitude d’ouvrir ces tombeaux, beaucoup d’entre eux en faisaient même un métier, et ce n’était pas sans raison. Ils en extrayaient, en grande quantité, des ornements ou des instruments d’or, d’argent et de cuivre. Il ne paraît pas que le fer s’y soit jamais montré. Dans les monuments construits pour le commun peuple, la trouvaille était de médiocre valeur ; aussi les chasseurs kirghizes ont-ils laissé subsister, jusqu’à nos jours, un grand nombre de ces constructions. Mais les plus belles, celles qui annonçaient, chez le mort, du rang ou de la richesse, ont été bouleversées sans pitié, non sans profit, car dans leur sein l’or a été recueilli avec profusion.\par
Les Cosaques prirent bientôt leur part de ces opérations destructives ; mais Pierre le Grand, l’ayant appris, défendit de fondre ni de détruire les objets déterrés dans les excavations, et ordonna de les lui envoyer à Saint-Pétersbourg. C’est ainsi que fut formé, dans cette capitale, le curieux musée des antiquités tchoudes, précieux par la matière et plus encore par la valeur historique. On appela ces monuments \emph{tchoudes} ou \emph{daours}, honneur peu mérité qu’on faisait aux Finnois, faute de connaître les véritables auteurs.\par
Les découvertes ne devaient pas se borner là. Bientôt on s’aperçut qu’on n’avait pas vu tout. À mesure qu’on avançait vers l’est, on trouvait des tombeaux par milliers, des fortifications, des mines. Dans l’Altaï, on remarqua même des restes de cités, et, de proche en proche, on put se convaincre que ces mystérieuses traces de la présence de l’homme civilisé embrassaient une zone immense, puisqu’elles s’étendaient depuis l’Oural moyen jusqu’au cours supérieur de l’Amour, prenant ainsi toute la largeur de l’Asie et couvrant de marques irrécusables d’une haute civilisation ces terribles plaines sibériennes aujourd’hui désertes, stériles et désolées. Vers le sud, on ne connaît pas la limite des monuments. À Semipalatinsk, sur l’Irtisch, dans le gouvernement de Tomsk, les campagnes sont hérissées de puissantes accumulations de terre et de pierres. Sur le Tarbagataï et la Chaïnda, des débris de cités nombreuses laissent contempler encore des ruines colossales \footnote{Ritter, \emph{ibid}., p. 325 et pass. Il semblerait que les monuments puissent se distinguer en deux classes, et celle à laquelle appartient la plus haute antiquité indique aussi la civilisation la plus complète. (\emph{Ibid.}, t. II, p. 333.)}.\par
Voilà les faits. À leur suite se présente cette question : à quels peuples nombreux et civilisés ont appartenu ces fortifications, ces villes, ces tombeaux, ces instruments d’or et d’argent ?\par
Pour obtenir une réponse, il faut ici procéder d’abord par exclusion. On ne saurait penser à attribuer toutes ces merveilles aux grands empires jaunes de la haute Asie. Eux aussi ont laissé des marques de leur existence. On les connaît, ces marques, et ce ne sont pas celles-là. Elles ont une tout autre apparence, une autre disposition. Il n’y a pas moyen de les confondre avec celles dont il est question ici. De même pour les restes de la grandeur passagère de certaines peuplades, comme les Kirghizes. Les couvents bouddhiques d’Ablaï-kitka ont leur caractère, qui ne saurait être confondu avec celui des constructions tchoudes \footnote{M. Ritter fait ici une observation pleine de sens et de profondeur. Comment, dit-il, se pourrait-il faire que des populations jaunes, que des Kalmouks, ces hommes absolument dénués d’imagination, eussent donné cours au mythe des Gryphons, et, devenus les Arimaspes, se fussent entourés de tant de peuples si singulièrement fabuleux ? En effet, le génie finnois n’atteint pas à de tels résultats. (Ritter, \emph{ibid.}, p. 336.)}.\par
Les temps modernes ainsi mis hors de cause, cherchons dans les temps anciens à quelle nation nous pouvons nous adresser. M. Ritter insinue que les habitants de ce mystérieux et vaste empire septentrional pourraient bien avoir été les Arimaspes d’Hérodote.\par
Je me permettrai de résister à l’opinion du grand érudit allemand, qui ne fait d’ailleurs qu’offrir cette solution sans paraître lui-même convaincu de sa valeur. Pour s’y tenir, il faudrait, ce me semble, forcer le texte du père de l’Histoire. Que dit-il ? Il raconte qu’au-dessus des Hindous demeurent les Arimaspes, et il décrit les Arimaspes ; mais au-dessus des Arimaspes résident les Gryphons, plus loin encore les Hyperboréens. Tous ces peuples sont les mêmes nations à demi fantastiques dont les poètes de l’Inde peuplent l’Uttara-Kourou \footnote{Lassen, \emph{Zeitschrift für d. K. d. Morgenl.}, t. II, p. 62 et 65. Les Grecs avaient puisé leurs connaissances à demi romanesques des peuples de l’Asie centrale à la source bactrienne à peu près identique avec celle du Mahabharata. L’Uttara-Kourou, le pays primitif des Kauravas, les Attacori de Pline, était aussi l’Hataka, la terre de l’or. Près de là demeuraient les Risikas qui, ayant des chevaux merveilleux, ressemblent fort aux Arimaspes. (Hérodote, IV, 13 et 17.)}. Je ne vois aucun motif d’attribuer à ces fantômes, qui cachent d’ailleurs des peuples réels et, sans nul doute, de race blanche, ce que l’on doit reporter à de vrais hommes, On serait plus près de la vérité en ne voyant dans les Issédons, les Arimaspes, les Gryphons, les Hyperboréens, que des fragments de l’antique société blanche, des peuples apparentés aux Arians zoroastriens, aux Sarmates \footnote{Il est incontestable que les Arimaspes portent, dans la première syllabe de leur nom une sorte de témoignage de leur origine blanche. Ne pourrait-on retrouver encore actuellement dans le nord de la Sibérie la même racine \emph{are} avec quelques-unes de ses conséquences ethnologiques ? Strahlenberg raconte que les Wotiaks se nomment, en leur langue, \emph{Arr}, et appellent leur pays \emph{Arima}. Il ne s’ensuivrait pas, sans doute, que les Wotiaks fussent un peuple de race ariane ; mais on pourrait conclure que ce sont des métis blancs et jaunes qui ont conservé le nom d’une partie de leurs ancêtres. Strahlenberg \emph{das Nord-und-œstliche Theil von Europa und Asien}, p. 76.) \emph{Nota. – Are} est le mot mongol pour dire \emph{homme}, par opposition à \emph{came, femme}. (\emph{Ibid.}, 137.) – De même, \emph{arian} signifie pur, etc.}. Ce qui appuie cette opinion, c’est que jusqu’ici les géographes avaient placé ces tribus en cercle autour de la Sogdiane et nullement dans le nord sibérien. C’est le vrai sens d’Hérodote, et rien ne porte à y être infidèle. De plus, les récits d’Aristée de Proconnèse, tels qu’Hérodote les rapporte, ont trait à une époque où les nations blanches de l’Asie étaient trop divisées, trop poursuivies pour pouvoir fonder de grandes choses, et laisser des traces d’une civilisation étendue sur de si immenses contrées.\par
Si ces peuples avaient été aussi puissants que M. Ritter le suppose, les Chinois n’auraient pu éviter de très nombreux rapports avec eux, et les Grecs, qui savaient de si belles choses de ces Chinois, que je ne fais pas difficulté de reconnaître dans les Argippéens chauves, sages et essentiellement pacifiques \footnote{Hérodote, IV, 23.}, auraient donné également des détails plus minutieux et plus exacts sur des faits aussi frappants que ceux dont les monuments tchoudes proclament l’existence. Il ne me paraît donc nullement possible qu’au VI\textsuperscript{e} siècle avant J.-C. tout le centre de l’Asie ait été la possession d’un grand peuple cultivé, s’étendant du Iénisséi à l’Amour, dont ni les Chinois, ni les Grecs, ni les Perses, ni les Hindous n’auraient jamais eu ni vent ni nouvelles, tous persuadés, au contraire, à l’exception des premiers, qui ont le privilège de ne rêver à rien, qu’il fallait peupler ces régions inconnues de créatures à moitié mythologiques.\par
Si l’on ne peut pas accorder de telles œuvres au temps d’Hérodote, comme il n’est pas possible non plus de les reporter, après lui, à l’époque d’Alexandre, par exemple, où ce prince, s’étant avancé jusqu’à l’extrémité de la Sogdiane, n’aurait rien appris des merveilles du nord, ce qui est inadmissible, il faut, de toute nécessité, se plonger intrépidement dans ce que l’antiquité a de plus reculé, de plus noir, de plus ténébreux, et ne pas hésiter à voir dans les contrées sibériennes le séjour primitif de l’espèce blanche, alors que les nations diverses de cette race, réunies et civilisées, occupaient des demeures voisines les unes des autres, alors qu’elles n’avaient pas encore de motifs de quitter leur patrie, et de s’éparpiller pour aller en chercher une autre au loin.\par
Tout ce qu’on a exhumé des tombeaux et des ruines tchoudes ou daouriennes confirme ce sentiment. Les squelettes sont toujours ou presque toujours accompagnés de têtes de chevaux. On observe à côté d’eux une selle, une bride, des étriers, des monnaies marquées d’une rose, des miroirs de cuivre, rencontre si commune parmi les reliques chinoises et étrusques, si fréquente encore sous les yourtes tongouses où ces instruments servent aux opérations magiques. Ils se trouvent abondamment dans les plus pauvres tombeaux daouriens \footnote{Chez les Bouriates, il est peu de temps où l’on ne rencontre de ces sortes de miroirs suspendus aux piliers. Le lama s’en sert en y faisant refléter l’image du Bouddha ; puis il verse dessus de l’eau qui, coulant de là dans un vase, est censée emporter l’image divine et devient consacrée. (Ritter, \emph{Erdkunde, Asien}, t. II, p. 119-120.)}. Chose plus remarquable : au siècle dernier, Pallas aperçut sur un monument en forme d’obélisque et sur des pierres tumulaires des inscriptions étendues. Un vase retiré d’un sépulcre en portait une également, et W. G. Grimm n’hésite pas à signaler entre les caractères de ces inscriptions et les runes ger­maniques, non pas une identité complète, mais une ressemblance imméconnaissable \footnote{W. C. Grimm, \emph{Ueber die deutschen Runen}, in-12, p. 128 ; Strahlenberg, \emph{das Nordund-œstliche Theil von Europa und Asien}, in-4° ; Stockholm, 1730. Le capitaine suédois, premier auteur qui ait parlé des monuments tchoudes, fait une remarque on ne peut plus intéressante : il dit qu’en Islande, dans les temps anciens, on écrivait sur des os de poissons avec une couleur rouge indélébile ; que des caractères tracés avec la même matière se rencontrent chez les Permiens et sur les bords du Iéniséi, puis à la source de l’Irbyht, et ailleurs encore (p. 363). On entrevoit sans peine les conclusions à tirer d’un fait aussi remarquable, et il est temps de se rappeler ici que le mot qui, chez les nations gothiques, signifiait \emph{écrire}, était \emph{mêljan ou gameljan} dont le sens véritable est \emph{peindre} ;\emph{ mèl}, peinture, et de là, écriture ; \emph{ufarmêli}, inscription. (W.C. Grimm, \emph{Ueber die deutschen Runen}, p. 47.)}. J’arrive au trait frappant, concluant, selon moi : au nombre des ornements les plus fréquents, comme les cornes de bélier, de cerf, d’élan, d’argali, en métal, or ou cuivre, le sujet le plus ordinaire, le plus répété, c’est le sphinx. Il se trouve au manche des miroirs et même taillé en relief sur des pierres \footnote{« Dans le vestibule du musée (à Barnaul) était un sphinx taillé en pierre, reposant sur un « bloc carré, et long de quatre pieds sur un pied et demi de large. Ce monument fut, pour « moi, d’un grand intérêt, ayant été découvert dans un tombeau tchoude. Le travail en « était, à la vérité, grossier ; mais trouver en ce lieu une production d’une si haute antiquité « me frappa beaucoup. Je vis aussi plusieurs pierres sépulcrales, provenant également de « tombeaux tchoudes, ornées de bas-reliefs représentant des figures d’hommes, peu « saillantes et d’une exécution également assez rude. » (C. F. von Ledebour, \emph{Reise durch das Altaï-Gebirge and die soongorische Kirgisen-Steppe}, 1\textsuperscript{re} partie ; Berlin, 1829, p. 371-372.)}.\par
Il sied bien aux énigmatiques habitants de la Sibérie antique de s’être rendu justice devant la postérité en lui léguant, comme leur plus parfait emblème, le symbole de l’impénétrable. Mais, trop prodigué, le sphinx finit par se révéler lui-même. Comme nous le trouvons chez les Perses sculpté aux murailles de Persépolis, comme nous le rencontrons en Égypte s’étendant silencieux en face du désert, et que sur les croupes du Cithéron des Grecs il erre encore tandis qu’Hérodote, ce soigneux observateur, le voit chez les Arimaspes, il devient possible de poser la main sur l’épaule de cette créature taciturne, et de lui dire, sinon qui elle est, du moins le nom de son maître. Elle appartient évidemment en commun à la race blanche. Elle fait partie de son patrimoine, et bien que le secret de ce qu’elle signifie n’ait pas encore été pénétré, on est autorisé à déclarer que, là où on l’aperçoit, là furent aussi des peuples arians.\par
Ces steppes du nord de l’Asie, aujourd’hui si tristes, si désertes, si dépeuplées, mais non pas stériles, comme on le croit généralement \footnote{Voir plus haut, p. 430 et suiv.}, sont donc le pays dont parlent les Iraniens, l’Airyanemvaëgo, berceau de leurs aïeux. Ils racontaient eux-mêmes qu’il avait été frappé d’hiver par Ahriman, et qu’il n’avait pas deux mois d’été. C’est l’Uttara-Kourou de la tradition brahmanique, région située, suivant elle, à l’extrême nord, où régnait la liberté la plus absolue pour les hommes et pour les femmes ; liberté réglée cependant par la sagesse, car là habitaient les Rischis, les saints de l’ancien temps \footnote{Lassen, \emph{Zeitschrift der deutsch. morgenl. Gesellsch.}, t. II, p. 39.}. C’est l’Hermionia des Hellènes, patrie des Hyperboréens, des gens de l’extrême nord, macrobiens, dont la vie était longue, la vertu profonde, la science infinie, l’existence heureuse. Enfin, c’était cette contrée de l’est dont les Suèves germaniques ne parlaient qu’avec un respect sans bornes, parce que, disaient-ils, elle était possédée par leurs glorieux ancêtres, les plus illustres des hommes, les Semnons \footnote{Mannert, \emph{Germania}, p. 2.}.\par
Ainsi, voilà quatre peuples arians qui, depuis la séparation de l’espèce, n’ont jamais communiqué ensemble, et qui s’accordent à placer dans le fond du nord, à l’est de l’Europe, le premier séjour de leurs familles. Si un pareil témoignage était repoussé, je ne sais plus sur quelle base solide pourrait compter l’histoire.\par
La terre de Sibérie garde donc dans ses solitudes les vénérables monuments d’une époque bien autrement ancienne que celle de Sémiramis, bien autrement majestueuse que celle de Nemrod. Ce n’est ni l’argile, ni la pierre taillée, ni le métal fondu que j’en admire. Je réfléchis que, dans une antiquité aussi haute, la civilisation que je constate touche de près aux âges géologiques, à cette époque encore troublée par les révoltes d’une nature mal soumise qui a vu la mise à sec de la grande mer intérieure dont le désert de Gobi faisait le fond. C’est vers le soixantième siècle avant J.-C. que les Chamites et les Hindous apparaissent au seuil du monde méridional. Il ne reste donc plus pour atteindre la limite que la religion et les sciences naturelles semblent imposer à l’âge du monde qu’un ou deux milliers d’années environ, et c’est pendant cette période que se développa avec une vigueur dont les preuves sont nombreuses et patentes un perfectionnement social qui ne laisse pas le moindre espace de durée à une barbarie primitive. Ce que j’ai répété plusieurs fois déjà sur la sociabilité et la dignité innées de l’espèce blanche, je crois que je viens de l’établir définitivement ici, et, en écartant, en repoussant dans un néant inexorable l’homme sauvage, le premier homme des philosophes matérialistes, celui dont le spectre constamment évoqué sert à combattre ce que les institutions sociales ont de plus respectable et de plus nécessaire, en chassant définitivement dans les kraals des Hottentots et jusqu’au fond des cabanes tongouses, et par delà encore, dans les cavernes des Pélagiens, cette misérable créature humaine qui n’est pas des nôtres, et qui se dit fille des singes, oublieuse d’une origine meilleure bien que défigurée, je ne fais autre chose que d’accepter ce que les découvertes de la science apportent de confirmation aux antiques paroles de la Genèse.\par
Le livre saint n’admet pas de sauvages à l’aurore du monde. Son premier homme agit et parle, non pas en vertu de caprices aveugles, non pas au gré de passions purement brutales, mais conformément à la règle préétablie, appelée par les théologiens loi \emph{naturelle}, et qui n’a d’autre source possible que la révélation, asseyant ainsi la morale sur un sol plus solide et plus immuable que ce droit ridicule de chasse et de pêche proposé par les docteurs du socialisme. J’ouvre la Genèse, et, au second chapitre, si les deux ancêtres sont nus, c’est qu’ils sont dans l’état d’innocence : « c’est », dit le livre saint, « qu’ils ne le prennent point à honte. » Aussitôt que l’état paradisiaque cesse, je ne vois pas les auteurs de l’espèce blanche se mettre à vaguer dans les déserts. Ils reconnaissent immédiatement la nécessité du travail, et ils la pratiquent. Immédiate­ment ils sont civilisés, puisque la vie agricole et les habitudes pastorales leur sont révélées. La pensée biblique est si ferme sur ce point, que le fondateur de la première ville est Caïn, le fils du premier homme, et cette ville porte le nom d’Hénoch, le petit-fils d’Adam \footnote{Gen., IV, 17 : « Caïn... ædificavit civitatem, vocavitque nomen ejus ex nomine filii sui, « Henoch. » La suite du récit n’est pas moins curieuse, et ne concorde pas moins avec ce que j’ai dit des mœurs primitives de la race blanche et de ses habitudes : 20. « Genuit Ada « Jabel, qui fuit pater habitantium in tentoriis, arque pastorum. » 21. « Et nomen fratris ejus « Jubal ; ipse fuit pater canentium cithara et organo. » 22. « Sella quoque genuit Tubalcaïn, « qui fuit malleator et faber in cuncta opera æris et ferri » Ainsi, cinq générations après Caïn, fondateur de la première ville, les peuples menaient la vie pastorale, connaissaient l’art du chant, c’est-à-dire conservaient des annales et savaient travailler les métaux. Je n’ai pas tiré des résultats différents de la série des témoignages physiologiques, philologiques et historiques que j’ai interrogés jusqu’ici dans ces pages.}.\par
Inutile de débattre ici la question de savoir si le récit sacré doit être entendu dans un sens littéral ou de toute autre façon : ce n’est pas de mon sujet. Je me borne à constater que, dans la tradition religieuse, qui est en même temps le récit le plus complet des âges primitifs de l’humanité, la civilisation naît, pour ainsi dire, avec la race, et cette donnée est pleinement confirmée par tous les faits qu’on peut grouper à l’entour.\par
Encore un mot sur l’espèce jaune. On la voit, dès les âges primordiaux, retenue par la digue épaisse et puissante que lui oppose la civilisation blanche, contrainte, avant d’avoir pu surmonter l’obstacle, de se partager en deux branches et d’inonder l’Europe et l’Asie orientale, en se coulant le long de la mer Glaciale, de la mer du Japon et des plages de la Chine. Mais il n’est pas possible de supposer, à voir quelles masses effrayantes se pressaient, au second siècle avant J.-C., dans le nord de la Mongolie actuelle, que ces multitudes aient pris naissance et continuassent à se former unique­ment dans les misérables territoires des Tongouses, des Ostiaks, des Yakouts, et dans la presqu’île du Kamtschatka.\par
Tout indique, en conséquence, que le siège originaire de cette race se trouve sur le continent américain. J’en déduis les faits suivants :\par
Les peuples blancs, isolés d’abord, à la suite des catastrophes cosmiques, de leurs congénères des deux autres espèces, et ne connaissant ni les hordes jaunes ni les tribus noires, n’eurent pas lieu de supposer qu’il existât d’autres hommes qu’eux. Cette manière de juger, loin d’être ébranlée par le premier aspect des Finnois et des nègres, s’en confirma au contraire. Les blancs ne purent s’imaginer voir des êtres égaux à eux dans ces créatures qui, par une hostilité méchante, une laideur hideuse, une inintelligence brutale et le titre de fils de singes qu’elles revendiquaient, semblaient se repousser d’elles-mêmes au rang des animaux. Plus tard, quand vinrent les conflits, la race d’élite flétrit les deux groupes inférieurs, surtout les peuplades noires, de ce nom de \emph{barbares}, qui resta comme le témoignage éternel d’un juste mépris.\par
Mais à côté de cette vérité se trouve encore celle-ci, que la race jaune, assaillante et victorieuse, tombant précisément au milieu des nations blanches, devint semblable à un fleuve qui traverse et détruit des gisements aurifères : il charge son limon de paillettes, et s’enrichit lui-même. Voilà pourquoi la race jaune apparaît si souvent, dans l’histoire, à demi civilisée et relativement civilisable, importante au moins comme instrument de destruction, tandis que l’espèce noire, plus isolée de tout contact avec la famille illustre, reste plongée dans une inertie profonde.
\chapterclose


\chapteropen
\chapter[{IV. Civilisations sémitisées du sud-ouest.}]{IV. \\
Civilisations sémitisées du sud-ouest.}\renewcommand{\leftmark}{IV. \\
Civilisations sémitisées du sud-ouest.}


\chaptercont
\section[{IV.1. L’histoire n’existe que chez les nations blanches. Pourquoi presque toutes les civilisations se sont développées dans l’occident du globe.}]{IV.1. \\
L’histoire n’existe que chez les nations blanches. Pourquoi presque toutes les civilisations se sont développées dans l’occident du globe.}
\noindent Nous abandonnons maintenant, jusqu’au moment d’aller, avec les conquérants espagnols, toucher le sol du continent américain, ces peuples isolés qui, moins exposés que les autres aux mélanges ethniques, ont pu conserver, pendant un long enchaînement de siècles, une organisation contre laquelle rien n’agissait. L’Inde et la Chine nous ont, dans leur séparation du reste du monde, présenté ce rare spectacle. Et de même que nous ne verrons plus désormais que des nations enchaînant leurs intérêts, leurs idées, leurs doctrines et leurs destinées à la marche de nations différemment formées, de même nous ne verrons plus durer les institutions sociales. Nulle part, nous n’aurons un seul moment l’illusion qui, dans le Céleste Empire et sur la terre des brahmanes, pourrait aisément porter l’observateur à se demander si la pensée de l’homme n’est pas immortelle. Au lieu de cette majestueuse durée, au lieu de cette solidité presque impérissable, magnifique prérogative que l’homogénéité relative des races garantit aux deux sociétés que je viens de nommer, nous ne contemplerons plus, à dater du VII\textsuperscript{e} siècle avant J.-C., dans la turbulente arène où va se ruer la majeure partie des peuples blancs, qu’instabilité, inconstance dans l’idée civilisatrice. Tout à l’heure, pour mesurer sur la longueur du temps la série des faits hindous ou chinois, il fallait compter par dizaines de siècles. Déshabitués de cette méthode, nous constaterons bientôt qu’une civilisation de cinq à six cents ans est comparativement très vénérable. Les plus splendides créations politiques n’auront de vie que pour deux cents, trois cents ans, et, ce terme passé, elles devront se transformer ou mourir. Éblouis un instant de l’éphémère éclat de la Grèce et de la Rome républicaine, ce nous sera une grande consolation, quand nous en viendrons aux temps modernes, de réfléchir que, si nos échafaudages sociaux durent peu, ils ont néanmoins autant de longévité que tout ce que l’Asie et l’Europe ont vu naître, ont admiré, redouté, puis, une fois mort, foulé aux pieds depuis cette ère du VII\textsuperscript{e} siècle avant J.-C., époque de renouvellement et de transformation quasi complète de l’influence blanche dans les affaires des terres occidentales.\par
L’Ouest fut toujours le centre du monde. Cette prétention, toutes les régions un tant soit peu apparentes l’ont, à la vérité, nourrie et affichée. Pour les Hindous, l’Aryavarta est au milieu des contrées sublunaires ; autour de ce pays saint s’étendent les Dwipas, rattachés au centre sacré, comme les pétales de lotus au calice de la divine plante. Selon les Chinois, l’univers rayonne autour du Céleste Empire. La même fantaisie amusa les Grecs ; leur temple de Delphes était le nombril de la Bonne Déesse. Les Égyptiens furent aussi fous. Ce n’est pas dans le sens de cette vieille vanité géographique qu’il est permis à une nation ou à un ensemble de nations de s’attribuer un rôle central sur le globe. Il ne lui est pas même accordé de réclamer la direction constante des intérêts civilisateurs et, sous ce rapport, je me permets de faire une critique bien radicale du célèbre ouvrage de M. Gioberti \footnote{\emph{Primato civile e morale dell’ Italiani} ; in-8°, Bruxelles.}. C’est, en se plaçant au seul point de vue moral, qu’il y a de l’exactitude à soutenir que, en dehors de toutes les préoccupations patriotiques, le centre de gravité du monde social a toujours oscillé dans les contrées occidentales, sans les quitter jamais, ayant, suivant les temps, deux limites extrêmes, Babylone et Londres de l’est à l’ouest, Stockholm et Thèbes d’Égypte du nord au sud ; au delà, isolement, personnalité restreinte, impuissance à exciter la sympathie générale, et finalement la barbarie sous toutes ses formes.\par
Le monde occidental, tel que je viens d’en marquer le contour, est comme un échiquier où les plus grands intérêts sont venus se débattre. C’est un lac qui a constamment débordé sur le reste du globe, parfois le ravageant, toujours le fertilisant. C’est une sorte de champ aux cultures bariolées où toutes les plantes, salubres et vénéneuses, nutritives et mortelles, ont trouvé des cultivateurs. La plus grande somme de mouvement, la plus étonnante diversité de faits, les plus illustres conflits et les plus intéressants par leurs vastes conséquences se concentrent là, tandis qu’en Chine et dans l’Inde il s’est produit bien des ébranlements considérables dont l’univers a été si peu averti que l’érudition, éveillée par certains indices, n’en découvre les traces qu’avec beaucoup d’efforts. Au contraire, chez les peuples civilisés de l’Occident, il n’est pas une bataille un peu sérieuse, pas une révolution un peu sanglante, pas un changement de dynastie un tant soit peu notable, qui, arrivé depuis trente siècles, n’ait percé jusqu’à nous, souvent avec des détails qui laissent le lecteur aussi étonné que le peut être l’antiquaire lorsque, sur les monuments des anciens âges, son œil retrouve intacte la délicatesse des sculptures les plus fines.\par
 D’où vient cette différence ? C’est que, dans la partie orientale du monde, la lutte permanente des causes ethniques n’eut lieu qu’entre l’élément arian, d’une part, et les principes noirs et jaunes, de l’autre. Je n’ai pas besoin de faire remarquer que, là où les races noires ne combattirent qu’avec elles-mêmes, où les races jaunes tournèrent également dans leur cercle propre ou bien là encore où les mélanges noirs et jaunes sont aux prises aujourd’hui, il n’y a pas d’histoire possible. Les résultats de ces conflits étant essentiellement inféconds, comme les agents ethniques qui les déterminent, rien n’en a paru, rien n’en est resté. C’est le cas de l’Amérique, de la plus grande partie de l’Afrique et d’une fraction trop considérable de l’Asie. L’histoire ne jaillit que du seul contact des races blanches.\par
Dans l’Inde, l’espèce noble n’a de frottement qu’avec deux antagonistes inférieurs. Compacte, en débutant, dans son essence ariane, toute son œuvre est de se défendre contre l’invasion, contre l’immersion au sein des principes étrangers. Ce travail préservateur se poursuit avec énergie, avec conscience du danger et par des moyens qu’on peut dire désespérés, et qui seraient vraiment romanesques, s’ils n’avaient donné des résultats si longuement pratiques. Cette lutte si réelle, si vraie, n’est pourtant pas de nature à produire l’histoire proprement dite. Comme le rameau blanc mis en action est, ainsi que je viens de le dire, compact, et qu’il a un but unique, une seule idée civilisatrice, une seule forme, c’est assez pour lui que de vaincre et de vivre. Peu de variété dans l’origine des mouvements enfante peu de désirs de conserver la trace des faits, et de même qu’on a remarqué avec raison que les peuples heureux n’ont pas d’annales, on peut ajouter qu’ils n’en ont pas, parce qu’ils n’ont à se raconter que ce que tout le monde sait chez eux. Ainsi le développement d’une civilisation unitaire telle que celle de l’Inde, n’offrant à la réflexion nationale que très peu d’innovations surprenantes, de renversements inattendus dans les pensées, dans les doctrines, dans les mœurs, n’a rien non plus de grave à narrer, et de là vient que les chroniques hindoues ont toujours revêtu la forme théologique, les couleurs de la poésie, et présentent une si complète absence de chronologie et de si considérables lacunes dans l’enregistrement des choses.\par
En Chine, recueillir des faits est un usage des plus anciens. On se l’explique en observant que la Chine a été de bonne heure en relation avec des peuples généralement trop peu nombreux pour la pouvoir conquérir, assez forts cependant pour l’inquiéter et l’émouvoir, et qui, formés, en tout ou en partie, d’éléments blancs, ne venaient pas seulement, lorsqu’ils l’attaquaient, heurter des sabres, mais aussi des idées. La Chine, bien qu’éloignée du contact européen, a eu pourtant beaucoup de part aux contre-coups des différentes migrations, et plus on lira les grandes compilations de ses écrivains, plus on y trouvera de renseignements sur nos propres origines, renseignements que l’histoire de l’Aryavarta ne nous fournit pas avec une précision comparable. Déjà, depuis plusieurs années, c’est par les livres des lettrés que l’on a modifié, de la manière la plus heureuse, nombre d’idées fausses sur les Huns et les Alains. On y a recueilli encore des détails précieux au sujet des Slaves, et peut-être le trop petit nombre de renseignements jusqu’ici obtenus sur les débuts des peuples sarmates s’augmentera-il, par cette voie, de nouvelles découvertes. Du reste, cette abondance de réalités antiques, conservée par la littérature du Céleste Empire, s’applique, et ceci est fort à remarquer, beaucoup plutôt aux contrées du nord-ouest de la Chine qu’à celles du sud de cet État. Il n’en faut pas chercher la cause ailleurs que dans le frottement des populations mélangées de blanc du Céleste Empire avec les tribus blanches ou demi blanches des frontières ; de sorte qu’en suivant une progression évidente, à partir de l’inerte silence des races noires ou jaunes, on trouve d’abord l’Inde, avec ses civilisateurs, n’ayant que peu d’histoire, parce qu’ils ont peu de rapports avec d’autres rameaux de même race. On rencontre ensuite l’Égypte, qui n’en a qu’un peu plus par la même raison. La Chine vient après, en en présentant davantage, parce que les frottements avec l’étranger arian ont été réitérés, et on arrive ainsi au territoire occidental du monde, à l’Asie antérieure, aux contrées européennes, où les annales alors se développent avec un caractère permanent et une activité infatigable. C’est parce que là ne s’affrontent plus seulement un ou deux ou trois rameaux de l’espèce noble, occupés à se défendre de leur mieux contre l’enlacement des branches inférieures de l’arbre humain. La scène est tout autre, et sur ce théâtre turbulent, à dater du septième siècle avant notre ère, de nombreux groupes de métis blancs doués de différentes manières, tous aux prises les uns avec les autres, combattant du poing et surtout de l’idée, modifient sans fin leurs civilisations réciproques au milieu d’un champ de bataille où les peuples noirs et jaunes ne paraissent plus que déguisés par des mélanges séculaires et n’agissent sur leurs vainqueurs que par une infusion latente et inaperçue, dont le seul auxiliaire est le temps. Si, en un mot, l’histoire s’épanouit dès ce moment dans les régions occidentales, c’est que désormais ce qui sera à la tête de tous les partis sera mélangé de blanc, qu’il ne sera question que d’Arians, de Sémites (les Chamites étant déjà fondus avec ceux-ci), de Celtes, de Slaves, tous peuples originairement nobles, ayant des idées spéciales, tous s’étant fait sur la civilisation un système plus ou moins raffiné, mais tous en possédant un, et se surprenant, s’étonnant les uns les autres par les doctrines qu’ils vont émettant en toutes choses, et dont ils cherchent le triomphe sur les doctrines rivales. Cet immense et incessant antagonisme intellectuel a semblé, de tout temps, à ceux qui l’accomplissaient, des plus dignes d’être observé, recueilli, enregistré heure par heure, tandis que d’autres peuples moins tourmentés n’estimaient pas utile de garder grand souvenir d’une existence sociale toujours uniforme, malgré les victoires gagnées sur des races à peu près muettes. Ainsi, l’ouest de l’Asie et de l’Europe est le grand atelier où se sont posées les plus importantes questions humaines. C’est là, en outre, que pour les besoins du combat civilisateur, tout ce qui, dans le monde, a été d’un prix capable d’exciter la convoitise a tendu inévitablement à se concentrer.\par
Si on n’y a pas tout créé, on a voulu tout y posséder, et toujours on y a réussi, dans la mesure où l’essence blanche exerçait son empire, car, il ne faut pas l’oublier, la race noble n’y est pure nulle part, et repose partout sur un fond ethnique hétérogène qui, dans la plupart des circonstances, la paralyse d’une manière qui pour être inaperçue n’en est pas moins décisive. Aux temps où l’action blanche s’est trouvée le plus libre, on a vu dans le milieu occidental, dans cet océan où se déversent tous les courants civilisateurs, on a vu les conquêtes intellectuelles des autres rameaux blancs agissant au centre des sphères les plus éloignées, venir tour à tour enrichir le trésor commun de la famille. C’est ainsi qu’aux belles époques de la Grèce, Athènes s’empara de ce que la science égyptienne connaissait de meilleur et de ce que la philosophie hindoue enseignait de plus subtil.\par
À Rome, de même, on eut l’art de se saisir des découvertes appartenant aux points les plus lointains du globe. Au moyen âge, où la société civile semble, à beaucoup de personnes, inférieure à ce qu’elle fut sous les Césars et les Augustes, on redoubla cependant de zèle et on obtint de plus grands succès pour la concentration des connaissances. On pénétra bien plus avant dans les sanctuaires de la sagesse orientale, on y recueillit bien plus de notions justes ; et, en même temps, d’intrépides voyageurs accomplissaient, poussés par le génie aventureux de leur race, des voyages lointains auprès desquels les périples de Scylax et d’Annon, ceux de Pythéas et de Néarque méritent médiocrement d’être cités. Et, cependant, un roi de France, et même un pape du douzième siècle, promoteurs et soutiens de ces généreuses entreprises, étaient-ils comparables aux colosses d’autorité qui gouvernèrent le monde romain ? C’est qu’au moyen âge, l’élément blanc était plus noble, plus pur, plus actif par conséquent que les palais de la Rome antique ne l’avaient connu.\par
Mais nous sommes au septième siècle avant l’ère chrétienne, à cette époque importante où, dans la vaste arène du monde occidental, l’histoire positive commence pour ne plus cesser, où les longues existences d’État ne vont plus être possibles, où les chocs des peuples et des civilisations se succéderont à de très courts intervalles, où la stérilité et la fécondité sociales devront se déplacer et se remplacer dans les mêmes pays, au gré de l’épaisseur plus ou moins considérable des éléments blancs qui recouvriront les fonds noirs ou jaunes. C’est ici le lieu de revenir sur ce que j’ai dit dans le premier livre, de l’importance accordée par quelques savants à la situation géographique.\par
Je ne renouvellerai pas mes arguments contre cette doctrine. Je ne répéterai pas que, si les emplacements d’Alexandrie, de Constantinople, étaient totalement indiqués pour devenir de grands centres de population, ils seraient demeurés et resteraient tels dans tous les temps, allégation démentie par les faits. Je ne rappellerai pas non plus que, à en juger ainsi, ni Paris, ni Londres, ni Vienne, ni Berlin, ni Madrid, n’auraient aucun titre à être les célèbres capitales que ces villes sont toutes devenues, et, qu’à leur place, nous aurions vu, dès la naissance des premiers marchands, Cadix ou peut-être mieux Gibraltar, Alexandrie beaucoup plus tôt que Tyr ou Sidon, Constantinople à l’exclusion éternelle d’Odessa, Venise, sans espoir pour Trieste, accaparer une suprématie naturelle, incommunicable, inaliénable, indomptable, si je puis employer ce mot, et l’histoire humaine tourner éternellement autour de ces points prédestinés. En effet, ce sont bien les lieux de l’Occident les plus favorablement placés pour servir la circulation. Mais, et la chose est fort heureuse, le monde a d’autres et plus grands intérêts que ceux de la marchandise. Ses affaires ne vont pas au gré de la secte économiste. Des mobiles plus élevés que les vues de \emph{doit} et \emph{avoir} président à ses actes, et la Providence a, dès l’aurore des âges, ainsi établi les règles de la gravitation sociale, que le lieu le plus important du globe n’est pas nécessairement le mieux disposé pour acheter ou pour vendre, pour faire transiter des denrées ou pour les fabriquer, pour recueillir ou cultiver les matières premières. C’est celui où habite, à un moment donné, le groupe blanc le plus pur, le plus intelligent et le plus fort. Ce groupe résidât-il, par un concours de circonstances politiques invincibles, au fond des glaces polaires ou sous les rayons de feu de l’équateur, c’est de ce côté que le monde intellectuel inclinerait. C’est là que toutes les idées, toutes les tendances, tous les efforts ne manqueraient pas de converger, et il n’y a pas d’obstacles naturels qui pussent empêcher les denrées, les produits les plus lointains d’y arriver à travers les mers, les fleuves et les montagnes.\par
Les changements perpétuels intervenus dans l’importance sociale des grandes villes sont une démonstration sans réplique de cette vérité sur laquelle les prétentieuses déclamations des théoriciens économistes ne peuvent mordre. Rien de plus détestable que le crédit où l’on voit être une prétendue science qui, de quelques observations générales appliquées par le bon sens de toutes les époques arianes positives, a su extraire, en voulant y donner une cohésion dogmatique, les plus grandes et les plus dangereuses inepties pratiques  ; qui, en ne s’emparant que trop de la confiance d’un public sensible à l’influence des \emph{sesquipedalia verba}, s’élève au rôle funeste d’une véritable hérésie en se donnant les airs de dominer, de gourmander, d’accommoder à ses vues la religion, les lois, les mœurs. Basant la vie humaine tout entière et, de même, la vie des peuples sur ces mots devenus cabalistiques dans ses écoles : produire et consommer, elle appelle honorable ce qui n’est que naturel et juste : le travail du manœuvre, et le mot honneur perd toute la sublimité de sa primitive signification. Elle fait de l’économie privée la plus haute des vertus, et, à force d’exalter les avantages de la prudence pour l’individu et les bienfaits de la paix pour l’État, le dévouement, la fidélité publique, le courage et l’intrépidité deviennent presque des vices au gré de ses maximes. Ce n’est pas une science, car la négation la plus misérable des véritables besoins de l’homme, et des plus saints, forme sa base étroite. C’est un mérite de meunier et de filateur déplacé de son rang modeste et proposé à l’admiration des empires. Mais, pour me borner à réfuter la moindre de ses erreurs, je dirai, encore une fois, que, malgré les convenances commerciales qui pouvaient recommander tel ou tel point topographique, les civilisations de l’antiquité n’ont jamais cessé de s’avancer vers l’ouest, simplement parce que les tribus blanches elles-mêmes ont suivi ce chemin, et ce n’est qu’arrivées sur notre continent qu’elles ont rencontré ces mélanges jaunes qui les ont acheminées vers les idées utilitaires adoptées avec plus de réserve par la race ariane et trop méconnues du monde sémitique. Aussi faudra-t-il s’attendre à voir les nations blanches de plus en plus réalistes, de moins en moins artistes à mesure qu’on les obser­vera plus avant dans l’ouest. Ce n’est pas, à coup sûr, pour des raisons empruntées à l’influence climatérique qu’elles seront telles. C’est uniquement parce qu’elles deviendront à la fois plus mêlées d’éléments jaunes et plus dégagées de principes mélaniens. Dressons ici, afin de nous en mieux convaincre, une liste de gradation des résultats que j’indique. Il est nécessaire que le lecteur y soit attentif. Les Iraniens, on va le constater tout à l’heure, furent plus réalistes, plus mâles que les Sémites, lesquels, l’étant plus que les Chamites, permettent d’établir cette progression :\par
Noirs,\par
Chamites,\par
 Sémites,\par
Iraniens.\par
On verra ensuite la monarchie de Darius couler au fond de l’élément sémitique et passer la palme au sang des Grecs, qui, bien que mélangés, étaient cependant, au temps d’Alexandre, plus libres d’alliages mélaniens.\par
Bientôt les Grecs, noyés dans l’essence asiatique, seront ethniquement inférieurs aux Romains, qui pousseront l’empire du monde d’une bonne distance de plus vers l’ouest, et qui, dans leur fusion faiblement jaune, blanche à un plus haut degré, et enfin sémitisée dans une progression croissante, auraient pourtant gardé la domination, si des compétiteurs plus blancs n’avaient encore une fois paru. Voilà pourquoi les Arians Germains fixèrent décidément la civilisation dans le nord-ouest.\par
De même que je viens de rappeler ce principe du livre premier, que la position géographique des nations ne fait nullement leur gloire et ne contribue (j’aurais pu l’ajouter) que dans une mesure minime à activer leur existence politique, intellectuelle, commerciale, de même encore pour les pays souverains les questions de climat restent non avenues, et ainsi que nous avons vu en Chine l’antique suprématie, donnée dans le premier temps au Yunnan, passer ensuite au Pé-tché-li ; que dans l’Inde les contrées du nord sont aujourd’hui les plus vivaces, quand, pendant de longs siècles, le sud, au contraire, l’emporta, ainsi il n’est pas, dans l’occident du monde, de climats qui n’aient eu leurs jours d’éclat et de puissance. Babylone où il ne pleut jamais, et l’Angleterre où il pleut toujours ; le Caire où le soleil est torride, Saint-Pétersbourg où le froid est mortel, voilà les extrêmes : la domination règne ou a régné dans ces différents lieux.\par
Je pourrais aussi, après ces questions, soulever celle de la fertilité : rien de plus inutile. La Hollande nous répond assez que le génie d’un peuple vient à bout de tout, crée de grandes cités dans l’eau, fait une patrie sur pilotis, attire l’or et les hommages de l’univers dans des marécages improductifs. Venise prouve plus encore : elle dit que, sans territoire aucun, pas même un marécage, pas même une lande, un État se peut fonder, qui lutte de splendeur avec les plus vastes et vit au delà des années accordées aux plus solides.\par
Il est donc établi que la question de race est majeure pour apprécier le degré du principe vital dans les grandes fondations ; que l’histoire s’est créée, développée et soutenue là seulement où plusieurs rameaux blancs se sont mis en contact ; qu’elle revêt le caractère positif d’autant plus qu’elle traite des affaires de peuples plus blancs, ce qui revient à dire que ceux-ci sont les seuls historiques, et que le souvenir de leurs actes importe uniquement à l’humanité. Il s’ensuit encore de là que l’histoire, aux différentes époques, tient plus de compte d’une nation à mesure que cette nation domine davantage, ou, autrement dit, que son origine blanche est plus pure.\par
Avant d’aborder l’étude des modifications introduites au VII\textsuperscript{e} siècle avant J.-C. dans les sociétés occidentales, j’ai dû constater l’application de certains principes posés précédemment et faire jaillir de nouvelles observations du terrain sur lequel je marchais. J’aborde maintenant l’analyse de ce que la composition ethnique des Zoroastriens présente de plus remarquable.
\section[{IV.2. Les Zoroastriens.}]{IV.2. \\
Les Zoroastriens.}
\noindent Les Bactriens, les Mèdes, les Perses, faisaient partie de ce groupe de peuples qui, en même temps que les Hindous et les Grecs, furent séparés des autres familles blanches de la haute Asie. Ils descendirent avec eux non loin des limites septentrionales de la Sogdiane \footnote{Lassen, \emph{Indische Alterthumskunde.}}. Là, les tribus helléniques abandonnèrent la masse de l’émigration et tournèrent à l’ouest, en suivant les montagnes et les bords inférieurs de la Caspienne. Les Hindous et les Zoroastriens continuèrent à vivre ensemble et à s’appeler du même nom d’\emph{Aryas} ou Airyas \footnote{Burnouf ne doute pas que les textes les plus anciens et les plus authentiques du Zend-Avesta ne fixent le séjour primitif des Zoroastriens au pied du Bordj, sur les bords de l’Arvanda, c’est-à-dire dans la partie occidentale des Monts Célestes. (\emph{Commentaire sur le Yaçna}, t. I, \emph{additions et corrections}, p. CLXXXV.)} pendant une période assez longue, jusqu’à ce que des querelles religieuses, qui paraissent avoir acquis un grand caractère d’aigreur, aient porté les deux peuples à se constituer en nationalités distinctes \footnote{Lassen, \emph{Indische Alterth}., t. i, p. 516 et passim. ‑ Le Zend-Avesta, livre de cette loi protestante, reconnaît lui-même qu’il y a eu, dans les temps antérieurs, une autre foi. C’est celle des \emph{hommes anciens}, les pischdadiens (alphabet étranger). Je doute que cette antique doctrine fût le brahmanisme. C’était beaucoup plutôt la source d’où le brahmanisme est sorti, le culte des purohitas, peut-être même de leurs prédécesseurs. ‑ Les pischdadiens sont appelés nettement par le Zend-Avesta les \emph{hommes anciens}, par opposition à ceux qui ont vécu postérieurement à la séparation d’avec les Hindous, et qui sont nommés en zend \emph{nabânazdita} (contemporains) et, en sanscrit, \emph{nabhanadichtra}, d’après un des fils de Manou, privé de sa part de l’héritage paternel, suivant le Rigvéda. (Burnouf, \emph{Commentaire sur le Yaçna}, t. I, p. 566 et passim.)}.\par
Les nations zoroastriennes occupaient d’assez grands territoires, dont il est difficile de préciser les bornes au nord-est. Probablement elles s’étendaient jusqu’au fond des gorges du Muztagh, et sur les plateaux intérieurs, d’où plus tard elles sont venues apporter aux contrées européennes les noms si célèbres des Sarmates, des Alains et des Ases. Vers le sud, on connaît mieux leurs limites. Elles envahirent successivement de­puis la Sogdiane, la Bactriane et le pays des Mardes jusqu’aux frontières de l’Arachosie, puis jusqu’au Tigre. Mais ces régions si vastes renferment aussi d’immenses espaces complètement stériles et inhabitables pour de grandes multitudes. Elles sont coupées par des déserts de sables, traversées par des montagnes d’une inexorable aridité. La population ariane ne pouvait donc y subsister en nombre. La force de la race se trouva ainsi rejetée à jamais hors du centre d’action que devaient embrasser un jour les monarchies des Mèdes et des Perses. Elle fut réservée par la Providence à fonder bien plus tard la civilisation européenne.\par
Quoique séparées des Hindous, les peuplades zoroastriennes de la frontière orientale ne s’en distinguaient pas aisément à leurs propres yeux ni à ceux des Grecs. Toutefois, les habitants de l’Aryavarta, en les acceptant pour consanguins, se refusaient, avec horreur, à les considérer comme compatriotes. Il était d’autant plus facile à ces tribus limitrophes de n’être qu’à demi zoroastriennes, que la nature de la réforme religieuse, origine du peuple entier, se basant sur la liberté, était loin de créer un lien social aussi fort que celui de l’Inde. On est en droit de croire, au contraire, puisque l’insurrection avait eu lieu contre une doctrine assez tyrannique, que, suivant l’effet naturel de toute réaction, l’esprit protestant, voulant abjurer la sévère discipline des brahmanes, avait donné à gauche et institué un peu de licence. En effet, les nations zoroastriennes nous apparaissent très hostiles les unes aux autres et s’opprimant mutuellement. Chacune, constituée à part, menait, suivant l’usage de la race blanche, une existence turbulente au milieu de grandes richesses pastorales, gouvernée par des magistrats soit électifs, soit héréditaires, mais forcés de compter de près avec l’opinion publique \footnote{Hérodote\emph{, Clio}, XCVI.}. Toutes ces tribus se piquaient donc d’indépendance. Ainsi organisées, elles descendaient graduellement vers le sud-ouest, où elles devaient finir par rencontrer les Assyriens.\par
Avant l’heure de ce contact, les premières colonnes trouvèrent, dans les environs de la Gédrosie, des populations noires ou du moins chamites, et se mêlèrent intimement à elles \footnote{Voir Klaproth, \emph{Asia polyglotta}, p. 62. ‑ Ce philologue remarque l’extrême fusion de tous les idiomes de l’Asie antérieure soit avec les principes arians ou sémitiques, soit aussi avec les éléments finniques. Il relève cette dernière circonstance pour l’arménien ancien, qui, suivant lui, a beaucoup de rapport avec les langues du nord de l’Asie. (\emph{Ouvr. cité}, p. 76. ‑ Cette assertion appuie le système d’interprétation des inscriptions médiques proposé par M. de Saulcy.}.\par
De là vint que les nations zoroastriennes du sud, celles qui prirent part à la gloire persique, furent de bonne heure atteintes par une certaine dose de sang mélanien. Le plus grand nombre, pénétré trop profondément par cet alliage, tomba, longtemps avant la conquête de Babylone, presque à l’état des Sémites. Ce qui l’indique, c’est que les Bactriens, les Mèdes et les Perses furent les seuls Zoroastriens qui jouèrent un rôle. Les autres se bornèrent à l’honneur d’appuyer ces familles d’élite.\par
Il peut paraître singulier que ces Arians, imprégnés ainsi du sang des noirs, directement ou par alliance avec les Chamites et les Sémites dégénérés, aient pu arriver à remplir le personnage important que leur attribue l’histoire.\par
 Si donc on se croyait en droit de supposer, chez toutes leurs tribus, une mesure égale dans la proportion du mélange, il deviendrait difficile d’expliquer cliniquement la domination des plus illustres de ces dernières sur les populations assyriennes.\par
Mais, pour fixer la certitude, il suffit de comparer entre elles les langues zoroastriennes, ainsi que je l’ai déjà fait ailleurs.\par
Le zend, ce fait n’est pas douteux, parlé chez les Bactriens, habitants de cette Balk appelée en Orient \emph{la mère des villes} \footnote{Les Bactriens, en zend \emph{Bakhdi}, sont les Bahlikas du Mahabharata. Ils étaient parents, suivant ce poème, du dernier des Kouravas et de Pandou. Ainsi leur caractère profondément arian est bien et dûment établi. (Lassen, \emph{Indische Alterthumskunde}, t. I, p. 297 ; voir aussi A F. v. Schack, \emph{Heldensagen von Firdusi}, in-8°, Berlin, 1851 ; \emph{Enleit.}, p. 16 et passim ; voir aussi Lassen, \emph{Zeitsch, f. K.. d. Morgenl.}, qui identifie les Bactriens avec les Afghans, dont le nom national est Pouschtou, t. II, p. 53.) ‑ Le nom de Balk, (alphabet étranger) donné à la cité des Bactriens, n’est pas le plus ancien qu’ait porté cette ville. Elle s’est appelée précédemment \emph{Zariaspe}. (Burnouf, \emph{Comment. sur le Yaçna}, notes et éclaircissements, t. I, p. CXII.}, les plus puissants des Zoroastriens primitifs, fut presque pur d’alliages sémitiques, et le dialecte de la Perside, qui ne jouit pas autant de cette prérogative, la posséda cependant dans un certain degré, supérieur au médique, moins sémitisé à son tour que le pehlvi, de sorte que le sang des futurs conquérants de l’Asie antérieure conservait, dans les plus nobles de ses rameaux du sud, un caractère assez arian pour expliquer la supériorité de ceux-ci,\par
Les Mèdes et surtout les Perses furent les successeurs de l’ancienne influence des Bactriens qui, après avoir dirigé les premiers pas de la famine dans les voies du magisme, avaient perdu leur prépondérance d’une manière aujourd’hui inconnue. Les héritiers méritaient l’honneur qui leur échut. Nous venons de voir qu’ils étaient restés Arians, moins complets sans doute que les Zoroastriens du nord-est, et même que les Grecs, tout autant néanmoins que les Hindous de la même époque, beaucoup plus que le groupe de leurs congénères, déjà presque absorbé sur les bords du Nil. Le grand et irrémédiable désavantage que les Mèdes et les Perses apportaient, en entrant sur la scène politique du monde, c’était leur chiffre restreint et la dégénération déjà avancée des autres tribus zoroastriennes du sud, leurs alliées naturelles. Toutefois, ils pouvaient commander quelque temps. Ils étaient encore en possession d’un des caractères les plus honorables de l’espèce noble, une religion plus rapprochée des sources véridiques que la plupart des Sémites, aux yeux desquels ils allaient être appelés à faire acte de force.\par
Déjà, à une époque reculée, une tribu médique avait régné sur l’Assyrie. Sa faiblesse numérique l’avait contrainte à se soumettre à une invasion chaldéenne-sémite venue des montagnes du nord-ouest. Dès ce temps, des doctrines religieuses, relativement vénéra­bles, se rattachent au nom de Zoroastre porté par le premier roi de cette dynastie ariane \footnote{Lassen, \emph{Indische Alterth.}, t. 1, p. 753 et passim.} : il n’y a pas moyen de confondre le prince ainsi appelé avec le réformateur religieux ; mais la présence d’un tel nom, à la date de 2234 avant J.-C., peut servir à montrer que les Mèdes et les Perses du VII\textsuperscript{e} siècle conservaient la même foi monothéistique que leurs plus anciens ancêtres.\par
Les Bactriens et les tribus arianes qui les limitaient au nord et à l’est avaient créé et développé ces dogmes. Ils en avaient vu naître le prophète dans cet âge bien éloigné où, sous les règnes nébuleux des rois kaïaniens, les nations zoroastriennes, y compris celles d’où devaient sortir un jour les Sarmates, étaient au lendemain de leur séparation d’avec les Hindous \footnote{Kaïanien, vient de Kaï, syllabe qui précède les noms de plusieurs rois de cette dynastie zoroastrienne : ainsi Kaï-Kaous et Kaï-Khosrou. Ce mot paraît avoir été le titre des monarques. En zend, il a la forme Kava, et est identique avec le sanscrit Kavi (soleil). Peut-être n’est-il pas sans intérêt de rapprocher ce sens de celui du Phra égyptien. (Voir Burnouf, Commentaire sur le Yaçna, t. I, p. 424 et passim.) ‑ Ces rois kaïniens donnèrent la première impulsion à la nationalité séparatiste des Zoroastriens. Ils ont jeté certainement un grand éclat, puisque, à travers tant de siècles, ils ont produit des traditions nombreuses et persistantes qui font la partie la plus notable du Schahnameh.}.\par
À ce moment, la religion nationale, bien que, par sa réforme, devenue étrangère au culte des purohitas, et même à ces notions théologiques plus simples, patrimoine primitif de toute la race blanche dans les régions septentrionales du monde. Cette religion était incomparablement plus digne, plus morale, plus élevée, que celle des Sémites. On en peut juger par ce fait, qu’au VII\textsuperscript{e} siècle elle valait mieux, malgré ses altérations, que le polythéisme, pourtant moins abject, adopté dès longtemps par les nations helléniques \footnote{Comme toutes les religions, aux époques de foi, le magisme était ce qu’on appelle, de nos jours, intolérant. Il détestait le polythéisme dans toutes ses formes. Xerxès enleva l’idole de Bel, qui trônait à Babylone, et détruisit ou dévasta tous les temples qu’il rencontra en Grèce. ‑ Ainsi Cambyse ne fit en Égypte qu’obéir à l’esprit général de sa nation lorsqu’il maltraita si fort les cultes du pays. (Voir Bœttiger, \emph{Ideen zur Kunstmythologie} (Dresde, In-8°, 1826), t. I, p. 25 et passim.)}. Sous la direction de cette croyance, les mœurs n’étaient pas non plus si dégradées et conservaient de la vigueur.\par
Conformément à l’organisation primitive des races arianes, les Mèdes vivaient, par tribus, dispersés dans des bourgades. Ils élisaient leurs chefs, comme jadis leurs pères avaient élu leurs viç-patis \footnote{Le mot employé par le Schahnameh pour désigner la dignité royale rappelle vivement les doctrines indépendantes des Arians primitifs. Féridoun porte le titre de schahr-jar, (alphabet étranger), (l’ami de la cité). ‑ Sur les sources antéislamitiques où Firdousi a puisé les traditions qu’il enchaîne, voir A. F. de Schack, \emph{Einl}., p. 52 et passim.}. Ils étaient belliqueux et remuants, toutefois, avec le sens de l’ordre, et ils le prouvèrent en faisant aboutir l’exercice de leur droit de suffrage à la fondation d’une monarchie régulière, basée sur le principe d’hérédité \footnote{Tous les faits qui composent l’histoire de la formation du royaume médique sont racontés par Hérodote, avec sa puissance de coloris ordinaire, \emph{Clio}, XCVIII et passim.}. Rien là que nous ne puissions également retrouver dans les Hindous antiques, chez les Égyptiens arians, chez les Macédoniens, les Thessaliens, les Épirotes, comme dans les nations germani­ques. Partout, le choix du peuple crée la forme de gouvernement, presque partout préfère la monarchie et la maintient dans une famille particulière. Pour tous ces peuples, la question de descendance et la puissance du fait établi sont deux principes, ou, pour mieux dire, deux instincts qui dominent les institutions sociales et les vivifient. Ces Mèdes, pasteurs et guerriers, restèrent des hommes libres, dans toute la force du terme, même pendant cette période où leur petit nombre les obligea de subir la suzeraineté des Chaldéens, et, si leur esprit exagéré d’indépendance, en les poussant au fractionnement et à l’antagonisme des forces, contribua certainement à prolonger leur temps de subordination, on ne peut admirer assez que cet état n’ait pas dégradé leur naturel, et qu’après de longs tâtonnements, la nation, ayant rallié toutes ses ressources dans sa forme monarchique, soit devenue capable, après seize cents ans, de reprendre la conquête du trône d’Assyrie et de l’exécuter.\par
Depuis qu’elle avait été chassée de Ninive, elle n’avait pas déchu. Elle avait persisté dans son culte, honneur bien rare, dû évidemment à son homogénéité persistante. Elle avait conservé son goût d’indépendance sous des chefs d’ailleurs par trop peu maîtres de leurs gouvernés : la nation médique était donc restée ariane. Quand une fois elle fut arrachée à son anarchie belliqueuse, le besoin de donner une application à sa vigueur, laissée sans emploi par l’heureux étouffement des discordes civiles, tourna ses vues vers les conquêtes extérieures. Commençant par soumettre les nations parentes établies dans son voisinage, entre autres, les Perses \footnote{Le Mahabharata connaît les Perses, il les appelle \emph{Parasikas.} Mais à cette époque lointaine des guerres des Pandavas et des fils de Kourou cette petite nation n’avait encore aucun renommée. C’est ce qui fait que, dans le poème hindou, elle a les simples honneurs d’une mention. (Lassen, \emph{Zeitschrift f. d. K. des Morgenl.}, t. II, p. 53.)}, elle se fortifia de leur adjonction. Puis, quand elle eut amené sous ses drapeaux et fondu en un seul corps de peuples dont elle était la tête tous les disciples méridionaux de sa religion, elle attaqua l’empire ninivite.\par
Beaucoup d’écrivains n’ont vu, dans ces guerres de l’Asie antérieure, dans ces rapides conquêtes, dans ces États si promptement construits, si subitement renversés, que des coups de main sans liaison, une série d’événements dénués de causes profon­des, et dès lors de portée. N’acceptons pas un tel jugement.\par
Les dernières émigrations sémitiques avaient cessé de descendre les montagnes de l’Arménie et de venir régénérer les populations assyriennes. Les contrées riveraines de la Caspienne et voisines du Caucase n’avaient plus d’hommes à envoyer au dehors. Dès longtemps, les colonnes voyageuses des Hellènes avaient achevé leur passage, et les Sémites, demeurés dans ces contrées, n’en étaient plus expulsés par personne. L’Assyrie ne renouvelait donc plus son sang depuis des siècles, et l’abondance des principes noirs, toujours en travail d’assimilation, avait effectué la décadence des races superposées \footnote{Movers, \emph{das Phœniz. Alterthum.}, t. I, 2\textsuperscript{e} partie, p. 415. ‑ Cette décadence était si profonde, et causée si évidemment par l’anarchie ethnique, que les Égyptiens, non moins dégénérés, mais plus compacts parce qu’il y avait en jeu, dans leur sang, moins d’éléments constitutifs, prirent un moment le dessus vis-à-vis de leurs anciens et redoutés adversaires. Au VII\textsuperscript{e} siècle, leur influence l’emportait en Phénicie. Les Mèdes eurent bientôt raison de cette énergie relative.}.\par
En Égypte, il s’était passé quelque chose d’analogue. Mais, comme le système des castes, malgré ses imperfections, conservait encore cette société dans ses principes constitutifs, les gouvernants de Memphis, se sentant d’ailleurs trop faibles pour résister à tous les chocs, tournaient leur politique à maintenir entre eux et la puissance ninivite, qu’ils redoutaient par-dessus tout, un rideau de petits royaumes syriens. Cachés derrière ce rempart, ils continuaient, tant bien que mal, à se traîner dans leurs ornières accoutumées, descendant la pente de la civilisation à mesure que le mélange noir les envahissait.\par
Si les Ninivites les épouvantaient par-dessus tout, ces peuples n’étaient pas les seuls à les tenir en émoi. Se reconnaissant également incapables de lutter contre l’imperceptible puissance des pirates grecs, (mot grec) Arians qui s’intitulaient \emph{rois de mer}, comme le firent plus tard leurs parents les Arians Scandinaves, les Égyptiens avaient eu recours à la prudente résolution de se séquestrer en fermant le Nil à ses embouchures. C’était au prix de précautions si excessives que les descendants de Rhamsès espéraient encore préserver longtemps leur tremblante existence.\par
À côté des deux grands empires du monde occidental ainsi affaiblis, les Hellènes se montraient à peu près dans l’état qu’avaient connu les Mèdes avant la fondation de la monarchie unitaire. Ils faisaient preuve de la même turbulence, du même amour de liberté, des mêmes sentiments belliqueux, d’une ambition égale de commander un jour aux autres peuples, et, retenus par leur fractionnement, ils restaient incapables d’entreprendre rien de plus vaste que des colonisations déjà assises aux embouchures des fleuves de l’Euxin, en Italie et sur la côte asiatique, où leurs villes, encouragées par la politique assyrienne à faire une concurrence heureuse au commerce des cités de Phénicie, dépendaient essentiellement, à ce titre, de la puissance souveraine à Ninive et à Babylone.\par
Ce fut à cette heure, où aucune des grandes puissances anciennes n’était plus en état d’attaquer ses voisins, que les Mèdes se présentèrent en candidats au gouvernement de l’univers. L’occasion était on ne peut mieux choisie : il s’en fallut de peu, cependant, qu’un acteur tout à fait inattendu, qui vint brusquement se précipiter sur la scène, ne dérangeât complètement la distribution des rôles.\par
Les Kimris, Cimmériens, Cimbres ou Celtes, comme on voudra les appeler, peuples blancs mêlés d’éléments jaunes, auxquels personne ne prenait garde, débouchèrent tout à coup dans l’Asie inférieure, venant de la Tauride, et, après avoir ravagé le Pont et toutes les contrées environnantes, mirent le siège devant Sardes et la prirent \footnote{Movers, t. II, 1\textsuperscript{re} partie, p, 419.}.\par
Ces farouches conquérants répandaient sur leur passage la stupeur et l’épouvante. Ils n’auraient, sans doute, pas demandé mieux que de justifier la haute opinion que la vue seule de leurs épées faisait concevoir de leur puissance. Malheureusement pour eux, ils reproduisaient un accident que nous avons déjà observé. Vainqueurs, ils n’étaient que des vaincus : poursuivants, c’étaient des fuyards. Ils ne dépossédaient que pour trouver un refuge. Attaqués dans les steppes, qui furent plus tard la Sarmatie asiatique, par un essaim de nations mongoles ou scythiques, et forcés de céder, ils s’étaient échappés jusqu’aux lieux où les Sémites tremblaient à leurs pieds, mais où, fatalement, leurs adversaires vinrent les poursuivre. De sorte que l’Asie antérieure avait à peine éprouvé les premières dévastations des Celtes, qu’elle tomba aux mains des hordes jaunes. Celles-ci, tout en continuant à guerroyer contre les fugitifs, s’attaquèrent aux villes et aux trésors des pays envahis, proie à coup sûr beaucoup plus attrayante \footnote{Movers, \emph{das Phœnizische Alterthum.}, t. II, 1\textsuperscript{re} partie, p. 401 et passim, et 419.}.\par
Les Celtes étaient moins nombreux que leurs antagonistes. Ils furent battus et dispersés. Les Scythes poursuivirent alors, sans compétiteurs, le cours de leurs victoi­res, nuisibles surtout aux desseins de la politique mède. Cyaxare venait, précisément, d’investir Ninive, et il n’avait plus qu’à franchir ce dernier obstacle pour se voir maître de l’Asie assyrienne. Irrité de cette intervention malencontreuse, il leva le siège et vint attaquer les Scythes. Mais la fortune ne le seconda pas, et, mis en déroute complète, il lui fallut laisser les barbares, comme il les appelait sans doute, libres de continuer leurs courses dévastatrices. Ceux-ci pénétrèrent jusque sur la lisière de l’Égypte, où les supplications et plus encore les présents obtinrent d’eux qu’ils n’entreraient pas. Satisfaits de la rançon, ils allèrent porter ailleurs leurs violences. Cette bacchanale mongole fut terrible, et pourtant dura peu. Vingt-huit ans en virent la fin. Les Mèdes, tout battus qu’ils avaient été dans une première rencontre, étaient trop réellement supérieurs aux Scythes pour supporter indéfiniment leur joug. Ils revinrent à la charge, et cette fois avec un plein succès \footnote{Hérodote, \emph{Clio}, CVI.}. Les cavaliers jaunes, chassés par les troupes de Cyaxare, s’enfuirent dans le pays au nord de l’Euxin. Ils allèrent y continuer, avec les peuples plus ou moins mélangés de sang finnois, les luttes anarchiques auxquelles ils sont propres, tandis que les Zoroastriens, débarrassés d’eux, reprenaient leur œuvre au point où elle avait été interrompue. L’invasion celto-scythe repoussée, Ninive fut assiégée de nouveau, et Cyaxare, vainqueur intelligent, entra dans ses murs.\par
Dès lors fut assurée la domination de la race ariane-zoroastrienne méridionale, à qui je puis désormais donner, sans inconvénients, le nom géographique d’iranienne. Il n’y eut plus que la seule question de savoir quel serait celui des rameaux de cette famille qui obtiendrait la suprématie. Le peuple mède n’était pas le plus pur. Pour ce motif, il ne pouvait garder la prédominance ; mais il était le plus civilisé par son contact avec la culture chaldéenne, et c’est là ce qui lui avait d’abord donné la place la plus éminente. Le premier, il avait préféré une forme de gouvernement régulière à de stériles agitations, et ses mœurs, ses habitudes, étaient plus raffinées que celles des autres branches parentes. Cependant, tous ces avantages résultant d’une affinité certaine avec les Assyriens, et que l’état de l’idiome accuse, avaient été achetés aux prix d’un hymen qui, en altérant le sang médique, avait aussi diminué sa vigueur vis-à-vis d’une autre tribu iranienne, celle des Perses, de sorte que, par droit de supériorité ethnique, la souve­raineté de l’Asie fut enlevée aux compagnons de Cyaxare, et passa dans la branche demeurée plus ariane. Un prince qui, par son père, appartenait à la nation perse, par sa mère à la maison royale de Déjocès, Cyrus, vint se substituer à la ligne directe et donner à ses compatriotes la supériorité sur la tribu fondatrice de l’empire et sur toutes les autres familles consanguines. Il n’y eut pas cependant substitution absolue : les deux peuples se trouvaient unis de trop près ; il s’établit seulement, entre les domina­teurs, une nuance, et qui encore ne dura pas longtemps ; car les Perses comprirent la nécessité de soumettre leur vigueur un peu inculte à l’école des Mèdes plus expérimentés. Ainsi, il se trouva bientôt que les rois de la maison de Cyrus \footnote{Les noms des premiers souverains perses sentent fortement la primitive identité des notions zoroastriennes avec les Hindous, et même avec les autres branches arianes. C’est ainsi que le père des Achéménides s’appelait Kourou, comme le chef des Kouravas blancs que nous avons vus envahir l’Inde à une époque très ancienne. Plus tard, Cambyse est nommé, dans l’inscription cunéiforme de Bi-Soutoum, \emph{Ka}(\emph{m})\emph{budya}, comme la tribu des kschattryas dissidents, habitant la rive droite de l’Indus, les Kambodyas. (Lassen, Indische Alterth., t. I, p. 598.) ‑ Il est curieux de remarquer que les habitants de l’Hindou‑Koh se nomment aujourd’hui Kamodje. Avant les conquêtes des Afghans, leur territoire allait jusqu’à l’Indus. (Lassen, \emph{Zeitschriht f. d. K. d. Morgenl.}, t. II, p. 56 et passim.)} ne se faisaient aucun scrupule de placer les plus habiles de ces derniers aux premiers rangs. Il y eut donc partage réel du pouvoir entre les deux tribus souveraines et les autres peuples iraniens plus sémitisés \footnote{Il faudrait même admettre que les Bactriens, ce rameau le plus anciennement civilisé de la famille zoroastrienne, eurent leur part de suprématie sous la dynastie de Darius, si l’on adoptait l’idée de M. Roth. Ce savant a avancé que les Achéménides étaient des vassaux bactriens des rois perses. (Roth, \emph{Geschichte der abendlædischen philosophie} (Mannheim, 1846, in-8°), t. I, p. 384 et passim.) Cependant, cette hypothèse a besoin d’être encore étudiée.}. Quant aux Sémites et autres groupes chamitisés ou noirs formant l’immense majorité des populations soumises, ils ne furent que le piédestal commun de la domination zoroastrienne.\par
Ce dut être pour les nations si dégénérées, si lâches, si perverties, et en même temps si artistes de l’Assyrie, un spectacle et une sensation bien étranges que de tomber sous le rude commandement d’une race guerrière, sérieuse et livrée aux inspira­tions d’un culte simple, moral, aussi idéaliste que leurs propres notions religieuses l’étaient peu.\par
Avec l’arrivée des Iraniens, les horreurs sacrées, les infamies théologiques prirent fin. L’esprit des mages ne pouvait s’en accommoder. On eut une preuve bien grande et bien singulière de cette intolérance lorsque, plus tard, le roi Darius, devenu maître de la Phénicie, envoya défendre aux Carthaginois de sacrifier des hommes à leurs dieux, offrandes doublement abominables aux yeux des Perses en ce qu’elles offensaient la piété envers des semblables et souillaient la pureté de la flamme sainte du bûcher \footnote{Darius Hystaspes leur interdit aussi de manger de la chair de chien. La coutume phénicienne des massacres hiératiques, qui, à l’époque des calamités publiques, porta les Carthaginois à égorger à la fois, sur leurs autels, des centaines d’enfants, coutume qui faisait dire à Ennius : « Et Poinei « solitei sos sacrificare puellos, » reprit quand tomba l’influence des Perses. Les Grecs cherchèrent en vain à décider les Carthaginois à renoncer à de telles monstruosités. Elles existaient encore secrètement au temps de Tibère, et s’étaient transmises, avec le sang sémitique, à la colonie romaine. (Bœttiger, \emph{Ideen zur Kunstmythologie}, t. i, p. 373.)}. Peut-être était-ce la première fois, depuis l’invention du polythéisme, que des prescriptions émanées du trône avaient parlé d’humanité. Ce fut un des caractères remarquables du nouveau gouvernement de l’Asie. On s’occupa désormais de rendre la justice à chacun et de faire cesser les atrocités publiques, sous quelque prétexte qu’elles eussent lieu. Particularité non moins nouvelle, le grand roi se soucia d’administrer. À dater de cette époque, le grandiose s’abaisse, et tout tend à devenir plus positif. Les intérêts sont plus régulièrement ménagés. Il y a du calcul, et du calcul raisonnable, terre à terre, dans les institutions de Cyrus et de ses successeurs. Pour bien dire, le sens commun inspire la politique, à côté et quelquefois un peu au-dessus des passions tumultueuses. Jusqu’alors ces dernières avaient beaucoup trop parlé \footnote{Le successeur du faux Smerdis s’exprimait ainsi dans l’inscription de Bi-Soutoun « Darius le « roi dit : Dans toutes ces provinces, j’ai donné faveur et protection à l’homme laborieux. Le « fainéant, je l’ai puni avec sévérité. » (Rawlinson, \emph{Journal of the Royal Asiatic Society}, vol. XVI, part. I, p. XXXV.) ‑ Ce Darius qui parlait ainsi portait dans son nom l’expression d’une idée utilitaire : \emph{Daryawus} signifie \emph{celui qui maintient l’ordre}. (Schack, \emph{Heldensagen von Firdusi}, p. 11.)}.\par
En même temps que l’impétuosité décroît chez les gouvernants, et que l’organisa­tion matérielle fait des progrès, le génie artistique décline d’une manière frappante Les monuments de l’époque perse ne sont qu’une reproduction médiocre de l’ancien style assyrien \footnote{Layard, \emph{Niniveh und seine Ueberreste}, Leipzig, 1850, p. 340. ‑ Je n’ai eu à ma disposition que la traduction de M. Meissner, excellente du reste. Le savant voyageur anglais discute d’une manière rare les rapports du style perse avec les modèles de l’Assyrie et de l’Égypte.}. Il n’y a plus d’invention dans les bas-reliefs de Persépolis. On n’y retrouve pas même la froide correction qui survit d’ordinaire aux grandes écoles. Les figures apparaissent gauches, lourdes, grossières. Ce ne sont plus les produits de sculpteurs, ce sont les ébauches imparfaites de manœuvres maladroits ; et puisque le grand roi, dans sa magnificence, ne se procurait pas des jouissances artistiques comparables à celles dont avaient joui ses prédécesseurs chaldéens, il faut nécessairement croire qu’il n’en éprouvait nullement le désir, et que les représentations médiocres étalées sur les murs de son palais pour célébrer sa gloire flattaient assez son orgueil et suffisaient à son goût.\par
On a souvent dit que les arts florissaient inévitablement sous un prince ami de la somptuosité, et que lorsque le luxe était recherché, les faiseurs de chefs-d’œuvre se montraient de toutes parts, encouragés par la perspective des hommages délicats et des gros salaires. Cependant voilà que les monarques de tant de régions, et qui avaient de quoi payer les plus fières renommées, ne purent établir autour d’eux que de bien faibles échantillons du génie artistique de leurs sujets N’eussent-ils pas eu de dispositions personnelles à concevoir le beau, puisqu’on copiait pour eux les chefs-d’œuvre des dynasties précédentes, et qu’eux-mêmes construisaient sur tous les points de leurs vastes possessions d’immenses édifices de toute nature, ils donnaient aux artistes, si les artistes avaient existé, toutes les occasions désirables de se signaler et de lutter de génie avec les générations éteintes. Pourtant rien ne jaillit des doigts de la Minerve. La monarchie perse fut opulente, rien de plus, et elle eut recours, en bien des occasions, à la décadence égyptienne pour obtenir chez elle des travaux d’une valeur secondaire sans doute, mais qui dépassaient pourtant les facultés de ses nationaux.\par
Essayons de trouver la clef de ce problème. Nous avons déjà vu que la nation ariane, portée au positif des faits et non pas au désordonné de l’imagination, n’est pas artiste en elle-même. Réfléchie, raisonnante, raisonneuse et raisonnable, elle l’est ; compréhensive au plus haut point, elle l’est encore ; habile à découvrir les avantages de toutes choses, même de ce qui lui est le plus étranger, oui, il faut aussi lui reconnaître cette prérogative, une des plus fécondes de son droit souverain. Mais quand la race ariane est pure de tout mélange avec le sang des noirs, pas de conception artistique pour elle : c’est ce que j’ai exposé ailleurs surabondamment. J’ai montré le noyau de cette famille composé des futures sociétés hindoues, grecques, iraniennes, sarmates, très inhabile à créer des représentations figurées d’un mérite réel, et, quelque grandes que soient les ruines des bords du Iénisséï et des croupes de l’Altaï, on n’y découvre aucun indice révélateur d’un sentiment délicat des arts. Si donc, en Égypte et en Assyrie, il y eut un puissant développement dans la reproduction matérialisée de la pensée, si, dans l’Inde, cette même aptitude ne manqua pas d’éclore, bien que plus tardivement, le fait ne s’explique que par l’action du mélange noir, abondant et sans frein en Assyrie, limité en Égypte, plus restreint sur le sol hindou, et créant ainsi les trois modes de manifestation de ces différents pays. Dans le premier, l’art atteignit promptement son apogée, puis il dégénéra non moins promptement, en tombant dans les monstruosités où la prédominance mélanienne trop hâtive le jeta. Avec le second, comme les éléments arians, sources de la vie et de la civilisation locales, étaient faibles, numériquement parlant, il fut promptement gagné aussi par l’infusion noire. Toutefois, il se défendit au moyen d’une séparation relative des castes, et le sentiment artistique, que le premier flux avait développé, resta stationnaire, cessa promptement de progresser, et ainsi put mettre beaucoup plus de temps qu’en Assyrie à s’avilir. Dans l’Inde, comme une barrière bien autrement forte et solide fut opposée aux invasions du principe nègre, le caractère artistique ne se développa que très lentement et pauvre­ment au sein du brahmanisme Il lui fallut attendre, pour devenir vraiment fort, la venue de Sakya-mouni : aussitôt que les bouddhistes, en appelant les tribus impures au partage du nirwana, leur eurent ouvert l’accès de quelques familles blanches, la passion des arts se développa à Salsette avec non moins d’énergie qu’à Ninive, atteignit promptement, comme là encore, son zénith, et, toujours pour la même cause, s’abîma presque subitement dans les folies que l’exagération, la prédominance du principe mélanien, amenèrent sur les bords du Gange comme partout ailleurs.\par
Lorsque les Iraniens prirent le gouvernement de l’Asie, ils se virent en présence de populations où les arts étaient complètement envahis et dégradés par l’influence noire. Eux-mêmes n’avaient pas toutes les facultés qu’il aurait fallu pour relever ce génie en décomposition,\par
On objectera que, précisément parce qu’ils étaient arians, ils rapportaient au sang corrompu des Sémites l’appoint blanc destiné à le régénérer et qu’ainsi, par une nouvelle infusion d’éléments supérieurs, ils devaient ramener le gros des nations assyriennes vers un équilibre de principes ethniques comparable à celui où s’étaient trouvés les Chamites noirs dans leur plus beau moment, ou, mieux encore, les Chaldéens de Sémiramis.\par
Mais les nations assyriennes étaient bien grandes et la population des tribus iraniennes dominatrices bien petite. Ce que ces tribus possédaient, dans leurs veines, d’essence féconde, déjà entamé, du reste, pouvait bien se perdre au milieu des masses asiatiques, mais non les relever, et, d’après ce fait incontestable, leur puissance même, leur prépondérance politique ne devait durer que le temps assez court où il leur serait possible de maintenir intacte une existence nationale isolée.\par
 J’ai parlé déjà de leur nombre restreint, et je recours là-dessus à l’autorité d’Hérodote. Lorsque l’historien trace, dans son VII\textsuperscript{e} livre, cet admirable tableau de l’armée de Xerxès traversant l’Hellespont, il déploie le magnifique dénombrement des nations appelées en armes par le grand roi, de toutes les parties de ses vastes États. Il nous montre des Perses ou des Mèdes commandant aux troupeaux de combattants qui passent les deux ponts du Bosphore en pliant sous les coups de fouet de leurs chefs iraniens. À part ces chefs de noble essence, gourmandant les esclaves que la victoire enchaînait sous leurs ordres, combien Hérodote énumère-t-il de soldats parmi les Mèdes proprement dits ? Combien de guerriers zoroastriens dans cette levée de boucliers que le fils de Darius avait voulu rendre si formidable ? Je n’en aperçois que 24 000, et qu’était-ce qu’un tel faisceau dans une armée de dix-sept cent mille hommes ? Au point de vue du nombre, rien ; à celui du mérite militaire, tout : car, si ces 24 000 Iraniens n’avaient pas été paralysés, dans leurs mouvements, par la cohue de leurs inertes auxiliaires, il est bien probable que la muse de Platée aurait célébré d’autres vainqueurs. Quoi qu’il en soit, puisque la nation régnante ne pouvait fournir des soldats en plus grande quantité, elle était peu considérable et ne pouvait suffire à la tâche de régénérer la masse épaisse des populations asiatiques. Elle n’avait donc que la perspective d’un seul avenir : se corrompre elle-même en s’engloutissant bientôt dans leur sein.\par
On ne découvre pas trace d’institutions fortes, destinées à créer une barrière entre les Iraniens et leurs sujets. La religion en aurait pu servir, si les mages n’avaient été animés de cet esprit de prosélytisme particulier à toutes les religions dogmatiques, et qui leur valut, bien des siècles après, la haine toute spéciale des musulmans. Ils vou­lurent convertir leurs sujets assyriens. Ils parvinrent à les arracher, en grande partie, aux atrocités religieuses des anciens cultes. Ce fut un succès presque regrettable : il ne fut bon ni pour les initiateurs ni pour les néophytes. Ceux-ci ne manquèrent pas de souiller le sang iranien par leur alliance, et quant à la religion meilleure qu’on leur don­nait, ils la pervertirent, afin de l’accommoder à leur incurable esprit de superstition \footnote{Burnouf, \emph{Commentaire sur le Yaçna}, t. I, p. 351 ‑ Ce savant, en citant le passage d’Hérodote sur lequel se base cette opinion, élève quelques doutes quant à sa portée. Je me bornerai à transcrire ici l’assertion de l’historien grec ; elle suffit entièrement à mon but : « \emph{Clio}, « CXXXI : Voici les coutumes qu’observent, à ma connaissance, les Perses. Leur usage « n’est pas d’élever aux dieux des statues, des temples, des autels. Ils traitent, au contraire, « d’insensés ceux qui le font. C’est, à mon avis, parce qu’ils ne croient pas, comme les « Grecs, que les dieux aient une forme humaine. Ils ont coutume de sacrifier à Jupiter sur le « sommet des plus hautes montagnes, et donnent le nom de Jupiter à toute la circonférence « du ciel. Ils font encore des sacrifices au soleil, à la lune, à la terre, au feu, à l’eau et aux « vents, et n’en offrent de tout temps qu’à ces divinités. \emph{Mais ils ont joint, dans la suite, le} « \emph{culte de Vénus Céleste ou Uranie, qu’ils ont emprunté des Assyriens et des Arabes.} Les « Assyriens donnent à Vénus le nom de \emph{Mylitta}, les Arabes celui d’\emph{Alitta}, et les Perses « l’appellent « \emph{Mitra} ».. Ainsi ce culte de Mithra, qui infecta plus tard tout l’occident romain, commença par saisir les Perses. C’est, en quelque sorte, le cachet de l’invasion du sang sémitique. ‑ Bœttiger dit que, sous le règne de Darius Ochus, le magisme s’était déjà très rapproché de l’hellénisme et du fétichisme par l’adoption du culte d’Anaïtis. (\emph{Ideen zur Kunstmythologie}, t. I, p. 27.)}.\par
La fin des nations iraniennes était ainsi marquée bien près du jour de leur triomphe. Toutefois, tant que leur essence n’était pas encore trop mélangée, leur supériorité sur l’univers civilisé était certaine et incontestable : ils n’avaient pas de compétiteurs. L’Asie inférieure entière se soumit à leur sceptre. Les petits royaumes d’au-delà de l’Euphrate, ce rempart soigneusement entretenu par les Pharaons, furent rapidement englobés dans les satrapies. Les villes libres de la côte phénicienne s’annexèrent à la monarchie perse, avec les États des Lydiens. Un jour vint où il ne resta que l’Égypte elle-même, antique rivale qui, pour les héritiers des dynastes chaldéens, put valoir la peine d’une campagne \footnote{On a vu ailleurs les Égyptiens se défendre, ou même quelquefois attaquer, quand il le fallait absolument, au moyen de leurs troupes mercenaires. Des Grecs en faisaient le nerf. (Wilkinson, \emph{Customs and Manners}, etc., t. I, p. 211.)}. C’était devant ce colosse vieilli que les conquérants sémites les plus vigoureux avaient constamment reculé.\par
Les Perses ne reculèrent pas. Tout favorisait leur domination. La décadence égyptienne était achevée. Le pays du Nil ne possédait plus de ressources personnelles de résistance. Il payait encore, à la vérité, des mercenaires pour faire la garde autour de sa caducité, et, par parenthèse, la dégénération générale de la race sémitique l’avait contraint de remplacer, presque absolument, les Cariens et les Philistins par des Arians Grecs. Là se bornait ce qu’il pouvait tenter. Il n’avait plus assez de souplesse ni de nerfs pour courir lui-même aux armes, et, battu, se relever d’une défaite \footnote{C’était le goût du gouvernement pour les auxiliaires étrangers qui avait déterminé l’émigration de l’armée nationale en Éthiopie. En 362-340, Nectanébo II envoya au secours des Chananéens, révoltés contre les Perses, Mentor le Rhodien avec 4,000 Grecs. Ce condottiere le trahit. (Wilkinson, \emph{Customs and Manners of the ancient Egyptians}, t. I, p. 211.)}.\par
Les Perses l’asservirent et insultèrent, de leur mieux, à cœur joie, à son culte, à ses lois et à ses mœurs.\par
Si l’on considère avec quelque attention le tableau si vivant qu’Hérodote a tracé de cette époque, on est frappé de voir que deux nations traitaient le reste de l’univers, soit vaincu, soit à vaincre, avec un égal mépris, et ces deux nations, qui sont les Perses et les Grecs, se considéraient aussi, l’une l’autre, comme barbares, oubliant à demi, à demi négligeant leur communauté d’origine. Il me semble que le point de vue où elles se plaçaient, pour juger si sévèrement les autres peuples, était à peu près le même. Ce qu’elles leur reprochaient, c’était également de manquer du sens de la liberté, d’être faibles devant le malheur, amollies dans la prospérité, lâches dans le combat ; et ni les Grecs ni les Perses ne tenaient beaucoup de compte aux Assyriens, aux Égyptiens, du passé glorieux qui avait abouti à tant de débilités répugnantes. C’est que les deux groupes méprisants se trouvaient alors à un niveau pareil de civilisation. Bien que séparés déjà par les immixtions qui avaient modifié leurs essences respectives, et, partant, leurs aptitudes, état dont leurs langues rendent témoignage, le commun principe arian qui, chez eux, dominait encore sur les alliages, suffisait à leur faire envisager d’une façon analogue les principales questions de la vie sociale. C’est pourquoi les pages du vieillard d’Halicarnasse représentent si vivement cette similitude de notions et de sentiments dont ils témoignaient. C’étaient comme deux frères de fortune différente, différents par le rang social, frères pourtant par le caractère et les tendances. Le peuple arian-iranien tenait dans l’Occident la place d’aîné de la famille : il dominait le monde. Le peuple grec était le cadet, réservé à porter un jour le sceptre, et se préparant à cette grande destinée par une sorte d’isonomie vis-à-vis de la branche régnante, isonomie qui n’était pas tout à fait de l’indépendance. Quant aux autres populations renfermées sous l’horizon des deux rameaux arians, elles demeuraient, pour le premier, objets de conquête et de domination, pour le second, matière à exploiter. Il est bon de ne pas perdre de vue ce parallélisme, sans quoi l’on comprendrait peu les déplacements du pouvoir arrivés plus tard.\par
Certainement, je conçois qu’on se mette de la partie dans le dédain ordinaire aux esprits vigoureux et positifs pour les natures artistes plutôt vouées à recueillir des apparences qu’à saisir des réalités. Il ne faut cependant pas oublier non plus que, si les Perses et les Grecs avaient tout sujet de mésestimer le monde sémitique, devenu leur pâture, ce monde possédait le trésor entier des civilisations, des expériences de l’Occident, et les souvenirs respectables de longs siècles de travaux, de conquêtes et de gloire. Les compagnons de Cyrus, les concitoyens de Pisistrate avaient en eux-mêmes j’en conviens, les gages d’une future rénovation de l’existence sociale ; mais ce n’était pas là une raison pour qu’on dût perdre ce que les Chamites noirs et les différentes couches de Sémites et les Égyptiens avaient de leur côté amassé de résultats. La moisson des deux groupes arians occidentaux, la moisson provenant de leur propre fonds, était encore à faire : les blés n’en étaient qu’en herbe, les épis pas encore mûrs ; tandis que les gerbes des nations sémitiques remplissaient les granges et approvision­naient les prochains réformateurs eux-mêmes. Il y a plus : les idées de l’Assyrie et de l’Égypte s’étaient répandues partout où le sang de leurs inventeurs avait pénétré, en Éthiopie, en Arabie, sur le pourtour de la Méditerranée, comme dans l’ouest de l’Asie, comme dans la Grèce méridionale, avec une opulence, une exubérance désespérante pour les civilisations encore à naître, et toutes les créations des sociétés postérieures allaient être à jamais contraintes de transiger avec ces notions et les opinions qui en ressortaient. Ainsi, malgré leur dédain pour les nations sémitiques et pour la paix efféminée des bords du Nil, les Arians Iraniens et les Arians Grecs devaient bientôt entrer dans le grand courant intellectuel de ces populations flétries par leur désordre ethnique et par l’exagération de leurs principes mélaniens. La part d’influence laissée à ces Iraniens si orgueilleux, à ces Grecs si actifs, se réduirait ainsi, en fin de compte, à jeter dans le lac immense et stagnant des multitudes asiatiques quelques éléments temporaires de mouvement, d’agitation et de vie.\par
Les Arians Iraniens, et après eux, les Arians Grecs, offrirent au monde d’Assyrie et d’Égypte ce que les Arians Germains donnèrent plus tard à la société romaine.\par
Quand l’Asie occidentale fut tout entière ralliée sous la main des Perses, il n’y eut plus de raison pour que la scission primitive entre sa civilisation et celle de l’Égypte subsistât. Le peu d’efforts tenté dans la vallée du Nil afin de reconquérir l’indépendance nationale ne compta plus que comme les convulsions d’une résistance expirante. Les deux sociétés primitives de l’Occident tendaient à se confondre, parce que les races qu’elles enfermaient ne se distinguaient plus assez nettement. Si les Perses avaient été très nombreux, si, à la manière des plus antiques envahisseurs, leurs tribus avaient pu lutter contre le chiffre des multitudes sémitiques, il n’en aurait pas été ainsi. Une organisation toute nouvelle se formant sur les débris méconnus des anciennes, on aurait vu quelques-uns de ces débris s’isoler, dans des extrémités de l’empire, avec des restes de la race, et se constituer à part, de manière à maintenir entre les inventions des nouveaux venus et l’état de choses aboli, pour la majorité des sujets, une ligne de démarcation perceptible.\par
Les Iraniens, n’étant qu’une poignée d’hommes, furent à peine en possession du pouvoir, que l’immense esprit assyrien les entoura de toutes parts, les saisit, les serra, et leur communiqua son vertige. On peut déjà se rendre compte sous le fils de Cyrus, sous Cambyse, de la part de parenté que la nature fatalement superbe et enflée des Sémites chamitisés pouvait déjà réclamer avec la personne du souverain. Heureuse­ment, cet alliage ne s’était pas encore généralisé. Le témoignage d’Hérodote vient nous prouver que l’esprit arian tenait bon contre les assauts de l’ennemi domestique. Rien ne le montre mieux que la fameuse conférence des sept chefs après la mort du faux Smerdis \footnote{Hérodote, \emph{Thalie}, LXXX et passim.}.\par
Il s’agissait de donner aux peuples délivrés une forme de gouvernement convenable. Le problème n’eût pas existé pour le génie assyrien qui, du premier mot, aurait proclamé l’éternelle légitimité du despotisme pur et simple ; mais il fut envisagé mûre­ment et résolu, non sans difficulté, par les guerriers dominateurs qui le soulevèrent. Trois opinions se trouvèrent en présence. Otanès opina pour la démocratie ; Mégabyzès parla en faveur de l’oligarchie. Darius, ayant loué l’organisation monarchique, qu’il affirma être la fin inévitable de toutes les formes de gouvernement possibles, gagna les suffrages à sa cause. Cependant il avait affaire à des associés tellement fous d’indépendance, qu’avant de remettre le pouvoir au roi élu, ils stipulèrent qu’Otanès et toute sa maison resteraient à jamais affranchis de l’action de l’autorité souveraine, et libres, sauf le respect des lois. Comme à l’époque d’Hérodote des sentiments de cette énergie n’existaient plus guère parmi les Perses, décidément déchus de leur primitive valeur ariane, l’écrivain d’Ionie prévient sagement ses lecteurs que le fait qu’il raconte va leur paraître étrange : il ne l’en maintient pas moins \footnote{Hérodote, \emph{Thalie}, LXXX.}.\par
Après l’extinction de cette grande fierté, il y eut encore quelques années illustres ; ensuite le désordre sémitique réussit à englober les Iraniens dans le sein croupissant des populations esclaves. Dès le règne du fils de Xerxès, il devient évident que les Perses ont perdu la force de rester les maîtres du monde, et, cependant, entre la prise de Ninive par les Mèdes et cette époque d’affaiblissement, il ne s’était encore écoulé qu’un siècle et demi.\par
L’histoire de la Grèce commence ici à se mêler plus intimement à celle du monde assyrien. Les Athéniens et les Spartiates se rencontrent désormais dans les affaires des colonies ioniennes. Je vais donc quitter le groupe iranien. pour m’occuper du nouveau peuple arian, qui s’annonce comme son plus digne et même son seul antagoniste.
\section[{IV.3. Les Grecs autochtones  ; les colons sémites ; les Arians Hellènes.}]{IV.3. \\
Les Grecs autochtones  ; les colons sémites ; les Arians Hellènes.}
\noindent La Grèce primordiale se présente moitié sémitique, moitié aborigène \footnote{Quelques mots sur ces aborigènes que les temps historiques ont à peine entrevus. Tous les souvenirs primitifs de l’Hellade sont remplis d’allusions à ces tribus mystérieuses. Hésiode appelle autochtones les plus anciennes populations de l’Arcadie, qualifiées de pélasgiques. Érechthée, Cécrops, étaient des chefs reconnus pour autochtones. Il en était de même des nations suivantes : la généralité des Pélasges, des Lélèges, les Kurètes, les Kaukons, les Aones, les Temmikes, les Hyantes, les Béotiens thraces, les Télèbes, les Éphyres, les Phlégyens, etc. (Voir Grote, \emph{History of Greece}, t. I, p. 238, 262, 268, et t. II, p. 349 ; Larcher, \emph{Chronol. d’Hérod}., t. VIII ; Niebuhr, \emph{Rœmische Geschichte}, t. I, p. 26 à 64 ; O. Müller, \emph{die Etrusker, Einleit}., p. 11 et 75 à 100.) ‑ Sur la rapidité avec laquelle les populations aborigènes disparurent aussitôt que les Arians Hellènes eurent paru au milieu d’elles, consulter Grote, t. II, p. 351. ‑ Hécatée, Hérodote et Thucydide sont d’accord sur ce point, qu’il y a eu une époque antéhellénique où différents langages étaient parlés entre le cap Malée et l’Olympe. (Grote, t. II, p. 317.) ‑ Dès l’an 771 avant J.-C., on ne trouve plus trace d’établissements non mêlés d’Arians Hellènes dans l’Hellade entière. ‑ Pour ce qui est de la nature ethnique des aborigènes, je suis obligé de renvoyer le lecteur au livre suivant, qui traite des populations absolument primitives de l’Europe.}. Ce sont des Sémites qui fondent le royaume de Sicyone, premier point civilisé du pays, ce sont des dynasties purement sémitiques ou autochtones que glorifient les noms caractéristiques d’Inachus, de Phoronée, d’Ogygès, d’Agénor, de Danaüs, de Codrus, de Cécrops, noms dont les légendes établissent la signification ethnique de la manière la plus claire. Tout ce qui ne vient pas d’Asie, à ces époques lointaines, se dit né sur le sol même, et forme la base populaire des États nouvellement éclos. Mais le fait remarquable, c’est que, aux âges primordiaux, on n’aperçoit nulle part la moindre trace historique des Arians Hellènes.\par
 Aucun récit mythique ne fait mention d’eux. Ils sont profondément inconnus dans toute la Grèce continentale, dans les îles à plus forte raison. Pour les rencontrer, il faut descendre jusqu’aux jours de Deucalion, qui, avec des troupes de Lélèges et de Curètes, c’est-à-dire avec des populations locales, par conséquent non arianes, vint, bien longtemps après la création des États de Sicyone, d’Argos, de Thèbes et d’Athènes, s’établir dans la Thessalie. Ce conquérant arrivait du nord.\par
Ainsi, depuis la fondation de Sicyone, placée par les chronologistes, comme Larcher, à l’an 2164 avant notre ère, jusqu’à l’arrivée de Deucalion en 1541, autrement dit pendant une période de six cents ans, on n’aperçoit en Grèce que des peuples antéarians aborigènes et des colonisateurs de race chamo-sémitique.\par
Où vivaient donc, que faisaient les Arians Hellènes pendant cette période de six cents ans ? Étaient-ils vraiment bien loin encore de leur future patrie ? La tradition les ignore d’une façon si complète, que l’on serait tenté de croire qu’ils ont exécuté leur apparition première avec Deucalion, brusquement, inopinément, et que, avant cette surprise, on n’avait jamais entendu parler d’eux. Puis soudain Deucalion, établi sur les terres de conquête, donne le jour à Hellen ; celui-ci a pour fils Dorus, Æolus, Xuthus, qui, à son tour, devient père d’Achæus et d’Ion : toutes les branches de la race, Doriens, Æoliens, Achéens et Ioniens, entrent en compétition des territoires jadis exclusivement acquis aux autochtones et aux Chananéens. Les Arians Hellènes sont trouvés.\par
Il ne faut pas s’étonner de ce défaut de précédents et de transition. Ce sont là les formes mnémoniques ordinaires des récits que conservent les peuples sur leurs origines. Cependant il n’y a pas le moindre doute que les invasions et les établisse­ments des multitudes blanches ne s’accomplissent point ainsi. Une nation menace longtemps un territoire avant de pouvoir s’y établir. Elle tourne autour des frontières du pays convoité sans les franchir. Elle épouvante d’abord et ne saisit que tardivement. Les Arians Hellènes n’ont pas procédé autrement que leurs frères : ils n’ont pas fait exception à la règle.\par
Puisque avant l’établissement de Deucalion en Thessalie il n’est pas question du nom de son peuple, cessons de rechercher ce nom et, nous attachant à d’autres ressour­ces, voyons ce qu’était Deucalion. Lui-même, bien reconnu comme Hellène, par les siècles postérieurs, puisqu’il est proclamé l’éponyme même de la race. Observons-le dans sa valeur ethnique, et d’abord, puisque nous procédons de bas en haut, commençons par préciser celle de ses fils, fondateurs des différentes tribus helléniques \footnote{Les noms des différents personnages de la généalogie ariane-hellénique, évidemment symboliques, sont plutôt des qualifications représentant le trait principal, résumant l’histoire de la vie de chacun de ces éponymes ; il en est constamment ainsi, chez toutes les nations, quant à ces êtres génésiaques. Ainsi Deucalion, non seulement l’auteur de la race hellénique, mais le patriarche qui concentre sur sa tête le résumé des antiques souvenirs cosmogoniques, le témoin du déluge (dans la tradition sémitique-grecque, Ogygès remplit ce rôle), Deucalion, qui répond au dieu-poisson, au Nô des Assyriens, au Noah hébraïque, est nommé ainsi du mot ancien (mot grec) (inusité), \emph{vin nouveau}, et à (mot grec) vieille \emph{forme d’}(mot grec) \emph{se rouler, l’homme qui se roule} (dans l’ivresse du) \emph{vin nouveau}. ‑ Le nom de (\emph{mot grec}) qui contient le sens de \emph{rouge}, ne présente pas une explication aussi nette. ‑\emph{ Pandore}, (mot grec) \emph{celle à qui on a tout donné}, est bien, en effet, un produit sans individualité propre ; c’est la femme qui appartient à celui qui l’a créée, ou civilisée.}.\par
Ils naquirent tous, au second degré, de Deucalion et de Pyrrha, fille de Pandore. Dorus commença par établir ses tribus autour de l’Olympe, près du Parnasse. Æolus régna dans la Thessalie, chez les Magnètes. Xuthus s’avança jusqu’au Péloponèse. Hellen, père de ces trois héros, les avait eus d’une fille dont l’origine autochtone est suffisamment indiquée par son nom : la légende l’appelle Orséis, \emph{la montagnarde}. Pandore également n’était pas née de la souche hellénique. Formée de limon, elle se trouvait être d’une autre espèce que les Arians : elle était autochtone, elle avait épousé le frère de son créateur. Ainsi, les patriarches de la famille hellénique ne se présentent pas comme étant de race pure. Quant à Pandore, cette femme aborigène mariée à un étranger ; quant à sa fille Pyrrha, mariée à un autre étranger ; quant à ce dernier couple qui, après le déluge, se fabrique un peuple avec les pierres du sol, il est difficile de ne pas se rappeler, en les observant, le mythe tout semblable de l’histoire chinoise, où Pan-Kou forme les premiers hommes avec de la glaise, bien qu’il soit homme lui-même. La pensée ariane-grecque et ariane-chinoise n’a trouvé, à des distances immenses, que le même mode de manifestation pour représenter deux idées complètement identiques, le mélange d’un rameau arian avec des aborigènes sauvages et l’appropriation de ces derniers aux notions sociales.\par
Deucalion, le premier des Grecs, à savoir, le premier d’une race mêlée, un demi-Sémite, à ce qu’il semble, était fils de Prométhée et de Klymène, issue de l’Océan \footnote{(Mot grec) \emph{le prévoyant}. Il est fils de Japet, le père commun de la famille blanche, au dire d’Hésiode et d’Apollonius, Sa mère était Asia. C’est la déclaration bien claire et de sa valeur ethnique et de son premier séjour. On donne encore une autre souche que j’accepterais également. Il serait, suivant quelques commentateurs, fils d’Ouranos. Je m’explique plus bas à ce sujet.}. On sent très bien ici la déviation de la source pure, d’où Prométhée était issu. Si Deucalion devient éponyme de ses descendants, c’est qu’il n’a pas la même composition, la même signification ethnique que son père. Rien de plus évident. Cependant les apports de sang sémitique ou aborigène ne peuvent constituer son originalité : c’est bien dès lors dans la ligne paternelle qu’il faut la chercher, sans quoi Deucalion ne serait nullement considéré par la légende hellénique comme l’homme type, et, dans les récits grecs d’origine sémitique, il serait classé bien après les héros chananéens qui l’ont, en effet, précédé suivant l’ordre des temps. Deucalion tire donc tout son mérite spécial de son père, et ainsi c’est la race de celui-ci qu’il importe de reconnaître. Or, Prométhée était un Titan, ainsi que son frère Épiméthée, d’où les Arians Hellènes descendent également par les femmes. En conséquence, personne, je crois, ne pourra combattre cette conclusion : les Arians Hellènes avant Deucalion, les Arians Hellènes encore à peu près intacts de tous mélanges soit sémitiques, soit aborigènes, ce sont les Titans \footnote{Hésiode dérive le mot (grec), de (mots grecs) \emph{ceux qui étendent les mains}. On donna à cette signification la portée de (mot grec) et on fit de ceux à qui on l’avait attribuée les \emph{Rois} par excellence. De même les Arians zoroastriens appelaient leurs ancêtres, probablement contemporains et frères des Titans, \emph{Kaï}, ou \emph{Kava}, les \emph{Rois}, Le Pseudo-Orphée et Diodore représentent les Titans comme les premiers des humains, les hommes types. (Diodore, III, 57 ; v, 66.) ‑ Le dialecte thessalien avait conservé fidèlement la trace de l’idée ancienne, et (mot grec)y désignait \emph{le seigneur, le chef}. (Voir Bœttiger, \emph{Ideen zur Kunstmythotogie} (Dresde, in-8°, 1826), t. II, p. 47 et passim.)}. La régularité de la filiation ne laisse rien à désirer.\par
Jusque-là, il est établi d’une manière irréfragable que les Grecs sont des descendants métis de cette nation glorieuse et terrible. Pourtant on pourrait douter encore que les Titans aient été, eux-mêmes, ces Hellènes, séparés jadis de la famille ariane sur les versants de l’Imaüs, et dont nous avons senti, plutôt que vu, la longue pérégrination dans les montagnes du nord de l’Assyrie, au long de la mer Caspienne. À la vérité, si la généalogie ascendante des Titans était complètement perdue, le fait n’en serait pas moins établi, avec toute la certitude possible, par la philologie et les arguments physiologiques : mais, puisque l’histoire est ici d’une clarté et d’une précision trop rares, je ne repousserai certes pas le secours qu’elle m’apporte, et je compléterai ma démonstration.\par
Les Titans étaient les fils directs de cet ancien dieu arian, déjà aperçu par nous dans l’Inde, aux origines védiques, de ce Varounas, expression vénérable de la piété des auteurs de la race blanche, et dont les Hellènes n’avaient même pas défiguré le nom en le conservant, après tant de siècles, sous la forme à peine altérée d’Ouranos. Les Titans, fils d’Ouranos, le dieu originel des Arians, étaient bien incontestablement eux-mêmes, on le voit, les Arians, et parlaient une langue dont les restes, survivant au sein des dialectes helléniques, se rapprochaient, sans nul doute, d’une façon très intime, et du sanscrit, et du zend, et du celtique, et du slave le plus ancien.\par
Les Titans, ces conquérants altiers des contrées montagneuses du nord de la Grèce, ces hommes violents et irrésistibles, laissèrent dans la mémoire des populations de l’Hellade, et, par contre-coup, dans celle de leurs propres descendants, exactement cette même idée de leur nature que les antiques Chamites blancs, que les premiers Hindous, que les Arians égyptiens, que les Arians chinois, tous conquérants, tous leurs parents, ont laissée dans le souvenir des autres peuples \footnote{Il est très vraisemblable qu’on peut considérer comme un monument de la législation titanique ces prescriptions de Busygès, qui, dit-on, furent la souche du code de Dracon. Trois commandements en formaient tout l’ensemble conservé à travers les siècles : « Honore tes parents ; offre aux dieux les prémices de la terre ; ne fais pas de mal au taureau. » C’est évidemment là toute la loi hindoue et zoroastrienne, c’est le pur esprit arian. ‑ On sait que les Grecs ne purent se défaire qu’avec peine du respect traditionnel pour le bœuf. Quand ils se laissèrent aller à sacrifier cet animal, ils imaginèrent, comme palliatif de la mauvaise action qu’ils commettaient, la cérémonie de la (mot grec)ou (mot grec), dans laquelle le sacrificateur, après avoir frappé sa victime, s’enfuyait en abandonnant la hache, à qui l’on faisait le procès. (Bœttiger, \emph{Ideen zur Kunstmythologie}, t. II, p. 267.)}. On les divinisa, on les plaça au-dessus de la créature humaine, on s’avoua plus petits qu’eux, et, ainsi que je l’ai dit quelquefois déjà, par une telle façon de comprendre les choses, on rendit exacte justice et aux nations primitives de race blanche pure et aux multitudes de valeur médiocre qui leur ont succédé.\par
Les Titans occupèrent donc le nord de la Grèce. Leur premier mouvement heureux vers le sud fut celui auquel présida Deucalion, menant à cette entreprise des troupes d’aborigènes, c’est-à-dire de gens étrangers à son sang \footnote{Qui d’ailleurs n’étaient point barbares. Elles paraissent avoir eu un degré respectable de culture utilitaire. Ces aborigènes labouraient le sol, prétendaient avoir inventé l’appropriation du bœuf aux travaux agricoles et l’usage du moulin à blé. (Mac Torrens Cullagh, \emph{The Industrial History of free Nations} (London, 1846, in-8°, t. I, p. 7.) ‑ Ce trait, et d’autres encore, qui les identifient aux autochtones d’Italie servira plus tard à démontrer qu’ils ne pouvaient être que des Celtes ou des Slaves, et, peut-être bien, l’un et l’autre.}. Lui-même d’ailleurs, on l’a vu, était un hybride. Ainsi, nous n’avons plus affaire désormais aux Titans. Ils restent, ils se mêlent, ils s’éteignent dans les contrées septentrionales de l’Hellade, dans la Chaonie, l’Épire, la Macédoine : ils disparaissent, mais non sans transmettre et assurer une valeur toute particulière aux populations parmi lesquelles ils se fondent \footnote{De là vont se dégager, avec mille nuances, les Arians Hellènes, peuple nouveau, dans un certain sens, bien que devant son énergie à des éléments anciens atténués. Ce que cette race eut de particulier est bien représenté par sa religion, de même âge que lui. Ce fut le culte de Zeus, dont Heyne, dans une note d’Apollodore, a pu dire avec vérité : « Inde a Jove novus mythorum ordo initium habet vere Hellenicus. » (Bœttiger, t. I, p. 195.)}.\par
Ces populations, non plus que celles de la Thrace et de la Tauride, n’étaient pas, je l’ai indiqué sommairement, de race jaune pure. Déjà les nations celtiques et slaves avaient incontestablement poussé leurs marches jusqu’à l’Euxin, jusqu’aux montagnes de la Grèce, jusqu’à l’Adriatique. Elles étaient même allées beaucoup plus loin. Les grands déplacements de peuples blancs septentrionaux, qui, sous l’effort violent des masses mongoles opérant au nord, avaient déterminé les Arians habitant plus au sud, sur les hauts plateaux asiatiques, à descendre le long des crêtes de l’Hindou-Koh, agissaient, dès longtemps, lorsque les Titans se montrèrent au delà de la Thrace. Les Celtes, que l’on trouve, au dix-septième siècle avant Jésus-Christ, fermement établis dans les Gaules, et les Slaves, que, pour des motifs à donner en leur lieu, j’aperçois en Espagne antérieurement à cette époque, avaient quitté depuis des siècles la patrie sibérienne et longé les bords supérieurs du Pont-Euxin. Pour toutes ces causes, une certaine somme de mélanges subis par les Titans avait apporté dans les veines des Arians Hellènes quelque proportion de principes jaunes dus seulement à l’intermédiaire des nations souillées d’un contact plus intime avec les peuples finnois \footnote{Très vraisemblablement le grec contient des racines thraces et illyriennes provenant du contact très ancien des Arians Hellènes et même des Titans avec les populations parlant ces idiomes. O. Müller remarque avec raison que les Hellènes rapportaient aux Thraces leur poésie et leur civilisation primordiales. Le pays au nord de l’Hémus était, pour les admirateurs d’Orphée, le berceau de la culture morale. (Pott, \emph{Encycl. Ersch u. Gruber}, p. 65.)}.\par
Après l’époque de Deucalion, à dater du seizième siècle avant Jésus-Christ \footnote{On s’aperçoit du premier coup d’œil combien les antiquités les plus lointaines de la Grèce sont humbles en comparaison de ce que l’on observe dans l’Inde, en Assyrie, en Égypte, même en Chine, et de ce que la Bactriane pourrait montrer. Ainsi Sicyone, ne date que de l’an 2164 avant J.-C. C’est une fondation chananéenne, et l’arrivée des Arians Hellènes, de six siècles plus tardive, rejette aux âges de maturité des sociétés primitives l’enfance encore antéhistorique de l’Hellade.}, les tribus fixées dans la Macédoine, l’Épire, l’Acarnanie, l’Étolie, le nord, en un mot, réunirent, à un degré tout particulier, les traits du caractère arian et furent les premières à faire connaître le nom des Hellènes.\par
Là surtout brilla l’esprit belliqueux. Le héros thessalien, le brave aux pieds légers, reste toujours le prototype du courage hellénique. Tel que l’\emph{Iliade} nous le montre, c’est un guerrier véhément, ami du danger, cherchant la lutte pour la lutte, et, dans sa religion de loyauté, ne transigeant pas avec le devoir qu’il s’impose. Ses nobles sentiments le font aimer. Les passions impétueuses qui le perdent le font plaindre. Il est digne d’être comparé aux vainqueurs de l’épopée hindoue, du Schahnameh et des chansons de geste.\par
L’énergie était le trait de cette famille. Cette vertu, quand l’intelligence l’éclaire et la conduit, est partout désignée d’avance pour le souverain pouvoir. Le nord de la Grèce fournit toujours au midi ses soldats les meilleurs, les plus intrépides, les plus nom­breux, et longtemps après que le reste du pays était étouffé sous l’élément sémitique, il s’entretenait encore dans cette région des pépinières de hardis combattants. D’autre part, il faut l’avouer, les habitants de ces contrées, si habiles à se battre, à commander, à organiser, à gouverner, ne le furent jamais à briller dans les travaux spéculatifs. Chez eux, pas d’artistes, pas de sculpteurs, de peintres, d’orateurs, de poètes, ni d’historiens célèbres. C’est tout ce que put faire le génie lyrique que de remonter du sud jusqu’à Thèbes pour y produire Pindare. Il n’alla pas au delà, parce que la race ne s’y prêtait pas, et Pindare lui-même fut une grande exception dans la Béotie. On sait ce qu’Athènes pensait de l’esprit cadméen, qui, pour n’avoir pas la langue déliée, ni la pensée fleurie, n’en suscitait pas moins des soldats mercenaires à toute l’Asie et, à l’occasion, un grand homme d’État à la patrie hellénique. Le sang de la Grèce septen­trionale avait à Thèbes sa frontière \footnote{Thèbes remplissait parfaitement l’emploi de limite entre deux races. Elle affichait sa double origine en racontant sur sa fondation deux légendes : l’une ariane, qui attribuait le fait à Amphion et à Zéthus ; l’autre sémitique, et par laquelle le Chananéen Cadmus était son premier roi. (Grote, \emph{History of Greece}, t. I, p. 350).) ‑ Ce sont ces mélanges de traditions asiatiques, helléniques-arianes et aborigènes qui ont rendu longtemps l’histoire primitive et la mythologie grecques presque incompréhensibles. Les époques savantes ont augmenté le désordre par la manie du symbolisme, de l’allégorie, et par les évhémérismes de toute espèce. Puis sont venus les modernes, qui, en généralisant les notions, ont réussi à les rendre absurdes au dernier chef.}.\par
Le nord fut donc toujours distingué par les instincts militaires et même grossiers de ses citoyens, et par leur génie pratique, double caractère dû incontestablement à un hymen de l’essence blanche ariane avec des principes jaunes. Il en résultait de grandes aptitudes utilitaires et peu d’imagination sensuelle. Nous apercevons ainsi, dans les parties de l’Europe les plus anciennement au pouvoir des Hellènes, l’antithèse ethnique et morale de ce que nous avons observé dans l’Inde, en Perse et en Égypte. Nous allons faire de même l’application de ce contraste aux nations de la Grèce méridionale. La différence sera plus saillante à mesure que nous passerons du continent dans les îles et des îles dans les colonies asiatiques.\par
Je me suis servi, il n’y a qu’un instant, de l’\emph{Iliade} pour caractériser le génie tout à la fois arian et finnique des Grecs du nord. Je n’y puise pas de moindres secours lorsque je cherche à me représenter l’esprit arian-sémitique des Grecs du sud, et il me suffira, dans ce but, d’opposer à Achille et à Pyrrhus le sage Ulysse. Voilà bien le type du Grec trempé de phénicien ; voilà l’homme qui nommerait certainement, dans sa généa­logie, plus de mères chananéennes que de femmes arianes. Courageux, mais seulement quand il le faut, astucieux par préférence, sa langue est dorée, et tout imprudent qui l’écoute plaider est séduit. Nul mensonge ne l’effraie, nulle fourberie ne l’embarrasse, aucune perfidie ne lui coûte. Il sait tout. Sa facilité de compréhension est étonnante, et sans bornes sa ténacité dans ses projets. Sous ce double rapport, il est Arian.\par
Poursuivons le portrait.\par
Le sang sémitique parle de nouveau en lui, quand il se montre sculpteur lui-même il a taillé son lit nuptial dans un olivier, et cet ouvrage incrusté d’ivoire est un chef-d’œuvre. Ainsi éloquent, artiste, fourbe et dangereux c’est un compatriote, un émule du pirate-marchand né à Sidon, du sénateur qui gouvernera Carthage, tandis qu’ingénieux à trouver des idées, inébranlable dans ses vues, habile à gouverner ses passions autant qu’à tempérer celles des autres, modéré quand il le veut, modeste parce que l’orgueil est une enflure maladroite de la raison, c’est un Arian. Il n’y a pas de doute qu’Ulysse doit l’emporter sur Ajax, véritable Arian Finnois. La nuance du type grec à laquelle appartient le fils de Laërte est destinée à une plus haute, plus rapide, mais aussi plus fragile fortune, que son opposite. La gloire de la Grèce fut l’œuvre de la fraction ariane, alliée au sang sémitique ; tandis que la grande prépondérance extérieure de ce pays résulta de l’action des populations quelque peu mongolisées du nord.\par
On le sait : de bonne heure, et longtemps avant que les premières tribus des Arians Grecs, provenant du mélange des aborigènes avec les Titans, fussent descendues dans l’Attique et le Péloponèse, des émigrants chananéens avaient déjà conduit leurs barques vers ces plages. On ne croit plus guère aujourd’hui, et cela pour des raisons irréfra­gables, que parmi ces étrangers se soient trouvés des Égyptiens. Les gens de Misr ne colonisaient pas : ils restaient chez eux, et même, bornés longtemps à la possession du cours supérieur du Nil, ils ne sont descendus qu’assez tard jusqu’aux bords de la mer. La partie inférieure du Delta était occupée par des peuples de race sémitique ou chamitique. C’était le grand chemin des expéditions vers l’Afrique occidentale. Si donc, ce que je n’ai nul motif de contester, certaines bandes, venues pour peupler la Grèce, sont parties de ce point, ce n’étaient pas des Égyptiens : c’étaient des congénères de ces autres envahisseurs qui, de l’aveu commun, sont accourus en grand nombre de Phénicie. Tous les noms des anciens chefs d’États grecs primitifs, qui ne présentent pas une apparence aborigène, sont uniquement sémitiques : ainsi Inachus, Azéus, Phégée, Niobé, Agénor, Cadmus, Codrus. On cite une exception, deux au plus : Phoronée, que l’on rapproche du Phra égyptien, et Apis. Mais Phoronée est le fils d’Inachus, le frère de Phégée, le père de Niobé. On trouve ce héros, dans sa famille même, entouré de noms clairement sémitiques, et il ne serait pas plus difficile de découvrir au sien une racine de même espèce qu’il ne l’est de l’identifier avec Phra \footnote{L’existence de colonies égyptiennes dans la Grèce primitive compte aujourd’hui beaucoup plus d’adversaires que de partisans. (Voir à ce sujet Pott, \emph{Encycl. Ersch u. Gruber, Indo}-g\emph{ermanischer Sprachstamm}, p. 23, et Grote, \emph{Hist. of Greece}, t. I, p. 32.) ‑ Ce dernier ne pense pas qu’avant le VII\textsuperscript{e} siècle il y ait eu des rapports suivis entre la Grèce et la terre des Pharaons.}.\par
On a rapproché le nom d’Inachus du mot \emph{Anak}, dont M. de Ewald et d’autres hébraïsants ont fait ressortir l’importance ethnique. Si ce nom devait avoir, quant au premier roi de l’Argolide, une signification de race, il indiquerait une parenté avec la tribu honteusement abrutie de ces noirs purs qui maîtres dépossédés du Chanaan, erraient dans les buissons et hantaient les cavernes de Seïr. Mais la vraisemblance n’en est pas grande, et je ne crois pas qu’il faille soit confondre le nom d’Inachus avec le mot Anak, soit, si l’on ne peut éviter ce rapport, y trouver un sens plus profond qu’une pure similitude de syllabes. C’est ainsi que, pour le mot \emph{Kabl}, (mot arabe) fréquent dans la composition des noms arabes, on aurait le plus grand tort de chercher le père de qui le porte parmi les individus de l’espèce canine \footnote{Le chananéen (chananéen) \emph{anak}, qui signifie un homme remarquable par l’élévation de la taille et la longueur du cou, c’est-à-dire un géant ou un homme fort, et de là un \emph{maître} est la véritable racine de ce nom ou plutôt de ce titre d’Inachus, considéré ensuite comme un appellatif, ainsi qu’on a fait de Brennus, de Boiorix, de Vercingétorix et de tant d’autres mots du même genre. Les Grecs sémitisés du sud l’ont fidèlement conservé dans le titre (en grec), donné aux dieux, principalement à Apollon, par Homère, et aux Dioscures, (en grec) puis aux chefs militaires. On peut aussi relever, comme une trace, entre tant d’autres, de l’énorme influence des Sémites sur l’esprit grec que (alphabet étranger), \emph{anér}, désignation que se donnaient les Chananéens, est l’étymologie de (en grec) qui, pour les contemporains de Périclès, voulait dire \emph{un homme, vir}. (Bœttiger, t. I, p. 206.)}.\par
Les colonies venues du sud et de l’est se composaient donc exclusivement de Chamites noirs et de Sémites différemment mélangés. Le degré de civilisation de cha­cune d’elles n’était pas moins nuancé, et les variétés de sang, créées par ces invasions dans les pays grecs, furent infinies.\par
Aucune contrée ne présente, aux époques primitives, plus de traces de convulsions ethniques, de déplacements subits et d’immigrations multipliées. On y venait par troupes de tous les coins de l’horizon, et souvent pour ne faire que passer ou se voir tellement assailli, que force était de se confondre aussitôt parmi les vainqueurs et de perdre son nom. Tandis que, à tout moment, des bandes saturées de noir accouraient soit des îles, soit du continent d’Asie, d’autres populations mêlées d’éléments jaunes, des Slaves, des Celtes, descendaient du nord sous mille dénominations imprégnées d’idées toutes spéciales \footnote{Cet état d’antagonisme ne prit jamais fin. Il continua à être représenté par l’existence d’innombrables dialectes. ‑ Inutile de rappeler que la classification en quatre branches, ionique, dorique, éolique et attique, est une œuvre artificielle des grammairiens et ne reproduit nullement un état de choses dans lequel chaque petite subdivision de territoire avait, à tout le moins, des idiotismes qui lui étaient absolument propres. (Grote, t. I, p. 318.)}. Pour expliquer ce concours de tant de nationalités sur une péninsule étroite et presque séparée du monde, il est besoin de ne jamais perdre de vue quelles perturbations énormes les agitations des peuples finnois amenaient dans les parties septentrionales du continent. Les guerriers arrivés de la Thessalie et de la Macédoine dans les parages de l’Acarnanie avaient été les victimes directes des dépossessions répétées de proche en proche, et, de même, les Chamites noirs et les Sémites venus de l’est et du sud fuyaient devant des événements analogues, et aban­donnaient, pour aller chercher fortune en Grèce, leurs territoires, devenus domaines des invasions hébraïques ou arabes, en un mot, chaldéennes de différentes dates.\par
Ces armées de fugitifs rejetés, le glaive à la main, dans le Péloponèse, l’Attique, l’Argolide, la Béotie, l’Arcadie, s’y heurtaient les unes contre les autres et s’y livraient bataille. Il résultait encore de ces nouveaux conflits de nouveaux vaincus et de nouveaux vainqueurs, des tribus asservies, d’autres chassées, de sorte que, après le combat, des cohues tumultueuses repartaient, soit pour se diriger vers l’ouest et gagner la Sicile, l’Italie, l’Illyrie, soit pour retourner sur la côte asiatique et y chercher une fortune meilleure \footnote{La race de Dardanus et de Teucer, une de celles qui portèrent l’élément arian-hellénique dans la Troade, fut dans ces derniers.}. L’Hellade ressemblait à un de ces abîmes profonds creusés dans le lit des fleuves, où les eaux, pressées par le courant, se précipitent en lourdes masses et ressortent en tourbillons.\par
Pas de repos, pas de trêve. Les temps héroïques sont à peine ouverts, l’épopée balbutie ses plus obscurs récits, et, dédaigneuse des hommes, remarque les dieux seuls, que déjà les expulsions violentes, les dépossessions de tribus entières, les révolutions de toutes sortes ont commencé. Puis, lorsque, mettant pied à terre, la Muse parle enfin de sang-froid et dans des termes que la raison peut discuter, elle nous montre les nations grecques composées à peu près ainsi :\par
1° Des Hellènes. – Arians modifiés par les principes jaunes, mais avec grande prépondérance de l’essence blanche et quelques affinités sémitiques ;\par
2° Des aborigènes. – Populations slavo-celtiques saturées d’éléments jaunes ;\par
3° Des Thraces. – Arians mêlés de Celtes et de Slaves ;\par
4° Des Phéniciens. – Chamites noirs ;\par
5° Des Arabes et des Hébreux. – Sémites très mêlés ;\par
6° Des Philistins. – Sémites peut-être plus purs ;\par
7° Des Libyens. – Chamites presque noirs ;\par
8° Des Crétois et autres insulaires. – Sémites assez semblables aux Philistins.\par
Ce tableau a besoin d’être commenté \footnote{Je suis de l’avis de Grote (\emph{Hist. of Greece}, t. II, p. 350 et passim) : je ne crois pas aux Pélasges, en tant que formant une race ou une nation distincte, et le mot signifie trop bien \emph{anciens habitants}, pour que je lui retire ce sens vague et lui en prête un plus spécial. On rencontre les Pélasges en tant d’endroits et pourvus de caractères si différents, qu’il me semble impossible de leur attribuer une nationalité unique. (Voir, à ce sujet, Grote, t. II, p. 349.) ‑ Pott exprime son sentiment d’une façon qui mérite d’être reproduite ici : « Les « Pélasges, dit-il, sont, quoi qu’on fasse, une simple fumée et dénués de toute réalité « historique, aussi bien que les \emph{Casci} c’est-à-dire les \emph{anciens}, les \emph{ancêtres} et les \emph{aborigènes} « c’est-à-dire \emph{habitants primitifs}. Le nom de Pélasges a été pris à tort pour une appellation « de peuple et de race. Il ne s’applique que chronologiquement aux premiers âges de la « Grèce et aux tribus qui habitaient alors ce pays, sans distinction d’origine. Si, plus tard, on « a cru trouver encore çà et là des peuplades qu’on a jugées propres à revêtir cette « désignation de Pélasges, c’est par un rapprochement tout semblable à l’idée admise au « siècle dernier que les Goths étaient des Scythes, des Gêtes, etc. On croyait alors qu’il « existait des restes de cette nation germanique dans la Crimée. » (\emph{Encyclop. Ersch u. Gruber}, 2\textsuperscript{e} sect. 18\textsuperscript{e} par., p. 18.)}. Il ne contient pas, à proprement parler, un seul élément pur. Sur sept, six renferment, à différents degrés, des principes méla­niens ; deux ont des principes jaunes ; deux encore contiennent l’élément blanc pris à la branche chamitique, et donc extrêmement affaibli ; trois le possèdent emprunté au rameau sémitique, deux autres au rameau arian ; trois, enfin, réunissent les deux dernières sources. J’en tire les conséquences suivantes :\par
Le principe blanc, en général, domine, et l’essence ariane y partage l’influence avec la sémitique, attendu que les invasions des Arians Hellènes, ayant été les plus nombreuses, ont formé le fond de la population nationale. Toutefois l’abondance du sang sémitique est telle, sur certains points en particulier, que l’on ne peut refuser à ce sang une action marquée, et c’est à lui qu’appartient une initiative tempérée par l’action ariane appuyée du contingent jaune. Il va sans dire que ce jugement a pour objet la Grèce méridionale, la Grèce de l’Attique, du Péloponèse, des colonies, la Grèce artiste et savante. Au nord, les éléments mélaniens sont presque nuls. Aussi, dans les siècles rapprochés de la guerre de Troie, ces régions excitèrent, beaucoup moins que les contrées asiatiques, les préoccupations des Grecs du sud.\par
C’est que, en effet, à ces époques, et vers le temps où Hérodote écrivait, la Grèce était elle-même un pays asiatique, et la politique qui l’intéressait le plus s’élaborait à la cour du grand roi. Tout ce qui avait trait à l’intérieur, agrandi, ennobli à nos yeux par l’admirable manière dont le souvenir nous en a été conservé, n’était pourtant que très secondaire en, comparaison des faits extérieurs dont les ressorts restaient aux mains des Perses.\par
Depuis que l’Égypte était tombée au rang de province ralliée aux États achéménides, il n’y avait plus dans le monde occidental deux civilisations comme jadis. L’antagonisme de l’Euphrate et du Nil avait cessé ; plus rien d’assyrien, plus rien d’égyptien, et, en place, un compromis auquel je ne trouve d’autre nom que celui d’asiatique. Cependant la grande place y appartenait encore au principe assyrien. Les Perses, trop peu nombreux, n’avaient pas transformé ce principe, ne l’avaient pas même renouvelé. Leur bras s’était trouvé assez fort pour lui donner une impulsion que les dynasties chaldéennes n’avaient pu créer à un même degré, et, sous l’atteinte de ce colosse en pourriture, la débile caducité égyptienne s’était réduite en poussière et mêlée à lui. Existait-il dans le monde une troisième civilisation pour prendre la place des cham­pions anciens ? Nullement : la Grèce ne représentait pas, vis-à-vis de l’Assyrie, une culture originale comme l’égyptienne, et bien que son intelligence eût des nuances très spéciales, la plupart des éléments qui la composaient se retrouvaient, avec le même sens et la même valeur, chez les peuples sémitiques du littoral méditerranéen. C’est une vérité qui n’a pas besoin de démonstration.\par
Dans leur opinion même, les Grecs faisaient beaucoup plus de cas de ce qu’ils appelaient, sans doute, en leur langage, les conquêtes de la civilisation, c’est-à-dire les importations de dieux, de dogmes, de rites asiatiques, et de rêveries monstrueuses venues des côtes voisines, que de la simplicité ariane professée jadis par leurs religieux ancêtres mâles. Ils s’enquéraient avec prédilection de ce qui s’était pensé et fait en Asie. Ils se mêlaient de leur mieux aux affaires, aux intérêts, aux querelles du grand continent, et, bien que pénétrés de leur propre importance, comme tout petit peuple doit l’être, bien qu’appelant même l’univers entier barbare, en dehors d’eux, leur regard ne se détachait pas de l’Asie.\par
Tant que les Assyriens furent indépendants, les Grecs, faibles et éloignés, ne comptèrent que peu dans le monde ; mais, comme le développement hellénique se trouva contemporain de la grande fortune des Arians Iraniens, ce fut à cette époque qu’en face des maîtres de l’Asie antérieure, ils eurent à opter entre l’antagonisme et la soumission. Le choix était indiqué par leur faiblesse. Ils acceptèrent l’influence victo­rieuse, dominatrice, irrésistible, du grand roi, et vécurent dans la sphère de sa puissance, sinon à l’état de sujets, du moins à celui de protégés.\par
Tout, je le répète, leur en faisait une obligation. La parenté avec les Asiatiques était étroite ; la civilisation presque identique dans ses bases, et, enfin, sans le bon vouloir des Perses, c’en était fait des colonies ioniennes, toujours et traditionnellement soutenues par la politique des souverains de l’Assyrie. Or, de la fortune des colonies dépendait celle des métropoles \footnote{Le fait qui démontre le mieux cet état de choses, c’est l’attitude de la majeure partie des États grecs pendant la guerre persique. À la bataille de Platée, 50.000 fantassins et une nombreuse cavalerie hellénique combattirent dans les rangs du grand roi, contre les Athéniens et leurs alliés. Ces troupes furent fournies, non pas par les Ioniens, que je mets à part, mais par les Béotiens, les Locriens, les Maliens, les Thessaliens, c’est-à-dire toute la Grèce orientale. Il faut y ajouter encore les Phocéens. Ces derniers envoyèrent 2.000 hommes aux Perses. Par conséquent, le Péloponèse et l’Attique, voilà tout ce qui résistait. On a fait depuis, de cette campagne d’une minorité contre la majorité de la Grèce, une gloire nationale. (Zumpt, \emph{Mémoires de l’Académie de Berlin, Ueber den Stand der Bevœlkerung und die Volksvermehrung im Alterthum}, p. 5.)}.\par
Il y avait ainsi accord entre les Arians Grecs et les Arians Iraniens. Le lien commun était ce vaste élément sémitique sur lequel, chacun chez soi, ils avaient dominé, et qui, tôt ou tard, par une voie ou par une autre, devait les absorber également dans son unité agrandie.\par
Il peut paraître singulier que je dise que les Arians Grecs eussent jamais dominé chez eux le principe sémitique, après avoir démontré que la plus grande partie de leur civilisation en était faite. Pour donner raison de cette contradiction apparente, je n’ai qu’à rappeler une réserve inscrite plus haut. En disant que la culture grecque était principalement d’origine sémitique, je réservais un certain état antérieur que je vais examiner maintenant, et qui contient, avec trois éléments tout à fait arians, l’histoire primitive de l’hellénisme épique. Ces éléments sont : la pensée gouvernementale, l’aptitude militaire, un genre bien particulier de génie littéraire. Tous les trois ressortent de l’hymen de ces deux instincts arians, la raison et la recherche de l’utile.\par
Le fondement de la doctrine gouvernementale des Arians Hellènes était la liberté personnelle. Tout ce qui pouvait garantir ce droit, dans la plus grande extension possi­ble, était bon et légitime. Ce qui le restreignait était à repousser. Voilà le sentiment, voilà l’opinion des héros d’Homère : voilà qui ne se retrouve qu’à l’origine des sociétés arianes.\par
 À l’aurore des âges héroïques, et même longtemps après, les États grecs sont gouvernés d’après les données, les notions déjà observées dans l’Inde, en Perse, et quelque peu à l’origine de la société chinoise, c’est-à-dire pourvus d’un gouvernement monarchique, limité par l’autorité des chefs de famille, par la puissance des traditions et la prescription religieuse. On y remarque un grand éparpillement national, de fortes traces de cette hiérarchie féodale si naturelle aux Arians, préservatif assez efficace contre les inconvénients principaux du fractionnement, conséquence de l’esprit d’indépendance \footnote{« Between the different degrees of hellenic chivalry a certain equality at all times prevailed, « which the fewness of their numbers comprend with the population amidst whom they « dwelt and the hereditary pride of a dominant race, alike tended to preserve. We find the « doric nobles, too in after times, assuming to themselves the epithet of \emph{the Equals.} » C’est un sentiment tout à fait pareil et d’une origine ethnique rigoureusement semblable, qui a rendu si cher à la noblesse du moyen âge le nom de \emph{pairs}, traduction exacte du grec (mot grec). (W. Torrens Mc. Cullagh, \emph{The industrial History of free Nations} (London, 1846, in-8°, t. I, p. 3.)}. Rien de plus surveillé dans l’exercice de son pouvoir qu’Agamemnon, le roi des rois ; rien de plus limité dans sa puissance que l’habile souverain d’Ithaque. L’opinion est maîtresse dans ces grands villages \footnote{Athènes avait commencé par être une agrégation de plusieurs hameaux. Sparte était un composé de cinq bourgades et ne fut jamais une ville ; Mantinée également ; Tégée en comptait huit ; Dymé, en Achaïe, et Élis de même ; de même encore Mégare et Tanagra. Jusqu’à la bataille de Leuctres, la plupart des Arcadiens n’eurent aussi que des villages, et les Épirotes les imitèrent. (Grote, t. II, p. 346.)}, où il n’existe pas, sans doute, de journaux \footnote{Les poètes, comme Hésiode et Homère, paraissent avoir eu leur franc parler contre les excès et probablement le simple usage aussi du pouvoir. (Hésiode, \emph{les Travaux et les jours}, p. 186.)}, mais où les ambitieux, plus ou moins éloquents, ne manquent pas à la perturbation des affaires. Pour bien comprendre ce que c’était qu’un roi grec aux prises avec les difficultés gouvernementales, il n’est rien de mieux que d’étudier le coup d’État d’Ulysse contre les amants de Pénélope. On y voit sur quel terrain scabreux opérait l’autorité du prince, même ayant de son côté le droit et le bon sens.\par
Dans cette société vive, jeune, altière, le génie arian inspirait richement la poésie épique. Les hymnes adressés aux dieux étaient des récits ou des nomenclatures plutôt que des effusions. Le jour du lyrisme n’était pas venu. Le héros grec combattait monté sur le char arian, ayant à ses côtés un écuyer de sang noble, souvent royal, bien semblable au souta brahmanique, et ses dieux étaient des dieux-esprits, indéfinis, peu nombreux et ramenés facilement à une unité qui, mieux que tout encore, sentait son origine voisine des monts hymalayens \footnote{Voir dans le premier volume la note sur le Vourounas arian, le \emph{Varouna} hindou et l’(mot grec) grec, et surtout ce qui a été dit sur le \emph{Deus}, puis sur les Titans.}.\par
À ce moment très ancien, la puissance civilisatrice, initiatrice, ne résidait pas dans le sud : elle émanait du nord. Elle venait de la Thrace avec Orphée, avec Musée, avec Linus. Les guerriers grecs apparaissaient grands de taille, blancs et blonds. Leurs yeux portaient leur arrogance dans l’azur, et ce souvenir resta tellement maître de la pensée des générations suivantes, que lorsque le polythéisme noir eut envahi, avec l’affluence croissante des immigrations sémitiques, toutes les contrées comme toutes les con­sciences, et eut substitué ses sanctuaires aux simples lieux de prière dont jadis les aïeux se contentaient, la plus haute expression de la beauté, de la puissance majestueuse, ne fut pas autre pour les Olympiens que la reproduction du type arian, yeux bleus, cheveux blonds, teint blanc, stature élevée, dégagée, élancée.\par
Autre signe d’identité non moins digne de remarque. En Égypte, en Assyrie, dans l’Inde, on avait eu l’idée que les hommes blancs étaient dieux ou pouvaient le devenir, et l’on admettait la possibilité du combat et de la victoire des guerriers blancs contre les puissances célestes. Les mêmes notions se retrouvent au sein des sociétés primitives de la Grèce, ainsi que je l’ai dit à propos des Titans, et je le répète ici de leurs descen­dants immédiats, les Deucalionides. Ces braves combattent audacieusement les êtres surnaturels et les forces personnifiées de la nature. Diomède blesse Vénus ; Hercule tue les oiseaux sacrés du lac Stymphalide, il étouffe les géants, enfants de la terre, et fait trembler d’épouvante la voûte des palais infernaux ; Thésée, parcourant le monde d’en bas l’épée à la main, est un vrai Scandinave. En un mot, les Arians Grecs, comme tous leurs parents, ont une si haute opinion des droits de la vigueur, que rien ne leur paraît trop au-dessus de leurs prétentions légitimes et d’une audace permise.\par
Des hommes si avides d’honneur, de gloire et d’indépendance étaient naturellement portés à se mettre au-dessus les uns des autres et à réclamer des égards extraordinaires. Il ne leur suffisait pas de limiter de leur mieux l’action du pouvoir social et de rendre ce pouvoir dépendant de leurs suffrages : ils voulaient se faire compter, estimer, honorer, non seulement comme Arians, libres et guerriers, mais, dans la masse des guerriers, des hommes libres, des Arians, comme des individualités d’élite. Cette prétention univer­selle obligeait chacun à de grands efforts, et puisque, pour atteindre à l’idéal proposé, il n’y avait d’autre voie que d’être le plus Arian possible, de résumer le plus les vertus de la race, l’on attacha une très grande importance à la pureté des généalogies.\par
Durant les temps historiques, cette notion se pervertit. On s’estima alors suffisam­ment noble, quand la famille put se dire vieille. Dans ce cas, elle mettait son orgueil à accuser une descendance asiatique \footnote{Certaines familles athéniennes semblent avoir pu se rendre, avec vérité, ce témoignage. Les Géphyres, d’où descendaient Harmodius et Aristogiton, portaient un nom chananéen (en chananéen) \emph{geber, geberim}, les forts, les puissants, les chefs. (Bœttiger, t. I, p. 206.)}. Mais, au début de la nation, avoir le droit de se vanter d’être un pur Arian fut le gage unique d’une supériorité incontestable. L’idée de la préexcellence de race existait aussi complète chez les Grecs primitifs que chez toutes les autres familles blanches. C’est un instinct qui ne se rencontre bien entier que dans ce cercle, et qui s’y altère par le mélange avec les races jaune et noire, auxquelles il fut toujours étranger.\par
Ainsi la société grecque, très neuve encore, se hiérarchisait suivant la supériorité de naissance. À côté de la liberté et de la liberté jalouse des Arians Hellènes, pas l’ombre d’égalité entre les autres occupants du sol et ces maîtres audacieux. Le sceptre, bien que donné en principe à l’élection, trouva, par le respect dont on entourait les grands lignages, une forte cause de se perpétuer exclusivement dans quelques descendances. Sous certains rapports même, l’idée de suprématie d’espèce, consacrée par celle de famille, conduisit les Arians Grecs à des résultats comparables à ceux que nous avons observés en Égypte et dans l’Inde, c’est-à-dire que, eux aussi, ils connurent les démar­cations de castes et les lois prohibitives des mélanges. Il y a plus : ils appliquèrent ces lois jusqu’aux derniers temps de leur existence politique. On cite des maisons sacerdotales qui ne s’alliaient qu’entre elles, et la loi civile fut toujours dure pour les rejetons des citoyens mariés à des étrangères. Cependant, je me hâte de le dire, ces restrictions étaient faibles. Elles ne pouvaient avoir la même portée que les lois du Nil et de l’Arya-varta. La race ariane-grecque, malgré la conscience de sa supériorité d’essence et de facultés sur les populations sémitiques qui la pénétraient de toutes parts, avait ce désavantage d’être jeune d’expérience et de savoir, tandis que les autres étaient vieilles de civilisation. Ces dernières jouissaient, à son détriment, d’une supé­riorité extérieure qui ne permettait pas de les dédaigner et de se refuser complètement à l’alliage. Le système des castes resta toujours à l’état d’embryon : il ne put se développer. L’hellénisme eut trop souvent intérêt à permettre les mésalliances, et d’autres fois il se vit forcé de les subir. Sous ce double rapport, sa situation ressembla beaucoup à ce que fut plus tard celle des Germains.\par
Quoi qu’il en soit, l’idée nobiliaire se montra extrêmement forte et puissante chez les Arians Grecs. Le classement des citoyens ne se faisait que d’après la valeur de chaque descendance ; les vertus individuelles venaient après \footnote{Il faut que cette doctrine ait été bien solidement attachée à l’esprit des tribus helléniques, par la partie ariane de leur sang, puisque, dans la période démocratique et à Athènes même, la naissance conservera toujours du prix. M. Mc. Cullagh le reconnaît sans difficulté : « Regard for ancient lineage was, through every change of plight and policy, fast rooted in « the Ionic mind. The old families remained every where, and even in the most democratic « states, preserved certain political privileges and what they doubtless prized still more, « certain social distinction. » (T. I, p. 239.)}. Je le répète donc : l’égalité était complètement proscrite. Chacun, se sentant fier de son extraction, ne voulait pas être confondu dans la foule.\par
Et de même que chacun prétendait être libre, honoré, admiré, chacun aussi visait à commander autant que possible. Il semble qu’une telle tendance dût être difficile à réaliser dans une société ainsi faite, que le roi lui-même, le pasteur du peuple, avant d’exprimer un avis, devait s’enquérir si cet avis convenait aux dieux, aux prêtres, aux gens de haute naissance, aux guerriers, au gros du peuple. Heureusement, il restait des ressources : il y avait l’esclave, l’ancien autochtone asservi, puis enfin les étrangers. Voyons d’abord ce qu’était l’esclave.\par
Pour premier point, la créature réduite à cette condition n’appartenait, dans aucun cas, à la cité. Tout homme né sur le sol consacré et de parents libres avait un droit imprescriptible à vivre libre lui-même. Sa servitude était illégitime, emportait le carac­tère de crime, ne durait pas, n’était pas. Si l’on réfléchit que la cité grecque primitive renfermait une nation, une tribu particulière, et que cette nation, cette tribu, se considérant comme unique en son espèce, ne voyait le monde qu’en elle-même, on découvre dans cette prescription fondamentale la proclamation du principe que voici : « L’homme blanc n’est fait que pour l’indépendance et la « domination ; il ne doit pas subir, dans la perpétration de ses actes, la direction « d’autrui. »\par
Cette loi, évidemment, n’est pas une invention locale. On la retrouve ailleurs, on la revoit dans toutes les constitutions sociales de la famille que l’on peut observer d’assez près pour se rendre compte des détails. J’en tire la conséquence que, suivant cette opinion, il n’était pas permis de réduire en servitude un homme blanc, c’est-à-dire un \emph{homme}, et que l’oppression, quand elle était limitée aux individus des espèces noire et jaune, n’était pas censée constituer une violation de ce dogme de la loi naturelle.\par
Après la séparation des différentes descendances blanches, chaque nation s’étant imaginé, dans son isolement au milieu de multitudes inférieures ou métisses, être l’unique représentant de l’espèce, ne se fit aucun scrupule d’user des prérogatives de la force dans toute leur étendue, même sur les parents que l’on rencontrait et qui n’étaient plus reconnus pour tels, du moment qu’ils appartenaient à d’autres rameaux. Ainsi, bien que, dans la règle, il ne dût y avoir que des esclaves jaunes ; et noirs, il s’en fit pourtant de métis et ensuite de blancs, par une corruption de la fâcheuse prescription antique dont on avait involontairement altéré le sens, en en restreignant le bénéfice aux seuls membres de la cité.\par
Une preuve sans réplique que cette interprétation est la bonne, c’est qu’en vertu d’une extension très anciennement appliquée, on ne voulut pas non plus pour esclaves les habitants des colonies, ni les alliés, ni les peuples avec lesquels on avait des rapports d’hospitalité ; et, plus tard encore, suivant une autre règle qui, au point de vue de la loi originelle, et dans un sens ethnique n’était qu’une assimilation arbitraire, on étendit cette franchise à toutes les nations grecques.\par
Je vois ici une preuve que, dans l’Asie centrale, les peuples blancs, au temps de leur réunion, s’interdisaient de posséder leurs congénères, c’est-à-dire les hommes blancs ; et les Arians Grecs, observateurs incorrects de cette loi primordiale, ne consentaient pas davantage à asservir leurs congénères, c’est-à-dire leurs concitoyens.\par
 En revanche, la situation des premiers possesseurs de l’Hellade, tels que les Hélotes et les Pénestes, ressemblait à du servage \footnote{« As a birthright the Hellenes claimed both in peace and war, exclusive sway ; and their « kings are depicted as endued with unlimited power over the earth-born multitude. » (Mc. Cullagh. t. I, p. 6.)}. La différence essentielle était que les populations soumises n’habitaient pas les demeures \footnote{Ces demeures étaient des citadelles chevaleresques entourées de cabanes. Elles dominaient les hauteurs et étaient construites en fragments énormes de rochers. Il est très vraisemblable que les cités, à proprement parler, n’étaient que l’œuvre des colons chananéens. (Mc. Cullagh, t. I, p. 22.) ‑ Disons à ce propos qu’en Italie on a trop longtemps attribué aux populations aborigènes ces vastes et solides constructions nommées pélasgiques ou cyclopéennes. Les tribus agricoles qui composaient ces races dites autochtones n’étaient nullement capables de concevoir ni d’exécuter de pareils labeurs, et on est d’autant plus autorisé à en reporter le mérite soit aux Arians Hellènes, soit même à leurs pères, les Titans, que, dans la Péninsule, le souvenir des murailles cyclopéennes est intimement uni à celui des Tyrrhéniens. La porte de Mycènes est aussi une construction essentiellement hellénique.} du guerrier ainsi que les esclaves : elles vivaient sous leurs toits particuliers, cultivant le sol et payant des redevances, comparables, en ceci, aux serfs du moyen âge. Pour achever la ressemblance, au-dessus de ces manants se plaçait une espèce de bourgeoisie également exclue de l’exercice des droits politiques, mais mieux traitée et plus riche que la classe des paysans. Ces hommes, \emph{Perrhèbes} et \emph{Magnètes} en Thessalie \footnote{Grote, \emph{History of Greece}, t. II, p. 370 et passim.}, et en Laconie \emph{Périœkes}, descendaient certainement de différentes catégories de vaincus. Ou bien ils avaient formé les classes supérieures de la société dissoute, ou bien ils s’étaient soumis volontairement et par capitulations.\par
Les étrangers domiciliés avaient des droits analogues ; mais en somme, esclaves, pénestes, périœkes, étrangers, portaient le poids de la suprématie hellénique.\par
Telles étaient les institutions par lesquelles les Arians Grecs, si amoureux de leur liberté personnelle et si jaloux de la conserver les uns vis-à-vis des autres, trouvaient à satisfaire, dans l’intérieur de l’État et hors des temps de guerre et de conquête, leur besoin de domination. Le guerrier renfermé dans sa maison y était roi. Sa compagne ariane, respectée de tous et de lui-même, avait aussi son parler franc devant le pasteur du peuple. Pareille à Clytemnestre, l’épouse grecque était assez hautaine. Froissée dans ses sentiments, elle savait punir comme la fille de Tyndare. Cette héroïne des temps primitifs \footnote{Grote, t. II, p. 113. ‑ La femme grecque d’Homère est infiniment supérieure à l’épouse des âges civilisés ou sémitisés. Voir Pénélope, Hélène, dans l’\emph{Odyssée}, et la reine des Phéaciens. Elle a, tout à la fois, plus de gravité, de considération et de liberté. Cette première institution s’était un peu conservée chez les Macédoniens, à en juger par le rôle que joue Olympias dans les affaires d’Alexandre. Comparer aussi les mœurs des Doriens de Sparte. (Bœttiger, t. II, p. 61.)} n’est pas autre que la femme altière aux cheveux blonds, aux yeux bleus, aux bras blancs, que nous avons déjà vue aux côtés des Pandavas, et que nous retrouverons chez les Celtes et dans les forêts germaniques. Pour elle, l’obéissance passive n’était pas faite.\par
Cette noble et généreuse créature, assise vis-à-vis de son belliqueux époux, auprès du foyer domestique, apparaissait entourée d’enfants soumis jusqu’à la mort inclusivement aux volontés paternelles. Les fils et les filles marquaient, dans la maison, le premier degré de l’obéissance : des représentations de leur part n’étaient pas de mise. Mais, une fois sorti de la demeure des aïeux, le fils allait fonder une autre souveraineté domestique, et pratiquait à son tour ce qu’il avait appris. Après les enfants venaient les esclaves : leur situation subordonnée n’avait rien de trop pénible. Qu’ils eussent été achetés pour un certain poids d’argent ou d’or, ou acquis par échange en retour de taureaux et de génisses, ou bien encore que le sort de la guerre les eût jetés aux mains de leurs vainqueurs comme épaves d’une ville prise d’assaut, les esclaves étaient plutôt des sujets que des êtres abandonnés à tous les caprices des propriétaires.\par
D’ailleurs, un des caractères saillants des sociétés jeunes, c’est la mauvaise entente de ce qui est productif \footnote{Le préjugé général des races arianes engendre d’ailleurs cette incapacité : pour elles, la première notion du droit de propriété, c’est la conquête, et, comme le dit très bien un historien anglais, « the hellenic idea of property was spoil whether acquired by land or sea. » (Mc. Cullagh, t. I, p. 18.)}, et cette heureuse ignorance rendait assez douce l’existence des esclaves grecs. Soit que, confondus avec les serfs, ils gardassent les troupeaux sur les rives du Pénée et de l’Achéloüs, soit que, dans l’intérieur du manoir, ils eussent à vaquer aux travaux sédentaires, ce qu’on exigeait d’eux était minime, parce que les maîtres avaient eux-mêmes peu de besoins. Les repas étaient promptement apprêtés. Le chef du logis se chargeait, le plus souvent, de tuer les bœufs ou les moutons, et de jeter leurs quartiers dans les chaudières d’airain. Il y prenait plaisir. C’était une politesse envers ses hôtes que de ne pas laisser à des mains serviles le soin de leur bien-être. Y avait-il à faire dans le domaine œuvre de maçon ou de charpentier, le maître encore ne dédaignait pas de manier la doloire et la hache. Fallait-il garder les troupeaux, il n’y répugnait pas davantage. Soigner les arbres du verger, les tailler, les émonder, il s’en chargeait volontiers. En somme, les travaux des esclaves ne s’accomplissaient pas sans la participation du guerrier, tandis que les femmes, réunies autour de l’épouse, tissaient avec elle à la même toile, ou préparaient la laine des mêmes toisons.\par
Rien donc ne contribuait nécessairement à empirer la condition de l’esclave, puisque tout labeur était assez honorable pour que le chef de la maison y prît une part constante. Puis il y avait au logis identité d’idées et de langage. Le guerrier n’en savait guère plus long que ses serviteurs sur les choses du monde et de la vie. S’il arrivait un poète, un voyageur, un sage, qui, après le repas, eût quelques récits à faire entendre, les esclaves, rassemblés autour du foyer, avaient leur part de l’enseignement. Leur expé­rience se formait comme celle du plus noble champion. Les conseils de leur vieillesse étaient aussi bien accueillis que s’ils étaient sortis d’une bouche libre et illustre.\par
Que restait-il donc au maître ? Il lui restait toutes les prérogatives d’honneur, et encore des avantages positifs. Il était le seul homme de la maison, le pontife du foyer. Il avait seul le droit d’offrir des sacrifices. Il défendait la communauté, et, couvert de ses armes, superbement vêtu, prenait sa part de la liberté commune et du respect rendu à tous les citoyens de la cité. Mais, encore une fois, à moins que son caractère ne fût exceptionnellement cruel, qu’il n’exerçât sur ses entours l’action d’un insensé, ni la cupidité ni la coutume ne le portaient à opprimer son esclave, qui ne subissait d’autre malheur réel que celui d’être dominé. Les dieux avaient-ils donné à ce serviteur un talent quelconque, de la beauté ou de l’esprit, il devenait le conseiller, tenait tête à chacun, et jouait le rôle du bossu phrygien chez Xanthus.\par
Ainsi l’Arian Grec, souverain chez lui, homme libre sur la place publique, vrai seigneur féodal, dominait sans réserve son entourage, enfants, serfs et bourgeois.\par
Tant que régna l’influence du Nord, les choses restèrent à peu près partout dans cette situation ; mais lorsque les immigrations asiatiques, les révolutions de toute espèce arrivées à l’intérieur eurent troublé les rapports originaires, et que l’instinct sémitique commença à se faire plus fortement sentir, la scène changea tout à fait.\par
Pour premier point, la religion se compliqua. Depuis longtemps les simples notions arianes avaient été abandonnées. Sans doute elles étaient altérées déjà à l’époque où les Titans commencèrent à pénétrer dans la Grèce. Mais les croyances qui leur avaient succédé, assez spiritualistes encore, perdirent pied de plus en plus. Kronos, usur­pateur, suivant la formule théologique, du sceptre d’Ouranos, fut à son tout détrôné par Jupiter. Des sanctuaires s’ouvrirent à l’infini, des pontificats inconnus jadis trouvèrent des croyants, et les rites les plus extravagants s’emparèrent de la faveur générale. On appelle, dans les écoles, cette fièvre d’idolâtrie l’\emph{aurore} de la civilisation.\par
Je n’y contredis pas : il est certain que le génie asiatique était aussi mûr et même pourri que le génie arian-grec était inexpérimenté et ignorant de ses voies futures. Ce dernier, encore étourdi de la longue traite que venaient de faire ses auteurs mâles à travers tant de pays et de hasards, n’avait pas encore trouvé le loisir de se raffiner. Je ne doute cependant pas que, s’il avait eu assez de temps pour se reconnaître avant de tomber sous l’influence assyrienne, il n’eût agi mieux, et de façon à devancer la civilisation européenne. Il aurait pu faire entrer une plus grande part de son originalité dans les destinées des peuples helléniques. Peu-être aura-il donné moins de hauteur à leurs triomphes artistiques ; mais leur vie politique, plus digne, moins agitée, plus noble, plus respectable, aurait été beaucoup plus longue. Malheureusement, les masses arianes-grecques n’étaient pas comparables en nombre aux immigrations d’Asie \footnote{On a fait d’immenses progrès dans la compréhension de la mythologie hellénique. La distinction est parfaitement établie entre les dogmes, les cultes et les rites venus d’Asie et ceux qui ont eu leurs sources dans des notions européennes. Ce qui reste à faire maintenant est d’une grande difficulté, mais aussi d’un grand intérêt. On sait que les mystères cabires et telchines sont sémitiques, et que l’oracle dodonéen est, pour le fond du moins, d’institution septentrionale. Ce qu’il faudrait maintenant, c’est séparer les données arianes des mélanges finnois. La proportion de ces éléments religieux divers, sémitique, arian, finnique, donnerait la composition exacte du sang grec.}.\par
Je ne date pas la révolution opérée dans les instincts des nations grecques du jour où se firent les mélanges avec les colonisations sémitiques, ou les établissements des Doriens dans le Péloponèse, et, plus anciennement, ceux des Ioniens dans l’Attique. Je me contente de partir du moment où les résultats de tous ces faits modifièrent la pondération des races. Alors l’ancien gouvernement monarchique prit fin. Cette forme de royauté équilibrée avec une grande liberté individuelle, par l’accord des pouvoirs publics, ne convenait plus au tempérament passionné, irréfléchi, incapable de modé­ration, de la race métisse alors produite. Désormais, il fallait du nouveau. L’esprit asiatique était en état d’imposer à ce qui restait d’esprit arian un compromis conforme à ses besoins, et il put, tant il était fort, ne laisser à son associé que des apparences pour satisfaire ce goût de liberté si indélébile dans la nature blanche, que, quand la chose n’existe pas, c’est alors surtout qu’on cherche à mettre le mot en relief.\par
Au lieu de la pondération, on voulut de l’excessif. Le génie de Sem poussait à l’absolutisme complet. Le mouvement était irrésistible. Il ne s’agissait que de savoir entre quelles mains la puissance allait résider. La confier, telle qu’on la voulait faire, à un roi, à un citoyen élevé au-dessus de tous les autres, c’était demander l’impossible à des groupes hétérogènes qui n’avaient pas assez d’unité pour se réunir sur un terrain aussi étroit. L’idée répugnait aux traditions libérales des Arians. L’esprit sémitique, de son côté, n’avait pas de fortes raisons d’y tenir : il était habitué aux formes républi­caines en vigueur sur la côte de Chanaan. Incapable d’ailleurs de se plier à la régularité de l’hérédité dynastique \footnote{« The heroic notion of the unity of the state being centred in the royal line was already « shaken. Many of the less potent nobles saw, in the greater distribution of authority, a « pathway opened to their ambition. » (Mc. Cullagh, t. I, p. 21.)}, il ne souhaitait pas une institution qui, chez lui, n’avait jamais puisé son origine dans le choix libre du peuple, mais toujours dans la conquête et la violence, et, souvent, dans la violence étrangère. Je ne fais d’exception que pour le royaume juif. On imagina donc, en Grèce, de créer une personne fictive, la \emph{Patrie} \footnote{« In the days of the monarchy the word which subsequently was used to denote a city (mot « grec) and finally a state, signified no mote than the castle of the prince. » (Mc. Cullagh, t. I, p. 22.) ‑ De même, à notre époque féodale, on n’employait guère le mot \emph{patrie}, qui ne nous est vraiment revenu que lorsque les couches gallo-romaines ont relevé la tête et joué un rôle dans la politique. C’est avec leur triomphe que le patriotisme a recommencé à être une vertu.}, et on ordonna au citoyen, par tout ce que l’homme peut imaginer de plus sacré et de plus redoutable, par la loi, le préjugé, le prestige de l’opinion publique, de sacrifier à cette abstraction ses goûts, ses idées, ses habitudes, jusqu’à ses relations les plus intimes, jusqu’à ses affections les plus naturelles, et cette abnégation de tous les jours, de tous les instants, ne fut que la menue monnaie de cette autre obligation qui consistait à donner, sur un signe, sans se permettre un murmure, sa dignité, sa fortune et sa vie, aussitôt que cette même patrie était censée vous les demander.\par
L’individu, la patrie l’enlevait à l’éducation domestique pour le livrer nu, dans un gymnase, aux immondes convoitises de maîtres choisis par elle. Devenu homme, elle le mariait quand elle voulait. Quand elle voulait aussi, elle lui reprenait sa femme pour la transmettre à un autre, ou lui attribuait des enfants qui n’étaient pas de lui, ou encore ses enfants propres, elle les envoyait continuer une famille près de s’éteindre. Possédait-il un meuble dont la forme n’agréait pas à la patrie, la patrie confisquait l’objet scandaleux et en punissait sévèrement le propriétaire. Votre lyre comptait une corde, deux de plus que la patrie ne le trouvait bon, l’exil. Enfin, le bruit se répandait-il que le triste citoyen ainsi morigéné obéissait trop bien aux caprices incessants, constamment renouvelés de son despote nerveux et acariâtre, en un mot, pouvait-on, non pas même prouver, mais penser qu’il était immodérément honnête homme, la patrie, perdant patience, lui mettait la besace sur le dos, le faisait saisir et conduire, malfaiteur d’un nouveau genre, à la frontière la plus voisine, en lui disant :Va et ne reviens plus !\par
Si, contre tant et de si effroyables exigences, la victime, cependant un peu émue, tentait de regimber, ne fût-ce qu’en paroles, il y avait la mort, souvent avec tortures, le déshonneur, la ruine certaine de la famille entière du coupable, qui, repoussée par tous les gens assez vertueux pour s’indigner du crime, mais non pas assez pour encourir le châtiment d’Aristide, devait s’estimer très heureuse d’échapper à l’indignation, aux pierres et aux couteaux de tous les patriotes de carrefours.\par
En récompense d’une abnégation si grande, on demande si la patrie accordait des compensations suffisamment magnifiques ? Sans doute : elle autorisait pleinement cha­cun à dire de lui-même, en délirant d’orgueil : je suis Athénien, je suis Lacédémonien, Thébain, Argien, Corinthien, titres fastueux, appréciés, au-dessus de tous les autres, au long d’un rayon de dix lieues carrées, et qui, au delà et dans le pays grec même, pouvait, sous certaines circonstances, valoir le fouet ou la corde à qui s’en serait pavané. En tout cas, c’était une garantie de haine et de mépris. Pour surcroît d’avantages, le citoyen se flattait hautement d’être libre, parce qu’il n’était pas soumis à un homme, et que, s’il rampait avec une servilité sans égale, c’était aux pieds de la patrie. Troisième et dernière prérogative : s’il obéissait à des lois qui n’émanaient pas de l’étranger, ce bonheur, tout à fait indépendant du mérite intrinsèque de la législation, s’appelait posséder l’isonomie, et passait pour incomparable. Voilà tous les dédommagements, et encore n’ai-je pas épuisé la liste des charges \footnote{Les modernes admirateurs du patriotisme grec l’exposent tous, à peu de choses près, comme M. Mc. Cullagh. Voilà la définition de cet économiste : « However they (the greek « state) may differ in internal forms, the but of all was to make every free man feel himself a « part of the state and so to organise the state as to concentrate its power, when required, in « favour of the least of its injured members or for the punishment of the most powerful « condemner of the law. » (Mc. Cullagh, t. I, p. 142.) ‑ Ces principes-là peuvent s’écrire ou se dire ; mais personne ayant le sens commun, n’ignore qu’ils sont impraticables, et, par conséquent ne valent pas ce qu’ils coûtent.}.\par
Le mot patrie couvrait en définitive une pure théorie. La patrie n’était pas de chair et d’os. Elle ne parlait pas, elle ne marchait pas, elle ne commandait pas de vive voix, et, quand elle rudoyait, on ne pouvait pas s’excuser parlant à sa personne. L’expérience de tous les siècles a démontré qu’il n’est pire tyrannie que celle qui s’exerce au profit des fictions, êtres de leur nature insensibles, impitoyables, et d’une impudence sans bornes dans leurs prétentions. Pourquoi ? C’est que les fictions, incapables de veiller elles-mêmes à leurs intérêts, délèguent leurs pouvoirs à des mandataires. Ceux-ci, n’étant pas censés agir par égoïsme, acquièrent le droit de commettre les plus grandes énormités. Ils sont toujours innocents lorsqu’ils frappent au nom de l’idole dont ils se disent les prêtres.\par
Il fallait des représentants à la patrie. Le sentiment arian, qui n’avait pu résister à l’importation de cette monstruosité chananéenne, fut assez séduit par la proposition de confier la délégation suprême aux plus nobles familles de l’État, point de vue conforme à ses idées naturelles. À la vérité, dans les époques où il avait été livré à lui-même, il n’avait jamais admis que les vénérables distinctions de la naissance constituassent un droit exclusif au gouvernement des citoyens. Désormais il était assez perverti pour admettre et subir les doctrines absolues, et, soit que l’on conservât, dans les nouvelles constitutions, un ou deux magistrats suprêmes appelés tantôt rois, tantôt archontes, soit que la puissance exécutive résidât dans un conseil de nobles, l’omnipotence acquise à la patrie fut exercée uniquement par les chefs des grandes familles ; en un mot, le gouvernement des cités grecques se modela complètement sur celui des villes phéniciennes.\par
Avant d’aller plus loin, il est indispensable d’intercaler ici une observation d’une haute importance. Tout ce qui précède s’applique à la Grèce savante, civilisée, à demi et même déjà plus qu’à demi sémitique. Pour la Grèce septentrionale, dominatrice aux premiers âges, et, en ce moment, retombée dans l’ombre, les faits que j’expose ne la concernent nullement. Cette partie du territoire, restée beaucoup plus ariane que l’autre, avait vu ses domaines se circonscrire.\par
La frontière sud, envahie par les populations sémitisées, s’était resserrée. Plus on montait vers le nord, plus l’ancien sang grec avait conservé de pureté. Mais, en somme, la Thessalie était elle-même déjà souillée, et il fallait arriver jusqu’à la Macédoine et à l’Épire pour se retrouver au milieu des traditions anciennes.\par
Au nord-est et au nord-ouest, ces provinces avaient également perdu un voisinage ami. Les Thraces et les Illyriens, envahis et transformés par les Celtes et les Slaves, ne se comptaient plus comme Arians. Cependant le contact de leurs éléments blancs, mêlés de jaunes, n’avait pas pour les Grecs septentrionaux les suites à la fois fébriles et débilitantes qui caractérisaient les immixtions asiatiques du sud.\par
Ainsi limités, les Macédoniens et les Épirotes se maintinrent plus fidèles aux instincts de la race primitive. Le pouvoir royal se conserva chez eux : la forme républicaine leur demeura inconnue aussi bien que l’exagération de puissance accordée au dominateur abstrait appelé la patrie. On ne pratiqua pas, dans ces contrées peu vantées, le grand perfectionnement attique. En revanche, on se gouverna noblement avec des notions de liberté qui possédaient en utilité réelle l’équivalent de ce qu’elles avaient de moins en arrogance. On ne fit pas tant parler de soi ; mais on ne vécut pas non plus d’une existence de catastrophes. Bref, même dans le temps où les Grecs du sud, ayant peu conscience de l’impureté de leur sang, se demandaient entre eux si vraiment les Macédoniens et leurs alliés valaient la peine d’être considérés comme des compatriotes et non comme des demi-barbares, ils n’osèrent jamais contester à ces peuples un grand et brillant courage et une habileté soutenue dans l’art de la guerre. Ces nations peu estimées avaient encore un autre mérite dont on ne s’apercevait pas alors, et qui, plus tard, devait se rendre de lui-même remarquable : c’est que, tandis que la Grèce sémitique ne pouvait, au prix de torrents de sang, souder ensemble ses antipathiques nationalités éparses, les Macédoniens possédaient une cohésion et une force d’attraction qui s’exerçaient avec succès, et, de proche en proche, tendaient à agrandir la sphère de leur puissance en y incorporant les peuples voisins. Sur ce point, ils suivaient exactement, et par les mêmes motifs ethniques, la destinée de leurs parents, les Arians Iraniens, que nous avons vus réunir de même et concentrer les populations congénères avant de marcher à la conquête des États assyriens. Ainsi, le flambeau arian, j’entends le flambeau politique, brûlait réellement, bien que sans éclairs et sans éclats, dans les montagnes macédoniennes. En cherchant dans toute la Grèce, on ne le voit plus exister que là.\par
Je reviens au sud. Le pouvoir absolu de la patrie fut donc délégué à des corps aristocratiques, \emph{aux meilleurs des hommes}, suivant l’expression grecque \footnote{On les appelait aussi, comme chez nous, les \emph{gens bien nés}, (mot grec) Ces nobles ont laissé quel­ques noms. On connaît encore les Codrides, les Médontides, les Alcméonides, les Géphyres d’Athènes, les Penthélides de Mitylène, les Basilides d’Erythrées, les Néléides de Milet, les Bacchiades de Corinthe, le, Ctésippides d’Épidaure, les Eratides de Rhodes, les Hippotadées de Cos et de Cnide, les Aleuades de Larisse, les Opheltiades et les Kléonymides de Thèbes ; les Deucalionides, qui avaient régné à Delphes depuis l’arrivée de leur éponyme. (Mc. Cullagh, t. I, p. 15.)}, et ils l’exercèrent naturellement, comme ce pouvoir absolu et sans réplique pouvait être exercé, avec une âpreté digne de la côte d’Asie. Si les populations avaient encore été arianes, il en serait résulté de grandes convulsions, et, après un temps d’essai plus ou moins prolongé, la race aurait rejeté unanimement un régime mal fait pour elle. Mais la tourbe plus qu’à demi sémitique ne pouvait avoir de ces délicatesses. Elle ne devait jamais s’en prendre à l’essence du système, et jamais, en effet, il n’y eut en Grèce, jusqu’aux derniers jours, la moindre insurrection ni des grands ni du peuple contre le régime arbitraire. Toute la discussion resta bornée à cette considération secondaire, de savoir à qui devait appartenir la délégation omnipotente.\par
Les nobles, arguant du droit de premier occupant, appuyaient leurs prétentions sur la possession traditionnelle, et ils éprouvèrent combien cette doctrine était difficile à maintenir en face d’un danger permanent, inhérent aux sources mêmes du système, et qui naissait de l’absolutisme. Toute chose violente possède en soi une force d’une nature spéciale : cette force, par ses écarts ou même son usage simple, produit des périls qui ne peuvent être conjurés qu’au prix d’une tension permanente. Or, l’unique moyen de réaliser cette immobilité se trouve dans une concentration énergique. C’est pourquoi la délégation des pouvoirs illimités de la patrie penchait constamment à se résumer entre les mains d’un seul homme. Ainsi, pour combattre une nuée d’inconvénients, on se mettait à perpétuité sous le coup d’un autre embarras jugé très redoutable, fort détesté, maudit par toutes les générations, et qu’on nomma la tyrannie.\par
L’origine et la fondation de la tyrannie étaient aussi faciles à découvrir et à prévoir qu’impossibles à empêcher. Lorsque, par suite de l’état de compétition perpétuelle des cités, la patrie périclitait, ce n’était plus un conseil de nobles qui se trouvait capable de faire face à une crise : c’était un citoyen seul qui, bon gré, mal gré, absorbait l’action gouvernementale. Dès ce moment, chacun pouvait se demander si, le danger passé, le sauveur consentirait à lâcher la délégation, et, au lieu de faire frémir tout le monde, s’en retournerait frémir lui-même du trop grand service qu’il avait rendu à la patrie.\par
Autre cas : un citoyen était riche, puissant, considéré ; sa haute position portait nécessairement ombrage aux nobles. Impossible de ne pas lui laisser deviner quelque chose de cette méfiance. À moins d’être aveugle, il s’apercevait qu’un jour ou l’autre un piège lui serait tendu, qu’il y tomberait, et qu’il serait victime d’une proscription pro­portionnée en dureté à l’éclat de ses mérites, à l’importance de sa fortune, à l’étendue de son crédit. Plus donc il avait de moyens de renverser l’autorité légitime et de prendre sa place, plus il avait de raisons de n’y pas manquer. À défaut d’ambition, il y allait de son bien et de sa tête \footnote{Tant que toutes les républiques furent aristocratiques, et là où elles le restèrent, les tyrans sortirent des maisons nobles. Le régime de la démocratie fit naître les tyrans parmi les meneurs libéraux, ceux qu’on appelait les Æsymnètes, gens d’esprit pour la plupart, beaux diseurs, amis des arts, possédés du goût de bâtir, mais qui n’avaient pas envie de se faire justicier par les jaloux et préféraient prendre les devants sur ces derniers. Avec la démagogie, les tyrans surgirent de la boue. (Mac Cullagh, t. I, p. 36.) ‑ C’est dans la peinture des despotes populaires qu’Aristophane excelle. Voir les \emph{Chevaliers}, la \emph{Paix}, etc., etc. La tyrannie fut la lèpre dont tous les gouvernements grecs eurent à souffrir sans pouvoir la guérir jamais. Elle était de leur essence.}. Il s’ensuivit que le prétendu état républicain des villes grecques fut presque constamment éclipsé par l’accident inévitable des tyrannies, et ce qui devait faire l’exception se trouva la règle.\par
Aussitôt que régnait un tyran, on se plaignait de ce qu’on ne remarquait pas sous le gouvernement légal : on se plaignait de voir l’autorité excessive, arbitraire, dégradante ; et, avec toute raison, on la déclarait différente de l’organisation régulière des Macédo­niens et des Perses, où la royauté, fixée et définie par les lois, se conformait aux mœurs et aux intérêts des races gouvernées.\par
En se montrant si sévère pour l’usurpation, on aurait dû réfléchir que le pouvoir des tyrans n’était pas une extension de l’ancien pouvoir : ce n’était rien de plus que les droits dont la patrie restait en tout temps investie. Le tyran, si atroce fût-il, n’aurait rien su pratiquer qui, un jour ou l’autre, n’eût déjà été mis en usage par l’administration normale. Ses prescriptions pouvaient sembler absurdes ou vexatoires ; toutefois, la patrie avait eu la primeur de l’invention. Le tyran ne se hasardait pas dans un seul sentier que les conseils républicains n’eussent frayé déjà.\par
On se rabattait sur ceci, que les excès de l’usurpateur ne profitaient qu’à lui, et qu’au contraire, les sacrifices demandés par les souverains à têtes multiples revenaient au bien général. L’objection est assez vide. Les gouvernements légaux, pour être composés d’une agrégation d’hommes, n’en étaient pas moins un assemblage sans frein d’ambi­tions, de vanités, de passions, de préjugés humains. L’oppression pratiquée par eux était d’aussi belle et bonne étoffe que celle d’un seul chef ; elle avait le même vice moral, elle dégradait tout autant ses victimes. Peu m’importe si c’est Pisistrate ou les Alcméonides qui, suivant leur caprice, peuvent me dépouiller, me violenter, me déshonorer, me tuer ; dès que je sais qu’une prérogative si épouvantable existe au-dessus de ma tête, je tremble, je m’abaisse ; mes mains se joignent suppliantes ; je n’ai plus la conscience d’être un homme, relevant de la raison et de l’équité. Auprès de Pisistrate, une fantaisie inattendue peut me perdre ; auprès des Alcméonides, c’est un hasard de majorité. Avec ou sans la tyrannie, le gouvernement des cités grecques était exécrable, honteux, parce que, dans quelques mains qu’il tombât, il ne supposait pas l’existence d’un droit inhérent à la personne du gouverné, parce qu’il était au-dessus de toute loi naturelle, parce qu’il venait en droite ligne de la théorie assyrienne, parce que ses racines premières, certaines, bien qu’inaperçues, plongeaient dans l’avilissante conception que les races noires se font de l’autorité.\par
Il arriva, mais très souvent, que ces tyrans, si exécrés, si abhorrés des peuples grecs, les gouvernèrent pourtant avec beaucoup plus de douceur et de sagesse que leurs assemblées politiques. Guidé par un sens juste, le possesseur unique d’un droit absolu se contente aisément d’une certaine part dans cette omnipotence, et trouve tout à la fois peu de plaisir et point d’intérêt à tendre ses prérogatives jusqu’à les faire rompre. Cette réserve heureuse n’a jamais chance de se rencontrer dans des corps constitués, toujours enclins, au contraire, à agrandir leurs attributions, et en Grèce tout y conviait les magistratures, rien ne les en écartait.\par
Néanmoins, malgré les services que les tyrans pouvaient rendre et la douceur de leur joug, le point d’honneur voulait qu’ils fussent maudits : il fallait donc que cela fût. Leurs règnes étaient un enchaînement de conspirations et de supplices. Rarement ils se maintenaient jusqu’à leur mort, plus rarement encore leurs enfants héritaient de leur sceptre \footnote{On ne cite pas un seul cas de tyrannie transmise à la troisième génération. les Cypsélides la gardèrent soixante-treize ans ; les Orthagorides, quatre-vingt-dix-neuf. C’est ce qu’on a de plus long. (Mac Cullagh, t. I, p. 40.)}. Cette terrible expérience n’empêchait pas que la nature même des choses ne suscitât sans cesse des successeurs aux tyrans dépossédés. C’est ainsi que ce que je disais tout à l’heure se vérifiait : le gouvernement était la règle, la tyrannie l’exception, et l’exception apparaissait beaucoup plus fréquemment que la règle.\par
Tandis que les pays grecs avaient ainsi tant de peine à conserver ou à reconquérir leur état légal, le courant sémitique y augmentait toujours. Il se continuait, s’accélérait et devait amener ainsi, dans la constitution de l’État, des modifications analogues à celles que nous avons observées dans les villes phéniciennes. De proche en proche, tous les pays helléniques du sud furent gagnés par sa prédominance. Cependant les points atteints les premiers, ce furent les établissements de la côte ionienne et l’Attique \footnote{« With the industrial growth of the commonwealth, the resident aliens, or, as they were « termed, \foreign{metoeci}, grew in number and consideration. They were more numerous at Athens « than in any other state. » (Mac Cullagh, t. I, p. 253.) ‑ Une preuve bien frappante de l’omnipotence de la civilisation asiatique, dans la Grèce méridionale, se trouve en ceci, que le système monétaire et des poids et mesures introduit en 947 par Phéidon, roi d’Argos, et qui s’appelait \emph{éginétique} pour avoir été pratiqué depuis longtemps à Égine, était tout à fait identique à celui que connaissaient les Assyriens, les Hébreux, etc. Bœckh l’a solidement établi. (Grote, \emph{History of Greece}, t. II, p. 429.)}.\par
Sans doute, les grandes immigrations, les colonisations compactes, avaient cessé depuis longtemps ; mais ce qui avait acquis à leur place une extension énorme, c’était l’établissement individuel de gens de toutes classes et de tous états. L’exclusivisme jaloux de la cité, né de l’instinct confus des prééminences ethniques, avait essayé en vain de rejeter tout nouveau venu en dehors des droits politiques : rien n’avait pu arrêter l’invasion du sang étranger. Il s’infiltrait par mille différentes voies dans les veines des citoyens. Les familles les plus nobles, déjà bien métisses, quand elles n’étaient pas purement chananéennes, comme les Géphyres, perdaient de plus en plus leur mérite généalogique. Le plus grand nombre d’ailleurs s’éteignait ; le reste s’appau­vrissait et tombait dans le flot dévorant de la population mélangée. Celle-ci allait se multipliant partout, grâce au mouvement créé par le commerce, le plaisir, la paix, la guerre.\par
L’aristocratie devint infiniment moins forte. Les classes moyennes gagnèrent en influence.\par
On se demanda un jour pourquoi les nobles représentaient seuls la patrie, et pourquoi les riches n’en pouvaient faire autant \footnote{Cette question fut posée un peu partout en Grèce au delà de la Thessalie ; mais les classes moyennes ne remportèrent pas partout la victoire. Dans le nord, à Thespies, à Orchomène, à Thèbes, après des conflits sanglants, la noblesse maintint sa suprématie. À Athènes, au contraire, elle se trahit elle-même. On remarquera que les villes que je nomme étaient beaucoup moins sémitisées que celles de l’extrême sud. (Mac Cullagh, t. I, p. 31.)}.\par
Les nobles, il est vrai, ne possédaient plus guère de noblesse, puisque beaucoup de leurs concitoyens en avaient autant qu’eux \footnote{Graduellement aussi, ils avaient perdu la prépondérance que donnent la possession du sol et la suprématie de richesse. Cependant la loi leur avait longtemps garanti le premier point, et, dans beaucoup d’États, à Milet, à Corinthe, à Samos, à Chalcis, à Égine, ils avaient, de bonne heure, admis que faire le commerce, ce n’était pas déroger. Ce principe ne fut cependant jamais accepté d’une manière générale (Mac Cullagh, t. I, p. 23.) ‑ Très promptement aussi, les grandes familles helléniques, considérant l’influence et les gros revenus de certaines races plébéiennes, s’étaient alliées à elles et ainsi dégradées. (\emph{Ibid.}, t. I, p. 25.)}. Le sang sémitique prédominait dans les chaumières : il avait gagné aussi les palais.\par
Il s’ensuivit des convulsions violentes, et les riches bientôt l’emportèrent \footnote{Sur quelques points, cette victoire ne s’opéra pas sans transition, et l’on vit certaines villes se faire une constitution où le pouvoir était remis à deux conseils : 1’un, la ghérousie (mot grec), était le collège des nobles ; l’autre, le \emph{boulé} (mot grec), l’assemblée des riches. (Mac Cullagh, t. I, p. 26.) ‑ Ce sont les deux chambres du système parlementaire anglais.}. Mais à peine étaient-ils maîtres de manœuvrer à leur tour le despotisme de la patrie, à peine avaient-ils entrepris, à la place de leurs rivaux dépossédés, l’éternelle et malheureuse défense de l’ordre légal contre la tyrannie pullulante, que le gros des citoyens posa de nouveau la question soumise naguère aux grands du pays \footnote{À Cumes, tout homme possédant un cheval avait voix dans l’assemblée. À Éphèse et à Erythrées, où l’on pratiquait une sorte de régime représentatif, des députés du peuple siégeaient avec la noblesse. (Mac Cullagh, t. I, p. 25.)}, se trouva également digne de gouverner et battit en brèche la position des timocrates. Et quand une fois le simple peuple eut mis le pied sur cette pente, l’État ne put s’y retenir. Il devint clair qu’après les citoyens pauvres allaient venir et réclamer les demi-citoyens, les étrangers domici­liés, les esclaves, la tourbe.\par
Arrêtons-nous ici un moment, et considérons une autre face du sujet.\par
La seule et souvent déterminante excuse que peut présenter de son existence prolongée un régime arbitraire et violent, c’est la nécessité d’être fort pour agir contre l’étranger ou dominer à l’intérieur. Le système grec donnait-il au moins ce résultat ?\par
 Il avait trois difficultés à résoudre : d’abord celle qui ressortait de sa situation vis-à-vis du reste du monde civilisé, c’est-à-dire de l’Asie ; puis les relations des États grecs entre eux ; enfin la politique intérieure de chaque cité souveraine.\par
Nous savons déjà que l’attitude de la Grèce entière envers le grand roi était toute de soumission et d’humilité. De Thèbes, de Sparte, d’Athènes, de partout, des ambassades ne faisaient qu’aller à Suse ou en revenir, sollicitant ou débattant les arrêts du souverain des Perses sur les démêlés des villes grecques entre elles. On ne courait même pas jusqu’au maître. La protection d’un satrape de la côte suffisait pour assurer à la politi­que d’une localité une grande prépondérance sur ses rivales. Tissapherne ordonnait, et, inquiètes des suites d’une désobéissance, les républiques silencieuses obéissaient à Tissapherne. Ainsi cette force extrême concentrée dans l’État ne contrariait pas la tendance de l’élément sémitique grec à subir l’influence de la masse asiatique. Si l’annexion tardait, c’est que les restes du sang arian maintenaient encore des motifs suffisants de séparation nationale. Mais ce préservatif allait s’épuisant dans le sud. On pouvait prévoir le jour où l’Hellade et la Perse allaient se réunir.\par
Avec leurs violents préjugés d’isonomie, les villes grecques, cramponnées à leurs petits despotismes patriotiques, marchaient à l’encontre des tendances arianes : il n’était pas question pour elles de simplifier les rapports politiques en agglomérant plusieurs États en un seul. Ce qui se faisait en Macédoine trouvait un contraste parfait dans le travail du reste de la Grèce. Aucune cité ne songeait à dominer un grand terri­toire. Toutes voulaient s’agrandir elles-mêmes matériellement, et n’avaient à proposer à leurs voisins que l’anéantissement. Ainsi, lorsque les expéditions des Lacédémoniens \footnote{C’est ce qui rendait les naturalisations d’étrangers fort difficiles dans les États doriens. « A « rigid exclusiveness characterised several greek communities, the most opposites in almost « every other political sentiment. The people of Megara boasted that they had never « conceded the right of citizenship to any foreigner but Hercules. But Sybaris and Athens « are said to have acted otherwise ; and the interest of Corinth, not to speak of less « important mercantile states, tended in the like direction. » (Mac Cullagh, t. I, p. 256.) ‑ Les mélanges n’en avaient pas moins lieu, bien que plus lentement, chez les nations de race dorique. Les constitutions et l’isonomie de ces peuples ne durèrent qu’un peu plus que celles des autres.} réussissaient, la fin était pour les vaincus d’aller grossir les troupeaux d’esclaves des triomphateurs. On conçoit que chacun se défendît jusqu’à la dernière extrémité. Pas de fusion possible. Ces Grecs élégants du temps de Périclès entendaient la guerre en sauvages. Le massacre couronnait toutes les victoires. C’était chose reçue que le dévouement si vanté à la patrie ne pouvait amener chaque ville qu’à se traîner dans un cercle étroit de succès inféconds et de défaites désastreuses \footnote{M. Bœckh, grand partisan de la liberté athénienne, fait le plus triste tableau des conséquences de la ligue hellénique formée sous la présidence de la ville de Minerve, et que la politique du Pnyx voulait faire tourner à l’avantage de l’État, tel qu’on le comprenait alors. Le trésor commun, d’abord déposé dans le temple de Délos, fut apporté à Athènes. On employa les contributions annuelles des villes alliées à payer le peuple affamé d’assemblées ; on en construisit des monuments, on en fit des statues, on en paya des tableaux. Tout naturellement on ne laissa passer guère de temps sans déclarer les contributions insuffisantes. Les cités confédérées furent accablées d’impôts, et, pour bien dire, pillées. Afin de les rendre souples, le peuple d’Athènes s’arrogea sur elles le droit de vie et de mort. Il y eut des révoltes ; on massacra ce qu’on put des populations rebelles, et le reste fut jeté en esclavage. Plusieurs nations, dégoûtées de ce genre de vie, s’embarquèrent sur leurs vaisseaux et s’enfuirent ailleurs. Les Athéniens, charmés, peuplèrent à leur gré les terrains vacants. Voilà ce qu’on appelait, dans l’antiquité grecque, le protectorat et l’alliance ; car, il ne faut pas s’y tromper, c’est l’état d’amitié que je viens de dépeindre d’après les doctes pages de M. Bœckh. De mille cités alliées que compte Aristophane dans les \emph{Guêpes}, il n’en restait plus que trois qui fussent libres à la fin de la guerre du Péloponèse : Chios, Mytilène de Lesbos et Méthymne. Le reste était non pas assimilé à ses maîtres, non pas même sujet, mais asservi dans toute la rigueur du mot. (\emph{Die Staatshaushaltung der Athener}, t. I, p. 443.)}.\par
Au bout des premiers, la ruine de l’ennemi ; au bout des secondes, celle des citoyens. Pas le moindre espoir de s’entendre jamais, et la certitude de ne rien fonder de grand.\par
Et à quoi aboutissait de son côté la politique intérieure ? Nous l’avons vu : sur dix ans, six de tyrannie, le reste de débats, de querelles, de proscriptions et de carnages entre l’aristocratie et les riches, entre les riches et le peuple. Quand, dans une ville, tel parti triomphait, tel autre errait au sein des cités voisines, recrutant des ennemis à ses adversaires trop heureux. Toujours un citoyen grec revenait d’exil ou faisait son paquet pour y aller. De sorte que ce gouvernement d’exigences, cette perpétuelle mise sur pied de la force publique, cette monstruosité morale que présentait l’existence d’un système politique dont la gloire était de ne rien respecter des droits de l’individu, aboutissait à quoi ? À laisser l’influence perse grossir sans obstacle, à perpétuer le fractionnement de nationalités qui, résultant de combinaisons inégales dans les éléments ethniques, empêchaient déjà les peuples grecs de marcher du même pas et de progresser dans la même mesure. Grâce à une terrible contraction de l’esprit de chaque localité, la réunion de la race était rendue impossible.\par
Enfin, à la puissance extérieure annulée ou paralysée venait aussi se joindre l’incapacité d’organiser la tranquillité intérieure. C’était un triste bilan, et, pour en faire l’objet de l’admiration des siècles, il a fallu l’éloquence admirable des historiens nationaux. Sous peine de passer pour des monstres, ces habiles artistes n’étaient pas libres de discuter, bien moins encore de blâmer le révoltant despotisme de la patrie. Je ne crois même pas que la magnificence de leurs périodes aurait suffi à elle seule à égarer le bon sens des époques modernes dans une puérile extase, si l’esprit tortu des pédants et la mauvaise foi des rêveurs théoriciens ne s’étaient ligués pour obtenir ce résultat et recommander l’anarchie athénienne à l’imitation de nos sociétés.\par
L’intérêt que prirent à cette affaire les entrepreneurs de renommées était bien naturel. Les uns trouvaient la chose belle, parce qu’elle était expliquée en grec ; les autres, parce qu’elle allait à l’encontre de toutes les idées nouvelles sur le juste et l’injuste. Toutes les idées, ce n’est pas trop dire : car, au tableau que je viens de tracer, il me reste encore à ajouter quels effroyables effets l’absolutisme patriotique produisait sur les mœurs.\par
En substituant l’orgueil factice du citoyen au légitime sentiment de dignité de la créature pensante, le système grec pervertissait complètement la vérité morale, et, comme, suivant lui, tout ce qui était fait en vue de la patrie était bien, également rien n’était bien qui n’avait pas obtenu l’approbation, la sanction de ce maître. Toutes les questions de conscience demeuraient irrésolues dans l’esprit aussi longtemps qu’on ne savait ce que la patrie ordonnait qu’on en pensât. On n’était pas libre de suivre là-dessus une donnée plus sérieuse, plus rigoureuse, moins variable, qu’à défaut d’une loi religieuse épurée, l’homme arian eût trouvée jadis dans sa raison.\par
Ainsi, par exemple, le respect de la propriété était-il, oui ou non, d’obligation stricte ? En général, oui ; mais, non, si l’on volait bien, si, pour déguiser le vol, on savait à propos et avec fermeté y ajouter le mensonge, la ruse, la fourberie ou la violence. Dans ce cas, le vol devenait une action d’éclat, recommandée, prisée, et le voleur ne passait pas pour un homme ordinaire. Était-il bien de garder la fidélité conjugale ? À dire vrai ce n’était pas crime. Mais si un époux s’attachait à tel point à sa femme, qu’il prît plaisir à vivre un peu plus sous son toit que sur la place publique, le magistrat s’en inquiétait et un châtiment exemplaire menaçait le coupable.\par
Je passe sur les résultats de l’éducation publique, je ne dis rien des concours de jeunes filles nues dans le stade, je n’insiste pas sur cette exaltation officielle de la beauté physique dont le but reconnu était d’établir pour l’Etat des haras à citoyens vertement taillés, corsés et vigoureux ; mais je dis que la fin de toute cette bestialité était de créer un ramas de misérables sans foi, sans probité, sans pudeur, sans humanité, capables de toutes les infamies, et façonnés d’avance, esclaves qu’ils étaient, à l’acceptation de toutes les turpitudes. Je renvoie là-dessus aux dialogues du Démos d’Aristophane avec ses valets \footnote{Il est facile de juger des résultats que le régime de la démocratie avait amenés à Athènes. À l’époque de Cécrops, l’Attique passe pour avoir eu 20,000 habitants. Sous Périclès, elle en comptait quelque chose de moins, et quand, avec les Macédoniens, l’isonomie véritable eut été remplacée par la domination étrangère, la cité présenta, dans les dénombrements, les chiffres que voici : 21,000 citoyens, 10,000 métœques ou étrangers domiciliés, et 400.000 esclaves. (Clarac\emph{, Manuel de l’histoire de l’art chez les anciens} (in-12, Paris, 1874), l\textsuperscript{re} partie, p. 318.) ‑ Ce renseignement statistique, comme ce que j’aurai occasion de dire plus tard de la situation de la Rome royale comparée à la Rome consulaire, fait, à lui seul, justice de toutes les opinions qui ont eu cours chez nous depuis trois cents ans sur le mérite relatif des différents gouvernements de l’antiquité. (Voir aussi Bœckh, \emph{Die Staatshaushaltung der Athener}, t, I p. 35 et passim.) ‑ Ce savant entre dans des détails qui concordent avec l’opinion de Clarac.}.\par
Le peuple grec, parce qu’il était arian, avait trop de bon sens, et, parce qu’il était sémite, avait trop d’esprit, pour ne pas sentir que sa situation ne valait rien et qu’il devait y avoir mieux en fait d’organisation politique. Mais par la raison que le contenu ne saurait embrasser le contenant, le peuple grec ne se mettait pas en dehors de lui-même et ne se haussait pas jusqu’à comprendre que la source du mal était dans l’absolutisme hébétant du principe gouvernemental. Il en cherchait vainement le remède dans les moyens secondaires. À la plus belle époque, entre la bataille de Marathon et la guerre du Péloponèse, tous les hommes éminents inclinaient vers l’opinion vague que nous appellerions aujourd’hui \emph{conservatrice.} Ils n’étaient pas aristocrates, dans le sens vrai du mot \footnote{Il y a des observations intéressantes sur ce point dans l’introduction que M. Droysen a mise en tête de sa traduction d’Eschyle. (\emph{Aschylose Werke}, in-12, zw. Aufl. ; Berlin, 1841.)}. Ni Eschyle ni Aristophane ne souhaitaient le rétablissement de l’archontat perpétuel ou décennal ; mais ils croyaient que, dans les mains des riches, le gouvernement avait quelque chance de fonctionner avec plus de régularité que lorsqu’il était abandonné aux matelots du Pirée et aux fainéants déguenillés du Pnyx.\par
Ils n’avaient certainement pas tort. Plus de lumières étaient à trouver dans la noble maison de Xénophon que chez l’intrigant corroyeur de la comédie des \emph{Chevaliers}. Mais, au fond, le gouvernement de la bourgeoisie et des riches se fût-il consolidé, le vice radical du système n’en subsistait pas moins. Je veux croire que les affaires auraient été conduites avec moins de passion, les finances gérées avec plus d’économie ; la nation n’en serait pas devenue d’un seul point meilleure, sa politique extérieure plus équitable et plus forte, et l’ensemble de sa destinée différent.\par
Personne ne s’aperçut du véritable mal et ne pouvait s’en apercevoir, puisque ce mal tenait à la constitution intime des races helléniques. Tous les inventeurs de systèmes nouveaux, à commencer par Platon, passèrent à côté, sans le soupçonner ; que dis-je ? ils le prirent, au contraire, pour élément principal de leurs plans de réforme. Socrate fournit peut-être l’unique exception. En cherchant à rendre l’idée du vice et de la vertu indépendante de l’intérêt politique, et à élever l’homme intérieur à côté et en dehors du citoyen, ce rhéteur avait au moins entrevu la difficulté. Aussi je comprends que la patrie ne lui ait pas fait grâce, et je ne m’étonne nullement de voir que dans tous les partis, et surtout parmi les conservateurs, il se soit trouvé des voix, au nombre desquelles on a compté injustement celle d’Aristophane, pour demander son châtiment et porter sa condamnation. Socrate était l’antagoniste du patriotisme absolu. À ce titre, il méritait que ce système le frappât. Pourtant, il y avait quelque chose de si pur et de si noble dans sa doctrine, que les honnêtes gens en étaient préoccupés malgré eux. Une fois dans le tombeau, on regretta le sage, et le peuple assemblé au théâtre de Bacchus fondit en larmes lorsque le chœur de la tragédie de Palamède, inspiré par Euripide, chanta ces tristes paroles : « Grecs, vous avez mis à mort le plus savant « rossignol des Muses, qui n’avait fait de mal à personne, le plus savant personnage « de la Grèce. » On le pleura ainsi disparu. Si le ciel l’eût soudain ressuscité, nul ne l’en aurait écouté davantage. C’était bien le rossignol des Muses que l’on regrettait, l’homme éloquent, discuteur habile, logicien ingénieux. Le dilettantisme artistique pleurait, le cœur s’affli­geait ; quant au sens politique, il était inconvertissable, parce qu’il fait partie intime, intégrante, de la nature même des races et reflète leurs défauts comme leurs qualités.\par
Je me suis montré assez peu admirateur des Hellènes au point de vue des institutions sociales pour avoir, maintenant, le droit de parler avec une admiration sans bornes de cette nation, lorsqu’il s’agit de la considérer sur un terrain où elle se montre la plus spirituelle, la plus intelligente, la plus éminente qui ait jamais paru. Je m’incline avec sympathie devant les arts qu’elle a si bien servis, qu’elle a portés si haut, tout en réservant mon respect pour des choses plus essentielles.\par
Si les Grecs devaient leurs vices à la portion sémitique de leur sang, ils lui devaient aussi leur prodigieuse impressionnabilité, leur goût prononcé pour les manifestations de la nature physique, leur besoin permanent de jouissances intellectuelles.\par
 Plus on s’enfonce vers les origines à demi blanches de l’antiquité assyrienne, plus on trouve de beauté et de noblesse, en même temps que de vigueur, dans les productions des arts. De même, en Égypte, l’art est d’autant plus admirable et puissant, que le mélange du sang arian, étant moins ancien et moins avancé, a laissé plus d’énergie à cet élément modérateur. Ainsi, en Grèce, le génie déploya toute sa force au temps où les infusions sémitiques dominèrent, sans l’emporter tout à fait, c’est-à-dire sous Périclès, et sur les points du territoire où ces éléments affluaient davantage, c’est-à-dire dans les colonies ioniennes et à Athènes \footnote{Movers, \emph{das Phœnizische Alterth.}, t. II. 1\textsuperscript{re} partie, p. 413.}.\par
Il n’est pas douteux aujourd’hui que, de même que les bases essentielles du système politique et moral venaient d’Assyrie, de même aussi les principes artistiques étaient fidèlement empruntés à la même contrée ; et, à cet égard, les fouilles et les découvertes de Khorsabad, en établissant un rapport évident entre les bas-reliefs de style ninivite et les productions du temple d’Égine et de l’école de Myron, ne laissent désormais subsis­ter aucune obscurité sur cette question \footnote{Bœttiger, à propos de la plus ancienne façon de représenter, sur les monuments, l’enlèvement de Ganymède, où le petit garçon est rudement emporté, tout en pleurs, par les cheveux serrés aux serres de l’aigle, remarque que les traits caractéristiques de l’art grec primitif sont la vivacité, la violence et la recherche de l’expression de la force (\emph{Heftigkeit, Gewaltsamkeit, bœchste Kraftaüsserung}). C’est bien nettement le principe assyrien et la marque de ses leçons. (Bœttiger, \emph{Ideen zur Kunstmythologie}, t. II, p. 64.)}. Mais parce que les Grecs étaient beaucoup plus trempés dans le principe blanc et arian que les Chamites noirs, la force régulatrice existant dans leur esprit était aussi plus considérable, et, outre l’expérience de leurs devanciers assyriens, la vue et l’étude de leurs chefs-d’œuvre, les Grecs avaient un surcroît de raison et un sentiment du naturel fort impérieux. Ils résistèrent vivement et avec bonheur aux excès où leurs maîtres étaient tombés. Ils eurent du mérite à s’en défendre parce qu’il y eut tentation d’y succomber ; car on connut aussi chez les Hellènes les poupées hiératiques à membres mobiles, les monstruosités de certaines images consacrées. Heureusement le goût exquis des masses protesta contre ces dépra­vations. L’art grec ne voulut généralement admettre ni symboles hideux ou révoltants, ni monuments puérils.\par
On lui a reproché pour ce fait d’avoir été moins spiritualiste que les sanctuaires d’Asie. Ce blâme est injuste, ou du moins repose sur une confusion d’idées. Si l’on appelle spiritualisme l’ensemble des théories mystiques, on a raison ; mais si, avec plus de vérité, l’on considère que ces théories ne prennent leur source que dans des poussées d’imagination délivrées de raison et de logique, et n’obéissant plus qu’aux éperons de la sensation, on conviendra que le mysticisme n’est pas du spiritualisme, et qu’à ce titre on a mauvaise grâce à accuser les Grecs d’avoir donné dans les voies sensualistes en s’en écartant. Ils furent, au contraire, beaucoup plus exempts que les Asiatiques des principales misères du matérialisme, et, culte pour culte, celui du Jupiter d’Olympie est moins dégradant que celui de Baal. J’ai, du reste, déjà touché ce sujet.\par
Cependant les Grecs n’étaient pas non plus très spiritualistes. L’idée sémitique régnait chez eux, bien que réduite, et s’exprimait par la puissance des mystères sacrés, exercés dans les temples. Les populations acceptaient ces rites en se bornant quel­quefois à les mitiger, suivant le sentiment d’horreur que la laideur physique inspirait. Quant à la laideur morale, nous savons qu’on était plus accommodant.\par
Cette rare perfection du sentiment artistique ne reposait que sur une pondération délicate de l’élément arian et sémitique avec une certaine portion de principes jaunes. Cet équilibre, sans cesse compromis par l’affluence des Asiatiques sur le territoire des colonies ioniennes et de la Grèce continentale, devait disparaître un jour pour faire place à un mouvement de déclin bien prononcé.\par
On peut calculer approximativement que l’activité artistique et littéraire des Grecs sémitisés naquit vers le VII\textsuperscript{e} siècle, au moment où fleurirent Archiloque, 718 ans avant J.-C., et les deux fondeurs en bronze Théodore et Rhœcus, 691 ans avant J.-C. La décadence commença après l’époque macédonienne, quand l’élément asiatique l’emporta décidément, autrement dit vers la fin du IV\textsuperscript{e} siècle, ce qui donne un laps de quatre cents ans. Ces quatre cents années sont marquées par une croissance ininterrompue de l’élément asiatique. Le style de Théodore paraît avoir été, dans la Junon de Samos, une simple reproduction des statues consacrées à Tyr et à Sidon. Rien n’indique que le fameux coffre de Cypsélus fût d’un travail différent ; du moins, les restitutions propo­sées par la critique moderne ne me paraissent pas rappeler quelque chose d’excellent. Pour trouver la révolution artistique qui créa l’originalité grecque, force est de descendre jusqu’à l’époque de Phidias, qui, le premier, sortit des données, soit du grand goût assyrien, retrouvé chez les Eginètes, et pratiqué dans toute la Grèce, soit des dégénérations de cet art en usage sur la côte phénicienne.\par
Or, Phidias termina la Minerve du Parthénon l’an 438 avant J.-C. Son école commençait avec lui, et le système ancien se perpétuait à ses côtés. Ainsi l’art grec fut simplement l’art sémitique jusqu’à l’ami de Périclès, et ne forma vraiment une branche spéciale qu’avec cet artiste. Par conséquent, depuis le commencement du VII\textsuperscript{e} siècle jusqu’au V\textsuperscript{e}, il n’y eut pas d’originalité, et le génie national proprement dit n’exista que depuis l’an 420 environ jusqu’à l’an 322, époque de la mort d’Aristote. Il va sans dire que ces dates sont vagues, et je ne les prends que pour enfermer tout le mouvement intellectuel, celui des lettres, comme celui des arts, dans un seul raisonnement. Aussi me montré-je plus généreux que de raison. Cependant, quoi que je fasse, il n’y a de l’an 420, où travaillait Phidias, à l’an 322, où mourut le précepteur d’Alexandre, qu’un espace de cent ans.\par
Le bel âge ne dura donc qu’un éclair, et s’intercala dans un court moment où l’équilibre fut parfait entre les principes constitutifs du sang national. L’heure une fois passée, il n’y eut plus de virtualité créatrice, mais seulement une imitation souvent heureuse, toujours servile, d’un passé qui ne ressuscita pas.\par
Je semble négliger absolument la meilleure part de la gloire hellénique, en laissant en dehors de ces calculs l’ère des épopées. Elle est antérieure à Archiloque, puisque Homère vécut au X\textsuperscript{e} siècle.\par
 Je n’oublie rien. Cependant je n’infirme pas non plus mon raisonnement, et je répète que la grande période de gloire littéraire et artistique de la Grèce fut celle où l’on sut bâtir, sculpter, fondre, peindre, composer des chants lyriques, des livres de philosophie et des annales crédules. Mais je reconnais en même temps qu’avant cette époque, bien longtemps avant, il y eut un moment où, sans se soucier de toutes ces belles choses, le génie arian, presque libre de l’étreinte sémitique, se bornait à la production de l’épopée, et se montrait admirable, inimitable sur ce point grandiose, autant qu’ignorant, inhabile et peu inspiré sur tous les autres \footnote{« It is the epic poetry which forms at once both the undoubted prerogative and the solitary « jewel of the earliest aera of Greece. » (Grote, t. II, p. 158 et 162.)}. L’histoire de l’esprit grec comprend donc deux phases très distinctes, celle des chants épiques sortis de la même source que les Védas, le Ramayana, le Mahabharata, les Sagas, le Schahnameh, les chansons de geste : c’est l’inspiration ariane. Puis vint, plus tard, l’inspiration sémitique, où l’épopée n’apparut plus que comme archaïsme, où le lyrisme asiatique et les arts du dessin triomphèrent absolument.\par
Homère, soit que ce fût un homme, soit que ce nom résume la renommée de plusieurs chanteurs \footnote{L’opinion de Wolf est appuyée sur des considérations décisives, Homère, lorsqu’il parle d’un chanteur, de Démodocus, par exemple, ne considère jamais les poèmes dont il charme les auditeurs comme étant des fragments d’un grand tout. Il dit : « Il chanta ceci, ou bien il chanta cela. » L’\emph{Iliade} et l’\emph{Odyssée} ne semblent être que des composés de ballades séparées. Dans le premier de ces ouvrages, observe un historien, en isolant les livres I, VIII, XI à XXII, on obtient une Achilléide complète. (Grote, t. II, p. 202 et 240.)}, composa ses récits au moment où la côte d’Asie était couverte par les descendants très proches des tribus arianes venues de la Grèce. Sa naissance prétendue tombe, suivant tous les avis, entre l’an 1102 et l’an 947. Les Æoliens étaient arrivés dans la Troade en 1162, les Ioniens en 1130. Je ferai le même calcul pour Hésiode, né en 944 en Béotie, contrée qui, de toutes les parties méridionales de la Grèce, conserva le plus tard l’esprit utilitaire, témoignage de l’influence ariane.\par
Dans la période où cette influence régna, l’abondance de ses productions fut extrême, et le nombre des œuvres perdues est extraordinaire. Pour l’\emph{Iliade} et l’\emph{Odyssée} que nous connaissons, nous n’avons plus les Éthiopiques d’Arctinus, la \emph{Petite Iliade} de Leschès, les \emph{Vers cypriotes}, la \emph{Prise d’Œchalie}, le \emph{Retour des vainqueurs de Troie}, la \emph{Thébaïd}e, les \emph{Épigones}, les \emph{Arimaspies} \footnote{La perte de ce poème est bien regrettable. Il nous aurait beaucoup appris sur les Arians de l’Asie centrale. (Grote, t. II, p. 158 et 162.)}, et une foule d’autres. Telle fut la littérature du passé le plus ancien des Grecs : elle resta didactique et narrative, positive et raisonnable, tant qu’elle fut ariane. L’infusion puissante du sang mélanien l’entraîna plus tard vers le lyrisme, en la rendant incapable de continuer dans ses premières et plus admirables voies.\par
Il serait inutile de s’étendre davantage sur ce sujet. C’est assez en dire que de reconnaître la supériorité de l’inspiration hellénique de l’une comme de l’autre époque sur tout ce qui s’est fait depuis. La gloire homérique, non plus qu’athénienne, n’a jamais été égalée. Elle atteignit le beau plutôt que le sublime. Certainement, elle restera à jamais sans rivale, parce que des combinaisons de race pareilles à celles qui la causèrent ne peuvent plus se présenter.
\section[{IV.4. Les Grecs sémitiques.}]{IV.4. \\
Les Grecs sémitiques.}
\noindent J’ai beaucoup devancé les temps et embrassé pour ainsi dire l’histoire de la Grèce hellénique dans son entier, après avoir montré les causes de son éternelle débilité politique. Maintenant je reviens en arrière, et, rentrant dans le domaine des questions d’État, je continuerai à suivre l’influence du sang sur les affaires de la Grèce et des peuples contemporains.\par
Après avoir mesuré la durée de l’aptitude artistique, j’en ferai autant de celle des différentes phases gouvernementales. On verra par là d’une manière nette quelle terrible agitation amène dans les destinées d’une société le mélange croissant des races.\par
Si l’on veut faire commencer à l’arrivée des Arians Hellènes avec Deucalion les temps héroïques où l’on vivait à peu près suivant le mode des ancêtres de la Sogdiane, sous un régime de liberté individuelle restreinte par des lois très flexibles, ces temps héroïques auraient leur début à l’an 1541 avant J.-C.\par
L’époque primitive de la Grèce est marquée par des luttes nombreuses entre les aborigènes, les colons sémites dès longtemps établis et affluant tous les jours, et les envahisseurs arians.\par
Les territoires méridionaux furent cent fois perdus et repris. Enfin, les Arians Hellènes, accablés par la supériorité de nombre et de civilisation, se virent chassés ou absorbés, moitié dans les masses aborigènes, moitié dans les cités sémitiques, et ainsi se constituèrent isolément la plupart des nations grecques \footnote{Les nations helléniques ont souvent la prétention d’être autochtones ; mais lorsque l’on en vient à la preuve, on trouve généralement qu’elles descendent d’un dieu, quand ce n’est pas d’une nymphe topique. Dans le premier cas, je vois un ancêtre arian ou sémite ; dans le second, un mélange initial avec les aborigènes. Ainsi, je conçois qu’on puisse appeler le pirate chananéen Inachus fils de l’Océan et de Téthys. Il avait surgi de la mer. Ainsi encore Dardanus était fils de Jupiter, de Zeus, du dieu arian par excellence. Il était donc Arian lui-même, et venait de la Samothrace, de l’Arcadie ou même d’Italie, bref du nord. Dans la Laconie, avant l’invasion dorienne, on rencontre des demi-autochtones, c’est-à-dire des peuples qui ne sont ni entièrement arians, ni entièrement sémites. Leurs généalogies remontent à Lélex et à la nymphe topique Kléocharia. (Voir Grote, t. I, p. 133, 230, 387.)}.\par
Grâce à l’invasion des Héraclides et des Doriens, le principe arian mongolisé reprit une supériorité passagère ; mais il finit encore Par céder à l’influence chananéenne, et le gouvernement tempéré des rois, aboli pour toujours, fit place au régime absolu de la république.\par
En 752, le premier archonte décennal gouverna Athènes. Le régime sémitique commençait dans la plus phénicienne des villes grecques. Il ne devait être complet que plus tard, chez les Doriens de Sparte et à Thèbes \footnote{Cumes, Argos et Cyrène conservèrent aussi le nom de roi (mot grec) à leur principal magistrat, investi d’ordinaire du commandement de l’armée et de la présidence générale (mot grec) (Mac Culagh, t. I, p. 15.)}. L’âge héroïque et ses conséquences immédiates, c’est-à-dire la royauté tempérée, avaient duré 800 ans. Je ne dis rien de l’époque bien plus pure, bien plus ariane des Titans ; il me suffit de parler de leurs fils, les Hellènes, pour montrer que le principe gouvernemental était resté longtemps établi entre leurs mains.\par
Le système aristocratique n’eut pas autant de longévité. Inauguré à Sparte en 867, et à Athènes en 753, il finit pour cette dernière cité, la ville brillante et glorieuse par excellence, il finit d’une manière régulière et permanente à l’archonte d’Isagoras, fils de Tisandre, en 508, ayant duré 245 ans. Depuis lors jusqu’à la ruine de l’indépendance hellénique, le parti aristocratique domina souvent, et persécuta même ses adversaires avec succès ; mais ce fut comme faction et en alternant avec les tyrans. L’état régulier depuis lors, si tant est que le mot régularité puisse s’appliquer à un affreux enchaînement de désordres et de violences, ce fut la démocratie.\par
 À Sparte, la puissance des nobles, abritée derrière un pauvre reste de monarchie, fut beaucoup plus solide. Le peuple aussi était plus arian \footnote{Ils avaient une certaine parenté avec les Thessaliens. Du moins les Aleuades se disaient Héraclides comme les rois de Sparte, et on observe de grandes analogies entre l’organisation servile des Hélotes et des Périakes des uns et celle des Phœnestes, des Perrhœbes et des Magnètes des autres. Les Doriens, bien supérieurs aux autres tribus helléniques au point de vue social, furent d’ailleurs les hommes d’une migration récente. Ils n’avaient aucun renom mythique, et ne sont pas même nommés dans l’\emph{Iliade}. Ce sont des espèces de Pandavas. (Grote, t. II, p. 2.) ‑ Ils paraissent avoir envahi le Péloponèse par mer, ainsi que les Arians Hindous ont fait du sud de l’Inde. (\emph{Ibid}., p. 4.) À cet égard, il est curieux d’observer comme les Arians, nation si méditerranéenne d’origine sont toujours facilement devenus des marins intrépides et habiles.}. La constitution de Lycurgue ne disparut complètement que vers 235, après une durée de 632 ans \footnote{M. Mac Cullagh attribue gravement le déclin et la chute de Sparte à la fâcheuse persistance des institutions aristocratiques. Il a aussi des paroles de pitié pour ces infortunés Doriens de la Crète, dont la constitution restera inébranlable pendant de longues séries de siècles. La comparaison des dates indiquées ici aurait dû le consoler ; ou du moins, s’il voulait persister à gémir sur le peu de longévité des lois de Lycurgue, ne se maintenant que le court espace de 632 ans, il eût pu réserver la plus grande part de sa sympathie pour la démocratie athénienne, encore bien plus promptement décédée. (Mac Cullagh, t. I, p. 208 et 227.) ‑ Mais M. Mac Cullagh, en sa qualité d’antiquaire libre-échangiste, a particulièrement l’horreur de la race dorienne. Je doute qu’il vienne à bout des préférences toutes contraires d’O. Müller (\emph{die Dorier}). L’érudit allemand est un bien rude antagoniste.}.\par
Pour l’état populaire à Athènes, je ne sais qu’en dire, sinon qu’il entasse tant de hontes politiques à côté de magnificences intellectuelles inimitables, qu’on pourrait croire au premier abord qu’il lui fallut bien des siècles pour accomplir une telle œuvre. Mais, en faisant commencer ce régime à l’archontat d’Isagoras en 508, on ne peut le prolonger que jusqu’à la bataille de Chéronée, en 339. Le gouvernement continua plus tard sans doute à s’intituler république ; toutefois l’isonomie était perdue, et, quand les gens d’Athènes s’avisèrent de prendre les armes contre l’autorité macédonienne, ils furent traités moins en ennemis qu’en rebelles. De 508 à 339, il y a 169 ans.\par
Sur ces 169 ans, il convient d’en déduire toutes les années où gouvernèrent les riches ; puis celles où régnèrent soit les Pisistratides, soit les trente tyrans institués par les Lacédémoniens. Il n’y faut pas comprendre non plus l’administration monarchique et exceptionnelle de Périclès, qui dura une trentaine d’années ; de sorte qu’il reste à peine pour le gouvernement démocratique la moitié des 169 ans ; encore cette période ne fut-elle pas d’un seul tenant. On la voit constamment interrompue par les conséquences des fautes et des crimes d’abominables institutions. Toute sa force s’employa à conduire la Grèce à la servitude.\par
Ainsi organisée, ainsi gouvernée, la société hellénique tomba, vers l’an 504, dans une attitude bien humble en face de la puissance iranienne. La Grèce continentale tremblait. Les colonies ioniennes étaient devenues tributaires ou sujettes.\par
Le conflit devait éclater par l’effet de l’attraction naturelle de la Grèce à demi sémitique vers la côte d’Asie, vers le centre assyrien, et de la côte d’Asie elle-même un peu arianisée vers l’Hellade. On allait voir le succès de la première tentative d’annexion. On y était préparé ; mais il trompa tout le monde, car il s’accomplit en sens contraire à ce qu’on avait dû prévoir.\par
La puissance perse, si démesurément grosse et redoutée, prit de mauvaises mesures. Xerxès se conduisit en Agramant. Sa \emph{giovenil furore} n’accorda aucun égard aux conseils des hommes sages. Les Grecs eurent beau, s’abandonnant les uns les autres, commettre des lâchetés impardonnables et les plus lourdes fautes, le roi s’obstina à être plus fou qu’ils n’étaient maladroits, et, au lieu de les attaquer avec des troupes régulières, il voulut s’amuser à repaître les yeux de sa vanité du spectacle de sa puissance. Dans ce but, il rassembla une cohue de 700.000 hommes, leur fit passer l’Hellespont sur des ouvrages gigantesques, s’irrita contre la turbulence des flots, et alla se faire battre, à la stupéfaction générale, par des gens plus étonnés que lui de leur bonheur et qui n’en sont jamais revenus.\par
Dans les pages des écrivains grecs, cette histoire des Thermopyles, de Marathon, de Platée, donne lieu à des récits bien émouvants. L’éloquence a brodé sur ce thème avec une abondance qui ne peut pas surprendre de la part d’une nation si spirituelle. Comme déclamation, c’est enthousiasmant ; mais, à parler sensément, tous ces beaux triomphes ne furent qu’un accident, et le courant naturel des choses, c’est-à-dire l’effet inévitable de la situation ethnique, n’en fut pas le moins du monde changé \footnote{Les dates sont persuasives : la bataille de Platée tut gagnée le 22 novembre 479 avant J.-C. et l’enivrement des Grecs dure encore et se perpétue dans nos collèges. Mais, outre que la plus grande partie de la Grèce avait été l’alliée des Perses, Sparte, le plus fort de leurs antagonistes, se hâta de conclure une paix séparée en 477, c’est-à-dire deux ans après la victoire. Si Athènes résista plus longtemps à cet entraînement naturel, c’est qu’elle trouvait du profit à maintenir la confédération pour avoir des alliés à opprimer et piller. (Mac Cullagh, t. I, p. 157.) ‑ On peut juger du caractère de cette politique par le décret rendu sur la proposition de Périclès et en vertu duquel le peuple athénien déclarait ne devoir aucun compte de l’emploi des fonds communs de la ligue. (\emph{Ibid.}, p. 161 ; Bœckh \emph{die Staatshaushaltung der Athener}, t. I, p. 429.)}.\par
Après comme avant la bataille de Platée, la situation se trouve celle-ci :\par
L’empire le plus fort doit absorber le plus faible ; et de même que l’Égypte sémitisée s’est agrégée à la monarchie perse, gouvernée par l’esprit arian, de même la Grèce, où le principe sémitique domine désormais, doit subir la prédominance de la grande famille d’où sont sorties les mères de ses peuples, parce que du moment qu’il n’existe pas à Athènes, à Thèbes et même à Lacédémone de plus purs Arians qu’à Suze, il n’y a pas de motifs pour que la loi prépondérante du nombre et de l’étendue du territoire suspende son action.\par
C’était une querelle entre deux frères. Eschyle n’ignorait pas ce rapport de parenté, lorsque, dans le songe d’Atossa, il fait dire à la mère de Xerxès :\par
« Il me semble voir deux vierges aux superbes vêtements.\par
« L’une richement parée à la mode des Perses, l’autre selon la coutume des « Doriens. Toutes deux dépassant en majesté les autres femmes. Sam défaut dans « leur beauté. Toutes deux sœurs d’une même race \footnote{Eschyle, \emph{les Perses}.}. »\par
 Malgré l’issue inespérée de la guerre persique, la Grèce était contrainte par la puissance sémitique de son sang de se rallier tôt ou tard aux destinées de l’Asie, elle qui avait subi si longtemps l’influence de cette contrée.\par
En vérité la conclusion fut telle ; mais les surprises continuèrent, et le résultat fut produit d’une manière différente encore de ce qu’on se croyait en droit d’attendre.\par
Aussitôt après la retraite des Perses, l’influence de la cour de Suze avait repris sur les cités helléniques ; comme auparavant, les ambassadeurs royaux donnaient des ordres. Ces ordres étaient suivis. Les nationalités locales s’exaspérant dans leur haine réciproque, ne négligeant rien pour s’entre-détruire, le moment approchait où la Grèce épuisée allait se réveiller province perse, peut-être bien heureuse de l’être et de connaître ainsi le repos.\par
De leur côté, les Perses, avertis par leurs échecs, se conduisaient avec autant de prudence et de sagesse que leurs petits voisins en montraient peu. Ils avaient soin d’entretenir dans leurs armées des corps nombreux d’auxiliaires hellènes ; ils les affectionnaient à leur service en les payant bien, en ne leur ménageant pas les honneurs. Souvent ils les employaient avec profit contre les populations ioniennes, et ils avaient alors la secrète satisfaction de ne pas voir s’alarmer la conscience calleuse de leurs mercenaires. Ils ne manquaient jamais d’incorporer dans ces troupes les bannis jetés sous leur protection par les révolutions incessantes de l’Attique, de la Béotie, du Péloponèse ; hommes précieux, car leurs villes natales étaient précisément celles contre qui s’exerçaient de préférence leur courage et leurs talents militaires. Enfin quand un illustre exilé, homme d’État célèbre, guerrier renommé, écrivain d’influence, rhéteur admiré, se réclamait du grand roi, les profusions de l’hospitalité n’avaient pas de bornes ; et qu’un revirement politique ramenât cet homme dans son pays, il rapportait au fond de sa conscience, fût-ce involontairement, un bout de chaîne dont l’extrémité était rivée au pied du trône des Perses. Tels étaient les rapports des deux nations. Le gouvernement raisonnable, ferme, habile de l’Asie avait certainement gardé plus de qualités arianes que celui des cités grecques méridionales, et celles-ci étaient à la veille d’expier durement leurs victoires de parade, lorsque l’état de faiblesse inouïe où elles gémissaient fut justement ce qui amena la péripétie la plus inattendue.\par
Tandis que les Grecs du sud se dégradaient en s’illustrant, ceux du nord, dont on ne parlait pas, et qui passaient pour des demi-barbares, bien loin de décliner, grandissaient à tel point, sous l’ombre de leur système monarchique, qu’un matin, se trouvant assez lestes, fermes et dispos, il gagnèrent les Perses de vitesse, et, s’emparant de la Grèce pour leur propre compte, firent front aux Asiatiques et leur montrèrent un adversaire tout neuf. Mais si les Macédoniens mirent la main sur la Grèce, ce fut d’une manière et avec des formes qui révélaient assez la nature de leur sang. Ces nouveaux venus différaient du tout au tout des Grecs du sud, et leurs procédés politiques le prouvèrent.\par
 Les Hellènes méridionaux, après la conquête, s’empressaient de tout bouleverser. Sous le prétexte le plus léger, ils rasaient une ville et transplantaient chez eux les habitants réduits en esclavage. C’était de la même manière que les Chaldéens sémites avaient agi à l’époque de leurs victoires. Les juifs en avaient su quelque chose lors du voyage forcé à Babylone ; les Syriens aussi, quand des bandes entières de leurs populations furent envoyées dans le Caucase. Les Carthaginois usaient du même système. La conquête sémitique pensait d’abord à l’anéantissement ; puis elle se rabattait tout au plus à la transformation. Les Perses avaient compris plus humaine­ment et plus habilement les profits de la victoire. Sans doute, on relève chez eux plusieurs imitations de la notion assyrienne ; cependant, en général, ils se contentaient de prendre la place des dynasties nationales, et ils laissaient subsister les États soumis par leur épée, dans la forme où ils les avaient trouvés.\par
Ce qui avait été royaume gardait ses formes monarchiques, les républiques restaient républiques, et les divisions par satrapies, moyen d’administrer et de concentrer certains droits régaliens, n’enlevaient aux peuples que l’isonomie : l’état des colonies ioniennes au temps de la guerre de Darius et au moment des conquêtes d’Alexandre en fait suffisamment foi.\par
Les Macédoniens restèrent fidèles au même esprit arian. Après la bataille de Chéronée, Philippe ne détruisit rien, ne réduisit personne en servitude, ne priva pas les cités de leurs lois, ni les citoyens de leurs mœurs. Il se contenta de dominer sur un ensemble, dont il acceptait les parties telles qu’il les trouvait, de le pacifier et d’en concentrer les forces de manière à s’en servir suivant ses vues. Du reste, on a vu que cette sagesse dans l’exploitation du succès avait été devancée, chez les Macédoniens, par la sagesse à conserver précieusement leurs propres institutions. Avec tous les droits possibles de faire commencer leur existence politique plus haut encore que la fondation du royaume de Sicyone, les Grecs du nord arrivèrent jusqu’au jour où ils se subordonnèrent le reste de la Grèce sans avoir jamais varié dans leurs idées sociales. Il me serait difficile d’alléguer une plus grande preuve de la pureté comparative de leur noble sang. Ils représentaient bien un peuple belliqueux, utilitaire, point artiste, point littéraire, mais doué de sérieux instincts politiques.\par
Nous avons trouvé un spectacle à peu près analogue chez les tribus iraniennes d’une certaine époque. Il ne faut pourtant pas en décider à la légère. Si nous comparons les deux nations au moment de leur développement, l’une quand, sous Philippe, elle déborda sur la Grèce, et l’autre, dans un temps antérieur, quand, avec Phraortes, elle commença ses conquêtes, les Iraniens nous apparaissent plus brillants et semblent à beaucoup d’égards plus vigoureux.\par
Cette impression est juste. Sous le rapport religieux, les doctrines spiritualistes des Mèdes et des Perses valaient mieux que le polythéisme macédonien, bien que celui-ci de son côté, attaché à ce qu’on nommait dans le sud les vieilles divinités, se tînt plus dégagé des doctrines sémitiques que les théologies athéniennes ou thébaines. Pour être exact, il faut néanmoins avouer que ce que les doctrines religieuses de la Macédoine perdaient en absurdités d’imagination, elles le regagnaient un peu en superstitions à demi finnoises, qui, pour être plus sombres que les fantaisies syriennes, n’en étaient guère moins funestes. En somme, la religion macédonienne ne valait pas celle des Perses, travaillée qu’elle était par les Celtes et les Slaves.\par
En fait de civilisation, l’infériorité existait encore. Les nations iraniennes, touchant d’un côté aux peuples vratyas, aux Hindous réfractaires, éclairés d’un reflet lointain du brahmanisme, de l’autre aux populations assyriennes, avaient vu se dérouler toute leur existence entre deux foyers lumineux qui n’avaient jamais permis à l’ombre de trop s’épaissir sur leurs têtes. Parents des Vratyas, les Iraniens de l’est n’avaient pas cessé de contracter avec eux des alliances de sang. Tributaires des Assyriens, les Iraniens de l’ouest s’étaient également imprégnés de cette autre race, et de tous côtés ainsi l’ensemble des tribus fit des emprunts aux civilisations qui les environnaient.\par
Les Macédoniens furent moins favorisés. Ils ne touchaient aux peuples raffinés que par leur frontière du sud. Partout ailleurs ils ne s’alliaient qu’à la barbarie. Ils n’avaient donc pas le frottement de la civilisation à un aussi grand degré que les Iraniens, qui, la recevant par un double hymen, lui donnaient une forme originale due à cette combinaison même.\par
En outre, l’Asie étant le pays vers lequel convergeaient les trésors de l’univers, la Macédoine demeurait en dehors des routes commerciales, et les Iraniens s’enrichissaient tandis que leurs remplaçants futurs restaient pauvres.\par
Eh bien, malgré tant d’avantages assurés jadis aux Mèdes de Phraortes, la lutte ne devait pas être douteuse entre leurs descendants, sujets de Darius, et les soldats d’Alexandre. La victoire appartenait de droit à ces derniers, car lorsque le démêlé commença, il n’y avait plus de comparaison possible entre la pureté ariane des deux races. Les Iraniens, qui déjà au temps de la prise de Babylone par Cyaxares étaient moins blancs que les Macédoniens, se trouvèrent bien plus sémitisés encore lorsque, 269 ans après, le fils de Philippe passa en Asie. Sans l’intervention du génie d’Alexandre, qui précipita la solution, le succès aurait hésité un instant, vu la grande différence numérique des deux peuples rivaux ; mais l’issue définitive ne pouvait en aucun cas être douteuse. Le sang asiatique attaqué était condamné d’avance à succomber devant le nouveau groupe arian, comme jadis il avait passé sous le joug des Iraniens eux-mêmes, désormais assimilés aux races dégénérées du pays, qui, elles également, avaient eu leurs jours de triomphe, dont la durée s’était mesurée à la conservation de leurs éléments blancs.\par
Ici se présente une application rigoureuse du principe de l’inégalité des races. À chaque nouvelle émission du sang des blancs en Asie, la proportion a été moins forte. La race sémitique, dans ses nombreuses couches successives, avait plus fécondé les populations chamites que ne le put l’invasion iranienne, exécutée par des masses beaucoup moindres. Quand les Grecs conquirent l’Asie, ils arrivèrent en nombre plus médiocre encore ; ils ne firent pas précisément ce qu’on appelle une colonisation. Isolés par petits groupes au milieu d’un immense empire, ils se noyèrent tout d’un coup dans l’élément sémitique. Le grand esprit d’Alexandre dut comprendre qu’après son triomphe, c’en était fait de l’Hellade ; que son épée venait d’accomplir l’œuvre de Darius et de Xerxès, en renversant seulement les termes de la proposition ; que, si la Grèce n’avait pas été asservie lorsque le grand roi avait été à elle, elle l’était maintenant qu’elle avait marché vers lui ; elle se trouvait absorbée dans sa propre victoire. Le sang sémitique engloutissait tout. Marathon et Platée s’effaçaient sous les vénéneux triom­phes d’Arbelles et d’Issus, et le conquérant grec, le roi macédonien, se transfigurant, était devenu le grand roi lui-même. Plus d’Assyrie, plus d’Égypte, plus de Perside, mais aussi plus d’Hellade : l’univers occidental n’avait désormais qu’une seule civilisation.\par
Alexandre mourut ; ses capitaines détruisirent l’unité politique ; ils n’empêchèrent pas que la Grèce entière, et, cette fois, avec la Macédoine comprimée, envahie, possédée par l’élément sémitique, ne devint le complément de la rive d’Asie. Une société unique, bien variée dans ses nuances, réunie cependant sous les mêmes formes générales, s’étendit sur cette portion du globe qui, commençant à la Bactriane et aux montagnes de l’Arménie, embrassa toute l’Asie inférieure, les pays du Nil, leurs annexes de l’Afrique, Carthage, les îles de la Méditerranée, l’Espagne, la Gaule phocéenne, l’Italie hellénisée, le continent hellénique. La longue querelle des trois civilisations parentes qui, avant Alexandre, avaient disputé de mérite et d’invention, se termina dans une fusion de forces également du sang sémitique amenant la proportion trop forte d’éléments noirs, et de cette vaste combinaison naquit un état de choses qu’il est aisé de caractériser.\par
La nouvelle société ne possédait plus le sentiment du sublime, joyau de l’ancienne Assyrie comme de l’antique Égypte ; elle n’avait pas non plus la sympathie de ces nations trop mélaniennes pour le monstrueux physique et moral. En bien comme en mal, la hauteur avait diminué par la double influence ariane des Iraniens et des Grecs. Avec ces derniers, elle prit de la modération dans les idées d’art, ce qui la conduisit à imiter les procédés et les formes helléniques ; mais d’un autre côté, et comme un cachet du goût sémitique raccourci, elle abonda dans l’amour des subtilités sophistiques, dans le raffinement du mysticisme, dans le bavardage prétentieux et les folles doctrines des philosophes. En cherchant le brillant, faux et vrai, elle eut de l’éclat, rencontra quelquefois la bonne veine, resta sans profondeur et montra peu de génie. Sa faculté principale, celle qui fait son mérite, c’est l’éclectisme ; elle ambitionna constamment le secret de concilier des éléments inconciliables, débris des sociétés dont la mort faisait sa vie. Elle eut l’amour de l’arbitrage. On reconnaît cette tendance dans les lettres, dans la philosophie, dans la morale, dans le gouvernement. La société hellénistique sacrifia tout à la passion de rapprocher et de fondre les idées, les intérêts les plus disparates, sentiment très honorable sans doute, indispensable dans un milieu de fusion, mais sans fécondité, et qui implique l’abdication un peu déshonorante de toute vocation et de toute croyance.\par
Le sort de ces sociétés de moyen terme, formées de décombres, est de se débattre dans les difficultés, d’épuiser leurs maigres forces, non pas à penser, elles n’ont pas d’idées propres ; non pas à avancer, elles n’ont pas de but ; mais à coudre et recoudre en soupirant des lambeaux bizarres et usés qui ne peuvent tenir ensemble. Le premier peuple un peu plus homogène qui leur met la main sur l’épaule, déchire sans peine le fragile et prétentieux tissu.\par
Le nouveau monde comprit l’espèce d’unité qui s’établissait. Il voulut que les choses fussent représentées par les mots. Dès lors, pour marquer le plus haut degré possible de perfection intellectuelle, on s’accoutuma à se servir du terme d’\emph{atticisme}, idéal auquel les contemporains et compatriotes de Périclès auraient eu peine à prétendre. On plaça au-dessous le nom d’Hellène ; plus bas, on étagea des dérivés comme \emph{hellénisant, hellénistique}, afin d’indiquer des mesures dans les degrés de civilisation. Un homme né sur la côte de la mer Rouge, dans la Bactriane, dans l’enceinte d’Alexandrie d’Égypte, au bord de l’Adriatique, se considéra et fut tenu pour un Hellène parfait. Le Péloponèse n’eut plus qu’une gloire territoriale ; ses habitants ne passaient pas pour des Grecs plus authentiques que les Syriens ou les gens de la Lydie, et ce sentiment était parfaitement justifié par l’état des races.\par
 Sous les premiers successeurs d’Alexandre, il n’existait plus dans la Grèce entière une nation qui eût le droit de refuser la parenté, je ne dis pas l’identité, avec les hellénisants les plus obscurs d’Olbia ou de Damas. Le sang barbare avait tout envahi. Au nord, les mélanges accomplis avec les populations slaves et celtiques attiraient les races hellénisées vers la rudesse et la grossièreté trônant sur les rives du Danube, tandis qu’au sud les mariages sémitiques répandaient une dépravation purulente pareille à celle de la côte d’Asie ; pourtant, ce n’étaient là au fond que des différences peu essentielles, et qui ne tournaient pas au profit des facultés arianes. Certes, les vainqueurs de Troie, s’ils fussent revenus des enfers, auraient en vain cherché leur descendance ; ils n’auraient vu que des bâtards sur l’emplacement de Mycènes et de Sparte \footnote{ \noindent On suit, avec une grande facilité, les transformations de la population lacédémonienne. À la bataille de Platée, la ville de Lycurgue avait mis en ligne 50,000 combattants, savoir : 5 000 Spartiates et 7 Hélotes par Spartiate, soit 35 000 Hélotes armés, 5 000 hoplitesPériœkes. 5 000 peltastes. Total 50 000. Sur le champ de bataille de Leuctres, il ne paraît plus que 1 000 Spartiates. Depuis longtemps, l’État ne soutenait ses guerres extérieures qu’au moyen d’Hélotes affranchis (mot grec). En 370, avant J.-C., lorsque Épaminondas envahit la Laconie, il fallut encore donner la liberté à 6.000 Hélotes pour pouvoir se défendre. Cent ans après, on ne comptait plus que 700 familles de citoyens, et 100 seulement possédaient des terres ; le reste était ruiné. On reforma alors une aristocratie avec des Périœkes, des étrangers et des Hélotes. À Sellasie, toute cette bourgeoisie nouvelle fut exterminée par le roi Antigone et les Achéens, sauf 200 hommes. Machanidas et son successeur Nabis employèrent le moyen ordinaire pour relever la république : il y eut une vaste promotion de citoyens. Mais peu après, malgré cette ressource, Sparte, encore vaincue et découragée, se fondit dans la ligne achéenne. Cette histoire est celle de tous les États grecs, d’Argos, de Thèbes, comme d’Athènes. (Zumpt, p. 7 et passim)
}.\par
Quoi qu’il en soit, l’unité du monde civilisé était fondée. À ce monde il fallait une loi, et cette loi où l’appuyer ? De quelle source la faire jaillir, quand les gouvernements ne présidaient plus qu’à un immense amas de détritus, où toutes les nationalités anciennes étaient venues éteindre leurs forces viriles ? Comment tirer des instincts mélaniens, qui désormais avaient pénétré jusqu’aux derniers replis de cet ordre social, la reconnaissance d’un principe intelligent et ferme, et en faire une règle stable ? Solution impossible ; et pour la première fois dans le monde on vit ce phénomène, qui depuis s’est reproduit deux fois encore, de grandes masses humaines conduites sans religion politique, sans principes sociaux définis, et sans autre but que de les aider à vivre. Les rois grecs adoptèrent, faute de pouvoir mieux, la tolérance universelle en tout et pour tout, et bornèrent leur action à exiger l’adoration des actes émanés de leur puissance. Qui voulait être république le restait ; telle ville tenait aux formes aristocratiques, à elle permis ; telle autre, un district, une province, choisissait la monarchie pure, on n’y contredisait pas. Dans cette organisation, les souverains ne niaient rien et n’affirmaient pas davantage. Pourvu que le trésor royal touchât ses revenus légaux et extralégaux, et que les citoyens ou les sujets ne fissent pas trop de bruit dans le coin où ils étaient censés se gouverner à leur guise, ni les Ptolémées, ni les Séleucides n’étaient gens à y trouver à redire.\par
La longue période qu’embrassa cette situation ne fut pas absolument vide d’individualités distinguées ; mais elle n’offrit pas à celles qui surgirent un public suffisamment sympathique, et dès lors tout resta dans le médiocre. On s’est souvent demandé pourquoi certains temps ne produisent pas telle catégorie de supériorité : on a répondu, tantôt que c’était par défaut de liberté, tantôt par pénurie d’encouragement. Les uns ont fait honneur à l’anarchie athénienne du mérite de Sophocle et de Platon, affirmé, et en conséquence, que sans les troubles perpétuels des communes d’Italie, Pétrarque, Boccace, le Dante surtout, n’auraient jamais étonné le monde par la magnificence de leurs écrits. D’autres penseurs, tout au rebours, attribuent la grandeur du siècle de Périclès aux générosités de cet homme d’État, l’élan de la muse italienne à la protection des Médicis, l’ère classique de notre littérature et ses lauriers à l’influence bienfaisante du soleil de Louis XIV. On voit qu’en s’en prenant aux circonstances ambiantes, on trouve des avis pour tous les goûts, tels philosophes reportant à l’anarchie ce que tels autres donnent au despotisme.\par
Il est encore un avis : c’est celui qui voit dans la direction prise par les mœurs d’une époque la cause de la préférence des contemporains pour tel ou tel genre de travaux, qui mène, comme fatalement, les natures d’élite à se distinguer, soit dans la guerre, soit dans la littérature, soit dans les arts. Ce dernier sentiment serait le mien, s’il concluait ; malheureusement il reste en route, et lorsqu’on lui demande la cause génératrice de l’état des mœurs et des idées, il ne sait pas répondre qu’elle est tout entière dans l’équilibre des principes ethniques. C’est, en effet, nous l’avons vu jusqu’ici, la raison déterminante du degré et du mode d’activité d’une population.\par
Lorsque l’Asie était partagée en un certain nombre d’États délimités par des différences réelles de sang entre les nations qui les habitaient, il existait sur chaque point particulier, en Égypte, en Grèce, en Assyrie, au sein des territoires iraniens, un motif à une civilisation spéciale, à des développements d’idées propres, à la concen­tration des forces intellectuelles sur des sujets déterminés, et cela parce qu’il y avait originalité dans la combinaison des éléments ethniques de chaque peuple. Ce qui donnait surtout le caractère national, c’était le nombre limité de ces éléments, puis la proportion d’intensité qu’apportait chacun d’eux dans le mélange. Ainsi, un Égyptien du XX\textsuperscript{e} siècle avant notre ère, formé, j’imagine, d’un tiers de sang arian, d’un tiers de sang chamite blanc et d’un tiers de nègre, ne ressemblait pas à un Égyptien du VIII\textsuperscript{e}, dans la nature duquel l’élément mélanien entrait pour une moitié, le principe chamite blanc pour un dixième, le principe sémitique pour trois, et le principe arian à peine pour un. Je n’ai pas besoin de dire que je ne vise pas ici à des calculs exacts ; je ne veux que mettre ma pensée en relief.\par
Mais l’Égyptien du VIII\textsuperscript{e} siècle, bien que dégénéré, avait pourtant encore une nationalité, une originalité. Il ne possédait plus, sans doute la virtualité des ancêtres dont il était le représentant ; néanmoins la combinaison ethnique dont il était issu continuait, en quelque chose, à lui être particulière. Dès le V\textsuperscript{e} siècle il n’en fut plus ainsi.\par
À cette époque l’élément arian se trouvait tellement subdivisé, qu’il avait perdu toute influence active. Son rôle se bornait à priver les autres éléments à lui adjoints de leur pureté, et dès lors de leur liberté d’action.\par
Ce qui est vrai pour l’Égypte s’applique tout aussi bien aux Grecs, aux Assyriens, aux Iraniens ; mais on pourrait se demander comment, puisque l’unité s’établissait dans les races, il n’en résultait pas une nation compacte, et d’autant plus vigoureuse qu’elle avait à disposer de toutes les ressources venues des anciennes civilisations fondues dans son sein, ressources multipliées à l’infini par l’étendue incomparablement plus considérable d’une puissance qui ne se voyait aucun rival extérieur. Pourquoi toute l’Asie antérieure, réunie à la Grèce et à l’Égypte, était-elle hors d’état d’accomplir la moindre partie des merveilles que chacune de ses parties constitutives avait multi­pliées, lorsque ces parties étaient isolées, et, de plus, lorsqu’elles auraient dû souvent être paralysées par leurs luttes intestines ?\par
La raison de cette singularité, réellement très étrange, gît dans ceci, que l’unité exista bien, mais avec une valeur négative. L’Asie était rassemblée, non pas compacte ; car d’où provenait la fusion ? Uniquement de ce que les principes ethniques supérieurs, qui jadis avaient créé sur tous les points divers des civilisations propres à ces points, ou qui, les ayant reçues déjà vivantes, les avaient modifiées et soutenues, quelquefois même améliorées, s’étaient, depuis lors, absorbés dans la masse corruptrice des éléments subalternes, et, ayant perdu toute vigueur, laissaient l’esprit national sans direction, sans initiative, sans force, vivant, sans doute, toutefois sans expression. Partout les trois principes, chamite, sémite et arian, avaient abdiqué leur ancienne initiative, et ne circulaient plus dans le sang des populations qu’en filets d’une ténuité extrême et chaque jour plus divisés. Néanmoins, les proportions différentes dans la combinaison des principes ethniques inférieurs se perpétuaient éternellement là où avaient régné les anciennes civilisations. Le Grec, l’Assyrien, l’Égyptien, l’Iranien du V\textsuperscript{e} siècle étaient à peine les descendants de leurs homonymes du XX\textsuperscript{e} : on les voyait de plus rapprochés entre eux par une égale pénurie de principes actifs ; ils l’étaient encore par la coexistence dans leurs masses diverses de beaucoup de groupes à peu près similaires ; et cependant, malgré ces faits très véritables, des contrastes généraux, sou­vent imperceptibles, cependant certains, séparaient les nations. Celles-ci ne pouvaient pas vouloir et ne voulaient pas des choses bien différentes ; mais elles ne s’entendaient pas entre elles, et dès lors, forcées de vivre ensemble, trop faible chacune pour faire prévaloir des volontés d’ailleurs à peine senties, elles penchaient toutes à considérer le scepticisme et la tolérance comme des nécessités, et la disposition d’âme que Sextus Empiricus vante sous le nom d’ataraxie comme la plus utile des vertus.\par
Chez un peuple restreint quant au nombre, l’équilibre ethnique ne parvient à s’établir qu’après avoir détruit toute efficacité dans le principe civilisateur, car ce principe, ayant nécessairement pris sa source chez une race noble, est toujours trop peu abondant pour être impunément subdivisé. Cependant, aussi longtemps qu’il reste à l’état de pureté relative, il y a prédominance de sa part, et donc pas d’équilibre avec les éléments inférieurs. Que peut-il arriver, dès lors, quand la fusion ne se fait plus qu’entre des races qui, ayant passé déjà par cette transformation première, sont en conséquence épuisées ? Le nouvel équilibre ne pourrait s’établir (je dis \emph{ne pourrait}, car l’exemple ne s’en est pas encore présenté dans l’histoire du monde) qu’en amenant non plus seulement la dégénération des multitudes, mais leur retour presque complet aux aptitudes normales de leur élément ethnique le plus abondant.\par
Cet élément ethnique le plus abondant, c’était pour l’Asie le noir. Les Chamites, dès les premières marches de leur invasion, l’avaient rencontré bien haut dans le nord, et probablement les Sémites, quoique plus purs, s’étaient, à leurs débuts, aussi laissé tacher par lui.\par
Plus nombreuses que toutes les émigrations blanches dont l’histoire ait fait mention, les deux premières familles venues de l’Asie centrale sont descendues si loin vers l’ouest et vers le sud de l’Afrique, que l’on ne sait encore où trouver la limite de leurs flots. Pourtant on peut attester, par l’analyse des langues sémitiques, que le principe noir a pris partout le dessus sur l’élément blanc des Chamites et de leurs associés.\par
Les invasions arianes furent, pour les Grecs comme pour leurs frères les Iraniens, peu fécondes en comparaison des masses plus d’aux deux tiers mélanisées dans lesquelles elles vinrent se plonger. Il était donc inévitable qu’après avoir modifié, pendant un temps plus ou moins long, l’état des populations qu’elles touchaient, elles se perdissent à leur tour dans l’élément destructeur où leurs prédécesseurs blancs s’étaient successivement absorbés avant elles. C’est ce qui arriva aux époques macédoniennes ; c’est ce qui est aujourd’hui.\par
Sous la domination des dynasties grecques ou hellénisées, l’épuisement, grand sans doute, était loin encore de ressembler à l’état actuel, amené par des mélanges ultérieurs d’une abondance extrême. Ainsi, la prédominance finale, fatale, nécessaire, de plus en plus forte, du principe mélanien a été le but de l’existence de l’Asie antérieure et de ses annexes. On pourrait affirmer que depuis le jour où le premier conquérant chamite se déclara maître, en vertu du droit de conquête, de ces patrimoines primitifs de la race noire, la famille des vaincus n’a pas perdu une heure pour reprendre sa terre et saisir du même coup ses oppresseurs. De jour en jour, elle y parvient avec cette inflexible et sûre patience que la nature apporte dans l’exécution de ses lois.\par
À dater de l’époque macédonienne, tout ce qui provient de l’Asie antérieure ou de la Grèce a pour mission ethnique d’étendre les conquêtes mélaniennes.\par
J’ai parlé des nuances persistant au sein de l’unité négative des Asiatiques et des hellénisants : de là, deux mouvements en sens contraire qui venaient encore augmenter l’anarchie de cette société. Personne n’étant fort, personne ne triomphait exclusivement. Il fallait se contenter du règne toujours chancelant, toujours renversé, toujours relevé d’un compromis aussi indispensable qu’infécond. La monarchie unique était impossible, parce qu’aucune race n’était de taille à la vivifier et à la faire durer. Il n’était pas moins impraticable de créer des États multiples, vivant d’une vie propre. La nationalité ne se manifestait en aucun lieu d’une façon assez tranchée pour être précise. On s’accom­modait donc de refontes perpétuelles de territoire ; on avait l’instabilité, et non le mouvement. Il n’y eut guère que deux courtes exceptions à cette règle : l’une causée par l’invasion des Galates ; la seconde par l’établissement d’un peuple plus important, les Parthes \footnote{Ils parlaient le pehlvi et y substituèrent ensuite le parsi, où affluèrent un plus grand nombre de racines sémitiques, résultant du long séjour des Arsacides à Ctésiphon et à Séleucie. Suivant Justin, le fond original est scythique ; mais les Scythes parlaient un dialecte arian. Le Mahabharata connaît les Parthes, qu’il nomme \emph{Parada}. Il les allie aux \emph{Saka} (Sacæ), certainement Mongols. Les Parthes donnent, par leur comparaison ethnique, une assez juste idée de ce que devaient être plusieurs races touraniennes.}, nation ariane mêlée de jaune, qui, sémitisée de bonne heure comme ses prédécesseurs, s’enfonça à son tour dans les masses hétérogènes.\par
En somme, cependant, les Galates et les Parthes étaient trop peu nombreux pour modifier longtemps la situation de l’Asie. Si une action plus vive de la puissance blanche n’avait pas dû se manifester, c’en était fait déjà, à cette époque, de l’avenir intellectuel du monde, de sa civilisation et de sa gloire. Tandis que l’anarchie s’établis­sait à demeure dans l’Asie antérieure, préludant avec une force irrésistible aux dernières conséquences de l’abâtardissement final, l’Inde allait de son côté, quoique avec une lenteur et une résistance sans pareilles, au-devant de la même destinée. La Chine seule continuait sa marche normale et se défendait avec d’autant plus de facilité contre toute déviation, que, parvenue moins haut que ses illustres sœurs, elle éprouvait aussi des dangers moins actifs et moins destructeurs. Mais la Chine ne pouvait représenter le monde ; elle était isolée, vivait pour elle-même, bornée surtout au soin modeste de régler l’alimentation de ses masses.\par
Les choses en étaient là quand, dans un coin retiré d’une péninsule méditerranéenne, une lueur commença à briller. Faible d’abord, elle s’accrut graduellement, et, s’étendant sur un horizon d’abord restreint, éclaira d’une aurore inattendue la région occidentale de l’hémisphère. Ce fut aux lieux mêmes où, pour les Grecs, le dieu Hélios descendait chaque soir dans la couche de la nymphe de l’Océan, que se leva l’astre d’une civilisation nouvelle. La victoire, sonnant de hautaines fanfares, proclama le nom du Latium et Rome se montra.
\chapterclose


\chapteropen
\chapter[{V. Civilisation européenne sémitisée}]{V. \\
Civilisation européenne sémitisée}\renewcommand{\leftmark}{V. \\
Civilisation européenne sémitisée}


\chaptercont
\section[{V.1. Populations primitives de l’Europe.}]{V.1. \\
Populations primitives de l’Europe.}
\noindent On a considéré longtemps comme impossible de découvrir entre le Bosphore de Thrace et la mer qui borde la Galice, et depuis le Sund jusqu’à la Sicile, un point quelconque où des hommes appartenant à la race jaune, mongole, ugrienne, finnoise, en un mot, à la race aux yeux bridés, au nez plat, à la taille obèse et ramassée, se soient jamais trouvés établis de manière à y former une ou plusieurs nations permanentes. Cette opinion, si bien acceptée qu’on ne l’a guère controversée que dans ces dernières années, ne reposait d’ailleurs sur aucune démonstration. Elle n’avait pas d’autre raison d’être qu’une ignorance à peu près absolue des faits concluants dont l’ensemble, aujourd’hui, la renverse et l’efface. Ces faits sont de différente nature, appartiennent à différents ordres d’observations, et le faisceau de preuves qu’ils composent est d’une complète rigueur \footnote{Schaffarik a été un des premiers à démontrer la présence primordiale et la diffusion des Finnois asiatiques en Europe ; mais il s’est borné à l’examen de la région septentrionale, en affirmant seulement que la race jaune était descendue beaucoup plus loin vers l’est et le sud qu’on ne le suppose généralement. (\emph{Slawische Alterthümer}, t. I, p. 88.) – Muller (\emph{Der ugrische Volksstamm}, t. I, p. 399) signale des traces d’établissements lapons dans la limite la plus méridionale de la Scandinavie et jusqu’à Schonen. ‑ Pott (\emph{Indogerm. Sprachstamm, Encycl. Ersch u. Gruber}, p. 23) pose en principe l’origine asiatique de toutes les tribus finnoises d’Europe, et pense que, dans des temps très anciens, cette famille s’étendait fort avant vers le sud. ‑ Rask mêle à des opinions plus hardies nombre d’assertions suspectes. ‑ Wormsaae est un des auteurs qui ont commencé avec beaucoup de sagacité et d’érudition à poser la question sur le véritable terrain.}.\par
Une certaine classe de monuments fort irréguliers, d’une antiquité très haute, et se montrant, à peu près, dans toutes les contrées de l’Europe, a depuis longtemps préoccupé les érudits. La tradition, de son côté, y rattache bon nombre de légendes. Ce sont tantôt des pierres brutes en forme d’obélisques dressées au milieu d’une lande ou sur le bord d’une côte, tantôt des espèces de boîtes de granit composées de quatre ou cinq blocs, dont un, deux au plus, servent de toiture. Ces blocs sont toujours de proportions gigantesques, et ne portent qu’exceptionnellement des traces de travail. Dans la même catégorie se rangent des amoncellements de cailloux souvent très considérables, ou des rochers posés en équilibre de manière à vibrer sous une très légère impulsion. Ces monuments, la plupart d’une forme extrêmement saisissante, même pour les yeux les plus inattentifs, ont engagé les savants à proposer plusieurs systèmes d’après lesquels il faudrait en faire honneur aux Phéniciens, ou bien aux Romains, peut-être aux Grecs, mieux encore aux Celtes, ou même aux Slaves. Mais les paysans, fidèles aux croyances de leurs pères, repoussent, sans le savoir, ces opinions si diverses, et adjugent les objets en litige aux fées et aux nains. On va voir que les paysans ont raison. Il en est des récits légendaires comme de la philosophie des Grecs, au jugement de saint Clément d’Alexandrie. Ce Père la comparait aux noix, âpres d’abord au goût du chrétien ; mais si l’on sait en briser l’écorce, on y trouve un fruit savoureux et nourrissant.\par
Les créations architecturales des Phéniciens, des Grecs, des Romains, des Celtes, ou même des Slaves n’offrent rien de commun avec les monuments dont il est ici question. On possède des œuvres de tous ces peuples à différents âges ; on connaît les procédés dont ils usaient : rien ne rappelle ce que nous avons ici sous les yeux. Puis, autre raison bien autrement puissante, et, même sans réplique, on rencontre des pierres debout, des cairns et des dolmens dans cent endroits où les conquérants de Tyr et de Rome, où les marchands de Marseille, où les guerriers celtes, où les laboureurs slaves n’ont jamais passé. Il faut donc envisager le problème à nouveau et de très près.\par
En partant de ce principe unanimement reconnu que toutes les antiquités de l’Europe occidentale ici mises en question sont, quant à leur style, antérieures à la domination romaine, on pose une base chronologique assurée, et l’on tient la clef du problème. J’insiste sur cette circonstance qu’il ne s’agit ici que de la date du style, et nullement de celle de la construction de tel monument en particulier, ce qui compliquerait la difficulté d’ensemble de beaucoup d’incertitudes de détail. Il faut s’en tenir d’abord à un exposé aussi général que possible, quitte à particulariser plus tard.\par
Puisque les armées des Césars occupaient la Gaule entière et une partie des îles Britanniques au premier siècle avant notre ère, le système générateur des antiquités gauloises et bretonnes remonte à des temps plus anciens. Mais l’Espagne aussi possède des monuments parfaitement identiques à ceux-là \footnote{Borrow, \emph{The Bible in Spain}, in-12, Lond., 1849, chap. VII, p. 35  : « Whilst toiling among « this wilds waste, I observed, a little way to my left, a pile of stones of rather a singular « appearance and rode up to it. It was a druidical altar and the most perfect and beautiful « one of the kind which I have never seen. It was circular, and consisted of stones « immensely larges and heavy at the bottom, which towards the top became thinner and « thinner, having been fashioned by the hand of art to something of the shape of scallop « shells. These were surmounted by a very large flat stone, which slanted down towards « the earth, where was a door. » ‑ Bien peu d’observations ont été faites en Espagne sur cette classe de monuments. M. Mérimée a visité cependant, près d’Antequera, un souterrain clairement marqué des caractères pseudo-celtiques.}. Or les Romains ont pris possession de cette contrée longtemps avant de s’établir dans les Gaules, et, avant eux, les Carthaginois et les Phéniciens y avaient jeté d’abondantes importations de leur sang et de leurs idées. Les peuples qui ont érigé les dolmens espagnols ne sauraient donc les avoir imaginés postérieurement à la première migration ou colonisation phénicienne. Pour ne pas déroger à une prudence même excessive, il est bon de ne pas user de cette certitude dans toute son étendue. Ne remontons pas plus haut que le troisième siècle avant Jésus-Christ.\par
Il faut être plus hardi en Italie. Nul doute que les constructions semblables aux monuments gaulois et espagnols qu’on y trouve ne soient antérieures à la période romaine, et, qui plus est, à la période étrusque. Les voilà repoussées du troisième siècle au huitième à tout le moins.\par
Mais, parce que les antiquités que nous venons d’apercevoir dans les îles Britanniques, la Gaule, l’Espagne et l’Italie, dérivent d’un type absolument le même, elles inspirent naturellement la pensée que leurs auteurs appartenaient à une même race. Aussitôt que cette idée se présente, on veut en éprouver la valeur en calculant la diffusion de cette race d’après celle des monuments qui révèlent son existence. On cesse donc de se tenir renfermé dans les quatre pays nommés ci-dessus, et l’on cherche, au dehors de leurs limites, si rien de semblable à ce qu’ils contiennent ne se peut rencontrer ailleurs. On arrive à un résultat qui d’abord effraie l’imagination.\par
La zone ouverte alors aux regards s’étend depuis les deux péninsules méridionales de l’Europe, en couvrant la Suisse, la Gaule et les îles Britanniques, sur toute l’Allemagne, enveloppe le Danemark et le sud de la Suède, la Pologne et la Russie, traverse l’Oural, embrasse la haute Sibérie, passe le détroit de Behring, enferme les prairies et les forêts de l’Amérique du Nord, et va finir vers les rives du Mississipi supérieur, si toutefois elle ne descend pas plus bas \footnote{Keferstein, \emph{Ansichten über die keltischen Alterthümer}, t. I, pass. ‑ Ouvrage qui témoigne des plus laborieuses recherches et du plus grand dévouement à la science. C’est un véritable et indispensable manuel pour la connaissance des antiquités primitives. ‑ Wormsaae, \emph{The Primeval Antiquities of Denmark}, translated by W. J. Thoms, Lond., in-8°, 1849. ‑ Schaffarik, \emph{Slawische Alterthümer}, t. I. ‑ Squier, \emph{Observations on the Aboriginal Monuments of the Mississipi Valley}, New-York, 1847. ‑ Abeken, \emph{Mittel Italien vor der Zeit der rœmischen Herrschatt}, Stuttgart u. Tübingen, etc., 1843. ‑ Dennis, \emph{Die Stædte und Begræbnisse Etruriens}, deutsch von Meissner, in-8°, Leipzig, 1852, t. I, pass., etc., etc. ‑ Pour ce qui concerne les monuments de la Suisse, je dois beaucoup aux obligeantes communications de M. Troyon, dont les investigations si habiles et si patientes agrandissent tous les jours le champ de l’archéologie primitive.}.\par
On conviendra que, s’il fallait adjuger soit aux Celtes, soit aux Slaves, pour ne parler ni des Phéniciens, ni des Grecs, ni des Romains, une si vaste série de régions, on devrait, en même temps, s’attendre à rencontrer toutes les autres catégories d’anti­quités que ces pays recèlent aussi identiques entre elles que le sont les monuments dont l’abondance conduit à tracer ces vastes limites. Que les aborigènes de tant de contrées aient été des Celtes ou des Slaves, ils auront laissé partout des restes de leur culture, aisément comparables à ceux que l’on décrit en France, en Angleterre, en Allemagne, en Danemark, en Russie, et que l’on sait, de science certaine, ne pouvoir être attribués qu’à eux. Mais, précisément, cette condition n’est pas remplie.\par
Sur les mêmes terrains que les constructions de pierre brute, abondent des dépôts de toute nature, gages de l’industrie humaine, qui, différant entre eux d’une manière radicale de contrée à contrée, accusent, d’une manière évidente, l’existence sporadique de nationalités très distinctes et auxquelles ils ont appartenu. De sorte que l’on contemple dans les Gaules des restes complètement étrangers à ceux des pays slaves, qui le sont à leur tour à des produits sibériens, comme ceux-ci à des produits américains.\par
Incontestablement donc l’Europe a possédé, avant tout contact avec les nations cultivées des rives de la Méditerranée, Phéniciens, Grecs ou Romains, plusieurs couches de populations différentes, dont les unes n’ont tenu que certaines provinces du continent, tandis que d’autres, ayant laissé partout des traces semblables, ont bien évidemment occupé la totalité du pays, et cela à une époque très certainement antérieure au huitième siècle avant Jésus-Christ.\par
La question qui se présente maintenant, c’est de savoir quelles sont les plus anciennes des diverses classes d’antiquités primitives, ou de celles qui sont sporadi­ques, ou de celles qui sont répandues partout.\par
Celles qui sont sporadiques accusent un degré d’industrie, de connaissances techniques et de raffinement social fort supérieur à celles qui occupent le plus vaste espace. Tandis que ces dernières ne montrent qu’exceptionnellement la trace de l’emploi des instruments de métal, les autres offrent deux époques où le bronze, puis le fer, se présentent sous les formes les plus habilement variées ; et ces formes, appli­quées comme elles le sont, ne peuvent pas laisser le moindre doute qu’elles n’aient été la propriété ici des Celtes, là des Slaves ; car le témoignage de la littérature classique exclut toute hésitation.\par
Conséquemment, puisque les Celtes et les Slaves sont d’ailleurs les derniers propriétaires connus de la terre européenne antérieurement au huitième siècle qui précéda notre ère, les deux périodes appelées par d’habiles archéologues les âges de bronze et de fer s’appliquent aussi à ces peuples. Elles embrassent les derniers temps de l’antiquité primordiale de nos contrées, et il faut reporter par delà leurs limites une époque plus ancienne, justement qualifiée d’âge de pierre par les mêmes classi­ficateurs \footnote{ \noindent Wormsaae, \emph{The Primeval Antiquities of Denmark}, p. 8.
 }. C’est à celle-là qu’appartiennent les monuments objets de notre étude.\par
Un point subsiste encore qui pourrait sembler obscur. L’habitude enracinée de ne rien apercevoir en Europe avant les Celtes et les Slaves peut induire certains esprits à se persuader que les trois âges de pierre, de bronze et de fer ne marquent que des gradations dans la culture des mêmes races. Ce seraient les aïeux encore sauvages des habiles mineurs, des artisans industrieux dont maintes découvertes récentes font admirer les œuvres, qui auraient produit les monuments bruts de la plus lointaine période. On s’expliquerait tant de barbarie par un état d’enfance sociale, encore ignorant des ressources techniques créées plus tard.\par
Une objection sans réplique renverse cette hypothèse d’ailleurs foncièrement inadmissible pour bien d’autres motifs \footnote{Keferstein\emph{, Ansichten}, t. I, p. 451 : « Si l’on observe la marche de la science et de l’art en « Europe, on n’aperçoit nulle part un développement graduel, mais bien une sorte de « fluctuation, et la condition des choses s’élève ou s’abaisse comme les flots de la mer. « Certaines circonstances amènent un progrès, d’autres une déchéance. Il est impossible « de découvrir aucune trace du passage des peuples complètement sauvages à l’état de « bergers et de chasseurs, puis d’habitants sédentaires, puis enfin d’agriculteurs et « d’artisans. Si haut que nous remontions dans les temps primitifs, au delà des périodes « héroïques, nous trouvons que les nations sédentaires et sociables ont été, de tout temps, « pourvues de ce caractère. » ‑ J’ai eu occasion, a la fin du deuxième livre de cet ouvrage, de démontrer l’exactitude de cette assertion ; comme elle va à l’encontre des opinions vulgaires, je ne me lasse pas de l’appuyer de témoignages imposants.}. Entre l’âge de bronze et l’âge de fer, il n’y a de différence que la plus grande variété des matières employées et la perfection croissante du travail. La pensée dirigeante ne change pas ; elle se continue, se modifie, se raffine, passe du bien au mieux, mais en se maintenant dans les mêmes données. Tout au contraire, entre les productions de l’âge de pierre et celles de l’âge de bronze, on relève, au premier coup d’œil, les contrastes les plus frappants ; pas de transition des unes aux autres, quant à l’essentiel : le sentiment créateur se transforme du tout au tout. Les instincts, les besoins auxquels il est satisfait, ne se correspondent pas. Donc l’âge de pierre et l’âge de bronze ne sont point dans les mêmes rapports de cohésion où ce dernier se trouve avec l’âge de fer \footnote{Wormsaae, \emph{The Primeval Antiquities of Denmark}, p. 124 et sqq.}. Dans le premier cas, il y a passage d’une race à une autre, tandis que, dans le second, il n’y a qu’un simple progrès au sein de races, sinon complètement identiques, du moins très près parentes. Or il n’est pas douteux que les Slaves sont établis en Europe depuis quatre mille ans au moins. D’autre part, les Celtes combattaient sur la Garonne au dix-huitième siècle avant nette ère. Nous voilà donc arrivés pied à pied à cette conviction, résultat mathématique de tout ce qui précède : les monuments de l’âge de pierre sont antérieurs, quant à leur style, à l’an 2000 avant J.-C. ; la race particulière qui les a construits occupait les contrées où on les trouve avant toute autre nation ; et comme, d’ailleurs, ils se présentent en plus grande abondance à mesure que l’observateur, quittant le sud, s’avance davantage vers le nord-ouest, le nord et le nord-est, cette même race était plus primitivement encore et, en tout cas, plus solidement souveraine dans ces dernières régions. Si l’on veut fixer d’une manière approximative l’époque probable de l’apogée de sa force, rien ne s’oppose à ce que l’on accepte la date de 3000 ans avant J.-C., proposée par un antiquaire danois, aussi ingénieux observateur que savant profond \footnote{Wormsaae, \emph{ouvr. cité}, p. 135: « If the Celts possessed settled abodes in the west of Europe « more than two thousand years ago, how much more ancient must be the populations « which preceded the arrival of the Celts ? A great number of years must pass away « before a people like the Celts could spread themselves in the west of Europe and render « the land productive. It is therefore no exaggeration if we attribute to the stone period an « antiquity of, at least, three thousand years. »}.\par
Ce qui reste maintenant à déterminer d’une manière positive, c’est la nature ethnique de ces populations primordiales si largement répandues dans notre hémis­ phère. Bien certainement elles se rattachent de la façon la plus intime aux groupes divers de l’espèce jaune, généralement petite, trapue, laide, difforme, d’une intelligence fort limitée, mais non nulle, grossièrement utilitaire et douée d’instincts mâles très prédominants \footnote{Je me suis étendu suffisamment ailleurs sur les traits caractéristiques de la race jaune, quant à ce qui est du domaine de la physiologie. Le tableau dressé par M. Morton donne tous les résultats désirables quant à la valeur comparative de cette race à l’égard des deux autres.}.\par
L’attention s’est portée récemment, en Danemark \footnote{Moniteur universel du 14 avril 1853, n° 104, Mérimée, Sur les Antiquités prétendues celtiques. ‑ Munch, Det norske Folkshistorie, deutsch von Claussen, in-8°, Lubeck, 1853, p. 3.} et en Norvège, sur d’énormes amoncellements d’écailles d’huîtres et de coquillages, mêlés de couteaux en os et en silex fort brutalement travaillés. On exhume aussi de ces détritus des squelettes de cerfs et de sangliers, d’où la moelle a été enlevée par fracture. M. Wormsaae, en analysant cette découverte, regrette que des recherches analogues à celles qui l’ont amenée n’aient pas eu lieu jusqu’ici sur les côtes de France. Il ne doute pas qu’il n’en dût sortir des observations semblables à celles qu’il a eu l’occasion de faire dans sa patrie, et il pense surtout que la Bretagne serait explorée avec grand avantage. Il ajoute : « Tout le monde sait combien ces amas de « coquillages et d’os sont fréquents en Amérique. Ils renferment des instruments « non moins grossiers (que ceux que l’on a trouvés dans les détritus danois et « norwégiens), et attestent le séjour des anciennes peuplades aborigènes. »\par
Ces monuments sont d’un genre si particulier, et si peu propre à frapper les yeux et à attirer l’attention, qu’on s’explique sans peine l’obscurité qui les a si longtemps couverts. Le mérite n’en est que plus grand pour les observateurs auxquels la science est redevable d’un présent, certes bien curieux, puisqu’il en résulte au moins une forte présomption que le nord de l’Europe possède des traces identiques à celles qu’offrent encore les plages du nouveau monde dans le voisinage du détroit de Behring. Il permet aussi de commenter une autre trouvaille du même genre, plus intéressante encore, faite, il y a peu de mois, aux environs de Namur. Un savant belge, M. Spring, a retiré d’une grotte à Chauvaux, village de la commune de Godine, un amas de débris doublement enterrés sous une couche de stalagmite et sous une autre de limon, parmi lesquels il a reconnu des fragments d’argile calcinée, du charbon végétal, puis des os de bœufs, de moutons, de porcs, de cerfs, de chevreuils, de lièvres, enfin de femmes, de jeunes hommes et d’enfants. Particularité curieuse qui se remarque aussi dans les détritus du Danemark et de la Norwège : tous les os à moelle sont rompus, aussi bien ceux qui ont appartenu à des individus de notre espèce que les autres, et M. Spring en conclut avec raison que les auteurs de ce dépôt comestible étaient anthropophages \footnote{\emph{Moniteur universel du} 18 mars 1854, n° 77. \emph{Communication faite par M. Spring à l’Académie royale de Belgique.}}. C’est là un goût étranger à toutes les tribus de la famille blanche, même les plus farouches, mais très fréquemment constaté chez les nations américaines.\par
Passant à un autre genre d’observations, on trouve comme objets remarquables certains tumulus de terre qui, par la rudesse de leur construction, n’ont rien de commun avec les sépultures arianes de la haute Asie, pas plus qu’avec ces tombeaux somptueux que l’on peut observer encore dans la Grèce, dans la Troade, dans la Lydie, dans la Palestine, et qui témoignent, sinon d’un goût artistique très raffiné chez leurs construc­teurs, du moins d’une haute conception de ce que sont la grandeur et la majesté \footnote{Von Prokesch Osten, \emph{Kleine Schriften, die Tumuli der Alten}, t. V, p. 317.}. Ceux dont il s’agit ici ne consistent, comme il vient d’être dit, qu’en simples accumulations de glaise ou de terre crayeuse, suivant la qualité du sol qui les porte. Cette enveloppe renferme des cadavres non brûlés, ayant à leurs côtés quelques tas de cendres \footnote{On considère généralement l’absence d’incinération des os comme un des caractères auxquels se peuvent reconnaître les sépultures finniques, car les Celtes et les Slaves brûlaient leurs morts. L’observation est juste, elle ne saurait néanmoins servir à fixer l’âge du monument où l’on trouve à l’appliquer. M. Troyon veut bien me communiquer à cet égard une opinion que je crois devoir consigner ici : « Je crois », m’écrivait ce savant, qu’on « peut poser en fait que les premiers habitants de l’Europe ont inhumé leurs morts sans les « brûler. Plus tard, dans l’âge de bronze, l’ustion est générale, mais bien des familles de la « race primitive ont poursuivi leur ancien mode de sépulture. C’est ainsi que, dans le « canton de Vaud, on rencontre tous les instruments en bronze, des tumuli, anneaux, « poignards, celts, épingles, etc., dans des tombes construites sous la surface du sol, « auprès de squelettes reployés ou étendus sur le dos. Le même fait se retrouve en quelques « parties de l’Allemagne et de l’Angleterre, et on le remarquera dans bien d’autres contrées « quand les observations seront complètes. »}. Souvent le corps paraît avoir été déposé sur un lit de branchages. Cette circonstance rappelle le fagot sépulcral des aborigènes de la Chine. Ce sont là des sépultures bien élémentaires, bien sauvages. Elles ont été rencontrées un peu partout au sein des régions européennes. Or des constructions toutes semblables, offrant les mêmes particularités, couvrent également la vallée supérieure du Mississipi. M. E.-G. Squier affirme que les squelettes enfouis dans ces tombes sont tellement fragiles que le moindre contact les résout en poussière. C’est pour lui un motif d’attribuer à ces cadavres et aux monuments qui les renferment une excessive antiquité \footnote{E. G. Squier, \emph{ouvr. cité}.}.\par
De tels tumulus, toujours semblables, érigés en Amérique, dans le nord de l’Asie et en Europe, viennent renforcer l’idée que ces contrées ont été possédées jadis par la même race, qui ne saurait être que la race jaune. Ils sont partout voisins de longs remparts de terre, quelquefois doubles et triples, couvrant des espaces de plusieurs milles en ligne droite. Il en existe de tels entre la Vistule et l’Elbe, dans l’Oldenbourg, dans le Hanovre. M. Squier donne sur ceux de l’Amérique du Nord des détails tellement précis, et, ce qui vaut mieux, des dessins si concluants, que l’on ne peut conserver le plus léger doute sur l’identité complète de la pensée qui a présidé à ces systèmes de défense.\par
On doit inférer de ces faits suffisamment nombreux et concordants :\par
Que les populations jaunes venant d’Amérique et accumulées dans le nord de l’Asie, ont jadis débordé sur l’Europe entière, et que c’est à elles qu’il faut attribuer l’ensemble de ces monuments grossiers de terre ou de pierre brute qui témoignent partout de l’unité de la population primordiale de notre continent. Il faut renoncer à voir dans de telles œuvres des résultats qui n’ont pu sortir de la culture sporadique, et d’ailleurs bien connue aujourd’hui pour avoir été plus développée, des nations celtiques et des tribus slaves. Ce point établi, il reste encore à suivre la marche des peuples finnois vers l’occident pour apercevoir, avec les moyens d’action dont ils disposaient, le détail des travaux qu’ils ont exécutés et qui nous étonnent aujourd’hui. Ce sera, en même temps, reconnaître les traits principaux de la condition sociale où se trouvaient les premiers habitants de notre terre d’Europe.\par
Cheminant avec lenteur à travers les steppes et les marais glacés des régions septentrionales, leurs hordes avaient devant elles un chemin le plus souvent plane et facile. Elles suivaient les bords de la mer et le cours des grands fleuves, lieux où les forêts étaient clairsemées, où les rochers et les montagnes s’abaissaient et livraient passage. Dénués de moyens énergiques pour se frayer des routes à travers des obstacles trop puissants, ou du moins n’en pouvant user qu’avec une grande dépense de temps et de forces individuelles, elles n’appliquaient à l’usage journalier que des haches de silex mal emmanchées d’une branche d’arbre. Pour opérer leur navigation côtière dans l’océan Arctique ou le long des rives fluviales, ou encore dans les contrées coupées de grands marécages, elles usaient de canots formés d’un unique tronc d’arbre abattu et creusé au feu, puis dégrossi tant bien que mal à l’aide de leurs instruments imparfaits. Les tourbières d’Angleterre et d’Écosse recelaient et ont livré à la curiosité moderne quelques-uns de ces véhicules. Plusieurs sont garnis à leurs extrémités de poignées en bois, destinées à faciliter le portage. Il en est un qui ne mesure pas moins de trente-cinq pieds de longueur.\par
On vient de voir que, lorsqu’il s’agissait de jeter à bas quelques arbres, les Finnois employaient le procédé encore en usage aujourd’hui chez les peuplades sauvages de leur continent natal. Les bûcherons pratiquaient de légères entailles dans un tronc de chêne ou de sapin, au moyen de leurs haches de silex, et suppléaient à l’insuffisance de ces outils par une application patiente de charbons enflammés introduits dans les trous ainsi préparés \footnote{Wormsaae, \emph{ouv. cité}, p. 13. Ceci n’est point une hypothèse, mais une observation confirmée par les faits.}.\par
À en juger d’après les vestiges aujourd’hui existants, les principaux établissements des hommes jaunes ont été riverains de la mer et des fleuves. Mais cette donnée ne saurait cependant fournir une règle sans exception. On rencontre des traces finniques assez nombreuses et fort importantes dans l’intérieur des terres. M. Mérimée, éclair­cissant ce point, a fort judicieusement signalé l’existence de monuments de ce genre dans le centre de la France \footnote{Moniteur universel du 14 avril 1853. Il s’agit de la Marche, du pays chartrain, du Vendômois, du Limousin, etc.}. On en constate plus loin encore. Les émigrants de race jaune primitive ont connu, en fait de pays d’un accès difficile, les solitudes des Vosges, les vallées du Jura, les bords du Léman. Leur séjour dans ces différentes parties de l’intérieur est attesté par des vestiges qui ne sauraient provenir que d’eux. On en reconnaît même d’une manière certaine dans quelques parties du nord de la Savoie \footnote{Keferstein, \emph{Ansichten}, t. I, p. 173 et 183. ‑ \emph{Mémoires et documents de la Société d’histoire et d’archéologie de Genève}, in-8°‑, 1847, t. V, p. 498 et pass.}, et les habiles recherches de M. Troyon sur des habitations très antiques, ensevelies aujourd’hui sous les eaux de plusieurs lacs de la Suisse, mettront probablement un jour hors de doute que les pêcheurs finnois avaient placé jusque sur les rives du lac de Zurich les pilotis de leurs misérables cabanes \footnote{Cette découverte est toute récente. Elle a eu lieu cette année, d’abord à Meilen, canton de Zurich, ensuite sur le lac de Bienne près de Nidau, enfin sur les lacs de Genève et de Neuchâtel. Ces restes consistent en pilotis qui portaient autrefois des habitations construites au-dessus de la surface de l’eau. On y trouve de nombreux fragments de poterie, et même des petits vases intacts, des ossements d’animaux, des charbons, des pierres destinées à moudre et à broyer, etc. Comme on y rencontre aussi çà et là quelques débris de bronze, il est à présumer que ces habitations datent de la période où les Celtes étaient déjà arrivés dans le pays. ‑ Je dois ces communications à M. Troyon.}.\par
Il convient de donner rapidement une nomenclature des principales espèces de débris qui ne peuvent avoir appartenu qu’aux aborigènes de race jaune, de ces débris que les archéologues du Nord considèrent unanimement comme portant le cachet de l’âge de pierre. Déjà j’ai cité les amoncellements de coquillages comestibles, d’os de quadrupèdes et d’êtres humains, mêlés de couteaux de pierre, d’os et de corne ; j’ai encore mentionné les haches, les marteaux de silex, les canots formés d’un seul tronc d’arbre, et les vestiges d’habitations sur pilotis qui viennent, pour la première fois, d’être observées sur les rives de plusieurs lacs helvétiques. À ce fond, on doit ajouter des têtes de flèches en caillou ou en arête de poisson, des pointes de lance et des hameçons pour la pêche en mêmes matières, des boutons destinés à assujettir des vêtements de peaux, des morceaux d’ambre, ou percés ou bruts, des boules d’argile teintes en rouge pour être enfilées et servir de colliers \footnote{Wormsaae, \emph{ouvr. cité}, p. 17 et pass. ‑ Keferstein, t. I, p. 314. ‑ Un beau dolmen, découvert à La Motte-Sainte-Héraye (Loire-Inférieure), en 1840, contenait, entre autres objets, un de ces colliers de terre cuite.}, enfin des poteries souvent fort grandes, puisqu’il en est qui servent de ’bières à des cadavres entiers, aux côtés desquels paraissent avoir été déposés des aliments.\par
Mais ce qui domine tout le reste, ce sont les productions architectoniques, côté surtout frappant de ces antiquités. Leur trait principal et dominant, celui qui crée leur style particulier, c’est l’absence complète, absolue, de maçonnerie. Dans ce mode de construction, il n’est fait usage que de blocs toujours considérables. Tels sont les menhirs, ou peulvens, appelés en Allemagne \emph{Hunenstein}e \footnote{Keferstein, \emph{ouvr. cité}, t. I, p. 265. Le mot \emph{Huns} ne signifie pas les Huns, comme on le croit généralement ; il vient du celtique \emph{hen, ancien, vieux}, ou de \emph{hun, le dormeur}. Il a passé dans le frison avec le sens de \emph{mort}. Ainsi \emph{Hunensteine} doit se traduire par pierres des anciens, des dormeurs, ou des morts. Peut-être faut-il appliquer cette observation à plus d’un passage de Sigebert et des chroniques gaëliques, où l’intervention des Huns, en tant que cavaliers d’Attila, est tout à fait absurde. ‑ Dieffenbach, \emph{Celtica II}, 2\textsuperscript{e} Abth., p. 269. Voir une citation de Fordun où l’Humber s’appelle \emph{Hunne}, et où le prince mythique Humber est nommé \emph{Rex Hynorum}. (\emph{Loc. cit}., p. 267). ‑ On trouve aussi dans Geoffroy de Monmouth, II, 1 : « Applicuit Humber, tex Hunnorum, in Albaniam. » ‑Les traditions germaniques, en se mêlant aux fables indigènes, n’ont pas hésité à déposer dans le mot \emph{hun} des souvenirs qui leur étaient très présents, et, par suite, à intercaler le nom d’Attila dans les généalogies irlando-milésiennes.} ; les obélisques de pierre brute, d’une hauteur plus ou moins grande, enfoncés dans le sol, ordinairement jusqu’au quart de leur élévation totale ; les cromlechs, \emph{Hunenbette}, cercles ou carrés formés par des séries de blocs posés à côté les uns des autres, et embrassant un espace souvent assez étendu. Ce sont encore des dolmens, lourdes cases, construites de trois ou quatre fragments de rocher accotés à angle droit, recouverts d’une cinquième masse, pavées en cailloux plats et quelquefois précédées d’un corridor de même style. Souvent ces monstrueuses masures sont ouvertes d’un côté ; dans d’autres cas, elles ne présentent pas d’issue. Ce ne peut être que des tombeaux. Sur certains points de la Bretagne, on les compte par groupes de trente à la fois ; le Hanovre n’en est pas moins richement pourvu\footnote{Moniteur universel déjà cité. M. Mérimée démontre le fait par une série d’arguments incontestables.}. La plupart contiennent ou contenaient, au moment où elles furent découvertes, des squelettes non brûlés.\par
Autant par leur masse, qui en fait le monument le plus apparent qu’ait produit la race finnoise, que par les débris qu’ils contiennent, les dolmens doivent être considérés comme un des témoignages les plus concluants de la présence des peuplades jaunes sur un point donné. Les fouilles les plus minutieuses n’ont jamais pu y faire apercevoir d’objets en métal, mais seulement ces sortes d’outils ou d’ustensiles, aussi élémentaires par la matière que par la forme, qui ont été énumérés plus haut. Les dolmens ont encore un caractère précieux, c’est leur vaste diffusion. On en connaît dans toute l’Europe.\par
Viennent maintenant les cairns, qui ne sont guère moins communs. Ce sont des amas de pierres de différentes dimensions \footnote{Keferstein, \emph{ouvr. cité}, t. I, p. 132. Cet auteur dénombre ainsi les monuments pseudo-celtiques du Hanovre : 290 constructions de pierre, 350 groupes de terre, 135 tumulus isolés, 65 remparts, etc. Il arrive au chiffre de 7 000.}. Plusieurs recèlent un cadavre, toujours non brûlé, avec quelques objets d’os ou de silex. Il est des exemples où le corps est déposé sous un petit dolmen érigé au centre du cairn \footnote{Très fréquemment le cadavre n’est pas posé à plat, mais assis et la tête reposant sur les genoux repliés. Cette coutume est extrêmement répandue chez les aborigènes américains. – Wormsaae, \emph{ouvr. cité}, p. 89.}. On voit aussi tel de ces monuments qui est à base pleine et ne semble avoir eu qu’une destination purement commémorative ou indicative. Il en est de fort petits, mais aussi d’énormes : celui de New-Grange, en Irlande, représente une masse de quatre millions de quintaux.\par
La combinaison du dolmen et du cairn n’est qu’une imitation, souvent suggérée par la nature du terrain, d’une réunion semblable du dolmen et du tumulus \footnote{Le cairn n’a guère été mis en usage que dans les contrées pierreuses. On en voit beaucoup dans le sud-ouest de la Suède, tandis qu’il ne s’en rencontre aucun en Danemark. ‑ Wormsaae, \emph{ouvr. cité}, p. 107.}. On signale des spécimens de cette espèce un peu partout, entre autres dans le Latium, près de Civita-Vecchia, à vingt-deux milles de Rome, non loin de l’ancienne Alsium et de Santa-Marinella. Il en est encore un à Chiusa, un autre près de Pratina, sur l’emplace­ment de Lavinium \footnote{Suivant Varron, toute chambre sépulcrale marquée des caractères du dolmen a été primitivement recouverte d’un tumulus de terre, détruit postérieurement. Ce passage est des plus importants pour établir l’existence des hordes finniques en Italie. ‑ Abeken, \emph{ouvr. cité}, p. 241.}.\par
Les squelettes tirés des dolmens ont permis de constater, chez les premiers habitants de la terre d’Europe, certains talents qu’assurément on n’aurait pas été enclin, \emph{a priori}, à leur supposer. Ils savaient pratiquer plusieurs opérations chirurgicales. Déjà les tumulus américains en avaient offert la preuve en livrant aux observateurs des têtes renfermant des dents fausses. Un dolmen ouvert récemment, près de Mantes, a fourni le corps d’un homme adulte dont le tibia, fracturé en flûte, présente une soudure artificielle.\par
Il est d’autant plus curieux de rencontrer chez la race jaune ce genre de savoir, que, parmi les descendants purs ou métis de la variété mélanienne, on n’en aperçoit pas vestige aux époques correspondantes. L’art de soulager les souffrances n’est guère allé, chez ces derniers, au delà de l’usage des simples et des topiques extérieurs. L’intérieur du corps humain et sa structure leur étaient complètement inconnus. C’est la suite de l’horreur que leur inspiraient les morts, horreur toute d’imagination, née des craintes superstitieuses qui ont de longtemps précédé le respect, et qui empêchait toute curio­sité de s’aventurer dans un domaine jugé redoutable. Au contraire, les jaunes, défendus par leur tempérament flegmatique contre l’excès des impressions de ce genre, envisagèrent très peu solennellement les dépouilles de leurs conquêtes. L’anthropo­phagie leur fournissait toutes les occasions désirables de s’instruire sur l’ostéologie de l’homme. Le soin même de leur sensualité en les portant à étudier la nature des os, afin de savoir, à point nommé, où trouver la moelle, leur procurait l’expérience pratique. C’est ainsi que se montrent si savants les habitants actuels de la Sibérie méridionale. Leurs connaissances anatomiques, en ce qui concerne les différentes catégories d’animaux, sont aussi sûres que détaillées \footnote{Huc, \emph{Souvenirs d’un voyage dans la Tartarie, le Thibet et la Chine}, t. II.}.\par
De l’habitude de voir des squelettes, de les manier, de les rompre, à l’idée de raccommoder un membre brisé ou de remplir un alvéole, le passage est extrêmement court. Il ne faut ni une intelligence extraordinaire ni un degré de culture générale bien avancé pour le franchir. Néanmoins il est intéressant de constater que les Finnois le savaient faire, parce qu’on s’explique ainsi un fait resté jusqu’à présent énigmatique, le plombage des dents malades chez les plus anciens Romains, habitude à laquelle fait allusion un article de la loi des XII Tables. Ce procédé médical, inconnu aux populations de la Grande-Grèce, provenait des tribus sabines ou des Rasênes, qui ne pouvaient l’avoir reçu que des anciens possesseurs jaunes de la péninsule. Voilà comment le bien sort du mal, et comment l’ostéologie, avec ses applications bienfai­santes, a sa source première dans l’anthropophagie.\par
Si l’on a quelque droit de s’étonner d’avoir pu tirer de pareilles conclusions de l’examen des squelettes trouvés dans les dolmens, on était fondé à en attendre les moyens de préciser physiologiquement le caractère ethnique des populations auxquel­les ils ont appartenu. Malheureusement les résultats obtenus jusqu’ici n’ont pas justifié cette espérance : ils sont des plus pauvres.\par
Pour première difficulté, on a peu de corps entiers. Le plus souvent les cadavres, altérés par des accidents inévitables, à la suite de si longs siècles d’inhumation, n’offrent qu’un objet d’examen fort incomplet. Trop fréquemment aussi, les explo­rateurs, ignorants ou maladroits, ne les ont pas assez ménagés en pénétrant dans leurs asiles. Bref, jusqu’à ce jour, la physiologie n’a rien ajouté de bien concluant aux preuves offertes par d’autres ordres de connaissances touchant le séjour primordial des Finnois sur toute la surface du continent d’Europe. Comme cette science n’est pas non plus parvenue à démontrer l’identité typique des squelettes trouvés en différents lieux, elle ne peut servir même à reconnaître si l’ancienne population a été ou non bien nombreuse. Pour se former une opinion à cet égard, il faut revenir aux témoignages fournis par les monuments que d’ailleurs on trouve en si étonnante abondance.\par
Déjà l’ubiquité du dolmen tendait à établir que les envahisseurs avaient pénétré jusque dans le centre, jusque dans les régions montagneuses de notre partie du monde. Mal pourvus des moyens matériels de rendre ces invasions faciles, ils n’ont dû y être déterminés que par une surabondance de nombre qui leur a rendu impossible de continuer à vivre tous agglomérés sur les premiers points de débarquement.\par
Cette induction puissante est renforcée encore par un argument direct, argument matériel qui saisit la conviction de la manière la plus forte, en augmentant la liste des monuments finniques de la description du plus vaste, du plus étonnant dont on ait encore eu connaissance \footnote{F. de Saulcy, \emph{Notice sur une Inscription découverte à Marsal}, Paris, in-8°, 1846. Se trouve aussi dans les \emph{Mémoires} de l’Académie des inscriptions. ‑ Ce travail n’est pas un des moins ingénieux ni des moins sagaces du savant académicien.}.\par
La vallée de la Seille, en Lorraine, occupée aujourd’hui par les villes de Dieuze, de Marsal, de Moyenvic et de Vic, ne formait, avant que l’homme y eût mis les pieds, qu’un immense marécage boueux et sans fond, créé et entretenu par une multitude de sources salines, qui, perçant de toutes parts sous la fange, ne laissaient pas un endroit stable et solide. Entouré de hauteurs, ce coin de pays était, en outre, aussi peu accessible qu’habitable. Une horde finnoise jugea qu’il lui serait possible de s’y faire une retraite à l’abri de toutes les agressions, si elle réussissait à y créer un terrain capable de la porter.\par
Pour y parvenir, elle fabriqua, avec l’argile des collines environnantes, une immense quantité de morceaux de terre pétris à la main. On retrouve encore aujourd’hui, sur ceux de ces fragments que l’on exhume de la vase, les traces recon­naissables de doigts d’hommes, de femmes et d’enfants. Quelquefois, pour abréger sa besogne, l’ouvrier sauvage s’est avisé de prendre un bloc de bois et de le recouvrir d’une faible couche de glaise. Tous ces fragments ainsi préparés furent ensuite soumis à l’action du feu et transformés en briques on ne peut plus irrégulières, dont les plus grandes, qui sont aussi les plus rares, ont environ 25 centimètres de circonférence sur une longueur à peu près égale. La plupart n’ont que des dimensions beaucoup plus faibles.\par
Les matériaux ainsi préparés furent transportés dans le marais, et jetés pêle-mêle sur la boue, sans mortier ni ciment. Le travail s’étendit de telle manière que le radier artificiel, recouvert aujourd’hui d’une couche de vase solidifiée de sept à onze pieds de profondeur, a, dans ses parties les plus minces, trois pieds de hauteur, et dans les plus épaisses sept environ. Ainsi fut créé sur l’abîme une espèce de croûte que le temps a rendue très compacte, et qui est évidemment très solide, puisqu’on la voit porter plusieurs villes, habitées par une population totale de vingt-neuf à trente mille âmes.\par
L’étendue de cet ouvrage bizarre, connu dans le pays sous le nom de \emph{briquetage de Marsal}, paraît être, autant que les sondages exécutés au dernier siècle par l’ingénieur La Sauvagère ont pu le faire connaître, de cent quatre-vingt-douze mille toises carrées sous la ville de Marsal, et de quatre-vingt-deux mille quatre cent quatre-vingt-dix-neuf toises sous Moyenvic.\par
En comparant entre elles les différentes mesures, M. de Saulcy a calculé approximativement, et en ayant soin de modérer, même à l’extrême, toutes ses appré­ciations, le nombre de bras et la durée de temps indispensables pour achever ce singulier monument de barbarie et de patience, et il a trouvé que quatre mille ouvriers actuels, usant des mêmes procédés, n’ayant d’ailleurs à s’occuper ni de l’extraction de l’argile, ni du charriage de cette matière sur les lieux de manutention, ni de la coupe, ni du transport du bois nécessaire à la cuisson des briques, ni enfin de celui de ces briques sur les points d’immersion, et opérant pendant huit heures par jour, mettraient vingt-cinq ans et demi pour arriver à la fin de leur tâche. On peut juger par là quelle est l’importance du travail exécuté.\par
Il est à peine utile de dire que ce ne sont pas de telles conditions qui ont présidé à la construction du briquetage de Marsal. Ce ne sont pas, dis-je, des ouvriers astreints régulièrement et uniquement à leur labeur qui l’ont exécuté. Il a été conduit à fin par des familles de travailleurs barbares, agissant lentement, maladroitement, mais avec une persévérance imperturbable qui comptait pour rien et le temps et la peine. Il est aussi vraisemblable que, dans la pensée de ceux qui les premiers se sont mis à l’œuvre, le briquetage ne devait pas acquérir l’extension qu’il a prise. Ce n’est qu’à mesure où la population, favorisée par la sécurité des lieux, s’y est recrutée et étendue, qu’on a pu sentir l’opportunité de faire à la demeure commune des augmentations correspon­dantes. Plusieurs siècles se sont donc passés avant que le radier en arrivât à pouvoir porter des masses d’habitants à coup sûr respectables, car tant de fatigues n’ont pas été dépensées pour créer des espaces vides.\par
S’il était possible d’organiser des fouilles intelligentes sur ce terrain, et de sonder avec un peu de bonheur les boues qui le recouvrent, ou mieux encore celles dont il cache les abîmes, il est à présumer que l’on y découvrirait beaucoup plus de restes finniques qu’on ne saurait l’espérer partout ailleurs \footnote{Je n’ai ici l’intention ni l’opportunité d’énumérer absolument toutes les catégories de monuments finniques répandus en Europe. Je ne m’attache qu’aux principaux. J’aurais pu mentionner, entre autres, certaines excavations en forme de plats ou de disques remarquées par M. Troyon sur plu­sieurs blocs erratiques du Jura. Ils appartiennent probablement à l’époque où les Finnois, entrés en rapport avec les peuples blancs, se trouvèrent pourvue de quelques instruments de métal qui leur rendirent ce travail possible. Je fais allusion plus bas à cette dernière circonstance.}.\par
Ces populations d’hommes d’autrefois, ces tribus dont les vestiges se retrouvent préférablement au bord des mers, des rivières, des lacs, au sein même des marais, et qui semblent avoir eu pour le voisinage des eaux un attrait tout particulier, doivent paraître bien grossières assurément ; toutefois on ne peut leur refuser ni les instincts d’un certain degré de sociabilité, ni la puissance de quelques conceptions qui ne sont pas dénuées d’énergie, bien qu’elles le soient totalement de beauté. Les arts n’étaient évidemment pas l’affaire de ces peuples, à en juger d’ailleurs par les dessins bien misérables que l’on connaît d’eux.\par
Des poteries ornementées sont trouvées assez souvent dans les dolmens. Les lignes spirales simples, doubles ou même triples s’y reproduisent presque constamment. Il est même rare qu’il s’y présente autre chose, à part quelques dentelures. L’aspect de ces arabesques rappelle complètement les compositions dont les indigènes américains embellissent encore leurs gourdes. Ces spirales, trait principal du goût finnique, et au delà desquelles une invention stérile n’a pu guère aller, se voient non seulement sur les vases, mais sur certains monuments architecturaux qui, faisant exception à la règle générale, portent quelques traces de taille. Il est vraisemblable que ces constructions appartiennent aux époques les plus récentes, à celles où les aborigènes ont eu à leur disposition soit les instruments, soit même le concours de quelques Celtes, circons­tance très ordinaire dans les temps de transition. Un grand dolmen, à New-Grange, dans le comté irlandais de Meath, est non seulement orné de lignes spirales, il a encore des entrées en ogives. Un autre, près de Dowth, est même embelli de quelques croix inscrites dans des cercles. C’est le \emph{nec plus ultra}. À Gavr-Innis, près de Lokmariaker, M. Mérimée a observé des sculptures ou plutôt des gravures du même genre. Il existe aussi, au musée de Cluny, un os sur lequel a été entaillée assez profondément l’image d’un cheval. Tout cela est fort mal fait, et sans rien qui révèle une imagination supérieure à l’exécution, observation que l’on a si souvent lieu de faire dans les œuvres les plus mauvaises des métis mélaniens. Encore n’est-il pas bien assuré que le dernier objet soit finnique, bien qu’il ait été trouvé dans une grotte et recouvert d’une sorte de gangue pierreuse qui semble lui assigner une assez lointaine antiquité.\par
Je n’ai démontré jusqu’ici que par voie de comparaison et d’élimination la présence primordiale des peuples jaunes en Europe. Quelle que soit la force de cette méthode, elle ne suffit pas. Il est nécessaire de recourir à des éléments de persuasion plus directs. Heureusement ils ne font pas défaut.\par
Les plus anciennes traditions des Celtes et des Slaves, les premiers des peuples blancs qui aient habité le nord et l’ouest de l’Europe, et, par conséquent, ceux qui ont gardé les souvenirs les plus complets de l’ancien ordre des choses sur ce continent, se montrent riches de récits confus ayant pour objets certaines créatures complètement étrangères à leurs races. Ces récits, en se transmettant de bouche en bouche, à travers les âges, et par l’intermédiaire de plusieurs générations hétérogènes, ont nécessai­rement perdu depuis longtemps leur précision et subi des modifications considérables. Chaque siècle a un peu moins compris ce que le passé lui livrait, et c’est ainsi que les Finnois, objets de ce qui n’était d’abord qu’un fragment d’histoire, sont devenus des héros de contes bleus, des créations surnaturelles.\par
 Ils sont passés de très bonne heure du domaine de la réalité dans le milieu nuageux et vague d’une mythologie toute particulière à notre continent. Ce sont désormais ces nains, le plus souvent difformes, capricieux, méchants, et dangereux, quelquefois, au contraire, doux, caressants, sympathiques et d’une beauté charmante \footnote{Shakespeare, \emph{Midsummer Night’s Dream} et \emph{The Tempest}, ‑ Robin Good Fellow dans les \emph{Relics of Ancient English Poetry}, de Thomas Percy, in-8°, Lond., 1847. Les nains abondent chez tous les peuples de l’Europe. ‑ Partout où les nains sont braves, bienveillants et aimables, on doit reconnaître l’influence de la mythologie scandinave ou des fables orientales, Les renseignements italiotes, celtiques et slaves les traitent constamment avec une extrême sévérité.}, cependant toujours nains, dont les bandes ne cessent pas d’habiter les monuments de l’âge de pierre, dormant le jour sous les dolmens, dans la bruyère, au pied des pierres levées, la nuit se répandant à travers les landes, au long des chemins creux, ou bien encore, errant au bord des lacs et des sources, parmi les roseaux et les grandes herbes.\par
C’est une opinion commune aux paysans de l’Écosse, de la Bretagne et des provinces allemandes que les nains cherchent surtout à dérober les enfants et à déposer à leur place leurs propres nourrissons \footnote{La Villemarqué, \emph{Chants populaires de la Bretagne}, t. I. Voir la ballade intitulée \emph{l’Enfant supposé}. « À sa place on avait mis un monstre ; sa face est aussi rousse que celle d’un crapeau. » (P. 51.)}. Quand ils ont réussi à mettre en défaut la surveillance d’une mère, il est très difficile de leur arracher leur proie. On n’y parvient qu’en battant à outrance le petit monstre qu’ils lui ont substitué. Leur but est de procurer à leur progéniture l’avantage de vivre parmi les hommes, et quant à l’enfant volé, les légendes sont partout unanimes sur ce qu’ils en veulent faire : ils veulent le marier à quelqu’un d’entre eux, dans le but précis d’améliorer leur race \footnote{\emph{Ibid., Introduction}, p. XLIX.}.\par
Au premier abord, on est tenté de les trouver bien modestes d’envier quelque chose à notre espèce, puisque, par la longévité et la puissance surnaturelle qu’on leur attribue d’ailleurs, ils sont très supérieurs et très redoutables aux fils d’Adam. Mais il n’y a pas à raisonner avec les traditions : telles quelles sont, il faut les écouter ou les rejeter. Ce dernier parti serait ici peu judicieux, car l’indication est précieuse. Cette ambition ethnique des nains, n’est autre que le sentiment qui se retrouve aujourd’hui chez les Lapons. Convaincus de leur laideur et de leur infériorité, ces peuples ne sont jamais plus contents que lorsque des hommes d’une meilleure origine, s’approchant de leurs femmes ou de leurs filles, donnent au père ou au mari, ou même au fiancé, l’espérance de voir sa hutte habitée un jour par un métis supérieur à lui\footnote{Regnard, \emph{Voyage en Laponie}.}.\par
Les pays de l’Europe où la mémoire des nains s’est conservée le plus vivace sont précisément ceux où le fond des populations est resté le plus purement celtique. Ces pays sont la Bretagne, l’Irlande, l’Écosse, l’Allemagne. La tradition s’est, au contraire, affaiblie dans le midi de la France, en Espagne, en Italie. Chez les Slaves, qui ont subi tant d’invasions et de bouleversements provenant de races très différentes, elle n’a pas disparu, tant s’en faut, mais elle s’est compliquée d’idées étrangères. Tout cela s’explique sans peine. Les Celtes du nord et de l’ouest, soumis principalement à des influences germaniques, en ont reçu et leur ont prêté des notions qui ne pouvaient faire disparaître absolument le fond des premiers récits. De même pour les Slaves. Mais les populations sémitisées du sud de l’Europe ont de bonne heure connu des légendes venues d’Asie, qui, tout à fait disparates avec celles de l’ancienne Europe, ont absorbé leur attention et exigé presque tout leur intérêt.\par
Ces petits nains, ces voleurs d’enfants, ces êtres si persuadés de leur infériorité vis-à-vis de la race blanche, et qui, en même temps, possèdent de si beaux secrets, un pouvoir immense, une sagesse profonde, n’en sont pas moins tenus, par l’opinion, dans une situation des plus humbles et même véritablement servile. Ce sont des ouvriers \footnote{Dieffenbach, \emph{Celtica II}, 2\textsuperscript{e} Abth., p. 210. Les montagnards gaëls de l’Écosse attribuent les monuments pseudo-celtiques de leur pays à un peuple mystérieux, antérieur à leur race et qu’ils nomment \emph{drinnach, les ouvriers}.}, et surtout des ouvriers mineurs. Ils ne dédaignent pas de battre de la fausse monnaie. Retirés dans les entrailles de la terre, ils savent fabriquer, avec les métaux les plus précieux, les armes de la plus fine trempe. Ce n’est pourtant jamais à des héros de leur race qu’ils destinent ces chefs-d’œuvre. Ils les font pour les hommes qui seuls savent s’en servir.\par
Il est arrivé parfois, dit la Fable, que des ménétriers, revenant tard de noces de village, ont rencontré, sur la lande, après minuit sonné, une foule de nains fort affairés aux carrefours des chemins creux. D’autres témoins rustiques les ont vus s’agitant par essaims au pied des dolmens, leurs demeures d’habitude, s’escrimant de lourds marteaux, de fortes tenailles, transportant les blocs de granit, et tirant du minerai d’or des entrailles de la terre. C’est surtout en Allemagne que l’on raconte des aventures de ce dernier genre. Presque toujours ces ouvriers laborieux ont donné lieu à la remarque qu’ils étaient singulièrement chauves. On se rappellera ici que la débilité du système pileux est un trait spécifique chez la plupart des Finnois.\par
Dans maintes occasions, ce ne sont plus des mineurs que l’on a surpris occupés à leur travail nocturne, mais des fileuses décrépites ou bien de petites lavandières battant le linge de tout leur cœur, sur le bord du marécage. Il n’est même pas besoin que le villageois irlandais, écossais, breton, allemand, scandinave ou slave, sorte de chez lui pour faire de pareilles rencontres. Bien des nains se blottissent dans les métairies, et y sont d’un grand secours à la buanderie, à la cuisine, à l’étable. Soigneux, propres et discrets, ils ne cassent ni ne perdent rien, ils aident les servantes et les garçons de ferme avec le zèle le plus méritoire. Mais de si utiles créatures ont aussi leurs défauts, et ces défauts sont grands. Les nains passent universellement pour être faux, perfides, lâches, cruels, gourmands à l’excès, ivrognes jusqu’à la furie, et aussi lascifs que les chèvres de Théocrite. Toutes les histoires d’ondines amoureuses, dépouillées des ornements que la poésie littéraire y a joints, sont aussi peu édifiantes que possible \footnote{Ces contes ont cours en Allemagne, absolument comme en Écosse et en Bretagne.}.\par
Les nains ont donc, par leurs qualités comme par leurs vices, la physionomie d’une population essentiellement servile, ce qui est une marque que les traditions qui les concernent se sont primitivement formées à une époque où, pour la plupart du moins, ils étaient déjà tombés sous le joug des émigrants de race blanche. Cette opinion est confirmée, ainsi que l’authenticité des récits de la légende moderne, par les traces très reconnaissables, très évidentes, que nous retrouvons de tous les faits qu’elle indique et attribue aux nains, de tous, sans exception aucune, dans l’antiquité la plus haute. La philologie, les mythes, et même l’histoire des époques grecques, étrusques et sabines, vont démontrer cette assertion.\par
Les nains sont connus, en Europe, sous quatre noms principaux, aussi vieux que la présence des peuples blancs, Ces noms appartiennent, par leurs racines, au fond le plus ancien des langues de l’espèce noble. Ce sont, sous réserve de quelques altérations de formes peu importantes, les mots pygmée \emph{fad, gen} et \emph{nar}.\par
Le premier se trouve dans une comparaison de l’Iliade, où le poète, parlant des cris et du tumulte qui s’élèvent des rangs des Troyens prêts à commencer le combat, s’exprime ainsi :\par
« De même montent vers le ciel les clameurs des grues, lorsque, fuyant l’hiver « et la pluie incessante, elles volent en criant vers le fleuve Océan, et apportent le « meurtre et la mort aux hommes pygmées. »\par
Le fait seul que cette allusion est destinée à faire bien saisir aux auditeurs du poème quelle était l’attitude des Troyens prêts à combattre, prouve que l’on avait, au temps d’Homère, une notion très générale et très familière de l’existence des pygmées. Ces petits êtres, demeurant du côté du fleuve Océan, se trouvaient à l’ouest du pays des Hellènes, et comme les grues allaient les chercher à la fin de l’hiver, ils étaient au nord ; car la migration des oiseaux de passage a lieu à cette époque dans cette direction. Ils habitaient donc l’Europe occidentale. C’est là, en effet, que nous les avons jusqu’à présent reconnus à leurs œuvres. Homère n’est pas le seul dans l’antiquité grecque qui ait parlé d’eux. Hécatée de Milet les mentionne, et en fait des laboureurs minuscules réduits à couper leurs blés à coups de hache. Eustathe place les pygmées dans les régions boréales, vers la hauteur de Thulé. Il les fait extrêmement petits, et ne leur assigne pas une vie très longue. Enfin Aristote lui-même s’occupe d’eux. Il déclare ne les considérer nullement comme fabuleux. Mais il explique la taille minime qu’on leur attribue par d’assez pauvres raisons, en disant qu’elle est due à la petitesse comparative de leurs chevaux ; et comme ce philosophe vivait à une époque où la mode scientifique voulait que tout vînt de l’Égypte, il les relègue aux sources du Nil. Après lui la tradition se corrompt de plus en plus dans ce sens, et Strabon, comme Ovide, ne donne que des renseignements complètement fantastiques, et qui ne sauraient ici trouver leur place.\par
Le mot de pygmée, (mot grec), indique la longueur du poing au coude. Telle aurait été la hauteur du petit homme ; mais il est facile de concevoir que les questions de grandeur et de quantité, tout ce qui exige de la précision, est surtout maltraité par les récits légendaires. L’histoire, même la plus correcte, n’est pas d’ailleurs à l’abri des exagérations et des erreurs de ce genre. (Mot grec) est donc le pendant du \emph{Petit Poucet} des contes français, et du \emph{Daumling} des contes allemands. En supposant cette étymologie irréprochable pour les époques historiques, qui ont su donner au mot la forme congruente à l’idée qu’elles lui faisaient rendre, il n’y a pas lieu d’en être pleinement satisfait et de s’y tenir pour ce qui appartient à une époque antérieure, et, par conséquent, à des notions plus saines. En se plaçant à ce point de vue, la forme primitive perdue de (mot grec) dérivait certainement d’une racine voisine du sanscrit \emph{pît}, au féminin \emph{pa}, qui veut dire \emph{jaune}, et d’une expression voisine des formes pronominales sanscrite, zende et grecque, \emph{aham, azem}, (mot grec) qui, renfermant surtout l’idée abstraite de l’\emph{être}, a donné naissance au gothique \emph{guma, homme}. (mot grec) ne signifie donc autre chose qu’\emph{homme jaune}.\par
Il est digne de remarque que la racine pronominale de ce mot \emph{guma}, se rappro­chant, dans les langues slaves, de l’expression sanscrite \emph{gan}, qui indique la production de l’être ou la génération, intercale un \emph{n} là où les autres idiomes d’origine blanche actuellement connus ont abandonné cette lettre. Elle survit cependant en allemand, dans une expression fort ancienne, qui est \emph{gnome}. Le \emph{gnome} est donc parfaitement identique et de nom et de fait au \emph{pygmée} ; dans sa forme actuelle, ce vocable ne signifie, au fond, pas autre chose qu’\emph{un être} ; c’est qu’il est mutilé, sort commun des choses intellectuelles et matérielles très antiques.\par
Après ces dénominations grecque et gothique de \emph{pygmée} et de \emph{gnome}, se présente l’expression celtique de \emph{fad}. Les Galls appelaient ainsi l’homme ou la femme qu’ils considéraient comme inspirés \footnote{\emph{Mémoires et documents publiés par la Société d’histoire et d’archéologie de Genève}, t. V, p. 496.}. C’est le \emph{vates} des peuples italiotes, et, par dérivation, c’est aussi cette puissance occulte dont les devins avaient le pouvoir de pénétrer les secrets, \emph{fatum} \footnote{Le nom des fées en italien, \emph{fata}, s’y rapporte étroitement. Il en est probablement de même de l’espagnol hada.}. Une telle identification originelle des deux mots n’est d’ailleurs point facultative. \emph{Fad}, devenu aujourd’hui, dans le patois du pays de Vaud, \emph{fatha} ou \emph{fada}, dans le dialecte savoyard du Chablais \emph{fihes}, dans le genevois \emph{faye}, dans le français \emph{fée}, dans le berrichon \emph{fadet}, au féminin \emph{fadette}, dans le marseillais \emph{fada}, désigne partout un homme ou une femme élevés au-dessus du niveau commun par des dons surnaturels, et rabaissés au-dessous de ce même niveau par la faiblesse de la raison. Le \emph{fada}, le \emph{fadet} est tout à la fois sorcier et idiot, un être fatal.\par
En suivant cette trace, on trouve les mêmes notions réunies sur le même être, sous une autre forme lexicologique, chez les races blanches aborigènes de l’Italie. C’est \emph{faunus}, au féminin \emph{fauna}. Il y a longtemps déjà que les érudits ont remarqué comme une singularité que ces divinités sont à la fois une et multiples, \emph{faunus} et \emph{fauni}, faune et les faunes, et, plus encore, que le nom de la déesse est identique à celui de son mari, circonstance dont, en effet, la mythologie classique n’offre peut-être pas un second exemple. D’autre explication n’est pas possible que d’admettre qu’il s’agit ici, non pas de dénomination de personnes, mais d’appellations génériques ou nationales. Faune et les faunes ont, en Grèce, leurs pareils dans Pan et les pans, les ægipans, transformation facile à expliquer d’un même mot. La permutation du \emph{p} et de l’\emph{f} est trop fréquente pour qu’il soit nécessaire de la justifier.\par
Le faune aussi bien que le pan étaient des êtres grotesques par leur laideur, touchant de près à l’animalité, ivrognes, débauchés, cruels, grossiers de toute façon, mais connaissant l’avenir et sachant le dévoiler \footnote{ \noindent Pan était sorcier dans toute la force du terme :\par
 
\begin{verse}
Munere sic niveo lanæ, si credere dignum est,\\
 Pan, deus Arcadiæ, captam te, Luna, fefellit,\\
 In nemora alta vocans ; nec tu adspernata vocantem.\\
\end{verse}
 
\bibl{Virg., \emph{Géorg.}, III, 391-393}
}. Qui ne voit ici le portrait moral et physique de l’espèce jaune, comme les premiers émigrants blancs se le sont repré­senté ? Un penchant invincible à toutes les superstitions, un abandon absolu aux pratiques magiques des sorciers, des jeteurs de sorts, des chamans, c’est encore là le trait dominant de la race finnique dans tous les pays où on peut l’observer. Les Celtes métis et les Slaves, en accueillant dans leur théologie, aux époques de décadence, les aberrations religieuses de leurs vaincus, appelèrent très naturellement du nom même de ces derniers leurs magiciens, héritiers ou imitateurs d’un sacerdoce barbare. On aperçoit dans la lasciveté des ondines ce vice si constamment reproché aux femmes de la race jaune, et qui est tel qu’il a, dit-on, fait naître l’usage de la mutilation des pieds, pratiquée comme précaution paternelle et maritale sur les filles chinoises, et que là où il ne rencontre pas les obstacles d’une société réglée, il donne lieu, comme au Kamtschatka, à des orgies trop semblables aux courses des Ménades de la Thrace, pour qu’on ne soit pas disposé à reconnaître dans les fougueuses meurtrières d’Orphée, des parentes de la courtisane actuelle de Sou-Tcheou-Fou et de Nanking \footnote{Callery et Ivan, \emph{l’Insurrection en Chine}, in-12, Paris, 1853, 224.}. On ne remarque pas moins chez les faunes le goût absorbant du vin et de la pâture, cette sensualité ignoble de la famille mongole, et, enfin, on y relève cette aptitude aux occupations rurales et ménagères \footnote{ 
\begin{verse}
Et vos, agrestum præsentia numina, Fauni,\\
 Ferte simul, Faunique, pedem, Dryadesque puellæ\\
 Munera vestra cano.\\
\end{verse}
 
\bibl{Virg., \emph{Géorg.}, (I, 10-12).}
 Pan, ovium custos.\\
 
\bibl{\emph{Ibid}., I, 17}
} que les légendes modernes attribuent à leurs pareils, et que, du temps des Celtes primitifs, on pouvait obtenir avec facilité d’une race utilitaire et essentiellement tournée vers les choses matérielles.\par
L’assimilation complète des deux formes, faunus et (mot grec), n’offre pas de difficultés. On doit la pousser plus loin. Elle est applicable également, quoique d’une manière d’abord moins évidente, aux mot\emph{s khorrigan} et \emph{khoridwen}. C’est ainsi que les paysans armoricains désignent les nains magiques de leurs pays. Les Gallois disent \emph{Gwrachan} \footnote{On nomme aussi quelquefois les khorrigans, \emph{duz, les dieux}, c’est un dérivé de l’arian \emph{déwa}. ‑La Villemarqué, \emph{ouvr. cité, Introduct}., t. I, p. XLVI. ‑ Voir l’article \emph{Dwergar}, dans l\emph{’Encycl. Ersch u. Gruber}, sect. I, 28 th., p. 190 et pass. ‑ Dieffenbach, \emph{Celtica II}, Abth. 2, p. 211.}. Ces expressions sont l’une et l’autre composées de deux parties. \emph{Khorr} et \emph{Gwr} ne valent autre chose que \emph{gon} et \emph{gwn}, ou \emph{gan} \footnote{Gan est encore un nom très communément appliqué, par les paysans bretons, aux khorrigans. Dans l’Inde, on connaît aussi les gâni pour être des démons malfaisants d’une espèce inférieure. ‑ Gorresio, Ramayana, t. VI, p. 125.} , chez les Latins \emph{genius}, en français \emph{génie}, employé dans le même sens. Je m’explique.\par
La lettre \emph{r}, dans les langues primitives de la famille blanche, a été d’une extrême débilité. L’alphabet sanscrit la possède trois fois, et, pas une seule ne lui accorde la force et la place d’une consonne. Dans deux cas, c’est une voyelle ; dans un, c’est une demi-voyelle comme 1’\emph{l} et le \emph{w} qui, pour nos idiomes modernes, a conservé par sa facilité à se confondre, même graphiquement, avec l’\emph{u} ou l’\emph{ou}, une égale mobilité.\par
Cette \emph{r} primordiale, si incertaine d’accentuation, paraît avoir eu les plus grands rapports avec l’\emph{aïn}, l’\emph{a} emphatique des idiomes sémitiques, et c’est ainsi seulement qu’on peut s’expliquer le goût marqué de l’ancien scandinave pour cette lettre. On la retrouve dans une grande quantité de mots où le sanscrit mettait un \emph{a}, comme, par exemple, dans \emph{gardhr}, synonyme de \emph{garta, enceinte, maison, ville}.\par
Cette faiblesse organique la rend plus susceptible qu’aucune autre des nombreuses permutations dont les principales ont lieu, comme on doit s’y attendre, avec des sons d’une faiblesse à peu près égale, avec 1’\emph{l}, avec le \emph{v}, avec l’\emph{s} ou l’\emph{n}, consonne à la vérité, mais reproduite trois fois en sanscrit, et, par conséquent, peu clairement marquée, enfin avec le \emph{g}, par suite de l’affinité intime qui unit ce dernier son au \emph{w}, principalement dans les langues celtiques \footnote{Bopp, \emph{Vergleichende Grammatik}, p. 39 et pass. ‑ Aufrecht u. Kirchhoff, \emph{Die umbrischen Sprachdenkmaeler}, p. 97, § 256. ‑ Le mot celtique \emph{bara}, pain, devenu \emph{panis}, offre un exemple certain de mutation de l’\emph{r} en \emph{n.}}. Citer trop d’exemples de l’application de cette loi de muabilité serait ici hors de place ; mais comme il n’est pas sans intérêt pour le sujet même que je traite, d’en alléguer quelques-uns, en voici des principaux :\par
(Mot grec) et \emph{faunus} sont corrélatifs de forme et de sens au persan (mot persan) \emph{péri}, une fée, et, en anglais, à \emph{fairy}, et en français, à la désignation générale de \emph{féerie}, et en suédois à \emph{alfar}, et en allemand à \emph{elfen} \footnote{La première syllabe \emph{al} ou \emph{el} n’est que l’article celtique. ‑ Richter, \emph{die Elfen, Encycl. Ersch. u. Gruber}, sect. I, 33, p. 301 et sqq.}. Dans le kymrique, on a l’adjectif \emph{ffyrnig, méchant, cruel, hostile, criminel}, qui se trouve en parenté étymologique bien remarquable avec \emph{ffur, sage, savant}, et \emph{furner, sagesse, prudence}, d’où est venu notre mot \emph{finesse} \footnote{Dieffenbach, \emph{Vergleichendes Woerterbuch der gothischen Sprache}, Frankfurt a. M., 1851, in-8°, t. I, p. 358-359.}. C’est ainsi que \emph{gan, wen, khorr} et \emph{genius}, et \emph{fen}, sont des reproduction altérées d’un seul et même mot.\par
Les dieux appelés par les aborigènes italiotes, et par les Étrusques, \emph{genii}, étaient considérés comme supérieurs aux puissances célestes les plus augustes. On les saluait des titres celtiques de \emph{lar} ou \emph{larth}, c’est-à-dire seigneurs, et de \emph{penates, penaeth}, les \emph{premiers}, les \emph{sublimes}. On les représentait sous la forme de nains chauves, fort peu avenants. On les disait doués d’une sagesse et d’une prescience infinies. Chacun d’eux veillait, en particulier, au salut d’une créature humaine, et le costume qui leur était attribué était une sorte de sac sans manches, tombant jusqu’à mi-jambes.\par
Les Romains les nommaient, pour cette raison, \emph{dii involuti}, les \emph{dieux enveloppés}. Qu’on se figure les grossiers Finnois revêtus d’un sayon de peaux de bêtes, et l’on a cet accoutrement peu recherché dont les auteurs de certaines pierres gravées ont probablement eu en vue de reproduire l’image \footnote{Tel est le personnage de Tagès. Le mythe qui le concerne est des plus significatifs. Un laboureur tyrrhénien ayant un jour creusé un sillon d’une profondeur peu commune, Tagès, fils d’un \emph{genius Jovialis}, d’un génie divin, d’un Gan, sortit tout à coup de la terre et adressa la parole au laboureur. Celui-ci effrayé, poussa des cris, et tous les Tyrrhéniens accoururent. Alors Tagès leur révéla les mystères de l’aruspicine. Il avait à peine fini de parler qu’il expira. Mais les auditeurs avaient soigneusement écouté ses paroles, et la science divinatoire leur fut acquise. De là, le pouvoir augural particulier aux Étrusques. Tagès était de la taille d’un enfant ; sa sagesse était profonde. Ainsi expliquaient les Rasènes l’héritage sacerdotal que leur avaient légué les peuples qui les avaient précédés en Italie. ‑ Cic., \emph{de Div}. ; 2, 23 ; Ovid., \emph{Metam}. ; 15, 558 ; Festus, S. v. Tagès, Isid., \emph{Orig}., 8. 9.}.\par
Ces \emph{genii}, ces \emph{larths}, esprits élémentaires, n’ont pas besoin d’être comparés longue­ment aux Finnois pour qu’on reconnaisse en eux ces derniers. L’identité s’établit d’elle-même. La haute antiquité de cette notion, son extrême généralisation, son ubiquité, dans toutes les régions européennes, sous les différentes formes d’une même dénomination, \emph{faunus}, (mot grec), \emph{gen} ou \emph{genius, fee, khorrigan, fairy}, ne permettent pas de douter qu’elle ne repose sur un fond parfaitement historique. Il n’y a donc nulle nécessité d’y insister davantage, et on peut passer à la dernière face de la question en examinant le mot \emph{nar}.\par
Il est identique avec \emph{nanus}, ou mieux encore avec le celtique \emph{nan}, par suite de la loi de permutation qui a été établie plus haut. Dans les dialectes tudesques modernes, il signifie un fou, comme jadis, chez les peuples italiotes, \emph{fatuus}, dérivé de \emph{fad}. Les langues néo-latines l’ont consacré à désigner exclusivement un nain, abstraction faite de toute idée de développement moral. Mais, dans l’antiquité, les deux notions aujourd’hui séparées se présentaient réunies. Le \emph{nan} ou le \emph{nar} était un être laborieux et doué d’un génie magique, mais sot, borné, fourbe, cruel et débauché, toujours de taille remarquablement petite, et généralement chauve.\par
Le \emph{casnar} des Étrusques était une sorte de polichinelle rabougri, contrefait, nain et aussi sot que méchant, gourmand et porté à s’enivrer. Chez les mêmes peuples, le \emph{nanus} était un pauvre hère sans feu ni lieu, un vagabond, situation qui était assuré­ment, sur plus d’un point, celle des Finnois dépossédés par les vainqueurs blancs ou métis, et, sous ce rapport, ces misérables fournissent aux annales primitives de l’Occident le pendant exact de ce que sont, dans les chroniques orientales, ces tristes Chorréens, ces Enakim, ces géants, ces Goliaths vagabonds, eux aussi dépouillés de leur patrimoine natal et réfugiés dans les villes des Philistins \footnote{Cf. t. I, p. 486, note. ‑ Dennis, \emph{ouvr. cité}, t. I, p. XIX.}.\par
Au sentiment de mépris qui s’attachait ainsi au \emph{nan}, réduit à errer de lieux en lieux, s’unissait, dans la péninsule italique, le respect des connaissances surhumaines qu’on prêtait à ce malheureux. On montrait à Cortone, avec une pieuse vénération le tom­beau d’un \emph{nan} voyageur \footnote{Le mot \emph{cas-nar} est lui-même composé des deux mots \emph{nar} et \emph{cas}, racine ariane qui en sanscrit, signifie \emph{aller, marcher}. Benfey, \emph{Glossarium}, p. 73. ‑ Voir, sur le tombeau de Cortone, Dionys. Halic., \emph{Antiq. rom.}, I, XXIII. ‑ Abeken, \emph{ouv, cité}, p. 26.}.\par
On avait les mêmes idées dans l’Aquitaine. Le pays de Néris révérait une divinité topique appelée Nen-nerio \footnote{Barailon, \emph{Recherches sur plusieurs monuments celtiques et romains}, in-8°, Paris, 1806, p. 143.}. Je relève en passant qu’il semble y avoir dans cette expression un pléonasme semblable à celui des mots \emph{koridwen} et \emph{khorrigan}. Peut-être aussi faut-il entendre l’un et l’autre dans un sens réduplicatif destiné à donner à ces titres une portée de superlatif ; ils signifieraient alors le \emph{gan} ou le \emph{nan} par excellence.\par
De l’Aquitaine passons au pays des Scythes, c’est-à-dire à la région orientale de l’Europe qui, dans le vague de sa dénomination, s’étend du Pont-Euxin à la Baltique. Hérodote y montre des sorciers fort consultés, fort écoutés, et qui portaient le nom d’\emph{Enarées} et de \emph{Neures} \footnote{Hérod., IV, 17, 67, 69, et ailleurs.}. Les peuples blancs au milieu desquels vivaient ces hommes, tout en accordant une confiance très grande à leurs prédictions, les traitaient avec un mépris outrageant, et, à l’occasion, avec une extrême cruauté. Lorsque les événements annoncés ne s’accomplissaient pas, on brûlait vivants les devins maladroits. La science des Enarées provenait, disaient-ils eux-mêmes, d’une disposition physique comparable à l’hystérie des femmes. Il est probable, en effet, qu’ils imitaient les convulsions nerveuses des sibylles. De telles maladies éclatent beaucoup plus fréquemment chez les peuples jaunes que dans les deux autres races. C’est pour cette raison que les Russes sont, de tous les peuples métis de l’Europe moderne, ceux qui en sont le plus atteints.\par
Cet être, rencontré par toutes les anciennes nations blanches de l’Europe sur l’étendue entière du continent, et appelé par elles \emph{pygmée, fad, genius} et \emph{nar}, décrit avec les mêmes caractères physiques, les mêmes aptitudes morales, les mêmes vices, les mêmes vertus, est évidemment partout un être primitivement très réel. Il est impossible d’attribuer à l’imagination collective de tant de peuples divers qui ne se sont jamais revus ni consultés, depuis l’époque immémoriale de leur séparation dans la haute Asie, l’invention pure et simple d’une créature si clairement définie et qui ne serait que fantastique. Le bon sens le plus vulgaire se refuse à une telle supposition. La linguistique n’y consent pas davantage ; on va le voir par le dernier mot qu’il faut encore lui arracher, et qui va bien préciser qu’il s’agit ici, à l’origine, d’êtres de chair et d’os, d’hommes très véritables.\par
Cessons un moment de lui demander quel sens spécial les Hellènes primitifs, peut-être même encore les Titans, attachaient au mot de \emph{pygmée}, les Celtes à celui de \emph{fad}, les Italiotes à celui de \emph{genius}, presque tous à celui de \emph{nan} et de \emph{nar}. Envisageons ces expressions uniquement en elles-mêmes. Dans toutes les langues, les mots commen­cent par avoir un sens large et peu défini, puis, avec le cours des siècles, ces mêmes mots perdent leurs flexibilité d’application et tendent à se limiter à la représentation d’une seule et unique nuance d’idée. Ainsi \emph{Haschaschi}, a voulu dire un Arabe soumis à la doctrine hérétique des princes montagnards du Liban, et qui, ayant reçu de son maître un ordre de mort, mangeait du haschisch pour se donner le courage du crime. Aujourd’hui, un assassin n’est plus un Arabe, n’est plus un hérétique musulman, n’est plus un sujet du Vieux de la Montagne, n’est plus un séide agissant sous l’impulsion d’un maître, n’est plus un mangeur de haschisch, c’est tout uniment un meurtrier. On pourrait faire des observations semblables sur le mot \emph{gentil}, sur le mot \emph{franc}, sur une foule d’autres ; mais, pour en revenir à ceux qui nous occupent plus particulièrement, nous trouverons que tous renferment dans leur sens absolu des applications très vagues, et que ce n’est que l’usage des siècles qui les a fixés peu à peu à un sens précis.\par
\emph{Pit-goma} serait encore celui qui pourrait le plus échapper à cette définition, car, formé de deux racines, il particularise, au premier aspect, l’objet auquel il s’applique. Il indique un \emph{homme jaune}, partant s’applique bien à un homme de la race finnique. Mais, en même temps, comme il ne contient rien qui fasse allusion aux qualités parti­culières de cette race, autres que la couleur, c’est-à-dire à la petitesse, à la sensualité, à la superstition, à l’esprit utilitaire, il ne suffit que faiblement à la désigner. D’ailleurs, il ne s’arrête pas à cette phase incomplète de son existence : il subit une modification, et, devenant (mot grec), il prend toutes les nuances qui lui manquaient pour se spécialiser. Un \emph{pygmée} n’est plus seulement un homme jaune, c’est un homme pourvu de tous les caractères de l’espèce finnique, et, dès lors, le mot ne saurait plus s’appliquer à personne autre. Dans le dialecte des Hellènes, la modification avait porté sur la lettre \emph{t}, de façon, en la rejetant, à contracter les deux mots \emph{Pit-goma} en une seule et même racine factice, parce que là où il n’y a pas une racine simple, factice ou réelle, il n’y a pas un sens précis. Mais, dans la région extra-hellénique, l’opération se fit autrement, et, pour atteindre à la forme concrète d’une racine, on rejeta tout à fait le mot \emph{pit}, qui aurait semblé pourtant devoir être considéré comme essentiel, et, se servant unique­ment de \emph{goma}, très légèrement altéré, on désigna les Finnois par une forme du mot homme, consacrée à eux seuls, et le but fut atteint. Bien que \emph{gnome} ne signifie pas autre chose qu’\emph{homme}, il ne saurait plus éveiller une autre idée que celle appliquée par la superstition aux Finnois errants cachés dans les rochers et les cavernes.\par
Il est peut-être plus difficile d’analyser à fond le mot fad. On doit croire que, mutilé comme \emph{pit-goma}, par la nécessité d’en faire une racine, il a perdu la partie que \emph{gnome} a conservée, et rejeté celle que ce dernier vocable a gardée. Dans cette hypothèse, \emph{fad} ne serait autre chose que \emph{pit}, en vertu de mutations d’autant plus admissibles que la voyelle, étant longue dans la forme sanscrite, était toute préparée à recevoir au gré d’un autre dialecte une prononciation plus large.\par
Avec le mot \emph{gen} ou \emph{gan} ou \emph{khorr}, la même modification de transformation que dans \emph{gnome} se retrouve. Le sens primitif est simplement la \emph{descendance}, la \emph{race}, les \emph{hommes, genus}. Il se peut aussi que la question ne soit pas aussi facile à résoudre, et qu’au lieu d’une mutilation, il s’agisse ici d’une contraction, aujourd’hui peu visible, et qui pourtant se laisse concevoir. L’affinité des sons \emph{p, f, w, g, ou, à}, permet de comprendre la progression suivante :\par
pit-gen,\par
fît-gen,\par
fî-gen,\par
fî-ouen,\par
gàn,\par
finn et fen.\par
 Ce dernier mot n’a rien de mythologique, c’est le nom antique des vrais et naturels Finnois, et Tacite le témoigne, non seulement par l’usage qu’il en fait mais par la description physique et morale donnée par lui des gens qui le portent. Ses paroles valent la peine d’être citées : « Chez les Finnois, dit-il, « étonnante sauvagerie, hideuse misère ; ni armes, ni chevaux, ni maisons. « Pour nourriture, de l’herbe ; pour vêtements, des peaux ; pour lit, le sol. « L’unique ressource, ce sont les flèches que, par manque de fer, on arme « d’os. Et la chasse repaît également hommes et femmes. Ils ne se quittent « pas, et chacun prend sa part du butin. Aux enfants, pas d’autre refuge contre « les bêtes et les pluies, que de s’abriter dans quelque entrelacs de branches. « Là reviennent les jeunes ; là se retirent les vieillards \footnote{\emph{De mor. Germ.}, XLVI.}. »\par
Aujourd’hui ce mot de \emph{Finnois} a perdu, dans l’usage ordinaire, sa véritable accep­tion, et les peuples auxquels on le donne sont, pour la plupart du moins, des métis germaniques ou slaves, de degrés très différents.\par
Avec \emph{nar} ou \emph{nan}, il y a évidemment mutilation. Ce mot, pour le sanscrit et le zend, signifie également \emph{homme} \footnote{En zend, c’est, au nominatif, \emph{nairya}.}. On a encore dans l’Inde la nation des Naïrs, comme on a eu dans la Gaule, à l’embouchure de la Loire, les \emph{Nannètes}. Ailleurs le même nom se présente fréquemment \footnote{J’ai sous les yeux quatre médailles gréco-bactriennes ou gréco-indiennes, deux de cuivre, deux d’argent. La première porte sur une face une figure debout, tournée de profil, vêtue d’une robe longue ; légende à droite, NONO, à gauche, \emph{effacée}. Au revers, \emph{figure de face, le bras droit étendu, le bras gauche relevé vers la tête, tunique courte} ; légende à gauche, \emph{illisible}. La seconde : face, \emph{figure nimbée sur un éléphant}, légende à droite, NANO ; à gauche, \emph{illisible}. Revers, \emph{divinité à plusieurs bras nimbée, debout, de profil}, traitée dans le style grec ; \emph{monogramme saytique}, légende à gauche : \emph{illisible}. La troisième, médaille d’argent : face, \emph{tête royale de profil, tournée à droite}, légende à droite : AIIAII (?) ; à gauche : OEPKIKOPAZ au revers, \emph{deux figures très effacées, se faisant face} ; au milieu légende à droite NAN ; à gauche : OKTO. La quatrième : face, \emph{tête royale de face, le bras droit levé} légende à droite ‑ AIIAIIO (?) ; à gauche : OEPKIKOP (?). ‑ \emph{Cabinet de S. E. M. le gén. baron de Prokesch-Osten}.}. Quant au mot perdu, il est retrouvé à l’aide de deux noms mythologiques, dont l’un est appliqué par le Ramayana aux aborigènes du Dekkhan, considérés comme des démons, les \emph{Naïrriti}, autrement dit les \emph{hommes horribles, redoutables} \footnote{On lit aussi \emph{Naïriti} ; Gorresio, \emph{Ramayana}, t. VI, \emph{introduct}., p. 7, et notes, p. 402.} ; dont l’autre est le nom d’une divinité celtique, adoptée par les Suèves Germains, riverains de la Baltique. C’est \emph{Nerthus} ou \emph{Hertha} ; son culte était des plus sauvages et des plus cruels, et tout ce qu’on en sait tend à le rattacher aux notions dégénérées que le sacerdoce druidique avait empruntées des sorciers jaunes.\par
Voici les aborigènes de l’Europe, considérés en personnes, décrits avec leurs caractères physiques et moraux. Nous n’avons pas à nous plaindre cette fois de la pénurie des renseignements. On voit que les témoignages et les débris abondent de toutes parts, et établissent les faits sous la pleine clarté d’une complète certitude. Pour que rien ne manque, il n’est plus besoin que de voir l’antiquité nous livrer des portraits matériels de ces nains magiques dont elle était si préoccupée. Nous avons déjà pu soupçonner que l’image de Tagès et d’autres, qui se rencontrent sur les pierres gravées, étaient propres à remplir ce but. En désirant davantage, on demande presque une espèce de miracle, et pourtant le miracle a lieu.\par
Entre Genève et le mont Salève, s’aperçoit, sur un monticule naturel, un bloc erratique qui porte sur une de ses faces un bas-relief grossier, représentant quatre figures debout, de stature rabougrie et ramassée, sans cheveux, à physionomie large et plate, tenant des deux mains un objet cylindrique dont la longueur dépasse de quelques pouces la largeur des doigts \footnote{Troyon,\emph{ Colline des sacrifices de Chavannes le Veuron}, in-4°, Londres, 1854, p. 14.}. Ce monument est encore uni dans le pays aux derniers restes de certaines cérémonies anciennes qui s’y pratiquent comme dans tous les cantons où se conserve un fond de population celtique \footnote{C’est là « qu’on allume le premier feu des \emph{brandons}, qui sert de signal pour le feu des autres contrées ». \emph{Ibid., note D}. ‑ Ces feux remontent aux mêmes usages païens que les bûchers de la Saint-Jean en France, et le jeu des torches qu’on lance en l’air en Bretagne. Les courses de flambeaux dans le Céramique, à Athènes, avaient aussi une origine non pas hellénique, mais pélasgique.}.\par
Ce bas-relief a ses analogues dans les statues grossières appelées \emph{baba}, que tant de collines des bords du Jenisseï, de l’Irtisch, du Samara, de la mer d’Azow, de tout le sud de la Russie, portent encore. Il est, comme elles, marqué d’une manière évidente du type mongol. Ammien Marcellin faisait foi de cette circonstance ; Ruysbock l’a encore remarquée au XIII\textsuperscript{e} siècle, et au XVIII\textsuperscript{e}, Pallas l’a relevée \footnote{\emph{Ibid.}}. Enfin, une coupe de cuivre, trouvée dans un tumulus du gouvernement d’Orenbourg, est ornée d’une figure semblable, et, pour qu’il ne subsiste pas le plus léger doute sur les personnages qu’on a voulu reproduire, un des \emph{babas} du musée de Moscou a une tête d’animal, et offre ainsi l’image incontestable d’un de ces Neures qui jouissaient de la faculté de se transformer en loups \footnote{Hérod., IV, 105.}.\par
Les deux particularités saillantes de ces représentations humaines sont la nature mongole, non moins fortement accusée sur le bas-relief du mont Salève que sur les monuments russes, et aussi cet objet cylindrique, de longueur moyenne, que l’on y remarque toujours tenu à deux mains par la figure. Or les légendes bretonnes consi­dèrent comme l’attribut principal des Khorrigans un petit sac de toile qui contient des crins, des ciseaux et autres objets destinés à des usages magiques. Le leur enlever, c’est les jeter dans le plus grand embarras, et il n’est pas d’efforts qu’ils ne fassent pour le ressaisir.\par
On ne peut voir dans ce sac que la poche sacrée où les Chamans actuels conservent leurs objets magiques, et qui, en effet, est absolument indispensable, ainsi que ce qu’elle contient, à l’exercice de leur profession. Les \emph{babas} et la pierre genevoise don­nent donc, indubitablement, le portrait matériel des premiers habitants de l’Europe \footnote{Il est encore évident que je ne me prononce pas plus sur l’âge de la pierre du mont Salève que sur celui des babas russes. Il me suffit de trouver dans ces monuments une représentation, soit réelle, soit légendaire, qui s’applique, avec une exactitude complète, aux êtres qu’elle a pour but de figurer.} : ils appartenaient aux tribus finniques.
\section[{V.2. Les Thraces. ‑ Les Illyriens. ‑ Les Étrusques. ‑ Les Ibères.}]{V.2. \\
Les Thraces. ‑ Les Illyriens. ‑ Les Étrusques. ‑ Les Ibères.}
\noindent Quatre peuples, dignes du nom de peuples, se montrent enfin dans les traditions de l’Europe méridionale, et viennent disputer aux Finnois la possession du sol. Il est impossible de déterminer, même approximativement, l’époque de leur apparition. Tout ce qu’on peut admettre, c’est que leurs plus anciens établissements sont bien antérieurs à l’an 2000 avant Jésus-Christ. Quant à leurs noms, la haute antiquité grecque et romaine les a connus et révérés, et même, en certains cas, honorés de mythes religieux. Ce sont les Thraces, les Illyriens, les Étrusques et les Ibères.\par
Les Thraces étaient, à leur début et probablement lorsqu’ils résidaient encore en Asie, un peuple grand et puissant, La Bible garantit le fait, puisqu’elle les nomme parmi les fils de Japhet \footnote{La \emph{Genèse} les appelle \emph{Thiras} (mot hébreu) Hérodote affirme qu’après les Indiens, les Thraces sont la nation la plus nombreuse de la terre, et qu’il ne leur manque pour être irrésistibles aux autres peuples que l’union. Ils étaient divisés autant que possible. (V, 3.)}.\par
Les tribus jaunes, quand on les trouve pures, étant, en général, peu guerrières, et le sentiment belliqueux diminuant dans un peuple à mesure que la proportion de leur sang y augmente, il y a lieu de croire que les Thraces n’appartenaient pas à leur parenté étroite. Puis les Grecs en parlent fort souvent aux temps historiques. Ils les employaient, concurremment avec des mercenaires issus des tribus scythiques, en qualité de soldats de police, et, s’ils se récrient sur leur grossièreté \footnote{ \noindent Horace reproduit cette opinion au début de l’ode XXVII du 1\textsuperscript{er} livre\par
 
\begin{verse}
Natis in usum lætitiæ scyphis\\
 Pugnare Thracum est ; tollite barbarum\\
 Morem...\\
\end{verse}
}, nulle part ils ne paraissent avoir été frappés de cette bizarre laideur qui est le partage de la race finnoise. Ils n’auraient pas manqué, s’il y avait eu lieu, de nous parler de la chevelure clairsemée, du défaut de barbe, des pommettes pointues, du nez camard, des yeux bridés, enfin de la carnation étrange des Thraces, si ceux-ci avaient appartenu à la race jaune \footnote{Une anecdote conservée par les polygraphes donne lieu de supposer, au contraire, que le type du Thrace était fort beau. C’est celle qui a trait au jeune Smerdiès, esclave issu de cette nation, aimé de Polycrate de Samos et d’Anacréon. Il était surtout remarquable par sa chevelure, que le tyran lui fit couper pour faire pièce au poète. Le nom même de Smerdiès est arian.}. Du silence des Grecs sur ce point, et de ce qu’ils ont toujours semblé considérer ces peuples comme pareils à eux-mêmes, sauf la rusticité, j’induis encore que les Thraces n’étaient pas des Finnois.\par
Si l’on avait conservé d’eux quelque monument figuré certain pour les époques vraiment anciennes, voire seulement des débris de leur langue, la question serait simple. Mais de la première classe de preuves, on est réduit à s’en passer tout à fait. Il n’y a rien. Pour la seconde, on ne possède guère qu’un petit nombre de mots, la plupart allégués par Dioscoride \footnote{\emph{Dioscor. lib. octo græce et latine}, in-12, Paris, 1589, 1 IV, cap. XV. ‑ Voir aussi quelques mots dans Strabon : (mot grec), \emph{scansores fumi} ; (mot grec), \emph{conditores} ; (mot grec), absque fœminis viventes. (VII, 33, etc.)}.\par
Ces faibles restes linguistiques semblent autoriser à assigner aux Thraces une origine ariane \footnote{M. Munsch trouve à tous les mots thraces une physionomie décidément indo-européenne. (\emph{Trad. all.} de Claussen, p. 13.) Suivant cet auteur, on les rapproche aisément de racines lettones et slaves. (\emph{Ibid}.) Plusieurs noms de lieux thraces sont clairement arians, comme, par exemple, le mot \emph{Hémus}, corrélatif au sanscrit \emph{hima}, neige. ‑ D’après \emph{Athénée}, 13, 1, Philippe de Macédoine, père d’Alexandre, avait épousé \emph{Méda}, fille d’un certain (mot grec), Thrace. ‑ Étienne de Byzance nomme cette femme (nom grec). Jornandès nomme le père \emph{Gothila}, et la fille \emph{Medopa}. Tous ces mots sont arians, mais l’époque où on les trouve est assez basse.}. D’autre part, ces peuples paraissent avoir éprouvé un vif attrait pour les mœurs grecques. Hérodote en fait foi. Il y voit la marque d’une parenté qui leur permettait de comprendre la civilisation au spectacle de laquelle ils assistaient ; or l’autorité d’Hérodote est bien puissante \footnote{Il n’hésite pas, non plus, un instant, à les confondre absolument avec les Gètes, Arians incontestables. (V, 3.)}. Il faut se rappeler, en outre, Orphée et ses travaux. Il faut tenir compte du respect profond avec lequel les chroniqueurs de la Grèce parlent des plus anciens Thraces, et de tout cela on devra conclure que, malgré une décadence irrémédiable, amenée par les mélanges, ces Thraces étaient une nation métisse de blanc et de jaune, où le blanc arian avait dominé jadis, puis s’était un peu trop effacé, avec le temps, au sein d’alluvions celtiques très puissantes et d’alliages slaves \footnote{Rask en fait des Arians sans donner aucune preuve à l’appui de son opinion. Il ne tient pas compte des différences notables existant entre ces peuples et les Hellènes, différences qui semblent s’opposer, jusqu’à présent, non pas à ce qu’on reconnaisse entre eux un degré d’affinité, mais à ce qu’on rapporte l’ensemble de leurs origines à la même source. ‑ Consulter à ce sujet Pott,\emph{ Encycl. Ersch u. Gruber, indo-germ. Sprachst}., p. 255. ‑ Comme indice à l’appui du mélange des Thraces avec des nations celtiques, je ferai remarquer combien se ressemblent les noms des villes de (nom grec), très antique cité de la Thrace, et de \emph{Vesuntio}, ville gallique dont la fondation se perd dans la nuit des temps. À la vérité, Byzance fut colonisé par Mégare, mais certainement sur l’emplacement d’une bourgade indigène. Le nom n’a rien de grec.}.\par
Pour découvrir le caractère ethnique des Illyriens, les difficultés ne sont pas moindres, mais elles se présentent autrement, et les moyens de les aborder sont tout autres. Des adorateurs de \emph{Xalmoxis} \footnote{Le nom de cette divinité paraît être de provenance slave, et se rattacher au mot szalmas, \emph{casque}. ‑ Munch, \emph{trad allem.} de Claussen, p. 13.} il n’est rien demeuré. Des Illyriens, au contraire, appelés aujourd’hui Arnautes ou Albanais, il reste un peuple et une langue qui, bien qu’altérés, offrent plusieurs singularités saisissables.\par
Parlons d’abord de l’individualité physique. L’Albanais, dans la partie vraiment nationale de ses traits, se distingue bien des populations environnantes. Il ne ressemble ni au Grec moderne ni au Slave. Il n’a pas plus de rapports essentiels avec le Valaque. Des alliances nombreuses, en le rapprochant physiologiquement de ses voisins, ont altéré considérablement son type primitif, sans en faire disparaître le caractère propre. On y reconnaît, comme signes fondamentaux, une taille grande et bien proportionnée, une charpente vigoureuse, des traits accusés et un visage osseux qui, par ses saillies et ses angles, ne rappelle pas précisément la construction du \emph{facies} kalmouk, mais fait penser au système d’après lequel ce \emph{facies} est conçu. On dirait que l’Albanais est au Mongol comme est à ce dernier le Turk, surtout le Hongrois. Le nez se montre saillant, proéminent, le menton large et fortement carré. Les lignes, belles d’ailleurs, sont rudement tracées comme chez le Madjar, et ne reproduisent, en aucune façon, la délicatesse du modelé grec. Or, puisqu’il est irrécusable que le Madjar est mêlé de sang mongol par suite de sa descendance hunnique \footnote{T. I, p. 221 et pass.}, de même je n’hésite pas à conclure que l’Albanais est un produit analogue.\par
Il serait à désirer que l’étude de la langue vînt donner son appui à cette conclusion. Malheureusement cet idiome mutilé et corrompu n’a pu jusqu’ici être analysé d’une manière pleinement satisfaisante \footnote{L’ouvrage de M. de Xylander, \emph{die Sprache der Albanesen oder Schkipetaren}, 1835, est à bon droit estimé ; mais le livre que vient de publier M. de Hahn, \emph{Albanesische Studien}, in-8°, Wien, 1853, est beaucoup plus complet. Écrit sur les lieux et loin de tout secours scientifique, cet ouvrage excellent sera d’un grand secours aux philologues qui vendront faire entrer l’albanais dans le cercle des études comparées.}. Il faut en élaguer d’abord les mots tirés du turk, du grec moderne, des dialectes slaves, qui s’y sont amalgamés récemment en assez grand nombre, Puis on aura encore à écarter les racines helléniques, celtiques et latines. Après ce triage délicat, il reste un fond difficile à apprécier, et dont jusqu’à présent on n’a pu rien affirmer de définitif, si ce n’est qu’il n’est rien moins que parent de l’ancien grec. On n’ose donc l’attribuer à une branche de la famille ariane. Est-on en droit de croire que cette affinité absente est remplacée par un rapport avec les langues finni­ques ? C’est une question jusqu’à présent irrésolue. Force est donc de s’accommoder provisoirement du doute, de rejeter toutes démonstrations philologiques trop hâtives et de se borner à celles que j’ai tirées précédemment de la physiologie. Je dirai donc que les Albanais sont un peuple blanc, arian, directement mélangé de jaune, et que, s’il est vrai qu’il ait accepté des nations au milieu desquelles il a vécu un langage étranger à son essence, il n’a fait en cela qu’imiter un assez grand nombre de tribus humaines, coupables du même tort \footnote{T. I., p. 329 et 344.}.\par
Les Thraces et les Illyriens \footnote{ \noindent L’Illyrie a changé très fréquemment d’étendue et de limites. Elle a embrassé les races les plus diverses sous une même dénomination. Cc fut d’abord le pays riverain de l’Adriatique, entre la Neretwa au nord et le Drinus au sud. Les Triballes formaient la frontière de l’Est.\par
 Ensuite, cette circonscription s’étendit depuis le territoire des Taurisques Celtes jusqu’à l’Épire et la Macédoine. La Mœsie y était comprise. Après le second siècle de notre ère, l’Illyrie, s’agrandissant encore, contint les deux Noriques, les deux Pannonies, la Valérie, la Savoie, la Dalmatie, les deux Dacies, la Mœsie et la Thrace. Enfin Constantin en détacha ces deux dernières provinces, mais y réunit la Macédoine, la Thessalie, l’Achaïe, les deux Épires, Prævallis et la Crète. À cette époque, l’Illyrie contenait dix-sept provinces. C’est probablement par suite de cette organisation administrative qu’à un certain moment on a confondu les Thraces et les Illyriens comme n’étant qu’un même peuple. Cette opinion est d’ailleurs soutenable ; quelques Grecs l’ont anciennement professée. ‑ Schaffarik, \emph{Slawische Alterthümer}, t. I, p. 257.
} ont assez noblement soutenu leur origine ariane pour n’en pas être déclarés indignes. Les premiers avaient pris une grande part à l’invasion des peuples arians hellènes dans la Grèce.\par
Les seconds, en se mêlant aux Grecs Épirotes, Macédoniens et Thessaliens, les ont aidés à gravir jusqu’à la domination de l’Asie antérieure \footnote{Pott, \emph{ouvr. cité}, p. 64.} Si, dans les temps historiques, les deux groupes auxquels sont donnés les noms de Thraces et d’Illyriens ont toujours, malgré leur énergie et leur intelligence reconnues, été réduits, en tant que nations, à un état subalterne, se contentant, au moins pour les derniers, de fournir en abondance des individualités illustres d’abord à la Grèce, puis aux empires romain et byzantin, enfin à la Turquie, il faut attribuer ce phénomène à leur fractionnement amené par des hymens locaux de valeurs différentes, à la faiblesse relative des groupes, et à leur séjour au milieu de tribus prolifiques, qui, les contenant dans des territoires montagneux et infertiles, ne leur ont jamais permis de se développer sur place. En tout état de cause, les Thraces et les Illyriens, considérés indépendamment de leurs alliages, représentent deux rameaux humains singulièrement bien doués, vigoureux et nobles, où l’essence ariane se fait très aisément deviner. Je me transporte maintenant à l’autre extrémité de l’Europe méridionale. J’y trouve les Ibères, et, avec eux, l’obscurité historique paraît s’amoindrir. Il serait oiseux de rappeler tous les efforts tentés jusqu’ici pour déterminer la nature de ce peuple mystérieux dont les Euskaras ou Basques actuels sont, avec plus ou moins de justesse, considérés comme les représentants. Le nom de ce peuple s’étant rencontré dans le Caucase, on a cherché à établir une sorte de ligne de route par laquelle il serait venu de l’Asie en Espagne \footnote{Ewald, \emph{Gescbichte des Volkes Israel}, t, I, p. 336. Ce savant ajoute que les Ibères du Caucase devaient appartenir à la souche de \emph{Hebr}. Ce qui rendrait le rapprochement avec les Ibères d’Espagne impossible ; mais rien ne prouve que la supposition soit exacte. ‑ Ce qui donne du prix au rapprochement du nom des Ibères du Caucase de celui des Ibères d’Espagne, c’est ce fait qu’une montagne de la Grèce continentale s’est très anciennement appelée les \emph{Pyrénées}, tandis qu’un fleuve de la Thrace se nommait l’\emph{Hèbre}. Ce sont là des jalons dignes d’être remarqués.}. Ces hypothèses sont demeurées fort obscures. On sait mieux que la famille ibérique a couvert la péninsule, habité la Sardaigne, la Corse, les îles Baléares, quelques points, sinon toute la côte occidentale de l’Italie. Ses enfants ont possédé le sud de la Gaule jusqu’à l’embouchure de la Garonne, couvrant ainsi l’Aquitaine et une partie du Languedoc.\par
Les Ibères n’ont laissé aucun monument figuré, et il serait impossible d’établir leur caractère physiologique, si Tacite ne nous en avait parlé \footnote{Dieffenbach, \emph{Celtica II}, 2\textsuperscript{e} Abth., p. 10, Toutefois le passage de Tacite n’est pas très concluant, et on peut lui opposer d’autres autorités, comme celle de Silius Italicus, qui fait les habitants de l’Espagne blonds. Mais à ces contradictions apparentes il y a à dire que l’Espagne contenait, à l’époque romaine, des populations de descendances bien diverses, et qu’il devait être fort difficile déjà d’y rencontrer un Ibère de race pure.}. Suivant lui, ils étaient bruns de peau et de petite taille. Les Basques modernes n’ont pas conservé cette apparence. Ce sont visiblement des métis blancs à la manière des populations voisines. Je n’en suis pas surpris. Rien ne garantit la pureté du sang chez les montagnards des Pyrénées, et je ne tirerai pas de l’examen qu’on en a pu faire les mêmes résultats que pour le guerrier albanais.\par
Dans celui-ci j’ai vu une différence marquée, un contraste notable avec les nations avoisinantes. Impossible de confondre des Arnautes avec des Turcs, des Grecs, des Bosniaques. Il est très difficile, au contraire, de démêler un Euskara parmi ses voisins de la France et de l’Espagne. La physionomie du Basque, très avenante assurément n’offre rien de particulier. Son sang est beau, son organisation énergique ; mais le mélange, ou plutôt la confusion des mélanges, est évidente chez lui. Il n’a nullement ce trait des races homogènes, la ressemblance des individus entre eux, ce qui a lieu à un haut degré chez les Albanais.\par
Comment d’ailleurs Tibère des Pyrénées serait-il de race pure ? La nation entière a été absorbée dans les mélanges celtiques, sémitiques, romains, gothiques. Quant au noyau, réfugié dans les vallées hautes des montagnes, on sait que des couches nom­breuses de vaincus sont venues successivement chercher un asile autour et auprès de lui. Il ne peut donc être resté plus intact que les Aquitains et les Roussillonais.\par
La langue euskara n’est pas moins énigmatique que l’albanais \footnote{Les Romains étaient extrêmement rebutés par sa rudesse. ‑ Dieffenbach, \emph{Celtica II}, 2\textsuperscript{e} Abth., p. 48-49.}. Les savants ont été frappés de l’obstination avec laquelle elle se refuse à toute annexion à une famille quelconque. Elle n’a rien de chamitique et peu d’arian. Les affinités jaunes paraissent exister chez elle \footnote{On croit apercevoir dans l’euskara quelques racines finnoises. ‑ Schaffarik, \emph{Stawische Alterthümer}, t. I, p. 35 et 293.}, mais cachées, et on ne les constate qu’approximativement. Le seul fait bien avéré jusqu’ici, c’est que, par son polysynthétisme, par sa tendance à incor­porer les mots les uns dans les autres, elle se rapproche des langues américaines \footnote{Prescott, \emph{History of the Conquest of Mexico}, t. III, p. 244, définit ainsi cette organisation idiomatique : « A system which bringing the greatest number of ideas within the smallest possible « compass, condenses whole sentences into a single word. » ‑ W. v. Humboldt, \emph{Prüfung der Untersuchungen über die Urbewohner Hispaniens}, p. 174 et sqq.}. Cette découverte a donné naissance à bien des romans plus hasardés les uns que les autres. Des hommes doués d’une imagination véhémente se sont empressés de faire passer le détroit de Gibraltar aux Ibères, de les acheminer au long de la côte occidentale de l’Afrique, de reconstruire, tout exprès pour eux, l’Atlantide, de pousser ces pauvres gens, bon gré, mal gré, et à pied sec, jusqu’aux rivages du nouveau continent. L’entreprise est hardie, et je n’oserais m’y associer. J’aime mieux penser que les affinités américaines de l’euskara peuvent avoir leur source dans le mécanisme primitivement commun à toutes les langues finniques \footnote{Dieffenbach, \emph{Celtica II}, 2\textsuperscript{e} Abth., p. 15 et sqq.}. Mais, comme ce point n’est pas encore éclairci de manière à produire une certitude, je préfère surtout le laisser à l’écart \footnote{M. Muller, \emph{Suggestions for the assistance of officers in learning the languages of the seat of war in the East}, London, 1854, considère l’agglutination comme le caractère distinctif de toutes les langues finniques. Peut-être y aura-t-il lieu, d’une part, à mieux s’expliquer sur les limites exactes de l’agglutination, et, d’une autre, à rechercher si les langues arianes elles-mêmes ne possèdent pas, de leur propre fonds, ce même procédé. L’étude des langues finniques est malheureusement bien peu avancée encore, et fait obstacle ainsi à toute connaissance définitive des autres familles d’idiomes.}.\par
Rejetons-nous sur ce que l’histoire nous apprend des habitudes et des mœurs de la nation ibère. Nous y trouverons plus de clartés conductrices.\par
Ici, la lumière saute aux yeux, et avec assez d’éclat pour détruire à peu près toutes les incertitudes. Les Ibères, lourds et rustiques, non pas barbares, avaient des lois, formaient des sociétés régulières \footnote{W. v. Humboldt, \emph{Prüfung der Untersuchungen über die Urbewohner Hispaniens}, p. 152 et pass.}. Leur humeur était taciturne, leurs habitudes étaient sombres. Ils allaient vêtus de noir ou de couleurs ternes, et n’éprouvaient pas cet amour de la parure si général chez les Mélaniens \footnote{\emph{Ibid}., p. 158.}. Leur organisation politique se montra peu vigoureuse ; car, après avoir occupé une étendue de pays à coup sûr considérable, ces peuples, chassés de l’Italie, chassés des îles et dépossédés d’une bonne partie de l’Espagne par les Celtes, le furent, plus tard encore et sans grand’peine, par les Phéniciens et les Carthaginois \footnote{Au temps de Strabon, on vantait beaucoup le développement intellectuel des habitants de la Bétique. On disait, entre autres choses, que les Turdétains avaient des poèmes et des lois dont la rédaction remontait à 6,000 ans. Il serait erroné d’attribuer à des Ibères cette littérature remarquable. Existant sur un point très anciennement sémitisé, elle n’offrait, sans aucun doute, que des originaux ou tout au plus des copies d’ouvrages chananéens ou puniques. ‑ Strabon, III, 1. ‑ D’après le géographe d’Apamée, les Ibères étaient, en guerre, plus rusés et plus adroits que braves et forts. ‑ W. v. Humboldt, \emph{ouvr. cité}, p. 153.}.\par
Enfin, et voici le point capital : ils se livraient avec succès au travail des mines \footnote{L’Espagne, dans la haute antiquité, produisait en quelques années 400 pouds d’or, c’est-à-dire autant que le Brésil et l’Oural réunis le font actuellement aux époques les plus prospères. ‑A. v. Humboldt, \emph{Asie centrale}, t. I, p. 540.}.\par
Ce labeur difficile, cette science compliquée qui consiste à extraire les métaux du sein de la terre et à leur faire subir des manipulations assez nombreuses, est incontes­tablement une des manifestations, un des emplois les plus raffinés de la pensée humaine. Aucun peuple noir ne l’a connue. Parmi les blancs, ceux qui l’ont pratiquée davantage, habitant en Asie, au-dessus des Arians, vers le nord, ont reçu dans leurs veines, par cette raison même, le mélange le plus considérable du sang des jaunes. À cette définition on reconnaît, je pense, les \emph{Slaves}. J’ajouterai que le sol de l’Espagne portait, dans son \emph{Mons Vindius}, le nom que, suivant Schaffarik, les nations étrangères, surtout les Celtes, ont toujours donné de préférence à ces mêmes Slaves, et je ne sais même si, invoquant la facilité que les langues wendes partagent avec les dialectes celtiques et italiotes pour retourner les syllabes, on ne serait pas en droit de reconnaître leur appellation nationale par excellence, le mot \emph{srb} dans le mot \emph{ibr} \footnote{La voyelle ouverte disparaît complètement dans le nom de fleuve, \emph{Ebre}.}\footnote{Le rapprochement entre \emph{srb} et \emph{ibr} n’est pas plus laborieux que celui établi par Schaffarik entre (mot grec) \emph{srb}. Quant à la signification du mot, je la trouverais volontiers dans \emph{obr, géant}, et par dérivation, \emph{un homme fort et redoutable}. Il est admissible que les émigrants blancs aient pris et conservé ce nom comme faisant contraste avec la faiblesse relative des indigènes finnois, et on verra plus tard que les énoncés scandinaves et germaniques attribuaient aux héros wendes la même exagération de taille avec le talent de forger des armes magiques.}. Cette étymo­logie tend la main à la mystérieuse peuplade homonyme reléguée dans le Caucase, et ajoute une apparence de plus à l’hypothèse que M. W. de Humboldt ne repoussait pas .\par
Les Ibères étaient donc des Slaves. J’en répète ici les raisons : peuple mélanco­lique, vêtu de sombre, peu belliqueux \footnote{Schaffarik insiste à plusieurs reprises sur l’esprit profondément pacifique et peu guerrier des nations slaves. Il les loue de se montrer, dès la plus haute antiquité, paisibles et très laborieuses. ‑ Schaffarik, t. I, p. 167.}, travailleur aux mines, utilitaire. Il n’est pas un de ces traits qui ne se laisse apercevoir aujourd’hui dans les masses du nord-est de l’Europe \footnote{Rask ne voit dans les Ibères que des Finnois, et il prétend fonder sa démonstration sur la linguistique. (Ursprung der altnordischen Sprachen, p. 112-146.)}.\par
Viennent maintenant les Rasènes \footnote{C’est le nom que ce groupe se donnait à lui-même, suivant O. Muller, \emph{die Etrusker}, p. 68. Mais Dennis, au contraire, prétend que cette dénomination appartient aux conquérants tyrrhéniens. (\emph{Die Stædte und Begræbnisse Etruriens}, t. I, p. IX.) Je le crois mal fondé dans cette opinion.} ou, autrement dit, les Étrusques de première formation. Par suite d’invasions pélasgiques, ce peuple extrêmement digne d’intérêt s’est trouvé, à une époque antérieure au X\textsuperscript{e} siècle avant notre ère, composé de deux éléments principaux, dont l’un, dernier venu, imprima à l’ensemble un élan civilisateur qui a produit des résultats importants. Je ne parle pas, en ce moment, de cette seconde période. Je m’attache uniquement à la plus grossière partie du sang, qui est en même temps la plus ancienne, et qui seule, à ce titre, doit figurer près des populations primordiales, thraces, illyriennes, ibères.\par
Les masses rasènes étaient certainement beaucoup plus épaisses que ne le furent celles de leurs civilisateurs. C’est là, d’ailleurs, un fait constant dans toutes les inva­sions suivies de conquêtes. Ce fut aussi leur langue qui étouffa celle des vainqueurs, et effaça chez ceux-ci presque toutes traces de l’ancien idiome. L’étrusque, tel que les inscriptions nous l’ont conservé, se montre assez étranger au grec et même au latin \footnote{O. Muller, \emph{die Etrusker.} Voir le monument de Pérouse et les observations de Vermiglioli. Les Romains appelaient l’étrusque une langue barbare, ce qu’ils ne disaient ni du sabin ni de l’osque. Preuve qu’ils ne le comprenaient pas.}. Il est remarquable par ses sons gutturaux et son aspect rude et sauvage \footnote{O. Muller, \emph{ouvr. cité}.}. Tous les efforts tentés pour interpréter ce qui en reste sont restés à peu près vains jusqu’à présent. M. W. de Humboldt inclinait à le considérer comme, une transition de l’ibère aux autres langues italiotes \footnote{Cette opinion est adoptée par O. Muller, \emph{ouvr. cité}, p. 68.}.\par
Quelques philologues ont émis la pensée qu’on en pourrait retrouver des vestiges dans le romansch des montagnes Rhétiennes. Peut-être ont-ils raison : cependant les trois dialectes parlés au canton des Grisons, en Suisse, sont des patois formés de débris latins, celtiques, allemands, italiens. Ils ne paraissent contenir que bien peu de mots issus d’autres sources, sauf des noms de lieux, en fort petit nombre.\par
Les monuments étrusques sont nombreux, et de différents âges. On en découvre tous les jours. Outre les ruines de villes et de châteaux, les tombeaux fournissent de précieux renseignements physiologiques. L’individu rasène, tel que le représente en ronde bosse le couvercle des sarcophages de pierre ou de terre cuite, est de petite taille \footnote{Prichard, \emph{Hist. natur. de l’homme}, t. I, p. 257. ‑ \emph{Verhandlungen der Academie von Berlin}, 1818-1819, p. 2. ‑ Abeken donne, dans son ouvrage, tabl. VIII, un dessin copié sur une peinture funéraire qui fait partie du musée de Berlin. Un des personnages surtout est remarquable par l’écrasement du visage, la protubérance d’un front très fuyant, la disposition des yeux extrêmement obliques, la grosseur des lèvres, les formes massives du corps. ‑ Voir aussi la représentation de la statuette 2-a, 2-b, tabl. VII et 4 et 5 de la même table, pour la forme pointue de la tête, qui rappelle beaucoup certains types américains. ‑ Consulter aussi Micali, \emph{Monuments antiques}, in-fol., Paris, 1824, tab. XVI, fig. 1, 2, 4 et 8 ; tab. XVII, fig. 3 ; tab. LXI, fig. 9.}. Il a la tête grosse, les bras épais et courts, le corps lourd et gros, les yeux bridés, obliques, de couleur brune, les cheveux jaunâtres. Le menton est sans barbe, fort et proéminent ; le visage plein et rond, le nez charnu. Un poète latin, en quatre mots, résume le portrait : \emph{obesos et pingues Etruscos}.\par
Toutefois, ni cette expression de Virgile, ni les images qu’elle commente si bien, ne s’appliquent, dans la pensée du poète, à des hommes de la race purement rasène. Images et descriptions poétiques se reportent aux Étrusques de l’époque romaine, de sang bien mêlé. C’est une nouvelle preuve, et preuve concluante, que l’immigration civilisatrice avait été comparativement faible, puisqu’elle n’avait pas modifié sensible­ment la nature des masses. Ainsi il suffit d’unir ces deux phénomènes de la conservation d’une langue étrangère à la famille blanche, et d’une constitution physiologique non moins distincte, pour être en droit de conclure que le sang de la race soumise a gardé le dessus dans la fusion, et s’est laissé guider, mais non pas absorber, par les vainqueurs de meilleure essence.\par
La démonstration de ce fait ressort encore mieux du mode de culture particulier aux Étrusques. Encore une fois, je ne parle pas ici de l’ensemble raséno-tyrrhénien ; je ne relève que ce qui peut m’aider à découvrir la nature véritable de la population rasène primitive.\par
La religion avait son type spécial. Ses dieux, bien différents de ceux des nations helléniques sémitisées, ne descendirent jamais sur la terre. Ils ne se montraient pas aux hommes, et se bornaient à faire connaître leurs volontés par des signes, ou par l’intermédiaire de certains êtres d’une nature toute mystérieuse \footnote{O. Muller, \emph{die Etrusker}, p. 266. Les Étrusques indigènes ne connaissaient pas le culte des héros topiques, et, par conséquent, n’avaient pas d’éponymes comme leurs vainqueurs, les Tyrrhéniens, ni comme les Grecs. Au-dessus de toutes leurs divinités, même de la plus grande, \emph{Tinia}, ils plaçaient ces êtres surnaturels que les Romains nommèrent \emph{dii involuti, les dieux enveloppés}. (Dennis, t. I, p. XXIV.) J’en ai parlé plus haut.}. En conséquence, l’art d’interpréter les obscures manifestations de la pensée céleste fut la principale occupation des sacerdoces. L’aruspicine et la science des phénomènes naturels, tels que les orages, la foudre, les météores \footnote{ \noindent Les sources minérales et leurs chaudes exhalaisons étaient aussi un grand objet d’épouvante religieuse ;\par
 
\begin{verse}
At tex sollicitus monstris, oracula Fauni\\
 Fatidici genitoris, adit, lucosque sub alta\\
 Consulit Albunea ; nemorum quæ maxima sacro\\
 Fonte sonat, sævamque exhalat opaca mephitim.\\
 Hinc Italæ gentes, omnisque OEnotria tellus,\\
 In dubiis responsa petunt. Huc dona sacerdos\\
 Quum tulit, et cæsarum ovium sub nocte silenti\\
 Pellibus incubuit stratis, somnosque petivit :\\
 Multa modis simulacra videt volitantia miris,\\
 Et varias audit voces, fruiturque deorum\\
 Colloquio, arque imis Acheronta affatur Avernis.\\
\end{verse}
 
\bibl{\emph{Æn.}, VII, 81-91}
}, absorbèrent les méditations des pontifes, et leur créèrent une superstition beaucoup plus étroite et plus sombre, plus méticuleuse, plus subtile, plus puérile que cette astrologie des Sémites, qui, au moins, avait pour elle de s’exercer dans un champ immense et de s’adonner à des mystères vraiment splendides. Tandis que le prêtre chaldéen, monté sur une des tours dont le relief de Babylone ou de Ninive était hérissé, suivait d’un œil curieux la marche régulière des astres semés à profusion dans les cieux sans limites, et apprenait peu à peu à calculer la courbe de leurs orbites, le devin étrusque, gros, gras, court, à large face, errant, triste et effaré, dans les forêts et les marécages salins qui bordent la mer Tyrrhénienne, interprétait le bruit des échos, pâlissait aux roulements de la foudre, frissonnait quand le bruissement des feuilles annonçait à sa gauche le passage d’un oiseau, et cherchait à donner un sens aux mille accidents vulgaires de la solitude. L’esprit du Sémite se perdait dans des rêveries absurdes sans doute, mais grandes comme la nature entière, et qui emportaient son imagination sur des ailes de la plus vaste envergure. Le Rasène traînait le sien dans les plus mesquines combinaisons, et, si l’un touchait à la folie en voulant lier la marche des planètes à celle de nos existences, l’autre rasait l’imbécillité en cherchant à découvrir une connexité entre la danse capricieuse d’un feu follet et tels événements qu’il lui importait de prévoir. C’est là précisément le rapport entre les égarements de la créature hindoue, suprême expression du génie arian mêlé au sang noir, et ceux de l’esprit chinois, type de la race jaune animée par une infusion blanche. En suivant cette indication, qui donne pour dernier terme aux erreurs des premiers la démence, et aux aberrations des seconds l’hébétement, on voit que les Rasènes tombent dans la même catégorie que les peuples jaunes, faiblesse d’imagination, tendance à la puérilité, habitudes peureuses.\par
Pour la faiblesse d’imagination, elle est démontrée par cette autre circonstance que la nation étrusque, si recommandable à quelques égards, et douée d’une véritable aptitude historique \footnote{Elle donna aux Romains le modèle de leurs annales ; mais il semble que ce n’étaient que des catalogues de faits sans autre liaison que la chronologie, et tout à fait dénués de grâces narratives. Valérius Flaccus, entre autres, et l’empereur Claude se servirent de chroniques étrusques pour composer leurs histoires. (Abeken, \emph{ouvr. cité}, p. 20.)}, n’a rien produit dans la littérature proprement dite que des traités de divination et de discipline augurale. Si l’on y ajoute des rituels, établissant avec les moindres détails l’enchaînement complexe des offices religieux, on aura tout ce qui occupait les loisirs intellectuels d’un peuple essentiellement formaliste \footnote{O. Muller, \emph{ouvr. cité}, p. 281 et peu.}. Pour unique poésie, la nation se contentait d’hymnes contenant plutôt des énumérations de noms divins que des effusions de l’âme. À la vérité, une époque assez postérieure nous montre dans une ville étrusque, Fescennium, un mode de compositions qui, sous forme dramatique, fit longtemps les délices de la population romaine. Mais ce genre de jouissance même démontre un goût peu délicat. Les vers fescennins n’étaient qu’une sorte de catéchisme poissard, un tissu d’invectives dont le mérite était la virulence, et qui n’empruntait aucune de ses qualités au charme de la diction, ni, bien moins encore, à l’élévation de la pensée. Enfin, tout pauvre que serait cet unique exemple d’aptitude poétique, on ne peut encore en attribuer complètement soit l’invention, soit la confection, aux Rasènes ‑ car, si Fescennium comptait parmi leurs villes, elle était surtout peuplée d’étrangers, et, en particulier, de Sicules \footnote{O. Muller, \emph{ouvr. cité}, p. 183. ‑ Sur l’incapacité poétique des Étrusques, voir Niebuhr, Rœm. Gescbichte, t. I, p. 88.}.\par
Ainsi, privés de besoins et de satisfactions d’esprit, il faut chercher le mérite des Rasènes sur un autre terrain. Il faut les voir agriculteurs, industriels, fabricants, marins et grands constructeurs d’aqueducs, de routes, de forteresses, de monuments utiles \footnote{O. Muller, \emph{ouvr. cité}, p. 260. Abeken, p. 31 et 164, et pass. ‑ On trouve des traces de ces travaux de mines si dignes de remarque, ethniquement parlant, à Populonia et à Massa Marittima. On en extrayait du cuivre.}. Les jouissances, et, pour me servir d’une expression devenue technique, les intérêts matériels étaient la grande préoccupation de leur société. Ils furent célèbres, dans l’antiquité la plus haute, par leur gourmandise et leur goût des plaisirs sensuels de toute espèce \footnote{Idem, \emph{ouvr. cité}. ‑ Les Étrusques employaient les femmes à la divination et aux choses du culte. C’est une coutume finnique, comme on le verra plus bas. ‑ Dennis, t. I, p. XXXII.}. Ce n’était pas un peuple héroïque, tant s’en faut ; mais je m’imagine que, s’il venait à sortir aujourd’hui de ses tombes, il serait, de toutes les nations du passé, celle qui comprendrait le plus vite la partie utilitaire de nos mœurs modernes et s’en accommoderait le mieux, Pourtant l’annexion à l’empire chinois lui conviendrait davantage encore.\par
De toute façon, l’Étrusque semblait un anneau détaché de ce peuple. Chez lui, par exemple, se présente avec éclat cette vertu spéciale des jaunes, le très grand respect du magistrat \footnote{O. Muller, \emph{die Etrusker}, p. 375.}, uni au goût de la liberté individuelle, en tant que cette liberté s’exerce dans la sphère purement matérielle. Il y a de cela chez les Ibères, tandis que les Illyriens et les Thraces paraissent avoir compris l’indépendance d’une manière beaucoup plus exigeante et plus absolue. On ne voit pas que les populations rasènes, dominées par leurs aristocraties de race étrangère, aient possédé une part régulière dans l’exercice du pouvoir. Cependant, comme on ne trouve pas non plus chez elles le despotisme sans frein et sans remords des États sémitiques, et que le subordonné y jouissait d’une somme suffisante de repos, de bien-être, d’instruction, l’instinct primordial de ce dernier devait se rapprocher beaucoup plus des dispositions à l’isolement individuel, qui caractérisent l’espèce finnique, que des tendances à l’agglomération, inhérentes à la race noire, et qui la privent tout aussi bien de l’instinct de la liberté physique que du goût de l’indépendance morale.\par
De toutes ces considérations, je conclus que les Rasènes, lorsqu’on les dégage de l’élément étranger apporté par la conquête tyrrhénienne, étaient un peuple presque entièrement jaune, ou, si l’on veut, une tribu slave médiocrement blanche \footnote{Abeken, assez empêché de trouver un nom à l’élément étrusque de première formation, l’appelle pélasgique, et, lorsqu’il veut définir ce qu’il entend par ce mot, il ne sait pas s’en tirer autrement qu’en l’expliquant par le mot plus obscur et plus vague encore d’\emph{urgriechisch} (\emph{hellénique primitif}). Chez lui, le sens définitif paraît être de rattacher les Étrusques indigènes à la souche ariane. Cette opinion semblera, je n’en doute pas, tout à fait inadmissible. (Abeken, \emph{Mittel-Italien vor der Zeit der rœmiscben Herrschaft}, p. 24.) ‑ Du reste, autant de savants qui se sont occupés de cette question, autant d’avis. Dans l’antiquité, Hérodote fait des Étrusques indigènes un peuple lydien, et la plupart des historiens se rangent à son opinion. Denys d’Halicarnasse s’en éloigna le premier et les déclara aborigènes, mais sans dire ce qu’il entendait par ce mot. O. Muller voit en eux une race à part, au milieu des populations italiotes. Lepsius n’admet ni des autochtones, ni même plus tard une conquête tyrrhénienne. À ses yeux, l’élément constitutif était formé de peuples umbriques qui, vaincus par des Pélasges, parvinrent à dominer leurs maîtres, et créèrent ainsi ure nouvelle combinaison nationale qui produisit les Étrusques. Sir William Betham assure que les Rasènes, les Tyrrhéniens, et autres groupes qu’on distingue dans ce peuple, sont autant de fantômes. Il n’aperçoit là que des Celtes, et passe légèrement sur les objections. Son but est de donner une illustre parenté aux Irlandais. Dennis, après avoir énuméré tous ces sentiments si divers, se rallie purement et simplement à la bannière d’Hérodote. (Dennis\emph{, die Stædte und Regræhnisse Etruriens}, t. I, p. IX et pass.) Niebuhr fait venir les Étrusques indigènes des montagnes Rhétiennes. (\emph{Rœmische Geschichte}, in-8°, Berlin, 1811, t. I, p. 74 et pass.)}.\par
J’ai porté un jugement analogue sur les Ibères, différents cependant des Étrusques par le nombre et la quotité des mélanges. De leur côté, les Illyriens et les Thraces, chacun avec des mœurs spéciales, m’ont présenté de fortes apparences d’alliages finnois. C’est une nouvelle démonstration, mais cette fois \emph{a posteriori}, et ce ne sera pas la dernière ni la plus frappante, que le fond primitif des populations de l’Europe méridionale est jaune. Il est bien clair que cet élément ethnique ne se trouvait pas à l’état pur chez les Ibères, ni même chez les Étrusques de première formation. Le degré de perfectionnement social auquel ces nations étaient parvenues, bien qu’assez humble, indique la présence d’un germe civilisateur qui n’appartient pas à l’élément finnois, et que cet élément a seulement la puissance de servir dans une certaine mesure.\par
Considérons donc les Ibères, puis, après eux, les Rasènes, les Illyriens et les Thraces, toutes nations de moins en moins mongolisées, comme ayant constitué les avant-gardes de la race blanche en marche vers l’Europe. Elles ont éprouvé avec les Finnois les contacts les plus directs ; elles ont acquis au plus haut degré l’empreinte spéciale qui devait distinguer l’ensemble des populations de notre continent de celles des régions méridionales du monde.\par
La première et la seconde émigration, Ibères et Rasènes, contraintes de se diriger vers l’extrême occident, attendu que le sud asiatique était déjà occupé par des déplacements arians, percèrent à travers des couches épaisses de nations finniques déjà éparpillées devant leurs pas. Par suite d’alliages inévitables, elles devinrent rapidement métisses, et l’élément jaune domina chez elles.\par
Les Illyriens, puis les Thraces, gravitèrent, à leur tour, sur des chemins plus rapprochés de la mer Noire. Ils eurent ainsi des contacts moins forcés, moins multi­pliés, moins dégradants avec les hordes jaunes. De là, une apparence physique et une énergie supérieure, et, tandis que les Ibères et les Rasènes furent destinés de bonne heure à l’asservissement, les Thraces maintinrent un rang convenable jusqu’au jour beaucoup plus tardif où ils se fondirent, non sans honneur encore, dans les populations ambiantes, Quant aux Illyriens, ils vivent aujourd’hui et se font respecter.
\section[{V.3. Les Galls.}]{V.3. \\
Les Galls.}
\noindent Puisque les émigrations des Ibères et des Rasènes, celles des Illyriens et des Thraces ont précédé tout autre établissement des familles blanches dans le sud de l’Europe, on doit considérer comme démontré que, lorsque les Ibères ont traversé la Gaule du nord au sud, et les Rasènes la Pannonie et un coin des Alpes Rhétiennes, pour gagner leurs demeures connues, aucune nation de race noble n’était sur leur chemin pour leur barrer le passage. Ibères et Rasènes ne formaient que des corps détachés des grandes multitudes slaves déjà établies dans le nord du continent, et que harcelaient en plus d’un lieu d’autres nations parentes, les Galls.\par
L’ensemble de la famille slave n’ayant joué aucun rôle de quelque importance aux époques antiques, il est inutile d’en parler en ce moment. Il suffit d’avoir indiqué son existence en Espagne, en Italie, et d’ajouter qu’établie, fortement au long de la mer Baltique, dans les régions comprises entre les monts Krapacks et l’Oural, et au delà encore, nous apercevrons bientôt quelques-unes de ses tribus entraînées au milieu du torrent celtique. À l’exception de ces détails que le récit fera naître naturellement, la personnalité de ce peuple restera dans l’ombre jusqu’au moment où l’histoire l’amènera tout entier sur la scène.\par
Déterminer, même vaguement, l’époque de l’acheminement des Galls vers le nord et l’ouest présente des difficultés insurmontables. Voici tout ce qu’on peut dire à ce sujet :\par
Au XVII\textsuperscript{e} siècle avant notre ère, on voit les Galls oocupés à forcer le passage des Pyrénées, défendu par les Ibères. C’est le premier renseignement positif sur leur existence dans l’ouest. Ils occupaient cependant les contrées situées entre la Garonne et le Rhin, et avaient parcouru et possédé les rives du Danube, longtemps avant cette époque.\par
D’autre part, il n’y a pas de doute qu’en quittant l’Asie, ils ne se résignèrent à s’avancer du côté de l’ouest, beaucoup moins attrayant que le sud, et, en outre, occupé déjà par des essaims de peuples jaunes, que parce que les routes méridionales leur étaient visiblement fermées et interdites par les encombrements d’Arians en marche vers l’Inde, l’Asie antérieure et la Grèce. Dès lors, leur arrivée dans l’Europe occiden­tale, si ancienne qu’on la suppose, est de beaucoup postérieure à l’apparition des Arians sur les crêtes de l’Himalaya et des Sémites du côté de l’Arménie. Or nous avons à peu près fixé, d’après des données convenables, l’âge de cette apparition à l’an 5000. C’est donc entre cette date et l’an 2000 environ, période de 3.000 ans, qu’il faut chercher l’époque de l’établissement des Celtes dans l’ouest.\par
La lutte des Ibères et des Galls, du côté de la Garonne, au XVII\textsuperscript{e} siècle, donne naissance, on l’a déjà vu, au plus ancien récit des annales de l’Occident. Là se confirme cette observation que l’histoire ne résulte jamais que du conflit des intérêts des blancs. Nous trouvons les Ibères, gens laborieux, mais relativement faibles, aux prises avec ces multitudes de guerriers hardis et turbulents, qui longtemps firent la loi dans notre partie du monde.\par
Le nom de ces guerriers vient de \emph{Gall, fort}. J’en rapporte l’origine à une ancienne racine de la race blanche, très reconnaissable encore dans le sanscrit \emph{wala} ou \emph{walya}, qui a le même sens. Les nations sarmates et, par suite, les gothiques restèrent fidèles à cette forme, et appelèrent les Galls Walah. Les Slaves altéraient le mot davantage, et en faisaient \emph{Wlach}. Les Grecs le prononçaient (mot grec) ou (mot grec), dont les Romains firent Celtæ, pour se rabattre ensuite, couramment, à la forme plus régulière \emph{Galli} \footnote{P. Wachter, \emph{Encycl. Ersch a. Gruber, Galli}, p. 47. ‑ Le bas breton emploie aussi la forme \emph{Gallaouet}, qui garde bien le \emph{t} originaire de (mot grec). Voir, à ce sujet, les médailles où l’on trouve les formes (mots grecs) et autres. ‑ \emph{Vischer, Keltische Münzen aus Hunningen}, in-4°, Bâle, p. 17. ‑ Voir aussi Schaffarik, \emph{Slattische Alterth.}, t. I, p. 236. Cet auteur indique quelques formes intéressantes du rom : \emph{Galedin}, que s’attribuaient les Belges et qui est la racine évidente de \emph{Caledonia} ; \emph{Gaoidheal}, en usage chez les Irlandais. Les Anglo-Saxons firent de \emph{walah} le gothique \emph{vealh}, fidèlement conservé dans notre \emph{valet}. Les Anglais ont depuis abandonné cette dérivation insultante, pour cette autre, \emph{gallant}, qui se rattache à notre vaillant. Ainsi, suivant l’humeur louangeuse ou méprisante de telle tribu de conquérants, la même racine ethnique a fourni l’éloge et l’injure. Une autre transformation de Gall, c’est \emph{Wallon}, appliquée à un peuple de Belgique. Une autre encore, c’est \emph{Welche}, dans la Suisse française, etc. ‑ Schaffarik, \emph{ouvr. cité}, t. I, p. 50 et pass. ‑ On observe la trace du nom des Celtes dans certaines appellations de localités modernes, comme dans \emph{Chaumont} = \emph{Kaldun}, où la dernière syllabe est traduite ; dans \emph{Châlons}, dans l’expression \emph{pays de Caux}. Voir aussi la longue et savante dissertation de P.-L. Dieffenbach, \emph{Celtica II}, in-8°, Stuttgart, 1840, 1\textsuperscript{re} Abth., p. 9 et sqq., qui me paraît épuiser la matière.}.\par
 Outre ce nom, les Galls en avaient un autre : celui de \emph{Gomer}, inscrit dans les généalogies bibliques, au nombre des fils de Japhet \footnote{(Mot hébreu) Les Arméniens, en transcrivant ce mot dans leurs chroniques, en ont fait \emph{Gamir}. Je n’ose décider s’ils le possèdent directement ou s’ils l’ont simplement emprunté à des traditions étrangères. Cependant la première hypothèse est d’autant plus soutenable qu’ils étaient eux-mêmes alliés de très près aux Celtes. Il y a plus : à examiner le nom que la Bible leur a appliqué à eux-mêmes, ils ne sont qu’une branche détachée de ces Gomers ou Gamirs ; ils s’appellent dans la \emph{Genèse} (X, 3), \emph{Thogarma}, (mot hébreu) les propres fils de Gourer. C’est ici le lieu de dire quelques mots de la généalogie japhétide. La chronique mosaïque ne la pousse pas très loin, et n’entend évidemment donner, à ce sujet, qu’un renseignement tout à fait fragmentaire. Il n’est question ni du gros des peuples zoroastriens, ni, à plus forte raison, des Hindous. Je ne signale que les deux lacunes les plus apparentes, En tête des fils de Japhet se trouve Gomer. C’est donc, dans la pensée biblique, le peuple le plus important, le plus considérable de la famille, par la puissance et le nombre. Au temps d’Ézéchiel, on pensait encore de même à Jérusalem et le prophète s’écriait : « Gomer et toutes ses troupes, la maison de Thogarma, les flancs de l’Aquilon et toute sa force et ses peuples nombreux. » (38,6.) ‑ Ainsi les Celtes unis aux Arméniens, comme ne formant qu’une seule race, c’est là pour les Hébreux la grande nation japhétide. Après elle vient \emph{Magog}. Ce sont les peuples de la région caucasienne, probablement arians, \emph{Gog} étant la transcription sémitique de l’arian \emph{kogh}. Le livre saint les place dans un rapport d’apposition ou d’opposition avec Gomer : car le chef qui doit conduire les armées cimmériennes s’appelle Gog. Il n’y a pas hostilité entre Gog et Magog. (Ezéch. 38, 2, 3, 4.) C’est le premier qui doit commander Magog tout comme Gomer. En conséquence, je vois dans Magog une nation géographiquement voisine des Cimmériens, une nation de la même souche, blanche comme eux, pouvant se réunir à eux ; je vois dans Magog des Slaves, et ne crois pas qu’on soit fondé à y voir autre chose. ‑ Après ce peuple s’offre \emph{Madaï}, qui s’explique aisément : ce sont les Mèdes, cette fraction des Zoroastriens, la plus anciennement connue, la seule connue même des Chamites noirs et des premiers Sémites (t. I, p. 469). Il est naturel que la Genèse ne cite qu’elle. Après Madaï se trouve \emph{Javan}. J’ai montré ailleurs (voir t. I\textsuperscript{er}) les différentes destinées de ce mot. On ne saurait lui attribuer ici un autre sens que celui d’occidental. Ainsi \emph{Javan} n’indique ni les Ioniens ni les Grecs, mais seulement des populations établies à l’ouest de la Palestine, soit qu’on entende par là le nord, le nord-ouest ou simplement l’ouest. ‑ \emph{Thubal} succède à Javan. Les commentateurs y voient un peuple insignifiant dans le Pont, les Tibaréniens. Il en est de même pour \emph{Meschesch}, placé entre l’Ibérie, l’Arménie et la Colchide. Ces deux groupes ont pu avoir, très anciennement, une importance qui se dissipa dans les siècles suivants comme celle des \emph{Thiras}, des Thraces, dont j’ai suffisamment parlé en leur lieu. Ce dernier nom clôt la liste des produits de la première génération de Japhet. Après eux viennent les fils de Gomer et les fils de Javan, c’est-à-dire les branches de la famille les moins inconnues. Les fils de Gomer sont \emph{Thogarma} dont j’ai déjà fait mention, les \emph{Arméniens}, cités (X, 3) les troisièmes et que je cite les premiers pour en finir avec eux, puis \emph{Aschkenas} et \emph{Riphath}. Aschkenas ne s’est prêté jusqu’ici à aucune explication. Rosenmuller incline à y voir une peuplade quelconque entre l’Arménie et la mer Noire. Il me semble que c’est supposer que la géographie biblique s’appesantit bien inutilement sur une région qui ne lui tenait pas fort à cœur et où elle avait déjà mis suffisamment d’habitants, si c’est à bon droit qu’on y place déjà Thubal et Meschesch. Puisque les Aschkenas sont des fils de Gomer, des Celtes véritables, et que Gomer lui-même, c’est-à-dire la souche de la nation, a déjà été reconnu dans son plus ancien gîte, sur la côte de la mer Noire, le parti le plus simple serait peut-être d’admettre qu’Aschkenas représente les groupes de même sang placés plus à l’ouest, indéfiniment, peut-être les Slaves. Quant à Riphath, les habitants des monts Riphées, ce sont encore des Celtes, s’allongeant du côté du nord dans des contrées froides, montagneuses, vaguement entrevues, et se confondant au milieu des Carpathes avec les Aschkenas. ‑ Si les fils de Gomer paraissent assez difficiles à reconnaître, ceux de Javan, l’\emph{occidental}, ne le sont pas moins, comme le promettait, du reste, le nom de leur père. Ils apparaissent au nombre de quatre : \emph{Elischah}, les habitants de la Grèce continentale, soit ceux de l’Élide, soit ceux d’Éleusis, non pas des Hellènes, mais, beaucoup plus vraisemblablement, des aborigènes, Celtes et Slaves. (Voir plus bas, chap. IV.) \emph{Tharschisch}, les Ibères d’Espagne et, peut-être aussi, des îles voisines. \emph{Kittim}, dans l’hypothèse la plus ordinaire, les habitants de Chypre et des archipels grecs ; mais j’en doute, les premiers colons de ces îles paraissant avoir été des Sémites. Enfin, \emph{Dodanim}, les gens de l’Épire, par conséquent les Illyriens. Consulter, entre autres, à ce sujet, Rosenmuller, \emph{Biblische Geographie}, in-8°, Berlin, 1823, t. I, p. 224 pass. ; plus récemment Delitsch, \emph{die Genesis}, p. 284 et sqq. ; et Knobel, Giessen, 1850. M. Richers a également publié un livre sur ce sujet, mais je ne l’ai pas eu entre les mains. On peut tirer de ce qui précède les conclusions suivantes : la géographie japhétide de la Genèse, basée sur les souvenirs antiques des Chamites et les connaissances acquises, très peu nombreuses, des Sémites de Chaldée, n’embrasse pas, tant s’en faut, tout l’ensemble des nations blanches du nord. Les Arians n’y figurent que par l’individualité médique, les races du Caucase, les Thraces, et une combinaison ethnique au second degré, les Illyriens. On peut distinguer trois parties dans le détail : 1° les noms de\emph{ Gomer}, de \emph{Magog}, de \emph{Thubal}, de \emph{Meschesch}, de \emph{Thiras} et d’\emph{Aschkenas}, sont des appellatifs patronymiques donnés à des peuples. Ils représentent probablement les produits de la plus ancienne tradition. 2° Les mots \emph{Javan, Kittim} et \emph{Dodanim} sont des noms collectifs de peuples, acquis après le temps des premières migrations. 3° Ceux de \emph{Madaï, Riphath, Thogarma, Elischah} et \emph{Thraschisch}, véritables dénominations géographiques, indiquent des contrées plutôt que des peuples, et résultent d’une connaissance topographique déjà plus expérimentée.}. On a ainsi la mesure de l’antique notoriété d’un si puissant rameau de la famille blanche. À cette période très ancienne, où les populations sémitiques étaient encore accumulées dans les montagnes de l’Arménie, et s’adossaient au Caucase, elles ont pu, sans doute, entretenir des relations directes avec les Celtes ou Gomers, dont plusieurs nations vivaient alors sur les côtes septentrionales de la mer Noire. Cependant il est également probable que les Celtes avaient eu des contacts avec les Sémites dès avant cette époque. Les rédacteurs de la Genèse ont puisé, sans doute, plus d’un renseignement cosmogonique et historique dans les annales des Chananéens \footnote{T. I, p. 441.}, mais rien ne s’oppose à ce qu’ils aient eu les moyens de compléter ces récits par des souvenirs qui leur étaient propres, et dont la source remontait à l’âge où toute l’espèce blanche se trouvait rassemblée au fond de la haute Asie.\par
Ces Gomers, connus traditionnellement des nations chananéennes du sud, le furent plus directement des Assyriens. Il y eut, à la fin du XIII\textsuperscript{e} siècle, entre les deux peuples, des conflits et des mêlées. Inhabiles à laisser à la postérité des monuments de leurs triomphes, les Celtes en perdirent la mémoire ; mais leurs rivaux asiatiques, plus soigneux, ont gardé des traces d’exploits dont ils s’honoraient. M. le lieutenant-colonel Rawlinson a trouvé très fréquemment dans les inscriptions cunéiformes le nom des \emph{Gumiris}, entre autres, sur les pierres de Bisoutoun \footnote{Lt-col. Rawlinson, \emph{Memoir on the babylonian and assyrian Inscriptions}, 1851, p. XXI.}. C’est donc dans l’Asie occidentale que se rencontrent les premières mentions du peuple qui devait se répandre le plus loin en Europe.\par
Outre la Bible et les témoignages assyriens, l’histoire grecque aussi parle de l’invasion cimmérienne au temps de Cyaxares \footnote{T. II, p. 379.}. Ces Cimmériens, ces Gumiris, qui firent alors tant de mal, et furent si rapidement dispersés par les Scythes, nous les suivons, dès lors, au delà de l’Euxin où ils retournent, et, montant avec eux vers l’ouest et le nord-ouest, nous ne perdons plus de vue leurs vastes pérégrinations.\par
 Ils s’enfoncent jusqu’aux contrées voisines de la met du Nord, et y portent leur nom de \emph{Kimbr} ou \emph{Cimri} \footnote{La nationalité celtique des plus anciens Cimbres n’est pas contestable. Ils nommaient l’Océan, sur les bords duquel ils résidaient, Mori-Marusa. Ce sont deux mots kymriques qui veulent dire mer morte. Ils lui donnèrent aussi le nom de \emph{crow}, reproduit en latin dans la formation \emph{cronium}, autre expression kymrique qui signifie \emph{glacé}. Lorsqu’ils vinrent attaquer Marius, un de leurs chefs se nommait \emph{Boiorix} ou le chef \emph{boïen}, et, les Boïens étant des Galla incontestables, il n’y aurait aucun motif qui eût pu porter un guerrier cimbre à prendre un titre celtique, s’il n’avait pas été Celte lui-même. On retrouve encore à côté de ce même Boïorix un \emph{Lucius} ou mieux \emph{Luk}, et ce nom, très connu des Latins, leur avait été transmis par les Umbres Celtes de la péninsule italique ; il était donc gallique comme ses possesseurs.}. Ils occupent la Gaule, et lui font connaître les \emph{Kymris}. Ils s’établissent dans la vallée du Pô, et y répandent la gloire des Umbri, des Ambrones \footnote{C’est une règle celtique que le \emph{k} et le \emph{g}, deux lettres qui paraissent avoir été tout à fait confondues dans la prononciation, s’effacent souvent devant une voyelle. ‑ Aufrecht et Kirchhoff, \emph{Die umbrischen Sprachdenkmæler, Lautlehre}, p. 15 et Pass. Il y en a beaucoup d’exemples : \emph{gwiper, vipère} ; \emph{win} et\emph{ gwin, vin} ; \emph{gwir} et \emph{fire, vrai} ; \emph{gwell}, devenu l’anglais \emph{well} ; \emph{alon} et \emph{galon, étranger} etc.}. En Écosse, on connaît encore le clan de Cameron ; en Angleterre, l’Humber et la Cambrie ; en France, les villes de Quimper, de Quimperlé, de Cambrai, comme, dans les plaines du pays de Posen, le souvenir des Ombrons est resté attaché, jusqu’à nos jours, à un territoire nommé Obrz \footnote{Schaffarik, \emph{ouvr. cité}, t. I, p. 51.}.\par
On a pensé que ce nom de \emph{Gumiri}, de \emph{Kymri}, de \emph{Cimbre}, pouvait indiquer une branche de la famille celtique, différente de celle des Galls, de même que dans les Celtes on ne savait pas reconnaître ces derniers. Mais il suffit de considérer combien les deux dénominations de \emph{Gall} et de \emph{Kymri} s’appliquent souvent aux mêmes tribus, aux mêmes peuplades, pour abandonner cette distinction. D’ailleurs, les deux mots ont le même sens ou à peu près : si \emph{Gall} veut dire \emph{fort, Kymri} signifie \emph{vaillant} \footnote{M. Amédée Thierry, Hist. des Gaulois, t. I, introduction. ‑ Le nom est resté dans le danois \emph{Kiemper}, avec la signification de \emph{combattant}. ‑ Salverte, \emph{Essai sur l’origine des noms d’hommes, de peuples et de lieux}, 1821, in-8°, Paris, t. II, p. 108.}.\par
En réalité, il n’existe aucun motif de scinder les masses celtiques en deux fractions radicalement distinctes, mais on n’aurait pas moins tort de croire que toutes les branches de la famille aient été absolument semblables. Ces multitudes, accumulées des rives de la Baltique et de la mer du Nord \footnote{Je n’affirme nullement que l’inondation celtique se soit arrêtée au Danemark. ‑ « Dans le « Nord (dit Wormsaae), c’est une opinion fort répandue que les Celtes ont habité la « Scandinavie méridionale, et, à défaut de renseignements historiques, on se fonde sur la « ressemblance des armes, des instruments et des bijoux en bronze et en or, trouvés dans « nos tumulus, avec ceux qui ont été découverts en Angleterre et en France. Cette opinion a « des partisans en Norwège, et les historiens de ce pays l’ont tenue pour démontrée. » ‑ \emph{Lettre à M. Mérimée, Moniteur du 14 avril 1853}. ‑ Voir aussi Munch, \emph{ouvr. cité}, p. 8.} au détroit de Gibraltar, et de l’Irlande à la Russie \footnote{En établissant les différents flux et reflux de la famille slave, Schaffarik donne d’excellentes indications sur l’étendue des établissements celtiques, principaux compétiteurs des Wendes. Un des points qui ressortent le mieux de cet examen, c’est que, sur plus d’une frontière, il est fort difficile de distinguer les deux groupes. (Schaffarik, \emph{ouvr. cité}, t. I, p. 56, 66, 89, 104, 207, 379.)}, différaient notablement entre elles, suivant qu’elles s’étaient plus ou moins alliées ici aux Slaves, là aux Thraces et aux Illyriens, partout aux Finnois. Bien qu’issues originairement d’une même souche, elles n’avaient souvent conservé qu’une simple et lointaine parenté dont l’identité de langue, altérée d’ailleurs par des modifi­cations infinies de dialectes, était l’insigne. Du reste, elles se traitaient à l’occasion en rivales et en ennemies, ainsi que plus tard on vit les Franks austrasiens guerroyer, en toute tranquillité de conscience, contre les Francs neustriens. Elles formaient donc des réunions politiques pleinement étrangères les unes aux autres \footnote{La monnaie d’or que frappaient les États celtiques n’avait cours que sur le territoire spécial de chaque nation, parce que le titre en était toujours particulier. Bien que cette observation ne puisse s’appliquer qu’au IV\textsuperscript{e} siècle avant Jésus-Christ, comme cette époque est un temps d’indépendance bien complète pour les peuples celtiques, je conclus qu’il y a là une preuve à ajouter à toutes celles qui, par ailleurs, témoignent de l’isonomie respective des différents peuples kymriques. ‑ Mommsen, \emph{Die nordetruskischen Alphabete}, dans les \emph{Mittheilungen der antiquarischen Geselischaft in Zurich}, VII B., 8 Heft, 1853, p. 265.}.\par
Qu’elles aient appartenu à la race blanche dans la partie originelle de leur essence, il n’y a pas à en douter. Chez elles, les guerriers avaient une carrure solide, des membres vigoureux et une taille gigantesque \footnote{Wachter, \emph{ouvr. cité}, p. 64.}, les yeux bleus ou gris, les cheveux blonds et rouges. C’étaient des hommes à passions turbulentes ; leur extrême avidité, leur amour du luxe, les faisaient volontiers recourir aux armes. Ils étaient doués d’une compréhension vive et facile, d’un esprit naturel très éveillé, d’une insatiable curiosité, très mous devant l’adversité, et, pour couronner le tout, d’une redoutable inconsistance d’humeur, résultat d’une inaptitude organique à rien respecter ni à rien aimer longtemps \footnote{César a ainsi dépeint les Gaulois en politique qui, prétendant se servir d’eux, voulait connaître et leur fort et leur faible. (Liv. II, 30 ; IV, 5, et VII, 20.) ‑ Strabon, les jugeant en littérateur désintéressé, est beaucoup plus indulgent. Il trouve les Gaulois bonnes gens et sans malice, ne se fâchant que quand ils sont les plus forts, et se laissant, du reste, persuader aisément. (Strab., IV, 4, 2.)}.\par
Ainsi faites, les nations galliques étaient parvenues de très bonne heure à un état social assez relevé, dont les mérites comme les défauts représentaient bien et la souche noble d’où ces nations tiraient leur origine, et l’alliage finnois qui avait modifié leur nature \footnote{Schaffarik, après avoir déclaré qu’il considère les Celtes comme le premier des peuples blancs établis en Europe, ajoute : « Déjà, dès les temps les plus anciens, ils étaient non « seulement riches et puissants à l’extrême, mais encore extraordinairement cultivés « (\emph{ungewœhnlich gebildet}). Ils occupaient un tiers de l’Europe, et, du III\textsuperscript{e} au II\textsuperscript{e} siècle avant « notre ère, ils s’étendaient d’un côté jusqu’à la Vistule, de l’autre, sur le bas Danube, « jusqu’au Dniester. » ‑ \emph{Slawische Alterthümer}, t. I, p. 89. ‑ Il montre, en plus d’un pays, les Slaves dominés par les Celtes, et vivant en sujets au milieu d’eux.}. Leur établissement politique présente le même spectacle que nous ont donné, à leurs origines, tous les peuples blancs.\par
Nous y retrouvons cette organisation sévèrement féodale et ce pouvoir incomplet d’un chef électif en usage chez les Hindous primitifs, chez les Iraniens, chez les Grecs homériques, chez les Chinois de la plus ancienne époque. L’inconsistance de l’autorité et la fierté ombrageuse du guerrier paralysent souvent l’action du mandataire de la loi. Dans le gouvernement des Galls, comme dans celui des autres peuples issus de la même souche, pas de vestiges de ce despotisme insensé d’une table d’airain ou de pierre, forte de l’abstraction qu’elle représente, aberration si familière aux républiques sémitiques, La loi était assez flottante, médiocrement respectée ; la prérogative des chefs incertaine. En un mot, le génie celtique maintenait ces droits hautains que l’élément noir détruit partout où il parvient à s’introduire.\par
Qu’on ne prenne pas ici le change en attribuant à un état de barbarie ces instincts peu disciplinables et cette organisation tourmentée. On n’a qu’à jeter les yeux sur la situation politique de l’Afrique actuelle pour se convaincre que la barbarie la plus radicale n’exclut pas, dans les sociétés, un développement monstrueux du despotisme. Être libre, être esclave, à un moment donné, ce sont là des faits qui dérivent souvent, pour un peuple, d’une série de combinaisons historiques fort longues ; mais, avoir une prédisposition naturelle à l’une ou à l’autre de ces situations, ce n’est jamais qu’un résultat ethnique. Le plus simple examen de la manière dont les idées sociales sont distribuées parmi les races ne permet pas de s’y tromper.\par
À côté du système politique se place naturellement le système militaire. Les Galls ne combattaient pas au hasard. Leurs armées, à l’image de celles des Arians Hindous, étaient composées de quatre éléments, l’infanterie \footnote{Ils avaient des archers excellents. (Cæsar\emph{, Comment. de Bello Gall}., VII, 31.)}, la cavalerie, les chariots de guerre \footnote{Le char de guerre, \emph{covinus}, était, comme celui des Assyriens, des Grecs homériques et des Hindous, monté par un guerrier et conduit par un écuyer. Fréquemment le guerrier, après avoir lancé ses javelots, mettait pied à terre pour combattre corps à corps. C’est absolument la même tactique que nous avons déjà observée en Asie. (César\emph{, ouvr. cité}, IV, 36.)} et les chiens de combat, qui tenaient la place des éléphants \footnote{Strabon, IV, 2.}. Ces troupes agissaient suivant les lois d’une stratégie sans doute médiocre, si l’on veut la considérer au point de vue perfectionné de la légion romaine, mais qui n’avait rien de commun avec l’élan grossier de la brute se précipitant sur sa proie. On en peut juger d’après la manière intelligente dont furent conduites les grandes invasions celtiques et le mode d’administration établi par les conquérants dans les pays occupés, régime original qui n’empruntait que des détails aux usages des vaincus. La Gallo-Grèce présente ce spectacle.\par
Les armes des Kymris étaient de métal \footnote{Keferstein, \emph{Ansichten über die keltischen Alterthümer}, t. I, p. 324 et passim. ‑ Wormsaae, \emph{Primeval antiquities of Denmark}, p. 23 et pass}, quelquefois de pierre, mais, en ce cas, très finement travaillées au moyen d’outils de bronze ou de fer. Il semblerait même que les épées et les haches de cette dernière espèce, qu’on a trouvées dans des tombes, étaient plutôt emblématiques ou vouées à des usages sacrés qu’à un emploi sérieux. À la même catégorie appartenaient, incontestablement, des glaives et des masses d’armes en argile cuite, richement dorées et peintes, qui ne peuvent avoir eu qu’une destination purement figurative \footnote{\emph{Ibidem}. ‑ Wormsaae donne la gravure d’une hache de cette espèce, qui est d’une grande élégance. \emph{(Ouvr. cité}, p. 39.)}. Du reste, il est bien probable aussi que les hommes de la plèbe la plus pauvre se faisaient arme de tout. Il leur était meilleur marché et plus facile d’emmancher un caillou percé dans un bâton que de se procurer une hache de bronze. Mais ce qui établit d’une manière irrécusable que cette circonstance n’implique nullement l’ignorance générale des métaux et l’inhabileté à les travailler, c’est que les langues galliques possèdent des mots propres pour dénommer ces produits, des mots dont on ne rencontre l’origine ni dans le latin, ni dans le grec, ni dans le phénicien. Si tels de ces vocables ont une affinité marquée avec leurs correspondants helléniques, ce n’est pas à dire qu’ils aient été fournis par les Massaliotes. Ces ressemblances prouvent seulement que les Arians Hellènes, pères des Phocéens et les aïeux des Celtes, étaient issus d’une race commune.\par
Le fer s’appelle \emph{ierne, irne, uirn, jarann} ; le cuivre \emph{copar}, et c’était le métal le plus en usage chez les Galls pour la fabrication des épées ; le plomb, \emph{luaid} ; le sel, \emph{hal, sal} \footnote{Keferstein, t. II, Erste Abtheilung, Verzeichniss. Les mots employés aujourd’hui dans l’art du mineur ont souvent l’avantage de fournir des notions fort anciennes. Keferstein fait cette réflexion pour l’Allemagne, et retrouve dans la langue actuelle des travailleurs souterrains du Harz des formes et des racines essentiellement celtiques, qui, en même temps que les procédés et les outils auxquels on les applique, ont passé des Galls aux métis germaniques. Quant à l’étymologie des noms de métaux, on peut remarquer que le mot celtique \emph{aes, ais}, qui devient dans le breton \emph{aren} et dans le latin \emph{aes}, avec la flexion \emph{aeris}, ne désigne pas proprement du bronze, mais bien, par excellence, \emph{le métal le plus dur}. C’est à ce titre seulement qu’on le trouve employé dans la plus haute antiquité pour désigner le bronze. Le sanscrit le possède sous la forme \emph{ayas} ou \emph{ayasa}, et lui donne le sens de \emph{fer}. L’allemand a de même \emph{Eisen}, dérivé du gothique \emph{eisarn}. L’anglo-saxon a \emph{iren}, l’anglais \emph{iron}, l’irlandais \emph{iarn}. Nous avons ici le celtique \emph{ierne}, et l’on peut voir que dans la forme \emph{jarann} il n’est pas trop loin d’\emph{aren}. ‑ Schlegel, \emph{Indische Bibliothek}, t. I, p. 243 et pass. ‑ Voir sur le sens de la racine primitive les recherches très curieuses de Dieffenbach, \emph{Vergleichendes Wœrterbuch der gothischen Sprache}, in-8°, Frankfurt a. M., t. I, p. 14, 15, n° 18. La signification de \emph{dur} parait être ici en corrélation avec l’idée de fondamental. ‑ Il résulte aussi de ce mot plusieurs applications plus ou moins directes, comme celles de \emph{métal} en génération, de \emph{richesses}, d’\emph{armes, harnais, harnisch}. On le découvre non seulement dans le sanscrit, les langues celtiques et gothiques, mais aussi dans le pouschtou ou afghan, le grec, le balouki, l’ossète, et on l’aperçoit jusque dans le chaldéen (mot chaldéen), \emph{asina hache}. On le remarque dans les langues slaves, avec une forme qui le rapproche de certains dialectes galliques.}.\par
Toutes ces expressions sont entièrement galliques, et c’est un témoignage qu’on ne peut récuser de l’antiquité du travail des métaux chez les Kymris. Il serait d’ailleurs bien étrange, on en conviendra, que dans cet Occident où les Ibères étaient en possession de l’art du mineur, où les Étrusques indigènes avaient le même avantage, les Galls en eussent été privés, eux venus les derniers du pays du nord-est, terre classique, terre natale des forgerons.\par
Les monuments des deux âges de bronze et de fer ont fourni une énorme quantité d’outils divers, qui donnent encore une haute idée de l’aptitude des nations celtiques au travail du minerai. Ce sont des épées, des haches, des fers de lance, des hallebardes, des jambards, des casques, le tout d’or ou doré, de bronze ou d’argent, ou de fer, ou de plomb, ou de zinc ; des baudriers, des chaînes précieuses, destinées aux hommes pour suspendre leurs glaives, et aux femmes pour attacher les clefs de la ménagère ; des bracelets de fil de métal tourné en spirales, des broderies appliquées sur des étoffes, des sceptres, des couronnes pour les chefs, etc. \footnote{Keferstein\emph{, ouvr. cité}, t. I, p. 330 et pass.}.\par
Les Galls pratiquaient la vie sédentaire. Ils vivaient dans de grands villages qui devenaient souvent des villes considérables. Avant l’époque romaine, plusieurs des capitales de leurs nations les plus opulentes avaient acquis un degré notable de puissance. Bourges comptait alors quarante mille habitants \footnote{Cæar, \emph{de Bello Gallico}, VII, 28.}. On peut juger, d’après ce seul fait, si ces cités étaient à dédaigner quant à leur étendue et à leur population \footnote{Les Celtes de Bourges, avant de s’insurger brûlèrent, en un seul jour, vingt de leurs villes qu’ils ne se jugeaient pas en état de défendre. Il s’en faut qu’aujourd’hui le Berry soit aussi peuplé.}, Autun, Reims, Besançon, dans les Gaules, Carrhodunum, en Pologne, bien d’autres bourgades, n’étaient certainement pas sans importance et sans éclat \footnote{Carrhodunum était dans le voisinage de Cracovie. Une autre ville celtique de Pannonie rappelle le nom des Carnutes du pays chartrain, c’est \emph{Carnuntum.} (Schaffarik, t. I, p. 104.)}.\par
L’antiquité latine nous a parlé de la forme des maisons. On en possède en France et dans l’Allemagne méridionale \footnote{On en a trouvé également dans le Brunswick et en Suisse, une première fois près de Bâle, plus tard dans les Grisons. (Keferstein, t. I, p. 292.)} de nombreux restes. Ce sont ces sortes d’excavations connues des antiquaires sous le nom de \emph{margelles}. Plusieurs mesurent cent pas de tour. Elles sont rondes et toujours réunies deux par deux. L’une servait d’habitation, l’autre de grange. Quelques-uns de ces emplacements semblent avoir porté un mur de soutènement en pierres, sur lequel s’élevait la bâtisse faite de planches et de torchis, souvent recouverte de plâtre. Les Galls usaient volontiers, dans leurs constructions, de la combinaison de la pierre ou du mortier avec le bois \footnote{Ils appliquaient même fort habilement ce système à l’architecture militaire. César loue beaucoup leur façon de construire certains remparts. (\emph{Comm. de Bello Gall.}, VII, 23.) En général, les traducteurs rendent mal ce passage. Un historien de la ville d’Orléans me paraît l’entendre mieux. Voici sa version : « Ces poutres sont placées à deux pieds l’une de l’autre « à angle droit avec le parement du rempart. Du côté de la ville, elles sont liées à l’aide de « terres extraites du fossé ; à l’extérieur, de grandes pierres remplissent l’intervalle qui les « sépare. Sur cette première assise on en établit une seconde, alternant en échiquier avec les « pierres, et ainsi de suite. » (L. de Buzonnière, \emph{Histoire architecturale de la ville d’Orléans}, 1849, In-8°, t. I, p. 22.}. Ces vieilles maisons, si communes encore dans presque toutes nos villes de province, comme en Allemagne, et formées de charpentes apparentes, dont les intervalles sont remplis de pierres ou de terre, sont des produits du système celtique.\par
Rien n’indique que les habitations aient comporté plusieurs étages. Elles ne sem­blent pas avoir eu beaucoup de luxe à l’intérieur. Les Celtes recherchaient plus que le beau, le bien-être.\par
Ils avaient des meubles travaillés en bois avec assez de soin, des ouvrages d’os et d’ivoire, tels que peignes, aiguilles de tête, cuillers, dés à jouer, cornes servant de vases à boire ; puis des harnais de chevaux garnis et ornés de plaques de cuivre ou de bronze doré, et surtout un grand nombre, de vases de toutes formes, tasses, amphores, coupes, etc. Les objets en verre n’étaient pas moins communs chez eux. On en trouve de blancs et de coloriés en bleu, en jaune, en orange. On a aussi des colliers de cette matière. On veut que ces ornements aient servi d’insignes au sacerdoce druidique pour distinguer les degrés de la hiérarchie \footnote{Keferstein, \emph{ouvr. cité}, t. I, p. 321 et pass.}.\par
La fabrication des étoffes avait lieu sur une grande échelle. On a découvert souvent, dans les tombeaux, des restes de drap de laine de différents degrés de finesse, et on sait, par les témoignages historiques, que les Celtes, s’ils étaient fort empressés à se chamarrer de chaînes et de bracelets de métal, ne l’étaient pas moins à se vêtir de ces étoffes bariolées dont les tartans écossais sont un souvenir direct \footnote{Tacite les décrit très bien, d’un seul mot : il nomme le \emph{sagum} celtique, \emph{versicolor}. (\emph{Histor.}, II, 20.)}.\par
De très bonne heure, cet amour des jouissances matérielles avait porté les Celtes au travail, et du travail productif naquit le goût du commerce. Si les Massaliotes prospé­rèrent, c’est qu’il trouvèrent dans les populations qui les entouraient, et dans celles qui couvraient derrière eux les pays du nord, un instinct mercantile qui, à sa façon, répondait au leur, et que cet instinct avait créé de nombreux éléments d’échange. Il avait aussi à sa disposition des moyens de transport abondants et faciles. Les Celtes possédaient une marine. Ce n’étaient pas les pirogues misérables des Finnois, mais de bons vaisseaux de haut bord, bien construits et solidement membrés, armés d’une forte mâture et de voiles de peaux, souples et bien cousues. Ces navires, dans l’opinion de César, étaient mieux entendus pour la navigation de l’Océan que les galères romaines. Le dictateur s’en servit pour la conquête de l’île de Bretagne, et put les apprécier d’autant mieux que, dans la guerre contre les Vénètes, il s’en fallut de peu que sa flotte ne succombât à la supériorité de celle de ce peuple. Il parle aussi avec admiration de la quantité de bâtiments dont disposaient les nations de la Saintonge et du Poitou \footnote{\emph{De Bello Gall}, III, 8, 9, 11.}.\par
De sorte que les Celtes avaient sur mer un puissant instrument d’activité et de fortune. Pour tant de raisons, leurs villes peu brillantes, étant d’ailleurs grandes, populeuses et bien pourvues de richesses de tout genre, le caractère belliqueux de la race leur faisait courir de fréquents dangers. La plupart étaient fortifiées, et non pas sommairement d’une palissade et d’un fossé, mais avec toutes les ressources d’un art d’ingénieur qui n’était pas méprisable. César rend justice au talent des Aquitains gaulois dans l’attaque des places au moyen de la mine. Il n’est pas à croire que les Celtes, habiles aux travaux souterrains, comme les Ibères, fussent plus maladroits que ces derniers dans l’application militaire de leurs connaissances \footnote{César dut renoncer à prendre Soissons, à cause de la largeur de ses fossés et de l’élévation de ses murailles. (\emph{De Bello Gall.}, II, 12.)}.\par
Les défenses des villes étaient donc très fortes. Elles consistaient en murs de bois et de pierres ainsi disposés, que, tandis que les poutres paralysaient l’emploi du bélier par leur élasticité, les moellons mettaient obstacle à l’action du feu \footnote{Bourges avait aussi des tours revêtues de cuir. (Cæsar, VII, 22.)}. Outre ce système, il y en avait un autre, probablement beaucoup plus ancien encore et dont on a trouvé de bien curieux vestiges en plusieurs endroits du nord de l’Écosse ; à Sainte-Suzanne, à Péran, en France ; à Görlitz, dans la Lusace. Ce sont de gros murs dont la surface, mise en fusion par l’action du feu, s’est recouverte d’une croûte vitrifiée qui fait du travail entier un seul bloc d’une dureté incomparable \footnote{Keferstein, t. I, p. 286. ‑ Geslin de Bourgogne, \emph{Notice sur l’enceinte de Péran, extrait du XVIII\textsuperscript{e} volume des Mémoires de la Société des Antiquaires de France}, p. 6 et sqq., et 39.}. Ce mode de construction est si étrange que longtemps on a douté qu’il fût dû à l’action de l’homme, et on l’a pris pour un produit volcanique, dans des contrées qui d’ailleurs ne révèlent pas une seule trace de l’existence de feux naturels. Mais on ne peut nier l’évidence. Le camp de Péran montre ses substructions vitrifiées sous une maçonnerie romaine, et il n’est pas douteux que ce genre impérissable de travail ne soit l’ouvrage des Celtes. L’antiquité en est certainement des plus reculées. J’en vois la preuve dans ce fait, qu’au temps des Romains l’Ecosse était tombée en décadence, et que de tels monuments dépassaient, de toute façon, ses besoins et les ressources dont elle disposait. On doit donc les attribuer à une époque où la population calédonienne n’avait pas encore subi, à un point dégradant, le mélange avec les hordes finniques qui l’entouraient \footnote{Au premier siècle avant notre ère, l’Angleterre proprement dite comptait deux espèces de popu­lations celtiques : l’une qui se disait autochtone, et qui habitait l’intérieur des terres ; l’autre était due à une immigration successive de Belges ou Galls germanisés, qui eut lieu vers le VII\textsuperscript{e} siècle à Rome. (Cæsar, \emph{de Bello Gall}, V, 12.) ‑ C’est à ces conquérants qu’appartiennent les monnaies celtiques de l’Angleterre. Ces restes numismatiques sont imités de ceux que l’on trouve depuis la Schelde jusqu’à Reims et à Soissons. Le type primitif en est le statère macédonien. On possède dans ce genre des exemplaires fort grossiers d’une monnaie d’or, marqués du \emph{cheval à gorge fourchue}, pesant de 6,1 gr. à 5,4 gr. ‑ Mommsen, \emph{Die nord-etruskischen Alphabete}, dans les \emph{Mittheilungen der antiquarischen Gesellschaft in Zürich}, VII B., 8 Heft, 1813, p. 245. ‑ Les Celtes de l’intérieur de l’Angleterre étaient devenus fort barbares. Ils allaient vêtus de peaux de bêtes. La polyandrie était presque générale parmi eux. Ils avaient déjà, en se mêlant aux Belges immigrés, communiqué à ceux-ci l’usage de se peindre le corps. Ces derniers les surpassaient de beaucoup par le raffinement des habitudes et par les richesses. Une population semblable à celle des Bretons de l’intérieur de l’île, et peut-être plus avilie encore, c’étaient les Irlandais. On peut admettre comme vraisemblable qu’à une époque fort ancienne leur île avait reçu quelques colonisations phéniciennes et carthaginoises ; mais, d’après ce qu’on a vu en Espagne d’établissements semblables, il est douteux que l’influence en ait dépassé les limites du comptoir. Toutefois M. Pictet pense avoir découvert dans l’erse des traces sémitiques. Peut-être encore y a-t-il eu des immigrations ibériques ou plutôt celtibériennes. Quoi qu’il en soit, Strabon dépeint les Irlandais comme des cannibales, mangeant leurs parents aînés. Diodore de Sicile et saint Jérôme racontent d’eux les mêmes choses. Les traditions locales avec leurs colonies antédiluviennes, commandées par César, leur Partholan, cinquième descendant de Magog, fils de Japhet, leur Clanna, leur Nemihidh, parents de ce héros, leurs Fir-Bolgs, tous originaires de Thrace, enfin leurs Milésiens, fils de Mileadh, venus d’Égypte en Espagne, et d’Espagne en Irlande, sont trop évidemment influencées par des romanciers bibliques et classiques pour qu’on puisse leur accorder beaucoup d’antiquité et, par suite, de confiance. C’est le pendant des histoires de France commençant à Francus, fils d’Hector. Il paraît certain que l’île n’a commencé à se relever que vers le IV\textsuperscript{e} siècle de l’ère chrétienne. Elle avait alors une marine. ‑ Dieffenbach, \emph{Celtica II}, Abth. 2, 371 et sqq., est peut-être l’écrivain le plus complet sur cette matière ardue, qui constitue un des chapitres des chroniques celtiques sur lesquels il a été débité le plus de folies et les extravagances les plus monstrueuses. Pour faire juger de l’esprit de ceux qui les ont mises en œuvre, je ne citerai qu’un trait : partant de ce point, que l’Irlande est une terre sacrée, qualité qu’en effet lui reconnaissaient les Druides, et qu’ont ensuite maintenue pour elle les Sculdées chrétiens, O’Connor raconte, dans ses \emph{Proleg}., II, 75, que de l’avis d’un savant allemand, l’erse était la seule langue inaccessible au diable, comme trop saint pour qu’il pût jamais l’apprendre et qu’à Rome un possédé, « aliis linguis locutum, at hibernice loqui, vel noluisse vel non potuisse. » Tout bien pesé cependant, il serait imprudent de rejeter absolument les traditions irlandaises ; elles contiennent çà et là des faits dignes d’être observés.}.\par
Des murs vitrifiés, construits en grosses pierres, supposent l’existence de l’archi­tecture fragmentaire. En effet, les Celtes, fort différents des peuplades jaunes, ne se bornaient pas à juxtaposer des quartiers de roches énormes ; ils élevaient, l’un sur l’autre, des blocs polygones qu’ils conservaient bruts, afin, a-t-on dit, de n’en pas diminuer la force \footnote{Keferstein, t. I. ‑ Suivant Abeken, les murs les plus rudement façonnés de l’Italie se trouvent dans l’Apennin. (\emph{Ouvr. cité}, p. 139.) Les constructions des Aborigènes, dans le Latium et l’Italie centrale, étant faites de tuf très tendre, présentèrent promptement des traces de taille. ‑ \emph{Ibid}. Dennis, \emph{ouvr. cité}, t. II, p. 571 et pass. ‑ Les ruines de Saturnia, une des plus anciennes villes de l’Étrurie, près d’Orbitello, renferment un tumulus bien évidemment celtique. Or, Saturnia, avant d’être aux Étrusques, appartenait aux aborigènes qui l’avaient fondée ; c’était une ville umbrique.}. C’est là l’origine du système connu sous les noms de pélasgique et de cyclopéen \footnote{Abeken, \emph{ouvr. cité}, p. 139. Cet auteur nomme \emph{pélasgiques} les maçonneries non taillées, celles où l’emploi de petites pierres pour boucher les interstices est le plus indispensable. Il rappelle que Pausanias se sert de cette expression en décrivant les murs de Tyrinthe et de Mycènes. Les murs cyclopéens marqueraient ainsi un perfectionnement dans le genre des constructions à blocs polygones.}. On en trouve en France, comme en Grèce, comme en Italie. À ordre de constructions appartiennent des enceintes découvertes dans nos provinces, et les chambres sépulcrales d’un grand nombre de tumulus, qui se distinguent ainsi nette­ment des ouvrages finniques, dans lesquels les blocs ne sont jamais superposés de manière à former muraille \footnote{Keferstein, \emph{Ansichten}, etc., t. IV, p. 287 Cet écrivain remarque qu’il y a fort peu de constructions celtiques maçonnées en Angleterre et en Scandinavie. Son observation s’accorde pleinement avec ce que dit César, que les Bretons de l’intérieur de l’île (non pas les Belges immigrés) appelaient \emph{ville} une de sorte de camp retranché formé de pieux et de branchages, au milieu des bois. (\emph{De Bello Gall.}, V, 21.) ‑ Les contrées où l’on en trouve le plus, soit à l’état de murailles, soit comme tombeaux recouverts ou ayant été recouverts d’un tumulus de terre, sont les pays que j’ai nommés déjà, la Bohême, la Wetteravie, la Franconie, la Thuringe, le Jura, l’Asie Mineure. Voir aussi, quant à l’existence des tumulus celtiques, Boettiger, Ideen zur Kunstmythologie, c. II, p. 294.}.\par
La puissance extraordinaire de ces débris massifs a résisté, en plus d’un lieu, à l’outrage des siècles. Les Romains s’en sont servis, comme des remparts de Sainte-Suzanne, et en ont fait la base de leurs propres travaux. Puis, les chevaliers du moyen âge, à leur tour, élevant leurs donjons sur cette double antiquité, sont venus compléter les archives matérielles de l’architecture militaire en Europe.\par
Outre la pierre et le bois, les Galls usaient aussi de la brique. Ils ont bâti des tours très remarquables, dont quelques-unes subsistent encore, une, entre autres, sur la Loire, et d’usage inconnu, mais probablement religieux \footnote{« Coram adire alloquique Velledam negatum. Arcebantur adspectu quo venerationis plus « inesset. Ipsa edita in turre ; delectus e propinquis consulta responsaque, ut internuncius « numinis, portabat. » Tacite, \emph{Hist.}, IV, 65.}.\par
Les cités, ainsi bien peuplées, bien bâties, bien défendues, bien fournies de meubles, d’ustensiles et de bijoux, communiquaient entre elles à travers le pays, non par des sentiers et des gués difficiles, mais par des routes régulières et des ponts. Les Romains n’ont pas été les premiers à établir des voies de communication dans les pays kymriques : ils en ont trouvé qui existaient avant eux, et plusieurs de leurs chemins les plus célèbres, parce qu’ils étaient les plus fréquentés, n’ont été que d’anciens ouvrages nationaux entretenus et réparés par leurs soins. Quant aux ponts, César en nomme que certes il n’avait pas bâtis \footnote{Keferstein, \emph{ouvr. cité}, t. I, p. 192. Sur plusieurs bornes milliaires antiques, on trouve, en France, l’indication de la lieue celtique au lieu du mille romain. Quant aux ponts, Orléans et Paris en avaient. Cæs., \emph{de Bello Gall.}, VII, 11.}.\par
Outre ces communications, les Celtes en avaient organisé de plus rapides encore pour les circonstances extraordinaires. Ils possédaient une télégraphie véritable. Des agents désignés se criaient de l’un à l’autre la nouvelle qu’il fallait transmettre : de cette façon, un ordre ou un avis parti d’Orléans, au lever du soleil, arrivait en Auvergne avant neuf heures du soir, ayant parcouru de la sorte quatre-vingts lieues de pays \footnote{Cæs., \emph{de Bello Gall.}, VII, 3.}.\par
Si les villes étaient nombreuses et rassemblaient beaucoup d’habitants, les campagnes paraissent n’avoir pas été moins peuplées. On le peut induire du nombre considérable de cimetières découverts dans les différentes contrées de l’Europe celtique. L’étendue de ces champs mortuaires est généralement remarquable. On n’y voit pas de tumulus. Cette construction, lorsqu’elle contient un dolmen, appartient aux premiers habitants finnois : il n’est pas question ici de cette variété. Lorsqu’elle renferme une chambre sépulcrale en maçonnerie, elle appartient aux princes, aux nobles, aux riches des nations. Les cimetières sont plus modestement le dernier asile des classes moyennes ou populaires. Ils ne fournissent à l’observateur que des tombeaux plats, la plupart construits avec soin, taillés souvent dans le roc ou établis dans la terre battue. Les tombes y sont couvertes de dalles. Les corps ont presque toujours été brûlés. Bien que ce fait ne soit pas absolument sans exception, sa fréquence établit une sorte de distinction supplémentaire entre les cadavres des plus anciens indigènes, toujours entiers, et ceux des Celtes. En tout cas, les tumulus à chambres funéraires, pélasgiques et cyclopéennes, monuments probablement contem­porains des cimetières, ne renferment jamais de squelettes intacts, mais toujours des ossements incinérés contenus dans des urnes.\par
Une autre différence existe encore entre celles de ces sépultures qui appartiennent à l’époque nationale, et celles qui ne remontent qu’à la période romaine : c’est que les objets trouvés dans ces dernières ont un caractère mixte où l’élément latin hellénisé se fait aisément apercevoir. Non loin de Genève, on voit un cimetière de cette espèce \footnote{Keferstein, \emph{ouvr. cité}, t. I.}.\par
Outre que l’abondance des cimetières purement celtiques donne une haute idée de l’ampleur des populations qui les ont fondés, elle inspire encore des réflexions d’un autre ordre. Le soin et, par suite, les frais qu’on y a employés, le nombre, la nature et la richesse des objets divers que renferment les tombes, tout cela, rapproché de l’observation qu’en les contemplant on n’a pas sous les yeux le lieu de repos des grands et des chefs, mais seulement des classes moyennes et inférieures, fait naître une très haute idée du bien-être de ces classes, et conséquemment de l’opulence générale des nations dont elles formaient la base \footnote{Keferstein, t. I, p. 304.}. Nous voilà bien loin de l’opinion si longtemps répandue, et si légèrement adoptée, sur la barbarie complète des tribus galliques, opinion qui prenait surtout son point d’appui dans la fausse allégation que les monuments finniques étaient leur œuvre.\par
Ce n’est pas encore fuir assez de si lourdes erreurs : plusieurs détails importants qui restent à dire vont allonger la distance. Les Celtes, habiles à tant de travaux divers, ne pouvaient pas être étrangers au besoin de les rémunérer et de leur reconnaître un prix. Ils connaissaient l’usage du numéraire, et, trois cents ans avant la venue de César, battaient monnaie pour les besoins du commerce extérieur. Ils avaient des pièces d’or, d’argent, d’or-argent et cuivre, de cuivre et plomb, de fer, de cuivre seul, rondes, carrées, radiées, concaves, sphériques, plates, épaisses, minces, frappées en creux ou en relief \footnote{Id., \emph{ouvr. cité}, t. I, p. 341.}. Un très grand nombre de ces monnaies ont été visiblement produites sous l’influence massaliote, macédonienne ou romaine \footnote{Les différentes catégories d’imitations paraissent se limiter à des territoires déterminés. Celles qui ont pour objet les monnaies massaliotes se trouvent dans la Narbonnaise, sur le cours supérieur du Rhône, dans la Lombardie entière, à Berne, à Genève, dans le Valais, le Tessin, les Grisons et le Tyrol italien ; mais, en France, on n’en a pas rencontré jusqu’ici au-dessus de Lyon. ‑Sur le penchant septentrional des Pyrénées et les côtes de l’Océan, ce sont les colonies grecques de Rhodæ et d’Emporiac qui ont fourni les types ; il s’en rencontre dans les pays de la Garonne, à Toulouse, dans le Poitou ; on en cite un exemplaire découvert en Sologne. Sur la Loire supérieure, sur le Rhin, sur la Schelde, se voient les contrefaçons grossières des statères macédoniens de Philippe II. Mommsen pense que cette habitude de copier, du moins mal possible, les types grecs pour la monnaie, a commencé au IV\textsuperscript{e} siècle avant J.-C., c’est-à-dire environ trois cents ans avant la conquête de César. C’est, à coup sûr, l’indice de relations commerciales fort étendues, fort suivies et telles qu’on les pourrait à peine dire supérieures aujourd’hui. ‑ Mommsen, \emph{Die nordetruskischen Alphabete}, dans les \emph{Mittheilungen der antiquarischen Gesellschaft in Zurich}, VII B. 8\textsuperscript{e} Heft., in-4° 1853, p. 204, 233, 236, 256.}. Mais d’autres échappent complète­ment au soupçon de cette parenté. Ce sont certainement les plus anciennes : elles remontent bien au delà de la date que je viens d’indiquer. Il en est, les radiées, qui ont leurs analogues en Étrurie, soit que les hommes de ce pays les aient empruntées aux peuples umbriques de leur voisinage, soit qu’un grand commerce entre les deux nations, commerce qui n’est pas à révoquer en doute, et que la présence fréquente du succin dans les tombeaux toscans les plus anciens suffirait à démontrer, ait de bonne heure engagé les deux groupes contractants à user de moyens d’échange parfaitement semblables \footnote{Abeken, \emph{ouvr. cité}, p. 284. ‑ On a découvert de ces monnaies radiées, d’origine étrusque, marquées de l’image d’une roue, à Posen et en Saxe. Elles se trouvaient mêlées à des médailles d’Égine et d’Athènes du VIII\textsuperscript{e} siècle avant notre ère.}.\par
Avec la monnaie, les Celtes possédaient encore l’art de l’écriture. Plusieurs inscrip­tions copiées sur des médailles celtibériennes, mais jusqu’à présent non déchiffrées, en font foi pour une époque lointaine.\par
Tacite signale, de son côté, un fait qui semble remonter à un âge au moins aussi éloigné. On disait de son temps qu’il existait, dans la Germanie et dans les Alpes Rhétiennes, des monuments antiques couverts d’inscriptions grecques. On ajoutait que ces monuments avaient été élevés par Ulysse, lors de ses grandes pérégrinations septentrionales, aventures dont nous n’avons pas le récit \footnote{\emph{Odyssée}, XXIII, 267 et pass.}. En rapportant cette tradition, Tacite, fort judicieusement, exprime le doute que le fils de Laërte ait jamais voyagé dans les Alpes et du côté du Rhin ; mais sa réserve devient excessive lorsqu’elle s’étend de la personne du voyageur à l’existence des inscriptions elles-mêmes \footnote{Tacite, \emph{de Moribus Germ}., 3. ‑ Mommsen considère comme démontré qu’avant l’époque romaine l’usage de l’écriture s’étendait, par delà les Alpes et le cours du Rhône, jusqu’au Danube. (\emph{Die nordetruskischen Alphabete}, p. 221.)}.\par
 Avec le témoignage de Tacite vient celui de César, qui, lorsqu’il eut défait les Helvétiens, trouva dans leur camp un état détaillé de la population émigrante, guerriers, femmes, enfants et vieillards. Ce registre était, à son dire, écrit en lettres grecques \footnote{Cæsar, \emph{de Bello Gall.}, I, 29.}.\par
Dans un autre passage des \emph{Commentaires}, le dictateur raconte que, pour toutes les affaires \emph{publiques} \footnote{Cæsar, \emph{de Bello Gall}., VI, 14 : « In reliquis fere rebus (\emph{publicis}) privatisque rationibus. » \emph{Publicis} n’est pas certain. Le mot semble interpolé, quoique la plupart des éditions le donnent.} et privées, les Celtes faisaient usage des lettres grecques. Par une singulière anomalie, les druides ne voulaient rien écrire de leurs doctrines ni de leurs rites, et forçaient leurs élèves à tout apprendre par cœur \footnote{Cæsar, \emph{de Bello Gall.}, VI, 14.}. C’était une règle stricte. D’après ces renseignements, il est hors de discussion qu’avant d’avoir passé par l’éducation romaine, les nations celtiques étaient accoutumées à la représentation graphique de leurs idées, et, ce qui est ici particulièrement intéressant, l’emploi qu’elles faisaient de cette science était tout autre que celui dont les grands peuples asiatiques de l’antiquité nous ont donné le spectacle. Chez ces derniers, l’écriture servait principale­ment aux prêtres, était révérée à l’égal d’un mystère religieux, et passait si difficilement dans l’usage familier que jusqu’à l’époque de Pisistrate, on n’écrivit pas même les poèmes d’Homère, objets, cependant, de l’admiration générale. Chez les Celtes, tout au rebours, ce sont les sanctuaires qui ne veulent pas de l’alphabet. La vie privée et l’administration profane s’en emparent : on s’en sert pour indiquer la valeur des monnaies et pour ce qui est d’intérêt personnel ou public. En un mot, chez les Celtes, l’écriture, dépouillée de tout prestige religieux, est une science essentiellement vulgarisée.\par
Mais Tacite et César ajoutent que ces lettres, que cet alphabet si usité, dont la présence n’est désormais pas douteuse en Allemagne \footnote{Mommsen (\emph{Die nordetruskischen Alphabete}) regarde le fait comme indubitable pour les contrées en deçà du Danube.}, est certaine dans la péninsule hispanique, les Gaules et l’Helvétie, que cet alphabet, dis-je, est hellénique, n’a rien de national, et provient d’une importation grecque. Aussitôt, pour expliquer cette asser­tion, les gens qui ne veulent voir partout que des civilisations importées, se tournent vers les Massaliotes. C’est leur grande ressource quand ils ne peuvent fermer les yeux sur la réalité d’un état de choses étranger à la barbarie dans les pays celtiques. Mais leur hypothèse n’est pas plus admissible cette fois que dans tant d’autres occasions où la saine critique en a fait justice.\par
Si les Massaliotes avaient eu le pouvoir d’agir sur les idées des nations galliques d’une manière assez constante, assez puissante, assez générale pour répandre partout l’usage de leur alphabet, à plus forte raison auraient-ils fait accepter les formes séduisantes de leurs armes et de leurs ornements. Cette victoire eût été certainement la plus facile de toutes. Cependant ils n’y réussirent pas. Lorsque les nations de la Gaule imaginèrent de copier les monnaies grecques, elles cédèrent à un sentiment d’utilité positif qui leur révélait tous les avantages attachés à l’unité du système monétaire ; mais, au point de vue artistique, elles s’y prirent avec une maladresse et une grossièreté qui montrent de la manière la plus évidente combien elles connaissaient peu les intentions du peuple dont elles cherchaient à contrefaire les œuvres, et le peu de fréquentation intellectuelle qu’elles avaient avec lui. Une race n’emprunte pas à une autre son alphabet sans lui prendre quelque chose de plus, des croyances religieuses, par exemple, et précisément les druides ne voulaient pas entendre parler de l’écriture. Donc l’écriture, chez les Celtes, n’était dépositaire d’aucun dogme. Ou bien, quelquefois, à défaut de doctrines théologiques, il pourrait être question d’importations littéraires. Nul écrivain de l’antiquité n’en a jamais remarqué la moindre trace \footnote{Je dois dire que Strabon, venant au-devant de cette objection, affirme que les Gaulois écrivaient leurs contrats en grec, non seulement avec les caractères, mais même dans la langue de l’Hellade : (mots grecs) (Strab., IV.) ‑ Mais, soit dit avec tout le respect possible pour l’autorité de Strabon, cette assertion n’est guère recevable. Si les Celtes avaient à tel point sympathisé avec les Grecs, qu’ils eussent fait de l’idiome de ces derniers l’instrument ordinaire de leurs transactions de toute nature, ils eussent mérité, non pas le nom de barbares, que les écrivains classiques ne leur ménageaient pas, mais celui de philologues, d’érudits consommés ; encore n’ai-je connaissance d’aucun docte personnage, soit ancien, soit moderne, pas même Scaliger, qui se soit amusé à passer des actes civils, par-devant notaire, dans une langue savante. Tout ce qu’il est possible d’accorder, c’est que Strabon, ou plutôt Posidonius, aura vu entre les mains de quelques négociants massaliotes des cédules grecques tracées par ces derniers, et souscrites par des commerçants gaulois.}. Enfin, cet usage de l’alphabet si répandu, si fort entré dans les mœurs des nations galliques qui avaient entre elles le moins de contact, par quelle voie aurait-il passé des Helvétiens aux gens de la Celtibérie ? Si ces derniers avaient été tentés de demander à des étrangers un moyen graphique de conserver le souvenir des faits, ils se fussent tournés certainement du côté des Phéniciens. Or, les \emph{letteras desconocidas} gravées sur les médailles indigènes de la Péninsule n’ont pas le moindre rapport avec l’alphabet chananéen ; elles n’en ont pas non plus avec celui de la Grèce.\par
Ce mot terminera la discussion quant à l’identité matérielle des deux familles de lettres, ce qui n’est pas vrai pour les Celtibériens ne l’est pas non plus pour la plupart des autres nations kymriques. je ne prétends pas néanmoins qu’il n’y eut qu’un seul alphabet pour elles toutes \footnote{Mommsen compte jusqu’à neuf alphabets différents, recueillis par lui au nord de l’Italie et dans les Alpes. Voici la liste topographique qu’il en donne : Todi, Provence, Étrurie, Valais, Tyrol, Styrie, Conegliano, Vérone, Padoue. ‑ Les déviations qui peuvent créer l’originalité de chacun de ces alphabets sont considérables, comme le déclare lui-même cet éminent et judicieux archéologue. (\emph{Die nordetruskischen Alphabete}, p. 221, taf. III.)}. Je m’arrête à cette limite que le système de l’agencement et des formes était identique en principe, bien que pouvant offrir des nuances et des variations locales fort tranchées.\par
On demandera comment il s’est pu faire que César, si accoutumé à la lecture des ouvrages grecs, se soit trompé sur l’apparence des registres helvétiens, et ait vu des lettres helléniques là où il n’y en avait pas ? Voici la réponse : César a tenu dans ses mains, probablement, ces manuscrits, mais c’est un interprète qui lui en a donné le sens. Ils étaient tracés, suivant ce secrétaire, en caractères grecs, c’est-à-dire en caractères qui ressemblaient fort aux grecs, mais la langue était gallique. L’apparence a suffi au dictateur, et, comme il regardait comme indubitable que les alphabets italiotes et étrusques étaient d’origine grecque, malgré leurs déviations de ce type, quand il a vu un ensemble qu’il ne comprenait pas, mais où son œil démêlait les mêmes analogies, il a conclu et dit ce qu’il a dit \footnote{Denys d’Halicarnasse raconte comme un fait admis que l’alphabet avait été apporté chez les Italiotes par les Pélasges arcadiens. Il ne tient nul compte des différences extrêmes que chacun peut remarquer entre les lettres grecques et celles de la Péninsule. (Dionys. Halic., \emph{Antiq. rom.}, 1, XXXIII.) ‑ C’était un axiome scientifique, indiscutable pour les lettrés grecs et romains, que tout, le bien, le mal, les vertus et les vices, l’ennui et le plaisir, l’art de marcher, de manger et de boire, avait été inventé dans l’Hellade et s’était de là répandu sur le reste du monde. Homère et Hérodote, comme Hésiode, sont complètement étrangers à cette puérile doctrine.}. Du reste, cette explication n’est pas facultative : il n’y a pas à hésiter : les monuments récemment découverts ont fait connaître les alphabets en usage, antérieurement aux Romains, chez les Salasses de la Provence, chez les Celtes du Saint-Bernard, chez les montagnards du Tessin : tous ces modes d’écriture sont originaux, ils n’ont que des affinités lointaines avec le grec \footnote{Mommsen, \emph{Die nordetruskischen Alphabete.}}.\par
Je ne nie pas en effet que, si l’alphabet ou les alphabets celtiques ne sont pas grecs, ils ne soient placés, à l’égard de l’alphabet hellénique, dans des rapports très intimes, en un mot, qu’ils ne puissent se reporter tous, eux et lui, à une même source. Ce ne sont pas des copies, mais ils se forment sur un même système, sur un mode primordial, antérieur à eux-mêmes comme au type hellénique, et qui leur a fourni leurs apparences communes, en même temps qu’un mécanisme identique.\par
L’ancien alphabet grec, celui qui, au dire des experts, fut employé le premier par les nations arianes helléniques, était composé de seize lettres. Ces lettres ont, il est vrai, des noms sémitiques, ont même plusieurs points de ressemblance avec les caractères chananéens et hébreux, mais rien ne prouve que l’origine des uns et des autres soit locale et n’ait pas été apportée du nord-est par les premiers émigrants de race blanche \footnote{ \noindent Je ne saurais me rendre à l’observation qui a été faite, que les alphabets sémitiques ne peuvent convenir qu’aux langues auxquelles ils sont adaptés, parce qu’ils ne comptent pas de voyelles proprement dites. Ces langues ont toutes : (alphabet sémitique) comme les Grecs ont (\emph{alphabet grec}). Les runes, destinées incontestablement à des dialectes qui traitent les voyelles tout autrement que les idiomes sémitiques, n’ont pas même tous ces caractères : il leur manque l’\emph{e}. Le rôle de consonnes attribué, dans les temps historiques, aux lettres chananéennes que je viens de citer, ne s’oppose nullement à ce qu’on admette que, primitivement, elles ont été considérées sous un autre point de vue. ‑ Consulter le travail de Gesenius, dans l’\emph{Encycl. Ersch un Gruber, Palæograpbie}, 3\textsuperscript{e} section, IX Theil, p. 287. et pass. ‑ Le problème de l’origine des alphabets est encore loin d’être éclairci comme il est désirable qu’il le devienne. Il tient d’aussi près que possible aux questions ethniques, et est destiné à prêter de grands secours à bien des solutions de détail. Il est, du reste, compliqué par une conception \emph{a priori}, inventée au XVIII\textsuperscript{e} siècle et sur laquelle on se heurte, à chaque instant, quand il s’agit des grands traits, des caractères principaux de l’histoire humaine. Les gens qui font ce qu’ils appellent de la philosophie de l’histoire ont imaginé que l’écriture avait commencé par le dessin, que du dessin elle était passée à la représentation symbolique, et qu’à un troisième degré, à un troisième âge, elle avait produit, comme terme final de ses développements, les systèmes phonétiques. C’est un enchaînement fort ingénieux, à coup sûr, et il est vraiment fâcheux que l’observation en démontre si complètement l’absurdité. Les systèmes figuratifs, c’est-à-dire ceux des Mexicains et des Égyptiens, sont devenus, ou plutôt ont été, dès les premiers moments de leur invention, idéographiques, parce qu’en même temps qu’on a eu à donner la forme d’un arbre, d’un fruit ou d’un animal, il a impérieusement fallu exprimer par un signe graphique l’idée incorporelle qui motivait la représentation de ces objets. Or voilà un des deux degrés de transition supprimé. Quant au troisième, il ne semble pas s’être produit nécessairement, puisque ni les Mexicains, ni les Chinois, ni les Égyptiens n’ont fait sortir de leurs hiéroglyphes un alphabet proprement dit. Le procédé que les deux derniers de ces peuples emploient pour rendre les noms propres est la plus grande preuve à offrir que le principe sur lequel se base leur système de reproduction du langage oppose des obstacles invincibles à ce prétendu développement. Les écritures idéographiques sont donc nécessairement symboliques, et, d’autre part, n’ont aucun rapport, ni passé, ni présent, ni futur, avec la méthode de décomposition élémentaire et de représentation abstraite des sons. Elles restent ce qu’elles sont, et n’atteignent pas à un but logiquement contraire au principe fondamental de leur construction primitive. – Peut-on affirmer de même que les alphabets phonétiques que nous possédons ne soient pas des descendants de systèmes idéographiques oubliés ? Poser une telle question, c’est, je le sais, affronter des axiomes qui ont acquis force de loi, mais qu’on juge de leur valeur. On part du type phénicien comme paradigme, comme souche de toutes les écritures phonétiques, et l’on veut que (alphabet étranger) représente le \emph{cou et la forme du chameau} ; (...), de même, est censé rappeler parfaitement un œil  ; (...) une maison ou une tente, etc. Pourquoi ? c’est que (...) et (...) sont les initiales de (...), de (...) et de (...). Mais (...) l’est également de (...), qui veut dire un \emph{puits}, de (...)qui signifie un \emph{bouc}, et, si l’on consent à examiner les choses sans prévention, on conviendra que (...) ressemble tout autant à un \emph{puits} ou à un \emph{bouc} qu’à un \emph{chameau}. On pourrait trouver, sans nulle peine, d’aussi nombreuses analogies pour toutes les lettres de l’alphabet. Il suffit d’un peu de bonne volonté. Voilà ce que c’est que le système qui fait dériver, inévitablement, les alphabets phonétiques des séries idéographiques, et voilà les puissantes raisons sur lesquelles il s’appuie. Aussi est-il nécessaire d’y renoncer, et au plus tôt.\par
 D’autant mieux que les études actuelles sur les alphabets assyriens font découvrir une nouvelle méthode graphique qui, de quelque façon qu’on la torture, ne saurait nullement être rapprochée du dessin symbolique. Ces combinaisons claviformes affichent, bien certainement, la prétention la mieux justifiée à ne présenter la pensée qu’au moyen de signes abstraits.\par
 Puis, au besoin, on pourrait citer encore tels modes d’écriture qui ne sont ni idéographiques, ni phonétiques, ni syllabiques, mais seulement mnémoniques, et qui se composent de traits sans autre signification que celle qui leur est attribuée par l’écrivain. Ce dernier système, fort imparfait, assurément, et privé du pouvoir d’exprimer des mots, rappelle seulement au lecteur certains objets ou certains faits déjà connus. L’écriture lenni-lenape est de ce genre.\par
 Voilà donc, la question étant prise en gros quatre catégories de ressources graphiques employées par les hommes pour garder la trace à leurs pensées. Ces quatre catégories sont fort inégales en mérite, et atteignent bien diversement le but pour lequel elles sont inventées. Elles résultent d’aptitudes très spéciales chez leurs créateurs, de façons très particulières de combiner les opérations de l’esprit et de déduire les rapports des choses. Leur étude approfondie mène à des résultats pleins d’intérêt, et sur les sociétés qui s’en servent, et sur les races dont elles émanent.
}. L’alphabet grec primitif s’écrivait tantôt de droite à gauche, tantôt de gauche à droite, et ce n’est que tard que sa marche actuelle a été fixée \footnote{Bœckh, \emph{Ueber die griechischen Inschriften auf Thera}, in-4°, Berlin, 1836, p. 17. ‑Généralement, et en dehors de l’influence romaine, les inscriptions osques, umbriques et étrusques vont donc de droite à gauche ; au contraire, l’alphabet sabellien, dans les deux seuls exemples connus jusqu’ici, suit la forme serpentine. ‑ Mommsen, \emph{Die nord etruskischen Alphabete}, p. 222.}.\par
Il n’y a là rien d’insolite. On a démontré que le dévanagari, qui suit aujourd’hui notre méthode, avait été inventé selon les besoins du système contraire. De même encore, les runes se placent de toutes les façons, de droite à gauche, de gauche à droite, de bas en haut, ou en cercle. On est même en droit d’affirmer qu’il n’existait pas primitivement de façon normale d’écrire les runes.\par
 Les seizes lettres du modèle grec ne rendaient pas tous les sons de la langue mixte formée d’éléments aborigènes, sémitiques et arians-helléniques. Elles ne pouvaient répondre davantage au besoin des idiomes de l’Asie antérieure, qui tous ont des alphabets beaucoup plus nombreux. Mais peut-être convenaient-elles mieux à l’idiome de ces habitants primitifs du pays, vaguement nommés Pélasges, dont je n’ai encore qu’indiqué l’origine celtique ou slave. Ce qui est certain, c’est que les runes du nord, que W. Grimm considère comme n’ayant point été inventées pour les dialectes teutoniques \footnote{W. C. Grimm, \emph{Ueber die teutsche Runen.}}, n’ont aussi que seize lettres, également insuffisantes pour reproduire toutes les modulations de la voix chez un Goth. W. Grimm \footnote{W. C. Grimm \emph{ouvr. cité}, p. 128. ‑ Strahlenberg, \emph{Der nord und œstliche Theil von Europa und Asien}, p. 407, 410 et 356, tab. v.}, comparant les runes aux caractères découverts par Strahlenberg et par Pallas sur les monuments arians des rives du Jenisseï, n’hésite pas à voir dans ces derniers le type originel. Il reporte, ainsi au berceau même de la race blanche la souche de tous nos alphabets actuels, et partant de l’alphabet grec ancien lui-même, sans parler des systèmes sémitiques. Cette considéra­tion deviendra dans l’avenir, je n’en doute pas, le point de départ des études les plus importantes pour l’histoire primitive.\par
Keferstein, poursuivant les traces de Grimm, relève, avec beaucoup de sagacité, que des lettres, des plus essentielles aux dialectes gothiques, manquent parmi les runes : ce sont les suivantes : c, d, e, f, g, h, q, w, x.\par
Appuyé sur cette observation, il complète fort bien la remarque de son devancier, en concluant que les runes ne sont autres que des alphabets à l’usage celtique \footnote{Keferstein Ansichten. etc., t. I, p. 353. ‑ Verelius, dans sa Runographia, avait déjà remarqué, il y a longtemps, ainsi que Rudbock, l’antériorité des runes à l’égard de la civilisation des Ases, et insisté sur l’interprétation fautive du Havamaal, qui semble attribuer à Odin l’invention des lettres sacrées, tandis que ce dieu ne peut prétendre qu’à celle de la poésie. Verelius a, de plus, fait observer que les runes étaient d’autant mieux tracées et mieux faites qu’elles étaient plus anciennes. ‑ Salverte, \emph{Essai sur l’origine des noms d’hommes, de peuples et de lieux}, t. II, p. 74, 75.}. Les caractères runiques, ainsi rendus à leurs véritables inventeurs, trouvent à l’instant un analogue très authentique chez un peuple de même race : c’est l’alphabet irlandais fort ancien, appelé \emph{bobelot} ou \emph{beluisnon}. Il est composé, comme les anciens prototypes, de seize lettres seulement, et offre avec les runes des ressemblances frappantes \footnote{Keferstein, t. I, p. 355. ‑ Dieffenbach, \emph{Celtica II}, 2\textsuperscript{e} Abth., p. 19.}.\par
Il ne faut pas perdre de vue que le système de tous ces modes d’écriture est absolument le même que celui de l’ancien grec, et que les rapports généraux de formes avec ce dernier ne cessent jamais d’exister. je termine cette revue générale en citant les alphabets italiotes, tels que 1’umbrique, l’osque, l’euganéen, le messapien \footnote{Dennis constate l’extrême similitude de tous ces alphabets. (T. I, p. XVIII.)} et les alphabets étrusques \footnote{On en compte plusieurs et dans lesquels le nombre de lettres varie. ‑ Dennis, \emph{ouvr. cité}, t. II, p. 399. ‑ Voir aussi Mommsen, \emph{Die nordetruskischen Alphabete.}}, également rapprochés du grec par leurs formes, et conséquem­ment ses alliés. Tous ces alphabets sont d’une date très reculée, et, bien qu’ayant entre eux de grandes ressemblances, ils ne présentent pas moins de diversités. Ils possèdent des lettres qui n’ont rien d’hellénique, et jouissent ainsi d’une physionomie vraiment nationale, dont il est fort difficile à la critique la plus systématique de les dépouiller \footnote{Niebuhr reconnaît que l’origine des alphabets étrusques et grecs est la même. Il la croit sémitique, à tort, suivant moi, si on veut admettre, ce qui me paraît discutable, que les écritures sémitiques soient elles-mêmes étrangères à l’invention ariane et nées sur le sol même de l’Asie antérieure après les grandes migrations. Mais le savant prussien déclare très positivement que, dans son opinion, les lettres étrusques ne se sont pas formées sur le type grec, et il en donne des raisons tout à fait concluantes. (\emph{Rœm. Geschichte}, t. I, p. 89.) Un argument à l’appui de cette assertion, qui ne me paraît pas sans valeur, c’est que le mot celtique, le mot latin et le mot grec qui signifient écrire, ont, avec une même racine, des physionomies si différentes, qu’ils doivent s’être formés sur place et ne pas provenir d’un emprunt opéré dans les âges où l’un de ces peuples a pu exercer une action sur les autres. Ainsi, (mot grec)\emph{, scribere}, et le gallois, \emph{crifellu, ysgriffen, ysgrifan}, ne se ressemblent que de loin, et on remarquera que le passage de (mot grec) à \emph{scribere} est assez bien marqué par les mots celtiques, tandis que \emph{scribere}, au contraire, n’est pas un intermédiaire entre ces mots et l’expression grecque.}. En outre, tous, sauf les étrusques, sont celtiques, comme on le verra plus tard. Pour le moment, personne n’en doutera quant à l’euganéen et à 1’umbrique.\par
Les monuments qui nous les ont conservés se montrent, pour la plupart, antérieurs à l’invasion de l’hellénisme dans la péninsule italique. Il faut donc conclure que ces alphabets européens, parents les uns des autres, parents du grec, ne sont pas formés d’après lui ; qu’ils remontent, ainsi que lui, à une origine plus ancienne ; que, comme le sang des races blanches, ils ont leur source dans les établissements primitifs de ces races au fond de la haute Asie ; que, comme les peuples qui les possèdent, ils sont originaux et vraiment indépendants de toute imitation grecque sur le territoire euro­péen où ils ont été employés ; enfin, que les nations celtiques, n’ayant pas emprunté leur genre de culture sociale à la Grèce, non plus que leur religion, non plus que leur sang, ne lui devaient pas davantage leurs systèmes graphiques \footnote{César, après avoir dit que les Celtes se servaient de caractères grecs, prouve, du reste, lui-même, l’inexactitude de son renseignement. Il raconte qu’ayant à envoyer une lettre à un de ses lieutenants, assiégé par les Belges, et ne voulant pas qu’elle pût être lue en route, il l’écrivit, non pas en langue grecque, mais en caractères grecs. Donc les caractères grecs étaient inconnus de ses adversaires. (Cæs., \emph{de Bello Gall.}, v.) ‑ Tout ce qu’il y a de peu satisfaisant dans l’assertion que les lettres en usage chez les Celtes étaient d’origine grecque a, du reste, frappé les commentateurs de César. Pour concilier les nombreuses difficultés qui leur sautaient aux yeux, ils ont eu recours à des subtilités infinies, mais dont ils se montrent, eux-mêmes tout les premiers, fort médiocrement satisfaits. ‑ Voir l’édition d’Oudendorp, in-8°, Lipsiæ, 1805. ‑ Il est effectivement inadmissible que les Celtes, ayant pour les légendes de leurs monnaies des alphabets nationaux, comme les médailles le démontrent, aient employé, dans les détails de leur vie, des caractères étrangers.}.\par
Ce qui est bien frappant chez elles, c’est l’emploi tout à fait utilitaire qui y était fait de la pensée écrite. Nous n’avons encore rien rencontré de semblable dans les sociétés féminines élevées à un degré correspondant sur l’échelle de la civilisation, et, l’esprit encore tout plein des faits que l’examen du monde asiatique a fournis aux pages du premier volume, nous devons nous reconnaître ici sur un terrain tout nouveau. Nous sommes au milieu de gens qui comprennent et éprouvent l’empire d’une raison plus sèche, et qui obéissent aux suggestions d’un intérêt plus terre à terre.\par
Les nations celtiques étaient guerrières et belliqueuses, sans doute ; mais, en définitive, beaucoup moins qu’on ne le suppose généralement. Leur renommée militaire se fonde sur les quelques invasions dont elles ont troublé la tranquillité des autres peuples. On oublie que ce furent là des convulsions passagères d’une multitude que des circonstances transitoires jetaient hors de ses voies naturelles, et que, pendant de très longs siècles, avant et après leurs grandes guerres, les États celtiques ont profondément respecté leurs voisins. En effet, leur organisation sociale avait elle-même besoin de repos pour se développer.\par
Ils étaient surtout agriculteurs, industriels et commerçants. S’il leur arrivait, comme à toutes les nations du monde, même les plus policées, de porter la guerre chez autrui, leurs citoyens s’occupaient, beaucoup plus ordinairement, de faire pâturer leurs bœufs et leurs immenses troupeaux de porcs dans les vastes clairières des forêts de chênes qui couvraient le pays. Ils étaient sans rivaux dans la préparation des viandes fumées et salées. Ils donnaient à leurs jambons un degré d’excellence qui rendit célèbre, au loin et jusqu’en Grèce, cet article de commerce \footnote{Strabon, IV, 3.}. Longtemps avant l’intervention des Romains, ils débitaient dans la péninsule italique, aussi bien que sur les marchés de Marseille, et leurs étoffes de laine, et leurs toiles de lin, et leurs cuivres, dont ils avaient inventé l’étamage. À ces différents produits ils joignaient la vente du sel, des esclaves, des eunuques, des chiens dressés pour la chasse ; ils étaient passés maîtres dans la charronnerie de toute espèce, chars de guerre, de luxe et de voyage \footnote{M. Amédée Thierry, \emph{Hist. des Gaulois, Introduct}.}. En un mot, les Kymris, comme je le faisais remarquer tout à l’heure, aussi avides marchands, pour le moins, que soldats intrépides, se classent, sans difficulté, dans le sein des peuples utilitaires, autrement dit, des nations mâles. On ne saurait les assigner à une autre catégorie. Supérieurs aux Ibères, militairement parlant, voués comme eux et plus qu’eux aux travaux lucratifs, ils ne semblent pas les avoir dépassés en besoins intellectuels. Leur luxe était surtout d’une nature positive : de belles armes, de bons habits, de beaux chevaux. Ils poussaient d’ailleurs ce dernier goût jusqu’à la passion, et faisaient venir à grands frais des coursiers de prix des pays d’outre-mer \footnote{Cæs., \emph{de Bello Gall.}, IV, 2.}.\par
 Ils paraissent cependant avoir possédé une littérature. Puisqu’ils avaient des bardes, ils avaient des chants. Ces chants exposaient l’ensemble des connaissances acquises par leur race, et conservaient les traditions cosmogoniques, théologiques, historiques.\par
La critique moderne n’a pas à la disposition de ses études des compositions écrites remontant à la véritable époque nationale. Toutefois il est, dans le fonds commun des richesses intellectuelles appartenant aux nations romanes comme aux peuples germaniques, un certain coin marqué d’une origine toute spéciale, que l’on peut revendiquer pour les Celtes. On trouve aussi, chez les Irlandais, les montagnards du nord de l’Écosse et les Bretons de l’Armorique, des productions en prose et en vers composées dans les dialectes locaux.\par
L’attention des érudits s’est fixée avec intérêt sur ces œuvres de la muse populaire. Elle leur a dû quelquefois de ressaisir les traces de quelques linéaments de l’ancienne physionomie du monde kymrique. Malheureusement, je le répète, ces compositions sont loin d’appartenir à la véritable antiquité. C’est tout ce que peuvent faire leurs admirateurs les plus enthousiastes, que d’en reporter quelques fragments au cinquième siècle \footnote{La Villemarqué\emph{, Barzaz Breiz}, t. I, p. XIV.}, date bien jeune pour permettre de juger de ce que pouvaient être les ouvrages celtiques à l’époque anté-romaine, au temps où l’esprit de la race était indépendant comme sa politique. En outre, on ressent, à l’aspect de ces œuvres, une défiance dont il n’est guère possible de se débarrasser, si l’on veut garder l’oreille ouverte à la voix de la raison. Bien que leur authenticité, en tant que produits des bardes gallois ou armoricains, des sennachies irlandais ou gaëliques, soit incontestable, on est frappé de leur ressemblance extrême avec les inspirations romaines et germaniques des siècles auxquels elles appartiennent.\par
La comparaison la plus superficielle rend cette vérité par trop notoire. Les allures de la pensée, les formes matérielles de la poésie, sont identiques \footnote{Voir le chant gallois attribué à Taliesin. (La Villemarqué, t. I, p. XIV). C’est un véritable sermon chrétien de l’époque}. Le goût est tout semblable pour la recherche énigmatique, pour la tournure sentencieuse du récit, pour l’obscurité sibyllienne, pour la combinaison ternaire des faits, pour l’allitération. À la vérité, on peut admettre que ces marques caractéristiques sont dues précisément à des emprunts primordiaux opérés sur le génie celtique par le monde germanique naissant. Tout porte à croire, en effet, que, dans le domaine moral, les Arians Germains ont dû prendre énormément des Kymris, puisque, dans l’ordre des faits ethniques et linguis­tiques, ils se sont laissé si puissamment modifier par eux. Mais, tout en reconnaissant comme admissible et même comme nécessaire ce point de départ, il n’en est pas moins très vraisemblable que les formes, les habitudes littéraires, désormais communes, ont pu, à la suite des invasions du V\textsuperscript{e} siècle, rentrer dans le patrimoine des Celtes, et, cette fois, fortement développées et enrichies par des apports dus à l’essence particulière des conquérants.\par
 Les Kymris des quatre premiers siècles de l’Église étaient, en tant que Kymris, tombés bien bas et devenus fort peu de chose. Leur vie intellectuelle, dépouillant son originalité, fut, comme le sang de la plupart de leurs nations, extrêmement altérée par l’influence romaine. La question n’en est pas une pour ce qui concerne la Gaule. Les compositions des ovates avaient péri en laissant peu de traces. Il n’en fut nullement de ces œuvres comme de celles des Étrusques, qui, bien que frappés d’impopularité auprès des vieux Sabins par la prétendue barbarie de la langue, n’en maintinrent pas moins leur importance et leur dignité, grâce à leur valeur historique. Le généalogiste et l’antiquaire se virent contraints d’en tenir compte, de les traduire, de les faire entrer, bien qu’en les transformant, dans la littérature dominante. La Gaule n’eut pas autant de bonheur. Ses peuples consentirent à l’abandon presque complet d’un patrimoine qu’ils apprirent rapidement à mépriser, et, sous toutes les faces où ils pouvaient s’examiner eux-mêmes, ils s’arrangèrent de façon à devenir aussi Latins que possible. Je veux que les idées de terroir, peut-être même quelques anciens chants, traduits et défigurés, se soient conservés dans la mémoire du peuple. Ce fonds, resté celtique au point de vue absolu, a cessé de l’être littérairement parlant, puisqu’il n’a vécu qu’à la condition de perdre ses formes.\par
Il faut donc considérer, à partir de l’époque romaine, les nations celtiques de la Gaule, de la Germanie, du pays helvétien, de la Rhétie, comme devenues étrangères à la nature spéciale de leur inspiration antique, et se borner à ne plus reconnaître chez elles que des traditions de faits et certaines dispositions d’esprit qui, persistant avec la mesure du sang des Kymris demeuré dans le nouveau mélange ethnique, ne gardaient d’autre puissance que de prédisposer les populations nouvelles à reprendre un jour quelques-unes des voies jadis familières à l’intelligence spéciale de la race gallique.\par
Les Celtes du continent, ainsi mis hors de cause longtemps avant la venue des Germains, il reste à examiner si ceux des îles de Bretagne, d’Irlande, ont conservé quelques débris du trésor intellectuel de la famille, et ce qu’ils en ont pu transmettre à leur colonie armoricaine.\par
César considère les indigènes de la grande île comme fort grossiers. Les Irlandais l’étaient encore davantage. À la vérité, les deux territoires passaient pour sacrés, et leurs sanctuaires étaient en vénération auprès des druides. Mais, autre chose est la science hiératique, autre la science profane. J’indiquerai plus bas les motifs qui me portent à croire la première très anciennement corrompue et avilie chez les Bretons. La seconde était évidemment peu cultivée par eux non pas parce que ces insulaires vivaient dans les bois ; non pas parce qu’ils n’avaient pour villes que des circonvalla­tions de branches d’arbres au milieu des forêts ; non pas parce que la dureté de leurs mœurs autorisait, à tort ou à raison, à les accuser d’anthropophagie ; mais parce que les traditions génésiaques qu’on leur attribue contiennent une trop faible proportion de faits originaux.\par
La prédominance des idées classiques y est évidente. Elle saute aux yeux, et elle ne nous apparaît même pas sous le costume latin ; c’est dans la forme chrétienne, dans la forme monacale, dans le style de pensée germano-romain, qu’elle s’offre à nos regards \footnote{Dieffenbach, \emph{Celtica II}, 2\textsuperscript{e} Abth., p. 55.}. Aucun observateur de bonne foi ne peut se refuser à reconnaître que les pieux cénobites du VI\textsuperscript{e} siècle ont, sinon composé toutes ses œuvres, du moins donné le ton à leurs compositeurs, même païens. Dans tous ces livres, à côté de César et de ses soldats, on voit apparaître les histoires bibliques : Magog et les fils de Japhet, les Pharaons et la terre d’Égypte ; puis le reflet des événements contemporains : les Saxons, la grandeur de Constantinople, la puissance redoutée d’Attila.\par
De ces remarques je ne tire pas la conséquence qu’il n’existe absolument aucun reste de souvenir véritablement ancien dans cette littérature ; mais je pense qu’elle appartient, totalement dans ses formes et presque entièrement dans le fond, à l’époque où les indigènes n’étaient plus seuls à habiter leurs territoires, à l’époque où leur race avait cessé d’être uniquement celtique, à celle où le christianisme et la puissance germanique, bien que trouvant encore parmi eux de grandes résistances, n’en étaient pas moins victorieux, dominateurs, et capables de plier à leurs vues l’intelligence intimidée des plus haineux ennemis.\par
Toutes ces raisons, en établissant que les groupes parlant, depuis l’ère chrétienne, des dialectes celtiques, avaient, depuis longtemps, perdu toute inspiration propre, appuient encore cette proposition, avancée tout à l’heure, que, si le génie germanique s’est, à son origine, enrichi d’apports kymriques, c’est sous son influence, c’est avec ce qu’il a rendu aux peuplades gaéliques, galloises et bretonnes, que s’est composée, vers le V\textsuperscript{e} siècle, la littérature de ces tribus, littérature que dès lors on est en droit d’appeler moderne. Celle-ci n’est plus qu’un dérivé de courants multiples, non pas une source originale. Je ne répéterai donc pas, avec tant de philologues, que les habitants celtiques de l’Angleterre possédaient, à l’aurore de l’âge féodal, des chants et des romans purement tirés de leur propre invention, et qui ont fait le tour de l’Europe ; mais, tout au contraire, je dirai que, de même que les moines irlandais, les sculdées ont brillé d’un éclat de science théologique, d’une énergie de prosélytisme tout à fait admirable et étranger aux habitudes égoïstes et peu enthousiastes des races galliques, de même leurs poètes, placés sous les mêmes influences étrangères, ont puisé dans le conflit d’idées et d’habitudes qui en résultèrent, dans le trésor des traditions si variées ouvert sous leurs yeux, enfin dans le faible et obscur patrimoine qui leur avait été légué par leurs pères, cette série de productions qui a, en effet, réussi dans toute l’Europe, mais qui a dû son vaste succès à ce motif même qu’elle ne reflétait pas les tendances absolues d’une race spéciale et isolée : tout au contraire, elle était à la fois le produit de la pensée celtique, romaine et germanique, et de là son immense popularité.\par
Cette opinion ne serait assurément pas soutenable, elle serait même opposée à toutes doctrines de ce livre, si la pureté de race qu’on attribue généralement aux populations parlant encore le celtique était prouvée. L’argument, et c’est le seul dont on se sert pour l’établir, consiste dans la persistance de la langue. On a déjà vu plusieurs fois, et notamment \footnote{\emph{Vid. supra} et livre I\textsuperscript{er}.} à propos des Basques, combien cette manière de raisonner est peu concluante . Les habitants des Pyrénées ne sauraient passer pour les descendants d’une race primitive, encore moins d’une race pure ; les plus simples considérations physio­logiques s’y opposent. Les mêmes raisons ne font pas moins de résistance à ce que les Irlandais, les montagnards de l’Écosse, les Gallois, les habitants de la Cornouaille anglaise et les Bretons soient considérés comme des peuples typiques et sans mélange. Sans doute, on rencontre, en général, parmi eux, et chez les Bretons surtout, des physionomies marquées d’un cachet bien particulier ; mais nulle part on n’aperçoit cette ressemblance générale des traits, apanage, sinon des races pures, au moins des races dont les éléments sont depuis assez longtemps amalgamés pour être devenus homogènes. Je n’insiste pas sur les différences très graves que présentent les groupes néo-celtiques quand on les compare entre eux. La persistance de la langue n’est donc pas, ici plus qu’ailleurs, une garantie certaine de pureté quant au sang. C’est le résultat des circonstances locales, fortement servies par les positions géographiques.\par
Ce que la physiologie ébranle, l’histoire le renverse. On sait de la manière la plus positive que les expéditions et les établissements des Danois et des Norwégiens dans les îles semées autour de la Grande-Bretagne et de l’Irlande ont commencé de très bonne heure \footnote{Dieffenbach, Celtica 11, 2\textsuperscript{e} Abth., p. 310 et pass. ‑ Tacite n’hésitait déjà pas à reconnaître parmi les habitants de la Calédonie la présence d’une race germanique : «\emph{ Rutilæ Caledoniam} « \emph{habitantium comæ, magni artus germanicum originem adseverant.} »  (\emph{Vita Agric}., II) – Je n’en conclus pas que tous les Calédoniens étaient des Germains ; mais rien ne s’oppose à ce qu’en effet il y eût alors des immigrants germains en Écosse.}. Dublin a appartenu à des populations et à des rois de race danoise, et un écrivain on ne peut plus compétent a solidement établi que les chefs des clans écossais étaient, au moyen âge, d’extraction danoise, comme leurs nobles ; que leur résistance à la couronne avait pour appuis les dominateurs danois des Orcades, et que leur chute, au XII\textsuperscript{e} siècle, fut la conséquence de celle de ces dynastes, leurs parents \footnote{\emph{Ibid.}}\par
Dieffenbach constate, en conséquence, l’existence d’un mélange scandinave et même saxon très prononcé chez les Highlanders. Avant lui, Murray avait reconnu l’accent danois dans le dialecte du Buchanshire, et Pinkerton, analysant les idiomes de l’île entière, avait également signalé, dans une province qui passe d’ordinaire pour essentiellement celtique, le pays de Galles, des traces si évidentes et si nombreuses du saxon, qu’il nomme le gallois \emph{a saxonised celtic} \footnote{Dieffenbach, \emph{Celtica II}, 2\textsuperscript{e} Abth., p. 286. Sur l’extrême appauvrissement du breton et les mutilations qu’il a subies en se rapprochant dans ses formes grammaticales du français moderne, voir La Villemarqué, \emph{Barzaz Breiz}, t. I, p. LXI.}.\par
Ce sont là les principaux motifs qui me semblent s’opposer à ce que l’on puisse considérer les ouvrages gallois, erses ou bretons comme reproduisant, même d’une manière approximative, soit les idées, soit le goût des populations kymriques de l’occident européen. Pour se former une idée juste à ce sujet, il me paraît plus exact de choisir un terrain d’abstraction. Prenons en bloc les productions romaines et germani­ques ; résumons, d’autre part, tout ce que les historiens et les polygraphes nous ont transmis d’aperçus et de détails sur le génie particulier des Celtes, et nous en pourrons tirer les conclusions suivantes.\par
 L’exaltation enthousiaste, observée en Orient, n’était pas le fait de la littérature des Galls. Soit dans les ouvrages historiques, soit dans les récits mythiques, elle aimait l’exactitude, ou, à défaut de cette qualité, ces formes affirmatives et précises qui, auprès de l’imagination, en tiennent lieu \footnote{M. de La Villemarqué relève avec raison, chez les auteurs des chants populaires de l’Europe, l’habitude de fixer aussi exactement que possible le lieu et la date des faits rapportés. (\emph{Barzaz Breiz}, t. I, p. XXVI.) Le but de ce qu’il appelle \emph{le poète de la nature} « est toujours, dit-il, de rendre la réalité. » (P. XXVIII.)}. Elle cherchait les faits plus que les sentiments ; elle tendait à produire l’émotion, non pas tant par la façon de dire, comme les Sémites, que par la valeur intrinsèque, soit tristesse, soit énergie, de ce qu’elle énonçait. Elle était positive, volontiers descriptive, ainsi que le voulait l’alliance intime qui la rapprochait du sang finnique, ainsi qu’on en voit l’exemple dans le génie chinois, et, par son défaut intime de chaleur et d’expansion, volontiers elliptique et concise. Cette austérité de forme lui permettait d’ailleurs une sorte de mélancolie vague et facilement sympathique qui fait encore le charme de la poésie populaire dans nos pays.\par
On trouvera, je l’espère, cette appréciation admissible, si l’on se rappelle qu’une littérature est toujours le reflet du peuple qui l’a produite, le résultat de son état ethnique, et si l’on compare les conclusions qui ressortent de cette vérité avec l’ensemble des qualités et des défauts que le contenu des pages précédentes a fait apercevoir dans le mode de culture des nations celtiques.\par
Il en résulte sans doute que les Kymris ne pouvaient pas être doués, intellec­tuellement, à la manière des nations mélanisées du sud. Si cette condition mettait son empreinte sur leurs productions littéraires, elle n’était pas moins sensible dans le domaine des arts plastiques. De tout le bagage que les Galls ont laissé derrière eux en ce genre, et que leurs tombes nous ont rendu, on peut admirer la variété, la richesse, la bonne et solide confection : il n’y a pas lieu de s’extasier sur la forme. Elle y est des plus vulgaires, et ne fournit aucune trace qui puisse faire reconnaître un esprit amusé, comme dans l’Asie antérieure, à donner de belles apparences aux moindres objets ou sentant le besoin de plaire à des yeux exigeants \footnote{Keferstein, \emph{Ansichten}, t. I p. 334.}.\par
Il est vraiment curieux que César, qui s’étend avec assez de complaisance sur tout ce qu’il a rencontré dans les Gaules, et qui loue avec beaucoup d’impartialité ce qui le mérite, ne se montre aucunement séduit par la valeur artistique de ce qu’il observe. Il voit des villes populeuses, des remparts très bien conçus et exécutés : il ne mentionne pas une seule fois un beau temple \footnote{Le fait que les Celtes élevaient des sanctuaires dans leurs villes, à Toulouse entre autres, prouve que les dolmens n’appartenaient pas à leur culte ordinaire. Strabon, parlant de l’ancienne splendeur des Tectosages, raconte qu’ils déposaient leurs trésors dans les chapelles, (mot grec) ou dans les étangs sacrés, (mots grecs). Si les dolmens avaient été ces (mot grec), leur forme les aurait rendus trop remarquables pour que Posidonius n’en eût pas fait la description. (Strab., IV, 13.)}. S’il parle des sanctuaires aperçus par lui dans les cités, cet aspect ne lui inspire ni éloge ni blâme, ni expression de curiosité. Il paraît que ces constructions étaient, comme toutes les autres, appropriées à leur but, et rien de plus. J’imagine que ceux de nos édifices modernes qui ne sont copiés ni du grec, ni du romain, ni du gothique, ni de l’arabe, ni de quelque autre style, inspirent la même indifférence aux observateurs désintéressés.\par
On a trouvé, outre les armes et les ustensiles, un très petit nombre de représenta­tions figurées de l’homme ou des animaux. J’avoue même que je n’en connais pas d’exemple bien authentique.\par
Le goût général, semblerait-il donc, ne portait pas les fabricants ou les artistes à ce genre de travail. Le peu qu’on en possède est fort grossier et tel que le moindre manœuvre en saurait faire autant. L’ornementation des vases, des objets en bronze ou en fer, des parures en or ou en argent, est de même dénuée de goût, à moins que ce ne soient des copies d’œuvres grecques ou plutôt romaines, particularité qui indique, lorsqu’elle se rencontre, que l’objet observé appartient à l’époque de la domination des Césars, ou du moins à un temps qui en est assez rapproché. Dans les périodes nationales, les dessins en spirales simples et doubles ou en lignes ondulées sont extrêmement communs : c’est même le sujet le plus ordinaire.\par
Nous avons vu que les gravures observées sur les plus beaux dolmens de construction finnique affectaient ordinairement cette forme. Il semblerait donc que les Celtes, tout en gardant leur supériorité vis-à-vis des habitants antérieurs du pays, se sont sentis assez pauvrement pourvus du côté de l’imagination pour ne pas dédaigner les leçons de ces malheureux \footnote{Telle est la persistance des goûts dans les races qu’aux environs de Francfort-sur-le-Main, où l’on trouve beaucoup de maisons construites à la manière celtique, les dessins dont ces maisons sont ornées reproduisent constamment les mêmes spirales qui se voient sur les monuments de Gavr-Innis.}. Mais, comme de pareils emprunts ne s’opèrent jamais qu’entre nations parentes, en trouver la marque peut servir à faire remarquer qu’outre les mélanges jaunes, déjà subis pendant la durée de la migration à travers l’Europe, les Celtes en contractèrent beaucoup d’autres avec les édificateurs des dolmens dans la plupart des contrées où ils s’établirent, sinon dans toutes. Cette conclusion n’a rien d’inattendu pour l’esprit du lecteur : de puissants indices l’ont déjà signalée.\par
Il en est d’ailleurs d’autres encore, et d’une nature plus relevée et plus importante que de simples détails d’éducation artistique. C’est ici le lieu d’en parler avec quelque insistance.\par
Quand j’ai dit que le système aristocratique était en vigueur chez les Galls, je n’ai pas ajouté, ce qui pourtant est nécessaire, que l’esclavage existait également parmi eux.\par
On voit que leur mode de gouvernement était assez compliqué pour mériter une sérieuse étude. Un chef électif, un corps de noblesse moitié sacerdotale, moitié mili­taire, une classe moyenne, bref l’organisation blanche, et, au-dessous, une population servile. Sauf le brillant des couleurs, on croit se retrouver dans l’Inde.\par
Dans ce dernier pays, les esclaves, aux temps primitifs, se composaient de noirs soumis par les Arians. En Égypte, les basses castes ayant été également formées, et presque en totalité, de nègres, force est d’en conclure qu’elles devaient de même leur situation à la conquête ou à ses conséquences. Dans les États chamo-sémitiques, à Tyr, à Carthage, il en était ainsi. En Grèce, les Hélotes lacédémoniens, les Pœnestes thessaliens et tant d’autres catégories de paysans attachés à la glèbe, étaient les descen­dants des aborigènes soumis. Il résulte de ces exemples que l’existence de populations serviles, même avec des nuances notables dans le traitement qui leur est infligé, dénote toujours des différences originelles entre les races nationales.\par
L’esclavage, ainsi que toutes les autres institutions humaines, repose sur d’autres conditions encore que le fait de la contrainte. On peut, sans doute, taxer cette institu­tion d’être l’abus d’un droit ; une civilisation avancée peut avoir des raisons philosophiques à apporter au secours de raisons ethniques, plus concluantes, pour la détruire : il n’en est pas moins incontestable qu’à certaines époques l’esclavage a sa légitimité, et on serait presque autorisé à affirmer qu’il résulte tout autant du consentement de celui qui le subit que de la prédominance morale et physique de celui qui l’impose.\par
On ne comprend pas qu’entre deux hommes doués d’une intelligence égale ce pacte subsiste un seul jour sans qu’il y ait protestation et bientôt cessation d’un état de choses illogique. Mais on est parfaitement en droit d’admettre que de tels rapports s’établissent entre le fort et le faible, ayant tous deux pleine conscience de leur position mutuelle, et ravalent ce dernier à une sincère conviction que son abaissement est justifiable en saine équité.\par
La servitude ne se maintient jamais dans une société dont les éléments divers se sont un tant soit peu fondus. Longtemps avant que l’amalgame arrive à sa perfection, cette situation se modifie, puis s’abolit. Bien moins encore est-il possible que la moitié d’une race dise à son autre moitié : « Tu me serviras, » et que l’autre obéisse \footnote{On opposera peut-être à ceci qu’en Russie comme en Pologne le servage est d’institution récente ; mais il faut observer, d’abord, que la situation du paysan de l’empire mérite à peine ce nom ; puis, dans les deux pays, elle se transforme rapidement en liberté complète, preuve qu’elle n’a jamais été subie sans protestation. Elle n’aura donc constitué qu’un accident transitoire, résultat naturel de la superposition de races différemment douées ; car, en Pologne aussi bien qu’en Russie, la noblesse est issue de conquérants étrangers. Aujourd’hui, cette ligne de démarcation ethnique disparaissant ou ayant disparu, le servage n’a plus de raison d’être et le prouve en s’éteignant.}.\par
De tels exemples ne se sont jamais produits, et ce que le poids des armes pourrait consacrer un moment, n’étant jamais ratifié par la conscience des opprimés, fragile et vacillant, s’anéantirait bientôt. Ainsi, partout où il y a esclavage, il y a dualité ou pluralité de races. Il y a des vainqueurs et des vaincus, et l’oppression est d’autant plus complète que les races sont plus distinctes. Les esclaves, les vaincus, chez les Galls, ce furent les Finnois. Je ne m’arrêterai pas à combattre l’opinion qui veut apercevoir dans la population servile de la Celtique des tribus ibériennes proprement dites. Rien n’indique que cette famille hispanique ait jamais occupé les provinces situées au nord de la Garonne \footnote{Le rapprochement que l’on peut établir entre le nom de la nation hispanique métisse des Ligures et celui du fleuve de Loire, \emph{Liger}, prouverait simplement que les Ligures avaient adopté le nom de la tribu austro-celtique paternelle, qui leur semblait plus honorable que celui de tout autre peuple, ibère d’origine, dont ils pouvaient également descendre. L’héritage de cette partie de leur généalogie se composait de souvenirs moins brillants. (Dieffenbach, \emph{Celtica II}, 1\textsuperscript{re} Abth., p. 22.) ‑ Voir encore le même auteur pour le nom des Llœgrwys, que les Triades gaéliques rattachent à la souche primitive des Kymris. (\emph{Ibid.}, 27‑ Abth., p. 71 et 130.)}. Puis les différences n’étaient pas telles entre les Galls et les maîtres de l’Espagne, que ces derniers aient pu être abaissés en masse au rôle d’esclaves vis-à-vis de leurs dominateurs. Quand des expéditions kymriques, pénétrant dans la Péninsule, allèrent y troubler tous les rapports antérieurs, nous en voyons résulter des expulsions et des mélanges ; mais tout démontre que, la guerre finie, il y eut, entre les deux parties contendantes, des relations généralement basées sur la reconnaissance d’une certaine égalité \footnote{Les Celtibériens, produit de l’hymen des deux peuples, se montrèrent peut-être un peu supérieurs aux familles d’où ils sortaient. J’ai déjà fait remarquer que ce fait était assez ordinaire dans les alliages d’espèces inférieures ou secondaires. (Voir t. I, livre I\textsuperscript{er}.) Dieffenbach (\emph{Celtica II}, 2\textsuperscript{e} Abth., p. 47) fait cette même observation, précisé ment à propos du sujet dont il s’agit ici.}.\par
Il en fut absolument de même pour d’autres groupes à demi blancs, apparentés aux Ibères d’assez près, et plus tard aux Galls. Ces groupes étaient composés de Slaves qui, semés sur plusieurs points des pays celtiques, y vivaient sporadiquement, côte à côte avec les Kymris. Les mêmes motifs qui empêchaient les Ibères d’Espagne, envahis par les Celtes, d’être réduits en esclavage, assuraient à ces Wendes, perdus loin du gros de leur race, une attitude d’indépendance. On les voit formant dans l’Armorique une nation distincte, et y portant leur nom national de \emph{Veneti}. Ces Vénètes avaient aussi dans le pays de Galles actuel une partie des leurs \footnote{Schaffarik, \emph{Slawische Alterth.}, t. I, p. 260.}, dont la résidence était Wenedotia ou Gwineth. La Vilaine s’appelait, d’après eux, \emph{Vindilis}. La ville de Vannes garde aussi bien dans son nom une trace de leur souvenir, et ce qui est assez curieux, c’est qu’elle le garde dans la forme que les Finnois donnent au mot \emph{Wende : Wane} \footnote{Schaffarik, \emph{ouvr. cité}, t. I, p. 260.}.\par
Une tribu gallique, parente des Vénètes, les Osismii, possédait un port qu’elle nommait Vindana \footnote{En breton, \emph{Gwenet} et \emph{Wenet}. C’est une règle curieuse que là où les Hellènes mettaient le digamma et où les Grecs modernes placent le \emph{C}, les Celtes, les Latins et les Slaves emploient le \emph{W}. Le digamma se confond avec l’esprit rude ; les dialectes gothiques, et le sanscrit même, remplacent le \emph{W} par le \emph{H}. (Shaffarik, \emph{Slawische Alterthümer}, t. I, p. 160.) On trouve encore en France la racine \emph{Vend} dans plusieurs autres noms de lieux à l’ouest, tels que Vendôme et la Vendée. Strabon nomme encore des (\emph{mot grec}) ou \emph{Vennones} au-dessus de Côme, à côté des Rhétiens, non loin, par conséquent, des Vénètes de l’Adriatique. (L. IV, 6.) ‑ Dieffenbach, \emph{Celtica II}, 1\textsuperscript{re} Abth., p. 342, 219, 220, 222.}. Bien loin de là encore, sur l’Adriatique et tout à côté des Celtes Euganéens, résidaient les \emph{Veneti, Heneti} ou \emph{Eneti}, dont la nationalité est un fait historiquement reconnu, mais qui, bien que parlant une langue particulière, avaient absolument les mêmes mœurs que les Galls, leurs voisins. Plusieurs autres populations slaves, celtisées dans des proportions diverses, vivaient au nord-est de l’Allemagne et sur la ligne des Krapacks, côte à côte avec les nations galliques.\par
Tous ces faits démontrent que les Slaves de la Gaule et de l’Italie, comme les Ibères d’Espagne, conservaient un rang assez digne et faisaient nombre parmi les États kymriques auxquels ils s’étaient alliés. Sans donc songer à déshonorer gratuitement leur mémoire, cherchons la race servile où elle put être : nous ne trouvons que les Finnois.\par
Leur contact immédiat devait nécessairement exercer sur leurs vainqueurs, bientôt leurs parents, une influence délétère. On en retrouve les preuves évidentes.\par
Au premier rang il faut mettre l’usage des sacrifices humains, dans la forme où on les pratiquait, et avec le sens qu’on leur donnait. Si l’instinct destructif est le caractère indélébile de l’humanité entière, comme de tout ce qui a vie dans la nature, c’est assurément parmi les basses variétés de l’espèce qu’il se montre le plus aiguisé. À ce titre, les peuples jaunes le possèdent tout aussi bien que les noirs. Mais, attendu que les premiers le manifestent au moyen d’un appareil spécial de sentiments et d’actions, il s’exerçait aussi chez les Galls, atteints par le sang finnique, d’une autre façon que chez les nations sémitiques, imbues de l’essence mélanienne. On ne voyait pas, dans les cantons celtiques, les choses se passer comme aux bords de l’Euphrate. Jamais, sur des autels publiquement élevés au milieu des villes, au centre de places inondées de la clarté du soleil, les rites homicides du sacerdoce druidique ne s’accomplirent impudemment, avec une sorte de rage bruyante, solennelle, délirante, joyeuse de nuire. Le culte morose et chagrin de ces prêtres d’Europe ne visait pas à repaître des imaginations ardentes par le spectacle enivrant de cruautés raffinées. Ce n’était pas à des goûts savants dans l’art des tortures qu’il fallait arracher des applaudissements. Un esprit de sombre superstition, amant des terreurs taciturnes, réclamait des scènes plus mystérieuses et non moins tragiques. À cette fin, on réunissait un peuple entier au fond des bois épais. Là, pendant la nuit, des hurlements poussés par des invisibles frappaient l’oreille effrayée des fidèles. Puis, sous la voûte consacrée du feuillage humide qui laissait à peine tomber sur une scène terrible la clarté douteuse d’une lune occidentale, sur un autel de granit grossièrement façonné, et emprunté à d’anciens rites barbares, les sacrificateurs faisaient approcher les victimes et leur enfonçaient, en silence, le couteau d’airain dans la gorge ou dans le flanc. D’autres fois, ces prêtres remplissaient de gigantesques mannequins d’osier de captifs et de criminels, et faisaient tout flamber dans une des clairières de leurs grandes forêts.\par
Ces horreurs s’accomplissaient comme secrètement ; et, tandis que le Chamite sortait de ses boucheries hiératiques ivre de carnage, rendu insensé par l’odeur du sang dont on venait de lui gonfler les narines et le cerveau, le Gall revenait de ses solennités religieuses, soucieux et hébété d’épouvante. Voilà la différence : à l’un, la férocité active et brûlante du principe mélanien ; à l’autre, la cruauté froide et triste de l’élément jaune. Le nègre détruit parce qu’il s’exalte, et s’exalte parce qu’il détruit. L’homme jaune tue sans émotion et pour répondre à un besoin momentané de son esprit. J’ai montré, ailleurs, qu’à la Chine l’adoption de certaines modes féroces, com­me d’enterrer des femmes et des esclaves avec le cadavre d’un prince, correspondait à des invasions de nouveaux peuples jaunes dans l’empire.\par
Chez les Celtes, tout l’ensemble du culte portait également témoignage de cette influence. Ce n’est pas que les dogmes et certains rites fussent absolument dépouillés de ce qu’ils devaient à l’origine primitivement noble de la famille. Les mythologues y ont découvert de frappantes analogies avec les idées hindoues, surtout quant aux théories cosmogoniques. Le sacerdoce lui-même, voué à la contemplation et à l’étude, façonné aux austérités et aux fatigues, étranger à l’usage des armes, placé au-dessus, sinon au dehors de la vie mondaine, et jouissant du droit de la guider, tout en ayant le devoir d’en faire peu de cas, ce sont là autant de traits qui rappellent assez bien la physionomie des purohitas.\par
Mais ces derniers ne dédaignaient aucune science et pratiquaient toutes les façons de perfectionner leur esprit. Les druides avilis s’en tenaient à des enseignements à jamais fermés et à des formes traditionnelles. Ils ne voulaient rien savoir au delà, ni surtout rien communiquer, et les terreurs dangereuses dont ils entouraient leurs sanctuaires, les périls matériels qu’ils accumulaient autour des forêts ou des landes qui leur servaient d’école, étaient moins rébarbatifs encore que les obstacles moraux apportés par eux à la pénétration de leurs connaissances. Des nécessités analogues à celles qui dégradèrent les sacerdoces chamitiques pesaient sur leur génie.\par
Ils craignaient l’usage de l’écriture. Leur doctrine entière était confiée à la mémoire. Bien différents des purohitas sur ce point capital, ils redoutaient tout ce qui aurait pu faire apprécier et juger leurs idées. Ils prétendaient, seuls de leurs nations, avoir les yeux ouverts sur les choses de la vie future. Forcés de reconnaître l’imbécillité religieuse des masses serviles, et plus tard des métis qui les entouraient, ils n’avaient pas pris garde que cette imbécillité les gagnait, parce qu’ils étaient des métis eux-mêmes. En effet, ils avaient omis ce qui aurait pu seul maintenir leur supériorité en face des laïques : ils ne s’étaient pas organisés en caste ; ils n’avaient pris nul soin de garder pure leur valeur ethnique. Au bout d’un certain temps, la barbarie, dont ils avaient cru sans doute se garantir par le silence, les avait envahis, et toutes les plates sottises et les atroces suggestions de leurs esclaves avaient pénétré au sein de leurs sanctuaires si bien clos, en s’y glissant dans le sang de leurs propres veines. Rien de plus naturel.\par
Comme tous les autres grands faits sociaux, la religion d’un peuple se combine d’après l’état ethnique. Le catholicisme lui-même condescend à se plier, quant aux détails, aux instincts, aux idées, aux goûts de ses fidèles. Une église de la Westphalie n’a pas l’apparence d’une cathédrale péruvienne ; mais, lorsque c’est de religions païennes qu’il s’agit, comme elles sont issues presque entièrement de l’instinct des races, au lieu de dominer cet instinct, elles lui obéissent sans réserve, reflétant son image avec la fidélité la plus scrupuleuse. Il n’y a pas de danger, d’ailleurs, qu’elles s’inspirent avec partialité de la partie la plus noble du sang. Existant surtout pour le plus grand nombre, c’est au plus grand nombre qu’elles doivent parler et plaire. S’il est abâtardi, la religion se conforme à la décomposition générale, et bientôt se fait fort d’en sanctifier toutes les erreurs, d’en refléter tous les crimes. Les sacrifices humains, tels qu’ils furent consentis par les druides, donnent une nouvelle démonstration de cette vérité.\par
Parmi les nations galliques du continent, les plus attachées à ce rite épouvantable étaient celles de l’Armorique. C’est, en même temps, une des contrées qui possèdent le plus de monuments finnois. Les landes de ce territoire, le bord de ses rivières, ses nombreux marécages, virent se conserver longtemps l’indépendance des indigènes de race jaune. Cependant les îles normandes, la Grande-Bretagne, l’Irlande et les archipels qui l’entourent, furent encore plus favorisés à cet égard  \footnote{Il ne serait pas impossible qu’au temps de César, les îles situées à l’embouchure du Rhin aient été encore occupées par des tribus purement finnoises. Le dictateur raconte que les hommes qui les habitaient étaient extrêmement barbares et féroces, et vivaient uniquement de poissons et d’œufs d’oiseaux. Il les distingue complètement des Belges. (\emph{De Bello Gall.}, IV, 10.) Quant à la situation ethnique des Celtes des îles de l’ouest, on peut juger combien elle était dégradée, par ce fait que certaines tribus avaient adopté le nom même des jaunes et s’appelaient les \emph{Féniens}. On trouve également l’indication d’un mélange avoué dans le nom caractéristique de Fin-gal.}.\par
Dans ses provinces intérieures, l’Angleterre possédait des populations celtiques inférieures de tout point à celles de la Gaule \footnote{Strabon (IV, chap. v, 2) raconte que plusieurs peuplades de la Grande-Bretagne étaient tellement grossières qu’ayant beaucoup de lait, elles ne savaient pas même en confectionner du fromage. Ce détail emprunte de l’intérêt à la même incapacité signalée chez plusieurs peuples jaunes. ‑ Voir plus loin.}, et qui, plus tard, ayant renvoyé à l’Armorique des habitants pour repeupler ses campagnes désertes, lui donnèrent cette colonie singulière qui, au milieu du monde moderne\textasciitilde a conservé l’idiome des Kymris. Certains Bas-Bretons, avec leur taille courte et ramassée, leur tête grosse, leur face carrée et sérieuse, généralement triste, leurs yeux souvent bridés et relevés à l’angle extrême, trahissent, pour l’observateur le moins exercé, la présence irrécusable du sang finnique à très forte dose.\par
Ce furent ces hommes si mélangés, tant de l’Angleterre que de l’Armorique, qui se montrèrent le plus longtemps attachés aux superstitions cruelles de leur religion nationale. De tels rites étaient abandonnés et oubliés par le reste de leur famille, qu’eux s’y cramponnaient avec passion. On peut juger du degré d’amour qu’ils lui portaient, en songeant qu’ils conservent actuellement, dans leur préoccupation pour le droit de bris, des notions tirées du code de morale honoré chez leurs antiques compatriotes, les Cimmériens de la Tauride.\par
Les druides avaient placé parmi ces Armoricains leur séjour de prédilection. C’était chez eux qu’ils entretenaient leurs principales écoles \footnote{Les réunions druidiques annuelles du pays Chartrain n’avaient pas pour but de traiter des questions religieuses ; il ne s’agissait là que d’affaires temporelles. (Cæs., de Bello Gall., vi, 13.) ‑Une singulière opinion des druides voulait que le peuple entier des Celtes descendît de Pluton. Cette doctrine, reproduite par une bouche et avec des formes romaines, pourrait bien se rattacher à des idées finnoises, et se rapprocher de celles qui mêlent constamment cette race de petite taille aux rochers, aux cavernes et aux mines. (Cæsar, \emph{de Bello Gall.}, VI, 18.) Peut-être aussi n’était-ce qu’un jeu de mots sur le nom commun à toutes les tribus : \emph{gal}, qui signifie aussi \emph{obscurité}, et qui, dans cette acception, est la racine des mots teutoniques : \emph{Hœlle} et \emph{Hell}, l’enfer, comme du latin : \emph{caligo, les ténèbres}.}.\par
Conformément à l’instinct le plus obstiné de l’espèce blanche, ils avaient admis les femmes au premier rang des interprètes de la volonté divine. Cette institution, impossi­ble à maintenir dans les régions du sud de l’Asie, devant les notions mélaniennes, leur avait été facile à conserver en Europe. Les hordes jaunes, tout en repoussant leurs mères et leurs filles dans un profond état d’abjection et de servilité, les emploient volontiers, aujourd’hui encore, aux œuvres magiques. L’extrême irritabilité nerveuse de ces créatures les rend propres à ces emplois. J’ai déjà dit qu’elles étaient, des trois races qui composent l’humanité, les femmes les plus soumises aux influences et aux maladies hystériques. De là, dans la hiérarchie religieuse de toutes les nations celtiques, ces druidesses, ces prophétesses qui, soit renfermées à jamais dans une tour solitaire, soit réunies en congrégations sur un îlot perdu dans l’océan du Nord, et dont l’abord était mortel pour les profanes, tantôt vouées à un éternel célibat, tantôt offertes à des hymens temporaires ou à des prostitutions fortuites, exerçaient sur l’imagination des peuples un prestige extraordinaire, et les dominaient surtout par l’épouvante,\par
C’est en employant de tels moyens que les prêtres, flattant la populace jaune de préférence aux classes moins dégradées, maintenaient leur pouvoir en l’appuyant sur des instincts dont ils avaient caressé et idéalisé les faiblesses. Aussi n’y a-t-il rien d’étrange à ce que la tradition populaire ait rattaché le souvenir des druides aux cromlechs et aux dolmens. La religion était de toutes les choses kymriques celle qui s’était mise le plus intimement en rapport avec les constructeurs de ces horribles monuments.\par
Mais ce n’était pas la seule. La grossièreté primitive avait pénétré de toutes parts dans les mœurs du Celte. Comme l’Ibère, comme l’Étrusque, le Thrace et le Slave, sa sensualité, dénuée d’imagination, le portait communément à se gorger de viandes et de liqueurs spiritueuses, simplement pour éprouver un surcroît de bien-être physique. Toutefois, disent les documents, cette habitude avait d’autant plus de prise sur le Gall qu’il se rapprochait davantage des basses classes \footnote{Am. Thierry, \emph{Hist. des Gaulois}, t. II, p. 62. ‑ Il ne faut pas confondre cet amour de la débauche avec la puissance de consommation dont s’honoraient les Arians Hellènes et les Scandinaves. Pour ces derniers peuples, c’était uniquement un signe de force chez les héros. On ne voit nulle part d’allusion qui puisse indiquer que l’ivresse en fût le résultat et parût excusable.}. Les chefs ne s’y abandonnaient qu’à demi. Dans le peuple, mieux assimilé aux populations esclaves, on rencontrait souvent des hommes qu’une constante ivrognerie avait conduits par degrés à un complet idiotisme. C’est encore de nos jours chez les nations jaunes que se trouvent les exemples les plus frappants de cette bestiale habitude. Les Galls l’avaient évidemment contractée par suite de leurs alliances finnoises, puisqu’ils y étaient d’autant moins soumis que le sang des individus était plus indépendant de ces mélanges \footnote{Dans les populations de l’Europe actuelle l’ivrognerie est surtout répandue chez les Slaves, les restes de la race kymrique, les Allemands slavisés du sud, et les Scandinaves métis de Finnois ; mais les Lapons y sont les plus abandonnés de tous.}.\par
À tous ces effets moraux ou autres, il ne reste plus qu’à joindre les résultats produits dans la langue des Kymris par l’association des éléments idiomatiques prove­nus de la race jaune. Ces résultats sont dignes de considération.\par
Bien que la conformation physique des Galls, très pareille à celle qu’on observa plus tard chez les Germains, ait conservé longtemps aux premiers la marque irréfragable d’une alliance étroite avec l’espèce blanche, la linguistique n’est arrivée que très tard à appuyer cette vérité de son assentiment \footnote{Il est bon de remarquer que la numismatique favorise ce doute. Je citerai, entre autres, une médaille d’or des Médiomatrices, dont la face porte une figure marquée du type le plus laid, le plus vulgaire, le plus commun, et dans lequel l’influence finnique est impossible à méconnaître. Nos rues et nos boutiques sont remplies aujourd’hui de ce genre de physionomies. ‑ \emph{Cabinet de S. E. M. le général baron de Prokesch-Osten.}}.\par
Les dialectes celtiques faisaient tant de résistance à se laisser assimiler aux langues arianes, que plusieurs érudits crurent même pouvoir les dire de source différente. Toutefois, après des recherches plus minutieuses, plus scrupuleuses, on a fini par casser le premier arrêt, et d’importantes conversions ont décidément révisé le juge­ment. Il est aujourd’hui reconnu et établi que le breton, le gallois, Perse d’Irlande, le gaëlique d’Écosse, sont bien des rameaux de la grande souche ariane, et parents du sanscrit, du grec et du gothique \footnote{Pott, \emph{Encl. Ersch u. Gruber ; Indo-germanischer Sprachst.}, p. 87. ‑ M. Bopp pense que le celtique ne le cède à aucune langue européenne en abondance de mots provenant de la souche indo-germanique. (\emph{Ueber die keltischem Sprachen}, et \emph{Mémoires de l’Académie de Berlin}, 1838, p. 189.) Il ajoute encore que, pour la désignation des rapports grammaticaux, les dialectes celtiques n’ont pas inventé de formes neuves non indo-germaniques, ni rien emprunté, sous ce même rapport, des familles de langues étrangères au sanscrit. Tous leurs idiomes proviennent uniquement de mutilations et de pertes. (\emph{Ouvr. cité}, p. 195.)}. Mais combien ne faut-il pas que les idiomes celtiques soient défigurés pour avoir rendu cette démonstration si lente et si labo­rieuse ! Combien ne faut-il pas que d’éléments hétérogènes se soient mêlés à leur contexture pour leur avoir donné un extérieur si différent de celui de toutes les langues de leur famille ! Et, en effet, une invasion considérable de mots étrangers, des mutilations nombreuses et bizarres, voilà les éléments de leur originalité.\par
Tels sont les dégâts accomplis dans le sang, les croyances, les habitudes, l’idiome des Celtes, par la population esclave qu’ils avaient d’abord soumise, et qui ensuite, suivant l’usage, les pénétra de toutes parts et les fit participer à sa dégradation. Cette population n’était pas restée et ne pouvait rester longtemps reléguée dans son abjection, loin du lit de ses maîtres. Les Celtes, par des mariages contractés avec elle, firent de bonne heure éclore, de leur propre abaissement, des séries nouvelles de capacités, d’aptitudes, et par suite de faits, qui ont, à leur tour, servi et serviront de mobile et de ressort à toute l’histoire du monde. Les antagonismes et les mélanges de ces forces hybrides ont, suivant les temps, favorisé le progrès social et la décadence transitoire ou définitive. De même que dans la nature physique les plus grandes oppositions contribuent mutuellement à se faire ressortir, de même ici les qualités spéciales des alliages jaunes et blancs forment un repoussoir des plus énergiques à celles des produits blancs et noirs. Chez ces derniers, sous leur sceptre, au pied de leurs trônes magnifiques, tout embrase l’imagination, la splendeur des arts, les inspirations de la poésie s’y décuplent et couvrent leurs créateurs des rayons étince­lants d’une gloire sans pareille. Les égarements les plus insensés, les plus lâches faiblesses, les plus immondes atrocités, reçoivent de cette surexcitation perpétuelle de la tête et du cœur un ébranlement, un je ne sais quoi favorable au vertige. Mais, quand on se retourne vers la sphère du mélange blanc et jaune, l’imagination se calme soudain. Tout s’y passe sur un fond froid.\par
Là, on ne rencontre plus que des créatures raisonnables, ou, à ce défaut, raison­neuses. On n’aperçoit plus que rarement, et comme des accidents remarqués, de ces despotismes sans bornes qui, chez les Sémites, n’avaient pas même besoin de s’excuser par le génie. Les sens ni l’esprit n’y sont plus étonnés par aucune tendance au sublime. L’ambition humaine y est toujours insatiable, mais de petites choses. Ce qu’on y appelle jouir, être heureux, se réduit aux proportions les plus immédiatement matérielles. Le commerce, l’industrie, les moyens de s’enrichir afin d’augmenter un bien-être physique réglé sur les facultés probables de consommation, ce sont là les sérieuses affaires de la variété blanche et jaune. À différentes époques, l’état de guerre et l’abus de la force, qui en est la suite, ont pu troubler la marche régulière des transactions et mettre obstacle au tranquille développement du bonheur de ces races utilitaires. Jamais cette situation n’a été admise par la conscience générale, comme devant être définitive. Tous les instincts en étaient blessés, et les efforts pour en amener la modification ont duré jusqu’au succès.\par
Ainsi, profondément distinctes dans leur nature, les deux grandes variétés métisses ont été au-devant de destinées qui ne pouvaient pas l’être moins. Ce qui s’appelle durée de force active, intensité de puissance, réalité d’action, la victoire, le royaume, devait, nécessairement, rester un jour aux êtres qui, voyant d’une manière plus étroite, touchaient, par cela même, le positif et la réalité ; qui, ne voulant que des conquêtes possibles et se conduisant par un calcul terre à terre, mais exact, mais précis, mais approprié rigoureusement à l’objet, ne pouvaient manquer de le saisir, tandis que leurs adversaires nourrissaient principalement leur esprit de bouffées d’exagérations et de non-sens.\par
Si l’on consulte les moralistes pratiques les mieux écoutés par les deux catégories, on est frappé de l’éloignement de leurs points de vue. Pour les philosophes asiatiques, se soumettre au plus fort, ne pas contredire qui peut vous perdre, se contenter de rien pour braver en sécurité la mauvaise fortune, voilà la vraie sagesse.\par
L’homme vivra dans sa tête ou dans son cœur, touchera la terre comme une ombre, y passera sans attache, la quittera sans regret.\par
Les penseurs de l’Occident ne donnent pas de telles leçons à leurs disciples. Ils les engagent à savourer l’existence le mieux et le plus longtemps possible. La haine de la pauvreté est le premier article de leur foi. Le travail et l’activité en forment le second. Se défier des entraînements du cœur et de la tête en est la maxime dominante : jouir, le premier et le dernier mot.\par
Moyennant l’enseignement sémitique, on fait d’un beau pays un désert dont les sables, empiétant chaque jour sur la terre fertile, engloutissent avec le présent l’avenir. En suivant l’autre maxime, on couvre le sol de charrues et la mer de vaisseaux ; puis un jour, méprisant l’esprit avec ses jouissances impalpables, on tend à mettre le paradis ici-bas, et finalement à s’avilir.
\section[{V.4. Les peuplades italiotes aborigènes.}]{V.4. \\
Les peuplades italiotes aborigènes.}
\noindent Les chapitres qui précèdent ont montré que les éléments fondamentaux de la population européenne, le jaune et le blanc, se sont combinés de bonne heure d’une manière très complexe. S’il est resté possible d’indiquer les groupes dominants, de dénommer les Finnois, les Thraces, les Illyriens, les Ibères, les Rasènes, les Galls, les Slaves, il serait complètement illusoire de prétendre spécifier les nuances, retrouver les particularités, préciser la quotité des mélanges dans les nationalités fragmentaires. Tout ce qu’on est en droit de constater avec certitude, c’est que ces dernières étaient déjà fort nombreuses avant toute époque historique, et cette seule indication suffira pour établir combien il est naturel que leur état linguistique porte dans sa confusion la trace irrécusable de l’anarchie ethnique du sang d’où elles étaient issues. C’est là le motif qui défigure les dialectes des Galls, et rend l’euskara, l’illyrien, le peu que nous savons du thrace, l’étrusque, même les dialectes italiotes, si difficiles à classer.\par
Cette situation problématique des idiomes se prononce d’autant mieux que l’on considère des contrées plus méridionales en Europe.\par
Les populations immigrantes, se poussant de ce côté et y rencontrant bientôt la mer et l’impossibilité de fuir plus loin, sont revenues sur leurs pas, se sont renversées les unes sur les autres, se sont déchirées, enveloppées, enfin mélangées plus confusément que partout ailleurs, et leurs langues ont eu le même sort.\par
Nous avons déjà contemplé ce jeu dans la Grèce continentale. Mais l’Italie surtout était réservée à devenir la grande impasse du globe. L’Espagne n’en approcha pas. Il y eut, dans cette dernière contrée, des tourbillonnements de peuples, mais de peuples grands et entiers quant au nombre, tandis qu’en Italie ce furent surtout des bandes hétérogènes qui se montrèrent et accoururent de toutes parts. De l’Italie on passa en Espagne, mais pour coloniser quelques points épars. D’Espagne on vint en Italie en masses diverses, comme on y venait de la Gaule, de l’Helvétie, des contrées du Danube, de l’Illyrie, comme on y vint de la Grèce continentale ou insulaire. Par la largeur de l’isthme qui la tient attachée au continent aussi bien que par le dévelop­pement étendu de ses côtes de l’est et de l’ouest, l’Italie semblait convier toutes les nations européennes à se réfugier sur ses territoires d’un aspect si séduisant et d’un abord si facile. Il semble qu’aucune peuplade errante n’ait résisté à cet appel.\par
Quand furent achevés les temps donnés à la domination obscure des familles finnoises, les Rasènes se présentèrent, et, après eux, ces autres nations qui devaient former la première couche des métis blancs, maîtres du pays depuis les Alpes jusqu’au détroit de Messine.\par
Elles se séparaient en plusieurs groupes qui comptaient plus ou moins de tribus. Les tribus, comme les groupes, portaient des noms distinctifs, et parmi ces noms le premier qui se montre, c’est, absolument comme dans la Grèce primitive, celui des Pélasges \footnote{Mommsen, \emph{Die unter-italischen Dialekte}, p. 206.}. À leur suite, les chroniqueurs amènent bientôt d’autres Pélasges sortis de l’Hellade, de sorte qu’aucun lieu ne saurait être mieux choisi et aucune occasion plus convenable pour examiner à fond ces multitudes qui, aux yeux des Grecs et des Romains, représentaient les sociétés primitivement cultivées, voyageuses et conqué­rantes de leur histoire.\par
La dénomination de \emph{Pélasge} n’a pas de sens ethnique. Elle ne suppose pas une nécessaire identité d’origine entre les masses auxquelles on l’attribue \footnote{Voir plus haut.}. Il se peut que cette identité ait existé  ; c’est même, dans certains cas, l’opinion plausible, mais assurément l’ensemble des Pélasges y échappe, et, par conséquent, le mot, en tant qu’indiquant une nationalité spéciale, est absolument sans valeur \footnote{Hérodote, parlant des Pélasges de Dodone, remarque qu’ils considéraient les dieux comme de simples régulateurs anonymes de l’univers, et nullement comme en étant les créateurs. C’est le naturalisme arian. Ces Pélasges semblent donc avoir été des Illyriens Arians, ce que n’étaient pas d’autres Pélasges. (Hérod., II, 52.)}.\par
Sous un certain point de vue cependant, il acquiert un mérite relatif. Tout ainsi que son synonyme\emph{ aborigène}, il n’a jamais été appliqué, par les \emph{annalistes anciens}, qu’à des populations blanches ou à demi blanches, de la Grèce ou de l’Italie, que l’on supposait primitives \footnote{Abeken, \emph{Mittel-italien vor der Zeit der rœmischen Herrschaft}, p. 18 et 125: « Si nous « considérons cette race \emph{grecque primitive} que l’Italie se partage avec l’Hellade, il est à « remarquer qu’on la reconnaît sur les deux points, non seulement aux bases des deux « langues, qui sont identiques, mais encore dans les plus anciens restes d’architecture. » – Voir encore \emph{même ouvrage}, p. 82. – O, Muller, \emph{die Etrusker}, p. 27 et 56\emph{. –} Mommsen, \emph{Die unter-italischen Dialekte}, p. 363\emph{. –} Strabon, V, 2, 4.}. Il est donc pourvu, au moins, d’une signification géographique, ce qui n’est pas dénué d’utilité pour élaborer l’éclaircissement de la question de race. Mais là s’arrêtent les services qu’il faut en attendre. Si ce n’est pas beaucoup, encore est-ce quelque chose.\par
En Grèce, les populations pélasgiques jouent le rôle d’opprimées, d’abord devant les colonisateurs sémites, ensuite devant les émigrants arians-hellènes. Il ne faut pas surfaire le malheur de ces victimes : la sujétion qu’on leur imposait avait des bornes \footnote{Voir plus haut.}. Dans son étendue la plus grande, elle s’arrêtait au servage. L’aborigène vaincu et soumis devenait le manant du pays. Il cultivait la terre pour ses conquérants, il travaillait à leur profit. Mais, ainsi que le comporte cette situation, il restait maître d’une partie de son travail et conservait suffisamment d’individualité \footnote{Voir plus haut.}. Toute subor­donnée qu’elle était, cette attitude valait mieux, à mille égards, que l’anéantissement civil auquel étaient réduites partout les peuplades jaunes. Puis, les Pélasges de la Grèce n’avaient pas été indistinctement asservis. Nous avons vu que la plupart des Sémites, puis des Arians Hellènes s’établirent sur l’emplacement des vinages aborigènes, en conservèrent souvent les noms anciens, et s’allièrent avec les vaincus de manière à produite bientôt un nouveau peuple. Ainsi les Pélasges ne furent pas traités en sauvages. On les subordonna sans les annihiler. On leur accorda un rang conforme à la somme et au genre de connaissances et de richesses qu’ils apportaient dans la communauté.\par
Cette dot était certainement d’une nature grossière : les aptitudes et les produits agricoles en faisaient le fond. Le poète de ces aborigènes, qui est Hésiode, non pas comme issu de leur race, mais parce qu’il a surtout envisagé et célébré leurs travaux, nous les montre fort attachés aux emplois rustiques. Ces pasteurs sont également habiles à élever de grands murs, à bâtir des chambres funéraires, à amonceler des tumulus de terre d’une imposante étendue \footnote{On ne doit pas oublier que ces constructions, formées de blocs entassés et encastrés l’un sur l’autre, d’après leurs formes naturelles, n’ont rien de commun avec les édifices arians-helléniques, où les pierres sont taillées d’une façon régulière.}. Or, toutes ces œuvres, nous les avons déjà observées dans les pays celtiques. Nous les reconnaissons pour semblables, quant aux traits généraux, à celles qui ont couvert le sol de la France et de l’Allemagne, sous l’action des premiers métis blancs.\par
Les auteurs grecs ont analysé les idées religieuses des aborigènes. Ils ont dit leur respect pour le chêne \footnote{Bœttiger, \emph{Ideen zur Kunstmythologie}, t. I, p. 203. Cette adoration se perpétua longtemps parmi les populations agricoles de l’Arcadie.  – « Habitæ Graiis oracula quercus. » (Georg., II, 16.)}, l’arbre druidique. Ils les ont montrés croyant aux vertus prophétiques de ce patriarche des bois, et cherchant dans la solitude des vertes forêts la présence de la Divinité. Ce sont là des habitudes, des notions toutes galliques. Ces mêmes Pélasges avaient encore l’usage d’écouter les oracles de femmes consacrées, de prophétesses semblables aux Alrunes, qui exerçaient sur leurs esprits une domination absolue \footnote{Bœttiger, \emph{loc. cit.}}. Ces devineresses furent les mères des sibylles, et, dans un rang moins élevé, elles eurent aussi pour postérité les magiciennes de la Thessalie \footnote{Parmi d’autres traces de la présence des Celtes dans la population primitive de la Grèce, on peut encore relever le nom tout à fait significatif du pays de \emph{Calydon}, (mot grec), et des \emph{Calydoniens}, (mot grec), qui l’habitent. Le mythe entier de Méléagre semble également faire partie de la tradition aborigène.}.\par
On ne doit pas non plus oublier que le théâtre des superstitions les moins confor­mes à la nature de l’esprit asiatique resta toujours fixé au sein des contrées septentrio­nales de la Grèce. Les ogres, les lémures, l’entrée du Tartare, toute cette fantasmagogie sinistre s’enferma dans l’Épire et la Chaonie, provinces où le sang sémitisé ne pénétra que très tard, et où les aborigènes maintinrent le plus longtemps leur pureté.\par
Mais, si ces derniers semblent, pour toutes ces causes, devoir être comptés au rang des nations celtiques, il y a des motifs d’admettre des exceptions pour d’autres tribus.\par
Hérodote a raconté que plusieurs langages étaient parlés, à une époque anté-hellénique, entre le cap Malée et l’Olympe \footnote{Voir plus haut.}. Le texte de l’historien, peu précis en cette occasion, se prête sans doute à des ambiguïtés. Il peut avoir voulu dire qu’il existait sur cet espace des dialectes chananéens et des dialectes kymriques. Toutefois une telle explication, n’étant qu’hypothétique, ne s’impose pas inévitablement, et on est autorisé à la prendre encore dans un autre sens non moins vraisemblable.\par
Les usages religieux de la Grèce primitive offrent plusieurs particularités absolu­ment étrangères aux habitudes kymriques, par exemple, celle qui existait à Pergame, à Samos, à Olympie, de construire des autels avec la cendre des victimes mêlée de monceaux d’ossements incinérés. Ces monuments dépassaient quelquefois une hauteur de cent pieds \footnote{Pausanias, in-8°, Lips., 1823, t. II, chap. XIII  – « Olympii quidem Jovis ara pari intervallo a « Pelopis et Junonis æde distat... Congesta illa est e cinere collecta ex adustis victimarum « femoribus. Talis et Pergami ara est, talis Samiæ Junonis, nihilo illa quidem ornatior quam « in Attica quos \emph{Rudes} appellant focos. Aræ olympicæ una crepido... ambitum peragit « centum et amplius quinque et viginti. »}. Ni en Asie, chez les Sémites, ni en Europe, chez les Celtes, nous n’avons rencontré trace d’une pareille coutume. En revanche, nous la trouvons chez les nations slaves. Là, il n’est pas une ruine de temple qui ne nous montre son tas de cendres consacré, et souvent même ce tas de cendres, entouré d’un mur et d’un fossé, forme tout le sanctuaire \footnote{Keferstein, \emph{ouvr. cité}, t. I, p. 236 et pass.}. Il devient ainsi très probable que parmi les aborigènes kymriques il se mêlait aussi des Slaves. Ces deux peuples, si fréquemment unis l’un à l’autre, avaient ainsi succédé aux Finnois, jadis parvenus en plus ou moins grand nombre sur ce point du continent, et s’étaient alliés à eux dans des mesures différentes \footnote{Les collines de sacrifices, de création slave se trouvent avec abondance jusqu’en Servie. M. Troyon pense qu’il faut en faire remonter l’époque au V\textsuperscript{e} et VI\textsuperscript{e} siècle de notre ère seulement. En tout cas, c’est un mode de construction fort antique et tout à fait semblable aux autels d’Olympie et de Samos.}.\par
Je ne trouve plus dès lors impossible que, dans les grandes révolutions amenées par la présence des colons sémites et des conquérants arians-titans, puis arians-hellènes, des fugitifs aborigènes de race slave aient pu passer en Asie à différentes époques, et y porter dans la Paphlagonie le nom wende des \emph{Enètes} ou \emph{Henètes} \footnote{Schaffarik, \emph{Slawische Alterthümer}, t. I, p. 159.  – Tite-Live contient ce passage digne de remarque : « Casibus deinde variis Antenorem, cum multitudine Henetum, qui seditione ex Paphlagonia pulsi, et sedes et ducem, rege Pylæmene ad Trojam amisso, quærebant. » – Liv. Gron., in-8°, Basileæ, 1740, t. I, p. 8.}. Ces malheureux Pélasges, Slaves, Celtes, Illyriens ou autres, mais toujours métis blancs, attaqués par des forces trop considérables, et souvent assez forts cependant pour ne pas accepter un esclavage absolu, émigraient de tous côtés, se faisaient à leur tour pillards, ou, si l’on veut, conquérants, et devenaient l’effroi des pays où ils portaient leur belliqueuse misère.\par
La terre italique était déjà peuplée de leurs pareils, appelés, comme eux, \emph{Pélasges} ou\emph{ aborigènes}, reconnus de même pour être les auteurs de grandes constructions massives en pierres brutes ou imparfaitement taillées, voués également aux travaux agricoles, ayant des prophétesses ou des sibylles toutes pareilles, enfin leur ressem­blant de tout point, et conséquemment identifiés de plein droit avec eux.\par
Ces aborigènes italiotes paraissent avoir appartenu le plus généralement à la famille celtique. Néanmoins ils n’étaient pas seuls, non plus que ceux de la Grèce, à occuper leurs provinces. Outre les Rasènes, dont le caractère slave a déjà été reconnu, on y aperçoit encore d’autres groupes de provenance wende, tels que les Vénètes \footnote{Hérodote les confond avec les Illyriens. Leur territoire s’étendait, au sud, jusqu’à l’embouchure de l’Etsch, et, à l’ouest jusqu’aux hauteurs qui vont de cette rivière au Bacciglione. (O. Muller, \emph{die Etrusker}, p. 134.)}. Il n’y a pas non plus de motifs pour refuser à Festus l’origine illyrienne des Peligni \footnote{Abeken, \emph{ouvr. cité} p 85.  – Cependant Ovide range cette nation parmi les tribus sabines. Les deux opinions peuvent se soutenir, et les Peligni n’être, comme la plupart des nations italiotes, que le résultat de nombreux mélanges où des émigrants illyriens, probablement Liburnes, auront eu leur place. Pour montrer combien les travaux auxquels donne lieu l’ethnographie d’un peuple sont épineux, et doivent tendre plutôt d’abord, à concilier qu’à rejeter les traditions, même les plus disparates, il n’y a qu’à étudier ce que Tacite dit des Juifs, lorsque, au livre V, ch. II des \emph{Histoires}, il recherche leur origine. Il énumère quatre opinions : la première les fait venir de Crète, et dérive le nom de Judaei du mont Ida. Ceux qui lui avaient donné cet avis confondaient tous les habitants en une seule race, et leur sentiment, juste par rapport aux Philistins, se trouvait inexact en ce qui avait trait aux Abrahamides. La seconde opinion les faisait venir d’Égypte, et les accusait de descendre des lépreux expulsés de ce pays qu’ils infectaient de leur mal. En laissant de côté le trait de haine nationale, il n’y a rien que de vrai dans cette assertion. Cependant elle ne détruit pas la valeur de la troisième, qui fait des Juifs une colonie d’Éthiopiens. Seulement Tacite paraît entendre, par ce mot, des Abyssins, et nous savons (voir t. I) que, dans la plus haute antiquité, il s’appliquait aux hommes de l’Assyrie. Cette vérité contribue à faire agréer du même coup la quatrième opinion citée par l’historien romain, et qui disait les juifs Assyriens d’origine. Ils l’étaient, sans doute, en tant que Chaldéens. Je n’ai voulu ici que donner un exemple de l’attention soutenue et scrupuleuse, de la réserve prudente qui doit diriger les élucidations et surtout les conclusions ethnologiques.}. Les japyges, venus vers l’an 1186 avant notre ère, et établis dans le sud-est du royaume de Naples, semblent avoir appartenu à la même famille. De son côté, M. W. de Humboldt a donné aussi de trop bonnes raisons pour qu’on puisse nier, après lui, que des populations ibériennes aient vécu et exercé une assez notable influence sur le sol de la Péninsule \footnote{Voir \emph{Prüfung der Untersuchungen über die Urbewohner Hispaniens}, p. 49.  – M. W. de Humboldt fait dériver le mot latin \emph{murus} de l’euskara \emph{murua}. (\emph{Ibid.}, p. 3 et pass.)}. Quant aux Troyens d’Énée, la question est plus difficile. Il semble plus que probable que l’ambition de se rattacher à cette souche épique ne vint aux Romains qu’à la suite de leurs rapports avec la colonie grecque de Cumes, qui leur en fit sentir la beauté.\par
Voilà, dès le début, une assez grande variété d’éléments ethniques. Mais, de tous le plus répandu, c’était incontestablement celui des Kymris ou des aborigènes, reconnus par les ethnographes, comme Caton, pour avoir appartenu à une seule et même race.\par
Ces aborigènes, lorsque les Grecs voulurent leur imposer un nom spécial et géographique, furent qualifiés d’abord d’Ausoniens \footnote{O. Muller, \emph{die Etrusker}, p. 27.}.\par
Ils étaient composés de différentes nations, telles que les Œnotriens, les Osques, les Latins, toutes subdivisées en fractions d’inégale puissance. C’est ainsi que le nom des Osques ralliait les Samnites, les Lucaniens, les Apuliens, les Calabrais, les Campaniens \footnote{\emph{Ouvr. cité}, p. 40.}.\par
Mais, comme les Grecs n’avaient noué leurs premiers rapports qu’avec l’Italie méridionale, le terme d’\emph{Ausonien} ne désignait que l’ensemble des masses trouvées dans cette partie du pays, et le sens ne s’en étendait pas aux habitants de la contrée moyenne.\par
L’appellation qui échut à ces derniers fut celle de \emph{Sabelliens} \footnote{Mommsen, \emph{Unter-ital. Dialekte}, p. 363.}. Au delà, vers le nord, on connut encore les Latins, puis les Rasènes et les Umbres \footnote{\emph{Ibidem.} Dont les trois subdivisions principales sont essentiellement celtiques, quant au nom : les Olombri, de ol, hauteur, habitaient les Alpes ; les Isombri, de is, bas, les plaines de la vallée du Pô ; les Vilombri, de bel, le rivage, l’Ombrie actuelle, sur l’Adriatique.}.\par
Cette classification, tout arbitraire qu’elle est, a pour premier et assez grand avantage de restreindre considérablement l’application du titre vague d’aborigène. En toutes circonstances, on croit connaître ce qu’on a dénommé. On mit donc à part les peuples déjà classés, Ausoniens, Sabelliens, Rasènes, Latins et Umbres, et on fit une catégorie spéciale de ceux qui ne restèrent aborigènes que parce qu’on n’avait pas eu de contact assez intime avec eux pour leur attribuer un nom. De ce nombre furent les Æques, les Volsques et quelques tribus de Sabins \footnote{Mommsen, \emph{ouvr. cité}, p. 324.}.\par
Les inconvénients du système étaient flagrants. Les Samnites, rangés parmi les Osques, et les Osques eux-mêmes, avec toutes celles de leurs peuplades citées plus haut, et ensuite les Mamertins et d’autres, n’étaient pas étrangers aux Sabelliens. Ces groupes tenaient à la souche sabine. Par conséquent, ils avaient des affinités certaines avec les gens de l’Italie moyenne, et tous, ce qui est significatif, avaient émigré, de proche en proche, de la partie septentrionale des montagnes Apennines \footnote{O. Muller, \emph{die Etrusker}, p. 45 et pass.}. Ainsi, en laissant à part les Rasènes et en remontant du sud au nord de la Péninsule, on arrivait, de parentés en parentés, à la frontière des Umbres, sans avoir remarqué une solution de continuité dans la partie dominante de cet enchaînement.\par
On a dit longtemps que les Umbres ne dataient, dans la Péninsule, que de l’invasion de Bellovèse, et qu’ils avaient remplacé une population qui ne portait pas le même nom qu’eux. Cette opinion est aujourd’hui abandonnée \footnote{O. Muller,\emph{ ouvr. cité}, p. 58.}. Les Umbres occupaient la vallée du Pô et le revers méridional des Alpes bien antérieurement à l’irruption des Kymris de la Gaule. Ils se rattachaient par leur race aux nations qui ont continué à être nommées aborigènes ou pélasgiques, tout comme les Osques et les Sabelliens \footnote{O. Muller,\emph{ ouvr. cité}, p. 56.  – Abeken, p. 82.  – Mommsen, p. 206.}, et même on les reconnaissait pour la souche d’où les Sabins étaient dérivés, et, avec ces derniers, les Osques.\par
Les Umbres donc, étant la racine même des Sabins, c’est-à-dire des Osques, c’est-à-dire encore des Ausoniens, et se trouvant ainsi germains des Sabelliens \footnote{Suivant Mommsen, les alphabets découverts dans la Provence, le Valais, le Tyrol, la Styrie, sont plus parents de l’alphabet sabellien que de tous les autres de l’Italie, c’est-à-dire que de ceux de l’Étrurie proprement dite et de la Campanie, et plus rapprochés du type grec archaïque. Cependant il établit, entre tous ces systèmes d’écriture, un caractère commun. (Mommsen, \emph{Die nord-etruskichen Alpbabete}, p. 222.) Il est utile de se reporter ici à ce qui a été dit plus haut des alphabets celtiques en général. Dans un sujet si difficile et si compliqué, les plus petits faits se portent mutuellement secours pour s’élever au rang de preuves, et il est indispensable de pouvoir compter sur l’attention soutenue du lecteur.} et de toutes les populations appelées du nom peu compromettant d’aborigènes, on serait, par cela seul, autorisé à affirmer que la masse entière de ces aborigènes, descendus du nord vers le sud, était de race umbrique, toujours à l’exception des Étrusques, des Ibères, des Vénètes et de quelques Illyriens. Ayant répandu sur la Péninsule les mêmes modes et le même style d’architecture, se réglant sur la même doctrine religieuse, montrant les mêmes mœurs agricoles, pastorales et guerrières, cette identification semblerait assez solidement justifiée pour ne devoir pas être révoquée en doute \footnote{Voir les autorités dénombrées par Dieffenbach, \emph{Celtica Il}, l \textsuperscript{re} Abth., p. 112 et sqq.}. Ce n’est pas assez cependant : l’examen des idiomes italiotes, autant qu’on le peut faire, enlève encore à la négative sa dernière ressource.\par
Mommsen pose en fait que la langue des aborigènes offre un mode de structure antérieur au grec, et il réunit dans un même groupe les idiomes umbriques, sabelliens et samnites, qu’il distingue de l’étrusque, du gaulois et du latin. Mais il ajoute ailleurs qu’entre ces six familles spéciales il existait de nombreux dialectes qui, se pénétrant les uns les autres, formaient autant de liens, établissaient la fusion et réunissaient l’ensemble \footnote{Mommsen, \emph{ouvr. cité}, p. 364.}.\par
En vertu de ce principe, il corrige son assertion séparatiste, et affirme que les Osques parlaient une langue très parente du latin \footnote{\emph{Ibidem}, p. 205.  –\emph{ Opici} ou \emph{Opsci.} Leur langue était encore en usage à Rome dans certaines pièces de théâtre, soixante ans après le début de l’ère chrétienne. (Strabon, V, 3, 6.) On trouve à Pompéi des inscriptions osques, et, comme l’ensevelissement de la ville ne date que de l’an 79 après J.-C., on peut comprendre, par cela seul, qu’elle fut la longévité de cet idiome. Peut-être y aurait-il grand profit à appliquer les dialectes populaires actuels de l’Italie au déchiffrement des inscriptions locales. On arriverait plus sûrement à un résultat qu’en se servant du latin, qui, en définitive, fut seulement la langue franque ou malaye, l’hindoustani de la Péninsule.}.\par
O. Muller remarque, dans cette langue composite, des rapports frappants avec l’umbrique, et le savant archéologue danois dont je viens d’invoquer le jugement donne leur véritable sens et toute leur portée à ces rapports, en affirmant que 1’umbrique est, de toutes les langues italiotes, celle qui est restée le plus près des sources aborigènes \footnote{Mommsen, \emph{ouvr. cité}, p. 206.  – C’est pourquoi il ajoute aussi que le Volsque avait de plus grands rapports avec 1’umbrique que l’osque (p. 322.)}. En d’autres termes, l’osque, comme le latin, tel que nous l’offrent la plupart des monuments, est d’un temps où les mélanges ethniques avaient exercé une grande influence et développé des corruptions considérables, tandis que, les circonstances géographiques ayant permis à 1’umbrique de recevoir moins d’éléments grecs et étrusques, ce dernier langage s’était tenu plus près de son origine et avait mieux conservé sa pureté. Il mérite, en conséquence, d’être pris comme prototype, lorsqu’il s’agit de juger dans leur essence les dialectes italiotes.\par
Nous avons donc bien conquis ce point capital : les populations aborigènes de l’Italie, sauf les exceptions admises, se rattachent fondamentalement aux Umbres  ; et quant aux Umbres, ce sont, ainsi que leur nom l’indique, des émissions de la souche kymrique, peut-être modifiées d’une manière locale par la mesure de l’infusion finnique reçue dans leur sein.\par
Il est difficile de demander à 1’umbrique même une confirmation de ce fait. Ce qui en reste est trop peu de chose, et jusqu’ici, ce qu’on en a déchiffré offre sans doute des racines appartenant au groupe des idiomes de la race blanche, mais défigurées par une influence qui n’a pas encore été déterminée dans ses véritables caractères. Adressons-nous donc d’abord aux noms de lieux, puis à la seule langue italiote qui nous soit pleinement accessible, c’est le latin.\par
Pour ce qui est des noms de lieux, l’étymologie du mot Italie est naturellement offerte par le celtique talamb, tellus, la terre par excellence, \emph{Saturnia tellus,Œnotria tellus} \footnote{Dieffenbach, \emph{Celtica II}, 1 \textsuperscript{re} Abth,, p. 114.}.\par
Deux peuplades umbriques, les Euganéens et les Taurisques, portent des noms purement celtiques \footnote{\emph{Euganéens, d’aguen, eau}  ; c’étaient les riverains des lacs de Lugano, Como et Garda. Les Taurisques, comme les Taurini, tirent leur nom de tor, montagne. Niebuhr, pour établir un lien intime entre les Rhétiens et les Rasènes, incline à faire des Euganéens des Étrusques. Mais il n’exprime cette idée que timidement et comme entraîné par le besoin de sa cause. (\emph{Rœmische Geschichte}, t. I, p. 70.)}. Les deux grandes chaînes de montagne qui partagent et bornent le sol italien, les Apennins et les Alpes, ont des dénominations empruntées à la même langue \footnote{\emph{A pen gwin, la crête, la montagne blanche}.}. Les villes d’Alba, si nombreuses dans la Péninsule et toujours de fondation aborigène, puisent l’étymologie de leur nom dans le celtique \footnote{\emph{Alb} ou \emph{Alp}, l’élévation, la montagne, la colline  ; \emph{Albany}, la contrée montagneuse de l’Écosse ; l’\emph{Albanie}, les montagnes de l’Illyrie ; \emph{Albania}, une partie du Caucase ; \emph{Albion, l’île aux grandes falaises}, et les nombreuses villes d’\emph{Alba}, placées sur des éminences. On connaissait aussi, dans la Narbonnaise, les Ligures \emph{albienses} et les \emph{Albæci}, peuples demi-celtiques. \emph{Alb} signifie également\emph{ blanc} et donne la racine d’\emph{albu}s.  – Consulter Dieffenbach, \emph{Celtica I}, p. 18, 13, et \emph{Celtica II}, l \textsuperscript{re} \emph{Abth}., p. 310, 6.}. Les faits de ce genre sont abondants. Je me borne à en indiquer la trace, et je passe de préférence à l’examen de quelques racines kymro-latines.\par
On remarque, en premier lieu, qu’elles appartiennent à cette catégorie d’expressions formant l’essence même du vocabulaire de tous les peuples, d’expressions qui, tenant au fond des habitudes d’une race, ne se laissent pas aisément expulser par des influen­ces passagères. Ce sont des noms de plantes, d’arbres, d’armes, Je ne m’étonnerais, dans aucun cas, de voir les dialectes celtiques et ceux des aborigènes de l’Italie posséder des racines semblables pour tous ces emplois, puisque, même en mettant à part la question actuelle, il faudrait toujours reconnaître qu’issus également de la souche blanche, ils ont assis leurs développements postérieurs sur une base unique. Mais, si les mêmes mots se présentent avec les mêmes formes, à peine altérées dans le celtique et dans l’italiote, il devient bien difficile de ne pas confesser l’évidence de l’identité d’origine secondaire.\par
Voyons d’abord le vocable employé pour désigner le \emph{chêne.} C’est un sujet digne d’attention. Chez les Celtes de l’Europe septentrionale, chez les aborigènes de la Grèce et de l’Italie, cet arbre jouait un grand rôle, et, par l’importance religieuse qui lui était attribuée, il tenait de près aux idées les plus intimes de ces trois groupes.\par
Le mot breton est \emph{cheingen}, qui, au moyen de la permutation locale de l’\emph{n} en \emph{r}, devient \emph{chergen}, d’où il y a peu de chemin jusqu’au latin \emph{quercus.}\par
Le mot \emph{guerre} fournit un rapport non moins frappant. La forme française reproduit presque pur le celtique, \emph{queir.} Le sabin \emph{queir} le garde tout entier. Mais, outre que ce mot, en celtique, a le sens que je viens d’indiquer, il a aussi celui de \emph{lance.} En sabin, il en est encore de même, et de là le nom et l’image du dieu héroïque \emph{Quirinus}, adoré sous l’aspect d’une lance chez les premiers Romains, vénéré encore chez les Falisques, qui avaient leur \emph{Pater curis}, et divinisé à Tibur, où la Junon Pronuba portait l’épithète de \emph{Curitis} ou \emph{Quiritis} \footnote{Bœttiger, \emph{Ideen zur Kunst-Mythologie}, t. I, p. 20 ; t. II, p. 227 et pass.}.\par
\emph{Arm} en breton, \emph{airm} en gaëlique, équivaut à l’\emph{arma} latin.\par
Le gallois pill est le latin \emph{pilum}, le \emph{trait} \footnote{Et le sanscrit \emph{pilu}.  – A. V. Schlegel, \emph{Indische Bibliothek}, t. I, p. 209.)  – D’ailleurs, MM. Aufrecht et Kirchhof, \emph{Die umbrischen Sprachdenkmæler}, établissent très bien le rapport de l’umbrique avec le sanscrit et les langues de la race blanche. Voir, \emph{Lautlehre}, p. 15 et pass. – Abeken exprime la même opinion : « Quant à la langue (umbrique), dit-il, elle est aussi « incompréhensible aujourd’hui que l’étrusque ; bien qu’en somme on y démêle beaucoup « mieux une souche \emph{grecque primitive} (on n’oublie pas que pour Abeken ce mot composé « est synonyme de \emph{pélasgique}). L’umbrique semble être une langue sœur de l’osque et du « latin. » (\emph{Ouvr. cité}, p. 28.)}.\par
Le bouclier, \emph{scutum}, apparaît dans le \emph{sgiath} gaëlique   \emph{gladius}, le \emph{glaive}, dans le \emph{cleddyf} gallois et le \emph{cledd} gaëlique ; l’arc, \emph{arcus}, dans \emph{l’archelte} breton ; la \emph{flèche, sagitta}, dans le \emph{saeth} gallois, le \emph{saighead} gaëlique ; le \emph{char, currus}, dans le \emph{car} gaëlique et le \emph{carr} breton et gallois.\par
Si je passe aux termes d’agriculture et de vie domestique, je trouve la maison, \emph{casa}, et l’erse \emph{cas} ;\emph{ ædes} et le gaëlique \emph{aite} ;\emph{ cella} et le gallois \emph{cell} ;\emph{ sedes} et le \emph{sedd} du même dialecte. Je trouve le \emph{bétail, pecus} ; et le gaëlique \emph{beo} ; car le bétail par excellence, ce sont les bêtes bovines. Je trouve le vieux latin \emph{bus}, le bœuf, et \emph{bo}, gaëlique, ou \emph{buh}, breton ; le \emph{bélier, aries}, et \emph{reithe}, gaëlique ; la \emph{brebis, ovis}, et le breton \emph{ovein}, avec le gallois \emph{oen} ; le \emph{cheval equus}, et le gallois\emph{ echw} ; la \emph{laine, lana}, et le gaëlique \emph{olann}, et le gallois \emph{gwlan} ;\emph{ l’eau, aqua}, et le breton \emph{aguen}, et le gallois \emph{aw} ; le \emph{lait, lactum}, et le gaëlique \emph{lachd} ; le \emph{chien, canis}, et le gallois \emph{can} ; le \emph{poisson, piscis}, et le gallois \emph{pysg} ; l\emph{’huître, ostrea}, et le breton \emph{oistr} ; la \emph{chair, caro}, et le gaëlique \emph{carn}, qui présente l’\emph{n} des flexions de \emph{caro} ; le verbe \emph{immoler, mactare}, et le gaëlique \emph{mactadh} ;\emph{ mouiller, madere}, et le gallois \emph{madrogi.}\par
Le verbe \emph{labourer, arare}, et le gaëlique \emph{ra} avec les deux formes galloises \emph{aru} et \emph{aredig} ; le champ, \emph{arvum}, avec le gaëlique \emph{ar} et le gallois \emph{arw} ; le \emph{blé, hordeum}, et le gaëlique \emph{eorma} ; la \emph{moisson, seges}, et le breton \emph{segall} ; la \emph{fève, f ba}, et le gallois \emph{ffa} ; la \emph{vigne, vitis}, et le gallois \emph{gwydd} ;\emph{ l’avoine, avena}, et le breton \emph{havre} ; le \emph{fromage, caseus}, et le gallique \emph{caise}, avec le breton \emph{casu} ;\emph{ butyrum}, le \emph{beurre}, et le gaëlique \emph{butar}  ; la \emph{chandelle, candela}, et le breton \emph{cantol} ; le \emph{hêtre, fagus}, et l’erse \emph{feagha}, avec le breton \emph{fao} et \emph{faouenn} ; la \emph{vipère, vipera}, et le gallois \emph{gwiper} ; le \emph{serpent, serpens}, et le gallois \emph{sarff} ; la \emph{noix, nux}, et le gaëlique \emph{cnu}, exemple notable de ces renversements de sons fréquemment subis par les monosyllabes, dans le passage d’un dialecte à un autre.\par
Puis j’énumère pêle-mêle des mots comme ceux-ci : la \emph{mer, mare}, gaëlique \emph{muir}, breton et gallois \emph{mor} ; se \emph{servir, uti}, gaëlique \emph{usinnich} ;\emph{ l’homme, vir}, gallois \emph{gwir} ;\emph{ l’année, annus}, gaëlique \emph{ann} ; la \emph{vertu}, gaëlique \emph{feart}, qui se confond bien avec le mot \emph{fortis, courageux} \footnote{Ce mot \emph{feart} se rapproche aussi du grec (mot grec) et de la racine typique \emph{ar.} (Voir tome I \textsuperscript{er}.)} le \emph{fleuve, amnis}, gaëlique \emph{amba, amhuin} ;\emph{ revenir, redire}, gallois \emph{rhetu} ; le \emph{roi, rex}, gaëlique \emph{righ} ;\emph{ mensis}, le \emph{mois}, gallois \emph{mis} ; la \emph{mort, murn}, gallois, et \emph{mourir, mori}, breton \emph{marheuein.} Je terminerai par \emph{penates}, qui n’a pas d’étymologie ailleurs qu’en celtique \footnote{Rien ne le saurait mieux prouver que la lecture du passage où Denys d’Halicarnasse à trouver à cette dénomination ethnologique un sens qui lui échappe, malgré tous ses efforts, ainsi qu’à ses commentateurs. (C. XLVII.)} : ce mot ne se dérive d’une manière simple et complètement satisfaisante que du gallois \emph{penaf}, qui veut dire \emph{élevé}, et qui a pour superlatif \emph{penaeth, très élevé}, le plus \emph{élevé} \footnote{J’aurais pu de même et, peut-être dû donner une liste semblable pour les Kymris Grecs, et montrer le grand nombre de mots celtiques demeurés dans les dialectes de l’Hellade ; mais ce soin me paraît superflu. je me borne à renvoyer le lecteur au vocabulaire de M. Keferstein (\emph{Ansichten}, etc., t. II, p. 3)  ; il ne contient pas moins de soixante pages, et, bien que plusieurs mots gréco-gallois ou gréco-bretons y soient évidemment d’importation très moderne, le fond est décisif et présente un tableau plus curieux encore, s’il est possible, que ce qui résulte de la comparaison que je fais ici.}.\par
On pourrait étendre ces exemples bien loin. Les trois cents mots allégués par le cardinal Maï, au tome V de sa collection des classiques édités sur les manuscrits du Vatican, seraient dépassés. Cependant c’en est assez, j’en ai la confiance, pour fixer toute indécision \footnote{ \noindent Je ne saurais cependant passer sous silence les noms de nombre :\par
 latins, celtiques : \textbf{1.} unus,un, aon. \textbf{2.} duo,dau. \textbf{3.} tres,tri. \textbf{4.} quatuor,ceither. \textbf{5.} quinque,cinq. \textbf{6.} sex,chuech. \textbf{7.} septem,saith. \textbf{8.} octo,ochd. \textbf{9.} novem,naw. \textbf{10.} decem,deich. Enfin, je ne ferai plus qu’une dernière observation : des liens généraux paraissent avoir uni assez étroitement les langues primitives de toute l’Europe occidentale, quelque différents que se présentent, aujourd’hui, l’un de l’autre l’ibère, l’étrusque les dialectes italiotes et les kymriques. On a vu que des règles analogues s’appliquent, dans toutes ces langues, à la permutation des consonnes. Il faut ajouter qu’elles pratiquaient, avec une égale facilité, le renversement des syllabes, si familier au latin et qu’on retrouve dans la manière d’écrire indifféremment \emph{Pratica ou Patrica}, nom d’une ville aborigène, \emph{Lanuvium ou Lavinium, Agendicum ou Agedincum.} Les dialectes slaves ne sont pas moins aptes que les celtiques à cette évolution.
}. On peut choisir des verbes tout aussi bien que des substantifs : les résultats de l’examen seront les mêmes, et lorsqu’on découvre des rapports aussi frappants, aussi intimes entre deux langues, que d’ailleurs les formes de l’oraison sont, de leur côté, parfaitement identiques, le procès est jugé : les Latins, descendants, en partie, des Umbres, étaient bien, comme leur nom l’indique, apparentés de près aux Galls, ainsi que leurs ancêtres, et, partant, les aborigènes de l’Italie, non moins que ceux de la Grèce, appartenaient, pour une forte part, à ce groupe de nations.\par
C’est ainsi, et seulement ainsi, que s’explique cette sorte de teinte uniforme, cette couleur terne qui couvre également, aux âges héroïques, tout ce que nous savons et pénétrons des faits et des actes de la masse appelée pélasgique, comme de celle qui porte son vrai nom de kymrique. On y observe une pareille allure grossière et soldatesque, une pareille façon de laboureur et de pasteur de bœufs. Quoi ! c’est une pareille manière de s’orner et de se parer. Nous ne retrouvons pas moins de bracelets et d’anneaux dans le costume des Sabins de la Rome primitive que dans celui des Arvernes et des Boïens de Vercingetorix \footnote{Liv., I, 129 : « Vulgo Sabini aureas armillas magni ponderis brachio lævo gemma tosque « magna specie annulos habuerint ».}. Chez les deux peuples, le brave se montre à nous sous le même aspect physique et moral, bataillant et travaillant, austère et sans rien de pompeux \footnote{Niebuhr signale chez les aborigènes de l’Italie cet usage, tout à fait étranger aux races sémitiques et sémitisées, de porter des noms propres permanents, qui maintenaient la notion généalogique de la famille. Probablement il en était ainsi chez les premiers habitants blancs de la Grèce, mais on ne possède plus aucun moyen de s’en assurer. Cette coutume fut conservée par les Romains. (Niebuhr, \emph{Rœm. Gescbichte}, t. I, p. 115.  – Salverte, \emph{Essai sur l’origine des noms propres d’hommes, de peuples et de lieux}, t. I, p. 187.) L’auteur de ce livre paraît croire que l’usage des noms propres permanents cessa vers le III\textsuperscript{e} siècle pour n’être repris que vers le X\textsuperscript{e} siècle. C’est, je crois, une opinion erronée, et j’inclinerais à penser que jamais l’habitude ne fut complètement abandonnée dans les couches celtiques de la population. II y avait à Bordeaux une famille de Paulins au IV\textsuperscript{e}siècle. (Voir Elle Vinet, \emph{l’Antiquité de Bourdeaus et de Bourg}, Bourdeaus, petit in-4°, 1554.)  – Notons en passant que cette habitude, très commode et très simple, de conserver indéfiniment aux descendants le nom du père, paraît faire partie des instincts de plusieurs groupes jaunes. Les Chinois la pratiquent de toute antiquité et avec une telle ténacité que certaines familles originaires de leur pays, qui se sont transportées et fixées en Arménie, ont bien pu, en changeant de langue, oublier leurs noms primitifs  ; mais elles en ont pris de locaux et les conservent fidèlement au milieu d’une population qui n’en a pas. Ce sont les Orpélians, les Mamigonéans, d’autres encore. Au japon, la même coutume existe, et, fait plus notable encore, elle est immémoriale chez les Lapons européens, chez les Bouriates, les Ostiaks, les Baschkirs. (Salverte, \emph{ouvr. cité}, t. I, p, 135, 141 et 144.)}.\par
Cependant les œuvres des aborigènes italiotes furent des plus considérables. Il n’y a pas dans la Péninsule de vieille ville en ruines, depuis des siècles, où l’on ne découvre encore la trace de leurs mains. Longtemps on a même attribué aux Étrusques telle de leurs œuvres. C’est ainsi que Pise \footnote{Deux ruines remarquables sont Testrina, la plus ancienne cité sabine, située sur une montagne au-dessus d’Amiternum. On y trouve des restes de murs gigantesques dont les blocs, extraits d’un tuf assez tendre, portent des marques d’une taille grossière. (Abeken, \emph{Mittel-Italien}, etc., p. 86 et 140.)}, Saturnia, Agylla, Alsium, très ancienne­ment acquises aux Rasènes, avaient commencé par être des villes kymriques, des cités fondées par les aborigènes. Il en était de même de Cortone \footnote{Abeken, \emph{Mittel-Italien}, etc., p. 125, Cortone présente une singularité remarquable. Comme d’autres villes métisses, et entre autres Thèbes, elle avait deux légendes : l’une probablement tyrrhénienne, qui lui attribuait un éponyme grec  ; puis une autre plus ancienne, et, quoi qu’en dise Abeken, aussi facilement kymrique que rasène, qui en faisait le lieu où avait été enterré ce personnage mystérieux appelé \emph{le Nain}, le (alphabet étranger), voyageur. (Dionys., Halic., I, XXIII Abeken, \emph{ouvr. cité}, p. 26.)}.\par
Dans un autre genre de construction, il paraît certain que la partie de la voie Appienne qui va de Terracine à Fondi était d’origine kymrique, et de beaucoup anté­rieure au tracé romain qui fit entrer ce tronçon dans un plan général \footnote{Abeken, \emph{ibidem}, p. 141.}.\par
Mais il n’était pas au pouvoir des races italiotes de maintenir en rien leur pureté. Ibères, Étrusques, Vénètes, Illyriens, Celtes, engagés dans des guerres permanentes, devaient tous, à chaque instant, perdre ou gagner du terrain. C’était l’état ordinaire. Cette situation s’empirait par l’effet des mœurs sociales qui avaient créé, sous le nom de \emph{printemps sacré}, une cause puissante de confusion ethnique. À l’occasion d’une disette ou d’un surcroît de population, une tribu vouait à un dieu quelconque une partie de sa jeunesse, lui mettait les armes à la main, et l’envoyait se faire une nouvelle patrie aux dépens du voisinage. Le dieu patron était chargé de l’y aider \footnote{Dionys. Halic., \emph{Ant. Rom.}, I, XVI.}. De là des conflits perpétuels qui, enfin, s’empirèrent par l’effet et le contre-coup de grands événements dont la source inconnue se cachait fort loin dans le nord-est du continent.\par
De tumultueuses nations de Galls transrhénans, probablement chassées par d’autres Galls que dérangeaient des Slaves harcelés par des Arians ou des peuples jaunes, firent invasion au delà du fleuve, poussèrent sur leurs congénères, entrèrent en partage de leurs territoires, et, bon gré, malgré, se culbutant avec eux, parvinrent, les armes à la main, jusque sur la Garonne, où leur avant-garde s’établit de force au milieu des vaincus. Puis ces derniers, mal contents d’un domaine devenu trop étroit, se portèrent en masse du côté des Pyrénées, les franchirent en longeant les côtes du golfe de Gascogne, et allèrent imposer aux Ibères une pression toute semblable à celle dont ils venaient de souffrir eux-mêmes.\par
Les Ibères, à leur tour, malmenés, s’ébranlèrent. Après s’être débattus et mêlés en partie à leurs conquérants, voyant leur pays insuffisant pour sa nouvelle population, ils partirent, non plus seulement Ibères, mais aussi Celtibères, sortirent par l’autre extrémité des montagnes, c’est-à-dire par les plages orientales de la Méditerranée, et, vers l’an 1600 avant notre ère, se répandirent sur les parties maritimes du Roussillon et de la Provence. Pénétrant ensuite en Italie par la côte génoise, se montrant en Toscane, enfin passant où ils purent mettre le pied, ils apprirent à ces vastes contrées à connaître leurs noms nouveaux de Ligures et de Sicules. Puis, confondus avec des aborigènes de diverses peuplades \footnote{O. Muller, \emph{die Etrusker}, p. 16.}, ils semèrent au loin un élément ou plutôt une combinaison ethnique destinée à jouer un rôle considérable dans l’avenir. Sous plus d’un rapport, ils ajoutaient un lien de plus à ceux qui unissaient déjà les Italiotes aux populations transalpines.\par
Ce que leur présence occasionna surtout, ce furent de terribles commotions dont toutes les parties de la Péninsule éprouvèrent le contre-coup. Les Étrusques, repoussés sur les provinces umbriques, y subirent des mélanges qui probablement ne furent par les premiers. Beaucoup de Sabelliens ou de Sabins, beaucoup d’Ausoniens eurent le même sort, et le sang ligure lui-même s’infiltra partout d’autant plus avant que la masse de cette nation immigrante, établie principalement dans la campagne de Rome \footnote{\emph{Ibid.}, p. 10.}, ne put jamais se créer une patrie suffisamment vaste. Elle n’eut pas la force de prévaloir contre toutes les résistances qui lui étaient opposées. Elle se contenta de vivre, à l’état flottant dans les contrées où les aborigènes, comme les Étrusques, surent se main­tenir  ; de sorte que les Ligures, intrus et tolérés en plus d’un lieu, ne purent que s’y confondre avec la plèbe \footnote{O. Muller, \emph{die Etrusker}, p. 11.}.\par
Tandis qu’ils supportaient ainsi les conséquences de leur origine, en se voyant forcés, tout envahisseurs qu’ils étaient, de rester au rang d’égaux, parfois d’inférieurs vis-à-vis des nations dont ils venaient troubler les rapports, une autre révolution s’opérait, mais presque en silence, à l’autre extrémité, à la pointe méridionale de la Péninsule. Vers le X\textsuperscript{e} siècle avant Jésus-Christ, des Hellènes, déjà sémitisés, commen­çaient à y établir des colonies, et, bien que formant, comparés aux masses ligures ou sicules, un contraste marqué par leur petit nombre, on les voyait déployer sur celles-ci et sur les aborigènes une telle supériorité de civilisation et de ressources, que la conquête de tout ce qu’ils voudraient prendre semblait d’avance leur être assurée.\par
Ils s’étendirent à leur aise. Ils placèrent des villes là où il leur plut. Ils traitèrent les Pélasges italiotes ainsi que leurs pères avaient traité les parents de ceux-ci dans l’Hellade. Ils les subjuguèrent ou les forcèrent de reculer, quand ils ne se mêlèrent pas à eux, comme il en advint avec les Osques. Ceux-ci, atteints, d’assez bonne heure, par l’alliage hellénique sémitisé, portèrent témoignage de cette situation dans leurs mœurs comme dans leur langue. Plusieurs de leurs tribus cessèrent d’être, à proprement parler, aborigènes. Elles offrirent un spectacle analogue à celui que présentèrent plus tard, vers le milieu du II\textsuperscript{e} siècle avant notre ère, les gens de la Provence soumis à l’hymen romain. C’est ce qu’on appelle la seconde formation des Osques \footnote{\emph{Ibidem.}}.\par
 Mais la plupart des nations pélasgiques éprouvèrent un traitement moins heureux. Chassées de leurs territoires par les colonisateurs hellènes, il ne leur resta que l’alternative de se porter sur des groupes de Sicules, établis un peu plus au nord dans le Latium \footnote{\emph{Ibid}.}, et elles se mêlèrent à eux. L’alliance, ainsi conclue, se renforça gra­duellement \footnote{Ammien Marcellin affirme (I, 15, 9) que les aborigènes du Latium étaient des Celtes.} de nouvelles victimes des colons grecs. À la fin, cette masse confuse, ballottée et pressée de tous côtés par des rassemblements rivaux, et surtout par des Sabins, demeurés plus Kymris que les autres, et, par conséquent, supérieurs en mérite guerrier aux Osques déjà sémitisés, comme aux Sicules demi-Ibères, comme aux Rasènes demi-Finnois, cette masse confuse, dis-je, recula pied à pied, et, un millier d’années à peu près avant l’ère chrétienne, s’en alla chercher un refuge en Sicile.\par
Voilà ce qu’on sait, ce que l’on peut voir des plus anciens actes de la population primitive de l’Italie, population qui, en général, échappe à l’accusation de barbarie, mais qui, à l’instar des Celtes du nord, bornait sa science sociale à la recherche de l’utilité matérielle. Bien des guerres la divisaient, et cependant l’agriculture florissait chez elle, ses champs étaient cultivés et productifs. Malgré la difficulté de passer les montagnes et les forêts, de traverser les fleuves, son commerce allait chercher les peuples les plus septentrionaux du continent. De nombreux morceaux de succin, conservés bruts ou taillés en colliers, se rencontrent fréquemment dans ses tombeaux \footnote{Abeken, \emph{Unter-Italien}, p. 267.  \emph{–} Voir la description que fait cet auteur du tumulus d’Alsium.}, et l’identité, déjà signalée, ainsi que ce fait, de certaines monnaies rasènes avec des monnaies de la Gaule, démontre irrésistiblement l’existence de relations régulières et permanentes entre les deux groupes \footnote{Abeken, \emph{Unter-Italien}, p. 282.  – Aristote assure qu’une route allait d’Italie dans la Celtique et en Espagne.}.\par
À cette époque si reculée, les souvenirs ethniques encore récents des races euro­péennes, leur ignorance des pays du sud, la similitude de leurs besoins et de leurs goûts, devaient tendre nécessairement à les rapprocher \footnote{Tite-Live a pu écrire au sujet du roi Mézence : « Cœre opulento tam, oppido imperitans. »}. Depuis la Baltique jusqu’à la Sicile \footnote{« Plus je m’avance profondément dans l’antiquité, dit Schaffarik, plus je demeure « convaincu de la fausseté complète des opinions émises et reçues jusqu’ici sur la « comparaison des peuples antiques du sud de l’Europe (des Grecs et des Romains) avec « ceux du nord, principalement des riverains de la Vistule et de la Baltique, comparaison « qui semblait convaincre ces derniers de sauvagerie, de rudesse et de misère, et rendre « inadmissible toute idée de relations commerciales entre les deux groupes. » (Schaffarik, \emph{Slawische Alterthümer}, t. I, p. 107, note 1.)  – Voici, sur le même propos, un jugement de Niebuhr : « Les aborigènes sont dépeints par Salluste et Virgile comme des sauvages qui « vivaient par bandes, sans lois, sans agriculture, se nourrissant des produits de la chasse « et de fruits sauvages. Cette façon de parler ne parait être qu’une pure spéculation « destinée à montrer le développement graduel de l’homme, depuis la rudesse bestiale « jusqu’à un état de culture complète. C’est l’idée que, dans le dernier demi-siècle, on a « ressassée jusqu’à donner le dégoût, sous le prétexte de faire de l’histoire philosophique. « On n’a pas même oublié la prétendue misère idiomatique qui rabaisse les hommes au « niveau de l’animal. Cette méthode a fait fortune, surtout à l’étranger (Niebuhr veut dire « en France). Elle s’appuie de myriades de récits de voyageurs soigneusement recueillis « par ces soi-disant philosophes. Mais ils n’ont pas pris garde qu’il n’existe pas un seul « exemple d’un peuple véritablement sauvage qui soit passé librement à la civilisation, et « que, là où la culture sociale a été imposée du dehors, elle a eu pour résultat la « disparition du groupe opprimé, comme on l’a vu, récemment, pour les Natticks, les « Guaranis, les tribus de la Nouvelle-Californie, et les Hottentots des Missions. Chaque « race humaine a reçu de Dieu son caractère, la direction qu’elle doit suivre et son « empreinte spéciale. De même, encore, la société existe avant l’homme isolé, comme le « dit très sagement Aristote ; le tout est antérieur à la partie et les auteurs du système du « développement successif de l’humanité ne voient pas que l’homme bestial n’est qu’une « créature dégénérée ou originairement un demi-homme. » (\emph{Rœm. Geschichte}, t. I, p. 121.)}, une civilisation existait incomplète, mais réelle et partout la même, sauf des nuances correspondantes aux nuances ethniques découlant des hymens, sporadique­ment contractés, entre des groupes issus des deux rameaux blanc et jaune.\par
Les Tyrrhéniens asiatiques vinrent troubler cette organisation sans éclat, et aider les colons de la Grande-Grèce dans la tâche de rallier l’Europe à la civilisation adoptée par les peuples de l’est de la Méditerranée\footnote{Les médailles grecques de la plus ancienne époque présentent, ainsi que quelques statues qui sont venues jusqu’à nous, un type fort étrange complètement différent de la physionomie hellénique, et que l’on ne peut attribuer qu’aux anciens Pélasges. Le nez est long, droit et pointu, courbé en dedans, au milieu, de façon que l’extrémité se relève légèrement. Les pommettes sont un peu saillantes ; les yeux montrent une légère tendance à l’obliquité ; la bouche est grande, et affecte une sorte de sourire singulier qu’on pourrait dire impitoyable. La tête est oblongue, le front bas et assez fuyant, sans exclure une certaine ampleur des tempes. Il n’y a pas de doute que ce type est pélasgique. Son centre paraît avoir été dans la Samothrace et les pays environnants, à Thasos, Lete, Orreskia, Selybria. Les médailles de Thasos l’offrent uni à la représentation d’une scène phallique qui fait allusion, sans doute, à quelque tradition d’enlèvement et de violence analogue à celle dont les Pélasges Tyrrhéniens, chassés de l’Attique, se rendirent coupables envers les femmes hellènes d’Athènes au milieu du XII\textsuperscript{e} siècle avant J.-C. On le contemple sur les vieilles monnaies de la ville de Minerve, sur celles d’Égine, d’Arcadie, d’Argos, de Potidée, de Pharsale  ; puis, en Asie, sur celles de Gergitus, de Mysie, d’Harpagia, de Lampsaque  ; enfin, en Italie, sur celles de Velia ; en Sicile, sur celles de Syracuse ; peut-être même, en Espagne, sur une médaille d’argent d’Obulco. Tous ces pays, sauf le dernier, ont été historiquement occupés par des populations soit aborigènes, soit immigrées, appartenant aux groupes pélasgiques, et toutes les médailles dont il est ici question et qui tranchent, de la manière la plus frappante, la plus impossible à méconnaître, avec le caractère hellénique, qui n’ont rien de commun avec sa régularité, sa beauté, appartiennent toutes à la plus ancienne époque. Certaines sculptures en Sicile, remarquables par leur laideur, s’y peuvent rapporter ; mais ce qui ne laisse pas le moindre doute sur cette corrélation, ce sont les statues du fronton d’Égine et quelques figures italiotes antéromaines. – Cabinet \emph{de S. E. M. le général baron de Prokesch-Osten.}}.\par
FIN DE LA NOTE DE BAS DE PAGE
\section[{V.5. Les Étrusques Tyrrhéniens. – Rome étrusque.}]{V.5. \\
Les Étrusques Tyrrhéniens. – Rome étrusque.}
\noindent Il semble peu naturel, au premier abord, de voir les souvenirs positifs en Étrurie ne remonter qu’au commencement du X\textsuperscript{e} siècle avant notre ère. C’est une antiquité en somme bien médiocre.\par
Cette particularité s’explique de deux manières qui ne s’excluent pas. Pour premier point, l’arrivée des nations blanches dans la partie occidentale du monde est postérieure à leur apparition dans le sud. Ensuite le mélange des blancs avec les noirs a donné, tout d’abord, naissance à la civilisation qu’on pourrait appeler apparente et visible, tandis que l’union des blancs avec les Finnois n’a créé qu’un mode de culture latente, cachée, utilitaire. Longtemps, confondant les apparences avec la réalité, on n’a voulu reconnaître le perfectionnement social que là où des formes extérieures très saillantes accusaient moins sa présence qu’une nature, qu’une façon d’être plus ornée dans sa manière de se produire. Mais, comme il n’est pas possible de nier que les Ibères et les Celtes aient eu le droit de se dire régulièrement constitués en sociétés civiles, il faut leur reconnaître, et, avec eux, à toute l’Europe primitive de l’ouest et du nord, un rang légitime dans la hiérarchie des peuples cultivés.\par
Je suis loin toutefois de traiter avec indifférence ce que j’appelle ici \emph{question de forme}, et, de même que je ne prendrai jamais pour type de l’homme social l’industriel consommé, ou le marchand le plus habile dans sa partie, et que je mettrai toujours au-dessus d’eux, mais certes à une hauteur incomparable, soit le prêtre, soit le guerrier, l’artiste, l’administrateur, ou ce qu’on appelle aujourd’hui l’homme du monde, et qu’on nommait au temps de Louis XIV l’\emph{honnête homme}  ; comme, de même, je préférerai toujours, dans l’ordre des hommes d’élite, saint Bernard à Papin ou à Watt, Bossuet à Jacques Cœur, Louvois, Turenne, l’Arioste ou Corneille à toutes les illustrations financières, je n’appelle pas civilisation active, civilisation de premier ordre, celle qui se contente de végéter obscurément, ne donnant à ses sectateurs que des satisfactions en définitive fort incomplètes et par trop humbles, confinant leurs désirs sous une sphère bornée, et tournant dans cette spirale de perfectionnements limités dont la Chine a atteint le sommet. Or, tant qu’un groupe de peuples est réduit, pour tout mélange, à l’élément jaune combiné avec le blanc, il n’acquiert dans les qualités, les capacités, les aptitudes, soit mixtes, soit nouvelles, que cet hymen procrée, rien qui l’attire dans le courant nécessaire de l’élément féminin, et lui fasse rechercher la divination de ce qu’il y a de transcendantalement utile à cultiver les jouissances que l’imagination pure répand sur une société.\par
Si donc les peuples occidentaux avaient dû rester bornés à la combinaison de leurs premiers principes ethniques, il est plus que probable qu’à force d’efforts ils auraient fini par arriver à un état comparable à celui du Céleste Empire, sans cependant trouver le même calme. Il y avait déjà trop d’affluents divers dans leur essence, et surtout trop d’apports blancs. Pour cette raison, le despotisme raisonné du Fils du Ciel ne se serait jamais établi. Les passions militaires auraient, à chaque instant, bouleversé cette société vouée ainsi à une culture médiocre et à de longs et inutiles conflits.\par
Mais les invasions du Sud vinrent apporter aux nations européennes ce qui leur manquait. Sans détruire encore leur originalité, cette heureuse immixtion alluma l’âme qui les fit marcher, et le flambeau qui, en les éclairant, les conduisit à associer leur existence au reste du monde.\par
Deux cent cinquante ans avant la fondation de Rome \footnote{Cette date est celle d’O. Muller. Abeken reporte l’arrivée des Tyrrhéniens à l’an 290 avant Rome. (Abeken, \emph{Mittel-Italien vor der Zeit der rœmischen Herrschaft}, p. 23.)}, des bandes pélasgiques sémitisées pénétrèrent en Italie par la voie de mer, et ayant fondé, au milieu des Étrusques conquis et domptés, la ville de Tarquinii, en firent le centre de leur puis­sance. De là ils s’étendirent, de proche en proche, sur une très grande partie de la Péninsule.\par
Ces civilisateurs, appelés plus particulièrement Tyrrhéniens ou Tyrséniens, venaient de la côte ionienne, où ils avaient appris beaucoup de choses des Lydiens, auxquels ils s’étaient alliés \footnote{Les peintures étrusques montrent ces Tyrrhéniens comme ayant parfaitement le type blanc. Ils ressemblent aux Celtes et aux Grecs, et cette ressemblance est d’autant plus saillante que l’on voit mêlés à eux les anciens Rasènes avec leurs statures et leurs visages de métis finnois. (Abeken, \emph{ouvr. cité}, tabl. IX et X.) Dans le n° 7 de la tabl. VII on peut constater la fusion des deux types.}. Ils apparurent aux yeux des Rasènes couverts d’armures d’airain, animant les combats du son des trompettes, ayant les flûtes pour égayer leurs banquets, et important une forme et des éléments de société inconnus partout ailleurs qu’en Asie et en Grèce, où les Sémites en avaient introduit de semblables.\par
 Au lieu d’imiter les constructions puissantes, mais grossières, des populations italiotes, les nouveaux venus, plus habiles parce qu’ils étaient métis de nations plus cultivées, apprirent à leurs sujets à bâtir sur les hauteurs, sur les crêtes de montagnes, des villes fortifiées avec un art tout nouveau, des refuges inexpugnables, aires redou­tées, d’où la domination planait sur les contrées environnantes \footnote{Ce fut probablement le genre de mérite qui éclata le plus en eux, et leur valut le surnom de \emph{Tyrrhéniens}, dont la racine semble se trouver dans le mot \emph{turs}, ou \emph{tour, fortitication}, et dériver primitivement de \emph{tur} ou \emph{tor, élévation, montagne}.  – On pourrait, du reste, tirer ainsi des habitudes architecturales des différentes populations pélasgiques certains noms encore, ou, au rebours, faire sortir ceux des nations de leur façon de se loger. \emph{Oppidum, le bourg ouvert}, serait en corrélation intime avec les habitudes des \emph{Opsci}, des \emph{Osques}, et \emph{arx, la forteresse fermée}, avec celui des Argiens. Abeken, \emph{ouvr. cité}, p. 128-135.)}. Les premiers dans l’Occident, ils taillèrent, au moyen de la règle de plomb, des blocs de pierre qui, s’encastrant les uns dans les autres par les angles rentrants et saillants adroitement ménagés \footnote{O. Muller, \emph{1. c.}}, formèrent des murailles épaisses et d’une solidité dont on peut juger encore, puisque, en plus d’un lieu, elles ont survécu à tout \footnote{\emph{Ibid.}, p. 260.}.\par
Après avoir ainsi créé des fortifications gigantesques, redoutables à leurs sujets autant qu’aux peuples rivaux \footnote{Dans plusieurs endroits, les Tyrrhéniens avaient construit leurs demeures à part de celles des vaincus et de manière à tenir en bride la ville ancienne. Ainsi Fidenæ et Veies avaient des citadelles placées en dehors de leurs murs. (Abeken, \emph{ouvr. cité}, p. 152.)}, les Tyrrhéniens ornèrent leurs villes de temples, de palais, et leurs palais et leurs temples de statues et de vases de terre cuite, dans ce qu’on appelle l’ancien style grec, et qui n’était autre que celui de la côte d’Asie \footnote{O. Muller, t. II, p. 247.}. C’est ainsi qu’un groupe pélasgique se trouvait en état, par ses alliances avec le sang sémitique, d’apporter aux Rasènes ce qui leur manquait, non pour devenir une nation, mais pour le paraître et le révéler à tout ce qui dans le monde tenait le même rang.\par
Il est probable que le nombre des Tyrrhéniens était petit en comparaison de celui des Rasènes. Ces vainqueurs parvinrent donc à donner à la société, pour le plus grand honneur de celle-ci, ses formes extérieures  ; cependant ils ne réussirent pas à l’entraî­ner jusqu’à une assimilation complète avec l’hellénisme. Ils ne le possédaient d’ailleurs eux-mêmes que sous une dose assez faible, n’étant pas Hellènes, mais seulement Kymris, Slaves ou Illyriens Grecs. Puis ils s’accommodèrent sans peine de partager nombre d’idées essentielles que la part sémitique de leur sang n’avait pas détruites dans leur propre sein, De là, cette continuité de l’esprit utilitaire chez la race étrusque  ; de là, cette prédominance du culte et des croyances antiques sur la mythologie importée  ; de là, en un mot, la persistance des aptitudes slaves. Le gros de la nation resta, sauf peu de différences, tel qu’il était avant la conquête. Comme cependant les vainqueurs se trouvèrent, malgré leurs concessions et leurs mélanges ultérieurs avec la population, marqués d’un cachet spécial dû à leur origine à demi asiatique, la fusion ne fut jamais complète, et des tiraillements nombreux préparèrent les révolutions et les déchire­ments.\par
 Les Tyrrhéniens, que j’appellerai aussi, d’après leurs titres, les \emph{lars} \footnote{Ce mot n’appartenait pas à l’étrusque proprement dit. Soit qu’il ait été Importé par les Tyrrhéniens eux-mêmes, soit que les anciennes alliances des Rasènes avec les Kymris italiotes l’eussent mis en usage avant l’arrivée des immigrants vainqueurs, ce mot était celtique  – c’est le \emph{larth} que l’on retrouve dans le \emph{laird} écossais, et le \emph{lord} anglais. Il est assez curieux de voir les grands seigneurs de l’empire britannique glorifier encore la qualification que se donnait le larth Porsenna.} les \emph{lucumons}, les \emph{nobles}, car, ayant perdu l’usage de leur langue primitive, remplacée par l’idiome de leurs sujets, et s’étant assez mariés à ces derniers, ils ne constituèrent bientôt plus une nation à part, les nobles, dis-je, avaient conservé le goût des idées grecques, et, comme un moyen d’y satisfaire, Tarquinii était restée leur ville de prédilection \footnote{Tarquinii, bâtie sur un rocher au bord de la Marta, n’était pas une ville maritime  ; mais Gravisæ, qui lui appartenait, lui servait de port. (Abeken, \emph{ouvr. cité}, p. 36.) Longtemps après la chute de l’Étrurie comme nation indépendante, Tarquinii conservait encore une assez grande valeur pour fournir les flottes romaines de toiles à voile lors de la seconde guerre punique. (Liv., XXVIII, 45.)}. Cette cité servait de lien à des communications constantes avec les nations helléniques \footnote{Ces relations étaient intimes, et Tite-Live a pu mettre en avant l’idée que la maison de Tarquin avait une origine hellénique. Ce roi même, au dire de l’historien, avait consulté, par députés, l’oracle de Delphes.  – Abeken signale des traces nombreuses de l’influence assyrienne dans les vases, les peintures murales et les ornements des tombeaux à une époque où cette influence ne pouvait s’exercer que par l’intermédiaire des Hellènes. (Abeken, \emph{ouvr. cité}, p. 274.) – Je ne parle pas des nombreuses productions égyptiennes que l’on rencontre dans les hypogées étrusques  ; elles appartiennent toutes à la période romaine avec les monuments qui les renferment. (\emph{Ibidem}, p. 268.  – Dennis, \emph{die Stædte und Begræbnisse Etruriens}, t. I, p. XLII.)}. On doit donc la considérer comme le siège de la culture naturelle en Étrurie, et le point d’appui de l’aristocratie et de sa puissance \footnote{\emph{Les Annales étrusques}, d’où le Romain Verrius Flaccus avait tiré les éléments de ses \emph{Libri rerum memoria dignarum}, affirmaient que le héros Tarchon avait fondé Tarquinii, puis les douze villes étrusques du pays plat, et en outre, tout le nomen etruscum Tarquinii était donc la cité historique et illustre par excellence, aux yeux de la famille tyrrhénienne. (Abeken, \emph{ouvr. cité}, p. 20.)}.\par
Tant que les Rasènes avaient été abandonnés à leurs seuls instincts, ils n’avaient pas dû être, pour les autres nations italiotes, des rivaux particulièrement à craindre. Occupés surtout de leurs travaux agricoles et industriels, ils aimaient la paix et cherchaient à la maintenir avec leur voisinage. Mais, lorsqu’une noblesse d’essence belliqueuse, se trouvant à leur tête, leur eut distribué des armes et construit de nobles forteresses, les Rasènes furent contraints de chercher aussi la gloire et les aventures : ils se jetèrent dans la vie de conquêtes.\par
L’Italie n’était pas encore devenue, tant s’en faut, une région tranquille. Au milieu des agitations incessantes des Italiotes aborigènes, des Illyriens, des Ligures, des Sicules, au milieu des déplacements de tribus, causés par les envahissements des colonies de la Grande-Grèce, les Étrusques s’emparèrent d’un rôle capital. Ils profitè­rent de tous les déchirements pour s’étendre à leur convenance. Ils s’agrandirent aux dépens des Umbres dans toute la vallée du Pô \footnote{O. Muller, \emph{die Etrusker}, p. 116.}. Conservant ce qu’avait déjà produit l’industrie de ce peuple dans les trois cents villes que l’histoire lui attribue \footnote{Ou 358.  – Nous savons déjà, pour parer à tout étonnement de ce côté, combien la race des Celtes était abondante et prolifique. (Keferstein, \emph{Ansichten}, etc., t. II, p. 323.)}, ils augmentèrent leur propre richesse et leur importance. Puis \footnote{Ils fondèrent Adria et Spezia entre le Pô et l’Etsch. (O. Muller, \emph{ouvr. cité}, p. 140.)}, du nord tournant leurs armes vers le sud et refoulant sur les montagnes les nations ou plutôt les fragments de nations réfractaires, ils s’étendirent jusque dans la Campanie \footnote{O. Muller, \emph{ouvr. cité}. p. 178.  – Ils restèrent fort longtemps à l’état de puissance prépondérante dans cette province, et n’en furent chassés que l’an 332 de Rome par les Samnites.}, en prenant pour limite occidentale le cours inférieur du Tibre. Ainsi ils touchaient aux deux mers \footnote{Il existe des monuments tyrrhéniens en Corse et en Sardaigne. On en trouve encore sur la côte méridionale de l’Espagne, et le nom de Tarraco, Tarragone, est très vraisemblablement un indice d’autant moins à négliger que, non loin de cette cité, s’élève Suessa, qui rappelle les villes campaniennes de Suessa, Veseia et Sinuessa. (Abeken, \emph{ouvr. cité}, p. 129.) Seulement, je ne suis pas aussi convaincu que cet auteur de l’origine tyrrhénienne des \emph{Sepolcri dei giganti} en Sardaigne. On peut les revendiquer, sans grande difficulté, pour les Rasènes de la première formation, ou pour les Ibères.  – Eu égard à la racine \emph{Tur, Turs, Tusc}, il est à noter aussi qu’on la retrouve, aujourd’hui même, chez les Albanais. Entre Durazzo et Alessio on connaît une ville appelée (nom grec). Une autre encore existe aux environs de Kroja, dans l’Albanie méridionale, qui elle-même se nomme (nom grec), et ses habitants (nom grec). (Voir Hahn, \emph{Albanesische Studien}, p. 232, 233. Cet auteur fait dériver ce mot de l’arnaute (mot grec), \emph{courir, se précipiter}, d’où (mot grec), le \emph{coureur, l’envahisseur.})}. L’État rasène devint, de la sorte, le plus puissant de la Péninsule, et même un des plus respec­tables de l’univers civilisé d’alors. Il ne se borna pas aux acquisitions continentales : il s’empara de plusieurs îles, porta des colonies sur la côte d’Espagne \footnote{O. Muller, p. 109 et pass. ; p. 178.}. Puissance maritime, il imita l’exemple des Phéniciens et des Grecs en couvrant les mers de navires tout à la fois commerçants et pirates \footnote{\emph{Ibid}., p. 105.}.\par
Avec des progrès si vastes, les Étrusques, déjà métis et fortement métis, soit qu’on les envisage dans leurs classes inférieures, soit qu’on décompose le sang de leur noblesse, ne s’étaient pas soustraits à de plus nombreux mélanges. Soumis au sort de toutes les nations dominatrices, ils avaient, à chacune de leurs conquêtes, annexé à leur individualité la masse des populations domptées, et des Umbres, des Sabins, des Ibères, des Sicules, probablement aussi beaucoup de Grecs, étaient venus se confondre dans la variété nationale, en en modifiant incessamment et les penchants et la nature.\par
À l’inverse de ce qui a lieu d’ordinaire, les altérations subies par l’espèce étrusque étaient, en général, de nature à l’améliorer. D’une part, le sang kymrique italiote, en se mêlant aux éléments rasènes, relevait leur énergie  ; de l’autre, l’essence ariane sémitisée, apportée par les Grecs, donnait à l’ensemble un mouvement, une ardeur, trop faible pour le jeter dans les frénésies helléniques ou asiatiques, mais suffisantes pour corriger quelque peu ce que les alliages occidentaux avaient de trop absolument utilitaire. Malheureusement ces transformations s’opéraient surtout dans les classes moyennes et basses, dont la valeur se trouvait ainsi rapprochée de celle des familles nobles, et ce n’était pas là de quoi maintenir l’équilibre politique intact et la puissance aristocratique incontestée.\par
Puis, cette grande bigarrure d’éléments ethniques créait trop de mélanges fragmentaires et de petits groupes séparés. Des antagonismes s’établirent dans le sein de la population, presque comme en Grèce, et jamais l’empire étrusque ne put parvenir à l’unité. Puissant pour la conquête, doué d’institutions militaires si parfaites que les Romains n’ont eu, plus tard, rien de mieux à faire que de les copier, tant pour l’organisation des légions que pour leur armement, les Étrusques n’ont jamais su concentrer leur gouvernement \footnote{La royauté existait de nom chez les Étrusques, mais elle resta de fait une magistrature très faiblement constituée  ; à Veies, elle était élective. (Niebuhr, \emph{Rœm. Geschichte}, t. I, p. 83.)}. Ils en sont toujours restés, dans les moments de crise, à la ressource celtique de \emph{l’embratur, l’imperator}, qui guidait leurs troupes confédérées avec un pouvoir absolu, mais temporaire. Hors de là, ils n’ont réalisé que des confédérations de villes principales, entraînant les cités inférieures dans l’orbite de leurs volontés. Chaque centre politique était le siège de quelques grandes races, maîtresses des pontificats, interprètes des lois, directrices des conseils souverains, commandant à la guerre, disposant du trésor public. Quand une de ces familles acquérait une prépondérance décidée sur ses rivales, il y avait, en quelque sorte, royauté, mais toujours entachée de ce vice originel, de cette fragilité implacable, qui constituait en Grèce le premier châtiment de la tyrannie. Pendant longtemps, il est vrai, la prédominance que toutes les cités étrusques s’accordaient à laisser à Tarquinii sembla corriger ce que cette constitution fédérative avait de bien débile. Mais une déférence si salutaire n’est jamais éternelle – en butte à mille accidents, elle périt au premier choc. Les peuples gardent plus longtemps le respect pour une dynastie, pour un homme, pour un nom que pour une enceinte de murailles. On le voit donc, les Tyrrhéniens avaient implanté en Italie quelque chose des vices inhérents aux gouver­nements républicains du monde sémitique. Néanmoins, comme ils n’eurent pas l’influence de modeler complètement l’esprit de leurs populations sur ce type dangereux, ils ne purent détruire une aptitude finnoise que j’ai déjà eu l’occasion de relever : les Étrusques professaient pour la personne des chefs et des magistrats un respect tout à fait illimité \footnote{O. Muller, \emph{die Etrusker}, p. 375.}.\par
Ni chez les Arians, ni chez les Sémites, il ne se rencontra jamais rien de semblable. Dans l’Asie antérieure, on vénère à l’excès, on idolâtre, pour ainsi dire, la puissance ; on se tient prêt à en supporter tous les caprices comme des calamités légitimes. Que le maître s’appelle roi ou patrie, on adore en lui jusqu’à sa démence. C’est qu’on redoute la possibilité de la contrainte, et qu’on se prosterne devant le principe abstrait de la souveraineté absolue. Quant à la personne revêtue du pouvoir et des prérogatives du principe, on n’en fait nul cas. C’est une notion commune aux nations serviles et aux démagogies que de considérer le magistrat comme un simple dépositaire de l’autorité qui, du jour où, par cessation régulière ou bien par dépossession violente, il est jeté hors de sa charge, n’est pas plus respectable que le dernier des hommes, et n’a pas plus de droits à la déférence. De ce sentiment naissent le proverbe oriental qui accorde tout au sultan vivant, rien au sultan mort, et encore cet axiome, cher aux révolutionnaires modernes, en vertu duquel on prétend honorer le magistrat en couvrant l’homme de bruyantes injures et d’outrages déclarés.\par
La notion étrusque, toute différente, aurait sévèrement réprimé chez Aristophane les attaques contre Cléon, chef de l’État, ou contre Lamachus, général de l’armée. Elle jugeait la personne même du représentant de la loi comme tellement sacrée, que le caractère auguste des fonctions publiques ne s’en séparait pas, ne pouvait en être distrait. J’insiste sur ce point, car cette vénération fut la source de la vertu que plus tard, on admira, à juste titre, chez les Romains.\par
Dans ce système, on admet que le pouvoir est, de soi, si salutaire et si vénérable, qu’il impose un caractère en quelque sorte indélébile à celui qui l’exerce ou l’a exercé. On ne croit pas que l’agent de la puissance souveraine redevienne jamais l’égal du vulgaire. Parce qu’il a participé au gouvernement des peuples, il reste à jamais au-dessus d’eux. Reconnaître un tel principe, c’est placer l’État dans une sphère d’éternelle admiration, donner une récompense incomparable aux services qu’on lui rend, et en proposer l’exemple aux émulations les plus nobles. Ainsi on n’accepte jamais qu’il soit loisible d’ouvrir, même respectueusement, la robe du juge, pour frotter de boue le cœur de celui qui la porte, et l’on pose une infranchissable barrière devant les emportements de cette prétendue liberté, avide de déshonorer qui commande, pour arriver d’un pas plus sûr à déshonorer le commandement même.\par
La nation étrusque, riche de son agriculture et de son industrie, agrandie par ses conquêtes, assise sur deux mers, commerçante, maritime \footnote{Les Tyrrhéniens exerçaient en grand la piraterie, et mirent en mer des flottes assez considérables pour lutter contre les villes grecques. Les Massaliotes n’osaient, à cause d’eux, traverser les mers occidentales qu’avec des convois armés. (Niebuhr, \emph{Rœm. Geschichte}, t. I, p. 84.) L’Étrurie avait conclu avec Carthage des traités de navigation et de commerce qui portaient encore leur plein effet au temps d’Aristote, vers 430 de Rome. (\emph{Ibid}., p. 85.)}, recevant, par Tarquinii et par les frontières du sud, tous les avantages intellectuels que sa constitution ethnique lui permettait d’emprunter à la race des Hellènes, exploitant les richesses que lui valaient ses travaux utiles et sa puissance territoriale, au profit des arts d’agrément, bien que, dans une mesure toute d’imitation \footnote{Voir, pour les détails des rapports intellectuels des Tyrrhéniens avec les Grecs, Niebuhr, \emph{Rœm. Geschichte}, t. I, p. 88.}, livrée à un grand luxe, à un vif entraî­nement sensuel vers les plaisirs de tout genre, la nation étrusque faisait honneur à l’Italie, et semblait n’avoir à craindre pour la perpétuité de sa puissance que le défaut essentiel d’une constitution fédérative et la pression des grandes masses de peuples celtiques, dont l’énergie pouvait un jour, dans le nord, lui porter de terribles coups.\par
Si ce dernier péril avait existé seul, il est probable qu’il eût été combattu avec avantage, et qu’après quelques essais d’invasion vigoureusement déjoués, les Celtes de la Gaule auraient été contraints de plier sous l’ascendant d’un peuple plus intelligent.\par
La variété étrusque formait certainement, prise en masse, une nation supérieure aux Kymris, puisque l’élément jaune y était ennobli par la présence d’alliages, sinon toujours meilleurs en fait, du moins plus avancés en culture. Les Celtes n’auraient donc eu d’autre instrument que leur nombre. Les Étrusques, déjà en voie de conquérir la Péninsule entière, avaient assez de forces pour résister, et auraient facilement rembarré les assaillants dans les Alpes. On aurait vu alors s’accomplir, et beaucoup plus tôt, ce que les Romains firent ensuite. Toutes les nations italiotes, enrôlées sous les aigles étrusques, eussent franchi, quelques siècles avant César, la limite des montagnes, et un résultat d’ailleurs semblable à celui qui eut lieu, puisque les éléments ethniques se seraient trouvés les mêmes, eût seulement avancé l’heure de la conquête et de la colonisation des Gaules. Mais cette gloire n’était pas réservée à un peuple qui devait laisser échapper de son propre sein un germe fécond dont l’énergie lui porta bientôt la mort.\par
Les Étrusques, pleins du sentiment de leur force, voulaient continuer leurs progrès. Apercevant du côté du sud les éclatants foyers de lumières que la colonisation grecque y avait allumés dans tant de cités magnifiques, c’était là que les confédérations tyrrhéniennes cherchaient surtout à s’étendre. Elles y trouvaient l’avantage de se mettre dans un rapport plus direct que par la voie de mer avec la civilisation la plus parente. Les lucumons avaient déjà porté les efforts de leurs armes vers la Campanie. Ils y avaient pénétré assez loin dans l’est. À l’ouest, ils s’étaient arrêtés au Tibre.\par
Désormais ils souhaitaient de franchir ce fleuve, ne fût-ce que pour se rapprocher du détroit, où Cumes les attirait tout autant que Vulturnum.\par
Ce n’était pas une entreprise facile. La rive gauche était longée par le territoire des Latins, peuple de la confédération sabine. Ces hommes avaient prouvé qu’ils étaient capables d’une résistance trop vigoureuse pour qu’on pût les déposséder à force ouverte. On préféra, avant de s’engager dans des hostilités sans issue, user de ces moyens à demi pacifiques, familiers à tous les peuples civilisés avides du bien d’autrui \footnote{Les populations italiotes tenaient beaucoup à ce que les Étrusques ne passassent pas le fleuve. Il y avait eu un traité entre les Latins et les Tyrrhéniens qui en stipulait la défense : « Pax ita « convenerat ut Etruscis Latinique fluvius Albula, quem nunc Tiberim vocant, finis esset. » (Liv, I, 12.)}.\par
Deux aventuriers latins, bâtards, disait-on, de la fille d’un chef de tribu, furent les instruments dont s’arma la politique rasène. Romulus et Rémus, c’étaient leurs noms, accostés de conseillers étrusques et d’une troupe de colons de la même nation, s’établirent dans trois bourgades obscures, déjà existantes sur la rive gauche du Tibre \footnote{Qui mérita dès lors le nom de \emph{Tuscum Tiberim} que lui donne Virgile (\emph{Georg.}, I, 499). – Suivant toute probabilité, les deux jumeaux se cantonnèrent sur l’Aventin, à côté d’une bourgade peuplée de Latins, \emph{prisci Latini}, qui occupait, antérieurement, le Janicule. (Abeken, \emph{Mittel-Italien vor der Zeit der rœmischen Herrsch}, p. 70.)\emph{ –} Un autre établissement latin couronnait le sommet du Palatin  – Des Étrusques prirent possession plus tard du \emph{mons Cœlius. Ibidem.  – Tac., Ann.}, IV, 65.)}, non pas au bord de la mer, on ne voulait pas faire un port  ; non pas sur le cours supérieur du fleuve, on ne pensait pas à créer une place de commerce qui ralliât plus tard les intérêts des deux parties nord et sud de l’Italie centrale, mais indifféremment sur le point qu’on put saisir, attendu que le résultat, pour les promoteurs de cette fondation, n’était que de faire passer le fleuve à leurs établissements. Ils s’en remet­taient ensuite aux circonstances pour développer ce premier avantage \footnote{Denys d’Halicarnasse remarque que plusieurs historiens ont appelé Rome une ville tyrrhénienne. Ces historiens avaient parfaitement raison de le faire, et ils exprimaient une vérité incontestable. (mots grecs). (I, XXIX.)}.\par
Comme il fallait agrandir trois hameaux destinés à devenir une ville, les deux fondateurs appelèrent, de toutes parts, les gens sans aveu. Ceux-ci, trop heureux de se créer des foyers, et, pour la plupart, Sabins ou Sicules errants, formèrent le gros des nouveaux citoyens.\par
Mais il n’aurait pas été conforme aux vues des directeurs de l’entreprise de laisser des races étrangères s’emparer de la tête de pont qu’ils jetaient dans le Latium. On donna donc à cette agglomération de vagabonds une noblesse tout étrusque. On reconnaît sa présence aux noms significatifs des Ramnes, des Luceres, des Tities \footnote{O. Muller, \emph{die Etrusker}, p. 381 et pass.  – Cette opinion me paraît avoir tout avantage sur celle d’Abeken, qui voit dans les \emph{Ramnes} les habitants primitifs du Palatin, dans les \emph{Luceres} ceux du Cœlius, dans les \emph{Tities} ceux du Capitole. (\emph{Ouvr. cité}, p. 136.) Les deux opinions peuvent du reste, se concilier, si l’on admet que les trois noms, également étrusques, ont été donnés non pas au gros des trois populations, mais seulement à leurs nobles, ce qui serait une conception parfaitement conforme aux idées italiotes et tyrthéniennes. (O. Muller, \emph{ouvr. cité}, p. 381 et pass.)}. Le gouvernement local porta la même empreinte \footnote{Niebuhr, \emph{Rœm. Geschichte}, t. I, p. 181.}. Il fut sévèrement aristocratique, et l’élément religieux, ou, pour mieux dire, pontifical, s’y présenta strictement uni au commandement militaire, ainsi que le voulaient les notions sémitisées des Tyrrhé­niens, si différentes, sur ce point, des idées galliques. Enfin, le pouvoir judiciaire, confondu avec les deux autres, fut également remis aux mains du patriciat, de sorte que, suivant le plan des organisateurs, il ne resta à la disposition des rois, sauf les bribes de despotisme, glanées dans les moments de crise, que l’action administrative \footnote{Niebuhr, \emph{Rœm. Geschichte}, t. I, p. 206.  – Il n’était pas indispensable que les rois fussent nés dans la ville. On les prenait comme on les trouvait, ou mieux, comme ils étaient imposés du dehors. (\emph{Ibidem.}, p. 213 et 220.)}.\par
Si le gouvernement s’institua dans tout étrusque, la forme extérieure de la civilisation, et même l’apparence de la nouvelle cité, ne le furent pas moins \footnote{Liv., I : « Me haud pœnitet eorum, sententiæ quibus et apparitores et hoc genus ab Etruscis « finitimis unde sella curilis unde rosa prætexta sumpta est, numerum quoque ipsum « ductum est : et ira habuisse Etruscos quod, ex duodecim populis communiter creato « rege, singulos singuli populi lectores dederint. »}. On construisit, sous le nom de Capitole, une citadelle de pierre à la mode tyrrhénienne, on bâtit des égouts et des monuments d’utilité publique, tels que les populations latines n’en connaissaient pas \footnote{O. Muller, \emph{die Etrusker}, p. 120.}. On érigea, pour les dieux importés, des temples ornés de vases et de statues de terre cuite fabriquées à Fregellæ \footnote{O. Muller, \emph{die Etrusker}, p. 247.  – Voir, sur la statue de Turanius de Fregellæ qui représentait un Jupiter, ce que dit Bœttiger, \emph{Ideen zur Kunstmythologie} (t. II, p. 193.)}. On créa des magistratures qui portèrent les mêmes insignes que celles de Tarquinii, de Falerii, de Volterra. On prêta à la ville naissante les armes, les aigles, les titres militaires \footnote{La tunique triomphale, le bâton de commandement du dictateur, en ivoire, surmonté d’un aigle, les jeux équestres, etc., etc. (O. Muller, \emph{ouvr. cité}, p. 121.)  – jusqu’à l’expulsion des rois, le système militaire, à Rome et en Étrurie, fut absolument le même dans les détails comme dans l’ensemble. (\emph{Ibidem}, p. 391.)}, on lui donna enfin le culte \footnote{Tite-Live déclare qu’on n’admit qu’une seule divinité non étrusque, c’était celle de la ville d’Albe à laquelle les deux maîtres nominaux de la ville avaient probablement conservé leur dévotion natale : « Sacra diis aliis, albano ritu, græco Herculi, ut ab Evandro instituta erant, facit. Hæc tum sacra Romulus una ex omnibus peregrina suscepit. » (Liv. I.)  – Toutefois, cette assertion de l’historien de Padoue me paraît ne devoir pas être prise au pied de la lettre. Elle s’applique, sans doute, au culte officiel seulement  ; car il est bien probable que les gens de races si diverses qui peuplaient Rome avaient conservé, dans l’intérieur de leurs maisons, leurs divinités nationales. Ainsi se prépara la vaste confusion des cultes qui devait avoir lieu au sein de Rome impériale.}, et, en un mot, Rome ne se distingua des établissements purement rasènes que par ce fait intime, très important d’ailleurs, que le gros de sa population, autrement composé, avait beaucoup plus de vigueur et de turbulence \footnote{ 
\bibl{Virg., \emph{Georg.}, II, 167 :}
 
\begin{verse}
Hæc genus acre virum : Marsos, pubemque Sabellam,\\
 Adsuetumque malo Ligurem, Volscosque verutos\\
 Extulit.\\
\end{verse}
}.\par
Les plébéiens n’y ressemblaient nullement à la masse pacifique et molle jadis soumise par les Tyrrhéniens, sans quoi les colonisateurs, plus heureux, auraient obtenu de leurs savantes combinaisons les résultats qu’ils s’en promettaient. Il y avait un élément de trop dans cette population plébéienne, qu’on avait si fort mélangée, peut-être avec l’intention de la rendre faible par le défaut d’homogénéité. Si ce calcul présida, en effet, au mode de recrutement adopté pour elle, on peut dire que les précautions de la politique étrusque allèrent tout à fait contre leur espoir de s’assurer une domination plus facile. Ce fut précisément ce qui inculqua dans le jeune établissement les premiers instincts d’émancipation, les premiers germes et mobiles de grandeur future, et cela par une voie si particulière, si bizarre, qu’un fait analogue ne s’est pas présenté deux fois dans l’histoire.\par
Au milieu du concours de gens sans aveu, de toutes tribus, appelés à devenir les habitants de la ville, on avait des Sicules. Cette nation métisse et errante possédait partout des représentants. Plusieurs des villes de l’Étrurie en comptaient en majorité dans leur plèbe ; des parties entières du Latium en étaient couvertes ; le pays sabin en renfermait des multitudes. Ces gens-là furent, en quelque sorte, le fil conducteur qui amena l’élément hellénique, plus ou moins sémitisé, dans la nouvelle fondation. Ce furent eux qui, en mêlant leur idiome au sabin, créèrent le latin proprement dit, commencèrent à lui donner une forte teinture grecque, et opposèrent ainsi l’obstacle le plus vigoureux à ce que la langue étrusque passât jamais le Tibre \footnote{O. Muller, \emph{die Etrusker}, p, 66. – Il est, en effet, très remarquable que l’étrusque, resté toujours pour les Romains, et même au temps des empereurs, une espèce de langue sacrée, n’ait jamais pu se répandre chez eux. Cependant, jusque vers l’époque de Jules, les patriciens l’apprenaient et en faisaient cas comme d’un instrument de civilisation. Plus tard elle fut abandonnée aux augures. À aucun moment elle n’avait pu devenir populaire.}. Le nouveau dialec­te, se pesant comme une digue devant l’idiome envahisseur, fut toujours considéré par les grammairiens romains comme un type dont l’osque et le sabin, altérés de leur valeur première, étaient devenus des variétés, mais qui se tenait dans un dédaigneux éloignement de la langue des lucumons, traitée d’idiome barbare. Ainsi les Sicules, en tant qu’habitants plébéiens de Rome, ont été surtout les adversaires du génie des fondateurs, comme l’importation de leur langue devait être le plus grand empêchement à l’adoption du rasène.\par
Il n’est pas nécessaire de faire remarquer, sans doute, qu’il ne s’agit ici que d’un antagonisme organique, instinctif, entre les Sicules et les Étrusques, et nullement d’une lutte ouverte et matérielle. Assurément cette dernière n’aurait pas eu de chance de succès. Ce fut l’Étrurie elle-même qui, bien malgré elle, se chargea de jeter Rome naissante dans la voie des agitations politiques.\par
La petite colonie était, depuis son premier jour, l’objet des haines déclarées des peuples du Latium. Bien que l’attrait des avantages divers qu’elle avait à offrir, sa construction étrusque, son organisation du même cru et la civilisation de son patriciat eussent porté quelques peuplades assez misérables, les Crustumini, les Antemnati, les Cæninenses \footnote{Liv., I, 28.  – Les Sabins de Tatius, pères des femmes enlevées, des\emph{ Sabinæ mulieres}, ne s’incorporèrent au nouvel État qu’après les trois tribus que je viens de nommer.}, et, un peu plus tard, les Albains, à se fondre dans ses habitants, les vrais possesseurs du sol sabin la considéraient de très mauvais œil. Ils reprochaient à ses fondateurs d’être des gens de rien, de ne représenter aucune nationalité, et de n’avoir d’autre droit à la patrie qu’ils s’étaient faite que le vol et l’usurpation. Ainsi sévèrement jugée, Rome était tenue en dehors de la confédération dont Amiternum était la cité principale, et exposée sur la rive gauche du Tibre, où elle se voyait isolée, à des attaques que très probablement elle n’aurait pas eu la force de repousser, si elle s’était trouvée sans soutiens.\par
Dans l’intérêt de son salut, elle se rattachait de toutes ses forces à la confédération étrusque dont elle était une émanation, et, quand les discordes civiles eurent éclaté au sein de ce corps politique, Rome ne put songer à rester neutre : il lui fallut prendre parti pour se conserver des amis actifs au milieu de ses périls.\par
L’Étrurie en était à cette phase politique où les races civilisatrices d’une nation se montrent abaissées par les mélanges avec les vaincus, et les vaincus relevés quelque peu par ces mêmes mélanges. Ce qui contribuait à hâter l’arrivée de cette crise, c’était la présence d’un trop grand nombre d’éléments kymriques plus ou moins hellénisés, et parfaitement de nature et de force à contester la suprématie aux descendants bâtards de la race tyrrhénienne. Il se développa, en conséquence, dans les cités rasènes un mouvement libéral qui déclara la guerre aux institutions aristocratiques, et prétendit substituer aux prérogatives de la naissance celles de la bravoure et du mérite.\par
C’est le caractère constant de toute décomposition sociale que de débuter par la négation de la suprématie de naissance. Seulement le programme de la sédition varie suivant le degré de civilisation des races insurgées. Chez les Grecs, ce furent les riches qui remplacèrent les nobles  ; chez les Étrusques, ce furent les braves, c’est-à-dire les plus hardis. Les métis raséno-tyrrhéniens, mêlés à la plèbe, sujets umbres, sabins, samnites, sicules, se déclarèrent candidats au partage de l’autorité souveraine. Les doctrines révolutionnaires obtinrent leurs plus nombreux partisans dans les villes de l’intérieur où les anciens vaincus abondaient. Volsinii paraît avoir été le principal point de ralliement des novateurs \footnote{Suivant Abeken, les villes principalement libérales auraient été Arretium, Volaterræ, Rusellæ et Clusium ; et ainsi s’expliquerait, pour le dernier de ces États, la promptitude avec laquelle son chef, le larth Porsenna, s’empressa de conclure la paix avec les Romains insurgés contre les Tarquiniens, après s’être laissé émouvoir à la commencer par un intérêt patriotique opposé à ses intérêts de parti. (\emph{Ouvr. cité}, p. 24.)  – Je remarquerai, en passant, que le nom de \emph{Volaterræ} est latin ; les Étrusques appelaient cette ville \emph{Felathri}, ce qui est beaucoup plus près du \emph{Velletri} moderne. C’est un argument de plus en faveur de l’étude des anciens idiomes de l’Italie au moyen des dialectes locaux actuels.}, tandis que le centre de la résistance aristocratique s’établit à Tarquinii, où le sang tyrrhénien avait conservé quelque force en gardant plus d’homogénéité. Le pays se partagea entre les deux partis. Il est même vraisemblable que chaque cité eut à la fois une majorité et une minorité au service de l’un et de l’autre. Ce qui occupait tout le \emph{nomen etruscum} eut son retentissement naturel dans la colonie transtibérine, et Rome, obéissant aux raisons que j’ai déduites plus haut, prit fait et cause dans le mouvement.\par
On devine déjà pour quel ordre d’idées elle devait se prononcer. Le caractère de sa population répondit d’avance de ses sympathies libérales. Son sénat étrusque, d’ailleurs mêlé déjà de Sabins, n’était pas en état de contenir l’opinion générale dans le camp de Tarquinii \footnote{O. Muller,,\emph{die Etrusker}, p. 316.}. L’esprit ambitieux et ardent des Sicules, des Quirites et des Albains y parlait trop haut. La majorité se prononça donc pour les novateurs, et le roi Servius Tullius essaya de réaliser la révolution en acheminant Rome vers le régime des doctrines anti-aristocratiques.\par
La constitution servienne donna satisfaction à l’élément populaire, en appelant à un rôle politique tout ce qui pouvait porter les armes \footnote{Niebuhr, \emph{Rœm. Geschichte}, t. I, p. 252 et pass.}. On demandait, il est vrai, au membre de \emph{l’exercitus urbanus} quelques conditions de fortune, mais non pas telles qu’elles constituassent une timocratie à la manière grecque. C’était plutôt un cens dans le genre de celui qui, au moyen âge, était exigé des bourgeois de plusieurs communes.\par
Le but n’était pas, dans ce dernier exemple, de créer chez le citoyen des garanties de puissance ou d’influence, mais seulement de moralité politique. Chez les plébéiens de Roma-Quirium, il s’agissait de moins encore : on ne voulait qu’obtenir des guerriers qui fussent en état de s’armer convenablement et de se suffire à eux-mêmes pendant une campagne.\par
Cette organisation, soutenue par les sympathies générales, ne put cependant que s’asseoir à côté des institutions tyrrhéniennes  ; elle ne parvint pas à les renverser. Il y avait encore trop de force dans la façon dont était combiné l’élément militaire et sacerdotal avec la puissance juridique. L’attaque, d’ailleurs, ne fut pas d’assez longue durée pour briser le faisceau et arracher le pouvoir aux races nobles. On y serait parvenu peut-être en recourant aux violences d’un coup de main. Il paraît qu’on ne voulut pas user de ce moyen contre des hommes que le pontificat revêtait d’un caractère sacré. Ce que les sociétés bien vivaces baissent davantage, c’est l’impiété, et évitent le plus longtemps, c’est le sacrilège.\par
Servius Tullius et ses partisans, manquant donc de ce qu’il eût fallu pour vaincre complètement leur noblesse étrusque, se contentèrent de placer le code militaire nouveau auprès de l’ancien, laissant aux progrès de leur cause dans les autres cités rasènes le soin de fournir la possibilité d’aller plus loin. Ces espérances furent trompées. Bientôt l’opposition libérale en Etrurie, battue par le parti aristocratique, se trouva réduite à la soumission. Volsinii fut prise, et un des chefs les plus éminents de la révolte, Cœlius, ne se trouva d’autre ressource que de fuir, d’aller chercher quelque part un asile pour ses plus chauds partisans et pour lui-même.\par
Cet asile, quel pouvait-il être, sinon la ville étrusque qui, après Volsinii, avait montré le plus de dévouement à la révolution, et dû très probablement à sa position territoriale excentrique, à son isolement au delà du Tibre, d’en pousser le plus loin les doctrines et d’en appliquer le plus ouvertement les idées ? Rome vit ainsi accourir Mastarna, Cœlius, et leur monde  ; et le \emph{tuscus vicus}, devenant le séjour de ces bannis \footnote{O. Muller, p. 116 et pass.}, agrandit encore l’enceinte d’une ville qui, au point de vue de ses fondateurs aristocratiques, comme à celui des réformateurs libéraux, était une espèce de camp ouvert à tous ceux qui cherchaient une patrie, et voulaient bien la prendre au sein de la négation de toutes les nationalités.\par
Mais l’arrivée de Mastarna, non moins que la réforme de Servius Tullius \footnote{L’origine latine de Servius, l’usurpation par laquelle il succédait à la dynastie étrusque, la façon dont il flattait les intérêts populaires le rendaient très propre à rallier et à protéger toutes les idées hostiles à la suprématie tyrrhénienne. (Dionys. Halic., 4, I-XL.)}, ne pouvaient être des faits indifférents à la réaction victorieuse. Les lucumons n’étaient pas disposés à souffrir qu’une ville fondée pour leur ouvrir le sud-ouest de l’Italie devînt une sorte de place d’armes aux mains de leurs ennemis intérieurs. Les nobles de Tarquinii se chargèrent d’étouffer l’esprit de sédition dans son dernier asile. Coryphées du parti qui avait créé la civilisation et la gloire nationales, ils en étaient restés les représentants ethniques les plus purs et les agents les plus vigoureux. Ils devaient à leurs relations plus constantes avec la Grèce et l’Asie Mineure de surpasser les autres Étrusques en richesse et en culture. C’était à eux d’achever la pacification en détruisant l’œuvre des niveleurs dans la colonie transtibérine.\par
Ils y parvinrent. La constitution de Servius Tullius fut renversée, l’ancien régime rétabli. La partie sabine du sénat et la population mélangée formant la plèbe rentrèrent dans leur état passif \footnote{Dionys. Halic., \emph{Antiq. Rom.}, XLII, XLIII.  – Le sénat fut renouvelé, et les pères nommés par Tullius, chassés. Les plébéiens rentrèrent dans leur condition de nullité primitive.}, rôle où la pensée étrusque les avait toujours voulu contenir, et les Tarquiniens se proclamèrent les arbitres suprêmes et les régulateurs du gouverne­ment restauré. Ce fut ainsi que le libéralisme vit se fermer son dernier asile \footnote{À ce moment, le parti qui conduisait les affaires à Tarquinii se trouva très fort dans tout le \emph{nomen etruscum.} Il tenait, d’un côté, sa capitale et Rome, puis Veies, Cæræ, Gabii, Tusculum, Antium, et, au sud, s’appuyait sur les sympathies de Cumes, colonie hellénique qui ne pouvait pas voir sans plaisir des efforts si soutenus pour maintenir la civilisation sémitisée dans la Péninsule. (Abeken, \emph{ouvrage cité}, p. 24.)}.\par
On ne sait trop l’histoire des luttes ultérieures de ce parti dans le reste du territoire rasène. Il est cependant certain qu’il releva la tête après un temps d’abattement. Les causes ethniques qui l’avaient suscité ne pouvaient que devenir plus exigeantes à mesure que les races sujettes gagnaient en importance par l’extinction graduelle du sang tyrrhénien. Toutefois, la race rasène du fond national étant de valeur médiocre, il eût fallu beaucoup de temps pour que le résultat égalitaire s’opérât, même avec l’appoint des vaincus, Umbres, Samnites et autres. De sorte que la résistance aristo­cratique avait des chances de se prolonger indéfiniment dans les villes anciennes \footnote{C’est ce qui fut en effet, et, même au temps de la guerre d’Annibal, le gouvernement de la plupart des cités étrusques était resté entier dans les mains de la noblesse, non pas toutefois sans résistances. (Niebuhr, \emph{Rœm. Geschichte}, t. I, p. 81.) Volsinii, la ville démocratique par excellence, réussit à maintenir une administration révolutionnaire entre les mains de la plèbe, depuis la campagne de Pyrrhus jusqu’à la première guerre punique. (\emph{Ouvr. cité}, t. I, p. 82.)}.\par
Mais précisément l’inverse de cette situation se rencontrait à Rome. Outre que les nobles étrusques, natifs de la ville, même appuyés par les Tarquiniens, n’étaient qu’une minorité, ils avaient contre eux une population qui valait infiniment plus que la plèbe rasène. La compression ne pouvait être que difficilement maintenue. Les idées de révolution continuaient à prendre un développement irrésistible en s’appuyant sur les idées d’indépendance, et, un jour ou l’autre, inévitablement, Rome allait secouer le joug. Si, par un coup du sort, Populonia, Pise ou toute autre ville étrusque, possédant jusqu’au fond de ses entrailles non seulement du sang tyrrhénien, mais surtout du sang rasène, avait réussi dans sa campagne contre les idées aristocratiques, l’usage que la cité victorieuse aurait fait de son triomphe se serait borné à changer sa constitution politique intérieure, et, du reste, elle serait restée fidèle à sa race en ne se séparant pas de la partie collective, en continuant à tenir au \emph{nomen etruscum.}\par
Rome n’avait, elle, aucun motif pour s’arrêter à ce point. Précisément les raisons qui la poussaient si chaudement dans le parti libéral, qui lui en avaient fait appliquer les théories, qui l’avaient désignée pour servir, en quelque sorte, de seconde capitale à la révolution, ces raisons-là, par leur énergie, la conduisaient bien au delà d’une simple réforme politique. Si elle ne goûtait pas la domination des lars et des lucumons, c’était, avant tout, parce que ceux-ci, avec les meilleurs droits de se dire ses fondateurs, ses éducateurs, ses maîtres, ses bienfaiteurs \footnote{Dans la guerre de Romulus contre les Sabins de Quirium, le roi romain avait été ouvertement soutenu par une armée étrusque sous le commandement d’un lucumon de Solonium  ; celui-ci avait partagé l’autorité avec lui. (Dionys. Halic., \emph{Antiq. Rom.}, 2, XXXVII)}, n’avaient pas celui d’ajouter qu’ils étaient ses concitoyens. Dans la débilité de ses premiers jours, elle avait trouvé un grand profit, une véritable nécessité à se faire protéger par eux  ; mais, pourtant, son sang ne s’était pas fondu avec le leur, leurs idées n’étaient pas devenues les siennes, ni leurs intérêts ses intérêts. Au fond, elle était sabine, elle était sicule, elle était hellénisée, puis encore elle était séparée géographiquement de l’Étrurie : elle lui était donc, en fait, étrangère, et voilà pourquoi la réaction des Tarquiniens ne pouvait avoir là qu’un temps de succès plus court que dans les autres villes, réellement étrusques, et pourquoi, l’aristocratie tyrthénienne une fois renversée, on devait s’attendre à ce que Rome se précipitât dans les nouveautés fort au delà de ce que souhaitaient les libéraux de l’Étrurie. Bien plus, nous allons voir, tout à l’heure, la ville émancipée revenir sur les théories libérales, source première de sa jeune indépendance, et rétablir l’aristocratie dans toute sa plénitude. Les révolutions, d’ailleurs, sont remplies de pareilles surprises.\par
 Ainsi Rome, après un temps de soumission aux Tarquiniens, réussit à accomplir un soulèvement heureux \footnote{La domination des Tarquiniens avait été, matériellement parlant, on ne peut plus heureuse pour Rome. Ces nobles pleins de génie l’avaient beaucoup embellie. Ils y avaient importé la construction en pierres quadrangulaires sans ciment. (Abeken, \emph{ouvr. cité}, p. 141.) Ils avaient étendu ses fortifications en agrandissant son enceinte. (O. Muller, \emph{ouvr. cité}, P. 120.) Ils y avaient fait venir des artisans habiles de toutes les villes d’Étrurie : « Fabris undique ex Etruria accitis. » (Liv., I.) Ils avaient placé Rome à la tête de la confédération latine, détruite de fait par la chute d’Alba Longs. (Abeken,\emph{ ouvr. cité}, p. 52.) Ils avaient même augmenté cette confédération en y réunissant quarante-sept villes nouvelles, tant en deça qu’au delà du Tibre. (\emph{Ibidem.}) Enfin, des cités telles que Circeii et Signia avaient été fondées, ou du moins agrandies par eux. Rome fit donc une très mauvaise affaire dès le premier moment où sa séparation d’avec Tarquinii fut consommée. L’œuvre entière de l’habileté tyrrhénienne s’écroula, du reste, en même temps. La confédération fut dissoute et le parti aristocratique très affaibli dans toute l’étendue de la domination étrusque. (O. Muller, \emph{ouvr. cité}, p. 124.)}. Elle chassa de ses murailles ses dominateurs, et, avec eux, cette partie du sénat qui, bien que née dans la cité, parlait la langue des maîtres et se vantait d’être de leur parentage. De cette façon, l’élément tyrrhénien disparut à peu près de sa colonie, et n’y exerça plus qu’une simple influence morale. À dater de cette époque, Rome cesse d’être un instrument dirigé par la politique étrusque contre l’indépendance des autres nations italiotes. La cité entre dans une phase où elle va vivre pour elle-même. Ses rapports avec ses fondateurs tourneront désormais au profit de sa grandeur et de sa gloire, et cela d’une façon que ceux-ci n’avaient certainement jamais soupçonnée.
\section[{V.6. Rome italiote.}]{V.6. \\
Rome italiote.}
\noindent J’ai déjà indiqué que, si l’aristocratie étrusque avait conservé sa prépondérance dans la Péninsule, il ne serait arrivé rien autre que ce qui s’est produit dans le monde sous le nom de Rome. Tarquinii aurait absorbé à la longue les indépendances des autres villes fédérées, et, ses éléments de pression sur les peuples voisins, comme sur ceux de l’Espagne, de la Gaule, de l’Asie et du nord de l’Afrique, étant les mêmes que ceux dont Rome disposa plus tard, le résultat final serait demeuré identique. Seulement la civilisation y aurait gagné de se développer plus tôt.\par
Il ne faut pas se le dissimuler : le premier effet de l’expulsion des Tarquiniens fut d’abaisser considérablement le niveau social dans l’ingrate cité \footnote{O. Muller, \emph{die Etrusker}, p. 259.\emph{ –} Les possessions de Rome s’arrêtaient à ce moment au Janicule. Elle avait perdu tout le reste. Servius avait partagé le peuple en trente tribus  ; il n’en restait plus que vingt en 271 de la ville. (Abeken, \emph{ouvr. cité}, p. 25.)}.\par
Qui possédait la science sous toutes formes, politique, judiciaire, militaire, reli­gieuse, augurale ? Les nobles étrusques, et presque personne avec eux. C’étaient eux qui avaient dirigé ces grandes constructions de la Rome royale dont plusieurs survivent encore, et qui dépassaient de si loin tout ce qu’on pouvait voir dans les capitales rustiques des autres nations italiotes. C’étaient eux qui avaient élevé les temples admirés du premier âge, eux encore qui avaient fourni le rituel indispensable pour l’adoration des dieux. On en tombait si bien d’accord que, sans eux, la Rome républicaine ne pouvait ni construire, ni juger, ni prier. Pour cette dernière et importante fonction de la vie domestique autant que sociale, leur concours resta toujours tellement nécessaire que, même sous les empereurs, quand depuis longtemps il n’y avait plus d’Étrurie, quand depuis des siècles les Romains, absorbés par les idées grecques, n’apprenaient plus même la langue, organe vénérable de l’ancienne civilisa­tion, il fallait encore, pour maints emplois du sanctuaire, se confier à des prêtres que la Toscane instruisait seule \footnote{Tac., \emph{Ann}., XI, 15 : « Retulit (Claudins) deinde ad senatum super collegio aruspicum « ne « vetustissima Italiæ disciplina per desidiam exolesceret : sæe adversis reipublicæ « temporibus accitos, quorum monitu redintegratas cærimonias et in posterum rectius « habitas ; primoresque Etruriæ, sponte aut patrum romanorum impulsu retinuisse « scientiam aut in familias propagasse  ; quod nunc segnius fieri, publica circa bonas artes « socordia et quia externæ superstitiones valescant : et iæta quidam in præsens omnia ; « sed benignitatideum gratiam referendam, ne ritus sacrorum, inter ambigua « culti,prospera oblitarentur.  – Factum ex eo senatusconsultum, viderent pontifices quæ « retinenda firmandaque aruspicum. »}. Mais, au dernier moment, il ne s’agissait que de rites ; sous la Rome républicaine, il s’agissait de tout. En chassant les fondateurs de l’État, on arracha les éléments les plus essentiels de la vie publique, et on n’eut d’autre ressource, après s’être assez félicité de la liberté acquise, que de s’accommoder de la misère et d’en faire l’éloge sous le nom de vertu austère. Au lieu des riches étoffes dont s’étaient habillés les seigneurs de la Rome royale, les patriciens de la Rome républicaine s’enveloppèrent dans de grossiers sayons. Au lieu de belles poteries, de plats de métal, entassés sur les tables, et pleins d’une nourriture somptueuse, ils n’eurent plus qu’une rude vaisselle, mal fabriquée par eux-mêmes, où ils s’offrirent leurs pois chiches et du lard. En place de maisons bien ornées \footnote{Un des griefs le, plus violents de la population romaine contre Tarquin le Superbe était qu’il employait la plèbe à construire des palais, des temples et des portiques afin d’embellir la ville. (Dionys. Halic., \emph{Antiq. Rom.}, 4, XLIV, LXI, etc.)}, ils durent se contenter de métairies sauvages, où, parmi les porcs et les poules, vivaient les consuls et les sénateurs qui se louaient judicieusement d’une pareille vie, faute de pouvoir l’échanger contre une meilleure. Bref, pour faire comprendre, par un seul trait, combien la Rome républicaine était au-dessous de son aînée, qu’on se rappelle que, lorsque, après l’invasion des Gaulois, la ville incendiée fut rétablie par Camille, on avait si bien oublié les nécessités d’une grande capitale, que l’on rebâtit les maisons au hasard, et sans tenir aucun compte de la direction des égouts construits par les fondateurs. On ne savait plus même l’existence de la \emph{cloaca maxima} \footnote{O. Muller, \emph{die Etrusker}, p . 259.}. C’est que, grâce à ces mœurs farouches, si admirées depuis, les Romains de cette époque étaient fort au-dessous de leurs pères, et tout autant que leur bourg l’était de la ville régulière fondée jadis par la noblesse étrusque.\par
Voilà cependant la civilisation partie avec le bagage des Tarquiniens. Eut-on au moins la liberté, je dis cette liberté dont les rêves des classes moyennes d’Étrurie avaient cru déposer le germe dans le système de Servius Tullius ? J’ai laissé entrevoir qu’il n’en fut rien, et, en effet, il n’en pouvait rien être.\par
Une fois les Tyrrhéniens chassés, la population se trouva composée en grande majorité de Sabins, gens rudes, austères, belliqueux, et qui, très susceptibles de se développer dans le sens matériel, très capables de résistance contre les agressions, très aptes à imposer leurs notions par la force, n’étaient pas disposés à céder du premier coup leurs droits de suprématie aux Sicules plus spirituels, mais moins vigoureux, aux Rasènes descendants des soldats de Mastarna, bref, au chaos de tant de races qui avaient les représentants dans les rues de Rome \footnote{O. Muller, \emph{ouvr. cité}, p. 204.}. De sorte qu’après s’être débarrassés de la partie étrusque de la nation, les libéraux se trouvèrent avoir sur les bras la partie sabine, et celle-ci fut assez forte pour attirer à elle tout le pouvoir.\par
Suivant l’esprit des blancs, l’amour et le culte de la famille étaient très forts chez les Sabins, et, pour être mal vêtus, mal nourris et assez ignorants, les nobles de cette descendance n’étaient pas moins aristocratiquement inspirés que les lucumons les plus orgueilleux. Les Valériens, les Fabiens, les Claudiens, tous de race sabine, ne souffrirent pas que d’autres que leurs égaux partageassent avec eux les soins du gouvernement, et la seule satisfaction qu’ils laissèrent aux plébéiens fut d’abolir cette royauté qu’eux-mêmes auraient difficilement soufferte. Du reste, ils s’ingénièrent à imiter de leur mieux les maîtres dépossédés en concentrant sous leurs mains jalouses toutes les prérogatives sociales \footnote{Id., \emph{ibid.}, p. 204.}.\par
Ils n’étaient pourtant pas dans cette position de supériorité complète où les Tyrthéniens, Pélasges sémitisés, s’étaient trouvés vis-à-vis des Rasènes, de sorte que les plébéiens ne reconnurent pas très explicitement la légitimité de leur puissance, et n’en supportèrent le joug qu’en murmurant. L’embarras ne se bornait pas là : eux-mêmes, pour peu qu’ils fussent illustres et puissants, gardaient des splendeurs de la royauté un souvenir secret qui leur faisait souhaiter le pouvoir suprême, et redouter que des compétiteurs ne le saisissent avant eux, de sorte que la république commença sa carrière avec toutes les difficultés que voici :\par
Une civilisation très abaissée ;\par
Une aristocratie qui voulait gouverner seule ;\par
Un peuple, tourmenté par elle, qui s’y refusait \footnote{Liv., I : « Civitas secum ipsa discors intestino inter patres plebemque flagrabat odio, « maxime propter nexos ob æs alienum. Fremebant se foris pro libertate et imperio « dimicantes, domi a civibus captos et oppressos esse : tutioremque in bello quam in pace, « inter hostes quam inter cives, libertatem plebis esse. »  – Tac., \emph{Ann.}, VI, 16 : « Sane vetus Urbi fœnebre malum, et seditionum discordiarumque creberrima causa. »} ;\par
L’usurpation imminente chez un noble quelconque ;\par
La révolte non moins imminente dans la plèbe ;\par
Des accusations perpétuelles contre tout ce qui s’élevait au-dessus du niveau vulgaire par le talent ou les services \par
Des ruses incessantes chez les gens d’en bas pour renverser ceux d’en haut sans employer la force ouverte.\par
 Une telle situation ne valait rien. La société romaine, placée dans de telles conditions, ne subsistait qu’à l’aide d’une compression permanente de tout le monde ; de là un despotisme qui n’épargnait personne, et cette anomalie que, dans un État qui fondait son plus cher principe sur l’absence du gouvernement d’un seul, qui proclamait son amour jaloux pour une légalité émanant de la volonté générale, et qui déclarait tous les patriciens égaux, le régime ordinaire fut l’autorité d’un dictateur, sans bornes, sans contrôle, sans rémission, et empruntant à son caractère soi-disant transitoire un degré de violence hautaine inconnu à l’administration de tout monarque avoué.\par
Au milieu de la terrible éruption des fureurs politiques, on est cependant surpris de voir cette Rome, ainsi faite qu’elle semblait une offrande à la discorde, ne pas repré­senter ce qu’on a observé chez les Grecs. Si la passion du pouvoir y tourmente toutes les têtes, c’est une passion qui tend chez les ambitieux, patriciens ou plébéiens, à s’emparer de la loi pour lui donner une forme régulatrice conséquente à telle et telle notion de l’utile ; mais on n’a pas le spectacle répugnant, si constamment étalé sur les places publiques d’Athènes, d’un peuple se ruant en forcené dans les horreurs de l’anarchie avec une sorte de conscience de cette tendance abominable. Ces Romains sont honnêtes, ce sont des hommes ; ils comprennent souvent mal le bien et donnent à gauche, mais au moins est-il évident qu’ils croient alors marcher à droite. Ils ne manquent ni de désintéressement ni de loyauté \footnote{Voir dans Tite-Live la violente insurrection apaisée par les consuls P. Servilius et Ap. Claudius, et l’affaire du mont Sacré. (Liv., I)}. Examinons la question dans le détail.\par
Les patriciens se supposent un droit natif à gouverner l’État exclusivement.\par
Ils ont tort. Les Étrusques pouvaient réclamer cette prérogative ; les Sabins, non, car il n’y a pas de leur côté de supériorité ethnique bien clairement prouvée sur les autres Italiotes qui les entourent et qui sont devenus leurs nationaux. Tout au plus, les Fabiens, les grandes familles possèdent-elles un degré de pureté de plus que la plèbe. En le concédant, on ne peut encore supposer ce mérite assez tranché pour conférer le pouvoir du civilisateur sur le peuple vaincu et dominé \footnote{Dès le temps des rois, il y avait eu des modifications très importantes dans la constitution ethnique du patriciat. Tarquin l’Ancien y avait appelé tout l’ordre équestre en masse. (Niebuhr, \emph{Rœm. Geschichte}, t. I, p. 239.) De sorte qu’aux premiers jours de la république, les plébéiens étaient fondés à se considérer comme du même sang ou d’un sang égal en valeur à celui de leurs gouvernants. Bien mieux, beaucoup de familles plébéiennes rivalisaient de noblesse reconnue avec les plus fières maisons sénatoriales, et formaient, réunies à l’ordre équestre, une classe en réalité aristocratique, avide de saisir les emplois, et toutefois forcée de faire cause commune avec la plèbe. (\emph{Ibid.}, t. I, p. 375.) Beaucoup de maisons plébéiennes, comme les Marciens, les Mamiliens, les Papiens, les Cilniens, les Marruciniens, se trouvaient dans les mêmes rapports vis-à-vis du patriciat où furent à Venise, dans les temps modernes, les nobles de terre ferme vis-à-vis des nobles de Saint-Marc.}. Il n’y avait pas, dans la Rome républicaine, deux races placées sous des rapports inégaux, mais uniquement un groupe plus nombreux que les autres. Ce genre de hiérarchie était de nature à disparaître assez promptement. La défaite du patriciat romain ne fut donc pas une révolution anormale et violant les lois ethniques, mais un fait malheureux et inop­portun, comme l’est constamment la chute d’une aristocratie.\par
La lutte des partis grecs tourna constamment autour des théories extrêmes. Les riches d’Athènes ne tendaient qu’à gouverner eux-mêmes, qu’à absorber les avantages de l’autorité ; le peuple d’Athènes ne visait qu’à la dilapidation des caisses publiques par les mains de l’écume démocratique. Quant aux gens impartiaux, ils imaginaient des doctrines toutes littéraires, toutes d’imagination, et voulaient solidifier des rêves pour corriger des faits. Dans tous les partis, à tous les points de vue, on ne désirait que table rase, et la tradition, l’histoire ne comptaient pour rien sur un sol où le sentiment du respect était absolument inconnu.\par
On n’aurait aucun droit à s’en étonner, Avec l’égrenage ethnique qui faisait le fond de la société athénienne, avec cette dissolution complète de la race qui réunissait, sans avoir jamais pu les fondre, les éléments les plus divers, avec cette prédominance, surtout, de l’élément spirituel, mais insensé, des Sémites, c’était bien là ce qui devait arriver. Une seule chose surnageait au milieu de l’anarchie des notions politiques, l’absolutisme du pouvoir incarné dans le mot de patrie.\par
Mais à Rome il en fut très différemment, et les partis eurent nécessairement d’autres allures. Les races étaient surtout utilitaires. Elles possédaient un sens pratique étranger à l’imagination grecque, et toutes comprenaient, à travers les passions engagées dans la défense de ce qu’on supposait le vrai bien de l’État, une égale horreur pour l’anarchie. C’est ce sentiment qui les rejeta bien souvent dans la ressource extrême de la dictature ; car nativement, il faut le reconnaître, elles étaient sincères, et beaucoup plus que les Grecs, quand elles protestaient de leur haine pour la tyrannie. Métisses de blanc et de jaune, elles avaient le goût de la liberté, et, malgré les sacrifices en ce genre, presque permanents, que les nécessités du salut social leur imposaient, on peut encore trouver la marque de leur esprit natif d’indépendance dans le rôle que le sentiment appelé par eux aussi \emph{l’amour de la patrie} jouait au milieu de leurs vertus politiques.\par
Cette passion, vive comme chez les nations helléniques, n’avait pas le même despotisme cassant. La délégation que la patrie faisait à la loi de ses pouvoirs donnait au culte des Romains pour cette divinité quelque chose de beaucoup plus régulier, de bien autrement grave, et, en somme, de plus modéré. La patrie régnait sans doute, mais ne gouvernait pas, et nul ne songeait, comme chez les Grecs, à justifier les caprices des factions, leurs énormités et leurs exactions en les couvrant de ce mot unique : la volonté de la patrie \footnote{Rien ne le montre mieux que la grande commotion civile qui porta les plébéiens à se retirer sur le mont Sacré, en laissant dans la ville les patriciens avec leurs clients et leurs esclaves. Toute cette affaire est admirablement exposée dans ses causes et sa conduite par Niebuhr. (\emph{Rœm. Geschichte}, t. I, p. 412.) C’est un des morceaux les plus remarquables qui aient jamais été écrits sur l’antiquité. L’élévation de la pensée, comme sa justesse, en donnant au style du grand historien une beauté inattendue, le fait échapper cette fois au jugement d’ailleurs équitable de M. Macaulay : « Niebuhr, a man who would have been the first « writer of his time. If his talent for communicating thoughts had borne any proportion to « his talent for investigating them. » (\emph{Lays of Ancient Rom.} Préface.)}. La loi, pour les Grecs, faite et défaite tous les jours, et constamment au nom du pouvoir supérieur, la loi n’avait ni prestige, ni autorité, ni force. Au contraire, à Rome, la loi ne s’abrogeait, pour ainsi dire, jamais ; elle était toujours vivante, toujours agissante, on la rencontrait partout, elle seule ordonnait, et, de fait, la patrie restait à son état d’abstraction, et n’avait pas le droit, bien que très honorée, de s’engouer tous les matins de quelque mauvais révolutionnaire nouveau comme cela n’avait lieu que trop souvent sur le Pnyx.\par
Il n’est rien de mieux, pour comprendre ce que c’était que \emph{l’omnipotence} de la loi dans la société romaine, que de voir le pouvoir des conventions augurales se perpétuer jusqu’à la fin de la république. Quand on lit qu’au temps de Cicéron, l’annonce d’un prodige météorologique suffisait encore pour faire rompre les comices et lever la séance, alors que les hommes politiques se moquaient non seulement des prodiges, mais des dieux même, on trouve là certainement un indice irrécusable d’un grand respect pour la loi, même jugée absurde \footnote{M. d’Erkstein (\emph{Recherches historiques sur l’humanité primitive}) a peint avec succès l’immobilité des idées romaines. Ses paroles s’adressent surtout à la religion, mais on peut sans difficulté en faire l’application à la loi. « Tandis que nous vivons, dit cet écrivain, dans « une plus ou moins heureuse inconséquence de nos œuvres et de nos pensées, les vieux « peuples poussaient l’esprit de conséquence souvent jusqu’aux dernières limites de « l’absurde... Seuls les Grecs ont pu s’affranchir jusqu’à un certain point de cette tyrannie « dans leurs temps religieux même ; jamais les Romains, esclaves absolus de leurs rites et « du forum sacré. » (p. 63.)}.\par
Les Romains furent ainsi le premier peuple d’Occident qui sut faire tourner au profit de sa stabilité, en même temps que de sa liberté, ces sortes de défauts de la législation qui sont ou organiques ou produits par les changements survenus dans les mœurs. Ils constatèrent qu’il y avait dans les constitutions politiques deux éléments nécessaires, l’action réelle et la comédie, vérité si bien reconnue et exploitée depuis par les Anglais. Ils surent pallier les inconvénients de leur système par leur patience à chercher et leur habileté à découvrir les moyens de paralyser les vices de la législation, sans toucher jamais à ce grand principe de vénération sans bornes dont ils avaient fait leur palladium, marque évidente d’une raison saine et d’une grande profondeur de jugement.\par
Enfin rien de tout ce qu’on pourrait accumuler d’exemples ne rendrait plus claires les différences de la liberté grecque et de la romaine que ce simple mot : les Romains étaient des hommes positifs et pratiques, les Grecs des artistes ; les Romains sortaient d’une race mâle, les Grecs s’étaient féminisés ; et c’est pourquoi les Romains Italiotes purent conduire leurs successeurs, leurs héritiers au seuil de l’empire du monde avec tous les moyens d’achever la conquête, tandis que les Grecs, au point de vue politique, n’eurent que la gloire d’avoir poussé la décomposition gouvernementale aussi loin qu’elle peut aller avant de rencontrer la barbarie ou la servitude étrangère.\par
Je reviens à l’examen de l’état du peuple de Rome, après l’expulsion des Étrusques, et à l’étude de ses destinées.\par
Les Sabins étaient, nous l’avons reconnu, la portion la plus nombreuse et la plus influente de cette nationalité de hasard. L’aristocratie sortait d’eux, et ce furent eux qui dirigèrent les premières guerres, Ils ne s’y épargnèrent pas ; cette justice leur est due \footnote{ \noindent XXXI\par
 
\begin{verse}
For Romans in Rome’s quarrel\\
 Spared neither land nor gold,\\
 Nor son, nor wife, nor limb, nor life,\\
 In the brave days of old.\\
\end{verse}
 \noindent XXXII\par
 
\begin{verse}
Then none was of a party ;\\
 Then all were for state, etc.\\
\end{verse}
 
\bibl{Macaulay’s Lays of Ancient Rom. Horatius}
}. En leur qualité de rameau kymrique, ils étaient naturellement hardis. Ils se portaient aisément aux entreprises militaires. Ils étaient très propres à présider aux périlleux travaux d’une république qui ne voyait guère autour de son territoire que des haines ou, à tout le moins, des malveillances.\par
On ne l’a pas oublié : les Romains, bien que de race italiote et sabine, étaient l’objet de la violente animadversion des tribus latines. Celles-ci ne trouvaient dans ce ramas de guerriers que des renégats de toutes les nationalités de la Péninsule, des gens sans foi ni loi, des bandits qu’il fallait exterminer, et d’autant plus détestables qu’ils étaient des proches parents. Tous ces peuples, ainsi animés, étaient sous les armes contre Rome, ou prêts à s’y mettre.\par
Autrefois, du temps des rois, la confédération étrusque avait constamment pris fait et cause pour sa colonie ; mais, depuis l’expulsion des Tarquiniens, l’amitié avait fait place à des sentiments tout différents \footnote{« Les Tarquiniens semblent avoir même un moment rallié contre les Romains, renégats de l’Étrurie, jusqu’aux villes libérales : Clusium, par exemple.  – Liv., I : « Incensus « Tarquinius non dolore solum tantæ ad irritum cadentis spei, sed etiam odio iraque... « bellum aperte moliendum ratus, circumire supplex Etruriæ urbes ; orare maxime Veientes « Tarquiniensesque, ne se ortum ejusdem sanguinis... perire sinerent. »}. Ainsi, n’ayant pas plus d’alliés sur la rive droite du Tibre que sur la rive gauche, Rome, malgré son courage, eût succombé, si la diversion la plus heureuse n’avait été faite en sa faveur par des masses puissantes qui, certes, ne songeaient pas à elle ; et ici vient se placer une de ces grandes périodes de l’histoire que les interprètes religieux des annales humaines, tels que Bossuet, ont coutume de considérer avec un saint respect comme le résultat admirable des longues et mystérieuses combinaisons de la Providence.\par
Les Galls d’au delà des Alpes, faisant un mouvement agressif hors de leur territoire, inondèrent tout à coup le nord de l’Italie, asservirent le pays des Umbres, et vinrent présenter la bataille aux Étrusques \footnote{O. Muller, \emph{ouvr. cité}, p. 165.  – Cet auteur fait très bien ressortir la nécessité où se trouvèrent les Étrusques, par suite de l’invasion gallique, de tolérer les agrandissements de Rome. Il les montre forcés de laisser prendre Véies, de voir, sans y parvenir, la soumission des Sabins, des Latins et des Osques, et cependant servant de rempart à ce cruel rival contre les ennemis qui les dévoraient eux-mêmes.}.\par
Les ressources diminuées de la confédération rasène suffirent à peine à résister à des antagonistes si nombreux, et Rome, quitte de son principal adversaire, prit autant de loisirs qu’il lui en fallut pour répondre à ses ennemis de la rive gauche.\par
 Elle réussit : elle les abaissa. Puis, lorsque de ce côté ses armes lui eurent assuré, non seulement le repos, mais la domination, elle mit à profit les embarras inextricables où les efforts des Galls plongeaient ses anciens maîtres, et, les prenant à dos, remporta sur eux des triomphes qui, sans cette circonstance, eussent probablement été mieux disputés et fort incertains.\par
Tandis que les Étrusques, culbutés dans le nord par les agresseurs sortis de la Gaule, fuyaient en bandes effarées jusqu’au fond de la Campanie \footnote{O. Muller, \emph{ouvr. cité}, p. 162.}, l’armée romaine, avec toute son ordonnance et son attirail jadis imités de ses victimes d’aujourd’hui, passait le fleuve et faisait sa main sur ce qui lui convenait. Elle n’était pas l’alliée des Gaulois, heureusement, car, n’ayant pas à partager le butin, elle le gardait tout entier ; mais elle combinait de loin ses entreprises avec les leurs, et, pour mieux assurer ses coups, ne les assenait qu’en même temps. Elle y trouva encore un autre profit.\par
Les Tyrrhéniens Rasènes, assaillis de toutes parts, défendirent leur indépendance aussi longtemps que faire se put. Mais, lorsque le dernier espoir de rester libres eut disparu pour eux, il leur fallut raisonnablement penser à quel vainqueur il valait mieux se rendre. Les Gaulois, on ne saurait trop insister sur cette vérité méconnue, n’avaient pas agi en barbares, car ils ne l’étaient pas. Après s’être abandonnés, dans la première ardeur de l’invasion, à saccager des cités umbriques, ils avaient à leur tour fondé des villes, comme Milan, Mantoue et autres \footnote{\emph{Ibid}., p. 139.}. Ils avaient adopté le dialecte des vaincus et, probablement, leur manière de vivre. Cependant, en somme, ils étaient étrangers au pays, avides, arrogants, brutaux. Les Étrusques espérèrent sans doute un sort moins dur sous la domination du peuple qui leur devait la vie. On vit donc des cités ouvrir aux consuls leurs citadelles, et se déclarer sujettes, quelquefois alliées, du peuple romain \footnote{Ibid., p. 128-130.  – Le dernier soupir de l’Étrurie indépendante fut recueilli par le consul Marcius Philippus, qui triompha en 471 de Rome. Cependant la nationalité se maintint jusqu’au temps de Sylla. Ce dictateur inonda le pays de colonies sémitisées. César continua, Octave acheva, et le sac de Pérouse mit le sceau à la dispersion de la race.}. C’était le meilleur parti à prendre. Le sénat, dans sa politique sérieuse et froide, eut longtemps la sagesse de ménager l’orgueil des nations soumises.\par
Une fois l’Étrurie annexée aux possessions de la république, comme les liaisons les plus voisines de Rome avaient, pendant ce temps, subi le même sort les unes après les autres le plus fort, le plus difficile du thème romain se trouva fait, et, quand l’invasion gauloise eut été rejetée loin des murs du Capitole, la conquête de la Péninsule tout entière ne fut plus qu’une question de temps pour les successeurs de Camille.\par
À la vérité, s’il avait alors existé dans l’Occident une nation énergique, issue de la race ariane, les destinées du monde eussent été différentes on eût vu bientôt les ailes de l’aigle tomber brisées ; mais la carte des États contemporains ne nous montre que trois catégories de peuples en situation de lutter avec la république.\par
1° Les Celtes,  – Brennus avait trouvé son maître, et ses bandes, après avoir dompté les Kymris métis de l’Umbrie et les Rasènes de l’Italie moyenne, avaient dû s’en tenir là. Les Celtes étaient divisés en trop de nations, et ces nations étaient chacune trop petites, pour qu’il leur fût loisible de recommencer des expéditions considérables. La migration de Bellovèse et de Sigovèse fut la dernière jusqu’à celle des Helvétiens au temps de César.\par
2° Les Grecs.  – Comme nationalité ariane, ils n’existaient plus depuis longtemps, et les brillantes armées de Pyrrhus n’auraient pas été en état de faire une trouée au milieu des redoutables bandes kymriques vaincues par les Romains. Que prétendre contre les Italiotes ?\par
3° Les Carthaginois.  – Ce peuple sémitique, appuyé sur l’élément noir, ne pouvait, dans aucune supposition, prévaloir contre une quantité moyenne de sang kymrique.\par
La prépondérance était donc assurée aux Romains. Ils n’auraient pu la perdre que si leur territoire, au lieu d’être situé dans l’occident du monde, les avait faits voisins de la civilisation brahmanique d’alors, ou, encore, s’ils avaient eu déjà sur les bras les populations germaniques qui ne vinrent qu’au V\textsuperscript{e} siècle.\par
Tandis que Rome marchait ainsi à la rencontre d’une gloire immense en s’appuyant sur la force respectée de ses constitutions, les crises les plus graves s’accomplissaient dans son enceinte, je ne dirai pas sans violences matérielles, car il y en eut beaucoup, mais sans destruction des lois. L’émeute triomphante ne fit jamais que modifier, et jamais ne renversa l’édifice légal de fond en comble, de telle sorte que ce patriciat si odieux à la plèbe, dès le lendemain de l’expulsion des Étrusques subsista jusque sous les empereurs, constamment détesté, constamment attaqué, affaibli par de perpétuelles atteintes, mais point assassiné : la loi ne le souffrait pas \footnote{Je n’ai pas besoin d’ajouter que le patriciat subsista, mais non pas les races nobles sabines, sauf un bien petit nombre. Elles furent graduellement remplacées par des familles plébéiennes. Sous Tibère, Gallus pouvait dire avec vérité dans le Sénat : « Distinctos « senatus et equitum census, \emph{non quia diversi natura}, sed ut locis, ordinibus, dignationibus « antistent et aliis quæ ad requiem animi sur salubritatem corpurum parentur. » (Tacit., \emph{Ann.}, II, 33.)}.\par
Ces luttes, ces querelles avaient pour causes véritables les modifications ethniques subies sans cesse par la population urbaine, et pour modérateur la parenté plus ou moins lointaine de tous les affluents ; autrement dit, les institutions se modifiaient parce que la race variait, mais elles ne se transformaient pas du tout au tout, elles ne passaient pas d’un extrême à l’autre, parce que ces variations de race, n’étant encore que relatives, tournaient à peu près dans le même cercle. Ce n’est pas à dire que les oscillations perpétuelles ainsi entretenues dans l’État ne fussent pas senties ni comprises. Le patriciat se rendait parfaitement compte du tort que les incessantes adjonctions d’étrangers causaient à son influence, et il prit pour maxime fondamentale de s’y opposer autant que possible, tandis que le peuple, au contraire, également éclairé sur ce qu’il gagnait en nombre, en richesses, en savoir, à tenir grandes ouvertes les portes de la cité devant des nouveaux venus qui, repoussés par la noblesse, n’avaient rien à faire qu’à s’adjoindre à lui, le peuple, la plèbe, se montra partisan déclaré des gens du dehors \footnote{Amédée Thierry,\emph{ Hist. de la Gaule sous l’admin. rom}., t. I, p. 3.}. Elle aspira toujours à les attirer, et rendit ainsi éternel le principe qui avait jadis fortifié la cité naissante, et qui consistait à inviter au festin de ses grandeurs tous les vagabonds du monde connu \footnote{« Ne vana urbis magnitudo esset, adficiendæ multitudinis causa... locum qui nunc septus « descendentibus inter duos lucos est, \emph{Asylum} aperit. Eo ex finitimis populis, turba omnis, « sine discrimine, liber an serves esset, avida novarum rerum perfugit. » (Liv., I) L’horreur que les gens de tous les ordres prirent de très bonne heure pour le mariage régulier ne contribua guère moins que la guerre à détruire la population de souche italiote. En 131 avant J.-C., Q. Métellus Macédonicus, censeur, porte plainte aux sénateurs, et un décret engage les citoyens à renoncer au célibat. Ce ne fut pas le seul effort de la loi ; et aucun n’eut de succès. (Zumpt, \emph{ouvr. cité}, p. 25.) Il faut encore tenir compte de l’usage qui permettait aux parents d’exposer leurs enfants, cause puissante de dépopulation.}. Comme l’univers d’alors était infirme, Rome ne pouvait manquer de devenir la sentine de toutes les maladies sociales \footnote{En principe, des citoyens seuls pouvaient entrer dans les légions. Lors de la seconde guerre punique, on y admit des affranchis. Marius y reçut indistinctement tous les prolétaires. (Zumpt, \emph{ouvr. cité}, p. 23 et 27.)}.\par
Cette soif immodérée d’agrandissement aurait paru monstrueuse dans les villes grecques, car il en résultait de terribles atteintes aux doctrines d’exclusivité de la patrie \footnote{Denys d’Halicarnasse fait ressortir la différence des points de vue hellénique et romain, et donne, comme de juste chez un homme de son temps, toute louange et tout avantage à la méthode qui lui avait conféré à lui-même son rang de citoyen. (\emph{Antiq. Rom}., 2, XVII.)}. Des multitudes toujours offrant, toujours prêtes à conférer le droit de cité à qui le souhaitait, n’avaient pas un patriotisme jaloux. Les grands historiens des siècles impériaux, ces panégyristes si fiers des temps anciens et de leurs mœurs, ne s’y trompent nullement. Ce qu’ils célèbrent dans leurs mâles et emphatiques périodes sur l’antique liberté, c’est le patricien romain, et non pas jamais l’homme de la plèbe \footnote{Il ne faut pas s’y méprendre lorsqu’on lit dans Tacite : « Igitur, verso civitatis statu, nihil « usquam prisci et integri moris : omnes, exuta æqualitate, jussa principis adspectare. » (\emph{Ann.}, I. t. 4.) Cette égalité, c’est l’égalité patricienne qui n’a que des inférieurs et pas de maîtres.}. Lorsqu’ils parlent avec adoration de ce citoyen vénérable dont les années se sont écoulées à servir l’État, qui porte sur son corps les cicatrices de tant de batailles gagnées contre les ennemis de la majesté romaine, qui a sacrifié non seulement ses membres, mais sa fortune, celle de sa famille, et quelquefois ses enfants, et, quelquefois même, a tué ses fils de sa propre main pour un manquement aux lois austères du devoir civique ; lorsqu’ils représentent cet homme des anciens âges, honoré jadis de la robe triomphale, une ou deux fois consul, questeur, édile, sénateur héréditaire, et préparant, de cette même main qui ne trouva jamais trop lourdes l’épée et la lance, les raves de son souper \footnote{ 
\begin{verse}
Gratus insigni referam Camœna,\\
 Fabriciumque\\
 Hunc, et incomptis Curtium capillis,\\
 Utilem bello tulit, et Camillum,\\
 Sæva paupertas, et avitus apto\\
 Cum lare fundus.\\
\end{verse}
 
\bibl{Hor., \emph{Od.} I, 12, 39.}
}, puis, avec cette rectitude de jugement, cette froide raison si utile à la république, calculant les intérêts de ses prêts usuraires, d’ailleurs méprisant les arts et les lettres, et ceux qui les cultivent, et les Grecs qui les aiment : ce vieillard, cet homme vénérable, ce citoyen idéal, ce n’est jamais qu’un patricien, qu’un vieux sabin. L’homme du peuple est, au contraire, ce personnage actif, hardi, intelligent, rusé, qui, pour renverser ses chefs, cherche d’abord à leur enlever le monopole judiciaire, y parvient, non pas par la violence, mais par l’infidélité et le vol ; qui, exaspéré de l’énergique résistance des nobles, prend enfin le parti, non de les attaquer, la loi ne le veut pas, et il faudrait les tuer tous sans espoir d’en faire céder un seul, mais le parti de s’en aller pour ne revenir qu’après avoir commenté avec profit la fable des \emph{membres et de l’estomac.} Le plébéien romain, c’est un homme qui n’aime pas la gloire autant que le profit \footnote{Il ne faut pas perdre de vue un seul instant, quand il s’agit de la Rome italiote, l’esprit profondément utilitaire de sa population. Les lois concernant les débiteurs, l’usure, le partage du butin et des terres conquises, voilà le fond, voilà l’essentiel de ses constitutions et les causes réelles de plus d’une de ses agitations politiques. (Niebuhr, \emph{Rœm. Geshichte}, t. I, p. 394 et pass. ; t. II, 22, 231, 310, etc.)}, et la liberté autant que ses avantages ; c’est le préparateur des grandes conquêtes, des grandes adjonctions par l’extension du droit civique aux villes étrangères ; c’est, en un mot, le politique pratique qui comprendra plus tard la nécessité du régime impérial, et se trouvera heureux de le voir éclore, échangeant volontiers l’honneur de se gouverner, et le monde avec soi, pour les mérites plus solides d’une administration mieux ordonnée. Les écrivains à grands sentiments n’ont jamais eu la moindre intention de louer ce plébéien toujours égoïste au milieu de son amour pour l’humanité, et si médiocre dans ses grandeurs.\par
Tant que le sang italiote, ou même gaulois, ou, encore, celui de la Grande-Grèce, se trouvèrent seuls à satisfaire les besoins de la politique plébéienne, en affluant dans Rome et dans les villes annexées, la constitution républicaine et aristocratique ne perdit pas ses traits principaux. Le plébéien d’origine sabine ou samnite désirait l’agrandissement de son rôle sans vouloir abroger complètement le régime du patriciat, dont ses idées ethniques sur la valeur relative des familles, dont ses doctrines raisonnables en matière de gouvernement lui faisaient apprécier les irremplaçables avantages. La dose de sang hellénique qui se glissait dans cet amalgame avivait le tout, et n’avait pas encore réussi à le dominer.\par
Après le coup d’éclat qui termina les guerres puniques, la scène changea. L’ancien sentiment romain commença à s’altérer d’une manière notable : je dis s’altérer, et non plus se modifier. Au sortir des guerres d’Afrique, vinrent les guerres d’Asie. L’Espagne était déjà acquise à la république. La Grande-Grèce et la Sicile tombèrent dans son domaine, et ce que l’hospitalité intéressée du parti plébéien \footnote{Am. Thierry, \emph{la Gaule sous l’administration romaine, Introduct.}, t. I, p. 62 : « Il serait « injuste, sans doute, de faire peser sur les hommes du parti patricien tout l’odieux de ces « abominables excès (les rapines de Verrès et de ses pareils). Le parti populaire ne « possédait assurément ni tant de désintéressement ni tant de vertu ; mais, comme les « accusations contre les vols publics et les réclamations en faveur des provinciaux « sortirent presque toujours de ses rangs, comme il promettait beaucoup de réformes, que « l’appui qu’il avait prêté aux Italiens avant et depuis la guerre sociale inspirait confiance « en sa parole, les provinces s’attachèrent à lui. Elles lui rendirent promesses pour « promesses, espérance pour espérance. Il se forma entre elles et les agitateurs des « derniers temps de la république des liens analogues à ceux qui avaient, un siècle « auparavant, compromis les alliés latins dans les entreprises des Gracques. On peut se « rappeler avec quel héroïsme l’Espagne adopta et défendit de son sang les derniers chefs « du parti de Marius. Catilina lui-même parvint à enrôler sous son drapeau la province « gauloise cisalpine, et déjà il entraînait quelques parties de la transalpine, réduites aussi « en province. »  – Le parti démocratique à Rome, outre qu’il tendait essentiellement à la destruction de la forme républicaine, résultat qu’il obtint, était aussi avec ferveur ce que la phraséologie moderne appellerait \emph{le parti de l’étranger.}} fit désormais affluer dans la ville, ce ne fut plus du sang celtique plus ou moins altéré, mais des éléments sémitiques ou sémitisés. La corruption s’accumula à grands flots. Rome, entrant en communion étroite avec les idées orientales, augmentait, avec le nombre de ses éléments constitutifs, la difficulté déjà grande de les amalgamer jamais. De là, ten­dances irrésistibles à l’anarchie pure, au despotisme, à l’énervement, et, pour conclure, à la barbarie ; de là, haine chaque jour mieux prononcée pour ce que le gouvernement ancien avait de stable, de conséquent et de réfléchi.\par
Rome sabine avait été marquée, vis-à-vis de la Grèce, d’une originalité tranchée dans sa physionomie ; désormais ses idées, ses mœurs, perdent graduellement cette empreinte. Elle devient à son tour hellénistique, comme jadis la Syrie, l’Égypte, bien qu’avec des nuances particulières. Jusqu’alors, bien modeste dans toutes les choses de l’esprit, quand ses armes commandaient aux provinces, elle s’était souvenue avec déférence que les Étrusques étaient la nation cultivée de l’Italie, et elle avait persisté à apprendre leur langue, à imiter leurs arts, à leur emprunter savants et prêtres, sans s’apercevoir que, sur beaucoup de points, l’Étrurie répétait assez mal la leçon des Grecs, et d’ailleurs que les Grecs eux-mêmes traitaient de suranné et de hors de mode ce que les Étrusques continuaient à admirer sur la foi des modèles anciens. Graduellement Rome ouvrit les yeux à ces vérités, elle renia ses antiques habitudes vis-à-vis des descendants asservis de ses fondateurs. Elle ne voulut plus entendre parler de leurs mérites, et prit un engouement de parvenue pour tout ce qui se taillait, se sculptait, s’écrivait, se pensait ou se disait dans le fond de la Méditerranée. Même au siècle d’Auguste, elle ne perdit jamais, dans ses rapports avec la Grèce dédaigneuse, cette humble et niaise attitude du provincial devenu riche qui veut passer pour connaisseur.\par
Mummius, vainqueur des Corinthiens, expédiait tableaux et statues à Rome en signifiant aux voituriers qu’ils auraient à remplacer les chefs-d’œuvre endommagés sur la route. Ce Mummius était un vrai Romain : un objet d’art n’avait pour lui que le prix vénal Saluons ce digne et vigoureux descendant des confédérés d’Amiternum. Il n’était pas dilettante, mais avait la vertu romaine, et on ne riait que tout bas dans les villes grecques qu’il savait si bien prendre.\par
Le latin, jusqu’alors, avait gardé une forte ressemblance avec les dialectes osques \footnote{Le livre de Meier présente cette vérité dans un jour vraiment frappant. (Voir Meier, \emph{Lateinische Anthologie}.)}. Il inclina davantage vers le grec, et si rapidement qu’il varia presque avec chaque génération. Il n’y a peut-être pas d’exemple d’une mobilité aussi extrême dans un idiome, comme il n’y en a pas non plus d’un peuple aussi constamment modifié dans son sang. Entre le langage des Douze Tables et celui que parlait Cicéron, la différence était telle que le savant orateur ne pouvait s’y reconnaître. Je ne parle pas des chants sabins, c’était encore pis. Le latin, depuis Ennius, tint à honneur de mettre en oubli ce qu’il avait d’italique.\par
 Ainsi, pas de langue vraiment et uniquement nationale, un engouement de plus en plus prononcé pour la littérature, les idées d’Athènes et d’Alexandrie, des professeurs helléniques, des maisons à l’asiatique, des meubles syriens, le dédain profond des usages locaux : voilà ce qu’était devenue la ville qui, ayant commencé par la domi­nation étrusque, avait grandi sous l’oligarchie sabine : le moment de la démocratie sémitique n’était pas loin désormais.\par
La foule entassée dans les rues s’abandonnait tout entière à l’étreinte de cet élément. L’âge des institutions libres et de la légalité allait se clore. L’époque qui succéda fut celle des coups d’État violents, des grands massacres, des grandes perver­sités, des grandes débauches. On se croit transporté à Tyr, aux jours de sa décadence ; et en effet, avec un plus grand espace aréal, la situation est pareille : un conflit des races les plus diverses ne pouvant parvenir à se mélanger, ne pouvant se dominer, ne pouvant pas transiger, et n’ayant de choix possible qu’entre le despotisme et l’anarchie.\par
Dans de pareils moments, les douleurs publiques trouvent souvent un théoricien illustre pour les comprendre et pour inventer un système supposé capable d’y mettre fin. Tantôt cet homme bien intentionné n’est qu’un simple particulier. Il ne devient alors qu’un écrivain de génie : tel fut, chez les Grecs, Platon. Il chercha un remède aux maux d’Athènes, et offrit, dans une langue divine, un résumé de rêveries admirables. D’autres fois, ce penseur se trouve, par sa naissance ou par les événements, placé à la tête des affaires. Si, attristé d’une situation tellement désastreuse, il est d’un naturel honnête, il voit avec trop d’horreur les maux et les ruines accumulés sous ses pas pour accepter l’idée de les agrandir encore, il reste impuissant. De telles gens sont médecins, non chirurgiens, et, comme Épaminondas et Philopœmen, ils se couvrent de gloire sans rien réparer.\par
Mais il apparut une fois, dans l’histoire des peuples en décadence, un homme mâlement indigné de l’abaissement de sa nation, apercevant d’un coup d’œil perçant, à travers les vapeurs des fausses prospérités, l’abîme vers lequel la démoralisation générale traînait la fortune publique, et qui, maître de tous les moyens d’agir, nais­sance, richesses, talents, illustration personnelle, grands emplois, se trouva être, en même temps, fort d’un naturel sanguinaire, déterminé à ne reculer devant aucune ressource. Ce chirurgien, ce boucher, si l’on veut, ce scélérat auguste, si on le préfère, ce Titan, se montra dans Rome au moment où la république, ivre de crimes, de domination et d’épuisement triomphal, rongée par la lèpre de tous les vices, s’en allait roulant sur elle-même et vers l’abîme. Ce fut Lucius Cornélius Sylla.\par
Véritable patricien romain, il était pétri de vertus politiques \footnote{Dion. Cass., \emph{Hist. rom.}, Hamb. (alphabet étranger), in-fol., t. (, p. 47, fragm. CXVII : (phrase grecque) – Dion Cassius est un écrivain très démocratique et fort ennemi du dictateur.}, vide de vertus privées ; sans peur pour lui, pour les autres ; pour les autres pas plus que pour lui, il n’avait de faiblesse. Un but à saisir, un obstacle à écarter, une volonté à réaliser, il n’apercevait rien en dehors. Ce qu’il fallait briser de choses ou d’hommes pour faire pont n’entrait pas dans ses calculs. Arriver, c’était tout, et, après, reprendre l’essor.\par
Les dispositions impitoyables de son sang, de sa race, s’étaient d’ailleurs fortifiées à l’odieux contact de ce soldat que, dans la personne bestiale de Marius, le parti populaire opposait à ses desseins.\par
Sylla n’était pas allé chercher dans les théories idéales le plan du régime régéné­rateur qu’il se proposait d’imposer. Il voulait simplement restaurer en son entier la domination patricienne, et, par ce moyen, rendre l’ordre avec la discipline à la république raffermie. Il s’aperçut bientôt que le plus difficile n’était pas de mettre en déroute les émeutes ou même les armées plébéiennes, mais bien de trouver une aristocratie digne de la grande tâche qu’il voulait lui livrer. Il lui fallait des Fabius, il lui fallait des Horaces ; il eut beau les appeler, il ne les fit pas sortir de ces maisons luxueuses où résidaient leurs images, et, comme il ne reculait devant rien, il voulut recréer les nobles qu’il ne trouvait plus.\par
On le vit alors, plus redoutable à ses amis qu’à ses rivaux, tailler et retailler d’un bras impitoyable l’arbre de la noblesse romaine. Pour rendre la virilité à un corps appauvri, il fit tomber les têtes par centaines, ruina, exila ceux qu’il ne mit pas à mort, et traita avec la dernière férocité bien moins les gens de la plèbe, francs ennemis, que les grands, obstacles directs de ses desseins par leur impuissance à les servir. À force de receper le vieux tronc, il s’imaginait en tirer des bourgeons nouveaux, porteurs d’autant de suc que ceux d’autrefois. Il espérait qu’après avoir élagué les branches indignes, il réussirait, à force d’effrayer, à faire des braves, et qu’ainsi la démocratie recevrait de sa main, pour être matée à jamais, des chefs inflexibles et des maîtres résolus.\par
Il serait dur d’avoir à reconnaître que de tels moyens se soient trouvés bons. Lui-même il cessa de le croire. Au bout d’une longue carrière, après des efforts dont l’intensité se mesure aux violences qu’ils accumulèrent, Sylla, désespérant de l’avenir, triste, épuisé, découragé, déposa de lui-même la hache de la dictature, et, se résignant à vivre inoccupé au milieu de cette population patricienne ou plébéienne que sa vue seule faisait encore frémir, il prouva du moins qu’il n’était pas un ambitieux vulgaire, et qu’ayant reconnu l’inanité de ses espérances, il ne tenait pas à garder un pouvoir stérile. Je n’ai pas d’éloges à donner à Sylla, mais je laisse à ceux que ne frappe pas d’une respectueuse admiration le spectacle d’un tel homme, échouant dans une telle entreprise, le soin de lui reprocher ses excès.\par
Il n’y avait pas moyen qu’il réussît. Le peuple qu’il voulait ramener aux mœurs et à la discipline des vieux âges ne ressemblait en rien au peuple républicain qui les avait pratiquées. Pour s’en convaincre, il suffit de comparer les éléments ethniques des temps de Cincinnatus à ceux qui existaient à l’époque où vécut le grand dictateur.\par

\tableopen{}
\begin{tabularx}{\linewidth}
{|l|X|X|X|X|X|}
\hline\multicolumn{3}{l}{ 
\begin{center}
\noindent temps de cicinnatus\par
\end{center}

 
\begin{center}
\noindent  
\end{center}

 } & \multicolumn{3}{l}{temps de sylla} \\
\hline
 \noindent Aristocratie
  &  \noindent Sabins, en majorité\par
 Quelques Étrusques\par
 Quelques Italiotes
  &  \noindent 1° Majorité\par
 métisse de blanc et de jaune ;
  &  \noindent  \par
  Arist.
  &  \noindent  Italiotes mêlés de sang hellénique*\par
 Italiotes
  &  \noindent 1° Majorité sémitisée;\par
 2° Minorité ariane ;
  \\
\hline
 & Sabins. &  &  & Grecs de la Grande- &  \\
\hline
 & Samnites. &  &  & Grèce et de la Sicile &  \\
\hline
Plèbe &  \noindent Sabelliens.\par
 Sicules\par
 Quelques Hellènes.
  & 2° Très faible apport sémitique. & Plèbe &  \noindent Hellénistes d’Asie.\par
 Sémites d’Asie.\par
 Sémites d’Afrique.\par
 Sémites d’Espagne.
  & 3° Subdivi-sion extrême du principe jaune. \\
\hline
\end{tabularx}
\tableclose{}

\noindent * Quand, sous Néron, il fut question au Sénat de restreindre les droits des affranchis, on rencontra beaucoup d’oppositions basées sur des raisons très dignes d’être rapportées ici comme aveux complets de la part des patriciens : « Disserebatur contra paucorum culpam « ipsis exitiosam esse debere, nihil universorum juri derogandum ; quippe late fusum id « corpus ; hinc plerumque tribus, decurias, ministeria magistratibus et sacerdotibus, « cohortes etiam in urbe conscriptas ; et plurimis equitum, plerisque senatoribus, non « aliunde originem trahi. Si separarentur libertini, manifestata fore penuriam ingenuorum. » (Tac., \emph{Ann}., XIII, 27.) Déjà du temps de Cicéron, l’usage s’était introduit d’affranchir un esclave après six ans de bons services et de bonne conduite. À dater de la même époque, un Romain de la classe riche se faisait un devoir en mourant de donner la liberté à toute sa maison, et l’opinion publique considérait cet acte comme une affaire de conscience. (Zumpt, \emph{loc. cit.}, p. 30.) Il me semble bien difficile de ne pas conclure de ces faits que la décadence de l’esclavage dans tout pays est correspondante à la confusion des races, et résulte directement de la parenté de plus en plus proche entre les maîtres et les serviteurs.\par
Impossible de ramener dans un même cadre deux nations qui, sous le même nom, se ressemblaient si peu \footnote{Denys d’Halicarnasse rend très bien compte de cette situation et de ses conséquences : (paragraphe en grec) \emph{Antiq. Rom.},I, LXXXIX.)}. Toutefois l’équité n’est pas aussi sévère pour l’œuvre de Sylla que le fut son auteur. Le dictateur eut raison de perdre courage, car il compara son résultat à ses plans. Il n’en avait pas moins donné au patriciat une vigueur factice, renforcée, il est vrai, par la terreur qui paralysait le parti contraire, et la république lui dut plusieurs années d’existence qu’elle n’aurait pas eues sans lui. Après la mort du réformateur, l’ombre cornélienne protégea encore quelque temps le sénat. Elle se dressait derrière Cicéron, lorsque ce rhéteur, devenu consul, défendait si maigrement la cause publique contre les audaces emportées des factions. Sylla réussit donc à entraver la course qui entraînait Rome vers d’incessantes transformations. Peut-être, sans lui, l’époque qui s’écoula jusqu’à la mort de César n’aurait-elle été qu’un enchaî­nement bien plus lamentable encore de proscriptions et de brigandages, qu’une lutte perpétuelle entre des Antoines et des Lépides prématurés, écrasés dans l’œuf par sa farouche intervention.\par
Voilà la part à lui faire ; mais il est incontestable que le plus terrible génie ne peut arrêter bien longtemps l’action des lois naturelles, pas plus que les travaux de l’homme ne sauraient empêcher le Gange de faire et de défaire les îles éphémères dont ce fleuve peuple son lit spacieux \footnote{Niebuhr s’indigne contre les écrivains modernes qui, prétendant signaler, au VII\textsuperscript{e} siècle de Rome, l’existence de factions patriciennes dans cet État, oublient ou ignorent que Sylla fut la dernière expression légitime de cet ordre d’idées. (Niebuhr, \emph{Rœm. Geschichte}, t. I, p. 375.)}.\par
Il s’agit maintenant de contempler Rome avec la nouvelle nationalité que les alluvions ethniques lui ont donnée. Voyons ce qu’elle devint quand un sang de plus en plus mêlé lui eut imprimé avec un nouveau caractère une nouvelle direction.
\section[{V.7. Rome sémitique.}]{V.7. \\
Rome sémitique.}
\noindent Depuis la conquête de la Sicile jusqu’assez avant dans les temps chrétiens, l’Italie n’a pas cessé de recevoir de nombreux, d’innombrables apports de l’élément sémitique, de telle façon que le sud entier fut hellénisé et que le courant des races asiatiques remontant vers le nord ne s’arrêta que devant les invasions germaniques \footnote{Les dernières immigrations hellénistiques dans le royaume de Naples, la Sicile, la basse Italie byzantines et arabes. En 1461, 1532 et 1744, il vint encore des Albanais en Sicile et en Calabre.}. Mais le mouvement de recul, le point où s’arrêtèrent les alluvions du sud dépassa Rome. Cette ville alla toujours perdant son caractère primitif. Il y eut gradation sans doute dans cette déchéance, jamais temps d’arrêt véritable. L’esprit sémitique étouffa sans rémission son rival. Le génie romain devint étranger au premier instinct italiote, et reçut une valeur où l’on reconnaît bien aisément l’influence asiatique.\par
Je ne mets pas au nombre des moins significatives manifestations de cet esprit importé la naissance d’une littérature marquée d’un sceau particulier, et qui mentait à l’instinct italiote déjà par cela seul qu’elle existait.\par
Ni les Étrusques, je l’ai dit, ni aucune tribu de la Péninsule, pas plus que les Galls, n’avaient eu de véritable littérature ; car on ne saurait appeler ainsi des rituels, des traités de divination, quelques chants épiques servant à conserver les souvenirs de l’histoire, des catalogues de faits, des satires, des farces triviales dont la malignité des Fescennins et des Atellans amusaient les rires des désœuvrés. Toutes ces nations utilitaires, capables de comprendre au point de vue social et politique le mérite de la poésie, n’y avaient pas de tendance naturelle, et, tant qu’elles n’étaient pas fortement modifiées par des mélanges sémitiques, elles manquaient des facultés nécessaires pour rien acquérir dans ce genre \footnote{ 
\bibl{Dyon. Halicarn., \emph{Antiq. Rom.}, (phrase en grec).}
 \noindent – Sans me faire le champion de la confiance vaniteuse d’Ennius dans son propre mérite, je suis tout disposé à croire avec lui qu’avant le temps où il se mit à écrire, en cherchant l’imitation des chefs-d’œuvre grecs, il y avait des chants, mais pas de poésie dans le Latium : « Quum neque « Musarum scopulos quisquam superarat, nec dicti studiesus erat. »
}. Ainsi ce ne fut que lorsque le sang hellénistique domina les anciens alliages dans les veines des Latins, que la plèbe la plus vile, ou de la bourgeoisie la plus humble, exposées surtout à l’action des apports sémitisés, sortirent les plus beaux génies qui ont fait la gloire de Rome. Certes, Mucius Scévola aurait tenu en bien petite estime l’esclave Plaute, le Mantouan Virgile, et Horace, Vénusien, l’homme qui jetait son boucher à la bataille et en racontait l’anecdote pour faire rire Pompéius Varus \footnote{ 
\begin{verse}
Tecum Philippes et celerem fugam\\
 Sensi, relicta non bene parmula,\\
 Quum fracta virtus et minaces\\
 Turpe solum tetigere mento.\\
\end{verse}
 
\bibl{Hor., \emph{Od}., II, 7, 9.}
}. Ces hommes étaient de grands esprits, mais non pas des Romains, à parler chimie.\par
Quoi qu’il en soit, la littérature naquit, et avec elle une bonne part, sans contredit, de l’illustration nationale, et la cause du bruit qu’a fait le reste ; car on ne disconviendra pas que la masse sémitisée d’où sont sortis les poètes et les historiens latins dût à son impureté seule le talent d’écrire avec éloquence, de sorte que ce sont les doctes emphases des bâtards collatéraux qui nous ont mis sur la voie d’admirer les hauts faits d’ancêtres qui, s’ils avaient pu réviser et consulter leurs généalogies, n’auraient rien eu de plus pressé à faire que de renier ces respectueux descendants \footnote{Voir, sur la richesse des annales latines, et la différence existant entre elles et les histoires grecques, Niebuhr, \emph{Rœm. Geschichte}, t. II, p. 1 et pass.  – La méthode hellénique offre la transition des épopées hindoues et persanes, complètement nulles sous le rapport de la chronologie et de l’exactitude matérielle, aux fastes italiotes, qui n’avaient, au contraire, que ces deux qualités.}.\par
Avec les livres, le goût du luxe et de l’élégance étaient de nouveaux besoins qui témoignaient aussi des changements survenus dans la race. Caton les dédaignait, mais il y mettait de l’affectation. N’en déplaise à la gloire de ce sage, les prétendues vertus romaines dont il se parait étaient plus consciencieuses encore chez les antiques patriciens, et toutefois plus modestes \footnote{Polybe rend justice entière à l’avarice sordide de l’esprit romain : (phrase en grec) (\emph{Fragm}., libr. XXXII c. 12.)}. De leur temps, il n’était pas besoin d’en faire parade pour se singulariser ; tout le monde était sage à leur manière. Au contraire, après avoir reçu le sang de mères orientales et d’affranchis grecs ou syriens, le marchand, devenu chevalier, riche de son trafic ou de ses extorsions, ne comprenait rien, pour sa part, aux mérites de l’austérité primitive, Il voulait jouir en Italie de ce que ses ancêtres méridionaux avaient créé chez eux, et il l’y transportait. Il poussa du pied sous sa table le banc de bois où s’était assis Dentatus ; il remplaça de telles misères par des lits de citronnier incrustés de nacre et d’ivoire. Il lui fallut, comme aux satrapes de Darius, des vases d’argent et d’or pour contenir les vins précieux dont se repaissait son intempérance, et des plats de cristal pour servir les sangliers farcis, les oiseaux rares, les gibiers exotiques que dévorait sa fastueuse gloutonnerie. Il ne se contenta plus, pour ses demeures particulières, des constructions que les gens d’autre­fois eussent trouvées assez splendides pour héberger les dieux ; il voulut des palais immenses avec des colonnades de marbre, de granit, de porphyre, des statues, des obélisques, des jardins, des basses-cours, des viviers \footnote{« Quid enim premium prohibere et pliscum ad morem recidere aggrediar ? Villarumne « infinita spatia ? familiarum numerum et nationes ? argenti et auri pondus ? æris « tabularumque miracula ? » (Tac., \emph{Ann.}, III, 53.)}, et, au milieu de ce luxe, afin d’animer l’aspect de tant de créations pittoresques, Lucullus faisait circuler des multi­tudes d’esclaves désœuvrés, d’affranchis et de parasites dont la servilité bassement intéressée n’avait rien de commun avec le dévouement martial et la sérieuse dépendance des clients d’un autre âge.\par
Mais, au milieu de ce débordement de splendeurs, persistait une souillure singu­lière qui, pour l’opinion même des contemporains, s’attachait à tout, enlaidissait tout. La gloire et la puissance, le pouvoir de faire des profusions et la volonté de s’y abandonner appartenaient, la plupart du temps, à des gens inconnus la veille \footnote{Am. Thierry, \emph{la Gaule sous l’adm. rom. Introd}., t. I, p. 145.}. On ne savait d’où sortaient tant d’opulents personnages \footnote{ \noindent « Petron., \emph{Satyr.}, XXXVII : « Uxor, inquit, Trimalchionis, Fortunata appellatur, quæ « nummos modio metitur. »  – « Ipse nescit quid habeat adeo zaplutus (mot grec) est. »\par
 – « Argentum in hostiarii illius plus jacet quam quisquam in fortunis habet. Familia vero « babæ ! babæ ! non me hercules puto decumam partem esse quæ dominum suum novit, « etc., etc. » – XXXVIII : « Reliquos autem collibertos ejus cave contemnas, valde succosi « sunt. Vides illum qui in imo imus recumbit ? Hodie sua octingenta possidet ; de nihilo « crevit ; solebat collo modo suo ligna portare. »
}, et tour à tour, soit que ce fussent les flatteurs ou les envieux qui parlassent, on prêtait à Trimalcion la plus illustre ou la plus immonde origine \footnote{Am. Thierry, \emph{ibid.}, t. I, p. 208 : « Cette nouvelle société qui se formait alors, et qui, en « Italie, depuis la guerre sociale, ne se recrutait plus que parmi les affranchis. Il n’y a rien « d’étonnant à ce que des hommes de cette étoffe répétassent volontiers avec Trimalcion : « Amici et servi homines sunt, et æque unum lactem biberunt. » (Petron., \emph{Satyr.}, LXXI.) Ils n’en étaient pas meilleurs pour cela, et n’écrivaient pas moins sur la porte de leur maison, comme ce même financier : \emph{Tout esclave qui, sans ma permission, sortira d’ici, recevra cent coups}. « Quisquis servus sine dominico jussu foras exierit, accipiet plagas centum. » (Petron., \emph{Satyr.}, XXVIII.)}. Toute cette brillante société était, en outre, un ramas d’ignorants ou d’imitateurs. Au fond, elle n’inventait rien, et tirait tout ce qu’elle savait des provinces helléniques. Les innovations qu’elle y mêlait étaient des altérations, non des embellissements. Elle s’habillait à la grecque ou à la phrygienne, se coiffait de la mitre persane, osait même, au grand scandale des louangeurs du temps passé, porter des caleçons à la mode asiatique sous une toge douteuse ; et tout cela qu’était-ce ? Des emprunts à l’hellénisme, et quoi de plus ? Rien, pas même les dieux nouveaux, les Isis, les Sérapis, les Astarté, et, plus tard, les Mithra et les Élagabal que Rome vit s’impatro­niser dans ses temples. Il ne perçait de tous côtés que ce sentiment d’une population asiatique transplantée, apportant dans le pays qui s’imposait à elle les usages, les idées, les préjugés, les opinions, les tendances, les superstitions, les meubles, les ustensiles, les vêtements, les coiffures, les bijoux, les aliments, les boissons, les livres, les tableaux, les statues, en un mot, toute l’existence de la patrie.\par
Les races italiotes s’étaient fondues dans cette masse amenée par ses défaites sur le sein des vainqueurs que son poids achevait d’étouffer ; ou bien les nobles Sabins, méconnus, croupissaient dans les plus obscurs bas-fonds de la populace, mourant de faim sur le pavé de la ville illustrée par leurs ancêtres. Ne vit-on pas les descendants des Gracques gagner leur pain, cochers du cirque \footnote{Am. Thierry,\emph{ Hist. de la Gaule sous l’administr. rom.}, t. I, p. 181.}, et ne fallut-il pas que les empereurs prissent en pitié la dégradante abjection où le patriciat était tombé ? Par une loi, ils refusèrent aux matrones issues des vieilles familles le droit de vivre de prostitution \footnote{« Eodem anno, gravibus senatus decretis libido feminarum coercita, cautumque ne « quæstum corpore faceret cui avus, aut pater aut maritus eques romanos fuisset. Nam « Vistilia, prætoria familia genita, licentiam stupri apud ædiles vulgaverat. » (Tacit., \emph{Ann.}, II, 85.)}. Du reste, la terre d’Italie elle-même était traitée comme ses indigènes par les vaincus devenus tout-puissants. Elle ne comptait plus parmi les régions dignes de nourrir les hommes. Elle n’avait plus de métairies, on n’y traçait plus de sillons, elle ne produisait plus de blé \footnote{« At, hercule, nemo refert quod Italia externæ opis indiget quod vita populi romani per « incerta maris et tempestatum quotidie volvitur, ac, nisi provinciarum copiæ et dominis et « servitiis et agris subvenerint, nostra nos scilicet nemora nostræque villæ tuebuntur ! » (Tac., \emph{Ann.}, III, 54.)}. C’était un vaste jardin semé de maisons de campagne et de châteaux de plaisance. On va voir bientôt le jour où il fut même défendu aux Italiotes de porter les armes \footnote{Dans la guerre Flavienne, Antonius traita bien dédaigneusement les prétoriens licenciés par Vitellius et recueillis par lui, lorsque, leur rappelant qu’ils étaient nés en Italie, à la différence des légionnaires de son armée, Germains ou Gaulois, il les appelle \emph{pagani, paysans}. (\emph{Hist.}, III, 24.) Ce fut dans cette garde spéciale, qui ne quittait jamais les résidences impériales et portait fort peu les armes, que les Italiotes continuèrent encore un certain temps à servir ; mais, à la fin, les empereurs se lassèrent d’eux, et les remplacèrent par de vrais soldats levés dans le Nord.}. Mais ne devançons pas les temps.\par
Lorsque l’Asie, prédominant ainsi dans la population de la Ville, eut enfin amené la nécessité prochaine du gouvernement d’un maître, César, pour illustrer d’habiles loisirs, s’en alla conquérir la Gaule. Le succès de son entreprise eut des conséquences ethniques tout opposées à celles des autres guerres romaines. Au lieu d’amener des Gaulois en Italie, la conquête entraîna surtout des Asiatiques au delà des Alpes, et, bien qu’un certain nombre de familles de race celtique ait, depuis lors, apporté leur sang à l’épouvantable tohu-bohu qui se mélangeait et se battait dans la métropole, cette immigration toujours restreinte n’eut pas une importance proportionnée à celle des colonisations sémitisées qui furent jetées à travers les provinces transalpines.\par
La Gaule, la proie future de César, n’avait pas l’étendue de la France actuelle, et, entre autres différences, le sud-est de ce territoire, ou, suivant l’expression romaine, la Province, avait dès longtemps subi le joug de la république, et n’en faisait plus réellement partie.\par
Depuis la victoire de Marius sur les Cimbres et leurs alliés, la Provence et le Languedoc étaient devenus le poste avancé de l’Italie contre les agressions du Nord \footnote{Am. Thierry, \emph{la Gaule sous l’administr. rom. Introd}., t. I, p. 119.}. Le sénat s’était laissé aller à cette fondation d’autant plus aisément que les Massaliotes, avec leurs colonies diverses, Toulon, Antibes, Nice, n’avaient rien épargné pour lui en prouver l’utilité. Ils espéraient gagner, à cette nouveauté, un repos plus profond et une extension notable de leur commerce.\par
 Il n’y a pas à douter non plus que les populations originairement phocéennes, mais très sémitisées, établies à l’embouchure du Rhône et dans les environs, n’aient modifié, à la longue, les populations galliques et ligures de leur voisinage immédiat en se mêlant à elles. Les tribus de ces contrées apparaissent dès lors comme les moins énergiques de toute leur parenté.\par
Les hommes d’État romains avaient annexé solidement tous ces territoires au domaine de la république, en y envoyant des colonies, en y établissant des légionnaires vétérans, en y faisant naître, pour tout dire, une multitude aussi romaine que possible. C’était, certes, le meilleur moyen de s’en rendre maîtres à jamais.\par
Mais avec quels éléments créa-t-on ces gens de la Province, ou, comme ils s’appelaient eux-mêmes, ces \emph{véritables} Romains ? Deux siècles plus tôt, on aurait pu composer leur sang d’un mélange italiote. Désormais, le mélange italiote lui-même étant presque absorbé dans les apports sémitisés, ce fut surtout de ces derniers que se forma la nouvelle population. On y mêla, en foule, d’anciens soldats recrutés en Asie ou en Grèce. Ceux-ci vinrent, avec leurs familles, déposséder les habitants du sol, leur prendre leurs chaumières et leurs cultures, et essayer, avec cette fortune conquise, de fonder pour l’avenir souche d’honnêtes gens. On donna aux villes gauloises une physionomie aussi romaine que possible ; on défendit aux habitants de conserver ce que les pratiques druidiques avaient de trop violent ; on les força de croire que leurs dieux n’étaient autres que les dieux romains ou grecs défigurés par des noms barbares, et, en mariant les jeunes Celtes aux filles des colons et des soldats, en obtint bientôt une génération qui aurait rougi de porter les mêmes noms que ses ancêtres paternels et qui trouvait les appellations latines bien plus belles.\par
Avec les groupes sémitiques attirés sur le sol gallique par l’action directe du gouvernement, il y eut encore plusieurs classes d’individus dont le séjour temporaire ou l’établissement fortuit et permanent vinrent contribuer à transformer le sang gallique. Les employés militaires et civils de la république apportèrent, avec leurs mœurs faciles, de grandes causes de renouvellement dans la race. Les marchands, les spéculateurs arrivèrent aussi ; ceux qui faisaient le commerce d’esclaves ne se rendirent pas les moins actifs, et la déroute morale des Galls fut achevée, comme l’est aujourd’hui celle des indigènes de l’Amérique, par le contact d’une civilisation inacceptable par ceux à qui elle était offerte, tant que leur sang restait pur, et partant leur intelligence fermée aux notions étrangères.\par
Tout ce qui était romain ou métis romain devint maître absolu. Les Celtes ou bien s’en allèrent chercher des mœurs analogues aux leurs chez leurs parents du centre des Gaules, ou bien tombèrent dans la foule des travailleurs ruraux, espèce d’hommes que l’on supposait libres, mais qui en réalité menaient la vie d’esclaves, En peu d’années, la Province se trouva aussi bien transfigurée et sémitisée que nous voyons aujourd’hui la ville d’Alger être devenue, après vingt ans, une ville française.\par
Ce que désormais on appela Gaulois ne désigna plus un Gall, mais seulement un habitant du pays possédé autrefois par les Galls, de même que lorsque nous disons un Anglais, nous n’entendons pas indiquer un fils direct des Saxons à longues barbes rouges, oppresseurs des tribus bretonnes, mais un homme issu du mélange breton, frison, anglais, danois, normand, et, par conséquent, moins Anglais que métis. Un Gaulois de la Province représenta, à prendre les choses au pied de la lettre, le produit sémitisé des éléments les plus disparates ; un homme qui n’était ni Italiote, ni Grec, ni Asiatique, ni Gall, mais de tout cela un peu, et qui portait dans sa nationalité, formée d’éléments inconciliables, cet esprit léger, ce caractère effacé et changeant, stigmate de toutes les races dégénérées. L’homme de la Province était peut-être le spécimen le plus mauvais de tous les alliages opérés dans le sein de la fusion romaine ; il se montrait, entre autres exemples, très inférieur aux populations du littoral hispanique.\par
Celles-ci avaient au moins plus d’homogénéité. Le fond ibère s’était marié avec un apport très puissant de sang directement sémitique où la dose des éléments mélaniens était forte. Au fond des provinces que les invasions anciennes avaient rendues celtiques, l’aptitude à embrasser la civilisation hellénisée resta toujours faible ; mais, sur le littoral, le penchant contraire se trouva très marqué. Les colonies implantées par les Romains, venant d’Asie et de Grèce, peut-être encore d’Afrique, trouvèrent assez facilement accueil, et, tout en gardant un caractère particulier que lui assuraient les mélanges ibères et celtiques, déposés au fond de sa nature, le groupe d’Espagne se haussa sur un degré honorable de la civilisation romano-sémitique \footnote{Am. Thierry,\emph{ la Gaule sous l’administr. rom. Introd}., t. I, p. 115 et pass., 166, 211.}. Même, à un certain moment, on le verra devancer l’Italie dans la voie littéraire, par cette raison que le voisinage de l’Afrique, en renouvelant incessamment la partie mélanienne de son essence le poussa vigoureusement dans cette voie. Rien donc de surprenant à ce que l’Espagne du sud fût un pays supérieur à la Province, et maintînt sa préséance aussi longtemps que la civilisation sémitisée eut la haute main dans le monde occidental.\par
Mais, de ce que la Gaule romaine se sémitisait, le sang celtique, loin de servir à rectifier ce que l’essence féminine asiatique apportait d’excessif dans la péninsule italique, était obligé, au contraire, de fuir devant sa puissance, et cette fuite-là ne devait jamais finir \footnote{À cette époque, il ne faut plus guère parler de nations celtiques indépendantes au delà du Rhin. Par conséquent, la race des Kymris n’occupait plus, avec sa liberté plus ou moins complète, que la Gaule au-dessus de la Province, l’Helvétie et les îles Britanniques. Toutes ces contrées étaient certainement fort peuplées, mais elles ne pouvaient entrer en comparaison sous ce rapport avec l’empire. Rome seule comptait pour le moins deux millions d’habitants. Alexandrie en avait 600.000 (58 avant J.-C.). Jérusalem, pendant le siège de Titus, perdit 1,100,000 personnes, et 97,000 ayant été réduites en esclavage par les Romains, cette multitude, qui représentait d’ailleurs à peu près la population de toute la Judée doit être considérée comme ayant formé, avant la guerre, 1,200.000 à 1,300,000 âmes pour cette très petite province. L’empire, sous les Antonins, comptait 160 millions d’âmes, et Gibbon, pour la même époque, n’en attribue que 107 à l’Europe entière. Il n’y avait donc aucune proportion entre la résistance que pouvaient offrir les nations galliques et l’énergie numérique dont Rome disposait contre elles.  – Voir Zumpt, dans les \emph{Mémoires de l’Académie des sciences de Berlin}, 1840, p. 20.}.\par
 César donc, ayant pour point d’appui la Province, complètement romanisée \footnote{On inventa, sous les empereurs, un mot spécial pour exprimer l’ensemble hétérogène de l’univers romain : ce fut celui de \emph{romanité, romanitas} ; on l’opposait à la \emph{barbaria}, qui comprenait toutes les nations, soit du sud, soit du nord, soit de l’Asie, soit de l’Europe, les Parthes comme les Germains, vivant en dehors de cette confusion. Voir Améd. Thierry, \emph{Hist. de la Gaule sous l’administrat. rom. Introd.}, t. I, p. 199.}, entreprit et conduisit à bien la conquête des Gaules supérieures. Lui et ses successeurs continuèrent à tenir les Celtes sous les pieds de la civilisation du sud. Toutes les colonies, en si grand nombre, qui s’abattirent sur le pays, devinrent de véritables garnisons, agissant vigoureusement pour la diffusion du sang et de la culture asiatiques. Dans ces municipes gaulois où tout, depuis la langue officielle jusqu’aux costumes, jusqu’aux meubles, était romain, où l’indigène était tellement considéré comme un barbare que ce pouvait être un sujet de vanité pour un grand que de devoir le jour à l’intrigue de sa mère avec un homme d’Italie \footnote{Am. Thierry, \emph{Hist. de la Gaule sous l’administrat. rom.}, t. I, p. 13. Tac., \emph{Hist.}, IV, 55 : « Sabinus, super insitam vanitatem, falsæ stirpis gloria incendebatur : proaviam suam divo « julio, pet Gallias bellanti, corpore atque adulterio placuisse. » Ce qui rendait cette prétention encore plus bizarre, c’est que Sabinus ne la faisait valoir que pour faire mieux sentir ses droits à diriger une insurrection contre la puissance romaine.} ; dans ces rues bordées de maisons à la mode grecque et latine, personne ne s’étonnait de voir, gardant le pays et circulant partout, des légionnaires nés en Syrie ou en Égypte, de la cavalerie cataphracte recrutée chez les Thessaliens, des troupes légères arrivant de Numidie, et des frondeurs baléares. Tous ces guerriers exotiques, au teint cuivré de mille nuances ou même noirs, passaient incessamment du Rhin au Pyrénées, et modifiaient la race à tous les degrés sociaux.\par
Tout en démontrant l’impuissance du sang celtique et sa passivité dans l’ensemble du monde romain, il ne faut pas pousser les choses trop avant, et méconnaître l’influence conservée par la civilisation kymrique sur les instincts de ses métis. L’esprit utilitaire des Galls, bien qu’agissant dans l’ombre, qui ne lui est d’ailleurs que favorable, continua à croître et à soutenir l’agriculture, le commerce et l’industrie. Pendant toute la période impériale, la Gaule eut dans ce genre, mais dans ce genre seul, de perpétuels succès. Ses étoffes communes, ses métaux travaillés, ses chars, continuèrent à jouir d’une vogue générale. Portant son intelligence sur les questions industrielles et mercantiles, le Celte avait gardé et même perfectionné ses antiques aptitudes. Pardessus tout, il était brave, et l’on en faisait aisément un bon soldat, qui allait tenir garnison le plus ordinairement en Grèce, dans la Judée, au bord de l’Euphrate. Sur ces différents points, il se mêlait à la population indigène. Mais là, en fait de désordre, tout était opéré depuis longtemps, et un peu plus, un peu moins d’alliage dans ces masses innombrables, n’était pas pour changer rien à leur incohé­rence, d’une part, à la prédominance foncière des éléments mélanisés, de l’autre.\par
On n’oubliera pas que ce n’est qu’épisodiquement si je parle en ce moment de la Gaule, et seulement pour expliquer comment son sang n’eut pas d’action pour empêcher Rome et l’Italie de se sémitiser. Par la même occasion, j’ai montré ce que cette province elle-même était devenue après sa conquête. Je rentre dans le courant du grand fleuve romain.\par
 Les races italiotes pures n’existaient donc plus, à l’époque de Pompée, en Italie : le pays était devenu jardin. Cependant, quelque temps encore, les multitudes jadis vaincues, glorifiées par leur défaite, n’osèrent pas proposer pour le gouvernement de l’univers des hommes nés dans leurs pays déshonorés. L’ancienne force d’impulsion subsistait, bien que mourante, et c’était sur le sol sacré par la victoire qu’on s’accom­modait encore de chercher le maître universel. Comme les institutions ne découlent jamais que de l’état ethnique des peuples, cette situation doit être bien assise avant que les institutions s’établissent et surtout se complètent. Jadis l’Italie n’avait obtenu le droit de cité romaine que longtemps après l’invasion complète de Rome par les Italiotes. Ce ne fut également que lorsque le désordre le plus complet dans la ville et la Péninsule eut effacé l’influence de leurs populations nationales que les provinces furent admises en masse aux droits civiques, et que l’on vit l’Arabe au fond de son désert, le Batave dans ses marais, s’intituler, mais sans trop d’orgueil, citoyen romain.\par
Néanmoins, avant qu’on en fût là, et que l’état des faits eût été confessé par celui de la loi, l’incohérence ethnique et la disparition des races italiotes s’étaient déjà affichées dans l’acte le plus considérable que pût amener la politique, je dis, dans le choix des empereurs.\par
Pour une société arrivée au même point que l’agglomération assyrienne, la royauté persane et le despotisme macédonien, et qui ne cherchait plus que la tranquillité, et, autant que possible, la stabilité, on peut être étonné que l’empire n’ait pas, dès le premier jour, accepté le principe de l’hérédité monarchique. Certainement, ce n’est pas le culte d’une liberté trop prude qui l’en tenait d’avance dégoûté. Ses répugnances provenaient de la même source qui avait ailleurs empêché la domination sur le monde gréco-asiatique de se perpétuer dans la famille du fils d’Olympias.\par
Les royaumes ninivites et babyloniens avaient pu inaugurer des dynasties. Ces États étaient dirigés par des conquérants étrangers qui imposaient aux vaincus une certaine forme, en se passant de tout assentiment, et ainsi la loi constitutive n’était pas assise sur un compromis, mais bien sur la force. Ce fait est si vrai que les dynasties ne se succédaient pas autrement que par le droit de victoire. Dans la monarchie persane, il en fut de même. La société macédonienne, issue elle-même d’un pacte entre les diverses nationalités de la Grèce, et englobée dès son premier pas dans l’anarchie des idées asiatiques, ne fonctionna pas d’une manière aussi aisée ni aussi simple. Elle ne put fonder rien d’unitaire ni même de stable, et, pour vivre, elle dut consentir à éparpiller ses forces. Toutefois son influence agit encore assez fortement sur les Asiatiques pour déterminer la fondation des différents royaumes de la Bactriane, des Lagides, des Séleucides. Il y eut là des dynasties, sans doute médiocrement régulières, quant à l’observation domestique des droits de successibilité, mais du moins inébranlables dans la possession du trône, et respectées de la race indigène. Cette circonstance fait bien voir à quel point étaient reconnus la suprématie ethnique des vainqueurs et les droits qui en découlaient.\par
C’est donc un fait incontestable que l’élément macédonien-arian parvenait à maintenir en Asie sa supériorité, et, bien que fort combattu et même annulé sur la plupart des points, demeurait capable de produire des résultats pratiques d’une assez notable importance \footnote{L’hellénisme avait encore assez d’individualité pour que les Séleucides fussent amenés par fanatisme religieux à persécuter les Juifs. (Voir Bœttiger, \emph{ouvr. cité}, t. I, p. 28.)}.\par
Mais il n’en pouvait être de même chez les Romains. Puisqu’il n’avait jamais existé au monde de nation romaine, de race romaine, il n’y avait jamais eu non plus, pour la cité qui ralliait le monde, de race paisiblement prédominante. Tour à tour, les Étrusques, mêlés au sang jaune, les Sabins, dont le principe kymrique était moins brillamment modifié que l’essence ariane des Hellènes, et enfin la tourbe sémitique avaient gagné le dessus dans la population urbaine. Les multitudes occidentales étaient vaguement réunies par l’usage commun du latin ; mais que valait ce latin, qui de l’Italie avait débordé sur l’Afrique, l’Espagne, les Gaules et le nord de l’Europe, en suivant la rive droite du Danube, et la dépassant quelquefois ? Ce n’était nullement le pendant du grec, même corrompu, répandu dans l’Asie antérieure jusqu’à la Bactriane, et même jusqu’au Pendjab ; c’était à peine l’ombre de la langue de Tacite ou de Pline ; un idiome élastique connu sous le nom de \emph{lingua rustica}, ici se confondant avec l’osque, là s’appariant avec les restes de l’umbrique, plus loin empruntant au celtique et des mots et des formes, et, dans la bouche des gens qui visaient à la politesse du langage, se rapprochant le plus possible du grec. Un langage d’une personnalité si peu exigeante convenait admirablement aux détritus de toutes nations forcées de vivre ensemble et de choisir un moyen de communiquer. Ce fut pour ce motif que le latin devint la langue universelle de l’Occident, et qu’en même temps on aura toujours quelque peine à décider s’il a expulsé les langues indigènes, et, dans ce cas, l’époque où il s’est substitué à elles, ou bien s’il s’est borné à les corrompre et à s’enrichir de leurs débris. La question demeure si obscure qu’on a pu soutenir en Italie cette thèse, vraie sous beaucoup de rapports, que la langue moderne exista de tous temps parallèlement au langage cultivé de Cicéron et de Virgile.\par
Ainsi cette nation qui n’en était pas une, cet amas de peuples dominé par un nom commun, mais non pas par une race commune, ne pouvait avoir et n’eut pas d’hérédité monarchique, et ce fut plutôt même le hasard qu’une conséquence des principes ethniques qui, en mettant pour le début le commandement dans la famille des Jules et les maisons ses parentes, conféra à une sorte de dynastie trop imparfaite, mais issue de la Ville, les premiers honneurs du pouvoir absolu. Ce fut hasard, car rien n’empêchait, dans les dernières années de la république, qu’un maître d’extraction italiote, ou asiatique, ou africaine, fît valoir avec succès les droits du génie \footnote{La population noble italiote commença à disparaître de Rome vers la seconde guerre punique. En 220 av. J.-C., deux ans avant l’ouverture des hostilités, le cens avait donné 270,213 citoyens romains. En 204, il n’y en avait plus que 214,000 ; cependant 8,000 esclaves avaient été affranchis pour pouvoir être incorporés dans les légions. (Zumpt, \emph{ouvr. cité}, p. 13.) Après la guerre, il se trouva que huit légions avaient été anéanties à Cannes, et deux autres, avec les alliés italiotes, si bien massacrées dans la forêt Litana qu’il n’en avait échappé que dix hommes. On combla ces vides terribles au moyen d’étrangers, et les familles plébéiennes d’ancienne extraction passèrent au sénat et dans l’ordre équestre. (\emph{Ibidem}, p. 25.) On voit à quel point les vieilles maisons d’origine sabine devaient être devenues rares parmi les patriciens au temps des premiers Césars.}. Aussi, ni le conquérant des Gaules, ni Auguste, ni Tibère, ni aucun des Césars, ne songea-t-il un instant au rôle de monarque héréditaire. Vaste comme était l’empire, on n’aurait pas reconnu à dix lieues de Rome, on n’aurait ni admis ni compris l’illustration d’une race sabine, et bien moins encore les droits universels que ses partisans eussent prétendu en faire découler. En Asie, au contraire, on connaissait encore les vieilles souches macédoniennes, et on ne leur contestait ni la gloire supérieure, ni les prérogatives dominatrices.\par
Le principat ne fut pas une dignité fondée sur les prestiges du passé, mais, au contraire, sur toutes les nécessités matérielles du présent. Le consulat lui apporta son contingent de forces ; la puissance tribunitienne y adjoignit ses droits énormes ; la préture, la questure, le censorat, les différentes fonctions républicaines vinrent tour à tour se fondre dans cette masse d’attributions aussi hétérogènes que les masses de peuples sur lesquelles elles devaient s’exercer \footnote{« ... Potestatem tribunitiam... Id summi fastigii vocabulum Augustus repperit, ne regis aut « dictatoris nomen assurneret, ac tamen appellatione aliqua cætera imperia præmineret. » (Tac., \emph{Ann.}, III, 56.)}, et quand plus tard on voulut joindre le brillant l’imposant à l’utile comme couronnement nécessaire, on put décerner au maître du monde les honneurs de l’apothéose, on put en faire un dieu \footnote{« ... Cuncta legum et magistratum munera in se trahens princeps... » (Tac., \emph{Ann.}, XI, 5.)  – Suet., \emph{Dom.}, 13 : « Dominus et deus noster sic fieri jubet. »}, mais jamais on ne parvint à introniser ses fils nés ou à naître dans la possession régulière de ses droits. Amasser sur sa tête des nuages d’honneurs, faire fouler à ses pieds l’humanité prosternée, concentrer dans ses mains tout ce que la science politique, la hiérarchie religieuse, la sagesse administrative, la discipline militaire avaient jamais créé de forces pour plier les volontés : ces prodiges s’accomplirent, et nulle réclamation ne s’éleva ; mais c’était à un homme que l’on prodiguait tous ces pouvoirs, jamais à une famille, jamais à une race. Le sentiment universel, qui ne reconnaissait plus nulle part de supériorité ethnique dans le monde dégénéré, n’y aurait pas consenti. On put croire un instant, sous les premiers Antonins, qu’une dynastie sacrée par ses bienfaits allait s’établir pour le bonheur du monde. Caracalla se montra soudain, et le monde, qui n’avait été qu’entraîné, non encore convaincu, reprit ses anciens doutes. La dignité impériale resta élective. Cette forme de commandement était décidément la seule possible, parce que, dans cette société sans principes fixes, sans besoins certains, enfin, en un mot qui dit tout, sans homogénéité de sang, on ne pouvait vivre, quoi qu’on en eût, qu’en laissant toujours la porte ouverte aux changements, et en prêtant les mains de bonne grâce à l’instabilité \footnote{On dit beaucoup que ce sont les guerres qui troublent la conscience des peuples, les ramènent vers l’ignorance et les empêchent de se créer une idée juste de leurs besoins. Or, depuis la bataille d’Actium jusqu’à la mort de Commode, il n’y eut dans l’intérieur de l’empire d’autre levée de boucliers que la lutte des Flaviens contre Vitellius. La prospérité matérielle fut très grande ; mais le pouvoir resta irrégulier, garda son inconsistance, et l’intelligence nationale alla toujours déclinant. (Voir Am. Thierry\emph{, Histoire de la Gaule sous l’administration romaine.} t. I, p. 241.)}.\par
Rien ne démontre mieux la variabilité ethnique de l’empire romain que le catalogue des empereurs. D’abord, et par le hasard assez ordinaire qui mit le génie sous le front d’un patricien démocrate, les premiers princes sortirent de la race sabine. Comment le pouvoir se perpétua un temps dans le cercle de leurs alliances, sans qu’une hérédité réelle pût s’établir jamais, c’est ce que Suétone raconte avec perfection. Les Jules, les Claude, les Néron eurent chacun leur jour, puis bientôt ils disparurent, et la famille italiote des Flavius les remplaça. Elle s’effaça promptement, et à qui fit-elle place ? À des Espagnols. Après les Espagnols, vinrent des Africains, après les Africains, dont Septime Sévère se montra le héros, et l’avocat Macrinus le représentant, non le plus fou, mais le plus vil, parurent les Syriens, bientôt supplantés par de nouveaux Africains, remplacés à leur tour par un Arabe, détrôné par un Pannonien. Je ne pousse pas plus loin la série, et je me contente de dire qu’après le Pannonien il y eut de tout sur le trône \footnote{Am. Thierry, \emph{la Gaule sous l’administration romaine. Introduction}, t. I, p. 163 et pass.} impérial, sauf un homme de famille urbaine.\par
Il faut considérer encore la manière dont le monde romain s’y prenait pour former l’esprit de ses lois \footnote{César avait désiré un code établi sur un principe unitaire. Il mourut trop tôt pour réaliser son projet. (Am. Thierry, \emph{la Gaule sous l’administr. rom. Introd.}, t. I, p. 73.) Je crois aussi que le temps n’en était pas encore arrivé. Il aurait eu à vaincre des résistances qui, un peu plus tard, n’existèrent plus. (Voir Am. Thierry, \emph{Hist. de la Gaule sous l’adm. rom. Introd.}, t. I, p. 253 et pass.)  – Savigny, \emph{Geschichte des rœmischen Rechtes im Mittelalter}, t. I, p. 4 et pass. « Très promptement, remarque « l’illustre écrivain, le droit romain cessa d’être animé d’un véritable esprit créateur. Les grands « jurisconsultes de l’époque de Caracalla et d’Alexandrie furent à peu près les derniers qui aient pu « répandre la vie dans la doctrine. » Cette opinion est encore trop favorable.}. Le demandait-il à l’ancien instinct, je ne dirai pas romain, puisqu’il n’y eut jamais rien de romain, mais du moins étrusque ou italique ? Nullement. Puisqu’il lui fallait une législation de compromis, il alla la chercher dans le pays qui offrait, après la ville éternelle, la population la plus mélangée : sur la côte syrienne, et il entoura, avec raison du reste, de toute son estime l’école d’où sortit Papinien. En fait de religion, il avait dès longtemps été large dans ses vues \footnote{L’étonnement des républicains peu idéalistes de la Rome sabine n’avait pas dû être médiocre en voyant Annibal mettre en avant contre lui des griefs théologiques. Le Carthaginois se présenta en apôtre de Milytta, et, au nom de cette divinité chananéenne, il détruisait les temples italiotes et faisait fondre les idoles en métal. (Voir Bœttiger, \emph{Ideen zur Kunst-Mythologie}, t. I, p. 29.)}. La Rome républicaine, avant de posséder un panthéon, s’était adressée à tous les coins de la terre pour se procurer des dieux \footnote{M. Am. Thierry félicite chaudement Adrien de ce que, dans ses voyages perpétuels à travers l’empire, le touriste-administrateur étudiait toutes les religions, et, pour bien en pénétrer l’esprit et les mérites, se faisait révéler tous leurs mystères en agréant toutes leurs initiations. (\emph{La Gaule sous l’administr. rom. Introd.}, t. I, p. 173.)  – Pétrone, \emph{Satyr.}, XVII, dit excellemment : « Nostra regio tain præsentibus plena est numinibus, ut facilius possis deum quam hominem « invenire. »}. Il vint un jour où, dans ce vaste éclectisme, on eut encore peur de s’être mis trop à l’étroit, et, pour ne pas sembler exclusif, on inventa ce mot vague de \emph{Providence}, qui est, en effet, chez des nations pensant différemment, mais ennemies des querelles, le meilleur à mettre en avant. Ne signifiant pas grand’chose, il ne peut choquer personne. La Providence devint le dieu officiel de l’empire \footnote{Avant l’invention de la \emph{Providence}, qui offrait cet avantage politique de ne trancher aucune question, les Grecs sémitisés avaient éprouvé le même besoin que les Romains et pour les mêmes causes, de réunir les cultes reconnus dans la sphère de l’action politique ; mais, au lieu de les accepter également, ils avaient cherché querelle à tous. Deux rhéteurs, Charax et Lampsacus, s’étaient fait fort de réduire tous les mythes au pied d’une explication rationnelle. Evhémère généralisa cette méthode, et il n’y eut plus pour lui dans les récits divins que des faits fort ordinaires, ou mal compris, ou défigurés ; enfin, à son avis, toutes les religions reposaient sur des malentendus de la nature la plus mesquine. Il avait découvert que Cadmus était un cuisinier du roi de Sidon, qui s’était enfui en Béotie avec Harmonia, joueuse de flûte de ce même monarque. (Bœttiger, \emph{Ideen zur Kunst-Mythologie}, t. I, p. 187 et pass.) Le grand écueil de l’évhémérisme, c’est d’avancer des explication qui ont autant besoin de preuves que les faits qu’ils prennent à partie.}.\par
Les peuples se trouvaient ainsi ménagés autant que possible dans leurs intérêts, dans leurs croyances, dans leurs notions du droit, dans leur répugnance à obéir toujours aux mêmes noms étrangers ; bref, il semblait qu’il ne leur manquât rien en fait de principes négatifs. On leur avait donné une religion qui n’en était pas une, une législation qui n’appartenait à aucune race, des souverains fournis par le hasard, et qui ne se réclamaient que d’une force momentanée. Et, cependant, que l’on s’en fût tenu là en fait de concessions, deux points auraient pu blesser encore. Le premier, si l’on eût conservé à Rome les anciens trophées : les provinciaux y auraient ravivé le souvenir de leurs défaites ; le second, si la capitale du monde était restée dans les mêmes lieux d’où s’étaient élancés les vainqueurs disparus. Le régime impérial comprit ces délicatesses et leur donna pleine satisfaction.\par
L’engouement des derniers temps de la république pour le grec, la littérature grecque et les gloires de la Grèce, avait été poussé jusqu’à l’extrême. Au temps de Sylla, il n’y avait homme de bien qui n’affectât de considérer la langue latine comme un patois grossier. On parlait grec dans les maisons qui se respectaient. Les gens d’esprit faisaient assaut d’atticisme, et les amants qui savaient vivre se disaient, dans leurs rendez-vous : (mots grecs), au lieu \emph{d’anima mea} \footnote{Pétrone, \emph{Satyr}., XXXVII : « Nunc nec quid nec quare in cœlum abiit et Trimalchionis tapanta est (mot grec). »}.\par
Après l’empire établi, cet hellénisme alla se renforçant ; Néron s’en fit le fanatique. Les héros antiques de la Ville furent considérés comme d’assez tristes hères, et on leur préféra tout haut le Macédonien Alexandre et les moindres porte-glaives de l’Hellade. Il est vrai qu’un peu plus tard une réaction se fit en faveur des vieux patriciens et de leur rusticité ; mais on peut soupçonner cet enthousiasme de n’avoir été qu’une mode littéraire : il n’eut, du moins, pour organes que des hommes fort éloquents sans doute, mais très étrangers au Latium, l’Espagnol Lucain, par exemple. Comme ces louangeurs inattendus ne purent déranger les préoccupations générales, le courant continua à pousser vers les illustrations grecques ou sémitiques. Chacun se sentait plus attiré, plus intéressé par elles. Ce que le gouvernement fit de mieux pour complaire à ces instincts fut accompli par Septime Sévère, lorsque ce grand prince érigea de riches monuments à la mémoire d’Annibal, et que son fils Antonin Caracalla dressa à ce même vainqueur de Cannes et de Trébie, des statues triomphales en grand nombre \footnote{Am. Thierry, \emph{la Gaule sous l’administr. rom. Introduct.}, t. I, p. 187 et pass.}. Ce qu’il faut admirer davantage, c’est qu’il en remplit Rome même. J’ai dit ailleurs que, si Cornélius Scipion avait été vaincu à Zama, la victoire n’aurait pu cependant changer l’ordre naturel des choses, et amener les Carthaginois à dominer sur les races italiotes. De même, le triomphe des Romains, sous l’ami de Lælius, n’empêcha pas non plus ces mêmes races, une fois leur œuvre accomplie, de s’engloutir dans l’élément sémitique, et Carthage, la malheureuse Carthage, une vague de cet océan, put savourer aussi son heure de joie dans le triomphe collectif, et dans l’outrage posthume appliqué sur la joue de la vieille Rome.\par
 Il semble que, le jour où les simulacres vermoulus des Fabius et des Scipions virent le borgne de la Numidie obtenir son marbre au milieu d’eux, il ne dut plus se trouver dans tout l’empireio un seul provincial humilié : chacun de ses citoyens put librement chanter les louanges des héros topiques. Le Gétule, le Maure célébra les vertus de Massinissa, et Jugurtha fut réhabilité. Les Espagnols vantèrent les incendies de Sagonte et de Numance, tandis que le Gaulois éleva plus haut que les nues la vaillance de Vercingétorix. Personne n’avait désormais à s’inquiéter des gloires urbaines insultées par ces gens qui se disaient citoyens, et le plus piquant, c’est que ces citoyens romains eux-mêmes, métis et bâtards qu’ils étaient à l’égard de toutes les vieilles races, n’avaient pas plus de droits à s’approprier les mérites des héros barbares dont il leur plaisait de se réclamer, que de honnir les grandes ombres patriciennes du Latium \footnote{Les gens réfléchis se rendaient bien compte de cette indignité des populations nouvelle vis-à-vis de la gloire des anciennes : « Cn. Pison, accusant indirectement Germanicus, lui « reprocha d’avoir, à la honte du nom romain, montré trop de bienveillance. non pour les « Athéniens, éteints par tant de désastres, mais pour l’écume des nations qui les avait « remplacés. » (Tac., \emph{Ann.}, II, 55.)}.\par
Reste la question de suprématie pour la Ville. Sur cet article, comme sur les autres, le monde de vaincus abrité sous les aigles impériales fut parfaitement traité.\par
Les Étrusques, constructeurs de Rome, n’avaient pas eu la prévision des hautes destinées qui attendaient leur colonie. Ils n’avaient pas choisi son territoire dans la vue d’en faire le centre du monde, ni même d’en rendre l’abord facile. Aussi, dès le règne de Tibère, on comprit que, puisque l’administration impériale se chargeait de surveiller les intérêts universels des nations amalgamées, il fallait qu’elle se rapprochât des pays où la vie était le plus active. Ces pays n’étaient pas les Gaules, nulles d’influence, n’étaient pas l’Italie dépeuplée : c’était l’Asie, où la civilisation croupissante, mais générale, et surtout l’accumulation de masses énormes d’habitants, rendaient nécessaire la surveillance incessante de l’autorité. Tibère, pour ne pas rompre du premier coup avec les anciennes habitudes, se contenta de s’établir à l’extrémité de la Péninsule. Il y avait alors plus d’un siècle que le dénouement des grandes guerres civiles et les résultats solides de la victoire ne s’acquéraient plus là, mais en Orient, ou, à tout le moins, en Grèce.\par
Néron, moins scrupuleux que Tibère, vécut le plus possible dans la terre classique, si douce à ce terrible ami des arts. Après lui, le mouvement qui entraînait les souve­rains vers l’est devint de plus en plus fort. Tels empereurs, comme Trajan ou Septime Sévère, passèrent leur vie à voyager ; tels autres, comme Héliogabale, visitèrent à peine et en étrangers, la ville éternelle. Un jour, la vraie métropole du monde fut Antioche. Quand les affaires du Nord prirent une importance majeure, Trèves devint la résidence ordinaire des chefs de l’État. Milan en reçut ensuite le titre officiel, et, cependant, que devenait Rome ? Rome gardait un sénat pour jouer dans les affaires un rôle triste, passif, et tel qu’un grand seigneur imbécile, produit adultérin des affranchis de ses aïeules, mais protégé par les souvenirs de son nom, peut encore l’avoir. De fait, ce sénat servait à peu de choses. Quelquefois, quand on y songeait, on le priait de reconnaître les empereurs issus de la volonté des légions. Des lois formelles interdisaient aux membres de la curie le métier des armes, et comme d’autres lois, en apparence bienveillantes, excluaient tous les Italiotes du service militaire actif, ces honnêtes sénateurs, qui d’ailleurs n’avaient rien de commun avec les pères conscrits des temps passés \footnote{« Iisdem diebus in numerum patriciorum adscivit Cæsar (Claudius) vetustissimum « quemque e senatu aut quibus clari parentes fuerant ; paucis jam reliquis familiarum « quas Romulus majorum et L. Brutus minorum gentium appellaverant ; exhaustis etiam « quæ dictator Cæsar lege Cassia et princeps Augustus lege Sænia, sublegere. » (Tac., \emph{Ann.}, XI, 25.) Claude venait de déclarer que, l’ancienne coutume de la république étant de s’adjoindre tous les chefs des peuples conquis, les Gaulois pouvaient être reçus dans le sénat, et il y avait admis les Éduens. (\emph{Ibidem}, 24.) Il est à remarquer que les plus vieilles maisons de Rome, les plus illustres avaient à peine six cents ans de durée, et on en comptait bien peu qui fussent dans ce cas, tant la fusion des races italiotes avait été rapide.}, n’auraient pas rencontré de soldats qui les connussent, s’ils avaient voulu de force se faire chefs d’une armée. Réduits pour toute occupation à la plus médiocre intrigue, ils ne trouvaient dans le monde personne qu’eux-mêmes pour croire à leur importance. Quand, par un malheur, quelque prince les employait dans ses combinaisons, leur autorité d’emprunt ne manquait jamais de les conduire à quelque abîme. Malheureux hommes, parvenus de hasard, vieillards sans dignité, ils aimaient encore à parader dans leurs séances oiseuses, combinant des périodes et jouant à l’éloquence dans ces jours terribles où l’empire n’appartenait qu’aux poignets vigoureux.\par
Ces sénateurs impuissants auraient pu s’avouer un défaut de plus, qui plus tard, du reste, leur porta grand préjudice, ce fut leur affectation de goûts littéraires, quand personne autre ne se souciait plus de savoir ce que c’était qu’un livre, Rome comptait parmi ses illustrations civiles des amateurs très prétentieux ; mais, sur ce point encore, Rome n’était plus le champ fécond de la littérature latine. Avouons aussi qu’elle ne l’avait jamais été.\par
À compter tous les beaux génies qui ont illustré les muses ausoniennes, poètes, prosateurs, historiens ou philosophes, depuis le vieux Ennius et Plaute, peu sont nés dans les murs de la Ville ou appartinrent à des familles urbaines. C’était une sorte de stérilité décidée, jetée comme une malédiction sur le sol de la cité guerrière, qui pourtant, il faut lui rendre cette justice, accueillit toujours noblement, et d’une façon conséquente au génie utilitaire du premier esprit italique, tout ce qui put rehausser sa splendeur. Ennius, Livius, Andronicus, Pacuvius, Plaute et Térence n’étaient pas Romains. Ne l’étaient pas non plus : Virgile, Horace, Tite-Live, Ovide, Vitruve, Cornélius Népos, Catulle, Valérius Flaccus, Pline. Encore bien moins cette pléiade espagnole venu-- à Rome avec ou après Portius Latro, les quatre Sénèque, le père et les trois fils, Sextilius Héna, Statorius Victor, Sénécion, Hygin, Columelle, Pomponius Méla, Silius Italicus, Quintilien, Martial, Florus, Lucain, et une longue liste encore \footnote{Am. Thierry, \emph{la Gaule sous l’administration romaine}, t. I, p. 200 et pass.}.\par
Les puristes urbains trouvaient toujours quelque chose à redire aux plus grands écrivains. Ceux de ces derniers qui venaient d’Italie avaient de trop la saveur du terroir, qui rendait leur style provincial. Ce reproche était plus mérité encore par les Espagnols. Toutefois la vogue de personne n’en était diminuée, et le mérite, quoi qu’on en ait dit depuis cent ans chez nous, était tout aussi reconnu chez les poètes de Cordoue que s’ils avaient écrit justement comme Cicéron. Nous ne pouvons trop juger la portée des critiques adressées au Padouan Tite-Live, mais nous sommes parfaitement en mesure de constater la vérité de celles qui poursuivaient les Sénèque, et Lucain, et Silius Italicus. Ces critiques se rattachent trop bien au sujet de ce livre pour n’en pas toucher un mot. On accusait donc l’école espagnole d’afficher à un degré choquant ce que je nomme le caractère sémitique, c’est-à-dire l’ardeur, la couleur, le goût du grandiose poussé jusqu’à l’emphase, et une vigueur dégénérant en mauvais goût et en dureté.\par
Acceptons toutes ces attaques. On a remarqué déjà combien elles étaient méritées par le génie des peuples mélanisés. Il n’y a donc pas lieu de les repousser quand il s’agit des œuvres de ce génie sur le sol espagnol, car on ne perd pas de vue que nous observons ici une poésie et une littérature qui ne florissaient dans la péninsule ibérique que là où il y avait du sang noir largement infusé, c’est-à-dire sur le littoral du sud. En conséquence, retournant le fait pour le faire entrer dans le rang de mes démonstrations, j’observe de nouveau combien la poésie, la littérature, sont plus fortes, et en même temps plus défectueuses par exubérance, partout où le sang mélanien se trouve abondamment, et, suivant cette veine, il n’y a qu’à passer jusqu’à la province qui marqua le plus dans les lettres après l’Espagne, ce fut l’Afrique \footnote{Am. Thierry, \emph{la Gaule sous l’administr. rom. Introd.}, t. I, p. 182 et seqq.}.\par
Là, autour de la Carthage romaine, la culture de l’imagination et de l’esprit était une habitude et, pour ainsi dire, un besoin général. Le philosophe Annæus Cornutus, né à Leptis, Septimius Sévérus, de la même ville, l’Adrumétain Salvius Julianus, le Numide Cornélius Fronton, précepteur de Marc-Aurèle, et enfin Apulée, élevèrent au plus haut point la gloire de l’Afrique dans la période païenne, tandis que l’Église militante dut à cette contrée de bien puissants et bien illustres apologistes dans la personne des Tertullien, des Minutius Félix, des saint Cyprien, des Arnobe, des Lactance, des saint Augustin. Chose plus remarquable encore : quand les invasions germaniques couvrirent de leurs masses régénératrices la face du monde occidental, ce fut sur les points où l’élément sémitique restait fort que les lettres romaines obtinrent leurs derniers succès. Je nomme donc cette même Afrique, cette même Carthage, sous le gouvernement des rois vandales \footnote{Meyer, \emph{Lateinische Anthologie}, t. II.}.\par
Ainsi, Rome ne fut jamais, ni sous l’empire, ni même sous la république, le sanctuaire des muses latines. Elle le sentait si bien que, dans ses propres murailles, elle n’accordait à sa langue naturelle aucune préférence. Pour instruite la population urbaine, le fisc impérial entretenait des grammairiens latins, mais aussi des grammai­riens grecs. Trois rhéteurs latins, mais cinq grecs, et, en même temps, comme les gens de lettres de langue latine trouvaient des honneurs et un salaire et un public partout ailleurs qu’en Italie, de même les écrivains helléniques étaient attirés et retenus à Rome par des avantages pareils : témoin Plutarque de Chéronée, Arrien de Nicomédie, Lucien de Samosate, Hérode Atticus de Marathon, Pausanias de Lydie, qui, tous, vinrent composer leurs ouvrages et s’illustrer au pied du Capitole.\par
 Ainsi, à chaque pas que nous faisons, nous nous enfonçons davantage dans les preuves accumulées de cette vérité que Rome n’avait rien en propre, ni religion, ni lois, ni langue, ni littérature, ni même préséance sérieuse et effective, et c’est ce que de nos jours on a proposé de considérer sous un point de vue favorable et d’approuver comme une nouveauté heureuse pour la civilisation. Tout dépend de ce qu’on aime et cherche, de ce qu’on blâme et réprouve \footnote{Savigny (\emph{Geschichte des rœmischen Rechtes im Mittelalter}) a très bien exprimé l’opinion ancienne en la raisonnant : « Lorsque Rome était petite, dit cet homme éminent, et qu’elle « rangeait sous sa dépendance quelques cités italiotes par l’octroi de son droit civique, en « pouvait supposer entre ces dernières et la ville conquérante une sorte d’égalité, et c’est « sur cette notion que reposa la constitution libre de ces villes. Mais, lorsque l’empire se « fut étendu sur trois parties du monde, cette égalité cessa complètement, de sorte que la « liberté locale dut diminuer. Vint ensuite la pression de l’administration impériale, qui, « en imposant partout un même niveau d’obéissance, fit disparaître peu à peu les « différences qui existaient entre l’Italie et les provinces. La Péninsule, jadis la partie du « territoire la plus favorisée, perdit de sa valeur individuelle, les terres autrefois conquises « se relevèrent quelque peu, puis enfin tout s’abîma ensemble dans un affaiblissement « incurable. Pour Rome même, cet énervement est de toute évidence... » (T. I, p. 31.)}.\par
Les détracteurs de la période impériale font remarquer, de leur côté, que, sur toute la face du monde romain depuis Auguste, aucune individualité illustre ne ressort plus. Tout est effacé ; plus de grandeur honorée, plus de bassesse flétrie ; tout vit en silence. Les anciennes gloires ne passionnent que les déclamateurs rhétoriciens à l’heure des classes ; elles n’appartiennent plus à personne, et les têtes vides seulement peuvent prendre feu pour elles. Plus de grandes familles ; toutes sont éteintes, et celles qui, occupant leur place, essayent de jouer leur rôle, sorties ce matin de la tourbe, y rentreront ce soir \footnote{Am. Thierry, \emph{la Gaule sous l’administr. rom. Introd.}, t. I, p. 181 : « Le parti des idées « républicaines et aristocratiques n’eut même bientôt plus pour chefs que des hommes « nouveaux ; ni Corbulon, ni Paetus Thraséas, ni Agricola, ni Helvidius, n’appartinrent à « l’ancien patriciat. Dès le second siècle, et surtout au troisième, les familles sénatoriales « étaient pour la plupart étrangères à l’Italie. »}. Puis cette antique liberté patricienne qui, avec ses inconvénients, avait aussi ses beaux et nobles côtés, c’en est fini d’elle. Personne n’y songe, et ceux-là qui, dans leurs livres, balancent encore devant son souvenir un encens théorique, recherchent, en bons courtisans, l’amitié des puissants de l’époque, et seraient désolés qu’on prît au mot leurs regrets. En même temps, les nationalités quittent leurs insignes. Elles vont les unes chez les autres porter le désordre de toutes les notions sociales, elles ne croient plus en elles-mêmes. Ce qu’elles ont gardé de personnel, c’est la soif d’empêcher l’une d’entre elles de se soustraire à la décadence générale.\par
Avec l’oubli de la race, avec l’extinction des maisons illustres dont les exemples guidaient jadis les multitudes, avec le syncrétisme des théologies, sont venus en foule, non pas les grands vices personnels, partage de tous les temps, mais cet universel relâchement de la morale ordinaire, cette incertitude de tous les principes, ce détachement de toutes les individualités de la chose publique, ce scepticisme tantôt riant, tantôt morose, indifféremment porté sur ce qui n’est pas d’intérêt ou d’usage quotidien, enfin ce dégoût effrayé de l’avenir, et ce sont là des malheurs bien autrement avilissants pour les sociétés. Quant aux éventualités politiques, interrogez la foule romaine. Plus rien ne lui répugne, plus rien ne l’étonne. Les conditions que les peuples homogènes exigent de qui veut les gouverner, elles en ont perdu jusqu’à l’idée. Hier c’était un Arabe qui montait sur le trône, demain ce sera le fouet d’un berger pannonien qui mènera les peuples. Le citoyen romain de la Gaule ou de l’Afrique s’en consolera en pensant qu’après tout ce ne sont pas là ses affaires, que le premier gouvernant venu est le meilleur, et que c’est une organisation acceptable que celle où son fils, sinon lui-même, peut à son tour devenir l’empereur.\par
Tel était le sentiment général au III\textsuperscript{e} siècle, et, pendant seize cents ans, tous ceux, païens ou chrétiens, qui ont réfléchi à cette situation ne l’ont pas trouvée belle. Les politiques comme les poètes, les historiens comme les moralistes, ont déversé leur mépris sur les immondes populations auxquelles on ne pouvait faire accepter un autre régime. C’est là le procès que des esprits d’ailleurs éminents, des hommes d’une érudition vaste et solide s’efforcent aujourd’hui de faire réviser. Ils sont emportés à leur insu par une sympathie bien naturelle et que les rapprochements ethniques n’expli­quent que trop.\par
Ce n’est pas qu’ils ne tombent d’accord de l’exactitude des reproches adressés aux multitudes de l’époque impériale ; mais ils opposent à ces défauts de prétendus avantages qui, à leurs yeux, les rachètent. De quoi se plaint-on ? du mélange des religions ? Il en résultait une tolérance universelle. Du relâchement de la doctrine offi­cielle sur ces matières ? Ce n’était rien que l’athéisme dans la loi \footnote{Tibère avait émis cette maxime toute moderne : « Deorum injurias diis curæ. » (Tacit., \emph{Ann.}, liv. I, 73.) C’était à propos de la loi sur les crimes de lèse-majesté, dont il cherchait à étendre les effets, non pour les dieux, mais pour lui.}. Qu’importent les effets d’un tel exemple partant de si haut ?\par
À ce point de vue, l’avilissement et la destruction des grandes familles, voire même des traditions nationales qu’elles conservaient, sont des résultats acceptables. Les classes moyennes du temps n’ont pu manquer de bien accueillir cet holocauste quand on l’a jeté sur leurs autels. Voir des hommes héritiers des plus augustes noms, des hommes dont les pères avaient donné à la patrie mille victoires et mille provinces, voir ces hommes, pour gagner leur vie, réduits à porter la balle et à faire les gladiateurs ; voir des matrones, nièces de Collatin, réduites au pain de leurs amants, ce ne sont pas là des spectacles à dédaigner pour les fils d’Habinas, pas plus que pour les cousins de Spartacus. La seule différence est que le fabricant de cercueils mis en scène par Pétrone désire en arriver là doucement et sans violence, tandis que la bête des ergastules savoure mieux la misère qu’elle-même, en personne, a faite, surtout si elle est ensanglantée. Un État sans noblesse, c’est le rêve de bien des époques. Il n’importe pas que la nationalité y perde ses colonnes, son histoire morale, ses archives : tout est bien quand la vanité de l’homme médiocre a abaissé le ciel à la portée de sa main.\par
Qu’importe la nationalité elle-même ? Ne vaut-il pas mieux pour les différents groupes humains perdre tout ce qui peut les séparer, les différencier ? À ce titre, en effet, l’âge impérial est une des plus belles périodes que l’humanité ait jamais parcourues.\par
 Passons aux avantages effectifs. D’abord, dit-on, une administration régulière et unitaire. Ici il faut examiner.\par
Si l’éloge est vrai, il est grand ; cependant on peut douter de son exactitude. J’entends bien qu’en principe tout aboutissait à l’empereur, que les moindres officiers civils et militaires devaient attendre hiérarchiquement l’ordre descendu du trône, et que, sur le vaste pourtour comme au centre de l’État, la parole du souverain était censée décisive. Mais que disait-elle, cette parole, et que voulait-elle ? Jamais qu’une seule et même chose : de l’argent, et, pourvu qu’elle en obtînt, l’intervention d’en haut ne prenait pas souci de l’administration intérieure des provinces, des royaumes, à plus forte raison des villes et des bourgades, qui, organisées sur l’ancien plan municipal, avaient le droit de n’être gouvernées que par leur curie. Ce droit survivait, énervé à la vérité, parce que le caprice d’en haut en troublait en mille occasions l’exercice, mais il existait seul, privé de bien des avantages et offrant tous les inconvénients de l’esprit de clocher.\par
Les écrivains démocratiques font grand éclat du titre de citoyen conféré à l’univers entier par Antonin Caracalla. J’en suis moins enthousiaste. La plus belle prérogative n’a de valeur que lorsqu’elle n’est pas prodiguée. Quand tout le monde est illustre, personne ne l’est plus, et ce fut ainsi qu’il en advint à la cohue innombrable des citoyens provinciaux \footnote{Rien ne fut changé par la constitution de Caracalla dans le mode d’administration des villes, aucun avantage nouveau ne fut introduit, et Savigny n’y aperçoit qu’une simple évolution de l’état personnel des gouvernés. (\emph{Geschichte des rœmischen Rechtes im Mittelalter}, t. I, p. 63.)}.\par
Tous ils furent astreints à payer l’impôt, tous ils devinrent passibles des peines que la jurisprudence impériale appliquait ; et, sans souci de ce qu’eût pensé de cette innovation le \emph{civis romanus} d’autrefois, on les soumettait à la torture quand s’en présentait la moindre tentation juridique. Saint Paul avait dû à sa qualité civique réclamée à propos un traitement d’honneur ; mais les confesseurs, les vierges de la primitive Église, bien que décorés du droit de cité, n’en étaient pas moins menés en esclaves. C’était désormais l’usage commun. L’édit de nivellement put donc plaire un jour aux sujets, en leur montrant abaissés ceux qu’ils enviaient naguère ; mais, pour eux, il ne les releva pas : ce fut simplement une grande prérogative abolie et jetée à l’eau \footnote{Pour n’en citer qu’un exemple, voir ce que dit Suétone de l’administration financière de Vespasien. (\emph{Vesp}., 16.)}.\par
Et quant aux sénats municipaux, maîtres, soi-disant, d’administrer leurs villes suivant l’opinion de la localité, leur félicité n’était pas non plus si grande qu’on le donne à croire \footnote{Consulter, sur l’organisation municipale pendant l’époque impériale, l’\emph{Histoire du droit municipal en France}, par M. Raynouard, Paris, 1829, 2 vol. in-8°, et l’\emph{Histoire critique du pouvoir municipal en France}, par C. Leber, Paris, 1829, in-8°. – Bien que spécialement destinés à l’examen des institutions gallo-romaines, ces deux ouvrages renferment un grand nombre d’observations générales. M. Raynouard, homme de cabinet et d’origine provençale, est un admirateur enthousiaste des idées et des procédés romains. M. Leber, érudit d’un immense savoir, mais en même temps administrateur pratique, et né dans une province moins complètement romanisée que M. Raynouard, est infiniment plus prudent dans ses éloges, et souvent cette prudence va jusqu’au blâme. Ce sont deux ouvrages curieux, bien que le second soit supérieur au premier. J’en ai beaucoup usé dans ces pages ; mais comme, malheureusement, je ne les ai pas sous les yeux, je suis réduit à citer de souvenir.  – Savigny, \emph{Geschichte des rœmischen Rechtes im Mittelalter}, in-8°, Heidelberg, 1815, t. I, p. 18 et pass.} Je veux que, dans les petites affaires, leur action demeurât assez libre. Il ne faut pas l’oublier, aussitôt qu’il s’agissait des demandes du fisc, plus de délibération, pas de raisonnements, bourse déliée ! Or ces demandes étaient fréquentes et peu discrètes \footnote{Je n’oserais ici me montrer aussi sévère, quoique je puisse le sembler beaucoup, qu’un écrivain dont le secours m’était assez inattendu dans une lutte contre des opinions dont M. Amédée Thierry est le principal propagateur. Je vais me couvrir de son autorité bien puissante en cette rencontre. Voici ce qu’il dit : « Sous le prétexte humain de gratifier le « monde d’un titre flatteur, un Antonin appela dans ses édits du nom de citoyens romains « les tributaires de l’empire romain, ces hommes qu’un consul pouvait légalement torturer, « battre de coups, écraser de corvées et d’impôts. Ainsi fut démentie la puissance de ce « titre autrefois inviolable, et devant lequel s’arrêtait la tyrannie la plus éhontée ; ainsi « périt ce vieux cri de sauvegarde qui faisait reculer les bourreaux : \emph{Je suis citoyen} « \emph{romain.} » (Augustin Thierry, \emph{Dix ans d’études historiques}, in-12, Paris, 1846, p. 188.)}. Pour quelques empereurs qui, dans un long principat, trouvèrent le loisir de régler leur appétit, combien n’en vit-on pas davantage qui, pressés de s’asseoir à la table du monde, n’eurent que le temps d’y dévorer ce que leurs mains purent saisir ? Et encore, parmi les princes favorisés d’un beau règne, combien y en eut-il que des guerres presque incessantes ne forcèrent pas de dévorer la substance de leurs peuples ? Et enfin, parmi les pacifiques, combien encore en peut-on citer dont les plus belles années ne se soient passées à diriger les meilleures ressources de l’empire contre les flots d’usurpateurs sans cesse renaissants, qui, de leur côté, emportaient aux villes tout ce qui était à prendre ? Le fisc ne fut donc presque jamais, excepté sous les Antonins, en disposition de ménager ses exigences ; et ainsi les magistrats municipaux avaient pour principale fonction, pour préoccupation première, de jeter de l’argent dans les caisses impériales, ce qui ôtait beaucoup au mérite de leur quasi-indépendance sur le reste, ou plutôt la réduisait à néant.\par
Le décurion, le sénateur, les vénérables membres de la curie, comme ils s’intitu­laient, car ces gens-là, descendus de quelques méchants affranchis, de marchands d’esclaves, de vétérans colonisés, tranchaient du patricien et du vieux Quirite, n’étaient pas toujours en mesure de remettre à l’agent du fisc la quote-part que celui-ci avait ordre d’exiger. Voter n’était rien, il fallait percevoir, et quand la commune était épuisée, à bout de voies, ruinée, les citoyens romains qui la composaient pouvaient sans doute être bâtonnés jusqu’à extinction de force par les appariteurs et gardes de police de la localité ; mais en espérer des sesterces, c’était illusoire. Alors l’officier impérial, victime lui-même de ses supérieurs, n’hésitait pas longtemps. Il faisait, à son tour, appel à ses propres licteurs, et demandait sans façon aux vénérables, aux illustres sénateurs de parfaire sur leurs propres fonds la somme à lui nécessaire pour établir ses comptes. Les illustres sénateurs refusaient, trouvant l’exigence mal placée, et alors, mettant de côté tout respect, on leur infligeait le même traitement, les mêmes ignominies dont ils se montraient si prodigues envers leurs libres administrés \footnote{Savigny, \emph{Geschichte des rœmischen Rechtes im Mittelalter}, t. I, p. 25.  – Certains dignitaires des curies municipales jouissaient d’heureux privilèges au point de vue des peines corporelles, auxquelles ils n’étaient pas astreints comme leurs collègues ; mais, en revanche, on était en droit de leur imposer de plus fortes amendes. (\emph{Ibid}., p. 71.)}.\par
 Il arriva de ce régime que bientôt les curiales, désabusés sur les mérites d’une toge qui ne les garantissait pas des meurtrissures, fatigués de siéger dans un capitole qui ne préservait pas leurs demeures des visites domiciliaires et de la spoliation, épouvantés des menaces de l’émeute qui, sans se préoccuper de rechercher les légitimes objets de sa colère, se ruait sur eux, tristes instruments, ces misérables curiales s’accordèrent à penser que leurs honneurs étaient trop lourds et qu’il valait mieux préférer une existence moins en vue, mais plus calme. Il s’en trouva qui émigrèrent et allèrent s’établir, simples citoyens, dans d’autres villes. Quelques-uns entrèrent dans la milice, et, quand le christianisme fut devenu religion légale, beaucoup se firent prêtres.\par
Mais ce n’était pas le compte du fisc. L’empereur rendit donc des lois pour dénier aux curiales, sous les peines les plus sévères, le droit d’abandonner jamais le lieu de leurs fonctions. Peut-être était-ce la première fois que des malheureux étaient cloués, de par la loi, au pilori des grandeurs \footnote{Voir, pour la situation quasi-aristocratique de \emph{l’ordo decurionum} sous les empereurs, Savigny, \emph{Geschichte des rœmischen Rechtes im Mittelalter}, t. I, p. 22 et seqq. Au même lieu, le détail de la vie misérable du curiale. L’auteur que je cite est d’avis que rien ne peut donner une plus juste idée de la décomposition intérieure de l’État sous les principats chrétiens que les constitutions théodosiennes ayant trait aux curies municipales. Non seulement les curiales ne voulaient pas l’être, mais ils préféraient même le servage, et il fallait une loi pour leur fermer ce refuge. On en vint même à cette étrange ressource de condamner des gens poursuivis pour crime à l’état de décurions. À la vérité, un décret impérial restreignit l’usage de cette singulière pénalité au châtiment des ecclésiastiques indignes, et des militaires qui, par lâcheté, s’étaient soustraits aux ordres de leurs chefs. (Savigny, \emph{loc. cit.})}. Puis, de même que, pour abaisser et avilir le sénat de Rome, on avait interdit à ses membres le métier de la guerre, de même, pour conserver au fisc les sénateurs provinciaux et l’exploitation de leurs fortunes, on défendit à ceux-là de se faire soldats, et par extension de quitter la profession de leurs pères, et, par extension encore, la même loi fut appliquée aux autres citoyens de l’empire ; de sorte que, par le plus singulier concours de convenances politiques, le monde romain, qui n’avait plus de races différentes à isoler les unes des autres, fit ce qu’avaient décrété le brahmanisme et le sacerdoce égyptien ; il prétendit créer des castes héréditaires, lui, le vrai génie de la confusion ! Mais il est des moments où la nécessité du salut force les États comme les individus aux plus monstrueuses inconséquences.\par
Voilà les curiales qui ne peuvent être ni soldats, ni marchands, ni grammairiens, ni marins ; ils ne peuvent être que curiales, et, tyrannie plus monstrueuse au milieu de la ferveur passionnée du christianisme naissant, on vit, au grand mépris de la conscience, la loi empêcher ces misérables d’entrer dans les ordres sacrés, toujours parce que le fisc, tenant en eux le meilleur de ses gages, ne voulait pas les lâcher \footnote{Tacite a pu mettre avec toute vérité ces mots dans la bouche d’Arminius : « Aliis gentibus, « ignorantia imperii romani, inexperta esse supplicia, nescia tributa. » (\emph{Ann}., 1. I, 59.)}.\par
De pareilles extrémités ne sauraient se produire chez des nations où un génie ethnique un peu noble souffle encore ses inspirations aux multitudes. La honte en retombe tout entière, non pas sur les gouvernements, que l’avilissement des peuples contraint d’y avoir recours, mais sur ces peuples dégénérés \footnote{Au milieu de ses déclamations, toujours défavorables à la puissance suprême, Tacite se laisse aller une fois à un singulier aveu. Il raconte qu’après avoir épié les délibérations du sénat, Tibère allait s’asseoir dans un angle du prétoire et assistait aux jugements ; puis il ajoute : « Bien des arrêts, par l’effet de sa présence, furent rendus contrairement aux « intrigues, aux prières des puissants ; mais, tandis que l’équité était sauve, la liberté se « perdait. » (\emph{Ann.}, I, 75.) La liberté de quoi ? la liberté de faire pendre l’innocent et de ruiner le pauvre ? Quand une nation en est au point des Romains de l’empire, le premier de ses besoins, c’est un maître ; un maître seul peut lui éviter des convulsions incessantes. Le génie de Tibère suppléait à la honteuse inertie du sénat et du peuple ; sa férocité était à tout le moins excusable par l’abjection sanguinaire de l’un et de l’autre. Ce qu’il tuait valait à peine la pitié, et il eût sans doute ménagé davantage des hommes qui n’eussent pas mérité de sa part cette réflexion empreinte du plus profond dégoût, et qui lui échappait chaque fois qu’il sortait du sénat : « O homines ad servitutem paratos ! » (Tac., \emph{Ann.}, III, 65.)}. Ceux-ci s’accommodaient de vivre sous ce joug. On connut à la vérité, dans le monde romain, quelques insurrections partielles, causées par l’excès des maux ; mais ces bagauderies, stimulées par la chair en révolte et ne s’appuyant sur rien de généreux, ne furent toujours qu’un surcroît de fléaux, qu’une occasion de pillage, de massacres, de viols, d’incendie. Les majorités n’en apprenaient l’explosion qu’avec une légitime horreur, et, la révolte une fois étouffée dans le sang, chacun s’en félicitait, et avait raison de le faire. Bientôt, n’y songeant plus, on continuait à souffrir le plus patiemment possible ; et, comme rien ne se prend plus vite que les mœurs de la servitude, il devint bientôt impossible aux gens du fisc d’obtenir le payement des impôts sans recourir à des violences. Les curiales ne tiraient rien de leurs administrés les plus solvables qu’en les faisant assommer, et, à leur tour, ils ne lâchaient guère que sur reçu de coups de verges. Morale particulière très comprise en Orient, où elle forme une sorte de point d’honneur. Même en temps ordinaire et sous des prétextes d’utilité locale, les curiales en arrivèrent à dépouiller leurs concitoyens, et les magistrats impériaux les en laissaient libres, trop heureux de savoir où trouver l’argent au jour du besoin.\par
Jusqu’ici, j’ai admis très bénévolement que les gens de l’empereur se tenaient immaculés de la corruption générale ; mais la supposition était gratuite. Ces hommes avaient tout autant de rapacité que les anciens proconsuls de la république. De plus, ils étaient bien autrement nombreux, et, quand les provinces épuisées prétendaient réclamer auprès du maître commun, on peut juger si la chose était facile. Tenant l’administration des postes impériales, dirigeant une police nombreuse et active, ayant seuls le droit d’accorder des passeports, les tyrans locaux rendaient presque impossible le départ de mandataires accusateurs. Si toutes ces précautions préalables se trouvaient déjouées, que venaient faire dans le palais du prince d’obscurs provinciaux, desservis par tous les amis, par les créatures, les protecteurs de leur ennemi ? Telle fut l’administration de la Rome impériale, et, bien que je concède aisément que tout le monde y jouissait du titre de citoyen, que l’empire était gouverné par un chef unique, et que les villes, maîtresses de leur régime intérieur, pouvaient s’intituler à leur gré autonomes, frapper monnaie, se dresser des statues et tout ce qu’on voudra, je n’en comprends pas davantage le bien qui en résultait pour personne \footnote{Les magistratures locales étaient, en principe, dispensatrices suprêmes du droit sur tout le territoire ; mais, en fait, elles n’exerçaient que le jugement en première instance ; l’appel se faisait aux officiers impériaux, et même elles n’appliquaient leur juridiction que dans les affaires minimes ne dépassant pas une certaine somme. Les contestations entre les cités, entre les autorités d’une même ville, le jugement au criminel, etc., ressortissaient aux tribunaux du souverain. (Savigny, \emph{Geschichte des rœmischen Rechtes im Mittelalter}, t. I, p. 35 et seqq.)}.\par
 Le suprême éloge adressé à ce système romain, c’est donc d’avoir été ce qu’on nomme régulier et unitaire. J’ai dit de quelle régularité ; voyons maintenant de quelle unité.\par
Il ne suffit pas qu’un pays ait un maître unique pour que le fractionnement et ses inconvénients en soient bannis. À ce titre, l’ancienne administration de la France aurait été unitaire, ce qui n’est l’avis de personne. Unitaire également se fût montré l’empire de Darius, autre chose fort contredite, et, à ce prix-là, ce qu’on avait connu sous telle monarchie assyrienne était aussi de l’unité. La réunion des droits souverains sur une seule tête, ce n’est donc pas assez ; il faut que l’action du pouvoir se répande d’une manière normale jusqu’aux dernières extrémités du corps politique ; qu’un même souffle circule dans tout cet être et le fasse tantôt mouvoir, tantôt dormir dans un juste repos. Or, quand les contrées les plus diverses s’administrent chacune d’après les idées qui leur conviennent, ne relèvent que financièrement et militairement d’une autorité lointaine, arbitraire, mal renseignée, il n’y a pas là cohésion véritable, amalgame réel. C’est une concentration approximative des forces politiques, si l’on veut ; ce n’est pas de l’unité.\par
Il est encore une condition indispensable pour que l’unité s’établisse et témoigne du mouvement régulier qui est son principal mérite ; c’est que le pouvoir suprême soit sédentaire, toujours présent sur un point désigné, et de là fasse diverger sa sollicitude, par des moyens, par des voies, autant que possible uniformes, sur les villes et les provinces. Alors seulement les institutions, bonnes ou mauvaises, fonctionnent comme une machine bien montée. Les ordres circulent avec facilité, et le temps, ce grand et indispensable agent de tout ce qui se fait de sérieux dans le monde, peut être calculé, mesuré et employé sans prodigalité inutile, comme aussi sans parcimonie désastreuse.\par
Cette condition manqua toujours à l’organisation impériale. J’ai montré comment la plupart des maîtres de l’État avaient, dès le principe, abandonné Rome, pour se fixer tantôt à l’extrémité méridionale de l’Italie, tantôt au nord des Gaules, tandis que d’autres voyagèrent pendant toute la durée de leur règne. Que pouvait être une administration dont les agents ne savaient où trouver sûrement le chef de qui émanait leur pouvoir, et dont ils étaient censés n’exécuter que les ordres ? Si l’empereur s’était constamment tenu à Antioche, il aurait fallu, sans doute, beaucoup de temps pour faire parvenir ses instructions aux prétoires de Cadix, de Trèves ou de l’île de Bretagne ; cependant, à tout prendre, on aurait pu calculer sur cet éloignement la constitution de ces provinces lointaines, l’étendue de la responsabilité accordée aux magistrats pour les régir et les défendre : on serait parvenu ainsi, tant bien que mal, à leur donner une organisation régulière.\par
Mais, quand un messager parti de Paris ou d’Italica pour prendre des ordres, arrivait lentement à Antioche, et apprenait là que l’empereur était parti pour Alexandrie ; que, le mandataire provincial parvenu dans cette ville, un nouveau départ l’amenait à Naples, et pouvait l’entraîner au delà du Rhin vers les limites décumates, en quoi, je le demande, une telle organisation avait-elle le caractère unitaire ? L’affirmer, c’est soutenir l’absurde ; l’empereur devait laisser, et laissait en effet, à l’initiative du préfet et des généraux une indépendance d’action d’où résultaient les conséquences les plus graves, tant pour la bonne administration du territoire que pour les plus hautes questions, l’hérédité impériale, par exemple.\par
Si le gouvernement avait été unitaire, ses forces vives étant rassemblées autour du trône, c’eût été à la cour même du prince décédé que la capacité de succession aurait été débattue ; il n’en était nullement ainsi. Quand l’empereur mourait en Asie, son héritier se révélait parfaitement en Illyrie, en Afrique ou dans l’île de Bretagne, suivant que, dans l’une ou l’autre de ces provinces, il s’improvisait un souverain qui avait su rattacher à sa cause plus d’intérêts, et qui ainsi jouissait d’un pouvoir plus étendu. Chaque grande circonscription de l’État possédait dans sa ville principale une cour en miniature où le pouvoir, tout délégué qu’il fût, prenait les allures d’une autorité suprême et absolue, disposait de tout en conséquence, et interprétait les lois mêmes, allant jusqu’à confisquer l’impôt, sans souci du trésor. Je ne nie pas que la foudre du dieu mortel, du héros souverain, n’éclatât quelquefois sur la tête des audacieux ; pourtant, dans la plupart des cas, ce n’était qu’après une longue tolérance d’où naissait l’excuse de l’abus. D’ailleurs, il n’était pas extrêmement rare que le magistrat récalcitrant, renvoyant la foudre d’où elle était partie et se déclarant empereur lui-même, ne démontrât le ridicule de ce fantôme d’unité monarchique qui cherchait, sans y parvenir, à embrasser et à féconder un monde soumis par son seul accablement. Ainsi, je ne saurais rien accorder de tout ce qu’on réclame désormais de sympathie théorique et de louanges pour l’époque impériale. Je me borne à être exact ; c’est pourquoi je termine en avouant que, si le régime inauguré par Auguste ne fut en lui-même ni beau, ni fécond, ni louable, il eut un genre de supériorité bien préférable encore : c’est qu’en face des populations multiples tombées au pouvoir des aigles, il était le seul possible. Tous les efforts, il les fit pour gouverner avec raison et honneur les masses qui lui étaient confiées. Il échoua. La faute n’en fut pas à lui : qu’elle retombe sur ces populations elles-mêmes.\par
Si le gouvernement fit sa religion d’une formule théologique sans valeur, d’un mot complètement vide de sens, je l’en absous. Il y avait été contraint par la nécessité de rester impartial entre mille croyances. Si, abolissant dans ses tribunaux d’appel les législations locales, il leur substitua une jurisprudence éclectique dont les trois bases étaient la servilité, l’athéisme et l’équité approximative, c’est qu’il s’était senti dominé par la même nécessité de nivellement. S’il avait, enfin, soumis ses procédés d’administration à une balance compliquée, relâchée, mal équilibrée entre la mollesse et la violence, c’est que, dans l’intelligence des masses sujettes, il n’avait pas trouvé de secours pour étayer un régime plus noble. Nulle part n’existait désormais la moindre trace d’aucune compréhension des devoirs sérieux. Les gouvernés n’étaient engagés à rien avec les gouvernants : faut-il donc accuser le chef, la tête de l’empire, de l’impuis­sance du corps \footnote{« \emph{Toute nation a le gouvernement qu’elle mérite.} De longues réflexions et une longue « expérience, payée bien cher, m’ont convaincu de cette vérité comme d’une proposition de « mathématiques. Toute loi est donc inutile et même funeste (quelque excellente qu’elle « puisse être en elle-même), si la nation n’est pas digne de la loi et faite pour la loi. » (Le comte de Maistre, \emph{Lettres et opuscules inédits}, t. I, p. 215.)} ? Ses défauts, ses vices, ses faiblesses, ses cruautés, ses oppressions, ses défaillances, et, de nouveau, ses enivrements furieux de domination, ses efforts insensés pour faire descendre le ciel sur la terre, et le mettre sous les pieds de son pouvoir que personne n’imaginait jamais assez énorme, assez divinisé, entouré d’assez de prestige, assez obéi, qui, avec tout cela, ne pouvait parvenir à se donner simplement l’hérédité, toutes ces folies ne provenaient d’autre chose que de l’épouvantable anarchie ethnique dominant cette société de décombres.\par
Les mots sont aussi impuissants à la rendre que la pensée à se la figurer. Essayons pourtant d’en prendre une idée en récapitulant à grands traits les principaux, seulement les principaux alliages auxquels avaient abouti les décadences assyrienne, égyptienne, grecque, celtique, carthaginoise, étrusque, et les colonisations de l’Espagne, de la Gaule et de l’Illyrie ; car c’est bien de tous ces détritus que l’empire romain était formé. Qu’on se rappelle que dans chacun des centres que j’indique il y avait déjà des fusions presque innombrables. Qu’on ne perde pas de vue que, si la première alliance du noir et du blanc avait donné le type chamitique, l’individualité des Sémites, des plus anciens Sémites, avait résulté de ce triple hymen noir, blanc et encore blanc, d’où était sortie une race spéciale ; que cette race, prenant un autre apport d’éléments noir, ou blanc, ou jaune, s’était, dans la partie atteinte, modifiée de manière à former une nouvelle combinaison. Ainsi à l’infini ; de sorte que l’espèce humaine, soumise à une telle variabilité de combinaisons, ne s’était plus trouvée séparée en catégories distinctes. Elle l’était désormais par groupes juxtaposés, dont l’économie se dérangeait à chaque instant, et qui, changeant sans cesse de conformation physique, d’instincts moraux et d’aptitudes, présentaient un vaste égrenage d’individus qu’aucun sentiment commun ne pouvait plus réunir, et que la violence seule parvenait à faire marcher d’un même pas \footnote{Dans ce pêle-mêle, les éléments septentrionaux étaient moins nombreux sans doute que ceux qui provenaient des régions méridionales. Ils méritent pourtant d’être remarqués plus qu’on ne l’a fait jusqu’ici. Beaucoup d’esclaves de race wende étaient répandus en Italie comme en Grèce bien avant le dernier siècle de la république. Les noms donnés aux personnages serviles par les poètes de la nouvelle comédie et par l’école latine de Plaute et de Térence en font foi. On peut aussi attribuer à des Slaves romanisés certaines inscriptions, gravées sur des tombes ou sur des instruments, que Mommsen et Lepsius ont citées et que M. Wolanski a interprétées d’une manière exacte par le slave. Je crois seulement que Mommsen, comme M. Wolanski, attribue une antiquité beaucoup trop haute à ces monuments d’ailleurs curieux en eux-mêmes.  – Voir Mommsen, \emph{Die unter-italischen Dialekte}, et Wolanski, \emph{Schriftdenkmale der Slawen.}}. J’ai appliqué à la période impériale le nom de sémitique. Il ne faut pas prendre ce mot comme indiquant une variété humaine identique à celle qui résulta des anciens mélanges chaldéens et chamites. J’ai seulement prétendu indiquer que, dans les multitudes répandues avec la fortune de Rome sur toutes les contrées soumises aux Césars, la majeure partie était affectée d’un alliage plus ou moins grand de sang noir, et représentait ainsi, à des degrés infinis, une combinaison, non pas équivalente, mais analogue à la fusion sémitique. Il serait impossible de trouver assez de noms pour en marquer les nuances innombrables et douées pourtant, chacune, d’une individualité propre que l’instabilité des alliances combinait à tout moment avec quelque autre. Cependant, comme l’élément noir se présentait en plus grande abondance dans la plupart de ces produits, certaines des aptitudes fondamentales de l’espèce mélanienne dominaient le monde, et l’on sait que, si, contenues dans de certaines limites d’intensité, et appariées avec des qualités blanches, elles servent au développement des arts et aux perfectionnements intellectuels de la vie sociale, elles se montrent peu favorables à la solidité d’une civilisation sérieuse.\par
Mais l’égrenage des races n’aboutissait pas uniquement à rendre impossible un gouvernement régulier, en détruisant les instincts et les aptitudes générales d’où seulement résulte la stabilité des institutions ; cet état de choses attaquait encore, d’une autre façon, la santé normale du corps social en faisant éclore une foule d’individualités pourvues fortuitement de trop de forces, et exerçant une action funeste sur l’ensemble des groupes dont elles faisaient partie. Comment la société serait-elle restée assise et tranquille quand, à tout instant, quelque combinaison des éléments ethniques en perpétuelle pérégrination et fusion créait en haut, en bas, au milieu de l’échelle, et plus souvent en bas qu’ailleurs, parce que là il y a plus de place pour les appariements de hasard, des individualités qui naissaient armées de facultés assez puissantes pour agir, chacune dans un sens différent, sur leurs voisins et leurs contemporains ?\par
Dans les époques où les races nationales se combinent harmonieusement, les hommes de talent jettent un plus vif éclat parce qu’ils sont plus rares, et ils sont plus rares parce que, ne pouvant, issus qu’ils sont d’une masse homogène, que reproduire des aptitudes et des instincts très répandus autour d’eux, leur distinction ne vient pas du disparate de leurs facultés avec celles des autres hommes, mais bien de l’opulence plus grande dans laquelle ils possèdent les mérites généraux. Ces créatures-là sont donc bien réellement grandes, et, comme leur pouvoir supérieur ne consiste qu’à mieux démêler les voies naturelles du peuple qui les entoure, elles sont comprises, elles sont suivies et font faire, non pas des phrases brillantes, non pas même toujours de très illustres choses, mais des choses utiles à leur groupe. Le résultat de cette concordance parfaite, intime, du génie ethnique d’un homme supérieur avec celui de la race qu’il guide, se manifeste par ceci, que, si le peuple est encore dans l’âge héroïque, le chef se confond plus tard, pour les annalistes, avec la population, ou bien la population avec le chef \footnote{Ainsi les récits mythologiques de la Grèce parlent des exploits d’Hercule sans jamais mentionner ses compagnons, et les chefs de différents peuples voyageurs ne sont autres que la personnification des nations elles-mêmes ; Leck ou Tschek, suivant les légendes, a dirigé les exploits des Lecks, Suap ceux des Souabes, Saxneat ceux des Saxons, Francus ceux des Franks, etc. (Schaffarik, \emph{Slawische Alterthümer}, t. I, p. 235.)}. C’est ainsi que l’on parle de l’Hercule Tyrien seul sans mentionner les compagnons de ses voyages, et, au rebours, dans les grandes migrations, on a oublié généralement le nom du guide pour ne se souvenir que de celui des masses conduites. Puis, lorsque la lumière de l’histoire, devenue trop intense, empêche de telles confusions, on a toujours bien de la peine à distinguer, dans les actions et les succès d’un souverain éminent, ce qui constitue son œuvre personnelle de ce qui appartient à l’intelligence de sa nation.\par
À de pareils moments de la vie des sociétés, il est très difficile d’être un grand homme, puisqu’il n’y a pas moyen d’être un homme étrange. L’homogénéité du sang s’y oppose, et pour se distinguer du vulgaire il faut, non pas être autrement fait que lui, mais, au contraire, en lui ressemblant, dépasser toutes ses proportions. Quand on n’est pas très grand, on se perd toujours plus ou moins dans la multitude, et les médiocrités ne sont pas remarquées, puisqu’elles ne font que reproduire un peu mieux la physionomie commune. Ainsi les hommes d’élite demeurent isolés, comme le sont des arbres de haute futaie au milieu d’un taillis. La postérité, les découvrant de loin dans leur stature immense, les admire plus qu’elle ne fait leurs analogues à des époques où les principes ethniques trop nombreux et mal amalgamés font sortit la puissance individuelle de faits complètement différents.\par
Dans ces derniers cas, ce n’est plus uniquement parce qu’un homme a des facultés supérieures qu’il peut être déclaré grand. Il n’existe plus de niveau ordinaire ; les masses n’ont plus une manière uniforme de voir et de sentir. C’est donc tantôt parce que cet homme a saisi un côté saillant des besoins de son temps, ou bien même parce qu’il a pris son époque à rebours, qu’il se rend glorieux. Dans la première alternative, je reconnais César ; dans la seconde, Sylla ou Julien. Puis, à la faveur d’une situation ethnique bien composite, des myriades de nuances se développent au sein des instincts et des facultés humaines ; de chacun des groupes formant les masses, sort nécessairement une supériorité quelconque. Dans l’état homogène, le nombre des hommes remarqués était restreint ; ici, au sein d’une société formée de disparates, ce nombre se montre tout à coup très considérable, bigarré de mille manières, et depuis le grand guerrier qui étend les bornes d’un empire jusqu’au joueur de violon qui réussit à faire grincer d’une manière acceptable deux notes jusque-là ennemies, des légions de gens acquièrent la renommée. Toute cette cohue s’élance au-dessus des multitudes en perpétuelle fermentation, les tire à droite, les tire à gauche, abuse de leur impossibilité fatalement acquise de discerner le vrai même d’avoir une vérité au-dessus d’elles, et fait pulluler les causes de désordre. C’est en vain que les supériorités sérieuses s’efforcent de remédier au mal : ou bien elles s’éteignent dans la lutte, ou bien elles ne parviennent, au prix d’efforts surhumains, qu’à bâtir une digue momentanée. À peine ont-elles quitté la place que le flot se désenchaîne et emporte leur ouvrage.\par
Dans la Rome sémitique, les natures grandioses ne manquèrent pas. Tibère savait, pouvait, voulait et faisait. Vespasien, Marc-Aurèle, Trajan, Adrien je compterais en foule les Césars dignes de la pourpre, mais tous, et le grand Septime Sévère lui-même, se reconnurent impuissants à guérir le mal incurable et rongeur d’une multitude incohérente, sans instincts ni penchants définis, rebelle à se laisser diriger longtemps vers le même but, et pourtant affamée de direction. Trop imbécile pour rien compren­dre d’elle-même, et d’ailleurs empoisonnée par les succès des coryphées infimes qui, se faisant un public d’abord, un parti ensuite, arrivaient à la fin où il plaisait au ciel : plusieurs à d’éminents emplois, le plus grand nombre à la plantureuse opulence des délateurs, pas assez à l’échafaud. Il faut encore distinguer dans ces supériorités subalternes deux classes exerçant une action fort différente : l’une suivait la carrière civile, l’autre prenait la casaque militaire, et entrait dans les camps. Je ne saurais faire de celle-là, au point de vue social, que des éloges \footnote{On m’objectera les perturbations que les révoltes militaires amenèrent souvent dans l’empire. Je répondrai que l’armée, pouvant tout, abusa souvent, et que c’est là un inconvénient de l’omnipotence ; mais je renvoie au spectacle même de ces commotions, par exemple, aux luttes sanglantes des légions de Germanie contre les Flaviens dans Rome, pour qu’on ait à se convaincre que les soldats étaient, malgré leur brutalité, bien supérieurs en toute manière à la population civile. Je n’en veux pour gage que leur bizarre fidélité à Vitellius. (Tac., \emph{Hist.}, III.)}.\par
 En effet, la nécessité unique, pour me servir de l’expression d’un antique chant des Celtes \footnote{La Villemarqué, \emph{Chants populaires de la Bretagne}, t. I, p. 1.}, n’admet pour les armées qu’un seul mode d’organisation, le classement hiérarchique et l’obéissance. Dans quelque état d’anarchie ethnique que se trouve un corps social, dès qu’une armée existe, il faut sans biaiser lui laisser cette règle invariable. Pour ce qui concerne le reste de l’organisme politique, tout peut être en question. On y doutera de tout ; on essayera, raillera, conspuera tout ; mais, quant à l’armée, elle restera isolée au milieu de l’État, peut-être mauvaise quant à son but principal, mais toujours plus énergique que son entourage, immobile, comme un peuple facticement homogène. Un jour, elle sera la seule partie saine et partant agissante de la nation \footnote{Toutefois l’armée n’aura de mérite réel, outre une plus grande subordination, ce qui est, après tout, une valeur négative, tout indispensable qu’elle soit, que si elle est composée de meilleurs éléments ethniques que le corps social auquel elle prête son appui. C’est précisément ce qui arriva pour les légions de Rome, ainsi que je l’expose en lieu utile. De même, en notre temps, les troupes mantchoues sont certainement supérieures aux populations chinoises ; mais, comme elles sont aussi recrutées un peu trop parmi ces populations, leur mérite militaire laisse beaucoup à désirer. Ce qu’il y a d’excellent dans la loi des camps ne saurait neutraliser que dans une certaine mesure les mauvaises conséquences des mélanges.}. C’est dire qu’après beaucoup de mouvement, de cris, de plaintes, de chants de triomphe étouffés bientôt sous les débris de l’édifice légal, qui, sans cesse relevé, sans cesse s’écroule, l’armée finit par éclipser le reste, et que les masses peuvent se croire encore quelquefois aux temps heureux de leur vigoureuse enfance où les fonctions les plus diverses se réunissaient sur les mêmes têtes, le peuple étant l’armée, l’armée étant le peuple. Il n’y a pas trop à s’applaudir, toutefois, de ces faux-semblants d’adolescence au sein de la caducité ; car, parce que l’armée vaut mieux que le reste, elle a pour premier devoir de contenir, de mater, non plus les ennemis de la patrie, mais ses membres rebelles, qui sont les masses.\par
Dans l’empire romain, les légions furent ainsi la seule cause de salut qui empêchait la civilisation de s’engloutir trop vite au milieu des convulsions sans cesse déterminées par le désordre ethnique, Ce furent elles seules qui fournirent les administrateurs de premier rang, les généraux capables de maintenir le bon ordre, d’étouffer les révoltes, de défendre les frontières, et, bref, ces généraux étaient la pépinière d’où sortaient les empereurs, la plupart assurément moins considérables encore par leur dignité que par leurs talents ou leur caractère. La raison en est transparente et facile à pénétrer. Sortis presque tous des rangs inférieurs de la milice, ils étaient, par la vertu de quelque grande qualité, montés de grade en grade, avaient dépassé le niveau commun par quelque heureux effort, et, portés aux alentours du dernier et plus sublime degré, s’étaient mesurés avant de le franchir avec des rivaux dignes d’eux et sortis des mêmes épreuves. Il y eut des exceptions à la règle ; mais je tiens le catalogue impérial sous mes yeux, et je ne me laisserai pas dire que la majorité des noms ne confirme pas ce que j’avance.\par
L’armée était donc non seulement le dernier refuge, le dernier appui, l’unique flambeau, l’âme de la société, c’était elle encore qui, seule, fournissait les guides suprêmes, et généralement les donnait bons. Par l’excellence du principe éternel sur lequel repose toute organisation militaire, principe qui n’est d’ailleurs que l’imitation imparfaite de cet ordre admirable résultant de l’homogénéité des races, l’armée faisait tourner à l’avantage général le mérite de ses supériorités de premier rang, et contenait l’action des autres d’une manière encore profitable par l’influence de la hiérarchie et de la discipline, Mais, dans l’ordre civil, il en était tout autrement : les choses ne s’y passaient pas si bien.\par
Là, un homme, le premier venu, qu’une combinaison fortuite des principes ethniques accumulés dans sa famille rendait quelque peu supérieur à son père et à ses voisins, se mettait le plus souvent à travailler dans un sens étroit et égoïste, indépendant du bien social. Les professions lettrées étaient naturellement la tanière où se tapissaient ces ambitions, car là, pour captiver l’attention et agiter le monde, il n’est besoin que d’une feuille de papier, d’un cornet d’encre et d’un médiocre bagage d’études. Dans une société forte, un écrivain ou un orateur ne se mettent pas en crédit sans être d’une haute volée. Personne ne s’arrêterait à écouter des massacres, car tout le monde a sur chaque chose le même parti pris et vit dans une atmosphère intellectuelle plus ou moins délicate, mais toujours sévère. Il n’en est pas de même aux temps des dégénérations. Chacun ne sachant que croire, ni que penser, ni qu’admirer, écoute volontiers celui qui l’interpelle, et ce n’est plus même ce que dit l’histrion qui plaît, c’est comme il le dit, et non pas s’il le dit bien, mais s’il le présente d’une manière nouvelle, et pas même nouvelle, mais bizarre, seulement inattendue. De sorte que, pour obtenir les bénéfices du mérite, il n’est pas nécessaire d’en avoir, il suffit de l’affirmer, tant on a affaire à des esprits appauvris, engourdis, dépravés, hébétés.\par
À Rome, depuis des siècles et à l’image de la Grèce croupissante, elle aussi dans la période sémitique, la carrière de tout adolescent sans fortune et sans courage était celle du grammairien. Le métier consistait à composer des pièces de vers pour les riches, à faire des lectures publiques, à prêter sa plume aux factums, aux pétitions, aux mémoi­res destinés aux curiales, voire aux préfets des provinces. Les téméraires risquaient des libelles, au risque de voir quelque jour leur dos et leur muse ressentir la mauvaise humeur d’un tribunal peu littéraire \footnote{Suet., \emph{Dom}., 8 : « Scripta famosa, vulgoque edita, quibus primores viri ac feminæ « notabantur, abolevit non sine auctorum ignominia. »}. Beaucoup encore se faisaient délateurs. La plupart de ces grammairiens menaient la vie d’Encolpe et d’Ascylte, héros débraillés du roman de Pétrone. On les rencontrait dans les bains publics, pérorant sous les colonnades \footnote{Bormanni, T. Perron., \emph{Satyr.}, VI : « Ingens scholasticorum turba in porticum venit. »}, chez les personnes qui donnaient à souper, et plus régulièrement dans les maisons de débauche, dont ils étaient les hôtes habituels et souvent les introducteurs. Ils menaient cette vie capricieuse et déhontée que l’euphémisme moderne appelle la vie d’artiste ou de bohème \footnote{Ibid.,X : « Quid ego, homo stultissime, facere debui, quum fame morerer ?... multo me « turpior es tu, hercule, qui, ut foris cœnares, poetam laudasti. Itaque ex turpissima lire in « risum diffusi, pacatius ad reliqua secessimus. »}. Ils s’introduisaient dans les familles opulentes à titre de précepteurs, et n’y donnaient pas toujours à leurs élèves les meilleures leçons de morale \footnote{\emph{Ibid.}, LXXXV.}.\par
 Plus tard, ceux qui ne s’arrêtaient pas aux débuts de cette existence de fantaisie, soit plus heureux, soit plus habiles, devenaient professeurs publics, rhéteurs patentés dans quelque municipe \footnote{Ce furent les méthodes d’enseignement adoptées par ces éducateurs d’enfants dont un personnage de Pétrone, rhéteur lui-même, parle en ces termes : « Et ideo ego « adolescentulos existimo in scholiis stultissimos fieri, quia nihil ex iis quæ in usu habemus « aut audiunt aut vident. Sed piratas cum catenis in littore stantes et tyrannos edicta « scribentes quibus imperent filiis, ut patrum eorum capita præcidant ; sed responsa in « pestilentia data ut virgines tres aut plures immolentur ; sed mellitos verborum globulos « et omnia dicta, factaque quasi papavere et sesamo sparsa. » (T. Petronii A., \emph{Satyricon}, I.)}. Alors ils se gourmaient en fonctionnaires, et ajoutaient un commentaire de leur façon aux milliers de gloses déjà publiées sur les auteurs. De cette catégorie sortaient les simples pédants ; ceux-là se mariaient et tenaient leur place au sein de la bourgeoisie. Mais le plus grand nombre ne se faisait pas jour dans ces fonctions laborieuses et enviées, bien que modestes ; il fallait donc continuer à vivre en dehors des classifications sociales. Avocats, rien ne distinguait les débutants romains des hommes de même profession dans tous les temps et tous les pays \footnote{Petron., \emph{Satyr.}, XV : « Advocati, tamen, jam pene nocturni, qui volebant pallium « luctifacere, flagitabant, uti apud se utraque depenerentur, ac posteto die judex querelam « inspiceret... Tam sequestri placebant, et nescio quis ex concionibus, calvus, « tuberosissimæ frontis, qui solebat aliquando et caussas agere, invaserat pallium, « exhibiturumque crastino die adfirmabat. »}. Ceux qui savaient marquer par l’éclat de leur parole ou la solidité de leur doctrine sortaient des barreaux obscurs et pouvaient prétendre aux augustes fonctions du prétoire. Plus d’un héros s’est trouvé parmi ceux-là. Les autres se nourrissaient de procès et gonflaient les basiliques de sophismes et d’arguties \footnote{ 
\bibl{Petron., \emph{Satyr.}, V :}
 
\begin{verse}
Det primos versibus annos,\\
 Mæoniumque bibat felici pectore fontem ;\\
 Mox et Socratico plenus grege, mater habenas\\
 Liber et ingentis quatiat magni Demosthenis arma.\\
\end{verse}
}. Mais l’avocature, le professorat, le métier de libelliste, ce n’était pas là ce qui attirait surtout la foule des lettrés, c’était la profession de philosophe.\par
On ne distinguait plus guère, quant aux mœurs, les différentes écoles : philosophe était l’homme portant barbe, besace et manteau à la grecque. Fût-il né dans les montagnes extrêmes de la Mauritanie, un manteau à la grecque était indispensable au vrai sage. Un tel vêtement donnait infailliblement cet air capable qui attirait le respect des amateurs. Du reste, on était platonicien, pyrrhonien, stoïcien, cynique ; on développait sous les portiques des villes les doctrines de Proclus, de Fronton ou, plus souvent, de leurs commentateurs, aujourd’hui ignorés, alors à la mode, peu importait ; l’essentiel était de savoir occuper les oisifs et mériter l’admiration du citadin, le mépris du soldat \footnote{Petron., \emph{Satyr}., III : « Minimum in his exercitationibus doctores peccant, qui necesse « habent cum insanientibus furere. Nam, nisi dixerint quæ adolescentuli probent, ut ait « Cicero, soli in scholiis relinquentur ; sicut ficti adulatores, quum cœnas divitura captant, « nihil prius meditantur quam id quod putant gratissimum auditoribus fore (nec enim « aliter impetrabunt, quod petunt, nisi quasdam insidias auribus fecerint) : sic eloquentiæ « magister, nisi, tamquam piscator, eam imposuerit hamis escam, quam scierit appetituros « esse pisciculos, sine spe prædæ moratur in scopulo. »}. La plupart de ces philosophes étaient des athées confirmés, et prêchaient des doctrines qui menaient là, ou pas loin. Quelques-uns, doués d’une éloquence hors ligne, parvenaient à plaire aux grands personnages, et, vivant à leurs frais, agissaient sur leurs résolutions ou sur leur conscience. Beaucoup, après avoir professé qu’il n’y avait pas de Dieu, ne trouvant pas leur métier assez lucratif, se faisaient isiaques, ou prêtres de Mithra, ou desservants d’autres divinités asiatiques découvertes par eux et qu’ils avaient l’ait d’inventer. C’était le goût dominant dans les hautes classes que d’aller jeter à la tête d’idoles, inconnues la veille, des flots d’adoration superstitieuse qui ne savaient plus où se répandre, depuis que les cultes réguliers n’étaient pas moins discrédités par la mode que les autres traditions nationales. Tous ces philosophes, tous ces savants, tous ces rhéteurs sémitisés étaient le plus souvent gens d’esprit. Ils tenaient généralement dans un coin de leur cervelle un système propre à régénérer le corps social ; mais, par un malheur fâcheux et qui paralysait tout, autant de têtes, autant d’avis, de sorte que les multitudes dont ils rêvaient de régler la vie intellectuelle se plongeaient de plus en plus, avec eux, dans un chaos inextricable.\par
Puis, effet naturel de l’abaissement des puissances ethniques et de l’énervement des races fortes, les aptitudes littéraires et artistiques avaient été chaque jour déclinant. Ce qu’on était contraint, par pauvreté, de considérer comme mérite, devenait très misérable. Les poètes ressassaient ce qu’avaient dit et redit les anciens. Bientôt le suprême talent se borna à copier d’aussi près que possible la forme de tel ou tel classique. On en arriva à s’extasier sur les centons. Le métier poétique en devint plus difficile. La palme appartenait à qui savait composer le plus de vers possible avec des hémistiches pris à Virgile ou à Lucain. De théâtres, depuis longtemps, plus l’ombre. Les mimes jadis avaient détrôné la comédie ; les acrobates, les gladiateurs, les coqs et les courses de chars avaient fait taire les mimes.\par
La sculpture et la peinture eurent le même sort  – ces deux arts se dégradèrent. D’un public sans idées il ne sortait plus de vrais artistes. Veut-on savoir dans quel genre d’écrits se réfugia la dernière étincelle de composition originale ? Dans l’histoire ; et par qui fut-elle le mieux écrite ? Par des militaires. Ce furent des soldats qui, surtout, rédigèrent l’Histoire Auguste. En dehors des camps, il y eut aussi sans doute des écrivains de génie et d’une rare élévation, mais ceux-là étaient inspirés par un sentiment surhumain, illuminés d’une flamme qui n’est pas terrestre : ce furent les Pères de l’Église.\par
On arguera peut-être, des œuvres de ces grands hommes, que, malgré ce qui précède, il était encore des cœurs fermes et honnêtes dans l’empire. Qui le nie ? Je parle des multitudes, et non des individualités. Bien certainement, au milieu de ces flots de misère, il subsistait encore çà et là, nageant dans le vaste gouffre, les plus belles vertus, les plus rares intelligences. Ces mêmes conjonctions fortuites d’éléments ethniques dispersés créaient, et, comme je l’ai remarqué dans le premier volume, en nombre même très considérable, les hommes les plus respectables par leur intégrité solide, leurs talents innés ou acquis. On en trouvait quelques-uns dans les sénats, on en voyait sous la saie des légionnaires, il s’en rencontrait à la cour. L’épiscopat, le service des basiliques, les réunions monacales en nourrissaient en foule, et déjà d’ailleurs des bandes de martyrs avaient certifié de leur sang que Sodome contenait encore bien des justes.\par
 Je ne prétends pas contredire cette évidence ; mais, je le demande, à quoi tant de vertus, à quoi tant de mérites, à quoi tant de génie servaient-ils au corps social ? Pouvaient-ils d’une minute arrêter sa pourriture ? Non ; les plus nobles esprits ne convertissaient pas la foule, ne lui donnaient pas du cœur. Si les Chrysostome et les Hilaire rappelaient à leurs contemporains l’amour de la patrie, c’était de celle d’en haut ; ils ne songeaient plus à la misérable terre que foulaient leurs sandales. Assurément on eût pu dénombrer beaucoup de gens de vertu qui, trop persuadés de leur impuissance, ou bien vivaient de leur mieux en sachant s’accommoder au temps, ou bien, et c’étaient les plus noblement inspirés, abandonnaient le monde à sa décrépitude et s’en allaient demander à la pratique de l’héroïsme catholique et au désert le moyen de se dégager sans faiblesse d’une société gangrenée. L’armée encore était un asile pour ces âmes froissées, un asile où l’honneur moral se conservait sous l’égide fraternelle de l’honneur militaire. Il s’y trouva en abondance des sages qui, le casque en tête, le glaive au côté et la lance à la main, allèrent par cohortes, sans regrets, tendre la gorge au couteau du sacrifice.\par
Aussi, quoi de plus ridicule que cette opinion, cependant consacrée, qui attribue à l’invasion des barbares du Nord la ruine de la civilisation ! Ces malheureux barbares, on les fait apparaître au V\textsuperscript{e} siècle comme des monstres en délire qui, se précipitant en loups affamés sur l’admirable organisation romaine, la déchirent pour déchirer, la brisent pour briser, la ruinent uniquement pour faire des décombres !\par
Mais, en acceptant même, fait aussi faux qu’il est bien admis, que les Germains aient eu ces instincts de brutes, il n’y avait pas de désordres à inventer au V\textsuperscript{e} siècle. Tout existait déjà en ce genre ; d’elle-même, la société romaine avait aboli depuis longtemps ce qui jadis avait fait sa gloire. Rien n’était comparable à son hébétement, sinon son impuissance. Du génie utilitaire des Étrusques et des Kymris Italiotes, de l’imagination chaude et vive des Sémites, il ne lui restait plus que l’art de construire encore avec solidité des monuments sans goût, et de répéter platement, comme un vieillard qui radote, les belles choses autrefois inventées. En place d’écrivains et de sculpteurs, on ne connaissait plus que des pédants et des maçons, de sorte que les barbares ne purent rien étouffer, par ce concluant motif que talents, esprit, mœurs élégantes, tout avait dès longtemps disparu \footnote{Au temps de Trajan, on avait déjà contracté l’habitude de se servir des anciennes statues pour glorifier les contemporains. On se contentait de changer les têtes, ce qui épargnait beaucoup de peine et d’invention.  – Voir, entre autres, la statue de Plotine, du musée du Louvre, n° 692. (Clarac, \emph{Manuel de l’Histoire de l’Art}, 1\textsuperscript{re} partie, p. 238.)  – Pétrone parle plusieurs fois de la profonde décadence des arts et surtout de la peinture, causée par l’amour exclusif que ses contemporains avaient pour le lucre : « Nolito ergo mirari, si pictura « deficit, quum omnibus diis hominibusque formosior videatur massa auri, quam « quidquid Apelles, Phidiasve, Græculi delirantes, fecerunt. » (Satyr., LXXXIX.)}. Qu’était, au physique et au moral, un Romain du III\textsuperscript{e} du IV\textsuperscript{e}, du V\textsuperscript{e} siècle ? Un homme de moyenne taille, faible de constitution et d’apparence, généralement basané, ayant dans les veines un peu du sang de toutes les races imaginables ; se croyant le premier homme de l’univers, et, pour le prouver, insolent, rampant, ignorant, voleur, dépravé, prêt à vendre sa sœur, sa fille, sa femme, son pays et son maître, et doué d’une peur sans égale de la pauvreté, de la souffrance, de la fatigue et de la mort. Du reste, ne doutant pas que le globe et son cortège de planètes n’eussent été faits pour lui seul.\par
En face de cet être méprisable, qu’était-ce que le barbare ? Un homme a blonde chevelure, au teint blanc et rosé, large d’épaules, grand de stature, vigoureux comme Alcide, téméraire comme Thésée, adroit, souple, ne craignant rien au monde, et la mort moins que le reste. Ce Léviathan possédait sur toutes choses des idées justes ou fausses, mais raisonnées, intelligentes et qui demandaient à s’étendre. Il s’était, dans sa nationalité, nourri l’esprit des sucs d’une religion sévère et raffinée, d’une politique sagace, d’une histoire glorieuse. Habile à réfléchir, il comprenait que la civilisation romaine était plus riche que la sienne, et il en cherchait le pourquoi. Ce n’était nullement cet enfant tapageur que l’on s’imagine d’ordinaire, mais un adolescent bien éveillé sur ses intérêts positifs, qui savait comment s’y prendre pour sentir, voir, comparer, juger, préférer. Quand le Romain vaniteux et misérable opposait sa fourberie à l’astuce rivale du barbare, qui décidait la victoire ? Le poing du second. Tombant comme une masse de fer sur le crâne du pauvre neveu de Rémus, ce poing musculeux lui apprenait de quel côté était passée la force. Et comment alors se vengeait le Romain écrasé ? Il pleurait, et criait d’avance aux siècles futurs de venger la civilisation opprimée en sa personne. Pauvre vermisseau ! Il ressemblait au contemporain de Virgile et d’Auguste comme Schylock au roi Salomon.\par
Le Romain mentait, et ceux qui, dans le monde moderne, par haine de nos origines germaniques et de leurs conséquences gouvernementales au moyen âge, ont amplifié ces hâbleries, n’ont pas été plus véridiques.\par
Bien loin de détruire la civilisation, l’homme du Nord a sauvé le peu qui en survivait. Il n’a rien négligé pour restaurer ce peu et lui rendre de l’éclat. C’est son intelligente sollicitude qui nous l’a transmis, et qui, lui donnant pour protection son génie particulier et ses inventions personnelles, nous a appris à en tirer notre mode de culture. Sans lui, nous ne serions rien. Mais ses services ne commencent pas là. Bien loin d’attendre l’époque d’Attila pour se précipiter, torrent aveugle et dévastateur, sur une société florissante, il était déjà depuis cinq cents ans l’unique soutien de cette société chaque jour plus caduque et plus avilie. À défaut de sa protection, de son bras, de ses armes, de son talent de gouverner, elle serait tombée, dès le II\textsuperscript{e} siècle, au point misérable où la réduisit Alaric, le jour qu’il culbuta si justement d’un trône ridicule l’avorton qui s’y prélassait. Sans les barbares du Nord, la Rome sémitique n’aurait pu maintenir la forme impériale qui la fit subsister, parce qu’elle ne serait jamais parvenue à créer cette armée qui seule conserva le pouvoir, lui recruta ses souverains, lui donna ses administrateurs, et, çà et là, sut allumer encore les derniers rayons de gloire qui enorgueillirent sa vieillesse.\par
Pour tout dire et sans rien outrer, presque tout ce que la Rome impériale connut de bien sortit d’une source germanique. Cette vérité s’étend si loin que les meilleurs laboureurs de l’empire, les plus braves artisans, on pourrait l’affirmer, furent ces lètes barbares colonisés en si grand nombre dans les Gaules et dans toutes les provinces septentrionales \footnote{Suivant Grimm, \emph{Deutsche Rechtsalterth.}, p. 305 et pass., les lètes formaient une classe intermédiaire entre les hommes libres et les esclaves. Schaffarik (t. I, p. 261, note 1) les considère comme descendus originairement des Lettes, Lettons ou Lithuaniens. Le mot allemand, \emph{Leute}, auquel M. Aug. Thierry rapporte cette étymologie, n’en serait que le dérivé. On disait \emph{læti Franci, læti Batavi, læti Suevi}, etc., probablement pour indiquer l’origine de ces différents lètes. (Guérard, \emph{Polyptique d’Irminon}, t. I, p. 251\emph{.  – Revue des Deux-Mondes}, 1\textsuperscript{er} mars 1852, p. 934 et 948.)}.\par
Quand enfin les nations gothiques vinrent en corps exercer un pouvoir qui, depuis des siècles, appartenait à leurs compatriotes, à leurs enfants mal romanisés, furent-elles coupables d’une révolution inique ? Non ; elles saisirent avec justice les fruits mûris par leurs soins, conservés par leurs labeurs, et que l’abâtardissement des races romaines laissait par trop corrompre. La prise de possession des Germains fut l’œuvre légitime d’une nécessité favorable. Depuis longtemps la démocratie énervée ne subsistait que grâce à la délégation perpétuelle du pouvoir absolu aux mains des soldats. Cet arrangement avait fini par ne plus suffire, l’abaissement général était devenu trop grand. Dieu alors, pour sauver l’Église et la civilisation, donna au monde ancien, non plus une troupe, mais des nations de tuteurs. Ces races nouvelles, le soutenant et le pétrissant de leurs larges mains, lui firent subir avec plein succès le rajeunissement d’Eson. Rien de plus glorieux dans les annales humaines que le rôle des peuples du Nord ; mais, avant de le caractériser avec l’exactitude qu’il exige, avant de montrer combien on a eu tort de clore la société romaine au jour des grandes invasions, puisqu’elle vécut encore longtemps après sous l’égide des envahisseurs, il convient de faire un temps d’arrêt et de rechercher une dernière fois ce que la réunion des anciens éléments ethniques du monde occidental, dans le vaste bassin de la romanité, avait, en définitive, offert de neuf à l’univers. On doit donc se demander si le colon romain avait su remanier de telle sorte ce que lui avaient légué les civilisations précédentes, qu’il en ait fait sortir des principes inconnus jusqu’à lui, et constituant ce qu’on aurait droit d’appeler une \emph{civilisation romaine.}\par
La question posée, qu’on entre dans les champs d’observation qu’elle ouvre aussitôt, vastes champs, démesurés comme les territoires ajoutés les uns aux autres qu’elle fait parcourir aux yeux. Tous sont déserts. Rome, n’ayant jamais eu de race originale, n’a jamais élaboré non plus une pensée qui le fût. L’Assyrie avait une empreinte particulière ; l’Égypte, la Grèce, l’Inde et la Chine de même. Les Perses avaient jadis dévoilé des principes aux regards des populations maîtrisées par leur glaive. Les Celtes, les aborigènes italiotes, les Étrusques possédèrent également leur patrimoine, à la vérité peu brillant, peu digne d’exciter l’admiration, mais réel, mais solide, mais positif et bien caractérisé.\par
Rome attira à elle un peu, un coin, un lambeau de toutes ces créations, à des moments où elles étaient déjà vieillies, salies, usées, à peu près hors de service. Dans ses murs, elle installa, non pas un atelier de civilisation où, d’un génie supérieur, elle ait jamais travaillé des œuvres frappées d’un cachet qui lui fût propre, mais un magasin d’oripeaux où elle entassa sans choix tout ce qu’elle déroba sans peine à l’impuissante vieillesse des nations de son temps. Imposante comme la fit la faiblesse de ses entours, elle ne le fut jamais assez pour combiner quoi que ce soit de général, ne fût-ce qu’un compromis étendu partout et à tout. Elle ne l’essaya même pas. Dans les localités diverses, elle laissa la religion, les mœurs, les lois, les constitutions politiques, à peu près comme elle les avait trouvées, se contentant d’énerver ce qui aurait pu gêner le contrôle dominateur que la nécessité la portait à se réserver.\par
Conduite par ce modèle unique, il lui fallut cependant déroger parfois plus gravement à ses habitudes d’inerte tolérance.\par
L’étendue de ses possessions constituait un fait qui, à lui seul, créait une situation et des obligations nouvelles. Ce fut donc sur ce terrain que, bon gré, mal gré, elle eut à montrer son savoir-faire. Il fut petit. Elle inventa très peu ; elle agit à la façon du jardinier qui taille les orangers et les buis de manière à leur faire prendre certaines formes, sans s’inquiéter autrement des lois naturelles qui dirigent la croissance de ces arbres.\par
L’action particulière de Rome se renferma dans l’administration et le droit civil \footnote{Tu, regere imperio populos, Romane, memento.}. Je ne sais jusqu’à quel point il serait jamais possible, en se bornant à ces deux spécialités, de donner naissance à des résultats réellement civilisateurs dans le sens large du mot. La loi n’est que la manifestation écrite de l’état des mœurs. C’est un des produits majeurs d’une civilisation, ce n’est pas la civilisation elle-même. Elle n’enrichit pas matériellement ni intellectuellement une société ; elle réglemente l’usage de ses forces, et son mérite est d’en amener une meilleure dispensation ; elle ne les crée pas. Cette définition est incontestable chez les nations homogènes. Toutefois il faut avouer qu’elle ne se présente pas d’une manière aussi claire, aussi immédiatement évidente, dans le cas particulier de la loi romaine. Il se pourrait, à la rigueur, que les éléments de ce code recueillis chez une multitude de nations vieillies, et partant expérimentées, résumassent une sagesse plus générale que ne faisait chacune des législations antérieures en son particulier, et de la constatation théorique de cette possibilité, on est facilement induit à conclure, sans y regarder de plus près, qu’en effet elle s’était réalisée dans la loi romaine. C’est l’opinion généralement reçue aujourd’hui. Cette opinion admet, fort à la légère, que le droit impérial découle d’une conception d’équité abstraite, dégagée de toute influence traditionnelle, hypothèse parfaitement gratuite. La philosophie du droit romain, comme la philosophie de toutes choses, a été faite après coup. Elle a surtout été inspirée par des notions complètement étrangères à l’antiquité, et qui eussent grandement surpris les légistes aux œuvres desquels elle se rattache.\par
Pour être nombreuses, les sources de cette jurisprudence ne sont pas infinies, et elles sont très positives. Les doctrines analytiques ont dû les influencer ; mais ces doctrines elles-mêmes, n’étant que des émanations de l’esprit italiote ou de l’imagination hellénistique, ne pouvaient rien y introduire de plus général. Quant au christianisme, il a été bien peu deviné par les juristes, car un des caractères remar­quables de leur monument, c’est l’indifférence religieuse. Certainement une telle donnée est des plus antipathiques aux tendances naturelles de l’Église, et elle l’a témoigné par la manière dont elle a réformé le droit romain, en en faisant le droit canonique.\par
Rome, étrangère dans ses propres murs, ne put, dès son origine, jamais avoir que des lois empruntées. Dans sa toute première période, sa législation était modelée sur celle du Latium, et, lorsque les \emph{Douze Tables} furent instituées pour répondre aux vues d’une population déjà composite, on y conserva quelques stipulations anciennes en les soutenant par une dose suffisante d’articles choisis dans les codes de la Grande-Grèce. Mais ce n’était pas encore satisfaire aux besoins d’une nation qui changeait à tout moment de nature et, par conséquent, de visées. Les immigrants abondant dans la Ville ne voulaient pas de cette compilation des décemvirs, étrangère en tour ou en partie à leurs idées nationales de justice. Les anciens habitants, qui, de leur côté, ne pouvaient modifier leur loi avec la même rapidité que leur sang, instituèrent un magistrat spécial chargé de régler les conflits entre les étrangers et les Romains, et les étrangers entre eux. Ce magistrat, le \emph{prætor peregrinus}, eut pour obligation distinctive de prendre sa jurisprudence en dehors des dispositions des \emph{Douze Tables.}\par
Quelques auteurs, trompés par la faveur dont jouissait, aux derniers temps de la république, la qualité de citoyen romain parmi les populations soumises, ont cru que cette préoccupation avait toujours existé, et ils l’ont supposée à tort pour les époques antérieures. C’est une faute grave. La concession du droit latin ou italiote n’était pas, à l’origine, une marque d’infériorité laissée par le sénat à ses vaincus. C’était, tout au contraire, un acte dicté par une prudente réserve vis-à-vis de peuples qui voulaient bien se soumettre à la suprématie politique des Romains, mais non pas à leur système juridique. Ces nations tenaient à leurs coutumes. On les laissa, et le \emph{prætor peregrinus}, qui devait juger ceux de leurs citoyens domiciliés dans la Ville, n’eut pas pour mission, en laissant de côté la loi locale, de chercher dans son imagination un idéal fantastique d’équité, mais d’appliquer de son mieux ce qu’il connaissait des principes de la justice positive en usage chez les Italiotes, les Grecs, les Africains, les Espagnols, les Gaulois amenés, pour la protection de leurs intérêts, devant son tribunal.\par
Et, en effet, si ce magistrat avait dû faire appel à sa force d’invention, celle-ci se fût adressée aussitôt à sa conscience. Or il était Romain, il avait les notions de son pays sur le juste et l’injuste ; il eût argumenté en Romain et, tout couramment, appliqué les prescriptions des Douze Tables, les plus belles du monde à ses yeux. C’était précisément là ce qu’il lui était commandé d’éviter. Il n’existait que pour ne pas prononcer ainsi. Il était donc tout naturellement forcé de s’enquérir des idées de ses justiciables, de les étudier, de les comparer, de les apprécier, et de tirer, pour son usage, des résultats de cette recherche, une conviction officielle, qui devenait pour lui le droit naturel, le droit des gens, le \emph{jus gentium.} Mais ce pot-pourri de doctrines positives ainsi combiné par un individu isolé, aujourd’hui magistrat, demain néant, n’avait rien d’évidemment juste et vrai. Aussi changeait-il avec les préteurs. Chacun d’eux arrivait en charge avec le sien, qui était contredit au bout de l’année d’exercice par celui d’un autre. Suivant que tel ou tel juge comprenait ou connaissait mieux telle législation étrangère, celle d’Athènes ou de Corinthe, de Padoue ou de Tarente, c’était la coutume d’Athènes, de Corinthe, de Padoue ou de Tarente qui composait la meilleure part de ce que, cette année-là, on nommait à Rome le droit des gens.\par
Quand le mélange romanisé fut à son comble, on s’ennuya avec raison de cette indigente mobilité. On força les \emph{prætores peregrini} à juger d’après des règles fixes, et, pour se procurer ces règles, on eut recours à la seule ressource admissible : on étudia, compila, amplifia des articles de lois pris dans tous les codes dont on put acquérir connaissance, et l’on produisit ainsi une législation sans nulle originalité, une législation qui ressemblait parfaitement aux races métisses et épuisées qu’elle était appelée à régir, qui avait gardé quelque chose de toutes, mais quelque chose d’indécis, d’incertain, d’à peine reconnaissable, et qui, dans cet état, se trouva convenir si bien à l’ensemble de la société qu’elle étouffa l’esprit sabin resté dans les Douze Tables, s’incorpora ce qu’elle en put conserver, peu de chose, et étendit son empire de toutes parts jusqu’aux points ou finissaient les voies romaines dans le dernier avant-poste des légions.\par
Pourtant une objection subsiste. Les grands légistes de la belle époque n’ont-ils pu réussir à extraire de tous ces lambeaux disparates, de tous ces membres arrachés à des codes souvent antipathiques, un suc tout nouveau devenu l’élément vital de ce corps de doctrines si laborieusement combiné, et donner à son ensemble une valeur que ses parties n’avaient pas ? Je répondrai que les plus éminents parmi les jurisconsultes ne s’appliquèrent pas à cette tâche. Pour la remplir, il leur aurait fallu sortir non seulement d’eux-mêmes, mais surtout de la société qui les absorbait. C’est une figure de rhétorique que de dire qu’un homme est plus grand que son siècle ; il n’est donné à personne d’avoir des yeux si perçants qu’ils dépassent l’horizon. Le \emph{nec plus ultra} du génie consiste à bien voir tout ce que cet horizon renferme. Les hommes spéciaux ne pouvaient acquérir et n’eurent de notions que celles existant autour d’eux. Il ne leur était pas loisible de prêter à leurs travaux une originalité qui ne s’offrait nulle part. Ils firent merveille dans l’appropriation des matériaux dont ils disposaient, dans l’art d’en tirer les conséquences pratiques que les plus subtils replis du texte pouvaient renfermer. Voilà ce qui les a faits grands, rien de plus, et c’est assez.\par
Mais, ajoutent quelques-uns, oubliez-vous ce suprême éloge mérité par le droit romain : son universalité ? Qu’est-ce à dire ? Il fut universel dans l’empire romain, oui. Il fut, il est en haute estime chez les peuples romanisés de tous les temps, j’en conviens. Mais, en dehors de ce cercle, nul esprit n’a jamais montré la moindre velléité de l’admettre. Lorsqu’il régnait avec toute sa plénitude sous la protection des aigles, il n’a pas fait une conquête hors de ses frontières. Les Germains l’ont vu pratiquer, l’ont même protégé chez leurs sujets, et ne l’ont jamais pris. Une grande partie de l’Europe actuelle, l’Amérique, l’étudient et ne l’adoptent pas. Que, dans les écoles, tel docteur lui voue son admiration, c’est une question de controverse ; mais, en mille endroits, en Angleterre, en Suisse, dans telles contrées de l’Allemagne, les mœurs le repoussent. En France même et en Italie, on ne saurait l’accepter sans des modifications profondes. Ce n’est donc pas la raison écrite, comme on l’a dit ambitieusement. C’est la raison d’un temps, d’un lieu, vaste sans doute, mais loin de l’être autant que la terre. C’est la raison spéciale d’une agglomération d’hommes, et nullement de la plupart des hommes ; en un mot, c’est une loi locale, comme toutes celles qui furent jusqu’ici. Ce n’est donc, en aucune manière, une invention qui mérite le nom d’universelle. Elle n’est pas suffisante pour se gagner toutes les consciences et réglementer tous les intérêts humains. Dès lors, puisqu’elle est si loin de pouvoir revendiquer avec justice un tel caractère ; puisque, d’ailleurs, elle ne contenait rien qui ne provienne d’une source qui, dans sa pureté, n’appartenait pas à Rome ; puisqu’elle n’a rien d’entier, de vivant, d’original, la loi romaine ne se trouve pas douée d’une action civilisatrice plus puissante que celle des autres législations. Elle ne fait donc pas exception, elle n’est qu’un résultat et non pas une cause de culture sociale ; elle ne saurait en aucune façon servir à caractériser une civilisation particulière.\par
Si le droit était ainsi dénué de principes vraiment nationaux, on en peut dire tout autant de l’administration, je l’ai montré ailleurs, et ce qu’on blâme aujourd’hui, avec tant de raison, dans les empires asiatiques modernes, cette indifférence profonde pour le gouverné, qui ne connaît le gouvernant et n’est connu de lui qu’à l’occasion de l’impôt et de la milice, existait absolument au même degré dans la Rome républicaine et dans la Rome impériale. La hiérarchie des fonctionnaires et leur manière de procéder étaient semblables, avec une nuance de despotisme de plus, à celle qui régissait les Perses, modèle que les Romains ont imité beaucoup plus souvent qu’on ne l’a dit. Du reste, l’administration comme la justice civile restaient soumises, dans la pratique, aux notions de moralité communément reçues. C’est sur ces points que l’on reconnaît combien l’empire des Césars est loin d’avoir rien produit de nouveau, d’avoir mis en circulation une idée ou un fait qui ne lui fût pas antérieur.\par
Un honnête homme romain, je l’ai dit en plus d’un lieu, n’était pas, très certainement, un phénix introuvable. Dans toutes les situations sociales, on rencontrait en abondance, au déclin de l’empire, de beaux et nobles caractères naturellement portés au bien et ne demandant pas mieux que de le faire. Mais l’honnête homme, dans toute société, se dirige en vue de l’idéal particulier créé par la civilisation au centre de laquelle il se trouve. Le vertueux Hindou, le Chinois intègre, l’Athénien de bonnes mœurs, sont des types qui se ressemblent surtout dans leur volonté commune de bien agir, et, de même que les différentes classes, les différentes professions, ont des devoirs spéciaux qui souvent s’excluent, de même la créature humaine est partout dominée, suivant les milieux qu’elle occupe, par une théorie préexistante au sujet des perfections dignes d’être recherchées. Le monde romain subissait cette loi comme les autres ; il avait, comme eux, son idéal du bien. Scrutons-le, et voyons s’il contenait ce principe nouveau que nous poursuivons, et qui jusqu’à ce moment nous a toujours échappé.\par
Hélas ! il en est ici de même que lorsqu’il s’est agi de la législation ; on n’aperçoit que des doctrines empruntées et écourtées. Tout ainsi que la philosophie venait en grande partie des Grecs, et n’abonda plus particulièrement vers le stoïcisme, dogme, en définitive, malgré ses beaux semblants, grossier et stérile, que sous l’influence du sang celtique-italiote, de même les vertus sabines, graduellement sémitisées, ne recelèrent rien que de très connu des premières races européennes. Le plus honnête homme et le plus doux ne croyait pas mal faire en exposant sa progéniture. Il eût estimé duperie et démence de pratiquer ou seulement de ressentir ces beaux mouvements d’abnégation qui font la base de la morale germanique et chevaleresque, et dont le christianisme tira si grand parti. J’ai beau regarder, je ne vois pas se développer dans la société romaine un seul sentiment, une seule idée morale dont je ne puisse retrouver l’origine, soit dans l’ancienne rudesse des aborigènes, soit dans la culture utilitaire des Étrusques, soit dans le raffinement composite des Grecs sémitisés, soit dans la spirituelle férocité de Carthage et de l’Espagne.\par
La tâche de Rome ne fut donc pas de donner au monde une floraison de nouveau­tés. L’immense puissance qui s’accumula dans ses mains ne produisit aucune amélioration, tout au contraire. Mais si l’on veut parler d’éparpillement de notions et de croyances, alors il faut tenir un bien autre langage. Rome exerça dans ce sens une action vraiment extraordinaire. Seuls, les Sémites et les Chinois seraient recevables à lui contester la prééminence. Rien de plus vrai, de plus évident. Si Rome n’éclaira pas, ne grandit pas les fractions de l’humanité tombées dans son orbite, elle hâta puissamment leur amalgame. J’ai dit les motifs qui m’empêchent d’applaudir à un tel résultat : le dénommer encore, c’est indiquer suffisamment que je suis loin de m’incliner devant la majesté du nom romain.\par
Cette majesté, cette grandeur ne dut la vie qu’à la prostration commune de tous les peuples antiques. Masse informe de corps expirants ou expirés, la force qui la soutint pendant la moitié de sa longue et pénible marche fut empruntée à ce qu’elle détestait le plus, à son antipode, à la barbarie, pour me servir de son expression. Acceptons, si l’on veut, et ce nom et l’intention insultante qui s’y attache. Laissons la tourbe romaine se hausser sur ses piédestaux ; il n’en est pas moins vrai que ce fut seulement à mesure que cette barbarie protectrice agrandit davantage et son influence et son action, qu’on voit poindre et régner enfin des notions dont le germe ne se trouvait plus nulle part dans l’ancien monde occidental, ni parmi les doctes concitoyens de Périclès, ni sous les ruines assyriennes, ni chez les premiers Celtes.\par
Cette action commença de bonne heure et se prolongea longtemps. De même, en effet, qu’il y avait eu une Rome étrusque, une Rome italiote, une Rome sémitique, il devait y avoir et il y eut une Rome germanique.
\chapterclose


\chapteropen
\chapter[{VI. la civilisation occidentale}]{VI. \\
la civilisation occidentale}\renewcommand{\leftmark}{VI. \\
la civilisation occidentale}


\chaptercont
\section[{VI.1. Les Slaves. – Domination de quelques peuples arians antégermaniques.}]{VI.1. \\
Les Slaves. – Domination de quelques peuples arians antégermaniques.}
\noindent Depuis le IV\textsuperscript{e} siècle jusque vers l’an 50 avant Jésus-Christ, les parties du monde qui se considéraient comme exclusivement civilisées, et qui nous ont fait partager cette opinion, c’est-à-dire les pays de sang et de coutumes helléniques, les contrées de sang et de coutumes italo-sémitiques, n’eurent que peu de contacts apparents avec les nations établies ou delà des Alpes. On eût pu croire que les seules de celles-ci qui eussent jamais menacé sérieusement le Sud, les Gaulois, s’étaient englouties dans les entrailles de la terre. Peu de bruit de ce qui se passait chez elles se répandait chez leurs voisins. Pour les savoir vivantes encore et même bien vivantes, il fallait être, comme les Massaliotes, involontairement soumis aux contrecoups de leurs discordes, ou, comme Posidonius, avoir voyagé dans ces régions qu’un peu bénévolement l’on avait peuplées jadis de terreurs plus fantastiques que réelles.\par
Les invasions celtiques ne s’étaient plus renouvelées. Leur fleuve dévastateur, qui jadis avait abouti à la fondation des États galates, était tari. Les descendants de Sigovèse avaient pris des allures si modestes que, quelques bandes d’entre eux s’étant pacifiquement transportées dans la haute Italie, avec l’intention d’y cultiver des terres vacantes, elles en sortirent sur une simple injonction du sénat, après avoir vu échouer les plus humbles supplications.\par
Ce repos que les Gaulois n’osaient plus troubler chez les autres peuples, ils n’en jouissaient pas eux-mêmes. La période de trois cents ans qui précéda la conquête de César fut pour eux une époque de douleur. Ils pratiquèrent, ils connurent à fond les phases les plus misérables de la décadence politique. Aristocratie, théocratie, royauté héréditaire ou élective, tyrannie, démocratie, démagogie, ils goûtèrent de tout, et tout fut transitoire \footnote{Cæs., de \emph{Bell. Gall}., VI.}. Leurs agitations ne réussissaient pas à produire de bons fruits. La raison en est que la généralité des nations celtiques en était arrivée à ce point de mélange, et partant de confusion, qui ne permet plus de progrès nationaux. Elles avaient dépassé le point culminant de leurs perfectionnements naturels et possibles  ; elles ne pouvaient désormais que descendre. Ce sont là cependant les masses qui servent de bases à notre société moderne, associées dans cet emploi avec d’autres multitudes, non moins considérables, qui sont les Slaves ou Wendes.\par
Ceux-ci, à l’époque dont il s’agit, étaient encore plus déprimés, dans la plupart de leurs nations, et l’étaient depuis beaucoup plus longtemps. Par la position topogra­phique qu’occupaient et occupent encore leurs principales branches, ils sont évidemment les derniers de tous les grands peuples blancs qui, dans la haute Asie, ont cédé sous les efforts des hordes finniques, et surtout ceux qui ont été le plus constamment en contact direct avec elles \footnote{Schaffarik, \emph{Slawische Alterth.}, t. I, p. 57.}. Ceci soit dit en faisant abstraction de quelques-unes de leurs bandes, entraînées dans les tourbillons voyageurs des Celtes, ou même les devançant, tels que les Ibères, les Rasènes, les Venètes des différentes contrées de l’Europe et de l’Asie. Mais, pour ce qui est du gros de leurs tribus, expulsées de la patrie primitive postérieurement au départ des Galls, elles n’ont plus trouvé à s’établir que dans les parties du nord-est de notre continent, et là jamais n’a cessé pour elles le voisinage dégradant de l’espèce jaune \footnote{Ouvr. cité, t. I, p. 74. – Schaffarik considère comme formant la première extension des Slaves en Europe, la région située entre l’Oder, le Niémen, le Bug, le Dnieper, le Dniester et le Danube. Mais ces limites ont très souvent changé.}. Plus elles en ont absorbé de familles, plus elles ont été constamment disposées à abonder dans de nouveaux hymens de même sorte \footnote{Ouvr. cité. – Le slave, pourvu des affinités originelles nécessaires avec les autres langues arianes montre la trace d’une grande influence exercée par la famille finnoise sur ses éléments constitutifs. (T. I, p. 47.)}. Aussi leurs caractères physiques sont-ils faciles à déchif­frer  ; les voici, tels que les décrit Schaffarik : « Tête approchant de la forme carrée, plus large que longue, front aplati, nez court avec tendance à la concavité ; les yeux horizontaux, mais creux et petits  ; sourcils minces rapprochés de l’œil à l’angle interne, et dès lors montants. Trait général, peu de poil \footnote{\emph{Ouvr. cité}, t, I, p. 33.}. »\par
Les aptitudes morales étaient en parfait accord, et n’ont jamais cessé de s’y maintenir, avec ces marques extérieures. Toutes leurs tendances principales aboutis­sent à la médiocrité, à l’amour du repos et du calme, au culte d’un bien-être peu exigeant, presque entièrement matériel, et aux dispositions les plus ordinairement pacifiques \footnote{\emph{Ibidem}, t. I, p. 66, 167.}. De même que le génie du Chamite, métis du noir et du blanc, avait tiré des aspirations véhémentes du nègre la sublimité des arts plastiques, de même le génie du Wende, hybride de blanc et de finnois, transforma le goût de l’homme jaune pour les jouissances positives en esprit industriel, agricole et commercial \footnote{\emph{Ibidem}, t. I, p. 1, 59.}. Les plus anciennes nations formées par cet alliage devinrent des nids de spéculateurs, moins ardents sans doute, moins véhéments, moins activement rapaces, moins généralement intelligents que les Chananéens, mais tout aussi laborieux et tout aussi riches, bien que d’une façon plus terne.\par
Dans une antiquité fort respectable, un affluent énorme de denrées diverses provenant des pays occupés par les Slaves appela vers le bassin de la mer Noire de nombreuses colonies sémitiques et grecques. L’ambre recueilli sur les rives de la Baltique, et que nous avons vu figurer dans le commerce des peuples galliques, passait aussi dans celui des nations wendes. Elles se le transmettaient de l’une à l’autre, l’amenaient jusqu’à l’embouchure du Borysthène et des autres fleuves de la contrée. Ce précieux produit répandait ainsi l’aisance chez ses différents facteurs, et faisait pénétrer jusqu’à eux une part des trésors métalliques et des objets fabriqués de l’Asie antérieure. À ce transit s’unissaient d’autres branches de spéculation non moins importantes, celle du blé, par exemple, qui, cultivé sur une très grande échelle dans les régions de la Scythie \footnote{Ouvr. cité, t. I, p. 271. – Schaffarik fait venir une grande partie de cette production des pays situés derrière les Karpathes. Mais il y avait aussi plus bas, dans la direction du sud-est, une nation à demi wende, celle des Alazons, qui se livrait au même commerce. (Hérod., IV, 17.)} et jusqu’à des latitudes impossibles à préciser, parvenait, au moyen d’une navigation fluviale organisée et exploitée par les indigènes, jusqu’aux entrepôts étrangers de l’Euxin. On le voit, les Slaves ne méritaient pas plus le reproche de barbarie que les Celtes \footnote{Ils vivaient dans des villages, à la façon des peuples blancs purs, leurs ancêtres. (Schaff., t. I, p. 59.) S’il était besoin d’en donner une preuve, on la trouverait dans le nom d’une tribu slave, les Budini, (alphabet étranger) dont la racine \emph{est budy, maison}  ; par conséquent, les hommes qui habitent des maisons, des demeures permanentes. Ce nom de Budini rappelle une des plus singulières erreurs auxquelles la science ait pu se complaire. Hérodote raconte que les gens ainsi nommés étaient (mot grec)  ; tous les traducteurs ont compris et dit qu’ils \emph{mangeaient de la vermine}, ou plus clairement des poux. Cette circonstance, qui parlait peu en faveur des Budini, n’a pas empêché les érudits allemands et les slavistes de se disputer ce peuple, les uns le réclamant pour germain, les autres pour wende. Larcher, Mannert, Buchon, bien d’autres, ont répété que les Budini mangeaient des poux  ; enfin Ritter, se rapportant à l’abréviateur de Tzetzès, et guidé par le sens commun, a démontré que, comme beaucoup de populations actuelles de l’extrême nord, ils se nourrissaient de \emph{jets de sapin}  ; mais l’habitude de l’absurde est si bien prise que Passow lui-même, dans son dictionnaire, tout en donnant les deux versions, montre une prédilection marquée pour la plus ancienne.}.\par
Ce ne sont pas non plus des peuples que l’on puisse dire avoir été civilisés, dans la haute signification du mot. Leur intelligence était trop obscurcie par la mesure du mélange où elle s’était absorbée, et, loin d’avoir développé les instincts natifs de l’espèce blanche, ils les avaient, au contraire, en grande partie émoussés ou perdus. Ainsi, leur religion et le naturalisme qui en fournissait l’étoffe s’étaient ravalés plus bas que ce qu’on voyait même chez les Galls. Le druidisme de ceux-ci, qui n’était assurément pas une doctrine exempte des influences corruptrices de l’alliance finnique, en était cependant moins pénétré que la théologie des Slaves. C’est en celle-ci que se montrait la source des opinions le plus grossièrement superstitieuses, la croyance à la lycanthropie, par exemple. Ils fournissaient aussi des sorciers de toutes les espèces désirables \footnote{Schaffarik, \emph{ouvr. cité}, t. I, p. 195.}.\par
Cette contemplation superstitieuse de la nature, qui n’était pas moins absorbante pour l’esprit des Slaves septentrionaux que pour celui de leurs parents, les Rasènes de l’Italie, tenait une très grande place dans l’ensemble de leurs notions. Les monuments nombreux qu’ils ont laissés, tout en attestant chez eux un certain degré d’habileté, et surtout un génie patient et laborieux, ne valent pas ce qu’on trouve sur les terres celtiques, et, ce qui met le sceau à la démonstration de leur infériorité, c’est qu’ils n’ont jamais pu agir sur les autres familles d’une façon dominatrice. La vie de conquête leur a été constamment inconnue. Ils n’ont pas même su créer pour eux-mêmes un État politique véritablement fort \footnote{Id., \emph{ibid.}, t. I, p. 167.}.\par
Quand, dans cette race prolifique, la tribu devenait quelque peu populeuse, elle se scindait. Trouvant par trop pénible pour sa dose de vigueur intellectuelle le gouverne­ment de trop de têtes réunies et l’administration de trop d’intérêts, elle s’empressait d’envoyer au dehors de ses limites une ou plusieurs communautés sur lesquelles elle ne prétendait conserver qu’une sorte de préséance maternelle, leur laissant d’ailleurs pleine liberté de se régir à leur guise. Les dispositions politiques du Wende, essentiel­lement sporadiques, ne lui permettaient pas de comprendre, encore moins de pratiquer le gouvernement nécessairement compliqué d’un empire vaste et compact. Vivre citoyen d’un municipe aussi étroit que possible, c’était là son rêve. Les conceptions orgueilleuses de domination, d’influence, d’action extérieure, y trouvaient sans doute peu leur compte  ; mais, précisément, le Slave ne les connaissait pas. L’agrandissement de son bien-être direct et personnel, la protection de son travail, l’assistance pour ses besoins physiques, la satisfaction de ses attachements, sentiment vif chez cet être doux et affectueux, bien que froid, tout cela lui était assuré par son régime municipal, avec une facilité, une liberté, une abondance qu’un état social plus perfectionné ne saurait jamais produire, il faut l’avouer. Il s’y tenait donc, et la modération de ces goûts si humbles doit lui mériter, au moins, l’hommage des moralistes, tandis que les politi­ques, plus difficiles à satisfaire, considèrent que les résultats en furent déplorables. L’antique gouvernement de la race blanche, si naturellement propre à servir toutes les dispositions d’indépendance, les plus dangereuses comme les plus utiles, se laissa énerver sans peine par tant de mollesse. On le voulait de plus en plus faible et incertain  ; il s’y prêta. Les magistrats, pères fictifs de la commune, continuèrent à ne devoir qu’à l’élection une autorité temporaire, étroitement limitée par le concours incessant d’une assemblée souveraine composée de tous les chefs de famille. Il est bien évident que ces aristocraties rurales et marchandes composaient les républiques les moins exposées aux usurpations de pouvoir que l’espèce blanche ait jamais réalisées  ; mais elles en étaient, en même temps, les plus faibles, les plus incapables de résister aux troubles intérieurs comme à l’agression étrangère.\par
 Il n’est pas même sans vraisemblance que les nombreux inconvénients de cet isolement si mesquin ne fissent parfois désirer, à ceux-là même qui en aimaient les douceurs, un changement d’état résultant de la conquête d un peuple plus habile. Cette calamité, au milieu du dommage qu’elle entraîne nécessairement, leur devait apporter d’une manière non moins sûre plusieurs avantages capables de les frapper, de leur plaire, \textasciitildet, jusqu’à un certain point, de leur fermer les yeux sur la perte de leur indépendance. On peut mettre de ce nombre l’accroissement des bénéfices matériels, conséquence facile d’un agrandissement de population et de territoire. Une commune isolée a peu de ressources ; deux réunies en ont davantage. La chute des barrières politiques trop rapprochées facilite les relations entre pays frontières ; elle les crée même souvent. Les denrées et les produits circulent plus abondamment, vont plus loin ; les gains et les profits s’accumulent, et l’instinct commercial émerveillé, séduit, gagné, renonçant à ses préjugés contre les concurrences pour se livrer tout entier au charme de la possession d’un marché plus étendu, renie un excès pour se jeter dans l’autre, et devient l’apôtre le plus ardent de cette fraternité universelle que des sentiments un peu plus nobles, que des opinions plus clairvoyantes repoussent comme n’étant autre chose que la mise en commun de tous les vices et l’avènement de toutes les servitudes.\par
Mais les conquérants des Slaves aux époques primitives n’étaient pas en état de pousser le système d’agglomération jusqu’à l’excès. Leurs groupes étaient trop peu considérables par le nombre et trop mal pourvus de moyens intellectuels ou matériels pour exécuter de si gigantesques fautes. Ils ne les imaginaient même pas, et leurs sujets, qui en auraient accepté sans doute les pires conséquences, pouvaient encore, assez raisonnablement, se réjouir de l’extension gagnée à leurs travaux économiques.\par
Puis, sous la loi d’un vainqueur dispensant de tels bienfaits, leur existence moins libre était, en définitive, mieux garantie. Tandis que l’isolement national les avait tou­jours livrés, presque sans défense, à toutes les agressions du dehors, leur constitution nouvelle, sous, des maîtres vigoureux, les soustrayait à ce genre de fléaux, et les envahisseurs rencontraient désormais, entre leur soif de pillage et les laboureurs qu’ils voulaient dépouiller, l’arc et l’épée d’un dominateur jaloux. Donc, pour bien des raisons, les Wendes étaient enclins à prendre la sujétion politique en patience, de même qu’ils avaient ignoré et repoussé les moyens d’y échapper. Et, d’ailleurs, cette sujétion qu’ils n’avaient pas l’orgueil ni même la fierté de haïr, le temps se chargeait, comme toujours, d’en adoucir les aspérités. À mesure qu’une longue cohabitation amenait entre les étrangers et leurs humbles tributaires les alliances inévitables, le rapprochement des esprits s’effectuait. Les relations mutuelles perdaient de leur rigueur première ; la protection se faisait mieux sentir, et le commandement beaucoup moins. À la vérité, les conquérants, victimes de ce jeu, devenaient graduellement des Slaves, et, s’affaissant à leur tour, à leur tour aussi subissaient la domination étrangère, qu’ils ne savaient plus écarter ni de leurs sujets ni d’eux-mêmes. Mais les mêmes mobiles poursuivant incessamment leur action, avec une régularité toute semblable aux mouvements du pendule, amenaient constamment des effets identiques, et les races wendes n’apprenaient pas, et même, arianisées au point médiocre où elles ont pu l’être, n’ont jamais appris que d’une manière imparfaite le besoin et l’art d’organiser un gouvernement qui fût à la fois national et plus complexe que celui d’une municipalité. Elles n’ont jamais pu se soustraire à la nécessité de subir un pouvoir étranger à leur race. Bien éloignées d’avoir rempli dans le monde antique un rôle souverain, ces familles, les plus anciennement dégénérées des groupes blancs d’Europe, n’ont même jamais eu, aux époques historiques, un rôle apparent \footnote{Schaff., \emph{ouvr. cité}, t. I, p. 128.}, et c’est tout ce que peut faire l’érudition la plus sagace que d’apercevoir leurs masses, cependant si nombreuses, si prolifiques, derrière les poignées d’aventuriers heureux qui les régissent pendant les périodes lointaines. En un mot, par suite des alliages jaunes immodérés d’où résulta pour elles cette situation éternellement passive, elles furent plus mal partagées, mora­lement parlant, que les Celtes, qui, du moins, outre de longs siècles d’indépendance et d’isonomie, eurent quelques moments bien courts, il est vrai, mais bien marqués, de prépondérance et d’éclat.\par
La situation subordonnée des Slaves, dans l’histoire, ne doit cependant pas faire prendre le change sur leur caractère. Lorsqu’un peuple tombe au pouvoir d’un autre peuple, les narrateurs de ses misères n’éprouvent généralement aucun scrupule de prononcer que l’un est vaillant et que l’autre ne l’est pas. Lorsqu’une nation, ou plutôt une race, s’adonne exclusivement aux travaux de la paix, et qu’une autre, déprédatrice et toujours armée, fait de la guerre son métier unique, les mêmes juges proclament hardiment que la première est lâche et amollie, la seconde virile, Ce sont là des arrêts rendus à la légère, et qui donnent aux conséquences qu’on en tire autant de maladresse que d’inexactitude.\par
Le paysan de la Beauce, plein d’aversion pour le service militaire et d’amour pour sa charrue, n’est certes pas le rejeton d’une souche héroïque, mais il est, à coup sûr, plus réellement brave que l’Arabe guerrier des environs du Jourdain. On l’amènera facilement, ou, pour mieux dire, il s’amènera lui-même, en un besoin, à faire des actions d’une intrépidité admirable pour défendre ses foyers, et, une fois enrégimenté, son drapeau, tandis que l’autre n’attaquera que rarement à force égale, n’affrontera que le danger le plus petit, et ce petit danger, il s’y soustraira même sans honte, en répétant à part lui l’adage favori du guerrier asiatique : « Se battre, ce n’est pas se faire tuer. » Cependant cet homme circonspect fait profession presque exclusive de manier le fusil. À son avis, c’est là le seul lot convenant à un homme, ce qui ne l’empêche pas, depuis des siècles, de se laisser subjuguer par qui veut s’en donner la peine.\par
Tous les peuples sont braves, en ce sens qu’ils sont tous également capables, sous une direction appropriée à leurs instincts, d’affronter certains périls et de s’exposer à la mort. Le courage, pris dans ses effets, n’est le caractère particulier d’aucune race. Il existe dans toutes les parties du monde, et c’est un tort que de le considérer comme la conséquence de l’énergie, encore plus de le confondre avec l’énergie elle-même : il en diffère essentiellement.\par
Ce n’est pas que l’énergie ne le produise aussi, mais d’une façon bien reconnais­sable. Surtout cette faculté est loin de n’avoir que cette manière de se manifester. En conséquence, si toutes les races sont braves, toutes ne sont pas énergiques, et, fondamentalement, il n’y a que l’espèce blanche qui le soit. On ne rencontre que chez elle la source de cette fermeté de la volonté, produite par la sûreté du jugement. Une nature énergique veut fortement, par la raison qu’elle a fortement saisi le point de vue le plus avantageux ou le plus nécessaire. Dans les arts de la paix, sa vertu s’exerce aussi naturellement que dans les fatigues d’une existence belliqueuse. Si les races blanches, fait incontestable, sont plus sérieusement braves que les autres familles, ce n’est aucunement parce qu’elles font moins de cas de l’existence, au contraire ; c’est que, tout aussi obstinées quand elles attendent du travail intellectuel ou matériel un résultat précieux que lorsqu’elles prétendent jeter bas les remparts d’une ville, elles sont surtout pratiquement intelligentes, et perçoivent le plus distinctement leur but. Leur bravoure résulte de là, et non pas de la surexcitation des organes nerveux, comme chez les peuples qui n’ont pas eu ou qui ont laissé perdre ce mérite distinctif.\par
Les Slaves, trop mélangés, étaient dans ce dernier cas. Ils y sont encore, et plus peut-être qu’autrefois. Ils déployaient beaucoup de valeur guerrière quand il le fallait, mais leur intelligence, affaiblie par les influences finniques, ne s’élevait que dans un cercle d’idées trop étroit, et ne leur montrait pas assez souvent ni assez clairement les grandes nécessités qui s’imposent à la vie des nations illustres. Quand le combat était inévitable, ils y marchaient, mais sans entraînement, sans enthousiasme, sans autre désir que celui de se retirer bien moins du péril que des fatigues, infructueuses à leurs yeux, dont l’état de guerre est hérissé. Ils souscrivaient à tout pour en finir, et retour­naient avec joie au travail des champs, au commerce, aux occupations domestiques. Toutes leurs prédilections se concentraient là.\par
Cette race, ainsi faite, ne posséda donc son isonomie que d’une manière fort obscure, puisque cette isonomie ne s’exerça que dans des centres trop petits pour être encore visibles à travers les ténèbres des âges, et ce n’est guère que par son association à ses conquérants mieux doués que l’on réussit à l’apercevoir et à juger ses qualités comme ses défauts. Trop faible et trop douce pour exciter de bien longues colères chez les hommes qui l’envahissent, sa facilité à accepter le rôle secondaire dans les nouveaux États fondés par la conquête, son naturel laborieux qui la rendait aussi utile à exploiter qu’elle était aisée à régir, toutes ces humbles facultés lui faisaient conserver la propriété du sol, en lui en laissant perdre le haut domaine. Les plus féroces agresseurs repoussaient bien vite la pensée de créer inutilement des solitudes qui ne leur auraient rien rapporté. Après avoir envoyé quelques milliers de captifs sur les marchés lointains de la Grèce, de l’Asie, des colonies italiotes, un moment arrivait où la soumission de leurs vaincus lassait leur furie \footnote{Schaff., \emph{ouvr. cité}, t. I, p. 244.}. Ils prenaient en pitié ce travailleur débonnaire qui opposait si peu de résistance, et désormais ils le laissaient cultiver ses champs. Bientôt la fécondité du Slave avait comblé les vides de la population. L’ancien habitant était plus solidement établi que jamais sur le sol qui lui était laissé, et, pour peu que ses souverains conservassent les faveurs de la victoire, il gagnait du terrain avec eux ; car il poussait l’obéissance jusqu’au point d’être intrépide à leur profit, quand on lui commandait une telle vertu.\par
 Ainsi, indissolublement mariés à la terre d’où rien ne pouvait les arracher, les Slaves occupaient dans l’orient de l’Europe le même emploi d’influence muette et latente, mais irrésistible, que remplissaient en Asie les masses sémitiques. Ils for­maient, comme ces dernières, le marais stagnant où s’engloutissaient, après quelques heures de triomphe, toutes les supériorités ethniques. Immobile comme la mort, actif comme elle, ce marais dévorait dans ses eaux dormantes les principes les plus chauds et les plus généreux, sans en éprouver d’autre modification, quant à lui-même, que çà et là une élévation relative du fond, mais pour en revenir finalement à une corruption générale plus compliquée.\par
Cette grande fraction métisse de la famille humaine, ainsi prolifique, ainsi patiente devant l’adversité, ainsi obstinée dans son amour utilitaire du sol, ainsi attentive à tous les moyens de le conquérir matériellement, avait étendu de fort bonne heure le réseau vivant de ses milliers de petites communes sur une énorme étendue de pays. Deux mille ans avant Jésus-Christ, des tribus wendes cultivaient les contrées du bas Danube et les rives septentrionales de la mer Noire, couvrant d’ailleurs, autant qu’on en peut juger, en concurrence avec les hordes finnoises, tout l’intérieur de la Pologne et de la Russie. Maintenant que nous les avons reconnues dans la véritable nature de leurs aptitudes et de leur tâche historique, laissons-les à leurs humbles travaux, et considé­rons leurs divers conquérants.\par
Au premier rang il convient de placer les Celtes. À l’époque très ancienne où ces peuples occupaient la Tauride et faisaient la guerre aux Assyriens, et, même au temps de Darius, ils avaient des sujets slaves dans ces régions \footnote{Hérodote (IV, 11) indique clairement cette situation, quand il raconte qu’au moment où les Scythes vinrent attaquer les Cimmériens, ceux-ci se consultèrent sur ce qu’il y avait à faire. Les \emph{rois} étaient d’avis de résister, le \emph{peuple} voulait émigrer ; les deux partis en vinrent aux mains, et, comme \emph{ils étaient égaux en nombre}, la bataille fut sanglante ; enfin le peuple eut le dessus, c’est-à-dire \emph{les Slaves}, et, après avoir enterré les morts, on s’enfuit devant les Scythes. – Ce passage donne le sens de cet autre du même livre (102) ou les Scythes, attaqués par Darius, demandent secours à leurs voisins. Alors se réunirent les \emph{rois} des Taures, des Agathyrses, des Neures, des Androphages, des Mélanchlènes, des Gélons, des Boudini et des Sauromates. Le mot \emph{rois}, ( en grec), doit être entendu ici comme au § 11. Il indique les tribus nobles, étrangères, qui régnaient sur les Taures Celtiques, les Agathyrses Slaves, les Neures, les Androphages, les Mélanchlènes Finnois, les Gélons, les Boudini, les Sauromates Slaves. Dans ces derniers, il y a à remarquer que c’étaient des Sarmates Satages ou servants qui formaient la couche inférieure de la population. Ces Satages, bien qu’ayant déjà pris le nom de leurs maîtres, étaient incontestablement de race wende. – Un roi des Agathyrses porte un nom arian : il s’appelle Spargapithès (IV, 78.)}. Plus tard ils en avaient également sur les Krapacks et dans la Pologne et probablement dans les contrées arrosées par l’Oder. Quand ils firent, venant de la Gaule, la grande expédition qui porta les bandes tectosages jusqu’en Asie \footnote{Schaff., I, 243.}, ils semèrent dans toute la vallée du Danube, et dans les pays des Thraces et des Illyriens, de nombreux groupes de noblesse qui restèrent à la tête des peuplades wendes, jusqu’à ce que des envahisseurs nouveaux fussent venus les soumettre eux-mêmes avec elles \footnote{Ce fut aux invasions kymriques que les poètes de la comédie grecque durent les noms de Davus et de Geta, si souvent appliqués par eux aux esclaves qui jouaient un rôle dans leurs fables. Les hommes portant ces noms appartenaient originairement à la classe supérieure des nations slaves vaincues, et provenaient d’une autre source première. (Schaff., t. I, p. 244.) – Ce même auteur pense que l’extension des Celtes, à cette dernière époque, alla jusqu’à la Save et à la Drave dans l’est, et au nord jusqu’aux sources de la Vistule et au Dniester. (T. I, p. 397.)}. En plusieurs occasions les Kymris avaient exercé, et ils exercèrent encore vers la fin du III\textsuperscript{e} siècle avant notre ère, une pression victorieuse sur telle ou telle des nations slaves.\par
Cependant, s’il faut les nommer en première ligne, c’est surtout parce que les raisons de voisinage multiplièrent les incursions de détail. Ils ne furent ni les plus puissants, ni les plus apparents, ni, peut-être même, les plus anciens des dominateurs que les Slaves virent abonder chez eux. Cette suprématie revient surtout à différentes nations fort célèbres qui, sous leurs noms divers, appartiennent toutes à la race ariane. Ce furent ces nations qui opérèrent avec le plus de force et d’autorité dans les contrées pontiques, et jusqu’au delà vers le plus extrême nord. C’est d’elles que les annales de ce pays s’entretiennent surtout, et c’est sur elles que l’attention doit ici se concentrer pour des causes plus graves encore.\par
Le fait que, malgré les mélanges qui déterminèrent successivement la chute et la disparition de la plupart d’entre elles, ces nations appartenaient originairement à la fraction la plus noble de l’espèce blanche serait déjà de nature à leur mériter le plus vif intérêt ; mais un si grand motif est encore renforcé par cette circonstance que c’est de leur sein, que c’est du milieu de leurs multitudes, et des plus pures et des plus puissantes, que se dégagèrent les groupes d’où sortirent les nations germaniques. Ainsi reconnues dans leur étroite intimité originelle avec le principe générateur de la société moderne, elles apparaissent comme plus importantes pour nous, et comme plus sympathiques, dans le sens général de l’histoire, que ne le peuvent être même les groupes de pareille origine, fondateurs ou restaurateurs des autres civilisations du monde.\par
Les premiers de ces peuples qui aient pénétré en Europe, à des époques extrême­ment obscures, et quand des groupes de Finnois, peut-être même des Celtes et des Slaves, occupaient déjà quelques contrées du nord de la Grèce, paraissent avoir été les Illyriens et les Thraces. Ces races subirent nécessairement les mélanges les plus considérables ; aussi leur prépondérance a-t-elle laissé le moins de vestiges. Il n’est vraiment utile d’en parler ici que pour montrer l’étendue approximative de la plus ancienne expansion des Arians extra-hindous et extra-iraniens. Vers l’ouest les Illyriens et des Thraces occupaient alors en maîtres les vallées et les plaines, de l’Hellade au Danube, et, poussant jusqu’en Italie, ils étaient surtout établis fortement sur les versants septentrionaux de l’Hémus \footnote{Schaffarik (I, 271) croit reconnaître des vestiges de leur domination jusque dans la Bessarabie.}.\par
Bientôt ils furent suivis par une autre branche de la famille, les Gètes, qui s’établi­rent à côté d’eux, souvent au milieu d’eux, et enfin beaucoup plus loin qu’eux, vers le nord-ouest et le nord \footnote{Pline (\emph{Hist. natur}., IV, 18) place une nation de Gètes après les Thraces, au nord de l’Hémus.}. Les Gètes se considéraient comme immortels, dit Hérodote. Ils pensaient que le passage au monde d’en bas, loin de les conduire au néant ou à une condition souffrante, les menait aux célestes et glorieuses demeures de Xamolxis \footnote{Hérod., IV, 93. – Il est à remarquer que, dans ce même paragraphe, il y a une identification complète des Gètes avec les Thraces ce qui peut servir d’argument supplémentaire pour appuyer l’origine ariane de ces derniers. – Les médailles apportent ici leur secours. Toutes celles qui appartiennent aux nations situées au nord de Mésons et à l’ouest de la Caspienne montrent des types souvent fort grossiers d’expression comme d’exécution ; la plupart sont évidemment arians, quelques-uns sont slaves, aucun ne montre la plus légère trace de la physionomie finnoise. je citerai, entre autres, les monnaies de Cotys V, type slave ; celles de la ville de Panticapée, type arian, etc.}. Ce dogme est purement arian.\par
Mais l’établissement des Gètes en Europe est tellement ancien qu’à peine est-il possible de les y entrevoir à l’état pur. La plupart de leurs tribus, telles qu’elles sont nommées dans les plus vieilles annales, avaient été profondément affectées déjà par des alliages slaves, kymriques, ou même jaunes. Les Thyssagètes ou Gètes géants, les Myrgètes ou apparentés à la tribu finnique des Merjans, les Samogètes à la race des Suomis, comme s’appellent eux-mêmes les Finnois, formaient de leur propre aveu, autant de tribus métisses qui, ayant uni le plus beau sang de l’espèce blanche à l’espèce mongole, en portaient la peine par l’infériorité relative dans laquelle elles étaient tombées vis-à-vis de leurs parents plus purs. Les Jutes de la Scandinavie, les Iotuns, pour employer l’expression de l’Edda, paraissent avoir été les plus septentrionaux, et, au point de vue moral, les plus dégradés de tous les Gètes \footnote{Au point de vue physique, ils étaient restés très vigoureux et très grands, puisqu’ils sont assimilés aux géants. (Schaff., I, 307.) – Wachter, qui tient aussi les Jotuns pour un peuple métis, les croit issus d’un mélange celte et finnois (\emph{Encycl. Ersch u. Gr.}, 83.) – Il est plus que vraisemblable qu’avec le temps toute espèce d’alliage s’opéra dans le sang des différentes tribus gètes ; mais que la base première ait été ariane, c’est ce dont il n’est pas possible de douter.}.\par
Du côté de l’Asie, du côté de la Caspienne, vivaient encore d’autres branches de la même nation, que les historiens grecs et romains connaissaient sous le nom de \emph{Massagètes} \footnote{Les Chinois les nommaient très régulièrement \emph{Ta-Yueti, grands Gètes} ; \emph{ta} est la traduction exacte de \emph{massa} ou \emph{maba, grand}. (Ritter, 7\textsuperscript{e} Th., 3\textsuperscript{e} Buch, V\textsuperscript{e} Band., page 609.) – Voir les deux notes qui suivent.}. Plus tard, on les nomma Scytho-Gètes ou Hindo-Gètes. Les écrivains chinois les nommaient \emph{Khou-te}, et l’authenticité, l’exactitude parfaite de cette trans­cription est garantie d’une manière rare par le témoignage décisif des poèmes hindous qui, à une époque infiniment plus ancienne, la produisent sous la forme du mot \emph{Khéta}. Les Khétas sont un peuple vratya, réfractaire aux lois du brahmanisme, mais incontestablement arian et vivant au nord de l’Himalaya \footnote{Les Chinois nommaient aussi certaines nations gétiques, et probablement les groupes les plus nombreux, \emph{Yueti} ou \emph{Yuei-tchi}. La première de ces formes se rapproche beaucoup de \emph{Jotun}, ce qui semble indiquer que, bien que cette dernière nous soit surtout connue par les Scandinaves, elle était déjà employée dès la noire antiquité au fond de la haute Asie. – (Ritter, \emph{Asien}, 7\textsuperscript{e} Th., 3\textsuperscript{e} Buch, V\textsuperscript{e} Band., p. 604.) Les renseignements si importants donnés par les écrivains du Céleste Empire sur les nations arianes de la haute Asie empruntent une nuance d’intérêt de plus à ce fait qu’ils ne datent que du II\textsuperscript{e} siècle avant J.-C., ce qui prouve qu’à cette époque encore, et, par conséquent, bien longtemps après le départ des peuples d’où sont sortis les Scandinaves, puis les Germains, il y avait encore de grandes masses blanches dans l’ouest de la Chine, et que ces masses portaient en partie ces mêmes noms que leurs parents européens, probablement bien oubliés par eux, allaient illustrer, quelques siècles plus tard, sur le Rhin et sur le Danube. – On peut ainsi se faire une idée de l’heureuse influence que les invasions et les infiltrations latentes de ces peuples eurent sur les races jaunes ou malayes de la Chine.}.\par
Au II\textsuperscript{e} siècle de notre ère, celles des tribus gétiques qui étaient restées dans la haute Asie se transportèrent sur le Sihoun, puis vers la Sogdiane, et eurent la gloire de substituer un empire de leur fondation à l’État bactro-macédonien. Ce succès toutefois fut peu de chose, comparé à l’éclat que leur nom acquit au IV\textsuperscript{e} et au V\textsuperscript{e} siècle en Europe. Un groupe descendu de leurs frères émigrés, et que nous allons retrouver tout à l’heure avec sa généalogie, partit alors des rives orientales de la Baltique et du sud du pays scandinave pour effacer tout ce que ses homonymes avaient pu faire de grand. La vaste confédération des Goths promena son étendard radieux en Russie, sur le Danube, en Italie, dans la France méridionale, et sur toute la face de la péninsule hispanique. Que les deux formes \emph{Goth} et \emph{Gète} soient absolument identiques, c’est ce dont témoi­gne au mieux un historien national fort instruit des antiquités de sa race, Jornandès. Il n’hésite pas à intituler les annales des rois et des tribus gothiques, \emph{Res geticæ}.\par
À côté des Gètes et un peu moins anciennement, se présente sur la Propontide et dans les régions avoisinantes un autre peuple également arian. Ce sont les Scythes, non pas les Scythes laboureurs, véritables Slaves \footnote{Le mot de (en grec) employé par Hérodote marque, de l’aveu commun, une catégorie de populations qui étaient soumises à des tribus militaires, et, par conséquent, une classe inférieure, une race différente et soumise. Il n’est pas sans intérêt de remarquer qu’elle se retrouvait chez d’autres nations arianes, les Sarmates, par exemple. C’étaient partout des Slaves, soit purs, soit mêlés de débris de noblesses subjuguées avec eux. (Schaff., t. I, p. 184-185, 350.) Un exemple de cette dernière situation existait au III\textsuperscript{e} siècle de notre ère dans la Dacie, où les Sarmates Yazyges dominaient des tribus gétiques, et, par contrecoup, les Slaves qui en formaient la base sociale. (Schaff., I, 250.)}, mais les Scythes belliqueux, les Scythes invincibles, les Scythes royaux, que l’écrivain d’Halicarnasse nous dépeint comme des hommes de guerre par excellence. Suivant lui, ils parlent une langue ariane ; leur culte est celui des plus anciennes tribus védiques, helléniques, iraniennes. Ils adorent le ciel, la terre, le feu, l’air. Ce sont bien là les différentes manifestations de ce naturalisme divinisé chez les plus anciens groupes blancs. Ils y joignent la vénération du génie inspirateur des batailles ; mais, dédaignant l’anthropomorphisme, à l’exemple de leurs ancêtres, ils se contentent de représenter l’abstraction qu’ils conçoivent par le symbole d’une épée plantée en terre.\par
Le territoire des Scythes en Europe s’étend dans la même direction que celui des Gètes, et, pour les connaissances italo-grecques, se confond avec cette région, comme les deux populations se confondaient en réalité \footnote{Les pays situés sur la Baltique et sur le golfe de Finlande s’appelaient, longtemps avant Ptolémée, la Scythie. Pythéas les nommait ainsi, et il était dans le vrai, comme on va le voir plus bas. (Schaff., I, 221.)}. Des Celto-Scythes, des Thraco-Scythes, voilà ce que les plus anciens géographes de l’Hellade connaissent dans le nord de l’Europe, et ils n’ont pas aussi tort qu’on le leur a reproché dans les temps modernes. Cependant leur terminologie n’était ni claire ni précise, il faut en convenir, et, bien qu’elle s’appliquât assez correctement à l’état réel des choses, c’était à leur insu : le vague servait leur ignorance et ne l’égarait pas.\par
 Dans la direction de l’est, les Scythes guerriers donnaient la main à leurs frères, les peuples du nord de la Médie, que les Grecs avaient tort de considérer comme étant leurs auteurs, mais qu’ils avaient raison de leur donner pour parents. Ils s’étendaient jusque dans les montagnes arméniennes où ils se nommaient \emph{Sakasounas}. Puis, au nord de la Bactriane, ils se confondaient avec les Indo-Scythes, appelés par les Chinois les \emph{Szou}. Ils recevaient là une dénomination légèrement altérée et évidemment offerte par ce dernier nom, et devenaient pour les Romains les \emph{Sacae} ; puis, en reprenant les traditions écrites du Céleste Empire, c’étaient ces Hakas, établis encore, à une époque assez basse, sur les rives du Jénisséi \footnote{Westergaard, dans ses études sur les inscriptions cunéiformes de la seconde espèce, observe que le mot \emph{Saka} doit y être lu avec deux \emph{k}, pour exprimer la palatale dure avec l’\emph{s} aspirée, que les Perses n’avaient pas. Ceci rapproche d’autant \emph{Haka} de \emph{Saka}, et semble indiquer que les tribus arianes du nord avaient conservé un dialecte plus rude, qui confondait volontiers la sibilante avec l’aspiration. (P. 32.) – Les Sakas ou Hakas sont aussi nommés, dans les annales chinoises, \emph{Sse}. (Ritter, \emph{l. c}., p. 605 et pass.)}. On ne peut voir en eux que les Sakas du \emph{Ramayana}, du \emph{Mahabharata}, des lois de Manou : des vratyas rebelles aux prescrip­tions sacrées de l’\emph{Arya-varta}, comme les Khétas, mais, comme eux aussi, incontestablement parents des Arians de l’Inde \footnote{Sur cette origine commune, ouvertement consentie par la tradition britannique, je ne puis que donner le passage du \emph{Ramayana} qui l’expose ; je me sers de l’admirable traduction de M. Gorresio : « Di nuovo ella (la vacca Sabalâ) produsse i fieri Saci, misti insieme cogli Yavani. Da questi Saci, commisti cogli Yavani, \emph{fu inondata la terra}. Erano scorridori, robustissimi, condensati, in frotte come fibre di loto ; \emph{portavano bipenni e lunghe spade}, avean armi e armadure d’oro. » – (Gorresio, \emph{Ramayana}, t. VI, \emph{Adicanda}, cap. LV, p. 150.) Voilà une description qui fait, avec justice, des Sakas tout autre chose qu’une horde misérable de pillards mongols. – Voir aussi \emph{Manava-Dharma-Sastra}, ch. X, 44.}. Ils l’étaient de même et d’une façon aussi reconnue de ceux de l’Iran ; et, s’il pouvait rester quelque doute que tous ces Scythes cavaliers de l’Asie et de l’Europe, ces Scythes que les Chinois voyaient errer sur les bords du Hoang-Ho et dans les solitudes du Gobi, que les Arméniens reconnaissaient pour maîtres sur plusieurs points de leur pays \footnote{Sharon-Turner, \emph{Hist. of the Anglo-Saxons}, t. I.}, et que les rivages de la Baltique, que les provinces kymriques \footnote{Une des stations avancées, non pas la plus avancée, des Arians vers le sud-ouest, était, au VIII\textsuperscript{e} siècle avant notre ère, celle des Sigynnes, qui, vêtus comme les Mèdes et vivant, disait-on, dans des chariots, se disaient colonie médique au temps d’Hérodote. Ils étaient voisins des Vénètes de l’Adriatique. (V, 9.)} redoutaient tout autant ; que ces Scythes, dis-je, errant dans le Touran \footnote{Spiegel, Benfey et Weber se sont récemment occupés de fixer la signification du mot persan (en persan) zend, \emph{tuirya}, sanscrit, \emph{tûrya}. Il est d’un grand intérêt de préciser, en effet, si cette dénomination, qui faisait naître dans les esprits des Hindous et des Iraniens de si fortes idées de haine et de crainte, renferme une notion de différence ethnique entre ces peuples et leurs adversaires. Il paraît qu’il n’en est rien, \emph{tûrya} ne signifie qu’\emph{ennemi}. – Voir Spiegel, \emph{Studien über das Zend-Avesta, Zeitschrift a. deutsch. morg. Gesellsch}., t. V, p. 223.} et dans le Pont, ces Skolotes \footnote{(Mot grec) Hérod., IV, 6. – Ce mot semble formé de \emph{Saka} et de lot, ou d’une racine parente de cette expression sanscrite qui signifie \emph{être hors de soi, exalté, furieux} ; les \emph{Saka lota} auraient été \emph{les Sakas au courage inspiré}, téméraire, sans bornes, pareils aux Berserkars scandinaves.}, comme ils se nommaient eux-mêmes, ne fussent absolument d’une même origine sur les points les plus divers où ils se montraient, sur l’Hémus, autant que sur le Bolor, il y aurait encore à alléguer le témoignage décisif des épigraphistes de la Perse. Les inscriptions achéménides connaissent en effet deux nations de Sakas, l’une résidant aux environs du Iaxartes, l’autre dans le voisinage des Thraces \footnote{Westergaard et Lassen, \emph{Inscript. de Darius}, p. 94-95. – Hérodote, Pline et Strabon se prononcent dans le même sens. Le dernier est encore plus péremptoire, puisqu’il confond nettement les Sakas avec les Massagètes et les Dahae : (phrases en grec) – Ainsi il est bien convenu pour Strabon que, sur les bords de la Caspienne, les Dahae et les Scythes sont un même peuple ; qu’à l’orient de ces contrées, les Massagètes et les Saces sont dans des rapports égaux d’identité, et que, de plus, le nom de \emph{Scythe} convient à l’un comme à l’autre de ces groupes. – J’ai longtemps hésité à classer les Scythes, les Skolotes comme ils doivent l’être, au nombre des groupes arians et non pas mongols, bien que soutenu par l’imposante autorité d’hommes tels que M. Ritter et M. A. de Humboldt. Je répugnais à rompre en visière, sans nécessité bien démontrée, à une opinion fortement établie, et, dans le premier volume de cet ouvrage, j’ai même raisonné dans le sens routinier ; mais il m’a fallu me rendre à l’évidence, et comprendre qu’une complaisance exagérée me jetterait dans des erreurs et des non-sens trop graves. Je me suis donc résigné. Ayant allégué déjà plusieurs des motifs sur lesquels j’appuie mon opinion, je me bornerai surtout, pour en bien établir la force, à résumer l’état de la question. D’une voix presque unanime, la science moderne considère les Scythes Skolotes comme des Finnois. Elle a pour cela trois raisons : d’abord, qu’Hippocrate les décrit comme tels ; ensuite que les Grecs appelaient Scythie tout le nord de l’Europe, et ne faisaient aucune distinction entre les populations de ce pays ; enfin que, puisqu’elle a prononcé une fois, elle ne veut pas se déjuger. Laissant respectueusement à l’écart le troisième motif, je ne m’occuperai que des deux premiers. Il est bien vrai qu’Hippocrate décrit des hommes habitant sur les rives de la Propontide comme ayant le caractère physiologique de la race finnoise, et ces hommes, il les qualifie de Scythes. Mais, de la façon dont il emploie ce nom, il est de toute évidence qu’il n’entend par là que des gens établis en Scythie parmi beaucoup d’autres qui ne leur ressemblaient pas. Or, qu’au temps d’Hippocrate, c’est-à-dire deux cents ans après Hérodote, des tribus jaunes pussent être descendues jusque dans le voisinage de la Propontide, et, y habitant pêle-mêle avec bien d’autres races, y eussent reçu des Grecs le nom de Scythes, il n’y a rien là que de très naturel et de très admissible. Il ne s’ensuit pas nécessairement qu’à une époque antérieure, ces mêmes gens fussent déjà dans le pays. Hérodote parle beaucoup de Scythes, il les avait visités, il avait conversé avec eux, il savait leur histoire ; nulle part il ne témoigne qu’ils eussent le moindre trait de la nature finnique ; tout au contraire, quand il décrit cette nature, à l’occasion du récit qu’il a fait des mœurs des Argippéens, il avoue qu’il n’a pas vu lui-même ces hommes chauves, au nez aplati, au menton allongé et que tout ce qu’il en rapporte, il ne le sait que par tradition des marchands et des voyageurs. Et non seulement il n’indique pas par un seul mot, lui, observateur si soigneux et si attentif, que les Scythes aient eu le moindre trait différent de la physionomie grecque ou thrace, mais aucun écrivain d’Athènes, de cette ville d’Athènes où la garde de police était composée, en partie, de soldats scythes, n’a jamais fait la moindre allusion à une particularité qui aurait, au moins, pu fournir l’étoffe d’une plaisanterie à Aristophane, lequel introduit un Scythe fort grossier dans une de ses pièces. Ce n’est pas tout : Hérodote, parlant de la Scythie, proteste contre l’usage de ses compatriotes de la considérer comme étant d’un seul tenant et habitée par une seule race ; il déclare. au contraire, que le nombre des Skolotes y est relativement très petit ; avec eux il nomme un grand nombre de nations qui ne leur sont apparentées en rien (IV, 20, 21, 22, 23, 46, 57, 99). Il les considère comme le peuple dominateur de la région pontique, et, en outre, comme le plus intelligent (IV, 46). Il leur attribue une langue médique, et, en effet, d’après tous les mots et tous les noms qu’il allègue, les Scythes parlaient incontestablement une langue ariane ; enfin, il n’y a pas de doute à conserver que, pour lui, les Skolotes ne soient les Sakas des Hindous et des Iraniens. Beaucoup plus tard, c’est encore l’avis de Strabon. Il est inévitable désormais de s’y ranger et de convenir, dans le cas actuel, comme dans bien d’autres, que c’est un mauvais système que de ne vouloir jamais apercevoir dans un pays qu’une seule race ; d’attribuer à cette race le premier type venu, en dépit des réclamations des gens mieux informés, et il faut donner raison, en l’affaire présente, au plus récent historien de la Norwège, M. Munch, qui, dans l’admirable préambule de son récit, montre les régions pontiques, avant le X\textsuperscript{e} siècle qui précéda notre ère, comme incessamment parcourues et dominées par des nations de cavaliers arians qui se succédaient les unes aux autres, courbant les populations slaves, finniques et métisses sous leur souffle, comme le vent d’est courbe les épis sous le sien. (Munch, \emph{Det norske folk Historie}, trad. all. p. 13.) – En dernier lieu, enfin, il faut en croire les médailles des rois scythes, qui ne portent jamais dans leurs effigies l’ombre d’un trait mongol, comme on peut s’en convaincre aisément en jetant un coup d’œil sur les monnaies de Leuko 1\textsuperscript{er}, de Phascuporis I\textsuperscript{er}, de Gegaepirès, de Rhaemetalcès, de Rhescuporis, etc. Toutes ces médailles montrent la physionomie ariane parfaitement évidente, ce qui constitue une démonstration matérielle à laquelle il n’y a pas de réplique. – Voir aussi toute la série des démonstrations appuyées sur des faits et des témoignages historiques, puisés dans les écrivains grecs, romains et chinois. Ritter, \emph{Asien}, I\textsuperscript{er} Th., VI\textsuperscript{e} Buch, \emph{West-Asien}, Band. V, P. 583 à p. 716.) J’ai emprumté de nombreux détails à cette admirable et féconde accumulation de recherches.}.\par
Ce nom antique des Sakas s’est maintenu non moins longtemps et a parcouru plus de régions encore que celui des Khétas. Aux époques des migrations germaniques, il était appliqué à la contrée noble par excellence, Skanzia, la Scandinavie, l’île ou la presqu’îles des Sakas. Enfin, une dernière transformation, qui fait dans ce moment l’orgueil de l’Amérique, après avoir brillé dans la haute Germanie et dans les îles Britanniques, est celle de Saxna, \emph{Sachsen}, les \emph{Saxons}, véritables \emph{Sakasunas}, fils des Sakas des dernières époques \footnote{À l’ordinaire on fait dériver le nom de Saxon du mot \emph{sax} ou \emph{seax couteau}. Cette étymologie convient d’autant moins que les Saxons étaient remarqués pour la grandeur de leurs épées, et se servaient d’ailleurs préférablement des haches d’armes : « Securibus gladiisque longis, » dit Henri de Huntingdon. – Kemble produit un passage d’un document ancien qui repousse de même cette opinion : « Incipit linea Saxonum et Anglorum descendes ab Adamo linealiter usque ad \emph{Sceafum de} « \emph{quo Saxones vocabantur} » Mullenhoff ne me paraît nullement bien fondé dans la critique qu’il fait de ce texte. (Voir \emph{Zeitschrift für das d. Alterth}., t. VII, p. 415.) – Sceaf est un personnage tellement ancien, au jugement de la légende germanique, qu’il est placé à la tête des aïeux d’Odin . Les Scandinaves chrétiens ont exprimé cette idée en le faisant naître dans l’arche de Noé. Mullenhoff lui-même considère les aventures qui sont attribuées à ce personnage comme un mythe de l’arrivée par mer des Roxolans dans la Suède. (\emph{Loc. cit}., p. 413 .)}.\par
 Les Sakas et les Khétas constituent, en fait, une seule et même chaîne de nations primitivement arianes. Quel qu’ait pu être, çà et là, le genre et le degré de dégradation ethnique subie par leurs tribus, ce sont deux grandes branches de la famille qui, moins heureuses que celles de l’Inde et de l’Iran, ne trouvèrent dans le partage du monde que des territoires déjà fortement occupés, relativement à ce qu’avaient eu leurs frères, et surtout bien inférieurs en beauté. Longtemps embarrassés de fixer leur existence tourmentée par les Finnois du nord, par leurs propres divisions et par l’antagonisme de leurs parents plus favorisés, la plupart de ces peuples périrent sans n’avoir pu fonder que des empires éphémères, bientôt médiatisés, absorbés ou renversés par des voisins trop puissants \footnote{On compte cependant dans ces États, souvent réduits à un bien faible périmètre, de nombreuses villes. On y remarque la présence de familles royales très respectées pour leur antiquité, une agriculture développée et surtout la mise en rapport de vignobles célèbres, l’élève de superbes races de chevaux, une grande réputation de bravoure militaire, une habileté commerciale dont les annalistes chinois, excellents juges en cette matière, se préoccupent beaucoup, et, ce qui est plus honorable encore, l’existence d’une littérature nationale et d’un ou plusieurs alphabets particuliers. (Ritter, \emph{loc, cit}., pass.) – Je rappellerai que les traits distinctifs physiologiques de tous ces peuples, aux yeux des écrivains chinois, sont d’avoir eu les yeux bleus, la barbe et la chevelure blondes et épaisses, et le nez proéminent. (\emph{Loc. cit}.)}. Tout ce qu’on aperçoit de leur existence dans ces régions vagues et illimitées du Touran, et des plaines pontiques, le Touran européen, qui étaient leurs lieux de passage, leurs stations inévitables, révèle autant d’infortune que de courage, une ardente intrépidité, la passion la plus chevaleresque des aventures, plus de grandeur idéale que de succès durables. En mettant à part celles de ces nations qui réussirent, mais beaucoup plus tard, à dominer notre continent, les Parthes furent encore une des plus chanceuses parmi les tribus arianes de l’ouest \footnote{Les médailles des rois barbares, des rois sakas, qui renversèrent l’empire gréco-macédonien, ne permettent pas non plus de douter que les conquérants ne parlassent une langue ariane, qu’ils n’eussent un culte arian, et enfin que leurs traits ne fussent tout à fait ceux de la famille blanche, sans rien qui rappelle le type mongol. (Benfey, \emph{Bemerkungen über die Gœtter-namen auf Indo-skythischen-münzen, Zeitsch. d. d. m. Gesellsch.}, t. VIII, p. 450 seqq.)}.\par
Ce n’est pas assez que de montrer par les faits que les Khétas, les Sakas, et les Arians, pris dans leur ensemble et à leurs origines, sont tout un. Les trois noms, analysés en eux-mêmes, donnent le même résultat : ils ont tous trois le même sens ; ce ne sont que des synonymes : ils veulent dire également \emph{les hommes honorables}, et, s’appliquant aux mêmes objets, exposent clairement que la même idée réside sous leurs apparences différentes \footnote{J’ai déjà parlé ailleurs du changement normal de l’\emph{r} en \emph{s} dans les langues arianes, et de la cause de cette loi. Je n’en donnerai ici que quelques exemples, amenés par le sujet, et pour montrer qu’elle s’exécute partout également. Dans les inscriptions achéménides de la seconde espèce, Westergaard observe que le mot \emph{asa} peut également être lu \emph{arsa} ; ainsi \emph{Parsa} ou \emph{Pasa}. Le savant indianiste ajoute que le médique n’admettait pas l’\emph{r} devant une consonne et le supprimait (pp. 87, 115.) On se rappelle involontairement ici la façon complexe dont Ammien Marcellin et Jornandès transcrivirent le nom des dieux scandinaves : au lieu d’ases, ils disent \emph{anses} ou \emph{anseis}. (On sait combien la mutation de l’\emph{r} en \emph{n} est d’ailleurs fréquente.) Cette forme \emph{ansi} était connue des Chinois, qui disent indifféremment \emph{asi} et \emph{ansi}. (Ritter, \emph{loc. cit.}, pass.) – Chez les Doriens, la même mobilité avait lieu entre l’\emph{s} et l’\emph{r}. On lit, dans le décret des Spartiates contre Timothée, (mots grecs), etc. – Chez les Latins, même observation, mais en sens inverse ; ainsi \emph{genus, generis, majosibus, majoribus, plurima, plusima, Papisius, Papirius, arbos, arbor}. On en trouve des traces dans un dialecte français, le poitevin, où on dit : \emph{il ertait} pour : \emph{il estait}, et dans les romans du XII\textsuperscript{e} siècle. – Ainsi, \emph{Arya} et \emph{Asa} sont identiques. L’Asie, \emph{Asia}, c’est \emph{le pays des Arians. Sak} ou \emph{hak} veut dire \emph{honorer}. (Lassen et Westergaard, p. 25.) – \emph{Ket}, (mot persan) en persan moderne, veut dire \emph{honorable}.}.\par
Ce point établi, suivons maintenant dans les phases ascendantes de leur histoire les tribus les mieux prédestinées de cette agglomération de maîtres que la Providence amenait graduellement au milieu des peuples de l’ancien monde, et, d’abord, des Slaves.\par
Il se trouvait parmi elles une branche particulière et fort étendue de nations d’essence très pure, du moins au moment où elles arrivèrent en Europe. Cette circonstance importante est garantie par les documents ; je parle des Sarmates. Ils descendaient, disaient les Grecs du Pont, d’une alliance entre les Sakis et les Amazones, autrement dit, \emph{les mères des Ases ou des Arians} \footnote{Le mot \emph{mère} est, en sanscrit, \emph{âmaba}. Il s’agit ici d’une forme dialectique plus courte.}. Les Sarmates, comme tous les autres peuples de leur famille, se reconnaissent des frères dans les contrées les plus distantes. Plusieurs de leurs nations habitaient au nord de la Paropamise, tandis que d’autres, connues des géographes du Céleste-Empire sous les noms de Suth, Suthle, Alasma et Jan-thsaï, vinrent au II\textsuperscript{e} siècle avant Jésus-Christ, occuper certains cantons orientaux de la Caspienne. Les Iraniens se mesurèrent maintes fois avec ces essaims de guerriers, et la crainte excessive qu’ils avaient de leur opiniâtreté martiale s’était perpétuée dans les traditions bactriennes et sogdes. C’est de là que Firdousi les a fait passer dans son poème \footnote{Les trois fils de Féridoun sont Iredj, Tour et Khawer. Ce sont les personnifications des trois rameaux blancs de la Perse, de l’Iran proprement dit, puis de l’intérieur de l’Asie, puis des contrées occidentales du monde. La parenté de ces trois groupes est ainsi rigoureusement reconnue. On ne manquera pas de retrouver dans la forme \emph{Khawer} une transcription toute naturelle de l’antique expression de \emph{Yavana}. C’est un témoignage de plus de l’antiquité des renseignements dont s’est servi Firdousi. (Voir tome I\textsuperscript{er}. – Schaffarik, \emph{Slawische Alterth}., t. I, p. 350-351.)}.\par
Ces vigoureuses populations, arrivées en Europe, pour la première fois, un millier d’années avant notre ère, pas davantage \footnote{Hérodote fournit trois traditions sur l’origine des Scythes et une sur celle des Sarmates. La première, considérant les Scythes comme autochtones, les déclarait les derniers nés de tous les peuples de la terre et leur donnait une antiquité de quinze cents ans environ avant J.-C. (Livre IV, 5.) La seconde, fournie par les Grecs du Pont, les faisant descendre d’Hercule et d’une nymphe du pays, ne leur assigne que treize cents et quelques années avant notre ère. (Livre IV, 8.) La troisième, due à Aristée de Proconnèse, qui l’avait rapportée de ses voyages dans l’Asie centrale, n’a rien de mythique, et fait simplement venir les Scythes de l’est, d’où ils avaient été chassés par les Issédons, fuyant à leur tour devant les Arimaspes. Il ne serait nullement difficile de montrer le point de concordance de ces trois manières d’envisager le même fait. Quant à la formation des peuples sarmates, nés des Scythes et des Amazones, je l’ai déjà indiquée. Ils parlaient un dialecte arian, différent de celui des Skolotes. (Livre IV, 17.) Pline, Pomponius Mela et Ammien Marcellin font les Sarmates beaucoup plus jeunes que je ne crois devoir l’admettre ici avec Hérodote. Ils supposent que les premiers groupes de leurs tribus furent établis sur le Don par les Scythes, au retour de l’expédition de ces derniers en Asie, vers la fin du VII\textsuperscript{e} siècle avant notre ère. Au fond, de telles questions sont peu réelles : 1° parce que les Sarmates ne sont qu’une simple variété des Sakas ; 2° parce que leurs nations, venant de l’est, dans la direction du Touran, se succédèrent à des époques très rapprochées, et qu’il n’y a pas lieu d’en choisir une à l’exclusion des autres pour servir aux éphémérides.}, avaient mis le pied dans le monde occidental avec des mœurs toutes semblables à celles des Sakas, leurs cousins et leurs antago­nistes principaux. Revêtus de l’équipage héroïque des champions du Schahnameh, leurs guerriers ressemblaient assez bien déjà à ces paladins du moyen âge germanique, dont ils étaient les lointains ancêtres. Un casque de métal sur le front, sur le corps une armure écailleuse de plaques de cuivre ou de corne, ajustées en manière de peau de dragon, l’épée au côté, l’arc et le carquois au dos, à la main une lance démesurément longue et pesante \footnote{Ces détails de costume et d’armement se trouvent dans les écrivains romains et grecs qui ont parlé des Sarmates avec détail. Quant à l’équipement général des autres peuples de la même famille, on a vu plus haut que le \emph{Ramayana} attribuait aux Sakas des armures d’or, de lourdes haches et de longues épées. Hérodote, en parfait accord avec ce livre, montre les Massagètes avec des baudriers, des cuirasses et des casques revêtus d’or, et employant le cuivre à forger les pointes de leurs lances, de leurs javelots et de leurs flèches. (Hérodote, II, 215.) – Dans l’expédition de Xerxès, les Arians Perses avaient des cuirasses de fer travaillées en écailles de poisson. (Hérodote, VII, 61.) Cette coutume, dit l’historien, avait été empruntée aux Mèdes. (Livre VII, 62.) – Les Arians Cissiens la suivaient aussi. (\emph{Ibidem}), ainsi que les Arians Hyrcaniens. (\emph{Ibidem}). Il en était de même des Parthes, des Chorasmiens, des Sogdiens, des Gandariens, des Dadices et des Bactriens. (\emph{Ibidem}., 64 et 66.) – Il n’y a donc nul doute possible que les armures complètes de métal. et en forme d’écailles ne fussent d’un usage général chez toutes les nations arianes désignées par les Hindous sous le nom de Sakas}, ils cheminaient à travers les solitudes sur des chevaux lourdement caparaçonnés, escortant et surveillant d’immenses chariots couverts d’un large toit. Dans ces vastes machines étaient renfermés leurs femmes, leurs enfants, leurs vieillards, leurs richesses. Des bœufs gigantesques les traînaient pesamment en faisant vaciller et crier leurs roues de bois plein sur le sable ou l’herbe courte de la steppe. Ces maisons roulantes étaient les pareilles de celles que la plus ténébreuse antiquité avait vues transporter vers le Pendjab, la contrée opulente des cinq fleuves, les familles des premiers Arians. C’étaient les pareilles encore de ces constructions ambulantes dont, plus tard, les Germains formèrent leurs camps ; c’était, sous des formes austères, l’arche véritable portant l’étincelle de vie aux civilisations à naître et le rajeunissement aux civilisations énervées, et, si les temps modernes peuvent encore fournir quelque image capable d’en évoquer le souvenir, c’est bien assurément la puissante charrette des émigrants américains, cet énorme véhicule, si connu dans l’ouest du nouveau continent, où il apporte sans cesse jusqu’au delà des montagnes Rocheuses, les audacieux défricheurs anglo-saxons et les viragos intrépides, compagnes de leurs fatigues et de leurs victoires sur la barbarie du désert.\par
L’usage de ces chariots décide un point d’histoire. Il établit une différence radicale entre les nations qui l’ont adopté et celles qui lui ont préféré la tente. Les premières sont voyageuses ; elles ne répugnent pas à changer absolument d’horizon et de climats ; les autres seules méritent la qualification de nomades. Elles ne sortent qu’avec peine d’une circonscription territoriale assez limitée. C’est être nomade que d’imaginer l’unique espèce d’habitation qui, par sa nature, soit éternellement mobile et présente le symbole le plus frappant de l’instabilité. Le chariot ne saurait jamais être une demeure définitive. Les Arians qui s’en sont servis, et qui, pendant un temps plus ou moins long, ou même jamais, n’ont pu se créer d’autres abris, ne possédaient pas et ne voulaient pas de tentes. Pourquoi ? C’est qu’ils voyageaient, non pour changer de place, mais, au contraire, pour trouver une patrie, une résidence fixe, une maison. Poussés par des événements contraires ou particulièrement excitants, ils ne réussis­saient à s’emparer d’aucun pays de manière à y pouvoir bâtir d’une manière définitive. Aussitôt que ce problème a pu se résoudre, l’habitation roulante s’est attachée au sol et n’en a plus bougé. Le mode de demeure encore en usage dans la plupart des pays européens qui ont possédé des établissements arians en offre la preuve : la maison nationale n’y est autre chose qu’un chariot arrêté. Les roues ont été remplacées par une base de pierre sur laquelle s’élève l’édifice de bois. Le toit est massif, avancé ; il enveloppe complètement l’habitation, à laquelle on ne parvient que par un escalier extérieur, étroit et tout semblable à une échelle. C’est bien, à très peu de modifications près, l’ancien chariot arian. Le chalet helvétique, la cabane du moujik moscovite, la demeure du paysan norwégien, sont également la maison errante du Saka, du Gète et du Sarmate, dont les événements ont enfin permis de dételer les bœufs et d’enlever les roues \footnote{Weinhold, \emph{Die deutschen Frauen in dem Mittelalter}, Wien 1851, p. 327. – A. de Haxthausen, dans son excellent ouvrage sur la Russie, fait une remarque qui aboutit au même résultat : « Les « ornements, dit-il, et les découpures qui ornent les toits (des maisons « des paysans russes aux environs de Moscou), les galeries et l’escalier conduisant à « l’intérieur, rappellent les habitations « des Alpes, et particulièrement les chalets suisses. » (T. I, p. 19-20.)}. En arriver là, c’était l’instinct permanent, sinon le vœu avoué des guerriers qui ont traîné en tant de lieux et si loin cette demeure vénérable par les héroïques souvenirs qu’elle rappelle. Malgré leurs pérégrinations multipliées, quelquefois séculaires, ces hommes n’ont jamais consenti à accepter l’abri définitivement mobile de la tente ; ils l’ont abandonné aux peuplades d’espèce ou de formation inférieure.\par
 Les Sarmates \footnote{Ce nom est formé des deux racines \emph{sâr} et \emph{mat}, qui signifient \emph{destructeur des peuples}. L’une, \emph{sâr}, est médique. (Westergaard, p. 81.) L’autre, \emph{mat}, répond au verbe sanscrit déchirer. – Je crois avoir déjà dit, mais je le répète encore, qu’il ne s’agit pas de trouver, pour des mots touraniens, une source directe dans le sanscrit, mais seulement des analogies de dialectes qui puissent faire entrevoir le sens à travers la forme peu concordante des vocables. – Le mot \emph{sâr, habitant}, est le même qui apparaît dans le nom de la capitale de la Lydie, (mot grec) de \emph{sâr} et de \emph{dhâ, Sarda, le lieu où l’on établit des habitants, la colonie}.}, les derniers venus des Arians, au X\textsuperscript{e} siècle avant notre ère, et conséquemment les plus purs, ne tardèrent pas à faire sentir aux anciens conquérants des Slaves la force supérieure de leur bras et de leur intelligence, dans les contestations qui ne manquèrent pas de s’élever. Bientôt ils se firent une grande place. Ils dominèrent entre la Caspienne et la mer Noire, et commencèrent à menacer les plaines du nord \footnote{Schaffarik, \emph{Slaw. Alterth.}, t. I, p. 120-121, 141.}. Longtemps, toutefois, les pentes septentrionales du Caucase demeu­rèrent leur point d’appui. C’est dans les défilés de cette grande chaîne que, plusieurs siècles après, quand ils eurent perdu l’empire exclusif des régions pontiques, celles de leurs tribus qui n’avaient pas émigré allèrent chercher un refuge parmi quelques peuplades parentes plus anciennement établies dans ces gorges \footnote{Les Ossètes du Caucase, nommés, dans les anciennes annales russes, \emph{Iasi} ou \emph{Osi}, et par Plan-Carpin, ou XIII\textsuperscript{e}\emph{ siècle, Alani} et \emph{Asses}, s’attribuent à eux-mêmes le titre d’\emph{Iron}, et à leur pays celui d’\emph{Ironistan}. C’est un nouvel exemple de permutation de l’\emph{r} en \emph{s}. (Schaff., \emph{Slaw. Alterth}., t. I, 141, 353.)}. Elles durent à cette circonstance, heureuse pour le maintien de leur intégrité ethnique, l’honneur dont elles jouissent aujourd’hui d’avoir été choisies par la science physiologique pour représenter le type le plus accompli de l’espère blanche. Les nations actuelles de ces montagnes continuent à être célèbres par leur beauté corporelle, par leur génie guerrier, par cette énergie indomptable qui intéresse les peuples les plus cultivés et les plus amollis aux chances de leurs combats, et par une résistance plus difficile encore à ce souffle d’avilissement qui, sans pouvoir les toucher, atteint autour d’elles les multitudes sémitiques, tatares et slaves. Loin de dégénérer, elles ont contribué, dans la proportion où leur sang s’est mêlé à celui des Osmanlis et des Persans, à réchauffer ces races. Il ne faut pas oublier non plus les hommes éminents qu’elles ont fournis à l’empire turc, ni la puissante et romanesque domination des beys circassiens en Égypte.\par
Il serait ici hors de place de prétendre suivre dans le détail les innombrables mouvements des groupes sarmates vers l’occident de l’Europe. Quelques-unes de ces migrations, comme celle des Limigantes, s’en allèrent disputer la Pologne à des noblesses celtiques, et, sur leur asservissement, fondèrent des États qui, parmi leurs villes principales, ont compté Bersovia, la Varsovie moderne, D’autres, les Iazyges, conquirent la Pannonie orientale, malgré les efforts des anciens vainqueurs de race thrace ou kymrique, qui déjà y dominaient les masses slaves. Ces invasions et bien d’autres n’intéressent que des histoires spéciales \footnote{Schaffarik reconnaît quelques faibles restes d’une tribu de Sarmates Iazyges dans la population d’aujourd’hui clairsemée sur la rive gauche de la Pialassa. Ils sont d’une carnation très brune, s’habillent de noir, et conservent des usages différents de ceux des races qui les entourent. Ils parlent le russe blanc, mais avec un accent lithuanien. Ils sont nommés par les gens du pays Iatwjèses ou Iodwezaj. C’est une formation de métis tout à fait tombés. (Schaff., Slawische Alterth., t. I, p. 338, 340, 343, 349.)}. Elles ne furent pas exécutées sur une assez grande échelle ni avec des forces suffisantes pour affecter d’une manière durable la valeur active des groupes subjugués. Il n’en est pas de même du mouvement qu’une vaste association des tribus de la même famille, issues de la grande branche des Alains, \emph{Alani}, peut-être, plus primitivement, \emph{Arani} ou Arians, et portant pour nom fédératif celui de\emph{ Roxolans} \footnote{Munch (\emph{Det Norske Folk Historie} (traduct. allem.), p. 63) cherche assez péniblement à établir l’étymologie de ce mot. Il veut que, de même que les Allemands sont appelés par les Slaves \emph{Njemzi, muets}, parce qu’on ne comprend pas ce qu’ils disent, ces mêmes Slaves, mieux instruits du langage des Sarmates, leur aient donné le nom de \emph{Ruotslaine, Rootslaine}, de la racine \emph{rot, le peuple de ceux qui parlent}.}, opéra du côté des sources de la Dwina, dans les contrées arrosées par le Wolga et le Dnieper, en un mot dans la Russie centrale, vers le VII\textsuperscript{e} ou le VIII\textsuperscript{e} siècle avant l’ère chrétienne \footnote{Munch, p. 14, 52-53.}. Cette époque, marquée par de grands change­ments dans la situation ethnique et topographique d’un grand nombre de nations asiatiques et européennes, constitue également pour les Arians du nord un nouveau point de départ, et par conséquent une date importante dans l’histoire de leurs migrations.\par
Il n’y avait guère que deux à trois cents ans qu’ils étaient arrivés en Europe, et cette période avait été remplie tout entière par les conséquences violentes de l’antagonisme qui les opposait aux nations limitrophes. Livrés sans réserve à leurs haines nationales, absorbés par les soins uniques de l’attaque et de la défense, ils n’avaient pas eu le temps sans doute de perfectionner leur état social ; mais cet inconvénient avait été largement compensé, au point de vue de l’avenir, par l’isolement ethnique, gage assuré de pureté, qui en avait été la conséquence. Maintenant ils se voyaient contraints de se transporter dans une nouvelle station. Cette station leur était assignée, exclusivement à toute autre, par des nécessités impérieuses.\par
La propulsion qui les jetait en avant venait du sud-est. Elle était donnée par des congénères, évidemment irrésistibles, puisqu’on ne leur résistait pas. Il n’y avait donc pas moyen que les Arians-Sarmates-Roxolans prissent leur marche contre cette direction. Ils ne pouvaient davantage s’avancer indéfiniment vers l’ouest, parce que les Sakas, les Gètes, les Thraces, les Kymris, y étaient demeurés par trop forts, et surtout par trop nombreux. C’eût été affronter une série de difficultés et d’embarras inextricables. Incliner vers le nord-est était non moins difficile. Outre les amoncelle­ments finnois qui opéraient sur ce point, des nations arianes encore considérables, des métis arians jaunes qui augmentaient chaque jour d’importance, devaient très légitimement faire repousser l’idée d’une marche rétrograde vers les anciens gîtes de la famille blanche. Restait l’accès du nord-ouest. De ce côté, les barrières, les empêche­ments étaient sérieux encore, mais pas insurmontables. Peu d’Arians, beaucoup de Slaves, des Finnois, en quantité moindre que dans l’est, il y avait là des probabilités de conquêtes plus grandes que partout ailleurs. Les Roxolans le comprirent ; le succès leur donna raison. Au milieu des populations diverses que leurs traditions conservées nous font encore connaître sous leurs noms significatifs de Wanes, de Iotuns et d’Alfars, ou fées, ou nains, ils réussirent à établir un état stable et régulier dont la mémoire, dont les dernières splendeurs projettent encore, à travers l’obscurité des temps, un éclat vif et glorieux sur l’aurore des nations scandinaves.\par
C’est le pays que l’Edda nomma le Gardarike, ou \emph{l’empire de la ville des Arians} \footnote{\emph{Garta} est employé dans les \emph{Védas} dans le double sens de chariot et de maison. On en voit la cause. Sur une inscription achéménide, \emph{karta} signifie \emph{château}. Dans ce sens, il fait partie de la composition du nom de plusieurs capitales asiatiques, entre autres \emph{Tigranocerta, le château de Tigrane}. En latin, en gothique, et dans toutes les langues dérivées de cette double source, \emph{hortus, gard, gardun, gurten, giœrd, giardino, jardin, garden}, veut dire principalement une enceinte, et c’est là, certainement, le sens intime du mot. (Dieffenbach, \emph{Vergleichendes Wœrterbuch der gotbischen Sprache}, t. II, p. 382.) – Lassen et Westergaard, \emph{Die Achem. Keilinschriften}, p. 29 et 72. – Weinhold, \emph{Die Deutschen Frauen in dem Mittelalter}, Wien, 1851, p. 327. – Pott Etymologische Forschun gen, th. I, p. 144) y joint très bien le (mot grec) grec et le mot italiote\emph{ chors}. J’y ajouterai le terme militaire de même origine\emph{ cohors}, qui garde dans ses flexions le \emph{t} primitif.}\footnote{Ptolémée nomme le peuple de ce pays (mot grec). Une inscription perse recueillie par Niebuhr, I, tabl. XXXI, le mentionne également. Hérodote compte huit mille Sagartes dans l’armée de Darius (VII, 85). (Lassen et Westergaard, \emph{Achem. Keilinschriften}, p. 54.)}. Les Sarmates Roxolans y purent dételer leurs bœufs voyageurs, y remiser leurs chariots. Ils connurent enfin des loisirs qu’ils n’avaient plus eus depuis bien des séries de siècles, et en profitèrent pour s’établir dans des demeures permanentes. Asgard, la ville des Ases ou des Arians, fut leur capitale. C’était probablement un grand village orné de palais à la façon des anciennes résidences des premiers conquérants de l’Inde et de la Bactriane. Son nom n’était d’ailleurs pas prononcé pour la première fois dans le monde. Entre autres applications qui en furent faites, il exista longtemps, non loin du rivage méridional de la Caspienne, un établissement médique appelé de même Açagarta .\par
Les traditions concernant Asgard sont nombreuses et même minutieuses. Elles nous montrent les pères des dieux, les dieux eux-mêmes, exerçant avec grandeur dans cette royale cité la plénitude de leur puissance souveraine, rendant la justice, décidant la paix ou la guerre, traitant avec une hospitalité splendide et leurs guerriers et leurs hôtes. Parmi ceux-ci nous apercevons quelques princes wanes \footnote{L’\emph{Edda} place les Ases, les Roxolans, sur la rive orientale du Don, tandis que les nations wendes indépendantes occupent la rive occidentale. (Schaffarik, t. I, p. 134, 307, 358.)} et iotuns, voire des chefs finnois. Les nécessités du voisinage, les hasards de la guerre forçaient les Roxolans de s’appuyer tantôt sur les uns, tantôt sur les autres, pour se maintenir contre tous. Des alliances ethniques furent alors contractées et étaient inévitables \footnote{Suivre la trace et l’indication de ces mélanges dans l’Edda, principalement dans la Vœluspa. La forme mythique du récit n’empêche en aucune façon d’apercevoir le noyau historique.}. Toutefois le nombre, et par conséquent l’importance, en resta minime, l’\emph{Edda} le démontre, parce que l’état de guerre moins constant que jadis, lorsque les Roxolans résidaient aux environs du Caucase, n’en fut pas moins très ordinaire, et surtout parce que le Gardarike, bien qu’ayant jeté beaucoup d’éclat sur l’histoire primitive des Arians Scandinaves, dura trop peu de temps pour que la race qui le possédait ait eu le temps de s’y corrompre. Fondé du VII\textsuperscript{e} au VIII\textsuperscript{e} siècle avant l’ère chrétienne, il fut renversé vers le IV\textsuperscript{e} \footnote{Munch attribue la ruine du Gardarike à la pression des nations de Sakas qui avaient remplacé les Sarmates dans les régions du Caucase, et qui étaient elles-mêmes dépossédées par les Achéménides. (P. 61.)}, malgré le courage et l’énergie de ses fondateurs, et ceux-ci, forcés encore une fois de céder à la fortune qui les conduisait à travers tant de catastrophes à l’empire de l’univers, remirent leurs familles et leurs biens dans leurs chariots, remontèrent sur leurs coursiers, et, abandonnant Asgard, s’enfoncèrent, à travers les marais désolés des régions septentrionales, au-devant de cette série d’aventures qui leur était réservée, et dont rien assurément ne pouvait leur faire présager les étonnantes péripéties et le succès final.
\section[{VI.2. Les Arians Germains.}]{VI.2. \\
Les Arians Germains.}
\noindent Arrivée à un certain point de sa route, l’émigration des nobles nations roxolanes se sépara en deux rameaux. L’un se dirigea vers la Poméranie actuelle, s’y établit, et de là conquit les îles voisines de la côte et le sud de la Suède \footnote{Munch, \emph{ouvr. cité}, p. 61.}. Pour la première fois les Arians devenaient navigateurs et s’emparaient d’un mode d’activité dans lequel il leur était réservé de dépasser un jour, en audace et en intelligence, tout ce que les autres civilisations avaient jamais pu exécuter. L’autre rameau, qui, à son heure, ne fut pas moins remarquable ni moins comblé dans ce genre, continua à marcher dans la direction de la mer Glaciale, et, arrivé sur ces tristes rivages, fit un coude, les longea, et, redescendant ensuite vers le midi, entra dans cette Norwège, \emph{Nord-wegr, le chemin septentrional} \footnote{Munch, p. 9 et 61. – Il donne, par extension, au mot \emph{Norwégien} le sens de \emph{gens qui marchent vers le nord}, et, par induction, \emph{de gens qui marchent vers le nord relativement à leurs compatriotes}, Suédois et Poméraniens, ou, autrement dit, Goths restés au sud.} contrée sinistre, peu digne de ces guerriers, les plus excellents des êtres. Ici l’ensemble des tribus qui s’arrêta abandonna les dénominations de Sarmates, de Roxolans, d’Ases, qui jusqu’alors avaient servi à le distinguer au milieu des autres races. Il reprit le titre de Sakas. Le pays s’appela Skanzia, la presqu’île des Sakas. Très probablement ces nations avaient toujours continué entre elles à se donner le titre d’\emph{hommes honorables}, et, sans un trop grand souci du mot qui rendait cette idée, elles ne nommaient indifféremment Khétas, Sakas, Arians ou Ases. Dans la nouvelle demeure, ce fut la seconde de ces dénominations qui prévalut, tandis que, pour le groupe établi dans la Poméranie et les terres adjacentes, celle de Khéta devint d’un usage commun \footnote{Munch, \emph{ouvr. cité}, p. 59.}. Néanmoins, les peuples voisins n’admirent jamais cette dernière modification, dont ils ne comprenaient pas sans doute la simplicité, et avec une ténacité de mémoire des plus précieuses pour la clarté des annales, les peuples finni­ques continuent encore d’appeler les Suédois d’aujourd’hui \emph{Ruotslaine} ou \emph{Rootslane}, tandis que les Russes ne sont pour eux que des \emph{Wænalnine} ou \emph{Wænelane}, des Wendes \footnote{\emph{Ibid}., p. 56.}.\par
Les nations scandinaves étaient à peine établies dans leur péninsule, quand un voyageur d’origine hellénique vint pour la première fois visiter ces latitudes, patrie redoutée de toutes les horreurs, au sentiment des nations de la Grèce et de l’Italie. Le Massaliote Pythias poussa ses voyages jusque sur la côte méridionale de la Baltique.\par
Il ne trouva encore dans le Danemark actuel que des Teutons, alors celtiques, comme leur nom en fait foi \footnote{Le nom de \emph{Teut}, que se donnent aujourd’hui les Allemands, est d’un usage fort ancien parmi les nations des Kyrnris, et n’a absolument rien de germanique. On trouve dans l’Italie aborigène \emph{Teuta} pour le nom primitif de Pise. Les habitants s’appelaient \emph{Teutanes, Teutani} ou\emph{ Teutæ}. (Pline, \emph{Hist. natur}., III, 8.) – Les guerriers de la Gaule avaient établi en Cappadoce la tribu des \emph{Teutobodiaci}, en Pannonie, la ville de (nom grec), dans le nord de la Grèce, les (nom grec) (Id., \emph{ibid}.) On connaît une foule de noms d’hommes celtiques dans la composition desquels entre ce mot, \emph{Teutobochus, Teutomalus}, etc. (Dieffenbach, \emph{Celtica II, I Abth}, p. 193, 338.) – Munch considère les Thjust du Smaaland comme des Celtes d’origine. (P. 46.) – Deutsch ne paraît pas avoir été pris collectivement avant le IX\textsuperscript{e} siècle de notre ère.}. Ces peuples possédaient le genre de culture utilitaire des autres nations de leur race ; mais à l’est de leur territoire se trouvaient les Guttons, et avec ceux-ci nous revoyons les Khétas ; c’était une fraction de la colonie poméranienne \footnote{Ils s’étaient établis sur les terres des nations slaves qu’ils avaient forcées au partage et dont ils paraissent avoir expulsé la noblesse. (Schaffarik, \emph{Slaw. Alterth.}, t. I, p. 106.)}. Le navigateur grec les visita dans un bassin intérieur de la mer qu’il nomme \emph{Mentonomon}. Ce bassin est, à ce qu’il semble, Frische-Haff, et la ville qui s’élève sur ses bords, Königsberg \footnote{Pythias, Ptolémée, Mela et Pline ont montré les Goths tendant vers la Vistule. Ce fut longtemps leur frontière. Ils touchaient là à des peuples arians qu’on nommait les Scytho-Sarmates, et qui, bien que de même souche qu’eux, faisaient partie d’un autre groupe d’invasion. (Munch, 36-37, 52-53.)}. Les Guttons s’étendaient alors très peu vers l’ouest ; jusqu’à l’Elbe, le pays était partagé entre des communes slaves et des nations celtiques \footnote{Munch, \emph{loc. cit.}, 31.}. En deçà du fleuve, jusqu’au Rhin d’une part, jusqu’au Danube de l’autre, et par delà ces deux cours d’eau, les Kymris régnaient à peu près seuls. Mais il n’était pas possible que les Sakas de la Norwège, que les Khétas de la Suède, des îles et du continent, avec leur esprit d’entreprise, leur courage et le mauvais lot territorial qui leur était échu, laissassent bien longtemps les deux amas de métis blancs qui bordaient leurs frontières en possession tranquille d’une isonomie qui n’était pas trop difficile à troubler.\par
Deux directions s’ouvraient à l’activité des groupes arians du nord. Pour la branche gothique, la façon la plus naturelle de procéder, c’était d’agir sur le sud-est et le sud, d’attaquer de nouveau les provinces qui avaient fait anciennement partie du Gardarike et les contrées où antérieurement encore tant de tribus arianes de toutes dénominations étaient venues commander aux Slaves et aux Finnois et avaient subi l’inévitable dépréciation qu’amènent les mélanges. Pour les Scandinaves, au contraire, la pente géographique était de s’avancer dans le sud et l’ouest, d’envahir le Danemark, encore kymrique, puis les terres inconnues de l’Allemagne centrale et occidentale, puis les Pays-Bas, puis la Gaule. Ni les Goths ni les Scandinaves no manquèrent aux avances de la fortune \footnote{Cette séparation des premières nations véritablement germaniques en Scandinaves et en Goths me paraît commandée par les faits, et je la préfère aux traditions généalogiques que nous ont conservées Tacite et Pline. Celles-ci font descendre les races du Nord d’un homme-type, appelé Tuisto, et de ses trois fils, Istæwo, Irmino et Ingævo. Tout prouve que ce mythe n’a jamais existé dans les pays purement germaniques, et s’est développé surtout dans l’Allemagne centrale et méridionale. Il paraît donc être d’origine celtique, bien qu’il ait été adopté et peut-être modifié dans quelques parties par les Germains métis. Les efforts de W. Muller pour retrouver dans les noms de Tuisto, d’Ingævo, d’Irmino et d’Istævo des surnoms de dieux scandinaves ne sont pas certainement très heureux. (\emph{Altdeutsche Religion}, p. 292 et seqq.) – Comme exemple des changements que cette tradition a subis dans le cours des temps, on peut présenter le tableau donné par Nemnius (éd. Gunn, p. 53-54), où, au lieu de Tuisto, dans lequel on ne peut, en tout cas, reconnaître que \emph{Teut}, transformé en éponyme de la race celtique, le chroniqueur donne \emph{Alanus}, et quant aux noms des trois héros fils de cet Alanus, il les écrit \emph{Hisicion, Armenon} et \emph{Neugio}.}.\par
Dès le second siècle avant notre ère, les nations norwégiennes donnaient des marques irrécusables de leur existence aux Kymris, qu’ils avaient pour plus proches voisins. De redoutables bandes d’envahisseurs, s’échappant des forêts, vinrent réveiller les habitants de la Chersonnèse cimbrique, et, franchissant toutes les barrières, traversant dix nations, passèrent le Rhin, entrèrent dans les Gaules, et ne s’arrêtèrent qu’à la hauteur de Reims et de Beauvais \footnote{Munch, \emph{ouvr. cité}, p. 18.}.\par
Cette conquête fut rapide, heureuse, féconde. Pourtant elle ne déplaça personne. Les vainqueurs, trop peu nombreux, n’eurent pas besoin d’expulser les anciens propriétaires du sol. Ils se contentèrent de les faire travailler à leur profit, comme toute leur race avait l’habitude de s’y prendre chez les métis blancs soumis. Bientôt même, nouvelle marque du peu d’épaisseur de cette couche d’arrivants, ils se mêlèrent suffisamment avec leurs sujets pour produire ces groupes germanisés si fort célébrés par César, comme représentant la partie la plus vivace des populations gauloises de son temps, et qui avaient conservé l’antique nom kymrique de Belges \footnote{Il se passa alors chez les populations celtiques de l’occident ce qui arrivait depuis des siècles, dans l’orient de l’Europe, à d’autres Celtes et surtout aux Slaves. Des maîtres arians commencèrent par s’imposer à elles, puis acceptèrent leur nom national en se mêlant. C’est là un des motifs qui portèrent si longtemps les Romains à confondre les deux groupes et Strabon à proposer cette singulière étymologie du mot de Germain, venu, disait-il, de ce que les Gaulois les appellent \emph{Frères}, (mot grec). (VII, 1, 2.) Ils étaient frères, en effet, au moment où le géographe d’Apamée les observait, mais non pas frères d’origine. (Voir Wachter, \emph{Encycl. Ersch u. Gruber, Galli}, p. 47 – Dieffenbach, \emph{Celtica II}, p. 68.) – De même que les premiers clans germaniques de l’Orient, ceux qui venaient de la Norwège, se mêlèrent aux Celtes, qu’ils trouvèrent sur leur chemin, de même les premières expéditions gothiques contractèrent des alliances qui les modifièrent profondément. Ainsi les \emph{Gothini} de la Silésie avaient adopté la langue de leurs sujets de la race kymrique. Tacite le dit expressément. (\emph{Germ}., 43.) J’insiste d’autant plus fortement sur les faits de ce genre, qu’ils forment la partie essentielle de l’histoire, qu’ils expliquent une multitude d’énigmes, jusqu’ici insolubles, et que jamais on ne les a pris en considération.}.\par
Cette première alluvion fit grand bien aux nations qu’elle pénétra. Elle restitua leur vitalité, atténua chez elles l’influence des alliages finniques, leur rendit pour un certain temps une activité conquérante, qui leur valut une partie des Gaules et les cantons orientaux de l’île de Bretagne ; bref, elle leur donna une supériorité si marquée sur tous les autres Galls que, lorsque les Cimbres et les Teutons, s’ébranlant à leur tour, franchirent le Rhin, ces émigrants passèrent à côté des territoires belges sans oser les attaquer, eux qui affrontaient sans crainte les légions romaines. C’est qu’ils reconnaissaient sur l’Escaut, la Somme et l’Oise des parents qui les valaient presque.\par
Le caractère de furie et de rage déployé par ces antagonistes de Marius, leur incroyable audace, leur pesante avidité sont tout à fait dignes de remarque, parce que rien de tout cela n’était plus ni dans les habitudes ni dans les moyens des peuples celtiques proprement dits. Toutes ces tribus cimbriques et teutonnes avaient été, plus particulièrement encore que les Celtes, fortifiées par des accessions scandinaves. Depuis que les Arians du nord vivaient dans leur voisinage immédiat et avaient commencé à leur faire sentir plus activement leur présence, depuis que les Jotuns avaient aussi pénétré dans leurs domaines, elles avaient subi de grandes transforma­tions, qui les mettaient au-dessus du reste de leur ancienne famille. C’étaient toujours des Celtes fondamentalement, mais des Celtes régénérés.\par
En cette qualité, ils n’étaient pas cependant devenus les égaux de ceux qui leur avaient communiqué une part de leur puissance ; et quand les Scandinaves, quittant un jour en nombre suffisant leur péninsule, étaient venus réclamer non plus seulement la suprématie souveraine, mais le domaine direct de ces métis, ces derniers s’étaient vus contraints de leur faire place. C’est ainsi qu’une grande partie d’entre eux, quittant un pays qui n’avait plus à leur offrir que la pauvreté et la sujétion, composèrent ces bandes exaspérées qui renouvelèrent un moment dans le monde romain la vision des jours désastreux de l’antique Brennus.\par
Tous les Teutons, tous les Cimbres n’eurent pas recours sans exception à ce violent parti et ne se jetèrent pas dans l’exil. Ce furent les plus hardis, les plus nobles, les plus germanisés qui le firent. S’il est dans les instincts des familles guerrières et dominantes d’abandonner en masse une contrée où l’attrait de leurs anciens droits ne les retient plus, il n’en est point ainsi des couches inférieures de la population, vouées aux travaux agricoles et à la soumission politique. Pas d’exemple qu’elles aient jamais été ni expulsées en masse, ni absolument détruites dans aucune contrée. Ce fut le cas des Cimbres et de leurs alliés. La couche germanisée disparut, pour faire place à une couche plus homogène dans sa valeur scandinave. Les substructions celtiques mêlées d’éléments finnois se conservèrent. La langue danoise moderne le révèle nettement \footnote{Munch (\emph{ouvr. cité}, p. 8) ne pense pas qu’avant le VIII\textsuperscript{e} siècle de notre ère on puisse affirmer que les populations aient été germaniques. L’extrême nord du Juttland paraît avoir porté un grand nombre de populations diverses, d’abord des Finnois, puis des Celtes, puis des Slaves, puis des Jotuns, enfin des Scandinaves. – Wachter (\emph{Gali}) considère les Danois comme un mélange primitif de Finnois et de Celtes.}. Elle a conservé des traces profondes du contact celtique, qui n’a pu s’opérer qu’à cette époque. Un peu plus tard on trouve encore, chez les diverses nations germaniques de ces pays, de nombreuses croyances et pratiques druidiques.\par
L’époque de l’expulsion des Teutons et des Cimbres constitue un second déplacement des Arians du nord, plus important déjà que le premier, celui qui avait créé les Belges de seconde formation. Il en résulta trois grandes conséquences, dont les Romains éprouvèrent les contrecoups. Je viens d’en citer une : ce fut la convulsion cimbrique. La seconde, en donnant pied aux Scandinaves de la Norwège sur la rive méridionale du Sund, fit arriver dans le nord de l’Allemagne, et peu à peu jusqu’au Rhin, des peuples nouveaux, de race mixte, plus arianisés que les Belges, pour la plupart, car ils apportèrent des dénominations nationales nouvelles au sein des masses celtiques qu’ils conquirent. Le troisième effet fut d’amener, au I\textsuperscript{er} siècle avant Jésus-Christ, jusqu’au centre de la Gaule, une conquête germanique bien caractérisée, bien nette, celle dont Arioviste se montra le seul meneur apparent. Ces deux derniers faits demandent quelque attention, et, nous occupant d’abord du premier, remarquons à quel point le dictateur connaît peu les nations transrhénanes de son temps. Ce ne sont plus pour lui, comme jadis pour Aristote, des populations kymriques, mais des groupes parlant une langue toute particulière, et que leur mérite, dont il a pu juger par expérience personnelle, rend fort supérieurs à la dégénération où sont en proie les Gaulois contemporains. La nomenclature donnée par lui de ces familles, si dignes d’intérêt, n’est pas plus riche que les détails qu’il rapporte sur leurs mœurs. Il n’en connaît et n’en cite que quelques tribus ; et encore si les Trévires et les Nerviens se déclarent Germains d’origine, comme ils en avaient le droit jusqu’à un certain point, il les range non moins légitimement parmi les Belges. Les Boïens vaincus avec les Helvètes sont à ses yeux demi-germains, mais d’une autre façon que les Rèmes ; et il n’a pas tort. Les Suèves, malgré l’origine celtique de leur nom, lui semblent pouvoir être comparés aux guerriers d’Arioviste \footnote{Les Suèves avaient une très grande réputation parmi les métis germaniques. Ils n’étaient cependant pas de race pure. Leur organisation politique était celle des Kymris, leur religion était druidique. Ils habitaient des villes, ce que ne faisait aucune nation scandinave ou gothique ; ils cultivaient même la terre, au dire de César.}. Enfin, il met absolument dans cette dernière catégorie d’autres bandes, également originaires d’outre-Rhin, qui un peu avant son consulat avaient pénétré, l’épée au poing, au sein du pays des Arvernes, et qui, s’y étant établies dans des terres concédées de gré, ou plutôt de force, par les indigènes, avaient ensuite appelé auprès d’eux un assez grand nombre de leurs compatriotes pour former là une colonisation de vingt mille âmes à peu près. Ce trait suffit, soit dit en passant, pour expliquer cette terrible résistance qui, parmi les habitants énervés de la Gaule, fit rivaliser les sujets de Vercingétorix avec le courage des plus hardis champions du Nord \footnote{Il paraît qu’avant l’époque de César les nations de la Gaule, les plus considérables, avaient eu recours, pour augmenter leur puissance, à ce moyen familier aux peuples en décadence, de coloniser chez eux des étrangers sous la condition du service militaire. Ce qu’avaient fait les Arvernes, peut-être un peu de force, leurs rivaux, les Éduens, l’avaient essayé de bonne grâce.}.\par
C’est à ce peu de renseignements que se bornait, au I\textsuperscript{er} siècle avant notre ère, la connaissance qu’on avait dans le monde romain de ces vaillantes nations qui allaient un jour exercer une si grande influence sur l’univers civilisé. Je ne m’en étonne pas : elles venaient d’arriver ou à peine de se former, et n’avaient pu encore révéler qu’à demi leur présence. On serait en droit de considérer ces détails incomplets comme à peu près nuls, quant au jugement à porter sur la nature spéciale des peuples germaniques de la seconde invasion, si, par la description spéciale que l’auteur de la guerre gallique a laissée du camp et de la personne d’Arioviste, il ne se trouvait heureusement avoir suppléé, dans une mesure utile, à ce que ses autres observations avaient de trop vague pour autoriser une conclusion.\par
Arioviste, aux yeux du grand homme d’État romain, n’est pas seulement un chef de bande, c’est un conquérant politique de la plus haute espèce, et ce jugement, à coup sûr, fait honneur à celui qui l’a mérité. Avant d’entrer en lutte avec le peuple-roi, il avait inspiré une bien forte idée de sa puissance au sénat, puisque celui-ci avait cru devoir le reconnaître déjà pour souverain et le déclarer ami et allié. Ces titres si recherchés, si appréciés des riches monarques de l’Asie, ne l’infatuaient pas. Lorsque le dictateur, avant d’en venir aux mains avec lui, cherche à l’étudier et, dans une négociation astucieuse, tente de discuter son droit à s’introduire dans les Gaules, il répond pertinemment que ce droit est égal et tout pareil à celui du Romain lui-même, qu’il est venu, comme lui, appelé par les peuples du pays, et pour intervenir dans leurs discordes. Il maintient sa position d’arbitre légitime ; puis, déchirant avec fierté les voiles hypocrites dont son compétiteur cherche à envelopper et à cacher le fond sérieux de la situation : « Il ne s’agit, dit-il, ni pour « toi ni pour moi, de protéger les cités gauloises, ni d’arranger leurs débats, en « pacificateurs désintéressés. Nous voulons, l’un et l’autre, les asservir. »\par
En parlant ainsi, il pose le débat sur son véritable terrain et se déclare digne de disputer la proie. Il connaît bien les affaires de la contrée, les partis qui la divisent, les passions, les intérêts de ceux-ci. Il parle le gaulois avec autant de facilité que sa propre langue. Bref, ce n’est pas plus un barbare par ses habitudes qu’un subalterne par son intelligence.\par
Il fut vaincu. Le sort prononça contre lui, contre son armée, mais non pas, on le sait, contre sa race. Ses hommes, qui n’appartenaient à aucune des nations riveraines du Rhin, se dispersèrent. Ceux que César, ébloui de leur valeur, ne put prendre à son service, allèrent se mêler, sans bruit, aux tribus mixtes qui couvraient derrière eux le terrain. Ils apportèrent de nouveaux éléments à leur génie martial.\par
C’étaient eux, bien qu’ils ne fussent pas une nation, mais seulement une armée \footnote{Arioviste dit à César que depuis quatorze ans que ces campagnes dans la Gaule avaient commencé, ni lui ni ses hommes n’avaient dormi sous un toit. Cette remarque indique bien la situation absolument militaire des gens de ce chef.}, qui avaient fait connaître les premiers dans l’Occident le nom des \emph{Germains}. C’était d’après la plus ou moins grande ressemblance que les Trévires, les Boïens, les Suèves, les Nerviens avaient avec eux, soit dans l’apparence corporelle, soit dans les mœurs et le courage, que César avait accordé à ceux-ci l’honneur de leur trouver quelque chose de germanique. C’est donc à leur propos qu’il faut s’enquérir de ce que signifie ce nom glorieux, que j’ai déjà employé en attendant l’occasion vraie de l’expliquer.\par
Puisque les gens d’Arioviste n’étaient pas un peuple et ne constituaient qu’une troupe en expédition, voyageant, suivant l’usage des nations arianes, avec ses femmes, ses enfants et ses biens, ils n’avaient pas lieu de se parer d’un nom national ; peut-être même, comme il arriva souvent depuis à leurs congénères, s’étaient-ils recrutés dans bien des tribus différentes. Ainsi privés d’un nom collectif, que pouvaient-ils répondre aux Gaulois qui leur demandaient : Qui êtes-vous ? Des guerriers, répliquaient-ils nécessairement, des hommes honorables, des nobles, des Arimanni, Heermanni, et suivant la prononciation kymrique, des Germanni. C’était en effet la dénomination générale et commune qu’ils donnaient à tous les champions de naissance libre \footnote{Savigny, \emph{D. Rœmische Recht im Mittelalter}, t. I, p. 193. – jusqu’aux IX\textsuperscript{e} et X\textsuperscript{e} siècles on a dit indifféremment \emph{Germanus} et \emph{Arimannus}, pour indiquer un homme libre parmi les populations germaniques de l’Italie. (\emph{Ibidem}, p. 166.) Il y en a même des exemples au XII\textsuperscript{e} siècle. On appelait alors \emph{Arimannia} l’ensemble des hommes libres d’une même circonscription et aussi la propriété libre d’un ariman. (\emph{Ibid}., 170-171.)}. Les noms synonymes de Saka, de Khéta, d’Arian, avaient cessé de désigner, comme autrefois, l’ensemble de leurs nations ; certaines branches particulières et quelques tribus se les appliquaient exclusivement \footnote{Outre les Oses Sarmates, qui habitaient encore la Pannonie, mais fort dégénérés et tributaires d’autres Sarmates et des Quades germaniques, on avait les Osyles dans la Baltique ; c’étaient des Roxolans d’origine. (Munch, p. 34.) On avait ainsi des \emph{Arii} germaniques au delà de la Vistule (Tac., 43), des Guttes, des Chattes, des Gotones, etc., etc. Pline, Strabon, Ptolémée et Méla donneraient, au besoin, tous les éléments d’une longue liste.}. Mais partout, comme dans l’Inde et la Perse, ce nom, dans une de ses expressions, et plus généralement dans celle d’Arian, continuait à s’appliquer à la classe la plus nombreuse de la société ou à la plus prépon­dérante. L’Arian chez les Scandinaves, c’était donc le chef de famille, le guerrier par excellence, ce que nous appellerions le citoyen. Quant au chef de l’expédition dont il s’agit ici, et qui, de même que Brennus, Vercingétorix et tant d’autres, paraît n’avoir reçu de l’histoire que son titre, et non pas son nom propre, Arioviste, c’était l’hôte des héros, celui qui les nourrissait, les payait, c’est-à-dire, d’après toutes les traditions, leur général. Arioviste, c’est Ariogast, ou Ariagast, l’hôte des Arians.\par
Avec le second siècle de l’ère chrétienne commence cette époque où les émissions scandinaves s’étant déjà multipliées dans la Germanie, l’instinct d’initiative y est devenu patent et éveille toutes les préoccupations des hommes d’État romains. L’âme de Tacite est en proie à de poignantes inquiétudes, et il ne sait qu’espérer de l’avenir. « Qu’elle persiste, s’écrie-t-il, qu’elle dure, j’en « adjure tous les dieux, non l’affection que ces peuples nous portent, mais la « haine dont ils s’entre-déchirent. Une société telle que la nôtre n’a rien de « mieux à attendre de la fortune que les discordes de ses voisins \footnote{« Maneat, quæso, duretque gentibus, sinon amor nostri, at certe odium sui ; quando « urgentibus imperii fatis, nihil jam præstare fortuna majus potest quam hostium « discordiam. » (\emph{Germ}., 33).}. »\par
Ces terreurs si naturelles furent cependant trompées par l’événement. Les Germains, limitrophes de l’empire au temps de Trajan, devaient, malgré leurs apparen­ces effrayantes, rendre à la chose romaine les plus éminents services et ne prendre guère de part à sa transformation future, si toutefois ils en ont pris. Ce n’était pas à eux qu’était promise la gloire de régénérer le monde et de constituer la société nouvelle. Tout énergiques qu’ils étaient comparativement aux hommes de la république, ils étaient déjà trop affectés par les mélanges celtiques et slaves pour accomplir une tâche qui exigeait tant de jeunesse et d’originalité dans les instincts. Les noms de la plupart de leurs tribus disparaissent sans éclat avant le X\textsuperscript{e} siècle. Un bien petit nombre se montre encore dans l’histoire de la grande migration ; encore sont-ils très loin d’y paraître aux premiers rangs. Ils s’étaient laissé gagner par la corruption romaine.\par
Pour trouver le foyer véritable des invasions décisives qui créèrent le germe de la société moderne, il faut se transporter sur la côte baltique et dans la péninsule scandinave. Voilà cette contrée que les plus anciens chroniqueurs nomment justement, et avec un ardent enthousiasme, la source des peuples, la matrice des nations \footnote{Jornandès, c. 4 : « Scandia insula, quasi officina gentium, aut certe velut vagina nationum. »}. Il faut lui associer aussi, dans une si illustre désignation, ces cantons de l’est où, depuis le départ du Gardarike de l’Asaland, la branche ariane des Goths avait fixé ses principales demeures. Au temps où nous les avons quittés, ces peuples étaient fugitifs et contraints à se contenter de misérables territoires. Nous les retrouvons à cette heure tout-puissants, dans d’immenses régions conquises par leurs armes.\par
Les Romains commencèrent à connaître non pas toutes leurs forces, mais celles des provinces extrêmes de leur empire, dans la guerre des Marcomans, autrement dit, \emph{des hommes de la frontière} \footnote{Munch, p. 31 et 38.}. Ces populations furent, à la vérité, contenues par Trajan ; mais la victoire coûta fort cher, et ne fut nullement définitive. Elle ne préjugea rien contre les destinées futures de cette grande agglomération germanique, qui, bien que touchant déjà au bas Danube, plongeait encore ses racines dans les terres les plus septentrionales, et partant les plus franches, les plus pures, les plus vivifiantes de la famille \footnote{\emph{Ibid.}, p. 40. – Keferstein, \emph{Keltische Alterth.}, t. I, p. XXXI.}.\par
En effet, quand, vers le V\textsuperscript{e} siècle, les grandes invasions commencent, ce sont des masses gothiques toutes nouvelles qui se présentent, en même temps que sur toute la ligne des limites romaines, depuis la Dacie jusqu’à l’embouchure du Rhin, des peuples, à peine connus naguère, et qui se sont graduellement rendus redoutables, deviennent irrésistibles. Leurs noms, indiqués par Tacite et Pline comme appartenant à des tribus extrêmement reculées vers le nord n’avaient paru à ces écrivains que très barbares ; ils avaient considéré les peuples qui les portaient comme les moins propres à éveiller leu sollicitude. Ils s’étaient trompés du tout au tout.\par
C’étaient, comme je viens de le dire, et en première ligne, les Goths, arrivés en masse de tous les coins de leurs possessions, d’où les expulsait la puissance d’Attila, appuyée plus encore sur des races arianes ou arianisées que sur ses hordes mongoles \footnote{M. Amédée Thierry, dans ses travaux sur le V\textsuperscript{e} siècle, est entré, le premier, dans une voie qui jette des lueurs toutes nouvelles sur les faits politiques de ces époques. On ne saurait trop louer la méthode employée par cet écrivain pour étudier et juger l’action d’Attila. – Schaffarik, \emph{Slaw. Alterth}., t. I, p. 124. – La grande migration fut surtout composée des Vandales, des Suèves et des Alains, quant aux masses envahissantes, mais non pas quant à la direction qui leur était donnée. (Munch., p. 40.)}. L’empire des Amalungs, la domination d’Hermanarik, s’étaient écroulés sous ces assauts terribles. Leur gouvernement, plus régulier, plus fort que celui des autres races germaniques \footnote{C’est à Tacite qu’on doit cette remarque.}, et qui reproduisait sans doute les mêmes formes en s’appuyant sur les mêmes principes que celui de l’antique Asgard, n’avait pu les sauver d’une ruine inévitable. Cependant ils avaient conservé leur grandeur entière ; leurs rois ne dégénéraient pas de la souche divine à laquelle remontait leur maison, non plus que du nom brillant qu’elle leur valait, les \emph{Amâls}, les \emph{Célestes}, les \emph{Purs} \footnote{Strahlenberg (\emph{Der nœrdl. u. oestl. Theil Europas u. Asiens}, p. 104) avait déjà remarqué que les Visigoths appelaient le ciel \emph{amal}. – Schlegel \emph{Ind. Biblioth}., t. I, p. 235) a fait observer, après lui, que le mot \emph{amala}, qui en gothique signifie \emph{pur, sans tache}, a exactement le même sens en sanscrit. – Les \emph{Amala}, en anglo-saxon, \emph{Amalunga}, dans le Nibelungenlied, \emph{Amalungen}, les Amalungs descendaient de \emph{Géat} ou Khéta. Suivant W. Muller (\emph{Alt. deutsche Religion}, p. 297), Géat est un surnom d’Odin. Je suis plutôt porté à voir dans ce nom une forme antique du nom national des Goths, comme \emph{Séaf} est une forme de \emph{Sako}. (Voir une note precédente.) Les Amalungs descendaient ainsi de la plus pure souche ariane.} ; enfin, la suprématie de la famille gothique était, en quelque sorte, avouée parmi les nations germaines, car elle éclate dans toutes les Pages de l’Edda, et ce livre, compilé en Islande d’après des chants et des récits norwégiens, célèbre principalement le Visigoth Théodorik. Ces honneurs extraordinaires étaient complètement mérités. Ceux auxquels ils étaient rendus aspirèrent à tous les genres de gloire. Ils comprirent beaucoup mieux que ne le faisaient les Romains l’importance et le prix des monuments de toute espèce provenus de l’ancienne civilisation ; ils exercèrent l’influence la plus noble dans tout l’Occident. Ils en furent récompensés par une gloire durable ; au XII\textsuperscript{e} siècle, un poète français se faisait encore honneur d’être issu de leur sang \footnote{Rigord, mort vers 1209, se qualifie, dans sa chronique : « Magister Rigordus, natione Gothu. » (\emph{Hist. litt. de France}, t. XVII, p. 7. )}, et, beaucoup plus tard, les derniers tressaillements de l’énergie gothique inspirèrent l’orgueil de la noblesse espagnole.\par
Après les Goths, les Vandales tiendraient un rang distingué dans l’œuvre du renouvellement social, si leur action avait pu se soutenir et durer davantage. Leurs bandes nombreuses n’étaient pas purement germaniques, ni par les recrues dont elles s’étaient renforcées, ni par l’origine même du noyau : l’élément slave tendait à y dominer \footnote{Schaffarik (\emph{Slaw. Alterth}., t. I, p. 163) pense que les Slaves, dans leurs établissements situés entre la Vistule et l’Oder, ayant reçu des immixtions des Suèves (Celtes germanisés), donnèrent naissance aux Vandales. La terminaison \emph{il, ul, al} indique un dérivé. Parmi les Vandales se mêlèrent plusieurs bandes dont l’origine purement germanique est incontestable. Cependant ces bandes étaient peu nombreuses.}. Bientôt la fortune les jeta au milieu de populations plus civilisées de beaucoup qu’ils ne l’étaient, et infiniment plus nombreuses. Les alliages particuliers qui s’opérèrent furent d’autant plus pernicieux, pour la partie germanique de leur essence, qu’étrangers à la combinaison première des éléments vandales, ces alliages y créèrent et y développèrent plus de désordres. Un mélange fondamentalement slave, jaune et arian, acceptant de proche en proche, en Italie et en Espagne, le sang romanisé de différentes formations pour prendre ensuite toutes les nuances mélanisées répandues sur le littoral africain, ne pouvait que dégénérer d’autant plus promptement qu’il cessa bientôt de recevoir tout affluent germanique. Carthage vit les Vandales accepter avec empressement sa civilisation décrépite et en mourir. Ils disparurent. Les Kabyles, que l’on prétend descendre d’eux, ont conservé en effet quelque chose de la physionomie septentrionale, et cela d’autant plus aisément que les habitudes sporadi­ques dans lesquelles leur décadence les a fait choir, en les rangeant au niveau des peuplades voisines, continuent à maintenir un certain équilibre entre les éléments ethniques dont ils sont actuellement formés. Mais, examinés avec quelque attention, ils laissent constater que le peu de traits teutoniques survivant dans leur physionomie est contrasté par beaucoup d’autres appartenant aux races locales. Et pourtant ces Kabyles si dégénérés sont encore les plus laborieux, les plus intelligents et les plus utilitaires des habitants de l’occident africain.\par
Les Longobards ont mieux défendu leur pureté que les Vandales ; ils ont eu aussi cet avantage de pouvoir se retremper à plusieurs reprises dans la source d’où sortait leur sang ; aussi ont-ils duré plus longtemps et exercé une plus grande action. Tacite les avait à peine remarqués aux environs de la Baltique, où ils vivaient de son temps. Ils y touchaient encore au berceau commun des nobles nations dont ils faisaient partie. Descendant ensuite plus au sud, ils gagnèrent les contrées moyennes du Rhin et le haut Danube, et ils y séjournèrent assez pour s’empreindre de la nature des races locales, ce dont le caractère celtisé de leur dialecte porte témoignage \footnote{Munch, p. 46 et 48.}. Malgré ces mélanges, ils n’avaient nullement oublié ce qu’ils étaient, et longtemps après qu’ils se furent établis dans la vallée du Pô, Prosper d’Aquitaine, Paul Diacre et l’auteur du poème anglo-saxon de \emph{Beowulf} voyaient encore en eux des descendants primitifs des Scandinaves \footnote{Munch, p. 46 et 48.}.\par
Les Burgondes, placés jadis par Pline dans le Jutland, peu de temps sans doute après qu’ils venaient d’y arriver, appartenaient, comme les Longobards, à la branche norwégienne \footnote{Keferstein (\emph{Keltische Alterth}., t. I, p. XXXI) signale dans leur composition, au moment où ils arrivèrent sur le Rhin, des mélanges gothiques et vandales. Il n’y a, en e effet, rien de plus vraisemblable. Je n’entends parler ici que de leur état premier.} ; ils s’étaient dirigés vers le sud, postérieurement au III\textsuperscript{e} siècle, et ayant dominé longtemps dans l’Allemagne méridionale, ils s’y étaient mariés aux Germains celtisés des invasions précédentes, comme aussi à tous les éléments divers, kymriques et slaves, qui pouvaient s’y trouver en fusion. Leur destinée ressembla en beaucoup de points à celle des Longobards, avec cette nuance cependant que leur sang put se conserver un peu davantage. Ils eurent le bonheur de se trouver directement, à dater du VII\textsuperscript{e} siècle, sous le coup d’un groupe germanique dont la pureté correspondait à celle des Goths, la nation des Franks. S’ils se virent promptement réduits à obéir à ces supérieurs, ils leur durent des immixtions ethniques très favorables.\par
Les Franks, qui survécurent comme nation puissante à presque toutes les autres branches de la souche commune, même à celle des Goths, n’avaient été qu’à peine entrevus, dans le noyau de leur race, par les historiens romains du I\textsuperscript{e} siècle de notre ère \footnote{Pline connaît ce peuple.}. Leur tribu royale, les Mérowings, habitait alors et jusqu’au VI\textsuperscript{e} siècle compta encore des représentants sur un territoire, assez borné, situé entre les embouchures de l’Elbe et de l’Oder, aux bords de la Baltique, au-dessus de l’ancien séjour des Longobards. Il est évident, d’après cette situation géographique, que les Mérowings étaient issus de la Norwège, et n’appartenaient pas à la branche gothique \footnote{ \noindent C’est le pays appelé par l’anonyme de Ravenne, \emph{Maurungania}, la terre des Mérowings. -Le poème de \emph{Beowulf} établit bien la relation entre les Mérowings et les Franks lorsqu’il dit, v. 5836 :\par
 
\begin{verse}
Us waes à-Syddan\\
 Mere-wionigas\\
 Milts un-gyfede,\\
\end{verse}
 \noindent « Depuis ce temps, la bienveillance des Mérovingiens nous a toujours été refusée », c’est-à-dire depuis que les Franks sont en guerre avec celui qui parle. (Kemble, \emph{Anglosaxon Poëm of Beowulf}, p. 206. – Ettmuller, \emph{Beowulfslied}, 21. – J. Bachlechner, \emph{Zeitschrift f. a. Alt}., t. VIII, p. 526.) – Keferstein montre bien comment, par la route qu’ils suivirent dans leur migration de l’extrême nord, les Franks ont pu arriver jusque dans la Gaule sans avoir été nullement mêlés aux Slaves et presque point aux Celtes purs. (T. I, p. XXXIV)
}. Ils acqui­ rent une grande prépondérance dans l’histoire des territoires gaulois postérieurement au V\textsuperscript{e} siècle. Toutefois, aucune des généalogies divines que l’on possède aujourd’hui ne les mentionne et ne permet de les rattacher à Odin, circonstance essentielle cependant, au gré des nations germaniques, pour fonder les droits à la royauté, et que remplirent, aussi bien que les Amalungs gothiques, les Skildings danois, les Astings suédois, et toutes les dynasties de l’heptatchie anglo-saxonne \footnote{Les généalogies héroïques qui nous ont été conservées, soit dans l’Edda, soit dans les annales compilées par des moines, soit dans les préambules des différents codes, constituent une des sources les plus importantes que l’on puisse consulter pour l’histoire germanique des plus anciennes époques. (Voir à ce sujet Grimm, W. Muller, Ettmuller, etc.) La forme des noms, l’ordre dans lequel ils sont placés, le nombre des aïeux donnés à Odin lui-même, enfin les traces d’allitération qui se retrouvent dans les compilations en prose sont autant de traits dignes d’être observés avec la plus extrême attention pour les résultats importants auxquels ils amènent. Je remarque surtout trois noms parmi les aieux d’Odin, \emph{Suaf, Heremod} et \emph{Géat} ; ce sont autant de souvenirs ethniques se rapportant aux grandes dénominations nationales de Saka, d’Arya, et de Khéta. On en peut signaler encore deux autres, indiquant des mélanges qui certainement ont eu lieu : \emph{Hwala, Gall}, et \emph{Funi, Fenn}.}. Malgré ce silence des documents, il n’y a pas à douter, en voyant la prééminence incontestée des Mérowings parmi les Franks, et la gloire de cette nation, que l’origine divine, la descendance odinique, autrement dit la condition de pureté ariane, ne faisait pas défaut à cette famille de rois, et que c’est uniquement par l’effet destructeur des temps que ses titres ne sont pas venus jusqu’à nous.\par
Les Franks étaient descendus assez promptement sur le Rhin inférieur où le poème de Beowulf les montre en possession des deux rives du fleuve, et séparés de la mer par les Flamands, Flæmings, et les Frisons, deux peuples avec lesquels leur alliance était étroite \footnote{Les Frisons s’étaient autrefois appelé \emph{Eotenas, Eotan} ou \emph{Jutæ}. C’étaient des \emph{Jotuns} germanisés. (Ettmuller, \emph{Beowulfslied}, p. 36.)} Là, ils ne trouvèrent sous leurs pas que des races extrêmement et de longue main germanisées \footnote{Parmi celles qui l’étaient le moins, on peut compter les Ubiens. Mais l’élément celtique n’en avait pas moins été très fortement affaibli chez cette nation par les mélanges d’autre nature qu’avaient apportées les Romains. (Dieffenbach, \emph{Celtica I}, p. 68.) Les Sicambres, dont le nom joue un rôle dans nos premières annales, étaient nécessairement germanisés à un très haut point, leur situation géographique le voulant ainsi. Cependant leur nom est celtique et rappelle celui des \emph{Segobrigi}, nation qui très anciennement était connue de la colonie phocéenne de Marseille. Ce nom paraît signifier les \emph{illustres Ambres} ou \emph{Kymris.}}, et de ce fait uni à leur départ tardif des pays les plus arians, ils emportèrent de puissantes garanties de force et de durée pour l’empire qu’ils allaient fonder. Cependant, sur le dernier point, plus favorisés que les Vandales, que les Longobards, que les Bourguignons, et même que les Goths, ils le furent moins que les Saxons, et, s’ils eurent plus d’éclat, ils leur cédèrent en longévité. Ceux-ci ne furent jamais portés par leurs conquêtes extérieures dans les parties vives du monde romain \footnote{Keferstein, \emph{ouvr. cité}, t. 1, p. XXXIV.}. En conséquence, ils n’eurent pas de contact avec les races les plus mélangées, le plus anciennement cultivées, mais aussi les plus affaiblissantes. À peine peut-on les compter au nombre des peuples envahisseurs de l’empire, bien que leurs mouvements aient commencé presque en même temps que ceux des Franks. Leurs principaux efforts se portèrent sur l’est de l’Allemagne et sur les îles bretonnes de l’Océan occidental. Ils ne contribuèrent donc nullement à régénérer les masses romaines. Ce défaut de contact avec les parties vives du monde civilisé, qui les priva d’abord de beaucoup d’illustration, leur a été avantageux au plus haut degré. Les Anglo-Saxons représentent, parmi tous les peuples sortis de la péninsule scandinave, le seul qui, dans les temps modernes, ait conservé une certaine portion apparente de l’essence ariane. C’est le seul qui, à proprement parler, vive encore de nos jours. Tous les autres ont plus ou moins disparu, et leur influence ne s’exerce plus qu’à l’état latent.\par
Dans le tableau que je viens de tracer, j’ai laissé de côté les détails. Je ne me suis pas arrêté à décrire les innombrables petits groupes qui, toujours en mouvement, sans cesse traversant et retraversant les voies des masses plus considérables, contribuent à donner aux invasions des IVe et V\textsuperscript{e} siècles cette apparence fiévreuse et tourmentée qui n’est pas une des moindres causes de leur grandeur. Il faudrait, pour bien faire, se représenter vivement et dans un incessant tumulte ces myriades de tribus, d’armées, de bandes en expédition, qui, poussées par les causes les plus diverses, tantôt la pression des nations rivales, tantôt le surcroît de population, ici la famine, là une ambition subitement éveillée, d’autres fois le simple amour de la gloire et du butin, se mettaient en marche, et, secondées par la victoire, déterminaient de proche en proche les plus terribles ébranlements \footnote{Dans ce nombre sont les Astings, les Scyrres, les Ruges, les Gépides et surtout les Hérules. Tous ces groupes, qui de même que les gens d’Arioviste, constituaient plutôt des armées, ou même des bandes en expédition, que des peuples à la recherche d’un gîte retournaient très souvent dans le Nord après avoir beaucoup épouvanté le Sud. (Munch, p. 44.)}. Depuis la mer Noire, depuis la Caspienne jusqu’à l’océan Atlantique, tout s’agitait. Le fond celtique et slave des populations rurales débordait incessamment d’un pays sur l’autre, emporté par l’impétuosité ariane ; et, au milieu de mille cohues, les cavaliers mongols d’Attila et de ses alliés, se faisant jour au travers de ces forêts d’épées et de ces troupeaux effarés de laboureurs, y traçaient dans tous les sens d’ineffaçables sillons. C’était un désordre extrême. Si à la surface apparaissaient de grandes causes de régénération, dans les profondeurs tombaient de nouveaux éléments ethniques d’abaissement et de ruine que l’avenir allait avoir beau jeu à développer.\par
Résumons maintenant l’ensemble des mouvements arians en Europe, je dis des mouvements qui aboutirent à la formation des groupes germaniques et à la descente de ceux-ci sur les frontières de l’empire romain. Vers le VIII\textsuperscript{e} siècle avant notre ère, les tribus sarmates roxolanes se dirigent vers les plaines du Volga. Au IV\textsuperscript{e}, elles occupent la Scandinavie et quelques points de la côte baltique vers le sud-est. Au III\textsuperscript{e}, elles commencent à refluer en deux directions vers les contrées moyennes du continent. Dans la région occidentale, leurs premières nappes rencontrent des Celtes et des Slaves ; à l’est, outre ces derniers, d’assez nombreux détritus arians, provenant des invasions très anciennes des Sarmates, des Gètes, des Thraces, bref des collatéraux de leurs propres ancêtres, sans compter les dernières nations de race noble qui continuaient à sortir de l’Asie. De là, supériorité marquée chez les tribus gothiques, que de tels mélanges ne pouvaient affaiblir. Peu à peu cependant l’égalité, l’équilibre ethnique entre les deux courants se rétablit. À mesure que les premières émissions occidentales sont recouvertes par de nouvelles plus pures, l’invasion scandinave s’élève aux plus majestueuses proportions ; de telle sorte que, si les Sicambres et les Chérusques avaient promptement cessé d’équivaloir aux hommes de l’empire gothique, les Franks peuvent être hardiment considérés comme les dignes frères des guerriers d’Hermanrik, et à plus forte raison les Saxons de la même époque ont droit au même éloge.\par
Mais, en même temps que tant de grandes races affluaient vers la Germanie méridionale, la Gaule et l’Italie, les catastrophes hunniques, arrachant les Goths et les derniers Alains à leurs sujets slaves, les reportaient en masse sur les points où les autres nations germaniques tendaient également à se concentrer. Il en résulta que l’orient de l’Europe, à peu près dépouillé de ses forces arianes, fut rendu au pouvoir des Slaves et des envahisseurs de race finnique, qui devaient plonger définitivement ces derniers dans l’abaissement irrémédiable dont de plus nobles dominateurs n’avaient jamais eu l’influence de les tirer. Il en résulta aussi que toutes les forces de l’essence germanique tendaient à s’accumuler d’une façon à peu près exclusive dans les parties les plus occidentales du continent, voire dans le nord-ouest. De cette disposition des principes ethniques devait résulter toute l’organisation de l’histoire moderne. Maintenant, avant d’aller plus loin, il convient d’examiner en elle-même cette famille ariane germanique dont nous venons de suivre les étapes. Rien de plus nécessaire que de préciser exactement sa valeur avant de l’introduire au milieu de la dégénération romaine.
\section[{VI.3. Capacité des races germaniques natives.}]{VI.3. \\
Capacité des races germaniques natives.}
\noindent Les nations arianes d’Europe et d’Asie, prises dans leur totalité, observées dans leurs qualités communes et typiques, nous ont également étonnés par cette attitude impérieuse et dominatrice qu’elles exercèrent constamment sur les autres peuples, même sur les peuples métis et blancs au milieu desquels ou auprès desquels elles vécurent. À ce seul aspect, il est déjà difficile de ne pas leur reconnaître à l’égard du reste de l’espèce humaine une suprématie réelle ; car en pareilles matières ce qui semble existe nécessairement. Il ne faudrait cependant pas prendre le change sur la nature de cette suprématie et la chercher ou prétendre la trouver dans des faits qui ne lui appartiendraient pas. Il ne faut pas davantage la croire obscurcie et mise en question par certains détails qui choquent les préventions vulgaires sur l’idée généra­lement admise de supériorité. Celle des Arians ne réside pas dans un développement exceptionnel et constant des qualités morales ; elle existe dans une plus grande provision des principes d’où ces qualités découlent.\par
Il ne faut jamais oublier que, lorsqu’on étudie l’histoire des sociétés, il ne s’agit en aucune façon de la moralité en elle-même. Ce n’est ni par des vices ni par des vertus que des civilisations se distinguent essentiellement les unes des autres, bien que, prises dans l’ensemble, elles valent mieux sous ce rapport que la barbarie ; mais c’est là une conséquence purement accessoire de leur travail. Ce qui fait essentiellement leur physionomie, ce sont les capacités qu’elles possèdent et développent.\par
L’homme est l’animal méchant par excellence. Ses besoins plus multipliés le harcèlent de plus d’aiguillons. Dans son espèce, il a d’autant plus de besoins, partant de souffrances, partant d’excitations au mal, qu’il est plus intelligent. Il semblerait donc naturel que ses mauvais instincts augmentassent en raison directe de la nécessité de briser plus d’obstacles pour arriver à un état de satisfaction. Mais, par un heureux retour, il n’en est pas ainsi. La raison, plus perfectionnée en même temps qu’elle vise plus haut et est plus exigeante, éclaire la créature qu’elle conduit sur les inconvénients matériels d’un abandon trop absolu à toutes les suggestions de l’intérêt. La religion, même imparfaite ou fausse, que cet être conçoit toujours d’une façon quelque peu élevée, lui interdit de céder en toute occasion à ses penchants destructeurs.\par
C’est ainsi que l’Arian est toujours sinon le meilleur des hommes au point de vue de la pratique morale, du moins le plus éclairé sur la valeur intrinsèque en ce genre des actes qu’il commet. Ses idées dogmatiques sont toujours en cette matière les plus développées et les plus complètes, bien que dépendant étroitement de l’état de sa fortune. Tant qu’il est le jouet d’une situation trop précaire, son corps reste cuirassé et son cœur de même ; dur envers sa propre personne, rien de moins étonnant qu’il soit impitoyable pour autrui, et c’est dans cette donnée inflexible qu’il pratique cette justice dont Hérodote vantait l’intégrité chez le Scythe belliqueux. Le mérite consiste ici dans la loyauté avec laquelle est acceptée une loi d’ailleurs si féroce peut-être, et qui ne s’adoucit que dans la proportion où l’atmosphère sociale ambiante réussit elle-même à se tempérer.\par
L’Arian est donc supérieur aux autres hommes, principalement dans la mesure de son intelligence et de son énergie ; et c’est par ces deux facultés que, lorsqu’il parvient à vaincre ses passions et ses besoins matériels, il lui est également donné d’arriver à une moralité infiniment plus haute, bien que, dans le cours ordinaire des choses, on puisse relever chez lui tout autant d’actes répréhensibles que chez les individus des deux autres espèces inférieures.\par
Cet Arian se présente maintenant à notre observation dans le rameau occidental de sa famille, et là il nous apparaît aussi vigoureusement bâti, aussi beau d’aspect, aussi belliqueux de cœur, que nous l’avons admiré jadis dans l’Inde \footnote{« L’inclito mio figlio Rama dagli occhi del color del loto. » (\emph{Ramayana}, t. VII, \emph{Ayodyacanda}, cap. III, p. 218.)} et dans la Perse, comme dans l’Hellade homérique. Une des premières considérations auxquelles l’aspect du monde germanique donne lieu, c’est encore celle-ci, que l’homme y est tout et la nation peu de chose. On y aperçoit l’individu avant de voir la masse associée, circonstance fondamentale, qui excitera d’autant plus l’intérêt qu’on prendra plus de soin de la comparer avec le spectacle offert par les agrégations de métis sémitiques, helléniques, romains, kymris et slaves. Là on ne voit presque que les multitudes ; l’homme ne compte pour rien, et il s’efface d’autant plus que, le mélange ethnique auquel il appartient étant plus compliqué, la confusion est devenue plus considérable.\par
Ainsi placé sur une sorte de piédestal, et se dégageant du fond sur lequel il agit, l’Arian Germain est une créature puissante, qui attire d’abord l’examen sur lui-même avant de permettre de le porter sur le milieu qui l’entoure. Tout ce que cet homme croit, tout ce qu’il dit, tout ce qu’il fait, acquiert de la sorte une importance majeure.\par
En matière de religion et de cosmogonie, voici quels sont ses dogmes la nature est éternelle, la matière infinie \footnote{W. Muller, \emph{Altdeutsche Religion}, p. 163.}. Cependant le vide béant\emph{, gap gunninga}, le chaos, a précédé toutes choses \footnote{\emph{Vœluspa}, 3.}. « En ce temps dit la Vœluspa, il n’y « avait ni sable, ni mer, ni les molles vagues. La terre ne se trouvait nulle part, ni le « ciel enveloppant. Du sein des ténèbres sortirent douze fleuves, qui en coulant « gelèrent. »\par
Alors l’air doux qui venait du sud, de la contrée du feu, fit fondre la glace ; ses gouttes d’eau prirent vie, et le géant Imir, personnification de la nature animée, apparut. Bientôt il s’endormit, et de sa main gauche ouverte, et de ses pieds fécondés l’un par l’autre, sortit la race des géants \footnote{W. Muller, p. 164.}.\par
Cependant la glace continuant à dégeler, il en provint la vache \emph{Audhumbha}. C’est le symbole de la force organique, qui donne le mouvement à toutes choses. À ce moment, un être nommé Buri sortit encore de ces gouttes d’eau, et il eut un fils, Börr, qui, s’unissant à la fille d’un géant, donna le jour aux trois premiers dieux, les plus anciens, les plus vénérables, Odhin, Vili et Ve \footnote{Ibid., p. 165. – Il est inutile de donner ici les développements ultérieurs de cette formule théologique, qui finit par contenir douze grands dieux et une foule de personnalités célestes de tout ordre et de toute provenance ; car il y eut des dieux wanes, jotuns et nanis, comme il y avait des dieux ases.}.\par
Cette trinité, ainsi venue quand les grandes créations cosmiques étaient déjà achevées, n’avait à réaliser qu’un travail d’organisation, et en effet ce fut là sa tâche. Elle ordonna le monde, et de deux troncs d’arbre échoués sur le rivage de la mer, elle façonna les durs auteurs de l’espèce humaine. Un chêne fut l’homme, un saule devint la femme \footnote{Ibid., ouvr. cité, p. 164. – Vœlusp, st. 17. – Je ne développe ici que les plus grands traits de la théologie et de la cosmogonie scandinaves, ne m’arrêtant surtout qu’aux parties les plus anciennes. La nouvelle Edda montre de nombreuses traces de mythes qui ne sont pas originairement arians ou qui ont été développés dans l’extrême Nord postérieurement à l’arrivée des Roxolans. – Le plus vénérable document scandinave, la Vœluspa, a été composé dans la première moitié du VIII\textsuperscript{e} siècle de notre ère. M. Dietrich y aperçoit des traces de cinq différents poèmes, beaucoup plus antiques. (Dietrich, Alter der Vœluspa, dans la Zeitschr. f. deutsch. Alterth., t. VIII, p. 318.)}.\par
Cette doctrine n’est toujours que le naturalisme arian, modifié par des idées développées dans l’extrême Nord \footnote{César pense que les Germains, ne reconnaissant pour dieux que les forces naturelles qui se manifestaient à leur vue, n’adoraient que le soleil, la lune et le feu, Sol, Luna, Vulcanus. (\emph{De Bello gall.}, VI, 21.)}. La matière vivante et intelligente, représentée encore par le mythe tout asiatique de la vache \emph{Audhumbha}, s’y maintient au-dessus des trois grands dieux eux-mêmes. Ils sont nés après elle : rien de moins étonnant qu’ils ne soient pas copartageants de son éternité. Ils doivent périr ; ils doivent disparaître un jour, vaincus par les géants, par les forces organiques de la nature, et cette organisation du monde dont ils sont les ordonnateurs est destinée à s’engloutir avec eux, avec les hommes leurs créatures, pour faire place à de nouveaux ordonnateurs, à un nouvel arrangement de toutes choses, à de nouvelles générations de mortels. Encore une fois, les antiques sanctuaires de l’Inde connaissaient l’essentiel de toutes ces notions \footnote{W. Muller, \emph{ouvr. cité}, p. 175.}.\par
Des dieux transitoires, si grands qu’ils fussent, n’étaient pas trop distants de l’homme. Aussi l’Arian Germain n’avait-il pas perdu l’habitude de s’élever jusqu’à eux. Sa vénération pour ses ancêtres confondait volontiers ceux-ci avec les puissances supérieures, et sans effort se changeait en adoration. Il aimait à se croire descendu de plus grand que lui, et de même que tant de races helléniques se rattachaient à Jupiter, à Neptune, au dieu de Chryse, de même le Scandinave traçait fièrement sa généalogie jusqu’à Odin, ou jusqu’aux autres individualités célestes que les conséquences naturelles du symbolisme firent monter sans peine autour de la trinité primitive \footnote{Les plus nobles familles, se rappelant le Gardarike, se représentaient leurs aïeux comme ayant vécu dans Asgard, que la tradition avait divinisée. (Munch, \emph{ouvr. cité}, p. 53.)}.\par
L’anthropomorphisme était complètement étranger à ces notions natives \footnote{W. Muller, \emph{ouvr. cité}, p. 64 et sqq. – Tac., \emph{Germ}., 9, 43.} ; il ne s’y associa que fort tard et sous l’influence irrésistible des mélanges ethniques. Tant que le fils des Roxolans resta pur, il se plaisait à ne voir les dieux que dans le miroir de son imagination, et répugna à se faire d’eux des images tangibles. Il aimait à se les figurer planant à demi cachés au sein des nuages rougis par les lueurs du couchant. Les bruits mystérieux des forêts lui révélaient leur présence \footnote{Tac., \emph{Ann.}, XIII, 55 ; \emph{Germ.}, 45. – Ils n’avaient pas et n’admettaient pas de temples, tandis que les populations celtiques de la Gaule et de l’Allemagne en avaient.}. Il croyait aussi trouver et il vénérait une émanation de leur nature dans certains objets précieux pour lui. Les Quades prêtaient serment sur des épées, ce qu’avaient déjà fait les Thraces. Les Longobards honoraient un serpent d’or ; les Saxons, un groupe mystique formé d’un lion, d’un dragon et d’un aigle ; les Franks avaient aussi des usages semblables \footnote{W. Muller, \emph{ouvr. cité}, p. 67, 70 et pass.}.\par
Mais des alliances avec les métis européens leur firent accepter plus tard, en tout ou en partie, le panthéon matériel des Slaves et des Celtes. Ils devinrent alors idolâtres. Chez les Suèves, ils admirent le culte sauvage de la déesse Nerthus, et apprirent à promener, une fois l’an, sa statue voilée dans un char \footnote{Tous les cultes indiqués par les écrivains romains portent la trace et révèlent la puissance de l’influence celtique. \emph{Nerthus, mater deum}, se retrouve dans le gallois \emph{neath}, force, secours, et dans le gaélique \emph{neart}, qui a le même sens. – L’usage de consacrer des îles principalement comme sanctuaires est tout à fait celtique. (W. Muller, \emph{ouvr. cité}, p. 37.) Cet auteur signale chez les Danois des usages religieux d’origine slave (p. 37). – L’Isis dont parle Tacite, et qu’il s’étonne de trouver chez les Suèves, c’était \emph{Hésu} ou \emph{Hu}, divinité celtique par excellence. (Tac., \emph{Germ}., 9.)}. Le sanglier de Freya, symbole favori des Galls, fut adopté par la plupart des nations germaniques, qui en surmontè­rent le cimier de leurs casques, et le firent briller sur les pignons de leurs palais. Jadis, dans les époques purement arianes, les Germains n’avaient pas même connu les temples. Ils finirent par en avoir, où ils entassèrent des idoles monstrueuses \footnote{Adam de Brème parle d’une statue de Wodan, qui se trouvait de son temps dans le temple d’Upsala. (W. Muller, p. 195.)}. Comme il était arrivé aux anciens Kymris, il leur fallut complaire, à leur tour, aux instincts les plus tenaces des races inférieures au milieu desquelles ils s’étaient établis \footnote{Il arriva même que tel dieu considéré en Scandinavie comme des plus puissants, Wodan, par exemple, fut à peu près inconnu chez les tribus demi-germanisées du sud de l’Allemagne. Les Bavarois ne le connaissaient pas, ou, pour mieux dire, ce qu’ils avaient de germanique dans leur sang ne l’avait pas conservé. (W. Muller, p. 76.)}.\par
Il en fut de même pour les formes du culte, cependant avec plus de mesure dans la dégénération. Primitivement l’Arian Germain était à lui-même son prêtre unique, et même longtemps après qu’on eut institué des pontifes nationaux, chaque guerrier conserva dans ses foyers la puissance sacerdotale \footnote{W. Muller, \emph{ouvr. cité}, p. 52, 81, 83.}. Elle resta même annexée à la propriété foncière, et l’aliénation d’un domaine entraîna celle du droit d’y sacrifier \footnote{Sous l’influence celtique, slave et finnique, les fonctions et, comme on dirait aujourd’hui, les \emph{spécialités} religieuses ou seulement superstitieuses se développèrent, avec le temps, d’une façon très surabondante. En même temps qu’il y eut chez les Goths, chez les Thuringiens, chez les Burgondes, chez les Anglo-Saxons, des grands prêtres, qui finirent même par exercer une certaine action politique, principalement chez les Burgondes, il y eut aussi des devins, des sorciers, des enchanteurs, des schamans de toute espèce. Les uns expliquaient les songes, les autres pénétraient l’avenir au moyen de cordes nouées. Or, appelait ces derniers \emph{caragni}, du gallois \emph{caraï, une cordelette}. (W Muller, \emph{ouvr. cité}, p. 83.) Mais tout cela ne concerne pas les nations germaniques.}. Lorsqu’on modifia cet état de choses, le prêtre germanique n’exerça d’action que pour l’ensemble de la tribu. Il ne fut d’ailleurs jamais que ce qu’avait été le purohita chez les Arians Hindous, dans les temps antévédiques. Il ne forma pas une caste distincte comme les brahmanes, un ordre puissant comme les druides, et, non moins sévèrement exclu des fonctions de la guerre, il ne lui fut pas laissé la moindre possibilité de dominer, ni même de diriger l’ordre social. Toutefois, par un sentiment empreint d’une haute et profonde sagesse, à peine les Arians eurent-ils reconnu des prêtres publics qu’ils leur confièrent les plus imposantes fonctions civiles, en les chargeant de maintenir l’ordre dans les assemblées politiques et d’exécuter les arrêts de la justice criminelle. De là chez ces peuples ce qu’on a appelé les sacrifices humains \footnote{W. Muller, \emph{ouvr. cité}, p. 52.}.\par
Le condamné, après avoir entendu sa sentence, était retranché de la société et livré au prêtre, c’est-à-dire au dieu. Une main sacrée, lui infligeant le dernier supplice, apaisait sur lui la colère céleste. Il tombait, non pas tant parce qu’il avait offensé l’humanité que parce qu’il avait irrité la divinité protectrice du droit. Le châtiment se trouvait de la sorte moins honteux pour la dignité de l’Arian et, il faut l’avouer, plus moral que ne le rendent nos coutumes juridiques, où un homme est égorgé simplement en compensation d’en avoir égorgé un autre, ou, suivant une opinion plus étroite encore, simplement pour le forcer à s’en tenir là \footnote{Les sacrifices humains sont attestés, par des témoignages positifs chez les Goths, chez les Hérules, chez les Saxons, chez les Frisons, chez les Thuringiens, chez les Franks, à l’époque où ces derniers étaient déjà chrétiens. (W. Muller, \emph{ouvr. cité}, p. 75-79.) – Le sacrifice des chevaux était aussi, dans la plus ancienne époque germanique, comme l’asvamédha, chez les Arians Hindous, une des cérémonies du culte les plus solennelles et les plus méritoires.}.\par
On s’est demandé, avec plus ou moins de raison, si les nations sémitiques avaient eu originairement une idée bien nette de l’autre vie. Chez aucune race ariane ce doute n’est possible. La mort ne fut jamais pour toutes qu’un passage étroit, à la vérité, mais insignifiant, ouvert sur un autre monde. Ils y entrevoyaient diverses destinées, qui, d’ailleurs, n’étaient pas déterminées par les mérites de la vertu ou le châtiment qu’aurait dû recevoir le vice. L’homme de noble race, le véritable Arian arrivait par la seule puissance de son origine à tous les honneurs du Walhalla, tandis que les pauvres, les captifs, les esclaves, en un mot, les métis et les êtres d’une naissance inférieure, tombaient indistinctement dans les ténèbres glaciales du Niflheimz \footnote{Cette notion se conserva très longtemps chez les Arians de l’Inde. À l’époque héroïque, elle régnait encore, ainsi que le passage suivant en fait foi. « Chi ha sortito il nascere da una « schialta pari alla tua, non puô ire in infimo, luogo ; par laqual cosa tu, privato della « terrestre sede, vanne ai mondi dove stella il neltare. » (\emph{Ramayana}, t. VI, Ayodhyacanda, cap. LXVI, p. 394)}.\par
Cette doctrine ne fut évidemment de mise que pendant les époques où toute gloire, toute puissance, toute richesse se trouva concentrée entre les mains des Arians et où nul Arian ne fut pauvre en même temps que nul métis ne fut riche. Mais lorsque l’ère des alliages ethniques eut complètement troublé cette simplicité primitive des rapports, et que l’on vit, ce qui aurait été jugé impossible autrefois, des gens de noble extraction dans la misère, et des Slaves et des Kymris, et même des Tchoudes, des Finnois opulents, les dogmes relatifs à l’existence future se modifièrent, et l’on accepta des opinions plus conformes à la distribution contemporaine des qualités morales dans les individus \footnote{W. Muller, \emph{ouvr. cité}, p. 410.}.\par
L’Edda partage l’univers en deux parties \footnote{\emph{Vœluspa}, st. 2.}. Au centre du système, la terre, résidence des hommes, formée comme un disque plat, ainsi que l’a décrite Homère, est entourée de tous côtés par l’Océan. Au-dessus d’elle s’étend le ciel, demeure des dieux. Au nord s’ouvre un monde sombre et glacé, d’où vient le froid ; au sud, un monde de feu, où s’engendre la chaleur. À l’est est Jotanheimz, le pays des géants ; à l’ouest, Svartalfraheimz, la demeure des nains noirs et méchants. Puis, dans une situation vague, Vanaheimz, la contrée habitée par les Wendes \footnote{Vœluspa pass. – On retrouve dans les noms des nains donnés par la Vœluspa, des appellations bien significatives, telles que Nar, Naïn, st. 11 ; Nori, Ann et Anar, puis encore une fois par Nar, puis Nyzardz, st. 12 ; Nali, et Hanar, st. 13 ; Alfr, st. 14, Funiar et Guinar, st. 16. – Il est à remarquer que les nains, non plus que les géants, n’ont pas été créés par les dieux comme l’homme, mais sont le produit direct des forces de la nature.}.\par
Si l’on arrête ici cette description, où s’unissent les idées cosmogoniques à la simple géographie, on a l’exacte reproduction du système des sept divissas brahmani­ques, ou, ce qui est pareil, des sept kischwers iraniens \footnote{C’est même à cette partie de la cosmogonie des Arians primitifs qu’il convient de rattacher celle des Scandinaves, descendants légitimes et directs des cavaliers du Touran. Quand on veut suivre la filiation des idées arianes, il importe de ne jamais perdre de vue que les Hindous, qui en ont, à la vérité, conservé jusqu’à nos jours le plus riche trésor, ne sont cependant pas l’intermédiaire auquel nous les devons. En marche vers la vallée du Gange, ils n’ont rien pu faire pour éclairer l’Occident ; c’est surtout aux groupes arians de la Sogdiane et des pays situés au-dessus que nous sommes redevables de ce que nous possédons, dans nos antiquités germaniques, de l’ancien fonds des connaissances primordiales. Malheureusement la philologie justement séduite, d’ailleurs, par l’importance des Védas, est tout occupée, en France surtout, à méconnaître cette vérité, et n’hésite même pas à faire émigrer les Germains des bords de la Yamouna, ce qui, en soi, constitue une absurdité au premier chef.}, et, comme on va le voir, un monde complet, au point de vue des premiers Arians Germains. Le territoire scandinave occupe le centre : c’est excellemment le pays des hommes. L’empyrée règne au-dessus. Le pôle nord lui envoie la froidure ; les régions méridionales, le peu de chaleur qui l’atteint. À l’est, c’est-à-dire tirant vers la côte de la Baltique, sont les principales tribus des Gètes métis ; à l’ouest, entre la Suède méridionale et la côte de l’Océan du Nord, les Lapons, un peu partout, des Wendes et des Celtes, justement confondus les uns avec les autres. Les connaissances positives de l’époque ne permettent pas d’ajouter rien. Mais les cosmographes nationaux, dans le travail de leurs idées, ne s’en tinrent pas à ces anciennes notions ; ils voulurent avoir neuf divissas, neuf kischwers, au lieu de sept qu’avaient connus leurs ancêtres, et, pour atteindre à ce chiffre, ils imaginèrent deux cieux nouveaux, placés au-dessus de celui des dieux, et les nommèrent, l’un Liôsâlfraheimz ou Andlanger, l’autre Vidhblacên \footnote{W Muller, \emph{ouvr. cité}, p. 163.}. Tous deux sont peuplés de nains lumineux. Cette conception serait absolument arbitraire et inutile, si elle ne se fondait pas, en quelque chose, sur la distinction que les plus anciens Arians de la haute Asie paraissent avoir faite entre l’atmosphère immédiate du globe et le ciel proprement dit, l’empyrée, où se meuvent les astres \footnote{Lorsque les doctrines scandinaves auront été comparées plus rigoureusement qu’on ne l’a fait encore aux idées iraniennes, on reconnaîtra sans doute que de grands rapports unissent les habitants célestes du Liôsâlfraheimz et du Adlanger aux Ireds et aux Amschespends du Zend-Avesta.}.\par
Telles étaient les opinions que l’Arian Germain entretenait sur les objets de considération les plus élevés. Il y puisait sans peine une haute idée de lui-même et de son rôle dans la création, d’autant plus qu’il s’y contemplait non seulement comme un demi-dieu, mais comme un possesseur absolu d’une portion de ce Mitgardhz, ou \emph{terre du milieu}, que la nature lui avait assigné pour demeure. Il avait constitué sa propriété foncière d’une manière toute conforme à ses fiers instincts. Deux modes de propriété étaient chez lui en usage.\par
Le plus ancien incontestablement est celui dont il avait apporté l’idée constitutive de la haute Asie, c’était l’\emph{odel} \footnote{Ce mot est un des plus anciens qui se puissent trouver, et la notion qu’il représente est vieille comme lui. C’est l’\emph{ædes} latin. – Voir, pour les différentes formes et significations dans les langues gothiques, Dieffenbach, \emph{Vergleichendes Wœrterbuch der gothischen Sprache}, t. I, p. 56.}. Ce mot emporte avec lui les deux idées de noblesse et de possession si intimement combinées, que l’on est fort embarrassé de découvrir si l’homme était propriétaire parce qu’il était noble, ou l’inverse \footnote{Chez les Anglo-Saxons il arriva même que la perte de l’odel entraînait celle des droits politiques, et par conséquent de la qualité d’homme libre. (Kemble, t. I, p. 70-71 et seqq.) On peut voir, du reste, avec toute raison, dans cette union étroite de la qualité légale d’Arian avec celle de propriétaire, à quel point les instincts de la race étaient éloignés des dispositions à la vie nomade.}. Mais il est peu douteux que l’organisation primordiale, ne reconnaissant pour homme véritable que l’Arian, ne voyait aussi de propriété régulière qu’entre ses mains et n’imaginait pas d’Arian privé de cet avantage.\par
L’odel appartenait sans restriction aucune à son maître. Ni la communauté ni le magistrat n’avaient qualité pour exercer sur cette sorte de possession la revendication la plus légère, le droit le plus minime. L’odel était absolument libre de toute charge  ; il ne payait pas d’impôts. Il constituait une véritable souveraineté, souveraineté inconnue aujourd’hui, où la nue-propriété, l’usufruit et le haut domaine se confondaient absolument. Le sacerdoce en était inséparable, et inséparable aussi la juridiction à tous ses degrés, au civil comme au criminel. L’Arian Germain siégeait à son foyer, disposait à son gré de la terre allodiale et de tout ce qui l’habitait. Femmes, enfants, serviteurs, esclaves, ne reconnaissaient que lui, ne vivaient que par lui, ne rendaient compte qu’à lui seul, qui ne rendait compte à personne. Soit qu’il eût construit sa demeure et mis ses champs en culture sur un terrain désert, soit que ses propres forces lui eussent suffi pour en dépouiller le Finnois, le Slave, le Celte ou le Jotun, tous gens placés nativement hors la loi, ses prérogatives ne rencontraient pas de limites.\par
Il n’en était pas tout à fait de même lorsque, en société avec d’autres Arians, agissant sous la direction commune d’un chef de guerre, il se trouvait être participant à la conquête d’un territoire dont une portion, grande ou petite, lui avait été adjugée. Cette autre situation créait un autre système de tenure tout différent ; et comme elle se réalisa presque seule quand furent venues les grandes migrations sur le continent d’Europe, on y doit chercher le germe véritable des principales institutions politiques de la race germanique. Mais pour pouvoir exposer clairement ce que c’était que cette forme de propriété et les conséquences qu’elle entraînait, il faut faire connaître auparavant les rapports de l’homme arian avec sa nation.\par
En tant qu’il était chef de famille et possesseur d’un odel, ces rapports se réduisaient à fort peu de chose. D’accord avec les autres guerriers pour conserver la paix publique, il élisait un magistrat, que les Scandinaves nommaient \emph{drottinn}, et que d’autres peuples sortis de leur sang appelèrent \emph{graff} \footnote{Palsgrave a eu pleine raison de dire que la royauté n’existait pas, dans les formes et avec la puissance qu’on lui a connues après le V\textsuperscript{e} siècle, aux époques véritablement germaniques. (\emph{The Rise and Progress of the English Commonwealth}, in-4°, Lond., 1832, t. I, p. 553.) Il est moins bien inspiré quand il ne voit dans le mot \emph{king} qu’un emprunt fait aux langues celtiques. C’est, de toute antiquité, un titre porté par les chefs militaires des nations arianes. Nous l’avons vu chez les Ou-douns. (Voir tome I\textsuperscript{er}). C’est le \emph{kava} de la première période iranienne. (Westergaard et Lassen, \emph{Die Achem. Keilinschriften}, p. 122), le \emph{ku} des inscriptions médiques (\emph{ibid.}, p. 57). Il est assez remarquable qu’on ne le donnât pas aux magistrats réguliers et ordinaires des tribus. – Quant au titre de \emph{graff}, ou\emph{ gereta}, chez les Anglo-Saxons \emph{gravio}, il n’est pas bien certain qu’on puisse le rapporter à une racine germanique. Peut-être faut-il en chercher l’origine chez les Celtes ou chez les Slaves.} Choisi dans les races les plus anciennes et les plus nobles, dans celles qui pouvaient réclamer une origine divine, ce pendant exact du viçampati hindou exerçait une autorité des plus restreintes, sinon des plus précaires. Son action légale ressemblait fort à celle des chefs chez les Mèdes avant l’époque d’Astyage, ou à celle des rois hellènes dans les temps homériques. Sous l’empire de cette règle facile, chaque Arian, au sein de son odel, n’était guère plus lié à son voisin de même nation que ne le sont entre eux les différents États formant un gouvernement fédératif.\par
Une telle organisation, admissible en présence de populations numériquement faibles ou complètement subjuguées par la conscience de leur infériorité, n’était nullement compatible avec l’état de guerre, ni même avec l’état de conquête au milieu de masses résistantes. L’Arian, qui, dans son humeur aventureuse, vivait principale­ment dans l’une ou l’autre de ces situations difficiles, avait trop de bon sens pratique pour ne pas apercevoir le remède du mal et chercher les moyens d’en concilier l’application avec les idées d’indépendance personnelle qui, avant tout, lui tenaient à cœur. Il imagina donc qu’au moment d’entrer en campagne, des rapports tout particuliers, tout spéciaux, complètement étrangers à l’organisation régulière du corps politique, devaient intervenir entre le chef et les soldats ; voici comment le nouvel ordre de choses se fondait :\par
Un guerrier connu se présentait à l’assemblée générale, et se proposait lui-même pour commander l’expédition projetée. Quelquefois, surtout dans les cas d’agression, il en ouvrait même la première idée. En d’autres circonstances, il ne faisait que soumettre un plan qui lui était propre et qu’il appliquait à la situation. Ce candidat au comman­dement prenait soin d’appuyer ses prétentions sur ses exploits antérieurs, et de faire valoir son habileté éprouvée ; mais, sur toutes choses, le moyen de séduction qu’il pouvait employer avec le plus de bonheur, et qui lui assurait la préférence sur ses concurrents, c’était l’offre et la garantie, pour tous ceux qui viendraient combattre sous ses ordres, de leur assurer des avantages individuels dignes de tenter leur courage et leur convoitise. Il s’établissait ainsi un débat et une surenchère entre les candidats et les guerriers. Ce n’était que par conviction ou par séduction que ceux-ci pouvaient être amenés à s’engager avec l’entrepreneur d’exploits, de gloire et de butin.\par
On conçoit que beaucoup d’éloquence et un passé quelque peu digne d’estime étaient absolument nécessaires à ceux qui voulaient commander. On ne leur demandait pas, comme aux drottinns, comme aux graffs, la grandeur de la naissance ; mais ce qu’il leur fallait indispensablement, c’était du talent militaire, et plus encore une libéralité sans bornes envers le soldat. Sans quoi il n’y aurait eu à suivre leur drapeau que des dangers, sans espérance de victoire ni de rémunération.\par
Mais une fois que l’Arian s’était laissé persuader que l’homme qui le sollicitait avait bien toutes les qualités requises, et qu’après avoir fait ses conditions il s’était engagé avec lui, aussitôt un état tout nouveau intervenait entre eux \footnote{Le droit de l’homme libre de choisir son chef se conserva très longtemps dans les lois anglo-saxonnes. C’est ce que les commentateurs du \emph{Domesday-Book} appellent \emph{Commendatio.} (Palsgrave, \emph{Rise and Progress of the English Commonwealth}, t. I, p. 15.)}. L’Arian libre, l’Arian souverain absolu de son odel, abdiquant pour un temps donné l’usage de la plupart de ses prérogatives, devenait, sauf le respect des engagements réciproques, l’homme de son chef, dont l’autorité pouvait aller jusqu’à disposer de sa vie, s’il manquait aux devoirs qu’il avait contractés.\par
L’expédition commençait ; elle était heureuse. En principe, le butin appartenait tout entier au chef, mais avec l’obligation stricte et rigoureuse de le partager avec ses compagnons, non pas seulement dans la mesure des promesses échangées, mais, comme je viens de le dire, avec une prodigalité extrême. Manquer à cette loi eût été aussi dangereux qu’impolitique. Les chants scandinaves appellent avec intention le chef de guerre illustre « l’ennemi de l’or », parce qu’il n’en doit pas garder ; « l’hôte des héros », parce qu’il doit mettre son orgueil à les loger dans sa demeure, à les réunir à sa table, à leur prodiguer les longs banquets, les amusements de toute espèce et les riches présents. Ce sont là les moyens, et les seuls, de conserver leur amitié, de s’assurer leur appui, et partant de maintenir sa renommée avec sa puissance. Un chef avare et égoïste est aussitôt abandonné de tout le monde, et il rentre dans le néant \footnote{Il y a similitude parfaite entre les vertus que l’on exigeait d’un chef de guerre et l’idéal du chef de famille arian-hindou, comme le décrit le Ramayana : « Capi di famiglia que vissero « casti colle lor consorti, coloro che donarono con larghezze vacche oro, alimienti, e « terre, quelli che « diedero, altrui sicuranza e colore, che furon veridici. » Gorresio, \emph{ouvr. cité}, t. VI, p. 394.)}.\par
Je viens de montrer là quel emploi le général vainqueur pouvait faire du butin mobilier, de l’argent, des armes, des chevaux, des esclaves. Mais lorsque, avec ces avantages, il y avait encore prise de possession d’une contrée, le principe des générosités recevait nécessairement des applications différentes. En effet, le pays conquis prenait le nom de \emph{rik}, c’est-à-dire pays gouverné absolument, pays soumis ; titre que les territoires vraiment arians, les pays à odels, se faisaient un point d’honneur de repousser, se considérant comme essentiellement libres \footnote{La Norwège n’a jamais porté le titre de \emph{rik}, ni l’Islande non plus, tandis qu’il y avait eu le Gardarike et que toutes les conquêtes germaniques dans le reste de l’Europe portèrent cette dénomination. (Munch, \emph{ouvr. cité}, p. 112 et note 2.)}. Dans le rik, les populations vaincues étaient entièrement placées sous la main du chef de guerre \footnote{Savigny, D. \emph{Rœm. Recht im Mittelalter}, t. I, p. 229.}, qui se parait de la qualification de \emph{konungr}, titre militaire, gage d’une autorité qui n’appartenait ni au drottinn ni au graff, et dont les souverains de l’extrême Nord n’osèrent s’emparer que très tard, car ils gouvernaient des provinces qui, n’ayant pas été acquises par le glaive à leur couronne, ne leur donnaient pas le droit de le prendre.\par
Le \emph{konungr} donc, le \emph{könig} allemand, le \emph{king} anglo-saxon, le \emph{roi}, pour tout dire \footnote{Il ne faut cependant pas perdre de vue que ce \emph{roi} n’avait nullement la physionomie du roi celtique ou italiote, bien qu’il ressemblât un peu mieux au (mot grec) macédonien des époques antérieures à Alexandre. Un roi, dans le poème de \emph{Beowulf}, s’appelle : \emph{folces hyrde, pasteur du peuple}, comme dans l’Iliade. (Kemble, \emph{The anglo-saxon Poem} of \emph{Beowulf}, v. 1213, p. 44.) \emph{–} Le \emph{theodr} gothique et l’anglo-saxon \emph{theoden} signifient de même celui qui mène le peuple. Ce sont autant de titres militaires, plutôt qu’administratifs.}, dans son obligation étroite de faire participer ses hommes à tous les avantages qu’il recueillait lui-même, leur concédait des biens-fonds. Mais comme les guerriers ne pouvaient emporter avec eux ce genre de présents, ils n’en jouissaient qu’aussi longtemps qu’ils restaient fidèles à leur conducteur, et cette situation comportait pour leur qualité de propriétaires toute une série de devoirs étrangers à la constitution de l’odel.\par
Le domaine ainsi possédé à condition s’appelait \emph{feod.} Il offrait plus d’avantages que la première forme de tenure pour le développement de la puissance germanique, parce qu’il contraignait l’humeur indépendante de l’Arian à abandonner au pouvoir dirigeant une autorité plus grande. Il préparait ainsi l’avènement d’institutions propres à mettre en accord les droits du citoyen et ceux de l’État, sans détruire les uns au profit exclusif des autres. Les peuples sémitisés du midi n’avaient jamais eu la moindre idée d’une telle combinaison. puisqu’il était de règle chez eux que l’État devait absorber tous les droits.\par
L’institution du féod amenait aussi des résultats latéraux qui méritent d’être enregistrés. Le roi qui le concédait, comme le guerrier qui le recevait, étaient égale­ment intéressés à n’en pas laisser péricliter la valeur vénale. Aux yeux du premier, c’était un don temporaire, qui pouvait rentrer dans ses mains au cas où l’usufruitier viendrait à mourir ou romprait son engagement pour aller chercher aventure sous un autre chef, circonstance assez commune. Dans cette prévision, il fallait que le domaine restât digne de servir d’appât à un remplaçant. Pour le second, posséder une terre n’était un avantage qu’autant que cette terre fructifiait ; et comme il n’avait ni le goût ni le temps de s’occuper par lui-même de la culture du sol, il ne manquait jamais de traiter, sous la garantie de son chef, avec les anciens propriétaires, auxquels il abandonnait l’entière et paisible possession d’une part, en leur donnant le reste à ferme. C’était une sage opération que les Doriens et les Thessaliens avaient très bien pratiquée jadis. Il en résulta que les conquêtes germaniques, malgré les excès des premiers moments, probablement un peu exagérés d’ailleurs par l’éloquente lâcheté des écri­vains de l’histoire Auguste, furent, en définitive, assez douces, médiocrement redoutées des peuples et, sans nulle comparaison, infiniment plus intelligentes, plus humaines et moins ruineuses que les colonisations brutales des légionnaires et l’administration féroce des proconsuls au temps où la politique romaine était dans toute la fleur de la civilisation \footnote{En thèse générale, les prétentions des Germains, arrivés dans les contrées de domination romaine, se bornèrent à prendre un tiers des terres. (Savigny, D. \emph{Rœm, Recht im Mittelalter}, t. I, p. 289.) – Les Burgondes furent des plus durs. Ils voulurent avoir la moitié de la maison et du jardin, les deux tiers de la terre cultivable, un tiers des esclaves ; les forêts restèrent en commun. Le Romain fut qualifié \emph{hospes} du Burgonde. Tout guerrier doté ailleurs par le roi dut abandonner à son \emph{hôte} la terre à laquelle il avait droit, et, s’il voulait vendre ce qui lui appartenait du fonds, \emph{l’hôte} était le premier acquéreur légal. (\emph{Ibid}., p. 254 et seqq.)}.\par
Il semblerait que le féod, récompense des travaux de la guerre, preuve éclatante d’un courage heureux, ait eu tout ce qu’il fallait pour se concilier les faveurs de l’opinion chez des races belliqueuses et fort sensibles au gain ; il n’en était cependant pas ainsi. Le service militaire à la solde d’un chef répugnait à beaucoup d’hommes, et surtout à ceux de haute naissance. Ces esprits arrogants trouvaient de l’humiliation à recevoir des dons de la main de leurs égaux, et quelquefois même de ceux qu’ils considéraient comme leurs inférieurs en pureté d’origine. Tous les profits imaginables ne les aveuglaient pas non plus sur l’inconvénient de laisser suspendre pour un temps, sinon de perdre pour toujours, l’action plénière de leur indépendance. Quand ils n’étaient pas appelés à commander eux-mêmes, par une incapacité d’une nature quelconque, ils préféraient ne prendre part qu’aux expéditions vraiment nationales ou à celles qu’ils se sentaient en état d’entreprendre avec les seules forces de leur odel.\par
Il est assez curieux de voir ce sentiment devancer l’arrêt sévère d’un savant historien qui, dans sa haine sentie envers les races germaniques, se fonde principale­ment sur les conditions du service militaire, et s’en autorise pour refuser aux Goths d’Hermantik, comme au Franks des premiers Mêrowings, toute notion véritable de liberté politique. Mais il ne l’est pas moins assurément de voir les Anglo-Saxons d’aujourd’hui, ce dernier rameau, bien défiguré il est vrai, mais encore ressemblant quelque peu aux antiques guerriers germains, les habitants indisciplinés du Kentucky et de l’Alabama, braver tout à la fois le verdict de leurs plus fiers aïeux et celui du savant éditeur du Polyptique d’Irminon. Sans croire porter la moindre atteinte à leurs principes de sauvage républicanisme, ils s’engagent en foule à la solde des pionniers qui s’offrent à leur faire tenter la fortune au milieu des indigènes du nouveau monde et dans les prairies les plus dangereuses de l’Ouest \footnote{L’homme qui prend à son service plusieurs chasseurs, laboureurs ou commis, et les mène dans les déserts, est appelé par eux du titre militaire de \emph{captain}, bien que ce soit, au fond, un marchand ou un défricheur de forêts.}. C’est là certainement de quoi répondre, d’une manière suffisante, aux exagérations anciennes et modernes.\par
Possesseur d’un odel, ou jouissant d’un féod, l’Arian Germain se montre à nous également étranger au sens municipal du Slave, du Celte et du Romain. La haute idée de sa valeur personnelle, le goût d’isolement qui en est la suite, dominent absolument sa pensée et inspirent ses institutions. L’esprit d’association ne saurait donc lui être familier. Il sait y échapper jusque dans la vie militaire ; car chez lui cette organisation n’est que l’effet d’un contrat passé entre chaque soldat et le général, abstraction faite des autres membres de l’armée. Très avare de ses droits et de ses prérogatives, il n’en fait jamais l’abandon, non pas même de la moindre parcelle ; et s’il consent à en restreindre, à en suspendre l’usage, c’est qu’il trouve dans cette concession temporaire un avantage direct, actuel et bien évident. Il a les yeux grands ouverts sur ses intérêts. Enfin, perpétuellement préoccupé de sa personnalité et de ce qui s’y rapporte d’une façon directe, il n’est pas matériellement patriote, et n’éprouve pas la passion du ciel, du sol, du lieu où il est né. Il s’attache aux êtres qu’il a toujours connus, et le fait avec amour et fidélité ; mais aux choses, point, et il change de province et de climat sans difficulté. C’est là une des clefs du caractère chevaleresque au moyen âge et le motif de l’indifférence avec laquelle l’Anglo-Saxon d’Amérique, tout en aimant son pays, quitte aisément sa contrée natale, et, de même, vend ou échange le terrain qu’il a reçu de son père.\par
Indifférent pour le génie des lieux, l’Arian Germain l’est aussi pour les nationalités, et ne leur porte d’amour ou de haine que suivant les rapports que ces milieux inévitables entretiennent avec sa propre personne. Il considère de prime abord tous les étrangers, fussent-ils de son peuple, sous un jour à peu près égal, et la supériorité qu’il s’arroge mise à part, une certaine partialité pour ses congénères également exceptée, il est assez libre de préjugés natifs contre ceux qui l’abordent, de quelque contrée éloignée qu’ils puissent venir ; de telle sorte que, s’il leur est donné de faire éclater à ses yeux des mérites réels, il ne refusera pas d’en reconnaître les bienfaits. De là vient que, dans la pratique, il accorda de très bonne heure aux Kymris et aux Slaves qui l’entouraient une estime proportionnée à ce qu’ils pouvaient lui montrer de vertus guerrières ou de talents domestiques. Dès les premiers jours de ses conquêtes, l’Arian mena à la guerre les serviteurs de son odel, et encore plus volontiers les hommes de son féod. Tandis qu’il était, lui, le compagnon gagé du chef de guerre, cette suite de rang inférieur combattait sous sa conduite et prenait part à tous ses profits. Il lui permit de recueillir de l’honneur, et reconnut cet honneur noblement quand il fut bien acquis ; il avoua l’illustration là où elle se trouva ; il fit mieux : il laissa son vaincu devenir riche, et l’achemina ainsi, pour toutes ces causes, à un résultat qui ne pouvait manquer d’arriver et qui arriva, que ce vaincu devint avec le temps son égal. Dès avant les invasions du V\textsuperscript{e} siècle, ces grands principes et toutes leurs conséquences avaient agi et porté leurs fruits \footnote{Voir plus haut. – Je renvoie à ce passage, où j’ai indiqué la double loi d’attraction et de répulsion qui préside aux mélanges ethniques, et qui est, dans sa première partie tout à la fois l’indice de l’aptitude à la civilisation chez une race et l’agent de sa décadence.}. On va en voir la démonstration.\par
Les nations germaniques ne s’étaient, dans l’origine, composées que de Roxolans, que d’Arians ; mais au temps où elles habitaient encore, à peu près compactes, la péninsule scandinave, la guerre avait déjà réuni dans les odels trois classes de personnes : les Arians proprement dits, ou les \emph{Jarls} : c’étaient les maîtres \footnote{\emph{Rigsmal}, st. 23-31.} ; les karls, agriculteurs, paysans domiciliés, tenanciers du jarl, hommes de famille blanche métisse, Slaves, Celtes ou Jotuns \footnote{\emph{Ibid.}, st. 14-18.} ; puis les \emph{traëlls}, les esclaves, race basanée et difforme, dans laquelle il est impossible de ne pas reconnaître les Finnois \footnote{\emph{Ibid}, st. 2-7.}.\par
Ces trois classes, formées aussi spontanément, aussi nécessairement dans les États germains que chez le anciens Hellènes, composèrent d’abord la société tout entière ; mais les mélanges, promptement opérés, firent naître des hybrides nombreux ; la liberté que les mœurs germaniques donnaient aux karls de marcher à la guerre, et, par suite, de s’enrichir, profita aux métis que cette classe de paysans avait produits en s’alliant à la classe dominatrice ; et tandis que la race pure, exposée surtout aux hasards des batailles, tendait à diminuer de nombre dans la plupart des tribus, et à se limiter aux familles qu’on nommait divines, et parmi lesquelles l’usage permettait seul de choisir les drottinns et les graffs, les demi-Germains voyaient sortir de leurs rangs d’innombrables chefs riches, vaillants, éloquents, populaires, et qui, libres de proposer à leurs concitoyens des plans d’expéditions et des projets d’aventures, ne trouvaient pas moins de compagnons prêts à les écouter que le pouvaient des héros d’une extraction plus noble. Il en advint des résultats de toute espèce, les plus divergents, les plus disparates, mais tous également faciles à comprendre. Dans certaines contrées, où la pureté de descendance, toujours estimée, était devenue extrêmement rare, le titre de jarl prit une valeur énorme, et finit par se confondre avec celui de konungr ou de roi ; mais là encore ce dernier fut rapidement égalé par les qualifications, d’abord fort modestes, de \emph{fylkir} et de \emph{hersir}, qui n’avaient été portées au début que par des capitaines d’un rang inférieur. Ce mode de confusion eut lieu en Scandinavie, et à l’ombre du gouvernement vraiment régulier, suivant le sens de la race, des anciens drottinns. Là, sur ce terrain, essentiellement arian, les jarls, les konungrs, les fylkirs, les hersirs n’étaient en fait que des héros sans emplois et, comme on dirait dans notre langue administrative, des généraux en disponibilité. Tout ce que le sentiment public pouvait leur accorder, c’était une part égale du respect qu’obtenait la noblesse du sang, bien qu’ils ne l’eussent pas tous ; mais on n’était nullement tenté de leur donner un commandement sur la population. Aussi fut-il très difficile à la monarchie militaire, qui est la monarchie moderne, issue des chefs de guerre germaniques, de s’établir dans les pays scandinaves. Elle n’y parvint qu’à force de temps et de luttes, et après avoir éliminé la foule des rois, au sein de laquelle elle était comme noyée, rois de terre, rois de mer, rois des bandes.\par
Les choses se passèrent tout autrement dans les pays de conquête, comme la Gaule et l’Italie. La qualité de jarl ou d’\emph{ariman}, ce qui est tout un, n’étant plus soutenue là par les formes libres du gouvernement national, ni rehaussée par la possession de l’odel, fut rapidement abaissée sous le fait de la royauté militaire, qui gouvernait les populations vaincues et commandait aux Arians vainqueurs. Donc, le titre d’ariman \footnote{Chez tes Anglo-Saxons, on disait \emph{sokeman.} (Palsgrave, ouvr, \emph{cité}, t. I, p, 15.)} au lieu d’augmenter d’importance comme en Scandinavie, s’abaissa, et ne s’appliqua bientôt plus qu’aux guerriers de naissance libre, mais d’un rang inférieur, les rois s’étant entourés d’une façon plus immédiate de leurs plus puissants compagnons, des hommes formant ce qu’ils nommaient leur \emph{truste}, de leurs \emph{fidèles}, tous gens qui, sous le nom de \emph{leudes}, ou possesseurs d’odels, domaines fictivement constitués suivant l’ancienne forme par la volonté du souverain, représentaient seuls et exclusivement la haute noblesse. Chez les Franks, les Burgondes, les Longobards, l’ariman, ou, suivant la traduction latine, le \emph{bonus homo}, en arriva à ne plus être qu’un simple propriétaire rural ; et pour empêcher le seigneur du fief de réduire en servage le représentant légal, mais non plus ethnique, des anciens Arians, il fallut l’autorité de plus d’un concile, qui d’ailleurs ne prévalut pas toujours contre la force des circonstances.\par
En somme, dans toutes les contrées originairement germaniques, comme dans celles qui ne le devinrent que par conquête, les principes des dominateurs furent identiquement les mêmes, et d’une extrême générosité pour les races vaincues.\par
En dehors de ce qu’on peut appeler les crimes sociaux, les crimes d’État, comme la trahison et la lâcheté devant l’ennemi, la législation germanique nous paraîtrait aujourd’hui indulgente et douce jusqu’à la faiblesse. Elle ne connaissait pas la peine de mort \footnote{Même pour le meurtre du roi, chez les Anglo-Saxons, la composition en argent était admise. On s’était contenté de la porter au plus haut degré. (Kemble, t. I, p. 123.) – Cependant les souverains de cette branche germanique s’étaient arrangés de façon à réunir sur leur tête au titre de \emph{theedr}, ou chef militaire, celui de \emph{dryht}, ou magistrat civil, ce que ne firent pas les chefs des Goths ni des Franks. (\emph{Ibid.}, t. II, p. 23.)} et pour les crimes de meurtre n’appliquait que la composition pécuniaire. C’était assurément une mansuétude bien remarquable, chez des hommes d’une aussi excessive énergie et dont les passions étaient assurément fort ardentes. On les en a loués, on les en a blâmés ; mais on a peut-être examiné la question un peu superficiellement. Pour asseoir avec pleine connaissance de cause une opinion définitive, il faut distinguer ici entre la justice rendue sous l’autorité ou plutôt sous la direction du drottinn, et plus tard, par assimilation, du konungr, ou roi militaire et celle qui, s’exerçant dans les odels, émanait, d’une manière bien autrement puissante et tout incontestée, de la volonté absolue et de l’initiative de l’Arian, chef de famille. Cette distinction est non seulement dans la nature des choses, mais nécessaire pour comprendre la théorie génératrice de la composition en argent dans les jugements criminels.\par
Le possesseur de l’odel, maître suprême de tous les habitants de sa terre et leur juge sans appel, suivait certainement dans ses arrêts les suggestions d’un esprit nativement rigide et porté à la doctrine du talion, cette loi la plus naturelle de toutes, et dont une sagesse très raffinée, appuyée sur l’expérience de cas très complexes, apprend seule à reconnaître l’injustice. Pas de doute que dans ce cercle de juridiction domestique on ne demandât œil pour œil et dent pour dent. Il n’y aurait pas même eu moyen de recourir à la composition pécuniaire, car rien n’établit que les membres inférieurs de l’odel aient eu le droit personnel de propriété dans les époques vraiment arianes.\par
Mais quand le crime, se produisant en dehors du cercle intérieur gouverné par le chef de famille, avait pour victime un homme libre, la répression se compliquait soudain de ces difficultés dirimantes qui hérissent toujours le redressement des torts d’un souverain envers son égal. On admettait bien en principe, dans l’intérêt évident du lien social, que la communauté, représentée par l’assemblée des hommes libres sous la présidence du drottinn ou du graff, avait le droit de punir les infractions à la tranquillité publique, état que ces pouvoirs avaient la mission de maintenir de leur mieux. Le point scabreux était de fixer l’étendue de ce droit. Il se trouvait, pour le circonscrire dans les plus étroites limites possibles, autant de volontés qu’il y avait de juges impartiaux, c’est-à-dire d’Arians Germains, attentifs à sauvegarder l’indépen­dance de chacun contre les empiétements éventuels de la communauté. On fut ainsi conduit à envisager sous un jour de compromis la position des coupables et à substi­tuer, dans le plus grand nombre de cas, à l’idée du châtiment celle de la réparation approximative. Placée sur ce terrain, la loi considéra le meurtre comme un fait accompli, sur lequel il n’y avait plus à revenir, et dont elle devait seulement borner les conséquences quant à la famille du mort. Elle écarta à peu près toute tendance à la vindicte, évalua matériellement le dommage, et, moyennant ce qu’elle jugea être un équivalent pour la perte de l’homme que l’action homicide avait rayé du nombre des vivants et arraché à ceux parmi lesquels il vivait, elle ordonna le pardon, l’oubli et le retour de la paix. Dans ce système, plus le défunt était d’un rang élevé, plus la perte était estimée considérable. Le chef de guerre valait plus que le simple guerrier, celui-ci plus que le laboureur, et certainement un Germain devait être mis à plus haut prix qu’un de ses vaincus.\par
Avec le temps, cette doctrine, pratiquée dans les camps comme dans les territoires scandinaves, devint la base de toutes les législations germaniques, bien qu’elle ne fût à l’origine qu’un résultat de l’impuissance de la loi à atteindre ceux qui faisaient la loi. Elle étouffa la coutume des odels à mesure que ceux-ci diminuèrent de nombre et virent ensuite restreindre leurs privilèges, à mesure que l’indépendance des membres de la nation fut moins absolue, que, le féod étant devenu le mode de tenure le plus ordinaire, les rois prirent plus d’empire, et enfin que les multitudes agrégées par la conquête et reconnues comme propriétaires du sol devinrent aptes à composer pour leurs délits et leurs crimes, comme les plus nobles personnages, comme les hommes de la plus haute lignée pour les leurs.\par
 L’Arian Germain n’habitait pas les villes ; il en détestait le séjour, et, par suite, en estimait peu les habitants. Toutefois il ne détruisait pas celles dont la victoire le rendait maître, et, au II\textsuperscript{e} siècle de notre ère, Ptolémée énumérait encore quatre-vingt-quatorze cités principales entre le Rhin et la Baltique, fondations antiques des Galls ou des Slaves, et encore occupées par eux \footnote{H. Leo, \emph{Vorlesungen über die Geschichte des deutschen Volkes und Reiches}, in-8°, Halle, 1854, t. I, p. 194.} À la vérité, sous le régime des conquérants venus du nord, ces villes entrèrent dans une période de décadence. Créées par la culture imparfaite de deux peuples métis, assez étroitement utilitaires, elles succombèrent à deux effets tout-puissants, bien qu’indirects, de la conquête qu’elles avaient subie. Les Germains, en attirant la jeunesse indigène à l’adoption de leurs mœurs, en conviant les guerriers du pays à prendre part à leurs expéditions, partant à leurs honneurs et à leur butin, firent goûter promptement leur genre de vie à la noblesse celtique. Celle-ci tendit à se mêler étroitement à eux. Quant à la classe commerçante, quant aux indus­triels, plus casaniers, l’imperfection de leurs produits ne pouvait que difficilement soutenir la concurrence contre ceux des fabricants de Rome, qui, établis de très bonne heure sur les limites décumates, livraient aux Germains des marchandises italiennes ou grecques beaucoup moins chères, ou du moins infiniment plus belles et meilleures que les leurs. C’est le double et constant privilège d’une civilisation avancée. Réduits à copier les modèles romains pour se prêter aux goûts de leurs maîtres, les ouvriers du pays ne pouvaient espérer un véritable profit de ce labeur qu’en se mettant directement au service des possesseurs d’odels et de féods, ceux-ci ayant une tendance naturelle à réunir dans leur clientèle immédiate et sous leur main tous les hommes qui pouvaient leur être de quelque utilité. C’est ainsi que les villes se dépeuplèrent peu à peu et devinrent d’obscures bourgades.\par
Tacite, qui ne veut absolument voir dans les héros de son pamphlet que d’esti­mables sauvages, a faussé tout ce qu’il raconte d’eux en matière de civilisation \footnote{Entre autres assertions contestables, on remarque celle-ci : « Litterarum secreta viri pariter ac fœminæ ignorant. » (\emph{Germ.}, 18.) \emph{–} On ne peut expliquer ce passage qu’en l’appliquant seulement à quelques tribus très mélangées et exceptionnellement pauvres. – Tous les mots qui se rapportent à l’écriture sont gothiques, et, si l’allemand moderne a emprunté au latin l’expression \emph{schreiben}, écrire, c’est que les Allemands ne sont pas d’essence germanique. – On trouve dans Ulfila \emph{spilda, planchette} pour tracer les caractères runiques ; \emph{vrits}, une \emph{fente}, une lettre formée par incision ; \emph{mêljan, gamêljan, écrire, peindre ; bôka}, un \emph{livre} formé d’écorce de hêtre, etc. (W. C. Grimm, \emph{Uber deutsche Runen}, p. 47.)} Il les représente comme des bandits philosophes. Mais, sans compter qu’il se contredit lui-même assez souvent, et que d’autres témoignages contemporains, d’une valeur au moins égale au sien, permettent de rétablir la vérité des faits, il ne faut que contempler le résultat des fouilles opérées dans les plus anciens tombeaux du Nord pour se convaincre que, malgré les emphatiques déclamations du gendre d’Agrippa, les Germains, ces héros qu’il célèbre d’ailleurs avec raison, n’étaient ni pauvres, ni ignorants, ni barbares \footnote{Ils avaient eu leur période de bronze avant d’arriver dans le Nord, et probablement avant de conquérir le Gardarike. (Munch, \emph{ouvr. cité}, p. 7.)\emph{ –} Toutes les antiquités de cet âge trouvées en Danemark sont celtiques. (\emph{Ibidem. –} Wormsaæ, \emph{Lettre à M. Mérimée, Moniteur universel} du 14 avril 1853.) – D’ailleurs, si les Germains avaient assez de goût pour apprécier les produits des arts, il est certain qu’ils n’avaient pas eux-mêmes, eux si richement doués sous le rapport de la poésie, l’inspiration des oeuvres plastiques. M. Wormsaæ a dit avec raison : « On remarquera que l’influence des arts de Rome est évidente « pour l’observateur attentif qui examine nos antiquités de l’âge de fer. Dès avant les « grandes expéditions normanniques, les Scandinaves imitaient des modèles romains, tout « en donnant par la fabrication un cachet particulier à leurs armes et à leurs bijoux. » – Il est inutile de répéter ici que les races les mieux douées ne deviennent artistes que par un contact quelconque avec l’essence mélanienne ; les Scandinaves ne l’avaient pas eu.}\par
La maison de l’odel ne ressemblait pas aux sordides demeures, à demi enfouies dans la terre, que l’auteur de la \emph{Germania} se plaît tant à décrire sous des couleurs stoï­ques. Cependant ces tristes retraites existaient ; mais c’était l’abri des races celtiques à peine germanisées ou des paysans, des karls, cultivateurs du domaine. On peut encore contempler leurs analogues dans certaines parties de l’Allemagne méridionale, et surtout dans le pays d’Appenzell, où les gens prétendent que leur mode de construction traditionnel est particulièrement propre à les préserver des rigueurs de l’hiver. C’était la raison qu’alléguaient déjà les anciens constructeurs ; mais les hommes libres, les guerriers arians étaient mieux logés, et surtout moins à l’étroit \footnote{On peut trouver sans peine la mention d’un certain nombre de palais ou châteaux germaniques dans les auteurs latins. – Le \emph{Scopes-Vidsidh} nomme encore \emph{Heorot}, dans le pays des Hadubards (Ettmuller, \emph{Beowulfied}, Eprileit, p. XXXIX) ; puis \emph{Hreosnabeorh}, dans le pays des Géates ; \emph{Finnesburh}, chez les Frisons ; \emph{Headhoraemens} et \emph{Hrones-næs}, en Suède. – Le poème de \emph{Beowulf} cite également toutes ces résidences.}\par
Lorsqu’on entrait dans leur résidence, on se trouvait d’abord dans une vaste cour, entourée de divers bâtiments, consacrés à tous les emplois de la vie agricole, étables, buanderies, forges, ateliers et dépendances de toute espèce, le tout plus ou moins considérable, suivant la fortune du maître. Cette réunion de bâtisses était entourée et défendue par une forte palissade. Au centre s’élevait le palais, l’odel proprement dit, que soutenaient et ornaient en même temps de fortes colonnes de bois, peintes de couleurs variées. Le toit, bordé de frises sculptées, dorées ou garnies de métal brillant, était d’ordinaire surmonté d’une image consacrée, d’un symbole religieux, comme, par exemple, le sanglier mystique de Freya \footnote{Tacite (\emph{Germ}., 45) parle de ce sanglier ; l’Edda de même, dans le \emph{Hyndluliodh}, st. 5. – On appelait cette figure emblématique \emph{hildisvin} ou \emph{hildigœltr, le porc des combats.} (Ettmuller, \emph{ouvr. cité, introd.}, p. 49.) \emph{–} Charlemagne avait fait mettre un aigle sur le faîte de son palais impérial d’Aix-la-Chapelle.} La plus grande partie de ce palais était occupée par une vaste salle, ornée de trophées et dont une table immense occupait le milieu.\par
C’était là que l’Arian Germain recevait ses hôtes, rassemblait sa famille, rendait la justice, sacrifiait aux dieux, donnait ses festins, tenait conseil avec ses hommes et leur distribuait ses présents. Quand, la nuit venue, il se retirait dans les appartements intérieurs, c’était là que ses compagnons, ranimant la flamme du foyer, se couchaient sur les bancs qui entouraient les murailles, et s’endormaient la tête appuyée sur leurs boucliers \footnote{Weinhold, \emph{Die deutsche Frauen in ; Mittelalt.}, p. 348-349.}.\par
On est sans doute frappé par la ressemblance de cette demeure somptueuse, de ses grandes colonnes, de ses toits élevés et ornés, de ses larges dimensions, avec les palais décrits dans l’Odyssée et les résidences royales des Mèdes et des Perses. En effet, les nobles manoirs des Achéménides étaient toujours situés en dehors des villes de l’Iran et composés d’un groupe de bâtiments affectés aux mêmes usages que les dépendances des palais germaniques. On y logeait également tous les ouvriers ruraux du domaine, une foule d’artisans, selliers, tisserands, forgerons, orfèvres, et jusqu’à des poètes, des médecins et des astrologues. Ainsi, les châteaux des Arians Germains décrits par Tacite, ceux dont les poèmes teutoniques parlent avec tant de détails, et, plus anciennement encore, la divine Asgard des bords de la Dwina, étaient l’image de l’iranienne Pasagard, au moins dans les formes générales, sinon dans la perfection de l’œuvre artistique \footnote{On a, dans les descriptions qui nous restent d’Ecbatane et de son palais, l’exacte reproduction d’une demeure ariane de l’extrême nord de l’Europe au V\textsuperscript{e} siècle. Rien ne manque au portrait : l’édifice médique était de bois, formé de grandes salles reposant sur des piliers peints de couleurs variées ; il n’y manque pas même les frises de métal au sommet des murs, ni les plaques argentées et dorées pour former la toiture. Ce genre de construction, opposé à celui de Persépolis et des villes de l’époque sassanide qui sont l’un et l’autre, des imitations assyriennes, est essentiellement arian. (Polybe, X, 24, 27.)\emph{ –} Cet auteur était tellement ébloui de la splendeur, de la richesse et de l’étendue (sept stades de tour) du palais d’Ecbatane, qu’il proteste d’avance contre ce que son récit peut avoir de semblable au fabuleux.}, ni dans la valeur des matériaux \footnote{Le palais d’Ecbatane était entièrement construit en bois de cyprès et de cèdre, et toutes les chambres étaient peintes, dorées et argentées. (Polybe, \emph{loc. cit.})\emph{ –} Ritter fait la remarque très juste que les palais persans de l’époque moderne se rapprochent beaucoup de ce style (\emph{West-Asien}, t. VI, 2\textsuperscript{e} Abth., p. 108.) J’ajouterai les palais chinois.}. Et après tant de siècles écoulés depuis que l’Arian Roxolan avait perdu de vue les frères qu’il avait quittés dans la Bactriane et peut-être même beaucoup plus haut dans le nord, après tant de siècles de voyages poursuivis par lui à travers tant de contrées, et, ce qui est plus remarquable encore, après tant d’années passées à n’avoir, dit-on, pour abri que le toit de son chariot, il avait si fidèlement conservé les instincts et les notions primitives de la culture propre à sa race, que l’on vit se mirer dans les eaux du Sund, et plus tard dans celles de la Somme, de la Meuse et de la Marne, des monuments construits d’après les mêmes données et pour les mêmes mœurs que ceux dont la Caspienne et même l’Euphrate avaient reflété les magnificences \footnote{ \noindent Cette réunion de bâtiments agglomérés, que nous ne savons, dans notre langage romano-celtique, autrement nommer que du mot \emph{ferme}, et qui éveille ainsi pour nous une idée fausse, est ce que les Allemands nomment très justement \emph{bof}. Cette expression s’applique à toute résidence patrimoniale héréditaire, à celle des rois comme à celle des nobles et même des paysans. C’est exactement le mot persan (mot persan) \emph{ivan}, qui se rapporte à la même racine et présente absolument le même sens partout où Firdousi l’emploie, comme, par exemple, dans ce vers :\par
 (vers persan)\par
 « Vous êtes en sûreté dans mon \emph{ivan}. »\\
 \noindent Du reste, le poème de Firdousi, à part le placage musulman, et dans ses éléments primitifs, peut être considéré, pour les mœurs, les caractères, les actions qu’il célèbre comme étant par excellence un poème germanique.
}.\par
Quand l’Arian Germain se tenait dans sa grand’salle, assis sur un siège élevé, au haut bout de la table, vêtu de riches habits, les flancs ceints d’une épée précieuse, forgée par les mains habiles et estimées magiques des ouvriers jotuns, slaves ou finnois, et qu’entouré de ses braves, il les conviait à se réjouir avec lui, au bruit des coupes et des cornes à boire, garnies d’argent ou dorées sur les bords, ni des esclaves, ni même des domestiques vulgaires, n’étaient admis à l’honneur de servir cette vaillante assemblée. De telles fonctions semblaient trop nobles et trop relevées pour être abandonnées à des mains si humbles ; et de même qu’Achille s’occupait lui-même du repas de ses hôtes, de même les héros germaniques se faisaient un honneur de conserver cette lointaine tradition de la courtoisie particulière à leur famille. Le glaive au côté, ils allaient quérir, ils plaçaient sur les tables les viandes, la bière, l’hydromel ; ensuite ils s’asseyaient librement, et parlaient sans crainte, suivant que leur pensée les inspirait.\par
Ils n’étaient pas tous sur le même pied dans la maison. Le maître estimait avant tous les autres son orateur, son porte-glaive, son écuyer, et, lorsqu’il était jeune encore, son père nourricier, celui qui lui avait appris le maniement des armes et l’avait préparé à l’expérience du commerce des hommes. Ces divers personnages, et le dernier surtout, avaient la préséance parmi leurs compagnons. On accordait aussi des égards particuliers au champion d’élite qui avait accompli des exploits hors ligne.\par
Le festin était commencé. La première faim s’apaisait ; les coupes se vidaient rapidement, la parole et la joie circulaient comme du feu dans toutes ces têtes violen­tes. Les actions de guerre racontées de toutes parts enflammaient ces imaginations combustibles et multipliaient les bravades. Tout à coup un convive se levait bruyamment ; il annonçait la volonté d’entreprendre telle expédition hasardeuse, et, la main étendue sur la corne qui contenait la bière, il jurait de réussir ou de tomber. Des applaudissements terribles éclataient de toutes parts. Les assistants, exaltés jusqu’à la folie, entre-choquaient leurs armes pour mieux célébrer leur allégresse ; ils entouraient le héros, le félicitaient, l’embrassaient. C’étaient là des délassements de lions.\par
Passant alors à d’autres idées, ils se mettaient au jeu, passion dominante et profonde chez des esprits amoureux d’aventures, avides de hasards, qui, dans leur façon de s’abandonner, sans réserve et sans mesure, à toutes les formes du danger, en arrivaient souvent à se jouer eux-mêmes et à affronter l’esclavage, plus redoutable dans leurs idées que la mort même. On conçoit que de longues séances ainsi employées pouvaient faire naître d’épouvantables orages, et il était des moments où le seigneur du lieu devait tenir à en écarter même l’occasion. Prenant donc ces imagina­tions actives par un de leurs côtés les plus accessibles, il avait recours aux récits des voyageurs, toujours écoutés avec une attention également vive et intelligente ; ou bien encore il proposait des énigmes, amusement favori \footnote{Ce goût des énigmes est un des traits principaux de la race ariane, et, comme il a été remarqué déjà ailleurs, il s’unit au personnage mystérieux du sphinx ou griffon, dont la patrie primitive est incontestablement l’Asie centrale ; c’est de là qu’il est descendu sur le Cythéron avec les Hellènes, après avoir habité le Bolor avec les Iraniens, qui l’appelèrent \emph{Simourgh}. Les énigmes font partie du génie national des Scythes et des Massagètes dans Hérodote, et c’est de là qu’elles ont continué à vivre dans les préoccupations du génie germanique.} ; ou enfin, profitant de l’influence incalculable dont jouissait la poésie, il ordonnait à son poète de remplir son office.\par
Les chants germaniques avaient, sous leurs formes ornées, le caractère et la portée de l’histoire, mais de l’histoire passionnée, préoccupée surtout de maintenir éternel­lement l’orgueil des journées de gloire, et de ne pas laisser périr la mémoire des outrages et le désir de les venger \footnote{Tac.,\emph{ Germ.}, 2. – W. Muller, \emph{ouvr. cité}, p. 297.} Elle proposait aussi les grands exemples des aïeux. On y trouve peu de traces de lyrisme. C’étaient des poèmes à la manière des compilations homériques, et, j’ose même le dire, les fragments mutilés qui en sont venus jusqu’à nous respirent une telle grandeur avec un tel enthousiasme, sont revêtus d’une si curieuse habileté de formes, que sous quelques rapports ils méritent presque d’être comparés aux chefs-d’œuvre du chantre d’Ulysse. La rime y est inconnue ; ils sont rythmés et allitérés \footnote{Wackernagel, \emph{Geschichte, d. d. Litteratur}, p. 8 et seqq. – L’allitération cesse d’être en usage en Allemagne au IX\textsuperscript{e} siècle. On la trouve dans les généalogies gothiques, vandales, burgondes, longobardes, frankes, anglo-saxonnes, dans les anciennes formules juridiques, dans quelques recettes d’incantation. C’est un mode d’harmonie poétique on ne peut plus ancien chez la race blanche ; les noms des trois éponymes Ingœvo, Irmino et Istæwo, cités par Tacite, sont allitérés. Il ne serait pas impossible d’en trouver des vestiges dans les généalogies bibliques.}. L’ancienneté de ce système de versification est incontes­table. Peut-être en pourrait-on retrouver des traces aux époques les plus primitives de la race blanche.\par
Ces poèmes, qui conservaient les traits mémorables des annales de chaque nation germanique, les exploits des grandes familles, les expéditions de leurs braves, leurs voyages et leurs découvertes sur terre et sur mer \footnote{Les Goths avaient des poèmes qui chantaient leur premier départ de l’île de Scanzia et les hauts faits des ancêtres de leurs chefs, les annales Ethrpamara, Hanala, Fridigern, Vidicula ou Vidicoja. (W. Muller\emph{, ouvr. cité}, p. 297.)}, tout enfin ce qui était digne d’être chanté, n’étaient pas seulement écoutés dans le cercle de l’odel, ni même de la tribu où ils avaient pris naissance et qu’ils célébraient Suivant qu’ils avaient un mérite supérieur, ils circulaient de peuple à peuple passant des forêts de la Norwège aux marais du Danube, apprenant aux Frisons, aux riverains du Weser les triomphes obtenus par les Amalungs sur les bords des fleuves de la Russie, et répandant chez les Bavarois et les Saxons les faits d’armes du Longobard Alboin dans les régions lointaines de l’Italie \footnote{M. Amédée Thierry a éloquemment et exactement décrit cette ubiquité des poèmes germaniques et, par suite, des grandes actions qui y étaient consacrées. (Revue des Deux-Mondes, 1\textsuperscript{er} déc. 1852, p. 844-845, 883. – Munch, ouvr. cité, p. 43-44.)}. L’intérêt que l’Arian Germain prenait à ces productions était tel, que souvent une nation demandait à une autre de lui prêter ses poètes et lui envoyait les siens. L’opinion voulait même rigoureusement qu’un jarl, un ariman, un véritable guerrier, ne se bornât pas à connaître le maniement des armes, du cheval et du gouvernail, l’art de la guerre, de toutes les sciences assurément les premières \footnote{La tactique germanique avait pour principe le coin ; on en attribuait l’invention à Odin. (W. Muller, Altdeutsche Religion, p. 197.)} ; il fallait encore qu’il eût appris par cœur et fût en état de réciter les compositions qui intéressaient sa race ou qui de son temps avaient le plus de célébrité. Il devait de plus être habile à lire les runes, à les écrire et à expliquer les secrets qu’elles renfermaient \footnote{Rigsmal, st. 39-42 : « Alors les fils du jarl grandirent ; ils domptèrent des étalons, « peignirent des boucliers, aiguisèrent des flèches, taillèrent des bois de lance. Korner, le « cadet, sut lire les runes, comprit les alphabets et les caractères divinatoires. Il apprit par là à dompter les hommes, à émousser les glaives, à contenir les mers. Il connut le langage « des oiseaux, sut apaiser l ’incendie, calmer les flots, guérir les chagrins. Quelquefois « aussi il put se donner la force de huit hommes. Il lutta avec Rigr (le dieu) dans la « science des runes et en toutes sortes de talents d’esprit ; il remporta la victoire. Alors il « lui fut donné, il lui fut accordé de s’appeler Rigr lui-même, et d’être savant en toutes les « choses de l’intelligence. » – Cette peinture hyperbolique de tout ce que devait savoir un jarl, ou noble, pour être digne de son titre, n’est assurément pas d’une race barbare.}.\par
 Qu’on juge de la puissante sympathie d’idées, de l’ardente curiosité intellectuelle qui, possédant toutes les nations germaniques, reliait entre eux les odels les plus éloignés, neutralisait chez leurs fiers possesseurs, et sous les rapports les plus nobles, l’esprit d’isolement, empêchait le souvenir de la commune origine de s’éteindre, et, si ennemis que les circonstances pussent les faire, leur rappelait constamment qu’ils pensaient, sentaient, vivaient sur le même fonds commun de doctrines, de croyances, d’espérances et d’honneur. Tant qu’il y eut un instinct qu’on put appeler germanique, cette cause d’unité fit son office. Charlemagne était trop grand pour la méconnaître ; il en comprenait toute la force et le parti qu’il en devait tirer. Aussi, malgré son admiration pour la romanité et son désir de restaurer de pied en cap le monde de Constantin, il n’eut jamais la moindre velléité de rompre avec ces traditions, bien que méprisées par la triste pédanterie gallo-romaine. Il fit réunir de toutes parts les poésies nationales, et il ne tint pas à lui qu’elles n’échappassent à la destruction. Malheureuse­ment, des nécessités d’un ordre supérieur contraignirent le clergé à tenir une conduite différente.\par
Il lui était impossible de tolérer que cette littérature, essentiellement païenne, troublât incessamment la conscience mal assurée des néophytes, et, les faisant rétrograder vers leurs affections d’enfance, ralentît le triomphe du christianisme. Elle mettait un tel emportement, une obstination si haineuse à célébrer les dieux du Walhalla et à préconiser leurs orgueilleuses leçons, que les évêques ne purent hésiter à lui déclarer la guerre. La lutte fut longue et pénible. La vieille attache des populations aux monuments de la gloire passée protégeait l’ennemi Mais enfin, la victoire étant restée à la bonne cause, l’Église ne se montra nullement désireuse de pousser son succès jusqu’à l’extermination totale. Lorsqu’elle n’eut plus rien à craindre pour la foi, elle tâcha elle-même de sauver des débris désormais inoffensifs. Avec cette tendre considération qu’elle a toujours montrée pour les œuvres de l’intelligence, même les plus opposées à ses sentiments, noble générosité dont on ne lui sait pas assez de gré, elle fit pour les œuvres germaniques exactement ce qu’elle faisait pour les livres profanes des Romains et des Grecs. Ce fut sous son influence que les Eddas furent recueillies en Islande. Ce sont des moines qui ont sauvé le poème de \emph{Beowulf}, les annales des rois anglo-saxons, leurs généalogies, les fragments du \emph{Chant du Voyageur}, de la \emph{Bataille de Finnesburh}, de \emph{Hiltibrant} \footnote{Dans sa forme actuelle, le poème de \emph{Beowulf} est du VIII\textsuperscript{e} siècle environ. (Ettmuller, \emph{Beowulfslied}, Einl. LXIII.) Les événements qu’il rapporte ne sont pas postérieurs à l’an 600 ; et même la mort d’Hygeiak, dont il fait mention, est placée par Grégoire de Tours entre 515 et 520. Ce poème semble avoir été formé de plusieurs chants différents ; on y remarque des espèces de sutures.}. D’autres religieux compilèrent tout ce que nous possédons des traditions du Nord, non comprises dans l’ouvrage de Sæmund, les chroniques d’Adam de Brême et du grammairien Saxon ; d’autres, enfin, transmirent à l’auteur du \emph{Nibelungenlied} les légendes d’Attila que le X\textsuperscript{e} siècle vit mettre en œuvre \footnote{Am. Thierry, \emph{Revue des Deux-Mondes},1er décembre 1852, p. 845.}. Ce sont là des services qui méritent d’autant plus de reconnaissance, que la critique ne doit qu’à eux seuls de pouvoir rattacher directement les parties originales des littératures modernes, les inspirations qui ne proviennent pas absolu­ment de l’influence hellénistique ou italiote, aux anciennes sources arianes, et par là aux grands souvenirs épiques de la Grèce primitive, de l’Inde, de l’Iran bactrien et des nations génératrices de la haute Asie.\par
Les poèmes odiniques avaient eu d’exaltés défenseurs, mais parmi ceux-ci les femmes s’étaient surtout fait distinguer. Elles avaient témoigné d’un attachement parti­culièrement opiniâtre aux anciennes mœurs et aux anciennes idées ; et, contrairement à ce qu’on suppose généralement de leur prédilection pour le christianisme, opinion vraie quant aux pays romanisés, mais dénuée de fondement dans les contrées germaniques, elles prouvèrent qu’elles aimaient du fond du cœur une religion et des coutumes assez austères peut-être, mais qui, leur attribuant un esprit sagace et pénétrant jusqu’à la divination, les avaient entourées de ces respects et armées de cette autorité que leur refusaient si dédaigneusement les paganismes du Sud sous l’empire de l’ancien culte. Bien loin qu’on les crût indignes de juger des choses élevées, on leur confiait les soins les plus intellectuels : elles avaient la charge de conserver les connaissances médicales, de pratiquer, en concurrence avec les thaumaturges de profession, la science des sortilèges et des recettes magiques. Instruites dans tous les mystères des runes \footnote{Weinhold, \emph{ouvr. cité}, p. 56. – W. C. Grimm, \emph{Deutsche Runen}, p. 51.}, elles les communiquaient aux héros, et leur prudence avait le droit de diriger, de hâter, de retarder les effets du courage de leurs maris ou de leurs frères. C’était une situation dont la dignité était faite pour leur plaire, et il n’y a rien de surprenant à ce qu’elles n’aient pas cru tout d’abord devoir gagner au change. Leur opposition, nécessairement limitée, se manifesta par leur entêtement pour la poésie germanique même. Devenues chrétiennes, elles en excusaient volontiers les défauts hétérodoxes ; et ces dispositions mutines persistèrent si bien chez elles, que, longtemps après avoir renoncé au culte de Wodan et de Freya, elles restèrent les dépositaires attitrées des chants des scaldes. Jusque sous les voûtes bénies des monastères, elles maintenaient cette habitude réprouvée, et un concile de 789 ne put même réussir, en fulminant les défenses les plus absolues et les menaces les plus effrayantes, à empêcher d’indisciplinables épouses du Seigneur de transcrire, d’appren­dre par cœur et de faire circuler ces œuvres antiques qui ne respiraient que les louanges et les conseils du panthéon scandinave \footnote{Weinhold, \emph{ouvr. cité}, p. 91. – Les canons de Chalcédoine avaient défendu aux femmes de s’approcher de l’autel et d’y remplir aucune fonction. Le pape Gélase renouvela cette interdiction dans ses décrétales, à cause des manquements fréquents qu’y faisaient les populations germanisées.}.\par
La puissance des femmes dans une société est un des gages les plus certains de la persistance des éléments arians. Plus cette puissance est respectée, plus on est en droit de déclarer la race qui s’y montre soumise rapprochée des vrais instincts de la variété noble ; or, les Germaines n’avaient rien à envier à leurs sœurs des branches antiques de la famille \footnote{Une marque singulière de la puissance que les races germaniques prêtaient aux femmes s’est empreinte dans cette tradition très tardive que Charlemagne, abattu par la défaite de Roncevaux, leva, d’après le conseil d’un ange, une armée de cinquante-trois mille vierges, auxquelles les païens n’osèrent résister. (Weinhold, \emph{ouvr. cité}, p. 44.)}.\par
La plus ancienne dénomination que leur applique la langue gothique est \emph{quino ;} c’est le corrélatif du grec (alphabet grec). Ces deux mots viennent d’un radical commun, \emph{gen}, qui signifie \emph{enfanter} \footnote{Gothique : \emph{ginan, genûm, gen} ; c’est le latin \emph{gignere}, et le grec (alphabet grec). C’est un radical fort ancien.}. La femme était donc essentiellement, aux yeux des Arians primitifs, \emph{la mère}, la source de la famille, de la race, et de là provenait la vénération dont elle était l’objet. Pour les deux autres variétés humaines et beaucoup de races métisses en décadence, bien que fort civilisées, la femme n’est que la femelle de l’homme.\par
De même que l’appellation de l’Arian Germain, du guerrier, \emph{jarl}, finit, dans la patrie du nord, par s’élever à la signification de gouvernant et de roi, de même le mot \emph{quino}, graduellement exalté, devint le titre exclusif des compagnes du souverain, de celles qui régnaient à ses côtés, en un mot, des \emph{reines.} Pour le commun des épouses, une appellation qui n’était guère moins flatteuse y succéda : c’est \emph{frau, frouwe}, mot divinisé dans la personnalité céleste de Freya \footnote{Sanscrit : \emph{prî} ; zend : \emph{frî} ; gothique : \emph{frijô, j’aime}. (Bopp, \emph{Vergleichende Grammatik}, p. 123.)}. Après ce mot, il en est d’autres encore qui sont tous frappés au même cachet. Les langues germaniques sont riches en désignations de la femme, et toutes sont empruntées à ce qu’il y a de plus noble et de plus respectable sur la terre et dans les cieux \footnote{Weinhold, \emph{ouvr. cité}, p. 20. – L’expression \emph{muine}, ancien féminin de \emph{mann}, n est pas germanique. Elle paraît être d’origine celtique. Elle ne s’est conservée que comme indiquant un démon femelle, dans les composés \emph{murmuine}, sirène, et \emph{wuldmuine}, dryade. (W. Muller, \emph{Altdeutsche Religion}, p. 366.)}. Ce fut sans doute par suite de cette tendance native à estimer à un haut degré l’influence exercée sur lui par sa compagne, que l’Arian du nord accepta, dans sa théologie, l’idée que chaque homme était dès sa naissance placé sous la protection particulière d’un génie féminin, qu’il appelait \emph{fylgja.} Cet ange gardien soutenait et consolait, dans les épreuves de la vie, le mortel qui lui était confié par les dieux, et, lorsque celui-ci touchait à l’heure suprême, il lui apparaissait pour l’avertir \footnote{Weinhold, \emph{ouvr. cité}, p. 49.}.\par
Cause ou résultat de ces habitudes déférentes, les mœurs étaient généralement si pures, que dans aucun des dialectes nationaux il ne se trouve un mot pour rendre l’idée de courtisane. Il semblerait que cette situation n’ait été connue des Germains qu’à la suite du contact avec les races étrangères, car les deux plus anciennes dénominations de ce genre sont le finnique \emph{kalkjô} et le celtique \emph{lenne} et \emph{laënia} \footnote{Ibid., p. 291. – Les crimes contre les femmes ne trouvaient même pas toujours d’excuse dans l’emportement de la conquête, et, au sac de Rome par Alaric, un Goth de grande naissance, ayant violé la fille d’un Romain, fut condamné à mort, malgré la résistance du roi, et exécuté. (Kemble, t. I, p. 190.)}.\par
L’épouse germanique apparaît, dans les traditions, comme un modèle de majesté et de grâce, mais de grâce imposante. On ne la confinait pas dans une solitude jalouse et avilissante ; l’usage voulait, au contraire, que, lorsque le chef de famille traitait des hôtes illustres, sa compagne, entourée de ses filles et de ses suivantes, toutes richement vêtues et parées, vînt honorer la fête de sa présence. C’est avec un enthousiasme bien caractéristique que des scènes de ce genre sont décrites par les poètes \footnote{Ettmuller, \emph{Beowulfslied}, Einl., p. XLVII.}.\par
« Le plaisir des héros était au comble, a chanté l’auteur de \emph{Beowulf.} La « grand’salle retentissait de paroles bruyantes. Alors entra Wealthéow, l’épouse « de Hrôdhgâr. Gracieuse pour les hommes de son mari, la noble créature, ornée « d’or, salua gaiement les guerriers attablés. Puis, charmante femme, elle offrit « d’abord la coupe au protecteur des odels danois et avec d’aimables paroles « l’encouragea à se réjouir et à bien traiter ses fidèles.\par
« Le chef magnanime saisit joyeusement la coupe. Puis la fille des nobles « Helmings salua, à la ronde, ceux des convives, jeunes ou vieux, à qui leur « valeur avait mérité d’illustres dons ; enfin, elle s’arrêta, la belle souveraine, « couverte de bracelets et de chaînes précieuses, la généreuse dame, devant le « siège de Beowulf. Elle salua en lui le soutien des Goths et lui versa la bière. « Pleine de sagesse, elle prit le ciel à témoin des vœux qu’elle formait pour « lui, car elle n’avait foi que dans ce champion valeureux pour punir les « crimes de Grendel \footnote{Kemble, \emph{The anglo-saxon Poem of Beowulf}, v. 1215 et seqq., p. 44-45.}. »\par
Après avoir accompli ses devoirs de courtoisie, la maîtresse du logis s’asseyait auprès de son époux et se mêlait aux entretiens. Mais avant que le banquet n’arrivât à sa période la plus animée, et quand les fumées de l’ivresse commençaient à gagner les héros, elle se retirait. C’est encore ainsi qu’on en use en Angleterre, le pays qui a le mieux conservé les débris des usages germaniques.\par
Retirées dans leur intérieur, les soins domestiques, les travaux de l’aiguille et du fuseau, la préparation des compositions pharmaceutiques, l’étude des runes, celle des compositions littéraires, l’éducation de leurs enfants, les entretiens intimes avec leurs époux, composaient aux femmes un cercle d’occupations qui ne manquait ni de variété ni d’importance. C’était dans le séjour particulièrement intime de la chambre nuptiale que ces sibylles de la famille rendaient leurs oracles écoutés du mari. Dans cette vie de confiance mutuelle, on jugeait que l’affection sérieuse et bien fondée sur le libre choix n’était pas de trop ; les filles avaient le droit de ne se marier qu’à leur convenance. C’était la règle ; et, lorsque la politique ou d’autres raisons la transgressaient, il n’était pas sans exemple que la victime apportât dans la demeure qu’on lui imposait une rancune implacable et n’y excitât de ces tempêtes qui finirent quelquefois, au dire de nombreuses légendes, par la ruine complète des plus puissantes familles, tant était grande et indomptable la fierté de l’épouse germanique.\par
Ce n’est pas à dire toutefois que les prérogatives féminines n’eussent leurs limites \footnote{La considération vouée aux femmes était plus religieuse que civile, plus passive qu’active. On les jugeait faibles de corps et grandes par l’esprit. On les consultait, mais on ne leur confiait pas l’action. (Weinhold, p. 149.)}. S’il est plus d’un exemple de la participation des femmes aux travaux guerriers, la loi les tenait, en principe, pour incapables de défendre la terre \footnote{Weinhold cite, d’après Luitprand et Jornandès, une foule de cas où les femmes germaniques prenaient les armes. (\emph{Ouvr. cité}, p. 42.)} ; par conséquent, elles n’héritaient pas de l’odel. Encore moins pouvaient-elles prétendre à être substituées aux droits de leurs époux défunts sur les féods \footnote{La notion germanique sur l’exercice des droits politiques était que celui-là seul y était admis qui pouvait remplir tous les devoirs de la communauté. La loi excluait donc les enfants, les esclaves, les vaincus et les femmes, tous par des causes inhérentes à leur situation. (Weinhold, \emph{ouvr. cité}, p. 120.)}. On les croyait propres au conseil, impropres à l’action. Si, en outre, on admettait chez elles l’esprit divinatoire, on ne pouvait leur confier les fonctions sacerdotales, puisque le glaive de la loi y était joint. Cette exclusion était si absolue, que dans plusieurs temples les rites voulaient que le pontife portât les habits de l’autre sexe ; néanmoins c’était toujours un prêtre. Les Arians Germains n’avaient pu accepter qu’avec cette modification les cultes que leur avaient fait adopter les nations celtiques parmi lesquelles ils vivaient \footnote{W. Muller, \emph{Altdeutsche Religion}, p. 53\emph{. –} Nerthus même avait un prêtre, et non une prêtresse.}.\par
Malgré ces restrictions et d’autres encore, l’influence des femmes germaines et leur situation dans la société étaient des plus considérables. Vis-à-vis de leurs pareilles de la Grèce et de Rome sémitisées, c’étaient de véritables reines en présence de serves, sinon d’esclaves. Quand elles arrivèrent avec leurs maris dans les pays du sud, elles se trouvèrent dans la meilleure des conditions pour transformer à l’avantage de la moralité générale les rapports de famille, et par suite la plupart des autres relations sociales. Le christianisme, qui, fidèle à son désintéressement de toutes formes et de toutes combinaisons temporelles, avait accepté la sujétion absolue de l’épouse orientale, et qui pourtant avait su ennoblir cette situation en y faisant entrer l’esprit de sacrifice, le christianisme, qui avait appris à sainte Monique à se faire de l’obéissance conjugale un échelon de plus vers le ciel, était loin de répugner aux notions nouvelles, et évidemment beaucoup plus pures, que les Arians Germains introduisaient. Néanm­oins il ne faut pas perdre de vue ce que nous avons observé tout à l’heure. L’Église eut d’abord assez peu à se louer de l’esprit d’opposition qui animait les Germaines. Il sembla que les derniers instincts du paganisme se fussent retranchés dans les institutions civiles qui les concernaient. Sans parler de la chevalerie, dont les idées sur cette matière appelèrent souvent la réprobation des conciles, il est curieux de voir toute la peine qu’éprouve le clergé à faire accepter comme indispensable son intervention dans la célébration des mariages \footnote{Les doubles mariages des Mérowings, qui produisaient régulièrement tous leurs effets civils, avaient lieu assurément sans la participation de l’Église. – jusqu’au XV\textsuperscript{e} siècle, il fut très difficile de faire accepter aux populations allemandes l’intervention d’un prêtre dans les cérémonies du mariage. Souvent même, lorsque sa présence fut requise, elle n’eut lieu qu’au milieu de la fête et sans qu’il fût question de se rendre à l’église. – On admit aussi la bénédiction ecclésiastique après la consommation du mariage. (Weinhold, \emph{ouvr. cité}, p. 260.)}. La résistance existait encore, chez certaines populations germanisées, dans le XVI\textsuperscript{e} siècle \footnote{On cite encore, en 1551, un cas de mariage dans la haute bourgeoisie protestante où n’intervint aucune action religieuse. (Weinhold, \emph{ouvr. cité}, p. 263.) – La bigamie de Philippe de Hesse pouvait se défendre à ce point de vue.}. On n’y voulait considérer le lien conjugal que comme un contrat purement civil, où l’action religieuse n’avait pas à s’exercer.\par
En combattant cette bizarrerie, dont les causes laissent entrevoir une bien singulière profondeur, l’Église ne perdit rien de sa bienveillance pour les conceptions très nobles auxquelles elle était jointe. En les épurant, elle s’y prêta, et ne contribua pas peu à les conserver dans les générations successives où désormais les mélanges ethniques tendent à les faire disparaître, surtout chez les peuples du midi de l’Europe.\par
Arrêtons-nous ici. C’en est assez sur les mœurs, les opinions, les connaissances, les institutions des Arians Germains pour faire comprendre que dans un conflit avec la société romaine cette dernière devait finir par avoir le dessous. Le triomphe des peuples nouveaux était infaillible. Les conséquences en devaient être bien autrement fécondes que les victoires des légions sous Scipion, Pompée et César. Que d’idées, non pas nées d’hier, très antiques au contraire, mais depuis longtemps disparues des contrées du midi, et oubliées avec les nobles races qui jadis les avaient pratiquées, allaient reparaître dans le monde ! Que d’instincts diamétralement opposés à l’esprit hellénistique ! Vertus et vices, défauts et qualités, tout dans les races arrivantes était combiné de façon à transformer la face de l’univers civilisé. Rien d’essentiel ne devait être détruit, tout devait être changé. Les mots même allaient perdre leur sens. La liberté, l’autorité, la loi, la patrie, la monarchie, la religion même, se dépouillant peu à peu de costumes et d’insignes usés, allaient pour plusieurs siècles en posséder d’autres, bien autrement sacrés.\par
Cependant les nations germaniques, procédant avec la lenteur qui est la condition première de toute œuvre solide, ne devaient pas débuter par cette restauration radi­cale ; elles commencèrent par vouloir maintenir et conserver, et cette tâche honorable, elles l’accomplirent sur la plus vaste échelle.\par
Pour assister à la manière dont elle s’exécuta, reportons-nous encore une fois à l’époque du premier César, et nous allons voir se dérouler sous nos yeux cet état de choses qu’annonçait la fin du livre précédent : nous allons contempler la Rome germanique.
\section[{VI.4. Rome germanique. – Les armées romano-celtiques et romano-germaniques. – Les empereurs germains.}]{VI.4. \\
Rome germanique. – Les armées romano-celtiques et romano-germaniques. – Les empereurs germains.}
\noindent Le rôle ethnique des populations septentrionales ne commence qu’au I\textsuperscript{er} siècle avant notre ère à prendre une importance générale et bien marquée.\par
Ce fut l’époque où le dictateur crut devoir traiter d’une manière si favorable les Gaulois, ces antiques ennemis du nom romain. Il fit d’eux les soutiens directs de son gouvernement, et ses successeurs, continuant dans la même voie, témoignèrent de leur mieux qu’ils avaient bien compris tous les services que les nations habitant entre les Pyrénées et le Rhin pouvaient rendre à un pouvoir essentiellement militaire. Ils s’étaient aperçus que c’était chez celles-ci une sorte d’instinct que de se dévouer sans réserve aux intérêts d’un général, quand surtout il était étranger à leur sang.\par
Cette condition était indispensable, et voici pourquoi : les Celtes de la Gaule, animés d’un esprit de localité bien franc, et plein de turbulence, s’attachaient beaucoup plus, dans les affaires de leurs cités, aux questions de personnes qu’aux questions de fait. La politique de leurs nations avait pris, dans cette habitude, une vivacité d’allures qui n’était guère proportionnée à la dimension des territoires. Des révolutions perpétuelles avaient épuisé la plupart de ces peuples. La théocratie, renversée presque partout, d’abord effacée devant la noblesse, puis, au moment où les Romains dépassaient les limites de la Provence, la démocratie et son inséparable sœur, la démagogie, faisant invasion à leur tour, avaient attaqué le pouvoir des nobles. La présence de ce genre d’idées annonçait clairement que le mélange des races était arrivé à ce point où la confusion ethnique crée la confusion intellectuelle et l’impossibilité absolue de s’entendre. Bref, les Gaulois, qui n’étaient point des barbares, étaient des gens en pleine voie de décadence, et, si leurs beaux temps avaient infiniment moins d’éclat que les périodes de gloire à Sidon et à Tyr, il n’en est pas moins indubitable que les cités obscures des Carnutes, des Rèmes et des Éduens mouraient du même mal qui avait terminé l’existence des brillantes métropoles chananéennes \footnote{Tacite, si grand admirateur des Germains, bien que souvent d’une manière un peu romanesque, traite les Gaulois de son temps avec une extrême sévérité. (\emph{Germ}., 28, 29.)}.\par
Les populations gaffiques, mêlées de quelques groupes slaves, s’étaient diverse­ment alliées aux aborigènes finnois. De là des différences fondamentales. Il en était résulté les séparations primitives les plus tranchées des tribus et des dialectes. Dans le nord, quelques peuples avaient été relevés par le contact avec les Germains ; d’autres, dans le sud-ouest, avaient subi celui des Aquitains ; sur la côte de la Méditerranée, le mélange s’était opéré avec des Ligures et des Grecs, et depuis un siècle les Germains sémitisés occupant la Province étaient venus compliquer encore ce désordre. Le développement du mal était d’ailleurs favorisé par la disposition sporadique de ces sociétés minuscules, où l’intercession du moindre élément nouveau développait presque instantanément ses conséquences.\par
Si chacune des petites communautés gauloises s’était trouvée subitement isolée, au moment même où les principes ethniques qui la composaient étaient parvenus à l’apogée de leur lutte, l’ordre et le repos, je ne dis pas de hautes facultés, auraient pu s’établir, parce que la pondération des races fusionnées s’accomplit plus facilement dans un moindre espace. Mais lorsqu’un groupe assez restreint reçoit de continuels apports de sang nouveau avant d’avoir en le temps d’amalgamer les anciens, les perturbations deviennent fréquentes, et sont plus rapides comme aussi plus doulou­reuses. Elles mènent à la dissolution finale. C’était la situation des États de la Gaule lorsque les légions romaines les envahirent.\par
Comme les populations y étaient braves, riches, pourvue, de beaucoup de ressources et, entre autres, de places de guerre fortes et nombreuses, l’envie de résister ne leur manquait pas ; mais ce qui leur manquait, on le voit, c’était la cohésion, non pas seulement entre nations, mais encore entre concitoyens. Presque partout les nobles trahissaient le peuple, quand le peuple ne vendait pas les nobles. Le camp romain était toujours encombré de transfuges de toutes les opinions, aveuglément acharnés à poignarder leurs ennemis politiques à travers la gorge de leur patrie. Il y eut des hommes dévoués, des intentions généreuses ; ce fut sans résultat. Les Celtes germa­nisés sauvèrent presque seuls l’antique réputation. Arvernes, ils s’élevèrent jusqu’aux prodiges ; Belges, ils furent presque déclarés indomptables par le vainqueur ; mais quant aux populations renommées comme les plus illustres, comme les plus intelligentes, celles précisément où les révolutions ne cessaient pas, les Rèmes, les Éduens, celles-là ou bien résistèrent à peine, ou bien s’abandonnèrent du premier coup à la générosité des conquérants, ou enfin, entrant sans honte dans les projets de l’étranger, reçurent avec joie, en échange de leur indépendance, le titre d’amies et d’alliées du peuple romain. En dix ans la Gaule fut domptée et à jamais soumise. Des armées qui valent bien celles de Rome n’ont pas obtenu de nos jours de si brillants succès chez les barbares de l’Algérie : triste comparaison pour les populations celtiques.\par
Mais ces gens si aisés à subjuguer devinrent immédiatement d’irrésistibles instru­ments de compression aux mains des empereurs. On les avait vus dans leurs cités, patriciens arrogants ou démocrates envieux, passer la majeure partie de leur vie dans la sédition ; ils furent à Rome du dévouement le plus utile au principat. Acceptant pour eux-mêmes le joug et l’aiguillon, ils servirent à y façonner les autres, ne sollicitant en retour de leur complaisance que les honneurs soldatesques et les émotions de la caserne. On leur prodigua ces biens par surcroît.\par
César avait composé sa garde de Gaulois. Il lui avait donné malicieusement le plus joli emblème de la légèreté et de l’insouciance, et les légionnaires kymris de l’Alauda, qui étalaient si fièrement sur leurs casques et sur leurs boucliers la figure de l’alouette, s’accordèrent avec tous leurs concitoyens pour chérir le grand homme qui les avait débarrassés de leur isonomie et leur faisait une existence si conforme à leurs goûts.\par
Ils étaient donc fort satisfaits ; mais ce ne serait pas rendre justice aux Gaulois que de supposer qu’ils aient été constants et inébranlables dans leur amour de l’autorité romaine. Maintes fois ils se révoltèrent, mais toujours pour revenir à l’obéissance, sous la pression d’une inexorable impossibilité de s’entendre. L’habitude d’être gouvernés par un maître ne leur apprit jamais le respect d’une loi. S’insurger, pour eux, c’était la moindre des difficultés et peut-être le plus vif des plaisirs. Mais aussitôt qu’il s’agissait d’organiser un gouvernement national à la place du pouvoir étranger que l’on venait de briser, aussitôt qu’il s’agissait de revenir à une règle quelconque et d’obéir à quelqu’un, l’idée que la prérogative souveraine allait appartenir à un Gaulois glaçait tous les esprits. Il eût semblé que c’était pourtant là le véritable but de l’insurrection ; mais non, les combinaisons les plus ingénieuses s’efforçaient en vain de tourner ce terrible écueil ; toutes s’y brisaient. Les assemblées, les conseils discutaient la question avec furie, et se séparaient tumultueusement sans réussir à passer outre. Alors les gens timides, qui s’étaient tenus à l’écart jusque-là, tous les amis secrets de la domination impériale reprenaient courage ; on allait répétant avec eux que le pouvoir des aigles pouvait être un mal mais qu’après tout Petilius Cerialis avait eu raison de dire aux Belges que c’était un mal nécessaire et qu’en dehors il n’y avait que la ruine. Cela dit, on rentrait la tête basse dans le bercail romain.\par
Cette singulière inaptitude d’indépendance se révéla sous toutes ses faces. On eût dit que le sort prenait plaisir à la pousser à bout. Il arriva un jour aux Gaulois de posséder un empereur à eux. Une femme le leur avait donné, et ne leur demandait que de le soutenir contre le concurrent d’Italie. Cet empereur, Tetricus, eut à lutter contre les mêmes impossibilités où s’étaient brisées les insurrections précédentes, et, bien qu’appuyé par les légions germaniques, qui le maintenaient contre le mauvais vouloir ou plutôt contre la légèreté chronique de ses peuples, il crut bien faire, et fit bien sans doute, d’échanger son diadème contre la préfecture de la Lucanie. Les États éphémères rentrèrent dans le devoir, en murmurant peut-être, au fond très satisfaits de n’avoir pas lâché un pouce de leurs jalousies municipales.\par
L’expérience journalière le démontrait donc : les Gaulois du I\textsuperscript{e} et du II\textsuperscript{e} siècle de notre ère n’avaient que des qualités martiales ; mais ils les avaient à un degré supérieur. Ce fut pour ce motif qu’impuissants dans leur propre cause, ils exercèrent une influence momentanée si considérable sur le monde romain sémitisé.\par
Certainement le Numide était un adroit cavalier, le Baléare un frondeur sans pareil ; les Espagnols fournissaient une infanterie qui bravait toute comparaison, et les Syriens, encore infatués des souvenirs d’Alexandre, donnaient des recrues d’une réputation aussi grande que justifiée. Cependant tous ces mérites pâlissaient devant celui des Gaulois. Ses rivaux de gloire, basanés et petits, ou du moins de moyenne taille, ne pouvaient lutter d’apparence martiale avec le grand corps du Trévire ou du Boïen, plus propre que personne à porter légèrement sur ses larges épaules le poids énorme dont la discipline réglementaire chargeait le fantassin des légions. C’était donc à bon droit que l’État cherchait à multiplier les enrôlements dans la Gaule, et surtout dans la Gaule germanisée. Sous les douze Césars, alors que l’action politique se concentrait encore chez les populations méridionales, c’était déjà le Nord qui était surtout chargé de maintenir par les armes le repos de l’empire.\par
Toutefois il est remarquable que cette estime, qui facilitait aux soldats de race celtique l’accès des grandes dignités militaires, voire de la chaire sénatoriale, ne les rendit pas participants au concours ouvert pour la pourpre souveraine. Les premiers provinciaux qui y parvinrent furent des Espagnols, des Africains, des Syriens, jamais des Gaulois, sauf les exemples irréguliers et peu encourageants de Tetricus et de Posthume. Décidément les Gaulois n’avaient pas d’aptitudes gouvernementales, et si Othon, Galba, Vitellius pouvaient en faire d’excellents suppôts de révolte, il ne venait à l’esprit de personne d’en tirer des administrateurs ni des hommes d’État, Gais et remuants, ils n’étaient ni instruits ni portés à le devenir. Leurs écoles, fécondes en pédants, fournissaient très peu d’esprits réellement distingués. Le premier rang ne leur était donc pas accessible et ce trône qu’ils gardaient si bien, ils n’étaient pas aptes à y monter.\par
Cette impuissance attachée à l’élément celtique cessa complètement de peser sur les armées septentrionales aussitôt qu’elles eurent commencé à se recruter beaucoup moins chez les Gaulois germanisés, bientôt atteints, comme les autres, par la lèpre romaine, que chez les Germains méridionaux, quoique ces derniers eux-mêmes fussent assez loin, pour la plupart, d’être de sang pur. Les effets de cette modification éclatèrent dès l’an 252, à l’avènement de Julius Verus Maximinus, lequel était fils d’un guerrier goth. La dépravation romaine, dans ses progrès sans remède, avait reconnu d’instinct l’unique moyen de prolonger sa vie, et tout en continuant de maudire et de dénigrer les barbares du Nord, elle consentait à leur laisser prendre toutes les positions qui la dominaient elle-même et d’où on pouvait la conduire.\par
 À dater de ce moment, l’essence germanique éclipse toutes les autres dans la romanité \footnote{« La Pannonie et la Mœsie romaines furent, aux III\textsuperscript{e} et IV\textsuperscript{e} siècles, la pépinière des légions, et, par les légions, celle des Césars. » (Amédée Thierry, \emph{Revue des Deux-Mondes}, 15 juillet 1954.)}. Elle anime les légions, possède les hautes charges militaires, décide dans les conseils souverains. La race gauloise, qui d’ailleurs n’était représentée vis-à-vis d’elle que par des groupes septentrionaux, ceux qui lui étaient déjà apparentés, lui cède absolument le pas. L’esprit des jarls, chefs de guerre, s’empare du gouvernement pratique, et l’on est déjà en droit de dire que Rome est germanisée, puisque le principe sémitique tombe au fond de l’océan social et se laisse visiblement remplacer à la surface par la nouvelle couche ariane.\par
Une révolution si extraordinaire, bien que latente, cette superposition contre nature d’une race ennemie, qui, plus souvent vaincue que victorieuse, et méprisée officielle­ment comme barbare, venait ainsi déprimer les races nationales, une si étrange anomalie avait beau s’effectuer par la force des choses, elle avait à percer trop de difficultés pour ne pas s’accompagner d’immenses violences.\par
Les Germains, appelés à diriger l’empire, trouvaient en lui un corps épuisé et moribond. Pour le faire vivre ce grand corps, ils étaient incessamment obligés de combattre ou les demandes d’un tempérament différent du leur, ou les caprices nés du malaise général, ou les exaspérations de la fièvre, également fatales au maintien de la paix publique. De là des sévérités d’autant plus outrées que ceux qui les jugeaient nécessaires, étant imparfaitement éclairés sur la nature complexe de la société qu’ils traitaient, poussaient aisément jusqu’à l’abus l’emploi des méthodes réactives, Ils exagéraient, avec toute la fougue intolérante de la jeunesse, la proscription dans l’ordre politique et la persécution dans l’ordre religieux. C’est ainsi qu’ils se montrèrent les plus ardents ennemis du christianisme. Eux qui devaient plus tard devenir les propagateurs de tous ses triomphes, ils débutèrent par le méconnaître ; ils se laissèrent prendre à la calomnie qui le poursuivait. Persuadés qu’ils tenaient dans ce culte nouveau une des expressions les plus menaçantes de l’incrédulité philosophique, leur amour inné d’une religion définie, considérée comme base de tout gouvernement régulier, le leur rendit d’abord odieux ; et ce qu’ils détestèrent en lui, ce ne fut pas lui, mais un fantôme qu’ils crurent voir. On est donc moins tenté de leur reprocher le mal qu’ils ont fait eux-mêmes que celui, beaucoup plus considérable, qu’ils ont laissé faire aux partisans sémitisés des anciens cultes. Cependant il faudrait craindre aussi de leur trop demander. Pouvaient-ils étouffer les conséquences inévitables d’une civilisation pourrie qu’ils n’avaient pas créée ? Réformer la société romaine sans la renverser, c’eût été beau sans doute. Substituer doucement, insensiblement, la pureté catholique à la dépravation païenne sans rien briser dans l’opération, c’eût été le bien idéal ; mais, qu’on y réfléchisse, un tel chef-d’œuvre n’aurait été possible qu’à Dieu.\par
Il n’appartient qu’à lui de séparer d’un geste la lumière des ténèbres et les eaux du limon. Les Germains étaient des hommes, et des hommes richement doués sans doute, mais sans nulle expérience du milieu où ils étaient appelés ; ils n’eurent pas cette puissance. Leur travail, depuis le milieu du III\textsuperscript{e} siècle jusqu’au V\textsuperscript{e}, se borna à conserver le monde tellement quellement, dans la forme où on le leur avait remis.\par
En considérant les choses sous ce point de vue, qui est le seul véritable, on n’accuse plus, on admire. De même encore, en reconnaissant sous leurs toges et leurs armures romaines Decius, Aurélien, Claude, Maximien, Dioclétien, et la plupart de leurs successeurs, sinon tous, jusqu’à Augustule, pour des Germains et fils de Germains, on convient que l’histoire est complètement faussée par ces écrivains, tant modernes qu’anciens, dont l’invariable système est de représenter comme un fait monstrueux, comme un cataclysme inattendu, l’arrivée finale des nations tudesques tout entières au sein de la société romanisée.\par
Rien, au contraire, de mieux annoncé et de plus facile à prévoir, rien de plus légitime, rien de mieux préparé que cette conclusion. Les Germains avaient envahi l’empire du jour où ils étaient devenus ses bras, ses nerfs et sa force. Le premier point qu’ils en avaient pris, ç’avait été le trône, et non pas par violence ou usurpation ; les populations indigènes elles-mêmes, se reconnaissant à bout de voies, les avaient appelés, les avaient payés, les avaient couronnés.\par
Pour gouverner à leur guise, comme ils en avaient incontestablement le droit et même le devoir, les empereurs ainsi installés s’étaient entourés d’hommes capables de comprendre et d’exécuter leur pensée, c’est-à-dire d’hommes de leur race. Ils ne trouvaient que chez ces Romains improvisés le reflet de leur propre énergie et la facilité nécessaire à les bien servir. Mais qui disait Germain, disait soldat. La profession des armes devint ainsi la condition première de l’admission aux grands emplois. Tandis que dans la vraie conception romaine, italique et romaine sémitique, la guerre n’avait été qu’un accident, et ceux qui la faisaient que des citoyens momentanément détournés de leurs fonctions régulières, la guerre fut pour la magistrature impériale la situation naturelle, sur laquelle durent se façonner l’éducation et l’esprit de l’homme d’État. En fait, la toge céda le pas à l’épée.\par
À la vérité, le profond bon sens des hommes du Nord ne voulut jamais que cette prédilection fût officiellement avouée, et telle fut à cet égard sa discrète et sage réserve, que cette convention se maintint à travers tout le moyen âge, et le dépassa pour venir jusqu’à nous. Le guerrier germain romanisé comprenait bien que la prépondérance au moins fictive de l’élément civil importait à la sécurité de la loi et pouvait seule maintenir la société existante.\par
L’empereur et ses généraux savaient donc, au besoin, dissimuler la cuirasse sous la robe de l’administrateur. Pourtant le déguisement n’était jamais si complet qu’il pût tromper des gens malveillants. L’épée montrait toujours sa pointe. Les populations s’en scandalisaient. Les demi-concessions ne les ramenaient pas. La protection qu’elles recevaient ne faisait pas naître leur gratitude. Les talents politiques de leurs gouvernants les trouvaient aveugles. Elles en riaient avec mépris, et murmuraient, depuis le Rhin jusqu’aux déserts de la Thébaïde, l’injure toujours renouvelée de barbare. On ne saurait dire qu’elles eussent tout à fait tort, suivant leurs lumières.\par
 Si les hommes germaniques admiraient l’ensemble de l’organisation romaine, sentiment qui n’est pas douteux, ils n’avaient pas autant de bienveillance pour tels détails qui précisément aux yeux des indigènes en faisaient la plus précieuse parure et composaient l’excellence de la civilisation. Les soldats couronnés et leurs compagnons ne demandaient pas mieux que de conserver la discipline morale, l’obéissance aux magistrats, de protéger le commerce, de continuer les grands travaux d’utilité publi­que ; ils consentaient encore à favoriser les œuvres de l’intelligence, en tant qu’elles produisaient des résultats appréciables pour eux. Mais la littérature à la mode, mais les traités de grammaire, mais la rhétorique, mais les poèmes lippogrammatiques, et toutes les gentillesses de même sorte qui faisaient les délices des beaux esprits du temps, ces chefs-d’œuvre-là les trouvaient, sans exception, plus froids que glace ; et comme, en définitive, les grâces venaient d’eux, et que toutes les faveurs tendaient à se concentrer, après les gens de guerre, sur les légistes, les fonctionnaires civils, les constructeurs d’aqueducs, de routes, de ponts, de forteresses, puis sur les historiens, quelquefois sur les panégyristes brûlant leur encens, par nuages compacts, aux pieds du maître, et qu’elles n’allaient guère plus loin, les classes lettrées ou soi-disant telles étaient en quelque sorte fondées à soutenir que César manquait de goût. Certes ils étaient barbares, ces rudes dominateurs qui, nourris des chants nerveux de la Germanie, restaient insensibles à la lecture comme à l’aspect de ces madrigaux écrits en forme de lyre ou de vase, devant lesquels se pâmaient d’admiration les gens bien élevés d’Alexandrie et de Rome. La postérité aurait bien dû en juger autrement, et prononcer que le barbare existait en effet, mais non pas sous la cuirasse du Germain.\par
Une autre circonstance blessait encore au vif l’amour-propre du Romain. Ses chefs, ignorants pour la plupart ses guerres passées, et jugeant des Romains d’autrefois d’après les contemporains, ne semblaient pas en prendre le moindre souci, et c’était bien dur pour des gens qui se considéraient si forts. Quand Néron avait plus honoré la Grèce que la ville de Quirinus, quand Septime Sévère avait élevé la gloire du borgne de Trasymène au-dessus de celle des Scipions, ces préférences n’étaient du moins pas sorties du territoire national. Le coup était plus rude quand on voyait tels des empereurs de rang nouveau, et les armées qui leur avaient donné la pourpre, ne s’occuper pas plus d’Alexandre le Grand que d’Horatius Coclès. On connut des Augustes qui de leur vie n’avaient entendu parler de leur prototype Octave, et ne savaient pas même son nom. Ces hommes-là sans nul doute savaient pas cœur les généalogies et les actions des héros de leur race.\par
Il ne résultait pas moins de ce fait, comme de tant d’autres, qu’au III\textsuperscript{e} siècle après Jésus-Christ la nation romaine armée et bien portante et la nation romaine pacifique et agonisante ne s’entendaient nullement ; et, quoique les chefs de cette combinaison, ou plutôt de cette juxtaposition de deux corps si hétérogènes, portassent des noms latins ou grecs et s’habillassent de la toge ou de la chlamyde, ils étaient foncièrement, et très heureusement pour cette triste société, de bons et authentiques Germains. C’était là leur titre et leur droit à dominer.\par
Le noyau qu’ils formaient dans l’empire avait d’abord été bien faible. Les deux cents cavaliers d’Arioviste que Jules César prit à sa solde en furent le germe. Des développements rapides succédèrent, et on les remarque surtout depuis que les armées, celles principalement qui avaient leurs cantonnements en Europe, établirent en principe de n’accepter guère que des recrues germaniques. Dès lors l’élément nouveau acquit une puissance d’autant plus considérable qu’elle se retrempa incessamment dans ses sources. Puis chaque jour de nouvelles causes apparurent et se réunirent pour l’entraîner dans les territoires romains, non plus par quantités relativement minimes, mais par masses.\par
Avant de passer à l’examen de cette terrible crise, on peut s’arrêter un moment devant une hypothèse dont la réalisation aurait paru bien séduisante aux populations romaines du V\textsuperscript{e} siècle. La voici : qu’on suppose un instant les nations germaniques qui à cette époque étaient limitrophes de l’empire beaucoup plus faibles, numériquement parlant, qu’elles ne l’ont été en effet ; elles auraient été très promptement absorbées dans le vaste réservoir social qui ne se lassait pas de leur demander des forces. Au bout d’un temps donné, ces familles auraient disparu parmi les éléments romanisés ; puis la corruption générale, poursuivant son cours, aurait abouti à une dégénération chronique qui aujourd’hui permettrait à peine à l’Europe de maintenir une sociabilité quelconque. Du Danube à la Sicile, et de la mer Noire à l’Angleterre, on en serait à peu près au point de décomposition pulvérulente où sont arrivées les provinces méridio­nales du royaume de Naples et la plupart des territoires de l’Asie antérieure.\par
Sur cette hypothèse qu’on en greffe une seconde. Si les nations jaunes et à demi jaunes, à demi slaves, à demi arianes, d’au delà de l’Oural avaient pu garder la possession de leurs steppes, les peuples gothiques, à leur tour, conservant les régions du nord-est jusqu’aux gorges hercyniennes d’une part, jusqu’à l’Euxin de l’autre, n’auraient eu aucune raison de passer le Danube. Elles auraient développé sur place une civilisation toute spéciale, enrichie de très faibles emprunts romains, livrés par l’inévitable absorption qu’elles auraient faite à la longue des colonies transrhénanes et transdanubiennes. Un jour, profitant de la supériorité de leurs forces actives, elles auraient éprouvé le désir de s’étendre pour s’étendre ; mais c’eût été bien tard. L’Italie, la Gaule et l’Espagne n’auraient plus été, comme elles le furent pour les vainqueurs du V\textsuperscript{e} siècle, des conquêtes instructives mais seulement des annexes propres à être exploitées matériellement, comme l’est aujourd’hui l’Algérie.\par
Cependant il y a quelque chose de si providentiel, de si fatal dans l’application des lois qui amènent les mélanges ethniques, qu’il ne serait résulté de cette différence, qui paraît si considérable à la première vue, qu’une simple perturbation de synchronismes. Un genre de culture comparable à celui qui a régné du X\textsuperscript{e} au XIII\textsuperscript{e} siècle environ aurait commencé beaucoup plus tôt et duré plus longtemps, parce que la pureté du sang germanique aurait résisté davantage. Elle aurait néanmoins fini par s’épuiser de même en subissant des contacts absolument semblables à ceux qui l’ont énervée. Les commotions sociales auraient été transportées à d’autres dates ; elles n’en auraient pas moins eu lieu. Bref, par un autre chemin, l’humanité serait arrivée identiquement au résultat qu’elle a obtenu.\par
 Venons à l’établissement des Germains par grandes masses au sein de la romanité, à la façon dont il s’opéra et à la manière dont il doit être jugé.\par
Les empereurs de race teutonique avaient à leur disposition, pour procurer à l’État des défenseurs de leur sang, un moyen infaillible, qui leur avait été enseigné par leurs prédécesseurs romains. Ceux-ci l’avaient appris du gouvernement de la république, qui le tenait des Grecs, lesquels, à travers l’exemple des Perses, l’avaient emprunté à la politique des plus anciens royaumes ninivites. Ce moyen, venu de si loin et d’un emploi si général, consistait à transplanter, au milieu des populations dont la fidélité ou l’aptitude militaire étaient douteuses, des colonisations étrangères destinées, suivant les circonstances, à défendre ou à contenir.\par
Le sénat, dans ses plus beaux jours d’habileté et d’omnipotence, avait fait de fréquentes applications de ce système ; les premiers Césars, tout autant. La Gaule entière, l’île de Bretagne, l’Helvétie, les champs décumates, les provinces illyriennes, la Thrace, avaient fini par être couverts de bandes de soldats libérés du service. On les avait mariés, on les avait pourvus d’instruments agricoles, on leur avait constitué des propriétés foncières, puis on leur avait démontré que la conservation de leur nouvelle fortune, la sécurité de leurs familles et le solide maintien de la domination romaine dans la contrée, c’était tout un. Rien de plus aisé à comprendre en effet, même pour les intelligences les plus rétives, d’après la manière dont on établissait les droits de ces nouveaux habitants à la possession du sol. Ces droits ne résidaient que dans l’expres­sion de la volonté du gouvernement qui expulsait l’ancien propriétaire et mettait à sa place le vétéran. Celui-ci, forcé de se roidir contre les réclamations de son prédécesseur, ne se sentait fort que de la bienveillance du pouvoir qui l’appuyait. Il était donc dans les meilleures dispositions imaginables pour se conserver cette bienveillance au prix d’un dévouement sans bornes.\par
Cette combinaison d’effets et de causes plaisait aux politiques de l’antiquité. Leur sagesse l’approuvait, et, si les gens qui avaient à en souffrir pouvaient s’en plaindre, la morale publique acceptait, sans plus de scrupules, un système jugé utile à la solidité de l’État, un système consacré par les lois, et qui de plus avait pour excuse d’avoir été toujours et partout pratiqué par les nations dont un esprit cultivé pouvait invoquer les exemples.\par
Dès le temps des premiers Césars, on crut devoir apporter quelques modifications à la simplicité brutale de ce mécanisme. L’expérience avait prouvé que les colonisations de vétérans italiotes, asiatiques ou même gaulois méridionaux, ne mettaient pas suffisamment les frontières du nord à l’abri des incursions de voisins trop redoutables. Les familles romanisées reçurent l’ordre de s’éloigner des limites extrêmes, puis l’on offrit à tous les Germains cherchant fortune, et le nombre n’en était pas médiocre, la libre disposition des terrains restés vacants, le titre un peu oppressif quelquefois d’amis du peuple romain et, ce qui semblait promettre davantage, l’appui des légions contre les agressions éventuelles des ennemis de l’empire.\par
 Ce fut ainsi que, par la propre volonté, par le choix libre du gouvernement impérial, des nations teutoniques furent installées tout entières sur les terres romaines. On espéra de si grands avantages de cette manière de procéder que bientôt l’on joignit aux aventuriers les prisonniers de guerre. Quand une tribu de Germains était vaincue, on l’adoptait, on en composait une nouvelle bande de gardes-frontières, en ayant soin seulement de la dépayser.\par
Les autres barbares n’assistaient pas sans jalousie au spectacle d’une situation si favorisée. Sans même avoir besoin de se rendre compte des avantages supérieurs auxquels ces Romains factices pouvaient prétendre, ni apercevoir d’une manière bien nette les sphères brillantes où cette élite disposait des destinées de l’univers, ils voyaient leurs pareils pourvus de propriétés depuis longtemps en bon état de culture ; ils les voyaient en contact avec un commerce opulent, et en jouissance de ce que les perfectionnements sociaux avaient pour eux de plus enviable. C’en était assez pour que les agressions redoublassent d’impétuosité, de fréquence. Obtenir des terres impériales devint le rêve obstiné de plus d’une tribu, lasse de végéter dans ses marais et dans ses bois.\par
Mais, d’un autre côté, à mesure que les attaques devenaient plus rudes, la situation des Germains colonisés était aussi plus précaire. Des rivaux les trouvaient trop riches ; eux, ils se sentaient trop peu tranquilles. Ils étaient souvent exposés à la tentation de tendre la main à leurs frères au lieu de les combattre, et, pour en obtenir la paix, de se liguer avec eux contre les vrais Romains, placés derrière leur douteuse protection.\par
L’administration impériale germanisée jugea le péril ; elle en comprit toute l’étendue, et, afin de le détourner en redoublant le zèle des auxiliaires, elle ne trouva rien de mieux que de leur proposer les modifications suivantes dans leur état légal :\par
Ils ne seraient plus considérés uniquement comme des colons, mais bien comme des soldats en activité de service. Conséquemment, à tous les avantages dont ils étaient déjà en possession, et qui ne leur seraient point retirés, ils verraient s’ajouter encore celui d’une solde militaire. Ils deviendraient partie intégrante des armées, et leurs chefs obtiendraient les grades, les honneurs et la paye des généraux romains.\par
Ces offres furent acceptées avec joie, comme elles devaient l’être. Ceux qui en furent les objets ne songèrent plus qu’à exploiter de leur mieux la faiblesse d’un empire qui en était réduit à de tels expédients. Quant aux tribus du dehors, elles n’en devinrent que plus possédées du désir d’obtenir des terres romaines, de devenir soldats romains, gouverneurs de province, empereurs. Il ne s’agissait plus désormais, dans la société civilisée, telle que le cours des événements l’avait faite, que d’antagonismes et de rivalités entre les Germains du dedans et ceux du dehors.\par
La question ainsi posée, le gouvernement fut entraîné à étendre sans fin le réseau des colonisations, et bientôt de frontières qu’elles étaient elles devinrent aussi intérieures. De gré ou de force, les peuplades chargées de la défense des limites, et qu’en cas de péril on était souvent contraint d’abandonner à elles-mêmes, ces peuplades faisaient de fréquentes transactions avec les assaillants. Il fallait bien que l’empereur finît par ratifier ces accords dont sa faiblesse était la première cause. De nouveaux soldats étaient enrôlés à la solde de l’État ; il leur fallait trouver les terres qu’on leur avait promises. Souvent mille considérations s’opposaient à ce qu’on les leur assignât sur des frontières qui, d’ailleurs, étaient encombrées de leurs pareils. Puis, ce n’était pas là qu’on avait chance de rencontrer des propriétaires maniables, disposés à se laisser déposséder sans résistance. On chercha cette espèce débonnaire où on savait qu’elle était, dans toutes les provinces intérieures. Par une sorte d’immunité résultant de la suprématie d’autrefois, l’Italie fut exceptée aussi longtemps que possible de cette charge ; mais on ne se gêna pas avec la Gaule. On mit des Teutons à Chartres ; Bayeux vit des Bataves ; Coutances, le Mans, Clermont furent entourés de Suèves ; des Alains et des Taïfales occupèrent les environs d’Autun et de Poitiers ; des Franks s’installèrent à Rennes \footnote{Dans l’île de Bretagne, les colons barbares, fort nombreux, ne portaient pas le nom ordinaire de \emph{læti}, on les appelait \emph{gentiles.} (Palsgrave, \emph{Rise and Progress of the English Commonwealth}, t. I, p. 355.)}. Les Gaulois romanisés étaient gens de bonne composition ; ils avaient appris la soumission avec les collecteurs impériaux. À plus forte raison n’avaient-ils rien à opposer au Burgonde ou au Sarmate, présentant d’un ton péremptoire l’invitation légale de céder la place.\par
Il ne faut pas oublier une minute que ces revirements de propriété étaient, suivant les notions romaines, parfaitement légitimes. L’État et l’empereur, qui le représentait, avaient le droit de tout faire au monde ; il n’existait pas de moralité pour eux : c’était le principe sémitique. Du moment donc que celui qui donnait avait le droit de donner, le barbare qui bénéficiait de cette concession avait un titre parfaitement régulier à prendre. Il se trouvait du jour au lendemain propriétaire, d’après la même règle dont avaient pu se réclamer jadis les Celtes romanisés eux-mêmes par la volonté du souverain.\par
Vers la fin du IV\textsuperscript{e} siècle, presque toutes les contrées romaines, sauf l’Italie centrale et méridionale, car la vallée du Pô était déjà concédée, possédaient un nombre notable de nations septentrionales colonisées, recevant la plupart une solde, et connues officiellement sous le nom de troupes au service de l’empire, avec l’obligation, d’ailleurs assez mal remplie, de se comporter paisiblement. Ces guerriers adoptaient rapidement les mœurs et les habitudes qu’ils voyaient pratiquer par les Romains ; ils se montraient fort intelligents, et, une fois pliés aux conséquences de la vie sédentaire, ils devenaient la partie la plus intéressante, la plus sage, la plus morale, la plus facilement chrétienne des populations.\par
Mais jusque-là, c’est-à-dire jusqu’au V\textsuperscript{e} siècle, toutes ces colonisations, tant intérieures que frontières, n’avaient amené les Germains sur les terres de l’empire que par groupes. L’amas immense accumulé avec les siècles dans le nord de l’Europe n’avait fait encore que ruisseler par jets comparativement minces à travers les digues de la romanité. Tout à coup il les effondra, et précipita toutes ses masses, fit rouler et écumer toutes ses vagues sur cette misérable société que des échappées de son génie faisaient seules vivre depuis trois siècles, et qui enfin ne pouvait plus aller. Il lui fallait une refonte complète.\par
La pression exercée par les Finnois ouraliens, par les Huns blancs et noirs, par des populations énormes où se présentaient à peu près purs, à tous les degrés de combinaisons, les éléments slaves, celtiques, arians, mongols ; cette pression était devenue si violente que l’équilibre toujours chancelant des États teutoniques avait été complètement renversé dans l’Est. Les établissements gothiques s’étant écroulés, les débris de la grande nation d’Hermanaric descendirent sur le Danube, et formulèrent à leur tour la demande ordinaire : des terres romaines, le service militaire et une solde.\par
Après des débats assez longs, comme ils n’obtenaient pas ce qu’ils voulaient, ils se décidèrent par provision à le prendre. Faisant une pointe depuis la Thrace jusqu’à Toulouse, ils s’abattirent comme une nuée de faucons sur le Languedoc et l’Espagne du nord, puis laissèrent les Romains parfaitement libres de les chasser, s’ils pouvaient.\par
Ceux-ci n’eurent garde d’essayer. La manière dont les Visigoths venaient de s’installer était un peu irrégulière ; mais une patente impériale ne tarda pas à réparer le mal, et de ce moment les nouveaux venus furent aussi légitimement établis sur les terres qu’ils avaient prises que les autres sujets dans les leurs. Les Franks et les Burgondes n’avaient pas attendu ce bon exemple pour se donner d’abord, se faire concéder ensuite des avantages pareils ; de sorte que vingt nations du nord, outre les anciennes tribus gardes-frontières, disparues sous cette épaisse alluvion, se virent dès lors acceptées et adoptées par les matricules militaires sur tout le territoire européen. Leurs chefs étaient consuls et patrices. On eut le patrice Théodorik et le patrice Khlodowig \footnote{Ces deux chefs devaient leurs titres romains à l’empereur Anastase, qui de fait n’était rien en Occident ; mais on verra tout à l’heure par quelle fiction les rois barbares tenaient à le considérer comme empereur national.}.\par
Maîtres absolus de tout, les Germains établis dans l’empire pouvaient désormais tout faire, assurés que leurs caprices seraient des lois irrésistibles. Deux partis s’offraient à eux : ou bien rompre avec les habitudes et les traditions conservées par leurs devanciers de même sang ; abolir la cohésion des territoires, et former de tous ces débris un certain nombre de souverainetés distinctes, libres de se constituer suivant les convenances de l’âge qui commençait ; ou bien rester fidèles à l’œuvre consacrée par les soins de tant d’empereurs issus de la race nouvelle, mais en modifiant cette œuvre par un certain appoint d’anomalies devenues indispensables.\par
Dans ce dernier système, l’organisation d’Honorius restait sauve quant à l’essentiel. La romanité, c’est-à-dire, suivant la ferme conviction des temps, la civilisation, poursuivait son cours.\par
Les barbares reculèrent devant l’idée de nuire à une chose si nécessaire ; ils persistèrent dans le rôle conservateur, adopté par les empereurs d’origine barbare, et choisirent le second parti ; ils ne découpèrent point le monde romain en autant de parcelles qu’ils étaient de nations. Ils le laissèrent bien entier, et, au lieu de s’en faire les destructeurs en en réclamant la possession, ils n’en voulurent avoir que l’usufruit.\par
Pour mettre cette idée à exécution, ils inaugurèrent un système politique d’une apparence extrêmement complexe. On y vit fonctionner tout à la fois et des règles empruntées à l’ancien droit germanique, et des maximes impériales, et des théories mixtes formées de ces deux ordres de conceptions.\par
Le roi, le konungr, car il ne s’agissait nullement ici ni du drottinn, ni du graff, mais bien du chef de guerre, conducteur d’invasion et hôte des guerriers, revêtit un double caractère. Pour les hommes de sa race, il devint un général perpétuel \footnote{Le droit de \emph{commendatio} se maintint si longtemps chez les Anglo-Saxons, la faculté de choisir librement son chef, se perdit de très bonne heure chez les Franks. Les leudes, antrustions ou fidèles, étaient tenus de rester attachés à leur roi et ne pouvaient, sans encourir des recherches légales, passer au service d’un autre. (Savigny, \emph{D. Rœm. Recht im Mittelalt.}, t. I, p. 186.) Cette modification importante à la liberté germanique avait eu lieu sous l’influence de la loi romaine.} ; pour les Romains, il fut un magistrat institué sous l’autorité de l’empereur. Vis-à-vis des premiers, ses succès avaient cette conséquence d’enrôler et de conserver plus de combattants autour de ses drapeaux ; vis-à-vis des seconds, d’étendre les limites géographiques de sa juridiction, D’ailleurs, le konungr germanique ne se considérait nullement comme le souverain des contrées tombées en sa puissance. La souveraineté n’appartenait qu’à l’empire ; elle était inaliénable et incommunicable ; mais comme magistrat romain, agissant au moyen d’une délégation du pouvoir suprême, le konungr disposait des propriétés avec une liberté absolue. Il usait pleinement du droit d’y coloniser ses compagnons, ce qui était simple aux yeux de tout le monde. Il leur distribuait, suivant les coutumes de sa nation, une partie des terres de rapport, et accordait ainsi l’usage romain avec l’usage germanique ; il organisait de la sorte un système mixte de tenures nouvelles des bénéfices réversibles en vertu de principes germaniques et de principes romains, ce qu’on appelait et ce qu’on appelle encore des féods ; ou même il constituait à son gré des terres allodiales, avec cette différence fondamentale, cependant, qui distinguait complètement ces concessions des odels anciens, que c’était la volonté royale qui les faisait, et non pas l’action libre du propriétaire \footnote{Ce fut probablement comme une conséquence de l’importation des alleux que certains possesseurs de terres furent exemptés par les rois du pouvoir des comtes. C’était un souvenir de l’ancienne liberté de l’Arian dans son odel. Mais cette immunité n’était jamais complète, et le possesseur de l’alleu fut toujours responsable devant le tribunal commun, devant le comte, des crimes de meurtre, de rapt et d’incendie. (Savigny, \emph{Das Rœm. Recht im Mittelalt}, t. I, p. 278.)}. Quoi qu’il en soit, féod ou odel, le chef qui les donnait à ses hommes avait sur la province le droit de propriété, ou plutôt de libre disposition, comme délégué de l’empereur, mais point le haut domaine.\par
Telle était la situation des Mérowings dans les Gaules. Lorsqu’un d’eux était à son lit de mort, il ne pouvait lui venir en idée de donner des provinces à son fils, puisqu’il n’en possédait pas lui-même. Il établissait donc la répartition de son héritage sur des principes tout autres. En tant que chef germanique, il ne disposait que du comman­dement d’un nombre plus ou moins considérable de guerriers, et de certaines propriétés rurales qui lui servaient à entretenir cette armée. C’étaient cette bande et ces domaines qui lui donnaient la qualité de roi, et il ne l’avait pas d’ailleurs. En tant que magistrat romain, il n’avait que le produit des impôts perçus dans les différentes parties de sa juridiction, d’après les données du cadastre impérial.\par
En face de cette situation, et voulant égaliser de son mieux les parts de ses enfants, le testateur assignait à chacun d’eux une résidence entourée d’hommes de guerre appartenant, autant que possible, à une même tribu. C’était là le domaine germanique, et il eût suffi d’une métairie et d’une vingtaine de champions pour autoriser le jeune Mérowing qui n’eût pas obtenu davantage à porter le titre de roi.\par
Quant au domaine romain, le chef mourant le fractionnait avec bien moins de scrupule encore, puisqu’il ne s’agissait que de valeurs mobilières. Il distribuait donc par portions diverses, à plusieurs héritiers, les revenus des douanes de Marseille, de Bordeaux ou de Nantes.\par
Les Germains n’avaient pas pour but principal de sauver ce qu’on nomme l’unité romaine. Ce n’était là à leurs yeux qu’une manière de maintenir la civilisation, et c’est pourquoi ils s’y soumettaient. Leurs efforts, pour ce but méritoire, furent des plus extraordinaires, et dépassèrent même ce qu’on avait pu observer dans ce sens chez un grand nombre d’empereurs. Il semblerait que depuis l’établissement en masse au sein de la romanité, la barbarie se repentit d’avoir donné trop peu d’attention aux niaiseries mêmes de l’état social qu’elle admirait. Tous les littérateurs étaient assurés de l’accueil le plus honorable à la cour des rois vandales, goths, franks, burgondes ou longobards. Les évêques, ces dépositaires véritables de l’intelligence poétique de l’époque, n’écrivaient pas que pour leurs moines. La race des conquérants elle-même se mit à manier la plume, et Jornandès, Paul Warnefrid, l’anonyme de Ravenne, bien d’autres dont les noms et les œuvres ont péri, témoignaient assez du goût de leur race pour l’instruction latine. D’un autre côté, les connaissances plus particulièrement nationales ne tombaient pas en oubli. On taillait des lunes chez le roi Hilpérik \footnote{La traduction mœso-gothique des évangiles par Ulfila est du IV\textsuperscript{e} siècle.}, qui, inquiet des imperfections de l’alphabet romain, occupait ses moments perdus à le réformer. Les poèmes du Nord se maintenaient en honneur, et les exploits des aïeux, fidèlement chantés par les générations nouvelles, servaient à prouver que ces dernières n’avaient point abdiqué les qualités énergiques de leur race \footnote{Théodorik III et ses successeurs promulguèrent plusieurs lois dans le but de protéger les monuments de Rome contre la destruction. Ce n’étaient pas les barbares qui les attaquaient, mais les Romains, soit par le zèle religieux, soit pour y prendre des matériaux de construction. – Les plus grands ravages ont été faits sous Constant II (Clarac, \emph{Manuel de l’histoire de l’art chez les anciens}, part. II, p. 857.) – Les Romains recherchaient beaucoup les statues de marbre, afin d’en faire de la chaux. Les rois visigoths et les papes, malgré les prescriptions les plus sévères, ne purent empêcher le plus grand nombre des objets d’art de périr ainsi. (\emph{Ouvr. cité}, p. 857.) – Athalaric s’efforça de réorganiser l’école de droit de Rome. (Cassiod., Var., IX, 31.) – Les rois visigoths, non contents de défendre la destruction des monuments, attribuèrent même des fonds à leur entretien. (Clarac, \emph{ouvr. cité}, part. II, p. 857.)}.\par
En même temps, les peuples germaniques, imitant ce qu’ils observaient chez leurs sujets, s’occupèrent activement de régulariser leur propre législation, suivant les nécessités de l’époque et du milieu où ils se trouvaient placés. Si leur attention fut mise en éveil par le travail d’autrui, ce ne fut nullement d’une manière servile, ni dans la méthode ni dans les résultats, que procéda leur intelligence.\par
S’étant imposé l’obligation de respecter et, par conséquent, de reconnaître les droits des Romains, ce leur fut une raison de se rendre un compte fort exact des leurs, et d’établir une sorte de concordance ou mieux de parallélisme entre les deux systèmes qu’ils avaient l’intention de faire vivre en face l’un de l’autre. Il résulta de cette dualité, si franchement acceptée et même cultivée, un principe d’une haute importance et dont l’influence ne s’est jamais complètement perdue. Ce fut de reconnaître, de constater, de stipuler qu’il n’existait pas de distinction organique entre les diverses tribus, les diverses nations venues du nord, en quelque lieu qu’elles fussent établies et quelques noms qu’elles pussent porter, du moment qu’elles étaient germaniques \footnote{C’était agir conformément aux indications de la race, de la langue, de la loi civile, et Palsgrave a dit avec vérité : « Like their various languages which are in truth but dialect of « one mother tongue, so their laws are but modifications of one primeval code... even now « we can mark the era when the same principles and doctrines were recognised at Upsala « and at Toledo, in Lombardy and in England. ) » (\emph{Ouvr. cité}, t. I, p. 3.)}. À la faveur de certaines alliances, un petit nombre de groupes plus qu’à demi slaves parvinrent à se faire accepter dans cette grande famille, et servirent plus tard de prétexte, d’inter­médiaire pour y rattacher, avec moins de fondement encore, plusieurs de leurs frères. Mais cette extension n’a jamais été bien sentie ni bien acceptée par l’esprit occidental. Les Slaves lui sont aussi étrangers que les peuples sémitiques de l’Asie antérieure, avec lesquels il est lié à peu près de la même façon par les populations de l’Italie et de l’Espagne.\par
On le voit, le génie germanique était aussi généralisateur que celui des rations antiques l’était peu. Bien qu’il partît d’une base en apparence plus étroite que les institutions hellénistiques, romaines ou celtiques, et que les droits de l’homme libre, pris individuellement, fussent pour lui ce qu’étaient les droits de la cité pour les autres, la notion qu’il en avait, et qu’il étendait avec une si superbe imprévoyance, le conduisit infiniment plus loin qu’il ne pensait lui-même aller. Rien de plus naturel : l’âme de ce droit personnel, c’était le mouvement, l’indépendance, la vie, l’appropriation facile à toutes les circonstances ambiantes ; l’âme du droit civique, c’était la servitude, comme sa suprême vertu était l’abnégation.\par
Malgré le profond désordre ethnique au milieu duquel l’Arian Germain appa­raissait, et bien que son propre sang ne fût pas absolument homogène, il mettait tous ses soins à circonscrire, à préciser deux grandes catégories idéales dans lesquelles il enfermait toutes les masses soumises à son arbitrage ; en principe, il ne reconnaissait que la romanité et la barbarie. C’était là le langage consacré. Il s’efforçait d’ajuster du moins mal possible ces deux éléments désormais constitutifs de la société occidentale, et dont le travail des siècles devait arrondir les angles, adoucir les contrastes, amener la fusion. Qu’un tel plan, que les germes qui y étaient déposés fussent supérieurs en fécondité et préparassent pour l’avenir de plus beaux fruits que les plus élégantes théories de la Rome sémitique, il serait oiseux de le discuter. Dans cette dernière organisation, on l’a pu constater, mille peuples rivaux, mille coutumes ennemies, mille débris de civilisations discordantes se faisaient une guerre intestine. Pas la moindre tendance n’existait à sortir d’une confusion si monstrueuse, sans courir le danger de tomber dans une autre plus horrible encore. Pour tous liens, le cadastre, les règlements niveleurs du fisc, l’impartialité négative de la loi ; mais rien de supérieur qui préparât, qui forçât l’avènement d’une moralité nouvelle, d’une communauté de vues, d’une tendance unanime parmi les hommes, ni qui annonçât cette civilisation sagace qui est la nôtre, et que nous n’aurions jamais obtenue si la barbarie germanique n’en avait apporté les plus précieuses greffes et n’avait pris la charge de les faire réussir sur la tige débile de la romanité, passive, dominée, contrainte, jamais sympathique.\par
J’ai rappelé quelquefois dans le cours de ces pages, et ce n’était pas inutilement, que les grands faits que je décris, les importantes évolutions que je signale, ne s’opèrent nullement par suite de la volonté expresse et directe des masses ou de tels ou tels personnages historiques. Causes et effets, tout se développe au contraire le plus ordinairement à l’insu ou à l’encontre des vues de ceux qui y contribuent. Je ne m’occupe nullement de retracer l’histoire des corps politiques, ni les actions belles ou mauvaises de leurs conducteurs. Tout entier attentif à l’anatomie des races, c’est uniquement de leurs ressorts organiques que je tiens compte et des conséquences prédestinées qui en résultent, ne dédaignant pas le reste, mais le laissant à l’écart lorsqu’il ne sert pas à expliquer le point en discussion. Si j’approuve ou si je blâme, mes paroles n’ont qu’un sens comparatif et, pour ainsi dire, métaphorique. En réalité, ce n’est pas un mérite moral pour les chênes que d’élever à travers les siècles leurs fronts majestueux, couronnés d’un vert diadème, comme ce n’est pas non plus une honte pour les herbes des gazons de se faner en quelques jours. Les uns et les autres ne font que tenir leurs places dans les séries végétales, et leur puissance ou leur humilité concourent également aux desseins du Dieu qui les a faits. Mais je ne me dissimule pas non plus que la libre action des lois organiques, auxquelles je borne mes recherches, est souvent retardée par l’immixtion d’autres mécanismes qui lui sont étrangers. Il faut passer sans étonnement par-dessus ces perturbations momentanées, qui ne sauraient changer le fond des choses. À travers tous les détours où les causes secondes peuvent entraîner les conséquences ethniques, ces dernières finissent toujours par retrouver leurs voies. Elles y tendent imperturbablement et ne manquent jamais d’y arriver, C’est ainsi qu’il en advint pour le sentiment conservateur des Germains envers la romanité. Il fut en vain combattu et souvent obscurci par les passions qui lui faisaient escorte ; à la fin il accomplit sa tâche. Il se refusa à la destruction de l’empire aussi longtemps que l’empire représenta un corps de peuples, un ensemble de notions sociales différentes de la barbarie. Il fut si ferme dans cette volonté et si inexpugnable, qu’il la maintint même pendant l’espace de quatre siècles où il se vit forcé de supprimer l’empereur dans l’empire.\par
Cette situation d’un État despotique subsistant sans avoir de tête n’était pas, du reste, aussi étrange qu’elle le peut sembler d’abord. Dans une organisation comme la romaine, où l’hérédité monarchique n’avait jamais existé et, où l’élection du chef suprême, indifféremment accomplie par le prédécesseur, par le sénat, par le peuple ou par une des armées, puisait sa validité dans le seul fait de sa maintenue ; en face d’un pareil ordre de choses, ce n’est pas la régularité des successions au trône qui peut faire connaître que le corps politique continue de vivre, encore bien moins le corps social. Le seul criterium admissible, c’est l’opinion des contemporains à cet égard. Et il n’importe pas que cette opinion soit fondée sur des faits spéciaux, comme, par exemple, la continuation d’institutions séculaires, chose de tout temps inconnue dans une société en perpétuelle refonte, ou bien la résidence du pouvoir continuée dans une même capitale, ce qui n’avait pas eu lieu davantage ; il suffit que la conviction existant sur ce sujet résulte de l’enchaînement d’idées, même transitoires et disparates, mais qui, s’engendrant les unes des autres, créent, malgré la rapidité de leur succession, une impression de durée pour le milieu assez vague dans lequel elles se développent, meurent et sont incessamment remplacées.\par
C’était l’état normal dans la romanité, et voilà pourquoi lorsque Odoacre eut déclaré le personnage d’un empereur d’Occident inutile, personne ne pensa, non plus que lui, que par suite de cette mesure l’empire d’Occident cessât d’être. Seulement, on jugea qu’une nouvelle phase commençait ; et de même que la société romaine avait été gouvernée d’abord par des chefs que ne désignait aucun titre, qu’elle en avait eu ensuite qui s’étaient décorés de leur nom de César, d’autres qui avaient établi une distinction entre les Césars et les Augustes, et, au lieu d’imposer une direction unique au corps politique, lui en avaient fourni deux, puis quatre, de même on s’accommoda de voir l’empire se passer d’un représentant direct, relever très superficiellement, et uniquement pour la forme, du trône de Constantinople, et obéir sans se dissoudre, et en restant toujours l’empire d’Occident, à des magistrats germaniques, qui, chacun dans les pays de son ressort, appliquaient aux populations les lois spéciales instituées jadis à leur usage par la jurisprudence romaine. Odoacre n’avait donc accompli qu’une pure révolution de palais, beaucoup moins importante qu’elle n’en avait l’air ; et la preuve la plus palpable qu’on en puisse donner, c’est la conduite que tint plus tard Charlemagne et la façon dont la restauration du porte-couronne impérial s’accomplit en sa personne.\par
Le roi des Hérules avait déposé le fils d’Oreste en 475 ; Charlemagne fut intronisé, et termina l’interrègne en 801. Les deux événements étaient séparés par une période de près de quatre siècles, et de quatre siècles remplis d’événements majeurs, bien capables d’effacer de la mémoire des hommes tout souvenir de l’ancienne forme de gouvernement. Quelle est, d’ailleurs, l’époque où il ne serait pas insensé de vouloir reprendre un ordre de choses qui aurait été interrompu depuis quatre cents ans ? Si donc Charlemagne le put faire, c’est qu’en réalité il ne ressuscitait pas le fond ni même la forme des institutions, c’est qu’il ne faisait que rétablir un détail qu’on avait pu négliger un temps sans péril, et qu’on reprenait sans anachronisme.\par
L’empire, la romanité, s’étaient constamment soutenus en face de la barbarie et par ses soins. Le couronnement du fils de Pépin ne faisait que lui rendre un des rouages qu’avec tant d’autres, disparus pour toujours, elle avait vus jadis fonctionner dans son sein. L’incident était remarquable, mais il n’avait rien de vital ; c’est ce que montre bien l’examen des motifs qui avaient prolongé si longtemps l’interrègne.\par
Après avoir jugé raisonnable, autrefois, que le chef de la société romaine fût issu d’une famille latine, on avait consenti bientôt à le prendre dans une partie quelconque de l’Italie, puis enfin et exclusivement dans les camps, et alors on ne s’était plus enquis de son origine. Cependant il était toujours resté convenu, et sur ce point le bon sens ne pouvait guère faiblir, que l’empereur devait avoir au moins les formes extérieures des populations qu’il régissait, porter un des noms familiers à leurs oreilles, s’habiller comme eux et parler la langue courante, la langue des décrets et des diplômes, tant bien que mal. À l’époque d’Odoacre, les distinctions extérieures entre les vainqueurs et les vaincus étaient encore trop accusées pour que la violation de ces règles ne fît pas scandale aux yeux de ceux-là même qui auraient pu vouloir l’essayer à leur profit.\par
Pour les chefs germaniques, pour les rois sortis du sang des Amâles ou des Mérowings, se faire instituer patrices et consuls, c’étaient là des ambitions permises et même nécessaires : le gouvernement des peuples était à ce prix. Mais, outre que la prise de possession de la pourpre augustale par un chef barbare, vêtu et vivant suivant les usages du Nord, entouré de sa truste, dans un palais de bois, aurait été passible de ridicule, l’ambitieux mal inspiré qui en eût fait l’essai aurait éprouvé la difficulté la plus grande à se faire reconnaître dans sa dignité suprême par de nombreux adversaires, tous ses rivaux, tous égaux à lui, ou croyant l’être, par l’illustration, tous à peu près aussi forts que lui. La coalition de mille vanités, de mille intérêts blessés aurait eu bientôt fait de le rabattre au rang commun, et peut-être au-dessous.\par
Pénétrés de cette évidence, les plus puissants monarques germaniques ne voulurent pas en essuyer les périls \footnote{Cependant on ne peut nier que la tentation de le faire n’existât pour eux très vive et qu’ils ne s’y abandonnassent quelquefois en partie. Klodowig, au dire de Grégoire de Tours (II, 38), s’était même fait donner le titre d’Auguste. Thédorik le Grand joua même le rôle de collègue d’Anastase. Mais ce furent plutôt des prétentions que des réalités, et ces deux circonstances ne sont guère que des curiosités historiques, tant elles furent peu suivies d’effets.}. Ils imaginèrent quelque temps le biais de donner à quelqu’un de leurs domestiques romains cette dignité qu’ils n’osaient revêtir eux-mêmes, et, quand le malheureux mannequin faisait mine d’essayer un peu d’indépen­dance, un mot, un geste, le faisait disparaître.\par
Tous les avantages semblaient se réunir dans cette combinaison. En dominant l’empereur on dominait l’empire, et cela sans se donner les apparences d’une usurpa­tion trop osée ; en un mot, c’était un expédient bien imaginé. Par malheur, comme tout expédient, il s’usa vite. La vérité perçait trop facilement sous le mensonge. Le Mérowing ne se souciait pas plus de reconnaître pour son souverain le serviteur d’Odoacre qu’Odoacre lui-même. Chacun protesta, chacun repoussa cette contrainte, puis chacun, ayant consulté ses forces, se rendit justice en silence, s’exécuta modeste­ment : l’interrègne fut proclamé, et l’on attendit que l’équilibre des forces eût cessé pour reconnaître à celui qui bien décidément l’emporterait le droit de recommencer la série des empereurs.\par
Ce ne fut qu’au bout de quatre cents ans que toutes les difficultés se trouvèrent aplanies. Au début de cette période nouvelle, les facilités les plus complètes apparurent à tous les yeux. La plupart des nations germaniques s’étaient laissé affaiblir, sinon incorporer par la romanité ; plusieurs même avaient cessé d’exister comme groupes distincts. Les Visigoths, appariés aux Romains de leurs territoires, ne conservaient plus entre eux et leurs sujets aucune distinction légale qui rappelât une inégalité ethnique. Les Longobards maintenaient une situation plus distincte, d’autres encore faisaient de même ; toutefois il était incontestable que le monde barbare n’avait plus qu’un seul représentant sérieux dans l’empire, et ce représentant, c’était la nation des Franks, à laquelle l’invasion des Austrasiens venait de rendre un degré d’énergie et de puissance évidemment supérieur à celui de toutes les autres races parentes. Le problème de la suprématie était donc résolu au profit de ce peuple.\par
Puisque les Franks dominaient tout, puisque en même temps le mariage de la barbarie et de la romanité était assez avancé déjà pour que les contrastes d’autrefois fussent devenus moins choquants, l’empire se retrouvait en situation de se donner un chef. Ce chef pouvait être un Germain, Germain de fait et de formes ; cet élu ne devait être qu’un Frank ; parmi les Franks, qu’un Austrasien, que le roi des Austrasiens, et donc que Charlemagne. Ce prince, acceptant tout le passé, se porta pour le successeur des empereurs d’Orient, dont le sceptre venait de tomber en quenouille, ce que la coutume d’Occident ne pouvait admettre suivant lui. Voilà par quel raisonnement il restaura le passé. D’ailleurs, les acclamations du peuple romain et les bénédictions de l’Église ne lui refusèrent pas leur concours \footnote{Les politiques du temps ne voulurent pas même avouer que le nouvel empereur restaurait un trône ancien. Ils prétendirent qu’il succédait, non pas à Augustule, mais à l’empereur d’Orient, Constantin V. Pendant tout le temps de l’interrègne, on avait, en effet, admis cette théorie, que le souverain siégeant à Constantinople était devenu le chef nominal de la romanité entière. Son pouvoir se bornait à accorder les investitures, quand on les lui demandait. Lorsque Charlemagne voulut prendre la pourpre, on rompit avec cette fiction, en lui en substituant une autre : ce fut d’imaginer que, par l’avènement d’Irène, l’empire d’Orient étant tombé en quenouille, celui d’Occident ne pouvait suivre le même sort, parce que la loi des Saliens s’y opposait, comme si la loi des Saliens eût eu quelque chose à dire dans un cas d’hérédité romaine, qui échappait même légalement aux règles de la jurisprudence civile. Il est, du reste, à remarquer que c’est ici la première application qui fut faite de la doctrine de l’inaptitude des femmes à succéder à la couronne de France et en ce cas de l’appel à la loi régissant la tenure du domaine salique. On a contesté à tort qu’il y eût corrélation réelle entre ces deux points.}.\par
Jusqu’à lui la barbarie avait fidèlement poursuivi son système de conservation à l’égard du monde romain. Tant qu’elle exista dans sa véritable et native essence, elle ne se départit pas de cette idée. Depuis comme avant l’arrivée des premiers grands peuples teutoniques, jusqu’à l’avènement des âges moyens vers le dixième siècle, c’est-à-dire pendant une période de sept cents ans environ, la théorie sociale, plus ou moins clairement développée et comprise, demeura celle-ci : la romanité, c’est l’ordre social. La barbarie n’est qu’un accident, accident vainqueur et dirigeant, à la vérité, mais enfin accident, et, comme tel, d’une nature transitoire.\par
Si l’on avait demandé aux sages de cette époque lequel des deux éléments devait survivre à l’autre, absorber l’autre, l’anéantir, incontestablement ils auraient répondu et ils répondaient effectivement en célébrant l’éternité du nom romain. Cette conviction était-elle erronée ? Oui, en ceci qu’on se représentait l’image incorrecte d’un avenir trop semblable au passé et beaucoup trop rapproché ; mais, au fond, elle n’était erronée qu’à la façon des calculs de Christophe Colomb par rapport à l’existence du nouveau monde. Le navigateur génois se trompait dans toutes ses supputations de temps, d’éloignement et d’étendue. Il se trompait sur la nature de ses découvertes à venir. Le globe terrestre n’était pas si petit qu’il le supposait ; les terres auxquelles il allait aborder étaient plus loin de l’Espagne et plus vastes qu’il ne l’imaginait ; elles ne faisaient point partie de l’empire chinois, et l’on n’y parlait pas l’arabe. Tous ces points étaient radicalement faux ; mais cette série d’illusions ne détruisait pas l’exactitude de l’assertion principale. Le protégé des rois catholiques avait raison de soutenir qu’il y avait un pays inconnu dans l’ouest.\par
De même aussi, la pensée générale de la romanité était dans le faux en considérant le mode de culture dont elle conservait les lambeaux comme le trésor et le dernier mot du perfectionnement possible ; elle l’était encore en ne voyant dans la barbarie qu’une anomalie destinée à promptement disparaître ; elle l’était bien davantage en annonçant comme prochaine la réapparition complète d’un ordre de choses qu’on s’imaginait admirable ; et cependant, malgré toutes ces erreurs si considérables, malgré ces rêves si rudement bafoués par les faits, la conscience publique devinait juste en ceci que, la romanité étant l’expression de masses humaines infiniment plus imposantes par leur nombre que la barbarie, cette romanité devait, à la longue, user sa dominatrice comme les flots usent le rocher, et lui survivre. Les nations germaniques ne pouvaient éviter de se dissoudre un jour dans les détritus accumulés et puissants des races qui les entouraient, et leur énergie était condamnée à s’y éteindre. Voilà ce qui était la vérité ; voilà ce que l’instinct révélait aux populations romaines. Seulement, je le répète, cette révolution devait s’opérer avec une lenteur dont les imaginations humaines n’aiment pas à mesurer les ennuis, vu la difficulté qu’elles éprouvent d’ailleurs à se soutenir au milieu d’espaces un peu larges. Il faut ajouter encore qu’elle ne pouvait jamais être si radicale que de ramener la société à son point de départ sémitisé. Les éléments germaniques devaient s’absorber, mais non pas disparaître à ce point.\par
Ils s’absorbent néanmoins, et d’une façon constante désormais. Leur décomposition au sein des autres éléments ethniques est bien facile à suivre. Elle fournit la raison d’être de tous les mouvements importants des sociétés modernes, ainsi qu’on en juge aisément en examinant les différents ordres de faits qui lui servent à se manifester.\par
Il a déjà été établi précédemment que toute société se fondait sur trois classes primitives, représentant chacune une variété ethnique : la noblesse, image plus ou moins ressemblante de la race victorieuse ; la bourgeoisie, composée de métis rappro­chés de la grande race ; le peuple, esclave, ou du moins fort déprimé, comme appartenant à une variété humaine inférieure, nègre dans le sud, finnoise dans le nord.\par
Ces notions radicales furent brouillées partout de très bonne heure. Bientôt on connut plus de trois catégories ethniques ; partant, beaucoup plus de trois subdivisions sociales. Cependant l’esprit qui avait fondé cette organisation est toujours resté vivant ; il l’est encore ; il ne s’est jamais donné de démenti à lui-même, et il se montre aujourd’hui aussi sévèrement logique que jamais.\par
Du moment que les supériorités ethniques disparaissent, cet esprit ne tolère pas longtemps l’existence des institutions faites pour elles et qui leur survivent. Il n’admet pas la fiction. Il abroge d’abord le nom national des vainqueurs, et fait dominer celui des vaincus ; puis il met à néant la puissance aristocratique. Tandis qu’il détruit ainsi par en haut toutes les apparences qui n’ont plus un droit réel et matériel à exister, il n’admet plus qu’avec une répugnance croissante la légitimité de l’esclavage ; il attaque, il ébranle cet état de choses. Il le restreint, enfin il l’abolit. Il multiplie, dans un désordre inextricable, les nuances infinies des positions sociales, en les rapprochant tous les jours davantage d’un niveau commun d’égalité ; bref, abaisser les sommets, exhausser les fonds, voilà son œuvre. Rien n’est plus propre à faire bien saisir les différentes phases de l’amalgame des races que l’étude de l’état des personnes dans le milieu qu’on observe. Ainsi, prenons ce côté de la société germanique du V\textsuperscript{e} au IX\textsuperscript{e} siècle, et, commençant par les points les plus culminants, considérons les rois.\par
Dès le II\textsuperscript{e} siècle avant notre ère, les Germains de naissance libre reconnaissaient entre eux des différences d’extraction. Ils qualifiaient de fils des dieux, de fils des Ases, les hommes issus de leurs plus illustres familles, de celles qui jouissaient seules du privilège de fournir aux tribus ces magistrats peu obéis, mais fort honorés, que les Romains appelaient leurs princes \footnote{Un des signes caractéristiques auxquels on reconnaissait un homme de race divine, c’était l’éclat extraordinaire de ses yeux. La même particularité s’attache, dans l’Inde, aux incarnations célestes. (H. Leo, \emph{Vorlesungen}, t. I, p. 40.)}. Les fils des Ases, ainsi que leur nom l’indique, descendaient de la souche ariane, et le fait seul qu’ils étaient mis à part du corps entier des guerriers et des hommes libres prouve qu’on reconnaissait dans le sang de ces derniers l’existence d’un élément qui n’était pas originairement national et qui leur assignait une place au-dessous de la première. Cette considération n’empêchait pas que ces hommes ne fussent fort importants, ne possédassent les odels, n’eussent même le droit de commander et de devenir chefs de guerre. C’est dire qu’il leur était loisible de se poser en conquérants et de se rendre plus véritablement rois que les fils des Ases, si ceux-ci consentaient à rester confinés dans leur grandeur au fond des territoires scandinaves.\par
C’était là le principe ; mais il ne paraît pas que les grandes nations germaniques de l’extrême nord, celles qui renouvelèrent la face du monde, aient jamais, tant qu’elles furent arianes, abandonné leurs plus importants établissements à des hommes d’une naissance commune \footnote{De là le respect dont étaient entourées certaines tribus royales : les Skilfinga chez les Suédois, les Nibelungs, \emph{Franci nebulones}, chez les Franks, les Herelinga, etc.}. Elles avaient trop de pureté de sang, quand elles apparurent au milieu de l’empire romain, pour admettre que leurs chefs pussent en manquer. Toutes pensèrent, à cet égard, comme les Hérules, et agirent de même. Elles ne placèrent à la tête de leurs bandes que des Arians purs, que des Ases, que des fils de dieux. Ainsi, postérieurement au V\textsuperscript{e} siècle, on doit considérer les tribus royales des nations teutoniques comme étant d’extraction pure. Cet état de choses ne dura pas longtemps. Ces familles d’élite ne s’alliaient pas qu’entre elles et ne suivaient pas, dans leurs mariages, des principes fort rigides ; leur race s’en ressentit, et, dans sa décadence, les reporta à tout le moins au rang de leurs guerriers. Les idées qu’elles possédaient, perdant du même coup leur valeur absolue, subirent des modifications analogues. Les rois germaniques devinrent accessibles à des notions inconnues de leurs ancêtres. Ils furent extrêmement séduits par les formes et les résultats de l’administration romaine, et beaucoup plus portés à les développer et à les mettre en pratique que favorables aux institutions de leurs peuples. Celles-ci ne leur donnaient qu’une autorité précaire, difficile et fatigante à maintenir ; elles ne leur conféraient que des droits hérissés de restrictions, Elles leur imposaient à tout moment le devoir de compter avec leurs hommes, de prendre leur avis, de respecter leurs volontés, de s’incliner devant leurs répugnances, leurs sympathies ou leurs préjugés. En chaque circonstance, il fallait que l’amalung des Goths ou le mérowing des Franks tâtât l’opinion avant d’agir, se donnât la peine de la flatter, de la persuader, ou, s’il la violentait, redoutât des explosions qui étaient autorisées par la loi à ne considérer le régicide que comme le maximum du meurtre ordinaire. Beaucoup de peines, de soucis, de fatigues, d’exploits obligés, de générosité, c’étaient là les dures conditions du commandement. Étaient-elles bien et dûment remplies, elles valaient des honneurs mesquins, des respects douteux qui ne mettaient pas celui auquel on les rendait à l’abri des admonestations brutalement sincères de ses fidèles.\par
Du côté de la romanité, quelle différence ! que d’avantages sur la barbarie ! La vénération pour celui qui portait le sceptre, quel qu’il fût, était sans limites ; des lois sévères, pressées comme un rempart autour de sa personne, punissaient du dernier supplice et de l’infamie la plus légère offense à cette rayonnante majesté. Où que tombât le regard du maître, prosternation, obéissance absolue ; jamais de contradic­tions, des empressements toujours. Il y avait bien une hiérarchie sociale. On distin­guait des sénateurs et une plèbe ; mais c’était là une organisation qui ne produisait pas, comme celle des tribus germaniques, des individualités fortes, en état de rembarrer la volonté du prince. Au contraire, les sénateurs, les curiales, n’existaient que pour être les ressorts passifs de la soumission générale. La crainte de la puissance matérielle des empereurs ne développait, ne maintenait pas seule de pareilles doctrines. Elles étaient naturelles à la romanité, et, prenant leur source dans la nature sémitique, elles se croyaient commandées, imposées, par la conscience publique. Il n’était pas possible à un homme honnête, à un bon citoyen de les répudier, sans manquer aussitôt à la règle, à la loi, à la coutume, à toute la théorie des devoirs politiques, partant sans blesser la conscience.\par
Les rois germaniques, contemplant ce tableau, le trouvèrent sans doute admirable. Ils comprirent que la plus satisfaisante de leurs attributions était celle de magistrat romain, et que le beau idéal serait de faire disparaître en eux-mêmes et dans leur entourage le caractère germanique pour parvenir à n’être plus que les heureux possesseurs d’une autorité nette et simple, et bien attrayante, puisqu’elle était illimitée. Rien de plus naturel que cette ambition ; mais, pour qu’elle se réalisât, il fallait que les éléments germaniques s’assouplissent. Le temps seul, amenant ce résultat des mélanges ethniques, y pouvait quelque chose.\par
En attendant, les rois montrèrent une faveur marquée à leurs sujets romains si respectueux, et ils les rapprochèrent, autant que possible, de leurs personnes. Ils les admirent très volontiers dans ce cercle intime des compagnons qu’ils appelaient leur truste, et cette faveur, en définitive inquiétante et blessante pour les guerriers natio­naux, ne paraît pas cependant avoir produit un tel effet. D’après la manière de voir de ceux-ci, le chef était en droit d’engager à son service tous ceux qu’il y jugeait propres. C’était chez eux un principe originel. Leur tolérance complète avait cependant des raisons plus profondes encore.\par
Les champions de naissance libre, qui n’étaient plus les égaux de leurs chefs par la naissance et n’appartenaient pas à la pure lignée des Ases, au moins pour la plupart \footnote{Chez les Franks, Khlodwig fit égorger tous les hommes de race salique, de sorte qu’après son règne il n’y eut plus personne dans les bandes germaniques de la contrée gauloise qui pût lutter de noblesse avec les Mérowings. (H. Leo, \emph{Vorlesungen}, etc., t. I, p. 156.)}, puisqu’ils avaient déjà subi quelques modifications ethniques avant le V\textsuperscript{e} siècle de notre ère, naturellement étaient disposés à en accepter de nouvelles. Certaines lois locales opposaient, à la vérité, quelques barrières à ce danger. Telles tribus nationales n’étaient pas autorisées à contracter des mariages entre elles \footnote{Weinhold, \emph{Die deutsch. Frauen im Mittelalt.}, p. 339 et seqq. – Dans ces nations les alliances avec des Romains passaient pour moins répréhensibles.} ; le code des Ripuaires, en le permettant entre les populations qu’il régissait et les Romains, stipulait toutefois une déchéance pour les produits de ces hymens mixtes \footnote{Les enfants issus d’un barbare et d’une Romaine étaient Romains. (\emph{Ibidem}.)\emph{ –} Au IX\textsuperscript{e} siècle, la loi saxonne prononçait la peine de mort contre les hommes coupables d’un mariage illégal. Mais il y a à remarquer que c’est à une époque bien tardive, et que rien n’indique que cette loi fût fort ancienne. En tout cas, elle n’a pas duré, (H. Leo, \emph{Vorlesungen}, etc., t. I, p. 160.)}. Il les dépouillait d’avance des immunités germaniques, et, les soumettant au régime des lois impériales, les rejetait dans la foule des sujets de l’empire. Cette logique et cette façon de procéder n’eussent pas été désavouées dans l’Inde ; mais, en somme, ce n’étaient que des restrictions très imparfaites ; elles n’eurent pas la puissance de neutraliser l’attraction que la romanité et la barbarie exerçaient l’une sur l’autre. Bientôt les concessions de la loi s’agrandirent, les réserves disparurent, et, avant l’extinction des Mérowings, le classement des habitants d’un territoire sous telle ou telle législation avait cessé de se régler sur l’origine \footnote{Bien que les ecclésiastiques fussent placés d’office sous la juridiction romaine, ils n’étaient pas partout forcés de l’accepter. Chez les Lombards, des prêtres et moines des communautés préférèrent et reçurent la loi barbare. Il y a des exemples de ce fait jusque dans les IX\textsuperscript{e}, X\textsuperscript{e} et XI\textsuperscript{e} siècles. (Savigny, \emph{ouvr. cité}, t. I, p. 117.) Les affranchis acquéraient la loi des peuples dont ils étaient issus. Chez les Ripuaires, il leur fallait suivre ou la loi ripuaire ou la loi romaine, au choix de leur patron. (\emph{Ibidem}., p. 118.) Chez les Lombards, ils restaient sous la loi du patron. (\emph{Ibid}.) Les enfants naturels choisissaient leur loi à leur gré. (\emph{Ibid}., p. 114.) Au-dessus de la loi romaine comme de la loi barbare, il y avait dans chaque territoire germanique une règle générale qui s’appliquait indifféremment à tous les habitants du pays, et qui, ayant pour objet les intérêts les plus généraux dérivait d’un compromis entre les diverses législations. Les Capitulaires sont la codification et le développement de cette règle suprême. (\emph{Ibid}., p. 143.)}. Rappelons que chez les Visigoths, bien plus avancés encore, toute distinc­tion légale entre barbare et Romain avait même cessé d’exister \footnote{Savigny, \emph{ouvr. cité}, p. 266.}.\par
Ainsi les vaincus se relevaient partout ; et, puisqu’ils pouvaient prétendre aux honneurs germaniques, c’est-à-dire à être admis parmi les leudes du roi, parmi ses affidés, ses confidents, ses lieutenants, il était bien naturel que le Germain, à son tour, pût avoir des motifs d’ambitionner leur alliance. Les Gaulois et les Italiens se trouvèrent ainsi de plain-pied avec leurs dominateurs, et, de plus, ils leur montrèrent encore qu’ils possédaient un joyau digne de rivaliser avec tous les leurs : c’était la dignité épiscopale. Les Germains comprirent à merveille la grandeur de cette situation ; ils la souhaitèrent ardemment, ils l’obtinrent, et l’on vit ainsi du même coup que des hommes sortis de la masse dominée devinrent les antrustions du fils d’Odin, tandis que plusieurs des dominateurs, dépouillant les ornements et les armes des héros germaniques pour prendre la crosse et le pallium du prêtre romain, s’instituaient les mandataires et, comme on disait, les défenseurs d’une population romaine, et, accep­tant avec elle la plus complète fraternité, répudiaient leur loi natale pour accepter la sienne.\par
En même temps, sur un autre point de l’organisation sociale, une autre innovation s’accomplissait. L’ariman, le \emph{bonus homo}, qui, aux premiers jours de la conquête, faisait profession de haïr et de mépriser le séjour des villes, se laissait aller peu à peu à quitter les champs pour devenir citadin. Il venait siéger à côté du curiale.\par
La position de celui-ci, épouvantable sous la verge de fer des prétoires impériaux, s’était améliorée de toutes manières \footnote{Savigny, \emph{ouvr. cité}, t. I, p. 250 et seqq. – Voici comment s’exprime à ce sujet M. Augustin Thierry, adversaire si prononcé, d’ailleurs, de la race et de l’action germaniques : « La curie, « le corps des décurions, cessa d’être responsable de la levée des impôts dus au fisc. « L’impôt fut levé par les soins du comte seul et d’après le dernier acte de contributions « dressé dans la cité. Il n’y eut plus d’autre garantie de l’exactitude des contribuables que « le plus ou moins de savoir-faire, d’activité et de violence du comte et de ses agents. « Ainsi les fonctions municipales cessèrent d’être une charge ruineuse, personne ne tint « plus à en être exempt, le clergé y entra. La liste des membres de la curie cessa d’être « invariablement fixe ; les anciennes conditions de propriété, nécessaires pour y être « admis, ne furent plus maintenues ; la simple notabilité suffit. Les corps de marchandise et « de métiers, jusque-là distincts de la corporation municipale, y entrèrent du moins par « leur sommité, et tendirent de plus en plus à se fondre avec elle... L’intervention de la « population entière de la cité dans ses affaires devint plus fréquente ; il y eut de grandes « assemblées de clercs et de laïques sous la présidence de l’évêque... » (\emph{Considérations sur l’histoire de France}, in-12°, Paris, 1846, t. I, p. 198-199.)}. Les exactions moins régulières, sinon moins fréquentes, étaient devenues plus supportables. Les évêques, chargés du lourd fardeau de la protection des villes, s’étaient attachés à rendre les sénats locaux capables de les seconder. Ils avaient plaidé la cause de ces aristocraties auprès des souverains de sang germanique, et ceux-ci ne trouvant rien que de naturel à leur commettre l’adminis­tration des intérêts de leurs concitoyens, leur donnèrent lieu de devenir infiniment plus importantes qu’elles ne l’avaient jamais été \footnote{Il se trouva même des points où l’administration provinciale fut conservée par les barbares : en Rhétie, par exemple, et dans les pays bourguignons, il y eut, pendant plusieurs siècles encore, un \emph{præses} et des patrices, au lieu des ceintes germaniques. (Savigny, \emph{ouvr. cité}, t. I, p. 278.)}. C’est, du reste, le résultat habituel de toutes les conquêtes opérées par des nations militaires, que l’accroissement d’influence des classes riches vaincues dans les municipalités, Du consentement des patrices barbares, les curiales se substituèrent aux nombreuses variétés et catégories de fonctionnaires impériaux, qui disparurent. La police, la justice, tout ce qui n’était pas expressément régalien tomba en leur pouvoir \footnote{En 543, le sénat de Vienne autorise la fondation d’un couvent. – En 573, les magistrats municipaux de Lyon ouvrent et reconnaissent le testament de saint Nicetius. – En 731, à Sémur, l’abbé de Flavigny, Widrad, parle, dans son testament, de la curie et du défenseur. Le cas est d’autant plus digne d’attention que Sémur n’était pas une ville proprement dite, mais un simple \emph{castrum. –} Autres faits analogues à Tours au VIII\textsuperscript{e} siècle, à Angers au VI\textsuperscript{e} et au IX\textsuperscript{e}, à Paris au VIII\textsuperscript{e}, dans toute l’Italie septentrionale et centrale au X\textsuperscript{e}, etc. (Savigny, \emph{ouvr. cité}, pass.) – Il n’est pas possible de douter que l’organisation municipale n’a jamais cessé d’exister, à aucune époque des âges moyens} ; et comme l’industrie et le commerce enrichissaient les villes, que c’était dans les villes que la religion et les études avaient leur siège, que les sanctuaires les plus vénérés attiraient et fixaient une foule dévote ou spéculatrice, sans compter les criminels qui s’y réunissaient par centaines pour profiter du droit d’asile, mille considérations opérèrent chez les arimans ce changement d’idées et d’humeur qui aurait tant indigné leurs aïeux. On les vit se complaire dans les villes, y prendre pied, s’y fixer ; et voilà comment ils y devinrent aussi curiales, voilà comment, sous leur influence, ce nom latin fut abandonné pour faire place à ceux de \emph{rachimbourgs} \footnote{Le rachimbourg est le même que le \emph{bonus homo} ; les deux termes sont employés indifféremment dans les textes. C’est le \emph{friting} des Saxons du continent, le \emph{freeman} des Anglo-Saxons, nommé aussi par eux \emph{friborgus.}} et de \emph{scabins.} On institua des scabins d’origine lombarde, franke, visigothique, tout comme des scabins d’origine romaine \footnote{Avec cette différence, que tous les Romains de naissance libre n’étaient pas d’abord aptes à être curiates, tandis que tous les barbares de la même catégorie n’admettaient pas entre eux de différence. Du reste, cette égalité finit par gagner aussi les Romains.}.\par
Pendant que les princes, les chefs et les hommes libres de la romanité et de la barbarie se rapprochaient, les classes inférieures faisaient de même, et de plus elles montaient. Le régime impérial avait jadis consacré l’existence de plusieurs situations intermédiaires entre l’esclavage complet et la liberté complète. Sous l’administration germanique ces nuances allèrent se multipliant, et l’esclavage absolu perdit tout d’abord beaucoup de terrain. Il était attaqué depuis bien des siècles par l’instinct général. La philosophie lui avait fait une rude guerre dès l’époque païenne ; l’Église lui avait porté des atteintes plus sérieuses encore. Les Germains ne se montrèrent disposés ni à le restaurer, ni même à le défendre ; ils laissèrent toute liberté aux affranchis­sements ; ils déclarèrent volontiers, avec les évêques, que retenir dans les fers des chrétiens, des membres de Jésus-Christ, était en soi un acte illégitime. Mais ils étaient en situation d’aller bien au delà, et ils le firent. La politique de l’antiquité, qui avait consisté surtout à agir dans l’enceinte des villes, et qui n’avait créé ses institutions principales que pour les populations urbaines, s’était toujours montrée médiocrement soucieuse du sort des travailleurs ruraux. Les Germains ont un point de départ tout autre, et, passionnés pour la vie des champs, considéraient leurs gouvernés d’une façon plus impartiale ; ils n’avaient de préférence théorique pour aucune catégorie d’entre eux, et par cela même étaient plus propres à régler d’une manière équitable les destinées de tous.\par
L’esclavage fut donc à pu près aboli sous leur administration \footnote{Voir, à ce sujet, Guérard, \emph{Polyptique de l’abbé Irminon}, in-4°, Paris, 1844, t. I p. 212 et seqq. – L’auteur de ce livre est doublement à accepter comme arbitre dans cette question, d’abord pour son grand et profond savoir, puis pour la haine consciencieuse et sans exemple dont il poursuit les populations germaniques. Le bien qu’il est obligé de dire de leur administration ne saurait être suspect.}. Ils le transformèrent en une condition mixte dans laquelle l’homme eut la libre disposition de son corps garantie par les lois civiles, l’Église et l’opinion publique. L’ouvrier rustique devint apte à posséder ; il le fut encore à entrer dans les ordres sacrés. La route des plus hautes dignités et des plus enviées lui fut ouverte. Il put aspirer à l’épiscopat, position supérieure à celle d’un général d’armée, dans la pensée des Germains eux-mêmes. Cette concession transformait d’une manière bien favorable la situation des personnes serviles habitant les domaines particuliers ; mais elle exerça une action plus puissante encore sur les esclaves des domaines royaux. Ces fiscalins,\emph{ fiscalini}, purent devenir et devinrent très souvent des marchands d’une grande opulence, des favoris du prince, des leudes, des comtes commandant à des guerriers d’extraction libre. Je ne parle pas de leurs filles, que les caprices de l’amour élevèrent plus d’une fois sur le trône même.\par
Les classes les plus infimes se trouvèrent ainsi avoir gagné le rang d’une autre série romaine, les colons, qui s’élevèrent du même coup dans une proportion égale. Au temps de Jules César, ils avaient été agriculteurs libres ; sous l’influence délétère de l’époque sémitisée, leur position était devenue fort triste. Des constitutions de Théodose et de Justinien les avaient indissolublement attachés à la glèbe. On leur avait laissé la faculté d’acquérir des immeubles, mais non pas celle de les vendre. Quand le sol changeait de propriétaire, ils en changeaient avec lui. L’accession aux fonctions publiques leur était rigoureusement fermée. Il leur était même interdit d’agir en justice contre leurs maîtres, tandis que ceux-ci pouvaient à leur gré les châtier corporellement. Par un dernier trait, on leur avait défendu le port et l’usage des armes ; c’était, dans les idées du temps, les déshonorer \footnote{Les âges moyens ne conservèrent pas même entièrement cette réserve : d’abord ils reconnurent les serfs eux-mêmes aptes à remplir certaines fonctions publiques ; ils eurent des \emph{servi vicarii} et des \emph{servi judices.} On leur accordait en cette qualité le droit de porter la lance et de chausser un éperon. Chez les Visigoths et chez les Lombards, on les armait même de toutes pièces, et on les appelait à concourir à la sûreté publique. (Guérard, \emph{ouvr. cité}, t. I, p. 335.) – Comparer cet état de choses à l’organisation romaine.}.\par
La domination germanique abolit presque toutes ces dispositions, et celles qu’elle négligea de faire disparaître, elle en toléra l’infraction constante. On vit sous les Mérowings des colons posséder eux-mêmes des serfs. Un ennemi fort animé des institutions et des races du nord a avoué que leur condition d’alors ne fut nullement mauvaise \footnote{Guérard, \emph{Polyptique d’lrminon}, t. I, pass.}.\par
Le travail des éléments teutoniques, agissant dans l’empire, tendit ainsi pendant quatre siècles, du V\textsuperscript{e} au IX\textsuperscript{e}, à améliorer la position des basses classes, et à relever la valeur intrinsèque de la romanité. C’était la conséquence naturelle du mélange ethni­que qui faisait circuler jusque dans le fond des multitudes le sang des vainqueurs. Quand Charlemagne apparut, l’œuvre était assez avancée pour que l’idée de reprendre les errements impériaux pût présider aux conceptions de cette forte tête ; mais il ne s’apercevait pas, non plus que personne, que les faits qui semblaient à première vue favoriser une restauration annonçaient, au contraire, une grande et profonde révolu­tion, amenaient l’avènement complet de rapports nouveaux dans la société. Il n’était au monde volonté ni génie qui pût empêcher l’explosion des causes parvenues en silence à toute leur maturité.\par
 La romanité avait repris de l’énergie, mais non pas partout en dose égale. La barbarie s’était presque effacée comme corps ; mais son influence dominait en plus d’une contrée, et sur ces points, bien qu’elle se fût annihilée sous l’élément latin, c’était, au contraire, celui-ci qui s’était résorbé en elle. Il en était résulté partout d’impérieuses dispositions sporadiques, et le pouvoir de les réaliser.\par
Dans le sud de l’Italie régnait une confusion plus profonde que jamais. Les populations anciennes, de faibles débris barbares, des alluvions grecques incessantes, puis des Sarrasins en foule, y entretenaient l’excès du désordre avec la prépondérance sémitique. Nulle pensée n’y était générale, nulle force n’y était assez grande pour s’imposer longtemps. C’était un pays voué pour toujours aux occupations étrangères, ou à une anarchie plus ou moins bien déguisée.\par
Dans le nord de la Péninsule, la domination des Lombards était incontestée. Ces Germains, peu assimilés à la population romanisée, ne partageaient pas son indiffé­rence pour la suprématie d’une race germanique différente de la leur. Comme ils n’étaient pas fort nombreux, Charlemagne pouvait les vaincre ; c’était tout, il ne pouvait pas étouffer leur nationalité \footnote{Savigny observe, avec vérité, que le nombre des groupes pourvus du droit personnel est beaucoup plus considérable en Italie qu’en France au VII\textsuperscript{e} siècle. il en conclut judicieusement que les différentes races y sont complètement représentées. (\emph{Ouvr. cité}, t. I, p. 104.)}.\par
En Espagne, le sud entier et le centre n’appartenaient plus à l’empire ; l’invasion musulmane en avait fait une annexe des vastes États du khalife. Quant au nord-ouest, où les descendants des Suèves et des Visigoths s’étaient cantonnés, il présentait dans les masses inférieures beaucoup plus d’éléments celtibères que de romains. De là une empreinte spéciale qui distinguait ces peuples des habitants de la France méridionale comme des Maures, bien qu’un peu moins.\par
Le sang de l’Aquitaine, pourvu de quelque affinité avec celui des Navarrais et des hommes de la Galice par ses éléments originairement indigènes, avait en outre une alluvion romaine fort riche, et une alluvion barbare de quelque épaisseur, sans équivaloir à celle de l’Espagne septentrionale.\par
En Provence et dans le Languedoc, la couche romaine était tellement considérable, le fond celtique sur lequel elle avait été établie était si fort primé par elle, que l’on aurait pu se croire là dans l’Italie centrale, d’autant mieux que les invasions sarrasines y entretenaient une infiltration sémitique qui n’était pas sans puissance \footnote{Reynaud, \emph{Invasions des Sarrasins en France, en Savoie et dans la Suisse}, Paris, 1836, in-8°.}. Les Visigoths, après un séjour où leur sang s’était beaucoup oblitéré, étaient en partie retirés en Espagne, en partie en voie de s’absorber définitivement dans la population native. Vers l’est, des groupes burgondes, et partout quelque peu de Franks, dirigeaient cet ensem­ble assez peu homogène, mais n’en étaient pas les maîtres absolus.\par
La Bourgogne et la Suisse occidentale, en y comprenant la Savoie et les vallées du Piémont, avaient conservé beaucoup d’éléments celtiques. Dans le premier de ces pays, à la vérité, l’élément romain était le plus fort, mais il l’était moins dans les autres, et surtout l’élément burgonde avait apporté beaucoup de détritus celtiques d’Allemagne qui s’étaient assez facilement alliés au vieux fonds du pays. Des Franks, des Longobards, des Goths, des Suèves et d’autres débris germaniques, des Slaves même \footnote{On en retrouve des traces au canton du Valais, à Granges (Gradec), dans les villages de Krimenza (Kremenica), Luc (Luka), Visoye, Grava, etc. Les Allemands des environs les appellent des Huns. (Schaffarik, \emph{Slawiche Alterth.}, t. I, p. 329.) – Le lac de Thun s’appelait, au VII\textsuperscript{e} siècle, \emph{lacus Vendalicus ;} on le nomma plus tard \emph{Wendensee.} (\emph{Ibid.}, p. 420, note 4.)} empêchaient ces contrées de présenter un tout bien homogène ; elles avaient néan­moins plus de rapports entre elles qu’avec leurs voisines. Sur leurs frontières du nord, elles ressemblaient fort aux peuples restés dans la Germanie.\par
La France centrale était surtout gallo-romaine. De tous les barbares qui y avaient pénétré, les Franks seuls régnaient. Les populations premières n’y avaient pas une couleur aussi sémitisée que dans la Provence ; elles ressemblaient davantage à celles de la haute Bourgogne. Il y avait de plus, dans le mélange général, la différence de mérite dans les éléments germaniques des deux pays, les Franks valant plus que les Burgondes ; du reste, les Franks, bien qu’en petit nombre chez ces derniers, les y primaient encore.\par
À l’ouest de la Gaule centrale s’ouvrait la petite Bretagne. Les populations à peine romanisées de cette péninsule avaient reçu, et plusieurs fois, des émigrations de la grande île. Elles n’étaient pas purement celtiques, mais d’origine belge, partant germanisées, et, dans le cours des temps, d’autres alliages germaniques avaient encore modifié leur essence. Les Bretons du continent représentaient un groupe mixte où l’élément celtique avait le dessus sans être aussi complètement libre d’alliage qu’on le pense communément.\par
Au delà de la haute Seine et dans les contrées qui se succédaient jusqu’à l’embouchure du Rhin d’un côté, de l’autre jusqu’au Mein et jusqu’au Danube, avec la Hongrie pour frontière à l’orient, s’aggloméraient des multitudes où les éléments germaniques exerçaient une prépondérance plus incontestée, mais non pas uniforme. La partie d’entre la Seine et la Somme appartenait à des Franks considérablement celtisés, avec une proportion relativement médiocre d’alliage romain sémitisé. Le pays riverain de la mer avait gardé, peut-être repris le nom kymrique de \emph{Picardaich.} Dans l’intérieur des terres, les Gallo-Romains mêlés aux Franks neustriens se distinguaient à peine de leurs voisins du sud et de l’est ; ils étaient cependant un peu moins énergiquement constitués que ces derniers, et surtout que ceux du nord. Plus on se rapprochait du Rhin et ensuite s’enfonçait dans la direction des anciennes limites décumates, plus on se trouvait entouré de véritables Franks de la branche austrasienne, où l’ancien sang germanique existait à son plus haut degré de verdeur. On était arrivé à son foyer. Aussi peut-on reconnaître bien aisément, en interrogeant les récits de l’histoire, que là étaient le cerveau, le cœur et la moelle de l’empire, que là résidait la force, que là se décidaient les destinées. Tout événement qui ne s’était pas préparé sur le Rhin moyen, ou dans les environs, n’avait et ne pouvait avoir qu’une portée locale assez peu riche en conséquences.\par
 En remontant le fleuve dans la direction de Bâle, les masses germaniques, revenant à se celtiser davantage, se rapprochaient du type bourguignon ; à l’est, le mélange gallo-romain se compliquait, dès la Bavière, de nuances slaves qui allaient se renfor­çant jusqu’aux confins de la Hongrie et de la Bohême, où, devenant plus marquées, elles finissaient par prendre le dessus, et formaient alors la transition entre les nations de l’occident et les peuples du nord-est et du sud-est jusqu’à la région byzantine.\par
Les groupes occidentaux devaient ainsi à l’élément teutonique, qui les animait tous à des degrés divers, une force disjonctive que les nations énervées du monde romain n’avaient pas possédée. L’époque finissait où les barbares n’avaient pu et dû voir dans le fonds ethnique régi par eux qu’une masse opposée à leur masse. Mêlés désormais à elle, ils avaient acquis un autre point de vue : ils n’étaient plus frappés que par des dissemblances toutes nouvelles, scindant l’ensemble des multitudes dont eux-mêmes se trouvaient désormais faire partie. Ce fut donc au moment même où la romanité croyait avoir conquis la barbarie qu’elle éprouva précisément les effets les plus graves de l’accession germanique. Jusqu’à Charlemagne, elle avait gardé tous les dehors en même temps que la réalité de la vie. Après lui, la forme matérielle cessa d’exister, et, bien que son esprit n’ait pas plus disparu du monde que l’esprit assyrien et l’esprit hellénistique, elle entra dans une phase comparable aux épreuves du rajeunissement d’Eson.\par
Quoi qu’il en soit, je le répète, son esprit ne périt pas. Ce génie, qui représentait la somme de tous les débris ethniques jusqu’alors amalgamés, résista, et, pendant le temps où il resta contraint de surseoir à des manifestations extérieures bien évidentes, il maintint au moins sa place par un moyen qui ne laisse pas que d’être digne d’avoir ici sa mention. Ce fut un phénomène tout opposé à celui qui avait eu lieu entre l’époque d’Odoacre et celle du fils de Pépin. Pendant cette période, l’empire avait subsisté sans l’empereur ; ici l’empereur subsista sans l’empire. Sa dignité, se rattachant tant bien que mal à la majesté romaine, s’efforça pendant plusieurs siècles de lui conserver une apparence de continuateur et d’héritier. Ce furent encore les populations germaniques qui, déployant en cette rencontre l’instinct, le goût obstiné de la conservation qui leur est naturel, donnèrent un nouvel exemple de cette logique et de cette ténacité que leurs frères de l’Inde n’ont pas possédées à un degré plus haut, bien qu’en l’appliquant d’une autre manière.\par
Il nous reste maintenant à voir pratiquer les vertus typiques de la race par les derniers rameaux arians que la Scandinavie envoya vers le sud : ce furent les Normands et les Anglo-Saxons.
\section[{VI.5. Dernières migrations arianes-scandinaves.}]{VI.5. \\
Dernières migrations arianes-scandinaves.}
\noindent Tandis que les grandes nations sorties de la Scandinavie après le I\textsuperscript{er} siècle de notre ère gravitaient successivement vers le sud, les masses encore considérables qui étaient demeurées dans la péninsule ou aux environs étaient loin de se vouer au repos. On doit les distinguer en deux grandes fractions : celle que produisit la confédération anglo-saxonne ; puis un autre amas dont les émissions furent plus indépendantes les unes des autres, commencèrent plus tôt, finirent plus tard, allèrent beaucoup plus loin, et auquel il convient de donner la qualification de \emph{normand}, que les hommes qui le composaient s’attribuaient à eux-mêmes.\par
Bien que, depuis le I\textsuperscript{er} siècle avant Jésus-Christ jusqu’au V\textsuperscript{e}, l’action de ces deux groupes se soit fait sentir à plusieurs reprises jusque dans les régions romaines, il n’y a pas lieu, sur ce terrain, d’en parler avec détail ; cette action s’y confond, de toutes manières, avec celle des autres peuples germaniques. Mais, après le V\textsuperscript{e} siècle, les conséquences de la domination d’Attila mirent fin à ces rapports antiques, ou du moins les relâchèrent très sensiblement \footnote{Schaffarik, \emph{Slawiche Alterth.}, t. I, p. 326 et seqq. – Amédée Thierry, \emph{Revue des Deux-Mondes}, 1\textsuperscript{er} décembre 1852, pass. On ne saurait trop louer cette belle appréciation de la confédération hunnique.}. Des multitudes slaves, entraînées par les convul­sions ethniques dont les Teutons et les Huns étaient les principaux agents, furent jetées entre les pays scandinaves et l’Europe méridionale, et c’est de ce moment seul que l’on peut faire dater la personnalité distincte des habitants arians de l’extrême nord de notre continent.\par
 Ces Slaves, victimes encore une fois des catastrophes qui agitaient les races supérieures, arrivèrent dans les contrées connues de leurs ancêtres, il y avait déjà bien des siècles ; peut-être même s’avancèrent-ils plus loin que ceux-ci ne l’avaient fait deux mille ans avant notre ère \footnote{Schaffarik, \emph{Slawische Alterth.}, t. I, p. 166 ; t. II, p. 411, 416, 427, 443, 503, 526, 565. – Kefestein, \emph{Keltische Alterth.}, t. I, p. XLV, XLVII, L et seqq.}. Ils repassèrent l’Elbe, rencontrèrent le Danube, apparurent dans le cœur de l’Allemagne. Conduits par leurs noblesses, formées de tant de mélanges gètes, sarmates, celtiques, par lesquels ils avaient été jadis asservis, et confondus avec quelques-unes des bandes hunniques qui les poussaient, ils occupè­rent, dans le nord, tout le Holstein jusqu’à l’Eider \footnote{Schaffarik incline même à penser que les Huns connus de \emph{l’Edda} sont tous des Slaves. Cette opinion est un peu absolue. (T. I, p. 328.)}. À l’ouest, gravitant vers la Saale, ils finirent par en faire leur frontière ; tandis qu’au sud ils se répandirent dans la Styrie, la Carniole, touchèrent d’un côté la mer Adriatique, de l’autre le Mein, et couvrirent les deux archiduchés d’Autriche, comme la Thuringe et la Souabe \footnote{Schaffarik, t. II, p. 310 et seqq. – Dans cette direction, les Slaves et leurs noblesses agissaient sous la pression spéciale des Avares, nation demi-mongole, demi-ariane. Beaucoup de ces derniers restèrent avec eux dans la Carniole et la Styrie. (P. 327.)}. Ensuite ils descendi­rent jusqu’aux contrées rhénanes, et pénétrèrent en Suisse. Ces nations wendes, toujours opprimées jusqu’alors, devinrent ainsi, bon gré mal gré, conquérantes, et les mélanges qui les distinguaient ne leur rendirent pas d’abord ce métier par trop difficile. Les circonstances, agissant avec énergie en leur faveur, amenèrent les choses à ce point que l’élément germanique s’affaiblit considérablement dans toute l’Allemagne, et ne resta quelque peu compact que dans la Frise, la Westphalie, le Hanovre et les contrées rhénanes depuis la mer jusque vers Bâle. Tel fut l’état des choses au VIII\textsuperscript{e} siècle.\par
Bien que les invasions saxonnes et les colonisations frankes des trois ou quatre siècles qui suivirent aient un peu modifié cette situation, il n’en demeura pas moins acquis, par la suite, que la masse des nations locales se trouva à jamais dépouillée de ses principaux éléments arians. Ce ne furent pas seulement les invasions slaves de l’époque hunnique qui contribuèrent à cette transformation ; elle fut en grande partie amenée par la constitution intime des groupes germaniques eux-mêmes. Essentielle­ment mixtes et éloignés de ne compter que des guerriers de noble origine, ils traînaient à leur suite, ainsi qu’on l’a vu, de nombreuses bandes serviles, celtiques et wendes. Quand leurs nations émigraient ou périssaient, c’était surtout la partie illustre qui, en elles, était frappée, et les traces subsistantes de leur occupation se retrouvaient infailli­blement dans la personne des karls et des traells, deux classes que les catastrophes politiques n’atteignaient que par contrecoup, mais qui possédaient une bien faible proportion de l’essence scandinave. Au contraire, les nations slaves perdaient-elles leurs nobles, elles n’en devenaient que plus émancipées de cette influence arianisée qui les détournait de leur véritable nature. Pour ces deux raisons, la disparition des Germains d’une part, de l’autre l’épuisement des aristocraties wendes, les populations de l’Allemagne, d’ailleurs composées sur les différents points des mêmes doses ethniques en quantités spéciales, ce qui est aussi l’origine de leurs dispositions faiblement sporadiques, se trouvèrent définitivement très peu germanisées. Tout en porte témoignage, les institutions commerciales, les habitudes rurales, les superstitions populaires, la physionomie des dialectes, les variétés physiologiques. De même qu’il n’est pas rare de trouver dans la forêt Noire, non plus qu’aux environs de Berlin, des types parfaitement celtiques ou slaves, de même il est facile d’observer que le naturel doux et peu actif de l’Autrichien et du Bavarois n’a rien de cet esprit de feu qui animait le Frank ou le Longobard \footnote{Haxthausen, \emph{Études sur la situation intérieure, la vie nationale et les institutions rurales de la Russie}, Hanovre, 1847, in-8°, t. I, p. III. – En recherchant l’origine de plusieurs coutumes qui exercent une influence décisive sur l’existence agricole en Allemagne, cet auteur démontre qu’on arrive immédiatement à une inspiration slave. – Quant aux dialectes allemands modernes, la présence d’abondants éléments celtiques dans leur contexture n’est pas mise en question. (Voir Grimm, \emph{Geschichte der deutschen Sprache}, t. I, p. 287 ; Mone, \emph{Th.} p. 352 ; Keferstein, \emph{Keltische Alterth.}, t. I, p. XXXVIII, etc.)}.\par
Ce fut sur ces populations que les Saxons et les Normands eurent à agir, absolument comme les Germains avaient agi sur des masses à peu de chose près semblables. Quant au théâtre des nouveaux exploits qui s’opérèrent, il fut identique­ment le même, avec cette différence que, les forces employées étant moins considérables, les résultats géographiques restèrent plus limités.\par
Les Normands reprirent d’abord l’œuvre des tribus gothiques. Navigateurs aussi hardis, ils poussèrent leurs expéditions principales dans l’est, franchirent la Baltique, vinrent aborder sur les plages où avaient débuté les ancêtres d’Hermanarik, et, traversant, l’épée au poing, toute la Russie, allèrent, d’un côté, lier des rapports de guerre, quelquefois d’alliance, avec les empereurs de Constantinople, tandis que, de l’autre, leurs pirates étonnaient et épouvantaient les riverains de la Caspienne \footnote{\emph{Mémoires de l’Académie de Saint-Pétersbourg}, 1848, t, IV, p. 182 et pass.}.\par
Ils se familiarisèrent si bien avec les contrées russes, ils y donnèrent une si haute idée de leur intelligence et de leur courage, que les Slaves de ce pays, faisant l’aveu officiel de leur impuissance et de leur infériorité, implorèrent presque unanimement leur joug. Ils fondèrent d’importantes principautés. Ils restaurèrent en quelque sorte Asgart, et le Gardarike, et l’empire des Goths. Ils créèrent l’avenir du plus imposant des États slaves, du plus étendu, du plus solide, en lui donnant pour premier et indispensable ciment leur essence ariane. Sans eux la Russie n’eût jamais existé \footnote{Ljudbrand de Ticino, évêque de Crémone, mort en 979, dit que le peuple appelé \emph{russe} par les Grecs est nommé \emph{normand} par les Occidentaux. (Munch, \emph{ouvr. cité}, p. 55.) Au X\textsuperscript{e} siècle, les Russes, et il faut comprendre sous ce nom la portion dominante de la nation, parlaient le scandinave. Le territoire de cet idiome comprenait les plaines du lac Ladoga, du lac Ilmen et le haut Dnieper. (Schaffarik, \emph{ouvr. cité}, t. I, p. 143.) Les Normands russes portaient plus particulièrement le nom de Warègues. Il est aussi ancien que le nom d’Ase, de Goth et de Saxon, et remonte comme eux à la pure souche ariane. Les Grecs connaissaient dans la Drangiane une nation sarmate appelée par eux (nom grec), et qui s’intitulait elle-même \emph{Zaranga ou Zaryanga}, dont la forme zend est \emph{Zarayangh.} Pline transcrit ce mot en en faisant \emph{Evergelæ.} (Westergaard et Lassen, \emph{Achemen Keilinschriften}, p. 55. – Niebuhr, \emph{Inscript. pers.}, table. I, XXXI.) Ce nom de (nom grec) Zaranga, Evergetæ, ou Waregh, fut aussi appelé en France, où il a laissé des traces qui survivent jusqu’à ce jour dans les noms de \emph{Varange}, de \emph{Varangeville} et autres. – Il est très important de ne rien négliger de tout ce qui démontre à quel point les Arians du nord restèrent, tant qu’ils vécurent, rapprochés, malgré les distances de leur souche originelle.}.\par
 Qu’on pèse bien cette proposition, et qu’on en examine les bases : il y a au monde un grand empire slave ; c’est le premier et le seul qui ait bravé l’épreuve des temps, et ce premier et unique monument d’esprit politique doit incontestablement son origine aux dynasties varègues, autrement dit normandes. Cependant cette fondation politique n’a de germanique que le fait même de son existence. Rien de plus aisé à concevoir. Les Normands n’ont pas transformé le caractère de leurs sujets ; ils étaient trop peu nombreux pour obtenir un pareil résultat. Ils se sont perdus au sein des masses populeuses qui n’ont fait qu’augmenter autour d’eux, et dans lesquelles les invasions tatares du moyen âge ont, sans cesse et sans mesure, augmenté l’influence énervante du sang finnique. Tout aurait fini, même l’instinct de cohésion, si une intervention providentielle n’avait ramené à temps cet empire sous l’action qui lui avait donné naissance : cette action a suffi jusqu’à présent pour neutraliser les pires effets du génie slave. L’accession des provinces allemandes, l’avènement des princes allemands, une foule d’administrateurs, de généraux, de professeurs, d’artistes, d’artisans allemands, anglais, français, italiens, émigration qui s’est faite lentement, mais sans interruption, a continué à tenir sous le joug les instincts nationaux, et à les réduire, malgré eux, à l’honneur de jouer un grand rôle en Europe. Tout ce qui en Russie présente quelque vigueur politique, dans le sens où l’Occident prend ce mot, tout ce qui rapproche ce pays, dans les formes du moins, de la civilisation germanisée, lui est étranger.\par
Il est possible que cette situation se soutienne pendant un temps plus ou moins long ; mais, au fond, elle n’a rien changé à l’inertie organique de la race nationale, et c’est gratuitement que l’on suppose la race wende dangereuse pour la liberté de l’Occident. On se l’est imaginée bien à tort conquérante. Quelques esprits abusés, la voyant peu capable de s’élever à des notions originales de perfectionnement social, se sont avisés de la déclarer neuve, vierge et pleine d’une sève qui n’a pas encore coulé. Ce sont là autant d’illusions. Les Slaves sont une des familles les plus vieilles, les plus usées, les plus mélangées, les plus dégénérées qui existent. Ils étaient épuisés avant les Celtes. Les Normands leur ont donné la cohésion qu’ils n’avaient pas en eux-mêmes. Cette cohésion se perdit quand l’invasion de sang scandinave fut absorbée ; des influences étrangères l’ont restituée et la maintiennent ; mais elles-mêmes valent, au fond, peu de chose : elles sont riches d’expérience, rompues à la routine de la civilisa­tion ; mais, dépouillées d’inspiration et d’initiative, elles ne sauraient donner à leurs élèves ce qu’elles ne possèdent pas.\par
Vis-à-vis de l’Occident, les Slaves ne peuvent occuper qu’une situation sociale toute subordonnée, et réduits, à ce point de vue, à la condition d’annexes et d’écoliers de la civilisation moderne, ils joueraient un personnage presque insignifiant dans l’histoire future comme dans l’histoire passée, si la situation physique de leurs territoi­res ne leur assurait un emploi qui est véritablement des plus considérables. Placés aux confins de l’Europe et de l’Asie, ils forment une transition naturelle entre leurs parents de l’ouest et leurs parents orientaux de race mongole. Ils rattachent ces deux masses qui croient s’ignorer. Ils forment des masses innombrables depuis la Bohême et les environs de Pétersbourg jusqu’aux confins de la Chine. Ils maintiennent ainsi, entre les métis jaunes des différents degrés, cette chaîne ininterrompue d’alliances ethniques qui fait aujourd’hui le tour de l’hémisphère boréal, et par laquelle circule un courant d’aptitudes et de notions analogues.\par
Voilà la part d’action dévolue aux Slaves, celle qu’ils n’auraient jamais acquise, si les Normands ne leur avaient donné la force de la prendre, et qui a son foyer principal en Russie, parce que c’est là que la plus considérable dose d’activité a été implantée par ces mêmes Normands qu’il faut suivre maintenant sur d’autres champs de bataille.\par
Je serai bref dans l’énumération de leurs hauts faits ; c’est surtout matière à considération pour l’histoire politique. Repoussés du centre de l’Allemagne par la foule des combattants qui s’y pressaient déjà, tenus en échec par les Saxons leurs égaux \footnote{Les Saxons du continent se mélangèrent si rapidement avec les populations celtiques ou slaves qui les entouraient, que, bien que leurs aïeux aient encore habité la Chersonèse cimbrique au V\textsuperscript{e} siècle et qu’ils n’aient envahi la Thuringe qu’au VI\textsuperscript{e}, une tradition connue aujourd’hui les dit autochtones du Harz. Ils prétendent être nés tout à coup au milieu des rochers et des forêts de cette contrée, au bord d’une fontaine, avec leur roi Aschanes. C’est là une confusion de mythes scandinaves avec des notions aborigènes, (W. Muller, \emph{ouvrage cité}, p. 298.)}, les Normands continuèrent néanmoins jusqu’au VII\textsuperscript{e} siècle à y pousser des incursions, mais sans autre résultat sensible que d’y augmenter le désordre. Effrayant les mers occidentales par le nombre et surtout par l’audace de leurs pirateries, ils allaient pénétrant jusque dans la Méditerranée, pillant l’Espagne, en même temps que, par un travail plus fécond, ils colonisaient les îles voisines de l’Angleterre, s’établissaient en Irlande et en Écosse, peuplaient les vallées d’Islande.\par
Un peu plus tard, ils firent mieux ; ils s’établirent à demeure dans cette Angleterre qu’ils avaient tant inquiétée, et en enlevèrent une grande partie aux Bretons, et surtout aux Saxons qui les avaient précédés sur cette terre. Plus tard encore, ils renouvelèrent le sang de la province française de Neustrie, et lui apportèrent une supériorité ethnique bien appréciable sur d’autres contrées de la Gaule. Elle la conserva longtemps, et en montre encore quelques restes. Parmi leurs titres de gloire les plus éclatants, et qui ne furent pas non plus sans de grands résultats, il faut compter surtout la découverte du continent américain, opérée au X\textsuperscript{e} siècle, et les colonisations qu’ils portèrent dans ces régions au XI\textsuperscript{e} et peut-être jusqu’au XII\textsuperscript{e}. Enfin je parlerai en son lieu de la conquête totale de l’Angleterre par les Normands français.\par
La Scandinavie, d’où sortaient ces guerriers, occupait encore dans la période héroïque des âges moyens le rang le plus distingué parmi les souvenirs de toutes les races dominantes de l’Europe. C’était le pays de leurs ancêtres vénérés, c’eût encore été le pays des dieux mêmes, si le christianisme l’eût permis. On peut comparer les grandes images que le nom de cette terre évoquait dans la pensée des Franks et des Goths à celles qui pour les brahmanes entouraient la mémoire de l’Ultara-Kourou. De nos jours, cette péninsule si féconde, cette terre si sacrée n’est plus habitée par une population égale à celles que son sein généreux a pendant si longtemps et avec tant de profusion répandues sur toute la surface du continent d’Europe \footnote{La langue des inscriptions runiques diffère considérablement, comme aussi le gothique d’Ulfila, des langues scandinaves actuelles. (Keferstein, \emph{Keltische Alterth.}, t. I, p. 351.) Ces dernières ont de nombreuses marques d’alliage avec les éléments finniques (Schaffarik,\emph{ ouvr. cité}, t. I, p. 140.)}. Plus les anciens guerriers étaient de race pure, moins ils étaient tentés de rester paresseusement dans leurs odels, quand tant d’aventures merveilleuses entraînaient leurs émules vers les contrées du midi. Bien peu y demeurèrent. Cependant quelques-uns y revinrent, Ils y trouvèrent les Finnois, les Celtes, les Slaves, soit descendants de ceux qui avaient autrefois occupé le pays, soit fils des captifs que les hasards de la guerre y avaient amenés, luttant avec quelque avantage contre les débris du sang des Ases. Cependant il n’est pas douteux que c’est encore en Suède, et surtout en Norwège, que l’on peut aujourd’hui retrouver le plus de traces physiologiques, linguistiques, politiques, de l’existence disparue de la race noble par excellence, et l’histoire des derniers siècles est là pour l’attester. Ni Gustave-Adolphe, ni Charles XII, ni leurs peuples ne sont des successeurs indignes de Ragnas Lodbrog et de Harald aux beaux cheveux. Si les populations norwégiennes et suédoises étaient plus nombreuses, l’esprit d’initiative qui les anime encore pourrait n’être pas sans conséquences ; mais elles sont réduites par leur chiffre à une véritable impuissance sociale : on peut donc affirmer que le dernier siège de l’influence germanique n’est plus au milieu d’elles. Il s’est transporté en Angleterre. C’est là qu’il déploie encore avec le plus d’autorité la part qu’il a gardée de son ancienne puissance.\par
Lorsqu’il a été question des Celtes, on a vu déjà que la population des îles Britanniques au temps de César était formée d’une couche primitive de Finnois, de plusieurs nations galliques différemment affectées par leur mélange avec ces indigè­nes, mais certainement très dégradées par leur contact, et de plus d’une immigration considérable de Belges germanisés, occupant le littoral de l’est et du sud.\par
Ce fut à ces derniers surtout que les Romains eurent affaire, tant pour la guerre que pour la paix. À côté de ces tribus d’origine étrangère vinrent se placer de très bonne heure, s’ils n’y étaient pas déjà lors de l’arrivée de César, des Germains plus purs, appelés par les documents gallois Coritaniens \footnote{Kemble \emph{die Sachsen in England}, übers. von Chr. Brandes, Leipzig, in-8°, 1853, t. I, p. 7. – Ptolémée appelle cette population (nom grec) (II, 3). Elle habitait les comtés actuels de Lincoln, Leicester, Rutland, Northampton, Nottingham et Derby. – Voir aussi Dieffenbach, \emph{Celtica I}.}. À dater de ce moment, les invasions et les immigrations partielles des groupes teutoniques ne cessèrent plus jusqu’à l’an 449, date ordinairement, bien qu’abusivement, assignée aux débuts de la période anglo-saxonne. Sous Probus, le gouvernement impérial colonisa dans l’île beaucoup de Vandales ; quelque temps après, il y amena des Quades et des Marcommans \footnote{Kemble, \emph{ouvr. cité}, p. 9.}. Honorius établit dans les cantons du nord plus de quarante cohortes de barbares qui amenèrent avec eux femmes et enfants. Ensuite des Tungres, en nombre considérable, reçurent encore des terres. Toutes ces accessions furent assez importantes pour couvrir d’une population nouvelle la côte de l’ouest, et nécessiter la création d’un fonctionnaire spécial qui, dans la hiérarchie romaine de l’île, portait le titre de \emph{préfet de la côte saxonne.} Ce titre démontre que, longtemps avant qu’il fût question des deux frères héroïques Hengest et Horsa, nombre d’hommes de leur nation vivaient déjà en Angleterre \footnote{Palsgrave, \emph{the Rise and Progress of the English Commonwealth}, t. I, p. 355.}.\par
 Ainsi la population bretonne se trouvait très anciennement affectée par des immixtions germaniques. Il est peu douteux que les tribus les moins bien douées, celles qui occupaient les provinces du centre, furent graduellement obligées de se confondre avec les masses environnantes, ou de se retirer au fond des montagnes, du nord, ou enfin d’émigrer dans l’île d’Irlande, qui devint ainsi le dernier asile des Celtes purs, si toutefois il en restait de tels.\par
Bientôt la population romaine était devenue à son tour importante. Lors de la révolte de Boadicée, soixante-dix mille Romains et alliés avaient été égorgés par les rebelles dans les trois seuls cantons de Londres, de Vérulam et de Colchester. Les causes qui avaient amené ces méridionaux dans la Grande-Bretagne continuant toujours d’agir, de nouveaux venus comblèrent bientôt les vides produits par l’insurrec­tion, et le nombre des Romains insulaires continua à suivre une progression ascendante.\par
Au III\textsuperscript{e} siècle, Marcien compte dans le pays cinquante-neuf villes de premier rang \footnote{Palsgrave, \emph{ouvr. cité}, t. I, p. 237. Beaucoup de ces villes n’étaient peuplées que de colons romains. On sait ce qu’il faut entendre par cette dénomination au point de vue ethnique. – César a dit deux choses contradictoires sur les villes de la Grande-Bretagne. Dans un passage, il déclare qu’elles ne sont que des camps palissadés. Dans un autre (v, 12), il décrit « creberrima ædificia fere gallicis consimilia. » – Il veut dire que les Bretons de l’intérieur, les plus grossiers, n’avaient que des retraites dans les bois, mais que les Belges germanisés venus de la Gaule avaient des villes comme leurs frères du continent. Il n’est pas douteux, en effet, qu’ils n’aient dû conserver cette coutume, puisqu’ils frappaient monnaie d’après les types belgiques, et que d’ailleurs, quarante ans après l’occupation romaine, sous Agricola, il y avait, au calcul de Ptolémée, cinquante-six villes dans le pays. C’étaient évidemment, pour la plupart, des cités nationales.}. Beaucoup n’étaient peuplées que de Romains, expression qu’il ne faut pas entendre dans ce sens que ces habitants n’avaient dans les veines que du sang d’outre-mer, mais dans celui-ci, que tous, d’origine bretonne ou étrangère, suivaient et prati­quaient la coutume romaine, obéissaient aux lois impériales, construisaient en abondance ces monuments, aqueducs, théâtres, arcs de triomphe, que l’on admirait encore au XIV\textsuperscript{e} siècle \footnote{Palsgrave, \emph{ouvr. cité}, t. I, p. 323. – Tacite, fort sévère pour les Gaulois à cause de la facilité avec laquelle ils s’étaient laissé aller à la corruption romaine, ne l’est pas moins pour les Bretons de la grande île à ce même point de vue. Ils avaient adopté dans leurs villes toute l’organisation municipale de l’empire. (Palsgrave, \emph{ouvr. cité}, t. I, p. 349.)}, bref, donnaient à tout le pays plat une apparence très analogue à celle des provinces de la Gaule.\par
Toutefois une grande différence subsistait. Les habitants de la Grande-Bretagne témoignaient d’une exubérance d’énergie politique tout à fait supérieure à celle de leurs voisins du continent, tout à fait disproportionnée à l’étendue de leur propre territoire, et en contradiction manifeste avec leur situation topographique qui, les rejetant sur le flanc de l’empire, semblait leur interdire l’espérance de pouvoir peser sur ses des­tinées. Mais ici s’offre encore une preuve manifeste du peu d’action qu’exerce la question géographique sur la puissance d’un pays. Les demi-Germains de la Grande-Bretagne furent les plus grands fabricateurs d’empereurs, reconnus ou refusés, qu’il y eut jamais dans le monde romain. Ce fut chez eux et avec leur concours que s’élaborèrent presque constamment les grandes trames ambitieuses. Ce fut de leur rivage et avec leurs cohortes que partirent presque par bandes les dominateurs de la romanité, et, trouvant encore cette gloire insuffisante, ils osèrent entreprendre la tâche dans laquelle leurs voisins les Gaulois avaient tant de fois échoué : ils prétendirent se donner des dynasties particulières, et ils y réussirent. Depuis Carausius, ils ne tintent plus que faiblement au grand corps romain \footnote{Palsgrave, \emph{ouvr. cité}, t. I, p. 375.} ; ils formèrent à part un centre politique orgueilleusement constitué sur le modèle et avec tous les insignes de la mère patrie. Ils se signalaient déjà dans leurs brouillards par cette auréole de liberté sévère et quelque peu égoïste qui fait encore la gloire de leurs neveux.\par
Je ne nommerai pas les empereurs britto-romains Allectus \footnote{Allectus soutint sa puissance absolument comme les vrais empereurs soutenaient la leur. Il colonisa dans son île un grand nombre de Franks et de Saxons. (Palsgrave, \emph{ouvr. cité}, t. I, p. 377.)}, Magnentius, Valentinius, Maxime, Constantin, avec qui Honorius fut contraint de pactiser ; je ne dirai rien de ce Marcus qui, de nom comme de fait, établit pour toujours l’isolement de son pays \footnote{Ce Marcus fut élu empereur avec la tâche spéciale de résister aux invasions saxonnes. On était alors en 407. (Palsgrave, \emph{ouvr. cité}, t. I, p. 386.)}. J’ai voulu montrer seulement à quelle antiquité remonte ce titre d’impérial donné par les Anglais modernes à leur État et à leur parlement. Les formes romaines prévalurent dans l’île pendant quatre cent cinquante ans à peu près. Cette période révolue, commencèrent les guerres civiles entre les Britto-Romains germanisés et les Saxons plus purs déjà établis depuis longues années sur plusieurs points du pays, mais qui, poussés et renforcés par des essaims de compatriotes accourus du continent, d’où les chassaient les agressions des Slaves, prétendirent tout à coup à la possession entière de l’île. Les historiens nous ont montré souvent ces fils des Scandinaves, ces \emph{Sakaï-Suna}, ou fils des Sakas, arrivant de la pointe de la Chersonèse cimbrique et des îles voisines montés sur des barques de cuir. Ils ont vu dans ce mode de navigation une preuve de la plus grande barbarie, et se sont trompés. Au V\textsuperscript{e} siècle, les hommes du Nord possédaient de grands vaisseaux sur la Baltique. Ils étaient habitués depuis longtemps à voir naviguer dans leurs mers les galères romaines, et l’étonnante expédition des Franks qui de la mer Noire étaient revenus dans la Frise, montés sur des navires enlevés à la flotte impériale, aurait suffi, s’il en avait été besoin, pour leur apprendre à construire des bâtiments de cette espèce ; mais ils n’en voulaient pas. Des embarcations tirant très peu d’eau, et pouvant être facilement transportées à bras, convenaient mieux à ces hommes intrépides pour passer de la mer dans les fleuves, des fleuves dans les plus petites rivières ; ils pouvaient remonter de la sorte jusqu’au cœur des provinces, ce qui leur aurait été fort difficile avec de grands navires, et c’est ainsi qu’ils achevèrent la conquête dans la mesure qui leur fut utile. Alors recommença la fusion des races, et le conflit des institutions \footnote{Prosper d’Aquitaine fixe à l’an 441 la conquête définitive par les Anglo-Saxons. Cette prise de possession se distingue de celle de la Gaule par les Franks en deux manières : d’abord, les Saxons ne reçurent pas d’investiture impériale et n’avaient pas à en recevoir, puisque la Grande-Bretagne formait un pays tout à fait indépendant ; ensuite, comme conséquence de ce premier fait, leurs chefs n’eurent jamais l’idée de solliciter les titres de patrices et de consuls, puisqu’ils n’avaient pas à jouer le personnage de magistrats romains.}.\par
 La population britto-romaine, infiniment plus énergique que les Gallo-Romains à cause de son origine en grande partie germanique, maintint en face de ses vainqueurs une situation beaucoup plus fière et beaucoup meilleure \footnote{Les Bretons, dans leurs batailles contre les Saxons, usaient de la tactique romaine. (Palsgrave, \emph{ouvr. cité}, t. I, p. 404.)}. Une partie resta presque indépendante, sauf le vasselage ; une autre, faisant de ses municipalités des espèces de républiques, se borna à une reconnaissance pure et simple du haut domaine saxon et au payement d’un tribut \footnote{Kemble, \emph{Die Sachsen in England}, t. II, pp. 231 et seqq. 249, 254.}. Le reste tomba, à la vérité, dans la situation subordonnée du iarl, du ceorl, suivant les dialectes des nouveaux maîtres ; mais là il fut soutenu et relevé par les lois mêmes de ceux-ci, et l’accession à la propriété foncière, le port des armes, le droit de commandation, ou de choisir son chef, lui restèrent acquis. La population britto-romaine put donc arriver ou prévoir qu’elle arriverait au rang des nobles, des iarls, des ceorls.\par
Le même sentiment qui portait les rois franks à s’entourer de préférence de leudes gaulois engageait également les princes de l’Heptarchie à recruter leurs bandes domestiques parmi les Britto-Romains. Ceux-ci revêtirent donc de très bonne heure des emplois importants à la cour de ces monarques, fils des Ases \footnote{Dans les documents anglo-saxons les plus anciens, on voit figurer, parmi les dignitaires, un grand nombre de noms bretons. (Kemble, \emph{ouvr. cité}, t. I, p. 17.)}. Ils leur enseignè­rent les lois romaines \footnote{Eux-mêmes tenaient cette science de la meilleure source, puisque Papinien avait été chef de l’administration de l’île. (Palsgrave, t. I, p. 322.)} ; ils leur en firent apprécier les avantages gouvernementaux, ils les initièrent à des idées de domination que les guerriers anglo-saxons n’auraient certainement pas contribué à répandre. Mais, et en ceci les conseillers britto-germains différaient essentiellement des leudes gaulois ou mérowings, ils ne sauvèrent pas de la destruction l’extérieur des mœurs romaines, attendu qu’eux-mêmes ne l’avaient jamais qu’assez imparfaitement possédé, et ils ne déposèrent pas dans l’administration le germe de la féodalité, parce que leur pays n’avait été que très passagèrement affecté par le régime des lois bénéficiales \footnote{Palsgrave,\emph{ ouvr. cité}, t. I, p. 495 et seqq.}. L’Angleterre se trouvait donc mise à part, dès le V\textsuperscript{e} siècle, du mode d’existence qui allait prévaloir dans tout le reste de l’Europe.\par
Ce que les ceorls britto-romains inspirèrent très bien aux descendants de Wodan et de Thor, ce fut l’envie de recueillir la succession entière des empereurs nationaux. On voit avec quelque étonnement les princes anglo-saxons les plus habiles, les plus forts, s’entourer des marques romaines de la souveraine puissance, frapper des médailles au type de la louve et des jumeaux, approprier les lois romaines à l’usage de leurs sujets, se plaire à entretenir avec la cour de Constantinople des rapports d’intimité, et revêtir un double titre, celui de \emph{bretwalda}, vis-à-vis de leurs sujets anglo-saxons et bretons, celui de \emph{basileus}, dans leurs documents écrits en langue latine \footnote{Palsgrave, \emph{ouvr. cité}, t. I, p. 420, 488, 563. – Le titre de \emph{bretwalda} entraînait la domination, au moins nominale, sur les nations bretonnes indépendantes de l’île. Plusieurs de ces nations, comme celle de la Cornouailles, par exemple, avaient au X\textsuperscript{e} siècle une noblesse d’origine germanique. (Palsgrave, t. I, p. 411.)}. Ce terme de \emph{basileus}, auquel les rois franks, wisigoths, lombards, n’osèrent jamais prétendre, donnait une situation de grandeur et d’indépendance toute particulière aux souverains qui le portaient. Dans l’île, comme sur le continent, on en comprenait parfaitement la portée, car, lorsque Charlemagne eut pris la succession de Constantin V, il se qualifia très bien, dans une lettre à Egbert, d’empereur des chrétiens orientaux, et salua son correspondant du titre d’empereur des chrétiens occidentaux \footnote{Guillaume le Conquérant porta encore le titre de \emph{basileus.} Il semblerait qu’il fût le dernier souverain anglais qui en ait fait usage. (Palsgrave, \emph{ouvr. cité}, t. I, p. CCCXLIII.)}.\par
Les rapports de race existant entre les Britto-Romains et les tribus germaniques venues du Jutland \footnote{Le titre \emph{d’Anglo-Saxons}, appliqué aux conquérants de l’Angleterre d’une certaine époque, n’implique pas l’idée que tous ces hommes fussent d’une seule nation. Ils avaient parmi eux des Warègues, des Juthungs, des Saxons de Thuringe, etc. (Kemble\emph{, ouvr. cité}, t. I, p. 50 et \emph{Anhang.} A) L’inspection des noms de lieux en Angleterre montre également que, de même que dans l’Europe occidentale, les tribus les plus diverses composaient de leurs contingents les armées de l’invasion.} servaient puissamment à amener entre elles le compromis qui se fondait nécessairement, du côté des vaincus, sur l’abandon de la plupart des importa­tions du sud, sur l’acceptation des idées germaniques. et, du côté des vainqueurs, sur certaines concessions à faire aux nécessités d’une administration plus sévère et plus fortement constituée que celle dont ils s’étaient fait gloire jusqu’alors de porter le joug facile \footnote{Palsgrave insiste avec beaucoup de sagacité sur les rapports d’origine qui existèrent toutes les époques entre les diverses couches des habitants de l’Angleterre, et il en tire les conséquences. (\emph{Ouvr. cité}, t. I, p. 35.)}. On vit s’établir des institutions tenant encore de très près à l’origine scandinave. La tenure des terres dans la forme de l’odel et du féod, l’usage des droits politiques basé exclusivement sur la possession territoriale, le goût de la vie agricole, l’abandon graduel de la plupart des villes \footnote{Kemble, \emph{Die Sachsen in England}, t. II, p. 259 et seqq. – Il arriva pour les villes bretonnes de l’Angleterre ce qui avait eu lieu pour les cités celtiques de la Germanie. Elles n’étaient pas assez riches ni assez fortement constituées pour résister à l’influence hostile du milieu où elles se trouvaient placées. Peu à peu leurs institutions romaines se germanisèrent, et dès lors la vie agricole, les envahissant, tendit à dissoudre leurs bourgeoisies, ou du moins à les transformer.}, l’accroissement du nombre des villages, surtout des métairies isolées, le maintien solide des franchises de l’homme libre, l’influence soutenue des conseils représentatifs, ce furent là autant de traits par lesquels l’esprit arian se donna à reconnaître et témoigna de sa persistance, tandis que des phénomènes d’une nature tout opposée, l’augmentation du nombre des villes, l’indiffé­rence croissante pour la participation aux affaires générales, la diminution du nombre des hommes absolument libres marquaient sur le continent les progrès d’un ordre d’idées d’une tout autre nature.\par
Il n’est pas étonnant que l’aspect assez digne du ceorl anglo-saxon, qui fut plus tard le yeoman, ait plu à la pensée de plusieurs historiens modernes, heureux de le voir libre dans sa vie rustique à une époque où ses analogues du continent, le karl, l’ariman, le \emph{bonus homo}, avaient contracté des obligations souvent fort dures et perdu presque toute ressemblance avec lui. Mais, en se plaçant au point de vue de ces écrivains, il faut, pour être tout à fait juste, considérer aussi ce qui doit constituer pour eux le mauvais côté de la question. L’organisation des classes moyennes, sous les rois saxons comme sous les premiers dynastes normands, n’étant que le résultat d’un concours de circonstances ethniques parachevé, ne prêtait à aucune espèce de perfectionnement \footnote{Et elle n’était pas très relevée. Les gens de la suite du roi, et que l’on nommait en Gaule, sous tes Mérowings, les antrustions, n’étaient pas autorisés à posséder des alods. Leurs armes même devaient, à leur mort, revenir au chef. (Kemble, \emph{ouvr. cité}, t. I, p. 149.)}. La société anglaise d’alors, avec ses avantages, avec ses inconvénients, présentait un tout complet qui n’était susceptible que de décadence. L’existence individuelle n’y était ni sans noblesse ni sans richesse incontestablement ; mais l’absence presque totale de l’élément romanisé la laissait sans éclat et l’éloignait de ce que nous appelons notre civilisation. À mesure que les alliages divers de la population se fondaient davantage, les éléments celtiques, très imbus d’essence finnoise, demeurés dans le fond breton, ceux que l’immigration anglo-saxonne avait jetés dans les masses, ceux que les invasions danoises apportaient encore, tendaient à envahir les éléments germaniques, et il ne faut pas oublier que, quelque abondants que fussent ceux-là, ils diminuaient beaucoup de leur énergie en continuant de se combiner avec une essence hétérogène. Du même coup leur fraîcheur s’en allait avec leurs qualités héroïques, absolument comme un fruit qui passe de main en main perd sa fleur et se flétrit tout en conservant sa pulpe. De là le spectacle que présenta l’Angleterre à l’Europe du XI\textsuperscript{e} siècle. À côté de remarquables mérites politiques une honteuse pauvreté dans le domaine de l’intelligence ; des instincts utilitaires extrêmement développés et qui avaient déjà accumulé dans l’île des richesses extraordinaires, mais nulle délicatesse, nulle élégance dans les mœurs ; des ceorls, plus heureux que les manants français, successeurs des \emph{boni homini} ; mais l’esclavage complet et l’esclavage assez dur, ce qui n’existait presque plus ailleurs \footnote{Palsgrave, \emph{ouvr. cité}, t. I, pp. 21, 30. – Kemble, \emph{Die Sachsen in England}, t. I, p. 150 et seqq. – Au temps de la conquête normande, les Anglo-Saxons en étaient encore à la première phase du servage, dépassée en France depuis les derniers Mérowings. – Le traell scandinave s’appelait dans la Grande-Bretagne \emph{lazzus} et \emph{laet, dio} et \emph{théow}, enfin \emph{wealh.} Les deux premiers noms indiquent la descendance slave des premiers esclaves, probablement amenés de la Germanie ; le dernier indique les Bretons. (T. I, pp. 150, 151, 171 et seqq.)}. Un clergé que l’ignorance et des mœurs basses et ignoblement sensuelles menaient lentement à l’hérésie ou, pour le moins, au schisme ; des souverains qui, ayant continué à gouverner un grand royaume comme jadis ils avaient fait leur odel et leur truste, avaient conservé, sans la déléguer, l’administration de la justice, et se faisaient payer la concession de leur sceau par une prévarication qui se trouvait être légale \footnote{Palsgrave, \emph{ouvr. cité}, t. I, p. 651. – Ce fait doit servir de commentaire, en quelque sorte justificatif, à certaines formes d’exactions de Guillaume le Roux et de Jean sans Terre. Ces souverains ne faisaient qu’appliquer de vieux usages anglo-saxons.}* ; enfin l’extinction de toutes les grandes races pures, et l’avènement au trône du fils d’un paysan, c’étaient là, au temps de la conquête normande, des ombres peu favorables dont le tableau était notablement enlaidi.\par
L’Angleterre eut ce bonheur que l’avènement de Guillaume, sans lui rien ôter de ce qu’elle avait d’organiquement bon \footnote{Palsgrave, \emph{ouvr. cité}, t. I, p. 653. – Cette déclaration d’un des publicistes les plus érudits de l’Angleterre est certainement digne d’être enregistrée. Elle se fonde, en fait, sur des considérations décisives. Guillaume ne toucha pas à l’organisation représentative ; il ne l’abolit pas ; en 1070, il convoqua lui-même un parlement, \emph{witanegemot}, où figurèrent les Saxons, d’après la règle légale. Dans le procès contre le comte normand Odon et l’archevêque Lanfranc de Canterbury, ce fut un tribunal saxon qui jugea la cause, à Pennenden Heath, sous la direction d’un witan anglais, versé dans la connaissance des lois, et d’Egilrik, évêque de Chicester. Enfin la ville d’Exeter déclara à Guillaume qu’en vertu de ses droits, elle lui payerait le tribut, \emph{gafol}, montant à dix-huit livres d’argent, et que, pour subsides de guerre, elle lui donnerait encore la somme des terrains imputable par la loi sur chaque terme de cinq hydes de terre ; qu’elle ne se refusait pas non plus à acquitter les rentes des marais appartenant au domaine royal, mais que les bourgeois ne lui devaient pas le serment d’hommage, qu’ils n’étaient pas ses vassaux, et qu’ils n’étaient pas astreints à le laisser entrer dans leurs murs. – Ces privilèges, qu’Exeter avait en commun avec Winchester, Londres, York et d’autres villes, ne furent pas abrogés par la conquête normande. (Palsgrave, \emph{ouvr. cité}, t. I, p. 631.)}, lui apporta, sous la forme d’une invasion gallo- scandinave, un nombre restreint d’éléments romanisés. Ceux-ci ne réagirent pas d’une manière ruineuse contre la prépondérance du fond teutonique ; ils ne lui enlevèrent pas son génie utilitaire, son esprit politique, mais ils lui infusèrent ce qui lui avait manqué jusqu’alors pour s’associer plus intimement à la croissance de la civilisation nouvelle. Avec le duc de Normandie arrivèrent des Bretons francisés, des Angevins, des Manceaux, des Bourguignons, des hommes de toutes les parties de la Gaule. Ce furent autant de liens qui rattachèrent l’Angleterre au mouvement général du continent et qui la tirèrent de l’isolement où le caractère de sa combinaison ethnique la renfermait, puisqu’elle était restée par trop celto-saxonne dans un temps où le reste du monde européen tendait à se dépouiller de la nature germanique\par
Les Plantagenets et les Tudors continuèrent cette marche civilisatrice en en propageant les causes d’impulsion. De leur temps, l’importation de l’essence romanisée n’eut pas lieu dans des proportions dangereuses ; elle n’atteignit pas au vif les couches inférieures de la nation ; elle agit principalement sur les supérieures, qui partout sont soumises, et le furent là comme ailleurs, à des agents incessants d’étiolement et de disparition. Il en est de l’infiltration d’une race civilisée, bien que corrompue, au milieu des masses énergiques, mais grossières, comme de l’emploi des poisons à faible dose dans la médecine. Le résultat ne saurait en être que salutaire. De sorte que l’Angleterre se perfectionna lentement, épura ses mœurs, polit quelque peu ses surfaces, se rapprocha de la communauté continentale, et, en même temps, comme elle continuait à rester surtout germanique, elle ne donna jamais à la féodalité la direction servile qui lui fut imprimée chez ses voisins \footnote{Palsgrave, \emph{ouvr. cité}, t. I, p. VI : « Allen, with profound erudition, has shown how much of « our monarchical theory is derived, not from the ancien Germans but from the government « of the Empire. » – Cette théorie monarchique ne se développa jamais fortement, et resta toujours exotique et traitée comme telle par l’instinct national, tandis que sur le continent elle acquit à la fin le plein indigénat, et étouffa ce qui lui faisait résistance. En somme, les droits des rois anglais ont toujours vacillé entre les différentes nations des Romains, des Bretons et des nations germaniques, mais avec prépondérance de ces derniers. (Palsgrave, t. I, p. 627.)} ; elle ne permit pas au pouvoir royal de dépasser certaines limites fixées par les instincts nationaux ; elle organisa les corporations municipales sur un plan qui ressembla peu aux modèles romains ; elle ne cessa pas de rendre sa noblesse accessible aux classes inférieures, et surtout elle n’attacha guère les privilèges du rang qu’à la possession de la terre. D’un autre côté, elle revint bientôt à se montrer peu sensible aux connaissances intellectuelles ; elle trahit toujours un dédain marqué pour ce qui n’est pas d’usage en quelque sorte matériel, et s’occupa très peu, au grand scandale des Italiens, de la culture des arts d’agrément \footnote{Sharon Turner, \emph{History of the Anglo-Saxons}, t. III, p. 389 : « The anglo-saxon nation... did « not altain a general or striking eminence in litterature. But society wants other blessings « besides these. The agencies that affected our ancestry took a different course. They « impelled them towards that of political melioration, the great fountain of human « improvement. »}.\par
Dans l’ensemble de l’histoire humaine, il y a peu de situations analogues à celle des populations de la Grande-Bretagne depuis le X\textsuperscript{e} siècle jusqu’à nos jours. On a vu ailleurs des masses arianes ou arianisées apporter leur énergie au milieu des multitudes de composition différente et les douer de puissance en même temps qu’elles en recevaient une culture déjà grande, que leur génie se chargeait de développer dans un sens nouveau ; mais on n’a pas contemplé ces natures d’élite, concentrées en nombre supérieur sur un territoire étroit et ne recevant les immixtions de races plus perfec­tionnées par l’expérience, bien que subalternes par le rang, que suivant des quantités tout à fait médiocres. C’est à cette circonstance exceptionnelle que les Anglais ont dû, avec la lenteur de leur évolution sociale, la solidité de leur empire ; il n’a certes pas été le plus brillant, ni le plus humain, ni le plus noble des États européens, mais il en est encore le plus vigoureux.\par
Cette marche circonspecte et si profitable s’accéléra cependant à dater de la fin du XVII\textsuperscript{e} siècle.\par
Le résultat des guerres religieuses de France avait apporté dans le Royaume-Uni une nouvelle affluence d’éléments français. Cette fois ils n’osèrent plus rentrer dans les classes aristocratiques ; l’effet de relations commerciales, qui partout allait croissant, en jeta une forte proportion au sein des masses plébéiennes, et le sang anglo-saxon fut sérieusement entamé. La naissance de la grande industrie vint encore accroître ce mouvement en appelant sur le sol national des ouvriers de toutes races non germani­ques, des Irlandais en foule, des Italiens, des Allemands slavisés ou appartenant à des populations fortement marquées du cachet celtique.\par
Alors les Anglais purent réellement se sentir entraînés dans la sphère des nations romanisées. Ils cessèrent d’occuper, aussi imperturbablement, ce médium qui auparavant les tenait autant rapprochés pour le moins du groupe scandinave que des nations méridionales, et qui, dans le moyen âge, les avait fait sympathiser surtout avec les Flamands et les Hollandais, leurs pareils sous beaucoup de rapports. À dater de ce moment, la France fut mieux comprise par eux. Ils devinrent plus littéraires dans le sens artiste du mot. Ils connurent l’attrait pour les études classiques ; ils les acceptèrent comme on le faisait de l’autre côté du détroit ; ils prirent le goût des statues, des tableaux, de la musique, et, bien que des esprits depuis longtemps initiés, et doués, par l’habitude, d’une délicatesse plus exigeante, les accusassent d’y porter encore une sorte de rudesse et de barbarie, ils surent recueillir, dans ce genre de travaux, une gloire que leurs ancêtres n’avaient ni connue ni enviée.\par
L’immigration continentale continua et s’agrandit. La révocation de l’édit de Nantes envoya de nombreux habitants de nos provinces méridionales rejoindre dans les villes britanniques la postérité des anciens réfugiés \footnote{Les recherches de M. Weill ont établi que plus de cent mille protestants français ont trouvé, à différentes époques, un refuge en Angleterre.}. La révolution française ne fut pas moins influente, ni dans ce triste sens moins généreuse, et, sans parler de ce courant tout récemment formé qui transporte maintenant en Angleterre une partie de la population de l’Irlande, les autres apports ethniques se multipliant sans relâche, les instincts opposés au sentiment germanique ont indéfiniment continué à abonder au sein d’une société qui, jadis si compacte, si logique, si forte, si peu littéraire, n’aurait pas pu naguère assister sans horreur à la naissance de Byron \footnote{ 
\begin{verse}
If…\\
 Of the great poet-rire of Italy\\
 I dare to build the imitation rhyme\\
 Harsh runic copy of the south’s sublime.\\
\end{verse}
 
\bibl{(Byron, \emph{Dedication of the Prophecy of Dante.})}
}.\par
La transformation est bien sensible ; elle marche d’un pas sûr et se trahit de mille manières. Le système des lois anglaises a perdu de sa solidité ; des réformateurs ne sont pas loin, et les Pandectes sont leur idéal. L’aristocratie trouve des adversaires ; la démocratie, jadis inconnue, proclame des prétentions qui n’ont pas été inventées sur le sol anglo-saxon. Les innovations qui trouvent faveur, les idées qui germent, les forces dissolvantes qui s’organisent, tout révèle la présence d’une cause de transformation apportée du continent. L’Angleterre est en marche pour entrer à son tour dans le milieu de la romanité.
\section[{VI.6. Derniers développements de la société germano-romaine.}]{VI.6. \\
Derniers développements de la société germano-romaine.}
\noindent Rentrons dans l’empire de Charlemagne, puisque c’est là, de toute nécessité, que la civilisation moderne doit naître. Les Germains non romanisés de la Scandinavie, du nord de l’Allemagne et des îles Britanniques ont perdu, par le frottement, la naïveté de leur essence ; leur vigueur est désormais sans souplesse. Ils sont trop pauvres d’idées pour obtenir une grande fécondité ni surtout une grande variété des résultats. Les pays slaves, à ce même inconvénient, ajoutent l’humilité des aptitudes, et cette cause d’incapacité se montrera si forte que, lorsque certains d’entre eux se trouveront en rapports étroits avec la romanité orientale, avec l’empire grec, rien ne sortira de cet hymen. Je me trompe ; il en sortira des combinaisons plus misérables encore que le compromis byzantin.\par
C’est donc au sein des provinces de l’empire d’Occident qu’il faut se transporter pour assister à l’avènement de notre forme sociale. La juxtaposition de la barbarie et de la romanité n’y existe plus d’une manière accusée ; ces deux éléments de la vie future du monde ont commencé à se pénétrer, et, comme pour rendre plus rapide l’achève­ment de la tâche, le travail s’est subdivisé ; il a cessé de se faire en commun sur toute l’étendue du territoire impérial. Des amalgames rudimentaires se sont empressés de se détacher partout de la grande masse ; ils s’enferment dans des limites incertaines, ils imaginent des nationalités approximatives ; la grande agglomération se fend de toutes parts ; la fusion dénature les éléments divers qui bouillonnent dans son sein.\par
Est-ce là un spectacle nouveau pour le lecteur de ce livre ? En aucune façon ; mais c’est un spectacle plus complet de ce qui lui fut déjà montré. L’immersion des races fortes au sein des sociétés antiques s’est opérée à des époques tellement lointaines et dans des régions si éloignées des nôtres, que nous n’en suivons les phases qu’avec difficulté. À peine quelquefois en pouvons-nous saisir plus que les catastrophes finales à de telles distances et de temps et de lieux, multipliées par les grands contrastes d’habitudes intellectuelles existant entre nous et les autres groupes. L’histoire, que soutient mal une chronologie imparfaite, et que souvent déguisent des formes mythi­ques, l’histoire, qui, dénaturée par des traducteurs intermédiaires aussi étrangers à la nation mise en jeu qu’à nous-mêmes, l’histoire, dis-je, reproduit bien moins les faits que leurs images. Encore ces images nous arrivent-elles par une succession de miroirs réfracteurs dont il est quelquefois difficile de rectifier les raccourcis.\par
Mais lorsqu’il s’agit de la civilisation qui nous touche, quelle différence ! Ce sont nos pères qui racontent, et qui racontent comme nous le ferions nous-mêmes. Pour lire leurs récits, nous nous asseyons à la place même où ils écrivirent ; nous n’avons qu’à lever les yeux, et nous contemplons le théâtre entier des événements qu’ils ont décrits. Il nous est d’autant plus facile de bien comprendre ce qu’ils nous disent et de deviner ce qu’ils nous taisent, que nous sommes nous-mêmes les résultats de leurs œuvres ; et, si nous éprouvons un embarras à nous rendre un compte exact et vrai de l’ensemble de leur action, à en suivre les développements, à en éprouver la logique, à en démêler exactement les conséquences, bien loin que nous en puissions accuser la pénurie des renseignements, c’est au contraire à l’opulence embarrassante des détails que notre débilité doit s’en prendre. Nous restons comme accablés sous le monceau des faits. Notre œil les distingue, les sépare, les pénètre avec une peine extrême, parce qu’ils sont trop nombreux et trop touffus, et c’est en nous efforçant de les classer que nos principales erreurs se commettent et nous fourvoient.\par
Nous sommes si directement en jeu dans les souffrances ou les joies, dans les gloires ou les humiliations de ce passé paternel, que nous avons peine à conserver en l’étudiant cette froide impassibilité sans laquelle il n’y a cependant pas de justesse de coup d’œil. En retrouvant dans les capitulaires carlovingiens, dans les chartes de l’âge féodal, dans les ordonnances de l’époque administrative, les premières traces de tous ces principes qui aujourd’hui excitent notre admiration ou soulèvent notre haine, nous ne savons pas le plus souvent contenir l’explosion de notre personnalité.\par
Ce n’est cependant pas avec des passions contemporaines, ce n’est pas avec des sympathies ou des répugnances du jour, qu’il convient d’aborder une pareille étude. Bien qu’il ne soit pas défendu de se réjouir ou de s’attrister des tableaux qu’elle présente, bien que le sort des hommes d’autrefois ne doive pas laisser insensibles les hommes d’aujourd’hui, il faut cependant savoir subordonner ces tressaillements du cœur à la recherche plus noble et plus auguste de la pure réalité. En imposant silence à ses prédilections, on n’est que juste, et partant plus humain. Ce n’est pas seulement une classe, ce ne sont plus quelques noms qui dès lors intéressent, c’est la foule entière des morts ; ainsi cette impartiale pitié que tous ceux qui vivent, que tous ceux qui vivront ont le droit d’exciter, s’attache aux actes de ceux qui ne sont plus, soit qu’ils aient porté la couronne des rois, le casque des nobles, le chaperon des bourgeois ou le bonnet des prolétaires. Pour arriver à cette sérénité de vue, il n’est d’autre moyen que de se refroidir en parlant de nos pères au même degré que nous le sommes en jugeant les civilisations moins directement parentes. Alors ces aïeux ne nous apparaissent plus, et c’est déjà fixer la vraie mesure des choses, que comme les représentants d’une agglomération d’hommes qui a subi précisément l’action des mêmes lois et qui a parcouru les mêmes phases auxquelles nous avons vu assujetties les autres grandes sociétés aujourd’hui mortes ou mourantes.\par
D’après tous les principes exposés et observés dans ce livre, la civilisation nouvelle doit se développer d’abord, dans ses premières formes, sur les points où la fusion de la barbarie et de la romanité possédera, du côté de la première, les éléments les plus chargés de principes hellénistiques, puisque ces derniers renferment l’essence de la civilisation impériale. En effet, trois contrées dominent moralement toutes les autres depuis le IX\textsuperscript{e} siècle jusqu’au XIII\textsuperscript{e} : la haute Italie, les contrées moyennes du Rhin, la France septentrionale.\par
Dans la haute Italie, le sang lombard se trouve avoir gardé une énergie réveillée à différentes fois par des immigrations de Franks. Cette condition remplie, la contrée possède la vigueur nécessaire pour bien servir les destinées ultérieures. D’autre part, la population indigène est chargée d’éléments hellénistiques autant qu’on peut le désirer, et, comme elle est fort nombreuse comparativement à la colonisation barbare, la fusion va promptement l’amener à la prépondérance. Le système communal romain se maintient, se développe avec rapidité. Les villes, Milan, Venise, Florence à leur tête, prennent une importance que, de longtemps encore, les cités n’auront pas ailleurs. Leurs constitutions affectent quelque chose des exigences de l’absolutisme propre aux républiques de l’antiquité. L’autorité militaire s’affaiblit ; la royauté germanique n’est qu’un voile transparent et fragile jeté sur le tout. Dès le XI\textsuperscript{e} siècle, la noblesse féodale est presque totalement anéantie, elle ne subsiste guère qu’à l’état de tyrannie locale et romanisée ; la bourgeoisie lui substitue, dans tous les lieux où elle domine, un patriciat à la manière antique ; le droit impérial renaît, les sciences de l’esprit reparaissent ; le commerce est respecté ; un éclat, une splendeur inconnue rayonne autour de la ligue lombarde. Mais il ne faut pas le méconnaître : le sang teutonique, instinctivement détesté et poursuivi dans toutes ces populations qui se ruent avec fureur vers le retour à la romanité, est précisément ce qui leur donne leur sève et les anime. Il perd chaque jour du terrain ; mais il existe, et l’on en peut voir la preuve dans la longue obstination avec laquelle le droit individuel se maintient, même parmi les hommes d’église, sur ce sol qui si avidement cherche à absorber ses régénérateurs \footnote{Sismondi, \emph{Histoire des républiques italiennes. –} Cet auteur, complètement inattentif aux questions de race, donne avec une exactitude qui n’en est que plus frappante une foule d’indications ethniques dans le sens indiqué ici. Mais ce qu’on peut lire de mieux à cet égard, c’est le poème d’un contemporain, le moine Gunther (\emph{Ligurinus, sive de rebus gestis imperatori, Cæsaris Friderici Primi Aug., cognomento Ænobarbi libri X}, Heydelbergæ, 1812, in-8°). Ce poème se trouve aussi imprimé dans des collections. Il peint avec une vérité admirable, et qui n’est ni sans grandeur ni sans beauté, l’antagonisme violent et irréconciliable des groupes romains et barbares. – Voir aussi Muratori, \emph{Script. rerum Italic.}}.\par
De nombreux États se modèlent de leur mieux, bien qu’avec des nuances innombrables, d’après le prototype lombard. Les provinces mal réunies du royaume de Bourgogne, la Provence, puis le Languedoc, la Suisse méridionale, lui ressemblent sans avoir son éclat. Généralement l’élément barbare est trop affaibli dans ces contrées pour prêter autant de forces à la romanité \footnote{Dans toutes ces contrées, des établissements germaniques de très faible étendue ont conservé leur individualité jusqu’à nos jours. Ce que sont, dans l’Italie orientale, la république de Saint-Marin et les VII et XIII Communes, les Teutons du mont Rosa et du Valais le sont également. – On trouve également des débris scandinaves dans certaines parties des petits cantons.}. Dans le centre et dans le sud de la Péninsule, il est presque absent ; aussi n’y voit-on que des agitations sans résultat et des convulsions sans grandeur. Sur ces territoires, les invasions teutoniques, n’ayant été que passagères, n’ont produit que des résultats incomplets, n’ont agi que dans un sens dissolvant. Le désordre ethnique n’en est devenu que plus considérable. De nombreux retours des Grecs et les colonisations sarrasines n’ont pas été de nature à y porter remède. Un moment, la domination normande a donné une valeur inattendue à l’extrémité de la Péninsule et à la Sicile. Malheureusement ce courant, toujours assez minime, se tarit bientôt, de sorte que son influence va se mourant, et les empereurs de la maison de Hohenstauffen en épuisent les derniers filons.\par
Lorsque le sang germanique eut presque achevé, au XV\textsuperscript{e} siècle, de se subdiviser dans les masses de la haute Italie, la contrée entra dans une phase analogue à celle que traversa la Grèce méridionale après les guerres persiques. Elle échangea sa vitalité politique contre un grand développement d’aptitudes artistiques et littéraires. Sous ce point de vue, elle atteignit à des hauteurs que l’Italie romaine, toujours courbée sur la copie des modèles athéniens, n’avait point atteintes. L’originalité manquant à cette devancière lui fut acquise dans une noble mesure ; mais ce triomphe fut aussi peu durable qu’il l’avait été chez les contemporains de Platon : à peine, comme pour ceux-ci, brilla-t-il une centaine d’années, et, lorsqu’il fut éteint, l’agonie de toutes les facultés recommença. Le XVII\textsuperscript{e} et le XVIII siècle n’ont rien ajouté à la gloire de l’Italie, et certes lui ont beaucoup ôté.\par
Sur les bords du Rhin et dans les provinces belgiques, les éléments romains étaient primés numériquement par les éléments germaniques. En outre, ils étaient nativement plus affectés par l’essence utilitaire des détritus celtiques que ne le pouvaient être les masses indigènes de l’Italie. La civilisation locale suivit la direction conforme aux causes qui la produisaient. Dans l’application qui y fut faite du droit féodal, le système impérial des bénéfices se montra peu puissant ; les liens par lesquels il rattachait le possesseur de fief à la couronne furent toujours très relâchés, tandis qu’au contraire les doctrines indépendantes de la législation primitivement germanique se maintinrent assez pour conserver longtemps aux propriétaires de châteaux une individualité libre qu’ils n’avaient plus ailleurs. La chevalerie du Hainaut, celle du Palatinat méritèrent, jusque dans le XVI\textsuperscript{e} siècle, d’être citées comme les plus riches, les plus indépendantes et les plus fières de l’Europe. L’empereur, leur suzerain immédiat, avait peu de prise sur elles, et les princes de second ordre, beaucoup plus nombreux qu’ailleurs dans ces provinces, étaient impuissants à leur faire plier le cou. Les progrès de la romanité s’effectuaient néanmoins, parce que la romanité était trop vaste pour ne pas être irrésistible à la longue ; ils amenèrent, bien que très laborieusement, la reconnaissance imparfaite des règles principales du droit de Justinien. Alors la féodalité perdit la plupart de ses prérogatives, mais elle en conserva cependant assez pour que l’explosion révolutionnaire de 1793 trouvât plus à niveler dans ces pays que dans aucun autre. Sans ce renfort, sans ce secours étranger apporté aux éléments locaux opposants, les restes de l’organisation féodale se seraient défendus longtemps encore dans les électorats de l’ouest, et ils auraient prouvé autant de solidité que sur les autres points de l’Allemagne, où ces dernières années seulement ont consommé leur destruction.\par
En face de cette noblesse si lente à succomber, la bourgeoisie fit son chef-d’œuvre en érigeant l’édifice hanséatique, combinaison d’idées celtiques et slaves où ces dernières dominaient, mais que toujours animait une somme suffisante de fermeté germanique. Couvertes de la protection impériale, on ne vit point les cités associées, impatientes de tutelle, protester à tout propos contre ce joug à la manière des villes d’Italie. Elles abandonnèrent volontiers les honneurs du haut domaine à leurs souverains, et ne surveillèrent avec jalousie que la libre administration de leurs intérêts communaux et les avantages de leur commerce. Chez elles, point de luttes intestines, point de tendances à l’absolutisme républicain, mais le prompt abandon des doctrines exagérées, qui ne se montrent dans leurs murs que comme un accident. L’amour du travail, la soif du profit, peu de passion, beaucoup de raison, un attachement fidèle à des libertés positives, voilà leur naturel. Ne méprisant ni les sciences ni les arts, s’associant d’une façon grossière mais active au goût de la noblesse pour la poésie narrative, elles avaient peu conscience de la beauté, et leur intelligence essentiellement attachée à des conquêtes pratiques n’offre guère les côtés brillants du génie italien à ses différentes époques. Cependant l’architecture ogivale leur dut ses plus beaux monuments. Les églises et les hôtels de ville des Flandres et de l’Allemagne occiden­tale montrent encore que ce fut la forme favorite et particulièrement bien comprise de l’art dans ces régions ; cette forme semble avoir correspondu directement à la nature intime de leur génie, qui ne s’en écarta guère sans perdre son originalité.\par
L’influence exercée par les contrées rhénanes fut très grande sur toute l’Allemagne ; elle se prolongea jusque dans l’extrême nord. C’est en elles que les royaumes scandinaves aperçurent longtemps la nuance de civilisation qui, se rappro­chant davantage de leur essence, leur convenait le mieux. À l’est, du côté des duchés d’Autriche, la dose du sang germanique étant plus faible, la mesure du sang celtique moins grande, et les couches slaves et romaines tendant à exercer une action prépondérante, l’imitation se tourna de bonne heure vers l’Italie, non toutefois sans être sensible aux exemples venus du Rhin, ni même, par ailleurs, aux suggestions slaves. Les contrées gouvernées par la maison de Habsbourg furent essentiellement un terrain de transition, comme la Suisse, qui, d’une manière moins compliquée sans doute, partageait son attention entre les modèles rhénans et ceux de la haute Italie. Dans les anciens territoires helvètes, le point mitoyen des deux systèmes était Zurich. Je répéterai ici, pour compléter le tableau, que, aussi longtemps que l’Angleterre demeura plus particulièrement germanique, après qu’elle eut à peu près absorbé les apports français de la conquête normande et avant que les immigrations protestantes eussent commencé à la rallier à nous, ce furent les formes flamandes et hollandaises qui lui furent les plus sympathiques. Elles rattachèrent de loin ses idées à celles du groupe rhénan.\par
Vient maintenant le troisième centre de civilisation, qui avait son foyer à Paris. La colonisation franke avait été puissante aux environs de cette ville. La romanité s’y était composée d’éléments celtiques au moins aussi nombreux que sur les bords du Rhin, mais beaucoup plus hellénisés, et, en somme, elle dominait l’action barbare par l’importance de sa masse. De bonne heure, les idées germaniques reculèrent devant elle \footnote{Les dernières traces en sont visibles dans les romans de Garin. Voir à ce sujet la savante dissertation de M. Paulin Pâris dans son édition d’une partie du poème, et quelques idées émises par M. Edelestand du Méril au début de la \emph{Mort de Garin. –} Voir aussi dom Calmet, \emph{Histoire de Lorraine} ; Wusseburg, \emph{Antiquités de la Gaule Belgique}, liv. III, p. 157.}. Dans les plus anciens poèmes du cycle carlovingien, les héros teutoniques sont pour la plupart oubliés ou représentés sous des couleurs odieuses, par exemple, les chevaliers de Mayence, tandis que les paladins de l’ouest, tels que Roland, Olivier, ou même du midi, comme Gérars de Roussillon, occupent les premières places dans l’estime générale. Les traditions du Nord n’apparaissent que de plus en plus défigurées sous un habit romain.\par
La coutume féodale pratiquée dans cette région s’inspire de plus en plus des notions impériales, et, circonvenant avec une infatigable activité la résistance de l’esprit contraire, complique à l’excès l’état des personnes, déploie une richesse de restrictions, de distinctions, d’obligations dont on n’avait pas l’idée ni en Allemagne, où la tenure des fiefs était plus libre, ni en Italie, où elle était plus soumise à la prérogative du souverain. Il n’y eut qu’en France où l’on vit le roi, suzerain de tous, pouvoir être en même temps l’arrière-vassal d’un de ses hommes, et, comme tel, soumis théoriquement à l’obligation de le servir contre lui-même, sous peine de forfaiture.\par
Mais la victoire de la prérogative royale était au fond de tous ces conflits, par la raison que leur action incessante favorisait l’élévation des basses classes de la population, et ruinait l’autorité des classes chevaleresques. Tout ce qui ne possédait pas de droits personnels ou territoriaux était en droit d’en acquérir, et, au rebours, tout ce qui avait à un degré quelconque les uns ou les autres, les voyait insensiblement s’atténuer \footnote{Guérard, \emph{le Polyptique d’Irminon}, t. I, p. 251 : « À partir de la fin du IXe siècle, le colon et « le lide deviennent de plus en plus rares dans les documents qui concernent la France, et « ces deux classes de personnes ne tardèrent pas à disparaître. Elles sont, en partie, « remplacées par celle des \emph{colliberti}, qui n’a pas une longue existence. Le serf, à son tour, « se montre moins fréquemment, et c’est le \emph{villanus}, le \emph{rusticus, l’homo potestatis} qui lui « succèdent. » On voit par là quelle rapidité de modifications, toutes favorables à la romanité, s’opérait clans cette société en fusion. (Voir aussi, \emph{même ouvr.}, t. I, p. 392.)}. Dans cette situation critique pour tout le monde, les antagonismes et les conflits éclatèrent avec une extrême vivacité et durèrent plus longtemps qu’ailleurs, parce qu’ils se prononcèrent plus tôt qu’en Allemagne et finirent plus tard qu’en Italie.\par
La catégorie des cultivateurs libres, des hommes de guerre indépendants, disparut peu à peu devant le besoin général de protection. De même on vit de moins en moins des chevaliers n’obéissant qu’au roi. Moyennant l’abandon d’une partie de ses droits, chacun voulut et dut acheter l’appui de plus fort que lui. De cet enchaînement universel des fortunes résultèrent beaucoup d’inconvénients pour les contemporains et pour leurs descendants, un acheminement irrésistible vers le nivellement universel \footnote{Les appréciations de Palsgrave sur la constitution politique de la Gaule dans la première partie des âges moyens sont, en grande partie, ce que l’on a écrit de plus vrai et de plus clair sur ce sujet, en apparence compliqué. Il montre très bien : 1° que l’idée d’étudier la France d’alors dans son étendue d’aujourd’hui est une erreur, et que nulle institution d’alors ne pouvait viser à satisfaire un tel ensemble, puisqu’il n’existait pas ; 2° il établit que les communes modernes n’ont jamais commencé, parce que les communes gallo-romaines et gallo-frankes n’ont jamais fini. (Palsgrave, \emph{the Rise and Progress of the English Commonwealth}, t. I, pp. 494, 545 et seqq.) – Voir également C. Leber, \emph{Histoire du pouvoir municipal en France}, Paris, 1829, in-8°. Ouvrage excellent et qui a été mis à contribution plus souvent que les emprunteurs ne l’ont avoué. – Raynouard, \emph{Histoire du droit municipal en France}, Paris, 1829, 2 vol in-8°. Livre tout romain.}.\par
Les communes n’atteignirent jamais un bien haut degré de puissance. Les grands fiefs eux-mêmes devaient à la longue s’affaiblir et cesser d’exister. De grandes indépendances personnelles, des individualités fortes et fières, constituaient autant d’anomalies, qui tôt ou tard allaient fléchir devant l’antipathie si naturelle de la romanité. Ce qui persista le plus longtemps, ce fut le désordre, dernière forme de protestation des éléments germaniques. Les rois, chefs instinctifs du mouvement romain, eurent encore bien de la peine à venir à bout de ces suprêmes efforts. Des convulsions générales et terribles, des douleurs universelles, déchirèrent ces temps héroïques. Personne n’y fut à l’abri des plus méchants coups de la fortune. Comment donc ne pas mettre un grain de mépris dans le sourire, à voir de nos jours ce qui s’appelle philanthropie croire légitime de s’apitoyer sur ce qu’étaient alors les basses classes, compter les chaumières détruites, et supputer le dommage des moissons ravagées ? Quel bon sens, quelle vérité, quelle justice de rapporter les choses du X\textsuperscript{e} siècle à la même mesure que les nôtres ! Il s’agit bien là de moissons, de chaumières et de paysans mal satisfaits ! Si l’on a des larmes en réserve, c’est à la société tout entière, c’est à toutes les classes, c’est à l’universalité des hommes qu’on les doit.\par
Mais pourquoi des larmes et de la pitié ? Cette époque n’appelle pas la compassion. Ce n’est pas le sentiment que fait naître la lecture attentive des chroniques ; soit que l’on s’arrête sur les pages austères et belliqueuses de Villehardouin, sur les récits merveilleux de l’Aragonais Raymond Muntaner, ou sur les souvenirs pleins de sérénité, de gaieté, de courage, du noble Joinville, soit qu’on parcoure la biographie passionnée d’Abélard, les notes plus monacales et plus calmes de Guibert de Nogent, ou tant d’autres écrits pleins de vie et de charme qui nous sont restés de ces temps, l’imagination est confondue par la dépense de cœur, d’intelligence et d’énergie qui s’y fait de toutes parts. Souvent plus enthousiaste que sèchement raisonnable dans ses applications, la pensée d’alors est toujours vigoureuse et saine. Elle est inspirée par une curiosité, par une activité sans bornes ; elle ne laisse rien sans y toucher. En même temps qu’elle a des forces inépuisables pour alimenter sans relâche la guerre étrangère et la guerre intérieure, qu’à demi fidèle encore à la prédilection des Franks pour le glaive, elle entretient le fracas des armes de royaume à royaume, de cité à cité, de village à village, de manoir à manoir, elle trouve le goût et le temps de sauver les trésors de la littérature classique, et de les méditer d’une manière erronée peut être à notre point de vue, mais à coup sûr originale. C’est là, en toutes choses, un suprême mérite, et, dans ce cas particulier, un mérite d’autant plus éclatant que nous en avons profité, et qu’il constitue toute la supériorité de la civilisation moderne sur l’ancienne romanité. Celle-ci n’avait rien inventé, n’avait fait que prendre, tant bien que mal et de toutes mains, des résultats des produits d’ailleurs flétris par le temps. Nous, nous avons créé des conceptions nouvelles, nous avons fait une civilisation, et c’est au moyen âge que nous sommes redevables de cette grande œuvre. L’ardeur féodale, infatigable dans ses travaux, ne se borne pas à persévérer de son mieux dans l’esprit conservateur des barbares pour ce qui touche au legs romain. Elle ressaisit encore, elle retouche incessamment ce qu’elle peut retrouver des traditions du Nord et des fables celtiques ; elle en compose la littérature illimitée de ses poèmes, de ses romans, de ses fabliaux, de ses chansons, ce qui serait incomparable, si la beauté de la forme répondait à la richesse illimitée du fond. Folle de discussion et de polémique, elle aiguise les armes déjà si subtiles de la dialectique alexandrine, elle épuise les thèmes théologiques, en extrait de nouvelles formules, fait naître dans tous les genres de philosophie les esprits les plus audacieux et les plus fermes, ajoute aux sciences naturelles, agrandit les sciences mathématiques, s’enfonce dans les profondeurs de l’algèbre. Secouant de son mieux la complaisance pour les hypothèses où s’est complue la stérilité romaine, elle sent déjà le besoin de voir de ses yeux et de toucher de ses mains avant que de prononcer. Les connaissances géographiques servent puissamment et exactement ces dispositions, et les petits royaumes du XIII\textsuperscript{e} siècle, sans ressources matérielles, sans argent, sans ces excitations accessoires et mesquines de lucre et de vanité qui déterminent tout de nos jours, mais ivres de foi religieuse et de juvénile curiosité, savent trouver chez eux des Plan-Carpin, des Maundevill, des Marco Polo, et pousser sur leurs pas des nuées de voyageurs intrépides vers les coins les plus reculés du monde, que ni les Grecs ni les Romains n’avaient même jamais eu la pensée d’aller visiter.\par
Cette époque a pu beaucoup souffrir, je le veux ; je n’examinerai pas si son imagination vive et sa statistique imparfaite, commentées par le dédain que nous aimons à éprouver pour tout ce qui n’est pas nous, n’en ont pas sensiblement exagéré les misères. Je prendrai les fléaux dans toute l’étendue vraie ou fausse qui leur est attribuée, et je demanderai seulement si, au milieu des plus grands désastres, on est vraiment bien malheureux quand on est si vivace ? Vit-on nulle part que le serf opprimé, le noble dépouillé, le roi captif aient jamais tourné de désespoir leur dernière arme contre eux-mêmes ? Il semblerait que ce qui est plus vraiment à plaindre, ce sont les nations dégénérées et bâtardes qui, n’aimant rien, ne voulant rien, ne pouvant rien, ne sachant où se prendre au sein des accablants loisirs d’une civilisation qui décline, considèrent avec une morne indulgence le suicide ennuyé d’Apicius.\par
La proportion spéciale des mélanges germaniques et gallo-romains dans les populations de la France septentrionale, en amenant par des voies douloureuses, mais sûres, l’agglomération en même temps que l’étiolement des forces, fournit aux différents instincts politiques et intellectuels le moyen d’atteindre à une hauteur moyenne, il est vrai, mais généralement assez élevée pour attirer à la fois les sympa­thies des deux autres centres de la civilisation européenne. Ce que l’Allemagne ne possédait pas, et qui se trouvait dans une trop grande plénitude en Italie, nous l’avions sous des proportions restreintes qui le rendaient compréhensible à nos voisins du nord ; et, d’autre part, telles provenances d’origine teutonique, très mitigées par nous, séduisaient les hommes du sud, qui les auraient repoussées, si elles leur fussent parvenues plus complètes. Cette sorte de pondération développa le grand crédit où l’on vit, aux XII\textsuperscript{e} et XIII\textsuperscript{e} siècles, parvenir la langue française chez les peuples du Nord comme chez ceux du midi, à Cologne comme à Milan. Tandis que les minnesingers traduisaient nos romans et nos poèmes, Brunetto Latini, le maître du Dante, écrivait en français, et de même les rédacteurs des mémoires du Vénitien Marco Polo. Ils considéraient notre idiome comme seul capable de répandre dans l’Europe entière les nouvelles connaissances qu’ils voulaient propager. Pendant ce temps, les écoles de Paris attiraient tout ce qu’il y avait de par le monde d’hommes savants et d’esprits studieux. Ainsi les âges féodaux furent spécialement pour la France d’au delà de la Seine une période de gloire et de grandeur morale, que n’obscurcirent nullement les difficultés ethniques dont elle était travaillée \footnote{ \noindent Au XIII\textsuperscript{e} siècle, on exigeait d’un chevalier accompli les mêmes perfections intellectuelles que les Scandinaves imposaient jadis à leurs jarls. Il devait surtout connaître plusieurs langues et les poésies qui les illustraient. Guillaume de Nevers parlait avec une égale facilité le bourguignon, le français, le flamand et le breton. En Allemagne, on faisait venir des maîtres de France pour instruire les enfants nobles dans la langue qu’ils ne devaient pas ignorer. Les vers suivants de \emph{Berthe aux grands piés} confirment cet usage :\par
 
\begin{verse}
« Tout droit a celui tems que je ci vous decris\\
 Avoit une coutume ens el Tyois païs\\
 Que tout li grand seignor, li conte et li marchis\\
 Avoient, entour aus, gent françoise tous-dis\\
 Pour aprendre françois leurs filles et leurs fils,\\
 Li rois et la royne et Berte o le cler vis\\
 Savent pres d’aussi bien le françois de Paris\\
 Com se il fussent nés el bour à Saint-Denis »\\
 « ... François savoit Aliste...\\
 C’est la fille à la Serve »\\
\end{verse}
 
\bibl{(Paulin Pâris, \emph{li Romans de Berte aux grans piés}, Paris, 1836, in-12, p. 10.)}
}.\par
Mais l’extension du royaume des premiers Valois vers le sud, en augmentant dans une proportion considérable l’action de l’élément gallo-romain, avait préparé et commença, avec le XIV\textsuperscript{e} siècle, la grande bataille qui, sous le couvert des guerres anglaises, fut de nouveau livrée aux éléments germanisés \footnote{La fusion du sud et du nord de la France fut assurée par le mélange ethnique qui eut lieu après la guerre des Albigeois. Dans un parlement tenu à Pamiers en 1212, Simon de Monfort fit décider que les veuves et les filles héritières de fiefs nobles, dans les provinces vaincues, ne pourraient épouser que des Français pendant les dix années qui allaient suivre. De là, transplantation d’un grand nombre de familles picardes, champenoises, tourangelles en Languedoc, et extinction de beaucoup de vieilles maisons gothiques.}. La législation féodale, alourdissant de plus en plus les obligations des possesseurs de terres envers la royauté, et diminuant de leurs droits, proclama bientôt, avec une entière franchise, sa prédilection pour des doctrines encore plus purement romaines. Les mœurs publiques, s’associant à cette tendance, portèrent à la chevalerie un coup terrible en transformant contre elle les idées jusqu’alors admises par elle-même au sujet du point d’honneur.\par
 L’honneur avait été jadis chez les nations arianes, était presque encore resté pour les Anglais et même pour les Allemands, une théorie du devoir qui s’accordait bien avec la dignité du guerrier libre. On peut même se demander si, sous ce mot \emph{d’honneur}, le gentilhomme immédiat de l’Empire et le tenancier des Tudors ne com­prenaient pas surtout la haute obligation de maintenir ses prérogatives personnelles au-dessus des plus puissantes attaques. Dans tous les cas, il n’admettait pas qu’il en dût faire le sacrifice à personne. Le gentilhomme français fut, au contraire, sommé de reconnaître que les obligations strictes de l’honneur l’astreignaient à tout sacrifier à son roi, ses biens, sa liberté, ses membres, sa vie. Dans un dévouement absolu consista pour lui l’idéal de sa qualité de noble, et, parce qu’il était noble, il n’y eut pas d’agres­sion de la part de la royauté qui pût le relever, en stricte conscience, de cette abnégation sans bornes. Cette doctrine, comme toutes celles qui s’élèvent à l’absolu, ne manquait certainement pas de beauté ni de grandeur. Elle était embellie par le plus brillant courage  ; mais ce n’était réellement qu’un placage germanique sur des idées impériales ; sa source, si l’on veut la rechercher à fond, n’était pas loin des inspirations sémitiques, et la noblesse française, en l’acceptant, devait à la fin tomber dans des habitudes bien voisines de la servilité.\par
Le sentiment général ne lui laissa pas le choix. La royauté, les légistes, la bour­geoisie, le peuple, se figurèrent le gentilhomme indissolublement voué à l’espèce d’honneur que l’on inventait : le propriétaire armé commença dès lors à ne plus être la base de l’État ; à peine en fut-il encore le soutien. Il tendit à en devenir la décoration.\par
Il est inutile d’ajouter que, s’il se laissa ainsi dégrader, c’est que son sang n’était plus assez pur pour lui donner la conscience du tort qu’on lui faisait, et lui fournir des forces suffisantes pour la résistance. Moins romanisé que la bourgeoisie, qui à son tour l’était moins que le peuple, il l’était beaucoup cependant ; ses efforts attestèrent, par la dose d’énergie qu’on y peut constater, la mesure dans laquelle il possédait encore les causes ethniques de sa primitive supériorité \footnote{La décomposition ethnique de la noblesse française avait commencé du jour où les leudes germaniques s’étaient alliés au sang des leudes gallo-romains ; mais elle avait marché vite, en partie parce que les guerriers germaniques s’étaient éteints en grand nombre dans les guerres incessantes, et parce que des révolutions fréquentes leur avaient substitué des hommes venus de plus bas. C’est ainsi que, sur l’autorité d’une chronique (\emph{Gesta Consul. Andegav}., 2), M. Guérard constate une des phases principales de cette dégénération : « Au milieu des troubles et des secousses de la société, il s’éleva de toutes parts des hommes nouveaux sous le règne de Charles le Chauve. De petits vassaux s’érigèrent en grands feudataires et les officiers publics du royaume en seigneurs presque indépendants. » (\emph{Ouvr. cité}. t. I, p. 205.)}. Ce fut dans les contrées où avaient existé les principaux établissements des Franks que l’opposition chevaleresque se signala davantage ; au delà de la Loire, il n’y eut pas, en général, une volonté aussi persistante. Enfin, avec le temps, à des nuances près, un niveau de soumission s’étendit partout, et la romanité commença à reparaître, presque reconnaissable, comme le XV\textsuperscript{e} siècle finissait.\par
Cette explosion des anciens éléments sociaux fut puissante, extraordinaire ; elle usa avec empire des alliages germaniques qu’elle avait réussi à dompter et à tourner en quelque sorte contre eux-mêmes ; elles les employa à battre en brèche les créations qu’ils avaient jadis produites en commun avec elle ; elle voulut reconstruire l’Europe sur un nouveau plan de plus en plus conforme à ses instincts, et avoua hautement cette prétention.\par
L’Italie du sud et celle du centre se retrouvaient à peu près à la même hauteur que la Lombardie déchue. Les rapports que cette dernière contrée avait, quelques siècles en çà, entretenus avec la Suisse et la Gaule méridionale étaient fort relâchés ; la Suisse était plus inclinée vers l’Allemagne rhénane, le sud de la Gaule vers les provinces moyennes. Et quel était le lien commun de ces rapprochements ? L’élément romain à coup sûr, mais, dans cet élément composite, plus particulièrement l’essence celtique qui reparaît de son côté. La preuve en est que, si la partie sémitisée avait agi en cette circonstance, la Suisse et le sud de la Gaule auraient resserré leurs anciens rapports avec l’Italie, au lieu de les rendre moins intimes.\par
L’Allemagne tout entière, agissant sous la même influence celtique, se chercha, et maria plus étroitement ses intérêts autrefois si sporadiques. L’élément romano-galli­que, dans sa résurrection, trouvait peu de difficultés à se combiner avec les principes slaves, en vertu de l’antique analogie. Les pays scandinaves devinrent plus attentifs pour un pays qui avait eu le temps de nouer avec eux des rapports ethniques non germains déjà suffisamment considérables. Au milieu de ce resserrement universel, les contrées rhénanes perdirent leur suprématie, et il devait nécessairement en être ainsi, puisque c’était la nature gallique qui désormais y avait le dessus.\par
Quelque chose de grossier et de commun, qui n’appartenait ni à l’élément germa­nique ni au sang hellénisé, s’infiltra partout. La littérature chevaleresque disparut des forteresses qui bordent le cours du Rhin ; elle fut remplacée par les compositions railleuses, bassement obscènes, lourdement grotesques de la bourgeoisie des villes. Les populations se complurent aux trivialités de Hans Sachs. C’est cette gaieté que nous appelons si justement la \emph{gaieté gauloise}, et dont la France produisit, à cette même époque, le plus parfait spécimen, comme, en effet, elle en avait le droit inné, en faisant naître les facéties de \emph{haulte graisse}, compilées par Rabelais, le géant de la facétie.\par
Toute l’Allemagne se trouva capable de rivaliser de mérite avec les villes rhénanes dans la nouvelle phase de civilisation dont cette bonne humeur frondeuse fut l’enseigne. La Saxe, la Bavière, l’Autriche, le Brandebourg même, se virent portés à peu près sur un même plan, tandis que du côté du sud, et la Bourgogne servant de lien, la France entière, dont l’Angleterre arrivait à goûter le génie, la France se sentait en plus parfaite harmonie d’humeur avec ses voisins du nord et de l’ouest, de qui elle reçut alors à peu près autant qu’elle leur donna.\par
L’Espagne, à son tour, fut atteinte par cette assimilation générale des instincts en voie de conquérir tous les pays de l’Occident. Jusqu’alors cette terre n’avait fait des emprunts à ses voisins du nord que pour les transformer d’une manière à peu près complète, unique moyen de les rendre accessibles au goût spécial de ses populations combinées d’une manière si particulière. Tant que l’élément gothique avait eu quelque force extérieurement manifestée, les relations de la péninsule ibérique avaient été au moins aussi fréquentes avec l’Angleterre qu’avec la France, tout en restant médiocres. Au XVI\textsuperscript{e} siècle, l’élément romano-sémitique prenant de la puissance, ce fut avec l’Italie, et l’Italie du sud, que les royaumes de Ferdinand s’entendirent le mieux, bien qu’ils tinssent aussi à nous par le lien du Roussillon. N’ayant qu’une assez faible teinte celtique, le genre d’esprit trivial des bourgeoisies du Nord ne prit que difficilement pied chez elle, comme aussi dans l’autre péninsule ; cependant il ne laissa pas de s’y montrer, mais avec une dose d’énergie et d’enflure toute sémitique, avec une verve locale qui n’était pas la force musculeuse de la barbarie germanique, mais qui, dans son espèce de délire africain, produisit encore de très grandes choses. Malgré ces restes d’originalité, on sent bien que l’Espagne avait perdu la meilleure part de ses forces gothiques, qu’elle éprouvait, comme tous les autres pays, l’influence restaurée de la romanité, par ce fait seul qu’elle sortait de son isolement.\par
Dans cette renaissance, comme on l’a appelée avec raison, dans cette résurrection du fond romain, les instincts politiques de l’Europe se montrant plus assouplis à mesure que l’on s’avançait au milieu de populations plus débarrassées de l’instinct germanique, c’était là que l’on trouvait moins de nuances dans l’état des personnes, une plus grande concentration des forces gouvernementales, plus de loisirs pour les sujets, une préoccupation plus exclusive du bien-être et du luxe, partant plus de civilisation à la mode nouvelle. Les centres de culture se déplacèrent donc. L’Italie, prise dans son ensemble, fut encore une fois reconnue pour le prototype sur lequel il fallait s’efforcer de se régler. Rome remonta au premier rang. Quant à Cologne, Mayence, Strasbourg, Liège, Gand, Paris même, toutes ces villes, naguère si admirées, durent se contenter de l’emploi d’imitateurs plus ou moins heureux. On ne jura plus que par les Latins et les Grecs, ces derniers, bien entendu, compris à la façon latine. On redoubla de haine pour tout ce qui sortait de ce cercle ; on ne voulut plus reconnaître ni dans la philosophie, ni dans la poésie, ni dans les arts, ce qui avait forme ou couleur germanique ; ce fut une croisade inexorable et violente contre ce qui s’était fait depuis un millier d’années. On pardonna à peine au christianisme.\par
Mais si l’Italie, par ses exemples, réussit à se maintenir à la tête de cette révolution pendant quelques années, où il ne fut encore question d’agir que dans la sphère intellectuelle, cette suprématie lui échappa aussitôt que la logique inévitable de l’esprit humain voulut de l’abstraction passer à la pratique sociale. Cette Italie si vantée était redevenue trop romaine pour pouvoir servir même la cause romaine ; elle s’affaissa promptement dans une nullité semblable à celle du IV\textsuperscript{e} siècle, et la France, sa plus proche parente, continua, par droit de naissance, la tâche que son aînée ne pouvait pas accomplir. La France poursuivit l’œuvre avec une vivacité de procédés qu’elle pouvait employer seule. Elle dirigea, exécuta en chef l’absorption des hautes positions sociales au sein d’une vaste confusion de tous les éléments ethniques que leur incohérence et leur fractionnement lui livraient sans défense. L’âge de l’égalité était revenu pour la plus grande partie des populations de l’Europe ; le reste n’allait pas cesser désormais de graviter de son mieux vers la même fin, et cela aussi rapidement que la constitution physique des différents groupes voudrait le permettre. C’est l’état auquel on est aujourd’hui parvenu \footnote{Amédée Thierry, \emph{Histoire de la Gaule sous l’administration romaine}, t. I, \emph{Introd.}, p. 347 « Nous-mêmes, Européens du XIX\textsuperscript{e} siècle, quels idiomes parlons-nous pour la plupart ? À « quel cachet est marqué notre génie littéraire ? Qui nous a fourni nos théories de l’art ? « Quel système de droit est écrit dans nos codes, ou se retrouve au fond de nos coutumes ? « Enfin, quelle est notre religion à tous ? La réponse à ces questions nous prouve la vitalité « de ces institutions romaines dont nous portons encore l’empreinte après quinze siècles, « empreinte qui, au lieu de s’effacer par l’action moderne, ne fait, en quelque sorte, que se « reproduire plus nette et plus éclatante, à mesure que nous nous dégageons de la barbarie féodale. »}.\par
Les tendances politiques ne suffiraient pas à caractériser cette situation d’une manière sûre ; elles pourraient, à la rigueur, être considérées comme transitoires et provenant de causes secondes. Mais ici, outre qu’il n’est guère possible de n’attribuer qu’une importance de passage à la persistante direction des idées pendant cinq à six siècles, nous voyons encore des marques de la réunion future des nations occidentales, au sein d’une romanité nouvelle, dans la ressemblance croissante de toutes leurs pro­ductions littéraires et scientifiques, et surtout dans le mode singulier de développement de leurs idiomes.\par
Les uns et les autres ils se dépouillent, autant qu’il est possible, de leurs éléments originaux et se rapprochent. L’espagnol ancien est incompréhensible pour un Français ou pour un Italien ; l’espagnol moderne ne leur offre presque plus de difficultés lexicologiques. La langue de Pétrarque et du Dante abandonne aux dialectes les mots, les formes non romaines, et, à première vue, n’a plus pour nous d’obscurités. Nous-mêmes, jadis riches de tant de vocables teutoniques, nous les avons abandonnés, et, si nous acceptons sans trop de répugnance des expressions anglaises, c’est que, pour la majeure partie, elles sont venues de nous ou appartiennent à une souche celtique. Pour nos voisins d’outre-Manche la proscription des éléments anglo-saxons marche vite ; le dictionnaire en perd tous les jours. Mais c’est en Allemagne que cette rénovation s’accomplit de la manière et par les voies les plus étranges.\par
Déjà, suivant un mouvement analogue à ce qu’on observe en Italie, les dialectes les plus chargés d’éléments germaniques, comme, par exemple, le frison et le bernois, sont relégués parmi les plus incompréhensibles pour la majorité. La plupart des langages provinciaux, riches d’éléments kymriques, se rapprochent davantage de l’idiome usuel. Celui-ci, connu sous le nom de haut allemand moderne, a relativement peu de ressemblances lexicologiques avec le gothique ou les anciennes langues du Nord, et des affinités de plus en plus étroites avec le celtique ; il y mêle aussi, çà et là, des emprunts slaves. Mais c’est surtout vers le celtique qu’il incline, et, comme il ne lui est pas possible d’en retrouver aisément les débris natifs dans l’usage moderne, il se rapproche avec effort du composé qui en est le plus voisin, c’est-à-dire du français. Il lui prend, sans nécessité apparente, des séries de mots dont il pourrait trouver sans peine les équivalents dans son propre fonds ; il s’empare de phrases entières qui produisent au milieu du discours l’effet le plus bizarre ; et, en dépit de ses lois grammaticales, dont il cherche d’ailleurs à modifier aussi la souplesse primitive pour se rapprocher de nos formes plus strictes et plus roides, il se romanise par toutes les voies qu’il peut se frayer ; mais il se romanise d’après la nuance celtique qui est le plus à sa portée, tandis que le français abonde de son mieux dans la nuance méridionale, et ne fait pas moins de pas vers l’italien que celui-ci n’en fait vers lui.\par
Jusqu’ici je n’ai éprouvé aucun scrupule à employer le mot de romanité pour indiquer l’état vers lequel retournent les populations de l’Europe occidentale. Cependant, afin d’être plus précis, il faut ajouter que sous cette expression on aurait tort d’entendre une situation complètement identique à celle d’aucune époque de l’ancien univers romain. De même que dans l’appréciation de celui-ci je me suis servi des mots de sémitique, \emph{d’hellénistique}, pour déterminer approximativement la nature des mélanges vers laquelle il abondait, en prévenant qu’il ne s’agissait pas de mixtures ethniques absolument pareilles à celles qui avaient jadis existé dans le monde assyrien et dans l’étendue des territoires syro-macédoniens, de même ici on ne doit pas oublier que la romanité nouvelle possède des nuances ethniques qui lui sont propres, et par conséquent développe des aptitudes inconnues à l’ancienne. Un fond complètement le même, un désordre plus grand, une assimilation croissante de toutes les facultés particulières par l’extrême subdivision des groupes primitivement distincts, voilà ce qui est commun entre les deux situations et ce qui ramène, chaque jour, nos sociétés vers l’imitation de l’univers impérial ; mais ce qui nous est propre, en ce moment du moins, et ce qui crée la différence, c’est que, dans la fermentation des parties consti­tutives de notre sang, beaucoup de détritus germaniques agissent encore et d’une manière fort spéciale, suivant qu’on les observe dans le Nord ou dans le Midi : ici, chez les Provençaux, en quantité dissolvante ; là, au contraire, chez les Suédois, avec un reste d’énergie qui retarde le mouvement prononcé de décadence.\par
Ce mouvement, opérant du sud au nord, a porté, depuis deux siècles déjà, les masses de la péninsule italique à un état très voisin de celui de leurs prédécesseurs du III\textsuperscript{e} siècle de notre ère, sauf des détails. Le Haut pays, à l’exception de certaines parties du Piémont, en diffère peu. L’Espagne, saturée d’éléments plus directement sémitiques, jouit dans ses races d’une sorte d’unité relative qui rend le désordre ethnique moins flagrant, mais qui est loin de donner le dessus aux facultés mâles ou utilitaires. Nos provinces françaises méridionales sont annulées ; celles du centre et de l’est, avec le sud-ouest de la Suisse, sont partagées entre l’influence du Midi et celle du Nord. La monarchie autrichienne maintient de son mieux, et avec une conscience de sa situation qu’on pourrait appeler scientifique, la prépondérance des éléments teutons dont elle dispose sur ses populations slaves. La Grèce, la Turquie d’Europe, sans force devant l’Europe occidentale, doivent au voisinage inerte de l’Anatolie un reste d’énergie relative, due aux infiltrations de l’élément germanique qu’à différentes reprises les âges moyens y ont apporté. On en peut dire autant des petits États voisins du Danube, avec cette différence que ceux-là doivent le peu d’immixtions arianes qui semblent les animer encore à une époque beaucoup plus ancienne, et que, chez eux, le désordre ethnique en est à sa plus douloureuse période. L’empire russe, terre de transition entre les races jaunes, les nations sémitisées et romanisées du sud et l’Allemagne, manque essentiellement d’homogénéité, n’a reçu jamais que de trop faibles apports de l’essence noble, et ne peut s’élever qu’à des appropriations imparfaites d’emprunts faits de tous côtés à la nuance hellénique, comme à la nuance italienne, comme à la nuance française, comme à la conception allemande. Encore ces appropriations ne dépassent-elles pas l’épiderme des masses nationales.\par
La Prusse, à la prendre d’après son extension actuelle, possède plus de ressources germaniques que l’Autriche, mais dans son noyau elle est inférieure à ce pays, où le groupe fortement arianisé des Madjars fait pencher la balance, non pas suivant la mesure de la civilisation, mais suivant celle de la vitalité, ce dont seulement il s’agit dans ce livre, on ne saurait trop s’en pénétrer.\par
En somme, la plus grande abondance de vie, l’agglomération de forces la plus considérable se trouve aujourd’hui concentrée et luttant avec désavantage contre le triomphe infaillible de la confusion romaine dans la série de territoires qu’embrasse un contour idéal qui, partant de Tornéo, enfermant le Danemark et le Hanovre, descen­dant le Rhin à une faible distance de sa rive droite jusqu’à Bâle, enveloppe l’Alsace et la haute Lorraine, serre le cours de la Seine, le suit jusqu’à son embouchure, se prolonge jusqu’à la Grande-Bretagne et rejoint à l’ouest l’Islande \footnote{Pour saisir dans sa véritable signification l’opinion exprimée ici, il faut se rappeler qu’il n’est question que d’une agglomération approximative. Des débris arians, plus ou moins bien conservés, se trouvent encore sur toutes les lignes de routes suivies par les races germaniques. De même qu’on en peut remarquer de très petits vestiges en Espagne, en Italie, en Suisse, partout où la configuration du sol a favorisé la formation et la conservation de ces dépôts, de même encore il s’en trouve dans le Tyrol, dans la Transylvanie, dans les montagnes de l’Albanie, dans le Caucase, dans l’Hindou-Koh, et jusqu’au fond des vallées hautes les plus orientales du Thibet. Il serait même imprudent d’affirmer qu’on n’en pourrait plus découvrir quelques-uns dans la haute Asie. Mais ce sont des spécimens fortement oblitérés déjà pour la plupart, impuissants, à peine perceptibles, qui n’échappent à une disparition, pour ainsi dire, instantanée, que grâce à l’inaction dans laquelle ils se maintiennent, et qui les défend heureusement de tout contact.}.\par
Dans ce centre subsistent les dernières épaves de l’élément arian, bien défigurées, bien dénudées, bien flétries sans doute, mais non pas encore tout à fait vaincues. C’est aussi là que bat le cœur de la société, et par suite de la civilisation moderne. Cette situation n’a jamais été analysée, expliquée, ni comprise jusqu’à présent ; néanmoins elle est vivement sentie par l’intelligence générale. Elle l’est si bien que beaucoup d’esprits en font instinctivement le point de départ de leurs spéculations sur l’avenir. Ils prévoient le jour où les glaces de la mort auront saisi les contrées qui nous semblent les plus favorisées, les plus florissantes ; et, supposant même peut-être cette cata­strophe plus prochaine qu’elle ne le sera, ils cherchent de là le lieu de refuge où l’humanité pourra, suivant leur désir, reprendre un nouveau lustre avec une nouvelle vie. Les succès actuels d’un des États situés en Amérique leur semblent présager cette ère si nécessaire. Le monde de l’ouest, voilà la scène immense sur laquelle ils imaginent que vont éclore des nations qui, héritant de l’expérience de toutes les civilisations passées, en enrichiront la nôtre et accompliront des œuvres que le monde n’a pu encore que rêver.\par
Examinons cette donnée avec tout l’intérêt qu’elle comporte. Nous allons trouver, dans l’examen approfondi des races diverses qui peuplent et ont peuplé les régions américaines, les motifs les plus décisifs de l’admettre ou de la rejeter.
\section[{VI.7. Les indigènes américains.}]{VI.7. \\
Les indigènes américains.}
\noindent En 1829, Cuvier ne se trouvait pas suffisamment informé pour émettre une opinion sur la nature ethnique des nations indigènes de l’Amérique, et il les laissait en dehors de ses nomenclatures. Les faits recueillis depuis lors permettent de se montrer plus hardi. Nombreux, ils deviennent concluants, et, si aucun n’apporte une certitude entière, une affirmation absolument sans réplique, l’ensemble en permet l’adoption de certaines bases complètement positives.\par
Il ne se trouvera plus désormais d’ethnologiste quelque peu renseigné qui puisse prétendre que les naturels américains forment une race pure, et qui leur applique la dénomination de variété rouge. Depuis le pôle jusqu’à la Terre-de-Feu, il n’est pas une nuance de la coloration humaine qui ne se manifeste, sauf le noir décidé du Congo et le blanc rosé de l’Anglais ; mais, en dehors de ces deux carnations, on observe les spécimens de toutes les autres \footnote{A. d’Orbigny, \emph{l’Homme américain}, t. I, p. 71 et seqq.}. Les indigènes, suivant leur nation, apparaissent bruns olivâtres, bruns foncés, bronzés, jaunes pâles, jaunes cuivrés, rouges, blancs, bruns, etc. Leur stature ne varie pas moins. Entre la taille non pas gigantesque, mais élevée, du Patagon, et la petitesse des Changos, il y a les mesures les plus multipliées. Les proportions du corps présentent les mêmes différences : quelques peuples ont le buste fort long, comme les tribus des Pampas ; d’autres, court et large, comme les habitants des Andes péruviennes \footnote{J’ai dit ailleurs que l’on cherchait à expliquer le développement extraordinaire du buste chez les Quichnas, dont il est ici question, par l’élévation de la chaîne où ils habitent, et j’ai montré pour quels motifs cette hypothèse était inacceptable. (Voir tome I\textsuperscript{er}) Voici une raison d’une autre sorte : les Umanas, placés dans les plaines qui bordent le cours supérieur de l’Amazone, ont la même conformation que les Quichnas montagnards. (Martius u. Spix, \emph{Reise in Brasilien}, t. III, p. 1255.)}. Il en est de même pour la forme et le volume de la tête. Ainsi la physiologie ne donne aucun moyen d’établir un type unique parmi les nations américaines.\par
En s’adressant à la linguistique, même résultat. Toutefois il faut y regarder de près. La grande majorité des idiomes possèdent chacun une originalité incontestable dans les parties lexicologiques ; à ce point de vue, ils sont étrangers les uns aux autres ; mais le système grammatical reste partout le même. On y remarque ce trait saillant d’une disposition commune à agglutiner les mots, et de plusieurs phrases à ne former qu’un seul vocable, faculté assurément très particulière, très remarquable, mais qui ne suffit pas à conquérir l’unité aux races américaines, d’autant moins que la règle ne va pas sans l’exception. On peut lui opposer l’othonis, très répandu dans la Nouvelle-Espagne, et qui, par sa structure nettement monosyllabique, tranche avec les disposi­tions fusionnaires des idiomes qui l’entourent \footnote{Prescott, History of the conquest of Mexico, t. III, p. 245.}. Peut-être rencontrera-t-on ultérieure­ment d’autres preuves que toutes les syntaxes américaines ne sont pas dérivées d’un même type, ni issues uniformément d’un seul et unique principe \footnote{Id., \emph{ibid.}, t. III p. 243.}.\par
Il n’y a donc plus moyen de classer parmi les divisions principales de l’humanité une prétendue race rouge qui n’existe évidemment qu’à l’état de nuance ethnique, que comme résultat de certaines combinaisons de sang, et qui ne saurait dès lors être prise que pour un sous-genre. Concluons avec M. Flourens et, avant lui, avec M. Garnot, qu’il n’existe pas en Amérique une famille indigène différente de celles qui habitent le reste du globe.\par
La question ainsi simplifiée n’en reste pas moins fort compliquée encore. S’il est acquis que les peuples du nouveau continent ne constituent pas une espèce à part, mille doutes s’élèvent quant à la façon de les rattacher aux types connus du vieux monde. Je vais tâcher d’éclairer de mon mieux ces ténébres, et, pour y parvenir, retournant la méthode dont j’ai usé tout à l’heure, je vais considérer si, à côté des diffé­rences profondes qui s’opposent à ce qu’on reconnaisse chez les nations américaines une unité particulière, il n’y a pas aussi des similitudes qui signalent dans leur organisation la présence d’un ou de plusieurs éléments ethniques semblables. Je n’ai pas besoin d’ajouter sans doute que, si le fait existe, ce ne peut être que dans des mesures très variées,\par
Les familles noire et blanche ne s’apercevant pas à l’état pur en Amérique, on a beau jeu pour constater, sinon leur absence totale, au moins leur effacement dans un degré notable. Il n’en est pas de même du type finnois ; il est irrécusable dans certaines peuplades du nord-ouest, telles que les Esquimaux \footnote{M. Morton (\emph{An Inquiry into the distinctive characteristics} of \emph{the aboriginal race of America}, Philadelphie, 1844) conteste la parenté des Esquimaux avec les Indiens Lenni-Lenapés ; mais ses arguments ne peuvent prévaloir contre ceux de Molina et de Humboldt. Son dessein est d’établir que la race américaine, sauf les peuplades polaires, dont il ne peut nier l’identité avec des groupes asiatiques, et que, pour ce motif, il range à part, est unitaire, ce qui est évident, mais de plus spéciale au continent qu’elle habite. (P. 6.)}. C’est donc là un point de jonction entre le vieux et le nouveau monde ; on ne peut mieux faire que de le choisir pour point de départ de l’examen. Après avoir quitté les Esquimaux, en descendant vers le sud, on arrive bientôt aux tribus appelées ordinairement rouges, aux Chinooks, aux Lenni-Lenapés, aux Sioux ; ce sont là des peuples qui ont eu un moment l’honneur d’être pris pour les prototypes de l’homme américain, bien que, ni par le nombre, ni par l’importance de leur organisation sociale, ils n’eussent le moindre sujet d’y prétendre. On constate sans peine des rapports étroits de parenté entre ces nations et les Esquimaux, partant les peuples jaunes, Pour les Chinooks, la question n’est pas un instant douteuse ; pour les autres, elle n’offrira plus d’obscurités du moment qu’on cessera de les comparer, ainsi qu’on le fait trop souvent, aux Chinois malais du sud de l’Empire Céleste, et qu’on les confrontera avec les Mongols. Alors on retrouvera sous la carnation cuivrée du Dahcota un fond évidemment jaune. On remarquera chez lui l’absence presque complète de barbe, la couleur noire des cheveux, leur nature sèche et roide, les dispositions lymphatiques du tempérament, la petitesse extraordinaire des yeux et leur tendance à l’obliquité. Cependant, qu’on y prenne garde aussi, ces divers caractères du type finnique sont loin d’apparaître chez les tribus rouges dans toute leur pureté.\par
Des contrées du Missouri on descend vers le Mexique, où l’on trouve ces signes spécifiques plus altérés encore, et néanmoins reconnaissables sous une carnation beaucoup plus bronzée. Cette circonstance pourrait égarer la critique, si, par un bonheur qui se reproduit rarement dans l’étude des antiquités américaines, l’histoire elle-même ne se chargeait d’affirmer la parenté des Astèques, et de leurs prédécesseurs les Toltèques, avec les hordes de chasseurs des noirs de la Colombia \footnote{Pickering, p. 41.}. C’est de ce fleuve que partirent les migrations des uns comme des autres vers le sud. La tradition est certaine : la comparaison des langues la confirme pleinement. Ainsi les Mexicains sont alliés à la race jaune par l’intermédiaire des Chinooks, mais avec immixtion plus forte d’un élément étranger \footnote{Pour les Californiens, M. Pickering s’exprime ainsi : « The first glame of the Californians satisfied me of their malay affinity. » (P. 100.)}.\par
Au delà de l’isthme commencent deux grandes familles qui se subdivisent en des centaines de nations dont plusieurs, devenues imperceptibles, sont réduites à douze ou quinze individus. Ces deux familles sont celle du littoral de l’océan Pacifique, et cette autre qui, s’étendant depuis le golfe du Mexique jusqu’au Rio de la Plata, couvre l’empire du Brésil, comme elle posséda jadis les Antilles. La première comprend les peuples péruviens. Ce sont les plus bruns, les plus rapprochés de la couleur noire de tout le continent, et, en même temps, ceux qui ont le moins de rapports généraux avec la race jaune. Le nez est long, saillant, fortement aquilin ; le front fuyant, comprimé sur les côtés, tendant à la forme pyramidale, et cependant on retrouve encore des stigmates mongols dans la disposition et la coupe oblique des yeux, dans la saillie des pommettes, dans la chevelure noire, grossière et lisse. C’en est assez pour tenir l’attention en éveil et la préparer à ce qui va lui être offert chez les tribus de l’autre groupe méridional qui embrasse toutes les peuplades guaranis. Ici le type finnique reparaît avec force et éclate d’évidence.\par
Les Guaranis, ou Caribes ou Caraïbes, sont généralement jaunes, à tel point que les observateurs les plus compétents n’ont pas hésité à les comparer aux peuples de la côte orientale d’Asie. C’est l’avis de Martius, de d’Orbigny, de Prescott. Plus variés peut-être dans leur conformation physique que les autres groupes américains, ils ont en commun « la couleur jaune, mélangée d’un peu de rouge très « pâle, gage, soit dit en passant, de leur migration du nord-est et de leur parenté « avec les Indiens chasseurs des États-Unis ; des formes très massives ; un front « non fuyant ; face pleine, circu­laire, nez court, étroit (généralement très épais), des « yeux souvent obliques, toujours relevés à l’angle extérieur, des traits « efféminés \footnote{D’Orbigny, \emph{ouvr. cité}, t. II, p. 347. D’après ce savant, les Botocudos ressemblent beaucoup au Mongol de Cuvier : « Nez court, bouche grande, barbe nulle, yeux relevés à l’angle externe. On peut, dit-il, les considérer comme le type de la race guarani. » – Martius u. Spix, \emph{ouvr. cité}, t. II, p. 819 : « Les Macams-Crans et les Aponeghi-Crans de la province de Maranhâo, les plus beaux des indigènes du Brésil, rentrent absolument dans la même classe. »}. »\par
J’ajouterai à cette citation que plus on s’avance vers l’est, plus la carnation des Guaranis devient forcée et s’éloigne du jaune rougeâtre.\par
La physiologie nous affirme donc que les peuples de l’Amérique ont, sous toutes les latitudes, un fond commun nettement mongol. La linguistique et la physiologie confirment de leur mieux cette donnée. Voyons la première.\par
Les langues américaines, dont j’ai remarqué tout à l’heure les dissemblances lexicologiques et les similitudes grammaticales, diffèrent profondément des idiomes de l’Asie orientale, rien n’est plus vrai ; mais Prescott ajoute, avec sa finesse et sa sagacité ordinaires, qu’elles ne se distinguent pas moins entre elles, et que, si cette raison suffisait pour faire rejeter toute parenté des indigènes du nouveau continent avec les Mongols, il faudrait aussi l’admettre pour isoler ces nations les unes des autres, système impossible. Puis, l’othonis enlève au fait sa portée absolue. Le rapport de cette langue avec les langues monosyllabiques de l’Asie orientale est évident ; la philologie ne peut donc, malgré bien des doutes que l’étude résoudra comme elle en a tant résolu, se refuser à admettre que, tout corrompus qu’ils peuvent être par des immixtions étrangères et un long travail intérieur, les dialectes américains ne s’oppo­sent nullement, dans leur état actuel, à une parenté du groupe qui les parle avec la race finnoise.\par
Quant aux dispositions intellectuelles de ce groupe, elles présentent plusieurs particularités caractéristiques faciles à dégager du chaos des tendances divergentes. Je voudrais, restant dans la vérité stricte, ne dire ni trop de bien ni trop de mal des indigènes américains. Certains observateurs les représentent comme des modèles de fierté et d’indépendance, et leur pardonnent à ce titre quelque peu d’anthropophagie \footnote{Cette opinion favorable a surtout pour propagateurs les romanciers américains.}. D’autres, au contraire, en faisant sonner bien haut des déclamations contre ce vice, reprochent à la race qui en est atteinte un développement monstrueux de l’égoïsme, d’où résultent les habitudes les plus follement féroces \footnote{Martius u. Spix, \emph{Reise in Brasilien}, t. I, p. 379, et t. III, p. 1033. – Carus, \emph{Ueber ungleiche Befæhigung der verschiedenen Menschheitsstæmme für næbere geistige} Entwickelung, p. 35. –Voir surtout les anciens auteurs espagnols.}.\par
 Avec la meilleure intention de rester impartial, on ne peut cependant pas mécon­naître que l’opinion a pour elle l’appui, l’aveu des plus anciens historiens de l’Amérique. Des témoins oculaires, frappés de la méchanceté froide et inexorable de ces sauvages qu’on fait par ailleurs si nobles, et qui sont, en effet, fort orgueilleux, ont voulu les reconnaître pour les descendants de Caïn. Ils les sentaient plus profondément mauvais que les autres hommes, et ils n’avaient pas tort.\par
L’Américain n’est pas à blâmer, entre les autres familles humaines, parce qu’il mange ses prisonniers, ou les torture et raffine leurs agonies. Tous les peuples en font ou en ont fait à peu près autant, et ne se distinguent de lui et entre eux sous ce rapport que par les motifs qui les mènent à de telles violences. Ce qui rend la férocité de l’Américain particulièrement remarquable à côté de celle du nègre le plus emporté, et du Finnois le plus bassement cruel, c’est l’impassibilité qui en fait la base et la durée du paroxysme, aussi long que sa vie. On dirait qu’il n’a pas de passion, tant il est capable de se modérer, de se contraindre, de cacher à tous les yeux la flamme haineuse qui le ronge ; mais, plus certainement encore, il n’a pas de pitié, comme le démontrent les relations qu’il entretient avec les étrangers, avec sa tribu, avec sa famille, avec ses femmes, avec ses enfants même \footnote{D’Orbigny, \emph{ouvr. cité}, t. II, p. 232 et pass.}.\par
En un mot, l’indigène américain, antipathique à ses semblables, ne s’en rapproche que dans la mesure de son utilité personnelle. Que juge-t-il rentrer dans cette sphère ? Des effets matériels seulement. Il n’a pas le sens du beau, ni des arts ; il est très borné dans la plupart de ses désirs, les limitant en général à l’essentiel des nécessités physiques. Manger est sa grande affaire, se vêtir après, et c’est peu de chose, même dans les régions froides. Ni les notions sociales de la pudeur, de la parure ou de la richesse, ne lui sont fortement accessibles.\par
Qu’on se garde de croire que ce soit par manque d’intelligence ; il en a, et l’applique bien à la satisfaction de sa forme d’égoïsme. Son grand principe politique, c’est l’indépendance, non pas celle de sa nation ou de sa tribu, mais la sienne propre, celle de l’individu même. Obéir le moins possible pour avoir peu à céder de sa fainéantise et de ses goûts, c’est la grande préoccupation du Guarani comme du Chinook. Tout ce qu’on prétend démêler de noble dans le caractère indien vient de là. Cependant plu­sieurs causes locales ont, dans quelques tribus, rendu la présence d’un chef nécessaire, indispensable. On a donc accepté le chef ; mais on ne lui accorde que la mesure de soumission la plus petite possible, et c’est le subordonné qui la fixe. On lui dispute jusqu’aux bribes d’une autorité si mince. On ne la confère que pour un temps, on la reprend quand on veut. Les sauvages d’Amérique sont des républicains extrêmes.\par
Dans cette situation, les hommes à talent ou ceux qui croient l’être, les ambitieux de toutes volées, emploient l’intelligence qu’ils possèdent, et j’ai dit qu’ils en avaient, à persuader à leur peuplade d’abord l’indignité de leurs concurrents, ensuite leur propre mérite ; et, comme il est impossible de former ce qui s’appelle ailleurs un parti solide, au moyen de ces individualités si farouches et si éparses, il leur faut user d’un recours journalier, d’un recours perpétuel à la persuasion et à l’éloquence pour maintenir cette influence si faible et si précaire, seul résultat pourtant auquel il leur soit permis d’aspirer. De là cette manie de discourir et de pérorer qui possède les sauvages, et tranche d’une manière si inattendue sur leur taciturnité naturelle. Dans leurs réunions de famille et même pendant leurs orgies, où il n’y a nul intérêt personnel mis en jeu, personne ne dit mot.\par
Par la nature de ce que des hommes trouvent utile, c’est-à-dire de pouvoir manger et de lutter contre les intempéries des saisons, de garder l’indépendance, non pour s’en servir à rechercher un but intellectuel, mais pour céder sans contrôle à des penchants purement matériels, par cette indifférente froideur dans les relations entre proches, je suis autorisé à reconnaître en eux la prédominance, ou du moins l’existence fonda­mentale de l’élément jaune. C’est bien là le type des peuples de l’Asie orientale, avec cette différence, pour ces derniers, que l’infusion constante et marquée du sang du blanc a modifié ces aptitudes étroites.\par
Ainsi la mythologie, comme la linguistique et surtout comme la physiologie, conclut que l’essence finnoise est répandue, en plus ou moins grande abondance, dans les trois grandes divisions américaines du nord, du sud-ouest et du sud-est. Il reste à trouver maintenant quelles causes ethniques, pénétrant ces masses, ont altéré, varié, contourné leurs caractères presque à l’infini, et de manière à les dégager en une série de groupes isolés. Pour parvenir à un résultat convenablement démontré, je continuerai à observer d’abord les caractères extérieurs, puis je passerai aux autres modes de la manifestation ethnique.\par
La modification du type jaune pur, lorsqu’elle a lieu par immixtion de principes blancs comme chez les Slaves et chez les Celtes, ou même chez les Kirghises, produit des hommes dont je ne trouve pas les semblables en Amérique. Ceux des indigènes de ce continent qui se rapprocheraient le plus, quant à l’extérieur, de nos populations galliques ou wendes, sont les Cherokees, et cependant il est impossible de s’y méprendre. Lorsqu’un mélange a lieu entre le jaune et le blanc, le second développe surtout son influence par la nouvelle mesure des proportions qu’il donne aux mem­bres ; mais, pour ce qui est du visage, il agit médiocrement et ne fait que modérer la nature finnoise. Or c’est précisément par les traits de la face que les Cherokees sont comparables au type européen. Ces sauvages n’ont pas même les yeux aussi bridés, ni aussi obliques, ni aussi petits que les Bretons et que la plupart des Russes orientaux ; leur nez est droit et s’éloigne notablement de la forme aplatie que rien n’efface dans les métis jaunes et blancs. Il n’y a donc nul motif d’admettre que les races américaines aient vu leurs éléments finniques influencés primitivement par des alliages venus de l’espèce noble.\par
Si l’observation physique se prononce de la sorte sur ce point, elle indique, en revanche, avec insistance, la présence d’immixtions noires. L’extrême variété des types américains correspond, d’une manière frappante, à la diversité non moins grande qu’il est facile d’observer entre les nations polynésiennes et les peuples malais du sud-est asiatique. On sera d’autant plus convaincu de la réalité de cette corrélation qu’on s’y arrêtera davantage. On découvrira, dans les régions américaines, les pendants exacts du Chinois septentrional, du Malais des Célèbes, du japonais, du Mataboulaï des îles Tonga, du Papou lui-même, dans les types de l’Indien du nord, du Guarani, de l’Aztèque, du Quichna, du Cafuso. Plus on descendra aux nuances, plus on rencontrera d’analogies ; toutes, certainement, ne correspondront pas d’une manière rigoureuse, il est bien facile de le prévoir, mais elles indiqueront si bien leur lien général de compa­raison que l’on conviendra sans difficulté de l’identité des causes. Chez les sujets les plus bruns, le nez prend la forme aquiline, et souvent d’une façon très accentuée ; les yeux deviennent droits, ou presque droits ; quelquefois la mâchoire se développe en avant : de tels cas sont rares. Le front cesse d’être bombé et affecte la forme fuyante. Tous ces indices réunis dénoncent la présence de l’immixtion noire dans un fond mongol. Ainsi l’ensemble des groupes aborigènes du continent américain forme un réseau de nations malaises, en tant que ce mot peut s’appliquer à des produits très différemment gradués du mélange finnomélanien, ce que personne ne conteste d’ailleurs pour toutes les familles qui s’étendent de Madagascar aux Marquises, et de la Chine à l’île de Pâques.\par
S’enquiert-on maintenant par quels moyens la communication entre les deux grands types noir et jaune a pu s’établir dans l’est de l’hémisphère austral ? Il est aisé, très aisé de tranquilliser l’esprit à cet égard. Entre Madagascar et la première île malaise, qui est Ceylan, il y a 12° au moins, tandis que du japon au Kamtschatka et de la côte d’Asie à celle d’Amérique, par le détroit de Behring, la distance est insigni­fiante. On n’a pas oublié que, dans une autre partie de cet ouvrage, l’existence de tribus noires sur les îles au nord de Niphon a déjà été signalée pour une époque très moderne. D’autre part, puisqu’il a été possible à des peuples malais de passer d’archipels en archipels jusqu’à l’île de Pâques, il n’y a nulle difficulté à ce que, parvenus à ce point, ils aient continué jusqu’à la côte du Chili, située vis-à-vis d’eux, et y soient arrivés, après une traversée rendue assez facile par les îles semées sur la route, Sala, Saint-Ambroise, Juan-Fernandez, circonstance qui réduit à deux cents lieues le plus court trajet d’un des points intermédiaires à l’autre. Or, on a vu que des hasards de mer entraînaient fréquemment des embarcations d’indigènes à plus du double de cette distance. L’Amérique était donc accessible, du côté de l’ouest, par ses deux extrémités nord et sud. Il est encore d’autres motifs pour ne pas douter que ce qui était matériellement possible a eu lieu en effet \footnote{Morton conteste la possibilité de l’arrivée de groupes malais jusqu’à la côte d’Amérique, parce que, dit-il, les vents d’est règnent le plus ordinairement dans ces parages. (\emph{Ouvr. cité}, p. 32.) En se prononçant ainsi, il oublie le fait incontestable de la colonisation de toutes les îles du Pacifique par une même race venue de l’ouest, et cette circonstance plus particulière, que lui-même signale (p. 17), qu’en 1833, une jonque japonaise a été jetée par les vents sur cette même côte d’Amérique qu’il déclare, un peu plus bas, inaccessible de ce côté. Il a vu lui-même des vases de porcelaine provenant de cette jonque, et il ajoute : « Such casualties may have occurred in the early period of american history. »}.\par
Les tribus d’aborigènes les plus bruns étant disposées sur la côte occidentale, on en doit conclure que là se firent les principales alliances du principe noir ou plutôt malais avec l’élément jaune fondamental. En présence de cette explication, on n’a plus à s’occuper de démonstrations appuyées sur la prétendue influence climatérique pour expliquer comment les Aztèques et les Quichnas sont plus basanés, bien qu’habitant des montagnes relativement très froides, que les tribus brésiliennes errant dans des pays plats et sur le bord des fleuves. On ne s’arrêtera plus à cette solution bizarre que, si ces sauvages sont d’un jaune pâle, c’est que l’abri des forêts leur conserve le teint. Les peuples de la côte occidentale sont les plus bruns, parce qu’ils sont les plus imbus de sang mélanien, vu le voisinage des archipels de l’océan Pacifique. C’est aussi l’opinion de la psychologie.\par
Tout ce qui a été dit plus haut du naturel de l’homme américain s’accorde avec ce que l’on sait des dispositions capitales de la race malaise. Égoïsme profond, noncha­lance, paresse, cruauté froide, ce fond identique des mœurs mexicaines, péruviennes, guaranis, huronnes, semble puisé dans les types offerts par les populations australiennes. On y observe de même un certain goût de l’utile médiocrement compris, une intelligence plus pratique que celle du nègre, et toujours la passion de l’indépen­dance personnelle. Parce que nous avons vu en Chine la variété métisse du Malais supérieure à la race noire et à la jaune, nous voyons également les populations d’Amérique posséder les facultés mâles avec plus d’intensité que les tribus du continent africain \footnote{D’Orbigny (\emph{ouvr. cité}, t. I, p. 143) déclare que le mélange des aborigènes américains, et ce sont surtout les Guaranis très mongolisés qu’il a observés, donne des produits supérieurs aux deux types qui les fournissent.}. Il a pu se développer chez elles, sous une influence supérieure, comme ailleurs chez les Malais de Java, de Sumatra, de Bali, des civilisations bien éphémères sans doute, mais non pas dénuées de mérite.\par
Ces civilisations, quelles qu’aient été leurs causes créatrices, n’ont eu l’étincelle nécessaire pour se former que là où la famille malaise, existant avec la plus grande somme d’éléments mélaniens, présentait l’étoffe la moins rebelle. On doit donc s’attendre à les trouver sur les points les plus rapprochés des archipels du Pacifique. Cette prévision n’est pas trompée : leurs plus complets développements nous sont offerts sur le territoire mexicain et sur la côte péruvienne.\par
Il est impossible de passer sous silence un préjugé commun à toutes les races américaines, et qui se rattache évidemment à une considération ethnique. Partout les indigènes admirent comme une beauté les fronts fuyants et bas. Dans plusieurs loca­lités, extrêmement distantes les unes des autres, telles que les bords de la Columbia et l’ancien pays des Aymaras péruviens, on a pratiqué ou l’on pratique encore l’usage d’obtenir cette difformité si appréciée, en aplatissant les crânes des enfants en bas âge par un appareil compressif formé de bandelettes étroitement serrées \footnote{Les Aymaras actuels n’ont pas la tête aplatie de leurs ancêtres, parce que l’influence espagnole les a fait renoncer à cet usage. (D’Orbigny, \emph{ouvr. cité}, t. I, p. 315.) Il n’avait commencé qu’avec la domination des Incas, vers le XIV\textsuperscript{e} siècle. (\emph{Ibid.}, p. 319.) Les Chinooks de la Colombie le maintiennent encore avec grand soin. Un voyageur choisi pour parrain d’un enfant, ne put décider les parents à ne pas remettre les bandelettes compressives aussitôt que le nourrisson eut été ondoyé par un missionnaire.}.\par
 Cette coutume n’est pas, d’ailleurs, exclusivement particulière au nouveau monde ; l’ancien en a vu des exemples. C’est ainsi que, chez plusieurs nations hunniques, d’extraction en partie étrangère au sang mongol, les parents employaient le même procédé qu’en Amérique pour repétrir la tête des nouveau-nés, et leur procurer plus tard une ressemblance factice avec la race aristocratique. Or, comme il n’est pas admissible que le fait d’avoir le front fuyant puisse répondre à une idée innée de belle conformation, on doit croire que les indigènes américains ont été amenés au désir de retoucher l’apparence physique de leurs générations par quelques indices qui les portaient à considérer les fronts fuyants comme la preuve d’un développement enviable des facultés actives, ou, ce qui revient au même, comme la marque d’une supériorité sociale quelconque. Il n’y a pas de doute que ce qu’ils voulaient imiter, c’était la tête pyramidale du Malais, forme mixte entre la disposition de la boîte crânienne du Finnois et celle du nègre. La coutume d’aplatir le front des enfants est ainsi une preuve de plus de la nature malaise des plus puissantes tribus américaines ; et je conclus en répétant qu’il n’y a pas de race d’Amérique proprement dite, ensuite que les indigènes de cette partie du monde sont de race mongole, différemment affectés par des immixtions soit de noirs purs, soit de Malais. Cette partie de l’espèce humaine est donc complètement métisse.\par
Il y a plus ; elle l’est depuis des temps incalculables, et il n’est guère possible d’admettre que jamais le soin de se maintenir pures ait inquiété ces nations. À en juger par les faits, dont les plus anciens sont malheureusement encore assez modernes, puisqu’ils ne s’élèvent pas au-dessus du X\textsuperscript{e} siècle de notre ère, les trois groupes américains, sauf de rares exceptions, ne se sont, en aucun temps, fait le moindre scrupule de mêler leur sang. Dans le Mexique, le peuple conquérant se rattachait les vaincus par des mariages pour agrandir et consolider sa domination. Les Péruviens, ardents prosélytes, prétendaient augmenter de la même manière le nombre des adorateurs du soleil. Les Guaranis, ayant décidé que l’honneur d’un guerrier consistait à avoir beaucoup d’épouses étrangères à sa tribu, harcèlent sans relâche leurs voisins dans le but principal, après avoir tué les hommes et les enfants, de s’attribuer les femmes \footnote{D’Orbigny, \emph{ouvr. cité}, t, I, p. 153. – Dans le Sud, les femmes sont vendues si cher par leurs parents, que les jeunes gens, procédant avec économie, préfèrent s’en procurer le casse-tête au poing. (\emph{Ibid.})}. Il résulta de cette habitude, chez ces derniers, un accident linguistique assez bizarre. Ces nouvelles compatriotes, important leurs langages dans leurs tribus d’adoption, y formèrent, au sein de l’idiome national, une partie féminine qui ne fut jamais à l’usage de leurs maris \footnote{D’Obigny, \emph{Ibid}}.\par
Tant de mélanges, venant s’ajouter incessamment à un fond déjà métis, ont amené la plus grande anarchie ethnique. Si l’on considère de plus que les mieux doués des groupes américains, ceux dont l’élément jaune fondamental est le plus chargé d’apports mélaniens, ne sont cependant et ne peuvent être qu’assez humblement placés sur l’échelle de l’humanité, on comprendra encore mieux que leur faiblesse n’est pas de la jeunesse, mais bien de la décrépitude, et qu’il n’y a jamais eu la moindre possibilité pour eux d’opposer une résistance quelconque aux attaques venues de l’Europe.\par
Il semblera étrange que ces tribus échappent à la loi ordinaire qui porte les nations, même celles qui sont déjà métisses, à répugner aux mélanges, loi qui s’exerce avec d’autant plus de force que les familles sont composées d’éléments ethniques grossiers. Mais l’excès de la confusion détruit cette loi chez les groupes les plus vils comme chez les plus nobles ; on en a vu bien des exemples ; et, quand on considère le nombre illimité d’alliages que toutes les peuplades américaines ont subis, il n’y a pas lieu de s’étonner de l’avidité avec laquelle les femmes guaranis du Brésil recherchent les embrassements du nègre. C’est précisément l’absence de tout sentiment sporadique dans les rapports sexuels qui démontre le plus complètement à quel bas degré les familles du nouveau monde sont descendues en fait de dépravation ethnique, et qui donne les plus puissantes raisons d’admettre que le début de cet état de choses remonte à une époque excessivement éloignée \footnote{Martius u. Spix., \emph{ouvr. cité}, t. III, p. 905. – Ces voyageurs vont jusqu’à affirmer que, dans la province du Para, il n’est peut-être pas une seule famille indienne qui ait laissé passer quelques générations sans se croiser, soit avec des blancs, soit avec des noirs.}.\par
Lorsque nous avons étudié les causes des migrations primitives de la race blanche vers le sud et l’ouest, nous avons constaté que ces déplacements étaient les consé­quences d’une forte pression exercée dans le nord-est par des multitudes innombrables de peuples jaunes. Antérieurement encore à la descente des Chamites blancs, des Sémites et des Arians, l’inondation finnique, trouvant peu de résistance chez les nations noires de la Chine, s’était répandue au milieu d’elles, et y avait poussé très loin ses conquêtes, par conséquent ses mélanges. Dans les dispositions dévastatrices, brutales, de cette race il y eut nécessairement excès de spoliation. En butte à des dépossessions impitoyables, des bandes nombreuses de noirs prirent la fuite et se dispersèrent où elles purent. Les unes gagnèrent les montagnes, les autres les îles Formose, Niphon, Yeso, les Kouriles, et, passant derrière les masses de leurs persécuteurs, vinrent à leur tour conquérir, soit en restant pures, soit mêlées au sang des agresseurs, les terres abandonnées par ceux-ci dans l’occident du monde. Là elles s’unirent aux traînards jaunes qui n’avaient pas suivi la grande émigration.\par
Mais le chemin pour passer ainsi de l’Asie septentrionale sur l’autre continent était hérissé de difficultés qui ne le rendaient pas attrayant ; puis, d’une autre part, les grandes causes qui expulsaient d’Amérique les multitudes énormes des jaunes n’avaient pas permis à beaucoup de tribus de ceux-ci de conserver l’ancien domicile. Pour ces motifs, la population resta toujours assez faible, et ne se releva jamais de la terrible catastrophe inconnue qui avait poussé ces masses natives à la désertion. Si les Mexicains, si les Péruviens présentèrent quelques dénombrements respectables à l’observation des Espagnols, les Portugais trouvèrent le Brésil peu habité, et les Anglais n’eurent devant eux, dans le nord, que des tribus errantes perdues au sein des solitudes. L’Américain n’est donc que le descendant clairsemé de bannis et de traînards. Son territoire représente une demeure abandonnée, trop vaste pour ceux qui l’occupent, et qui ne sauraient pas se dire absolument les héritiers directs et légitimes des maîtres primordiaux.\par
 Les observateurs attentifs, qui tous, d’un commun accord, ont reconnu chez les naturels du nouveau monde les caractères frappants et tristes de la décomposition sociale, ont cru, pour la plupart, que cette agonie était celle d’une société jadis constituée, était celle de l’intelligence vieillie, de l’esprit usé. Point. C’est celle du sang frelaté, et encore n’ayant été primitivement formé que d’éléments infimes. L’impuis­sance de ces peuples était telle, à ce moment même où des civilisations nationales les éclairaient de tous leurs feux, qu’ils n’avaient pas même la connaissance du sol sur lequel ils vivaient. Les empires du Mexique et du Pérou, ces deux merveilles de leur génie, se touchaient presque, et on n’a jamais pu découvrir la moindre liaison de l’un à l’autre. Tout porte à croire qu’ils s’ignoraient. Cependant ils cherchaient à étendre leurs frontières, à se grossir de leur mieux. Mais les tribus qui séparaient leurs frontières étaient si mauvaises conductrices des impressions sociales qu’elles ne les propageaient pas même à la plus faible distance. Les deux sociétés constituaient donc deux îlots qui ne s’empruntaient et ne se prêtaient rien.\par
Cependant elles avaient longtemps été cultivées sur place, et avaient acquis toute la force qu’elles devaient jamais avoir. Les Mexicains n’étaient pas les premiers civilisateurs de leur contrée. Avant eux, c’est-à-dire avant le X\textsuperscript{e} siècle de notre ère \footnote{Prescott, \emph{ouvr. cité}, t. III, p. 255) ne fait même remonter qu’au X\textsuperscript{e} siècle l’arrivée des Toltèques.}, les Toltèques avaient fondé de grands établissements sur le même sol, et avant les Toltèques on reporte encore l’âge des Olmécas, qui seraient les véritables fondateurs de ces grands et imposants édifices dont les ruines dorment ensevelies au plus profond des forêts du Yucatan. D’énormes murailles formées de pierres immenses, des cours d’une étonnante étendue, impriment à ces monuments un aspect de majesté auquel la mélancolie grandiose et les profusions végétales de la nature viennent ajouter leurs charmes. Le voyageur qui, après plusieurs jours de marche à travers les forêts vierges de Chiapa, le corps fatigué par les difficultés de la route, l’âme émue par la conscience de mille dangers, l’esprit exalté par cette interminable succession d’arbres séculaires, les uns debout, les autres tombés, d’autres encore cachant la poussière de leur vétusté sous des monceaux de lianes, de verdure et de fleurs étincelantes ; l’oreille remplie du cri des bêtes de proie ou du frissonnement des reptiles ; ce voyageur qui, à travers tant de causes d’excitation, arrive à ces débris inespérés de la pensée humaine, ne mériterait pas sa fortune, si son enthousiasme ne lui jurait qu’il a sous les yeux des beautés incomparables.\par
Mais, quand un esprit froid examine ensuite dans le cabinet les esquisses et les récits de l’observateur exalté, il a le devoir d’être sévère, et, après mûres réflexions, il conclura sans doute que ce n’est pas l’œuvre d’un artiste, ni même d’une nation grandement utilitaire que l’on peut reconnaître dans les restes de Mitla, d’Izalanca, de Palenqué, des ruines de la vallée d’Oaxaca.\par
Les sculptures tracées sur les murailles sont grossières, aucune idée d’art élevé n’y respire. On n’y voit pas, comme dans les œuvres des Sémites d’Assyrie, l’apothéose heureuse de la matière et de la force. Ce sont d’humbles efforts pour imiter la forme de l’homme et des animaux. Il en résulte des créations qui, de bien loin, n’atteignent pas à l’idéal ; et cependant elles ne sauraient pas non plus avoir été commandées par le sentiment de l’utile. Les races mâles n’ont pas coutume de se donner tant de peine pour amonceler des pierres ; nulle part les besoins matériels ne commandent de pareils travaux. Aussi n’existe-t-il rien de semblable en Chine ; et, quand l’Europe des âges moyens a dressé ses cathédrales, l’esprit romanisé lui avait fait déjà, pour son usage, une notion du beau et une aptitude aux arts plastiques que les races blanches peuvent bien adopter, qu’elles poussent à une perfection unique, mais que seules et d’elles-mêmes elles ne sont pas aptes à concevoir. Il y a donc du nègre dans la création des monuments du Yucatan, mais du nègre qui, en excitant l’instinct jaune et en le portant à sortir de ses goûts terre à terre, n’a pas réussi à lui faire acquérir ce que l’initiateur même n’avait pas, le goût, ou, pour mieux dire, le vrai génie créateur \footnote{D’Orbigny observe que c’est chez les Aymaras péruviens que l’on peut trouver, dans les œuvres architecturales, le plus d’idéalité ; encore n’est-ce jamais beau. (\emph{Ouvr. cité}, t. I, p. 203 et seqq.) On a essayé de découvrir l’âge des monuments de Palenqué d’après la nature des stalactites déposées sur quelques murailles, d’après les couches concentriques formées par la végétation sur de très vieux arbres et par l’observation des couches de détritus accumulées à une hauteur de neuf pieds dans les cours. Cette méthode n’a pas donné de résultats sous un ciel aussi fécond que celui du Yucatan. (Prescott, \emph{ouvr cité} t. III, p. 254.)}.\par
On doit tirer encore une conséquence de la vue de ces monuments. C’est que le peuple malais par lequel ils furent construits, outre qu’il ne possédait pas le sens artistique dans la signification élevée du mot, était un peuple de conquérants qui disposait souverainement des bras de multitudes asservies \footnote{Dans une des cours d’Uxmal, le pavé de granit, sur lequel sont figurées en relief des figures de tortues, est presque uni par les pas des anciennes populations. (Prescott, \emph{ibid}.)}. Une nation homogène et libre ne s’impose jamais de pareilles créations ; il lui faut des étrangers pour les imaginer, lorsque sa puissance intellectuelle est médiocre, et pour les accomplir, lorsque cette même puissance est grande. Dans le premier cas, il lui faut des Chamites, des Sémites, des Arians Iraniens ou Hindous, des Germains, c’est-à-dire, pour employer des termes compris chez tous les peuples, des dieux, des demi-dieux, des héros, des prêtres ou des nobles omnipotents. Dans le second, cette série de maîtres ne peut se passer de masses serviles pour réaliser les conceptions de son génie. L’aspect des ruines du Yucatan induit donc à conclure que les populations mixtes de cette contrée étaient dominées, lorsque ces palais s’élevèrent, par une race métisse comme elles, mais d’un degré un peu plus élevé, et surtout plus affectée par l’alliage mélanien.\par
Les Toltèques et les Aztèques se reconnaissent également au peu de largeur du front et à la couleur olivâtre. Ils venaient du nord-ouest, où l’on retrouve encore leurs tribus natales dans les environs de Nootka ; ils s’installèrent au milieu des peuplades indigènes, qui avaient déjà connu la domination des Olmécas, et ils leur enseignèrent une sorte de civilisation bien faite pour nous étonner ; car elle a conservé, tant qu’elle a vécu, les caractères résultant de la vie des forêts à côté de ceux dont l’existence des villes rend les raffinements nécessaires.\par
En détaillant la splendeur de Mexico au temps des Aztèques, on y remarque de somptueux bâtiments, de belles étoffes, des mœurs élégantes et recherchées. Dans le gouvernement on y voit cette hiérarchie monarchique, mêlée d’éléments sacerdotaux, qui se reproduit partout où des masses populaires sont assujetties par une nation de vainqueurs. On y constate encore de l’énergie militaire chez les nobles, et des tendances très accusées à comprendre l’administration publique d’une façon toute propre à la race jaune. Le pays n’était pas non plus sans littérature. Malheureusement les historiens espagnols ne nous ont rien conservé qu’ils n’aient défiguré en l’amplifiant. Il y a cependant du goût chinois dans les considérations morales, dans les doctrines régulatrices et édifiantes des poésies aztèques, comme ce même goût apparaît aussi dans la recherche contournée et énigmatique des expressions. Les chefs mexicains, pareils en ce point à tous les caciques de l’Amérique, se montraient grands parleurs, et cultivaient fort cette éloquence ampoulée, nuageuse, séductrice, que les Indiens des prairies du nord connaissent et pratiquent si bien au gré des romanciers qui les ont décrits de nos jours. J’ai déjà indiqué la source de ce genre de talent. L’éloquence politique, ferme, simple, brève, qui n’est que l’exposition des faits et des raisons, assure le plus grand honneur à la nation qui en fait usage. Chez les Arians de tous les âges, comme encore chez les Doriens et dans le vieux sénat sabin de la Rome latine, c’est l’instrument de la liberté et de la sagesse. Mais l’éloquence politique ornée, verbeuse, cultivée comme un talent spécial, élevée à la hauteur d’un art, l’éloquence qui devient la rhétorique, c’est tout autre chose. On ne saurait la considérer que comme un résultat direct du fractionnement des idées chez une race, et de l’isolement moral où sont tombés tous les esprits. Ce que l’on a vu chez les Grecs méridionaux, chez les Romains sémitisés, j’allais dire dans les temps modernes, démontre assez que le talent de la parole, cette puissance en définitive grossière, puisque ses œuvres ne peuvent être conservées qu’à la condition rigoureuse de passer dans une forme supérieure à celle où elles ont produit leurs effets ; qui a pour but de séduire, de tromper, d’entraî­ner, beaucoup plus que de convaincre, ne saurait naître et vivre que chez des peuples égrenés qui n’ont plus de volonté commune, de but défini, et qui se tiennent, tant ils sont incertains de leurs voies, à la disposition du dernier qui leur parle. Donc, puisque les Mexicains honoraient si fort l’éloquence, c’est une preuve que leur aristocratie même n’était pas très compacte, très homogène. Les peuples, sans contredit, ne différaient pas des nobles sous ce rapport.\par
Quatre grandes lacunes affaiblissaient l’éclat de la civilisation aztèque. Les massa­cres hiératiques étaient considérés comme l’une des bases de l’organisation sociale, comme un des buts principaux de la vie publique. Cette férocité normale tuait sans choix, comme sans scrupule, les hommes, les femmes, les vieillards, les enfants ; elle tuait par troupeaux, et y prenait un plaisir ineffable. Il est inutile de signaler combien ces exécutions différaient des sacrifices humains dont le monde germanique nous a présenté l’usage. On comprend que le mépris de la vie et de l’âme était la source dégradante de cet usage, et résultait naturellement du double courant noir et jaune qui avait formé la race.\par
Les Aztèques n’avaient jamais songé à réduire des animaux en domesticité ; ils ne connaissaient pas l’usage du lait. C’est une singularité qui se retrouve çà et là chez certains groupes de la famille jaune \footnote{Voir plus haut.}.\par
 Ils possédaient un système graphique, mais des plus imparfaits. Leur écriture ne consistait qu’en une série de dessins grossièrement idéographiques. Il y a bien loin de là aux hiéroglyphes proprement dits. On se servait de cette méthode pour conserver le souvenir des grands faits historiques, transmettre les ordres du gouvernement, les renseignements fournis par les magistrats au roi. C’était un procédé très lent, très incommode ; cependant les Aztèques n’avaient pas su mieux faire. Ils étaient inférieurs sous ce rapport aux Olmécas, leurs prédécesseurs, si tant est qu’il faille les prendre, avec M. Prescott, pour les fondateurs de Palenqué, et admettre que certaines inscrip­tions observées sur les murailles de ces ruines constituent des signes phonétiques \footnote{Prescott, \emph{ouvr. cité}, t. III, p. 253.}.\par
Enfin, dernière défectuosité chronique de la société mexicaine, il est certain, bien qu’à peine croyable, que ce peuple riverain de la mer, et dont le territoire n’est pas privé de cours d’eau, ne pratiquait pas la navigation, et se servait uniquement de piro­gues fort mal construites et de radeaux plus imparfaits encore.\par
Voilà quelle était la civilisation renversée par Cortez : et il est bon d’ajouter que ce conquérant la trouva dans sa fleur et dans sa nouveauté ; car la fondation de la capitale, Tenochtitlan, ne remontait qu’à l’an 1325. Combien donc les racines de cette organisation étaient courtes et peu tenaces ! Il a suffi de l’apparition et du séjour d’une poignée de métis blancs sur son terrain pour la précipiter immédiatement au sein du néant. Quand la forme politique eut péri, il n’y eut plus de trace des inventions sur lesquelles elle s’appuyait. La culture péruvienne ne se montra pas plus solide.\par
La domination des Incas, comme celle des Toltèques et des Aztèques, succédait à un autre empire, celui des Aymaras, dont le siège principal avait existé dans les régions élevées des Andes, sur les rives du lac de Titicaca. Les monuments qu’on voit encore dans ces lieux permettent d’attribuer à la nation aymara des facultés supérieures à celles des Péruviens qui l’ont suivie, puisque ceux-ci n’ont été que des copistes. M. d’Orbigny fait observer avec raison que les sculptures de Tihuanaco révèlent un état intellectuel plus délicat que les ruines des âges postérieurs, et qu’on y découvre même une certaine propension à l’idéalité tout à fait étrangère à ceux-ci \footnote{D’Orbigny, \emph{ouvr. cité}, t. I, p. 325.}.\par
Les Incas, reproduction affaiblie d’une race civilisatrice, arrivèrent des montagnes en en couvrant vers l’ouest toutes les pentes, occupant les plateaux et agglomérant sous leur conduite un certain nombre de peuplades. Ce fut au XI\textsuperscript{e} siècle de notre ère que cette puissance naquit \footnote{D’Orbigny, \emph{ouvr. cité}, t. I, p. 296. – C’est l’époque où parut Manco-Capac.}, et, véritable singularité en Amérique, la famille régnante semble avoir été extrêmement préoccupée du soin de conserver la pureté de son sang. Dans le palais de Cuzco, l’empereur n’épousait que ses sœurs légitimes, afin d’être plus assuré de l’intégrité de sa descendance, et il se réservait, ainsi qu’à un petit nombre de parents très proches, l’usage exclusif d’une langue sacrée, qui vraisemblablement était l’aymara \footnote{D’Orbigny, \emph{ouvr. cité}, t. I, p. 297.}.\par
 Ces précautions ethniques de la famille souveraine démontrent qu’il y avait beaucoup à redire à la valeur généalogique de la nation conquérante elle-même. Les Incas éloignés du trône ne se faisaient qu’un très mince scrupule de prendre des épouses où il leur plaisait. Toutefois, si leurs enfants avaient pour aïeux maternels les aborigènes du pays, la tolérance ne s’étendait pas jusqu’à admettre dans les emplois les descendants en ligne paternelle de cette race soumise. Ces derniers étaient donc peu attachés au régime sous lequel ils vivaient, et voilà un des motifs pour lesquels Pizarre renversa si aisément toute la couche supérieure de cette société, tout le couronnement des institutions, et pourquoi les Péruviens n’essayèrent jamais d’en retrouver ni d’en faire revivre les restes.\par
Les Incas ne se sont pas souillés des institutions homicides de l’Anahuac mexicain ; leur régime était au contraire fort doux. Ils avaient tourné leurs principales idées vers l’agriculture, et, mieux avisés que les Aztèques, ils avaient apprivoisé de nombreux troupeaux d’alpacas et de lamas. Mais chez eux, pas d’éloquence, pas de luttes de parole : l’obéissance passive était la suprême loi. La formule fondamentale de l’État avait indiqué une route à suivre à l’exclusion de toute autre, et n’admettait pas la discussion dans ses moyens de gouvernement. Au Pérou, on ne raisonnait pas, on ne possédait pas, tout le monde travaillait pour le prince. La fonction capitale des magistrats consistait à répartir dans chaque famille une quote-part convenable du labeur commun. Chacun s’arrangeait de façon à se fatiguer le moins possible, puisque l’application la plus acharnée ne pouvait jamais procurer aucun avantage exceptionnel. On ne réfléchissait pas non plus. Un talent surhumain n’était pas capable d’avancer son propriétaire dans les distinctions sociales. On buvait, on mangeait, on dormait, et surtout on se prosternait devant l’empereur et ses préposés ; de sorte que la société péruvienne était assez silencieuse et très passive.\par
En revanche, elle se montrait encore plus utilitaire que la mexicaine. Outre les grands ouvrages agricoles, le gouvernement faisait exécuter des routes magnifiques, et ses sujets connaissaient l’usage des ponts suspendus, qui est si nouveau pour nous. La méthode dont ils usaient pour fixer et transmettre la pensée était des plus élémentaires, et peut-être faut-il préférer les peintures de l’Anahuac aux quipos.\par
Pas plus que chez les Aztèques, la construction navale n’était connue. La mer qui bordait la côte restait déserte \footnote{D’Orbigny, \emph{ouvr. cité}, t. I, p. 215. – Les Guaranis ou Caraïbes, conquérants des Antilles, n’avaient eux-mêmes que des pirogues faites d’un tronc d’arbre creusé. (\emph{Ibid}.)}.\par
Avec ses qualités et ses défauts, la civilisation péruvienne inclinait vers les molles préoccupations de l’espèce jaune, tandis que l’activité féroce du Mexicain accuse plus directement la parenté mélanienne. On comprend assez qu’en présence de la profonde confusion ethnique des races du nouveau continent, ce serait une insoutenable prétention que de vouloir aujourd’hui préciser les nuances qui ressortent de l’amalgame de leurs éléments.\par
 Il resterait à examiner une troisième nation américaine établie dans les plaines du nord, au pied des monts Alléghanis, à une époque fort obscure. Des restes de travaux considérables et des tombeaux sans nombre se font apercevoir au sein de cette région. Ils se divisent en plusieurs classes indicatives de dates et de races fort différentes. Mais les incertitudes s’accumulent sur cette question. jusqu’à présent rien de positif n’a encore été découvert. S’attacher à un problème encore si peu et si mal étudié, ce serait s’enfoncer gratuitement dans des hypothèses inextricables \footnote{Des monuments de différentes espèces, mais extrêmement grossiers, sont répandus jusque dans le Nouveau-Mexique et la Californie. (L. G. Squier, \emph{Extract from the American Review} for nov. 1848.) Plusieurs de ces constructions remontaient à une époque excessivement reculée, et ne concernent pas les races américaines actuelles. C’est aux Finnois primitifs qu’il faut les rapporter ; aussi n’est-ce pas à cette classe qu’il est fait ici allusion. – Les Alléghaniens paraissent avoir transmis aux Lenni-Lenapes actuels ce mode d’écriture mnémonique qui consiste en signes arbitraires tracés sur une planchette dans le but de rappeler les détails d’un récit à ceux qui le savent et à les empêcher de se tromper dans l’ordre de succession des idées. C’est dans ce système qu’est reproduit le chant mythique intitulé : \emph{Wolum-0lum, la Création}, donné par E. G. Squier, dans le\emph{ Historical and mythological traditions of the Algonquino}, p. 6.}. Je laisserai donc les nations alléghaniennes absolument à l’écart, et je passerai immédiatement à l’examen d’une difficulté qui pèse sur la naissance de leur mode de culture, quel qu’ait pu être son degré, tout comme sur celle de la culture des empires du Mexique et du Pérou des différents âges. On doit se demander pourquoi quelques nations américaines ont été induites à s’élever au-dessus de toutes les autres, et pourquoi le nombre de ces nations a été si limité, en même temps que leur grandeur relative est, en fait, restée si médiocre ?\par
C’est déjà avoir une réponse que d’observer, comme on a pu le faire d’après les remarques précédentes, que ces développements partiels avaient été déterminés en partie par des combinaisons fortuites entre les mélanges jaunes et noirs. En voyant combien les aptitudes résultant de ces combinaisons étaient en définitive bornées, et les singulières lacunes qui caractérisent leurs travaux et leurs œuvres, on a pu se convaincre que les civilisations américaines ne s’élevaient pas, dans le détail, beaucoup au-dessus de ce que les meilleures races malaises de la Polynésie ont réussi à produire. Toutefois il ne faut pas se le dissimuler non plus, si défectueuses que nous apparaissent les organisations aztèque et quichna, il est cependant en elles quelque chose d’essentiellement supérieur à la science sociale pratiquée à Tonga-Tabou et dans l’île d’Hawaii ; on y aperçoit un lien national plus fortement tendu, une conscience plus nette d’un but qui est, de lui-même, d’une nature plus complexe ; de sorte que l’on est en droit de conclure, malgré beaucoup d’apparences contraires, que le mélange polynésien le mieux doué n’arrive pas encore tout à fait à égaler ces civilisations du grand continent occidental, et, en conséquence, on est amené à croire que, pour déterminer cette différence, il a fallu l’intervention locale d’un élément plus énergique, plus noble que ceux dont les espèces jaune et noire ont la disposition. Or il n’est dans le monde que l’espèce blanche qui puisse fournir cette qualité suprême. Il y a donc\emph{, a priori}, lieu de soupçonner que des infiltrations de cette essence pré-excellente ont quelque peu vivifié les groupes américains, là où des civilisations ont existé. Quant à la faiblesse de ces civilisations, elle s’explique par la pauvreté des filons qui les ont fait naître. J’insiste sur cette dernière idée.\par
 Les éléments blancs, s’ils ont paru créer les principales parties de la charpente sociale, ne se révèlent nullement dans la structure de la totalité. Ils ont fourni la force agrégative, et presque rien de plus. Ainsi ils n’ont pas réussi à consolider l’œuvre qu’ils rendaient possible, puisque nulle part ils ne lui ont assuré la durée. L’empire de l’Anahuac ne remontait qu’au X\textsuperscript{e} siècle, tout au plus ; celui du Pérou, au XI\textsuperscript{e} ; et rien ne démontre que les sociétés précédentes s’enfoncent à une distance bien lointaine dans la nuit des temps. C’est l’avis de M. de Humboldt, que la période du mouvement social en Amérique n’a pas dépassé cinq siècles. Quoi qu’il en soit, les deux grands États que les mains violentes de Cortez et de Pizarre ont détruits marquaient déjà l’ère de la décadence, puisqu’ils étaient inférieurs, dans l’Anahuac, à celui des Olmécas, et, sur le plateau des Andes péruviennes, à celui que les Aymaras avaient autrefois fondé \footnote{Jomard, \emph{les Antiquités américaines au point de vue de la géographie}, p. 6.}.\par
La présence de quelques éléments blancs rendue nécessaire, affirmée d’office par l’état des choses, est confirmée par le double témoignage des traditions américaines elles-mêmes, et d’autres récits datant de la fin du X\textsuperscript{e} siècle et du commencement du XI\textsuperscript{e}, qui nous sont transmis par les Scandinaves. Les Incas déclarèrent aux Espagnols qu’ils tenaient leur religion et leurs lois d’un homme étranger de race blanche. Ils ajoutaient même cette observation si caractéristique, que ces hommes avaient une longue barbe, fait complètement anormal chez eux. Il n’y aurait aucune raison pour repousser un récit traditionnel de ce genre, quand même il serait isolé \footnote{Pickering, p. 113. – La même tradition, avec les mêmes détails, se retrouve chez les Muyscas, dans le Bogota, par conséquent à une distance considérable du Mexique.}.\par
Voici qui lui donne une force irrésistible. Les Scandinaves de l’Islande et du Groënland tenaient, au X\textsuperscript{e} siècle, pour indubitable que des relations fort anciennes avaient eu lieu entre l’Amérique du Nord et l’Islande. Ils avaient d’autant plus de motifs de ne pas douter de la possibilité des faits que leur racontaient à cet égard les habitants de Limerick, que plusieurs de leurs propres expéditions avaient été rejetées par les tempêtes soit sur la côte islandaise, en allant en Amérique, soit sur la côte américaine, en allant en Islande. Ils racontaient donc, d’après ce qui leur avait été dit, qu’un guerrier gallois appelé Madok, parti de l’île de Bretagne, avait navigué très loin dans l’ouest \footnote{« Cambro-Britannos, ibidem, anno 1170, duce Madoco concedisse, nonnullis probatum « habetur et alios quoque Europæos, tam ante quam post hoc tempus, notitiam terræ « habuisse, non amplius absurdum aut improbabile existimatur. » (Rafn, \emph{Antiq. americanœ}, Hafniæ, 1837, in-4°, p. III-IV.)}. Qu’ayant rencontré là une terre inconnue, il y avait fait un court séjour. Mais, de retour dans sa patrie, il n’avait plus eu d’autre pensée que d’aller s’établir dans le pays transmarin dont la nature mystérieuse lui avait plu ; il avait réuni des colons, hommes et femmes, fait des provisions, armé des vaisseaux, était parti et n’était plus jamais revenu. Cette histoire avait pris un tel développement chez les Scandinaves du Groënland qu’en 1121 \footnote{Rafn, \emph{Antiq. americ}., p. 262 : « Excerpta ex annalibus Islandarum : ann. 1121 « Eiriker « Biskup af graenlandi for at leita Vinlands. »} l’évêque Éric s’embarqua pour aller porter, à ce qu’on suppose, à l’antique colonisation islandaise les consolations et les secours de la religion, et les maintenir dans la foi, où on se plaisait à croire qu’ils étaient demeurés fermes.\par
Ce ne fut pas seulement au Groënland et en Islande que cette tradition s’établit. De l’Islande, où elle avait évidemment vu le jour, elle était passée en Angleterre, et y avait si bien pris créance, que les premiers colons britanniques du Canada ne cherchaient pas moins activement, dans leur nouvelle possession, les descendants de Madok, que les Espagnols, sous Christophe Colomb, avaient cherché les sujets du grand khan de la Chine à Hispaniola. On crut même avoir trouvé la postérité des émigrants gallois dans la tribu indienne des Mandans. Tous ces récits, encore une fois, sont obscurs sans doute ; mais on ne peut contester leur antiquité, et il existe encore bien moins de raisons de douter de leur parfaite et irréprochable exactitude.\par
Il en résulte pour les Islandais, mais très probablement pour les Islandais d’origine scandinave, une certaine auréole de courage aventureux et de goût des entreprises lointaines. Cette opinion est appuyée par la circonstance incontestable qu’en 795 des navigateurs de la même nation avaient débarqué dans l’Islande, encore inoccupée et y avaient établi des moines \footnote{A. de Humboldt, \emph{Examen critique de l’histoire de la géographie du nouveau continent}, t. II, p. 90 et pass.}. Trois Norwégiens, le roi de mer Naddok et les deux héros Ingulf et Hiorleïf, suivirent cet exemple, et amenèrent sur l’île, en 874, une colonie composée de nobles scandinaves qui, fuyant devant les prétentions despotiques d’Harald aux beaux cheveux, cherchaient une terre où ils pussent continuer l’existence indépendante et fière des antiques odels arians. Habitués que nous sommes à considérer l’Islande dans son état actuel, stérilisée par l’action volcanique et l’invasion croissante des glaces, nous nous la figurons, au début des âges moyens, peu peuplée comme nous la voyons aujourd’hui, réduite au rôle d’annexe des autres pays normands, et nous méconnaissons l’activité dont elle était alors le foyer. Il est facile de rectifier d’aussi fausses préventions.\par
Cette terre, choisie par l’élite des nobles norwégiens, était un foyer de grandes entreprises où abondaient constamment tous les hommes énergiques du monde scandinave \footnote{Les preuves abondent de toutes parts dans les annales des royaumes scandinaves, mais ce sont surtout les chroniques islandaises qui présentent le tableau le plus vivant des faits. Il suffit de les feuilleter pour être convaincu.}. Il en partait, chaque jour, des expéditions qui s’en allaient à la pêche de la baleine et à la recherche de nouvelles contrées, tantôt dans l’extrême nord-ouest, tantôt dans le sud-ouest. Cet esprit remuant était entretenu par la foule des scaldes et des moines érudits qui, d’une part, avaient porté au plus haut degré la science des antiquités du Nord et fait de leur nouveau séjour la métropole poétique de la race, et qui, de l’autre, y attiraient incessamment la connaissance des littératures méridionales, et traduisaient dans le langage usuel les principales productions des pays romans \footnote{Weinhold, \emph{Die deutschen Frauen im Mittelalter}, p. 187 et ailleurs.}.\par
L’Islande était donc, au X\textsuperscript{e} siècle, un territoire très intelligent, très populeux, très actif, très puissant, et ses habitants le démontrèrent bien par ce fait, qu’arrivés et établis dans leur île en 874, ils fondaient leurs premiers établissements groënlandais en 986. Nous n’avons eu d’exemple d’une pareille exubérance de forces que chez les Carthaginois. C’est que l’Islande était, en effet, comme la cité de Didon, l’œuvre d’une race aristocratique parvenue, avant d’agir, à tout son développement, et cherchant dans l’exil non seulement le maintien, mais encore le triomphe de ses droits.\par
Quand une fois les Scandinaves eurent pris pied dans le Groënland, leurs colonisations s’y succédèrent, s’y multiplièrent rapidement, et en même temps des voyages de découverte commencèrent vers le sud \footnote{A. de Humboldt remarque que le Groënland Oriental est si rapproché de la péninsule scandinave et du nord de l’Écosse, qu’il n’existe d’un point à l’autre qu’une distance de 269 lieues marines, trajet qui, par un vent frais et continu, peut être franchi en moins de quatre jours de navigation (\emph{Op. cit}., t. II, p. 76.)}. L’Amérique fut ainsi trouvée par les rois de mer, comme si la Providence avait voulu qu’aucune gloire ne manquât à la plus noble des races.\par
On connaît très peu, très mal, très obscurément, l’histoire des rapports du Groënland avec le continent occidental. Deux points seulement sont fixés avec la dernière évidence par quelques chroniques domestiques parvenues jusqu’à nous. Le premier, c’est que les Scandinaves avaient pénétré, au X\textsuperscript{e} siècle, jusqu’à la Floride, au sud de la contrée où ils avaient trouvé des vignes, et qu’ils avaient appelée Vinland. Dans le voisinage était, suivant eux, l’ancien pays des colons irlandais, que leurs documents nomment \emph{Hirttramanhaland}, le \emph{pays des blancs} : c’était l’expression dont s’étaient servis les Indiens, premiers auteurs de ce renseignement, et que ceux qui le recevaient n’avaient pas hésité à traduire par le mot \emph{: Island it mikla, la grande Islande} \footnote{Chronique d’Islande, intitulée \emph{Islingabok}, composée vers 1080 ou 1090 ; \emph{Antiquit. americ.} p. 211.}.\par
Le second point est celui-ci : jusqu’en 1347 les communications entre le Groënland et le bas Canada étaient fréquentes et faciles. Les Scandinaves allaient y charger des bois de construction \footnote{\emph{Antiquit. americ}. p. 265.}.\par
Vers la même époque un changement remarquable s’opère dans l’état des popula­tions groënlandaises et islandaises. Les glaces, gagnant plus de terrain, rendent le climat par trop dur et la terre trop stérile. La population décroît rapidement, et si bien que le Groënland se trouve tout à coup absolument abandonné et désert, sans qu’on puisse dire ce que ses habitants sont devenus. Cependant ils n’ont pas été détruits subitement par des convulsions de la nature. On peut contempler encore aujourd’hui des restes d’habitations et d’églises fort nombreuses qui évidemment ont été quittées, et ne s’écroulent que sous l’action du temps et de l’abandon. Ces restes ne révèlent aucune trace d’un cataclysme qui aurait englouti ceux qui les habitaient jadis. Il faut donc de toute nécessité que ces derniers, en désertant leurs demeures, aient été chercher ailleurs un autre séjour. Où sont-ils allés ?\par
On a voulu à toute force les retrouver individuellement, un à un, dans les États du nord de l’Europe, et on a oublié qu’il ne s’agissait pas d’hommes isolés, mais de véritables populations qui, arrivant en masse en Norwège, en Hollande, en Allemagne, auraient excité une attention dont les récits des chroniqueurs auraient conservé la trace, ce qui n’est pas. Il est plus admissible, il est plus raisonnable de croire que les Scandinaves Groënlandais et une partie des hommes de l’Islande, ayant depuis de longues années connaissance des territoires fertiles et bien boisés, du climat doux et attrayant du Vinland, et s’étant fait une habitude de parcourir les mers occidentales, échangèrent peu à peu pour cette résidence, de tous points préférable, des contrées qui leur devenaient inhabitables, et qu’ils émigrèrent en Amérique, absolument comme leurs compatriotes de Suède et de Norwège avaient naguère passé de leurs rochers du nord dans la Russie et dans les Gaules \footnote{Les Scandinaves de l’Islande et du Groënland, vivant sous le régime des odels, s’occupaient beaucoup plus de l’histoire des familles que de celle de la nation. Aussi la plupart des documents dont je me suis servi ne sont-ils que des chroniques domestiques et des chants destinés à célébrer les exploits d’un héros. Dans cet état de choses, on conçoit que presque toutes les relations de voyages se soient perdues et aient disparu avec les familles qu’elles glorifiaient. Il ne nous reste d’un peu étendu que ce qui a rapport à la race d’Érik le Roux. Il est donc extrêmement possible que, si les marins de cette maison se sont toujours préoccupés du Vinland, qu’ils avaient découvert et qui était pour eux une sorte de possession, d’autres se soient dirigés de préférence sur divers points leur appartenant au même titre. C’est une hypothèse, sans doute, mais elle est naturelle, et voici qui la soutient : un planisphère islandais de la fin du XIII\textsuperscript{e} siècle divise la terre en quatre parties : l’Europe, l’Asie, l’Afrique, et une quatrième qui occupe à elle seule tout un hémisphère et qui est appelée \emph{Synnri-bigd} ; ou\emph{ région méridionale de la terre habitée.} Cette carte a été publiée déjà dans plusieurs occasions. Elle n’est pas d’ailleurs unique, et démontre que les islandais attribuaient une très grande étendue vers le sud au continent américain : donc ils ne s’étaient pas bornés à en visiter l’hémisphère boréal.}.\par
C’est ainsi que les races aborigènes du nouveau continent ont pu s’enrichir de quelques apports du sang des blancs, et que celles qui possédèrent au milieu d’elles des métis islandais ou des métis scandinaves se virent douées du pouvoir de créer des civilisations, tâche glorieuse à laquelle leurs congénères moins heureux étaient nativement et restèrent à perpétuité inhabiles. Mais, comme l’affluent ou les affluents d’essence noble mis en circulation dans les masses malaises étaient trop faibles pour produire rien de vaste ni de durable, les sociétés qui en résultèrent furent peu nom­breuses, et surtout très imparfaites, très fragiles, très éphémères, et, à mesure qu’elles se succédèrent, moins intelligentes, moins marquées au sceau de l’élément dont elles étaient issues, de telle sorte que, si la découverte nouvelle de l’Amérique par Christophe Colomb, au lieu de s’accomplir au XV\textsuperscript{e} siècle, n’avait été réalisée qu’au XIX\textsuperscript{e}, nos marins n’auraient vraisemblablement trouvé ni Mexico, ni Cuzco, ni temples du Soleil, mais des forêts partout, et dans ces forêts des ruines hantées par les mêmes sauvages qui les traversent aujourd’hui \footnote{A. de Humboldt, \emph{ouvr. cité}, t. I. – L’illustre auteur place l’état de civilisation connue des Aztèques et des Incas entre l’époque des expéditions scandinaves et le XV\textsuperscript{e} siècle. Ces deux suprêmes efforts de la sociabilité américaine étaient, suivant lui, fort débiles et très inférieurs à ceux qui les avaient précédés d’environ cinq cents ans en moyenne. C’est ici le lieu de dire quelques mots d’une hypothèse très répandue et très admissible qui attribue aux populations de l’Asie orientale, Chinois et japonais, une grande influence sur la naissance des civilisations de l’ancien continent. A. de Humboldt (\emph{Vue des Cordillères}), Prescott, dans son troisième volume de son histoire de la conquête du Mexique. Morton et la plupart des archéologues actuels, ou appuient fortement ou discutent à peine la possibilité des faits. Rien de plus naturel, en effet, que des communications fortuites ou même préméditées aient eu lieu de ce côté, et on démontrera peut-être un jour d’une manière satisfaisante que le pays de Fon-dang, cité par quelques écrivains chinois comme existant à l’ouest, n’est autre que le continent d’Amérique. Je n’ai pas cru devoir cependant rattacher directement mes démonstrations à ce système, le considérant comme susceptible, pour ce qui a trait au Japon, de développements très considérables qu’il est dangereux de prévenir. Lorsque le fait sera établi, il en résultera que l’Amérique, outre ce qu’elle a reçu des Scandinaves, a encore recueilli par l’intermédiaire d’aventuriers malais, faiblement arianisés, une petite portion de plus d’essence noble. Aucun des principes posés ici n’en sera ébranlé.}.\par
Les civilisations américaines étaient si débiles qu’elles sont tombées en poussière au premier choc. Les tribus spécialement douées qui les soutenaient se sont dispersées sans difficulté devant le sabre d’un vainqueur imperceptible, et les masses populaires qui les avaient subies, sans les comprendre, se sont retrouvées libres de suivre les directions de leurs nouveaux maîtres ou de continuer leur antique barbarie. La plupart ont préféré prendre le dernier parti ; elles rivalisent d’abrutissement avec ce qu’on voit de mieux en ce genre en Australie. Quelques-unes possèdent même la conscience de leur abaissement, et elles en agréent toutes les conséquences. De ce nombre est la tribu brésilienne, qui s’est fait, pour ses fêtes, un air de danse dont voici les paroles :\par

\begin{verse}
Quand je serai mort,\\
Ne me pleure pas ;\\
Il y a le vautour\\
Qui me pleurera.\\
Quand je serai mort,\\
Jette-moi dans la forêt ;\\
Il y a l’armadille\\
Qui m’enterrera.\\
\end{verse}
\noindent On n’est pas plus philosophe \footnote{Cette chanson en langue géral est donnée par Martius. u. Spix, \emph{ouvr. cité}, t. III, p. 1085.} ; les bêtes de proie sont des fossoyeurs acceptés. Les nations américaines n’ont donc obtenu qu’à un seul moment, et sous un jour bien sombre, la lumière civilisatrice. Maintenant les voilà revenues à leur état normal : c’est une sorte de demi-néant intellectuel, et rien ne les en doit arracher que la mort physique \footnote{Humboldt, \emph{Histoire critique}, etc., t. II, p. 128. – Les observations de cet écrivain s’appliquent surtout aux peuples chasseurs de l’hémisphère septentrional.}.\par
Je me trompe. Beaucoup de ces nations semblent, au contraire, à l’abri de cette fin misérable. Il ne s’agit, pour entrer en goût de le soutenir, que d’envisager la question sous une face nouvelle.\par
De même que les mélanges opérés entre les indigènes et les colons islandais et scandinaves ont pu créer des métis relativement civilisables, de même les descendants des conquérants espagnols et portugais, en se mariant aux femmes des pays occupés par eux, ont donné naissance à une race mixte supérieure à l’ancienne population. Mais, si l’on veut considérer le sort des naturels américains sous cet aspect, il faut en même temps tenir compte de la dépression manifestée, par le fait de cet hymen, dans les facultés des groupes européens qui ont consenti à le contracter. Si les Indiens des pays espagnols et portugais sont, çà et là, un peu moins abâtardis, et surtout infiniment plus nombreux \footnote{M. A. de Humboldt démontre même que la population indigène des contrées espagnoles est en voie de prospérité et d’augmentation, au détriment, bien entendu, de la descendance des conquérants immergés dans cette masse. (\emph{Ouvr. cité}, t. II, p. 129.) Cet état de choses trouble beaucoup la sécurité de conscience des observateurs américains dans le pays desquels se manifeste un phénomène tout opposé. Il ébranle presque leur confiance dans ce qu’on appelle \emph{les bienfaits de la civilisation}, et M. Pickering, confondant du reste toutes notions raisonnables, se pose cette question : « By an exception to the usual tendency of european civilisation, there are grounds for questioning whether Peru has altogether gained by the change. » (P. 21.) – C’est plutôt au sujet des tribus de Lennis-Lenapés que le savant Américain devrait soulever ce doute.} que ceux des autres parties du nouveau continent, il faut considérer que cette amélioration dans l’état de leurs aptitudes est bien minime et que la consé­quence la plus pratique en a été l’avilissement des races dominatrices. L’Amérique du Sud, corrompue dans son sang créole, n’a nul moyen désormais d’arrêter dans leur chute ses métis de toutes variétés et de toutes classes. Leur décadence est sans remède.\par

\begin{center}
\noindent \centerline{Livre sixième}
\end{center}

\section[{VI.8. Les colonisations européennes, en Amérique.}]{VI.8. \\
Les colonisations européennes \\
en Amérique.}
\noindent Les relations des indigènes américains avec les nations européennes, à la suite de la découverte de 1495, ont été marquées de caractères très différents, déterminés par la mesure de parenté primitive entre les groupes mis en présence. Parler des rapports de parenté entre les nations du nouveau monde et les navigateurs de l’ancien, semblera d’abord hasardé. En y réfléchissant mieux, on se rendra compte que rien n’est plus réel, et on va en voir les effets.\par
Les peuples d’outre-mer qui ont le plus agi sur les Indiens sont les Espagnols, les Portugais, les Français et les Anglais.\par
Dès le début de leur établissement, les sujets des rois catholiques se sont intimement rapprochés des gens du pays. Sans doute ils les ont pillés, battus, et très souvent massacrés. De tels événements sont inséparables de toute conquête, et même de toute domination. Il n’en est pas moins vrai que les Espagnols rendaient hommage à l’organisation politique de leurs vaincus, et la respectaient en ce qui n’était pas con­traire à leur suprématie. Ils concédaient le rang de gentilhomme et le titre de \emph{don} à leurs princes ; ils usaient des formules impériales quand ils s’adressaient à Montézuma ; et même après avoir proclamé sa déchéance et exécuté sa condamnation à mort, ils ne parlaient de lui qu’en se servant du mot de majesté. Ils recevaient ses parents au rang de leur grandesse, et en faisaient autant pour les Incas. D’après ce principe, ils épousèrent sans difficulté des filles de caciques, et, de tolérance en tolérance, en arrivèrent à allier librement une famille d’hidalgos à une famille de mulâtres. On pourrait croire que cette conduite, que nous appellerions libérale, était imposée aux Espagnols par la nécessité de s’attacher des populations trop nombreuses pour ne pas être ménagées ; mais dans telles contrées où ils n’avaient affaire qu’à des tribus sauvages et clairsemées, dans l’Amérique centrale, à Bogota, dans la Californie, ils agissaient absolument de même. Les Portugais les imitèrent sans réserve. Après avoir déblayé un certain rayon autour de Rio-Janeiro, ils se mêlèrent sans scrupule aux anciens possesseurs de la contrée, sans se scandaliser de l’abrutissement de ceux-ci. Cette facilité de mœurs provenait, sans aucun doute, des points d’attraction que la composition des races respectives laissait subsister entre les maîtres et les sujets.\par
Chez les aventuriers sortis de la péninsule hispanique, et qui appartenaient pour la plupart à l’Andalousie \footnote{Il y a une exception à faire en faveur de la population européenne du Chili. Elle est venue en majorité du nord de l’Espagne, elle s’est moins mêlée aux aborigènes ; elle est donc très naturellement supérieure aux habitants des républiques voisines, et son état politique s’en ressent.}, le sang sémitique dominait, et quelques éléments jaunes, provenus des parties ibériennes et celtiques de la généalogie, donnaient à ces groupes une certaine portée malaise. Ses principes blancs étaient là en minorité devant l’essence mélanienne. Une affinité véritable existait donc entre les vainqueurs et les vaincus, et il en résultait une assez grande facilité de s’entendre, et, par suite, pro­pension à se mêler.\par
Pour les Français, il en était à peu près de même, quoique par un autre côté, et nullement par ce côté. Dans le Canada, nos émigrants ont très fréquemment accepté l’alliance des aborigènes et, ce qui fut toujours assez rare de la part des colonisateurs anglo-saxons, ils ont adopté souvent et sans peine le genre de vie des parents de leurs femmes. Les mélanges ont été si faciles, que l’on trouve peu d’anciennes familles canadiennes qui n’aient touché, au moins de loin, à la race indienne ; et cependant ces mêmes Français, si accommodants dans le nord, n’ont jamais voulu, dans le sud, admettre la possibilité d’une alliance avec l’espèce nègre que comme une flétrissure, ni voir dans les mulâtres que des avortons réprouvés. La cause de cette inconséquence apparente est aisée à expliquer. La plupart des familles qui se sont les premières établies, tant au Canada qu’aux Antilles, appartenaient aux provinces de Bretagne ou de Normandie. Une affinité existait, pour la partie gallique de leur origine, avec les tribus malaises très jaunes du Canada, tandis que tout leur naturel répugnait à contracter alliance avec l’espèce noire sur les terrains où ils se trouvaient rapprochés d’elle, bien différents en cela, comme on l’a vu, des colons espagnols, qui, dans l’Amérique du Sud, l’Amérique centrale, le Mexique, se trouvent aujourd’hui, grâce aux mélanges de toute nature qu’ils ont aisément acceptés, dans des conditions de concordances fâcheuses avec les groupes indigènes qui les entourent.\par
Il y aurait assurément injustice à prétendre que le citoyen de la république mexicaine, ou le général improvisé qui apparaît à chaque instant dans la confédération argentine, soient sur le même plan que le Botoendo anthropophage ; mais on ne saurait nier non plus que la distance qui sépare ces deux termes de la proposition n’est pas indéfinie, et que, sous bien des aspects, le cousinage se laisse découvrir. Tout ce monde indien habitant les forêts, chercheur d’or, à demi blanc, militaire de hasard, mulâtre à moitié indigène ; tout ce monde, depuis le président de l’État jusqu’au dernier vagabond, se comprend à merveille et peut vivre ensemble. On s’en aperçoit, du reste, à la façon dont s’y prend le farouche cavalier des pampas pour manier les institutions européennes que notre folie propagandiste l’a induit à accepter. Les gouvernements de l’Amérique du Sud ne sont guère comparables qu’à l’empire d’Haïti, il faut bien consentir désormais à s’en apercevoir, et ce sont les hommes qui naguère applaudissaient avec le plus d’emportement à la prétendue émancipation de ces peuples, et qui en attendaient les plus beaux résultats, ce sont ceux-là mêmes qui aujourd’hui, devenus justement incrédules sur un avenir qu’ils ont tant hâté de leurs vœux, de leurs écrits et de leurs efforts, prédisent le plus haut qu’il faut un joug à ces amas de métis, et qu’une domination étrangère peut seule leur donner l’éducation forte dont ils ont besoin. En parlant ainsi, ils indiquent du doigt, avec un sourire satisfait, le point de l’horizon d’où viennent déjà les envahisseurs prédestinés ; ils montrent les Anglo-Saxons des États-Unis d’Amérique. Ce nom d’Anglo-Saxons parait flatter l’imagination des habitants de la grande confédération transatlantique ; malgré le droit de plus en plus équivoque que la population actuelle peut avoir à le réclamer, commençons par le lui donner un moment, ne serait-ce que pour faciliter l’examen des premiers temps de l’agrégation dont les colons anglais forment le noyau.\par
Ces Anglo-Saxons, ces gens d’origine britannique, représentent la nuance la plus éloignée tout à la fois du sang des aborigènes et de celui des nègres d’Afrique. Ce n’est pas qu’on ne pût trouver dans leur essence quelques traces d’affinités finniques ; mais elles sont contre-balancées par la nature germanique, à la vérité ossifiée, un peu flétrie, dépouillée de ses côtés grandioses, toutefois encore rigide et vigoureuse, qui survit en leur organisme. Ce sont donc, pour les représentants purs ou métis des deux grandes variétés inférieures de l’espèce, des antagonistes irréconciliables. Voilà leur situation sur leur propre territoire. À l’égard des autres contrées indépendantes de l’Amérique, ils composent un État fort en face d’États agonisants. Ces derniers, au lieu d’opposer à l’Union américaine, au défaut d’une organisation ethnique quelque peu compacte, au moins une certaine expérience de la civilisation, et l’énergie apparente ou transitoire d’un gouvernement despotique, ne possèdent que l’anarchie à tous les degrés ; et quelle anarchie, puisqu’elle réunit les disparates de l’Amérique malaise à ceux de l’Europe romanisée !\par
Le noyau anglo-saxon existant aux États-Unis n’a donc nulle peine à se faire reconnaître pour l’élément vivace du nouveau continent. Il est placé, vis-à-vis des autres populations, dans cette attitude de supériorité accablante où furent jadis toutes les branches de la famille ariane, Hindous, Kchattryas Chinois, Iraniens, Sarmates, Scandinaves, Germains, à l’égard des multitudes métisses. Bien que ce dernier représentant de la grande race soit fortement déchu, il offre cependant un tableau assez curieux des sentiments de celle-ci pour le reste de l’humanité. Les Anglo-Saxons se comportent en maîtres envers les nations inférieures ou même seulement étrangères à la leur, et il n’est pas sans utilité de profiter de cette occasion d’étudier dans le détail ce que c’est que le contact d’un groupe fort avec un groupe faible. L’éloignement des temps et l’obscurité des annales ne nous a pas toujours permis de saisir avec l’exactitude qui nous est maintenant offerte les linéaments de ce tableau.\par
Les restes anglo-saxons, dans l’Amérique du Nord, forment un groupe qui ne doute pas un seul instant de sa supériorité innée sur le reste de l’espèce humaine, et des droits de naissance que cette supériorité lui confère. Imbu de tels principes, qui sont plutôt encore des instincts que des notions, et dominé par des besoins bien autrement exigeants que ceux des siècles où la civilisation n’existait qu’à l’état d’aptitude, ce groupe ne s’est pas même accommodé, comme les Germains, de partager la terre avec les anciens possesseurs. Ceux-ci, il les a dépouillés, il les a refoulés de solitudes en solitudes ; il leur a acheté de force et à vil prix le sol qu’ils ne voulaient pas vendre, et le misérable lambeau de champ que, par des traités solennels et répétés, il leur a garanti, parce qu’il fallait pourtant que ces misérables pussent poser le pied quelque part, il n’a pas tardé à le leur prendre, impatient, non plus de leur présence, mais de leur vie. Sa nature raisonnante et amie des formes légales lui a fait trouver mille subterfuges pour concilier le cri de l’équité avec le cri plus impérieux encore d’une rapacité sans bornes. Il a inventé des mots, des théories, des déclamations pour innocenter sa conduite. Peut-être a-t-il reconnu, au fond du dernier retrait de sa conscience, l’impropriété de ces tristes excuses. Il n’en a pas moins persévéré dans l’exercice du droit de tout envahir, qui est sa première loi, et la plus nettement gravée dans son cœur.\par
Vis-à-vis des nègres il ne se montre pas moins impérieux qu’avec les aborigènes : ceux-ci, il les dépouille jusqu’à l’os ; ceux-là, il les courbe sans hésitation jusqu’au niveau du sol qu’ils travaillent pour lui, et cette façon d’agir est d’autant plus remarquable qu’elle n’est pas en accord avec les principes d’humanité professés par ceux qui la pratiquent. Cette inconséquence veut une explication. Au point où elle est poussée, elle est toute nouvelle sur la terre. Les Germains n’en ont pas donné l’exemple ; se contentant d’une portion de la terre, ils ont garanti le libre usage du reste à leurs vaincus. Ils avaient trop peu de besoins pour se sentir l’envie de tout envahir. Ils étaient trop grossiers pour concevoir la pensée d’imposer à leurs sujets ou à des nations étrangères l’usage de liqueurs ou de matières pernicieuses. C’est là une idée moderne. Ce que ni les Vandales, ni les Goths, ni les Franks, ni les premiers Saxons n’ont imaginé de faire, les civilisations du monde antique, qui, plus raffinées, étaient aussi plus perverses, n’y avaient cependant pas songé davantage. Ce n’est pas le brahmane, ce n’est pas le mage qui ont senti le besoin de faire disparaître autour d’eux, avec une parfaite précision, tout ce qui ne s’associait pas à leur pensée. Notre civilisation est la seule qui ait possédé cet instinct et en même temps cette puissance homicide ; elle est la seule qui, sans colère, sans irritation, et en se croyant, au contraire, douce et compatissante à l’excès, en proclamant la mansuétude la plus illimitée, travaille incessamment à s’entourer d’un horizon de tombes. La raison en est qu’elle ne vit que pour trouver l’utile ; que tout ce qui ne la sert pas dans ses tendances lui nuit, et que, logiquement, tout ce qui nuit est d’avance condamné, et, le moment arrivé, détruit.\par
Les Anglo-Américains, représentants convaincus et fidèles de ce mode de culture, ont agi conformément à ses lois. Ils ne sont pas répréhensibles. C’est sans hypocrisie qu’ils se sont cru le droit de se joindre au concert de réclamations élevé par le XVIII\textsuperscript{e} siècle contre toute espèce de contrainte politique, contre l’esclavage des noirs en particulier. Les partis et les nations jouissent, comme les femmes, de l’avantage de braver la logique, d’associer les dispararates intellectuelles et morales les plus surprenantes, sans pour cela manquer de sincérité. Les concitoyens de Washington, en déclamant avec énergie pour l’affranchissement de l’espèce nègre, ne se sont pas crus obligés de donner l’exemple ; comme les Suisses, leurs émules théoriques dans l’amour de l’égalité, qui savent maintenir encore contre les juifs la législation du moyen âge, ils ont traité les noirs attachés à leur glèbe avec la dernière rigueur, avec le dernier mépris. Plus d’un héros de leur indépendance leur a donné l’exemple de ce désaccord instinctif entre les maximes et les actes. Jefferson, dans ses rapports avec ses négresses esclaves et les enfants qui en provenaient, a laissé des souvenirs qui, en petit, ne ressemblent pas mal aux excès des premiers Chamites blancs.\par
Les Anglo-Saxons d’Amérique sont religieux : ce trait leur est resté assez bien empreint de la noble partie de leur origine. Cependant ils n’acceptent ni les terreurs ni le despotisme de la foi. Chrétiens, on ne les voit pas sans doute, comme les anciens Scandinaves, rêver d’escalader le ciel, ni combattre de plain-pied avec la Divinité ; mais ils la discutent librement, et, particularité véritablement typique, en la discutant toujours, semblables encore en ceci à leurs aïeux arians, ils ne la nient jamais, et restent dans ce remarquable milieu qui, touchant à la superstition d’une part, à l’athéisme de l’autre, se maintient avec un égal dégoût, une horreur égale, au-dessus de ces deux abîmes.\par
Possédés de la soif de régner, de commander, de posséder, de prendre et de s’étendre toujours, les Anglo-Saxons d’Amérique sont primitivement agriculteurs et guerriers ; je dis guerriers, et non pas militaires : leur besoin d’indépendance s’y oppose. Ce dernier sentiment fut, à toutes les époques, la base et le mobile de leur existence politique. Ils ne l’ont point acquis à la suite de leur rupture avec la mère patrie ; ils l’ont toujours possédé. Ce qu’ils ont gagné à leur révolution est considé­rable, puisque à dater de ce moment ils se sont trouvés, quant à leur action extérieure, maîtres absolus et libres d’employer leurs forces à leur gré pour s’étendre indéfiniment. Mais, en ce qui concerne l’essentiel de leur organisation intérieure, aucun germe nouveau n’a paru. Avec ou sans la participation de la métropole, les peuples des États-Unis actuels étaient constitués de façon à se développer dans la direction communale où on les voit agir. Leurs magistratures électives et temporaires, leur jalouse surveil­lance du chef de l’État, leur goût pour le fractionnement fédératif, rappellent bien les vicampatis des Hindous primitifs, la séparation par tribus, les ligues des peuples parents, anciens dominateurs de la Perse septentrionale, de la Germanie, de l’Heptarchie saxonne. Il n’est pas jusqu’à la constitution de la propriété foncière qui n’ait encore beaucoup de traits de la théorie de l’odel.\par
 On attache donc ordinairement une importance inconsidérée à la crise où brilla Washington. Assurément ce fut une évolution considérable dans les destinées du groupe anglo-saxon transplanté en Amérique ; ce fut une phase brillante et en même temps fortifiante ; mais y apercevoir une naissance, une fondation de la nationalité, c’est faire tort tout à la fois à la gloire des compagnons de Penn ou des gentilshommes de la Virginie, et à l’exacte appréciation des faits. L’émancipation n’a été qu’une applica­tion nécessaire de principes existant déjà, et la véritable année climatérique des États-Unis n’est pas encore arrivée.\par
Ce peuple républicain témoigne de deux sentiments qui tranchent d’une manière complète avec les tendances naturelles de toutes les démocraties issues de l’excès des mélanges. C’est d’abord le goût de la tradition, de ce qui est ancien, et, pour employer un terme juridique, des précédents ; penchant si prononcé que, dans l’ordre des affections, il défend même l’image de l’Angleterre contre de nombreuses causes d’ani­mosité. En Amérique, on modifie beaucoup et sans cesse les institutions ; mais il y a, parmi les descendants des Anglo-Saxons, une répugnance marquée aux transfor­mations radicales et subites. Beaucoup de lois importées de la métropole, au temps où le pays était sujet, sont restées en vigueur. Plusieurs exhalent même, au milieu des émanations modernes qui les entourent, une saveur de vétusté qui s’allie chez nous aux souvenirs féodaux. En second lieu, les mêmes Américains sont beaucoup plus préoccupés qu’ils ne l’avouent des distinctions sociales ; seulement, tous veulent les posséder. Le nom de citoyen n’est pas plus popularisé parmi eux que le titre chevaleresque de \emph{squire}, et cette préoccupation instinctive de la position personnelle, apportée par des colons de même souche qu’eux dans le Canada, y a déterminé les mêmes effets. On lit très bien dans les journaux de Montréal, à la page des annonces, que M***, épicier, \emph{gentilhomme}, tient telle denrée à la disposition du public.\par
Ce n’est pas là un usage indifférent ; il indique chez les démocrates du nouveau monde une disposition à se rehausser qui fait un contraste bien complet avec les goûts tout opposés des révolutionnaires de l’ancien. Chez ces derniers, la tendance est, au contraire, à descendre au plus bas possible, afin de ravaler les essences ethniques les plus hautes et les moins nombreuses au niveau des plus basses, qui, par leur abondance, donnent le ton et dirigent tout.\par
Le groupe anglo-saxon ne représente donc pas parfaitement ce qu’on entend, de ce côté de l’Atlantique, par le mot \emph{démocratie.} C’est plutôt un état-major sans troupes. Ce sont des hommes propres à la domination, qui ne peuvent pas exercer cette faculté sur leurs égaux, mais qui la feraient volontiers sentir à leurs inférieurs. Ils sont, sous ce rapport, dans une situation analogue à celle des nations germaniques peu de temps avant le V\textsuperscript{e} siècle. Ce sont, en un mot, des aspirants à la royauté, à la noblesse, armés des moyens intellectuels de légitimer leurs vues. Reste à savoir si les circonstances ambiantes s’y prêteront. Quoi qu’il en soit, veut-on aujourd’hui considérer en face et examiner à son aise l’homme redouté qui s’appelle un barbare dans le langage des peuples dégénérés qui le redoutent ? Qu’on se place à côté du Mexicain, qu’on l’écoute parler, et, suivant la direction de son regard effrayé, on contemplera le chasseur du Kentucky. C’est la dernière expression du Germain ; c’est là le Frank, le Longobard de nos jours ! Le Mexicain a raison de le qualifier de barbare sans héroïsme et sans générosité ; mais il ne faut pas sans doute qu’il soit sans énergie et sans puissance.\par
Ici cependant, quoi qu’en disent les populations effrayées, le barbare est plus avancé dans les branches utiles de la civilisation qu’elles ne le sont elles-mêmes. Cette situation n’est pas sans précédents. Quand les armées de la Rome sémitique conqué­raient les royaumes de l’Asie inférieure, les Romains et les hellénisés se trouvaient avoir puisé leur mode de culture aux mêmes sources. Les gens des Séleucides et des Ptolémées se croyaient infiniment plus raffinés et plus admirables, parce qu’ils avaient croupi plus de temps dans la corruption et qu’ils étaient plus artistes. Les Romains, se sentant plus utilitaires, plus positifs, bien que moins brillants que leurs ennemis, en auguraient la victoire. Ils avaient raison, et l’événement le prouva.\par
Le groupe anglo-saxon est autorisé à entrevoir les mêmes perspectives. Soit par conquête directe, soit par influence sociale, les Américains du Nord semblent destinés à se répandre en maîtres sur toute la face du nouveau monde. Qui les arrêterait ? Leurs propres divisions peut-être, si elles venaient à éclater trop tôt. En dehors de ce péril, ils n’ont rien à craindre ; mais il faut avouer aussi qu’il n’est pas sans gravité.\par
On s’est aperçu déjà que, pour obtenir une vue plus nette du degré d’intensité auquel pouvait parvenir l’action du peuple des États-Unis sur les autres groupes du nouveau monde, il n’a encore été question que de la race qui a fondé la nation, et que, par une supposition tout à fait gratuite, j’ai considéré cette race comme étant encore conservée aujourd’hui dans sa valeur ethnique spéciale et devant y persister indéfiniment. Or, rien de plus fictif. L’Union américaine représente, tout au contraire, entre les pays du monde celui qui, depuis le commencement du siècle, et surtout dans ces dernières années, a vu affluer sur son territoire la plus grande masse d’éléments hétérogènes. C’est un nouvel aspect qui peut, sinon changer, du moins modifier gravement les conclusions présentées plus haut.\par
Sans doute, les alluvions considérables de principes nouveaux qu’apportent les émigrations ne sont pas de nature à créer à l’Union une infériorité quelconque vis-à-vis des autres groupes américains. Ceux-ci, mêlés aux natifs et aux nègres, sont bien résolument déprimés, et, quelque basse que soit la valeur de certains des apports venus d’Europe, encore ces derniers sont-ils moins entachés de dégénération que le fond des populations mexicaines ou brésiliennes. Il n’y a donc rien, dans les observations qui vont suivre, qui infirme ce qui a été dit précédemment de la prépondérance morale des États du nord de l’Amérique vis-à-vis des autres corps politiques du même continent ; mais en ce qui concerne la situation de la république de Washington vis-à-vis de l’Europe, il en est tout autrement.\par
La descendance anglo-saxonne des anciens colons anglais ne compose plus la majeure partie des habitants de la contrée, et, pour peu que le mouvement qui pousse chaque année les Irlandais et les Allemands, par centaines de mille, sur le sol américain se soutienne encore quelque temps, avant la fin du siècle, la race nationale sera en partie éteinte. Du reste, elle est déjà fortement affaiblie par les mélanges. Elle continuera sans doute quelque temps encore à donner l’apparence de l’impulsion ; puis cette apparence s’effacera, et l’empire sera tout à fait aux mains d’une famille mixte, où l’élément anglo-saxon ne jouera plus qu’un rôle des plus subordonnés. Je remarquerai incidemment que déjà le gros de la variété primitive s’éloigne des côtes de la mer, et s’enfonce dans l’ouest, où le genre de vie convient mieux à son activité et à son courage aventureux.\par
Mais les nouveaux arrivés, que sont-ils ? Ils représentent les échantillons les plus variés de ces races de la vieille Europe dont il y a le moins à attendre. Ce sont les produits du détritus de tous les temps : des Irlandais, des Allemands, tant de fois métis, quelques Français qui ne le sont pas moins, des Italiens qui les surpassent tous. La réunion de tous ces types dégénérés donne et donnera nécessairement la naissance à de nouveaux désordres ethniques ; ces désordres n’ont rien d’inattendu, rien de nouveau ; ils ne produiront aucune combinaison qui ne se soit réalisée déjà ou ne puisse l’être sur notre continent. Pas un élément fécond ne saurait s’en dégager, et même le jour où des produits résultant de séries indéfiniment combinées entre des Allemands, des Irlandais, des Italiens, des Français et des Anglo-Saxons, iront par surcroît se réunir, s’amal­gamer dans le sud avec le sang composé d’essence indienne, nègre, espagnole et portugaise qui y réside, il n’y a pas moyen de s’imaginer que d’une si horrible confusion il résulte autre chose que la juxtaposition incohérente des êtres les plus dégradés.\par
J’assiste avec intérêt, bien qu’avec une sympathie médiocre, je l’avoue, au grand mouvement que les instincts utilitaires se donnent en Amérique. Je ne méconnais pas la puissance qu’ils déploient ; mais, tout bien compté, qu’en résulte-t-il d’inconnu ? et même que présentent-ils de sérieusement original ? Se passe-t-il là quelque chose qui au fond soit étranger aux conceptions européennes ? Existe-t-il là un motif déterminant auquel se puisse rattacher l’espérance de futurs triomphes pour une jeune humanité qui serait encore à naître ? Qu’on pèse mûrement le pour et le contre, et on ne doutera pas de l’inanité de semblables espérances. Les États-Unis d’Amérique ne sont pas le premier État commercial qu’il y ait eu dans le monde. Ceux qui l’ont précédé n’ont rien produit qui ressemblât à une régénération de la race dont ils étaient issus.\par
Carthage a jeté un éclat qui sera difficilement égalé par New-York. Carthage était riche, grande en toutes manières. La côte septentrionale de l’Afrique dans son entier développement, et une vaste partie de la région intérieure, étaient sous sa main. Elle avait été plus favorisée à sa naissance que la colonie des puritains d’Angleterre, car ceux qui l’avaient fondée étaient les rejetons des familles les plus pures du Chanaan. Tout ce que Tyr et Sidon perdirent, Carthage en hérita. Et cependant Carthage n’a pas ajouté la valeur d’un grain à la civilisation sémitique, ni empêché sa décadence d’un jour.\par
Constantinople fut à son tour une création qui semblait bien devoir effacer en splendeur le présent, le passé, et transformer l’avenir. Jouissant de la plus belle situa­ tion qui soit sur la terre, entourée des provinces les plus fertiles et les plus populeuses de l’empire de Constantin, elle paraissait affranchie, comme on le veut imaginer pour les États-Unis, de tous les empêchements que l’âge mûr d’un pays se plaint d’avoir reçus de son enfance. Peuplée de lettrés, gorgée de chefs-d’œuvre en tous genres, fami­liarisée avec tous les procédés de l’industrie, possédant des manufactures immenses et absorbant un commerce sans limites avec l’Europe, avec l’Asie, avec l’Afrique, quelle rivale eut jamais Constantinople ? Pour quel coin du monde le ciel et les hommes pourront-ils jamais faire ce qui fut fait pour cette majestueuse métropole ? Et de quel prix paya-t-elle tant de soins ? Elle ne fit rien, elle ne créa rien ; aucun des maux que les siècles avaient accumulés sur l’univers romain, elle ne le sut guérir ; pas une idée réparatrice ne sortit de sa population. Rien n’indique que les États-Unis d’Amérique, plus vulgairement peuplés que cette noble cité, et surtout que Carthage, doivent se montrer plus habiles.\par
Toute l’expérience du passé est réunie pour prouver que l’amalgame de principes ethniques déjà épuisés ne saurait fournir une combinaison rajeunie. C’est déjà beaucoup prévoir, beaucoup accorder, que de supposer dans la république du nouveau monde une assez longue cohésion pour que la conquête des pays qui l’entourent lui reste possible. À peine ce grand succès, qui leur donnerait un droit certain à se compa­rer à la Rome sémitique, est-il même probable ; mais il suffit qu’il le soit pour qu’il faille en tenir compte. Quant au renouvellement de la société humaine, quant à la création d’une civilisation supérieure ou au moins différente, ce qui, au jugement des masses intéressées, revient toujours au même, ce sont là des phénomènes qui ne sont produits que par la présence d’une race relativement pure et jeune. Cette condition n’existe pas en Amérique. Tout le travail de ce pays se borne à exagérer certains côtés de la culture européenne, et non pas toujours les plus beaux, à copier de son mieux le reste, à ignorer plus d’une chose \footnote{Une observation de Pickering donne un indice curieux de la grossièreté du génie des Anglo-Saxons d’Amérique en matière d’art. Il assure que la plupart des chants populaires, d’ailleurs si peu nombreux, que possèdent ses compatriotes ont été empruntés par ces derniers aux esclaves nègres, faute de pouvoir mieux. (Pickering, p. 185.) Il y a un grand rapport entre ce fait et l’imitation que firent jadis les Kymris des dessins en spirale inventés par les Finnois.}. Ce peuple qui se dit jeune, c’est le vieux peuple d’Europe, moins contenu par des lois plus complaisantes, non pas mieux inspiré. Dans le long et triste voyage qui jette les émigrants à leur nouvelle patrie, l’air de l’Océan ne les transforme pas. Tels ils étaient partis, tels ils arrivent. Le simple transfert d’un point à un autre ne régénère pas les races plus qu’à demi épuisées.
\chapterclose


\chapteropen
\chapter[{Conclusion générale}]{Conclusion générale}\renewcommand{\leftmark}{Conclusion générale}


\chaptercont
\noindent L’histoire humaine est semblable à une toile immense. La terre est le métier sur lequel elle est tendue. Les siècles assemblés en sont les infatigables artisans. Ils ne naissent que pour saisir aussitôt la navette et la faire courir sur la trame ; ils ne la posent que pour mourir. Ainsi, sous ces doigts affairés, va croissant d’ampleur le large tissu.\par
L’étoffe n’en revêt pas une seule couleur ; elle ne se compose pas d’une unique matière. Bien loin que l’inspiration de la sobre Pallas en ait décidé les dessins, l’aspect en rappelle plutôt la méthode des artistes du Kachemyr. Les bigarrures les plus étranges et les enroulements les plus bizarres s’y compliquent sans cesse des caprices les plus inattendus, et ce n’est qu’à force de diversité et de richesse que, contrairement à toutes les lois du goût, cet ouvrage, incomparable en grandeur, devient également incomparable en beauté.\par
Les deux variétés inférieures de notre espèce, la race noire, la race jaune, sont le fond grossier, le coton et la laine, que les familles secondaires de la race blanche assouplissent en y mêlant leur soie tandis que le groupe arian, faisant circuler ses filets plus minces à travers les générations ennoblies, applique à leur surface, en éblouissant chef-d’œuvre, ses arabesques d’argent et d’or.\par
 C’est ainsi que l’histoire est une, et que tant d’anomalies qu’elle présente peuvent trouver leur explication et rentrer dans des règles communes, si l’œil et la pensée, cessant de se concentrer avec une obstination irréfléchie sur des points isolés, consentent à embrasser l’ensemble, à y recueillir les faits semblables, à les rapprocher, à les comparer, et à tirer une conclusion rigoureuse des causes mieux étudiées et dès lors mieux comprises de leur identité fondamentale ; mais l’esprit de l’homme est de sa nature si débile qu’en s’approchant des sciences, son premier instinct est de les simplifier, ce qui d’ordinaire signifie les mutiler, les amoindrir, les débarrasser de tout ce qui gêne et déroute sa faiblesse, et, lorsqu’il a réussi à les défigurer pour des yeux qui seraient plus clairvoyants que les siens, c’est à ce moment seul qu’il les trouve belles, parce qu’elles sont devenues faciles ; cependant, dépouillées d’une partie de leurs trésors, elles n’en sauraient plus livrer que des restes trop souvent privés de vie. À peine s’en aperçoit-il. L’histoire n’est pas une science autrement constituée que les autres. Elle se présente composée de mille éléments en apparence hétérogènes, qui, sous des entrelacements multipliés, cachent ou déguisent une racine plongeant à de grandes profondeurs. En élaguer ce qui trouble la vue, c’est faire jaillir peut-être un peu plus de clarté sur les débris qu’on aura conservés ; mais c’est aussi altérer inévitablement la mesure et partant l’importance relative des parties, et rendre impossible de jamais pénétrer le sens réel du tout.\par
Pour obvier à ce mal qui frappe toute connaissance de stérilité, il faut se résoudre à renoncer à de pareils moyens, et à accepter la tâche avec ses difficultés natives. Si, bien résolu à le faire, on se borne d’abord à chercher sans rien omettre les principales sources du sujet, on découvrira d’une manière certaine qu’il en est trois d’où surgissent les phénomènes les plus dignes d’attirer l’attention. La première de ces sources, c’est l’activité de l’homme prise isolément ; la seconde, c’est l’établissement des centres politiques ; la troisième, la plus influente, celle qui vivifie les deux autres, c’est la manifestation d’un mode donné d’existence sociale. Que l’on ajoute maintenant à ces trois sources de mouvement et de transformation le fait de la pénétration mutuelle des sociétés, les contours généraux du travail seront tracés. L’histoire avec ses causes, avec ses mobiles, avec ses résultats principaux, sera renfermée dans un vaste cercle, et l’on pourra aborder les détails de la plus minutieuse analyse sans craindre de s’être préparé, par une dissection indiscrète, l’inévitable moisson d’erreurs qui résulte des autres façons de procéder.\par
L’activité de l’homme, prise isolément, s’exprime par les inventions de l’intelli­gence et le jeu des passions. L’observation de ce travail et des résultats dramatiques qu’il amène absorbe exclusivement l’attention du commun des penseurs. Ceux-là ne s’appliquent qu’à voir la créature s’agiter, céder ou résister à ses penchants, les diriger avec sagesse ou tomber engloutie dans leurs torrents fougueux. Rien d’émouvant, sans doute, comme les péripéties d’une pareille lutte entre l’homme et lui-même. Dans les deux éventualités posées devant ses pas, qui pourrait douter qu’il n’agisse en maître ? Le Dieu qui le contemple, et le jugera d’après le bien moral qu’il aura fait, le mal moral qu’il aura repoussé, nullement d’après la mesure de génie qu’il aura reçue, appesantit sur lui sa liberté, et le spectateur de ses hésitations, comparant les actes qu’il observe avec le code ouvert entre ses mains par la religion ou la philosophie, ne s’égare dans l’intérêt qu’il y prend que lorsqu’il leur suppose une étendue d’action que les efforts de l’homme isolé ne sauraient usurper.\par
Ces efforts n’opèrent jamais que dans une sphère étroitement limitée. Qu’on imagine le plus puissant des hommes, le plus éclairé, le plus énergique : la longueur de son bras reste toujours peu de chose. Faites sortir les plus hautes pensées imaginables du cerveau de César ; elles ne sauraient embrasser dans leur vol toute la circonférence du globe. Leurs œuvres, bornées à certains lieux, n’atteignent tout au plus qu’un nombre restreint d’objets ; elles ne sauraient affecter, pendant un temps donné, que l’organisme d’un ou tout au plus de quelques centres politiques. Aux yeux des contemporains, c’est beaucoup ; mais pour l’histoire il n’en résulte le plus souvent que d’imperceptibles effets. Imperceptibles, dis-je ; car, du vivant même de leurs auteurs, on en voit la majeure partie s’effacer, et la génération suivante en cherche vainement les traces. Considérons les plus vastes sphères qui furent jamais abandonnées à la volonté d’un prince illustre, soit les conquêtes immenses du Macédonien, soit les États superbes de ce monarque espagnol où le soleil ne se couchait jamais. Qu’a fait la volonté d’Alexandre ? que créa celle de Charles Quint ? Sans énumérer les causes indépendantes de leur génie qui réunirent tant de sceptres aux mains de ces grands hommes, et permirent au moins favorisé des deux d’en ramasser plus qu’il n’en arracha, l’essentiel de leur rôle a consisté en définitive à n’être que les conducteurs dociles ou les contradicteurs abandonnés de ces multitudes que l’on suppose soumises à leur empire. Entraînés dans une impulsion qu’ils ne donnaient pas, leur plus beau succès fut de l’avoir suivie ; et, lorsque le dernier des deux, armé de toutes ses gloires, prétendit à son tour guider le torrent, le torrent qui l’emportait se gonfla contre ses défenses, grandit contre ses menaces, effondra toutes ses digues, et, poursuivant son cours, le renversa dans sa honte, et trop bien convaincu de sa faiblesse, sur l’obscur parvis de Saint-Just.\par
Ce ne sont pas les grands hommes qui se croient omnipotents ; il leur est trop facile de mesurer ce qu’ils font sur ce qu’ils voudraient faire. Ils savent bien, ceux dont la taille dépasse le niveau commun, que l’action permise à leur autorité n’a jamais atteint dans sa plus vaste expansion l’étendue d’un continent ; que, dans leur palais même, on ne vit pas comme ils le souhaitent ; que, si leur intervention retarde ou précipite le pas des événements, c’est de la même façon qu’un enfant contrarie le ruisseau qu’il ne saurait empêcher de couler. La meilleure partie de leurs récits est faite non d’invention, mais de compréhension. Là s’arrête la puissance historique de l’homme agissant dans les plus favorables conditions de développement. Elle ne constitue pas une cause, ce n’est pas non plus un terme, c’est quelquefois un moyen transitoire ; le plus souvent on ne saurait la considérer que comme un enjolivement. Mais, telle qu’elle est, il lui faut reconnaître pourtant le suprême mérite d’appeler sur la marche de l’humanité cette sympathie générale que le tableau d’évolutions purement impersonnelles n’aurait jamais éveillée. Les différentes écoles lui ont attribué une influence omnipotente, en méconnaissant grossièrement son incapacité réelle. Elle fut cependant jusqu’ici l’unique mobile de cet attrait irraisonné qui a porté les hommes à recueillir les reliques du passé.\par
 On vient d’entrevoir que la limite immédiate devant laquelle elle s’arrête est fournie par la résistance du centre politique au sein duquel elle se meut. Un centre politique, réunion collective de volontés humaines, aurait donc par lui-même une volonté ; incontestablement il en est ainsi. Un centre politique, autrement dit un peuple, a ses passions et son intelligence. Malgré la multiplicité des têtes qui le forment, il possède une individualité mixte, résultant de la mise en commun de toutes les notions, de toutes les tendances, de toutes les idées, que la masse lui suggère. Tantôt il en est la moyenne, tantôt l’exagération ; tantôt il parle comme la minorité, tantôt la majorité l’entraîne, ou bien encore c’est une inspiration morbide qui n’était attendue et n’est avouée de personne. Bref, un peuple pris collectivement est, dans de nombreuses fonctions, un être aussi réel que si on le voyait condensé en un seul corps. L’autorité dont il dispose est plus intense, plus soutenue, et en même temps moins sûre et moins durable, parce qu’elle est plutôt instinctive que volontaire, qu’elle est plutôt négative qu’affirmative, et que, dans tous les cas, elle est moins directe que celle des individualités isolées. Un peuple est exposé à changer de visées dix fois et plus dans l’intervalle d’un siècle, et c’est là ce qui explique les fausses décadences et les fausses régénérations. Dans un intervalle de peu d’années, il se montre propre à conquérir ses voisins, puis à être conquis par eux ; aimant ses lois et leur étant soumis, puis ne respirant que révolte pour aspirer quelques heures plus tard à la servitude. Mais, dans le malaise, l’ennui ou le malheur, on l’entend sans cesse accuser ses gouvernants de ce qu’il souffre ; preuve évidente qu’il a le sentiment d’une faiblesse organique qui réside en lui, et qui provient de l’imperfection de sa personnalité.\par
Un peuple a toujours besoin d’un homme qui comprenne sa volonté, la résume, l’explique, et le mène où il doit aller. Si l’homme se trompe, le peuple résiste, et se lève ensuite pour suivre celui qui ne se trompe pas. C’est la marque évidente de la nécessité d’un échange constant entre la volonté collective et la volonté individuelle. Pour qu’il y ait un résultat positif, il faut que ces deux volontés s’unissent ; séparées, elles sont infécondes. De là vient que la monarchie est la seule forme de gouvernement rationnelle.\par
Mais on s’aperçoit sans peine que le prince et la nation réunis ne font jamais que mettre en valeur des aptitudes ou des capacités, ne font jamais que conjurer des influences néfastes provenant d’un domaine extérieur à l’un comme à l’autre. Dans bien des cas où un chef voit la route que son monde voudrait prendre, ce n’est pas sa faute si ce monde manque des forces nécessaires pour accomplir la tâche indispensable ; et de même encore un peuple, une multitude ne peut se donner les compréhensions qu’elle n’a pas et qu’elle devrait avoir, pour éviter des catastrophes vers lesquelles elle court tout en les concevant, tout en les redoutant, tout en en gémissant.\par
Cependant voilà que le plus terrible malheur est tombé sur une nation. L’impré­voyance, ou la folie, ou l’impuissance de ses guides, conjurés avec ses propres torts, font éclater sa ruine. Elle tombe sous le sabre d’un plus fort, elle est envahie, annexée à d’autres États. Ses frontières s’effacent, et ses étendards déchirés vont triomphalement agrandir de leurs lambeaux les étendards du vainqueur. Sa destinée finit-elle là ?\par
Suivant les annalistes, l’affirmation n’est pas douteuse. Tout peuple subjugué ne compte plus, et, s’il s’agit d’époques reculées et quelque peu ténébreuses, la plume de l’écrivain n’hésite pas même à le rayer du nombre des vivants, et à le déclarer matériellement disparu.\par
Mais qu’avec un juste dédain pour une conclusion aussi superficielle, on se mette en quête de la réalité, on trouvera qu’une nation, politiquement abolie, continue à subsister sans autre modification que de porter un nom nouveau ; qu’elle conserve ses allures propres, son esprit, ses facultés, et qu’elle influe, d’une manière conforme à sa nature ancienne, sur les populations auxquelles elle est réunie. Ce n’est donc pas la forme politiquement agrégative qui donne la vie intellectuelle à des multitudes, qui leur fait une volonté, qui leur inspire une manière d’être. Elles ont tout cela sans posséder de frontières propres. Ces dons résultent d’une impulsion suprême qu’elles reçoivent d’un domaine plus haut qu’elles-mêmes. Ici s’ouvrent ces régions inexplorées où l’horizon élargi dans une mesure incomparable ne livre plus seulement aux regards le territoire borné de tel royaume ou de telles républiques, ni les fluctuations étroites des populations qui les habitent, mais étale toutes les perspectives de la société qui les contient, avec les grands rouages et les puissants mobiles de la civilisation qui les anime.\par
La naissance, les développements, l’éclipse d’une société et de sa civilisation constituent des phénomènes qui transportent l’observateur bien au-dessus des horizons que les historiens lui font ordinairement apercevoir. Ils ne portent, dans leurs causes initiales, aucune empreinte des passions humaines ni des déterminations populaires, matériaux trop fragiles pour prendre place dans une œuvre d’aussi longue durée. Seuls, les différents modes d’intelligence départis aux différentes races et à leurs combinai­sons s’y font reconnaître. Encore ne les aperçoit-on que dans leurs parties les plus essentielles, les plus dégagées de l’autorité du libre arbitre, les plus natives, les plus raréfiées, en un mot, les plus fatales, celles que l’homme ou la nation ne peuvent ni se donner ni se retirer, et dont ils ne sauraient s’interdire ou se commander l’usage. Ainsi se déploient, au-dessus de toute action transitoire et volontaire émanant soit de l’individu, soit de la multitude, des principes générateurs qui produisent leurs effets avec une indépendance et une impassibilité que rien ne peut troubler. De la sphère libre, absolument libre, où ils se combinent et opèrent, le caprice de l’homme ou d’une nation ne saurait faire tomber aucun résultat fortuit. C’est, dans l’ordre des choses immatérielles, un milieu souverain où se meuvent des forces actives, des principes vivifiants en communication perpétuelle avec l’individu comme avec la masse, dont les intelligences respectives, contenant quelques parcelles identiques à la nature de ces forces, sont ainsi préparées et éternellement disposées à en recevoir l’impulsion.\par
Ces forces actives, ces principes vivifiants, ou, si l’on veut les concevoir sous une idée concrète, cette âme, demeurée jusqu’à présent inaperçue et anonyme, doit être mise au rang des agents cosmiques du premier degré. Elle remplit, au sein du monde intangible, des emplois analogues à ceux que l’électricité et le magnétisme exercent sur d’autres points de la création, et, comme ces deux influences, elle se laisse constater par ses fonctions, ou plus exactement, par quelques-unes de ses fonctions, mais non pas saisir, décrire et apprécier, en elle-même, dans sa nature propre et abstraite, dans sa totalité.\par
Rien ne prouve que ce soit une émanation de l’homme et des corps politiques. Elle vit par eux en apparence, elle vit pour eux certainement. La mesure de vigueur et de santé des civilisations est aussi la mesure de sa vigueur et de sa santé ; mais, si l’on observe que c’est dans le temps même où les civilisations s’éclipsent qu’elle atteint souvent son plus haut degré de dilatation et de force chez certains individus et chez certaines nations, on sera porté à en conclure qu’elle peut être comparée à une atmosphère respirable qui, dans le plan de la création, n’a de raison d’être que tant que la société qu’elle enveloppe et anime doit vivre ; qu’elle lui est, au fond, étrangère aussi bien qu’extérieure, et que c’est sa raréfaction qui amène la mort de cette société malgré la provision d’air que celle-ci pouvait avoir encore, et dont la source est cependant tarie.\par
Les manifestations appréciables de cette grande âme partent de la double base que j’ai appelée ailleurs masculine et féminine. On se souvient, d’ailleurs, que je n’ai eu en vue, dans le choix de ces dénominations, qu’une attitude subjective, d’une part, et, de l’autre, une faculté objective, sans corrélation à aucune idée de suprématie d’un de ces foyers sur l’autre. Elle se répand de là, en deux courants de qualités diverses, jusque dans les plus minimes fractions, jusque dans les dernières molécules de l’agglomé­ration sociale que son incessante circulation dirige, et ce sont les deux pôles vers lesquels ils gravitent et dont ils s’éloignent tour à tour.\par
L’existence d’une société étant, en premier ressort, un effet qu’il ne dépend pas de l’homme de produire ni d’empêcher, n’entraîne pour lui aucun résultat dont il soit responsable. Elle ne comporte donc pas de moralité. Une société n’est, en elle-même, ni vertueuse ni vicieuse ; elle n’est ni sage ni folle ; elle est. Ce n’est pas de l’action d’un homme, ce n’est pas de la détermination d’un peuple que se dégage l’événement qui la fonde. Le milieu à travers lequel elle passe pour arriver à l’existence positive doit être riche des éléments ethniques nécessaires, absolument comme certains corps, pour employer encore une comparaison qui se représente sans cesse à l’esprit, absorbent facilement et abondamment l’agent électrique, et sont bons pour le disperser, tandis que d’autres ont peine à s’en laisser pénétrer, et plus de peine encore à le faire rayonner autour d’eux. Ce n’est pas la volonté d’un monarque ou de ses sujets qui modifie l’essence d’une société ; c’est, en vertu des mêmes lois, un mélange ethnique subséquent. Une société enfin enveloppe ses nations comme le ciel enveloppe la terre, et ce ciel, que les exhalaisons des marais ou les jets de flammes du volcan n’atteignent pas, est encore, dans sa sérénité, l’image parfaite des sociétés que leur contenu ne saurait affecter de ses tressaillements, tandis qu’irrésistiblement, bien que d’une façon insensible, elles l’assouplissent à toutes leurs influences.\par
 Elles imposent aux populations leurs modes d’existence. Elles les circonscrivent entre les limites dont ces esclaves aveugles n’éprouvent pas même la velléité de sortir, et n’en auraient pas la puissance. Elles leur dictent les éléments de leurs lois, elles inspirent leurs volontés, elles désignent leurs amours, elles attisent leurs haines, elles conduisent leur mépris. Toujours soumises à l’action ethnique, elles produisent les gloires locales par ce moyen immédiat ; par la même voie elles implantent le germe des malheurs nationaux, puis, à jour dit, elles entraînent vainqueurs et vaincus sur une même pente, qu’une nouvelle action ethnique peut seule les empêcher elles-mêmes de descendre indéfiniment.\par
Si elles tiennent avec tant d’énergie les membres des peuples, elles ne régissent pas moins les individus. En leur laissant, et sans nulle réserve, ce point est de toute importance, les mérites d’une moralité dont néanmoins elles règlent les formes, elles manient, elles pétrissent en quelque sorte leurs cerveaux au moment de la naissance, et, leur indiquant certaines voies, leur ferment les autres dont elles ne leur permettent pas même d’apercevoir les issues.\par
Ainsi donc, avant d’écrire l’histoire d’un pays distinct et de prétendre expliquer les problèmes dont une pareille tâche est semée, il est indispensable de sonder, de scruter, de bien connaître les sources et la nature de la société dont ce pays n’est qu’une fraction. Il faut étudier les éléments dont elle se compose, les modifications qu’elle a subies, les causes de ces modifications, l’état ethnique obtenu par la série des mélanges admis dans son sein.\par
On s’établira ainsi sur un sol positif contenant les racines du sujet. On les verra d’elles-mêmes pousser, fructifier et porter graine. Comme les combinaisons ethniques ne sont jamais répandues à doses égales sur tous les points géographiques compris dans le territoire d’une société, il conviendra de particulariser davantage ses recherches et d’en contrôler plus sévèrement les découvertes à mesure que l’on se rapprochera de son objet. Tous les efforts de l’esprit, tous les secours de la mémoire, toute la perspicacité méfiante du jugement sont ici nécessaires. Peines sur peines, rien n’est de trop. Il s’agit de faire entrer l’histoire dans la famille des sciences naturelles, de lui donner, en ne l’appuyant que sur des faits empruntés à tous les ordres de notions capables d’en fournir, toute la précision de cette classe de connaissances, enfin de la soustraire à la juridiction intéressée dont les fractions politiques lui imposent jusqu’à aujourd’hui l’arbitraire.\par
Faire quitter à la muse du passé les sentiers douteux et obliques pour conduire son char dans une voie large et droite, explorée à l’avance et jalonnée de stations connues, ce n’est rien enlever à la majesté de son attitude, et c’est beaucoup ajouter à l’autorité de ses conseils. Certes elle ne viendra plus, par des gémissements enfantins, accuser Darius d’avoir causé la perte de l’Asie, ni Persée l’humiliation de la Grèce ; mais on ne la verra pas davantage saluer follement, dans d’autres catastrophes, les effets du génie des Gracques, ni l’omnipotence oratoire des Girondins. Désaccoutumée de ces misères, elle proclamera que les causes irréconciliables de pareils événements, planant bien haut au-dessus de la participation des hommes, n’intéressent point la polémique des partis. Elle dira quel concours de motifs invincibles les fait naître, sans que personne à leur sujet ait de blâme à recevoir ou d’éloge à demander. Elle distinguera ce que la science ne peut que constater de ce que la justice doit saisir.\par
De son trône superbe tomberont dès lors des jugements sans appel et des leçons salutaires pour les bonnes consciences. Soit qu’on aime, soit qu’on réprouve telle évolution d’une nationalité, ses arrêts, en réduisant la part que l’homme y peut prendre à déplacer quelques dates, à irriter ou à adoucir d’inévitables blessures, rendront le libre arbitre de chacun sévèrement responsable de la valeur de tous les actes. Pour le méchant plus de ces vaines excuses, de ces nécessités factices dont on prétend aujourd’hui ennoblir des crimes trop réels. Plus de pardon pour les atrocités ; de soi-disant services ne les innocenteront pas. L’histoire arrachera tous les masques fournis par les théories sophistiques ; elle s’armera, pour flétrir les coupables, des anathèmes de la religion. Le rebelle ne sera plus devant son tribunal qu’un ambitieux impatient et nuisible : Timoléon, qu’un assassin ; Robespierre, un immonde scélérat.\par
Pour donner aux annales de l’humanité ce souffle, ces allures et cette portée inaccoutumée, il est temps de changer la façon dont on les compose, en entrant coura­geusement dans les mines de vérités que tant d’efforts laborieux viennent d’ouvrir. Des méfiances mal raisonnées n’excuseraient pas l’hésitation.\par
Les premiers calculateurs qui entrevirent l’algèbre, effrayés des profondeurs dont elle révélait les ouvertures, lui prêtèrent des vertus surnaturelles et de la plus rigou­reuse des sciences firent l’enveloppe des plus folles imaginations. Cette vision rendit quelque temps les mathématiques suspectes aux esprits sensés ; puis l’étude sérieuse perça l’écorce et prit le fruit.\par
Les premiers physiciens qui remarquèrent les ossements fossiles et les débris marins échoués sur les cimes des montagnes, ne manquèrent pas de s’abandonner aux divagations les plus répugnantes. Leurs successeurs, repoussant les rêves, ont fait de la géologie la genèse de l’exposition des trois règnes. Il n’est plus permis de discuter ce qu’elle affirme. Il en est de l’ethnologie comme de l’algèbre et de la science des Cuvier et des Beaumont. Asservie par les uns à la complicité des plus sottes fantaisies philanthropiques, elle est repoussée par les autres, qui confondent dans l’injustice d’un même mépris et le charlatan, et sa drogue, et l’aromate précieux dont il abuse.\par
Sans doute, l’ethnologie est jeune. Elle a toutefois passé l’âge des premiers bégayements. Elle est assez avancée pour disposer d’un nombre suffisant de démons­trations solides sur lesquelles on peut bâtir en toute sécurité. Chaque jour lui apporte de plus riches contributions. Entre les diverses branches de connaissances qui rivalisent à l’en pourvoir, l’émulation est si productive, qu’à peine lui est-il possible de recueillir et de classer les découvertes avec la même rapidité qu’elles s’accumulent. Plût au ciel que ses progrès ne fussent plus embarrassés que par ce genre d’obstacles ! Mais elle en rencontre de pires. On se refuse encore à apprécier avec netteté sa véritable nature, et par conséquent on ne la traite pas régulièrement d’après les seules méthodes qui lui conviennent.\par
C’est la frapper de stérilité que de l’appuyer avec prédilection sur une science isolée, et principalement sur la physiologie. Ce domaine lui est ouvert, sans nul doute ; mais, pour que les matériaux qu’elle lui emprunte acquièrent le degré d’authenticité nécessaire et revêtent son caractère spécial, il est presque toujours indispensable qu’elle leur fasse subir le contrôle de témoignages venus d’ailleurs, et que l’étude comparée des langues, l’archéologie, la numismatique, la tradition ou l’histoire écrite, aient garanti leur valeur, soit directement, soit par induction, \emph{a priori} ou \emph{a posteriori}. En second lieu, un fait ne saurait passer d’une science dans une autre sans se présenter sous un jour nouveau dont il convient encore de constater la nature avant d’être en droit de s’en prévaloir ; donc l’ethnologie ne peut considérer comme incontestablement entrés dans son domaine que les documents physiologiques ou autres qui ont subi cette dernière épreuve dont elle seule possède la direction et les critériums. Comme elle n’a pas que la matière pour objet, et qu’elle embrasse en même temps les manifestations de l’espèce la plus intellectuelle, il n’est pas permis de la confiner une seule minute dans une sphère étrangère et surtout dans la sphère physique, sans l’égarer au milieu de lacunes que les plus audacieuses et les plus vaines hypothèses ne parviendront jamais à combler. En réalité, elle n’est autre que la racine et la vie même de l’histoire. C’est artificiellement, arbitrairement, et au grand détriment de celle-ci que l’on parvient à l’en séparer. Maintenons-la donc à la fois sur tous les terrains où l’histoire a le droit de frapper sa dîme.\par
Ne la détournons pas trop non plus des travaux positifs, en lui posant des questions dont il n’est pas bien certain que l’esprit de l’homme ait le pouvoir de percer les ténèbres. Le problème d’unité ou de multiplicité des types primitifs est de ce nombre. Cette recherche a donné jusqu’à présent peu de satisfaction à ceux qui s’y sont absorbés. Elle est tellement dépourvue d’éléments de solution, qu’elle semble plutôt destinée à amuser l’esprit qu’à éclairer le jugement, et à peine doit-elle être considérée comme scientifique. Plutôt que de se perdre avec elle dans des rêveries sans issue, mieux vaut jusqu’à nouvel ordre, la tenir à l’écart de tous les travaux sérieux, ou du moins ne lui accorder là qu’une place très subalterne. Ce qu’il importe seulement de constater, c’est jusqu’à quel point les variétés sont organiques et la mesure de la ligne qui les sépare. Si des causes quelconques peuvent ramener les différents types à se confondre, si, par exemple, en changeant de nourriture et de climat, un blanc peut devenir un nègre, et un nègre un mongol, l’espèce entière, serait-elle issue de plusieurs millions de pères complètement dissemblables, doit être déclarée sans hésitation unitaire, elle en a le trait principal et vraiment pratique.\par
Mais si, au contraire, les variétés sont renfermées dans leur constitution actuelle, de telle sorte qu’elles soient inhabiles à perdre leurs caractères distinctifs autrement que par des hymens contractés hors de leurs sphères, et si aucune influence externe ou interne n’est apte à les transformer dans leurs parties essentielles ; si enfin elles possèdent d’une manière permanente, et ce point n’est plus douteux, leurs particularités physiques et morales, coupons court aux divagations frivoles, et proclamons le résultat, la conséquence rigoureuse et seule utile : fussent-elles nées d’un seul couple, les variétés humaines, éternellement distinctes, vivent sous la loi de la multiplicité des types, et leur unité primordiale ne saurait exercer et n’exerce pas sur leurs destinées la plus impondérable conséquence. C’est ainsi que, pour satisfaire dignement aux impérieux besoins d’une science parvenue à sa virilité, il faut savoir se borner et diriger ses recherches vers les buts abordables en répudiant le reste. Et maintenant, nous plaçant au centre du vrai domaine de la véritable histoire, de l’histoire sérieuse et non point fantastique, de l’histoire tissue de faits, et non pas d’illusions ou d’opinions, examinons, pour la dernière fois, par grandes masses, non point ce que nous croyons pouvoir être, mais ce que, de science certaine, nos yeux voient, nos oreilles entendent, nos mains touchent.\par
À une époque toute primordiale de la vie de l’espèce entière, époque qui précède les récits des plus lointaines annales, on découvre, en se plaçant en imagination sur les plateaux de l’Altaï, trois amas de peuples immenses, mouvants, composés chacun de différentes nuances, formés, dans les régions qui s’étendent à l’ouest autour de la montagne, par la race blanche ; au nord-est, par les hordes jaunes arrivant des terres américaines ; au sud, par les tribus noires ayant leur foyer principal dans les lointaines régions de l’Afrique. La variété blanche, peut-être moins nombreuse que ses deux sœurs, d’ailleurs douée d’une activité combattante qu’elle tourne contre elle-même et qui l’affaiblit, étincelle de supériorités de tout genre.\par
Poussée par les efforts désespérés et accumulés des nains, cette race noble s’ébranle, déborde ses territoires du côté du midi, et ses tribus d’avant-garde tombent au milieu des multitudes mélaniennes, y éclatent en débris, et commencent à se mêler aux éléments circulant autour d’elles. Ces éléments sont grossiers, antipathiques, fugaces ; mais la ductilité de l’élément qui les aborde parvient à les saisir. Elle leur communique, partout où elle les atteint, quelque chose de ses qualités, ou du moins les dépouille d’une partie de leurs défauts ; surtout elle leur donne la puissance nouvelle de se coaguler, et bientôt au lieu d’une série de familles, de tribus incultes et ennemies qui se disputaient le sol sans en tirer nul avantage, une race mixte se répand depuis les contrées bactriennes sur la Gédrosie, les golfes de Perse et d’Arabie, bien au delà des lacs nubiens, pénètre jusqu’à des latitudes inconnues vers les contrées centrales du continent d’Afrique, longe la côte septentrionale par delà les Syrtes, dépasse Calpé, et, sur toute cette étendue, la variété mélanienne diversement atteinte, ici complètement absorbée, là absorbant à son tour, mais surtout modifiant à l’infini l’essence blanche et étant modifiée par elle, perd sa pureté et quelques traits de ses caractères primitifs. De là certaines aptitudes sociales qui se manifestent aujourd’hui dans les parties les plus reculées du monde africain : ce ne sont que les résultats lointains d’une antique alliance avec la race blanche. Ces aptitudes sont faibles, incohérentes, indécises, comme le lien lui-même est devenu, pour ainsi dire, imperceptible.\par
Pendant ces premières invasions, pendant que ces premières générations de mulâtres se développaient du côté de l’Afrique, un travail analogue s’opérait à travers la presqu’île hindoue, et se compliquait au delà du Gange, et plus encore, du Brahmapoutra, en passant des peuplades noires aux hordes jaunes, déjà parvenues, plus ou moins pures, jusque dans ces régions. En effet les Finnois s’étaient multipliés sur les plages de la mer de Chine avant même d’avoir pu déterminer aucun déplacement sérieux des nations blanches dans l’intérieur du continent. Ils avaient trouvé plus de facilités à étreindre, à pénétrer l’autre race inférieure. Ils s’étaient mêlés à elle comme ils avaient pu. La variété malaise avait alors commencé à sortir de cette union, qui ne s’opérait ni sans efforts ni sans violences. Les premiers produits métis remplirent d’abord les provinces centrales du Céleste Empire. À la longue, ils se formèrent de proche en proche dans toute l’Asie orientale, dans les îles du Japon, dans les archipels de la mer des Indes ; ils touchèrent l’est de l’Afrique, ils enveloppèrent toutes les îles de la Polynésie, et, placés de la sorte en face des terres américaines, dans le nord comme dans le sud, aux Kouriles comme à l’île de Pâques, ils rentrèrent fortuitement, par petites bandes peu nombreuses, et en abordant aux points les plus divers, dans ces régions quasi désertes où n’habitaient plus que des descendants clairsemés de quelques traînards détachés de l’arrière-garde des multitudes jaunes, auxquelles, race mixte qu’ils étaient, ces Malais devaient en partie leur naissance, leur aspect physique et leurs aptitudes morales.\par
Du côté de l’ouest, et en tirant indéfiniment vers l’Europe, pas de peuples méla­niens, mais le contact le plus forcé, le plus inévitable entre les Finnois et les blancs. Tandis qu’au sud, ces derniers, fugitifs heureux, forçaient tout à plier sous leur empire et s’alliaient en maîtres aux populations indigènes, dans le nord, au contraire, ils commencèrent l’hymen en opprimés. Il est douteux que les nègres, maîtres de choisir, eussent beaucoup envié leur alliance physique ; il ne l’est pas que les jaunes l’aient ardemment souhaitée. Soumis à l’influence directe de l’invasion finnique, les Celtes, et surtout les Slaves, qu’on en distingue avec peine, furent assaillis, tourmentés, puis forcés de transporter leur séjour en Europe, par des déplacements graduels. Ainsi, bon gré mal gré, ils commencèrent de bonne heure à s’allier aux petits hommes venus d’Amérique ; et, lorsque leurs pérégrinations ultérieures leur eurent fait rencontrer dans les différents pays occidentaux de nouveaux établissements de mêmes créatures, ils eurent d’autant moins de raisons de répugner à leur alliance.\par
Si l’espèce blanche tout entière avait été expulsée de ses domaines primitifs dans l’Asie centrale, le gros des peuples jaunes n’aurait eu rien à faire qu’à se substituer, à elle dans les domaines abandonnés. Le Finnois aurait dressé son wigwam de branchages sur les ruines des monuments anciens, et, agissant suivant son naturel, il s’y serait assis, engourdi, endormi, et le monde n’aurait plus entendu parler de ses masses inertes. Mais l’espèce blanche n’avait pas déserté en masse la patrie originelle. Brisée sous le choc épouvantable des masses finnoises, elle avait emmené, à la vérité, dans différentes directions, le gros de ses peuples ; mais d’assez nombreuses de ses nations étaient cependant restées qui, en s’incorporant avec le temps à plusieurs, à la plupart des tribus jaunes, leur communiquèrent une activité, une intelligence, une force physique, un degré d’aptitude sociale tout à fait étrangers à leur essence native, et par là les rendirent propres à continuer indéfiniment de verser sur les régions environ­nantes, même en dépit de résistances assez fortes, l’abondance de leurs éléments ethniques.\par
Au milieu de ces transformations générales qui atteignent l’ensemble des races pures, et comme résultat nécessaire de ces alliages, la culture antique de la famille blanche disparaît, et quatre civilisations mixtes la remplacent l’assyrienne, l’hindoue, l’égyptienne, la chinoise ; une cinquième prépare son avènement peu lointain, la grecque, et l’on est déjà en droit d’affirmer que tous les principes qui posséderont à l’avenir les multitudes sociales sont trouvés, car les sociétés subséquentes, ne leur ajoutant rien, n’en ont jamais présenté que des combinaisons nouvelles.\par
L’action la plus évidente de ces civilisations, leur résultat le plus remarquable, le plus positif, n’est autre que d’avoir continué sans se ralentir jamais l’œuvre de l’amal­game ethnique. À mesure qu’elles s’étendent, elles englobent nations, tribus, familles jusque-là isolées, et, sans pouvoir jamais les approprier toutes aux formes, aux idées dont elles vivent elles-mêmes, elles réussissent cependant à leur faire perdre le cachet d’une individualité propre.\par
Dans ce qu’on pourrait appeler un second âge, dans la période des mélanges, les Assyriens montent jusqu’aux limites de la Thrace, peuplent les îles de l’Archipel, s’établissent dans la basse Égypte, se fortifient en Arabie, s’insinuent chez les Nubiens. Les gens d’Égypte s’étendent dans l’Afrique centrale, poussent leurs établissements dans le sud et l’ouest, se ramifient dans l’Hedjaz, dans la presqu’île du Sinaï. Les Hindous disputent le terrain aux Hymyarites Arabes, débarquent à Ceylan, colonisent Java, Bali, continuent à se mêler aux Malais d’outre-Gange. Les Chinois se marient aux peuples de la Corée, du Japon ; ils touchent aux Philippines, tandis que les métis noirs et jaunes, formés sur toute la Polynésie et faiblement impressionnés par les civilisations qu’ils aperçoivent, font circuler depuis Madagascar jusqu’en Amérique le peu qu’ils en peuvent comprendre.\par
Quant aux populations reléguées dans le monde occidental, quant aux blancs d’Europe, les Ibères, les Rasènes, les Illyriens, les Celtes, les Slaves, ils sont déjà affectés par des alliages finniques. Ils continuent à s’assimiler les tribus jaunes répan­dues autour de leurs établissements ; puis, entre eux, ils se marient encore, et encore aux Hellènes, métis sémitisés, accourus de toutes parts sur leurs côtes.\par
Ainsi mélange, mélange partout, toujours mélange, voilà l’œuvre la plus claire, la plus assurée, la plus durable des grandes sociétés et des puissantes civilisations, celle qui, à coup sûr, leur survit ; et plus les premières ont d’étendue territoriale et les secondes de génie conquérant, plus loin les flots ethniques qu’elles soulèvent vont saisir d’autres flots primitivement étrangers, ce dont leur nature et la sienne s’altèrent également.\par
Mais, pour que ce grand mouvement de fusion générale embrasse jusqu’aux dernières races du globe et n’en laisse pas une seule intacte, ce n’est pas assez qu’un milieu civilisateur déploie toute l’énergie dont il est pourvu ; il faut encore que dans les différentes régions du monde ces ateliers ethniques s’établissent de manière à agir sur place, sans quoi l’œuvre générale resterait nécessairement incomplète. La force négative des distances paralyserait l’expansion des groupes les plus actifs. La Chine et l’Europe n’exercent l’une sur l’autre qu’une faible action, bien que le monde slave leur serve d’intermédiaire. L’Inde n’a jamais influé fortement sur l’Afrique, ni l’Assyrie sur le Nord asiatique ; et, dans le cas où les sociétés auraient à jamais conservé les mêmes foyers, jamais l’Europe n’aurait pu être directement et suffisamment saisie, ni tout à fait entraînée dans le tourbillon. Elle l’a été parce que les éléments de création d’une civilisation propre à servir l’action générale avaient été répandus d’avance sur son sol. Avec les races celtiques et slaves, elle posséda en effet, dès les premiers âges, deux courants amalgamateurs qui lui permirent d’entrer, au moment nécessaire, dans le grand ensemble.\par
Sous leur influence, elle avait vu disparaître dans une immersion complète l’essence jaune et la pureté blanche. Avec l’intermédiaire fortement sémitisé des Hellènes, puis avec les colonisations romaines, elle acquit de proche en proche les moyens d’associer ses masses au compartiment asiatique le plus voisin de ses rivages. Celui-ci, à son tour, reçut le contrecoup de cette évolution ; car, tandis que les groupes d’Europe se teignaient d’une nuance orientale en Espagne, dans la France méridionale, en Italie, en Illyrie, ceux d’Orient et d’Afrique prenaient quelque chose de l’Occident romain sur la Propontide, dans l’Anatolie, en Arabie, en Égypte. Ce rapprochement effectué, l’effort des Slaves et des Celtes, combiné avec l’action hellénique, avait produit tous ses effets ; il ne pouvait aller au delà ; il n’avait nul moyen de dépasser de nouvelles limites géographiques ; la civilisation de Rome, la sixième dans l’ordre du temps, qui avait pour raison d’être la réunion des principes ethniques du monde occidental, n’eut pas la force de rien opérer seule après le III\textsuperscript{e} siècle de notre ère.\par
Pour agrandir désormais l’enceinte où tant de multitudes se combinaient déjà, il fallait l’intervention d’un agent ethnique d’une puissance considérable, d’un agent qui résultât d’un hymen nouveau de la meilleure variété humaine avec les races déjà civilisées. En un mot, il fallait une infusion d’Arians dans le centre social le mieux placé pour opérer sur le reste du monde, sans quoi les existences sporadiques de tous degrés, répandues encore sur la terre, allaient continuer indéfiniment sans plus rencontrer des eaux d’amalgamation.\par
Les Germains apparurent au milieu de la société romaine. En même temps, ils occupèrent l’extrême nord-ouest de l’Europe, qui peu à peu devint le pivot de leurs opérations. Des mariages successifs avec les Celtes et les Slaves, avec les populations gallo-romaines, multiplièrent la force d’expansion des nouveaux arrivants, sans dégrader trop rapidement leur instinct naturel d’initiative. La société moderne naquit ; elle s’attacha, sans désemparer, à perfectionner de toutes parts, à pousser en avant l’œuvre agrégative de ses devancières. Nous l’avons vue, presque de nos jours, décou­vrir l’Amérique, s’y unir aux races indigènes ou les pousser vers le néant ; nous la voyons faire refluer les Slaves chez les dernières tribus de l’Asie centrale, par l’impulsion qu’elle donne à la Russie ; nous la voyons s’abattre au milieu des Hindous, des Chinois ; frapper aux portes du Japon ; s’allier, sur tout le pourtour des côtes africaines, aux naturels de ce grand continent ; bref, augmenter sur ses propres terres et étendre sur tout le globe, dans une indescriptible proportion, les principes de confusion ethnique dont elle dirige maintenant l’application.\par
 La race germanique était pourvue de toute l’énergie de la variété ariane. Il le fallait pour qu’elle pût remplir le rôle auquel elle était appelée. Après elle, l’espèce blanche n’avait plus rien à donner de puissant et d’actif : tout était dans son sein à peu près également souillé, épuisé, perdu. Il était indispensable que les derniers ouvriers envoyés sur le terrain ne laissassent rien de trop difficile à terminer ; car personne n’existait plus, en dehors d’eux, qui fût capable de s’en charger. Ils se le tinrent pour dit. Ils achevèrent la découverte du globe ; ils s’en emparèrent par la connaissance avant d’y répandre leurs métis ; ils en firent le tour dans tous les sens. Aucun recoin ne leur échappa, et maintenant qu’il ne s’agit plus que de verser les dernières gouttes de l’essence ariane au sein des populations diverses, devenues accessibles de toutes parts, le temps servira suffisamment ce travail qui se continuera de lui-même, et qui n’a pas besoin d’un surcroît d’impulsion nouvelle pour se perfectionner.\par
En présence de ce fait, on s’explique, non pas pourquoi il ne se trouve pas d’Arians purs, mais l’inutilité de leur présence. Puisque leur vocation générale était de produire les rapprochements et la confusion des types en les unissant les uns aux autres, malgré les distances, ils n’ont plus rien à faire désormais, cette confusion étant accomplie quant au principal, et toutes les dispositions étant prises pour l’accessoire. Voilà donc que l’existence de la plus belle variété humaine, de l’espèce blanche tout entière, des facultés magnifiques concentrées dans l’une et dans l’autre, que la création, le développement et la mort des sociétés et de leurs civilisations, résultat merveilleux du jeu de ces facultés, révèlent un grand point qui est comme le comble, comme le sommet, comme le but suprême de l’histoire. Tout cela naît pour rapprocher les variétés, se développe, brille, s’enrichit pour accélérer leur fusion, et meurt quand le principe ethnique dirigeant est complètement fondu dans les éléments hétérogènes qu’il rallie, et par conséquent lorsque sa tâche locale est suffisamment faite. De plus, le principe blanc, et surtout arian, dispersé sur la face du globe, y est cantonné de façon à ce que les sociétés et les civilisations qu’il anime ne laissent finalement aucune terre, et, par conséquent, aucun groupe en dehors de son action agrégative. La vie de l’humanité prend ainsi une signification d’ensemble qui rentre absolument dans l’ordre des manifestations cosmiques. J’ai dit qu’elle était comparable à une vaste toile composée de différentes matières textiles, et étalant les dessins les plus différemment contournés et bariolés ; elle l’est encore à une chaîne de montagnes relevées en plusieurs sommets qui sont les civilisations, et la composition géologique de ces som­mets est représentée par les divers alliages auxquels ont donné lieu les combinaisons multiples des trois grandes divisions primordiales de l’espèce et de leurs nuances secondaires. Tel est le résultat dominant du travail humain. Tout ce qui sert la civilisation attire l’action de la société ; tout ce qui l’attire l’étend, tout ce qui l’étend la porte géographiquement plus loin, et le dernier terme de cette marche est l’accession ou la suppression de quelques noirs ou de quelques Finnois de plus dans le sein des masses déjà amalgamées. Posons en axiome que le but définitif des fatigues et des souffrances, des plaisirs et des triomphes de notre espèce, est d’arriver un jour à la suprême unité. Ce point acquis va nous livrer ce qu’il nous reste à savoir.\par
L’espèce blanche, considérée abstractivement, a désormais disparu de la face du monde. Après avoir passé l’âge des dieux, où elle était absolument pure ; l’âge des héros, où les mélanges étaient modérés de force et de nombre ; l’âge des noblesses, où des facultés, grandes encore, n’étaient plus renouvelées par des sources taries, elle s’est acheminée plus ou moins promptement, suivant les lieux, vers la confusion définitive de tous ses principes, par suite de ses hymens hétérogènes. Partant, elle n’est plus maintenant représentée que par des hybrides ; ceux qui occupent les territoires des premières sociétés mixtes ont eu naturellement le temps et les occasions de se dégrader le plus. Pour les masses qui, dans l’Europe occidentale et dans l’Amérique du Nord, représentent actuellement la dernière forme possible de culture, elles offrent encore d’assez beaux semblants de force, et sont en effet moins déchues que les habitants de la Campanie, de la Susiane et de l’Iémen. Cependant cette supériorité relative tend constamment à disparaître ; la part de sang arian, subdivisée déjà tant de fois, qui existe encore dans nos contrées, et qui soutient seule l’édifice de notre société, s’achemine chaque jour vers les termes extrêmes de son absorption.\par
Ce résultat obtenu, s’ouvrira l’ère de l’unité. Le principe blanc, tenu en échec dans chaque homme en particulier, y sera vis-à-vis des deux autres dans le rapport de 1 à 2, triste proportion qui, dans tous les cas, suffirait à paralyser son action d’une manière presque complète, mais qui se montre encore plus déplorable quand on réfléchit que cet état de fusion, bien loin d’être le résultat du mariage direct des trois grands types pris à l’état pur, ne sera que le \emph{caput mortuum} d’une série infinie de mélanges, et par conséquent de flétrissures ; le dernier terme de la médiocrité dans tous les genres : médiocrité de force physique, médiocrité de beauté, médiocrité d’aptitudes intellec­tuelles, on peut presque dire néant. Ce triste héritage, chacun en possédera une portion égale ; nul motif n’existe pour que tel homme ait un lot plus riche que tel autre ; et, comme dans ces îles polynésiennes où les métis malais, confinés depuis des siècles, se partagent équitablement un type dont nulle infusion de sang nouveau n’est jamais venue troubler la première composition, les hommes se ressembleront tous. Leur taille, leurs traits, leurs habitudes corporelles, seront semblables. Ils auront même dose de forces physiques, directions pareilles dans les instincts, mesures analogues dans les facultés, et ce niveau général, encore une fois, sera de la plus révoltante humilité.\par
Les nations, non, les troupeaux humains, accablés sous une morne somnolence, vivront dès lors engourdis dans leur nullité, comme les buffles ruminants dans les flaques stagnantes des marais Pontins. Peut-être se tiendront-ils pour les plus sages, les plus savants et les plus habiles des êtres qui furent jamais ; nous-mêmes, lorsque nous contemplons ces grands monuments de l’Égypte et de l’Inde, que nous serions si incapables d’imiter, ne sommes-nous pas convaincus que notre impuissance même prouve notre supériorité ? Nos honteux descendants n’auront aucune peine à trouver quelque argument semblable au nom duquel ils nous dispenseront leur pitié et s’honoreront de leur barbarie. C’était là, diront-ils en montrant d’un geste dédaigneux les ruines chancelantes de nos derniers édifices, c’était là l’emploi insensé des forces de nos ancêtres. Que faire de ces inutiles folies ? Elles seront, en effet, inutiles pour eux ; car la vigoureuse nature aura reconquis l’universelle domination de la terre, et la créature humaine ne sera plus devant elle un maître, mais seulement un hôte, comme les habitants des forêts et des eaux.\par
 Cet état misérable ne sera pas de longue durée non plus ; car un effet latéral des mélanges indéfinis, c’est de réduire les populations à des chiffres de plus en plus minimes. Quand on jette les yeux sur les époques antiques, on s’aperçoit que la terre était alors bien autrement couverte par notre espèce qu’elle ne l’est aujourd’hui. La Chine n’a jamais eu moins d’habitants qu’à présent ; l’Asie centrale était une fourmilière, et on n’y rencontre plus personne. La Scythie, au dire d’Hérodote, était pleine de nations, et la Russie est un désert. L’Allemagne est bien fournie d’hommes, mais elle ne l’était pas moins au II\textsuperscript{e}, au IV\textsuperscript{e}, au V\textsuperscript{e} siècle de notre ère, quand elle jetait sans s’épuiser, sur le monde romain, des océans de guerriers, suivis de leurs femmes et de leurs enfants. La France et l’Angleterre ne nous paraissent ni vides ni incultes ; mais la Gaule et la Grande-Bretagne ne l’étaient pas davantage à l’époque des émigrations kymriques. L’Espagne et l’Italie ne possèdent plus le quart des hommes qui les cou­vraient dans l’antiquité. La Grèce, l’Égypte, la Syrie, l’Asie Mineure, la Mésopotamie, regorgeaient de monde, les villes s’y pressaient aussi nombreuses que des épis dans un champ ; ce sont des solitudes mortuaires, et l’Inde, bien que populeuse encore, n’est plus sous ce rapport que l’ombre d’elle-même. L’Afrique occidentale, cette terre qui nourrissait l’Europe et où tant de métropoles étalaient leurs splendeurs, ne porte plus que les tentes clairsemées de quelques nomades et les villes moribondes d’un petit nombre de marchands. Les autres parties de ce continent languissent de même partout où les Européens et les musulmans ont porté ce qu’ils appellent, les uns le progrès, les autres la foi, et il n’y a que l’intérieur des terres, où personne n’a presque pénétré, qui garde encore un noyau bien compact. Mais ce n’est pas pour durer. Quant à l’Amérique, l’Europe y verse ce qu’elle a de sang ; elle s’appauvrit, si l’autre s’enrichit. Ainsi, du même pas que l’humanité se dégrade, elle s’efface.\par
On ne saurait prétendre à calculer avec rigueur le nombre des siècles qui nous séparent encore de la conclusion certaine. Cependant il n’est pas impossible d’entrevoir un à peu près. La famille ariane, et, à plus forte raison, le reste de la famille blanche, avait cessé d’être absolument pure à l’époque ou naquit le Christ. En admettant que la formation actuelle du globe soit de six à sept mille ans antérieure à cet événement, cette période avait suffi pour flétrir dans son germe le principe visible des sociétés, et, lorsqu’elle finit, la cause de toute décrépitude avait déjà pris la haute main dans le monde. Par ce fait que la race blanche s’était absorbée de manière à perdre la fleur de son essence dans les deux variétés inférieures, celles-ci avaient subi des modifications correspondantes, qui, pour la race jaune, s’étaient étendues fort avant. Dans les dix-huit cents ans qui se sont écoulés depuis, le travail de fusion, bien qu’incessamment continué et préparant ses conquêtes ultérieures sur une échelle plus considérable que jamais, n’a pas été aussi directement efficace. Mais, outre ce qu’il s’est créé de moyens d’action pour l’avenir, il a beaucoup augmenté la confusion ethnique dans l’intérieur de toutes les sociétés, et, par conséquent, hâté d’autant l’heure finale de la perfection de l’amalgame. Ce temps-là est donc bien loin d’avoir été perdu ; et, puisqu’il a préparé l’avenir, et que d’ailleurs les trois variétés ne possèdent plus de groupes purs, ce n’est pas exagérer la rapidité du résultat que de lui donner pour se produire un peu moins de temps qu’il n’en a fallu pour que ses préparations en arrivassent au point où elles sont aujourd’hui. On serait donc tenté d’assigner à la domination de l’homme sur la terre une durée totale de douze à quatorze mille ans, divisée en deux périodes : l’une, qui est passée, aura vu, aura possédé la jeunesse, la vigueur, la grandeur intellectuelle de l’espèce ; l’autre, qui est commencée, en connaîtra la marche défaillante vers la décrépitude \footnote{Cf. JANINE, BUENZOD : La formation de la pensée de Gobineau et \emph{l’Essai sur l’inégalité des races humaines}, Nizet, 1967.}.\par
En s’arrêtant même aux temps qui doivent quelque peu précéder le dernier soupir de notre espèce, en se détournant de ces âges envahis par la mort, où le globe, devenu muet, continuera, mais sans nous, à décrire dans l’espace ses orbes impassibles, je ne sais si l’on n’est pas en droit d’appeler la fin du monde cette époque moins lointaine qui verra déjà l’abaissement complet de notre espèce. Je n’affirmerai pas non plus qu’il fût bien facile de s’intéresser avec un reste d’amour aux destinées de quelques poignées d’êtres dépouillés de force, de beauté, d’intelligence, si l’on ne se rappelait qu’il leur restera du moins la foi religieuse, dernier lien, unique souvenir, héritage précieux des jours meilleurs.\par
Mais la religion elle-même ne nous a pas promis l’éternité ; mais la science, en nous montrant que nous avons commencé, semblait toujours nous assurer aussi que nous devions finir. Il n’y a donc lieu ni de s’étonner ni de s’émouvoir en trouvant une confirmation de plus d’un fait qui ne pouvait passer pour douteux. La prévision attristante, ce n’est pas la mort, c’est la certitude de n’y arriver que dégradés ; et peut-être même cette honte réservée à nos descendants nous pourrait-elle laisser insensibles, si nous n’éprouvions, par une secrète horreur, que les mains rapaces de la destinée sont déjà posées sur nous.
\chapterclose

 


% at least one empty page at end (for booklet couv)
\ifbooklet
  \pagestyle{empty}
  \clearpage
  % 2 empty pages maybe needed for 4e cover
  \ifnum\modulo{\value{page}}{4}=0 \hbox{}\newpage\hbox{}\newpage\fi
  \ifnum\modulo{\value{page}}{4}=1 \hbox{}\newpage\hbox{}\newpage\fi


  \hbox{}\newpage
  \ifodd\value{page}\hbox{}\newpage\fi
  {\centering\color{rubric}\bfseries\noindent\large
    Hurlus ? Qu’est-ce.\par
    \bigskip
  }
  \noindent Des bouquinistes électroniques, pour du texte libre à participation libre,
  téléchargeable gratuitement sur \href{https://hurlus.fr}{\dotuline{hurlus.fr}}.\par
  \bigskip
  \noindent Cette brochure a été produite par des éditeurs bénévoles.
  Elle n’est pas faîte pour être possédée, mais pour être lue, et puis donnée.
  Que circule le texte !
  En page de garde, on peut ajouter une date, un lieu, un nom ; pour suivre le voyage des idées.
  \par

  Ce texte a été choisi parce qu’une personne l’a aimé,
  ou haï, elle a en tous cas pensé qu’il partipait à la formation de notre présent ;
  sans le souci de plaire, vendre, ou militer pour une cause.
  \par

  L’édition électronique est soigneuse, tant sur la technique
  que sur l’établissement du texte ; mais sans aucune prétention scolaire, au contraire.
  Le but est de s’adresser à tous, sans distinction de science ou de diplôme.
  Au plus direct ! (possible)
  \par

  Cet exemplaire en papier a été tiré sur une imprimante personnelle
   ou une photocopieuse. Tout le monde peut le faire.
  Il suffit de
  télécharger un fichier sur \href{https://hurlus.fr}{\dotuline{hurlus.fr}},
  d’imprimer, et agrafer ; puis de lire et donner.\par

  \bigskip

  \noindent PS : Les hurlus furent aussi des rebelles protestants qui cassaient les statues dans les églises catholiques. En 1566 démarra la révolte des gueux dans le pays de Lille. L’insurrection enflamma la région jusqu’à Anvers où les gueux de mer bloquèrent les bateaux espagnols.
  Ce fut une rare guerre de libération dont naquit un pays toujours libre : les Pays-Bas.
  En plat pays francophone, par contre, restèrent des bandes de huguenots, les hurlus, progressivement réprimés par la très catholique Espagne.
  Cette mémoire d’une défaite est éteinte, rallumons-la. Sortons les livres du culte universitaire, cherchons les idoles de l’époque, pour les briser.
\fi

\ifdev % autotext in dev mode
\fontname\font — \textsc{Les règles du jeu}\par
(\hyperref[utopie]{\underline{Lien}})\par
\noindent \initialiv{A}{lors là}\blindtext\par
\noindent \initialiv{À}{ la bonheur des dames}\blindtext\par
\noindent \initialiv{É}{tonnez-le}\blindtext\par
\noindent \initialiv{Q}{ualitativement}\blindtext\par
\noindent \initialiv{V}{aloriser}\blindtext\par
\Blindtext
\phantomsection
\label{utopie}
\Blinddocument
\fi
\end{document}
