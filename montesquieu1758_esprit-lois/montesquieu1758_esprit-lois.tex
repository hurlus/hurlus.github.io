%%%%%%%%%%%%%%%%%%%%%%%%%%%%%%%%%
% LaTeX model https://hurlus.fr %
%%%%%%%%%%%%%%%%%%%%%%%%%%%%%%%%%

% Needed before document class
\RequirePackage{pdftexcmds} % needed for tests expressions
\RequirePackage{fix-cm} % correct units

% Define mode
\def\mode{a4}

\newif\ifaiv % a4
\newif\ifav % a5
\newif\ifbooklet % booklet
\newif\ifcover % cover for booklet

\ifnum \strcmp{\mode}{cover}=0
  \covertrue
\else\ifnum \strcmp{\mode}{booklet}=0
  \booklettrue
\else\ifnum \strcmp{\mode}{a5}=0
  \avtrue
\else
  \aivtrue
\fi\fi\fi

\ifbooklet % do not enclose with {}
  \documentclass[french,twoside]{book} % ,notitlepage
  \usepackage[%
    papersize={105mm, 297mm},
    inner=12mm,
    outer=12mm,
    top=20mm,
    bottom=15mm,
    marginparsep=0pt,
  ]{geometry}
  \usepackage[fontsize=9.5pt]{scrextend} % for Roboto
\else\ifav
  \documentclass[french,twoside]{book} % ,notitlepage
  \usepackage[%
    a5paper,
    inner=25mm,
    outer=15mm,
    top=15mm,
    bottom=15mm,
    marginparsep=0pt,
  ]{geometry}
  \usepackage[fontsize=12pt]{scrextend}
\else% A4 2 cols
  \documentclass[twocolumn]{report}
  \usepackage[%
    a4paper,
    inner=15mm,
    outer=10mm,
    top=25mm,
    bottom=18mm,
    marginparsep=0pt,
  ]{geometry}
  \setlength{\columnsep}{20mm}
  \usepackage[fontsize=9.5pt]{scrextend}
\fi\fi

%%%%%%%%%%%%%%
% Alignments %
%%%%%%%%%%%%%%
% before teinte macros

\setlength{\arrayrulewidth}{0.2pt}
\setlength{\columnseprule}{\arrayrulewidth} % twocol
\setlength{\parskip}{0pt} % classical para with no margin
\setlength{\parindent}{1.5em}

%%%%%%%%%%
% Colors %
%%%%%%%%%%
% before Teinte macros

\usepackage[dvipsnames]{xcolor}
\definecolor{rubric}{HTML}{800000} % the tonic 0c71c3
\def\columnseprulecolor{\color{rubric}}
\colorlet{borderline}{rubric!30!} % definecolor need exact code
\definecolor{shadecolor}{gray}{0.95}
\definecolor{bghi}{gray}{0.5}

%%%%%%%%%%%%%%%%%
% Teinte macros %
%%%%%%%%%%%%%%%%%
%%%%%%%%%%%%%%%%%%%%%%%%%%%%%%%%%%%%%%%%%%%%%%%%%%%
% <TEI> generic (LaTeX names generated by Teinte) %
%%%%%%%%%%%%%%%%%%%%%%%%%%%%%%%%%%%%%%%%%%%%%%%%%%%
% This template is inserted in a specific design
% It is XeLaTeX and otf fonts

\makeatletter % <@@@


\usepackage{blindtext} % generate text for testing
\usepackage[strict]{changepage} % for modulo 4
\usepackage{contour} % rounding words
\usepackage[nodayofweek]{datetime}
% \usepackage{DejaVuSans} % seems buggy for sffont font for symbols
\usepackage{enumitem} % <list>
\usepackage{etoolbox} % patch commands
\usepackage{fancyvrb}
\usepackage{fancyhdr}
\usepackage{float}
\usepackage{fontspec} % XeLaTeX mandatory for fonts
\usepackage{footnote} % used to capture notes in minipage (ex: quote)
\usepackage{framed} % bordering correct with footnote hack
\usepackage{graphicx}
\usepackage{lettrine} % drop caps
\usepackage{lipsum} % generate text for testing
\usepackage[framemethod=tikz,]{mdframed} % maybe used for frame with footnotes inside
\usepackage{pdftexcmds} % needed for tests expressions
\usepackage{polyglossia} % non-break space french punct, bug Warning: "Failed to patch part"
\usepackage[%
  indentfirst=false,
  vskip=1em,
  noorphanfirst=true,
  noorphanafter=true,
  leftmargin=\parindent,
  rightmargin=0pt,
]{quoting}
\usepackage{ragged2e}
\usepackage{setspace} % \setstretch for <quote>
\usepackage{tabularx} % <table>
\usepackage[explicit]{titlesec} % wear titles, !NO implicit
\usepackage{tikz} % ornaments
\usepackage{tocloft} % styling tocs
\usepackage[fit]{truncate} % used im runing titles
\usepackage{unicode-math}
\usepackage[normalem]{ulem} % breakable \uline, normalem is absolutely necessary to keep \emph
\usepackage{verse} % <l>
\usepackage{xcolor} % named colors
\usepackage{xparse} % @ifundefined
\XeTeXdefaultencoding "iso-8859-1" % bad encoding of xstring
\usepackage{xstring} % string tests
\XeTeXdefaultencoding "utf-8"
\PassOptionsToPackage{hyphens}{url} % before hyperref, which load url package

% TOTEST
% \usepackage{hypcap} % links in caption ?
% \usepackage{marginnote}
% TESTED
% \usepackage{background} % doesn’t work with xetek
% \usepackage{bookmark} % prefers the hyperref hack \phantomsection
% \usepackage[color, leftbars]{changebar} % 2 cols doc, impossible to keep bar left
% \usepackage[utf8x]{inputenc} % inputenc package ignored with utf8 based engines
% \usepackage[sfdefault,medium]{inter} % no small caps
% \usepackage{firamath} % choose firasans instead, firamath unavailable in Ubuntu 21-04
% \usepackage{flushend} % bad for last notes, supposed flush end of columns
% \usepackage[stable]{footmisc} % BAD for complex notes https://texfaq.org/FAQ-ftnsect
% \usepackage{helvet} % not for XeLaTeX
% \usepackage{multicol} % not compatible with too much packages (longtable, framed, memoir…)
% \usepackage[default,oldstyle,scale=0.95]{opensans} % no small caps
% \usepackage{sectsty} % \chapterfont OBSOLETE
% \usepackage{soul} % \ul for underline, OBSOLETE with XeTeX
% \usepackage[breakable]{tcolorbox} % text styling gone, footnote hack not kept with breakable


% Metadata inserted by a program, from the TEI source, for title page and runing heads
\title{\textbf{ De l’esprit des lois }}
\date{1758}
\author{Montesquieu, Charles-Louis de Secondat (1689-1755 ; baron de La Brède et de) }
\def\elbibl{Montesquieu, Charles-Louis de Secondat (1689-1755 ; baron de La Brède et de) . 1758. \emph{De l’esprit des lois}}
\def\elsource{Montesquieu, {\itshape De l’esprit des lois} [éd. définitive, 1758]}

% Default metas
\newcommand{\colorprovide}[2]{\@ifundefinedcolor{#1}{\colorlet{#1}{#2}}{}}
\colorprovide{rubric}{red}
\colorprovide{silver}{lightgray}
\@ifundefined{syms}{\newfontfamily\syms{DejaVu Sans}}{}
\newif\ifdev
\@ifundefined{elbibl}{% No meta defined, maybe dev mode
  \newcommand{\elbibl}{Titre court ?}
  \newcommand{\elbook}{Titre du livre source ?}
  \newcommand{\elabstract}{Résumé\par}
  \newcommand{\elurl}{http://oeuvres.github.io/elbook/2}
  \author{Éric Lœchien}
  \title{Un titre de test assez long pour vérifier le comportement d’une maquette}
  \date{1566}
  \devtrue
}{}
\let\eltitle\@title
\let\elauthor\@author
\let\eldate\@date


\defaultfontfeatures{
  % Mapping=tex-text, % no effect seen
  Scale=MatchLowercase,
  Ligatures={TeX,Common},
}


% generic typo commands
\newcommand{\astermono}{\medskip\centerline{\color{rubric}\large\selectfont{\syms ✻}}\medskip\par}%
\newcommand{\astertri}{\medskip\par\centerline{\color{rubric}\large\selectfont{\syms ✻\,✻\,✻}}\medskip\par}%
\newcommand{\asterism}{\bigskip\par\noindent\parbox{\linewidth}{\centering\color{rubric}\large{\syms ✻}\\{\syms ✻}\hskip 0.75em{\syms ✻}}\bigskip\par}%

% lists
\newlength{\listmod}
\setlength{\listmod}{\parindent}
\setlist{
  itemindent=!,
  listparindent=\listmod,
  labelsep=0.2\listmod,
  parsep=0pt,
  % topsep=0.2em, % default topsep is best
}
\setlist[itemize]{
  label=—,
  leftmargin=0pt,
  labelindent=1.2em,
  labelwidth=0pt,
}
\setlist[enumerate]{
  label={\bf\color{rubric}\arabic*.},
  labelindent=0.8\listmod,
  leftmargin=\listmod,
  labelwidth=0pt,
}
\newlist{listalpha}{enumerate}{1}
\setlist[listalpha]{
  label={\bf\color{rubric}\alph*.},
  leftmargin=0pt,
  labelindent=0.8\listmod,
  labelwidth=0pt,
}
\newcommand{\listhead}[1]{\hspace{-1\listmod}\emph{#1}}

\renewcommand{\hrulefill}{%
  \leavevmode\leaders\hrule height 0.2pt\hfill\kern\z@}

% General typo
\DeclareTextFontCommand{\textlarge}{\large}
\DeclareTextFontCommand{\textsmall}{\small}

% commands, inlines
\newcommand{\anchor}[1]{\Hy@raisedlink{\hypertarget{#1}{}}} % link to top of an anchor (not baseline)
\newcommand\abbr[1]{#1}
\newcommand{\autour}[1]{\tikz[baseline=(X.base)]\node [draw=rubric,thin,rectangle,inner sep=1.5pt, rounded corners=3pt] (X) {\color{rubric}#1};}
\newcommand\corr[1]{#1}
\newcommand{\ed}[1]{ {\color{silver}\sffamily\footnotesize (#1)} } % <milestone ed="1688"/>
\newcommand\expan[1]{#1}
\newcommand\foreign[1]{\emph{#1}}
\newcommand\gap[1]{#1}
\renewcommand{\LettrineFontHook}{\color{rubric}}
\newcommand{\initial}[2]{\lettrine[lines=2, loversize=0.3, lhang=0.3]{#1}{#2}}
\newcommand{\initialiv}[2]{%
  \let\oldLFH\LettrineFontHook
  % \renewcommand{\LettrineFontHook}{\color{rubric}\ttfamily}
  \IfSubStr{QJ’}{#1}{
    \lettrine[lines=4, lhang=0.2, loversize=-0.1, lraise=0.2]{\smash{#1}}{#2}
  }{\IfSubStr{É}{#1}{
    \lettrine[lines=4, lhang=0.2, loversize=-0, lraise=0]{\smash{#1}}{#2}
  }{\IfSubStr{ÀÂ}{#1}{
    \lettrine[lines=4, lhang=0.2, loversize=-0, lraise=0, slope=0.6em]{\smash{#1}}{#2}
  }{\IfSubStr{A}{#1}{
    \lettrine[lines=4, lhang=0.2, loversize=0.2, slope=0.6em]{\smash{#1}}{#2}
  }{\IfSubStr{V}{#1}{
    \lettrine[lines=4, lhang=0.2, loversize=0.2, slope=-0.5em]{\smash{#1}}{#2}
  }{
    \lettrine[lines=4, lhang=0.2, loversize=0.2]{\smash{#1}}{#2}
  }}}}}
  \let\LettrineFontHook\oldLFH
}
\newcommand{\labelchar}[1]{\textbf{\color{rubric} #1}}
\newcommand{\milestone}[1]{\autour{\footnotesize\color{rubric} #1}} % <milestone n="4"/>
\newcommand\name[1]{#1}
\newcommand\orig[1]{#1}
\newcommand\orgName[1]{#1}
\newcommand\persName[1]{#1}
\newcommand\placeName[1]{#1}
\newcommand{\pn}[1]{\IfSubStr{-—–¶}{#1}% <p n="3"/>
  {\noindent{\bfseries\color{rubric}   ¶  }}
  {{\footnotesize\autour{ #1}  }}}
\newcommand\reg{}
% \newcommand\ref{} % already defined
\newcommand\sic[1]{#1}
\newcommand\surname[1]{\textsc{#1}}
\newcommand\term[1]{\textbf{#1}}

\def\mednobreak{\ifdim\lastskip<\medskipamount
  \removelastskip\nopagebreak\medskip\fi}
\def\bignobreak{\ifdim\lastskip<\bigskipamount
  \removelastskip\nopagebreak\bigskip\fi}

% commands, blocks
\newcommand{\byline}[1]{\bigskip{\RaggedLeft{#1}\par}\bigskip}
\newcommand{\bibl}[1]{{\RaggedLeft{#1}\par\bigskip}}
\newcommand{\biblitem}[1]{{\noindent\hangindent=\parindent   #1\par}}
\newcommand{\dateline}[1]{\medskip{\RaggedLeft{#1}\par}\bigskip}
\newcommand{\labelblock}[1]{\medbreak{\noindent\color{rubric}\bfseries #1}\par\mednobreak}
\newcommand{\salute}[1]{\bigbreak{#1}\par\medbreak}
\newcommand{\signed}[1]{\bigbreak\filbreak{\raggedleft #1\par}\medskip}

% environments for blocks (some may become commands)
\newenvironment{borderbox}{}{} % framing content
\newenvironment{citbibl}{\ifvmode\hfill\fi}{\ifvmode\par\fi }
\newenvironment{docAuthor}{\ifvmode\vskip4pt\fontsize{16pt}{18pt}\selectfont\fi\itshape}{\ifvmode\par\fi }
\newenvironment{docDate}{}{\ifvmode\par\fi }
\newenvironment{docImprint}{\vskip6pt}{\ifvmode\par\fi }
\newenvironment{docTitle}{\vskip6pt\bfseries\fontsize{18pt}{22pt}\selectfont}{\par }
\newenvironment{msHead}{\vskip6pt}{\par}
\newenvironment{msItem}{\vskip6pt}{\par}
\newenvironment{titlePart}{}{\par }


% environments for block containers
\newenvironment{argument}{\itshape\parindent0pt}{\vskip1.5em}
\newenvironment{biblfree}{}{\ifvmode\par\fi }
\newenvironment{bibitemlist}[1]{%
  \list{\@biblabel{\@arabic\c@enumiv}}%
  {%
    \settowidth\labelwidth{\@biblabel{#1}}%
    \leftmargin\labelwidth
    \advance\leftmargin\labelsep
    \@openbib@code
    \usecounter{enumiv}%
    \let\p@enumiv\@empty
    \renewcommand\theenumiv{\@arabic\c@enumiv}%
  }
  \sloppy
  \clubpenalty4000
  \@clubpenalty \clubpenalty
  \widowpenalty4000%
  \sfcode`\.\@m
}%
{\def\@noitemerr
  {\@latex@warning{Empty `bibitemlist' environment}}%
\endlist}
\newenvironment{quoteblock}% may be used for ornaments
  {\begin{quoting}}
  {\end{quoting}}

% table () is preceded and finished by custom command
\newcommand{\tableopen}[1]{%
  \ifnum\strcmp{#1}{wide}=0{%
    \begin{center}
  }
  \else\ifnum\strcmp{#1}{long}=0{%
    \begin{center}
  }
  \else{%
    \begin{center}
  }
  \fi\fi
}
\newcommand{\tableclose}[1]{%
  \ifnum\strcmp{#1}{wide}=0{%
    \end{center}
  }
  \else\ifnum\strcmp{#1}{long}=0{%
    \end{center}
  }
  \else{%
    \end{center}
  }
  \fi\fi
}


% text structure
\newcommand\chapteropen{} % before chapter title
\newcommand\chaptercont{} % after title, argument, epigraph…
\newcommand\chapterclose{} % maybe useful for multicol settings
\setcounter{secnumdepth}{-2} % no counters for hierarchy titles
\setcounter{tocdepth}{5} % deep toc
\markright{\@title} % ???
\markboth{\@title}{\@author} % ???
\renewcommand\tableofcontents{\@starttoc{toc}}
% toclof format
% \renewcommand{\@tocrmarg}{0.1em} % Useless command?
% \renewcommand{\@pnumwidth}{0.5em} % {1.75em}
\renewcommand{\@cftmaketoctitle}{}
\setlength{\cftbeforesecskip}{\z@ \@plus.2\p@}
\renewcommand{\cftchapfont}{}
\renewcommand{\cftchapdotsep}{\cftdotsep}
\renewcommand{\cftchapleader}{\normalfont\cftdotfill{\cftchapdotsep}}
\renewcommand{\cftchappagefont}{\bfseries}
\setlength{\cftbeforechapskip}{0em \@plus\p@}
% \renewcommand{\cftsecfont}{\small\relax}
\renewcommand{\cftsecpagefont}{\normalfont}
% \renewcommand{\cftsubsecfont}{\small\relax}
\renewcommand{\cftsecdotsep}{\cftdotsep}
\renewcommand{\cftsecpagefont}{\normalfont}
\renewcommand{\cftsecleader}{\normalfont\cftdotfill{\cftsecdotsep}}
\setlength{\cftsecindent}{1em}
\setlength{\cftsubsecindent}{2em}
\setlength{\cftsubsubsecindent}{3em}
\setlength{\cftchapnumwidth}{1em}
\setlength{\cftsecnumwidth}{1em}
\setlength{\cftsubsecnumwidth}{1em}
\setlength{\cftsubsubsecnumwidth}{1em}

% footnotes
\newif\ifheading
\newcommand*{\fnmarkscale}{\ifheading 0.70 \else 1 \fi}
\renewcommand\footnoterule{\vspace*{0.3cm}\hrule height \arrayrulewidth width 3cm \vspace*{0.3cm}}
\setlength\footnotesep{1.5\footnotesep} % footnote separator
\renewcommand\@makefntext[1]{\parindent 1.5em \noindent \hb@xt@1.8em{\hss{\normalfont\@thefnmark . }}#1} % no superscipt in foot
\patchcmd{\@footnotetext}{\footnotesize}{\footnotesize\sffamily}{}{} % before scrextend, hyperref


%   see https://tex.stackexchange.com/a/34449/5049
\def\truncdiv#1#2{((#1-(#2-1)/2)/#2)}
\def\moduloop#1#2{(#1-\truncdiv{#1}{#2}*#2)}
\def\modulo#1#2{\number\numexpr\moduloop{#1}{#2}\relax}

% orphans and widows
\clubpenalty=9996
\widowpenalty=9999
\brokenpenalty=4991
\predisplaypenalty=10000
\postdisplaypenalty=1549
\displaywidowpenalty=1602
\hyphenpenalty=400
% Copied from Rahtz but not understood
\def\@pnumwidth{1.55em}
\def\@tocrmarg {2.55em}
\def\@dotsep{4.5}
\emergencystretch 3em
\hbadness=4000
\pretolerance=750
\tolerance=2000
\vbadness=4000
\def\Gin@extensions{.pdf,.png,.jpg,.mps,.tif}
% \renewcommand{\@cite}[1]{#1} % biblio

\usepackage{hyperref} % supposed to be the last one, :o) except for the ones to follow
\urlstyle{same} % after hyperref
\hypersetup{
  % pdftex, % no effect
  pdftitle={\elbibl},
  % pdfauthor={Your name here},
  % pdfsubject={Your subject here},
  % pdfkeywords={keyword1, keyword2},
  bookmarksnumbered=true,
  bookmarksopen=true,
  bookmarksopenlevel=1,
  pdfstartview=Fit,
  breaklinks=true, % avoid long links
  pdfpagemode=UseOutlines,    % pdf toc
  hyperfootnotes=true,
  colorlinks=false,
  pdfborder=0 0 0,
  % pdfpagelayout=TwoPageRight,
  % linktocpage=true, % NO, toc, link only on page no
}

\makeatother % /@@@>
%%%%%%%%%%%%%%
% </TEI> end %
%%%%%%%%%%%%%%


%%%%%%%%%%%%%
% footnotes %
%%%%%%%%%%%%%
\renewcommand{\thefootnote}{\bfseries\textcolor{rubric}{\arabic{footnote}}} % color for footnote marks

%%%%%%%%%
% Fonts %
%%%%%%%%%
\usepackage[]{roboto} % SmallCaps, Regular is a bit bold
% \linespread{0.90} % too compact, keep font natural
\newfontfamily\fontrun[]{Roboto Condensed Light} % condensed runing heads
\ifav
  \setmainfont[
    ItalicFont={Roboto Light Italic},
  ]{Roboto}
\else\ifbooklet
  \setmainfont[
    ItalicFont={Roboto Light Italic},
  ]{Roboto}
\else
\setmainfont[
  ItalicFont={Roboto Italic},
]{Roboto Light}
\fi\fi
\renewcommand{\LettrineFontHook}{\bfseries\color{rubric}}
% \renewenvironment{labelblock}{\begin{center}\bfseries\color{rubric}}{\end{center}}

%%%%%%%%
% MISC %
%%%%%%%%

\setdefaultlanguage[frenchpart=false]{french} % bug on part


\newenvironment{quotebar}{%
    \def\FrameCommand{{\color{rubric!10!}\vrule width 0.5em} \hspace{0.9em}}%
    \def\OuterFrameSep{\itemsep} % séparateur vertical
    \MakeFramed {\advance\hsize-\width \FrameRestore}
  }%
  {%
    \endMakeFramed
  }
\renewenvironment{quoteblock}% may be used for ornaments
  {%
    \savenotes
    \setstretch{0.9}
    \normalfont
    \begin{quotebar}
  }
  {%
    \end{quotebar}
    \spewnotes
  }


\renewcommand{\headrulewidth}{\arrayrulewidth}
\renewcommand{\headrule}{{\color{rubric}\hrule}}

% delicate tuning, image has produce line-height problems in title on 2 lines
\titleformat{name=\chapter} % command
  [display] % shape
  {\vspace{1.5em}\centering} % format
  {} % label
  {0pt} % separator between n
  {}
[{\color{rubric}\huge\textbf{#1}}\bigskip] % after code
% \titlespacing{command}{left spacing}{before spacing}{after spacing}[right]
\titlespacing*{\chapter}{0pt}{-2em}{0pt}[0pt]

\titleformat{name=\section}
  [block]{}{}{}{}
  [\vbox{\color{rubric}\large\raggedleft\textbf{#1}}]
\titlespacing{\section}{0pt}{0pt plus 4pt minus 2pt}{\baselineskip}

\titleformat{name=\subsection}
  [block]
  {}
  {} % \thesection
  {} % separator \arrayrulewidth
  {}
[\vbox{\large\textbf{#1}}]
% \titlespacing{\subsection}{0pt}{0pt plus 4pt minus 2pt}{\baselineskip}

\ifaiv
  \fancypagestyle{main}{%
    \fancyhf{}
    \setlength{\headheight}{1.5em}
    \fancyhead{} % reset head
    \fancyfoot{} % reset foot
    \fancyhead[L]{\truncate{0.45\headwidth}{\fontrun\elbibl}} % book ref
    \fancyhead[R]{\truncate{0.45\headwidth}{ \fontrun\nouppercase\leftmark}} % Chapter title
    \fancyhead[C]{\thepage}
  }
  \fancypagestyle{plain}{% apply to chapter
    \fancyhf{}% clear all header and footer fields
    \setlength{\headheight}{1.5em}
    \fancyhead[L]{\truncate{0.9\headwidth}{\fontrun\elbibl}}
    \fancyhead[R]{\thepage}
  }
\else
  \fancypagestyle{main}{%
    \fancyhf{}
    \setlength{\headheight}{1.5em}
    \fancyhead{} % reset head
    \fancyfoot{} % reset foot
    \fancyhead[RE]{\truncate{0.9\headwidth}{\fontrun\elbibl}} % book ref
    \fancyhead[LO]{\truncate{0.9\headwidth}{\fontrun\nouppercase\leftmark}} % Chapter title, \nouppercase needed
    \fancyhead[RO,LE]{\thepage}
  }
  \fancypagestyle{plain}{% apply to chapter
    \fancyhf{}% clear all header and footer fields
    \setlength{\headheight}{1.5em}
    \fancyhead[L]{\truncate{0.9\headwidth}{\fontrun\elbibl}}
    \fancyhead[R]{\thepage}
  }
\fi

\ifav % a5 only
  \titleclass{\section}{top}
\fi

\newcommand\chapo{{%
  \vspace*{-3em}
  \centering % no vskip ()
  {\Large\addfontfeature{LetterSpace=25}\bfseries{\elauthor}}\par
  \smallskip
  {\large\eldate}\par
  \bigskip
  {\Large\selectfont{\eltitle}}\par
  \bigskip
  {\color{rubric}\hline\par}
  \bigskip
  {\Large TEXTE LIBRE À PARTICPATION LIBRE\par}
  \centerline{\small\color{rubric} {hurlus.fr, tiré le \today}}\par
  \bigskip
}}

\newcommand\cover{{%
  \thispagestyle{empty}
  \centering
  {\LARGE\bfseries{\elauthor}}\par
  \bigskip
  {\Large\eldate}\par
  \bigskip
  \bigskip
  {\LARGE\selectfont{\eltitle}}\par
  \vfill\null
  {\color{rubric}\setlength{\arrayrulewidth}{2pt}\hline\par}
  \vfill\null
  {\Large TEXTE LIBRE À PARTICPATION LIBRE\par}
  \centerline{{\href{https://hurlus.fr}{\dotuline{hurlus.fr}}, tiré le \today}}\par
}}

\begin{document}
\pagestyle{empty}
\ifbooklet{
  \cover\newpage
  \thispagestyle{empty}\hbox{}\newpage
  \cover\newpage\noindent Les voyages de la brochure\par
  \bigskip
  \begin{tabularx}{\textwidth}{l|X|X}
    \textbf{Date} & \textbf{Lieu}& \textbf{Nom/pseudo} \\ \hline
    \rule{0pt}{25cm} &  &   \\
  \end{tabularx}
  \newpage
  \addtocounter{page}{-4}
}\fi

\thispagestyle{empty}
\ifaiv
  \twocolumn[\chapo]
\else
  \chapo
\fi
{\it\elabstract}
\bigskip
\makeatletter\@starttoc{toc}\makeatother % toc without new page
\bigskip

\pagestyle{main} % after style

  \frontmatter \noindent {\itshape … Prolem sine matre creatam. }\\
                     {\scshape Ovide}.
\mainmatter \section[{Avertissement de l’auteur}]{Avertissement de l’auteur}\renewcommand{\leftmark}{Avertissement de l’auteur}

\noindent 1° Pour l’intelligence des quatre premiers livres de cet ouvrage, il faut observer que ce que j’appelle la {\itshape vertu} dans la république est l’amour de la patrie, c’est-à-dire l’amour de l’égalité. Ce n’est point une vertu morale, ni une vertu chrétienne ; c’est la vertu {\itshape politique} ; et celle-ci est le ressort qui fait mouvoir le gouvernement républicain, comme l’hon{\itshape neur} est le ressort qui fait mouvoir la monarchie. J’ai donc appelé {\itshape vertu politique} l’amour de la patrie et de l’égalité. J’ai eu des idées nouvelles ; il a bien fallu trouver de nouveaux mots, ou donner aux anciens de nouvelles acceptions. Ceux qui n’ont pas compris ceci m’ont fait dire des choses absurdes, et qui seraient révoltantes dans tous les pays du monde, parce que, dans tous les pays du monde, on veut de la morale.\par
2° Il faut faire attention qu’il y a une très grande différence entre dire qu’une certaine qualité, modification de l’âme, ou vertu, n’est pas le ressort qui fait agir un gouvernement, et dire qu’elle n’est point dans ce gouvernement. Si je disais : telle roue, tel pignon ne sont point le ressort qui fait mouvoir cette montre, en conclurait-on qu’ils ne sont point dans la montre ? Tant s’en faut que les vertus morales et chrétiennes soient exclues de la monarchie, que même la vertu politique ne l’est pas. En un mot, l’honneur est dans la république, quoique la vertu politique en soit le ressort ; la vertu politique est dans la monarchie, quoique l’honneur en soit le ressort.\par
Enfin, l’homme de bien dont il est question dans le livre III, chapitre V, n’est pas l’homme de bien chrétien, mais l’homme de bien politique, qui a la vertu politique dont j’ai parlé. C’est l’homme qui aime les lois de son pays, et qui agit par l’amour des lois de son pays. J’ai donné un nouveau jour à toutes ces choses dans cette édition-ci, en fixant encore plus les idées : et, dans la plupart des endroits où je me suis servi du mot de {\itshape vertu}, j’ai mis {\itshape vertu politique.}
\section[{Préface}]{Préface}\renewcommand{\leftmark}{Préface}

\noindent Si, dans le nombre infini de choses qui sont dans ce livre, il y en avait quelqu’une qui, contre mon attente, pût offenser, il n’y en a pas du moins qui y ait été mise avec mauvaise intention. Je n’ai point naturellement l’esprit désapprobateur. Platon remerciait le ciel de ce qu’il était né du temps de Socrate ; et moi, je lui rends grâces de ce qu’il m’a fait naître dans le gouvernement où je vis, et de ce qu’il a voulu que j’obéisse à ceux qu’il m’a fait aimer.\par
Je demande une grâce que je crains qu’on ne m’accorde pas : c’est de ne pas juger, par la lecture d’un moment, d’un travail de vingt années ; d’approuver ou de condamner le livre entier, et non pas quelques phrases. Si l’on veut chercher le dessein de l’auteur, on ne le peut bien découvrir que dans le dessein de l’ouvrage.\par
J’ai d’abord examiné les hommes, et j’ai cru que, dans cette infinie diversité de lois et de mœurs, ils n’étaient pas uniquement conduits par leurs fantaisies.\par
J’ai posé les principes, et j’ai vu les cas particuliers s’y plier comme d’eux-mêmes, les histoires de toutes les nations n’en être que les suites, et chaque loi particulière liée avec une autre loi, ou dépendre d’une autre plus générale.\par
Quand j’ai été rappelé à l’antiquité, j’ai cherché à en prendre l’esprit, pour ne pas regarder comme semblables des cas réellement différents, et ne pas manquer les différences de ceux qui paraissent semblables.\par
Je n’ai point tiré mes principes de mes préjugés, mais de la nature des choses.\par
Ici, bien des vérités ne se feront sentir qu’après qu’on aura vu la chaîne qui les lie à d’autres. Plus on réfléchira sur les détails, plus on sentira la certitude des principes. Ces détails même, je ne les ai pas tous donnés ; car, qui pourrait dire tout sans un mortel ennui ?\par
On ne trouvera point ici ces traits saillants qui semblent caractériser les ouvrages d’aujourd’hui. Pour peu qu’on voie les choses avec une certaine étendue, les saillies s’évanouissent ; elles ne naissent d’ordinaire que parce que l’esprit se jette tout d’un côté, et abandonne tous les autres.\par
Je n’écris point pour censurer ce qui est établi dans quelque pays que ce soit. Chaque nation trouvera ici les raisons de ses maximes ; et on en tirera naturellement cette conséquence, qu’il n’appartient de proposer des changements qu’à ceux qui sont assez heureusement nés pour pénétrer d’un coup de génie toute la constitution d’un État.\par
Il n’est pas indifférent que le peuple soit éclairé. Les préjugés des magistrats ont commencé par être les préjugés de la nation. Dans un temps d’ignorance, on n’a aucun doute, même lorsqu’on fait les plus grands maux ; dans un temps de lumière, on tremble encore lorsqu’on fait les plus grands biens. On sent les abus anciens, on en voit la correction ; mais on voit encore les abus de la correction même. On laisse le mal, si l’on craint le pire ; on laisse le bien, si on est en doute du mieux. On ne regarde les parties que pour juger du tout ensemble ; on examine toutes les causes pour voir tous les résultats.\par
Si je pouvais faire en sorte que tout le monde eût de nouvelles raisons pour aimer ses devoirs, son prince, sa patrie, ses lois ; qu’on pût mieux sentir son bonheur dans chaque pays, dans chaque gouvernement, dans chaque poste où l’on se trouve ; je me croirais le plus heureux des mortels.\par
Si je pouvais faire en sorte que ceux qui commandent augmentassent leurs connaissances sur ce qu’ils doivent prescrire, et que ceux qui obéissent trouvassent un nouveau plaisir à obéir, je me croirais le plus heureux des mortels.\par
Je me croirais le plus heureux des mortels, si je pouvais faire que les hommes pussent se guérir de leurs préjugés. J’appelle ici préjugés, non pas ce qui fait qu’on ignore de certaines choses, mais ce qui fait qu’on s’ignore soi-même.\par
C’est en cherchant à instruire les hommes, que l’on peut pratiquer cette vertu générale qui comprend l’amour de tous. L’homme, cet être flexible, se pliant dans la société aux pensées et aux impressions des autres, est également capable de connaître sa propre nature lorsqu’on la lui montre, et d’en perdre jusqu’au sentiment lorsqu’on la lui dérobe.\par
J’ai bien des fois commencé, et bien des fois abandonné cet ouvrage ; j’ai mille fois envoyé aux\footnote{Ludibria ventis.} vents les feuilles que j’avais écrites, je sentais tous les jours les mains paternelles tomber\footnote{Bis patriae cecidere manus…} ; je suivais mon objet sans former de dessein ; je ne connaissais ni les règles ni les exceptions ; je ne trouvais la vérité que pour la perdre. Mais, quand j’ai découvert mes principes, tout ce que je cherchais est venu à moi ; et, dans le cours de vingt années, j’ai vu mon ouvrage commencer, croître, s’avancer et finir.\par
Si cet ouvrage a du succès, je le devrai beaucoup à la majesté de mon sujet ; cependant je ne crois pas avoir totalement manqué de génie. Quand j’ai vu ce que tant de grands hommes, en France, en Angleterre et en Allemagne, ont écrit avant moi, j’ai été dans l’admiration ; mais je n’ai point perdu le courage : Et moi aussi, je suis peintre\footnote{Ed io anche son pittore.}, ai-je dit avec le Corrège.
\section[{Première partie}]{Première partie}\renewcommand{\leftmark}{Première partie}

\subsection[{Livre premier. Des lois en général}]{Livre premier. Des lois en général}
\subsubsection[{Chapitre I. Des lois, dans le rapport qu’elles ont avec les divers êtres}]{Chapitre I. Des lois, dans le rapport qu’elles ont avec les divers êtres}
\noindent Les lois, dans la signification la plus étendue, sont les rapports nécessaires qui dérivent de la nature des choses ; et, dans ce sens, tous les êtres ont leurs lois, la divinité\footnote{La loi, dit Plutarque, est la reine de tous mortels et immortels. Au traité {\itshape Qu’il est requis qu’un prince soit savant}.} a ses lois, le monde matériel a ses lois, les intelligences supérieures à l’homme ont leurs lois, les bêtes ont leurs lois, l’homme a ses lois.\par
Ceux qui ont dit {\itshape qu’une fatalité aveugle a produit tous les effets que nous voyons dans le monde}, ont dit une grande absurdité : car quelle plus grande absurdité qu’une fatalité aveugle qui aurait produit des êtres intelligents ?\par
Il y a donc une raison primitive ; et les lois sont les rapports qui se trouvent entre elle et les différents êtres, et les rapports de ces divers êtres entre eux.\par
Dieu a du rapport avec l’univers, comme créateur et comme conservateur : les lois selon lesquelles il a créé sont celles selon lesquelles il conserve. Il agit selon ces règles, parce qu’il les connaît ; il les connaît parce qu’il les a faites ; il les a faites, parce qu’elles ont du rapport avec sa sagesse et sa puissance.\par
Comme nous voyons que le monde, formé par le mouvement de la matière, et privé d’intelligence, subsiste toujours, il faut que ses mouvements aient des lois invariables ; et, si l’on pouvait imaginer un autre monde que celui-ci, il aurait des règles constantes, ou il serait détruit.\par
Ainsi la création, qui paraît être un acte arbitraire, suppose des règles aussi invariables que la fatalité des athées. Il serait absurde de dire que le créateur, sans ces règles, pourrait gouverner le monde, puisque le monde ne subsisterait pas sans elles.\par
Ces règles sont un rapport constamment établi. Entre un corps mû et un autre corps mû, c’est suivant les rapports de la masse et de la vitesse que tous les mouvements sont reçus, augmentés, diminués, perdus ; chaque diversité est {\itshape uniformité}, chaque changement est {\itshape constance}.\par
Les êtres particuliers intelligents peuvent avoir des lois qu’ils ont faites ; mais ils en ont aussi qu’ils n’ont pas faites. Avant qu’il y eût des êtres intelligents, ils étaient possibles ; ils avaient donc des rapports possibles, et par conséquent des lois possibles. Avant qu’il y eût des lois faites, il y avait des rapports de justice possibles. Dire qu’il n’y a rien de juste ni d’injuste que ce qu’ordonnent ou défendent les lois positives, c’est dire qu’avant qu’on eût tracé de cercle, tous les rayons n’étaient pas égaux.\par
Il faut donc avouer des rapports d’équité antérieurs à la loi positive qui les établit : comme, par exemple, que, supposé qu’il y eût des sociétés d’hommes, il serait juste de se conformer à leurs lois ; que, s’il y avait des êtres intelligents qui eussent reçu quelque bienfait d’un autre être, ils devraient en avoir de la reconnaissance ; que, si un être intelligent avait créé un être intelligent, le créé devrait rester dans la dépendance qu’il a eue dès son origine ; qu’un être intelligent, qui a fait du mal à un être intelligent, mérite de recevoir le même mal ; et ainsi du reste.\par
Mais il s’en faut bien que le monde intelligent soit aussi bien gouverné que le monde physique. Car, quoique celui-là ait aussi des lois qui par leur nature sont invariables, il ne les suit pas constamment comme le monde physique suit les siennes. La raison en est que les êtres particuliers intelligents sont bornés par leur nature, et par conséquent sujets à l’erreur ; et, d’un autre côté, il est de leur nature qu’ils agissent par eux-mêmes. Ils ne suivent donc pas constamment leurs lois primitives ; et celles même qu’ils se donnent, ils ne les suivent pas toujours.\par
On ne sait si les bêtes sont gouvernées par les lois générales du mouvement, ou par une motion particulière. Quoi qu’il en soit, elles n’ont point avec Dieu de rapport plus intime que le reste du monde matériel ; et le sentiment ne leur sert que dans le rapport qu’elles ont entre elles, ou avec d’autres êtres particuliers, ou avec elles-mêmes.\par
Par l’attrait du plaisir, elles conservent leur être particulier ; et, par le même attrait, elles conservent leur espèce. Elles ont des lois naturelles, parce qu’elles sont unies par le sentiment ; elles n’ont point de lois positives, parce qu’elles ne sont point unies par la connaissance. Elles ne suivent pourtant pas invariablement leurs lois naturelles : les plantes, en qui nous ne remarquons ni connaissance ni sentiment, les suivent mieux.\par
Les bêtes n’ont point les suprêmes avantages que nous avons ; elles en ont que nous n’avons pas. Elles n’ont point nos espérances, mais elles n’ont pas nos craintes ; elles subissent comme nous la mort, mais c’est sans la connaître ; la plupart même se conservent mieux que nous, et ne font pas un aussi mauvais usage de leurs passions.\par
L’homme, comme être physique, est, ainsi que les autres corps, gouverné par des lois invariables. Comme être intelligent, il viole sans cesse les lois que Dieu a établies, et change celles qu’il établit lui-même. Il faut qu’il se conduise ; et cependant il est un être borné : il est sujet à l’ignorance et à l’erreur, comme toutes les intelligences finies ; les faibles connaissances qu’il a, il les perd encore. Comme créature sensible, il devient sujet à mille passions. Un tel être pouvait à tous les instants oublier son créateur ; Dieu l’a rappelé à lui par les lois de la religion. Un tel être pouvait à tous les instants s’oublier lui-même ; les philosophes l’ont averti par les lois de la morale. Fait pour vivre dans la société, il y pouvait oublier les autres ; les législateurs l’ont rendu à ses devoirs par les lois politiques et civiles.
\subsubsection[{Chapitre II. Des lois de la nature}]{Chapitre II. Des lois de la nature}
\noindent Avant toutes ces lois, sont celles de la nature, ainsi nommées, parce qu’elles dérivent uniquement de la constitution de notre être. Pour les connaître bien, il faut considérer un homme avant l’établissement des sociétés. Les lois de la nature seront celles qu’il recevrait dans un état pareil.\par
Cette loi qui, en imprimant dans nous-mêmes l’idée d’un créateur, nous porte vers lui, est la première des lois {\itshape naturelles} par son importance, et non pas dans l’ordre de ces lois. L’homme, dans l’état de nature, aurait plutôt la faculté de connaître, qu’il n’aurait des connaissances. Il est clair que ses premières idées ne seraient point des idées spéculatives : il songerait à la conservation de son être, avant de chercher l’origine de son être. Un homme pareil ne sentirait d’abord que sa faiblesse ; sa timidité serait extrême : et, si l’on avait là-dessus besoin de l’expérience, l’on a trouvé dans les forêts des hommes sauvages\footnote{Témoin le sauvage qui fut trouvé dans les forêts de Hanover, et que l’on vit en Angleterre sous le règne de George I\textsuperscript{er}.} ; tout les fait trembler, tout les fait fuir.\par
Dans cet état, chacun se sent inférieur ; à peine chacun se sent-il égal. On ne chercherait donc point à s’attaquer, et la paix serait la première loi naturelle.\par
Le désir que Hobbes donne d’abord aux hommes de se subjuguer les uns les autres, n’est pas raisonnable. L’idée de l’empire et de la domination est si composée, et dépend de tant d’autres idées, que ce ne serait pas celle qu’il aurait d’abord.\par
Hobbes demande {\itshape pourquoi, si les hommes ne sont pas naturellement en état de guerre, ils vont toujours armés, et pourquoi ils ont des clefs pour fermer leurs maisons}. Mais on ne sent pas que l’on attribue aux hommes avant l’établissement des sociétés, ce qui ne peut leur arriver qu’après cet établissement, qui leur fait trouver des motifs pour s’attaquer et pour se défendre.\par
Au sentiment de sa faiblesse, l’homme joindrait le sentiment de ses besoins. Ainsi une autre loi naturelle serait celle qui lui inspirerait de chercher à se nourrir.\par
J’ai dit que la crainte poilerait les hommes à se fuir : mais les marques d’une crainte réciproque les engageraient bientôt à s’approcher. D’ailleurs ils y seraient portés par le plaisir qu’un animal sent à l’approche d’un animal de son espèce. De plus, ce charme que les deux sexes s’inspirent par leur différence, augmenterait ce plaisir ; et la prière naturelle qu’ils se font toujours l’un à l’autre, serait une troisième loi.\par
Outre le sentiment que les hommes ont d’abord, ils parviennent encore à avoir des connaissances ; ainsi ils ont un second lien que les autres animaux n’ont pas. Ils ont donc un nouveau motif de s’unir ; et le désir de vivre en société est une quatrième loi naturelle.
\subsubsection[{Chapitre III. Des lois positives}]{Chapitre III. Des lois positives}
\noindent Sitôt que les hommes sont en société, ils perdent le sentiment de leur faiblesse ; l’égalité, qui était entre eux, cesse, et l’état de guerre commence.\par
Chaque société particulière vient à sentir sa force ; ce qui produit un état de guerre de nation à nation. Les particuliers, dans chaque société, commencent à sentir leur force ; ils cherchent à tourner en leur faveur les principaux avantages de cette société ; ce qui fait entre eux un état de guerre.\par
Ces deux sortes d’état de guerre font établir les lois parmi les hommes. Considérés comme habitants d’une si grande planète, qu’il est nécessaire qu’il y ait différents peuples, ils ont des lois dans le rapport que ces peuples ont entre eux ; et c’est le DROIT DES GENS. Considérés comme vivant dans une société qui doit être maintenue, ils ont des lois dans le rapport qu’ont ceux qui gouvernent avec ceux qui sont gouvernés ; et c’est le DROIT POLITIQUE. Ils en ont encore dans le rapport que tous les citoyens ont entre eux ; et c’est le DROIT CIVIL.\par
Le {\itshape droit des gens} est naturellement fondé sur ce principe, que les diverses nations doivent se faire, dans la paix, le plus de bien, et, dans la guerre, le moins de mal qu’il est possible, sans nuire à leurs véritables intérêts.\par
L’objet de la guerre, c’est la victoire ; celui de la victoire, la conquête ; celui de la conquête, la conservation. De ce principe et du précédent doivent dériver toutes les lois qui forment le droit des gens.\par
Toutes les nations ont un droit des gens ; et les Iroquois même, qui mangent leurs prisonniers, en ont un. Ils envoient et reçoivent des ambassades ; ils connaissent des droits de la guerre et de la paix : le mal est que ce droit des gens n’est pas fondé sur les vrais principes.\par
Outre le droit des gens, qui regarde toutes les sociétés, il y a un droit politique pour chacune. Une société ne saurait subsister sans un gouvernement. La réunion de toutes les forces particulières, dit très bien Gravina, forme ce qu’on appelle l’ÉTAT POLITIQUE.\par
La force générale peut être placée entre les mains d’un seul, ou entre les mains de plusieurs. Quelques-uns ont pensé que, la nature ayant établi le pouvoir paternel, le gouvernement d’un seul était le plus conforme à la nature. Mais l’exemple du pouvoir paternel ne prouve rien. Car, si le pouvoir du père a du rapport au gouvernement d’un seul, après la mort du père, le pouvoir des frères ou, après la mort des frères, celui des cousins germains ont du rapport au gouvernement de plusieurs. La puissance politique comprend nécessairement l’union de plusieurs familles.\par
Il vaut mieux dire que le gouvernement le plus conforme à la nature est celui dont la disposition particulière se rapporte mieux à la disposition du peuple pour lequel il est établi.\par
Les forces particulières ne peuvent se réunir sans que toutes les volontés se réunissent. La réunion de ces volontés, dit encore très bien Gravina, est ce qu’on appelle l’ÉTAT CIVIL.\par
La loi, en général, est la raison humaine, en tant qu’elle gouverne tous les peuples de la terre ; et les lois politiques et civiles de chaque nation ne doivent être que les cas particuliers où s’applique cette raison humaine.\par
Elles doivent être tellement propres au peuple pour lequel elles sont faites, que c’est un très grand hasard si celles d’une nation peuvent convenir à une autre.\par
Il faut qu’elles se rapportent à la nature et au principe du gouvernement qui est établi, ou qu’on veut établir ; soit qu’elles le forment, comme font les lois politiques ; soit qu’elles le maintiennent, comme font les lois civiles.\par
Elles doivent être relatives au physique du pays ; au climat glacé, brûlant ou tempéré ; à la qualité du terrain, à sa situation, à sa grandeur ; au genre de vie des peuples, laboureurs, chasseurs ou pasteurs ; elles doivent se rapporter au degré de liberté que la constitution peut souffrir ; à la religion des habitants, à leurs inclinations, à leurs richesses, à leur nombre, à leur commerce, à leurs mœurs, à leurs manières. Enfin elles ont des rapports entre elles ; elles en ont avec leur origine, avec l’objet du législateur, avec l’ordre des choses sur lesquelles elles sont établies. C’est dans toutes ces vues qu’il faut les considérer.\par
C’est ce que j’entreprends de faire dans cet ouvrage. J’examinerai tous ces rapports : ils forment tous ensemble ce que l’on appelle l’ESPRIT DES LOIS.\par
Je n’ai point séparé les lois {\itshape politiques} des {\itshape civiles} : car, comme je ne traite point des lois, mais de l’esprit des lois, et que cet esprit consiste dans les divers rapports que les lois peuvent avoir avec diverses choses, j’ai dû moins suivre l’ordre naturel des lois, que celui de ces rapports et de ces choses.\par
J’examinerai d’abord les rapports que les lois ont avec la nature et avec le principe de chaque gouvernement : et, comme ce principe a sur les lois une suprême influence, je m’attacherai à le bien connaître ; et, si je puis une fois l’établir, on en verra couler les lois comme de leur source. Je passerai ensuite aux autres rapports, qui semblent être plus particuliers.
\subsection[{Livre deuxième. Des lois qui dérivent directement de la nature du gouvernement}]{Livre deuxième. Des lois qui dérivent directement de la nature du gouvernement}
\subsubsection[{Chapitre I. De la nature des trois divers gouvernements}]{Chapitre I. De la nature des trois divers gouvernements}
\noindent Il y a trois espèces de gouvernements : le RÉPUBLICAIN, le MONARCHIQUE et le DESPOTIQUE. Pour en découvrir la nature, il suffit de l’idée qu’en ont les hommes les moins instruits. Je suppose trois définitions, ou plutôt trois faits : l’un que {\itshape le gouvernement républicain est celui où le peuple en corps, ou seulement une partie du peuple, à la souveraine puissance ; le monarchique, celui où un seul gouverne, mais par des lois fixes et établies ; au lieu que, dans le despotique, un seul, sans loi et sans règle, entraîne tout par sa volonté et par ses caprices}.\par
Voilà ce que j’appelle la nature de chaque gouvernement. Il faut voir quelles sont les lois qui suivent directement de cette nature, et qui par conséquent sont les premières lois fondamentales.
\subsubsection[{Chapitre II. Du gouvernement républicain et des lois relatives à la démocratie}]{Chapitre II. Du gouvernement républicain et des lois relatives à la démocratie}
\noindent Lorsque, dans la république, le peuple en corps a la souveraine puissance, c’est une démocratie. Lorsque la souveraine puissance est entre les mains d’une partie du peuple, cela s’appelle une aristocratie.\par
Le peuple, dans la démocratie, est, à certains égards, le monarque ; à certains autres, il est le sujet.\par
Il ne peut être monarque que par ses suffrages qui sont ses volontés. La volonté du souverain est le souverain lui-même. Les lois qui établissent le droit de suffrage sont donc fondamentales dans ce gouvernement. En effet, il est aussi important d’y régler comment, par qui, à qui, sur quoi, les suffrages doivent être donnés, qu’il l’est dans une monarchie de savoir quel est le monarque, et de quelle manière il doit gouverner.\par
Libanios\footnote{Déclamations 17 et 18.} dit qu’à Athènes un étranger qui se mêlait dans l’assemblée du peuple, était puni de mort. C’est qu’un tel homme usurpait le droit de souveraineté.\par
Il est essentiel de fixer le nombre des citoyens qui doivent former les assemblées ; sans cela, on pourrait ignorer si le peuple a parlé, ou seulement une partie du peuple. À Lacédémone, il fallait dix mille citoyens. À Rome, née dans la petitesse pour aller à la grandeur ; à Rome, faite pour éprouver toutes les vicissitudes de la fortune ; à Rome, qui avait tantôt presque tous ses citoyens hors de ses murailles, tantôt toute l’Italie et une partie de la terre dans ses murailles, on n’avait point fixé ce nombre\footnote{Voyez les {\itshape Considérations sur les causes de la grandeur des Romains et de leur décadence}, ch. IX.} ; et ce fut une des grandes causes de sa ruine.\par
Le peuple qui a la souveraine puissance doit faire par lui-même tout ce qu’il peut bien faire ; et ce qu’il ne peut pas bien faire, il faut qu’il le fasse par ses ministres.\par
Ses ministres ne sont point à lui s’il ne les nomme : c’est donc une maxime fondamentale de ce gouvernement, que le peuple nomme ses ministres, c’est-à-dire ses magistrats.\par
Il a besoin, comme les monarques, et même plus qu’eux, d’être conduit par un conseil ou sénat. Mais, pour qu’il y ait confiance, il faut qu’il en élise les membres ; soit qu’il les choisisse lui-même, comme à Athènes ; ou par quelque magistrat qu’il a établi pour les élire, comme cela se pratiquait à Rome dans quelques occasions.\par
Le peuple est admirable pour choisir ceux à qui il doit confier quelque partie de son autorité. Il n’a à se déterminer que par des choses qu’il ne peut ignorer, et des faits qui tombent sous les sens. Il sait très bien qu’un homme a été souvent à la guerre, qu’il y a eu tels ou tels succès : il est donc très capable d’élire un général. Il sait qu’un juge est assidu, que beaucoup de gens se retirent de son tribunal contents de lui, qu’on ne l’a pas convaincu de corruption ; en voilà assez pour qu’il élise un préteur. Il a été frappé de la magnificence ou des richesses d’un citoyen ; cela suffit pour qu’il puisse choisir un édile. Toutes ces choses sont des faits dont il s’instruit mieux dans la place publique, qu’un monarque dans son palais. Mais saura-t-il conduire une affaire, connaître les lieux, les occasions, les moments, en profiter ? Non : il ne le saura pas.\par
Si l’on pouvait douter de la capacité naturelle qu’a le peuple pour discerner le mérite, il n’y aurait qu’à jeter les yeux sur cette suite continuelle de choix étonnants que firent les Athéniens et les Romains ; ce qu’on n’attribuera pas sans doute au hasard.\par
On sait qu’à Rome, quoique le peuple se fût donné le droit d’élever aux charges les plébéiens, il ne pouvait se résoudre à les élire ; et quoiqu’à Athènes on pût, par la loi d’Aristide, tirer les magistrats de toutes les classes, il n’arriva jamais, dit Xénophon\footnote{Pp. 691 et 692, édition de Wechelius, de l’an 1596.}, que le bas peuple demandât celles qui pouvaient intéresser son salut ou sa gloire.\par
Comme la plupart des citoyens, qui ont assez de suffisance pour élire, n’en ont pas assez pour être élus ; de même le peuple, qui a assez de capacité pour se faire rendre compte de la gestion des autres, n’est pas propre à gérer par lui-même.\par
Il faut que les affaires aillent, et qu’elles aillent un certain mouvement qui ne soit ni trop lent ni trop vite. Mais le peuple a toujours trop d’action, ou trop peu. Quelquefois avec cent mille bras il renverse tout ; quelquefois avec cent mille pieds il ne va que comme les insectes.\par
Dans l’État populaire, on divise le peuple en de certaines classes. C’est dans la manière de faire cette division que les grands législateurs se sont signalés ; et c’est de là qu’ont toujours dépendu la durée de la démocratie et sa prospérité.\par
Servius Tullius suivit, dans la composition de ses classes, l’esprit de l’aristocratie. Nous voyons dans Tite-Live\footnote{Liv. I.} et dans Denys d’Halicarnasse\footnote{Liv. IV, art. 15 et suiv.}) comment il mit le droit de suffrage entre les mains des principaux citoyens. Il avait divisé le peuple de Rome en cent quatre-vingt-treize centuries, qui formaient six classes. Et mettant les riches, mais en plus petit nombre, dans les premières centuries ; les moins riches, mais en plus grand nombre, dans les suivantes, il jeta toute la foule des indigents dans la dernière : et chaque centurie n’ayant qu’une voix\footnote{Voyez dans les {\itshape Considérations sur les causes de la grandeur des Romains et de leur décadence}, ch. IX, comment cet esprit de Servius Tullius se conserva dans la République.} c’étaient les moyens et les richesses qui donnaient le suffrage, plutôt que les personnes.\par
Solon divisa le peuple d’Athènes en quatre classes. Conduit par l’esprit de la démocratie, il ne les fit pas pour fixer ceux qui devaient élire, mais ceux qui pouvaient être élus : et, laissant à chaque citoyen le droit d’élection, il voulut\footnote{Denys d’Halicarnasse, {\itshape Éloge d’Isocrate}, p. 97, t. II, édition de Wechelius. Pollux, liv. VIII, chap. X, art. 130.} que, dans chacune de ces quatre classes, on pût élire des juges ; mais que ce ne fût que dans les trois premières, où étaient les citoyens aisés, qu’on pût prendre les magistrats.\par
Comme la division de ceux qui ont droit de suffrage est, dans la république, une loi fondamentale, la manière de le donner est une autre loi fondamentale.\par
Le suffrage par le sort est de la nature de la démocratie ; le suffrage par choix est de celle de l’aristocratie.\par
Le sort est une façon d’élire qui n’afflige personne ; il laisse à chaque citoyen une espérance raisonnable de servir sa patrie.\par
Mais, comme il est défectueux par lui-même, c’est à le régler et à le corriger que les grands législateurs se sont surpassés.\par
Solon établit à Athènes que l’on nommerait par choix à tous les emplois militaires, et que les sénateurs et les juges seraient élus par le sort.\par
Il voulut que l’on donnât par choix les magistratures civiles qui exigeaient une grande dépense, et que les autres fussent données par le sort.\par
Mais, pour corriger le sort, il régla qu’on ne pourrait élire que dans le nombre de ceux qui se présenteraient ; que celui qui aurait été élu serait examiné par des juges\footnote{Voy. l’oraison de Démosthène, {\itshape De falsa legatione}, et l’oraison contre Timarque.}, et que chacun pourrait l’accuser d’en être indigne\footnote{On tirait même pour chaque place deux billets : l’un qui donnait la place, l’autre qui nommait celui qui devait succéder, en cas que le premier fût rejeté.} : cela tenait en même temps du sort et du choix. Quand on avait fini le temps de sa magistrature, il fallait essuyer un autre jugement sur la manière dont on s’était comporté. Les gens sans capacité devaient avoir bien de la répugnance à donner leur nom pour être tirés au sort.\par
La loi qui fixe la manière de donner les billets de suffrage est encore une fois fondamentale dans la démocratie. C’est une grande question si les suffrages doivent être publics ou secrets. Cicéron\footnote{Liv. I et III des {\itshape Lois}.} écrit que les lois\footnote{Elles s’appelaient lois tabulaires. On donnait à chaque citoyen deux tables : la première marquée d’un A, pour dire {\itshape antiquo} ; l’autre d’un U et d’un R, {\itshape uti rogas}.} qui les rendirent secrets dans les derniers temps de la République romaine, furent une des grandes causes de sa chute. Comme ceci se pratique diversement dans différentes républiques, voici, je crois, ce qu’il en faut penser.\par
Sans doute que, lorsque le peuple donne ses suffrages, ils doivent être publics\footnote{À Athènes, on levait les mains.} \footnote{Comme à Venise.}; et ceci doit être regardé comme une loi fondamentale de la démocratie. Il faut que le petit peuple soit éclairé par les principaux, et contenu par la gravité de certains personnages. Ainsi, dans la République romaine, en rendant les suffrages secrets, on détruisit tout ; il ne fut plus possible d’éclairer une populace qui se perdait. Mais lorsque, dans une aristocratie, le corps des nobles donne les suffrages, ou dans une démocratie, le sénat\footnote{Les trente tyrans d’Athènes voulurent que les suffrages des aréopagites fussent publics, pour les diriger à leur fantaisie : Lysias, {\itshape Orat. contra Agorat.}, cap. VIII.} ; comme il n’est là question que de prévenir les brigues, les suffrages ne sauraient être trop secrets.\par
La brigue est dangereuse dans un sénat ; elle est dangereuse dans un corps de nobles : elle ne l’est pas dans le peuple, dont la nature est d’agir par passion. Dans les États où il n’a point de part au gouvernement, il s’échauffera pour un acteur, comme il aurait fait pour les affaires. Le malheur d’une république, c’est lorsqu’il n’y a plus de brigues ; et cela arrive lorsqu’on a corrompu le peuple à prix d’argent : il devient de sang-froid, il s’affectionne à l’argent, mais il ne s’affectionne plus aux affaires : sans souci du gouvernement et de ce qu’on y propose, il attend tranquillement son salaire.\par
C’est encore une loi fondamentale de la démocratie, que le peuple seul fasse des lois. Il y a pourtant mille occasions où il est nécessaire que le sénat puisse statuer ; il est même souvent à propos d’essayer une loi avant de l’établir. La constitution de Rome et celle d’Athènes étaient très sages. Les arrêts du sénat\footnote{Voyez Denys d’Halicarnasse, liv. IV et IX.} avaient force de loi pendant un an ; ils ne devenaient perpétuels que par la volonté du peuple.
\subsubsection[{Chapitre III. Des lois relatives à la nature de l’aristocratie}]{Chapitre III. Des lois relatives à la nature de l’aristocratie}
\noindent Dans l’aristocratie, la souveraine puissance est entre les mains d’un certain nombre de personnes. Ce sont elles qui font les lois et qui les font exécuter ; et le reste du peuple n’est tout au plus à leur égard que, comme dans une monarchie, les sujets sont à l’égard du monarque.\par
On n’y doit point donner le suffrage par sort ; on n’en aurait que les inconvénients. En effet, dans un gouvernement qui a déjà établi les distinctions les plus affligeantes, quand on serait choisi par le sort, on n’en serait pas moins odieux : c’est le noble qu’on envie, et non pas le magistrat.\par
Lorsque les nobles sont en grand nombre, il faut un sénat qui règle les affaires que le corps des nobles ne saurait décider, et qui prépare celles dont il décide. Dans ce cas, on peut dire que l’aristocratie est en quelque sorte dans le sénat, la démocratie dans le corps des nobles, et que le peuple n’est rien.\par
Ce sera une chose très heureuse dans l’aristocratie si, par quelque voie indirecte, on fait sortir le peuple de son anéantissement : ainsi à Gênes la banque de Saint-Georges, qui est administrée en grande partie par les principaux du peuple\footnote{Voyez M. Addisson, {\itshape Voyages d’Italie}, p. 16.}, donne à celui-ci une certaine influence dans le gouvernement, qui en fait toute la prospérité.\par
Les sénateurs ne doivent point avoir le droit de remplacer ceux qui manquent dans le sénat ; rien ne serait plus capable de perpétuer les abus. À Rome, qui fut dans les premiers temps une espèce d’aristocratie, le sénat ne se suppléait pas lui-même ; les sénateurs nouveaux étaient nommés\footnote{Ils le furent d’abord par les consuls.} par les censeurs.\par
Une autorité exorbitante, donnée tout à coup à un citoyen dans une république, forme une monarchie, ou plus qu’une monarchie. Dans celles-ci les lois ont pourvu à la constitution, ou s’y sont accommodées ; le principe du gouvernement arrête le monarque ; mais, dans une république où un citoyen se fait donner\footnote{C’est ce qui renversa la république romaine. Voyez les {\itshape Considérations sur les causes de la grandeur des Romains et de leur décadence}.} un pouvoir exorbitant, l’abus de ce pouvoir est plus grand, parce que les lois, qui ne l’ont point prévu, n’ont rien fait pour l’arrêter.\par
L’exception à cette règle est lorsque la constitution de l’État est telle qu’il a besoin d’une magistrature qui ait un pouvoir exorbitant. Telle était Rome avec ses dictateurs, telle est Venise avec ses inquisiteurs d’État ; ce sont des magistratures terribles, qui ramènent violemment l’État à la liberté. Mais, d’où vient que ces magistratures se trouvent si différentes dans ces deux républiques ? C’est que Rome défendait les restes de son aristocratie contre le peuple ; au lieu que Venise se sert de ses inquisiteurs d’État pour maintenir son aristocratie contre les nobles. De là il suivait qu’à Rome la dictature ne devait durer que peu de temps, parce que le peuple agit par sa fougue, et non pas par ses desseins. Il fallait que cette magistrature s’exerçât avec éclat, parce qu’il s’agissait d’intimider le peuple, et non pas de le punir ; que le dictateur ne fût créé que pour une seule affaire, et n’eût une autorité sans bornes qu’à raison de cette affaire, parce qu’il était toujours créé pour un cas imprévu. À Venise, au contraire, il faut une magistrature permanente : c’est là que les desseins peuvent être commencés, suivis, suspendus, repris ; que l’ambition d’un seul devient celle d’une famille, et l’ambition d’une famille celle de plusieurs. On a besoin d’une magistrature cachée, parce que les crimes qu’elle punit, toujours profonds, se forment dans le secret et dans le silence. Cette magistrature doit avoir une inquisition générale, parce qu’elle n’a pas à arrêter les maux que l’on connaît, mais à prévenir même ceux que l’on ne connaît pas. Enfin, cette dernière est établie pour venger les crimes qu’elle soupçonne ; et la première employait plus les menaces que les punitions pour les crimes, même avoués par leurs auteurs.\par
Dans toute magistrature, il faut compenser la grandeur de la puissance par la brièveté de sa durée. Un an est le temps que la plupart des législateurs ont fixé ; un temps plus long serait dangereux, un plus court serait contre la nature de la chose. Qui est-ce qui voudrait gouverner ainsi ses affaires domestiques ? À Raguse\footnote{{\itshape Voyages} de Tournefort.}, le chef de la république change tous les mois, les autres officiers toutes les semaines, le gouverneur du château tous les jours. Ceci ne peut avoir lieu que dans une petite république\footnote{À Lucques, les magistrats ne sont établis que pour deux {\itshape mois.}} environnée de puissances formidables, qui corrompraient aisément de petits magistrats.\par
La meilleure aristocratie est celle où la partie du peuple qui n’a point de part à la puissance, est si petite et si pauvre, que la partie dominante n’a aucun intérêt à l’opprimer. Ainsi, quand Antipater\footnote{Diodore, liv. XVIII, p. 601, édition de Rhodoman.} établit à Athènes que ceux qui n’auraient pas deux mille drachmes seraient exclus du droit de suffrage, il forma la meilleure aristocratie qui fût possible ; parce que ce cens était si petit qu’il n’excluait que peu de gens, et personne qui eût quelque considération dans la cité.\par
Les familles aristocratiques doivent donc être peuple autant qu’il est possible. Plus une aristocratie approchera de la démocratie, plus elle sera parfaite ; et elle le deviendra moins, à mesure qu’elle approchera de la monarchie.\par
La plus imparfaite de toutes est celle où la partie du peuple qui obéit est dans l’esclavage civil de celle qui commande, comme l’aristocratie de Pologne, où les paysans sont esclaves de la noblesse.
\subsubsection[{Chapitre IV. Des lois dans leur rapport avec la nature du gouvernement monarchique}]{Chapitre IV. Des lois dans leur rapport avec la nature du gouvernement monarchique}
\noindent Les pouvoirs intermédiaires subordonnés et dépendants constituent la nature du gouvernement monarchique, c’est-à-dire de celui où un seul gouverne par des lois fondamentales. J’ai dit les pouvoirs intermédiaires, subordonnés et dépendants : en effet, dans la monarchie, le prince est la source de tout pouvoir politique et civil. Ces lois fondamentales supposent nécessairement des canaux moyens par où coule la puissance : car, s’il n’y a dans l’État que la volonté momentanée et capricieuse d’un seul, rien ne peut être fixe, et par conséquent aucune loi fondamentale.\par
Le pouvoir intermédiaire subordonné le plus naturel est celui de la noblesse. Elle entre en quelque façon dans l’essence de la monarchie, dont la maxime fondamentale est : {\itshape point de monarque, point de noblesse ; point de noblesse, point de monarque} ; {\itshape mais} on {\itshape a} un despote.\par
Il y a des gens qui avaient imaginé, dans quelques États en Europe, d’abolir toutes les justices des seigneurs. Ils ne voyaient pas qu’ils voulaient faire ce que le parlement d’Angleterre a fait. Abolissez dans une monarchie les prérogatives des seigneurs, du clergé, de la noblesse et des villes ; vous aurez bientôt un État populaire, ou bien un État despotique.\par
Les tribunaux d’un grand État en Europe frappent sans cesse, depuis plusieurs siècles, sur la juridiction patrimoniale des seigneurs, et sur l’ecclésiastique. Nous ne voulons pas censurer des magistrats si sages ; mais nous laissons à décider jusqu’à quel point la constitution en peut être changée.\par
Je ne suis point entêté des privilèges des ecclésiastiques : mais je voudrais qu’on fixât bien une fois leur juridiction. Il n’est point question de savoir si on a eu raison de l’établir : mais si elle est établie ; si elle fait une partie des lois du pays, et si elle y est partout relative ; si, entre deux pouvoirs que l’on reconnaît indépendants, les conditions ne doivent pas être réciproques ; et s’il n’est pas égal à un bon sujet de défendre la justice du prince, ou les limites qu’elle s’est de tout temps prescrites.\par
Autant que le pouvoir du clergé est dangereux dans une république, autant est-il convenable dans une monarchie, surtout dans celles qui vont au despotisme. Où en seraient l’Espagne et le Portugal depuis la perte de leurs lois, sans ce pouvoir qui arrête seul la puissance arbitraire ? Barrière toujours bonne, lorsqu’il n’y en a point d’autre : car, comme le despotisme cause à la nature humaine des maux effroyables, le mal même qui le limite est un bien.\par
Comme la mer, qui semble vouloir couvrir toute la terre, est arrêtée par les herbes et les moindres graviers qui se trouvent sur le rivage ; ainsi les monarques, dont le pouvoir paraît sans bornes, s’arrêtent par les plus petits obstacles, et soumettent leur fierté naturelle à la plainte et à la prière.\par
Les Anglais, pour favoriser la liberté, ont ôté toutes les puissances intermédiaires qui formaient leur monarchie. Ils ont bien raison de conserver cette liberté ; s’ils venaient à la perdre, ils seraient un des peuples les plus esclaves de la terre.\par
M. Law, par une ignorance égale de la constitution républicaine et de la monarchique, fut un des plus grands promoteurs du despotisme que l’on eût encore vu en Europe. Outre les changements qu’il fit, si brusques, si inusités, si inouïs, il voulait ôter les rangs intermédiaires, et anéantir les corps politiques : il dissolvait\footnote{Ferdinand, roi d’Aragon, se fit grand maître des ordres, et cela seul altéra la constitution.} la monarchie par ses chimériques remboursements, et semblait vouloir racheter la constitution même.\par
Il ne suffit pas qu’il y ait, dans une monarchie, des rangs intermédiaires ; il faut encore un dépôt de lois. Ce dépôt ne peut être que dans les corps politiques, qui annoncent les lois lorsqu’elles sont faites et les rappellent lorsqu’on les oublie. L’ignorance naturelle à la noblesse, son inattention, son mépris pour le gouvernement civil, exigent qu’il y ait un corps qui fasse sans cesse sortir les lois de la poussière où elles seraient ensevelies. Le Conseil du prince n’est pas un dépôt convenable. Il est, par sa nature, le dépôt de la volonté momentanée du prince qui exécute, et non pas le dépôt des lois fondamentales. De plus, le Conseil du monarque change sans cesse ; il n’est point permanent ; il ne saurait être nombreux ; il n’a point à un assez haut degré la confiance du peuple : il n’est donc pas en état de l’éclairer dans les temps difficiles, ni de le ramener à l’obéissance.\par
Dans les États despotiques, où il n’y a point de lois fondamentales, il n’y a pas non plus de dépôt de lois. De là vient que, dans ces pays, la religion a ordinairement tant de force ; c’est qu’elle forme une espèce de dépôt et de permanence : et, si ce n’est pas la religion, ce sont les coutumes qu’on y vénère, au lieu des lois.
\subsubsection[{Chapitre V. Des lois relatives à la nature de l’état despotique}]{Chapitre V. Des lois relatives à la nature de l’état despotique}
\noindent Il résulte de la nature du pouvoir despotique, que l’homme seul qui l’exerce le fasse de même exercer par un seul. Un homme à qui ses cinq sens disent sans cesse qu’il est tout, et que les autres ne sont rien, est naturellement paresseux, ignorant, voluptueux. Il abandonne donc les affaires. Mais, s’il les confiait à plusieurs, il y aurait des disputes entre eux ; on ferait des brigues pour être le premier esclave ; le prince serait obligé de rentrer dans l’administration. Il est donc plus simple qu’il l’abandonne à un vizir\footnote{Les rois d’Orient ont toujours des vizirs, dit M. Chardin.} qui aura d’abord la même puissance que lui. L’établissement d’un vizir est, dans cet État, une loi fondamentale.\par
On dit qu’un pape, à son élection, pénétré de son incapacité, fit d’abord des difficultés infinies. Il accepta enfin et livra à son neveu toutes les affaires. il était dans l’admiration, et disait : « Je n’aurais jamais cru que cela eût été si aisé. » Il en est de même des princes d’Orient. Lorsque de cette prison, où des eunuques leur ont affaibli le cœur et l’esprit, et souvent leur ont laissé ignorer leur état même, on les tire pour les placer sur le trône, ils sont d’abord étonnés : mais, quand ils ont fait un vizir, et que dans leur sérail ils se sont livrés aux passions les plus brutales ; lorsqu’au milieu d’une cour abattue ils ont suivi leurs caprices les plus stupides, ils n’auraient jamais cru que cela eût été si aisé.\par
Plus l’empire est étendu, plus le sérail s’agrandit, et plus, par conséquent, le prince est enivré de plaisirs. Ainsi, dans ces États, plus le prince a de peuples à gouverner, moins il pense au gouvernement ; plus les affaires y sont grandes, et moins on y délibère sur les affaires.
\subsection[{Livre troisième. Des principes des trois gouvernements}]{Livre troisième. Des principes des trois gouvernements}
\subsubsection[{Chapitre I. Différence de la nature du gouvernement et de son principe}]{Chapitre I. Différence de la nature du gouvernement et de son principe}
\noindent Après avoir examiné quelles sont les lois relatives à la nature de chaque gouvernement, il faut voir celles qui le sont à son principe.\par
Il y a cette différence\footnote{Cette distinction est très importante, et j’en tirerai bien des conséquences ; elle est la clef d’une infinité de lois.} entre la nature du gouvernement et son principe, que sa nature est ce qui le fait être tel, et son principe ce qui le fait agir. L’une est sa structure particulière, et l’autre les passions humaines qui le font mouvoir.\par
Or les lois ne doivent pas être moins relatives au principe de chaque gouvernement qu’à sa nature. Il faut donc chercher quel est ce principe. C’est ce que je vais faire dans ce livre-ci.
\subsubsection[{Chapitre II. Du principe des divers gouvernements}]{Chapitre II. Du principe des divers gouvernements}
\noindent J’ai dit que la nature du gouvernement républicain est que le peuple en corps, ou de certaines familles, y aient la souveraine puissance ; celle du gouvernement monarchique, que le prince y ait la souveraine puissance, mais qu’il l’exerce selon des lois établies ; celle du gouvernement despotique, qu’un seul y gouverne selon ses volontés et ses caprices. Il ne m’en faut pas davantage pour trouver leurs trois principes ; ils en dérivent naturellement. Je commencerai par le gouvernement républicain, et je parlerai d’abord du démocratique.
\subsubsection[{Chapitre III. Du principe de la démocratie}]{Chapitre III. Du principe de la démocratie}
\noindent Il ne faut pas beaucoup de probité pour qu’un gouvernement monarchique ou un gouvernement despotique se maintiennent ou se soutiennent. La force des lois dans l’un, le bras du prince toujours levé dans l’autre, règlent ou contiennent tout.\par
Mais, dans un État populaire, il faut un ressort de plus, qui est la VERTU.\par
Ce que je dis est confirmé par le corps entier de l’histoire, et est très conforme à la nature des choses. Car il est clair que dans une monarchie, où celui qui fait exécuter les lois se juge au-dessus des lois, on a besoin de moins de vertu que dans un gouvernement populaire, où celui qui fait exécuter les lois sent qu’il y est soumis lui-même, et qu’il en portera le poids.\par
Il est clair encore que le monarque qui, par mauvais conseil ou par négligence, cesse de faire exécuter les lois, peut aisément réparer le mal : il n’a qu’à changer de conseil, ou se corriger de cette négligence même. Mais lorsque, dans un gouvernement populaire, les lois ont cessé d’être exécutées, comme cela ne peut venir que de la corruption de la république, l’État est déjà perdu.\par
Ce fut un assez beau spectacle, dans le siècle passé, de voir les efforts impuissants des Anglais pour établir parmi eux la démocratie. Comme ceux qui avaient part aux affaires n’avaient point de vertu, que leur ambition était irritée par le succès de celui qui avait le plus osé\footnote{Cromwell.}, que l’esprit d’une faction n’était réprimé que par l’esprit d’une autre, le gouvernement changeait sans cesse ; le peuple étonné cherchait la démocratie et ne la trouvait nulle part. Enfin, après bien des mouvements, des chocs et des secousses, il fallut se reposer dans le gouvernement même qu’on avait proscrit.\par
Quand Sylla voulut rendre à Rome la liberté, elle ne put plus la recevoir ; elle n’avait plus qu’un faible reste de vertu, et, comme elle en eut toujours moins, au lieu de se réveiller après César, Tibère, Caïus, Claude, Néron, Domitien, elle fut toujours plus esclave ; tous les coups portèrent sur les tyrans, aucun sur la tyrannie.\par
Les politiques grecs, qui vivaient dans le gouvernement populaire, ne reconnaissaient d’autre force qui pût les soutenir que celle de la vertu. Ceux d’aujourd’hui ne nous parlent que de manufactures, de commerce, de finances, de richesses et de luxe même.\par
Lorsque cette vertu cesse, l’ambition entre dans les cœurs qui peuvent la recevoir, et l’avarice entre dans tous. Les désirs changent d’objets : ce qu’on aimait, on ne l’aime plus ; on était libre avec les lois, on veut être libre contre elles. Chaque citoyen est comme un esclave échappé de la maison de son maître ; ce qui était {\itshape maxime}, on l’appelle {\itshape rigueur} ; ce qui était {\itshape règle}, on l’appelle {\itshape gêne} ; ce qui y était {\itshape attention}, on l’appelle {\itshape crainte}. C’est la frugalité qui y est l’avarice, et non pas le désir d’avoir. Autrefois le bien des particuliers faisait le trésor public ; mais pour lors le trésor public devient le patrimoine des particuliers. La république est une dépouille ; et sa force n’est plus que le pouvoir de quelques citoyens et la licence de tous.\par
Athènes eut dans son sein les mêmes forces pendant qu’elle domina avec tant de gloire, et pendant qu’elle servit avec tant de honte. Elle avait vingt mille citoyens\footnote{Plutarque, {\itshape in Pericle} ; Platon, {\itshape in Critia}.} lorsqu’elle défendit les Grecs contre les Perses, qu’elle disputa l’empire à Lacédémone, et qu’elle attaqua la Sicile, Elle en avait vingt mille lorsque Démétrius de Phalère les dénombra\footnote{Il s’y trouva vingt et un mille citoyens, dix mille étrangers, quatre cent mille esclaves. Voyez Athénée, liv. VI.} comme dans un marché l’on compte les esclaves. Quand Philippe osa dominer dans la Grèce, quand il parut aux portes d’Athènes\footnote{Elle avait vingt mille citoyens. Voyez Démosthène, {\itshape in Aristog.}}, elle n’avait encore perdu que le temps. On peut voir dans Démosthène quelle peine il fallut pour la réveiller : on y craignait Philippe, non pas comme l’ennemi de la liberté, mais des plaisirs\footnote{Ils avaient fait une loi pour punir de mort celui qui proposerait de convertir aux usages de la guerre l’argent destiné pour les théâtres.}. Cette ville, qui avait résisté à tant de défaites, qu’on avait vue renaître après ses destructions, fut vaincue à Chéronée, et le fut pour toujours. Qu’importe que Philippe renvoie tous les prisonniers ? Il ne renvoie pas des hommes. Il était toujours aussi aisé de triompher des forces d’Athènes qu’il était difficile de triompher de sa vertu.\par
Comment Carthage aurait-elle pu se soutenir ? Lorsque Annibal, devenu préteur, voulut empêcher les magistrats de piller la république, n’allèrent-ils pas l’accuser devant les Romains ? Malheureux, qui voulaient être citoyens sans qu’il y eût de cité, et tenir leurs richesses de la main de leurs destructeurs ! Bientôt Rome leur demanda pour otages trois cents de leurs principaux citoyens ; elle se fit livrer les armes et les vaisseaux, et ensuite leur déclara la guerre. Par les choses que fit le désespoir dans Carthage désarmée\footnote{Cette guerre dura trois ans.} on peut juger de ce qu’elle aurait pu faire avec sa vertu, lorsqu’elle avait ses forces.
\subsubsection[{Chapitre IV. Du principe de l’aristocratie}]{Chapitre IV. Du principe de l’aristocratie}
\noindent Comme il faut de la vertu dans le gouvernement populaire, il en faut aussi dans l’aristocratique. Il est vrai qu’elle n’y est pas si absolument requise.\par
Le peuple, qui est à l’égard des nobles ce que les sujets sont à l’égard du monarque, est contenu par leurs lois. Il a donc moins besoin de vertu que le peuple de la démocratie. Mais comment les nobles seront-ils contenus ? Ceux qui doivent faire exécuter les lois contre leurs collègues sentiront d’abord qu’ils agissent contre eux-mêmes. Il faut donc de la vertu dans ce corps, par la nature de la constitution.\par
Le gouvernement aristocratique a par lui-même une certaine force que la démocratie n’a pas. Les nobles y forment un corps, qui, par sa prérogative et pour son intérêt particulier, réprime le peuple : il suffit qu’il y ait des lois, pour qu’à cet égard elles soient exécutées.\par
Mais autant qu’il est aisé à ce corps de réprimer les autres, autant est-il difficile qu’il se réprime lui-même\footnote{Les crimes publics y pourront être punis, parce que c’est l’affaire de tous ; les crimes particuliers n’y seront pas punis, parce que l’affaire de tous est de ne les pas punir.}. Telle est la nature de cette constitution, qu’il semble qu’elle mette les mêmes gens sous la puissance des lois, et qu’elle les en retire.\par
Or, un corps pareil ne peut se réprimer que de deux manières : ou par une grande vertu, qui fait que les nobles se trouvent en quelque façon égaux à leur peuple, ce qui peut former une grande république ; ou par une vertu moindre, qui est une certaine modération qui rend les nobles au moins égaux à eux-mêmes, ce qui fait leur conservation.\par
La {\itshape modération} est donc l’âme de ces gouvernements. J’entends celle qui est fondée sur la vertu, non pas celle qui vient d’une lâcheté et d’une paresse de l’âme.\par
  \textbf{Chapitre V. {\itshape Que la vertu n’est point le principe du gouvernement monarchique} } \par
Dans les monarchies, la politique fait faire les grandes choses avec le moins de vertu qu’elle peut ; comme, dans les plus belles machines, l’art emploie aussi peu de mouvements, de forces et de roues qu’il est possible.\par
L’État subsiste indépendamment de l’amour pour la patrie, du désir de la vraie gloire, du renoncement à soi-même, du sacrifice de ses plus chers intérêts, et de toutes ces vertus héroïques que nous trouvons dans les anciens, et dont nous avons seulement entendu parler.\par
Les lois y tiennent la place de toutes ces vertus, dont on n’a aucun besoin ; l’État vous en dispense : une action qui se fait sans bruit, y est en quelque façon sans conséquence.\par
Quoique tous les crimes soient publics par leur nature, on distingue pourtant les crimes véritablement publics d’avec les crimes privés, ainsi appelés, parce qu’ils offensent plus un particulier, que la société entière.\par
Or, dans les républiques, les crimes privés sont plus publics, c’est-à-dire choquent plus la constitution de l’État, que les particuliers ; et, dans les monarchies, les crimes publics sont plus privés, c’est-à-dire choquent plus les fortunes particulières que la constitution de l’État même.\par
Je supplie qu’on ne s’offense pas de ce que j’ai dit ; je parle après toutes les histoires. Je sais très bien qu’il n’est pas rare qu’il y ait des princes vertueux ; mais je dis que, dans une monarchie, il est très difficile que le peuple le soit\footnote{Je parle ici de la vertu politique, qui est la vertu morale, dans le sens qu’elle se dirige au bien général, fort peu des vertus morales particulières, et point du tout de cette vertu qui a du rapport aux vérités révélées. On verra bien ceci au liv. V, chap. II.}.\par
Qu’on lise ce que les historiens de tous les temps ont dit sur la cour des monarques ; qu’on se rappelle les conversations des hommes de tous les pays sur le misérable caractère des courtisans : ce ne sont point des choses de spéculation, mais d’une triste expérience.\par
L’ambition dans l’oisiveté, la bassesse dans l’orgueil, le désir de s’enrichir sans travail, l’aversion pour la vérité, la flatterie, la trahison, la perfidie, l’abandon de tous ses engagements, le mépris des devoirs du citoyen, la crainte de la vertu du prince, l’espérance de ses faiblesses, et plus que tout cela, le ridicule perpétuel jeté sur la vertu, forment, je crois, le caractère du plus grand nombre des courtisans, marqué dans tous les lieux et dans tous les temps. Or il est très malaisé que la plupart des principaux d’un État soient malhonnêtes gens, et que les inférieurs soient gens de bien ; que ceux-là soient trompeurs, et que ceux-ci consentent à n’être que dupes.\par
Que si, dans le peuple, il se trouve quelque malheureux honnête homme\footnote{Entendez ceci dans le sens de la note précédente.}, le cardinal de Richelieu, dans son {\itshape Testament politique}, insinue qu’un monarque doit se garder de s’en servir\footnote{Il ne faut pas, y est-il dit, se servir des gens de bas lieu ; ils sont trop austères et trop difficiles.}. Tant il est vrai que la vertu n’est pas le ressort de ce gouvernement ! Certainement elle n’en est point exclue ; mais elle n’en est pas le ressort.
\subsubsection[{Chapitre VI. Comment on supplée à la vertu dans le gouvernement monarchique}]{Chapitre VI. Comment on supplée à la vertu dans le gouvernement monarchique}
\noindent Je me hâte, et je marche à grands pas, afin qu’on ne croie pas que je fasse une satire du gouvernement monarchique. Non ; s’il manque d’un ressort, il en a un autre : L’HONNEUR, c’est-à-dire le préjugé de chaque personne et de chaque condition, prend la place de la vertu politique dont j’ai parlé, et la représente partout. Il y peut inspirer les plus belles actions ; il peut, joint à la force des lois, conduire au but du gouvernement comme la vertu même.\par
Ainsi, dans les monarchies bien réglées, tout le monde sera à peu près bon citoyen, et on trouvera rarement quelqu’un qui soit homme de bien ; car, pour être homme de bien\footnote{Ce mot, {\itshape homme de bien}, ne s’entend ici que dans un sens politique.}, il faut avoir intention de l’être\footnote{Voyez la note a de la page 119.}, et aimer l’État moins pour soi que pour lui-même.
\subsubsection[{Chapitre VII. Du principe de la monarchie}]{Chapitre VII. Du principe de la monarchie}
\noindent Le gouvernement monarchique suppose, comme nous avons dit, des prééminences, des rangs, et même une noblesse d’origine. La nature de {\itshape l’honneur} est de demander des préférences et des distinctions ; il est donc, par la chose même, placé dans ce gouvernement.\par
L’ambition est pernicieuse dans une république. Elle a de bons effets dans la monarchie ; elle donne la vie à ce gouvernement ; et on y a cet avantage, qu’elle n’y est pas dangereuse, parce qu’elle y peut être sans cesse réprimée.\par
Vous diriez qu’il en est comme du système de l’univers, où il y a une force qui éloigne sans cesse du centre tous les corps, et une force de pesanteur qui les y ramène. L’honneur fait mouvoir toutes les parties du corps politique ; il les lie par son action même ; et il se trouve que chacun va au bien commun, croyant aller à ses intérêts particuliers.\par
Il est vrai que, philosophiquement parlant, c’est un honneur faux qui conduit toutes les parties de l’État ; mais cet honneur faux est aussi utile au public, que le vrai le serait aux particuliers qui pourraient l’avoir.\par
Et n’est-ce pas beaucoup d’obliger les hommes à faire toutes les actions difficiles, et qui demandent de la force, sans autre récompense que le bruit de ces actions ?
\subsubsection[{Chapitre VIII. Que l’honneur n’est point le principe des états despotiques}]{Chapitre VIII. Que l’honneur n’est point le principe des états despotiques}
\noindent Ce n’est point {\itshape l’honneur} qui est le principe des États despotiques : les hommes y étant tous égaux, on n’y peut se préférer aux autres ; les hommes y étant tous esclaves, on n’y peut se préférer à rien.\par
De plus, comme l’honneur a ses lois et ses règles, et qu’il ne saurait plier ; qu’il dépend bien de son propre caprice, et non pas de celui d’un autre, il ne peut se trouver que dans des États où la constitution est fixe, et qui ont des lois certaines.\par
Comment serait-il souffert chez le despote ? Il fait gloire de mépriser la vie, et le despote n’a de force que parce qu’il peut l’ôter. Comment pourrait-il souffrir le despote ? Il a des règles suivies et des caprices soutenus ; le despote n’a aucune règle, et ses caprices détruisent tous les autres.\par
L’honneur, inconnu aux États despotiques, où même souvent on n’a pas de mot pour l’exprimer\footnote{Voyez Perry, p. 447.}, règne dans les monarchies ; il y donne la vie à tout le corps politique, aux lois et aux vertus même.
\subsubsection[{Chapitre IX. Du principe du gouvernement despotique}]{Chapitre IX. Du principe du gouvernement despotique}
\noindent Comme il faut de la vertu dans une république, et dans une monarchie, de l’honneur, il faut de la CRAINTE dans un gouvernement despotique : pour la vertu, elle n’y est point nécessaire, et l’honneur y serait dangereux.\par
Le pouvoir immense du prince y passe tout entier à ceux à qui il le confie. Des gens capables de s’estimer beaucoup eux-mêmes seraient en état d’y faire des révolutions. Il faut donc que la crainte y abatte tous les courages, et y éteigne jusqu’au moindre sentiment d’ambition.\par
Un gouvernement modéré peut, tant qu’il veut, et sans péril, relâcher ses ressorts. Il se maintient par ses lois et par sa force même. Mais lorsque, dans le gouvernement despotique, le prince cesse un moment de lever le bras ; quand il ne peut pas anéantir à l’instant ceux qui ont les premières places\footnote{Comme il arrive souvent dans l’aristocratie militaire.}, tout est perdu : car le ressort du gouvernement, qui est la crainte, n’y étant plus, le peuple n’a plus de protecteur.\par
C’est apparemment dans ce sens que des cadis ont soutenu que le grand seigneur n’était point obligé de tenir sa parole ou son serment, lorsqu’il bornait par là son autorité\footnote{Ricaut, De l’Empire ottoman.}.\par
Il faut que le peuple soit jugé par les lois, et les grands par la fantaisie du prince ; que la tête du dernier sujet soit en sûreté, et celle des bachas toujours exposée. On ne peut parler sans frémir de ces gouvernements monstrueux. Le sophi de Perse, détrôné de nos jours par Mirivéis, vit le gouvernement périr avant la conquête, parce qu’il n’avait pas versé assez de sang\footnote{Voyez l’histoire de cette révolution, par le père Du Cerceau.}.\par
L’histoire nous dit que les horribles cruautés de Domitien effrayèrent les gouverneurs, au point que le peuple se rétablit un peu sous son règne\footnote{Son gouvernement était militaire ; ce qui est une des espèces du gouvernement despotique.}. C’est ainsi qu’un torrent, qui ravage tout d’un côté, laisse de l’autre des campagnes où l’œil voit de loin quelques prairies.
\subsubsection[{Chapitre X. Différence de l’obéissance dans les gouvernements modérés, et dans les gouvernements despotiques}]{Chapitre X. Différence de l’obéissance dans les gouvernements modérés \\
et dans les gouvernements despotiques}
\noindent Dans les États despotiques la nature du gouvernement demande une obéissance extrême ; et la volonté du prince, une fois connue, doit avoir aussi infailliblement son effet qu’une boule jetée contre une autre doit avoir le sien.\par
Il n’y a point de tempérament, de modifications, d’accommodements, de termes, d’équivalents, de pourparlers, de remontrances ; rien d’égal ou de meilleur à proposer. L’homme est une créature qui obéit à une créature qui veut.\par
On n’y peut pas plus représenter ses craintes sur un événement futur, qu’excuser ses mauvais succès sur le caprice de la fortune. Le partage des hommes, comme des bêtes, y est l’instinct, l’obéissance, le châtiment.\par
Il ne sert de rien d’opposer les sentiments naturels, le respect pour un père, la tendresse pour ses enfants et ses femmes, les lois de l’honneur, l’état de sa santé ; on a reçu l’ordre, et cela suffit.\par
En Perse, lorsque le roi a condamné quelqu’un, on ne peut plus lui en parler, ni demander grâce. S’il était ivre ou hors de sens, il faudrait que l’arrêt s’exécutât tout de même\footnote{Voyez Chardin.} ; sans cela, il se contredirait, et la loi ne peut se contredire. Cette manière de penser y a été de tout temps : l’ordre que donna Assuérus d’exterminer les Juifs ne pouvant être révoqué, on prit le parti de leur donner la permission de se défendre.\par
Il y a pourtant une chose que l’on peut quelquefois opposer à la volonté du prince\footnote{{\itshape Ibid.}} : c’est la religion. On abandonnera son père, on le tuera même, si le prince l’ordonne : mais on ne boira pas de vin, s’il le veut et s’il l’ordonne. Les lois de la religion sont d’un précepte supérieur, parce qu’elles sont données sur la tête du prince comme sur celle des sujets. Mais, quant au droit naturel, il n’en est pas de même ; le prince est supposé n’être plus un homme.\par
Dans les États monarchiques et modérés la puissance est bornée par ce qui en est le ressort ; je veux dire l’honneur, qui règne, comme un monarque, sur le prince et sur le peuple. On n’ira point lui alléguer les lois de la religion. Un courtisan se croirait ridicule. On lui alléguera sans cesse celles de l’honneur. De là résultent des modifications nécessaires dans l’obéissance ; l’honneur est naturellement sujet à des bizarreries, et l’obéissance les suivra toutes.\par
Quoique la manière d’obéir soit différente dans ces deux gouvernements, le pouvoir est pourtant le même. De quelque côté que le monarque se tourne, il emporte et précipite la balance, et est obéi. Toute la différence est que, dans la monarchie, le prince a des lumières, et que les ministres y sont infiniment plus habiles et plus rompus aux affaires que dans l’État despotique.
\subsubsection[{Chapitre XI. Réflexions sur tout ceci}]{Chapitre XI. Réflexions sur tout ceci}
\noindent Tels sont les principes des trois gouvernements : ce qui ne signifie pas que, dans une certaine république, on soit vertueux ; mais qu’on devrait l’être. Cela ne prouve pas non plus que, dans une certaine monarchie, on ait de l’honneur ; et que, dans un État despotique particulier, on ait de la crainte ; mais qu’il faudrait en avoir : sans quoi le gouvernement sera imparfait.
\subsection[{Livre quatrième. Que les lois de l’éducation doivent être relatives aux principes du gouvernement}]{Livre quatrième. Que les lois de l’éducation doivent être relatives aux principes du gouvernement}
\subsubsection[{Chapitre I. Des lois de l’éducation}]{Chapitre I. Des lois de l’éducation}
\noindent Les lois de l’éducation sont les premières que nous recevons. Et, comme elles nous préparent à être citoyens, chaque famille particulière doit être gouvernée sur le plan de la grande famille qui les comprend toutes.\par
Si le peuple en général a un principe, les parties qui le composent, c’est-à-dire les familles, l’auront aussi. Les lois de l’éducation seront donc différentes dans chaque espèce de gouvernement. Dans les monarchies, elles auront pour objet {\itshape l’honneur} ; dans les républiques, la {\itshape vertu} ; dans le despotisme, la {\itshape crainte}.
\subsubsection[{Chapitre II. De l’éducation dans les monarchies}]{Chapitre II. De l’éducation dans les monarchies}
\noindent Ce n’est point dans les maisons publiques où l’on instruit l’enfance, que l’on reçoit dans les monarchies la principale éducation ; c’est lorsque l’on entre dans le monde, que l’éducation en quelque façon commence. Là est l’école de ce que l’on appelle {\itshape honneur}, ce maître universel qui doit partout nous conduire.\par
C’est là que l’on voit et que l’on entend toujours dire trois choses : {\itshape qu’il faut mettre dans les vertus une certaine noblesse, dans les mœurs une certaine franchise, dans les manières une certaine politesse}.\par
Les vertus qu’on nous y montre sont toujours moins ce que l’on doit aux autres, que ce que l’on se doit à soi-même : elles ne sont pas tant ce qui nous appelle vers nos concitoyens, que ce qui nous en distingue.\par
On n’y juge pas les actions des hommes comme bonnes, mais comme belles ; comme justes, mais comme grandes ; comme raisonnables, mais comme extraordinaires.\par
Dès que l’honneur y peut trouver quelque chose de noble, il est ou le juge qui les rend légitimes, ou le sophiste qui les justifie.\par
Il permet la galanterie lorsqu’elle est unie à l’idée des sentiments du cœur, ou à l’idée de conquête ; et c’est la vraie raison pour laquelle les mœurs ne sont jamais si pures dans les monarchies que dans les gouvernements républicains.\par
Il permet la ruse lorsqu’elle est jointe à l’idée de la grandeur de l’esprit ou de la grandeur des affaires, comme dans la politique, dont les finesses ne l’offensent pas.\par
Il ne défend l’adulation que lorsqu’elle est séparée de l’idée d’une grande fortune, et n’est jointe qu’au sentiment de sa propre bassesse.\par
À l’égard des mœurs, j’ai dit que l’éducation des monarchies doit y mettre une certaine franchise. On y veut donc de la vérité dans les discours. Mais est-ce par amour pour elle ? point du tout. On la veut, parce qu’un homme qui est accoutumé à la dire paraît être hardi et libre. En effet, un tel homme semble ne dépendre que des choses, et non pas de la manière dont un autre les reçoit.\par
C’est ce qui fait qu’autant qu’on y recommande cette espèce de franchise, autant on y méprise celle du peuple, qui n’a que la vérité et la simplicité pour objet.\par
Enfin, l’éducation dans les monarchies exige dans les manières une certaine politesse. Les hommes, nés pour vivre ensemble, sont nés aussi pour se plaire ; et celui qui n’observerait pas les bienséances, choquant tous ceux avec qui il vivrait, se décréditerait au point qu’il deviendrait incapable de faire aucun bien.\par
Mais ce n’est pas d’une source si pure que la politesse a coutume de tirer son origine. Elle naît de l’envie de se distinguer. C’est par orgueil que nous sommes polis : nous nous sentons flattés d’avoir des manières qui prouvent que nous ne sommes pas dans la bassesse, et que nous n’avons pas vécu avec cette sorte de gens que l’on a abandonnés dans tous les âges.\par
Dans les monarchies, la politesse est naturalisée à la cour. Un homme excessivement grand rend tous les autres petits. De là les égards que l’on doit à tout le monde ; de là naît la politesse, qui flatte autant ceux qui sont polis que ceux à l’égard de qui ils le sont ; parce qu’elle fait comprendre qu’on est de la cour, ou qu’on est digne d’en être.\par
L’air de la cour consiste à quitter sa grandeur propre pour une grandeur empruntée. Celle-ci flatte plus un courtisan que la sienne même. Elle donne une certaine modestie superbe qui se répand au loin, mais dont l’orgueil diminue insensiblement, à proportion de la distance où l’on est de la source de cette grandeur.\par
On trouve à la cour une délicatesse de goût en toutes choses, qui vient d’un usage continuel des superfluités d’une grande fortune, de la variété, et surtout de la lassitude des plaisirs, de la multiplicité, de la confusion même des fantaisies, qui, lorsqu’elles sont agréables, y sont toujours reçues.\par
C’est sur toutes ces choses que l’éducation se porte pour faire ce qu’on appelle l’honnête homme, qui a toutes les qualités et toutes les vertus que l’on demande dans ce gouvernement.\par
Là l’honneur, se mêlant partout, entre dans toutes les façons de penser et toutes les manières de sentir, et dirige même les principes.\par
Cet honneur bizarre fait que les vertus ne sont que ce qu’il veut, et comme il les veut : il met, de son chef, des règles à tout ce qui nous est prescrit ; il étend ou il borne nos devoirs à sa fantaisie, soit qu’ils aient leur source dans la religion, dans la politique, ou dans la morale.\par
Il n’y a rien dans la monarchie que les lois, la religion et l’honneur prescrivent tant que l’obéissance aux volontés du prince : mais cet honneur nous dicte que le prince ne doit jamais nous prescrire une action qui nous déshonore, parce qu’elle nous rendrait incapables de le servir.\par
Crillon refusa d’assassiner le duc de Guise, mais il offrit à Henri III de se battre contre lui. Après la Saint-Barthélemy, Charles IX ayant écrit à tous les gouverneurs de faire massacrer les huguenots, le vicomte d’Orte, qui commandait dans Bayonne, écrivit au roi\footnote{Voyez l’{\itshape Histoire} de d’Aubigné.} : « Sire, je n’ai trouvé parmi les habitants et les gens de guerre que de bons citoyens, de braves soldats, et pas un bourreau ; ainsi, eux et moi, supplions Votre Majesté d’employer nos bras et nos vies à choses faisables. » Ce grand et généreux courage regardait une lâcheté comme une chose impossible.\par
Il n’y a rien que l’honneur prescrive plus à la noblesse que de servir le prince à la guerre. En effet, c’est la profession distinguée, parce que ses hasards, ses succès et ses malheurs même conduisent à la grandeur. Mais, en imposant cette loi, l’honneur veut en être l’arbitre ; et, s’il se trouve choqué, il exige ou permet qu’on se retire chez soi.\par
Il veut qu’on puisse indifféremment aspirer aux emplois, ou les refuser ; il tient cette liberté au-dessus de la fortune même.\par
L’honneur a donc ses règles suprêmes, et l’éducation est obligée de s’y conformer\footnote{On dit ici ce qui est et non pas ce qui doit être : l’honneur est un préjugé que la religion travaille tantôt à détruire, tantôt à régler.}. Les principales sont qu’il nous est bien permis de faire cas de notre fortune, mais qu’il nous est souverainement défendu d’en faire aucun de notre vie.\par
La seconde est que, lorsque nous avons été une fois placés dans un rang, nous ne devons rien faire ni souffrir qui fasse voir que nous nous tenons inférieurs à ce rang même.\par
La troisième, que les choses que l’honneur défend sont plus rigoureusement défendues, lorsque les lois ne concourent point à les proscrire ; et que celles qu’il exige sont plus fortement exigées, lorsque les lois ne les demandent pas.
\subsubsection[{Chapitre III. De l’éducation dans le gouvernement despotique}]{Chapitre III. De l’éducation dans le gouvernement despotique}
\noindent Comme l’éducation dans les monarchies ne travaille qu’à élever le cœur, elle ne cherche qu’à l’abaisser dans les États despotiques. Il faut qu’elle y soit servile. Ce sera un bien, même dans le commandement, de l’avoir eue telle, personne n’y étant tyran sans être en même temps esclave.\par
L’extrême obéissance suppose de l’ignorance dans celui qui obéit ; elle en suppose même dans celui qui commande : il n’a point à délibérer, à douter, ni à raisonner ; il n’a qu’à vouloir.\par
Dans les États despotiques, chaque maison est un empire séparé. L’éducation, qui consiste principalement à vivre avec les autres, y est donc très bornée : elle se réduit à mettre la crainte dans le cœur, et à donner à l’esprit la connaissance de quelques principes de religion fort simples. Le savoir y sera dangereux, l’émulation funeste ; et, pour les vertus, Aristote ne peut croire qu’il y en ait quelqu’une de propre aux esclaves\footnote{{\itshape Politique}, liv. I.} ; ce qui bornerait bien l’éducation dans ce gouvernement.\par
L’éducation y est donc en quelque façon nulle. Il faut ôter tout, afin de donner quelque chose ; et commencer par faire un mauvais sujet, pour faire un bon esclave.\par
Eh ! pourquoi l’éducation s’attacherait-elle à y former un bon citoyen qui prit pari au malheur public ? S’il aimait l’État, il serait tenté de relâcher les ressorts du gouvernement : s’il ne réussissait pas, il se perdrait ; s’il réussissait, il courrait risque de se perdre, lui, le prince, et l’empire.
\subsubsection[{Chapitre IV. Différence des effets de l’éducation chez les anciens et parmi nous}]{Chapitre IV. Différence des effets de l’éducation chez les anciens et parmi nous}
\noindent La plupart des peuples anciens vivaient dans des gouvernements qui ont la vertu pour principe ; et, lorsqu’elle y était dans sa force, on y faisait des choses que nous ne voyons plus aujourd’hui, et qui étonnent nos petites âmes.\par
Leur éducation avait un autre avantage sur la nôtre ; elle n’était jamais démentie. Épaminondas, la dernière année de sa vie, disait, écoutait, voyait, faisait les mêmes choses que dans l’âge où il avait commencé d’être instruit.\par
Aujourd’hui, nous recevons trois éducations différentes ou contraires : celle de nos pères, celle de nos maîtres, celle du monde. Ce qu’on nous dit dans la dernière renverse toutes les idées des premières. Cela vient, en quelque partie, du contraste qu’il y a parmi nous entre les engagements de la religion et ceux du monde ; chose que les anciens ne connaissaient pas.
\subsubsection[{Chapitre V. De l’éducation dans le gouvernement républicain}]{Chapitre V. De l’éducation dans le gouvernement républicain}
\noindent C’est dans le gouvernement républicain que l’on a besoin de toute la puissance de l’éducation. La crainte des gouvernements despotiques naît d’elle-même parmi les menaces et les châtiments ; l’honneur des monarchies est favorisé par les passions, et les favorise à son tour : mais la vertu politique est un renoncement à soi-même, qui est toujours une chose très pénible.\par
On peut définir cette vertu, l’amour des lois et de la patrie. Cet amour, demandant une préférence continuelle de l’intérêt public au sien propre, donne toutes les vertus particulières : elles ne sont que cette préférence.\par
Cet amour est singulièrement affecté aux démocraties. Dans elles seules, le gouvernement est confié à chaque citoyen. Or, le gouvernement est comme toutes les choses du monde ; pour le conserver, il faut l’aimer.\par
On n’a jamais ouï dire que les rois n’aimassent pas la monarchie, et que les despotes haïssent le despotisme.\par
Tout dépend donc d’établir dans la république cet amour ; et c’est à l’inspirer que l’éducation doit être attentive. Mais, pour que les enfants puissent l’avoir, il y a un moyen sûr : c’est que les pères l’aient eux-mêmes.\par
On est ordinairement le maître de donner à ses enfants ses connaissances ; on l’est encore plus de leur donner ses passions.\par
Si cela n’arrive pas, c’est que ce qui a été fait dans la maison paternelle est détruit par les impressions du dehors.\par
Ce n’est point le peuple naissant qui dégénère ; il ne se perd que lorsque les hommes faits sont déjà corrompus.
\subsubsection[{Chapitre VI. De quelques institutions des grecs}]{Chapitre VI. De quelques institutions des grecs}
\noindent Les anciens Grecs, pénétrés de la nécessité que les peuples qui vivaient sous un gouvernement populaire fussent élevés à la vertu, firent, pour l’inspirer, des institutions singulières. Quand vous voyez, dans la {\itshape Vie de Lycurgue}, les lois qu’il donna aux Lacédémoniens vous croyez lire l’His{\itshape toire des Sévarambes}. Les lois de Crète étaient l’original de celles de Lacédémone ; et celles de Platon en étaient la correction.\par
Je prie qu’on fasse un peu d’attention à l’étendue de génie qu’il fallut à ces législateurs pour voir qu’en choquant tous les usages reçus, en confondant toutes les vertus, ils montreraient à l’univers leur sagesse. Lycurgue, mêlant le larcin avec l’esprit de justice, le plus dur esclavage avec l’extrême liberté les sentiments les plus atroces avec la plus grande modération, donna de la stabilité à sa ville. Il sembla lui ôter toutes les ressources, les arts, le commerce, l’argent, les murailles : on y a de l’ambition, sans espérance d’être mieux : on y a les sentiments naturels, et on n’y est ni enfant, ni mari, ni père : la pudeur même est ôtée à la chasteté. C’est par ces chemins que Sparte est menée à la grandeur et à la gloire ; mais avec une telle infaillibilité de ses institutions, qu’on n’obtenait rien contre elle en gagnant des batailles, si on ne parvenait à lui ôter sa police\footnote{Philopœmen contraignit les Lacédémoniens d’abandonner la manière de nourrir leurs enfants, sachant bien que, sans cela, ils auraient toujours une âme grande et le cœur haut. Plutarque, {\itshape Vie de Philopœmen}. Voyez Tite-Live, liv. XXXVIII.}.\par
La Crète et la Laconie furent gouvernées par ces lois. Lacédémone céda la dernière aux Macédoniens, et la Crète\footnote{Elle défendit, pendant trois ans, ses lois et sa liberté. Voyez les livres XCVIII, XCIX et C de Tite-Live, dans l’{\itshape Epitome} de Florus. Elle fit plus de résistance que les plus grands rois.} fut la dernière proie des Romains. Les Samnites eurent ces mêmes institutions, et elles furent pour ces Romains le sujet de vingt-quatre triomphes\footnote{Florus, liv. I.}.\par
Cet extraordinaire que l’on voyait dans les institutions de la Grèce, nous l’avons vu dans la lie et la corruption de nos temps modernes\footnote{In fece Romuli, Cicéron.}. Un législateur honnête homme a formé un peuple, où la probité paraît aussi naturelle que la bravoure chez les Spartiates. M. Penn est un véritable Lycurgue ; et, quoique le premier ait eu la paix pour {\itshape objet}, comme l’autre a eu la guerre, ils se ressemblent dans la vole singulière où ils ont mis leur peuple, dans l’ascendant qu’ils ont eu sur des hommes libres, dans les préjugés qu’ils ont vaincus, dans les passions qu’ils ont soumises.\par
Le Paraguay peut nous fournir un autre exemple. On a voulu en faire un crime à la Société, qui {\itshape regarde le plaisir} de commander comme le seul bien de la vie ; mais il sera toujours beau de gouverner les hommes en les rendant plus {\itshape heureux}\footnote{Les Indiens du Paraguay ne dépendent point d’un seigneur particulier, ne payent qu’un cinquième des tributs, et ont des armes à feu pour se défendre.}.\par
Il est glorieux pour elle d’avoir été la première qui ait montré dans ces contrées l’idée de la religion jointe à celle de l’humanité. En réparant les dévastations des Espagnols, elle a commencé à guérir une des grandes plaies qu’ait encore reçues le genre humain.\par
Un sentiment exquis qu’a cette Société pour tout ce qu’elle appelle honneur, son zèle pour une religion qui humilie bien plus ceux qui l’écoutent que ceux qui la prêchent, lui ont fait entreprendre de grandes choses ; et elle y a réussi. Elle a retiré des bois des peuples dispersés ; elle leur a donné une subsistance assurée ; elle les a vêtus ; et, quand elle n’aurait fait par là qu’augmenter l’industrie parmi les hommes, elle aurait beaucoup fait.\par
Ceux qui voudront faire des institutions pareilles établiront la communauté de biens de la {\itshape République} de Platon, ce respect qu’il demandait pour les dieux, cette séparation d’avec les étrangers pour la conservation des mœurs, et la cité faisant le commerce, et non pas les citoyens {\itshape ;} ils donneront nos arts sans notre luxe, et nos besoins sans nos désirs.\par
Ils proscriront l’argent, dont l’effet est de grossir la fortune des hommes au-delà des bornes que la nature y avait mises, d’apprendre à conserver inutilement ce qu’on avait amassé de même, de multiplier à l’infini les désirs, et de suppléer à la nature, qui nous avait donné des moyens très bornés d’irriter nos passions, et de nous corrompre les uns les autres.\par
« Les Épidamniens\footnote{Plutarque, Demande des choses grecques.}, sentant leurs mœurs se corrompre par leur communication avec les Barbares, élurent un magistrat pour faire tous les marchés au nom de la cité et pour la cité. » Pour lors, le commerce ne corrompt pas la constitution, et la constitution ne prive pas la société des avantages du commerce.
\subsubsection[{Chapitre VII. En quel cas ces institutions singulières peuvent être bonnes}]{Chapitre VII. En quel cas ces institutions singulières peuvent être bonnes}
\noindent Ces sortes d’institutions peuvent convenir dans les républiques, parce que la vertu politique en est le principe : mais, pour porter à l’honneur dans les monarchies, ou pour inspirer de la crainte dans les États despotiques, il ne faut pas tant de soins.\par
Elles ne peuvent d’ailleurs avoir lieu que dans un petit État\footnote{Comme étaient les villes de la Grèce.}, où l’on peut donner une éducation générale, et élever tout un peuple comme une famille.\par
Les lois de Minos, de Lycurgue et de Platon supposent une attention singulière de tous les citoyens les uns sur les autres. On ne peut se promettre cela dans la confusion, dans les négligences, dans l’étendue des affaires d’un grand peuple.\par
Il faut, comme on l’a dit, bannir l’argent dans ces institutions. Mais, dans les grandes sociétés, le nombre, la variété, l’embarras, l’importance des affaires, la facilité des achats, la lenteur des échanges, demandent une mesure commune. Pour porter partout sa puissance, ou la défendre partout, il faut avoir ce à quoi les hommes ont attaché partout la puissance.
\subsubsection[{Chapitre VIII. Explication d’un paradoxe des anciens par rapport aux mœurs}]{Chapitre VIII. Explication d’un paradoxe des anciens par rapport aux mœurs}
\noindent Polybe, le judicieux Polybe, nous dit que la musique était nécessaire pour adoucir les mœurs des Arcades, qui habitaient un pays où l’air est triste et froid ; que ceux de Cynète, qui négligèrent la musique, surpassèrent en cruauté tous les Grecs, et qu’il n’y a point de ville où l’on ait vu tant de crimes. Platon ne craint point de dire que l’on ne peut faire de changement dans la musique, qui n’en soit un dans la constitution de l’État. Aristote, qui semble n’avoir fait sa {\itshape Politique} que pour opposer ses sentiments à ceux de Platon, est pourtant d’accord avec lui touchant la puissance de la musique sur les mœurs. Théophraste, Plutarque\footnote{Vie de Pélopidas.}, Strabon\footnote{Liv. I.}, tous les anciens ont pensé de même. Ce n’est point une opinion jetée sans réflexion ; c’est un des principes de leur politique\footnote{Platon (liv. IV des {\itshape Lois}), dit que les préfectures de la musique et de la gymnastique sont les plus importants emplois de la cité ; et, dans sa {\itshape République}, liv. III : « Damon vous dira, dit-il, quels sont les sons capables de faire naître la bassesse de l’âme, l’insolence, et les vertus contraires. »}. C’est ainsi qu’ils donnaient des lois, c’est ainsi qu’ils voulaient qu’on gouvernât les cités.\par
Je crois que je pourrais expliquer ceci. Il faut se mettre dans l’esprit que, dans les villes grecques, surtout celles qui avaient pour principal objet la guerre, tous les travaux et toutes les professions qui pouvaient conduire à gagner de l’argent, étaient regardés comme indignes d’un homme libre. « La plupart des arts, dit Xénophon\footnote{Liv. V, Dits mémorables.}, corrompent le corps de ceux qui les exercent ; ils obligent de s’asseoir à l’ombre, ou près du feu : on n’a de temps ni pour ses amis, ni pour la république. » Ce ne fut que dans la corruption de quelques démocraties, que les artisans parvinrent à être citoyens. C’est ce qu’Aristote\footnote{Politique, liv. III, chap. IV.} nous apprend ; et il soutient qu’une bonne république ne leur donnera jamais le droit de cité\footnote{Diophante, dit Aristote ({\itshape Politique}, liv. II, chap. VII), établit autrefois à Athènes que les artisans seraient esclaves du public.}.\par
L’agriculture était encore une profession servile, et ordinairement c’était quelque peuple vaincu qui l’exerçait : les Ilotes, chez les Lacédémoniens ; les Périéciens, chez les Crétois ; les Pénestes, chez les Thessaliens ; d’autres\footnote{Aussi Platon et Aristote veulent-ils que les esclaves cultivent les terres, {\itshape Lois}, liv. VII ;{\itshape  Politique}, liv. VII, chap. X. Il est vrai que l’agriculture n’était pas partout exercée par des esclaves : au contraire, comme dit Aristote, les meilleures républiques étaient celles où les citoyens s’y attachaient ; mais cela n’arriva que par la corruption des anciens gouvernements, devenus démocratiques, car, dans les premiers temps, les villes de Grèce vivaient dans l’aristocratie.} peuples esclaves, dans d’autres républiques.\par
Enfin, tout bas commerce\footnote{Cauponatio (27).} était infâme chez les Grecs. Il aurait fallu qu’un citoyen eût rendu des services à un esclave, à un locataire, à un étranger : cette idée choquait l’esprit de la liberté grecque. Aussi Platon\footnote{Liv. II.} veut-il, dans ses {\itshape Lois}, qu’on punisse un citoyen qui ferait le commerce.\par
On était donc fort embarrassé dans les républiques grecques. On ne voulait pas que les citoyens travaillassent au commerce, à l’agriculture, ni aux arts ; on ne voulait pas non plus qu’ils fussent oisifs\footnote{Aristote, {\itshape Politique}, liv. X.}. Ils trouvaient une occupation dans les exercices qui dépendaient de la gymnastique, et dans ceux qui avaient du rapport à la guerre\footnote{Ars corporum exercendorum, gymnastica ; variis certaminibus terendorum, paedotribica. Aristote, Politique, liv. VIII, chap. III, 13.}. L’institution ne leur en donnait point d’autres. Il faut donc regarder les Grecs comme une société d’athlètes et de combattants. Or, ces exercices, si propres à faire des gens durs et sauvages\footnote{Aristote dit que les enfants des Lacédémoniens, qui commençaient ces exercices dès l’âge le plus tendre, en contractaient trop de férocité. {\itshape Politique}, liv. VIII, chap. IV.}, avaient besoin d’être tempérés par d’autres qui pussent adoucir les mœurs. La musique, qui tient à l’esprit par les organes du corps, était très propre à cela. C’est un milieu entre les exercices du corps qui rendent les hommes durs, et les sciences de spéculation qui les rendent sauvages. On ne peut pas dire que la musique inspirât la vertu ; cela serait inconcevable : mais elle empêchait l’effet de la férocité de l’institution, et faisait que l’âme avait dans l’éducation une part qu’elle n’y aurait point eue.\par
Je suppose qu’il y ait parmi nous une société de gens si passionnés pour la chasse, qu’ils s’en occupassent uniquement ; il est sûr qu’ils en contracteraient une certaine rudesse. Si ces mêmes gens venaient à prendre encore du goût pour la musique, on trouverait bientôt de la différence dans leurs manières et dans leurs mœurs. Enfin, les exercices des Grecs n’excitaient en eux qu’un genre de passions, la rudesse, la colère, la cruauté. La musique les excite toutes, et peut faire sentir à l’âme la douceur, la pitié, la tendresse, le doux plaisir. Nos auteurs de morale, qui, parmi nous, proscrivent si fort les théâtres, nous font assez sentir le pouvoir que la musique a sur nos âmes.\par
Si à la société dont j’ai parlé, on ne donnait que des tambours et des airs de trompette, n’est-il pas vrai que l’on parviendrait moins à son but, que si l’on donnait une musique tendre ? Les anciens avaient donc raison, lorsque, dans certaines circonstances, ils préféraient pour les mœurs un mode à un autre.\par
Mais, dira-t-on, pourquoi choisir la musique par préférence ? C’est que, de tous les plaisirs des sens, il n’y en a aucun qui corrompe moins l’âme. Nous rougissons de lire dans Plutarque\footnote{Vie de Pélopidas.}, que les Thébains, pour adoucir les mœurs de leurs jeunes gens, établirent par les lois un amour qui devrait être proscrit par toutes les nations du monde.
\subsection[{Livre cinquième. Que les lois que le législateur donne doivent être relatives au principe de gouvernement}]{Livre cinquième. Que les lois que le législateur donne doivent être relatives au principe de gouvernement}
\subsubsection[{Chapitre I. Idée de ce livre}]{Chapitre I. Idée de ce livre}
\noindent Nous venons de voir que les lois de l’éducation doivent être relatives au principe de chaque gouvernement. Celles que le législateur donne à toute la société sont de même. Ce rapport des lois avec ce principe tend tous les ressorts du gouvernement ; et ce principe en reçoit à son tour une nouvelle force. C’est ainsi que, dans les mouvements physiques, l’action est toujours suivie d’une réaction.\par
Nous allons examiner ce rapport dans chaque gouvernement ; et nous commencerons par l’État républicain, qui a la vertu pour principe.
\subsubsection[{Chapitre II. Ce que c’est que la vertu dans l’état politique}]{Chapitre II. Ce que c’est que la vertu dans l’état politique}
\noindent La vertu, dans une république, est une chose très simple : c’est l’amour de la république ; c’est un sentiment, et non une suite de connaissances ; le dernier homme de l’État peut avoir ce sentiment, comme le premier. Quand le peuple a une fois de bonnes maximes, il s’y tient plus longtemps que ce qu’on appelle les honnêtes gens. Il est rare que la corruption commence par lui. Souvent il a tiré de la médiocrité de ses lumières un attachement plus fort pour ce qui est établi.\par
L’amour de la patrie conduit à la bonté des mœurs, et la bonté des mœurs mène à l’amour de la patrie. Moins nous pouvons satisfaire nos passions particulières, plus nous nous livrons aux générales. Pourquoi les moines aiment-ils tant leur ordre ? C’est justement par l’endroit qui fait qu’il leur est insupportable. Leur règle les prive de toutes les choses sur lesquelles les passions ordinaires s’appuient : reste donc cette passion pour la règle même qui les afflige. Plus elle est austère, c’est-à-dire, plus elle retranche de leurs penchants, plus elle donne de force à ceux qu’elle leur laisse.
\subsubsection[{Chapitre III. Ce que c’est que l’amour de la république dans la démocratie}]{Chapitre III. Ce que c’est que l’amour de la république dans la démocratie}
\noindent L’amour de la république, dans une démocratie, est celui de la démocratie ; l’amour de la démocratie est celui de l’égalité.\par
L’amour de la démocratie est encore l’amour de la frugalité. Chacun devant y avoir le même bonheur et les mêmes avantages, y doit goûter les mêmes plaisirs, et former les mêmes espérances ; chose qu’on ne peut attendre que de la frugalité générale.\par
L’amour de l’égalité, dans une démocratie, borne l’ambition au seul désir, au seul bonheur de rendre à sa patrie de plus grands services que les autres citoyens. Ils ne peuvent pas lui rendre tous des services égaux ; mais ils doivent tous également lui en rendre. En naissant, on contracte envers elle une dette immense dont on ne peut jamais s’acquitter.\par
Ainsi les distinctions y naissent du principe de l’égalité, lors même qu’elle paraît ôtée par des services heureux, ou par des talents supérieurs.\par
L’amour de la frugalité borne le {\itshape désir d’avoir} à l’attention que demande le nécessaire pour sa famille et même le superflu pour sa patrie. Les richesses donnent une puissance dont un citoyen ne peut pas user pour lui ; car il ne serait pas égal.\par
Elles procurent des délices dont il ne doit pas jouir non plus parce qu’elles choqueraient l’égalité tout de même.\par
Aussi les bonnes démocraties, en établissant la frugalité domestique, ont-elles ouvert la poile aux dépenses publiques, comme on fit à Athènes et à Rome. Pour lors la magnificence et la profusion naissaient du fond de la frugalité même : et, comme la religion demande qu’on ait les mains pures pour faire des offrandes aux dieux, les lois voulaient des mœurs frugales pour que l’on pût donner à sa patrie.\par
Le bon sens et le bonheur des particuliers consistent beaucoup dans la médiocrité de leurs talents et de leurs fortunes. Une république où les lois auront formé beaucoup de gens médiocres, composée de gens sages, se gouvernera sagement ; composée de gens heureux, elle sera très heureuse.
\subsubsection[{Chapitre IV. Comment on inspire l’amour de l’égalité et de la frugalité}]{Chapitre IV. Comment on inspire l’amour de l’égalité et de la frugalité}
\noindent L’amour de {\itshape l’égalité} et celui de la {\itshape frugalité} sont extrêmement excités par l’égalité et la frugalité mêmes, quand on vit dans une société où les lois ont établi l’une et l’autre.\par
Dans les monarchies et les États despotiques, personne n’aspire à l’égalité ; cela ne vient pas même dans l’idée : chacun y tend à la supériorité.\par
Les gens des conditions les plus basses ne désirent d’en sortir que pour être les maîtres des autres.\par
Il en est de même de la frugalité. Pour l’aimer, il faut en jouir. Ce ne seront point ceux qui sont corrompus par les délices qui aimeront la vie frugale ; et, si cela avait été naturel ou ordinaire, Alcibiade n’aurait pas fait l’admiration de l’univers. Ce ne seront pas non plus ceux qui envient ou qui admirent le luxe des autres qui aimeront la frugalité : des gens qui n’ont devant les yeux que des hommes riches, ou des hommes misérables comme eux, détestent leur misère, sans aimer ou connaître ce qui fait le terme de la misère.\par
C’est donc une maxime très vraie que, pour que l’on aime l’égalité et la frugalité dans une république, il faut que les lois les y aient établies.
\subsubsection[{Chapitre V. Comment les lois établissent l’égalité dans la démocratie}]{Chapitre V. Comment les lois établissent l’égalité dans la démocratie}
\noindent Quelques législateurs anciens, comme Lycurgue et Romulus, partagèrent également les terres. Cela ne pouvait avoir lieu que dans la fondation d’une république nouvelle ; ou bien lorsque l’ancienne loi était si corrompue, et les esprits dans une telle disposition, que les pauvres se croyaient obligés de chercher, et les riches obligés de souffrir un pareil remède.\par
Si, lorsque le législateur fait un pareil partage, il ne donne pas des lois pour le maintenir, il ne fait qu’une constitution passagère ; l’inégalité entrera par le côté que les lois n’auront pas défendu, et la république sera perdue.\par
Il faut donc que l’on règle, dans cet objet, les dots des femmes, les donations, les successions, les testaments, enfin toutes les manières de contracter. Car, s’il était permis de donner son bien à qui on voudrait et comme on voudrait, chaque volonté particulière troublerait la disposition de la loi fondamentale.\par
Solon, qui permettait à Athènes de laisser son bien à qui on voulait par testament, pourvu qu’on n’eût point d’enfants\footnote{Plutarque, {\itshape Vie de Solon}.}, contredisait les lois anciennes, qui ordonnaient que les biens restassent dans la famille du testateur\footnote{Plutarque, {\itshape Vie de Solon}.}. Il contredisait les siennes propres ; car, en supprimant les dettes, il avait cherché l’égalité.\par
C’était une bonne loi pour la démocratie, que celle qui défendait d’avoir deux hérédités\footnote{Philolaüs de Corinthe établit à Athènes que le nombre des portions de terre et celui des hérédités serait toujours le même. Aristote, {\itshape Politique}, liv. II, chap. XII.}. Elle prenait son origine du partage égal des terres et des portions données à chaque citoyen. La loi n’avait pas voulu qu’un seul homme eût plusieurs portions.\par
La loi qui ordonnait que le plus proche parent épousât l’héritière, naissait d’une source pareille. Elle est donnée chez les Juifs après un pareil partage. Platon\footnote{{\itshape République}, liv. VIII.}, qui fonde ses lois sur ce partage, la donne de même ; et c’était une loi athénienne.\par
Il y avait à Athènes une loi, dont je ne sache pas que personne ait connu l’esprit. Il était permis d’épouser sa sœur consanguine, et non pas sa sœur utérine\footnote{Cornelius Nepos, {\itshape in praefat}. Cet usage était des premiers temps. Aussi Abraham dit-il de Sara : {\itshape Elle est ma sœur, fille de mon père, et non de ma mère}. Les mêmes raisons avaient fait établir une même loi chez différents peuples.}. Cet usage tirait son origine des républiques, dont l’esprit était de ne pas mettre sur la même tête deux portions de fonds de terre, et par conséquent deux hérédités. Quand un homme épousait sa sœur du côté du père, il ne pouvait avoir qu’une hérédité, qui était celle de son père : mais, quand il épousait sa sœur utérine, il pourrait arriver que le père de cette sœur, n’ayant pas d’enfants mâles, lui laissât sa succession ; et que, par conséquent, son frère, qui l’avait épousée, en eût deux.\par
Qu’on ne m’objecte pas ce que dit Philon\footnote{{\itshape De specialibus legibus quae pertinent ad praecepta Decalogi}.}, que, quoiqu’à Athènes on épousât sa sœur consanguine, et non pas sa sœur utérine, on pouvait à Lacédémone épouser sa sœur utérine, et non pas sa sœur consanguine. Car je trouve dans Strabon\footnote{Liv. X.} que, quand à Lacédémone une sœur épousait son frère, elle avait pour sa dot la moitié de la portion du frère. Il est clair que cette seconde loi était faite pour prévenir les mauvaises suites de la première. Pour empêcher que le bien de la famille de la sœur ne passât dans celle du frère, on donnait en dot à la sœur la moitié du bien du frère.\par
Sénèque\footnote{{\itshape Athenis dimidium licet, Alexandriae totum}. Sénèque, {\itshape De morte Claudii}.}, parlant de Silanus qui avait épousé sa sœur, dit qu’à Athènes la permission était restreinte, et qu’elle était générale à Alexandrie. Dans le gouvernement d’un seul, il n’était guère question de maintenir le partage des biens.\par
Pour maintenir ce partage des terres dans la démocratie, c’était une bonne loi que celle qui voulait qu’un père qui avait plusieurs enfants en choisît un pour succéder à sa portion\footnote{Platon fait une pareille loi, liv. III des {\itshape Lois.}}, et donnât les autres en adoption à quelqu’un qui n’eût point d’enfants afin que le nombre des citoyens pût toujours se maintenir égal à celui des partages.\par
Phaléas de Chalcédoine\footnote{Aristote, {\itshape Politique}, liv. II, chap. VII.} avait imaginé une façon de rendre égales les fortunes dans une république où elles ne l’étaient pas. Il voulait que les riches donnassent des dots aux pauvres, et n’en reçussent pas ; et que les pauvres reçussent de l’argent pour leurs filles, et n’en donnassent pas. Mais je ne sache point qu’aucune république se soit accommodée d’un règlement pareil. Il met les citoyens sous des conditions dont les différences sont si frappantes, qu’ils haïraient cette égalité même que l’on chercherait à introduire. Il est bon quelquefois que les lois ne paraissent pas aller si directement au but qu’elles se proposent.\par
Quoique, dans la démocratie, l’égalité réelle soit l’âme de l’État, cependant elle est si difficile à établir, qu’une exactitude extrême à cet égard ne conviendrait pas toujours. Il suffit que l’on établisse un cens\footnote{Solon fit quatre classes : la première, de ceux qui avaient cinq cents mines de revenu, tant en grains qu’en Fruits liquides ; la seconde, de ceux qui en avaient trois cents, et pouvaient entretenir un cheval ; la troisième, de ceux qui n’en avaient que deux cents ; la quatrième, de tous ceux qui vivaient de leurs bras. Plutarque, {\itshape Vie de Solon}.} qui réduise ou fixe les différences à un certain point ; après quoi, c’est à des lois particulières à égaliser, pour ainsi dire, les inégalités, par les charges qu’elles imposent aux riches, et le soulagement qu’elles accordent aux pauvres. Il n’y a que les richesses médiocres qui puissent donner ou souffrir ces sortes de compensations : car, pour les fortunes immodérées, tout ce qu’on ne leur accorde pas de puissance et d’honneur, elles le regardent comme une injure.\par
Toute inégalité dans la démocratie doit être tirée de la nature de la démocratie et du principe même de l’égalité. Par exemple, on y peut craindre que des gens qui auraient besoin d’un travail continuel pour vivre, ne fussent trop appauvris par une magistrature, ou qu’ils n’en négligeassent les fonctions ; que des artisans ne s’enorgueillissent ; que des affranchis trop nombreux ne devinssent plus puissants que les anciens citoyens. Dans ces cas, l’égalité entre les citoyens\footnote{Solon exclut des charges tous ceux du quatrième cens.} peut être ôtée dans la démocratie pour l’utilité de la démocratie. Mais ce n’est qu’une égalité apparente que l’on ôte : car un homme ruiné par une magistrature serait dans une pire condition que les autres citoyens ; et ce même homme, qui serait obligé d’en négliger les fonctions, mettrait les autres citoyens dans une condition pire que la sienne ; et ainsi du reste.
\subsubsection[{Chapitre VI. Comment les lois doivent entretenir la frugalité dans la démocratie}]{Chapitre VI. Comment les lois doivent entretenir la frugalité dans la démocratie}
\noindent Il ne suffit pas, dans une bonne démocratie, que les portions de terre soient égales ; il faut qu’elles soient petites, comme chez les Romains. « À Dieu ne plaise, disait Curius à ses soldats\footnote{Ils demandaient une plus grande portion de la terre conquise. Plutarque, {\itshape Œuvres morales, Vies des anciens rois et capitaines.}}, qu’un citoyen estime peu de terre, ce qui est suffisant pour nourrir un homme. »\par
Comme l’égalité des fortunes entretient la frugalité, la frugalité maintient l’égalité des fortunes. Ces choses, quoique différentes, sont telles qu’elles ne peuvent subsister l’une sans l’autre ; chacune d’elles est la cause et l’effet ; si l’une se retire de la démocratie, l’autre la suit toujours.\par
Il est vrai que, lorsque la démocratie est fondée sur le commerce, il peut fort bien arriver que des particuliers y aient de grandes richesses, et que les mœurs n’y soient pas corrompues. C’est que l’esprit de commerce entraîne avec soi celui de frugalité, d’économie, de modération, de travail, de sagesse, de tranquillité, d’ordre et de règle. Ainsi, tandis que cet esprit subsiste, les richesses qu’il produit n’ont aucun mauvais effet. Le mal arrive, lorsque l’excès des richesses détruit cet esprit de commerce : on voit tout à coup naître les désordres de l’inégalité, qui ne s’étaient pas encore fait sentir.\par
Pour maintenir l’esprit de commerce, il faut que les principaux citoyens le fassent eux-mêmes ; que cet esprit règne seul, et ne soit point croisé par un autre ; que toutes les lois le favorisent ; que ces mêmes lois, par leurs dispositions, divisant les fortunes à mesure que le commerce les grossit, mettent chaque citoyen pauvre dans une assez grande aisance, pour pouvoir travailler comme les autres ; et chaque citoyen riche dans une telle médiocrité, qu’il ait besoin de son travail pour conserver ou pour acquérir.\par
C’est une très bonne loi, dans une république commerçante, que celle qui donne à tous les enfants une portion égale dans la succession des pères. Il se trouve par là que, quelque fortune que le père ait faite, ses enfants, toujours moins riches que lui, sont portés à fuir le luxe, et à travailler comme lui. Je ne parle que des républiques commerçantes ; car, pour celles qui ne le sont pas, le législateur a bien d’autres règlements à faire\footnote{On y doit borner beaucoup les dots des femmes.}.\par
Il y avait dans la Grèce deux sortes de républiques : les unes étaient militaires, comme Lacédémone ; d’autres étaient commerçantes, comme Athènes. Dans les unes, on voulait que les citoyens fussent oisifs ; dans les autres, on cherchait à donner de l’amour pour le travail. Solon fit un crime de l’oisiveté, et voulut que chaque citoyen rendît compte de la manière dont il gagnait sa vie. En effet, dans une bonne démocratie où l’on ne doit dépenser que pour le nécessaire, chacun doit l’avoir ; car de qui le recevrait-on ?
\subsubsection[{Chapitre VII. Autres moyens de favoriser le principe de la démocratie}]{Chapitre VII. Autres moyens de favoriser le principe de la démocratie}
\noindent On ne peut pas établir un partage égal des terres dans toutes les démocraties. Il y a des circonstances où un tel arrangement serait impraticable, dangereux, et choquerait même la constitution. On n’est pas toujours obligé de prendre les voies extrêmes. Si l’on voit, dans une démocratie, que ce partage, qui doit maintenir les mœurs, n’y convienne pas, il faut avoir recours à d’autres moyens.\par
Si l’on établit un corps fixé qui soit par lui-même la règle des mœurs, un sénat où l’âge, la vertu, la gravité, les services donnent entrée, les sénateurs, exposés à la vue du peuple comme les simulacres des dieux, inspireront des sentiments qui seront portés dans le sein de toutes les familles.\par
Il faut surtout que ce sénat s’attache aux institutions anciennes, et fasse en sorte que le peuple et les magistrats ne s’en départent jamais.\par
Il y a beaucoup à gagner, en fait de mœurs, à garder les coutumes anciennes. Comme les peuples corrompus font rarement de grandes choses, qu’ils n’ont guère établi de sociétés, fondé de villes, donné de lois ; et qu’au contraire ceux qui avaient des mœurs simples et austères ont fait la plupart des établissements ; rappeler les hommes aux maximes anciennes, c’est ordinairement les ramener à la vertu.\par
De plus, s’il y a eu quelque révolution, et que l’on ait donné à l’État une forme nouvelle, cela n’a guère pu se faire qu’avec des peines et des travaux infinis, et rarement avec l’oisiveté et des mœurs corrompues. Ceux mêmes qui ont fait la révolution ont voulu la faire goûter, et ils n’ont guère pu y réussir que par de bonnes lois. Les institutions anciennes sont donc ordinairement des corrections, et les nouvelles, des abus. Dans le cours d’un long gouvernement, on va au mal par une pente insensible, et on ne remonte au bien que par un effort.\par
On a douté si les membres du sénat dont nous parlons, doivent être à vie, ou choisis pour un temps. Sans doute qu’ils doivent être choisis pour la vie, comme cela se pratiquait à Rome\footnote{Les magistrats y étaient annuels, et les sénateurs pour la vie.}, à Lacédémone\footnote{Lycurgue, dit Xénophon, {\itshape De republ. Lacedaem}., voulut « qu’on élût les sénateurs parmi les vieillards, pour qu’ils ne se négligeassent pas, même à la fin de la vie ; et en les établissant juges du courage des jeunes gens, il a rendu la vieillesse de ceux-là plus honorable que la force de ceux-ci ».}, et à Athènes même. Car il ne faut pas confondre ce qu’on appelait le sénat à Athènes, qui était un corps qui changeait tous les trois mois, avec l’Aréopage, dont les membres étaient établis pour la vie, comme des modèles perpétuels.\par
Maxime générale : dans un sénat fait pour être la règle, et, pour ainsi dire, le dépôt des mœurs, les sénateurs doivent être élus pour la vie ; dans un sénat fait pour préparer les affaires, les sénateurs peuvent changer.\par
L’esprit, dit Aristote, vieillit comme le corps. Cette réflexion n’est bonne qu’à l’égard d’un magistrat unique, et ne peut être appliquée à une assemblée de sénateurs.\par
Outre l’Aréopage, il y avait à Athènes des gardiens des mœurs et des gardiens des lois\footnote{L’Aréopage lui-même était soumis à la censure.}. À Lacédémone, tous les vieillards étaient censeurs. À Rome, deux magistrats particuliers avaient la censure. Comme le sénat veille sur le peuple, il faut que des censeurs aient les yeux sur le peuple et sur le sénat. Il faut qu’ils rétablissent dans la république tout ce qui a été corrompu, qu’ils notent la tiédeur, jugent les négligences, et corrigent les fautes, comme les lois punissent les crimes.\par
La loi romaine qui voulait que l’accusation de l’adultère fût publique, était admirable pour maintenir la pureté des mœurs ; elle intimidait les femmes, elle intimidait aussi ceux qui devaient veiller sur elles.\par
Rien ne maintient plus les mœurs qu’une extrême subordination des jeunes gens envers les vieillards. Les uns et les autres seront contenus, ceux-là par le respect qu’ils auront pour les vieillards, et ceux-ci par le respect qu’ils auront pour eux-mêmes.\par
Rien ne donne plus de force aux lois, que la subordination extrême des citoyens aux magistrats. « La grande différence que Lycurgue a mise entre Lacédémone et les autres cités, dit Xénophon\footnote{République de Lacédémone.}, consiste en ce qu’il a surtout fait que les citoyens obéissent aux lois ; ils courent lorsque le magistrat les appelle. Mais, à Athènes, un homme riche serait au désespoir que l’on crût qu’il dépendît du magistrat. »\par
L’autorité paternelle est encore très utile pour maintenir les mœurs. Nous avons déjà dit que, dans une république, il n’y a pas une force si réprimante que dans les autres gouvernements. Il faut donc que les lois cherchent à y suppléer : elles le font par l’autorité paternelle.\par
À Rome, les pères avaient droit de vie et de mort sur leurs enfants\footnote{On peut voir, dans l’histoire romaine, avec quel avantage pour la république on se servit de cette puissance. Je ne parlerai que du temps de la plus grande corruption. Aulus Fulvius s’était mis en chemin pour aller trouver Catilina ; son père le rappela et le fit mourir. Salluste, {\itshape De bello Catilinae}. Plusieurs autres citoyens firent de même, Dion, liv. XXXVII.}. À Lacédémone, chaque père avait droit de corriger l’enfant d’un autre.\par
La puissance paternelle se perdit à Rome avec la république. Dans les monarchies, où l’on n’a que faire de mœurs si pures, on veut que chacun vive sous la puissance des magistrats.\par
Les lois de Rome, qui avaient accoutumé les jeunes gens à la dépendance, établirent une longue minorité. Peut-être avons-nous eu tort de prendre cet usage : dans une monarchie on n’a pas besoin de tant de contrainte.\par
Cette même subordination dans la république y pourrait demander que le père restât, pendant sa vie, le maître des biens de ses enfants, comme il fut réglé à Rome. Mais cela n’est pas de l’esprit de la monarchie.
\subsubsection[{Chapitre VIII. Comment les lois doivent se rapporter au principe du gouvernement dans l’aristocratie}]{Chapitre VIII. Comment les lois doivent se rapporter au principe du gouvernement dans l’aristocratie}
\noindent Si, dans l’aristocratie, le peuple est vertueux, on y jouira à peu près du bonheur du gouvernement populaire, et l’État deviendra puissant. Mais, comme il est rare que, là où les fortunes des hommes sont si inégales, il y ait beaucoup de vertu, il faut que les lois tendent à donner, autant qu’elles peuvent, un esprit de modération, et cherchent à rétablir cette égalité que la constitution de l’État ôte nécessairement.\par
L’esprit de modération est ce qu’on appelle la vertu dans l’aristocratie ; il y tient la place de l’esprit d’égalité dans l’État populaire.\par
Si le faste et la splendeur qui environnent les rois font une partie de leur puissance, la modestie et la simplicité des manières font la force des nobles aristocratiques\footnote{De nos jours, les Vénitiens, qui, à bien des égards, se sont conduits très sagement, décidèrent, sur une dispute entre un noble Vénitien et un gentilhomme de terre ferme, pour une préséance dans une église, que, hors de Venise, un noble Vénitien n’avait point de prééminence sur un autre citoyen.}. Quand ils n’affectent aucune distinction, quand ils se confondent avec le peuple, quand ils sont vêtus comme lui, quand ils lui font partager tous leurs plaisirs, il oublie sa faiblesse.\par
Chaque gouvernement a sa nature et son principe. Il ne faut donc pas que l’aristocratie prenne la nature et le principe de la monarchie ; ce qui arriverait, si les nobles avaient quelques prérogatives personnelles et particulières, distinctes de celles de leur corps : les privilèges doivent être pour le sénat, et le simple respect pour les sénateurs.\par
Il y a deux sources principales de désordres dans les États aristocratiques : l’inégalité extrême entre ceux qui gouvernent et ceux qui sont gouvernés ; et la même inégalité entre les différents membres du corps qui gouverne. De ces deux inégalités résultent des haines et des jalousies que les lois doivent prévenir ou arrêter.\par
La première inégalité se trouve principalement lorsque les privilèges des principaux ne sont honorables que parce qu’ils sont honteux au peuple. Telle fut à Rome la loi qui défendait aux patriciens de s’unir par mariage aux plébéiens\footnote{Elle fut mise par les décemvirs dans les deux dernières tables. Voyez Denys d’Halicarnasse, liv. X.} ; ce qui n’avait d’autre effet que de rendre d’un côté les patriciens plus superbes, et de l’autre plus odieux. Il faut voir les avantages qu’en tirèrent les tribuns dans leurs harangues.\par
Cette inégalité se trouvera encore, si la condition des citoyens est différente par rapport aux subsides ; ce qui arrive de quatre manières : lorsque les nobles se donnent le privilège de n’en point payer ; lorsqu’ils font des fraudes pour s’en exempter\footnote{Comme dans quelques aristocraties de nos jours. Rien n’affaiblit tant l’État.} ; lorsqu’ils les appellent à eux, sous prétexte de rétributions ou d’appointements pour les emplois qu’ils exercent ; enfin, quand ils rendent le peuple tributaire, et se partagent les impôts qu’ils lèvent sur eux. Ce dernier cas est rare ; une aristocratie, en cas pareil, est le plus dur de tous les gouvernements.\par
Pendant que Rome inclina vers l’aristocratie, elle évita très bien ces inconvénients. Les magistrats ne tiraient jamais d’appointements de leur magistrature. Les principaux de la République furent taxés comme les autres ; ils le furent même plus ; et quelquefois ils le furent seuls. Enfin, bien loin de se partager les revenus de l’État, tout ce qu’ils purent tirer du trésor public, tout ce que la fortune leur envoya de richesses, ils le distribuèrent au peuple pour se faire pardonner leurs honneurs\footnote{Voyez dans Strabon, liv. XIV, comment les Rhodiens se conduisirent à cet égard.}.\par
C’est une maxime fondamentale, qu’autant que les distributions faites au peuple ont de pernicieux effets dans la démocratie, autant en ont-elles de bons dans le gouvernement aristocratique. Les premières font perdre l’esprit de citoyen, les autres y ramènent.\par
Si l’on ne distribue point les revenus au peuple, il faut lui faire voir qu’ils sont bien administrés : les lui montrer, c’est, en quelque manière, l’en faire jouir. Cette chaîne d’or que l’on tendait à Venise, les richesses que l’on portait à Rome dans les triomphes, les trésors que l’on gardait dans le temple de Saturne étaient véritablement les richesses du peuple.\par
Il est surtout essentiel, dans l’aristocratie, que les nobles ne lèvent pas les tributs. Le premier ordre de l’État ne s’en mêlait point à Rome ; on en chargea le second, et cela même eut dans la suite de grands inconvénients. Dans une aristocratie où les nobles lèveraient les tributs, tous les particuliers seraient à la discrétion des gens d’affaires ; il n’y aurait point de tribunal supérieur qui les corrigeât. Ceux d’entre eux préposés pour ôter les abus, aimeraient mieux jouir des abus. Les nobles seraient comme les princes des États despotiques, qui confisquent les biens de qui il leur plaît.\par
Bientôt les profits qu’on y ferait seraient regardés comme un patrimoine, que l’avarice étendrait a sa fantaisie. On ferait tomber les fermes, on réduirait à rien les revenus publics. C’est par là que quelques États, sans avoir reçu d’échec qu’on puisse remarquer, tombent dans une faiblesse dont les voisins sont surpris, et qui étonne les citoyens mêmes.\par
Il faut que les lois leur défendent aussi le commerce : des marchands si accrédités feraient toutes sortes de monopoles. Le commerce est la profession des gens égaux ; et, parmi les États despotiques, les plus misérables sont ceux où le prince est marchand.\par
Les lois de Venise\footnote{Amelot de La Houssaye, {\itshape Du gouvernement de Venise}, partie III. La loi Claudia défendait aux sénateurs d’avoir en mer aucun vaisseau qui tînt plus de quarante muids. Tite-Live, liv. XXI, 63, 3.} défendent aux nobles le commerce qui pourrait leur donner, même innocemment, des richesses exorbitantes.\par
Les lois doivent employer les moyens les plus efficaces pour que les nobles rendent justice au peuple. Si elles n’ont point établi un tribun, il faut qu’elles soient un tribun elles-mêmes.\par
Toute sorte d’asile contre l’exécution des lois perd l’aristocratie ; et la tyrannie en est tout près.\par
Elles doivent mortifier, dans tous les temps, l’orgueil de la domination. Il faut qu’il y ait, pour un temps ou pour toujours, un magistrat qui fasse trembler les nobles, comme les éphores à Lacédémone, et les inquisiteurs d’État à Venise, magistratures qui ne sont soumises à aucunes formalités. Ce gouvernement a besoin de ressorts bien violents. Une bouche de pierre\footnote{Les délateurs y jettent leurs billets.} s’ouvre à tout délateur à Venise ; vous diriez que c’est celle de la tyrannie.\par
Ces magistratures tyranniques, dans l’aristocratie, ont du rapport à la censure de la démocratie, qui, par sa nature, n’est pas moins indépendante. En effet, les censeurs ne doivent point être recherchés sur les choses qu’ils ont faites pendant leur censure ; il faut leur donner de la confiance, jamais du découragement. Les Romains étaient admirables ; on pouvait faire rendre à tous les magistrats\footnote{Voyez Tite-Live, liv. XLIX. Un censeur ne pouvait pas même être troublé par un censeur : chacun faisait sa note sans prendre l’avis de son collègue ; et quand on fit autrement, la censure fut, pour ainsi dire, renversée.} raison de leur conduite, excepté aux censeurs\footnote{À Athènes, les logistes, qui faisaient rendre compte à tous les magistrats, ne rendaient point compte eux-mêmes.}.\par
Deux choses sont pernicieuses dans l’aristocratie : la pauvreté extrême des nobles, et leurs richesses exorbitantes. Pour prévenir leur pauvreté, il faut surtout les obliger de bonne heure à payer leurs dettes. Pour modérer leurs richesses, il faut des dispositions sages et insensibles ; non pas des confiscations, des lois agraires, des abolitions de dettes, qui font des maux infinis.\par
Les lois doivent ôter le droit d’aînesse entre les nobles\footnote{Cela est ainsi établi à Venise. Amelot de La Houssaye, p. 30 et 31.}, afin que, par le partage continuel des successions, les fortunes se remettent toujours dans l’égalité.\par
Il ne faut point de substitutions, de retraits lignagers, de majorats, d’adoptions. Tous les moyens inventés pour perpétuer la grandeur des familles dans les États monarchiques, ne sauraient être d’usage dans l’aristocratie\footnote{Il semble que l’objet de quelques aristocraties soit moins de maintenir l’État, que ce qu’elles appellent leur noblesse.}.\par
Quand les lois ont égalisé les familles, il leur reste à maintenir l’union entre elles. Les différends des nobles doivent être promptement décidés ; sans cela, les contestations entre les personnes deviennent des contestations entre les familles. Des arbitres peuvent terminer les procès, ou les empêcher de naître.\par
Enfin, il ne faut point que les lois favorisent les distinctions que la vanité met entre les familles, sous prétexte qu’elles sont plus nobles ou plus anciennes ; cela doit être mis au rang des petitesses des particuliers.\par
On n’a qu’à jeter les yeux sur Lacédémone ; on verra comment les éphores surent mortifier les faiblesses des rois, celles des grands et celles du peuple.
\subsubsection[{Chapitre IX. Comment les lois sont relatives à leur principe dans la monarchie}]{Chapitre IX. Comment les lois sont relatives à leur principe dans la monarchie}
\noindent L’honneur étant le principe de ce gouvernement, les lois doivent s’y rapporter.\par
Il faut qu’elles y travaillent à soutenir cette noblesse, dont l’honneur est, pour ainsi dire, l’enfant et le père.\par
Il faut qu’elles la rendent héréditaire, non pas pour être le terme entre le pouvoir du prince et la faiblesse du peuple, mais le lien de tous les deux.\par
Les substitutions, qui conservent les biens dans les familles, seront très utiles dans ce gouvernement, quoiqu’elles ne conviennent pas dans les autres.\par
Le retrait lignager rendra aux familles nobles les terres que la prodigalité d’un parent aura aliénées.\par
Les terres nobles auront des privilèges, comme les personnes. On ne peut pas séparer la dignité du monarque de celle du royaume ; on ne peut guère séparer non plus la dignité du noble de celle de son fief.\par
Toutes ces prérogatives seront particulières à la noblesse, et ne passeront point au peuple, si l’on ne veut choquer le principe du gouvernement, si l’on ne veut diminuer la force de la noblesse et celle du peuple.\par
Les substitutions gênent le commerce ; le retrait lignager fait une infinité de procès nécessaires ; et tous les fonds du royaume vendus sont au moins, en quelque façon, sans maître pendant un an. Des prérogatives attachées à des fiefs donnent un pouvoir très à charge à ceux qui les souffrent. Ce sont des inconvénients particuliers de la noblesse, qui disparaissent devant l’utilité générale qu’elle procure. Mais quand on les communique au peuple, on choque inutilement tous les principes.\par
On peut, dans les monarchies, permettre de laisser la plus grande partie de ses biens à un de ses enfants ; cette permission n’est même bonne que là.\par
Il faut que les lois favorisent tout le commerce\footnote{Elle ne le permet qu’au peuple. Voyez la loi troisième, au Code {\itshape De commercio et mercatoribus}, qui est pleine de bon sens.} que la constitution de ce gouvernement peut donner ; afin que les sujets puissent, sans périr, satisfaire aux besoins toujours renaissants du prince et de sa cour.\par
Il faut qu’elles mettent un certain ordre dans la manière de lever les tributs, afin qu’elle ne soit pas plus pesante que les charges mêmes.\par
La pesanteur des charges produit d’abord le travail ; le travail, l’accablement, l’esprit de paresse.
\subsubsection[{Chapitre X. De la promptitude de l’exécution dans la monarchie}]{Chapitre X. De la promptitude de l’exécution dans la monarchie}
\noindent Le gouvernement monarchique a un grand avantage sur le républicain : les affaires étant menées par un seul, il y a plus de promptitude dans l’exécution. Mais, comme cette promptitude pourrait dégénérer en rapidité, les lois y mettront une certaine lenteur. Elles ne doivent pas seulement favoriser la nature de chaque constitution, mais encore remédier aux abus qui pourraient résulter de cette même nature.\par
Le cardinal de Richelieu\footnote{{\itshape Testament politique}.} veut que l’on évite, dans les monarchies, les épines des compagnies, qui forment des difficultés sur tout. Quand cet homme n’aurait pas eu le despotisme dans le cœur, il l’aurait eu dans la tête.\par
Les corps qui ont le dépôt des lois n’obéissent jamais mieux que quand ils vont à pas tardifs, et qu’ils apportent, dans les affaires du prince, cette réflexion qu’on ne peut guère attendre du défaut de lumières de la cour sur les lois de l’État, ni de la précipitation de ses Conseils\footnote{{\itshape Barbaris cunctatio servilis : statim exsequi regium videtur.} Tacite, {\itshape Annal.}, liv. V.}.\par
Que serait devenue la plus belle monarchie du monde si les magistrats, par leurs lenteurs, par leurs plaintes, par leurs prières, n’avaient arrêté le cours des venus même de ses rois, lorsque ces monarques, ne consultant que leur grande âme, auraient voulu récompenser sans mesure des services rendus avec un courage et une fidélité aussi sans mesure ?
\subsubsection[{Chapitre XI. De l’excellence du gouvernement monarchique}]{Chapitre XI. De l’excellence du gouvernement monarchique}
\noindent Le gouvernement monarchique à un grand avantage sur le despotique. Comme il est de sa nature qu’il y ait sous le prince plusieurs ordres qui tiennent à la constitution, l’État est plus fixe, la constitution plus inébranlable, la personne de ceux qui gouvernent plus assurée.\par
Cicéron\footnote{Liv. III des {\itshape Lois}.} croit que l’établissement des tribuns de Rome fut le salut de la république. « En effet, dit-il, la force du peuple qui n’a point de chef est plus terrible. Un chef sent que l’affaire roule sur lui, il y pense ; mais le peuple, dans son impétuosité, ne connaît point le péril où il se jette. » On peut appliquer cette réflexion à un État despotique, qui est un peuple sans tribuns ; et à une monarchie, où le peuple a, en quelque façon, des tribuns.\par
En effet, on voit partout que, dans les mouvements du gouvernement despotique, le peuple, mené par lui-même, porte toujours les choses aussi loin qu’elles peuvent aller ; tous les désordres qu’il commet sont extrêmes ; au lieu que, dans les monarchies, les choses sont très rarement portées à l’excès. Les chefs craignent pour eux-mêmes ; ils ont peur d’être abandonnés ; les puissances intermédiaires dépendantes\footnote{Voyez ci-dessus la première note du livre II, chap. IV.} ne veulent pas que le peuple prenne trop le dessus. Il est rare que les ordres de l’État soient entièrement corrompus. Le prince tient à ces ordres : et les séditieux, qui n’ont ni la volonté ni l’espérance de renverser l’État, ne peuvent ni ne veulent renverser le prince.\par
Dans ces circonstances, les gens qui ont de la sagesse et de l’autorité s’entremettent ; on prend des tempéraments, on s’arrange, on se corrige ; les lois reprennent leur vigueur et se font écouter.\par
Aussi toutes nos histoires sont-elles pleines de guerres civiles sans révolutions ; celles des États despotiques sont pleines de révolutions sans guerres civiles.\par
Ceux qui ont écrit l’histoire des guerres civiles de quelques États, ceux mêmes qui les ont fomentées, prouvent assez combien l’autorité que les princes laissent à de certains ordres pour leur service, leur doit être peu suspecte ; puisque, dans l’égarement même, ils ne soupiraient qu’après les lois et leur devoir, et retardaient la fougue et l’impétuosité des factieux plus qu’ils ne pouvaient la servir\footnote{{\itshape Mémoires} du cardinal de Retz et autres histoires.}.\par
Le cardinal de Richelieu, pensant peut-être qu’il avait trop avili les ordres de l’État, a recours, pour le soutenir, aux vertus du prince et de ses ministres\footnote{{\itshape Testament politique}.} ; et il exige d’eux tant de choses, qu’en vérité il n’y a qu’un ange qui puisse avoir tant d’attention, tant de lumières, tant de fermeté, tant de connaissances ; et on peut à peine se flatter que, d’ici à la dissolution des monarchies, il puisse y avoir un prince et des ministres pareils.\par
Comme les peuples qui vivent sous une bonne police sont plus heureux que ceux qui, sans règle et sans chefs, errent dans les forêts ; aussi les monarques qui vivent sous les lois fondamentales de leur État, sont-ils plus heureux que les princes despotiques, qui n’ont rien qui puisse régler le cœur de leurs peuples, ni le leur.
\subsubsection[{Chapitre XII. Continuation du même sujet}]{Chapitre XII. Continuation du même sujet}
\noindent Qu’on n’aille point chercher de la magnanimité dans les États despotiques ; le prince n’y donnerait point une grandeur qu’il n’a pas lui-même : chez lui, il n’y a pas de gloire.\par
C’est dans les monarchies que l’on verra autour du prince les sujets recevoir ses rayons ; c’est là que chacun, tenant, pour ainsi dire, un plus grand espace, peut exercer ces vertus qui donnent à l’âme, non pas de l’indépendance, mais de la grandeur.
\subsubsection[{Chapitre XIII. Idée du despotisme}]{Chapitre XIII. Idée du despotisme}
\noindent Quand les sauvages de la Louisiane veulent avoir du fruit, ils coupent l’arbre au pied, et cueillent le fruit\footnote{{\itshape Lettres édifiantes}, recueil I, p. 315.}. Voilà le gouvernement despotique.
\subsubsection[{Chapitre XIV. Comment les lois sont relatives au principe du gouvernement despotique}]{Chapitre XIV. Comment les lois sont relatives au principe du gouvernement despotique}
\noindent Le gouvernement despotique a pour principe la crainte : mais à des peuples timides, ignorants, abattus, il ne faut pas beaucoup de lois.\par
Tout y doit rouler sur deux ou trois idées : il n’en faut donc pas de nouvelles. Quand vous instruisez une bête, vous vous donnez bien de garde de lui faire changer de maître, de leçon et d’allure ; vous frappez son cerveau par deux ou trois mouvements, et pas davantage.\par
Lorsque le prince est enfermé, il ne peut sortir du séjour de la volupté sans désoler tous ceux qui l’y retiennent. Ils ne peuvent souffrir que sa personne et son pouvoir passent en d’autres mains. Il fait donc rarement la guerre en personne, et il n’ose guère la faire par ses lieutenants.\par
Un prince pareil, accoutumé dans son palais à ne trouver aucune résistance, s’indigne de celle qu’on lui fait les armes à la main ; il est donc ordinairement conduit par la colère ou par la vengeance. D’ailleurs il ne peut avoir d’idée de la vraie gloire. Les guerres doivent donc S’Y faire dans toute leur fureur naturelle, et le droit des {\itshape gens} y avoir moins d’étendue qu’ailleurs.\par
Un tel prince a tant de défauts qu’il faudrait craindre d’exposer au grand jour sa stupidité naturelle. Il est caché, et l’on ignore l’état où il se trouve. Par bonheur, les hommes sont tels dans ce pays, qu’ils n’ont besoin que d’un nom qui les gouverne.\par
Charles XII, étant à Bender, trouvant quelque résistance dans le sénat de Suède, écrivit qu’il leur enverrait une de ses bottes pour commander. Cette botte aurait commandé comme un roi despotique.\par
Si le prince est prisonnier, il est censé être mort, et un autre monte sur le trône. Les traités que fait le prisonnier sont nuls ; son successeur ne les ratifierait pas. En effet, comme il est les lois, l’État et le prince, et que sitôt qu’il n’est plus le prince, il n’est rien ; s’il n’était pas censé mort, l’État serait détruit.\par
Une des choses qui détermina le plus les Turcs à faire leur paix séparée avec Pierre I\textsuperscript{er}, fut que les Moscovites dirent au vizir qu’en Suède on avait mis un autre roi sur le trône\footnote{Suite de Pufendorf, {\itshape Histoire universelle}, au traité de la Suède, chap. X.}.\par
La conservation de l’État n’est que la conservation du prince, ou plutôt du palais où il est enfermé. Tout ce qui ne menace pas directement ce palais ou la ville capitale ne fait point d’impression sur des esprits ignorants, orgueilleux et prévenus ; et, quant à l’enchaînement des événements, ils ne peuvent le suivre, le prévoir, y penser même. La politique, ses ressorts et ses lois y doivent être très bornées ; et le gouvernement politique y est aussi simple que le gouvernement civil\footnote{Selon M. Chardin, il n’y a point de Conseil d’État en Perse.}.\par
Tout se réduit à concilier le gouvernement politique et civil avec le gouvernement domestique, les officiers de l’État avec ceux du sérail.\par
Un pareil État sera dans la meilleure situation, lorsqu’il pourra se regarder comme seul dans le monde ; qu’il sera environné de déserts, et séparé des peuples qu’il appellera barbares. Ne pouvant compter sur la milice, il sera bon qu’il détruise une partie de lui-même.\par
Comme le principe du gouvernement despotique est la crainte, le but en est la tranquillité ; mais ce n’est point une paix, c’est le silence de ces villes que l’ennemi est près d’occuper.\par
La force n’étant pas dans l’État, mais dans l’armée qui l’a fondé, il faudrait, pour défendre l’État, conserver cette armée ; mais elle est formidable au prince. Comment donc concilier la sûreté de l’État avec la sûreté de la personne ?\par
Voyez, je vous prie, avec quelle industrie le gouvernement moscovite cherche à sortir du despotisme, qui lui est plus pesant qu’aux peuples mêmes. On a cassé les grands corps de troupes ; on a diminué les peines des crimes ; on a établi des tribunaux ; on a commencé à connaître les lois ; on a instruit les peuples. Mais il y a des causes particulières, qui le ramèneront peut-être au malheur qu’il voulait fuir.\par
Dans ces États, la religion a plus d’influence que dans aucun autre ; elle est une crainte ajoutée à la crainte. Dans les empires mahométans, c’est de la religion que les peuples tirent en partie le respect étonnant qu’ils ont pour leur prince.\par
C’est la religion qui corrige un peu la constitution turque. Les sujets, qui ne sont pas attachés à la gloire et à la grandeur de l’État par honneur, le sont par la force et par le principe de la religion.\par
De tous les gouvernements despotiques, il n’y en a point qui s’accable plus lui-même, que celui où le prince se déclare propriétaire de tous les fonds de terre, et l’héritier de tous ses sujets. Il en résulte toujours l’abandon de la culture des terres ; et, si d’ailleurs le prince est marchand, toute espèce d’industrie est ruinée.\par
Dans ces États, on ne répare, on n’améliore rien\footnote{Voyez Ricaut, {\itshape État de l’empire ottoman}, p. 196.}. On ne bâtit de maisons que pour la vie, on ne fait point de fossés, on ne plante point d’arbres ; on tire tout de la terre, on ne lui rend rien ; tout est en friche, tout est désert.\par
Pensez-vous que des lois qui ôtent la propriété des fonds de terre et la succession des biens, diminueront l’avarice et la cupidité des grands ? Non : elles irriteront cette cupidité et cette avarice. On sera porté à faire mille vexations, parce qu’on ne croira avoir en propre que l’or ou l’argent que l’on pourra voler ou cacher.\par
Pour que tout ne soit pas perdu, il est bon que l’avidité du prince soit modérée par quelque coutume. Ainsi, en Turquie, le prince se contente ordinairement de prendre trois pour cent sur les successions\footnote{Voyez, sur les successions des Turcs, {\itshape Lacédémone ancienne et moderne}. Voyez aussi Ricaut, {\itshape De l’Empire ottoman}.} des gens du peuple. Mais, comme le grand seigneur donne la plupart des terres à sa milice, et en dispose à sa fantaisie ; comme il se saisit de toutes les successions des officiers de l’empire ; comme, lorsqu’un homme meurt sans enfants mâles, le grand seigneur a la propriété, et que les filles n’ont que l’usufruit, il arrive que la plupart des biens de l’État sont possédés d’une manière précaire.\par
Par la loi de Bantam\footnote{{\itshape Recueil des voyages qui ont servi à l’établissement de la Compagnie des Indes}, t. I. La loi de Pégu est moins cruelle ; si on a des enfants, le roi ne succède qu’aux deux tiers. {\itshape Ibid., t.} III, p. 1.} le roi prend la succession, même la femme, les enfants et la maison. On est obligé, pour éluder la plus cruelle disposition de cette loi, de marier les enfants à huit, neuf ou dix ans, et quelquefois plus jeunes, afin qu’ils ne se trouvent pas faire une malheureuse partie de la succession du père.\par
Dans les États où il n’y a point de lois fondamentales, la succession à l’empire ne saurait être fixe. La couronne y est élective par le prince, dans sa famille, ou hors de sa famille. En vain serait-il établi que l’aîné succéderait ; le prince en pourrait toujours choisir un autre. Le successeur est déclaré par le prince lui-même, ou par ses ministres, ou par une guerre civile. Ainsi cet État a une raison de dissolution de plus qu’une monarchie.\par
Chaque prince de la famille royale ayant une égale capacité pour être élu, il arrive que celui qui monte sur le trône fait d’abord étrangler ses frères, comme en Turquie ; ou les fait aveugler, comme en Perse ; ou les rend fous, comme chez le Mogol : ou, si l’on ne prend point ces précautions, comme à Maroc, chaque vacance de trône est suivie d’une affreuse guerre civile.\par
Par les constitutions de Moscovie\footnote{Voyez les différentes constitutions, surtout celle de 1722.} le czar peut choisir qui il veut pour son successeur, soit dans sa famille, soit hors de sa famille. Un tel établissement de succession cause mille révolutions, et rend le trône aussi chancelant que la succession est arbitraire. L’ordre de succession étant une des choses qu’il importe le plus au peuple de savoir, le meilleur est celui qui frappe le plus les yeux, comme la naissance, et un certain ordre de naissance. Une telle disposition arrête les brigues, étouffe l’ambition ; on ne captive plus l’esprit d’un prince faible, et l’on ne fait point parler les mourants.\par
Lorsque la succession est établie par une loi fondamentale, un seul prince est le successeur, et ses frères n’ont aucun droit réel ou apparent de lui disputer la couronne. On ne peut présumer ni faire valoir une volonté particulière du père. Il n’est donc pas plus question d’arrêter ou de faire mourir le frère du roi, que quelque autre sujet que ce soit.\par
Mais dans les États despotiques, où les frères du prince sont également ses esclaves et ses rivaux, la prudence veut que l’on s’assure de leurs personnes, surtout dans les pays mahométans, où la religion regarde la victoire ou le succès comme un jugement de Dieu ; de sorte que personne n’y est souverain de droit, mais seulement de fait.\par
L’ambition est bien plus irritée dans des États où des princes du sang voient que, s’ils ne montent pas sur le trône, ils seront enfermés ou mis à mort, que parmi nous où les princes du sang jouissent d’une condition qui, si elle n’est pas si satisfaisante pour l’ambition, l’est peut-être plus pour les désirs modérés.\par
Les princes des États despotiques ont toujours abusé du mariage. Ils prennent ordinairement plusieurs femmes, surtout dans la partie du monde où le despotisme est, pour ainsi dire, naturalisé, qui est l’Asie. Ils en ont tant d’enfants, qu’ils ne peuvent guère avoir d’affection pour eux, ni ceux-ci pour leurs frères.\par
La famille régnante ressemble à l’État : elle est trop faible, et son chef est trop fort ; elle paraît étendue, et elle se réduit à rien. Artaxerxès\footnote{Voyez Justin.} fit mourir tous ses enfants, pour avoir conjuré contre lui. Il n’est pas vraisemblable que cinquante enfants conspirent contre leur père ; et encore moins qu’ils conspirent, parce qu’il n’a pas voulu céder sa concubine à son fils aîné. Il est plus simple de croire qu’il y a là quelque intrigue de ces sérails d’Orient ; de ces lieux où l’artifice, la méchanceté, la ruse règnent dans le silence, et se couvrent d’une épaisse nuit ; où un vieux prince, devenu tous les jours plus imbécile, est le premier prisonnier du palais.\par
Après tout ce que nous venons de dire, il semblerait que la nature humaine se soulèverait sans cesse contre le gouvernement despotique. Mais, malgré l’amour des hommes pour la liberté, malgré leur haine contre la violence, la plupart des peuples y sont soumis. Cela est aisé à comprendre. Pour former un gouvernement modéré, il faut combiner les puissances, les régler, les tempérer, les faire agir ; donner, pour ainsi dire, un lest à l’une, pour la mettre en état de résister à une autre ; c’est un chef-d’œuvre de législation, que le hasard fait rarement, et que rarement on laisse faire à la prudence. Un gouvernement despotique, au contraire, saute, pour ainsi dire, aux yeux ; il est uniforme partout : comme il ne faut que des passions pour l’établir, tout le monde est bon pour cela.
\subsubsection[{Chapitre XV. Continuation du même sujet}]{Chapitre XV. Continuation du même sujet}
\noindent Dans les climats chauds, où règne ordinairement le despotisme, les passions se font plus tôt sentir, et elles sont aussi plus tôt amorties\footnote{Voyez le livre {\itshape Des Lois}, dans le rapport avec la nature du climat.} \footnote{La Guilletière, {\itshape Lacédémone ancienne et nouvelle}, p. 463.}; l’esprit y est plus avancé ; les périls de la dissipation des biens y sont moins grands ; il y a moins de facilité de se distinguer, moins de commerce entre les jeunes gens renfermés dans la maison ; on s’y marie de meilleure heure : on y peut donc être majeur plus tôt que dans nos climats d’Europe. En Turquie, la majorité commence à quinze ans.\par
La cession des biens n’y peut avoir lieu. Dans un gouvernement où personne n’a de fortune assurée, on prête plus à la personne qu’aux biens.\par
Elle entre naturellement dans les gouvernements modérés\footnote{Il en est de même des atermoiements dans les banqueroutes de bonne foi.}, et surtout dans les républiques, à cause de la plus grande confiance que l’on doit avoir dans la probité des citoyens, et de la douceur que doit inspirer une forme de gouvernement que chacun semble s’être donnée lui-même.\par
Si dans la république romaine les législateurs avaient établi la cession de biens\footnote{Elle ne fut établie que par la loi Julie, {\itshape De cessione bonorum.} On évitait la prison et la ignominieuse des biens.}, on ne serait pas tombé dans tant de séditions et de discordes civiles, et on n’aurait point essuyé les dangers des maux, ni les périls des remèdes.\par
La pauvreté et l’incertitude des fortunes, dans les États despotiques, y naturalisent l’usure ; chacun augmentant le prix de son argent à proportion du péril qu’il y a à le prêter. La misère vient donc de toutes parts dans ces pays malheureux ; tout y est ôté, jusqu’à la ressource des emprunts.\par
Il arrive de là qu’un marchand n’y saurait faire un grand commerce ; il vit au jour la journée : s’il se chargeait de beaucoup de marchandises, il perdrait plus par les intérêts qu’il donnerait pour les payer, qu’il ne gagnerait sur les marchandises. Aussi les lois sur le commerce n’y ont-elles guère de lieu ; elles se réduisent à la simple police.\par
Le gouvernement ne saurait être injuste sans avoir des mains qui exercent ses injustices : or il est impossible que ces mains ne s’emploient pour elles-mêmes. Le péculat est donc naturel dans les États despotiques.\par
Ce crime y étant le crime ordinaire, les confiscations y sont utiles. Par là on console le peuple ; l’argent qu’on en tire est un tribut considérable que le prince lèverait difficilement sur des sujets abîmés : il n’y a même dans ce pays aucune famille qu’on veuille conserver.\par
Dans les États modérés, c’est tout autre chose. Les confiscations rendraient la propriété des biens incertaine ; elles dépouilleraient des enfants innocents ; elles détruiraient une famille, lorsqu’il ne s’agirait que de punir un coupable. Dans les républiques, elles feraient le mal d’ôter l’égalité qui en fait l’âme, en privant un citoyen de son nécessaire physique\footnote{Il me semble qu’on aimait trop les confiscations dans la république d’Athènes.}.\par
Une loi romaine\footnote{{\itshape Authentica, Bona Damnatorum}. Code, {\itshape De bonis proscriptorum seu damnatorum}.} veut qu’on ne confisque que dans le cas du crime de lèse-majesté au premier chef. Il serait souvent très sage de suivre l’esprit de cette loi, et de borner les confiscations à de certains crimes. Dans les pays où une coutume locale a disposé des {\itshape propres}, Bodin\footnote{Liv. V, chap. III.} dit très bien qu’il ne faudrait confisquer que les {\itshape acquêts}.
\subsubsection[{Chapitre XVI. De la communication du pouvoir}]{Chapitre XVI. De la communication du pouvoir}
\noindent Dans le gouvernement despotique, le pouvoir passe tout entier dans les mains de celui à qui on le confie. Le vizir est le despote lui-même ; et chaque officier particulier est le vizir. Dans le gouvernement monarchique, le pouvoir s’applique moins immédiatement ; le monarque, en le donnant, le tempère\footnote{{\itshape Ut esse Phoebi dulcius lumen solet} / {\itshape Jamjam cadentis…}}. Il fait une telle distribution de son autorité, qu’il n’en donne jamais une partie, qu’il n’en retienne une plus grande.\par
Ainsi, dans les États monarchiques, les gouverneurs particuliers des villes ne relèvent pas tellement du gouverneur de la province, qu’ils ne relèvent du prince encore davantage ; et les officiers particuliers des corps militaires ne dépendent pas tellement du général, qu’ils ne dépendent du prince encore plus.\par
Dans la plupart des États monarchiques, on a sagement établi que ceux qui ont un commandement un peu étendu ne soient attachés à aucun corps de milice ; de sorte que, n’ayant de commandement que par une volonté particulière du prince, pouvant être employés et ne l’être pas, ils sont en quelque façon dans le service, et en quelque façon dehors.\par
Ceci est incompatible avec le gouvernement despotique. Car, si ceux qui n’ont pas un emploi actuel avaient néanmoins des prérogatives et des titres, il y aurait dans l’État des hommes grands par eux-mêmes ; ce qui choquerait la nature de ce gouvernement.\par
Que si le gouverneur d’une ville était indépendant du bacha, il faudrait tous les jours des tempéraments pour les accommoder ; chose absurde dans un gouvernement despotique. Et, de plus, le gouverneur particulier pouvant ne pas obéir, comment l’autre pourrait-il répondre de sa province sur sa tête ?\par
Dans ce gouvernement, l’autorité ne peut être balancée ; celle du moindre magistrat ne l’est pas plus que celle du despote. Dans les pays modérés, la loi est partout sage, elle est partout connue, et les plus petits magistrats peuvent la suivre. Mais dans le despotisme, où la loi n’est que la volonté du prince, quand le prince serait sage, comment un magistrat pourrait-il suivre une volonté qu’il ne connaît pas ? Il faut qu’il suive la sienne.\par
Il y a plus : c’est que la loi n’étant que ce que le prince veut, et le prince ne pouvant vouloir que ce qu’il connaît, il faut bien qu’il y ait une infinité de gens qui veuillent pour lui et comme lui.\par
Enfin, la loi étant la volonté momentanée du prince, il est nécessaire que ceux qui veulent pour lui, veuillent subitement comme lui.
\subsubsection[{Chapitre XVII. Des présents}]{Chapitre XVII. {\itshape Des présents}}
\noindent C’est un usage, dans les pays despotiques, que l’on n’aborde qui que ce soit au-dessus de soi, sans lui faire un présent, pas même les rois. L’empereur du Mogol\footnote{{\itshape Recueil des voyages qui ont servi à l’établissement de la Compagnie des Indes}, t. I, p. 80.} ne reçoit point les requêtes de ses sujets, qu’il n’en ait reçu quelque chose. Ces princes vont jusqu’à corrompre leurs propres grâces.\par
Cela doit être ainsi dans un gouvernement où personne n’est citoyen ; dans un gouvernement où l’on est plein de l’idée que le supérieur ne doit rien à l’inférieur ; dans un gouvernement où les hommes ne se croient liés que par les châtiments que les uns exercent sur les autres ; dans un gouvernement où il y a peu d’affaires, et où il est rare que l’on ait besoin de se présenter devant un grand, de lui faire des demandes, et encore moins des plaintes.\par
Dans une république, les présents sont une chose odieuse, parce que la vertu n’en a pas besoin. Dans une monarchie, l’honneur est un motif plus fort que les présents. Mais, dans l’État despotique, où il n’y a ni honneur ni vertu, on ne peut être déterminé à agir que par l’espérance des commodités de la vie.\par
C’est dans les idées de la république que Platon\footnote{Liv. XII des {\itshape Lois}.} voulait que ceux qui reçoivent des présents pour faire leur devoir, fussent punis de mort : Il n’en faut prendre, disait-il, ni pour les choses bonnes, ni pour les mauvaises.\par
C’était une mauvaise loi que cette loi romaine\footnote{Leg. 6, §2, {\itshape Dig. ad leg. Jul. repet}.} qui permettait aux magistrats de prendre de petits présents\footnote{{\itshape Munuscula}.}, pourvu qu’ils ne passassent pas cent écus dans toute l’année. Ceux à qui on ne donne rien, ne désirent rien ; ceux à qui on donne un peu, désirent bientôt un peu plus, et ensuite beaucoup. D’ailleurs, il est plus aisé de convaincre celui qui, ne devant rien prendre, prend quelque chose, que celui qui prend plus, lorsqu’il devrait prendre moins, et qui trouve toujours, pour cela, des prétextes, des excuses, des causes et des raisons plausibles.
\subsubsection[{Chapitre XVIII. Des récompenses que le souverain donne}]{Chapitre XVIII. Des récompenses que le souverain donne}
\noindent Dans les gouvernements despotiques, où, comme nous avons dit, on n’est déterminé à agir que par l’espérance des commodités de la vie, le prince qui récompense n’a que de l’argent à donner. Dans une monarchie, où l’honneur règne seul, le prince ne récompenserait que par des distinctions, si les distinctions que l’honneur établit n’étaient jointes à un luxe qui donne nécessairement des besoins : le prince y récompense donc par des honneurs qui mènent à la fortune. Mais, dans une république, où la vertu règne, motif qui se suffit à lui-même et qui exclut tous les autres, l’État ne récompense que par des témoignages de cette vertu.\par
C’est une règle générale, que les grandes récompenses dans une monarchie et dans une république sont un signe de leur décadence, parce qu’elles prouvent que leurs principes sont corrompus ; que, d’un côté, l’idée de l’honneur n’y a plus tant de force ; que, de l’autre, la qualité de citoyen s’est affaiblie.\par
Les plus mauvais empereurs romains ont été ceux qui ont le plus donné : par exemple, Caligula, Claude, Néron, Othon, Vitellius, Commode, Héliogabale et Caracalla. Les meilleurs, comme Auguste, Vespasien, Antonin Pie, Marc Aurèle et Pertinax, ont été économes. Sous les bons empereurs, l’État reprenait ses principes ; le trésor de l’honneur suppléait aux autres trésors.
\subsubsection[{Chapitre XIX. Nouvelles conséquences des principes des trois gouvernements}]{Chapitre XIX. Nouvelles conséquences des principes des trois gouvernements}
\noindent Je ne puis me résoudre à finir ce livre sans faire encore quelques applications de mes trois principes.\par
PREMIÈRE QUESTION. Les lois doivent-elles forcer un citoyen à accepter les emplois publics ? Je dis qu’elles le doivent dans le gouvernement républicain, et non pas dans le monarchique. Dans le premier, les magistratures sont des témoignages de vertu, des dépôts que la patrie confie à un citoyen, qui ne doit vivre, agir et penser que pour elle ; il ne peut donc pas les refuser\footnote{Platon, dans sa {\itshape République}, liv. VIII, met ces refus au nombre des marques de la corruption de la république. Dans ses {\itshape Lois}, liv. VI, il veut qu’on les punisse par une amende. À Venise, on les punit par l’exil.}. Dans le second, les magistratures sont des témoignages d’honneur ; or telle est la bizarrerie de l’honneur, qu’il se plaît à n’en accepter aucun que quand il veut, et de la manière qu’il veut.\par
Le feu roi de Sardaigne\footnote{Victor Amédée.} punissait ceux qui refusaient les dignités et les emplois de son État ; il suivait, sans le savoir, des idées républicaines. Sa manière de gouverner, d’ailleurs, prouve assez que ce n’était pas là son intention.\par
SECONDE QUESTION. Est-ce une bonne maxime qu’un citoyen puisse être obligé d’accepter, dans l’armée, une place inférieure à celle qu’il a occupée ? On voyait souvent, chez les Romains, le capitaine servir, l’année d’après, sous son lieutenant\footnote{Quelques centurions ayant appelé au peuple pour demander l’emploi qu’ils avaient eu : {\itshape Il est juste, mes compagnons}, dit un centurion, {\itshape que vous regardiez comme honorables tous les postes où vous défendrez la république.} Tite-Live, liv. XLII.}. C’est que, dans les républiques, la vertu demande qu’on fasse à l’État un sacrifice continuel de soi-même et de ses répugnances. Mais, dans les monarchies, l’honneur, vrai ou faux, ne peut souffrir ce qu’il appelle se dégrader.\par
Dans les gouvernements despotiques, où l’on abuse également de l’honneur, des postes et des rangs, on fait indifféremment d’un prince un goujat, et d’un goujat un prince.\par
TROISIÈME QUESTION. Mettra-t-on sur une même tête les emplois civils et militaires ? Il faut les unir dans la république, et les séparer dans la monarchie. Dans les républiques, il serait bien dangereux de faire de la profession des armes un état particulier, distingué de celui qui a les fonctions civiles ; et, dans les monarchies, il n’y aurait pas moins de péril à donner les deux fonctions à la même personne.\par
On ne prend les armes, dans la république, qu’en qualité de défenseur des lois et de la patrie ; c’est parce que l’on est citoyen qu’on se fait, pour un temps, soldat. S’il y avait deux états distingués, on ferait sentir à celui qui, sous les armes, se croit citoyen, qu’il n’est que soldat.\par
Dans les monarchies, les gens de guerre n’ont pour objet que la gloire, ou du moins l’honneur, ou la fortune. On doit bien se garder de donner les emplois civils à des hommes pareils ; il faut, au contraire, qu’ils soient contenus par les magistrats civils, et que les mêmes gens n’aient pas en même temps la confiance du peuple et la force pour en abuser\footnote{{\itshape Ne imperium ad optimos nobilium transferretur, senatum militia vetuit Gallienus ; etiam adire exercitum.} Aurelius Victor, {\itshape De viris illustribus}.}.\par
Voyez, dans une nation où la république se cache sous la forme de la monarchie, combien l’on craint un état particulier de gens de guerre, et comment le guerrier reste toujours citoyen, ou même magistrat, afin que ces qualités soient un gage pour la patrie, et qu’on ne l’oublie jamais.\par
Cette division de magistratures en civiles et militaires, faite par les Romains après la perte de la république, ne fut pas une chose arbitraire. Elle fut une suite du changement de la constitution de Rome, elle était de la nature du gouvernement monarchique. Et ce qui ne fut que commencé sous Auguste\footnote{Auguste ôta aux sénateurs, proconsuls et gouverneurs, le droit de porter les armes. Dion, liv. XXXIII.}, les empereurs suivants\footnote{Constantin. Voyez Zozime, liv. II.} furent obligés de l’achever, pour tempérer le gouvernement militaire.\par
Ainsi Procope, concurrent de Valens à l’empire, n’y entendait rien, lorsque, donnant à Hormisdas, prince du sang royal de Perse, la dignité de proconsul\footnote{Ammien Marcellin, liv. XXVI. {\itshape More veterum, et bella rectum}.}, il rendit à cette magistrature le commandement des armées qu’elle avait autrefois ; à moins qu’il n’eût des raisons particulières. Un homme qui aspire à la souveraineté cherche moins ce qui est utile à l’État que ce qui l’est à sa cause.\par
QUATRIÈME QUESTION. Convient-il que les charges soient vénales ? Elles ne doivent pas l’être dans les États despotiques, où il faut que les sujets soient placés ou déplacés dans un instant par le prince.\par
Cette vénalité est bonne dans les États monarchiques, parce qu’elle fait faire, comme un métier de famille, ce qu’on ne voudrait pas entreprendre pour la vertu ; qu’elle destine chacun à son devoir, et rend les ordres de l’État plus permanents. Suidas\footnote{Fragments tirés des {\itshape Ambassades} de Constantin Porphyrogénète.} dit très bien qu’Anastase avait fait de l’empire une espèce d’aristocratie en vendant toutes les magistratures.\par
Platon\footnote{{\itshape République}, liv. VIII.} ne peut souffrir cette vénalité. « C’est, dit-il, comme si, dans un navire, on faisait quelqu’un pilote ou matelot pour son argent. Serait-il possible que la règle fût mauvaise dans quelque autre emploi que ce fût de la vie, et bonne seulement pour conduire une république ? » Mais Platon parle d’une république fondée sur la vertu, et nous parlons d’une monarchie. Or, dans une monarchie où, quand les charges ne se vendraient pas par un règlement public, l’indigence et l’avidité des courtisans les vendraient tout de même ; le hasard donnera de meilleurs sujets que le choix du prince. Enfin, la manière de s’avancer par les richesses inspire et entretient l’industrie\footnote{Paresse de l’Espagne ; on y donne tous les emplois.} {\itshape ;} chose dont cette espèce de gouvernement a grand besoin.\par
CINQUIÈME QUESTION. Dans quel gouvernement faut-il des censeurs ? Il en faut dans une république, où le principe du gouvernement est la vertu. Ce ne sont pas seulement les crimes qui détruisent la vertu, mais encore les négligences, les fautes, une certaine tiédeur dans l’amour de la patrie, des exemples dangereux, des semences de corruption ; ce qui ne choque point les lois, mais les élude ; ce qui ne les détruit pas, mais les affaiblit : tout cela doit être corrigé par les censeurs.\par
On est étonné de la punition de cet aréopagite qui avait tué un moineau qui, poursuivi par un épervier, s’était réfugié dans son sein. On est surpris que l’Aréopage ait fait mourir un enfant qui avait crevé les yeux à son oiseau. Qu’on fasse attention qu’il ne s’agit point là d’une condamnation pour crime, mais d’un jugement de mœurs dans une république fondée sur les mœurs.\par
Dans les monarchies, il ne faut point de censeurs ; elles sont fondées sur l’honneur, et la nature de l’honneur est d’avoir pour censeur tout l’univers. Tout homme qui y manque est soumis aux reproches de ceux mêmes qui n’en ont point.\par
Là, les censeurs seraient gâtés par ceux mêmes qu’ils devraient corriger. Ils ne seraient pas bons contre la corruption d’une monarchie ; mais la corruption d’une monarchie serait trop forte contre eux.\par
On sent bien qu’il ne faut point de censeurs dans les gouvernements despotiques. L’exemple de la Chine semble déroger à cette règle ; mais nous verrons, dans la suite de cet ouvrage, les raisons singulières de cet établissement.
\subsection[{Livre sixième. Conséquences des principes des divers gouvernements par rapport à la simplicité des lois civiles et criminelles, la forme des jugements et l’établissement des peines}]{Livre sixième. Conséquences des principes des divers gouvernements par rapport à la simplicité des lois civiles et criminelles, la forme des jugements et l’établissement des peines}
\subsubsection[{Chapitre I. De la simplicité des lois civiles dans les divers gouvernements}]{Chapitre I. De la simplicité des lois civiles dans les divers gouvernements}
\noindent Le gouvernement monarchique ne comporte pas des lois aussi simples que le despotique. Il y faut des tribunaux. Ces tribunaux donnent des décisions. Elles doivent être conservées ; {\itshape elles} doivent être apprises, pour que l’on y juge aujourd’hui comme l’on y jugea hier, et que la propriété et la vie des citoyens y soient assurées et fixes comme la constitution même de l’État.\par
Dans une monarchie, l’administration d’une justice qui ne décide pas seulement de la vie et des biens, mais aussi de l’honneur, demande des recherches scrupuleuses. La délicatesse du juge augmente à mesure qu’il a un plus grand dépôt, et qu’il prononce sur de plus grands intérêts.\par
Il ne faut donc pas être étonné de trouver dans les lois de ces États tant de règles, de restrictions, d’extensions, qui multiplient les cas particuliers, et semblent faire un art de la raison même.\par
La différence de rang, d’origine, de condition, qui est établie dans le gouvernement monarchique, entraîne souvent des distinctions dans la nature des biens ; et des lois relatives à la constitution de cet État peuvent {\itshape augmenter le} nombre de ces distinctions. Ainsi, parmi nous, les biens sont propres, acquêts ou conquêts ; dotaux, paraphernaux ; paternels et maternels ; meubles de plusieurs espèces ; libres, substitués ; du lignage ou non ; nobles, en franc-alleu, ou roturiers ; rentes foncières, ou constituées à prix d’argent. Chaque sorte de bien est soumise à des règles particulières ; il faut les suivre pour en disposer : ce qui ôte encore de la simplicité.\par
Dans nos gouvernements, les fiefs sont devenus héréditaires. Il a fallu que la noblesse eût une certaine consistance, afin que le propriétaire du fief fût en état de servir le prince. Cela a dû produire bien des variétés : par exemple, il y a des pays où l’on n’a pu partager les fiefs {\itshape entre les} frères ; dans d’autres, les cadets ont pu avoir leur subsistance avec plus d’étendue.\par
Le monarque, qui connaît chacune de ses provinces, peut établir diverses lois, ou souffrir différentes coutumes. Mais le despote ne connaît rien, et ne peut avoir d’attention sur rien ; il lui faut une allure générale ; il gouverne par une volonté rigide qui est partout la même ; tout s’aplanit sous ses pieds.\par
À mesure que les jugements des tribunaux se multiplient dans les monarchies, la jurisprudence se charge de décisions qui quelquefois se contredisent, ou parce que les juges qui se succèdent pensent différemment, ou parce que les mêmes affaires sont tantôt bien, tantôt mal défendues, ou enfin par une infinité d’abus qui se glissent dans tout ce qui passe par la main des hommes. C’est un mal nécessaire, que le législateur corrige de temps en temps, comme contraire même à l’esprit des gouvernements modérés. Car, quand on est obligé de recourir aux tribunaux, il faut que cela vienne de la nature de la constitution, et non pas des contradictions et de l’incertitude des lois.\par
Dans les gouvernements où il y a nécessairement des distinctions dans les personnes, il faut qu’il y ait des privilèges. Cela diminue encore la simplicité, et fait mille exceptions.\par
Un des privilèges le moins à charge à la société, et surtout à celui qui le donne, c’est de plaider devant un tribunal plutôt que devant un autre. Voilà de nouvelles affaires ; c’est-à-dire, celles où il s’agit de savoir devant quel tribunal il faut plaider.\par
Les peuples des États despotiques sont dans un cas bien différent. Je ne sais sur quoi, dans ces pays, le législateur pourrait statuer, ou le magistrat juger. Il suit de ce que les terres appartiennent au prince, qu’il n’y a presque point de lois civiles sur la propriété des terres. Il suit du droit que le souverain a de succéder, qu’il n’y en a pas non plus sur les successions. Le négoce exclusif qu’il fait, dans quelques pays, rend inutiles toutes sortes de lois sur le commerce. Les mariages que l’on y contracte avec des filles esclaves, font qu’il n’y a guère de lois civiles sur les dots et sur les avantages des femmes. Il résulte encore de cette prodigieuse multitude d’esclaves, qu’il n’y a presque point de gens qui aient une volonté propre, et qui par conséquent doivent répondre de leur conduite devant un juge. La plupart des actions morales, qui ne sont que les volontés du père, du mari, du maître, se règlent par eux, et non par les magistrats.\par
J’oubliais de dire que ce que nous appelons l’honneur, étant à peine connu dans ces États, toutes les affaires qui regardent cet honneur, qui est un si grand chapitre parmi nous, n’y ont point de lieu. Le despotisme se suffit à lui-même ; tout est vide autour de lui. Aussi, lorsque les voyageurs nous décrivent les pays où il règne, rarement nous parlent-ils de lois civiles\footnote{Au Mazulipatan, on n’a pu découvrir qu’il y eût de loi écrite. Voyez le {\itshape Recueil des voyages qui ont servi à l’établissement de la Compagnie des Indes}, t. IV, part. I, p. 391. Les Indiens ne se règlent, dans les jugements, que sur de certaines coutumes. Le {\itshape Vedam} et autres livres pareils ne contiennent point de lois civiles, mais des préceptes religieux. Voyez {\itshape Lettres édifiantes}, quatorzième recueil.}.\par
Toutes les occasions de dispute et de procès y sont donc ôtées. C’est ce qui fait en partie qu’on y maltraite si fort les plaideurs : l’injustice de leur demande paraît à découvert, n’étant pas cachée, palliée, ou protégée par une infinité de lois.
\subsubsection[{Chapitre II. De la simplicité des lois criminelles dans les divers gouvernements}]{Chapitre II. De la simplicité des lois criminelles dans les divers gouvernements}
\noindent On entend dire sans cesse qu’il faudrait que la justice fût rendue partout comme en Turquie. Il n’y aura donc que les plus ignorants de tous les peuples qui auront vu clair dans la chose du monde qu’il importe le plus aux hommes de savoir ?\par
Si vous examinez les formalités de la justice par rapport à la peine qu’a un citoyen à se faire rendre son bien, ou à obtenir satisfaction de quelque outrage, vous en trouverez sans doute trop. Si vous les regardez dans le rapport qu’elles ont avec la liberté et la sûreté des citoyens, vous en trouverez souvent trop peu ; et vous verrez que les peines, les dépenses, les longueurs, les dangers mêmes de la justice, sont le prix que chaque citoyen donne pour sa liberté.\par
En Turquie, où l’on fait très peu d’attention à la fortune, à la vie, à l’honneur des sujets, on termine promptement, d’une façon ou d’une autre, toutes les disputes. La manière de les finir est indifférente, pourvu qu’on finisse. Le bacha, d’abord éclairci, fait distribuer, à sa fantaisie, des coups de bâton sur la plante des pieds des plaideurs, et les renvoie chez eux.\par
Et il serait bien dangereux que l’on y eût les passions des plaideurs : elles supposent un désir ardent de se faire rendre justice, une haine, une action dans l’esprit, une constance à poursuivre. Tout cela doit être évité dans un gouvernement où il ne faut avoir d’autre sentiment que la crainte, et où tout mène tout à coup, et sans qu’on le puisse prévoir, à des révolutions. Chacun doit connaître qu’il ne faut point que le magistrat entende parler de lui, et qu’il ne tient sa sûreté que de son anéantissement.\par
Mais, dans les États modérés, où la tête du moindre citoyen est considérable, on ne lui ôte son honneur et ses biens qu’après un long examen : on ne le prive de la vie que lorsque la Patrie elle-même l’attaque ; et elle ne l’attaque qu’en lui laissant tous les moyens possibles de la défendre.\par
Aussi, lorsqu’un homme se rend plus absolu\footnote{César, Cromwell et tant d’autres.}, songe-t-il d’abord à simplifier les lois. On commence, dans cet État, à être plus frappé des inconvénients particuliers, que de la liberté des sujets dont on ne se soucie point du tout.\par
On voit que dans les républiques il faut pour le moins autant de formalités que dans les monarchies. Dans l’un et dans l’autre gouvernement, elles augmentent en raison du cas que l’on y fait de l’honneur, de la fortune, de la vie, de la liberté des citoyens.\par
Les hommes sont tous égaux dans le gouvernement républicain ; ils sont égaux dans le gouvernement despotique : dans le premier, c’est parce qu’ils sont tout ; dans le second, c’est parce qu’ils ne sont rien.
\subsubsection[{Chapitre III. Dans quels gouvernements et dans quels cas on doit juger selon un texte précis de la loi}]{Chapitre III. Dans quels gouvernements et dans quels cas on doit juger selon un texte précis de la loi}
\noindent Plus le gouvernement approche de la république, plus la manière de juger devient fixe ; et c’était un vice de la république de Lacédémone, que les éphores jugeassent arbitrairement, sans qu’il y eût des lois pour les diriger. À Rome, les premiers consuls jugèrent comme les éphores : on en sentit les inconvénients, et l’on fit des lois précises.\par
Dans les États despotiques, il n’y a point de loi : le juge est lui-même sa règle. Dans les États monarchiques, il y a une loi : et là où elle est précise, le juge la suit ; là où elle ne l’est pas, il en cherche l’esprit. Dans le gouvernement républicain, il est de la nature de la constitution que les juges suivent la lettre de la loi. Il n’y a point de citoyen contre qui on puisse interpréter une loi, quand il s’agit de ses biens, de son honneur, ou de sa vie.\par
À Rome, les juges prononçaient seulement que l’accusé était coupable d’un certain crime, et la peine se trouvait dans la loi, comme on le voit dans diverses lois qui furent faites. De même, en Angleterre, les jurés décident si l’accusé est coupable, ou non, du fait qui a été porté devant eux ; et, s’il est déclaré coupable, le juge prononce la peine que la loi inflige pour ce fait ; et pour cela il ne lui faut que des yeux.
\subsubsection[{Chapitre IV. De la manière de former les jugements}]{Chapitre IV. De la manière de former les jugements}
\noindent De là suivent les différentes manières de former les jugements. Dans les monarchies, les juges prennent la manière des arbitres ; ils délibèrent ensemble, ils se communiquent leurs pensées, ils se concilient ; on modifie son avis pour le rendre conforme à celui d’un autre ; les avis les moins nombreux sont rappelés aux deux plus grands. Cela n’est point de la nature de la république. À Rome et dans les villes grecques, les juges ne se communiquaient point : chacun donnait son avis d’une de ces trois manières : {\itshape J’absous, Je condamne, Il ne me paraît pas}\footnote{{\itshape Non liquet.}} : c’est que le peuple jugeait, ou était censé juger. Mais le peuple n’est pas jurisconsulte ; toutes ces modifications et tempéraments des arbitres ne sont pas pour lui ; il faut lui présenter un seul objet, un fait, et un seul fait, et qu’il n’ait qu’à voir s’il doit condamner, absoudre, ou remettre le jugement.\par
Les Romains, à l’exemple des Grecs, introduisirent des formules d’actions\footnote{{\itshape Quas actiones, ne populus, prout vellet, institueret, certas solemnesque esse voluerunt.} Leg. 2, § 6, Digest., {\itshape De orig. jur.}}, et établirent la nécessité de diriger chaque affaire par l’action qui lui était propre. Cela était nécessaire dans leur manière de juger : il fallait fixer l’état de la question, pour que le peuple l’eût toujours devant les yeux. Autrement, dans le cours d’une grande affaire, cet état de la question changerait continuellement, et on ne le reconnaîtrait plus.\par
De là il suivait que les juges, chez les Romains, n’accordaient que la demande précise, sans rien augmenter, diminuer, ni modifier. Mais les préteurs imaginèrent d’autres formules d’actions qu’on appela {\itshape de bonne} foi\footnote{Dans lesquelles on mettait ces mots : {\itshape ex bonâ fide.}}, où la manière de prononcer était plus dans la disposition du juge. Ceci était plus conforme à l’esprit de la monarchie. Aussi les jurisconsultes français disent-ils {\itshape : En France}\footnote{On y condamne aux dépens celui-là même à qui on demande plus qu’il ne doit, s’il n’a offert et consigné ce qu’il doit.}{\itshape , toutes les actions sont de bonne} foi.
\subsubsection[{Chapitre V. Dans quel gouvernement le souverain peut être juge}]{Chapitre V. Dans quel gouvernement le souverain peut être juge}
\noindent Machiavel\footnote{{\itshape Discours sur la première décade de Tite-Live}, liv. I, chap. VII.} attribue la perte de la liberté de Florence à ce que le peuple ne jugeait pas en corps, comme à Rome, des crimes de lèse-majesté commis contre lui. Il y avait pour cela huit juges établis : {\itshape Mais}, dit Machiavel, {\itshape peu sont corrompus par peu}. J’adopterais bien la maxime de ce grand homme : mais comme dans ces cas l’intérêt politique force, pour ainsi dire, l’intérêt civil (car c’est toujours un inconvénient que le peuple juge lui-même ses offenses), il faut, pour y remédier, que les lois pourvoient, autant qu’il est en elles, à la sûreté des particuliers.\par
Dans cette idée, les législateurs de Rome firent deux choses : ils permirent aux accusés de s’exiler\footnote{Cela est bien expliqué dans l’oraison de Cicéron, {\itshape Pro Caecina} à la fin.} avant le jugement\footnote{C’était une loi d’Athènes, comme il paraît par Démosthène. Socrate refusa de s’en servir.}, et ils voulurent que les biens des condamnés fussent consacrés, pour que le peuple n’en eût pas la confiscation. On verra, dans le livre XI, les autres limitations que l’on mit à la puissance que le peuple avait de juger.\par
Solon sut bien prévenir l’abus que le peuple pourrait faire de sa puissance dans le jugement des crimes : il voulut que l’Aréopage revît l’affaire ; que, s’il croyait l’accusé injustement absous\footnote{Démosthène, {\itshape Sur la Couronne}, p. 494, édit. de Francfort, de l’an 1604.}, il l’accusât de nouveau devant le peuple ; que, s’il le croyait injustement condamné\footnote{Voyez Philostrate, {\itshape Vie des sophistes}, liv. I, {\itshape Vie d’Eschine}.}, il arrêtât l’exécution, et lui fît rejuger l’affaire : loi admirable, qui soumettait le peuple à la censure de la magistrature qu’il respectait le plus, et à la sienne même !\par
Il sera bon de mettre quelque lenteur dans des affaires pareilles, surtout du moment que l’accusé sera prisonnier, afin que le peuple puisse se calmer et juger de sang-froid.\par
Dans les États despotiques, le prince peut juger lui-même. Il ne le peut dans les monarchies : la constitution serait détruite, les pouvoirs intermédiaires dépendants, anéantis : on verrait cesser toutes les formalités des jugements ; la crainte s’emparerait de tous les esprits ; on verrait la pâleur sur tous les visages ; plus de confiance, plus d’honneur, plus d’amour, plus de sûreté, plus de monarchie.\par
Voici d’autres réflexions. Dans les États monarchiques, le prince est la partie qui poursuit les accusés et les fait punir ou absoudre ; s’il jugeait lui-même, il serait le juge et la partie.\par
Dans ces mêmes États, le prince a souvent les confiscations : s’il jugeait les crimes, il serait encore le juge et la partie.\par
De plus, il perdrait le plus bel attribut de sa souveraineté, qui est celui de faire grâce\footnote{Platon ne pense pas que les rois, qui sont, dit-il, prêtres, puissent assister au jugement où l’on condamne à la mort, à l’exil, à la prison.}. Il serait insensé qu’il fit et défit ses jugements : il ne voudrait pas être en contradiction avec lui-même.\par
Outre que cela confondrait toutes les idées, on ne saurait si un homme serait absous ou s’il recevrait sa grâce.\par
Lorsque Louis XIII voulut être juge dans le procès du duc de La Valette\footnote{Voyez la relation du procès fait à M. le duc de La Valette. Elle est imprimée dans les {\itshape Mémoires} de Montrésor, t. II, p. 62.}, et qu’il appela pour cela dans son cabinet quelques officiers du parlement et quelques conseillers d’État, le roi les ayant forcés sur le décret de prise de corps, le président de Bellièvre dit : « Qu’il voyait dans cette affaire une chose étrange, un prince opiner au procès d’un de ses sujets ; que les rois ne s’étaient réservé que les grâces, et qu’ils renvoyaient les condamnations vers leurs officiers. Et Votre Majesté voudrait bien voir sur la sellette un homme devant Elle, qui, par son jugement, irait dans une heure à la mort ! Que la face du prince, qui porte les grâces, ne peut soutenir cela ; que sa vue seule levait les interdits des églises ; qu’on ne devait sortir que content de devant le prince. » Lorsqu’on jugea le fond, le même président dit dans son avis : « Cela est un jugement sans exemple, voire contre tous les exemples du passé jusqu’à aujourd’hui, qu’un roi de France ait condamné en qualité de juge, par son avis, un gentilhomme à mort\footnote{Cela fut changé dans la suite. Voyez la même relation.}. »\par
Les jugements rendus par le prince seraient une source intarissable d’injustices et d’abus ; les courtisans extorqueraient, par leur importunité, ses jugements. Quelques empereurs romains eurent la fureur de juger ; nuls règnes n’étonnèrent plus l’univers par leurs injustices.\par
« Claude, dit Tacite\footnote{{\itshape Annales}, liv. XI.}, ayant attiré à lui le jugement des affaires et les fonctions des magistrats, donna occasion à toutes sortes de rapines. » Aussi Néron, parvenant à l’empire après Claude, voulant se concilier les esprits, déclara-t-il : « Qu’il se garderait bien d’être le juge de toutes les affaires, pour que les accusateurs et les accusés, dans les murs d’un palais, ne fussent pas exposés à l’inique pouvoir de quelques affranchis\footnote{Tacite, {\itshape Annales}, liv. XIII.}. »\par
« Sous le règne d’Arcadius, dit Zozime\footnote{{\itshape Histoire}, liv. V.}, la nation des calomniateurs se répandit, entoura la cour et l’infecta. Lorsqu’un homme était mort, on supposait qu’il n’avait point laissé d’enfants\footnote{Même désordre sous Théodose le Jeune.} ; on donnait ses biens par un rescrit. Car, comme le prince était étrangement stupide, et l’impératrice entreprenante à l’excès, elle servait l’insatiable avarice de ses domestiques et de ses confidentes ; de sorte que, pour les gens modérés, il n’y avait rien de plus désirable que la mort. »\par
« Il y avait autrefois, dit Procope\footnote{{\itshape Histoire secrète}.}, fort peu de gens à la cour ; mais, sous Justinien, comme les juges n’avaient plus la liberté de rendre justice, leurs tribunaux étaient déserts, tandis que le palais du prince retentissait des clameurs des parties qui y sollicitaient leurs affaires. » Tout le monde sait comment on y vendait les jugements, et même les lois.\par
Les lois sont les yeux du prince ; il voit par elles ce qu’il ne pourrait pas voir sans elles. Veut-il faire la fonction des tribunaux ? il travaille non pas pour lui, mais pour ses séducteurs contre lui.
\subsubsection[{Chapitre VI. Que, dans la monarchie, les ministres ne doivent pas juger}]{Chapitre VI. Que, dans la monarchie, les ministres ne doivent pas juger}
\noindent C’est encore un grand inconvénient, dans la monarchie, que les ministres du prince jugent eux-mêmes les affaires contentieuses. Nous voyons encore aujourd’hui des États où il y a des juges sans nombre pour décider les affaires fiscales, et où les ministres, qui le croirait ! veulent encore les juger. Les réflexions viennent en foule ; je ne ferai que celle-ci.\par
Il y a, par la nature des choses, une espèce de contradiction entre le Conseil du monarque et ses tribunaux. Le Conseil des rois doit être composé de peu de personnes, et les tribunaux de judicature en demandent beaucoup. La raison en est que, dans le premier, on doit prendre les affaires avec une certaine passion et les suivre de même ; ce qu’on ne peut guère espérer que de quatre ou cinq hommes qui en font leur affaire. Il faut au contraire des tribunaux de judicature de sang-froid, et à qui toutes les affaires soient en quelque façon indifférentes.
\subsubsection[{Chapitre VII. Du magistrat unique}]{Chapitre VII. Du magistrat unique}
\noindent Un tel magistrat ne peut avoir lieu que dans le gouvernement despotique. On voit, dans l’histoire romaine, à quel point un juge unique peut abuser de son pouvoir. Comment Appius, sur son tribunal, n’aurait-il pas méprisé les lois, puisqu’il viola même celle qu’il avait faite\footnote{Voyez la loi 2, § 24, Dig., {\itshape de orig. jur.}} \footnote{{\itshape Quod pater puellae abesset, locum injuriae esse ratus.} Tite-Live, Décade I, liv. III.}? Tite-Live nous apprend l’inique distinction du décemvir. Il avait aposté un homme qui réclamait devant lui Virginie comme son esclave ; les parents de Virginie lui demandèrent, qu’en vertu de sa loi, on la leur remit jusqu’au jugement définitif. Il déclara que sa loi n’avait été faite qu’en faveur du père, et que, Virginius étant absent, elle ne pouvait avoir d’application.
\subsubsection[{Chapitre VIII. Des accusations dans les divers gouvernements}]{Chapitre VIII. Des accusations dans les divers gouvernements}
\noindent À Rome\footnote{Et dans bien d’autres cités.}, il était permis à un citoyen d’en accuser un autre. Cela était établi selon l’esprit de la république, où chaque citoyen doit avoir pour le bien public un zèle sans bornes, où chaque citoyen est censé tenir tous les droits de la patrie dans ses mains. On suivit, sous les empereurs, les maximes de la république ; et d’abord on vit paraître un genre d’hommes funestes, une troupe de délateurs. Quiconque avait bien des vices et bien des talents, une âme bien basse et un esprit ambitieux, cherchait un criminel dont la condamnation pût plaire au prince ; c’était la voie pour aller aux honneurs et à la fortune\footnote{Voyez dans Tacite les récompenses accordées à ces délateurs.}, chose que nous ne voyons point parmi nous.\par
Nous avons aujourd’hui une loi admirable : c’est celle qui veut que le prince, établi pour faire exécuter les lois, prépose un officier dans chaque tribunal, pour poursuivre, en son nom, tous les crimes : de sorte que la fonction des délateurs est inconnue parmi nous ; et, si ce vengeur publie était soupçonné d’abuser de son ministère, on l’obligerait de nommer son dénonciateur.\par
Dans les Lois de Platon\footnote{Liv. IX.}, ceux qui négligent d’avertir les magistrats, ou de leur donner du secours, doivent être punis. Cela ne conviendrait point aujourd’hui. La partie publique veille pour les citoyens ; elle agit, et ils sont tranquilles.
\subsubsection[{Chapitre IX. De la sévérité des peines dans les divers gouvernements}]{Chapitre IX. De la sévérité des peines dans les divers gouvernements}
\noindent La sévérité des peines convient mieux au gouvernement despotique, dont le principe est la terreur, qu’à la monarchie et à la république, qui ont pour ressort l’honneur et la vertu.\par
Dans les États modérés, l’amour de la patrie, la honte et la crainte du blâme, sont des motifs réprimants, qui peuvent arrêter bien des crimes. La plus grande peine d’une mauvaise action sera d’en être convaincu. Les lois civiles y corrigeront donc plus aisément, et n’auront pas besoin de tant de force.\par
Dans ces États, un bon législateur s’attachera moins à punir les crimes qu’à les prévenir ; il s’appliquera plus à donner des mœurs qu’à infliger des supplices.\par
C’est une remarque perpétuelle des auteurs chinois\footnote{Je ferai voir dans la suite que la Chine, à cet égard, est dans le cas d’une république ou d’une monarchie.} que plus, dans leur empire, on voyait augmenter les supplices, plus la révolution était prochaine. C’est qu’on augmentait les supplices à mesure qu’on manquait de mœurs.\par
Il serait aisé de prouver que, dans tous ou presque tous les États d’Europe, les peines ont diminué ou augmenté à mesure qu’on s’est plus approché ou plus éloigné de la liberté.\par
Dans les pays despotiques, on est si malheureux, que l’on y craint plus la mort qu’on ne regrette la vie ; les supplices y doivent donc être plus rigoureux. Dans les États modérés, on craint plus de perdre la vie qu’on ne redoute la mort en elle-même ; les supplices qui ôtent simplement la vie y sont donc suffisants.\par
Les hommes extrêmement heureux, et les hommes extrêmement malheureux, sont également portés à la dureté ; témoins les moines et les conquérants. Il n’y a que la médiocrité et le mélange de la bonne et de la mauvaise fortune, qui donnent de la douceur et de la pitié.\par
Ce que l’on voit dans les hommes en particulier se trouve dans les diverses nations. Chez les peuples sauvages, qui mènent une vie très dure, et chez les peuples des gouvernements despotiques, où il n’y a qu’un homme exorbitamment favorisé de la fortune, tandis que tout le reste en est outragé, on est également cruel. La douceur règne dans les gouvernements modérés.\par
Lorsque nous lisons, dans les histoires, les exemples de la justice atroce des sultans, nous sentons, avec une espèce de douleur, les maux de la nature humaine.\par
Dans les gouvernements modérés, tout, pour un bon législateur, peut servir à former des peines. N’est-il pas bien extraordinaire qu’à Sparte une des principales fût de ne pouvoir prêter sa femme à un autre, ni recevoir celle d’un autre, de n’être jamais dans sa maison qu’avec des vierges ? En un mot, tout ce que la loi appelle une peine est effectivement une peine.
\subsubsection[{Chapitre X. Des anciennes lois françaises}]{Chapitre X. Des anciennes lois françaises}
\noindent C’est bien dans les anciennes lois françaises que l’on trouve l’esprit de la monarchie. Dans les cas où il s’agit de peines pécuniaires, les non-nobles sont moins punis que les nobles\footnote{« Si, comme pour briser un arrêt, les non-nobles doivent une amende de quarante sous, et les nobles de soixante livres. » Somme {\itshape rurale}, liv. II, p. 198, édit. goth. de l’an 1512 ; et Beaumanoir, chap. LXI, p. 309.}. C’est tout le contraire dans les crimes\footnote{Voyez le {\itshape Conseil} de Pierre Desfontaines, chap. XIII, surtout l’article 22.} {\itshape : le} noble perd l’honneur et réponse en cour, pendant que le vilain, qui n’a point d’honneur, est puni en son corps.
\subsubsection[{Chapitre XI. Que, lorsqu’un peuple est vertueux, il faut peu de peines}]{Chapitre XI. Que, lorsqu’un peuple est vertueux, il faut peu de peines}
\noindent Le peuple romain avait de la probité. Cette probité eut tant de force, que souvent le législateur n’eut besoin que de lui montrer le bien pour le lui faire suivre. Il semblait qu’au lieu d’ordonnances, il suffisait de lui donner des conseils.\par
Les peines des lois royales et celle des lois des douze Tables furent presque toutes ôtées dans la république, soit par une suite de la loi Valérienne\footnote{Elle fut faite par Valerius Publicola, bientôt après l’expulsion des rois ; elle fut renouvelée deux fois, toujours par des magistrats de la même famille, comme le dit Tite-Live, liv. X. Il n’était pas question de lui donner plus de force, mais d’en perfectionner les dispositions. {\itshape Diligentius sanctam}, dit Tite-Live, {\itshape ibid.}}, soit par une conséquence de la loi Porcie\footnote{{\itshape Lex Porcia pro tergo civium lata.} Elle fut faite en 454 de la fondation de Rome.}. On ne remarqua pas que la république en fût plus mal réglée, et il n’en résulta aucune lésion de police.\par
Cette loi Valérienne, qui défendait aux magistrats toute voie de fait contre un citoyen qui avait appelé au peuple, n’infligeait à celui qui y contreviendrait que la peine d’être réputé méchant\footnote{{\itshape Nihil ultra quam improbe factum adjecit}. Tite-Live.}.
\subsubsection[{Chapitre XII. De la puissance des peines}]{Chapitre XII. De la puissance des peines}
\noindent L’expérience a fait remarquer que, dans les pays où les peines sont douces, l’esprit du citoyen en est frappé, comme il l’est ailleurs par les grandes.\par
Quelque inconvénient se fait-il sentir dans un État ? Un gouvernement violent veut soudain le corriger ; et, au lieu de songer à faire exécuter les anciennes lois, on établit une peine cruelle qui arrête le mal sur-le-champ. Mais on use le ressort du gouvernement : l’imagination se fait à cette grande peine, comme elle s’était faite à la moindre ; et comme on diminue la crainte pour celle-ci, l’on est bientôt forcé d’établir l’autre dans tous les cas. Les vols sur les grands chemins étaient communs dans quelques États ; on voulut les arrêter ; on inventa le supplice de la roue, qui les suspendit pendant quelques temps. Depuis ce temps, on a volé comme auparavant sur les grands chemins.\par
De nos jours, la désertion fut très fréquente ; on établit la peine de mort contre les déserteurs, et la désertion n’est pas diminuée. La raison en est bien naturelle : un soldat, accoutumé tous les jours à exposer sa vie, en méprise ou se flatte d’en mépriser le danger. Il est tous les jours accoutumé à craindre la honte : il fallait donc laisser une peine\footnote{On fendait le nez, on coupait les oreilles.} qui faisait porter une flétrissure pendant la vie. On a prétendu augmenter la peine, et on l’a réellement diminuée.\par
Il ne faut point mener les hommes par les voies extrêmes ; on doit être ménager des moyens que la nature nous donne pour les conduire. Qu’on examine la cause de tous les relâchements, on verra qu’elle vient de l’impunité des crimes, et non pas de la modération des peines.\par
Suivons la nature, qui a donné aux hommes la honte comme leur fléau ; et que la plus grande partie de la peine soit l’infamie de la souffrir.\par
Que s’il se trouve des pays où la honte ne soit pas une suite du supplice, cela vient de la tyrannie, qui a infligé les mêmes peines aux scélérats et aux gens de bien.\par
Et si vous en voyez d’autres où les hommes ne sont retenus que par des supplices cruels, comptez encore que cela vient en grande partie de la violence du gouvernement, qui a employé ces supplices pour des fautes légères.\par
Souvent un législateur qui veut corriger un mal ne songe qu’à cette correction ; ses yeux sont ouverts sur cet objet, et fermés sur les inconvénients. Lorsque le mal est une fois corrigé, on ne voit plus que la dureté du législateur ; mais il reste un vice dans l’État, que cette dureté a produit : les esprits sont corrompus, ils se sont accoutumés au despotisme.\par
Lysandre\footnote{Xénophon, {\itshape Histoire}, liv. II.} ayant remporté la victoire sur les Athéniens, on jugea les prisonniers ; on accusa les Athéniens d’avoir précipité tous les captifs de deux galères, et résolu en pleine assemblée de couper le poing aux prisonniers qu’ils feraient. Ils furent tous égorgés, excepté Adymante, qui s’était opposé à ce décret. Lysandre reprocha à Philoclès, avant de le faire mourir, qu’il avait dépravé les esprits et fait des leçons de cruauté à toute la Grèce.\par
« Les Argiens, dit Plutarque\footnote{{\itshape Œuvres morales}, De ceux qui manient les affaires d’État.}, ayant fait mourir quinze cents de leurs citoyens, les Athéniens firent apporter les sacrifices d’expiation, afin qu’il plût aux dieux de détourner du cœur des Athéniens une si cruelle pensée. »\par
Il y a deux genres de corruption : l’un, lorsque le peuple n’observe point les lois ; l’autre, lorsqu’il est corrompu par les lois ; mal incurable, parce qu’il est dans le remède même.
\subsubsection[{Chapitre XIII. Impuissance des lois japonaises}]{Chapitre XIII. Impuissance des lois japonaises}
\noindent Les peines outrées peuvent corrompre le despotisme même. Jetons les yeux sur le Japon.\par
On y punit de mort presque tous les crimes\footnote{Voyez Kempfer.}, parce que la désobéissance à un si grand empereur que celui du Japon, est un crime énorme. Il n’est pas question de corriger le coupable, mais de venger le prince. Ces idées sont tirées de la servitude, et viennent surtout de ce que l’empereur étant propriétaire de tous les biens, presque tous les crimes se font directement contre ses intérêts.\par
On punit de mort les mensonges qui se font devant les magistrats\footnote{{\itshape Recueil des voyages qui ont servi à l’établissement de la Compagnie des Indes}, t. III, part. II, p. 428.}, chose contraire à la défense naturelle.\par
Ce qui n’a point l’apparence d’un crime est là sévèrement puni ; par exemple, un homme qui hasarde de l’argent au jeu est puni de mort.\par
Il est vrai que le caractère étonnant de ce peuple opiniâtre, capricieux, déterminé, bizarre, et qui brave tous les périls et tous les malheurs, semble, à la première vue, absoudre ses législateurs de l’atrocité de leurs lois. Mais, des gens qui naturellement méprisent la mort, et qui s’ouvrent le ventre pour la moindre fantaisie, sont-ils corrigés ou arrêtés par la vue continuelle des supplices ? Et ne s’y familiarisent-ils pas ?\par
Les relations nous disent, au sujet de l’éducation des Japonais, qu’il faut traiter les enfants avec douceur, parce qu’ils s’obstinent contre les peines ; que les esclaves ne doivent point être trop rudement traités, parce qu’ils se mettent d’abord en défense. Par l’esprit qui doit régner dans le gouvernement domestique, n’aurait-on pas pu juger de celui qu’on devait porter dans le gouvernement politique et civil ?\par
Un législateur sage aurait cherché à ramener les esprits par un juste tempérament des peines et des récompenses ; par des maximes de philosophie, de morale et de religion, assorties à ces caractères ; par la juste application des règles de l’honneur ; par le supplice de la honte ; par la jouissance d’un bonheur constant et d’une douce tranquillité ; et, s’il avait craint que les esprits, accoutumés à n’être arrêtés que par une peine cruelle, ne pussent plus l’être par une plus douce, il aurait agi\footnote{Remarquez bien ceci, comme une maxime de pratique, dans les cas où les esprits ont été gâtés par des peines trop rigoureuses.} d’une manière sourde et insensible ; il aurait, dans les cas particuliers les plus graciables, modéré la peine du crime, jusqu’à ce qu’il eût pu parvenir à la modifier dans tous les cas.\par
Mais le despotisme ne connaît point ces ressorts ; il ne mène pas par ces voies. Il peut abuser de lui, mais c’est tout ce qu’il peut faire. Au Japon, il a fait un effort, il est devenu plus cruel que lui-même.\par
Des âmes partout effarouchées et rendues plus atroces, n’ont pu être conduites que par une atrocité plus grande.\par
Voilà l’origine, voilà l’esprit des lois du Japon. Mais elles ont eu plus de fureur que de force. Elles ont réussi à détruire le christianisme : mais des efforts si inouïs sont une preuve de leur impuissance. Elles ont voulu établir une bonne police, et leur faiblesse a paru encore mieux.\par
Il faut lire la relation de l’entrevue de l’empereur et du deyro à Méaco\footnote{{\itshape Recueil des voyages qui ont servi à l’établissement de la Compagnie des Indes}, t. V, part. II.}. Le nombre de ceux qui y furent étouffés, ou tués par des garnements, fut incroyable ; on enleva les jeunes filles et les garçons ; on les retrouvait tous les jours exposés dans des lieux publics, à des heures indues, tout nus, cousus dans des sacs de toile, afin qu’ils ne connussent pas les lieux par où ils avaient passé ; on vola tout ce qu’on voulut ; on fendit le ventre à des chevaux pour faire tomber ceux qui les montaient ; on renversa des voitures pour dépouiller les dames. Les Hollandais, à qui l’on dit qu’ils ne pouvaient passer la nuit sur des échafauds sans être assassinés, en descendirent, etc.\par
Je passerai vite sur un autre trait. L’empereur, adonné à des plaisirs infâmes, ne se mariait point : il courait risque de mourir sans successeur. Le deyro lui envoya deux filles très belles : il en épousa une par respect, mais il n’eut aucun commerce avec elle. Sa nourrice fit chercher les plus belles femmes de l’empire, tout était inutile. La fille d’un armurier étonna son goût\footnote{{\itshape Ibid.}} ; il se détermina, il en eut un fils. Les dames de la cour, indignées de ce qu’il leur avait préféré une personne d’une si basse naissance, étouffèrent l’enfant. Ce crime fut caché à l’empereur ; il aurait versé un torrent de sang. L’atrocité des lois en empêche donc l’exécution. Lorsque la peine est sans mesure, on est souvent obligé de préférer l’impunité.\par
Livre VI : conséquences des principes des divers gouvernements par rapport à la simplicité des lois civiles et criminelles, la forme des jugements et l’établissement des peines
\subsubsection[{Chapitre XIV. De l’esprit du sénat de Rome}]{Chapitre XIV. De l’esprit du sénat de Rome}
\noindent Sous le consulat d’Acilius Glabrio et de Pison, on fit la loi Acilia\footnote{Les coupables étaient condamnés à une amende ; ils ne pouvaient plus être admis dans l’ordre des sénateurs, et nommés à aucune magistrature. Dion, liv. XXXVI.} pour arrêter les brigues. Dion dit\footnote{{\itshape Ibid.}} que le sénat engagea les consuls à la proposer, parce que le tribun C. Cornelius avait résolu de faire établir des peines terribles contre ce crime, à quoi le peuple était fort porté. Le sénat pensait que des peines immodérées jetteraient bien la terreur dans les esprits ; mais qu’elles auraient cet effet qu’on ne trouverait plus personne pour accuser ni pour condamner ; au lieu qu’en proposant des peines modiques, on aurait des juges et des accusateurs.
\subsubsection[{Chapitre XV. Des lois des Romains à l’égard des peines}]{Chapitre XV. Des lois des Romains à l’égard des peines}
\noindent Je me trouve fort dans mes maximes, lorsque j’ai pour moi les Romains ; et je crois que les peines tiennent à la nature du gouvernement, lorsque je vois ce grand peuple changer à cet égard de lois civiles, à mesure qu’il changeait de lois politiques.\par
Les lois royales, faites pour un peuple composé de fugitifs, d’esclaves et de brigands, furent très sévères. L’esprit de la république aurait demandé que les décemvirs n’eussent pas mis ces lois dans leurs douze Tables ; mais des gens qui aspiraient à la tyrannie n’avaient garde de suivre l’esprit de la république.\par
Tite-Live\footnote{Liv. I.} dit, sur le supplice de Métius Suffétius, dictateur d’Albe, qui fut condamné par Tullus Hostilius à être tiré par deux chariots, que ce fut le premier et le dernier supplice où l’on témoigna avoir perdu la mémoire de l’humanité. Il se trompe ; la loi des Douze Tables est pleine de dispositions très cruelles\footnote{On y trouve le supplice du feu, des peines presque toujours capitales, le vol puni de mort, etc.}.\par
Celle qui découvre le mieux le dessein des décemvirs est la peine capitale, prononcée contre les auteurs des libelles et les poètes. Cela West guère du génie de la république, où le peuple aime à voir les grands humiliés. Mais des gens qui voulaient renverser la liberté craignaient des écrits qui pouvaient rappeler l’esprit de la liberté\footnote{Sylla, animé du même esprit que les décemvirs, augmenta, comme eux, les peines contre les écrivains satiriques.}.\par
Après l’expulsion des décemvirs, presque toutes les lois qui avaient fixé les peines furent ôtées. On ne les abrogea pas expressément. mais la loi Porcia ayant défendu de mettre à mort un citoyen romain, elles n’eurent plus d’application.\par
Voilà le temps auquel on peut rappeler ce que Tite-Live\footnote{Liv. I.} dit des Romains, que jamais peuple n’a plus aimé la modération des peines.\par
Que si l’on ajoute à la douceur des peines le droit qu’avait un accusé de se retirer avant le jugement, on verra bien que les Romains avaient suivi cet esprit que j’ai dit être naturel à la république.\par
Sylla, qui confondit la tyrannie, l’anarchie et la liberté, fit les lois Cornéliennes. Il sembla ne faire des règlements que pour établir des crimes. Ainsi, qualifiant une infinité d’actions du nom de meurtre, il trouva partout des meurtriers ; et, par une pratique qui ne fut que trop suivie, il tendit des pièges, sema des épines, ouvrit des abîmes sur le chemin de tous les citoyens.\par
Presque toutes les lois de Sylla ne portaient que l’interdiction de l’eau et du feu. César y ajouta la confiscation des biens\footnote{{\itshape Poenas facinorum auxit, cùm locupletes eo facilius scelere se obligarent, quod integris patrimoniis exularent.} Suétone, {\itshape in Julio Caesare}.}, parce que les riches gardant, dans l’exil, leur patrimoine, ils étaient plus hardis à commettre des crimes.\par
Les empereurs ayant établi un gouvernement militaire, ils sentirent bientôt qu’il n’était pas moins terrible contre eux que contre les sujets ; ils cherchèrent à le tempérer : ils crurent avoir besoin des dignités et du respect qu’on avait pour elles.\par
On s’approcha un peu de la monarchie, et l’on divisa les peines en trois classes\footnote{Voyez la loi 3, § {\itshape legis ad legem Cornel de sicariis}, et un très grand nombre d’autres, au Digeste et au Code.} \footnote{{\itshape Sublimiores}.}: celles qui regardaient les premières personnes de l’État, et qui étaient assez douces ; celles qu’on infligeait aux personnes d’un rang inférieur\footnote{{\itshape Medios}.}, et qui étaient plus sévères ; enfin, celles qui ne concernaient que les conditions basses\footnote{{\itshape Infimos}. Leg. 3, § {\itshape legis ad leg. Cornel. de sicariis}.}, et qui furent les plus rigoureuses.\par
Le féroce et insensé Maximin irrita, pour ainsi dire, le gouvernement militaire qu’il aurait fallu adoucir. Le sénat apprenait, dit Capitolin\footnote{Jul. Cap., {\itshape Maximini duo.}}, que les uns avaient été mis en croix, les autres exposés aux bêtes, ou enfermés dans des peaux de bêtes récemment tuées, sans aucun égard pour les dignités. Il semblait vouloir exercer la discipline militaire, sur le modèle de laquelle il prétendait régler les affaires civiles.\par
On trouvera dans les {\itshape Considérations sur la grandeur des Romains, et leur décadence}, comment Constantin changea le despotisme militaire en un despotisme militaire et civil, et en un despotisme militaire et civil, et s’approcha de la monarchie. On y peut suivre les diverses révolutions de cet État, et voir comment on y passa de la rigueur à l’indolence, et de l’indolence à l’impunité.
\subsubsection[{Chapitre XVI. De la juste proportion des peines avec le crime}]{Chapitre XVI. De la juste proportion des peines avec le crime}
\noindent Il est essentiel que les peines aient de l’harmonie entre elles, parce qu’il est essentiel que l’on évite plutôt un grand crime qu’un moindre, ce qui attaque plus la société, que ce qui la choque moins.\par
« Un imposteur\footnote{{\itshape Histoire} de Nicéphore, patriarche de Constantinople.}, qui se disait Constantin Ducas, suscita un grand soulèvement à Constantinople. Il fut pris et condamné au fouet ; mais, ayant accusé des personnes considérables, il fut condamné, comme calomniateur, à être brûlé. » Il est singulier qu’on eût ainsi proportionné les peines entre le crime de lèse-majesté et celui de calomnie.\par
Cela fait souvenir d’un mot de Charles II, roi d’Angleterre. Il vit, en passant, un homme au pilori ; il demanda pourquoi il était là. « Sire, {\itshape lui} dit-on, {\itshape c’est parce qu’il a fait des libelles contre vos ministres. — Le grand sot} ! dit le roi : {\itshape que ne les écrivait-il contre moi ? on ne lui aurait rien fait. »}\par
« Soixante-dix personnes conspirèrent contre l’empereur Basile\footnote{{\itshape Histoire} de Nicéphore.} {\itshape ;} il les fit fustiger ; on leur brûla les cheveux et le poil. Un cerf l’ayant pris avec son bois par la ceinture, quelqu’un de sa suite tira son épée, coupa sa ceinture et le délivra : il lui fit trancher la tête, parce qu’il avait, disait-il, tiré l’épée contre lui. » Qui pourrait penser que, sous le même prince, on eût rendu ces deux jugements ?\par
C’est un grand mal, parmi nous, de faire subir la même peine à celui qui vole sur un grand chemin, et à celui qui vole et assassine. Il est visible que, pour la sûreté publique, il faudrait mettre quelque différence dans la peine.\par
À la Chine, les voleurs cruels sont coupés en morceaux\footnote{Le P. Du Halde, t. I, p. 6.}, les autres non : cette différence fait que l’on y vole, mais qu’on n’y assassine pas.\par
En Moscovie, où la peine des voleurs et celle des assassins sont les mêmes, on assassine\footnote{{\itshape État présent de la grande Russie}, par Perry.} toujours. Les morts, y dit-on, ne racontent rien.\par
Quand il n’y a point de différence dans la peine, il faut en mettre dans l’espérance de la grâce. En Angleterre, on n’assassine point, parce que les voleurs peuvent espérer d’être transportés dans les colonies, non pas les assassins.\par
C’est un grand ressort des gouvernements modérés que les lettres de grâce. Ce pouvoir que le prince a de pardonner, exécuté avec sagesse, peut avoir d’admirables effets. Le principe du gouvernement despotique, qui ne pardonne pas, et à qui on ne pardonne jamais, le prive de ces avantages.
\subsubsection[{Chapitre XVII. De la torture ou question contre les criminels}]{Chapitre XVII. De la torture ou question contre les criminels}
\noindent Parce que les hommes sont méchants, la loi est obligée de les supposer meilleurs qu’ils ne sont. Ainsi la déposition de deux témoins suffit dans la punition de tous les crimes. La loi les croit, comme s’ils parlaient par la bouche de la vérité. L’on juge aussi que tout enfant conçu pendant le mariage est légitime ; la loi a confiance en la mère comme si elle était la pudicité même. Mais la {\itshape question} contre les criminels n’est pas dans un cas forcé comme ceux-ci. Nous voyons aujourd’hui une nation\footnote{La nation anglaise.} très bien policée la rejeter sans inconvénient. Elle n’est donc pas nécessaire par sa nature\footnote{Les citoyens d’Athènes ne pouvaient être mis à la question (Lysias, {\itshape Orat. in Agorat)}, excepté dans le crime de lèse-majesté. On donnait la question trente jours après la condamnation (Curius Fortunatus, {\itshape Rhetor. scol.}, liv. II). Il n’y avait pas de question préparatoire. Quant aux Romains, la loi 3 et 4{\itshape  ad leg. Juliam majest.} fait voir que la naissance, la dignité, la profession de la milice garantissaient de la question, si ce n’est dans le cas de crime de lèse-majesté. Voyez les sages restrictions que les lois des Wisigoths mettaient à cette pratique.}.\par
Tant d’habiles gens et tant de beaux génies ont écrit contre cette pratique, que je n’ose parler après eux. J’allais dire qu’elle pourrait convenir dans les gouvernements despotiques, où tout ce qui inspire la crainte entre plus dans les ressorts du gouvernement ; j’allais dire que les esclaves chez les Grecs et chez les Romains… Mais j’entends la voix de la nature qui crie contre moi.
\subsubsection[{Chapitre XVIII. Des peines pécuniaires et des peines corporelles}]{Chapitre XVIII. Des peines pécuniaires et des peines corporelles}
\noindent Nos pères les Germains n’admettaient guère que des peines pécuniaires. Ces hommes guerriers et libres estimaient que leur sang ne devait être versé que les armes à la main. Les Japonais\footnote{Voyez Kempfer.}, au contraire, rejettent ces sortes de peines, sous prétexte que les gens riches éluderaient la punition. Mais les gens riches ne craignent-ils pas de perdre leurs biens ? Les peines pécuniaires ne peuvent-elles pas se proportionner aux fortunes ? Et, enfin, ne peut-on pas joindre l’infamie à ces peines ?\par
Un bon législateur prend un juste milieu : il n’ordonne pas toujours des peines pécuniaires ; il n’inflige pas toujours des peines corporelles.
\subsubsection[{Chapitre XIX. De la loi du Talion}]{Chapitre XIX. De la loi du Talion}
\noindent Les États despotiques, qui aiment les lois simples, usent beaucoup de la loi du talion\footnote{Elle est établie dans l’Alcoran. Voyez le chapitre De la {\itshape vache} II, 78£.}. Les États modérés la reçoivent quelquefois : mais il y a cette différence, que les premiers la font exercer rigoureusement, et que les autres lui donnent presque toujours des tempéraments.\par
La loi des Douze Tables en admettait deux ; elle ne condamnait au talion que lorsqu’on n’avait pu apaiser celui qui se plaignait\footnote{{\itshape Si membrum rupit, ni cum eo pacit, talio esteo}. Aulu-Gelle, liv. XX, chap. I.}. On pouvait, après la condamnation, payer les dommages et intérêts\footnote{{\itshape Ibid.}}, et la peine corporelle se convertissait en peine pécuniaire\footnote{Voyez aussi la loi des Wisigoths, liv. VI, tit. IV, § 3 et 5.}.
\subsubsection[{Chapitre XX. De la punition des pères pour leurs enfants}]{Chapitre XX. De la punition des pères pour leurs enfants}
\noindent On punit à la Chine les pères pour les fautes de leurs enfants. C’était l’usage du Pérou\footnote{Voyez Garcilasso, {\itshape Histoire des guerres civiles des Espagnols}.}. Ceci est encore tiré des idées despotiques.\par
On a beau dire qu’on punit à la Chine le père pour n’avoir fait usage de ce pouvoir paternel que la nature a établi, et que les lois même y ont augmenté ; cela suppose toujours qu’il n’y a point d’honneur chez les Chinois. Parmi nous, les pères dont les enfants sont condamnés au supplice, et les enfants\footnote{Au lieu de les punir, disait Platon, il faut les louer de ne pas ressembler à leur père. Liv. IX des {\itshape Lois}.} dont les pères ont subi le même sort, sont aussi punis par la honte, qu’ils le seraient à la Chine par la perte de la vie.
\subsubsection[{Chapitre XXI. De la clémence du prince}]{Chapitre XXI. De la clémence du prince}
\noindent La clémence est la qualité distinctive des monarques. Dans la république, où l’on a pour principe la vertu, elle est moins nécessaire. Dans l’État despotique, où règne la crainte, elle est moins en usage, parce qu’il faut contenir les grands de l’État par des exemples de sévérité. Dans les monarchies, où l’on est gouverné par l’honneur, qui souvent exige ce que la loi défend, elle est plus nécessaire. Le disgrâce y est un équivalent à la peine ; les formalités même des jugements y sont des punitions. C’est là que la honte vient de tous côtés pour former des genres particuliers de peines.\par
Les grands y sont si fort punis par la disgrâce, par la perte souvent imaginaire de leur fortune, de leur crédit, de leurs habitudes, de leurs plaisirs, que la rigueur à leur égard est inutile ; elle ne peut servir qu’à ôter aux sujets l’amour qu’ils ont pour la personne du prince, et le respect qu’ils doivent avoir pour les places.\par
Comme l’instabilité des grands est de la nature du gouvernement despotique, leur sûreté entre dans la nature de la monarchie.\par
Les monarques ont tant à gagner par la clémence, elle est suivie de tant d’amour, ils en tirent tant de gloire, que c’est presque toujours un bonheur pour eux d’avoir l’occasion de l’exercer ; et on le peut presque toujours dans nos contrées.\par
On leur disputera peut-être quelque branche de l’autorité, presque jamais l’autorité entière ; et si quelquefois ils combattent pour la couronne, ils ne combattent point pour la vie.\par
Mais, dira-t-on, quand faut-il punir ? quand faut-il pardonner ? C’est une chose qui se fait mieux sentir qu’elle ne peut se prescrire. Quand la clémence a des dangers, ces dangers sont très visibles. On la distingue aisément de cette faiblesse qui mène le prince au mépris et à l’impuissance même de punir.\par
L’empereur Maurice\footnote{Évagre, {\itshape Histoire}.} prit la résolution de ne verser jamais le sang de ses sujets. Anastase\footnote{Fragment de Suidas dans Constantin Porphyrogénète.} ne punissait point les crimes. Isaac l’Ange jura que, de son règne, il ne ferait mourir personne. Les empereurs grecs avaient oublié que ce n’était pas en vain qu’ils portaient l’épée.
\subsection[{Livre septième. Conséquences des différents principes des trois gouvernements, par rapport aux lois somptuaires, au luxe et à la condition des femmes}]{Livre septième. Conséquences des différents principes des trois gouvernements, par rapport aux lois somptuaires, au luxe et à la condition des femmes}
\subsubsection[{Chapitre I. Du luxe}]{Chapitre I. {\itshape Du luxe}}
\noindent Le luxe est toujours en proportion avec l’inégalité des fortunes. Si, dans un État, les richesses sont également partagées, il n’y aura point de luxe ; car il n’est fondé que sur les commodités qu’on se donne par le travail des autres.\par
Pour que les richesses restent également partagées, il faut que la loi ne donne à chacun que le nécessaire physique. Si l’on a au-delà, les uns dépenseront, les autres acquerront, et l’inégalité s’établira.\par
Supposant le nécessaire physique égal à une somme donnée, le luxe de ceux qui n’auront que le nécessaire sera égal à zéro ; celui qui aura le double aura un luxe égal à un ; celui qui aura le double du bien de ce dernier aura un luxe égal à trois ; quand on aura encore le double, on aura un luxe égal à sept ; de sorte que le bien du particulier qui suit, étant toujours supposé double de celui du précédent, le luxe croîtra du double plus une unité, dans cette progression 0, 1, 3, 7, 15, 31, 63, 127.\par
Dans la république de Platon\footnote{Le premier cens était le sort héréditaire en terres, et Platon ne voulait pas qu’on pût avoir, en autres effets, plus du triple du sort héréditaire. Voyez ses {\itshape Lois}, liv. V.}, le luxe aurait pu se calculer au juste. Il y avait quatre sortes de cens établis. Le premier était précisément le terme où finissait la pauvreté ; le second était double, le troisième triple, le quatrième quadruple du premier. Dans le premier cens, le luxe était égal à zéro ; il était égal à un dans le second, à deux dans le troisième, à trois dans le quatrième ; et il suivait ainsi la proportion arithmétique.\par
En considérant le luxe des divers peuples les uns à l’égard des autres, il est dans chaque État en raison composée de l’inégalité des fortunes qui est entre les citoyens, et de l’inégalité des richesses des divers États. En Pologne, par exemple, les fortunes sont d’une inégalité extrême ; mais la pauvreté du total empêche qu’il y ait autant de luxe que dans un État plus riche.\par
Le luxe est encore en proportion avec la grandeur des villes, et surtout de la capitale ; en sorte qu’il est en raison composée des richesses de l’État, de l’inégalité des fortunes des particuliers et du nombre d’hommes qu’on assemble dans de certains lieux.\par
Plus il y a d’hommes ensemble, plus ils sont vains et sentent naître en eux l’envie de se signaler par de petites choses\footnote{Dans une grande ville, dit l’auteur de {\itshape La Fable des abeilles}, t. I, p. 133, on s’habille au-dessus de sa qualité, pour être estimé plus qu’on n’est par la multitude. C’est un plaisir pour un esprit faible, presque aussi grand que celui de l’accomplissement de ses désirs.}. S’ils sont en si grand nombre que la plupart soient inconnus les uns aux autres, l’envie de se distinguer redouble, parce qu’il y a plus d’espérance de réussir. Le luxe donne cette espérance ; chacun prend les marques de la condition qui précède la sienne. Mais à force de vouloir se distinguer, tout devient égal, et on ne se distingue plus : comme tout le monde veut se faire regarder, on ne remarque personne.\par
Il résulte de tout cela une incommodité générale. Ceux qui excellent dans une profession mettent à leur art le prix qu’ils veulent ; les plus petits talents suivent cet exemple ; il n’y a plus d’harmonie entre les besoins et les moyens. Lorsque je suis forcé de plaider, il est nécessaire que je puisse payer un avocat ; lorsque je suis malade, il faut que je puisse avoir un médecin.\par
Quelques gens ont pensé qu’en assemblant tant de peuple dans une capitale, on diminuait le commerce, parce que les hommes ne sont Plus à une certaine distance les uns des autres. Je ne le crois pas ; on a plus de désirs, plus de besoins, plus de fantaisies quand on est ensemble.
\subsubsection[{Chapitre II. Des lois somptuaires dans la démocratie}]{Chapitre II. Des lois somptuaires dans la démocratie}
\noindent Je viens de dire que, dans les républiques où les richesses sont également partagées, il ne peut point y avoir de luxe ; et comme on a vu au livre cinquième\footnote{Chap. III et IV.} que cette égalité de distribution faisait l’excellence d’une république, il suit que moins il y a de luxe dans une république, plus elle est parfaite. Il n’y en avait point chez les premiers Romains ; il n’y en avait point chez les Lacédémoniens ; et dans les républiques où l’égalité n’est pas tout à fait perdue, l’esprit de commerce, de travail et de vertu fait que chacun y peut et que chacun y veut vivre de son propre bien, et que par conséquent il y a peu de luxe.\par
Les lois du nouveau partage des champs, demandées avec tant d’instance dans quelques républiques, étaient salutaires par leur nature. Elles ne sont dangereuses que comme action subite. En ôtant tout à coup les richesses aux uns, et augmentant de même celles des autres, elles font dans chaque famille une révolution, et en doivent produire une générale dans l’État.\par
À mesure que le luxe s’établit dans une république, l’esprit se tourne vers l’intérêt particulier.\par
À des gens à qui il ne faut rien que le nécessaire, il ne reste à désirer que la gloire de la patrie et la sienne propre. Mais une âme corrompue par le luxe a bien d’autres désirs : bientôt elle devient ennemie des lois qui la gênent. Le luxe que la garnison de Rhège commença à connaître, fit qu’elle en égorgea les habitants.\par
Sitôt que les Romains furent corrompus, leurs désirs devinrent immenses. On en peut juger par le prix qu’ils mirent aux choses. Une cruche de vin de Falerne\footnote{Fragment du livre XXXVI de Diodore, rapporté par Constantin Porphyrogénète, {\itshape Extrait des vertus et des vices.}} se vendait cent deniers romains ; un baril de chair salée du Pont en coûtait quatre cents ; un bon cuisinier, quatre talents ; les jeunes garçons n’avaient point de prix. Quand, par une impétuosité\footnote{{\itshape Cum maximus omnium impetus ad luxuriam esset. Ibid.}} générale, tout le monde se portait à la volupté, que devenait la vertu ?
\subsubsection[{Chapitre III. Des lois somptuaires dans l’aristocratie}]{Chapitre III. Des lois somptuaires dans l’aristocratie}
\noindent L’aristocratie mal constituée a ce malheur, que les nobles y ont les richesses, et que cependant ils ne doivent pas dépenser ; le luxe contraire à l’esprit de modération en doit être banni. Il n’y a donc que des gens très pauvres qui ne peuvent pas recevoir, et des gens très riches qui ne peuvent pas dépenser.\par
À Venise, les lois forcent les nobles à la modestie. Ils se sont tellement accoutumés à l’épargne, qu’il n’y a que les courtisanes qui puissent leur faire donner de l’argent. On se sert de cette voie pour entretenir l’industrie ; les femmes les plus méprisables y dépensent sans danger, pendant que leurs tributaires y mènent la vie du monde la plus obscure.\par
Les bonnes républiques grecques avaient, à cet égard, des institutions admirables. Les riches employaient leur argent en fêtes, en chœurs de musique, en chariots, en chevaux pour la course, en magistratures onéreuses. Les richesses y étaient aussi à charge que la pauvreté.
\subsubsection[{Chapitre IV. Des lois somptuaires dans les monarchies}]{Chapitre IV. Des lois somptuaires dans les monarchies}
\noindent « Les Suions, nation germanique, rendent honneur aux richesses, dit Tacite\footnote{{\itshape De moribus Germanorum}.} ; ce qui fait qu’ils vivent sous le gouvernement d’un seul. » Cela signifie bien que le luxe est singulièrement propre aux monarchies, et qu’il n’y faut point de lois somptuaires.\par
Comme, par la constitution des monarchies, les richesses y sont inégalement partagées, il faut bien qu’il y ait du luxe. Si les riches n’y dépensent pas beaucoup, les pauvres mourront de faim. il faut même que les riches y dépensent à proportion de l’inégalité des fortunes, et que, comme nous avons dit, le luxe y augmente dans cette proportion. Les richesses particulières n’ont augmenté que parce qu’elles ont ôté à une partie des citoyens le nécessaire physique ; il faut donc qu’il leur soit rendu.\par
Ainsi, pour que l’État monarchique se soutienne, le luxe doit aller en croissant, du laboureur à l’artisan, au négociant, aux nobles, aux magistrats, aux grands seigneurs, aux traitants principaux, aux princes ; sans quoi tout serait perdu.\par
Dans le sénat de Rome, composé de graves magistrats, de jurisconsultes et d’hommes pleins de l’idée des premiers temps, on proposa, sous Auguste, la correction des mœurs et du luxe des femmes. Il est curieux de voir dans Dion\footnote{Dion Cassius, liv. LIV.} avec quel ail il éluda les demandes importunes de ces sénateurs. C’est qu’il fondait une monarchie, et dissolvait une république.\par
Sous Tibère, les édiles proposèrent dans le sénat le rétablissement des anciennes lois somptuaires\footnote{Tacite, {\itshape Annales}, liv. III.}. Ce prince, qui avait des lumières, s’y opposa : « L’État ne pourrait subsister, disait-il, dans la situation où sont les choses. Comment Rome pourrait-elle vivre ? comment pourraient vivre les provinces ? Nous avions de la frugalité lorsque nous étions citoyens d’une seule ville ; aujourd’hui nous consommons les richesses de tout l’univers ; on fait travailler pour nous les maîtres et les esclaves. » Il voyait bien qu’il ne fallait plus de lois somptuaires.\par
Lorsque, sous le même empereur, on proposa au sénat de défendre aux gouverneurs de mener leurs femmes dans les provinces, à cause des dérèglements qu’elles y apportaient, cela fut rejeté. On dit « que les exemples de la dureté des anciens avaient été changés en une façon de vivre plus agréable\footnote{{\itshape Multa duritiae veterum melius et laetius mutata}. Tacite, {\itshape Annales}, liv. III.} ». On sentit qu’il fallait d’autres mœurs.\par
Le luxe est donc nécessaire dans les États monarchiques ; il l’est encore dans les États despotiques. Dans les premiers, c’est un usage que l’on fait de ce qu’on possède de liberté. Dans les autres, c’est un abus qu’on fait des avantages de sa servitude, lorsqu’un esclave, choisi par son maître pour tyranniser ses autres esclaves, incertain pour le lendemain de la fortune de chaque jour, n’a d’autre félicité que celle d’assouvir l’orgueil, les désirs et les voluptés de chaque jour.\par
Tout ceci mène à une réflexion : les républiques finissent par le luxe ; les monarchies, par la pauvreté\footnote{{\itshape Opulentia paritura inox egestatem}. Florus, liv. III.}.
\subsubsection[{Chapitre V. Dans quels cas les lois somptuaires sont utiles dans une monarchie}]{Chapitre V. Dans quels cas les lois somptuaires sont utiles dans une monarchie}
\noindent Ce fut dans l’esprit de la république, ou dans quelques cas particuliers, qu’au milieu du \textsc{xiii}\textsuperscript{e} siècle on fit en Aragon des lois somptuaires. Jacques Ier ordonna que le roi, ni aucun de ses sujets, ne pourraient manger plus de deux sortes de viandes à chaque repas, et que chacune ne serait préparée que d’une seule manière, à moins que ce ne fût du gibier qu’on eût tué soi-même\footnote{Constitution de Jacques I\textsuperscript{er}, de l’an 1234, art. 6, dans {\itshape Marca Hispanica}, p. 1429.}.\par
On a fait aussi de nos jours, en Suède, des lois somptuaires ; mais elles ont un objet différent de celles d’Aragon.\par
Un État peut faire des lois somptuaires dans l’objet d’une frugalité absolue ; c’est l’esprit des lois somptuaires des républiques ; et la nature de la chose fait voir que ce fut l’objet de celles d’Aragon.\par
Les lois somptuaires peuvent avoir aussi pour objet une frugalité relative, lorsqu’un État, sentant que des marchandises étrangères d’un trop haut prix demanderaient une telle exportation des siennes, qu’il se priverait plus de ses besoins par celles-ci, qu’il n’en satisferait par celles-là, en défend absolument l’entrée ; et c’est l’esprit des lois que l’on a faites de nos jours en Suède\footnote{On y a défendu les vins exquis et autres marchandises précieuses.}. Ce sont les seules lois somptuaires qui conviennent {\itshape aux} monarchies.\par
En général, plus un État est pauvre, plus il est ruiné par son luxe relatif ; et plus, par conséquent, il lui faut de lois somptuaires relatives. Plus un État est riche, plus son luxe relatif l’enrichit ; et il faut bien se garder d’y faire des lois somptuaires relatives. Nous expliquerons mieux ceci dans le livre sur le commerce\footnote{Voyez t. II, liv. XX, chap. XX.}. Il n’est ici question que du luxe absolu.
\subsubsection[{Chapitre VI. Du luxe à la Chine}]{Chapitre VI. Du luxe à la Chine}
\noindent Des raisons particulières demandent des lois somptuaires dans quelques États. Le peuple, par la force du climat, peut devenir si nombreux, et d’un autre côté les moyens de le faire subsister peuvent être si incertains, qu’il est bon de l’appliquer tout entier à la culture des terres. Dans ces États, le luxe est dangereux, et les lois somptuaires y doivent être rigoureuses. Ainsi, pour savoir s’il faut encourager le luxe ou le proscrire, on doit d’abord jeter les yeux sur le rapport qu’il y a entre le nombre du peuple et la facilité de le faire vivre. En Angleterre, le sol produit beaucoup plus de grain qu’il ne faut pour nourrir ceux qui cultivent les terres, et ceux qui procurent les vêtements ; il peut donc y avoir des arts frivoles, et par conséquent du luxe. En France, il croît assez de blé pour la nourriture des laboureurs et de ceux qui sont employés aux manufactures. De plus, le commerce avec les étrangers peut rendre pour des choses frivoles tant de choses nécessaires, qu’on n’y doit guère craindre le luxe.\par
À la Chine, au contraire, les femmes sont si fécondes, et l’espèce humaine s’y multiplie à un tel point, que les terres, quelque cultivées qu’elles soient, suffisent à peine pour la nourriture des habitants. Le luxe y est donc pernicieux, et l’esprit de travail et d’économie y est aussi requis que dans quelque république que ce soit\footnote{Le luxe y a toujours été arrêté.}. Il faut qu’on s’attache aux arts nécessaires, et qu’on fuie ceux de la volupté.\par
Voilà l’esprit des belles ordonnances des empereurs chinois. « Nos anciens, dit un empereur de la famille des Tang\footnote{Dans une ordonnance rapportée par le P. Du Halde, t. II, p. 497.}, tenaient pour maxime que, s’il y avait un homme qui ne labourât point, une femme qui ne s’occupât point à filer, quelqu’un souffrait le froid ou la faim dans l’empire… » Et sur ce principe, il fit détruire une infinité de monastères de bonzes.\par
Le troisième empereur de la vingt-unième dynastie\footnote{{\itshape Histoire de la Chine}, vingt-unième dynastie, dans l’ouvrage du P. Du Halde, t. I.}, à qui on apporta des pierres précieuses trouvées dans une mine, la fit fermer, ne voulant pas fatiguer son peuple à travailler pour une chose qui ne pouvait ni le nourrir ni le vêtir.\par
« Notre luxe est si grand, dit Kiayventi\footnote{Dans un discours rapporté par le P. Du Halde, t. II, p. 418.}, que le peuple orne de broderies les souliers des jeunes garçons et des filles, qu’il est obligé de vendre. » Tant d’hommes étant occupés à faire des habits pour un seul, le moyen qu’il n’y ait bien des gens qui manquent d’habits ? Il y a dix hommes qui mangent le revenu des terres, contre un laboureur : le moyen qu’il n’y ait bien des gens qui manquent d’aliments ?
\subsubsection[{Chapitre VII. Fatale conséquence du luxe à la Chine}]{Chapitre VII. Fatale conséquence du luxe à la Chine}
\noindent On voit dans l’histoire de la Chine qu’elle a eu vingt-deux dynasties qui se sont succédé ; c’est-à-dire qu’elle a éprouvé vingt-deux révolutions générales, sans compter une infinité de particulières. Les trois premières dynasties durèrent assez longtemps, parce qu’elles furent sagement gouvernées, et que l’empire était moins étendu qu’il ne le fut depuis. Mais on peut dire en général que toutes ces dynasties commencèrent assez bien. La vertu, l’attention, la vigilance sont nécessaires à la Chine ; elles y étaient dans le commencement des dynasties, et elles manquaient à la fin. En effet, il était naturel que des empereurs nourris dans les fatigues de la guerre, qui parvenaient à faire descendre du trône une famille noyée dans les délices, conservassent la vertu qu’ils avaient éprouvée si utile, et craignissent les voluptés qu’ils avaient vues si funestes. Mais, après ces trois ou quatre premiers princes, la corruption, le luxe, l’oisiveté, les délices, s’emparent des successeurs ; ils s’enferment dans le palais, leur esprit s’affaiblit, leur vie s’accourcit, la famille décline ; les grands s’élèvent, les eunuques s’accréditent, on ne met sur le trône que des enfants ; le palais devient ennemi de l’empire ; un peuple oisif qui l’habite ruine celui qui travaille ; l’empereur est tué ou détruit par un usurpateur, qui fonde une famille, dont le troisième ou quatrième successeur va dans le même palais se renfermer encore.
\subsubsection[{Chapitre VIII. De la continence publique}]{Chapitre VIII. De la continence publique}
\noindent Il y a tant d’imperfections attachées à la perte de la vertu dans les femmes, toute leur âme en est si fort dégradée, ce point principal ôté en fait tomber tant d’autres, que l’on peut regarder, dans un État populaire, l’incontinence publique comme le dernier des malheurs, et la certitude d’un changement dans la constitution.\par
Aussi les bons législateurs y ont-ils exigé des femmes une certaine gravité de mœurs. Ils ont proscrit de leurs républiques non seulement le vice, mais l’apparence même du vice. Ils ont banni jusqu’à ce commerce de galanterie qui produit l’oisiveté, qui fait que les femmes corrompent avant même d’être corrompues, qui donne un prix à tous les riens, et rabaisse ce qui est important, et qui fait que l’on ne se conduit plus que sur les maximes du ridicule, que les femmes entendent si bien à établir.
\subsubsection[{Chapitre IX. De la condition des femmes dans les divers gouvernements}]{Chapitre IX. De la condition des femmes dans les divers gouvernements}
\noindent Les femmes ont peu de retenue dans les monarchies, parce que la distinction des rangs les appelant à la cour, elles y vont prendre cet esprit de liberté qui est à peu près le seul qu’on y tolère. Chacun se sert de leurs agréments et de leurs passions pour avancer sa fortune ; et comme leur faiblesse ne leur permet pas l’orgueil, mais la vanité, le luxe y règne toujours avec elles.\par
Dans les États despotiques, les femmes n’introduisent point le luxe ; mais elles sont elles-mêmes un objet du luxe. Elles doivent être extrêmement esclaves. Chacun suit l’esprit du gouvernement, et porte chez soi ce qu’il voit établi ailleurs. Comme les lois y sont sévères et exécutées sur-le-champ, on a peur que la liberté des femmes n’y fasse des affaires. Leurs brouilleries, leurs indiscrétions, leurs répugnances, leurs penchants, leurs jalousies, leurs piques, cet art qu’ont les petites âmes d’intéresser les grandes, n’y sauraient être sans conséquence.\par
De plus, comme dans ces États, les princes se jouent de la nature humaine, ils ont plusieurs femmes, et mille considérations les obligent de les renfermer.\par
Dans les républiques, les femmes sont libres par les lois, et captivées par les mœurs ; le luxe en est banni, et avec lui la corruption et les vices.\par
Dans les villes grecques, où l’on ne vivait pas sous cette religion qui établit que, chez les hommes même, la pureté des mœurs est une partie de la vertu ; dans les villes grecques, où un vice aveugle régnait d’une manière effrénée, où l’amour n’avait qu’une forme que l’on n’ose dire, tandis que la seule amitié s’était retirée dans les mariages\footnote{Quant au vrai amour, dit Plutarque, les femmes n’y ont aucune part. {\itshape Œuvres morales}, Traité {\itshape de l’Amour}, p. 600. Il parlait comme son siècle. Voyez Xénophon, au dialogue intitulé {\itshape Hieron}.} \footnote{À Athènes, il y avait un magistrat particulier qui veillait sur la conduite des femmes.}; la vertu, la simplicité, la chasteté des femmes y étaient telles, qu’on n’a guère jamais vu de peuple qui ait eu à cet égard une meilleure police.
\subsubsection[{Chapitre X. Du tribunal domestique chez les romains}]{Chapitre X. Du tribunal domestique chez les romains}
\noindent Les Romains n’avaient pas, comme les Grecs, des magistrats particuliers qui eussent inspection sur la conduite des femmes. Les censeurs n’avaient l’œil sur elles que comme sur le reste de la république. L’institution du tribunal domestique\footnote{Romulus institua ce tribunal, comme il paraît par Denys d’Halicarnasse, liv. II, p. 96.} suppléa à la magistrature établie chez les Grecs\footnote{Voyez dans Tite-Live, liv. XXXIX, l’usage que l’on fit de ce tribunal lors de la conjuration des bacchanales : on appela conjuration contre la république, des assemblées où l’on cor-rompait les mœurs des femmes et des jeunes gens.}.\par
Le mari assemblait les parents de la femme, et la jugeait devant eux\footnote{Il paraît par Denys d’Halicarnasse, liv. II, que par l’institution de Romulus, le mari, dans les cas ordinaires, jugeait seul devant les parents de la femme ; et que, dans les grands crimes, il la jugeait avec cinq d’entre eux. Aussi Ulpien, au titre VI, § 9, 12 et 13, distingue-t-il, dans les jugements des mœurs, celles qu’il appelle graves, d’avec celles qui l’étaient moins : {\itshape mores graviores, mores leviores}.}. Ce tribunal maintenait les mœurs dans la république. Mais ces mêmes mœurs maintenaient ce tribunal, Il devait juger non seulement de la violation des lois, mais aussi de la violation des mœurs. Or, pour juger de la violation des mœurs, il faut en avoir.\par
Les peines de ce tribunal devaient être arbitraires, et l’étaient en effet : car, tout ce qui regarde les mœurs, tout ce qui regarde les règles de la modestie, ne peut guère être compris sous un code de lois. Il est aisé de régler par des lois ce qu’on doit aux autres ; il est difficile d’y comprendre tout ce qu’on se doit à soi-même.\par
Le tribunal domestique regardait la conduite générale des femmes. Mais il y avait un crime qui, outre l’animadversion de ce tribunal, était encore soumis à une accusation publique : c’était l’adultère ; soit que, dans une république, une si grande violation de mœurs intéressât le gouvernement ; soit que le dérèglement de la femme pût faire soupçonner celui du mari ; soit enfin que l’on craignit que les honnêtes gens mêmes n’aimassent mieux cacher ce crime que le punir, l’ignorer que le venger.
\subsubsection[{Chapitre XI. Comment les institutions changèrent à Rome avec le gouvernement}]{Chapitre XI. Comment les institutions changèrent à Rome avec le gouvernement}
\noindent Comme le tribunal domestique supposait des mœurs, l’accusation publique en supposait aussi ; et cela fit que ces deux choses tombèrent avec les mœurs, et finirent avec la république\footnote{{\itshape Judicio de moribus (quod antea quidem in antiquis legibus positum erat, non autem frequentabatur) penitus abolito.} Leg. II, § 2, Cod. {\itshape de repudiis}.}.\par
{\itshape L’établissement des questions perpétuelles, c’est-à-dire, du Partage de la juridiction entre les préteurs, et la coutume qui s’introduisit de plus en plus que ces préteurs jugeassent eux-mêmes}\footnote{{\itshape Judicia extraordinaria}.} toutes les affaires, affaiblirent l’usage du tribunal domestique ; ce qui paraît par la surprise des historiens, qui regardent comme des faits singuliers et comme un renouvellement de la pratique ancienne, les jugements que Tibère fit rendre par ce tribunal.\par
L’établissement de la monarchie et le changement des mœurs firent encore cesser l’accusation publique. On pouvait craindre qu’un malhonnête homme, piqué des mépris d’une femme, indigné de ses refus, outré de sa vertu même, ne formât le dessein de la perdre. La loi Julie ordonna qu’on ne pourrait accuser une femme d’adultère, qu’après avoir accusé son mari de favoriser ses dérèglements ; ce qui restreignit beaucoup cette accusation, et l’anéantit pour ainsi dire\footnote{Constantin l’ôta entièrement : « C’est une chose indigne, disait-il, que des mariages tranquilles soient troublés par l’audace des étrangers. »}.\par
Sixte V sembla vouloir renouveler l’accusation publique\footnote{Sixte V ordonna qu’un mari qui n’irait point se plaindre à lui des débauches de sa femme serait puni de mort. Voyez Leti.}. Mais il ne faut qu’un peu de réflexion pour voir que cette loi, dans une monarchie telle que la sienne, était encore plus déplacée que dans toute autre.
\subsubsection[{Chapitre XII. De la tutelle des femmes chez les romains}]{Chapitre XII. De la tutelle des femmes chez les romains}
\noindent Les institutions des Romains mettaient les femmes dans une perpétuelle tutelle, à moins qu’elles ne fussent sous l’autorité d’un mari\footnote{{\itshape Nisi convenissent in manum viri}.}. Cette tutelle était donnée au plus proche des parents par mâles ; et il paraît, par une expression vulgaire\footnote{{\itshape Ne sis mihi patruus oro}.}, qu’elles étaient très gênées. Cela était bon pour la république, et n’était point nécessaire dans la monarchie\footnote{La loi Papienne ordonna, sous Auguste, que les femmes qui auraient eu trois enfants seraient hors de cette tutelle.}.\par
Il paraît, par les divers codes des lois des barbares, que les femmes, chez les premiers Germains, étaient aussi dans une perpétuelle tutelle\footnote{Cette tutelle s’appelait chez les Germains {\itshape mundeburdium.}}. Cet usage passa dans les monarchies qu’ils fondèrent ; mais il ne subsista pas.
\subsubsection[{Chapitre XIII. Des peines établies par les empereurs contre les débauches des femmes}]{Chapitre XIII. Des peines établies par les empereurs contre les débauches des femmes}
\noindent La loi Julie établit une peine contre l’adultère. Mais, bien loin que cette loi, et celles que l’on fit depuis là-dessus, fussent une marque de la bonté des mœurs, elles furent, au contraire, une marque de leur dépravation.\par
Tout le système politique à l’égard des femmes changea dans la monarchie. Il ne fut plus question d’établir chez elle la pureté des mœurs, mais de punir leurs crimes. On ne faisait de nouvelles lois pour punir ces crimes, que parce qu’on ne punissait plus les violations, qui n’étaient point ces crimes.\par
L’affreux débordement des mœurs obligeait bien les empereurs de faire des lois pour arrêter à un certain point l’impudicité ; mais leur intention ne fut pas de corriger les mœurs en général. Des faits positifs, rapportés par les historiens, prouvent plus cela que toutes ces lois ne sauraient prouver le contraire. On peut voir dans Dion la conduite d’Auguste à cet égard, et comment il éluda, et dans sa préture et dans sa censure, les demandes qui lui furent faites\footnote{Comme on lui eut amené un jeune homme qui avait épousé une femme avec laquelle il avait eu auparavant un mauvais commerce, il hésita longtemps, n’osant ni approuver ni punir ces choses. Enfin, reprenant ses esprits : « Les séditions ont été cause de grands maux, dit-il, oublions-les. » Dion, liv. LIV. Les sénateurs lui ayant demandé des règlements sur les mœurs des femmes, il éluda cette demande, en leur disant qu’ils corrigeassent leurs femmes, comme il corrigeait la sienne. Sur quoi ils le prièrent de leur dire comment il en usait avec sa femme ; question, ce me semble, fort indiscrète.}.\par
On trouve bien dans les historiens des jugements rigides rendus, sous Auguste et sous Tibère, contre l’impudicité de quelques dames romaines : mais en nous faisant connaître l’esprit de ces règnes, ils nous font connaître l’esprit de ces jugements.\par
Auguste et Tibère songèrent principalement à punir les débauches de leurs parentes. Ils ne punissaient point le dérèglement des mœurs, mais un certain crime d’impiété ou de lèse-majesté\footnote{{\itshape Culpam inter viros et feminas vulgatam, gravi nomine laesarum religionum, ac violatae majestatis appellando, clementiam majorum suasque ipse leges egrediebatur.} Tacite, {\itshape Annales}, liv. III.} qu’ils avaient inventé, utile pour le respect, utile pour leur vengeance. De là vient que les auteurs romains s’élèvent si fort contre cette tyrannie.\par
La peine de la loi Julie était légère\footnote{Cette loi est rapportée au Digeste ; mais on n’y a pas mis la peine. On juge qu’elle n’était que de la relégation, puisque celle de l’inceste n’était que de la déportation. Leg. {\itshape Si quis viduam}, ff. {\itshape de quaestionibus.}}. Les empereurs voulurent que, dans les jugements, on augmentât la peine de la loi qu’ils avaient faite. Cela fut le sujet des invectives des historiens. Ils n’examinaient pas si les femmes méritaient d’être punies, mais si l’on avait violé la loi pour les punir.\par
Une des principales tyrannies de Tibère\footnote{{\itshape Proprium id Tiberio fuit, scelera nuper reperta priscis verbis obtegere.} Tacite.} fut l’abus qu’il fit des anciennes lois. Quand il voulut punir quelque dame romaine au-delà de la peine portée par la loi Julie, il rétablit contre elle le tribunal domestique\footnote{{\itshape Adulterii graviorem paenam deprecatus, ut, exemplo majorum, propinquis suis ultra ducentesimum lapidem removeretur, suasit. Adultero Manlio Italia atque Africa interdictum est.} Tacite, {\itshape Annales}, liv. II.}.\par
Ces dispositions à l’égard des femmes ne regardaient que les familles des sénateurs, et non pas celles du peuple. On voulait des prétextes aux accusations contre les grands, et les déportements des femmes en pouvaient fournir sans nombre.\par
Enfin ce que j’ai dit, que la bonté des mœurs n’est pas le principe du gouvernement d’un seul, ne se vérifia jamais mieux que sous ces premiers empereurs ; et si l’on en doutait, on n’aurait qu’à lire Tacite, Suétone, Juvénal et Martial.
\subsubsection[{Chapitre XIV. Lois somptuaires chez les Romains}]{Chapitre XIV. Lois somptuaires chez les Romains}
\noindent Nous avons parlé de l’incontinence publique, parce qu’elle est jointe avec le luxe, qu’elle en est toujours suivie, et qu’elle le suit toujours. Si vous laissez en liberté les mouvements du cœur, comment pourrez-vous gêner les faiblesses de l’esprit ?\par
À Rome, outre les institutions générales, les censeurs firent faire, par les magistrats, plusieurs lois particulières, pour maintenir les femmes dans la frugalité. Les lois {\itshape Fannienne, Licinienne} et {\itshape Oppienne} eurent cet objet. Il faut voir dans Tite-Live\footnote{Décade IV, liv. IV.} comment le sénat fut agité, lorsqu’elles demandèrent la révocation de la loi {\itshape Oppienne}. Valère-Maxime met l’époque du luxe chez les Romains à l’abrogation de cette loi.
\subsubsection[{Chapitre XV. Des dots et des avantages nuptiaux dans les diverses constitutions}]{Chapitre XV. Des dots et des avantages nuptiaux dans les diverses constitutions}
\noindent Les dots doivent être considérables dans les monarchies, afin que les maris puissent soutenir leur rang et le luxe établi. Elles doivent être médiocres dans les républiques, où le luxe ne doit pas régner\footnote{Marseille fut la plus sage des républiques de son temps les dots ne pouvaient passer cent écus en argent, et cinq en habits, dit Strabon, liv. IV.}. Elles doivent être à peu près nulles dans les États despotiques, où les femmes sont, en quelque façon, esclaves.\par
La communauté des biens introduite par les lois françaises entre le mari et la femme, est très convenable dans le gouvernement monarchique, parce qu’elle intéresse les femmes aux affaires domestiques, et les rappelle, comme malgré elles, au soin de leur maison. Elle l’est moins dans la république, où les femmes ont plus de vertu. Elle serait absurde dans les États despotiques, où presque toujours les femmes sont elles-mêmes une partie de la propriété du maître.\par
Comme les femmes, par leur état, sont assez portées au mariage, les gains que la loi leur donne sur les biens de leur mari sont inutiles. Mais ils seraient très pernicieux dans une république, parce que leurs richesses particulières produisent le luxe. Dans les États despotiques, les gains de noces doivent être leur subsistance, et rien de plus.
\subsubsection[{Chapitre XVI. Belle coutume des Samnites}]{Chapitre XVI. Belle coutume des Samnites}
\noindent Les Samnites avaient une coutume qui, dans une petite république, et surtout dans la situation où était la leur, devait produire d’admirables effets. On assemblait tous les jeunes gens, et on les jugeait. Celui qui était déclaré le meilleur de tous prenait pour sa femme la fille qu’il voulait ; celui qui avait les suffrages après lui choisissait encore ; et ainsi de Suite\footnote{Fragm. de Nicolas de Damas, tiré de Stobée, dans le {\itshape Recueil} de Constantin Porphyrogénète.}. Il était admirable de ne regarder entre les biens des garçons que les belles qualités, et les services rendus à la patrie. Celui qui était le plus riche de ces sortes de biens choisissait une fille dans toute la nation. L’amour, la beauté, la chasteté, la vertu, la naissance, les richesses mêmes, tout cela était, pour ainsi dire, la dot de la vertu. Il serait difficile d’imaginer une récompense plus noble, plus grande, moins à charge à un petit État, plus capable d’agir sur l’un et l’autre sexe.\par
Les Samnites descendaient des Lacédémoniens ; et Platon, dont les institutions ne sont que la perfection des lois de Lycurgue, donna à peu près une pareille loi\footnote{Il leur permet même de se voir plus fréquemment.}.
\subsubsection[{Chapitre XVII. De l’administration des femmes}]{Chapitre XVII. De l’administration des femmes}
\noindent Il est contre la raison et contre la nature que les femmes soient maîtresses dans la maison, comme cela était établi chez les Égyptiens ; mais il ne l’est pas qu’elles gouvernent un empire. Dans le premier cas, l’état de faiblesse où elles sont ne leur permet pas la prééminence ; dans le second, leur faiblesse même leur donne plus de douceur et de modération ; ce qui peut faire un bon gouvernement, plutôt que les vertus dures et féroces.\par
Dans les Indes, on se trouve très bien du gouvernement des femmes ; et il est établi que, si les mâles ne viennent pas d’une mère du même sang, les filles qui ont une mère du sang royal, succèdent\footnote{{\itshape Lettres édifiantes}, 14\textsuperscript{e} recueil.}. On leur donne un certain nombre de personnes pour les aider à porter le poids du gouvernement. Selon M. Smith\footnote{{\itshape Voyage de Guinée}, seconde partie, p. 165 de la traduction sur le royaume d’Angona, sur la Côte d’Or.}, on se trouve aussi très bien du gouvernement des femmes en Afrique. Si l’on ajoute à cela l’exemple de la Moscovie et de l’Angleterre, on verra qu’elles réussissent également et dans le gouvernement modéré, et dans le gouvernement despotique.
\subsection[{Livre huitième. De la corruption des principes des trois gouvernements}]{Livre huitième. De la corruption des principes des trois gouvernements}
\subsubsection[{Chapitre I. Idée générale de ce livre}]{Chapitre I. Idée générale de ce livre}
\noindent La corruption de chaque gouvernement commence presque toujours par celle des principes.
\subsubsection[{Chapitre II. De la corruption du principe de la démocratie}]{Chapitre II. De la corruption du principe de la démocratie}
\noindent Le principe de la démocratie se corrompt, non seulement lorsqu’on perd l’esprit d’égalité, mais encore quand on prend l’esprit d’égalité extrême, et que chacun veut être égal à ceux qu’il choisit pour lui commander. Pour lors le peuple, ne pouvant souffrir le pouvoir même qu’il confie, veut tout faire par lui-même, délibérer pour le sénat, exécuter pour les magistrats, et dépouiller tous les juges.\par
Il ne peut plus y avoir de vertu dans la république. Le peuple veut faire les fonctions des magistrats : on ne les respecte donc plus. Les délibérations du sénat n’ont plus de poids ; on n’a donc plus d’égards pour les sénateurs, et par conséquent pour les vieillards. Que si l’on n’a pas du respect pour les vieillards, on n’en aura pas non plus pour les pères ; les maris ne méritent pas plus de déférence, ni les maîtres plus de soumission. Tout le monde parviendra à aimer ce libertinage : la gêne du commandement fatiguera comme celle de l’obéissance. Les femmes, les enfants, les esclaves n’auront de soumission pour personne. Il n’y aura plus de mœurs, plus d’amour de l’ordre, enfin plus de vertu.\par
On voit, dans le {\itshape Banquet} de Xénophon, une peinture bien naïve d’une république où le peuple a abusé de l’égalité. Chaque convive donne à son tour la raison pourquoi il est content de lui. « je suis content de moi, dit Charmides, à cause de ma pauvreté. Quand j’étais riche, j’étais obligé de faire ma cour aux calomniateurs, sachant bien que j’étais plus en état de recevoir du mal d’eux que de leur en faire ; la république me demandait toujours quelque nouvelle somme ; je ne pouvais m’absenter. Depuis que je suis pauvre, j’ai acquis de l’autorité ; personne ne me menace, je menace les autres ; je puis m’en aller ou rester. Déjà les riches se lèvent de leurs places, et me cèdent le pas. Je suis un roi, j’étais esclave ; je payais un tribut à la république, aujourd’hui elle me nourrit ; je ne crains plus de perdre, j’espère d’acquérir. »\par
Le peuple tombe dans ce malheur, lorsque ceux à qui il se confie, voulant cacher leur propre corruption, cherchent à le corrompre. Pour qu’il ne voie pas leur ambition, ils ne lui parlent que de sa grandeur ; pour qu’il n’aperçoive pas leur avarice, ils flattent sans cesse la sienne.\par
La corruption augmentera parmi les corrupteurs, et elle augmentera parmi ceux qui sont déjà corrompus. Le peuple se distribuera tous les deniers publics ; et, comme il aura joint à sa paresse la gestion des affaires, il voudra joindre à sa pauvreté les amusements du luxe, Mais, avec sa paresse et son luxe, il n’y aura que le trésor public qui puisse être un objet pour lui.\par
Il ne faudra pas s’étonner si l’on voit les suffrages se donner pour de l’argent. On ne peut donner beaucoup au peuple, sans retirer encore plus de lui ; mais, pour retirer de lui, il faut renverser l’État. Plus il paraîtra tirer d’avantage de sa liberté, plus il s’approchera du moment où il doit la perdre. Il se forme de petits tyrans qui ont tous les vices d’un seul. Bientôt ce qui reste de liberté devient insupportable ; un seul tyran s’élève ; et le peuple perd tout, jusqu’aux avantages de sa corruption.\par
La démocratie a donc deux excès à éviter : l’esprit d’inégalité, qui la mène à l’aristocratie, ou au gouvernement d’un seul ; et l’esprit d’égalité extrême, qui la conduit au despotisme d’un seul, comme le despotisme d’un seul finit par la conquête.\par
Il est vrai que ceux qui corrompirent les républiques grecques ne devinrent pas toujours tyrans. C’est qu’ils s’étaient plus attachés à l’éloquence qu’à l’art militaire : outre qu’il y avait dans le cœur de tous les Grecs une haine implacable contre ceux qui renversaient le gouvernement républicain ; ce qui fit que l’anarchie dégénéra en anéantissement, au lieu de se changer en tyrannie.\par
Mais Syracuse, qui se trouva placée au milieu d’un grand nombre de petites oligarchies changées en tyrannies\footnote{Voyez Plutarque, dans les {\itshape Vies de Timoléon} et {\itshape de Dion}.} \footnote{C’est celui des six cents, dont parle Diodore.}; Syracuse, qui avait un sénat dont il n’est presque jamais fait mention dans l’histoire, essuya des malheurs que la corruption ordinaire ne donne pas. Cette ville, toujours dans la licence\footnote{Ayant chassé les tyrans, ils firent citoyens des étrangers et des soldats mercenaires, ce qui causa des guerres civiles, Aristote, {\itshape Politique}, liv. V, chap. III. Le peuple ayant été cause de la victoire sur les Athéniens, la république fut changée, {\itshape ibid.}, chap. IV. La passion de deux jeunes magistrats, dont l’un enleva à l’autre un jeune garçon, et celui-ci lui débaucha sa femme, fit changer la forme de cette république, {\itshape ibid.}, liv. VII, chap. IV.} ou dans l’oppression, également travaillée par sa liberté et par sa servitude, recevant toujours l’une et l’autre comme une tempête, et malgré sa puissance au-dehors, toujours déterminée à une révolution par la plus petite force étrangère, avait dans son sein un peuple immense, qui n’eut jamais que cette cruelle alternative de se donner un tyran, ou de l’être lui-même.
\subsubsection[{Chapitre III. De l’esprit d’égalité extrême}]{Chapitre III. De l’esprit d’égalité extrême}
\noindent Autant que le ciel est éloigné de la terre, autant le véritable esprit d’égalité l’est-il de l’esprit d’égalité extrême. Le premier ne consiste point à faire en sorte que tout le monde commande, ou que personne ne soit commandé ; mais à obéir et à commander à ses égaux. Il ne cherche pas à n’avoir point de maître, mais à n’avoir que ses égaux pour maîtres.\par
Dans l’état de nature, les hommes naissent bien dans l’égalité ; mais ils n’y sauraient rester. La société la leur fait perdre, et ils ne redeviennent égaux que par les lois.\par
Telle est la différence entre la démocratie réglée et celle qui ne l’est pas, que, dans la première, on n’est égal que comme citoyen, et que, dans l’autre, on est encore égal comme magistrat, comme sénateur, comme juge, comme père, comme mari, comme maître.\par
La place naturelle de la vertu est auprès de la liberté ; mais elle ne se trouve pas plus auprès de la liberté extrême qu’auprès de la servitude.
\subsubsection[{Chapitre IV. Cause particulière de la corruption du peuple}]{Chapitre IV. Cause particulière de la corruption du peuple}
\noindent Les grands succès, surtout ceux auxquels le peuple contribue beaucoup, lui donnent un tel orgueil, qu’il n’est plus possible de le conduire. Jaloux des magistrats, il le devient de la magistrature ; ennemi de ceux qui gouvernent, il l’est bientôt de la constitution. C’est ainsi que la victoire de Salamine sur les Perses corrompit la république d’Athènes\footnote{Aristote, {\itshape Politique}, liv. V, chap. IV.} {\itshape ;} c’est ainsi que la défaite des Athéniens perdit la république de Syracuse\footnote{{\itshape Ibid.}}.\par
Celle de Marseille n’éprouva jamais ces grands passages de l’abaissement à la grandeur : aussi se gouverna-t-elle toujours avec sagesse ; aussi conserva-t-elle ses principes.
\subsubsection[{Chapitre V. De la corruption du principe de l’aristocratie}]{Chapitre V. De la corruption du principe de l’aristocratie}
\noindent L’aristocratie se corrompt lorsque le pouvoir des nobles devient arbitraire : il ne peut plus y avoir de vertu dans ceux qui gouvernent, ni dans ceux qui sont gouvernés.\par
Quand les familles régnantes observent les lois, c’est une monarchie qui a plusieurs monarques, et qui est très bonne par sa nature ; presque tous ces monarques sont liés par les lois. Mais quand elles ne les observent pas, c’est un État despotique qui a plusieurs despotes.\par
Dans ce cas, la république ne subsiste qu’à l’égard des nobles, et entre eux seulement. Elle est dans le corps qui gouverne, et l’État despotique est dans le corps qui est gouverné ; ce qui fait les deux corps du monde les plus désunis.\par
L’extrême corruption est lorsque les nobles deviennent héréditaires\footnote{L’aristocratie se change en oligarchie.} {\itshape ;} ils ne peuvent plus guère avoir de modération. S’ils sont en petit nombre, leur pouvoir est plus grand, mais leur sûreté diminue ; s’ils sont en plus grand nombre, leur pouvoir est moindre, et leur sûreté plus grande : en sorte que le pouvoir va croissant, et la sûreté diminuant, jusqu’au despote, sur la tête duquel est l’excès du pouvoir et du danger.\par
Le grand nombre des nobles dans l’aristocratie héréditaire rendra donc le gouvernement moins violent ; mais comme il y aura peu de vertu, on tombera dans un esprit de nonchalance, de paresse, d’abandon, qui fera que l’État n’aura plus de force ni de ressort\footnote{Venise est une des républiques qui a le mieux corrigé, par ses lois, les inconvénients de l’aristocratie héréditaire.}.\par
Une aristocratie peut maintenir la force de son principe, si les lois sont telles qu’elles fassent plus sentir aux nobles les périls et les fatigues du commandement que ses délices ; et si l’État est dans une telle situation qu’il ait quelque chose à redouter ; et que la sûreté vienne du dedans, et l’incertitude du dehors.\par
Comme une certaine confiance fait la gloire et la sûreté d’une monarchie, il faut au contraire qu’une république redoute quelque chose\footnote{Justin attribue à la mort d’Épaminondas l’extinction de la vertu à Athènes. N’ayant plus d’émulation, ils dépensèrent leurs revenus en fêtes, {\itshape frequentius coenam quam castra visentes.} Pour lors, les Macédoniens sortirent de l’obscurité. Liv. VI.}. La crainte des Perses maintint les lois chez les Grecs. Carthage et Rome s’intimidèrent l’une l’autre, et s’affermirent. Chose singulière ! plus ces États ont de sûreté, plus, comme des eaux trop tranquilles, ils sont sujets à se corrompre.
\subsubsection[{Chapitre VI. De la corruption du principe de la monarchie}]{Chapitre VI. De la corruption du principe de la monarchie}
\noindent Comme les démocraties se perdent lorsque le peuple dépouille le sénat, les magistrats et les juges de leurs fonctions, les monarchies se corrompent lorsqu’on ôte peu à peu les prérogatives des corps ou les privilèges des villes. Dans le premier cas, on va au despotisme de tous ; dans l’autre, au despotisme d’un seul.\par
« Ce qui perdit les dynasties de Tsin et de Souï, dit un auteur chinois, c’est qu’au lieu de se borner, comme les anciens, à une inspection générale, seule digne du souverain, les princes voulurent gouverner tout immédiatement par eux-mêmes\footnote{Compilation d’ouvrages faits sous les Ming, rapportés par le P. Du Halde.}. » L’auteur chinois nous donne ici la cause de la corruption de presque toutes les monarchies.\par
La monarchie se perd, lorsqu’un prince croit qu’il montre plus sa puissance en changeant l’ordre des choses qu’en le suivant ; lorsqu’il ôte les fonctions naturelles des uns pour les donner arbitrairement à d’autres ; et lorsqu’il est plus amoureux de ses fantaisies que de ses volontés.\par
La monarchie se perd, lorsque le prince, rapportant tout uniquement à lui, appelle l’État à sa capitale, la capitale à sa cour, et la cour à sa seule personne.\par
Enfin elle se perd, lorsqu’un prince méconnaît son autorité, sa situation, l’amour de ses peuples ; et lorsqu’il ne sent pas bien qu’un monarque doit se juger en sûreté, comme un despote doit se croire en péril.
\subsubsection[{Chapitre VII. Continuation du même sujet}]{Chapitre VII. Continuation du même sujet}
\noindent Le principe de la monarchie se corrompt lorsque les premières dignités sont les marques de la première servitude, lorsqu’on ôte aux grands le respect des peuples, et qu’on les rend de vils instruments du pouvoir arbitraire.\par
Il se corrompt encore plus, lorsque l’honneur a été mis en contradiction avec les honneurs, et que l’on peut être à la fois couvert d’infamie\footnote{{\itshape Sous} le règne de Tibère, on éleva des statues et l’on donna les ornements triomphaux aux délateurs : ce qui avilit tellement ces honneurs, que ceux qui les avaient mérités, les dédaignèrent. Fragment de Dion, liv. LVIII, chap. XIV, tiré de l’{\itshape Extrait des vertus et des vices}, de Constantin Porphyrogénète. Voyez dans Tacite comment Néron, sur la découverte et la punition d’une prétendue conjuration, donna à Petronius Turpilianus, à Nerva, à Tigellinus, les ornements triomphaux, {\itshape Annales}, liv. XV. Voyez aussi comment les généraux dédaignèrent de faire la guerre, parce qu’ils en méprisaient les honneurs. {\itshape Pervulgatis triumphi insignibus.} Tacite, {\itshape Annales}, liv. XIII.} et de dignités.\par
Il se corrompt lorsque le prince change sa justice en sévérité ; lorsqu’il met, comme les empereurs romains, une tête de Méduse sur sa poitrine\footnote{Dans cet état, le prince savait bien quel était le principe de son gouvernement.} {\itshape ; lorsqu’il} prend cet air menaçant et terrible que Commode faisait donner à ses statues\footnote{Hérodien.}.\par
Le principe de la monarchie se corrompt lorsque des âmes singulièrement lâches tirent vanité de la grandeur que pourrait avoir leur servitude ; et qu’elles croient que ce qui fait que l’on doit tout au prince, fait que l’on ne doit rien à sa patrie.\par
Mais s’il est vrai (ce que l’on a vu dans tous les temps) qu’à mesure que le pouvoir du monarque devient immense, sa sûreté diminue ; corrompre ce pouvoir, jusqu’à le faire changer de nature, n’est-ce pas un crime de lèse-majesté contre lui ?
\subsubsection[{Chapitre VIII. Danger de la corruption du principe du gouvernement monarchique}]{Chapitre VIII. Danger de la corruption du principe du gouvernement monarchique}
\noindent L’inconvénient n’est pas lorsque l’État passe d’un gouvernement modéré à un gouvernement modéré, comme de la république à la monarchie, ou de la monarchie à la république ; mais quand il tombe et se précipite du gouvernement modéré au despotisme.\par
La plupart des peuples d’Europe sont encore gouvernés par les mœurs. Mais si, par un long abus du pouvoir, si, par une grande conquête, le despotisme s’établissait à un certain point, il n’y aurait pas de mœurs ni de climat qui tinssent ; et, dans cette belle partie du monde, la nature humaine souffrirait, au moins pour un temps, les insultes qu’on lui fait dans les trois autres.
\subsubsection[{Chapitre IX. Combien la noblesse est portée à défendre le trône}]{Chapitre IX. Combien la noblesse est portée à défendre le trône}
\noindent La noblesse anglaise s’ensevelit avec Charles I\textsuperscript{er} sous les débris du trône ; et, avant cela, lorsque Philippe II fit entendre aux oreilles des Français le mot de liberté, la couronne fut toujours soutenue par cette noblesse, qui tient à honneur d’obéir à un roi, mais qui regarde comme la souveraine infamie de partager la puissance avec le peuple.\par
On a vu la maison d’Autriche travailler sans relâche à opprimer la noblesse hongroise. Elle ignorait de quel prix elle lui serait quelque jour, elle cherchait chez ces peuples de l’argent qui n’y était pas ; elle ne voyait pas des hommes qui y étaient. Lorsque tant de princes partageaient entre eux ses États, toutes les pièces de sa monarchie, immobiles et sans action, tombaient, pour ainsi dire, les unes sur les autres. Il n’y avait de vie que dans cette noblesse, qui s’indigna, oublia tout pour combattre, et crut qu’il était de sa gloire de périr et de pardonner.
\subsubsection[{Chapitre X. De la corruption du principe du gouvernement despotique}]{Chapitre X. De la corruption du principe du gouvernement despotique}
\noindent Le principe du gouvernement despotique se corrompt sans cesse, parce qu’il est corrompu par sa nature. Les autres gouvernements périssent, parce que des accidents particuliers en violent le principe : celui-ci périt par son vice intérieur, lorsque quelques causes accidentelles n’empêchent point son principe de se corrompre. Il ne se maintient donc que quand des circonstances tirées du climat, de la religion, de la situation ou du génie du peuple, le forcent à suivre quelque ordre, et à souffrir quelque règle. Ces choses forcent sa nature sans la changer ; sa férocité reste ; elle est pour quelque temps apprivoisée.\par
 \textbf{Chapitre XI. {\itshape Effets naturels de la bonté et de la corruption des principes} }  \par
Lorsque les principes du gouvernement sont une fois corrompus, les meilleures lois deviennent mauvaises, et se tournent contre l’État ; lorsque les principes en sont sains, les mauvaises ont l’effet des bonnes ; la force du principe entraîne tout.\par
Les Crétois, pour tenir les premiers magistrats dans la dépendance des lois, employaient un moyen bien singulier : c’était celui de {\itshape l’insurrection}. Une partie des citoyens se soulevait\footnote{Aristote, {\itshape Politique}, liv. II, chap. X.}, mettait en fuite les magistrats, et les obligeait de rentrer dans la condition privée. Cela était censé fait en conséquence de la loi. Une institution pareille, qui établissait la sédition pour empêcher l’abus du pouvoir, semblait devoir renverser quelque république que ce fût ; elle ne détruisit pas celle de Crète. Voici pourquoi\footnote{On se réunissait toujours d’abord contre les ennemis du dehors, ce qui s’appelait {\itshape syncrétisme.} Plutarque, {\itshape Œuvres morales}, p. 88.} :\par
Lorsque les Anciens voulaient parler d’un peuple qui avait le plus grand amour pour la patrie, ils citaient les Crétois. {\itshape La patrie}, disait Platon\footnote{{\itshape République}, liv. IX.}, {\itshape nom si tendre aux Crétois.} Ils l’appelaient d’un nom qui exprime l’amour d’une mère pour ses enfants\footnote{Plutarque, {\itshape Œuvres morales}, au traité : {\itshape Si l’homme d’âge doit se mêler des affaires publiques}.}. Or, l’amour de la patrie corrige tout.\par
Les lois de Pologne ont aussi leur {\itshape insurrection}. Mais les inconvénients qui en résultent font bien voir que le seul peuple de Crète était en état d’employer avec succès un pareil remède.\par
Les exercices de la gymnastique établis chez les Grecs ne dépendirent pas moins de la bonté du principe du gouvernement. « Ce furent les Lacédémoniens et les Crétois, dit Platon\footnote{République, liv. V.}, qui ouvrirent ces académies fameuses qui leur firent tenir dans le monde un rang si distingué. La pudeur s’alarma d’abord ; mais elle céda à l’utilité publique. » Du temps de Platon, ces institutions étaient admirables\footnote{La gymnastique se divisait en deux parties : la danse et la lutte. On voyait en Crète les danses armées des Curètes ; à Lacédémone, celles de Castor et de Pollux ; à Athènes, les danses armées de Pallas, très propres pour ceux qui ne sont pas encore en âge d’aller à la guerre. La lutte est l’image de la guerre, dit Platon, {\itshape Des Lois}, liv. VII. Il loue l’Antiquité de n’avoir établi que deux danses : la pacifique et la pyrrhique. Voyez comment cette dernière danse s’appliquait à l’art militaire. Platon, {\itshape ibid.}} \footnote{……. {\itshape Aut libidinosae} / {\itshape Ledaeas Lacedaemonis palestras.} / (MARTIAL, liv. IV, epig. 55.)}: elles se rapportaient à un grand objet, qui était l’art militaire. Mais, lorsque les Grecs n’eurent plus de vertu, elles détruisirent l’art militaire même : on ne descendit plus sur l’arène pour se former, mais pour se corrompre.\par
Plutarque nous dit\footnote{{\itshape Œuvres morales}, au traité : {\itshape Des demandes des choses romaines}.} que, de son temps, les Romains pensaient que ces jeux avaient été la principale cause de la servitude où étaient tombés les Grecs. C’était, au contraire, la servitude des Grecs qui avait corrompu ces exercices. Du temps de Plutarque\footnote{Plutarque, {\itshape ibid.}}, les parcs où l’on combattait à nu, et les jeux de la lutte, rendaient les jeunes gens lâches, les portaient à un amour infâme, et n’en faisaient que des baladins ; mais du temps d’Épaminondas, l’exercice de la lutte faisait gagner aux Thébains la bataille de Leuctres\footnote{Plutarque, {\itshape Œuvres morales. Propos de table}, liv. II.}.\par
Il y a peu de lois qui ne soient bonnes, lorsque l’État n’a point perdu ses principes ; et, comme disait Épicure en parlant des richesses : « Ce n’est point la liqueur qui est corrompue, c’est le vase. »
\subsubsection[{Chapitre XII. Continuation du même sujet}]{Chapitre XII. Continuation du même sujet}
\noindent On prenait à Rome les juges dans l’ordre des sénateurs. Les Gracques transportèrent cette prérogative aux chevaliers. Drusus la donna aux sénateurs et aux chevaliers ; Sylla, aux sénateurs seuls ; Cotta, aux sénateurs, aux chevaliers et aux trésoriers de l’épargne. César exclut ces derniers. Antoine fit des décuries de sénateurs, de chevaliers et de centurions.\par
Quand une république est corrompue, on ne peut remédier à aucun des maux qui naissent, qu’en ôtant la corruption et en rappelant les principes : toute autre correction est ou inutile, ou un nouveau mal. Pendant que Rome conserva ses principes, les jugements purent être sans abus entre les mains des sénateurs ; mais quand elle fut corrompue, à quelque corps que ce fût qu’on transportât les jugements, aux sénateurs, aux chevaliers, aux trésoriers de l’épargne, à deux de ces corps, à tous les trois ensemble, à quelque autre corps que ce fût, on était toujours mal. Les chevaliers n’avaient pas plus de vertu que les sénateurs, les trésoriers de l’épargne pas plus que les chevaliers, et ceux-ci aussi peu que les centurions.\par
Lorsque le peuple de Rome eut obtenu qu’il aurait part aux magistratures patriciennes, il était naturel de penser que ses flatteurs allaient être les arbitres du gouvernement. Non : l’on vit ce peuple, qui rendait les magistratures communes aux plébéiens, élire toujours des patriciens. Parce qu’il était vertueux, il était magnanime ; parce qu’il était libre, il dédaignait le pouvoir. Mais lorsqu’il eut perdu ses principes, plus il eut de pouvoir, moins il eut de ménagements ; jusqu’à ce qu’enfin, devenu son propre tyran et son propre esclave, il perdit la force de la liberté pour tomber dans la faiblesse de la licence.
\subsubsection[{Chapitre XIII. Effet du serment chez un peuple vertueux}]{Chapitre XIII. Effet du serment chez un peuple vertueux}
\noindent Il n’y a point eu de peuple, dit Tite-Live\footnote{Liv. I.}, où la dissolution se soit plus tard introduite que chez les Romains, et où la modération et la pauvreté aient été plus longtemps honorées.\par
Le serment eut tant de force chez ce peuple, que rien ne l’attacha plus aux lois. Il fit bien des fois pour l’observer ce qu’il n’aurait jamais fait pour la gloire ni pour la patrie.\par
Quintius Cincinnatus, consul, ayant voulu lever une armée dans la ville contre les Èques et les Volsques, les tribuns s’y opposèrent. « Eh bien ! dit-il, que tous ceux qui ont fait serment au consul de l’année précédente marchent sous mes enseignes\footnote{Tite-Live, liv. III.}. » En vain les tribuns s’écrièrent-ils qu’on n’était plus lié par ce serment ; que, quand on l’avait fait, Quintius était un homme privé : le peuple fut plus religieux que ceux qui se mêlaient de le conduire ; il n’écouta ni les distinctions ni les interprétations des tribuns.\par
Lorsque le même peuple voulut se retirer sur le Mont-Sacré, il se sentit retenir par le serment qu’il avait fait aux consuls de les suivre à la guerre\footnote{Tite-Live, liv. II.}. Il forma le dessein de les tuer ; on lui fit entendre que le serment n’en subsisterait pas moins. On peut juger de l’idée qu’il avait de la violation du serment, par le crime qu’il voulait commettre.\par
Après la bataille de Cannes, le peuple effrayé voulut se retirer en Sicile : Scipion lui fit jurer qu’il resterait à Rome ; la crainte de violer leur serment surmonta toute autre crainte. Rome était un vaisseau tenu par deux ancres dans la tempête : la religion et les mœurs.
\subsubsection[{Chapitre XIV. Comment le plus petit changement dans la constitution entraîne la ruine des principes}]{Chapitre XIV. Comment le plus petit changement dans la constitution entraîne la ruine des principes}
\noindent Aristote nous parle de la république de Carthage comme d’une république très bien réglée. Polybe nous dit qu’à la seconde guerre punique\footnote{Environ cent ans après.} il y avait à Carthage cet inconvénient, que le sénat avait perdu presque toute son autorité. Tite-Live nous apprend que, lorsque Annibal retourna à Carthage, il trouva que les magistrats et les principaux citoyens détournaient à leur profit les revenus publics, et abusaient de leur pouvoir. La vertu des magistrats tomba donc avec l’autorité du sénat ; tout coula du même principe.\par
On connaît les prodiges de la censure chez les Romains. Il y eut un temps où elle devint pesante ; mais on la soutint, parce qu’il y avait plus de luxe que de corruption. Claudius l’affaiblit ; et par cet affaiblissement, la corruption devint encore plus grande que le luxe ; et la censure\footnote{Voyez Dion, liv. XXXVIII ; la {\itshape Vie de Cicéron} dans Plutarque ; Cicéron à Atticus, liv. IV, lettres X et XV ; Asconius sur Cicéron, {\itshape De divinatione}.} s’abolit, pour ainsi dire, d’elle-même. Troublée, demandée, reprise, quittée, elle fut entièrement interrompue jusqu’au temps où elle devint inutile, je veux dire les règnes d’Auguste et de Claude.
\subsubsection[{Chapitre XV. Moyens très efficaces pour la conservation des trois principes}]{Chapitre XV. Moyens très efficaces pour la conservation des trois principes}
\noindent Je ne pourrai me faire entendre que lorsqu’on aura lu les quatre chapitres suivants.
\subsubsection[{Chapitre XVI. Propriétés distinctives de la république}]{Chapitre XVI. Propriétés distinctives de la république}
\noindent Il est de la nature d’une république qu’elle n’ait qu’un petit territoire : sans cela elle ne peut guère subsister. Dans une grande république, il y a de grandes fortunes, et par conséquent peu de modération dans les esprits : il y a de trop grands dépôts à mettre entre les mains d’un citoyen ; les intérêts se particularisent ; un homme sent d’abord qu’il peut être heureux, grand, glorieux, sans sa patrie ; et bientôt, qu’il peut être seul grand sur les ruines de sa patrie.\par
Dans une grande république, le bien commun est sacrifié à mille considérations ; il est subordonné à des exceptions ; il dépend des accidents. Dans une petite, le bien public est mieux senti, mieux connu, plus près de chaque citoyen ; les abus y sont moins étendus, et par conséquent moins protégés.\par
Ce qui fit subsister si longtemps Lacédémone, c’est qu’après toutes ses guerres, elle resta toujours avec son territoire. Le seul but de Lacédémone était la liberté ; le seul avantage de sa liberté, c’était la gloire.\par
Ce fut l’esprit des républiques grecques de se contenter de leurs terres, comme de leurs lois. Athènes prit de l’ambition, et en donna à Lacédémone : mais ce fut plutôt pour commander à des peuples libres, que pour gouverner des esclaves ; plutôt pour être à la tête de l’union, que pour la rompre. Tout fut perdu lorsqu’une monarchie s’éleva ; gouvernement dont l’esprit est plus tourné vers l’agrandissement.\par
Sans des circonstances particulières\footnote{Comme quand un petit souverain se maintient entre deux grands États par leur jalousie mutuelle ; mais il n’existe que précairement.}, il est difficile que tout autre gouvernement que le républicain puisse subsister dans une seule ville. Un prince d’un si petit État chercherait naturellement à opprimer, parce qu’il aurait une grande puissance et peu de moyens pour en jouir, ou pour la faire respecter : il foulerait donc beaucoup ses peuples. D’un autre côté, un tel prince serait aisément opprimé par une force étrangère, ou même par une force domestique : le peuple pourrait à tous les instants s’assembler et se réunir contre lui. Or, quand un prince d’une ville est chassé de sa ville, le procès est fini ; s’il a plusieurs villes, le procès n’est que commencé.
\subsubsection[{Chapitre XVII. Propriétés distinctives de la monarchie}]{Chapitre XVII. Propriétés distinctives de la monarchie}
\noindent Un État monarchique doit être d’une grandeur médiocre. S’il était petit, il se formerait en république ; s’il était fort étendu, les principaux de l’État, grands par eux-mêmes, n’étant point sous les yeux du prince, ayant leur cour hors de sa cour, assurés d’ailleurs contre les exécutions promptes par les lois et par les mœurs, pourraient cesser d’obéir ; ils ne craindraient pas une punition trop lente et trop éloignée.\par
Aussi Charlemagne eut-il à peine fondé son empire, qu’il fallut le diviser ; soit que les gouverneurs des provinces n’obéissent pas ; soit que, pour les faire mieux obéir, il fût nécessaire de partager l’empire en plusieurs royaumes.\par
Après la mort d’Alexandre, son empire fut partagé. Comment ces grands de Grèce et de Macédoine, libres, ou du moins chefs des conquérants répandus dans cette vaste conquête, auraient-ils pu obéir ?\par
Après la mort d’Attila, son empire fut dissous : tant de rois qui n’étaient plus contenus, ne pouvaient point reprendre des chaînes.\par
Le prompt établissement du pouvoir sans bornes est le remède qui, dans ces cas, peut prévenir la dissolution : nouveau malheur après celui de l’agrandissement !\par
Les fleuves courent se mêler dans la mer : les monarchies vont se perdre dans le despotisme.
\subsubsection[{Chapitre XVIII. Que la monarchie d’Espagne était dans un cas particulier}]{Chapitre XVIII. Que la monarchie d’Espagne était dans un cas particulier}
\noindent Qu’on ne cite point l’exemple de l’Espagne ; elle prouve plutôt ce que je dis. Pour garder l’Amérique, elle fit ce que le despotisme même ne fait pas ; elle en détruisit les habitants. Il fallut, pour conserver sa colonie, qu’elle la tint dans la dépendance de sa subsistance même.\par
Elle essaya le despotisme dans les Pays-Bas ; et sitôt qu’elle l’eut abandonné, ses embarras augmentèrent. D’un côté, les Wallons ne voulaient pas être gouvernés par les Espagnols ; et de l’autre, les soldats espagnols ne voulaient pas obéir aux officiers wallons\footnote{Voyez l’{\itshape Histoire des Provinces-Unies}, par M. Le Clerc.}.\par
Elle ne se maintint dans l’Italie, qu’à force de l’enrichir et de se ruiner : car ceux qui auraient voulu se défaire du roi d’Espagne n’étaient pas pour cela d’humeur à renoncer à son argent.
\subsubsection[{Chapitre XIX. Propriétés distinctives du gouvernement despotique}]{Chapitre XIX. Propriétés distinctives du gouvernement despotique}
\noindent Un grand empire suppose une autorité despotique dans celui qui gouverne. Il faut que la promptitude des résolutions supplée à la distance des lieux où elles sont envoyées ; que la crainte empêche la négligence du gouverneur ou du magistrat éloigné ; que la loi soit dans une seule tête ; et qu’elle change sans cesse, comme les accidents, qui se multiplient toujours dans l’État, à proportion de sa grandeur.
\subsubsection[{Chapitre XX. Conséquence des chapitres précédents}]{Chapitre XX. Conséquence des chapitres précédents}
\noindent Que si la propriété naturelle des petits États est d’être gouvernés en république, celle des médiocres, d’être soumis à un monarque, celle des grands empires, d’être dominés par un despote, il suit que, pour conserver les principes du gouvernement établi, il faut maintenir l’État dans la grandeur qu’il avait déjà ; et que cet État changera d’esprit, à mesure qu’on rétrécira, ou qu’on étendra ses limites.
\subsubsection[{Chapitre XXI. De l’empire de la Chine}]{Chapitre XXI. De l’empire de la Chine}
\noindent Avant de finir ce livre, je répondrai à une objection qu’on peut faire sur tout ce que j’ai dit jusqu’ici.\par
Nos missionnaires nous parlent du vaste empire de la Chine comme d’un gouvernement admirable, qui mêle ensemble dans son principe la crainte, l’honneur et la vertu. J’ai donc posé une distinction vaine, lorsque j’ai établi les principes des trois gouvernements.\par
J’ignore ce que c’est que cet honneur dont on parle chez des peuples à qui on ne fait rien faire qu’à coups de bâton\footnote{C’est le bâton qui gouverne la Chine, dit le P. Du Halde.}.\par
De plus, il s’en faut beaucoup que nos commerçants nous donnent l’idée de cette vertu dont nous parlent nos missionnaires : on peut les consulter sur les brigandages des mandarins\footnote{Voyez, entre autres, la relation de Lange.}. Je prends encore à témoin le grand homme mylord Anson.\par
D’ailleurs, les lettres du P. Parennin sur le procès que l’empereur fit faire à des princes du sang néophytes\footnote{De la famille de Sourniama, {\itshape Lettres édifiantes}, 18\textsuperscript{e} recueil.}, qui lui avaient déplu, nous font voir un plan de tyrannie constamment suivi, et des injures faites à la nature humaine avec règle, c’est-à-dire de sang-froid.\par
Nous avons encore les lettres de M. de Mairan et du même P. Parennin sur le gouvernement de la Chine. Après des questions et des réponses très sensées, le merveilleux s’est évanoui.\par
Ne pourrait-il pas se faire que les missionnaires auraient été trompés par une apparence d’ordre ; qu’ils auraient été frappés de cet exercice continuel de la volonté d’un seul, par lequel ils sont gouvernés eux-mêmes, et qu’ils aiment tant à trouver dans les cours des rois des Indes, parce que n’y allant que pour y faire de grands changements, il leur est plus aisé de convaincre les princes qu’ils peuvent tout faire que de persuader aux peuples qu’ils peuvent tout souffrir\footnote{Voyez dans le P. Du Halde comment les missionnaires se servirent de l’autorité de Canhi pour faire taire les mandarins, qui disaient toujours que, par les lois du pays, un culte étranger ne pouvait être établi dans l’empire.} ?\par
Enfin, il y a souvent quelque chose de vrai dans les erreurs mêmes. Des circonstances particulières, et peut-être uniques, peuvent faire que le gouvernement de la Chine ne soit pas aussi corrompu qu’il devrait l’être. Des causes, tirées la plupart du physique du climat, ont pu forcer les causes morales dans ce pays, et faire des espèces de prodiges.\par
Le climat de la Chine est tel qu’il favorise prodigieusement la propagation de l’espèce humaine. Les femmes y sont d’une fécondité si grande, que l’on ne voit rien de pareil sur la terre. La tyrannie la plus cruelle n’y arrête point le progrès de la propagation. Le prince n’y peut pas dire comme Pharaon : {\itshape Opprimons-les avec sagesse.} Il serait plutôt réduit à former le souhait de Néron, que le genre humain n’eût qu’une tête. Malgré la tyrannie, la Chine, par la force du climat, se peuplera toujours, et triomphera de la tyrannie.\par
La Chine, comme tous les pays ou croît le riz\footnote{Voyez ci-dessous, liv. XXIII, chap. XIV.}, est sujette à des famines fréquentes. Lorsque le peuple meurt de faim, il se disperse pour chercher de quoi vivre ; il se forme de toutes parts des bandes de trois, quatre ou cinq voleurs. La plupart sont d’abord exterminées ; d’autres se grossissent, et sont exterminées encore. Mais, dans un si grand nombre de provinces, et si éloignées, il peut arriver que quelque troupe fasse fortune. Elle se maintient, se fortifie, se forme en corps d’armée, va droit à la capitale, et le chef monte sur le trône.\par
Telle est la nature de la chose, que le mauvais gouvernement y est d’abord puni. Le désordre y naît soudain, parce que ce peuple prodigieux y manque de subsistance. Ce qui fait que, dans d’autres pays, on revient si difficilement des abus, c’est qu’ils n’y ont pas des effets sensibles ; le prince n’y est pas averti d’une manière prompte et éclatante, comme il l’est à la Chine.\par
Il ne sentira point, comme nos princes, que, s’il gouverne mal, il sera moins heureux dans l’autre vie, moins puissant et moins riche dans celle-ci. Il saura que, si son gouvernement n’est pas bon, il perdra l’empire et la vie.\par
Comme, malgré les expositions d’enfants, le peuple augmente toujours à la Chine\footnote{Voyez le mémoire d’un tsongtou, pour qu’on défriche, {\itshape Lettres édifiantes}, 21\textsuperscript{e} recueil.}, il faut un travail infatigable pour faire produire aux terres de quoi le nourrir : cela demande une grande attention de la part du gouvernement. Il est à tous les instants intéressé à ce que tout le monde puisse travailler sans crainte d’être frustré de ses peines. Ce doit moins être un gouvernement civil qu’un gouvernement domestique.\par
Voilà ce qui a produit les règlements dont on parle tant. On a voulu faire régner les lois avec le despotisme : mais ce qui est joint avec le despotisme n’a plus de force. En vain ce despotisme, pressé par ses malheurs, a-t-il voulu s’enchaîner ; il s’arme de ses chaînes, et devient plus terrible encore.\par
La Chine est donc un État despotique, dont le principe est la crainte. Peut-être que dans les premières dynasties, l’empire n’étant pas si étendu, le gouvernement déclinait un peu de cet esprit. Mais aujourd’hui cela n’est pas.
\section[{Seconde partie}]{Seconde partie}\renewcommand{\leftmark}{Seconde partie}

\subsection[{Livre neuvième. Des lois dans le rapport qu’elles ont avec la force défensive}]{Livre neuvième. Des lois dans le rapport qu’elles ont avec la force défensive}
\subsubsection[{Chapitre I. Comment les républiques pourvoient à leur sûreté}]{Chapitre I. Comment les républiques pourvoient à leur sûreté}
\noindent Si une république est petite, elle est détruite par une force étrangère ; si elle est grande, elle se détruit par un vice intérieur.\par
Ce double inconvénient infecte également les démocraties et les aristocraties, soit qu’elles soient bonnes, soit qu’elles soient mauvaises. Le mal est dans la chose même ; il n’y a aucune forme qui puisse y remédier.\par
Ainsi il y a grande apparence que les hommes auraient été à la fin obligés de vivre toujours sous le gouvernement d’un seul, s’ils n’avaient imaginé une manière de constitution qui a tous les avantages intérieurs du gouvernement républicain, et la force extérieure du monarchique. Je parle de la république fédérative.\par
Cette forme de gouvernement est une convention par laquelle plusieurs corps politiques consentent à devenir citoyens d’un État plus grand qu’ils veulent former. C’est une société de sociétés, qui en font une nouvelle, qui peut s’agrandir par de nouveaux associés qui se sont unis.\par
Ce furent ces associations qui firent fleurir si longtemps le corps de la Grèce. Par elles les Romains attaquèrent l’univers, et par elles seules l’univers se défendit contre eux ; et quand Rome fut parvenue au comble de sa grandeur, ce fut par des associations derrière le Danube et le Rhin associations que la frayeur avait fait faire, que les Barbares purent lui résister.\par
C’est par là que la Hollande\footnote{Elle est formée par environ cinquante républiques, toutes différentes les unes des autres. {\itshape État des Provinces-Unies}, par M. Janisson.}, l’Allemagne, les Ligues suisses, sont regardées en Europe comme des républiques éternelles.\par
Les associations des villes étaient autrefois plus nécessaires qu’elles ne le sont aujourd’hui. Une cité sans puissance courait de plus grands périls. La conquête lui faisait perdre, non seulement la puissance exécutrice et la législative, comme aujourd’hui, mais encore tout ce qu’il y a de propriété parmi les hommes\footnote{Liberté civile, biens, femmes, enfants, temples et sépultures même.}.\par
Cette sorte de république, capable de résister à la force extérieure, peut se maintenir dans sa grandeur sans que l’intérieur se corrompe : la forme de cette société prévient tous les inconvénients.\par
Celui qui voudrait usurper ne pourrait guère être également accrédité dans tous les États confédérés. S’il se rendait trop puissant dans l’un, il alarmerait tous les autres ; s’il subjuguait une partie, celle qui serait libre encore pourrait lui résister avec des forces indépendantes de celles qu’il aurait usurpées, et l’accabler avant qu’il eût achevé de s’établir.\par
S’il arrive quelque sédition chez un des membres confédérés, les autres peuvent l’apaiser. Si quelques abus s’introduisent quelque part, ils sont corrigés par les parties saines. Cet État peut périr d’un côté sans périr de l’autre ; la confédération peut être dissoute, et les confédérés rester souverains.\par
Composé de petites républiques, il jouit de la bonté du gouvernement intérieur de chacune ; et, à l’égard du dehors, il a, par la force de l’association, tous les avantages des grandes monarchies
\subsubsection[{Chapitre II. Que la constitution fédérative doit être composée d’états de même nature, surtout d’états républicains}]{Chapitre II. Que la constitution fédérative doit être composée d’états de même nature, surtout d’états républicains}
\noindent Les Cananéens furent détruits, parce que c’étaient de petites monarchies qui ne s’étaient point confédérées, et qui ne se défendirent pas en commun. C’est que la nature des petites monarchies n’est pas la confédération.\par
La république fédérative d’Allemagne est composée de villes libres et de petits États soumis à des princes. L’expérience fait voir qu’elle est plus imparfaite que celle de Hollande et de Suisse.\par
L’esprit de la monarchie est la guerre et l’agrandissement ; l’esprit de la république est la paix et la modération. Ces deux sortes de gouvernements ne peuvent que d’une manière forcée subsister dans une république fédérative.\par
Aussi voyons-nous dans l’histoire romaine que lorsque les Véiens eurent choisi un roi, toutes les petites républiques de Toscane les abandonnèrent. Tout fut perdu en Grèce, lorsque les rois de Macédoine obtinrent une place parmi les Amphictyons.\par
La république fédérative d’Allemagne, composée de princes et de villes libres, subsiste parce qu’elle a un chef, qui est en quelque façon le magistrat de l’union, et en quelque façon le monarque.
\subsubsection[{Chapitre III. Autres choses requises dans la république fédérative}]{Chapitre III. Autres choses requises dans la république fédérative}
\noindent Dans la république de Hollande, une province ne peut faire une alliance sans le consentement des autres. Cette loi est très bonne, et même nécessaire dans la république fédérative. Elle manque dans la constitution germanique, où elle préviendrait les malheurs qui y peuvent arriver à tous les membres, par l’imprudence, l’ambition, ou l’avarice d’un seul. Une république qui s’est unie par une confédération politique, s’est donnée entière, et n’a plus rien à donner.\par
il est difficile que les États qui s’associent soient de même grandeur, et aient une puissance égale. La république des Lyciens\footnote{Strabon, liv. XIV.} était une association de vingt-trois villes ; les grandes avaient trois voix dans le conseil commun ; les médiocres, deux ; les petites, une. La république de Hollande est composée de sept provinces, grandes ou petites, qui ont chacune une voix.\par
Les villes de Lycie\footnote{{\itshape Ibid.}} payaient les charges selon la proportion des suffrages. Les provinces de Hollande ne peuvent suivre cette proportion ; il faut qu’elles suivent celle de leur puissance.\par
En Lycie\footnote{{\itshape Ibid.}}, les juges et les magistrats des villes étaient élus par le conseil commun, et selon la proportion que nous avons dite. Dans la république de Hollande, ils ne sont point élus par le conseil commun, et chaque ville nomme ses magistrats. S’il fallait donner un modèle d’une belle république fédérative, je prendrais la république de Lycie.
\subsubsection[{Chapitre IV. Comment les états despotiques pourvoient à leur sûreté}]{Chapitre IV. Comment les états despotiques pourvoient à leur sûreté}
\noindent Comment les républiques pourvoient à leur sûreté en s’unissant, les États despotiques le font en se séparant, et en se tenant, pour ainsi dire, seuls. Ils sacrifient une partie du pays, ravagent les frontières et les rendent désertes ; le corps de l’empire devient inaccessible.\par
Il est reçu en géométrie que plus les corps ont d’étendue, plus leur circonférence est relativement petite. Cette pratique de dévaster les frontières est donc plus tolérable dans les grands États que dans les médiocres.\par
Cet État fait contre lui-même tout le mal que pourrait faire un cruel ennemi, mais un ennemi qu’on ne pourrait arrêter.\par
L’État despotique se conserve par une autre sorte de séparation, qui se fait en mettant les provinces éloignées entre les mains d’un prince qui en soit feudataire. Le Mogol, la Perse, les empereurs de la Chine ont leurs feudataires ; et les Turcs se sont très bien trouvés d’avoir mis entre leurs ennemis et eux, les Tartares, les Moldaves, les Valaques, et autrefois les Transylvains.
\subsubsection[{Chapitre V. Comment la monarchie pourvoit à sa sûreté}]{Chapitre V. Comment la monarchie pourvoit à sa sûreté}
\noindent La monarchie ne se détruit pas elle-même comme l’État despotique ; mais un État d’une grandeur médiocre pourrait être d’abord envahi. Elle a donc des places fortes qui défendent ses frontières, et des armées pour défendre ses places fortes. Le plus petit terrain s’y dispute avec art, avec courage, avec opiniâtreté. Les États despotiques font entre eux des invasions ; ils n’y a que les monarchies qui fassent la guerre.\par
Les places fortes appartiennent aux monarchies ; les États despotiques craignent d’en avoir. Ils n’osent les confier à personne ; car personne n’y aime l’État et le prince.
\subsubsection[{Chapitre VI. De la force défensive des états en général}]{Chapitre VI. De la force défensive des états en général}
\noindent Pour qu’un État soit dans sa force, il faut que sa grandeur soit telle, qu’il y ait un rapport de la vitesse avec laquelle on peut exécuter contre lui quelque entreprise, et la promptitude qu’il peut employer pour la rendre vaine. Comme celui qui attaque peut d’abord paraître partout, il faut que celui qui défend puisse se montrer partout aussi ; et par conséquent que l’étendue de l’État soit médiocre, afin qu’elle soit proportionnée au degré de vitesse que la nature a donné aux hommes pour se transporter d’un lieu à un autre.\par
La France et l’Espagne sont précisément de la grandeur requise. Les forces se communiquent si bien qu’elles se portent d’abord là où l’on veut ; les armées s’y joignent, et passent rapidement d’une frontière à l’autre ; et l’on n’y craint aucune des choses qui ont besoin d’un certain temps pour être exécutées.\par
En France, par un bonheur admirable, la capitale se trouve plus près des différentes frontières justement à proportion de leur faiblesse ; et le prince y voit mieux chaque partie de son pays, à mesure qu’elle est plus exposée.\par
Mais lorsqu’un vaste État, tel que la Perse, est attaqué, il faut plusieurs mois pour que les troupes dispersées puissent s’assembler ; et on ne force pas leur marche pendant tant de temps, comme on fait pendant quinze jours. Si l’armée qui est sur la frontière est battue, elle est sûrement dispersée, parce que ses retraites ne sont pas prochaines. L’armée victorieuse, qui ne trouve pas de résistance, s’avance à grandes journées, paraît devant la capitale et en forme le siège, lorsque à peine les gouverneurs des provinces peuvent être avertis d’envoyer du secours. Ceux qui jugent la révolution prochaine la hâtent en n’obéissant pas. Car des gens, fidèles uniquement parce que la punition est proche, ne le sont plus dès qu’elle est éloignée ; ils travaillent à leurs intérêts particuliers. L’empire se dissout, la capitale est prise, et le conquérant dispute les provinces avec les gouverneurs.\par
La vraie puissance d’un prince ne consiste pas tant dans la facilité qu’il y a à conquérir que dans la difficulté qu’il y a à l’attaquer ; et, si j’ose parler ainsi, dans l’immutabilité de sa condition. Mais l’agrandissement des États leur fait montrer de nouveaux côtés par ou on peut les prendre.\par
Ainsi, comme les monarques doivent avoir de la sagesse pour augmenter leur puissance, ils ne doivent pas avoir moins de prudence afin de la borner. En faisant cesser les inconvénients de la petitesse, il faut qu’ils aient toujours l’œil sur les inconvénients de la grandeur.
\subsubsection[{Chapitre VII. Réflexions}]{Chapitre VII. {\itshape Réflexions}}
\noindent Les ennemis d’un grand prince qui a si longtemps régné l’ont mille fois accusé, plutôt, je crois, sur leurs craintes que sur leurs raisons, d’avoir formé et conduit le projet de la monarchie universelle. S’il y avait réussi, rien n’aurait été plus fatal à l’Europe, a ses anciens sujets, à lui, à sa famille. Le ciel, qui connaît les vrais avantages, l’a mieux servi par des défaites qu’il n’aurait fait par des victoires. Au lieu de le rendre le seul roi de l’Europe, il le favorisa plus en le rendant le plus puissant de tous.\par
Sa nation qui, dans les pays étrangers, n’est jamais touchée que de ce qu’elle a quitté ; qui, en partant de chez elle, regarde la gloire comme le souverain bien, et, dans les pays éloignés, comme un obstacle à son retour ; qui indispose par ses bonnes qualités mêmes, parce qu’elle paraît y joindre du mépris ; qui peut supporter les blessures, les périls, les fatigues, et non pas la perte de ses plaisirs ; qui n’aime rien tant que sa gaieté, et se console de la perte d’une bataille lorsqu’elle a chanté le général, n’aurait jamais été jusqu’au bout d’une entreprise qui ne peut manquer dans un pays sans manquer dans tous les autres, ni manquer un moment sans manquer pour toujours.
\subsubsection[{Chapitre VIII. Cas où la force défensive d’un état est inférieure à sa force offensive}]{Chapitre VIII. Cas où la force défensive d’un état est inférieure à sa force offensive}
\noindent C’était le mot du sire de Coucy au roi Charles V, « que les Anglais ne sont jamais si faibles, ni si aisés à vaincre que chez eux ». C’est ce qu’on disait des Romains ; c’est ce qu’éprouvèrent les Carthaginois ; c’est ce qui arrivera à toute puissance qui a envoyé au loin des armées pour réunir par la force de la discipline et du pouvoir militaire ceux qui sont divisés chez eux par des intérêts politiques ou civils. L’État se trouve faible à cause du mal qui reste toujours, et il a été encore affaibli par le remède.\par
La maxime du sire de Coucy est une exception à la règle générale qui veut qu’on n’entreprenne point des guerres lointaines. Et cette exception confirme bien la règle, puisqu’elle n’a lieu que contre ceux qui ont eux-mêmes violé la règle.
\subsubsection[{Chapitre IX. De la force relative des états}]{Chapitre IX. De la force relative des états}
\noindent Toute grandeur, toute force, toute puissance est relative. Il faut bien prendre garde qu’en cherchant à augmenter la grandeur réelle, on ne diminue la grandeur relative.\par
Vers le milieu du règne de Louis XIV, la France fut au plus haut point de sa grandeur relative. L’Allemagne n’avait point encore les grands monarques qu’elle a eus depuis. L’Italie était dans le même cas. L’Écosse et l’Angleterre ne formaient point un corps de monarchie. L’Aragon n’en formait pas un avec la Castille ; les parties séparées de l’Espagne en étaient affaiblies, et l’affaiblissaient. La Moscovie n’était pas plus connue en Europe que la Crimée.
\subsubsection[{Chapitre X. De la faiblesse des états voisins}]{Chapitre X. De la faiblesse des états voisins}
\noindent Lorsqu’on a pour voisin un État qui est dans sa décadence, on doit bien se garder de hâter sa ruine, parce qu’on est, à cet égard, dans la situation la plus heureuse où l’on puisse être ; n’y ayant rien de si commode pour un prince que d’être auprès d’un autre qui reçoit pour lui tous les coups et tous les outrages de la fortune. Et il est rare que, par la conquête d’un pareil État, on augmente autant en puissance réelle qu’on a perdu en puissance relative.
\subsection[{Livre dixième. Des lois dans le rapport qu’elles ont avec la force offensive}]{Livre dixième. Des lois dans le rapport qu’elles ont avec la force offensive}
\subsubsection[{Chapitre I. De la force offensive}]{Chapitre I. De la force offensive}
\noindent La force offensive est réglée par le droit des gens, qui est la loi politique des nations considérées dans le rapport qu’elles ont les unes avec les autres.
\subsubsection[{Chapitre II. De la guerre}]{Chapitre II. {\itshape De la guerre}}
\noindent La vie des États est comme celle des hommes. Ceux-ci ont droit de tuer dans le cas de la défense naturelle ; ceux-là ont droit de faire la guerre pour leur propre conservation.\par
Dans le cas de la défense naturelle, j’ai droit de tuer, parce que ma vie est à moi, comme la vie de celui qui m’attaque est à lui : de même un État fait la guerre, parce que sa conservation est juste comme toute autre conservation.\par
Entre les citoyens, le droit de la défense naturelle n’emporte point avec lui la nécessité de l’attaque. Au lieu d’attaquer, ils n’ont qu’à recourir aux tribunaux. Ils ne peuvent donc exercer le droit de cette défense que dans les cas momentanés où l’on serait perdu si l’on attendait le secours des lois. Mais, entre les sociétés, le droit de la défense naturelle entraîne quelquefois la nécessité d’attaquer, lorsqu’un peuple voit qu’une plus longue paix en mettrait un autre en état de le détruire, et que l’attaque est dans ce moment le seul moyen d’empêcher cette destruction.\par
Il suit de là que les petites sociétés ont plus souvent le droit de faire la guerre que les grandes, parce qu’elles sont plus souvent dans le cas de craindre d’être détruites.\par
Le droit de la guerre dérive donc de la nécessité et du juste rigide. Si ceux qui dirigent la conscience ou les conseils des princes ne se tiennent pas là, tout est perdu ; et lorsqu’on se fondera sur des principes arbitraires de gloire, de bienséance, d’utilité, des flots de sang inonderont la terre.\par
Que l’on ne parle pas surtout de la gloire du prince : sa gloire serait son orgueil ; c’est une passion et non pas un droit légitime.\par
Il est vrai que la réputation de sa puissance pourrait augmenter les forces de son État ; mais la réputation de sa justice les augmenterait tout de même.
\subsubsection[{Chapitre III. Du droit de conquête}]{Chapitre III. Du droit de conquête}
\noindent Du droit de la guerre dérive celui de conquête, qui en est la conséquence ; il en doit donc suivre l’esprit.\par
Lorsqu’un peuple est conquis, le droit que le conquérant a sur lui suit quatre sortes de lois : la loi de la nature, qui fait que tout tend à la conservation des espèces ; la loi de la lumière naturelle, qui veut que nous fassions à autrui ce que nous voudrions qu’on nous fît ; la loi qui forme les sociétés politiques, qui sont telles que la nature n’en a point borné la durée ; enfin la loi tirée de la chose même. La conquête est une acquisition ; l’esprit d’acquisition porte avec lui l’esprit de conservation et d’usage, et non pas celui de destruction.\par
Un État qui en a conquis un autre le traite d’une des quatre manières suivantes : il continue à le gouverner selon ses lois, et ne prend pour lui que l’exercice du gouvernement politique et civil ; ou il lui donne un nouveau gouvernement politique et civil ; ou il détruit la société, et la disperse dans d’autres ; ou enfin il extermine tous les citoyens.\par
La première manière est conforme au droit des gens que nous suivons aujourd’hui, la quatrième est plus conforme au droit des gens des Romains : sur quoi je laisse à juger à quel point nous sommes devenus meilleurs. Il faut rendre ici hommage à nos temps modernes, à la raison présente, à la religion d’aujourd’hui, à notre philosophie, à nos mœurs.\par
Les auteurs de notre droit public, fondés sur les histoires anciennes, étant sortis des cas rigides, sont tombés dans de grandes erreurs. Ils ont donné dans l’arbitraire ; ils ont supposé dans les conquérants un droit, je ne sais quel, de tuer : ce qui leur a fait tirer des conséquences terribles comme le principe, et établir des maximes que les conquérants eux-mêmes, lorsqu’ils ont eu le moindre sens, n’ont jamais prises. Il est clair que, lorsque la conquête est faite, le conquérant n’a plus le droit de tuer, puisqu’il n’est plus dans le cas de la défense naturelle, et de sa propre conservation.\par
Ce qui les a fait penser ainsi, c’est qu’ils ont cru que le conquérant avait droit de détruire la société : d’où ils ont conclu qu’il avait celui de détruire les hommes qui la composent ; ce qui est une conséquence faussement tirée d’un faux principe. Car, de ce que la société serait anéantie, il ne s’ensuivrait pas que les hommes qui la forment dussent aussi être anéantis. La société est l’union des hommes, et non pas les hommes ; le citoyen peut périr, et l’homme rester.\par
Du droit de tuer dans la conquête, les politiques ont tiré le droit de réduire en servitude : mais la conséquence est aussi mal fondée que le principe.\par
On n’a droit de réduire en servitude que lorsqu’elle est nécessaire pour la conservation de la conquête. L’objet de la conquête est la conservation : la servitude n’est jamais l’objet de la conquête ; mais il peut arriver qu’elle soit un moyen nécessaire pour aller à la conservation.\par
Dans ce cas, il est contre la nature de la chose que cette servitude soit éternelle. Il faut que le peuple esclave puisse devenir sujet. L’esclavage dans la conquête est une chose d’accident. Lorsqu’après un certain espace de temps, toutes les parties de l’État conquérant se sont liées avec celles de l’État conquis, par des coutumes, des mariages, des lois, des associations, et une certaine conformité d’esprit, la servitude doit cesser. Car les droits du conquérant ne sont fondés que sur ce que ces choses-là ne sont pas, et qu’il y a un éloignement entre les deux nations, tel que l’une ne peut pas prendre confiance en l’autre.\par
Ainsi, le conquérant qui réduit le peuple en servitude doit toujours se réserver des moyens (et ces moyens sont sans nombre) pour l’en faire sortir.\par
Je ne dis point ici des choses vagues. Nos pères, qui conquirent l’empire romain, en agirent ainsi. Les lois qu’ils firent dans le feu, dans l’action, dans l’impétuosité, dans l’orgueil de la victoire, ils les adoucirent ; leurs lois étaient dures, ils les rendirent impartiales. Les Bourguignons, les Goths et les Lombards voulaient toujours que les Romains fussent le peuple vaincu ; les lois d’Euric, de Gondebaud et de Rotharis firent du Barbare et du Romain des concitoyens\footnote{Voyez le code des lois des Barbares, et le livre XXVIII ci-après.}.\par
Charlemagne, pour dompter les Saxons, leur ôta l’ingénuité et la propriété des biens. Louis le Débonnaire les affranchit\footnote{Voyez l’auteur incertain de la vie de Louis le Débonnaire, dans le {\itshape Recueil} de Duchesne, t. II, p. 296.} : il ne fit rien de mieux dans tout son règne. Le temps et la servitude avaient adouci leurs mœurs ; ils lui furent toujours fidèles.
\subsubsection[{Chapitre IV. Quelques avantages du peuple conquis}]{Chapitre IV. Quelques avantages du peuple conquis}
\noindent Au lieu de tirer du droit de conquête des conséquences si fatales, les politiques auraient mieux fait de parler des avantages que ce droit peut quelquefois apporter au peuple vaincu. Ils les auraient mieux sentis, si notre droit des gens était exactement suivi, et s’il était établi dans toute la terre.\par
Les États que l’on conquiert ne sont pas ordinairement dans la force de leur institution : la corruption s’y est introduite ; les lois y ont cessé d’être exécutées ; le gouvernement est devenu oppresseur. Qui peut douter qu’un État pareil ne gagnât et ne tirât quelques avantages de la conquête même, si elle n’était pas destructrice ? Un gouvernement parvenu au point où il ne peut plus se réformer lui-même, que perdrait-il à être refondu ? Un conquérant qui entre chez un peuple où, par mille ruses et mille artifices, le riche s’est insensiblement pratiqué une infinité de moyens d’usurper ; où le malheureux qui gémit, voyant ce qu’il croyait des abus devenir des lois, est dans l’oppression, et croit avoir tort de la sentir ; un conquérant, dis-je, peut dérouter tout ; et la tyrannie sourde est la première chose qui souffre la violence.\par
On a vu, par exemple, des États opprimés par les traitants, être soulagés par le conquérant, qui n’avait ni les engagements ni les besoins qu’avait le prince légitime. Les abus se trouvaient corrigés, sans même que le conquérant les corrigeât.\par
Quelquefois la frugalité de la nation conquérante l’a mise en état de laisser aux vaincus le nécessaire, qui leur était ôté sous le prince légitime.\par
Une conquête peut détruire les préjugés nuisibles, et mettre, si j’ose parler ainsi, une nation sous un meilleur génie.\par
Quel bien les Espagnols ne pouvaient-ils pas faire aux Mexicains ? Ils avaient à leur donner une religion douce ; ils leur apportèrent une superstition furieuse. Ils auraient pu rendre libres les esclaves ; et ils rendirent esclaves les hommes libres. Ils pouvaient les éclairer sur l’abus des sacrifices humains ; au lieu de cela, ils les exterminèrent. Je n’aurais jamais fini, si je voulais raconter tous les biens qu’ils ne firent pas, et tous les maux qu’ils firent.\par
C’est à un conquérant à réparer une partie des maux qu’il a faits. Je définis ainsi le droit de conquête : un droit nécessaire, légitime et malheureux, qui laisse toujours à payer une dette immense, pour s’acquitter envers la nature humaine.
\subsubsection[{Chapitre V. Gélon, roi de Syracuse}]{Chapitre V. Gélon, roi de Syracuse}
\noindent Le plus beau traité de paix dont l’histoire ait parlé est, je crois, celui que Gélon fit avec les Carthaginois. Il voulut qu’ils abolissent la coutume d’immoler leurs enfants\footnote{Voyez le {\itshape Recueil} de M. de Barbeyrac, art. 112.}. Chose admirable ! Après avoir défait trois cent mille Carthaginois, il exigeait une condition qui n’était utile qu’à eux, ou plutôt il stipulait pour le genre humain.\par
Les Bactriens faisaient manger leurs pères vieux à de grands chiens : Alexandre le leur défendit\footnote{Strabon, liv. XI.} {\itshape ;} et ce fut un triomphe qu’il remporta sur la superstition.
\subsubsection[{Chapitre VI. D’une république qui conquiert}]{Chapitre VI. D’une république qui conquiert}
\noindent Il est contre la nature de la chose que, dans une constitution fédérative, un État confédéré conquière sur l’autre, comme nous avons vu de nos jours chez les Suisses\footnote{Pour le Tockembourg.}. Dans les républiques fédératives mixtes, où l’association est entre de petites républiques et de petites monarchies, cela choque moins.\par
Il est encore contre la nature de la chose qu’une république démocratique conquière des villes qui ne sauraient entrer dans la sphère de la démocratie. Il faut que le peuple conquis puisse jouir des privilèges de la souveraineté, comme les Romains l’établirent au commencement. On doit borner la conquête au nombre des citoyens que l’on fixera pour la démocratie.\par
Si une démocratie conquiert un peuple pour le gouverner comme sujet, elle exposera sa propre liberté, parce qu’elle confiera une trop grande puissance aux magistrats qu’elle enverra dans l’État conquis.\par
Dans quel danger n’eût pas été la république de Carthage, si Annibal avait pris Rome ? Que n’eût-il pas fait dans sa ville après la victoire, lui qui y causa tant de révolutions après sa défaite\footnote{Il était à la tête d’une faction.} ?\par
Hannon n’aurait jamais pu persuader au sénat de ne point envoyer de secours à Annibal, s’il n’avait fait parler que sa jalousie. Ce sénat, qu’Aristote nous dit avoir été si sage (chose que la prospérité de cette république nous prouve si bien), ne pouvait être déterminé que par des raisons sensées. Il aurait fallu être trop stupide pour ne pas voir qu’une armée, à trois cents lieues de là, faisait des pertes nécessaires qui devaient être réparées.\par
Le parti d’Hannon voulait qu’on livrât Annibal aux Romains\footnote{Hannon voulait livrer Annibal aux Romains, comme Caton voulait qu’on livrât César aux Gaulois.}. On ne pouvait pour lors craindre les Romains ; on craignait donc Annibal.\par
On ne pouvait croire, dit-on, les succès d’Annibal ; mais comment en douter ? Les Carthaginois, répandus par toute la terre, ignoraient-ils ce qui se passait en Italie ? C’est parce qu’ils ne l’ignoraient pas, qu’on ne voulait pas envoyer de secours à Annibal.\par
Hannon devient plus ferme après Trébie, après Trasimène, après Cannes : ce n’est point son incrédulité qui augmente, c’est sa crainte.
\subsubsection[{Chapitre VII. Continuation du même sujet}]{Chapitre VII. Continuation du même sujet}
\noindent Il y a encore un inconvénient aux conquêtes faites par les démocraties. Leur gouvernement est toujours odieux aux États assujettis. Il est monarchique par la fiction : mais, dans la vérité, il est plus dur que le monarchique, comme l’expérience de tous les temps et de tous les pays l’a fait voir.\par
Les peuples conquis y sont dans un état triste ; ils ne jouissent ni des avantages de la république, ni de ceux de la monarchie.\par
Ce que j’ai dit de l’État populaire se peut appliquer à l’aristocratie.
\subsubsection[{Chapitre VIII. Continuation du même sujet}]{Chapitre VIII. Continuation du même sujet}
\noindent Ainsi, quand une république tient quelque peuple sous sa dépendance, il faut qu’elle cherche à réparer les inconvénients qui naissent de la nature de la chose, en lui donnant un bon droit politique et de bonnes lois civiles.\par
Une république d’Italie tenait des insulaires sous son obéissance ; mais son droit politique et civil à leur égard était vicieux. On se souvient de cet acte\footnote{Du 18 octobre 1738, imprimé à Gênes, chez Franchelli : {\itshape Vietamo al nostro general governatore in detta isola, di condannare in avenire solamente ex informatâ conscientiâ persona alcuna nazionale in pena afflittiva. Potrà ben si far arrestare ed incarcerare le persone che gli saranno sospette ; salvo di renderne poi a noi conto sollecitamente}, art. VI.} d’amnistie, qui porte qu’on ne les condamnerait plus à des peines afflictives {\itshape sur la conscience informée du gouverneur}. On a vu souvent des peuples demander des privilèges : ici le souverain accorde le droit de toutes les nations.
\subsubsection[{Chapitre IX. D’une monarchie qui conquiert autour d’elle}]{Chapitre IX. D’une monarchie qui conquiert autour d’elle}
\noindent Si une monarchie peut agir longtemps avant que l’agrandissement l’ait affaiblie, elle deviendra redoutable ; et sa force durera tout autant qu’elle sera pressée par les monarchies voisines.\par
Elle ne doit donc conquérir que pendant qu’elle reste dans les limites naturelles à son gouvernement. La prudence veut qu’elle s’arrête sitôt qu’elle passe ces limites.\par
Il faut, dans cette sorte de conquête, laisser les choses comme on les a trouvées : les mêmes tribunaux, les mêmes lois, les mêmes coutumes, les mêmes privilèges ; rien ne doit être change que l’armée et le nom du souverain.\par
Lorsque la monarchie a étendu ses limites par la conquête de quelques provinces voisines, il faut qu’elle les traite avec une grande douceur.\par
Dans une monarchie qui a travaillé longtemps à conquérir, les provinces de son ancien domaine seront ordinairement très foulées. Elles ont à souffrir les nouveaux abus et les anciens ; et souvent une vaste capitale, qui engloutit tout, les a dépeuplées. Or si, après avoir conquis autour de ce domaine, on traitait les peuples vaincus comme on fait ses anciens sujets, l’État serait perdu : ce que les provinces conquises enverraient de tributs à la capitale ne leur reviendrait plus ; les frontières seraient ruinées, et par conséquent plus faibles ; les peuples en seraient mal affectionnés ; la subsistance des armées, qui doivent y rester et agir, serait plus précaire.\par
Tel est l’état nécessaire d’une monarchie conquérante ; un luxe affreux dans la capitale, la misère dans les provinces qui s’en éloignent, l’abondance aux extrémités. Il en est comme de notre planète : le feu est au centre, la verdure à la surface, une terre aride, froide et stérile, entre les deux.
\subsubsection[{Chapitre X. D’une monarchie qui conquiert une autre monarchie}]{Chapitre X. D’une monarchie qui conquiert une autre monarchie}
\noindent Quelquefois une monarchie en conquiert une autre. Plus celle-ci sera petite, mieux on la contiendra par des forteresses ; plus elle sera grande, mieux on la conservera par des colonies.
\subsubsection[{Chapitre XI. Des mœurs du peuple vaincu}]{Chapitre XI. Des mœurs du peuple vaincu}
\noindent Dans ces conquêtes, il ne suffit pas de laisser à la nation vaincue ses lois ; il est peut-être plus nécessaire de lui laisser ses mœurs, parce qu’un peuple connaît, aime et défend toujours plus ses mœurs que ses lois.\par
Les Français ont été chassés neuf fois de l’Italie, à cause, disent les historiens\footnote{Parcourez {\itshape l’Histoire de l’univers}, par M. Pufendorf.}, de leur insolence à l’égard des femmes et des filles. C’est trop pour une nation d’avoir à souffrir la fierté du vainqueur, et encore son incontinence, et encore son indiscrétion, sans doute plus fâcheuse, parce qu’elle multiplie à l’infini les outrages.
\subsubsection[{Chapitre XII. D’une loi de Cyrus}]{Chapitre XII. D’une loi de Cyrus}
\noindent Je ne regarde pas comme une bonne loi celle que fit Cyrus pour que les Lydiens ne pussent exercer que des professions viles, ou des professions infâmes. On va au plus pressé ; on songe aux révoltes, et non pas aux invasions. Mais les invasions viendront bientôt ; les deux peuples s’unissent, ils se corrompent tous les deux. J’aimerais mieux maintenir par les lois la rudesse du peuple vainqueur qu’entretenir par elles la mollesse du peuple vaincu.\par
Aristodème, tyran de Cumes \footnote{Denys d’Halicarnasse, liv. VII.} chercha à énerver le courage de la jeunesse. Il voulut que les garçons laissassent croître leurs cheveux, comme les filles ; qu’ils les ornassent de fleurs, et portassent des robes de différentes couleurs jusqu’aux talons ; que, lorsqu’ils allaient chez leurs maîtres de danse et de musique, des femmes leur portassent des parasols, des parfums et des éventails ; que, dans le bain, elles leur donnassent des peignes et des miroirs. Cette éducation durait jusqu’à l’âge de vingt ans. Cela ne peut convenir qu’à un petit tyran, qui expose sa souveraineté pour défendre sa vie.
\subsubsection[{Chapitre XIII. Charles XII}]{Chapitre XIII. {\itshape Charles XII}}
\noindent Ce prince, qui ne fit usage que de ses seules forces, détermina sa chute en formant des desseins qui ne pouvaient être exécutés que par une longue guerre ; ce que son royaume ne pouvait soutenir.\par
Ce n’était pas un État qui fût dans la décadence qu’il entreprit de renverser, mais un empire naissant. Les Moscovites se servirent de la guerre qu’il leur faisait, comme d’une école. À chaque défaite ils s’approchaient de la victoire ; et, perdant au-dehors, ils apprenaient à se défendre au-dedans.\par
Charles se croyait le maître du monde dans les déserts de la Pologne, où il errait, et dans lesquels la Suède était comme répandue, pendant que son principal ennemi se fortifiait contre lui, le serrait, s’établissait sur la mer Baltique, détruisait ou prenait la Livonie.\par
La Suède ressemblait à un fleuve dont on coupait les eaux dans sa source, pendant qu’on les détournait dans son cours.\par
Ce ne fut point Pultava qui perdit Charles : s’il n’avait pas été détruit dans ce lieu, il l’aurait été dans un autre. Les accidents de la fortune se réparent aisément ; on ne peut pas parer à des événements qui naissent continuellement de la nature des choses.\par
Mais la nature ni la fortune ne furent jamais si fortes contre lui que lui-même.\par
Il ne se réglait point sur la disposition actuelle des choses, mais sur un certain modèle qu’il avait pris ; encore le suivit-il très mal. Il n’était point Alexandre ; mais il aurait été le meilleur soldat d’Alexandre.\par
Le projet d’Alexandre ne réussit que parce qu’il était sensé. Les mauvais succès des Perses dans les invasions qu’ils firent de la Grèce, les conquêtes d’Agésilas et la retraite des Dix mille avaient fait connaître au juste la supériorité des Grecs dans leur manière de combattre, et dans le genre de leurs armes ; et l’on savait bien que les Perses étaient trop grands pour se corriger.\par
Ils ne pouvaient plus affaiblir la Grèce par des divisions : elle était alors réunie sous un chef, qui ne pouvait avoir de meilleur moyen pour lui cacher sa servitude que de l’éblouir par la destruction de ses ennemis éternels et par l’espérance de la conquête de l’Asie.\par
Un empire cultivé par la nation du monde la plus industrieuse, et qui travaillait les terres par principe de religion, fertile et abondant en toutes choses, donnait à un ennemi toutes sortes de facilités pour y subsister.\par
On pouvait juger par l’orgueil de ces rois, toujours vainement mortifiés par leurs défaites, qu’ils précipiteraient leur chute en donnant toujours des batailles, et que la flatterie ne permettrait jamais qu’ils pussent douter de leur grandeur.\par
Et non seulement le projet était sage, mais il fut sagement exécuté. Alexandre, dans la rapidité de ses actions, dans le feu de ses passions mêmes, avait, si j’ose me servir de ce terme, une saillie de raison qui le conduisait, et que ceux qui ont voulu faire un roman de son histoire, et qui avaient l’esprit plus gâté que lui, n’ont pu nous dérober. Parlons-en tout à notre aise.
\subsubsection[{Chapitre XIV. Alexandre}]{Chapitre XIV. {\itshape Alexandre}}
\noindent Il ne partit qu’après avoir assuré la Macédoine contre les peuples barbares qui en étaient voisins, et achevé d’accabler les Grecs ; il ne se servit de cet accablement que pour l’exécution de son entreprise ; il rendit impuissante la jalousie des Lacédémoniens ; il attaqua les provinces maritimes ; il fit suivre à son armée de terre les côtes de la mer, pour n’être point séparé de sa flotte ; il se servit admirablement bien de la discipline contre le nombre ; il ne manqua point de subsistances ; et s’il est vrai que la victoire lui donna tout, il fit aussi tout pour se procurer la victoire.\par
Dans le commencement de son entreprise, c’est-à-dire dans un temps où un échec pouvait le renverser, il mit peu de chose au hasard ; quand la fortune le mit au-dessus des événements, la témérité fut quelquefois un de ses moyens. Lorsque avant son départ, il marche contre les Triballiens et les Illyriens, vous voyez une guerre\footnote{Voyez Arrien, {\itshape De exped. Alex.}, liv. III.} comme celle que César fit depuis dans les Gaules. Lorsqu’il est de retour dans la Grèce\footnote{{\itshape Ibid.}}, c’est comme malgré lui qu’il prend et détruit Thèbes : campé auprès de leur ville, il attend que les Thébains veuillent faire la paix ; ils précipitent eux-mêmes leur ruine. Lorsqu’il s’agit de combattre\footnote{{\itshape Ibid.}} les forces maritimes des Perses, c’est plutôt Parménion qui a de l’audace ; c’est plutôt Alexandre qui a de la sagesse. Son industrie fut de séparer les Perses des côtes de la mer, et de les réduire à abandonner eux-mêmes leur marine, dans laquelle ils étaient supérieurs. Tyr était, par principe, attachée aux Perses, qui ne pouvaient se passer de son commerce et de sa marine ; Alexandre la détruisit. Il prit l’Égypte, que Darius avait laissée dégarnie de troupes pendant qu’il assemblait des armées innombrables dans un autre univers.\par
Le passage du Granique fit qu’Alexandre se rendit maître des colonies grecques ; la bataille d’Issus lui donna Tyr et l’Égypte ; la bataille d’Arbelles lui donna toute la terre.\par
Après la bataille d’Issus, il laisse fuir Darius, et ne s’occupe qu’à affermir et à régler ses conquêtes ; après la bataille d’Arbelles, il le suit de si près\footnote{Voyez Arrien, {\itshape De exped. Alex.}, liv. I.}, qu’il ne lui laisse aucune retraite dans son empire. Darius n’entre dans ses villes et dans ses provinces que pour en sortir : les marches d’Alexandre sont si rapides, que vous croyez voir l’empire de l’univers plutôt le prix de la course, comme dans les jeux de la Grèce, que le prix de la victoire.\par
C’est ainsi qu’il fit ses conquêtes ; voyons comment il les conserva.\par
Il résista à ceux qui voulaient qu’il traitât\footnote{C’était le conseil d’Aristote. Plutarque, {\itshape Œuvres morales : De la fortune d’Alexandre}.} les Grecs comme maîtres, et les Perses comme esclaves ; il ne songea qu’à unir les deux nations, et à faire perdre les distinctions du peuple conquérant et du peuple vaincu. Il abandonna, après la conquête, tous les préjugés qui lui avaient servi à la faire. Il prit les mœurs des Perses, pour ne pas désoler les Perses en leur faisant prendre les mœurs des Grecs. C’est ce qui fit qu’il marqua tant de respect pour la femme et pour la mère de Darius, et qu’il montra tant de continence. Qu’est-ce que ce conquérant qui est pleuré de tous les peuples qu’il a soumis ? Qu’est-ce que cet usurpateur, sur la mort duquel la famille qu’il a renversée du trône verse des larmes ? C’est un trait de cette vie dont les historiens ne nous disent pas que quelque autre conquérant puisse se vanter.\par
Rien n’affermit plus une conquête que l’union qui se fait des deux peuples par les mariages. Alexandre prit des femmes de la nation qu’il avait vaincue ; il voulut que ceux de sa cour\footnote{Voyez Arrien, {\itshape De exped. Alex}., liv. VII.} en prissent aussi ; le reste des Macédoniens suivit cet exemple. Les Francs et les Bourguignons\footnote{Voyez la loi des Bourguignons, tit. XII, art. V.} permirent ces mariages ; les Wisigoths les défendirent\footnote{Voyez la loi des Wisigoths, liv. III, tit. I, § 1, qui abroge la loi ancienne, qui avait plus d’égards, y est-il dit, à la différence des nations que des conditions.} en Espagne, et ensuite ils les permirent ; les Lombards ne les permirent pas seulement, mais même les favorisèrent\footnote{Voyez la loi des Lombards, liv. II, tit. VII, § 1 et 2.}. Quand les Romains voulurent affaiblir la Macédoine, ils y établirent qu’il ne pourrait se faire d’union par mariages entre les peuples des provinces.\par
Alexandre, qui cherchait à unir les deux peuples, songea à faire dans la Perse un grand nombre de colonies grecques. Il bâtit une infinité de villes, et il cimenta si bien toutes les parties de ce nouvel empire, qu’après sa mort, dans le trouble et la confusion des plus affreuses guerres civiles, après que les Grecs se furent pour ainsi dire anéantis eux-mêmes, aucune province de Perse ne se révolta.\par
Pour ne point épuiser la Grèce et la Macédoine, il envoya à Alexandrie une colonie de Juifs\footnote{Les rois de Syrie, abandonnant le plan des fondateurs de l’empire, voulurent obliger les Juifs à prendre les mœurs des Grecs ; ce qui donna à leur État de terribles secousses.} : il ne lui importait quelles mœurs eussent ces peuples, pourvu qu’ils lui fussent fidèles.\par
Il ne laissa pas seulement aux peuples vaincus leurs mœurs, il leur laissa encore leurs lois civiles, et souvent même les rois et les gouverneurs qu’il avait trouvés. Il mettait les Macédoniens\footnote{Voyez Arrien, {\itshape De exped. Alex.}, liv. III et autres.} à la tête des troupes, et les gens du pays à la tête du gouvernement ; aimant mieux courir le risque de quelque infidélité particulière (ce qui lui arriva quelquefois) que d’une révolte générale. Il respecta les traditions anciennes et tous les monuments de la gloire ou de la vanité des peuples. Les rois de Perse avaient détruit les temples des Grecs, des Babyloniens et des Égyptiens ; il les rétablit\footnote{Voyez Arrien, {\itshape De exped. Alex}.} ; peu de nations se soumirent à lui, sur les autels desquelles il ne fît des sacrifices. Il semblait qu’il n’eût conquis que pour être le monarque particulier de chaque nation, et le premier citoyen de chaque ville. Les Romains conquirent tout pour tout détruire : il voulut tout conquérir pour tout conserver ; et quelque pays qu’il parcourût, ses premières idées, ses premiers desseins furent toujours de faire quelque chose qui pût en augmenter la prospérité et la puissance. Il en trouva les premiers moyens dans la grandeur de son génie ; les seconds dans sa frugalité et son économie particulière\footnote{{\itshape Ibid.}, liv. VII.} ; les troisièmes dans son immense prodigalité pour les grandes choses. Sa main se fermait pour les dépenses privées ; elle s’ouvrait pour les dépenses publiques. Fallait-il régler sa maison ? C’était un Macédonien. Fallait-il payer les dettes des soldats, faire part de sa conquête aux Grecs, faire la fortune de chaque homme de son armée ? Il était Alexandre.\par
Il fit deux mauvaises actions : il brûla Persépolis, et tua Clitus. Il les rendit célèbres par son repentir : de sorte qu’on oublia ses actions criminelles, pour se souvenir de son respect pour la vertu ; de sorte qu’elles furent considérées plutôt comme des malheurs que comme des choses qui lui fussent propres ; de sorte que la postérité trouve la beauté de son âme presque à côté de ses emportements et de ses faiblesses ; de sorte qu’il fallut le plaindre, et qu’il n’était plus possible de le haïr.\par
Je vais le comparer à César. Quand César voulut imiter les rois d’Asie, il désespéra les Romains pour une chose de pure ostentation ; quand Alexandre voulut imiter les rois d’Asie, il fit une chose qui entrait dans le plan de sa conquête.
\subsubsection[{Chapitre XV. Nouveaux moyens de conserver la conquête}]{Chapitre XV. Nouveaux moyens de conserver la conquête}
\noindent Lorsqu’un monarque conquiert un grand État il y a une pratique admirable, également propre à modérer le despotisme et à conserver la conquête ; les conquérants de la Chine l’ont mise en usage.\par
Pour ne point désespérer le peuple vaincu, et ne point enorgueillir le vainqueur, pour empêcher que le gouvernement ne devienne militaire, et pour contenir les deux peuples dans le devoir, la famille tartare, qui règne présentement à la Chine, a établi que chaque corps de troupes, dans les provinces, serait composé de moitié Chinois et moitié Tartares, afin que la jalousie entre les deux nations les contienne dans le devoir. Les tribunaux sont aussi moitié chinois, moitié tartares. Cela produit plusieurs bons effets : 1° les deux nations se contiennent l’une l’autre ; 2° elles gardent toutes les deux la puissance militaire et civile, et l’une n’est pas anéantie par l’autre ; 3° la nation conquérante peut se répandre partout sans s’affaiblir et se perdre ; elle devient capable de résister aux guerres civiles et étrangères. Institution si sensée, que c’est le défaut d’une pareille qui a perdu presque tous ceux qui ont conquis sur la terre.
\subsubsection[{Chapitre XVI. D’un état despotique qui conquiert}]{Chapitre XVI. D’un état despotique qui conquiert}
\noindent Lorsque la conquête est immense, elle suppose le despotisme. Pour lors, l’armée répandue dans les provinces ne suffit pas. Il faut qu’il y ait toujours autour du prince un corps particulièrement affidé, toujours prêt à fondre sur la partie de l’empire qui pourrait s’ébranler. Cette milice doit contenir les autres, et faire trembler tous ceux à qui on a été obligé de laisser quelque autorité dans l’empire. Il y a autour de l’empereur de la Chine un gros corps de Tartares toujours prêt pour le besoin. Chez le Mogol, chez les Turcs, au Japon, il y a un corps à la solde du prince, indépendamment de ce qui est entretenu du revenu des terres. Ces forces particulières tiennent en respect les générales.
\subsubsection[{Chapitre XVII. Continuation du même sujet}]{Chapitre XVII. Continuation du même sujet}
\noindent Nous avons dit que les États que le monarque despotique conquiert doivent être feudataires. Les historiens s’épuisent en éloges sur la générosité des conquérants qui ont rendu la couronne aux princes qu’ils avaient vaincus. Les Romains étaient donc bien généreux, qui faisaient partout des rois, pour avoir des instruments de servitude\footnote{{\itshape Ut haberent instrumenta servitutis et reges.}}. Une action {\itshape pareille} est un acte nécessaire. Si le conquérant garde l’État conquis, les gouverneurs qu’il {\itshape enverra} ne sauront contenir les sujets, ni lui-même ses gouverneurs. Il sera obligé de dégarnir de troupes son ancien patrimoine pour garantir le nouveau. Tous les malheurs des deux États seront communs ; la {\itshape guerre civile} de l’un sera la guerre civile de l’autre. Que si, au contraire, le conquérant rend le trône au prince légitime, il aura un allié {\itshape nécessaire} qui, avec les forces qui lui seront propres, augmentera les siennes. Nous venons de voir Schah-Nadir conquérir les trésors du Mogol, et lui laisser l’Indoustan.
\subsection[{Livre onzième. Des lois qui forment la liberté politique dans son rapport avec la constitution}]{Livre onzième. Des lois qui forment la liberté politique dans son rapport avec la constitution}
\subsubsection[{Chapitre I. Idée générale}]{Chapitre I. Idée générale}
\noindent Je distingue les lois qui forment la liberté politique dans son rapport avec la constitution, d’avec celles qui la forment dans son rapport avec le citoyen. Les premières seront le sujet de ce livre-ci ; je traiterai des secondes dans le livre suivant.
\subsubsection[{Chapitre II. Diverses significations données au mot de liberté}]{Chapitre II. Diverses significations données au mot de liberté}
\noindent Il n’y a point de mot qui ait reçu plus de différentes significations, et qui ait frappé les esprits de tant de manières, que celui de {\itshape liberté}. Les uns l’ont pris pour la facilité de déposer celui à qui ils avaient donné un pouvoir tyrannique ; les autres, pour la faculté d’élire celui à qui ils devaient obéir ; d’autres, pour le droit d’être armés, et de pouvoir exercer la violence ; ceux-ci, pour le privilège de n’être gouvernés que par un homme de leur nation, ou par leurs propres lois\footnote{« J’ai, dit Cicéron, copié l’édit de Scévola, qui permet aux Grecs de terminer entre eux leurs différends selon leurs lois, ce qui fait qu’ils se regardent comme des peuples libres. »}. Certain peuple a longtemps pris la liberté pour l’usage de porter une longue barbe\footnote{Les Moscovites ne pouvaient souffrir que le czar Pierre la leur fît couper.}. Ceux-ci ont attaché ce nom à une forme de gouvernement, et en ont exclu les autres. Ceux qui avaient goûté du gouvernement républicain l’ont mise dans ce gouvernement ; ceux qui avaient joui du gouvernement monarchique l’ont placée dans la monarchie\footnote{Les Cappadociens refusèrent l’État républicain que leur offrirent les Romains.}. Enfin chacun a appelé {\itshape liberté} le gouvernement qui était conforme à ses coutumes ou à ses inclinations ; et comme dans une république on n’a pas toujours devant les yeux, et d’une manière si présente, les instruments des maux dont on se plaint, et que même les lois paraissent y parler plus, et les exécuteurs de la loi y parler moins, on la place ordinairement dans les républiques, et on l’a exclue des monarchies. Enfin, comme, dans les démocraties, le peuple paraît à peu près faire ce qu’il veut, on a mis la liberté dans ces sortes de gouvernements, et on a confondu le pouvoir du peuple avec la liberté du peuple.
\subsubsection[{Chapitre III. Ce que c’est que la liberté}]{Chapitre III. Ce que c’est que la liberté}
\noindent Il est vrai que, dans les démocraties, le peuple paraît faire ce qu’il veut ; mais la liberté politique ne consiste point à faire ce que l’on veut. Dans un État, c’est-à-dire dans une société où il y a des lois, la liberté ne peut consister qu’à pouvoir faire ce que l’on doit vouloir, et à n’être point contraint de faire ce que l’on ne doit pas vouloir.\par
Il faut se mettre dans l’esprit ce que c’est que l’indépendance, et ce que c’est que la liberté. La liberté est le droit de faire tout ce que les lois permettent ; et si un citoyen pouvait faire ce qu’elles défendent, il n’aurait plus de liberté, parce que les autres auraient tout de même ce pouvoir.
\subsubsection[{Chapitre IV. Continuation du même sujet}]{Chapitre IV. Continuation du même sujet}
\noindent La démocratie et l’aristocratie ne sont point des États libres par leur nature. La liberté politique ne se trouve que dans les gouvernements modérés. Mais elle n’est pas toujours dans les États modérés ; elle n’y est que lorsqu’on n’abuse pas du pouvoir ; mais c’est une expérience éternelle que tout homme qui a du pouvoir est porté à en abuser ; il va jusqu’à ce qu’il trouve des limites. Qui le dirait ! la vertu même a besoin de limites.\par
Pour qu’on ne puisse abuser du pouvoir, il faut que, par la disposition des choses, le pouvoir arrête le pouvoir. Une constitution peut être telle que personne ne sera contraint de faire les choses auxquelles la loi ne l’oblige pas, et à ne point faire celles que la loi lui permet.
\subsubsection[{Chapitre V. De l’objet des états divers}]{Chapitre V. De l’objet des états divers}
\noindent Quoique tous les États aient en général un même objet, qui est de se maintenir, chaque État en a pourtant un qui lui est particulier. L’agrandissement était l’objet de Rome ; la guerre, celui de Lacédémone ; la religion, celui des lois judaïques ; le commerce, celui de Marseille ; la tranquillité publique, celui des lois de la Chine\footnote{Objet naturel d’un État qui n’a point d’ennemis au-dehors, ou qui croit les avoir arrêtés par des barrières.} ; la navigation, celui des lois des Rhodiens ; la liberté naturelle, l’objet de la police des sauvages ; en général, les délices du prince, celui des États despotiques ; sa gloire et celle de l’État, celui des monarchies ; l’indépendance de chaque particulier est l’objet des lois de Pologne ; et ce qui en résulte, l’oppression de tous\footnote{Inconvénient du {\itshape Liberum veto.}}.\par
Il y a aussi une nation dans le monde qui a pour objet direct de sa constitution la liberté politique. Nous allons examiner les principes sur lesquels elle la fonde. S’ils sont bons, la liberté y paraîtra comme dans un miroir.\par
Pour découvrir la liberté politique dans la constitution, il ne faut pas tant de peine. Si on peut la voir où elle est, si on l’a trouvée, pourquoi la chercher ?
\subsubsection[{Chapitre VI. De la constitution d’Angleterre}]{Chapitre VI. De la constitution d’Angleterre}
\noindent Il y a dans chaque État trois sortes de pouvoirs : la puissance législative, la puissance exécutrice des choses qui dépendent du droit des gens, et la puissance exécutrice de celles qui dépendent du droit civil.\par
Par la première, le prince ou le magistrat fait des lois pour un temps ou pour toujours, et corrige ou abroge celles qui sont faites. Par la seconde, il fait la paix ou la guerre, envoie ou reçoit des ambassades, établit la sûreté, prévient les invasions. Par la troisième, il punit les crimes, ou juge les différends des particuliers. On appellera cette dernière la puissance de juger, et l’autre simplement la puissance exécutrice de l’État.\par
La liberté politique dans un citoyen est cette tranquillité d’esprit qui provient de l’opinion que chacun a de sa sûreté ; et pour qu’on ait cette liberté, il faut que le gouvernement soit tel qu’un citoyen ne puisse pas craindre un autre citoyen.\par
Lorsque, dans la même personne ou dans le même corps de magistrature, la puissance législative est réunie à la puissance exécutrice, il n’y a point de liberté ; parce qu’on peut craindre que le même monarque ou le même sénat ne fasse des lois tyranniques pour les exécuter tyranniquement.\par
Il n’y a point encore de liberté si la puissance de juger n’est pas séparée de la puissance législative et de l’exécutrice. Si elle était jointe à la puissance législative, le pouvoir sur la vie et la liberté des citoyens serait arbitraire : car le juge serait législateur. Si elle était jointe à la puissance exécutrice, le juge pourrait avoir la force d’un oppresseur.\par
Tout serait perdu, si le même homme, ou le même corps des principaux, ou des nobles, ou du peuple, exerçaient ces trois pouvoirs : celui de faire des lois, celui d’exécuter les résolutions publiques, et celui de juger les crimes ou les différends des particuliers.\par
Dans la plupart des royaumes de l’Europe, le gouvernement est modéré, parce que le prince, qui a les deux premiers pouvoirs, laisse à ses sujets l’exercice du troisième. Chez les Turcs, où ces trois pouvoirs sont réunis sur la tête du sultan, il règne un affreux despotisme.\par
Dans les républiques d’Italie, où ces trois pouvoirs sont réunis, la liberté se trouve moins que dans nos monarchies. Aussi le gouvernement a-t-il besoin, pour se maintenir, de moyens aussi violents que le gouvernement des Turcs ; témoins les inquisiteurs d’État\footnote{À Venise.}, et le tronc où tout délateur peut, à tous les moments, jeter avec un billet son accusation.\par
Voyez quelle peut être la situation d’un citoyen dans ces républiques. Le même corps de magistrature a, comme exécuteur des lois, toute la puissance qu’il s’est donnée comme législateur. Il peut ravager l’État par ses volontés générales, et, comme il a encore la puissance de juger, il peut détruire chaque citoyen par ses volontés particulières.\par
Toute la puissance y est une ; et, quoiqu’il n’y ait point de pompe extérieure qui découvre un prince despotique, on le sent à chaque instant.\par
Aussi les princes qui ont voulu se rendre despotiques ont-ils toujours commencé par réunir en leur personne toutes les magistratures ; et plusieurs rois d’Europe, toutes les grandes charges de leur État.\par
Je crois bien que la pure aristocratie héréditaire des républiques d’Italie ne répond pas précisément au despotisme de l’Asie. La multitude des magistrats adoucit quelquefois la magistrature ; tous les nobles ne concourent pas toujours aux mêmes desseins ; on y forme divers tribunaux qui se tempèrent. Ainsi, à Venise, le {\itshape grand conseil} a la législation ; le {\itshape prégady}, l’exécution ; les {\itshape quaranties}, le pouvoir de juger. Mais le mal est que ces tribunaux différents sont formés par des magistrats du même corps ; ce qui ne fait guère qu’une même puissance.\par
La puissance de juger ne doit pas être donnée à un sénat permanent, mais exercée par des personnes tirées du corps du peuple\footnote{Comme à Athènes.} dans certains temps de l’année, de la manière prescrite par la loi, pour former un tribunal qui ne dure qu’autant que la nécessité le requiert.\par
De cette façon, la puissance de juger, si terrible parmi les hommes, n’étant attachée ni à un certain état, ni à une certaine profession, devient, pour ainsi dire, invisible et nulle. On n’a point continuellement des juges devant les yeux ; et l’on craint la magistrature, et non pas les magistrats.\par
Il faut même que, dans les grandes accusations, le criminel, concurremment avec la loi, se choisisse des juges ; ou du moins qu’il en puisse récuser un si grand nombre, que ceux qui restent soient censés être de son choix.\par
Les deux autres pouvoirs pourraient plutôt être donnés à des magistrats ou à des corps permanents, parce qu’ils ne s’exercent sur aucun particulier ; n’étant, l’un, que la volonté générale de l’État, et l’autre, que l’exécution de cette volonté générale.\par
Mais, si les tribunaux ne doivent pas être fixes, les jugements doivent l’être à un tel point, qu’ils ne soient jamais qu’un texte précis de la loi. S’ils étaient une opinion particulière du juge, on vivrait dans la société, sans savoir précisément les engagements que l’on y contracte.\par
Il faut même que les juges soient de la condition de l’accusé, ou ses pairs, pour qu’il ne puisse pas se mettre dans l’esprit qu’il soit tombé entre les mains de gens portés à lui faire violence.\par
Si la puissance législative laisse à l’exécutrice le droit d’emprisonner des citoyens qui peuvent donner caution de leur conduite, il n’y a plus de liberté, à moins qu’ils ne soient arrêtés pour répondre, sans délai, à une accusation que la loi a rendue capitale ; auquel cas ils sont réellement libres, puisqu’ils ne sont soumis qu’à la puissance de la loi.\par
Mais, si la puissance législative se croyait en danger par quelque conjuration secrète contre l’État, ou quelque intelligence avec les ennemis du dehors, elle pourrait, pour un temps court et limité, permettre à la puissance exécutrice de faire arrêter les citoyens suspects, qui ne perdraient leur liberté pour un temps que pour la conserver pour toujours.\par
Et c’est le seul moyen conforme à la raison de suppléer à la tyrannique magistrature des {\itshape éphores} et aux {\itshape inquisiteurs d’État} de Venise, qui sont aussi despotiques.\par
Comme, dans un État libre, tout homme qui est censé avoir une âme libre doit être gouverné par lui-même, il faudrait que le peuple en corps eût la puissance législative. Mais comme cela est impossible dans les grands États, et est sujet à beaucoup d’inconvénients dans les petits, il faut que le peuple fasse par ses représentants tout ce qu’il ne peut faire par lui-même.\par
L’on connaît beaucoup mieux les besoins de sa ville que ceux des autres villes ; et on juge mieux de la capacité de ses voisins que de celle de ses autres compatriotes. Il ne faut donc pas que les membres du corps législatif soient tirés en général du corps de la nation ; mais il convient que, dans chaque lieu principal, les habitants se choisissent un représentant.\par
Le grand avantage des représentants, c’est qu’ils sont capables de discuter les affaires. Le peuple n’y est point du tout propre ; ce qui forme un des grands inconvénients de la démocratie.\par
Il n’est pas nécessaire que les représentants, qui ont reçu de ceux qui les ont choisis une instruction générale, en reçoivent une particulière sur chaque affaire, comme cela se pratique dans les diètes d’Allemagne. Il est vrai que, de cette manière, la parole des députés serait plus l’expression de la voix de la nation ; mais cela jetterait dans des longueurs infinies, rendrait chaque député le maître de tous les autres, et, dans les occasions les plus pressantes, toute la force de la nation pourrait être arrêtée par un caprice.\par
Quand les députés, dit très bien M. Sidney, représentent un corps de peuple, comme en Hollande, ils doivent rendre compte à ceux qui les ont commis ; c’est autre chose lorsqu’ils sont députés par des bourgs, comme en Angleterre.\par
Tous les citoyens, dans les divers districts, doivent avoir droit de donner leur voix pour choisir le représentant ; excepté ceux qui sont dans un tel état de bassesse, qu’ils sont réputés n’avoir point de volonté propre.\par
Il y avait un grand vice dans la plupart des anciennes républiques : c’est que le peuple avait droit d’y prendre des résolutions actives, et qui demandent quelque exécution, chose dont il est entièrement incapable. Il ne doit entrer dans le gouvernement que pour choisir ses représentants, ce qui est très à sa poilée. Car, s’il y a peu de gens qui connaissent le degré précis de la capacité des hommes, chacun est pourtant capable de savoir, en général, si celui qu’il choisit est plus éclairé que la plupart des autres.\par
Le corps représentant ne doit pas être choisi non plus pour prendre quelque résolution active, chose qu’il ne ferait pas bien ; mais pour faire des lois, ou pour voir si l’on a bien exécuté celles qu’il a faites, chose qu’il peut très bien faire, et qu’il n’y a même que lui qui puisse bien faire.\par
Il y a toujours dans un État des gens distingués par la naissance les richesses ou les honneurs ; mais s’ils étaient confondus parmi le peuple, et s’ils n’y avaient qu’une voix comme les autres, la liberté commune serait leur esclavage, et ils n’auraient aucun intérêt à la défendre, parce que la plupart des résolutions {\itshape seraient contre eux}. La part qu’ils {\itshape ont} à la législation doit donc être proportionnée aux autres avantages qu’ils ont dans l’État : ce qui arrivera s’ils forment un corps qui ait droit d’arrêter les entreprises du peuple, comme le peuple a droit d’arrêter les leurs.\par
Ainsi, la puissance législative sera confiée, et au corps des nobles, et au corps qui sera choisi pour représenter le peuple, qui auront chacun leurs assemblées et leurs délibérations à part, et des vues et des intérêts séparés.\par
Des trois puissances dont nous avons parlé, celle de juger est en quelque façon nulle. Il n’en reste que deux ; et comme elles ont besoin d’une puissance réglante pour les tempérer, la partie du corps législatif qui est composée de nobles est très propre à produire cet effet.\par
Le corps des nobles doit être héréditaire. Il l’est premièrement par sa nature ; et d’ailleurs il faut qu’il ait un très grand intérêt à conserver ses prérogatives, odieuses par elles-mêmes, et qui, dans un État libre, doivent toujours être en danger.\par
Mais, comme une puissance héréditaire pourrait être induite à suivre ses intérêts particuliers et à oublier ceux du peuple, il faut que dans les choses où l’on a un souverain intérêt à la corrompre, comme dans les lois qui concernent la levée de l’argent, elle n’ait de part à la législation que par sa faculté d’empêcher, et non par sa faculté de statuer.\par
J’appelle {\itshape faculté de statuer}, le droit d’ordonner par soi-même, ou de corriger ce qui a été ordonné par un autre. J’appelle {\itshape faculté d’empêcher}, le droit de rendre nulle une résolution prise par quelque autre ; ce qui était la puissance des tribuns de Rome. Et, quoique celui qui a la faculté d’empêcher puisse avoir aussi le droit d’approuver, pour lors cette approbation n’est autre chose qu’une déclaration qu’il ne fait point d’usage de sa faculté d’empêcher, et dérive de cette faculté.\par
La puissance exécutrice doit être entre les mains d’un monarque, parce que cette partie du gouvernement, qui a presque toujours besoin d’une action momentanée, est mieux administrée par un que par plusieurs ; au lieu que ce qui dépend de la puissance législative est souvent mieux ordonné par plusieurs que par un seul.\par
Que s’il n’y avait point de monarque, et que la puissance exécutrice fût confiée à un certain nombre de personnes tirées du corps législatif, il n’y aurait plus de liberté, parce que les deux puissances seraient unies ; les mêmes personnes ayant quelquefois, et pouvant toujours avoir part à l’une et à l’autre.\par
Si le corps législatif était un temps considérable sans être assemblé, il n’y aurait plus de liberté. Car il arriverait de deux choses l’une : ou qu’il n’y aurait plus de résolution législative, et l’État tomberait dans l’anarchie ; ou que ces résolutions seraient prises par la puissance exécutrice, et elle deviendrait absolue.\par
Il serait inutile que le corps législatif fût toujours assemblé. Cela serait incommode pour les représentants, et d’ailleurs occuperait trop la puissance exécutrice, qui ne penserait point à exécuter, mais à défendre ses prérogatives, et le droit qu’elle a d’exécuter.\par
De plus : si le corps législatif était continuellement assemblé, il pourrait arriver que l’on ne ferait que suppléer de nouveaux députés à la place de ceux qui mourraient ; et, dans ce cas, si le corps législatif était une fois corrompu, le mal serait sans remède. Lorsque divers corps législatifs se succèdent les uns aux autres, le peuple, qui a mauvaise opinion du corps législatif actuel, porte, avec raison, ses espérances sur celui qui viendra après. Mais si c’était toujours le même corps, le peuple, le voyant une fois corrompu, n’espérerait plus rien de ses lois ; il deviendrait furieux, ou tomberait dans l’indolence.\par
Le corps législatif ne doit point s’assembler lui-même ; car un corps n’est censé avoir de volonté que lorsqu’il est assemblé ; et, s’il ne s’assemblait pas unanimement, on ne saurait dire quelle partie serait véritablement le corps législatif : celle qui serait assemblée, ou celle qui ne le serait pas. Que s’il avait droit de se proroger lui-même, il pourrait arriver qu’il ne se prorogerait jamais ; ce qui serait dangereux dans le cas où il voudrait attenter contre la puissance exécutrice. D’ailleurs, il y a des temps plus convenables les uns que les autres pour l’assemblée du corps législatif : il faut donc que ce soit la puissance exécutrice qui règle le temps de la tenue et de la durée de ces assemblées, par rapport aux circonstances qu’elle connaît.\par
Si la puissance exécutrice n’a pas le droit d’arrêter les entreprises du corps législatif, celui-ci sera despotique ; car, comme il pourra se donner tout le pouvoir qu’il peut imaginer, il anéantira toutes les autres puissances.\par
Mais il ne faut pas que la puissance législative ait réciproquement la faculté d’arrêter la puissance exécutrice. Car, l’exécution ayant ses limites par sa nature, il est inutile de la borner ; outre que la puissance exécutrice s’exerce toujours sur des choses momentanées. Et la puissance des tribuns de Rome était vicieuse, en ce qu’elle arrêtait non seulement la législation, mais même l’exécution : ce qui causait de grands maux.\par
Mais si, dans un État libre, la puissance législative ne doit pas avoir le droit d’arrêter la puissance exécutrice, elle a droit, et doit avoir la faculté d’examiner de quelle manière les lois qu’elle a faites ont été exécutées ; et c’est l’avantage qu’a ce gouvernement sur celui de Crète et de Lacédémone, où les {\itshape cosmes} et les {\itshape éphores} ne rendaient point compte de leur administration.\par
Mais, quel que soit cet examen, le corps législatif ne doit point avoir le pouvoir de juger la personne, et par conséquent la conduite de celui qui exécute. Sa personne doit être sacrée, parce qu’étant nécessaire à l’État pour que le corps législatif n’y devienne pas tyrannique, dès le moment qu’il serait accusé ou jugé, il n’y aurait plus de liberté.\par
Dans ce cas, l’État ne serait point une monarchie, mais une république non libre. Mais, comme celui qui exécute ne peut exécuter mal sans avoir des conseillers méchants et qui haïssent les lois comme ministres, quoiqu’elles les favorisent comme hommes, ceux-ci peuvent être recherchés et punis. Et c’est l’avantage de ce gouvernement sur celui de Gnide, où la loi ne permettant point d’appeler en jugement les {\itshape amimones}\footnote{C’étaient des magistrats que le peuple élisait tous les ans. Voyez Étienne de Byzance.}, même après leur administration\footnote{On pouvait accuser les magistrats romains après leur magistrature. Voyez, dans Denys d’Halicarnasse, liv. IX, l’affaire du tribun Génutius.}, le peuple ne pouvait jamais se faire rendre raison des injustices qu’on lui avait faites.\par
Quoiqu’en général la puissance de juger ne doive être unie à aucune partie de la législative, cela est sujet à trois exceptions, fondées sur l’intérêt particulier de celui qui doit être jugé.\par
Les grands sont toujours exposés à l’envie ; et s’ils étaient jugés par le peuple, ils pourraient être en danger, et ne jouiraient pas du privilège qu’a le moindre des citoyens, dans un État libre, d’être jugé par ses pairs. Il faut donc que les nobles soient appelés, non pas devant les tribunaux ordinaires de la nation, mais devant cette partie du corps législatif qui est composée de nobles.\par
Il pourrait arriver que la loi, qui est en même temps clairvoyante et aveugle, serait, en de certains cas, trop rigoureuse. Mais les juges de la nation ne sont, comme nous avons dit, que la bouche qui prononce les paroles de la loi ; des êtres inanimés qui n’en peuvent modérer ni la force ni la rigueur. C’est donc la partie du corps législatif, que nous venons de dire être, dans une autre occasion, un tribunal nécessaire, qui l’est encore dans celle-ci ; c’est à son autorité suprême à modérer la loi en faveur de la loi même, en prononçant moins rigoureusement qu’elle.\par
Il pourrait encore arriver que quelque citoyen, dans les affaires publiques, violerait les droits du peuple, et ferait des crimes que les magistrats établis ne sauraient ou ne voudraient pas punir. Mais, en général, la puissance législative ne peut pas juger ; et elle le peut encore moins dans ce cas particulier, où elle représente la partie intéressée, qui est le peuple. Elle ne peut donc être qu’accusatrice. Mais devant qui accusera-t-elle ? Ira-t-elle s’abaisser devant les tribunaux de la loi, qui lui sont inférieurs, et d’ailleurs composés de gens qui, étant peuple comme elle, seraient entraînés par l’autorité d’un si grand accusateur ? Non : il faut, pour conserver la dignité du peuple et la sûreté du particulier, que la partie législative du peuple accuse devant la partie législative des nobles, laquelle n’a ni les mêmes intérêts qu’elle, ni les mêmes passions.\par
C’est l’avantage qu’a ce gouvernement sur la plupart des républiques anciennes, où il y avait cet abus, que le peuple était en même temps et juge et accusateur.\par
La puissance exécutrice, comme nous avons dit, doit prendre part à la législation par sa faculté d’empêcher ; sans quoi elle sera bientôt dépouillée de ses prérogatives. Mais si la puissance législative prend part à l’exécution, la puissance exécutrice sera également perdue.\par
Si le monarque prenait part à la législation par la faculté de statuer, il n’y aurait plus de liberté. Mais, comme il faut pourtant qu’il ait part à la législation pour se défendre, il faut qu’il y prenne part par la faculté d’empêcher.\par
Ce qui fut cause que le gouvernement changea à Rome, c’est que le Sénat, qui avait une partie de la puissance exécutrice, et les magistrats, qui avaient l’autre, n’avaient pas, comme le peuple, la faculté d’empêcher.\par
Voici donc la constitution fondamentale du gouvernement dont nous parlons. Le corps législatif y étant composé de deux parties, l’une enchaînera l’autre par sa faculté mutuelle d’empêcher. Toutes les deux seront liées par la puissance exécutrice, qui le sera elle-même par la législative.\par
Ces trois puissances devraient former un repos ou une inaction. Mais comme, par le mouvement nécessaire des choses, elles sont contraintes d’aller, elles seront forcées d’aller de concert.\par
La puissance exécutrice ne faisant partie de la législative que par sa faculté d’empêcher, elle ne saurait entrer dans le débat des affaires. Il n’est pas même nécessaire qu’elle propose, parce que, pouvant toujours désapprouver les résolutions, elle peut rejeter les décisions des propositions qu’elle aurait voulu qu’on n’eût pas faites.\par
Dans quelques républiques anciennes, où le peuple en corps avait le débat des affaires, il était naturel que la puissance exécutrice les proposât et les débattît avec lui ; sans quoi il y aurait eu dans les résolutions une confusion étrange.\par
Si la puissance exécutrice statue sur la levée des deniers publics autrement que par son consentement, il n’y aura plus de liberté, parce qu’elle deviendra législative dans le point le plus important de la législation.\par
Si la puissance législative statue, non pas d’année en année, mais pour toujours, sur la levée des deniers publics, elle court risque de perdre sa liberté, parce que la puissance exécutrice ne dépendra plus d’elle ; et quand on tient un pareil droit pour toujours, il est assez indifférent qu’on le tienne de soi ou d’un autre. Il en est de même si elle statue, non pas d’année en année, mais pour toujours, sur les forces de terre et de mer qu’elle doit confier à la puissance exécutrice.\par
Pour que celui qui exécute ne puisse pas opprimer, il faut que les armées qu’on lui confie soient peuple, et aient le même esprit que le peuple, comme cela fut à Rome jusqu’au temps de Marius. Et, pour que cela soit ainsi, il n’y a que deux moyens : ou que ceux que l’on emploie dans l’armée aient assez de bien pour répondre de leur conduite aux autres citoyens, et qu’ils ne soient enrôlés que pour un an, comme il se pratiquait à Rome ; ou, si on a un corps de troupes permanent, et où les soldats soient une des plus viles parties de la nation, il faut que la puissance législative puisse le casser sitôt qu’elle le désire ; que les soldats habitent avec les citoyens, et qu’il n’y ait ni camp séparé, ni casernes, ni place de guerre.\par
L’armée étant une fois établie, elle ne doit point dépendre immédiatement du corps législatif, mais de la puissance exécutrice ; et cela par la nature de la chose, son fait consistant plus en action qu’en délibération.\par
Il est dans la manière de penser des hommes que l’on fasse plus de cas du courage que de la timidité ; de l’activité que de la prudence ; de la force que des conseils. L’armée méprisera toujours un sénat et respectera ses officiers. Elle ne fera point cas des ordres qui lui seront envoyés de la part d’un corps composé de gens qu’elle croira timides, et indignes par là de lui commander. Ainsi, sitôt que l’armée dépendra uniquement du corps législatif, le gouvernement deviendra militaire. Et si le contraire est jamais arrivé, c’est l’effet de quelques circonstances extraordinaires ; c’est que l’armée y est toujours séparée ; c’est qu’elle est composée de plusieurs corps qui dépendent chacun de leur province particulière ; c’est que les villes capitales sont des places excellentes, qui se défendent par leur situation seule, et où il n’y a point de troupes.\par
La Hollande est encore plus en sûreté que Venise ; elle submergerait les troupes révoltées, elle les ferait mourir de faim. Elles ne sont point dans les villes qui pourraient leur donner la subsistance ; cette subsistance est donc précaire.\par
Que si, dans le cas où l’année est gouvernée par le corps législatif, des circonstances particulières empêchent le gouvernement de devenir militaire, on tombera dans d’autres inconvénients ; de deux choses l’une : ou il faudra que l’armée détruise le gouvernement, ou que le gouvernement affaiblisse l’armée.\par
Et cet affaiblissement aura une cause bien fatale : il naîtra de la faiblesse même du gouvernement.\par
Si l’on veut lire l’admirable ouvrage de Tacite {\itshape Sur les mœurs des Germains}\footnote{{\itshape De minoribus rebus principes consultant, de majoribus omnes ; ita tamen ut ea quoque quorum penes plebem arbitrium est apud principes pertractentur.}}, on verra que c’est d’eux que les Anglais ont tiré l’idée de leur gouvernement politique. Ce beau système a été trouvé dans les bois.\par
Comme toutes les choses humaines ont une fin, l’État dont nous parlons perdra sa liberté, il périra. Rome, Lacédémone et Carthage ont bien péri. Il périra lorsque la puissance législative sera plus corrompue que l’exécutrice.\par
Ce n’est point à moi à examiner si les Anglais jouissent actuellement de cette liberté, ou non. Il me suffit de dire qu’elle est établie par leurs lois, et je n’en cherche pas davantage.\par
Je ne prétends point par là ravaler les autres gouvernements, ni dire que cette liberté politique extrême doive mortifier ceux qui n’en ont qu’une modérée. Comment dirais-je cela, moi qui crois que l’excès même de la raison n’est pas toujours désirable, et que les hommes s’accommodent presque toujours mieux des milieux que des extrémités ?\par
Harrington, dans son {\itshape Oceana}, a aussi examiné quel était le plus haut point de liberté où la constitution d’un État peut être portée. Mais on peut dire de lui qu’il n’a cherché cette liberté qu’après l’avoir méconnue, et qu’il a bâti Chalcédoine, ayant le rivage de Byzance devant les yeux.
\subsubsection[{Chapitre VII. Des monarchies que nous connaissons}]{Chapitre VII. Des monarchies que nous connaissons}
\noindent Les monarchies que nous connaissons n’ont pas, comme celle dont nous venons de parler, la liberté pour leur objet direct ; elles ne tendent qu’à la gloire des citoyens, de l’État et du prince. Mais de cette gloire il résulte un esprit de liberté qui, dans ces États, peut faire d’aussi grandes choses, et peut-être contribuer autant au bonheur que la liberté même.\par
Les trois pouvoirs n’y sont point distribués et fondus sur le modèle de la constitution dont nous avons parlé. Ils ont chacun une distribution particulière, selon laquelle ils approchent plus ou moins de la liberté politique ; et, s’ils n’en approchaient pas, la monarchie dégénérerait en despotisme.
\subsubsection[{Chapitre VIII. Pourquoi les anciens n’avaient pas une idée bien claire de la monarchie}]{Chapitre VIII. Pourquoi les anciens n’avaient pas une idée bien claire de la monarchie}
\noindent Les anciens ne connaissaient point le gouvernement fondé sur un corps de noblesse, et encore moins le gouvernement fondé sur un corps législatif formé par les représentants d’une nation. Les républiques de Grèce et d’Italie étaient des villes qui avaient chacune leur gouvernement, et qui assemblaient leurs citoyens dans leurs murailles. Avant que les Romains eussent englouti toutes les républiques, il n’y avait presque point de roi nulle part, en Italie, Gaule, Espagne, Allemagne ; tout cela était de petits peuples ou de petites républiques. L’Afrique même était soumise à une grande ; l’Asie Mineure était occupée par les colonies grecques. Il n’y avait donc point d’exemple de députés de villes, ni d’assemblées d’États ; il fallait aller jusqu’en Perse pour trouver le gouvernement d’un seul.\par
Il est vrai qu’il y avait des républiques fédératives ; plusieurs villes envoyaient des députés à une assemblée. Mais je dis qu’il n’y avait point de monarchie sur ce modèle-là.\par
Voici comment se forma le premier plan des monarchies que nous connaissons. Les nations germaniques qui conquirent l’empire romain étaient, comme l’on sait, très libres. On n’a qu’à voir là-dessus Tacite sur {\itshape Les Mœurs des Germains}. Les conquérants se répandirent dans le pays ; ils habitaient les campagnes, et peu les villes. Quand ils étaient en Germanie, toute la nation pouvait s’assembler. Lorsqu’ils furent dispersés dans la conquête, ils ne le purent plus. Il fallait pourtant que la nation délibérât sur ses affaires, comme elle avait fait avant la conquête : elle le fit par des représentants. Voilà l’origine du gouvernement gothique parmi nous. Il fut d’abord mêlé de l’aristocratie et de la monarchie. Il avait cet inconvénient que le bas peuple y était esclave. C’était un bon gouvernement qui avait en soi la capacité de devenir meilleur. La coutume vint d’accorder des lettres d’affranchissement ; et bientôt la liberté civile du peuple, les prérogatives de la noblesse et du clergé, la puissance des rois, se trouvèrent dans un tel concert, que je ne crois pas qu’il y ait eu sur la terre de gouvernement si bien tempéré que le fut celui de chaque partie de l’Europe dans le temps qu’il y subsista. Et il est admirable que la corruption du gouvernement d’un peuple conquérant ait formé la meilleure espèce de gouvernement que les hommes aient pu imaginer.
\subsubsection[{Chapitre IX. Manière de penser d’Aristote}]{Chapitre IX. Manière de penser d’Aristote}
\noindent L’embarras d’Aristote paraît visiblement quand il traite de la monarchie\footnote{Politique, liv. III, chap. XIV.}. Il en établit cinq espèces : il ne les distingue pas par la forme de la constitution, mais par des choses d’accident, comme les vertus ou les vices du prince ; ou par des choses étrangères, comme l’usurpation de la tyrannie, ou la succession à la tyrannie.\par
Aristote met au rang des monarchies et l’empire des Perses et le royaume de Lacédémone. Mais qui ne voit que l’un était un État despotique, et l’autre, une république ?\par
Les anciens, qui ne connaissaient pas la distribution des trois pouvoirs dans le gouvernement d’un seul, ne pouvaient se faire une idée juste de la monarchie.
\subsubsection[{Chapitre X. Manière de penser des autres politiques}]{Chapitre X. Manière de penser des autres politiques}
\noindent Pour tempérer le gouvernement d’un seul, Arribas\footnote{Voyez Justin, liv. XVII.}, roi d’Épire, n’imagina qu’une république. Les Molosses ne sachant comment borner le même pouvoir, firent deux rois\footnote{Aristote, {\itshape Politique}, liv. V, chap. IX.} : par là on affaiblissait l’État plus que le commandement ; on voulait des rivaux, et on avait des ennemis.\par
Deux rois n’étaient tolérables qu’à Lacédémone ; ils n’y formaient pas la constitution, mais ils étaient une partie de la constitution.
\subsubsection[{Chapitre XI. Des rois des temps héroïques chez les grecs}]{Chapitre XI. Des rois des temps héroïques chez les grecs}
\noindent Chez les Grecs, dans les temps héroïques, il s’établit une espèce de monarchie qui ne subsista pas\footnote{Aristote, {\itshape Politique}, liv. III, chap. XIV.}. Ceux qui avaient inventé des arts, fait la guerre pour le peuple, assemblé des hommes dispersés, ou qui leur avaient donné des terres, obtenaient le royaume pour eux, et le transmettaient à leurs enfants. Ils étaient rois, prêtres et juges. C’est une des cinq espèces de monarchie dont nous parle Aristote\footnote{{\itshape ibid.}} {\itshape ;} et c’est la seule qui puisse réveiller l’idée de la constitution monarchique. Mais le plan de cette constitution est opposé à celui de nos monarchies d’aujourd’hui.\par
Les trois pouvoirs y étaient distribués de manière que le peuple y avait la puissance législative\footnote{Voyez ce que dit Plutarque, {\itshape Vie de Thésée}. Voyez aussi Thucydide, liv. I.} ; et le roi, la puissance exécutrice avec la puissance de juger, au lieu que, dans les monarchies que nous connaissons, le prince a la puissance exécutrice et la législative, ou du moins une partie de la législative, mais il ne juge pas.\par
Dans le gouvernement des rois des temps héroïques, les trois pouvoirs étaient mal distribués. Ces monarchies ne pouvaient subsister, car, dès que le peuple avait la législation, il pouvait, au moindre caprice, anéantir la royauté, comme il fit partout.\par
Chez un peuple libre, et qui avait le pouvoir législatif ; chez un peuple renfermé dans une ville, où tout ce qu’il y a d’odieux devient plus odieux encore, le chef-d’œuvre de la législation est de savoir bien placer la puissance de juger. Mais elle ne le pouvait être plus mal que dans les mains de celui qui avait déjà la puissance exécutrice. Dès ce moment, le monarque devenait terrible. Mais en même temps, comme il n’avait pas la législation, il ne pouvait pas se défendre contre la législation ; il avait trop de pouvoir, et il n’en avait pas assez.\par
On n’avait pas encore découvert que la vraie fonction du prince était d’établir des juges, et non pas de juger lui-même. La politique contraire rendit le gouvernement d’un seul insupportable. Tous ces rois furent chassés. Les Grecs n’imaginèrent point la vraie distribution des trois pouvoirs dans le gouvernement d’un seul ; ils ne l’imaginèrent que dans le gouvernement de plusieurs, et ils appelèrent cette sorte de constitution, {\itshape police}\footnote{Voyez Aristote, {\itshape Politique}, liv. IV, chap. VIII.}.
\subsubsection[{Chapitre XII. Du gouvernement des rois de Rome et comment les trois pouvoirs y furent distribués}]{Chapitre XII. Du gouvernement des rois de Rome et comment les trois pouvoirs y furent distribués}
\noindent Le gouvernement des rois de Rome avait quelque rapport à celui des rois des temps héroïques chez les Grecs. Il tomba, comme les autres, par son vice général ; quoiqu’en lui-même, et dans sa nature particulière, il fût très bon.\par
Pour faire connaître ce gouvernement, je distinguerai celui des cinq premiers rois, celui de Servius Tullius et celui de Tarquin.\par
La couronne était élective ; et sous les cinq premiers rois, le sénat eut la plus grande part à l’élection.\par
Après la mort du roi, le sénat examinait si l’on garderait la forme du gouvernement qui était établie. S’il jugeait à propos de la garder, il nommait un magistrat\footnote{Denys d’Halicarnasse, liv. II, p. 120 ; et liv. IV, p. 242 et 243.} tiré de son corps, qui élisait un roi ; le sénat devait approuver l’élection ; le peuple, la confirmer ; les auspices, la garantir. Si une de ces trois conditions manquait, il fallait faire une autre élection.\par
La constitution était monarchique, aristocratique et populaire ; et telle fut l’harmonie du pouvoir, qu’on ne vit ni jalousie ni dispute, dans les premiers règnes. Le roi commandait les armées, et avait l’intendance des sacrifices ; il avait la puissance de juger les affaires civiles\footnote{Voyez le discours de Tanaquil, dans Tite-Live, liv. I, décade 1 et le règlement de Servius Tullius, dans Denys d’Halicarnasse, liv. IV, p. 229.} et criminelles\footnote{Voyez Denys d’Halicarnasse, liv. II, p. 118 ; et liv. III, p. 171.} {\itshape ;} il convoquait le sénat ; il assemblait le peuple ; il lui portait de certaines affaires, et réglait les autres avec le sénat\footnote{Ce fut par un sénatus-consulte que Tullus Hostilius envoya détruire Albe. Denys d’Halicarnasse, liv. III, p. 167 et 172.}.\par
Le sénat avait une grande autorité. Les rois prenaient souvent des sénateurs pour juger avec eux : ils ne portaient point d’affaires au peuple qu’elles n’eussent été délibérées\footnote{{\itshape Ibid.}, liv. IV, p. 176.} dans le sénat.\par
Le peuple avait le droit d’élire\footnote{{\itshape Ibid.}, liv. II. Il fallait pourtant qu’il ne nommât pas à toutes les charges puisque Valerius Publicola fit la fameuse loi qui défendait à tout citoyen d’exercer aucun emploi, s’il ne l’avait obtenu par le suffrage du peuple.} les magistrats, de consentir aux nouvelles lois, et, lorsque le roi le permettait, celui de déclarer la guerre et de faire la paix. Il n’avait point la puissance de juger. Quand Tullus Hostilius renvoya le jugement d’Horace au peuple, il eut des raisons particulières, que l’on trouve dans Denys d’Halicarnasse\footnote{{\itshape Ibid.}, liv. III, p. 159.}.\par
La constitution changea sous\footnote{{\itshape Ibid.}, liv. IV.} Servius Tullius. Le sénat n’eut point de part à son élection ; il se fit proclamer par le peuple. Il se dépouilla des jugements\footnote{Il se priva de la moitié de la puissance royale, dit Denys d’Halicarnasse, liv. IV, p. 229.} civils, et ne se réserva que les criminels ; il porta directement au peuple toutes les affaires, il le soulagea des taxes, et en mit tout le fardeau sur les patriciens. Ainsi, à mesure qu’il affaiblissait la puissance royale et l’autorité du sénat il augmentait le pouvoir du peuple\footnote{On croyait que, s’il n’avait pas été prévenu par Tarquin, il aurait établi le gouvernement populaire. Denys d’Halicarnasse, liv. IV, p. 243.}.\par
Tarquin ne se fit élire ni par le sénat ni par le peuple. Il regarda Servius Tullius comme un usurpateur, et prit la couronne comme un droit héréditaire ; il extermina la plupart des sénateurs ; il ne consulta plus ceux qui restaient, et ne les appela pas même à ses jugements\footnote{Denys d’Halicarnasse, liv. IV.}. Sa puissance augmenta ; mais ce qu’il y avait d’odieux dans cette puissance devint plus odieux encore : il usurpa le pouvoir du peuple ; il fit des lois sans lui, il en fit même contre lui\footnote{{\itshape Ibid.}}. Il aurait réuni les trois pouvoirs dans sa personne, mais le peuple se souvint un moment qu’il était législateur, et Tarquin ne fut plus.
\subsubsection[{Chapitre XIII. Réflexions générales sur l’État de Rome après l’expulsion des rois}]{Chapitre XIII. Réflexions générales sur l’État de Rome après l’expulsion des rois}
\noindent On ne peut jamais quitter les Romains : c’est ainsi qu’encore aujourd’hui, dans leur capitale, on laisse les nouveaux palais pour aller chercher des ruines ; c’est ainsi que l’œil qui s’est reposé sur l’émail des prairies, aime à voir les rochers et les montagnes.\par
Les familles patriciennes avaient eu, de tout temps, de grandes prérogatives. Ces distinctions, grandes sous les rois, devinrent bien plus importantes après leur expulsion. Cela causa la jalousie des plébéiens, qui voulurent les abaisser. Les contestations frappaient sur la constitution sans affaiblir le gouvernement : car, pourvu que les magistrats conservassent leur autorité, il était assez indifférent de quelle famille étaient les magistrats.\par
Une monarchie élective, comme était Rome, suppose nécessairement un corps aristocratique puissant qui la soutienne, sans quoi elle se change d’abord en tyrannie ou en État populaire. Mais un État populaire n’a pas besoin de cette distinction de familles pour se maintenir. C’est ce qui fit que les patriciens, qui étaient des parties nécessaires de la constitution du temps des rois, en devinrent une partie superflue du temps des consuls ; le peuple put les abaisser sans se détruire lui-même, et changer la constitution sans la corrompre.\par
Quand Servius Tullius eut avili les patriciens, Rome dut tomber des mains des rois dans celles du peuple. Mais le peuple, en abaissant les patriciens, ne dut point craindre de retomber dans celles des rois.\par
Un État peut changer de deux manières : ou parce que la constitution se corrige, ou parce qu’elle se corrompt. S’il a conservé ses principes, et que la constitution change, c’est qu’elle se corrige ; s’il a perdu ses principes, quand la constitution vient à changer, c’est qu’elle se corrompt.\par
Rome, après l’expulsion des rois, devait être une démocratie. Le peuple avait déjà la puissance législative : c’était son suffrage unanime qui avait chassé les rois ; et, s’il ne persistait pas dans cette volonté, les Tarquins pouvaient à tous les instants revenir. Prétendre qu’il eût voulu les chasser pour tomber dans l’esclavage de quelques familles, cela n’était pas raisonnable. La situation des choses demandait donc que Rome fût une démocratie ; et cependant elle ne l’était pas. Il fallut tempérer le pouvoir des principaux, et que les lois inclinassent vers la démocratie.\par
Souvent les États fleurissent plus dans le passage insensible d’une constitution à une autre, qu’ils ne le faisaient dans l’une ou l’autre de ces constitutions. C’est pour lors que tous les ressorts du gouvernement sont tendus ; que tous les citoyens ont des prétentions ; qu’on s’attaque ou qu’on se caresse ; et qu’il y a une noble émulation entre ceux qui défendent la constitution qui décline, et ceux qui mettent en avant celle qui prévaut.
\subsubsection[{Chapitre XIV. Comment la distribution des trois pouvoirs commença à changer après l’expulsion des rois}]{Chapitre XIV. Comment la distribution des trois pouvoirs commença à changer après l’expulsion des rois}
\noindent Quatre choses choquaient principalement la liberté de Rome. Les patriciens obtenaient seuls tous les emplois sacrés, politiques, civils et militaires ; on avait attaché au consulat un pouvoir exorbitant ; on faisait des outrages au peuple ; enfin on ne lui laissait presque aucune influence dans les suffrages. Ce furent ces quatre abus que le peuple corrigea.\par
1° Il fit établir qu’il y aurait des magistratures où les plébéiens pourraient prétendre ; et il obtint peu à peu qu’il aurait part à toutes, excepté à celle d’entre-roi.\par
2{\itshape °} On décomposa le consulat, et on en forma plusieurs magistratures. On créa des préteurs\footnote{Tite-Live, décade I, liv. VI.}, à qui on donna la puissance de juger les affaires privées ; on nomma des questeurs\footnote{{\itshape Quaestores parricidii}, Pomponius, leg. 2, § 23, ff. {\itshape De orig. jur.}}, pour faire juger les crimes publics ; on établit des édiles, à qui on donna la police ; on fit des trésoriers\footnote{Plutarque, {\itshape Vie de Publicola}.}, qui eurent l’administration des deniers publics ; enfin, par la création des censeurs, on ôta aux consuls cette partie de la puissance législative qui règle les mœurs des citoyens, et la police momentanée des divers corps de l’État. Les principales prérogatives qui leur restèrent furent de présider aux grands États du peuple\footnote{{\itshape Comitiis centuriatis}.}, d’assembler le sénat et de commander les armées.\par
3° Les lois sacrées établirent des tribuns qui pouvaient, à tous les instants, arrêter les entreprises des patriciens, et n’empêchaient pas seulement les injures particulières, mais encore les générales.\par
[4°] Enfin les plébéiens augmentèrent leur influence dans les décisions publiques. Le peuple romain était divisé de trois manières : par centuries, par curies et par tribus ; et quand il donnait son suffrage, il était assemblé et formé d’une de ces trois manières.\par
Dans la première, les patriciens, les principaux, les gens riches, le sénat, ce qui était à peu près la même chose, avaient presque toute l’autorité ; dans la seconde, ils en avaient moins ; dans la troisième, encore moins.\par
La division par centuries était plutôt une division de cens et de moyens, qu’une division de personnes. Tout le peuple était partagé en cent quatre-vingt-treize centuries\footnote{Voyez là-dessus Tite-Live, liv. I, chap. XLIII ; et Denys d’Halicarnasse, liv. IV et VII.} qui avaient chacune une voix. Les patriciens et les principaux formaient les quatre-vingt-dix-huit premières centuries ; le reste des citoyens étant répandu dans les quatre-vingt-quinze autres. Les patriciens étaient donc, dans cette division, les maîtres des suffrages.\par
Dans la division par curies\footnote{Denys d’Halicarnasse, liv. IX, p. 598.} les patriciens n’avaient pas les mêmes avantages. Ils en avaient pourtant. Il fallait consulter les auspices, dont les patriciens étaient les maîtres ; on n’y pouvait faire de proposition au peuple, qui n’eût été auparavant portée au sénat, et approuvée par un sénatus-consulte. Mais, dans la division par tribus, il n’était question ni d’auspices, ni de sénatus-consulte, et les patriciens n’y étaient pas admis.\par
Or le peuple chercha toujours à faire par curies les assemblées qu’on avait coutume de faire par centuries, et à faire par tribus les assemblées qui se faisaient par curies ; ce qui fit passer les affaires des mains des patriciens dans celles des plébéiens.\par
Ainsi, quand les plébéiens eurent obtenu le droit de juger les patriciens, ce qui commença lors de l’affaire de Coriolan\footnote{{\itshape Id.}, liv. VII.}, les plébéiens voulurent les juger assemblés par tribus\footnote{Contre l’ancien usage, comme on le voit dans Denys d’Halicarnasse, liv. V, p. 320.}, et non par centuries ; et lorsqu’on établit en faveur du peuple les nouvelles magistratures\footnote{{\itshape Ibid.}, liv. VI, p. 4 10 et 411.} de tribuns et d’édiles, le peuple obtint qu’il s’assemblerait par curies pour les nommer ; et quand sa puissance fut affermie, il obtint\footnote{{\itshape Ibid.}, liv. IX, p. 605.} qu’ils seraient nommés dans une assemblée par tribus.
\subsubsection[{Chapitre XV. Comment, dans l’état florissant de la république, Rome perdit tout à coup sa liberté}]{Chapitre XV. Comment, dans l’état florissant de la république, Rome perdit tout à coup sa liberté}
\noindent Dans le feu des disputes entre les patriciens et les plébéiens, ceux-ci demandèrent que l’on donnât des lois fixes, afin que les jugements ne fussent plus l’effet d’une volonté capricieuse, ou d’un pouvoir arbitraire. Après bien des résistances, le sénat y acquiesça. Pour composer ces lois, on nomma les décemvirs. On crut qu’on devait leur accorder un grand pouvoir, parce qu’ils avaient à donner des lois à des partis qui étaient presque incompatibles. On suspendit la nomination de tous les magistrats ; et dans les comices, ils furent élus seuls administrateurs de la république. Ils se trouvèrent revêtus de la puissance consulaire et de la puissance tribunitienne. L’une leur donnait le droit d’assembler le sénat ; l’autre, celui d’assembler le peuple ; mais ils ne convoquèrent ni le sénat ni le peuple. Dix hommes dans la république eurent seuls toute la puissance législative, toute la puissance exécutrice, toute la puissance des jugements. Rome se vit soumise à une tyrannie aussi cruelle que celle de Tarquin. Quand Tarquin exerçait ses vexations, Rome était indignée du pouvoir qu’il avait usurpé ; quand les décemvirs exercèrent les leurs, elle fut étonnée du pouvoir qu’elle avait donné.\par
Mais quel était ce système de tyrannie, produit par des gens qui n’avaient obtenu le pouvoir politique et militaire que par la connaissance des affaires civiles ; et qui, dans les circonstances de ces temps-là, avaient besoin au-dedans de la lâcheté des citoyens pour qu’ils se laissassent gouverner, et de leur courage au-dehors pour les défendre ?\par
Le spectacle de la mort de Virginie, immolée par son père à la pudeur et à la liberté, fit évanouir la puissance des décemvirs. Chacun se trouva libre, parce que chacun fut offensé : tout le monde devint citoyen, parce que tout le monde se trouva père. Le sénat et le peuple rentrèrent dans une liberté qui avait été confiée à des tyrans ridicules.\par
Le peuple romain, plus qu’un autre, s’émouvait par les spectacles. Celui du corps sanglant de Lucrèce fit finir la royauté. Le débiteur qui parut sur la place couvert de plaies, fit changer la forme de la république. La vue de Virginie fit chasser les décemvirs. Pour faire condamner Manlius, il fallut ôter au peuple la vue du Capitole. La robe sanglante de César remit Rome dans la servitude.
\subsubsection[{Chapitre XVI. De la puissance législative dans la république romaine}]{Chapitre XVI. De la puissance législative dans la république romaine}
\noindent On n’avait point de droits à se disputer sous les décemvirs ; mais, quand la liberté revint, on vit les jalousies renaître : tant qu’il resta quelques privilèges aux patriciens, les plébéiens les leur ôtèrent.\par
Il y aurait eu peu de mal, si les plébéiens s’étaient contentés de priver les patriciens de leurs prérogatives, et s’ils ne les avaient pas offensés dans leur qualité même de citoyens. Lorsque le peuple était assemblé par curies ou par centuries, il était composé de sénateurs, de patriciens et de plébéiens. Dans les disputes, les plébéiens gagnèrent ce point\footnote{Denys d’Halicarnasse, liv. XI, p. 725.}, que seuls, sans les patriciens et sans le sénat, ils pourraient faire des lois qu’on appela plébiscites ; et les comices où on les fit s’appelèrent comices par tribus. Ainsi il y eut des cas où les patriciens n’eurent point de part à la puissance législative\footnote{Par les lois sacrées, les plébéiens purent faire des plébiscites, seuls et sans que les patriciens fussent admis dans leur assemblée. Denys d’Halicarnasse, liv. VI, p. 410 ; et liv. VII, p. 430.}, et où ils furent soumis à la puissance législative d’un autre corps de l’État\footnote{Par la loi faite après l’expulsion des décemvirs, les patriciens furent soumis aux plébiscites, quoiqu’ils n’eussent pu y donner leurs voix. Tite-Live, liv. III, et Denys d’Halicarnasse, liv. XI, p. 725. — Et cette loi fut confirmée par celle de Publilius Philo, dictateur, l’an de Rome 416. Tite-Live, liv. VIII.}.\par
Ce fut un délire de la liberté. Le peuple, pour établir la démocratie, choqua les principes mêmes de la démocratie. Il semblait qu’une puissance aussi exorbitante aurait dû anéantir l’autorité du sénat ; mais Rome avait des institutions admirables. Elle en avait deux surtout : par l’une, la puissance législative du peuple était réglée ; par l’autre, elle était bornée.\par
Les censeurs, et avant eux les consuls\footnote{L’an 312 de Rome, les consuls faisaient encore le cens, comme il paraît par Denys d’Halicarnasse, liv. XI.}, formaient et créaient, pour ainsi dire, tous les cinq ans, le corps du peuple ; ils exerçaient la législation sur le corps même qui avait la puissance législative : « Tiberius Gracchus, censeur, dit Cicéron, transféra les affranchis dans les tribus de la ville, non par la force de son éloquence, mais par une parole et par un geste ; et s’il ne l’eût pas fait, cette république, qu’aujourd’hui nous soutenons à peine, nous ne l’aurions {\itshape plus. »}\par
D’un autre côté, le sénat avait le pouvoir d’ôter, pour ainsi dire, la république des mains du peuple, par la création d’un dictateur, devant lequel le souverain baissait la tête, et les lois les plus populaires restaient dans le silence\footnote{Comme celles qui permettaient d’appeler au peuple des ordonnances de tous les magistrats.}.\par
  \textbf{Chapitre XVII. {\itshape De la puissance exécutrice dans la même république} } \par
Si le peuple fut jaloux de sa puissance législative, il le fut moins de sa puissance exécutrice. Il la laissa presque tout entière au sénat et aux consuls ; et il ne se réserva guère que le droit d’élire les magistrats, et de confirmer les actes du sénat et des généraux.\par
Rome, dont la passion était de commander, dont l’ambition était de tout soumettre, qui avait toujours usurpé, qui usurpait encore, avait continuellement de grandes affaires, ses ennemis conjuraient contre elle, ou elle conjurait contre ses ennemis.\par
Obligée de se conduire, d’un côté, avec un courage héroïque, et de l’autre avec une sagesse consommée, l’état des choses demandait que le sénat eût la direction des affaires. Le peuple disputait au sénat toutes les branches de la puissance législative, parce qu’il était jaloux de sa liberté ; il ne lui disputait point les branches de la puissance exécutrice, parce qu’il était jaloux de sa gloire.\par
La part que le sénat prenait à la puissance exécutrice était si grande, que Polybe\footnote{Liv. VI.} dit que les étrangers pensaient tous que Rome était une aristocratie. Le sénat disposait des deniers publics et donnait les revenus à ferme ; il était l’arbitre des affaires des alliés ; il décidait de la guerre et de la paix, et dirigeait à cet égard les consuls ; il fixait le nombre des troupes romaines et des troupes alliées, distribuait les provinces et les armées aux consuls ou aux préteurs ; et, l’an du commandement expiré, il pouvait leur donner un successeur ; il décernait les triomphes ; il recevait des ambassades et en envoyait ; il nommait les rois, les récompensait, les punissait, les jugeait, leur donnait ou leur faisait perdre le titre d’alliés du peuple romain.\par
Les consuls faisaient la levée des troupes qu’ils devaient mener à la guerre ; ils commandaient les armées de terre ou de mer, disposaient des alliés : ils avaient dans les provinces toute la puissance de la république ; ils donnaient la paix aux peuples vaincus, leur en imposaient les conditions, ou les renvoyaient au sénat.\par
Dans les premiers temps, lorsque le peuple prenait quelque part aux affaires de la guerre et de la paix, il exerçait plutôt sa puissance législative que sa puissance exécutrice. Il ne faisait guère que confirmer ce que les rois, et, après eux, les consuls ou le sénat avaient fait. Bien loin que le peuple fût l’arbitre de la guerre, nous voyons que les consuls ou le sénat la faisaient souvent malgré l’opposition de ses tribuns. Mais, dans l’ivresse des prospérités, il augmenta sa puissance exécutrice. Ainsi il\footnote{L’an de Rome 444, Tite-Live, première décade, liv. IX. La guerre contre Persée paraissant périlleuse, un sénatus-consulte ordonna que cette loi serait suspendue, et le peuple y consentit. Tite-Live, cinquième décade, liv. II.} créa lui-même les tribuns des légions, que les généraux avaient nommés jusqu’alors ; et quelque temps avant la première guerre punique, il régla qu’il aurait seul le droit de déclarer la guerre\footnote{Il l’arracha du sénat, dit Freinshemius, deuxième décade, liv. VI.}.
\subsubsection[{Chapitre XVIII. De la puissance de juger dans le gouvernement de Rome}]{Chapitre XVIII. De la puissance de juger dans le gouvernement de Rome}
\noindent La puissance de juger fut donnée au peuple, au sénat, aux magistrats, à de certains juges. Il faut voir comment elle fut distribuée. Je commence par les affaires civiles.\par
Les consuls\footnote{On ne peut douter que les consuls, avant la création des préteurs, n’eussent eu les jugements civils. Voyez Tite-Live, première décade, liv. II, p. 19 ; Denys d’Halicarnasse, liv. X, p. 627 ; et même livre, p. 645.} jugèrent après les rois, comme les préteurs jugèrent après les consuls. Servius Tullius s’était dépouillé du jugement des affaires civiles ; les consuls ne les jugèrent pas non plus, si ce n’est dans des cas très rares\footnote{Souvent les tribuns jugèrent seuls ; rien ne les rendit plus odieux. Denys d’Halicarnasse, liv. XI, p. 709.} que l’on appela, pour cette raison, {\itshape extraordinaires}\footnote{{\itshape Judicia extraordinaria}. Voyez les {\itshape Institutes}, liv. IV.}. Ils se contentèrent de nommer les juges, et de former les tribunaux qui devaient juger. Il paraît, par le discours d’Appius Claudius, dans Denys d’Halicarnasse\footnote{Liv. VI, p. 360.}, que, dès l’an de Rome 259, ceci était regardé comme une coutume établie chez les Romains ; et ce n’est pas la faire remonter bien haut que de la rapporter à Servius Tullius.\par
Chaque année, le préteur formait une liste\footnote{{\itshape Album judicum}.} ou tableau de ceux qu’il choisissait pour faire la fonction de juges pendant l’année de sa magistrature. On en prenait le nombre suffisant pour chaque affaire. Cela se pratique à peu près de même en Angleterre. Et ce qui était très favorable à la liberté\footnote{« Nos ancêtres n’ont pas voulu, dit Cicéron, {\itshape Pro Cluentio}, c. XLIII, qu’un homme, dont les parties ne seraient pas convenues, pût être juge non seulement de la réputation d’un citoyen, mais même de la moindre affaire pécuniaire. »} c’est que le préteur prenait les juges du consentement\footnote{Voyez dans les Fragments de la loi Servilienne, de la Cornélienne et autres, de quelle manière ces lois donnaient des juges dans les crimes qu’elles se proposaient de punir. Souvent ils étaient pris par le choix, quelquefois par le sort, ou enfin par le sort mêlé avec le choix.} des parties. Le grand nombre de récusations que l’on peut faire aujourd’hui en Angleterre, revient à peu près à cet usage.\par
Ces juges ne décidaient que des questions de fait\footnote{Sénèque, {\itshape De beneficiis}, liv. III, chap. VII, {\itshape in fine.}} {\itshape : par} exemple, si une somme avait été payée, ou non ; si une action avait été commise, ou non. Mais pour les questions de droit\footnote{Voyez Quintilien, liv. IV, p. 54, in-fol., édit. de Paris, 1541.}, comme elles demandaient une certaine capacité, elles étaient portées au tribunal des centumvirs\footnote{Leg. 2, § 24, ff. {\itshape De orig. jur.} Des magistrats, appelés décemvirs, présidaient au jugement, le tout sous la direction d’un préteur.}.\par
Les rois se réservèrent le jugement des affaires criminelles, et les consuls leur succédèrent en cela. Ce fut en conséquence de cette autorité que le consul Brutus fit mourir ses enfants et tous ceux qui avaient conjuré pour les Tarquins. Ce pouvoir était exorbitant. Les consuls ayant déjà la puissance militaire, ils en portaient l’exercice même dans les affaires de la ville ; et leurs procédés, dépouillés des formes de la justice, étaient des actions violentes plutôt que des jugements.\par
Cela fit faire la loi Valérienne, qui permit d’appeler au peuple de toutes les ordonnances des consuls qui mettraient en péril la vie d’un citoyen. Les consuls ne purent plus prononcer une peine capitale contre un citoyen romain, que par la volonté du peuple\footnote{{\itshape Quoniam de capite civis Romani, injussu populi Romani, non erat permissum consulibus jus dicere.} Voyez Pomponius, leg. 2, § 16, ff. {\itshape De origine juris}.}.\par
On voit, dans la première conjuration pour le retour des Tarquins, que le consul Brutus juge les coupables ; dans la seconde, on assemble le sénat et les comices pour juger\footnote{Denys d’Halicarnasse, liv. V, p. 322.}.\par
Les lois qu’on appela {\itshape sacrées} donnèrent aux plébéiens des tribuns, qui formèrent un corps qui eut d’abord des prétentions immenses. On ne sait quelle fut plus grande, ou dans les plébéiens la lâche hardiesse de demander, ou dans le sénat la condescendance et la facilité d’accorder. La loi Valérienne avait permis les appels au peuple, c’est-à-dire au peuple composé de sénateurs, de patriciens et de plébéiens. Les plébéiens établirent que ce serait devant eux que les appellations seraient portées. Bientôt on mit en question si les plébéiens pourraient juger un patricien : cela fut le sujet d’une dispute que l’affaire de Coriolan fit naître, et qui finit avec cette affaire. Coriolan, accusé par les tribuns devant le peuple, soutenait, contre l’esprit de la loi Valérienne, qu’étant patricien, il ne pouvait être jugé que par les consuls : les plébéiens, contre l’esprit de la même loi, prétendirent qu’il ne devait être jugé que par eux seuls, et ils le jugèrent.\par
La loi des Douze Tables modifia ceci. Elle ordonna qu’on ne pourrait décider de la vie d’un citoyen que dans les grands États du peuple\footnote{Les comices par centuries. Aussi Manlius Capitolinus fut-il jugé dans ces comices. Tite-Live, décade I, liv. VI, p. 68.}. Ainsi, le corps des plébéiens, ou, ce qui est la même chose, les comices par tribus, ne jugèrent plus que les crimes dont la peine n’était qu’une amende pécuniaire. Il fallait une loi pour infliger une peine capitale ; pour condamner à une peine pécuniaire, il ne fallait qu’un plébiscite.\par
Cette disposition de la loi des Douze Tables fut très sage. Elle forma une conciliation admirable entre le corps des plébéiens et le sénat. Car, comme la compétence des uns et des autres dépendit de la grandeur de la peine et de la nature du crime, il fallut qu’ils se concertassent ensemble.\par
La loi Valérienne ôta tout ce qui restait à Rome du gouvernement qui avait du rapport à celui des rois grecs des temps héroïques. Les consuls se trouvèrent sans pouvoir pour la punition de crimes. Quoique tous les crimes soient publics, il faut pourtant distinguer ceux qui intéressent plus les citoyens entre eux, de ceux qui intéressent plus l’État dans le rapport qu’il a avec un citoyen. Les premiers sont appelés privés ; les seconds sont les crimes publics. Le peuple jugea lui-même les crimes publics ; et, à l’égard des privés, il nomma pour chaque crime, par une commission particulière, un questeur pour en faire la poursuite. C’était souvent un des magistrats, quelquefois un homme privé, que le peuple choisissait. On l’appelait {\itshape questeur du parricide}. Il en est fait mention dans la loi des Douze Tables\footnote{Dit Pomponius, dans la loi 2, au Digeste {\itshape De orig. jur.}}.\par
Le questeur nommait ce qu’on appelait le juge de la question, qui tirait au sort les juges, formait le tribunal, et présidait sous lui au jugement\footnote{Voyez un fragment d’Ulpien, qui en rapporte un autre de la loi Cornélienne ; on le trouve dans la {\itshape Collation des lois mosaïques et romaines}, tit. I, {\itshape de sicariis et homicidiis}.}.\par
Il est bon de faire remarquer ici la part que prenait le sénat dans la nomination du questeur, afin que l’on voie comment les Puissances étaient, à cet égard, balancées. Quelquefois le sénat faisait élire un dictateur, pour faire la fonction de questeur\footnote{Cela avait surtout lieu dans les crimes commis en Italie, où le sénat avait une principale inspection. Voyez Tite-Live, première décade, liv. IX sur les conjurations de Capoue.} ; quelquefois il ordonnait que le peuple serait convoqué par un tribun, pour qu’il nommât un questeur\footnote{Cela fut ainsi dans la poursuite de la mort de Posthumius, l’an 340 de Rome. Voyez Tite-Live.} ; enfin le peuple nommait quelquefois un magistrat pour faire son rapport au sénat sur un certain crime, et lui demander qu’il donnât un questeur, comme on voit dans le jugement de Lucius Scipion\footnote{Ce jugement fut rendu l’an de Rome 567.}, dans Tite-Live\footnote{Liv. VIII.}.\par
L’an de Rome 604, quelques-unes de ces commissions furent rendues permanentes\footnote{Cicéron, {\itshape in Bruto}.}. On divisa peu à peu toutes les matières criminelles en diverses parties, qu’on appela des {\itshape questions perpétuelles}. On créa divers préteurs, et on attribua à chacun d’eux quelqu’une de ces questions. On leur donna, pour un an, la puissance de juger les crimes qui en dépendaient ; et ensuite ils allaient gouverner leur province.\par
À Carthage, le sénat des cent était composé de juges qui étaient pour la vie\footnote{Cela se prouve par Tite-Live, liv. XLIII, qui dit qu’Annibal rendit leur magistrature annuelle.}. Mais à Rome les préteurs étaient annuels ; et les juges n’étaient pas même pour un an, puisqu’on les prenait pour chaque affaire. On a vu, dans le chapitre VI de ce livre, combien, dans de certains gouvernements, cette disposition était favorable à la liberté.\par
Les juges furent pris dans l’ordre des sénateurs, jusqu’au temps des Gracques. Tiberius Gracchus fit ordonner qu’on les prendrait dans celui des chevaliers : changement si considérable, que le tribun se vanta d’avoir, par une seule rogation, coupé les nerfs de l’ordre des sénateurs.\par
Il faut remarquer que les trois pouvoirs peuvent être bien distribués par rapport à la liberté de la constitution, quoiqu’ils ne le soient pas si bien dans le rapport avec la liberté du citoyen. À Rome, le peuple ayant la plus grande partie de la puissance législative, une partie de la puissance exécutrice, et une partie de la puissance de juger, c’était un grand pouvoir qu’il fallait balancer par un autre. Le sénat avait bien une partie de la puissance exécutrice ; il avait quelque branche de la puissance législative\footnote{Les sénatus-consultes avaient force pendant un an, quoiqu’ils ne fussent pas confirmés par le peuple. Denys d’Halicarnasse, liv. IX, p. 59 ; et liv. XI, p. 735.} {\itshape ;} mais cela ne suffisait pas pour contrebalancer le peuple. Il fallait qu’il eût part à la puissance de juger ; et il y avait part lorsque les juges étaient choisis parmi les sénateurs. Quand les Gracques privèrent les sénateurs de la puissance de juger\footnote{En l’an 630.}, le sénat ne put plus résister au peuple. Ils choquèrent donc la liberté de la constitution, pour favoriser la liberté du citoyen ; mais celle-ci se perdit avec celle-là.\par
Il en résulta des maux infinis. On changea la constitution dans un temps où, dans le feu des discordes civiles, il y avait à peine une constitution. Les chevaliers ne furent plus cet ordre moyen qui unissait le peuple au sénat ; et la chaîne de la constitution fut rompue.\par
Il y avait même des raisons particulières qui devaient empêcher de transporter les jugements aux chevaliers. La constitution de Rome était fondée sur ce principe, que ceux-là devaient être soldats, qui avaient assez de bien pour répondre de leur conduite à la république. Les chevaliers, comme les plus riches, formaient la cavalerie des légions. Lorsque leur dignité fut augmentée, ils ne voulurent plus servir dans cette milice ; il fallut lever une autre cavalerie : Marius prit toute sorte de gens dans les légions, et la république fut perdue\footnote{{\itshape Capite censos plerosque.} Salluste, {\itshape Guerre de Jugurtha}.}.\par
De plus, les chevaliers étaient les traitants de la république ; ils étaient avides, ils semaient les malheurs dans les malheurs, et faisaient naître les besoins publics des besoins publics. Bien loin de donner à de telles gens la puissance de juger, il aurait fallu qu’ils eussent été sans cesse sous les yeux des juges. Il faut dire cela à la louange des anciennes lois françaises ; elles ont stipulé avec les gens d’affaires, avec la méfiance que l’on garde à des ennemis. Lorsqu’à Rome les jugements furent transportés aux traitants, il n’y eut plus de vertu, plus de police, plus de lois, plus de magistrature, plus de magistrats.\par
On trouve une peinture bien naïve de ceci dans quelques fragments de Diodore de Sicile et de Dion. « Mucius Scévola, dit Diodore\footnote{Fragment de cet auteur, liv. XXXVI, dans le recueil de Constantin Porphyrogénète, {\itshape Des vertus et des vices.}}, voulut rappeler les anciennes mœurs et vivre de son bien propre avec frugalité et intégrité. Car ses prédécesseurs ayant fait une société avec les traitants, qui avaient pour lors les jugements à Rome, ils avaient rempli la province de toutes sortes de crimes. Mais Scévola fit justice des publicains, et fit mener en prison ceux qui y traînaient les autres. »\par
Dion nous dit\footnote{Fragment de son histoire, tiré de l’{\itshape Extrait des vertus et des vices.}} que Publius Rutilius, son lieutenant, qui n’était pas moins odieux aux chevaliers, fut accusé, à son retour, d’avoir reçu des présents, et fut condamné à une amende. Il fit sur-le-champ cession de biens. Son innocence parut, en ce qu’on lui trouva beaucoup moins de bien qu’on ne l’accusait d’en avoir volé, et il montrait les titres de sa propriété. Il ne voulut plus rester dans la ville avec de telles gens.\par
« Les Italiens, dit encore Diodore\footnote{Fragment du livre XXXIV dans l’{\itshape Extrait des vertus et des vices}.}, achetaient en Sicile des troupes d’esclaves pour labourer leurs champs et avoir soin de leurs troupeaux ; ils leur refusaient la nourriture. Ces malheureux étaient obligés d’aller voler sur les grands chemins, armés de lances et de massues, couverts de peaux de bêtes, de grands chiens autour d’eux. Toute la province fut dévastée, et les gens du pays ne pouvaient dire avoir en propre que ce qui était dans l’enceinte des villes. Il n’y avait ni proconsul, ni préteur, qui pût ou voulût s’opposer à ce désordre, et qui osât punir ces esclaves, parce qu’ils appartenaient aux chevaliers qui avaient à Rome les jugements\footnote{{\itshape Penes quos Romae tum judicia erant, atque ex equestri ordine solerent sortito judices eligi in causa praetorum et proconsulum, quibus, post administratam provinciam, dies dicta erat.}}. » Ce fut pourtant une des causes de la guerre des esclaves. Je ne dirai qu’un mot : une profession qui n’a ni ne peut avoir d’objet que le gain ; une profession qui demandait toujours, et à qui on ne demandait rien ; une profession sourde et inexorable, qui appauvrissait les richesses et la misère même, ne devait point avoir à Rome les jugements.
\subsubsection[{Chapitre XIX. Du gouvernement des provinces romaines}]{Chapitre XIX. Du gouvernement des provinces romaines}
\noindent C’est ainsi que les trois pouvoirs furent distribués dans la ville ; mais il s’en faut bien qu’ils le fussent de même dans les provinces. La liberté était dans le centre, et la tyrannie aux extrémités.\par
Pendant que Rome ne domina que dans l’Italie, les peuples furent gouvernés comme des confédérés : on suivait les lois de chaque république. Mais lorsqu’elle conquit plus loin, que le sénat n’eut pas immédiatement l’œil sur les provinces, que les magistrats qui étaient à Rome ne purent plus gouverner l’empire, il fallut envoyer des préteurs et des proconsuls. Pour lors, cette harmonie des trois pouvoirs ne fut plus. Ceux qu’on envoyait avaient une puissance qui réunissait celle de toutes les magistratures romaines ; que dis-je ? celle même du sénat, celle même du peuple\footnote{Ils faisaient leurs édits en entrant dans les provinces.}. C’étaient des magistrats despotiques, qui convenaient beaucoup à l’éloignement des lieux où ils étaient envoyés. Ils exerçaient les trois pouvoirs ; ils étaient, si j’ose me servir de ce terme, les bachas de la république.\par
Nous avons dit ailleurs\footnote{Liv. V, chap. XIX. Voyez aussi les liv. II, III, IV et V.}, que les mêmes citoyens dans la république avaient, par la nature des choses, les emplois civils et militaires. Cela fait qu’une république qui conquiert ne peut guère communiquer son gouvernement, et régir l’État conquis selon la forme de sa constitution. En effet, le magistrat qu’elle envoie pour gouverner, ayant la puissance exécutrice, civile et militaire, il faut bien qu’il ait aussi la puissance législative, car qui est-ce qui ferait des lois sans lui ? Il faut aussi qu’il ait la puissance de juger, car qui est-ce qui jugerait indépendamment de lui ? Il faut donc que le gouverneur qu’elle envoie ait les trois pouvoirs, comme cela fut dans les provinces romaines.\par
Une monarchie peut plus aisément communiquer son gouvernement, parce que les officiers qu’elle envoie ont, les uns la puissance exécutrice civile, et les autres la puissance exécutrice militaire ; ce qui n’entraîne pas après soi le despotisme.\par
C’était un privilège d’une grande conséquence pour un citoyen romain, de ne pouvoir être jugé que par le peuple. Sans cela, il aurait été soumis dans les provinces au pouvoir arbitraire d’un proconsul ou d’un propréteur. La ville ne sentait point la tyrannie, qui ne s’exerçait que sur les nations assujetties.\par
Ainsi, dans le monde romain, comme à Lacédémone, ceux qui étaient libres étaient extrêmement libres, et ceux qui étaient esclaves étaient extrêmement esclaves.\par
Pendant que les citoyens payaient des tributs, ils étaient levés avec une équité très grande. On suivait l’établissement de Servius Tullius, qui avait distribué tous les citoyens en six classes, selon l’ordre de leurs richesses, et fixé la part de l’impôt à proportion de celle que chacun avait dans le gouvernement. Il arrivait de là qu’on souffrait la grandeur du tribut à cause de la grandeur du crédit, et que l’on se consolait de la petitesse du crédit par la petitesse du tribut.\par
Il y avait encore une chose admirable : c’est que la division de Servius Tullius par classes étant, pour ainsi dire, le principe fondamental de la constitution, il arrivait que l’équité, dans la levée des tributs, tenait au principe fondamental du gouvernement, et ne pouvait être ôtée qu’avec lui.\par
Mais, pendant que la ville payait les tributs sans peine, ou n’en payait point du tout\footnote{Après la conquête de la Macédoine, les tributs cessèrent à Rome.}, les provinces étaient désolées par les chevaliers, qui étaient les traitants de la république. Nous avons parlé de leurs vexations, et toute l’histoire en est pleine.\par
« Toute l’Asie m’attend comme son libérateur, disait Mithridate\footnote{Harangue tirée de Trogue Pompée, rapportée par Justin, liv. XXXVIII.} {\itshape ;} tant ont excité de haine contre les Romains les rapines des proconsuls\footnote{Voyez les Oraisons contre Verrès.}, les exactions des gens d’affaires et les calomnies des jugements\footnote{On sait que ce fut le tribunal de Varus qui fit révolter les Germains.}. »\par
Voilà ce qui fit que la force des provinces n’ajouta rien à la force de la république, et ne fit au contraire que l’affaiblir. Voilà ce qui fit que les provinces regardèrent la perte de la liberté de Rome comme l’époque de l’établissement de la leur.
\subsubsection[{Chapitre XX. Fin de ce livre}]{Chapitre XX. Fin de ce livre}
\noindent Je voudrais rechercher, dans tous les gouvernements modérés que nous connaissons, quelle est la distribution des trois pouvoirs, et calculer par là les degrés de liberté dont chacun d’eux peut jouir. Mais il ne faut pas toujours tellement épuiser un sujet, qu’on ne laisse rien à faire au lecteur. Il ne s’agit pas de faire lire, mais de faire penser.
\subsection[{Livre douzième. Des lois qui forment la liberté politique dans son rapport avec le citoyen}]{Livre douzième. Des lois qui forment la liberté politique dans son rapport avec le citoyen}
\subsubsection[{Chapitre I. Idée de ce livre}]{Chapitre I. Idée de ce livre}
\noindent Ce n’est pas assez d’avoir traité de la liberté politique dans son rapport avec la constitution ; il faut la faire voir dans le rapport qu’elle a avec le citoyen.\par
J’ai dit que, dans le premier cas, elle est formée par une certaine distribution des trois pouvoirs ; mais, dans le second, il faut la considérer sous une autre idée. Elle consiste dans la sûreté, ou dans l’opinion que l’on a de sa sûreté.\par
Il pourra arriver que la constitution sera libre, et que le citoyen ne le sera point. Le citoyen pourra être libre, et la constitution ne l’être pas. Dans ces cas, la constitution sera libre de droit, et non de fait ; le citoyen sera libre de fait, et non pas de droit.\par
Il n’y a que la disposition des lois, et même des lois fondamentales, qui forme la liberté dans son rapport avec la constitution. Mais, dans le rapport avec le citoyen, des mœurs, des manières, des exemples reçus peuvent la faire naître ; et de certaines lois civiles la favoriser, comme nous allons voir dans ce livre-ci.\par
De plus, dans la plupart des États, la liberté étant plus gênée, choquée ou abattue, que leur constitution ne le demande, il est bon de parler des lois particulières qui, dans chaque constitution, peuvent aider ou choquer le principe de la liberté dont chacun d’eux peut être susceptible.
\subsubsection[{Chapitre II. De la liberté du citoyen}]{Chapitre II. De la liberté du citoyen}
\noindent La liberté philosophique consiste dans l’exercice de sa volonté, ou du moins (s’il faut parler dans tous les systèmes) dans l’opinion où l’on est que l’on exerce sa volonté. La liberté politique consiste dans la sûreté, ou du moins dans l’opinion que l’on a de sa sûreté.\par
Cette sûreté n’est jamais plus attaquée que dans les accusations publiques ou privées. C’est donc de la bonté des lois criminelles que dépend principalement la liberté du citoyen.\par
Les lois criminelles n’ont pas été perfectionnées tout d’un coup. Dans les lieux mêmes où l’on a le plus cherché la liberté, on ne l’a pas toujours trouvée. Aristote\footnote{{\itshape Politique}, liv. II.} nous dit qu’à Cumes, les parents de l’accusateur pouvaient être témoins. Sous les rois de Rome, la loi était si imparfaite, que Servius Tullius prononça la sentence contre les enfants d’Ancus Martius, accusés d’avoir assassiné le roi son beau-père\footnote{Tarquinius Priscus. Voyez Denys d’Halicarnasse, liv. IV.}. Sous les premiers rois des Francs, Clotaire fit une loi\footnote{De l’an 560.} pour qu’un accusé ne pût être condamné sans être ouï ; ce qui prouve une pratique contraire dans quelque cas particulier, ou chez quelque peuple barbare. Ce fut Charondas qui introduisit les jugements contre les faux témoignages\footnote{Aristote, {\itshape Politique}, liv. II, chap. XII. Il donna ses lois à Thurium dans la quatre-vingt-quatrième olympiade.}. Quand l’innocence des citoyens n’est pas assurée, la liberté ne l’est pas non plus.\par
Les connaissances que l’on a acquises dans quelques pays, et que l’on acquerra dans d’autres, sur les règles les plus sûres que l’on puisse tenir dans les jugements criminels, intéressent le genre humain plus qu’aucune chose qu’il y ait au monde.\par
Ce n’est que sur la pratique de ces connaissances que la liberté peut être fondée ; et dans un État qui aurait là-dessus les meilleures lois possibles, un homme à qui on ferait son procès, et qui devrait être pendu le lendemain, serait plus libre qu’un bacha ne l’est en Turquie.
\subsubsection[{Chapitre III. Continuation du même sujet}]{Chapitre III. Continuation du même sujet}
\noindent Les lois qui font périr un homme sur la déposition d’un seul témoin sont fatales à la liberté. La raison en exige deux ; parce qu’un témoin qui affirme et un accusé qui nie font un partage ; et il faut un tiers pour le vider.\par
Les Grecs\footnote{Voyez Aristide, {\itshape Oratio in Minervam}.} et les Romains\footnote{Denys d’Halicarnasse, sur le jugement de Coriolan, liv. VII.} exigeaient une voix de plus pour condamner. Nos lois françaises en demandent deux. Les Grecs prétendaient que leur usage avait été établi par les dieux\footnote{{\itshape Minervae calculus}.} ; mais c’est le nôtre.
\subsubsection[{Chapitre IV. Que la liberté est favorisée par la nature des peines et leur proportion}]{Chapitre IV. Que la liberté est favorisée par la nature des peines et leur proportion}
\noindent C’est le triomphe de la liberté, lorsque les lois criminelles tirent chaque peine de la nature particulière du crime. Tout l’arbitraire cesse ; la peine ne descend point du caprice du législateur, mais de la nature de la chose ; et ce n’est point l’homme qui fait violence à l’homme.\par
Il y a quatre sortes de crimes : ceux de la première espèce choquent la religion ; ceux de la seconde, les mœurs ; ceux de la troisième, la tranquillité ; ceux de la quatrième, la sûreté des citoyens. Les peines que l’on inflige doivent dériver de la nature de chacune de ces espèces.\par
Je ne mets dans la classe des crimes qui intéressent la religion que ceux qui l’attaquent directement, comme sont tous les sacrilèges simples. Car les crimes qui en troublent l’exercice sont de la nature de ceux qui choquent la tranquillité des citoyens ou leur sûreté, et doivent être renvoyés à ces classes.\par
Pour que la peine des sacrilèges simples soit tirée de la nature\footnote{Saint Louis fit des lois si outrées contre ceux qui juraient, que le pape se crut obligé de l’en avertir. Ce prince modéra son zèle et adoucit ses lois. Voyez ses ordonnances.} de la chose, elle doit consister dans la privation de tous les avantages que donne la religion : l’expulsion hors des temples ; la privation de la société des fidèles, pour un temps ou pour toujours ; la fuite de leur présence, les exécrations, les détestations, les conjurations.\par
Dans les choses qui troublent la tranquillité ou la sûreté de l’État, les actions cachées sont du ressort de la justice humaine. Mais dans celles qui blessent la divinité, là où il n’y a point d’action publique, il n’y a point de matière de crime : tout s’y passe entre l’homme et Dieu, qui sait la mesure et le temps de ses vengeances. Que si, confondant les choses, le magistrat recherche aussi le sacrilège caché, il porte une inquisition sur un genre d’action où elle n’est point nécessaire : il détruit la liberté des citoyens, en armant contre eux le zèle des consciences timides, et celui des consciences hardies.\par
Le mal est venu de cette idée, qu’il faut venger la divinité. Mais il faut faire honorer la divinité, et ne la venger jamais. En effet, si l’on se conduisait par cette dernière idée, quelle serait la fin des supplices ? Si les lois des hommes ont à venger un être infini, elles se régleront sur son infinité, et non pas sur les faiblesses, sur les ignorances, sur les caprices de la nature humaine.\par
Un historien de Provence\footnote{Le P. Bougerel.} rapporte un fait qui nous peint très bien ce que peut produire sur des esprits faibles cette idée de venger la divinité. Un Juif, accusé d’avoir blasphémé contre la Sainte Vierge, fut condamné à être écorché. Des chevaliers masqués, le couteau à la main, montèrent sur l’échafaud, et en chassèrent l’exécuteur, pour venger eux-mêmes l’honneur de la Sainte Vierge… Je ne veux point prévenir les réflexions du lecteur.\par
La seconde classe est des crimes qui sont contre les mœurs. Telles sont la violation de la continence publique ou particulière ; c’est-à-dire, de la police sur la manière dont on doit jouir des plaisirs attachés à l’usage des sens et à l’union des corps. Les peines de ces crimes doivent encore être tirées de la nature de la chose. La privation des avantages que la société a attachés à la pureté des mœurs, les amendes, la honte, la contrainte de se cacher, l’infamie publique, l’expulsion hors de la ville et de la société ; enfin, toutes les peines qui sont de la juridiction correctionnelle suffisent pour réprimer la témérité des deux sexes. En effet, ces choses sont moins fondées sur la méchanceté que sur l’oubli ou le mépris de soi-même.\par
Il n’est ici question que des crimes qui intéressent uniquement les mœurs, non de ceux qui choquent aussi la sûreté publique, tels que l’enlèvement et le viol, qui sont de la quatrième espèce.\par
Les crimes de la troisième classe sont ceux qui choquent la tranquillité des citoyens ; et les peines en doivent être tirées de la nature de la chose, et se rapporter à cette tranquillité, comme la prison, l’exil, les corrections et autres peines qui ramènent les esprits inquiets et les font rentrer dans l’ordre établi.\par
Je restreins les crimes contre la tranquillité aux choses qui contiennent une simple lésion de police : car celles qui, troublant la tranquillité, attaquent en même temps la sûreté, doivent être mises dans la quatrième classe.\par
Les peines de ces derniers crimes sont ce qu’on appelle des supplices. C’est une espèce de talion, qui fait que la société refuse la sûreté à un citoyen qui en a privé, ou qui a voulu en priver un autre. Cette peine est tirée de la nature de la chose, puisée dans la raison et dans les sources du bien et du mal. Un citoyen mérite la mort lorsqu’il a violé la sûreté au point qu’il a ôté la vie, ou qu’il a entrepris de l’ôter. Cette peine de mort est comme le remède de la société malade. Lorsqu’on viole la sûreté à l’égard des biens, il peut y avoir des raisons pour que la peine soit capitale ; mais il vaudrait peut-être mieux, et il serait plus de la nature, que la peine des crimes contre la sûreté des biens fût punie par la perte des biens ; et cela devrait être ainsi, si les fortunes étaient communes ou égales. Mais, comme ce sont ceux qui n’ont point de biens qui attaquent plus volontiers celui des autres, il a fallu que la peine corporelle suppléât à la pécuniaire.\par
Tout ce que je dis est puisé dans la nature, et est très favorable à la liberté du citoyen.
\subsubsection[{Chapitre V. De certaines accusations qui ont particulièrement besoin de modération et de prudence}]{Chapitre V. De certaines accusations qui ont particulièrement besoin de modération et de prudence}
\noindent Maxime importante : il faut être très circonspect dans la poursuite de la magie et de l’hérésie. L’accusation de ces deux crimes peut extrêmement choquer la liberté, et être la source d’une infinité de tyrannies, si le législateur ne sait la borner. Car, comme elle ne porte pas directement sur les actions d’un citoyen, mais plutôt sur l’idée que l’on s’est faite de son caractère, elle devient dangereuse à proportion de l’ignorance du peuple ; et pour lors un citoyen est toujours en danger, parce que la meilleure conduite du monde, la morale la plus pure, la pratique de tous les devoirs, ne sont pas des garants contre les soupçons de ces crimes.\par
Sous Manuel Comnène, le {\itshape protestator}\footnote{Nicétas, {\itshape Vie de Manuel Comnène}, liv. IV.} fut accusé d’avoir conspiré contre l’empereur, et de s’être servi pour cela de certains secrets qui rendent les hommes invisibles. Il est dit, dans la vie de cet empereur\footnote{{\itshape Ibid.}}, que l’on surprit Aaron lisant un livre de Salomon, dont la lecture faisait paraître des légions de démons. Or, en supposant dans la magie une puissance qui arme l’enfer, et en partant de là, on regarde celui que l’on appelle un magicien, comme l’homme du monde le plus propre à troubler et à renverser la société, et l’on est porté à le punir sans mesure.\par
L’indignation croît lorsque l’on met dans la magie le pouvoir de détruire la religion. L’histoire de Constantinople\footnote{{\itshape Histoire de l’empereur Maurice}, par Théophylacte chap. XI.} nous apprend que, sur une révélation qu’avait eue un évêque qu’un miracle avait cessé à cause de la magie d’un particulier, lui et son fils furent condamnés à mort. De combien de choses prodigieuses ce crime ne dépendait-il pas ? Qu’il ne soit pas rare qu’il y ait des révélations ; que l’évêque en ait eu une ; qu’elle fût véritable ; qu’il y eût eu un miracle ; que ce miracle eût cessé ; qu’il y eût de la magie ; que la magie pût renverser la religion ; que ce particulier fût magicien ; qu’il eût fait enfin cet acte de magie.\par
L’empereur Théodore Lascaris attribuait sa maladie à la magie. Ceux qui en étaient accusés n’avaient d’autre ressource que de manier un fer chaud sans se brûler. Il aurait été bon, chez les Grecs, d’être magicien pour se justifier de la magie. Tel était l’excès de leur idiotisme, qu’au crime du monde le plus incertain, ils joignaient les preuves les plus incertaines.\par
Sous le règne de Philippe le Long, les Juifs furent chassés de France, accusés d’avoir empoisonné les fontaines par le moyen des lépreux. Cette absurde accusation doit bien faire douter de toutes celles qui sont fondées sur la haine publique.\par
Je n’ai point dit ici qu’il ne fallait point punir l’hérésie ; je dis qu’il faut être très circonspect à la punir.
\subsubsection[{Chapitre VI. Du crime contre nature}]{Chapitre VI. Du crime contre nature}
\noindent Dieu ne plaise que je veuille diminuer l’horreur que l’on a pour un crime que la religion, la morale et la politique condamnent tour à tour. Il faudrait le proscrire quand il ne ferait que donner à un sexe les faiblesses de l’autre, et préparer à une vieillesse infâme par une jeunesse honteuse. Ce que j’en dirai lui laissera toutes ses flétrissures, et ne portera que contre la tyrannie qui peut abuser de l’horreur même que l’on en doit avoir.\par
Comme la nature de ce crime est d’être caché, il est souvent arrivé que des législateurs l’ont puni sur la déposition d’un enfant. C’était ouvrir une porte bien large à la calomnie. « Justinien, dit Procope\footnote{{\itshape Histoire secrète}.}, publia une loi contre ce crime ; il fit rechercher ceux qui en étaient coupables, non seulement depuis la loi, mais avant. La déposition d’un témoin, quelquefois d’un enfant, quelquefois d’un esclave, suffisait, surtout contre les riches et contre ceux qui étaient de la faction des verts. »\par
Il est singulier que, parmi nous, trois crimes, la magie, l’hérésie et le crime contre nature, dont on pourrait prouver, du premier, qu’il n’existe pas ; du second, qu’il est susceptible d’une infinité de distinctions, interprétations, limitations ; du troisième, qu’il est très souvent obscur, aient été tous trois punis de la peine du feu.\par
Je dirai bien que le crime contre nature ne fera jamais dans une société de grands progrès, si le peuple ne s’y trouve porté d’ailleurs par quelque coutume, comme chez les Grecs, où les jeunes gens faisaient tous leurs exercices nus ; comme chez nous, où l’éducation domestique est hors d’usage ; comme chez les Asiatiques, où des particuliers ont un grand nombre de femmes qu’ils méprisent, tandis que les autres n’en peuvent avoir. Que l’on ne prépare point ce crime, qu’on le proscrive par une police exacte, comme toutes les violations des mœurs, et l’on verra soudain la nature, ou défendre ses droits, ou les reprendre. Douce, aimable, charmante, elle a répandu les plaisirs d’une main libérale ; et, en nous comblant de délices, elle nous prépare, par des enfants qui nous font, pour ainsi dire, renaître, à des satisfactions plus grandes que ces délices mêmes.
\subsubsection[{Chapitre VII. Du crime de lèse-majesté}]{Chapitre VII. Du crime de lèse-majesté}
\noindent Les lois de la Chine décident que quiconque manque de respect à l’empereur doit être puni de mort. Comme elles ne définissent pas ce que c’est que ce manquement de respect, tout peut fournir un prétexte pour ôter la vie à qui l’on veut, et exterminer la famille que l’on veut.\par
Deux personnes chargées de faire la gazette de la cour, ayant mis dans quelque fait des circonstances qui ne se trouvèrent pas vraies, on dit que mentir dans une gazette de la cour, c’était manquer de respect à la cour ; et on les fit mourir\footnote{Le P. Du Halde, t. I, p. 43.}. Un prince du sang ayant mis quelque note par mégarde sur un mémorial signé du pinceau rouge par l’empereur, on décida qu’il avait manqué de respect à l’empereur, ce qui causa contre cette famille une des terribles persécutions dont l’histoire ait jamais parlé\footnote{Lettres du P. Parennin, dans les {\itshape Lettres édifiantes}.}.\par
C’est assez que le crime de lèse-majesté soit vague, pour que le gouvernement dégénère en despotisme. Je m’étendrai davantage là-dessus dans le livre {\itshape de la composition des lois}.
\subsubsection[{Chapitre VIII. De la mauvaise application du nom de crime de sacrilège et de lèse-majesté}]{Chapitre VIII. De la mauvaise application du nom de crime de sacrilège et de lèse-majesté}
\noindent C’est encore un violent abus de donner le nom de crime de lèse-majesté à une action qui ne l’est pas. Une loi des empereurs\footnote{Gratien, Valentinien et Théodose. C’est la troisième au Code {\itshape De crimin. sacril.}} poursuivait comme sacrilèges ceux qui mettaient en question le jugement du prince, et doutaient du mérite de ceux qu’il avait choisis pour quelque emploi\footnote{{\itshape Sacrilegii instar est dubitare an is dignus sit quem elegerit imperator, ibid.} Cette loi a servi de modèle à celle de Roger, dans les constitutions de Naples, tit. IV.}. Ce furent bien le cabinet et les favoris qui établirent ce crime. Une autre loi avait déclaré que ceux qui attentent contre les ministres et les officiers du prince sont criminels de lèse-majesté, comme s’ils attentaient contre le prince même\footnote{La loi cinquième, au Code, {\itshape ad leg. Jul. maj.}}. Nous devons cette loi à deux princes\footnote{Arcadius et Honorius.} dont la faiblesse est célèbre dans l’histoire ; deux princes qui furent menés par leurs ministres, comme les troupeaux sont conduits par les pasteurs ; deux princes, esclaves dans le palais, enfants dans le conseil, étrangers aux armées ; qui ne conservèrent l’empire que parce qu’ils le donnèrent tous les jours. Quelques-uns de ces favoris conspirèrent contre leurs empereurs. Ils firent plus : ils conspirèrent contre l’empire ; ils y appelèrent les Barbares ; et quand on voulut les arrêter, l’État était si faible qu’il fallut violer leur loi et s’exposer au crime de lèse-majesté pour les punir.\par
C’est pourtant sur cette loi que se fondait le rapporteur de M. de Cinq-Mars\footnote{{\itshape Mémoires} de Montrésor, t. I.}, lorsque, voulant prouver qu’il était coupable du crime de lèse-majesté pour avoir voulu chasser le cardinal de Richelieu des affaires, il dit : « Le crime qui touche la personne des ministres des princes est réputé, par les constitutions des empereurs, de pareil poids que celui qui touche leur personne. Un ministre sert bien son prince et son État ; on l’ôte à tous les deux ; c’est comme si l’on privait le premier d’un bras\footnote{{\itshape Nam ipsi pars corporis nostri sunt.} Même loi au code {\itshape ad. leg. Jul. maj.}} et le second d’une partie de sa puissance. » Quand la servitude elle-même viendrait sur la terre, elle ne parlerait pas autrement.\par
Une autre loi de Valentinien, Théodose et Arcadius\footnote{C’est la neuvième au code Théodosien, {\itshape de falsa moneta}.} déclare les faux-monnayeurs coupables du crime de lèse-majesté. Mais n’était-ce pas confondre les idées des choses ? Porter sur un autre crime le nom de lèse-majesté, n’est-ce pas diminuer l’horreur du crime de lèse-majesté ?
\subsubsection[{Chapitre IX. Continuation du même sujet}]{Chapitre IX. Continuation du même sujet}
\noindent Paulin ayant mandé à l’empereur Alexandre « qu’il se préparaît à poursuivre comme criminel de lèse-majesté un juge qui avait prononcé contre ses ordonnances » ; l’empereur lui répondit « que, dans un siècle comme le sien, les crimes de lèse-majesté indirects n’avaient point de lieu\footnote{{\itshape Etiam ex aliis causis majestatis crimina cessant meo saeculo}. Leg. I, Code, {\itshape ad leg. Jul. maj.}} ».\par
Faustinien ayant écrit au même empereur qu’ayant juré, par la vie du prince, qu’il ne pardonnerait jamais à son esclave, il se voyait obligé de perpétuer sa colère, pour ne pas se rendre coupable du crime de lèse-majesté : « Vous avez pris de vaines terreurs\footnote{{\itshape Alienam sectae meae sollicitudinem concepisti}. Leg. 2, Code, {\itshape ad leg. Jul. maj.}}, lui répondit l’empereur, et vous ne connaissez pas mes maximes. »\par
Un sénatus-consulte\footnote{Voyez la loi 4, § 1, ff. {\itshape ad leg. Jul. maj.}} ordonna que celui qui avait fondu des statues de l’empereur, qui auraient été réprouvées, ne serait point coupable de lèse-majesté. Les empereurs Sévère et Antonin écrivirent à Pontius\footnote{Voyez la loi 5, § 2.} que celui qui vendrait des statues de l’empereur non consacrées, ne tomberait point dans le crime de lèse-majesté. Les mêmes empereurs écrivirent à Julius Cassianus que celui qui jetterait, par hasard, une pierre contre une statue de l’empereur, ne devait point être poursuivi comme criminel de lèse-majesté\footnote{{\itshape Ibid.} §, 1.}. La loi Julie demandait ces sortes de modifications : car elle avait rendu coupable de lèse-majesté, non seulement ceux qui fondaient les statues des empereurs, mais ceux qui commettaient quelque action semblable\footnote{{\itshape Aliudve quid simile admiserint}. Leg. 6, ff. {\itshape ad leg. Jul. maj.}} ce qui rendait ce crime arbitraire. Quand on eut établi bien des crimes de lèse-majesté, il fallut nécessairement distinguer ces crimes. Aussi le jurisconsulte Ulpien, après avoir dit que l’accusation du crime de lèse-majesté ne s’éteignait point par la mort du coupable, ajoute-t-il que cela ne regarde pas tous\footnote{Dans la loi dernière, ff. {\itshape ad leg. Jul. de adulteriis}.} les crimes de lèse-majesté établis par la loi Julie ; mais seulement celui qui contient un attentat contre l’empire, ou contre la vie de l’empereur.
\subsubsection[{Chapitre X. Continuation du même sujet}]{Chapitre X. Continuation du même sujet}
\noindent Une loi d’Angleterre, passée sous Henri VIII, déclarait coupables de haute trahison tous ceux qui prédiraient la mort du roi. Cette loi était bien vague. Le despotisme est si terrible, qu’il se tourne même contre ceux qui l’exercent. Dans la dernière maladie de ce roi, les médecins n’osèrent jamais dire qu’il fût en danger ; et ils agirent, sans doute, en conséquence\footnote{Voyez l’{\itshape Histoire de la réformation}, par M. Burnet.}.
\subsubsection[{Chapitre XI. Des pensées}]{Chapitre XI. {\itshape Des pensées}}
\noindent Un Marsyas songea qu’il coupait la gorge à Denys\footnote{Plutarque, {\itshape Vie de Denys.}}. Celui-ci le fit mourir, disant qu’il n’y aurait pas songé la nuit s’il n’y eût pensé le jour. C’était une grande tyrannie : car, quand même il y aurait pensé, il n’avait pas attenté\footnote{Il faut que la pensée soit jointe à quelque sorte d’action.}. Les lois ne se chargent de punir que les actions extérieures.
\subsubsection[{Chapitre XII. Des paroles indiscrètes}]{Chapitre XII. Des paroles indiscrètes}
\noindent Rien ne rend encore le crime de lèse-majesté plus arbitraire que quand des paroles indiscrètes en deviennent la matière. Les discours sont si sujets à interprétation, il y a tant de différence entre l’indiscrétion et la malice, et il y en a si peu dans les expressions qu’elles emploient, que la loi ne peut guère soumettre les paroles à une peine capitale, à moins qu’elle ne déclare expressément celles qu’elle y soumet\footnote{{\itshape Si non tale sit delictum, in quod vel scriptura legis descendit, vel ad exemplum legis vindicandum est}, dit Modestinus dans la loi 7, § 3, {\itshape in fine}, ff. {\itshape ad leg. Jul. maj.}}.\par
Les paroles ne forment point un corps de délit ; elles ne restent que dans l’idée. La plupart du temps, elles ne signifient point par elles-mêmes, mais par le ton dont on les dit. Souvent, en redisant les mêmes paroles, on ne rend pas le même sens : ce sens dépend de la liaison qu’elles ont avec d’autres choses. Quelquefois le silence exprime plus que tous les discours. Il n’y a rien de si équivoque que tout cela. Comment donc en faire un crime de lèse-majesté ? Partout où cette loi est établie, non seulement la liberté n’est plus, mais son ombre même.\par
Dans le manifeste de la feue czarine, donné contre la famille d’Olgourouki\footnote{En 1740.}, un de ces princes est condamné à mort pour avoir proféré des paroles indécentes qui avaient du rapport à sa personne ; un autre, pour avoir malignement interprété ses sages dispositions pour l’empire, et offensé sa personne sacrée par des paroles peu respectueuses.\par
Je ne prétends point diminuer l’indignation que l’on doit avoir contre ceux qui veulent flétrir la gloire de leur prince ; mais je dirai bien que, si l’on veut modérer le despotisme, une simple punition correctionnelle conviendra mieux dans ces occasions, qu’une accusation de lèse-majesté toujours terrible à l’innocence même\footnote{{\itshape Nec lubricum linguae ad poenam facile trahendum est.} Modestin, dans la loi 7, § 3, ff. {\itshape ad leg. Jul. maj.}}.\par
Les actions ne sont pas de tous les jours ; bien des gens peuvent les remarquer : une fausse accusation sur des faits peut être aisément éclaircie. Les paroles qui sont jointes à une action, prennent la nature de cette action. Ainsi un homme qui va dans la place publique exhorter les sujets à la révolte, devient coupable de lèse-majesté, parce que les paroles sont jointes à l’action, et y participent. Ce ne sont point les paroles que l’on punit ; mais une action commise, dans laquelle on emploie les paroles. Elles ne deviennent des crimes que lorsqu’elles préparent, qu’elles accompagnent, ou qu’elles suivent une action criminelle. On renverse tout, si l’on fait des paroles un crime capital, au lieu de les regarder comme le signe d’un crime capital.\par
Les empereurs Théodose, Arcadius et Honorius, écrivirent à Ruffin, préfet du prétoire : « Si quelqu’un parle mal de notre personne ou de notre gouvernement, nous ne voulons point le punir : s’il a parlé par légèreté, il faut le mépriser ; si c’est par folie, il faut le plaindre ; si c’est une injure, il faut lui pardonner\footnote{{\itshape Si id ex levitate processerit, contemnendum est ; si ex insania, miseratione dignissimum ; si ab injuria, remittendum.} Leg. unica, Code {\itshape si quis imperatori maledixerit}.}. Ainsi, laissant les choses dans leur entier, vous nous en donnerez connaissance, afin que nous jugions des paroles par les personnes, et que nous pesions bien si nous devons les soumettre au jugement, ou les négliger. »
\subsubsection[{Chapitre XIII. Des écrits}]{Chapitre XIII. {\itshape Des écrits}}
\noindent Les écrits contiennent quelque chose de plus permanent que les paroles ; mais, lorsqu’ils ne préparent pas au crime de lèse-majesté, ils ne sont point une matière du crime de lèse-majesté.\par
Auguste et Tibère y attachèrent pourtant la peine de ce crime\footnote{Tacite, {\itshape Annales}, liv. I. Cela continua sous les règnes suivants. Voyez la loi première au Code {\itshape de famosis libellis}.} ; Auguste, à l’occasion de certains écrits faits contre des hommes et des femmes illustres ; Tibère, à cause de ceux qu’il crut faits contre lui. Rien ne fut plus fatal à la liberté romaine. Crémutius Cordus fut accusé, parce que, dans ses annales, il avait appelé Cassius le dernier des Romains\footnote{Tacite, {\itshape Annales}, liv. IV.}.\par
Les écrits satiriques ne sont guère connus dans les États despotiques, où l’abattement d’un côté et l’ignorance de l’autre ne donnent ni le talent ni la volonté d’en faire. Dans la démocratie, on ne les empêche pas, par la raison même qui, dans le gouvernement d’un seul, les fait défendre. Comme ils sont ordinairement composés contre des gens puissants, ils flattent dans la démocratie la malignité du peuple qui gouverne. Dans la monarchie, on les défend ; mais on en fait plutôt un sujet de police que de crime. Ils peuvent amuser la malignité générale, consoler les mécontents, diminuer l’envie contre les places, donner au peuple la patience de souffrir, et le faire rire de ses souffrances.\par
L’aristocratie est le gouvernement qui proscrit le plus les ouvrages satiriques. Les magistrats y sont de petits souverains qui ne sont pas assez grands pour mépriser les injures. Si, dans la monarchie, quelque trait va contre le monarque, il est si haut que le trait n’arrive point jusqu’à lui. Un seigneur aristocratique en est percé de part en part. Aussi les décemvirs, qui formaient une aristocratie, punirent-ils de mort les écrits satiriques\footnote{La loi des Douze Tables.}.
\subsubsection[{Chapitre XIV. Violation de la pudeur dans la punition des crimes}]{Chapitre XIV. Violation de la pudeur dans la punition des crimes}
\noindent Il y a des règles de pudeur observées chez presque toutes les nations du monde : il serait absurde de les violer dans la punition des crimes, qui doit toujours avoir pour objet le rétablissement de l’ordre.\par
Les Orientaux, qui ont exposé des femmes à des éléphants dressés pour un abominable genre de supplice, ont-ils voulu faire violer la loi par la loi ?\par
Un ancien usage des Romains défendait de faire mourir les filles qui n’étaient pas nubiles. Tibère trouva l’expédient de les faire violer par le bourreau avant de les envoyer au supplice\footnote{Suetonius, {\itshape in Tiberio}.} ; tyran subtil et cruel, il détruisait les mœurs pour conserver les coutumes.\par
Lorsque la magistrature japonaise a fait exposer dans les places publiques les femmes nues, et les a obligées de marcher à la manière des bêtes, elle a fait frémir la pudeur\footnote{{\itshape Recueil des voyages qui ont servi à l’établissement de la Compagnie des Indes}, t. V, part. II.} ; mais lorsqu’elle a voulu contraindre une mère.… lorsqu’elle a voulu contraindre un fils.… je ne puis achever, elle a fait frémir la nature même\footnote{{\itshape Ibid.}, p. 496.}.
\subsubsection[{Chapitre XV. De l’affranchissement de l’esclave pour accuser le maître}]{Chapitre XV. De l’affranchissement de l’esclave pour accuser le maître}
\noindent Auguste établit que les esclaves de ceux qui auraient conspiré contre lui seraient vendus au public, afin qu’ils pussent déposer contre leur maître\footnote{Dion, dans Xiphilin.}. On ne doit rien négliger de ce qui mène à la découverte d’un grand crime. Ainsi, dans un État où il y a des esclaves, il est naturel qu’ils puissent être indicateurs ; mais ils ne sauraient être témoins.\par
Vindex indiqua la conspiration faite en faveur de Tarquin ; mais il ne fut pas témoin contre les enfants de Brutus. Il était juste de donner la liberté à celui qui avait rendu un si grand service à sa patrie ; mais on ne la lui donna pas afin qu’il rendît ce service à sa patrie.\par
Aussi l’empereur Tacite ordonna-t-il que les esclaves ne seraient pas témoins \textsubscript{contre} leur maître, dans le crime même de lèse-majesté\footnote{Flavius Vopiscus, dans sa {\itshape Vie.}} : loi qui n’a pas été mise dans la compilation de Justinien.
\subsubsection[{Chapitre XVI. Calomnie dans le crime de lèse-majesté}]{Chapitre XVI. Calomnie dans le crime de lèse-majesté}
\noindent Il faut rendre justice aux Césars ; ils n’imaginèrent pas les premiers les tristes lois qu’ils firent. C’est Sylla\footnote{Sylla fit une loi de majesté, dont il est parlé dans les {\itshape Oraisons} de Cicéron, {\itshape Pro Cluentio}, art. 3 ; {\itshape In Pisonem}, art. 21 ; {\itshape Deuxième contre Verrès}, art. 5, {\itshape Épîtres familières}, liv. III, lettre II. César et Auguste les insérèrent dans les lois Julies ; d’autres y ajoutèrent.} qui leur apprit qu’il ne fallait point punir les calomniateurs. Bientôt on alla jusqu’à les récompenser\footnote{{\itshape Et quo quis distinctior accusator, eo magis honores assequebatur, ac veluti sacrosanctus erat.} Tacite.}.
\subsubsection[{Chapitre XVII. De la révélation des conspirations}]{Chapitre XVII. De la révélation des conspirations}
\noindent « Quand ton frère, ou ton fils, ou ta fille, ou ta femme bien-aimée, ou ton ami, qui est comme ton âme, te diront en secret : Allons à {\itshape d’autres dieux}, tu les lapideras : d’abord ta main sera sur lui, ensuite celle de tout le peuple. » Cette loi du Deutéronome\footnote{Chap. XIII, vers. 6, 7, 8 et 9.} ne peut être une loi civile chez la plupart des peuples que nous connaissons, parce qu’elle y ouvrirait la porte à tous les crimes.\par
La loi qui ordonne dans plusieurs États, sous peine de la vie, de révéler les conspirations auxquelles même on n’a pas trempé, n’est guère moins dure. Lorsqu’on la porte dans le gouvernement monarchique, il est très convenable de la restreindre.\par
Elle n’y doit être appliquée, dans toute sa sévérité, qu’au crime de lèse-majesté au premier chef. Dans ces États, il est très important de ne point confondre les différents chefs de ce crime.\par
Au Japon, où les lois renversent toutes les idées de la raison humain, le crime de non-révélation s’applique aux cas les plus ordinaires.\par
Une relation\footnote{{\itshape Recueil des voyages qui ont servi à l’établissement de la Compagnie des Indes}, p. 423, liv. V, part. II.} nous parle de deux demoiselles qui furent enfermées jusqu’à la mort dans un coffre hérissé de pointes ; l’une, pour avoir eu quelque intrigue de galanterie ; l’autre, pour ne l’avoir pas révélée.
\subsubsection[{Chapitre XVIII. Combien il est dangereux dans les républiques de trop punir le crime de lèse-majesté}]{Chapitre XVIII. Combien il est dangereux dans les républiques de trop punir le crime de lèse-majesté}
\noindent Quand une république est parvenue à détruire ceux qui voulaient la renverser, il faut se hâter de mettre fin aux vengeances, aux peines et aux récompenses mêmes.\par
On ne peut faire de grandes punitions, et par conséquent de grands changements, sans mettre dans les mains de quelques citoyens un grand pouvoir. Il vaut donc mieux, dans ce cas, pardonner beaucoup que punir beaucoup ; exiler peu qu’exiler beaucoup ; laisser les biens que multiplier les confiscations. Sous prétexte de la vengeance de la république, on établirait la tyrannie des vengeurs. Il n’est pas question de détruire celui qui domine, mais la domination. Il faut rentrer le plus tôt que l’on peut dans ce train ordinaire du gouvernement, où les lois protègent tout, et ne s’arment contre personne.\par
Les Grecs ne mirent point de bornes aux vengeances qu’ils prirent des tyrans ou de ceux qu’ils soupçonnèrent de l’être. Ils firent mourir les enfants\footnote{Denys d’Halicarnasse, {\itshape Antiquités romaines}, liv. VIII.}, quelquefois cinq des plus proches parents\footnote{{\itshape Tyranno occiso, quinque ejus proximos cognatione, magistratus necato} ; Cicéron, {\itshape De Inventione}, liv. II.}. Ils chassèrent une infinité de familles. Leurs républiques en furent ébranlées ; l’exil ou le retour des exilés furent toujours des époques qui marquèrent le changement de la constitution.\par
Les Romains furent plus sages. Lorsque Cas{\itshape sius} fut condamné pour avoir aspiré à la tyrannie, on mit en question si l’on ferait mourir ses enfants : ils ne furent condamnés à aucune peine. « Ceux qui ont voulu, dit Denys d’Halicarnasse\footnote{Liv. VIII, p. 547.}, changer cette loi à la fin de la guerre des Marses et de la guerre civile, et exclure des charges les enfants des proscrits par Sylla, sont bien criminels. »\par
On voit dans les guerres de Marius et de Sylla jusqu’à quel point les âmes chez les Romains s’étaient peu à peu dépravées. Des choses si funestes firent croire qu’on ne les reverrait plus. Mais sous les triumvirs on voulut être plus cruel et le paraître moins : on est désolé de voir les sophismes qu’employa la cruauté. On trouve dans Appien\footnote{{\itshape Des guerres civiles}, liv. IV} la formule des proscriptions. Vous diriez qu’on n’y a d’autre objet que le bien de la république, tant on y parle de sang-froid, tant on y montre d’avantages, tant les moyens que l’on prend sont préférables à d’autres, tant les riches seront en sûreté, tant le bas peuple sera tranquille, tant on craint de mettre en danger la vie des citoyens, tant on veut apaiser les soldats, tant enfin on sera heureux\footnote{{\itshape Quod felix faustumque sit.}}.\par
Rome était inondée de sang quand Lépidus triompha de l’Espagne, et, par une absurdité sans exemple, sous peine d’être proscrit\footnote{{\itshape Sacris et epulis dent hunc diem : qui secus faxit, inter proscriptos esto.}}. Il ordonna de se réjouir.
\subsubsection[{Chapitre XIX. Comment on suspend l’usage de la liberté dans la république}]{Chapitre XIX. Comment on suspend l’usage de la liberté dans la république}
\noindent Il y a, dans les États où l’on fait le plus de cas de la liberté, des lois qui la violent contre un seul, pour la garder à tous. Tels sont, en Angleterre, les bills appelés {\itshape d’atteindre}\footnote{Il ne suffit pas, dans les tribunaux du royaume, qu’il y ait une preuve telle que les juges soient convaincus : il faut encore que cette preuve soit formelle, c’est-à-dire légale : et la loi demande qu’il y ait deux témoins contre l’accusé ; une autre preuve ne suffirait pas. Or, si un homme, présumé coupable de ce qu’on appelle haut crime, avait trouvé moyen d’écarter les témoins, de sorte qu’il fût impossible de le faire condamner par la loi, on pourrait porter contre lui un {\itshape bill} particulier {\itshape d’atteindre ;} c’est-à-dire faire une loi singulière sur sa personne. On y procède comme pour tous les autres {\itshape bills : il} faut qu’il passe dans deux chambres, et que le roi y donne son consentement, sans quoi il n’y a point de {\itshape bill}, c’est-à-dire de jugement. L’accusé peut faire parler ses avocats contre le {\itshape bill}, et on peut parler dans la chambre pour le {\itshape bill.}}. Ils se rapportent à ces lois d’Athènes qui statuaient contre un particulier\footnote{{\itshape Legem de singulari aliquo ne rogato, nisi sex millibus ita visum.} Ex Andocide, {\itshape de mysteriis}. C’est l’ostracisme.}, pourvu qu’elles fussent faites par le suffrage de six mille citoyens. Ils se rapportent à ces lois qu’on faisait à Rome contre des citoyens particuliers, et qu’on appelait {\itshape privilèges}\footnote{{\itshape De privis hominibus latae.} Cicéron, {\itshape De Leg.}, liv. III.}. Elles ne se faisaient que dans les grands États du peuple. Mais, de quelque manière que le peuple les donne, Cicéron veut qu’on les abolisse, parce que la force de la loi ne consiste qu’en ce qu’elle statue sur tout le monde\footnote{{\itshape Scitum est jussum in omnes.} Cicéron, {\itshape ibid.}}. J’avoue pourtant que l’usage des peuples les plus libres qui aient jamais été sur la terre me fait croire qu’il y a des cas où il faut mettre, pour un moment, un voile sur la liberté, comme l’on cache les statues des dieux.
\subsubsection[{Chapitre XX. Des lois favorables à la liberté du citoyen dans la république}]{Chapitre XX. Des lois favorables à la liberté du citoyen dans la république}
\noindent Il arrive souvent dans les États populaires, que les accusations sont publiques, et qu’il est permis à tout homme d’accuser qui il veut. Cela a fait établir des lois propres à défendre l’innocence des citoyens. À Athènes, l’accusateur qui n’avait point pour lui la cinquième partie des suffrages, payait une amende de mille dragmes. Eschine, qui avait accusé Ctésiphon, y fut condamné\footnote{Voyez Philostrate, liv. I, {\itshape Vie des sophistes, Vie d’Eschine}. Voyez aussi Plutarque et Photius.}. À Rome, l’injuste accusateur était noté d’infamie\footnote{Par la loi Remnia.} ; on lui imprimait la lettre K sur le front. On donnait des gardes à l’accusateur, pour qu’il fût hors d’état de corrompre les juges ou les témoins\footnote{Plutarque, au traité : {\itshape Comment on pourrait recevoir de l’utilité de ses ennemis}.}.\par
J’ai déjà parlé de cette loi athénienne et romaine qui permettait à l’accusé de se retirer avant le jugement.
\subsubsection[{Chapitre XXI. De la cruauté des lois envers les débiteurs dans la République}]{Chapitre XXI. De la cruauté des lois envers les débiteurs dans la République}
\noindent Un citoyen s’est déjà donné une assez grande supériorité sur un citoyen, en lui prêtant un argent que celui-ci n’a emprunté que pour s’en défaire, et que par conséquent il n’a plus. Que sera-ce dans une république, si les lois augmentent cette servitude encore davantage ?\par
À Athènes et à Rome\footnote{Plusieurs vendaient leurs enfants pour payer leurs dettes. Plutarque, {\itshape Vie de Solon}.}, il fut d’abord permis de vendre les débiteurs qui n’étaient pas en état de payer. Solon corrigea cet usage à Athènes\footnote{{\itshape Ibid.}} : il ordonna que personne ne serait obligé par corps pour dettes civiles. Mais les décemvirs \footnote{Il paraît par l’histoire que cet usage était établi chez les Romains avant la loi des Douze Tables. Tite-Live, première Décade, liv. II.}ne réformèrent pas de même l’usage de Rome ; et, quoiqu’ils eussent devant les yeux le règlement de Solon, ils ne voulurent pas le suivre. Ce n’est pas le seul endroit de la loi des Douze Tables où l’on voit le dessein des décemvirs de choquer l’esprit de la démocratie.\par
Ces lois cruelles contre les débiteurs mirent bien des fois en danger la république romaine. Un homme couvert de plaies s’échappa de la maison de son créancier et parut dans la place\footnote{Denys d’Halicarnasse, {\itshape Antiquités romaines}, liv. VI.}. Le peuple s’émut à ce spectacle. D’autres citoyens, que leurs créanciers n’osaient plus retenir, sortirent de leurs cachots. On leur fit des promesses ; on y manqua : le peuple se retira sur le Mont-Sacré. Il n’obtint pas l’abrogation de ces lois, mais un magistrat pour le défendre. On sortait de l’anarchie, on pensa tomber dans la tyrannie. Manlius, pour se rendre populaire, allait retirer des mains des créanciers les citoyens qu’ils avaient réduits en esclavage\footnote{Plutarque, {\itshape Vie de Furius Camillus}.}. On prévint les desseins de Manlius ; mais le mal restait toujours. Des lois particulières donnèrent aux débiteurs des facilités de payer\footnote{Voyez ci-dessous le chap. XXIV du liv. XXII.}, et l’an de Rome 428 les consuls portèrent une loi\footnote{Cent vingt ans après la loi des Douze Tables. {\itshape Eo anno plebi romanae, velut aliud initium libertatis, factum est quod necti desierunt.} Tite-Live, liv. VIII.} qui ôta aux créanciers le droit de tenir les débiteurs en servitude dans leurs maisons\footnote{{\itshape Bona debitoris, non corpus obnoxium esset. Ibid.}}. Un usurier nommé Papirius avait voulu corrompre la pudicité d’un jeune homme nommé Publius, qu’il tenait dans les fers. Le crime de Sextus donna à Rome la liberté politique ; celui de Papirius y donna la liberté civile.\par
Ce fut le destin de cette ville, que des crimes nouveaux y confirmèrent la liberté que des crimes anciens lui avaient procurée. L’attentat d’Appius sur Virginie remit le peuple dans cette horreur contre les tyrans que lui avait donnée le malheur de Lucrèce. Trente-sept ans\footnote{L’an de Rome 465.} » après le crime de l’infâme Papirius, un crime pareil\footnote{Celui de Plautius, qui attenta contre la pudicité de Veturius. Valère Maxime, liv. VI, art. IX. On ne doit point confondre ces deux événements : ce ne sont ni les mêmes personnes, ni les mêmes temps.} fit que le peuple se retira sur le Janicule\footnote{Voyez un fragment de Denys d’Halicarnasse, dans l’{\itshape Extrait des vertus et des vices}, l’{\itshape Epitome} de Tite-Live, liv. XI, et Freinshemius, liv. XI.}, et que la loi faite pour la sûreté des débiteurs reprit une nouvelle force.\par
Depuis ce temps, les créanciers furent plutôt poursuivis par les débiteurs pour avoir violé les lois faites contre les usures, que ceux-ci ne le furent pour ne les avoir pas payées.
\subsubsection[{Chapitre XXII. Des choses qui attaquent la liberté dans la monarchie}]{Chapitre XXII. Des choses qui attaquent la liberté dans la monarchie}
\noindent La chose du monde la plus inutile au prince a souvent affaibli la liberté dans les monarchies : les commissaires nommés quelquefois pour juger un particulier.\par
Le prince tire si peu d’utilité des commissaires, qu’il ne vaut pas la peine qu’il change l’ordre des choses pour cela. Il est moralement sûr qu’il a plus l’esprit de probité et de justice que ses commissaires, qui se croient toujours assez justifiés par ses ordres, par un obscur intérêt de l’État, par le choix qu’on a fait d’eux, et par leurs craintes mêmes.\par
Sous Henri VIII, lorsqu’on faisait le procès à un pair, on le faisait juger par des commissaires tirés de la chambre des pairs : avec cette méthode on fit mourir tous les pairs qu’on voulut.
\subsubsection[{Chapitre XXIII. Des espions dans la monarchie}]{Chapitre XXIII. Des espions dans la monarchie}
\noindent Faut-il des espions dans la monarchie ? Ce n’est pas la pratique ordinaire des bons princes. Quand un homme est fidèle aux lois, il a satisfait à ce qu’il doit au prince. il faut au moins qu’il ait sa maison pour asile, et le reste de sa conduite en sûreté. L’espionnage serait peut-être tolérable s’il pouvait être exercé par d’honnêtes gens ; mais l’infamie nécessaire de la personne peut faire juger de l’infamie de la chose. Un prince doit agir avec ses sujets avec candeur, avec franchise, avec confiance. Celui qui a tant d’inquiétudes, de soupçons et de craintes, est un acteur qui est embarrassé à jouer son rôle. Quand il voit qu’en général les lois sont dans leur force, et qu’elles sont respectées, il peut se juger en sûreté. L’allure générale lui répond de celle de tous les particuliers. Qu’il n’ait aucune crainte, il ne saurait croire combien on est porté à l’aimer. Eh ! pourquoi ne l’aimerait-on pas ? Il est la source de presque tout le bien qui se fait ; et quasi toutes les punitions sont sur le compte des lois. Il ne se montre jamais au peuple qu’avec un visage serein : sa gloire même se communique à nous, et sa puissance nous soutient. Une preuve qu’on l’aime, c’est que l’on a de la confiance en lui, et que, lorsqu’un ministre refuse, on s’imagine toujours que le prince aurait accordé. Même dans les calamités publiques, on n’accuse point sa personne ; on se plaint de ce qu’il ignore, ou de ce qu’il est obsédé par des gens corrompus. {\itshape Si le prince savait} ! dit le peuple. Ces paroles sont une espèce d’invocation, et une preuve de la confiance qu’on a en lui.
\subsubsection[{Chapitre XXIV. Des lettres anonymes}]{Chapitre XXIV. Des lettres anonymes}
\noindent Les Tartares sont obligés de mettre leur nom sur leurs flèches, afin que l’on connaisse la main dont elles partent. Philippe de Macédoine ayant été blessé au siège d’une ville, on trouva sur le javelot : {\itshape Aster a porté ce coup mortel à Philippe}\footnote{Plutarque, {\itshape Œuvres morales, Collat. de quelques histoires romaines et grecques}, t. II, p. 487.}. Si ceux qui accusent un homme le faisaient en vue du bien public, ils ne l’accuseraient pas devant le prince, qui peut être aisément prévenu, mais devant les magistrats, qui ont des règles qui ne sont formidables qu’aux calomniateurs. Que s’ils ne veulent pas laisser les lois entre eux et l’accusé, c’est une preuve qu’ils ont sujet de les craindre ; et la moindre peine qu’on puisse leur infliger, c’est de ne les point croire. On ne peut y faire d’attention que dans les cas qui ne sauraient souffrir les lenteurs de la justice ordinaire, et où il s’agit du salut du prince. Pour lors, on peut croire que celui qui accuse a fait un effort qui a délié sa langue, et l’a fait parler. Mais, dans les autres cas, il faut dire avec l’empereur Constance : « Nous ne saurions soupçonner celui à qui il a manqué un accusateur, lorsqu’il ne lui manquait pas un ennemi\footnote{Leg. 6, code Théodosien {\itshape de famosis libellis}.}. »
\subsubsection[{Chapitre XXV. De la manière de gouverner dans la monarchie}]{Chapitre XXV. De la manière de gouverner dans la monarchie}
\noindent L’autorité royale est un grand ressort qui doit se mouvoir aisément et sans bruit. Les Chinois vantent un de leurs empereurs, qui gouverna, disent-ils, comme le ciel, c’est-à-dire, par son exemple.\par
Il y a des cas où la puissance doit agir dans toute son étendue ; il y en a où elle doit agir par ses limites. Le sublime de l’administration est de bien connaître quelle est la partie du pouvoir, grande ou petite, que l’on doit employer dans les diverses circonstances.\par
Dans nos monarchies, toute la félicité consiste dans l’opinion que le peuple a de la douceur du gouvernement. Un ministre mal habile veut toujours vous avertir que vous êtes esclaves. Mais, si cela était, il devrait chercher à le faire ignorer. Il ne sait vous dire ou vous écrire, si ce n’est que le prince est fâché ; qu’il est surpris ; qu’il mettra ordre. Il y a une certaine facilité dans le commandement : il faut que le prince encourage, et que ce soient les lois qui menacent\footnote{Nerva, dit Tacite, augmenta la facilité de l’empire.}.
\subsubsection[{Chapitre XXVI. Que, dans la monarchie, le prince doit être accessible}]{Chapitre XXVI. Que, dans la monarchie, le prince doit être accessible}
\noindent Cela se sentira beaucoup mieux par les contrastes. « Le czar Pierre I\textsuperscript{er}, dit le sieur Perry\footnote{{\itshape État de la grande Russie}, p. 173, édit. de Paris, 1717.}, a fait une nouvelle ordonnance qui défend de lui présenter de requête qu’après en avoir présenté deux à ses officiers. On peut, en cas de déni de justice, lui présenter la troisième ; mais celui qui a tort, doit perdre la vie. Personne depuis n’a adressé de requête au czar. »
\subsubsection[{Chapitre XXVII. Des mœurs du Monarque}]{Chapitre XXVII. Des mœurs du Monarque}
\noindent Les mœurs du prince contribuent autant à la liberté que les lois : il peut, comme elles, faire des hommes des bêtes, et des bêtes faire des hommes. S’il aime les âmes libres, il aura des sujets ; s’il aime les âmes basses, il aura des esclaves. Veut-il savoir le grand art de régner ? Qu’il approche de lui l’honneur et la vertu, qu’il appelle le mérite personnel. Il peut même jeter quelquefois les yeux sur les talents. Qu’il ne craigne point ces rivaux qu’on appelle les hommes de mérite ; il est leur égal, dès qu’il les aime. Qu’il gagne le cœur, mais qu’il ne captive point l’esprit. Qu’il se rende populaire. Il doit être flatté de l’amour du moindre de ses sujets ; ce sont toujours des hommes. Le peuple demande si peu d’égards, qu’il est juste de les lui accorder : l’infinie distance, qui est entre le souverain et lui, empêche bien qu’il ne le gêne. Qu’exorable à la prière, il soit ferme contre les demandes ; et qu’il sache que son peuple jouit de ses refus, et ses courtisans de ses grâces.
\subsubsection[{Chapitre XXVIII. Des égards que les monarques doivent à leurs sujets}]{Chapitre XXVIII. Des égards que les monarques doivent à leurs sujets}
\noindent Il faut qu’ils soient extrêmement retenus sur la raillerie. Elle flatte lorsqu’elle est modérée, parce qu’elle donne les moyens d’entrer dans la familiarité ; mais une raillerie piquante leur est bien moins permise qu’au dernier de leurs sujets, parce qu’ils sont les seuls qui blessent toujours mortellement.\par
Encore moins doivent-ils faire à un de leurs sujets une insulte marquée : ils sont établis pour pardonner, pour punir ; jamais pour insulter.\par
Lorsqu’ils insultent leurs sujets, ils les traitent bien plus cruellement que ne traite les siens le Turc ou le Moscovite. Quand ces derniers insultent, ils humilient et ne déshonorent point ; mais pour eux, ils humilient et déshonorent.\par
Tel est le préjugé des Asiatiques qu’ils regardent un affront fait par le prince comme l’effet d’une bonté Paternelle, et telle est notre manière de penser, que nous joignons au cruel sentiment de l’affront le désespoir de ne pouvoir nous en laver jamais.\par
Ils doivent être charmés d’avoir des sujets à qui l’honneur est plus cher que la vie, et n’est pas moins un motif de fidélité que de courage.\par
On peut se souvenir des malheurs arrivés aux princes pour avoir insulté leurs sujets ; des vengeances de Chéréas, de l’eunuque Narsès, et du comte Julien ; enfin, de la duchesse de Montpensier, qui, outrée contre Henri III, qui avait révélé quelqu’un de ses défauts secrets, le troubla pendant toute sa vie.
\subsubsection[{Chapitre XXIX. Des lois civiles propres à mettre un peu de liberté dans le gouvernement despotique}]{Chapitre XXIX. Des lois civiles propres à mettre un peu de liberté dans le gouvernement despotique}
\noindent Quoique le gouvernement despotique, dans sa nature, soit partout le même, cependant des circonstances, une opinion de religion, un préjugé, des exemples reçus, un tour d’esprit, des manières, des mœurs, peuvent y mettre des différences considérables.\par
Il est bon que de certaines idées s’y soient établies. Ainsi, à la Chine, le prince est regardé comme le père du peuple ; et, dans les commencements de l’empire des Arabes, le prince en était le prédicateur\footnote{Les califes.}.\par
Il convient qu’il y ait quelque livre sacré qui serve de règle, comme l’Alcoran chez les Arabes, les livres de Zoroastre chez les Perses, le {\itshape Védam} chez les Indiens, les livres classiques chez les Chinois. Le code religieux supplée au code civil, et fixe l’arbitraire.\par
Il n’est pas mal que, dans les cas douteux, les juges consultent les ministres de la religion\footnote{{\itshape Histoire des Tatars}, III\textsuperscript{e} partie, p. 277, dans les remarques.}. Aussi, en Turquie, les cadis interrogent-ils les mollachs. Que si le cas mérite la mort, il peut être convenable que le juge particulier, s’il y en a, prenne l’avis du gouverneur, afin que le pouvoir civil et l’ecclésiastique soient encore tempérés par l’autorité politique.
\subsubsection[{Chapitre XXX. Continuation du même sujet}]{Chapitre XXX. Continuation du même sujet}
\noindent C’est la fureur despotique qui a établi que la disgrâce du père entraînerait celle des enfants et des femmes. Ils sont déjà malheureux sans être criminels ; et, d’ailleurs, il faut que le prince laisse entre l’accusé et lui des suppliants pour adoucir son courroux, ou pour éclairer sa justice.\par
C’est une bonne coutume des Maldives\footnote{Voyez François Pyrard.}, que lorsqu’un seigneur est disgracié, il va tous les jours faire sa cour au roi, jusqu’à ce qu’il rentre en grâce ; sa présence désarme le courroux du prince.\par
Il y a des États despotiques\footnote{Comme aujourd’hui en Perse, au rapport de M. Chardin. Cet usage est bien ancien. « On mit Cavade, dit Procope, dans le château de l’oubli. Il y a une loi qui défend de parler de ceux qui y sont enfermés, et même de prononcer leur nom. »} où l’on pense que de parler à un prince pour un disgracié, c’est manquer au respect qui lui est dû. Ces princes semblent faire tous leurs efforts pour se priver de la vertu de clémence.\par
Arcadius et Honorius, dans la loi\footnote{La loi, au Code {\itshape ad leg. Jul. maj.}} dont j’ai tant parlé\footnote{Au chapitre VIII de ce livre.}, déclarent qu’ils ne feront point de grâce à ceux qui oseront les supplier pour les coupables\footnote{Frédéric copia cette loi dans les {\itshape Constitutions de Naples}, liv.{\itshape  I.}}. Cette loi était bien mauvaise, puisqu’elle est mauvaise dans le despotisme même.\par
La coutume de Perse qui permet à qui veut de sortir du royaume est très bonne ; et, quoique l’usage contraire ait tiré son origine du despotisme, où l’on a regardé les sujets comme des esclaves\footnote{Dans les monarchies, il y a ordinairement une loi qui défend à ceux qui ont des emplois publics de sortir du royaume sans la permission du prince. Cette loi doit être encore établie dans les républiques. Mais dans celles qui ont des institutions singulières, la défense doit être générale, pour qu’on n’y rapporte pas les mœurs étrangères.} et ceux qui sortent comme des esclaves fugitifs, cependant la pratique de Perse est très bonne pour le despotisme, où la crainte de la fuite ou de la retraite des redevables, arrête ou modère les persécutions des bachas et des exacteurs.
\subsection[{Livre treizième. Des rapports que la levée des tributs et la grandeur des revenus publics ont avec la liberté}]{Livre treizième. Des rapports que la levée des tributs et la grandeur des revenus publics ont avec la liberté}
\subsubsection[{Chapitre I. Des revenus de l’état}]{Chapitre I. Des revenus de l’état}
\noindent Les revenus de l’État sont une portion que chaque citoyen donne de son bien pour avoir la sûreté de l’autre, ou pour en jouir agréablement.\par
Pour bien fixer ces revenus, il faut avoir égard et aux nécessités de l’État, et aux nécessités des citoyens. Il ne faut point prendre au peuple sur ses besoins réels, pour des besoins de l’État imaginaires.\par
Les besoins imaginaires sont ce que demandent les passions et les faiblesses de ceux qui gouvernent, le charme d’un projet extraordinaire, l’envie malade d’une vaine gloire, et une certaine impuissance d’esprit contre les fantaisies. Souvent ceux qui, avec un esprit inquiet, étaient sous le prince à la tête des affaires, ont pensé que les besoins de l’État étaient les besoins de leurs petites âmes.\par
Il n’y a rien que la sagesse et la prudence doivent plus régler que cette portion qu’on ôte et cette portion qu’on laisse aux sujets.\par
Ce n’est point à ce que le peuple peut donner qu’il faut mesurer les revenus publics, mais à ce qu’il doit donner ; et si on les mesure à ce qu’il peut donner, il faut que ce soit du moins à ce qu’il peut toujours donner.
\subsubsection[{Chapitre II. Que c’est mal raisonner de dire que la grandeur des tributs soit bonne par elle-même}]{Chapitre II. Que c’est mal raisonner de dire que la grandeur des tributs soit bonne par elle-même}
\noindent On a vu, dans de certaines monarchies, que de petits pays exempts de tributs étaient aussi misérables que les lieux qui, tout autour, en étaient accablés. La principale raison est que le petit État entouré ne peut avoir d’industrie, d’arts, ni de manufactures, parce qu’à cet égard il est gêné de mille manières par le grand État dans lequel il est enclavé. Le grand État qui l’entoure a l’industrie, les manufactures et les arts ; et il fait des règlements qui lui en procurent tous les avantages. Le petit État devient donc nécessairement pauvre, quelque peu d’impôts qu’on y lève.\par
On a pourtant conclu de la pauvreté de ces petits pays que, pour que le peuple fût industrieux, il fallait des charges pesantes. On aurait mieux fait d’en conclure qu’il n’en faut pas. Ce sont tous les misérables des environs qui se retirent dans ces lieux-là pour ne rien faire : déjà découragés par l’accablement du travail, ils font consister toute leur félicité dans leur paresse.\par
L’effet des richesses d’un pays, c’est de mettre de l’ambition dans tous les cœurs. L’effet de la pauvreté est d’y faire naître le désespoir. La première s’irrite par le travail ; l’autre se console par la paresse.\par
La nature est juste envers les hommes ; elle les récompense de leurs peines ; elle les rend laborieux, parce qu’à de plus grands travaux elle attache de plus grandes récompenses. Mais, si un pouvoir arbitraire ôte les récompenses de la nature, on reprend le dégoût pour le travail, et l’inaction paraît être le seul bien.
\subsubsection[{Chapitre III. Des tributs dans les pays où une partie du peuple est esclave de la glèbe}]{Chapitre III. Des tributs dans les pays où une partie du peuple est esclave de la glèbe}
\noindent L’esclavage de la glèbe s’établit quelquefois après une conquête. Dans ce cas, l’esclave qui cultive doit être le colon partiaire du maître. Il n’y a qu’une société de perte et de gain qui puisse réconcilier ceux qui sont destinés à travailler, avec ceux qui sont destinés à jouir.
\subsubsection[{Chapitre IV. D’une République en cas pareil}]{Chapitre IV. D’une République en cas pareil}
\noindent Lorsqu’une république a réduit une nation à cultiver les terres pour elle, on n’y doit point souffrir que le citoyen puisse augmenter le tribut de l’esclave. On ne le permettait point à Lacédémone : on pensait que les Élotes\footnote{Plutarque.} cultiveraient mieux les terres lorsqu’ils sauraient que leur servitude n’augmenterait pas ; on croyait que les maîtres seraient meilleurs citoyens lorsqu’ils ne désireraient que ce qu’ils avaient coutume d’avoir.
\subsubsection[{Chapitre V. D’une Monarchie en cas pareil}]{Chapitre V. D’une Monarchie en cas pareil}
\noindent Lorsque, dans une monarchie, la noblesse fait cultiver les terres à son profit par le peuple conquis, il faut encore que la redevance ne puisse augmenter\footnote{C’est ce qui fit faire à Charlemagne ses belles institutions là-dessus. Voyez le liv. V des {\itshape Capitulaires}, art. 303.}. De plus, il est bon que le prince se contente de son domaine et du service militaire.\par
Mais s’il veut lever des tributs en argent sur les esclaves de sa noblesse, il faut que le seigneur soit garant\footnote{Cela se pratique ainsi en Allemagne.} du tribut, qu’il le paie pour les esclaves, et le reprenne sur eux ; et si l’on ne suit pas cette règle, le seigneur et ceux qui lèvent les revenus du prince vexeront l’esclave tour à tour, et le reprendront l’un après l’autre, jusqu’à ce qu’il périsse de misère ou fuie dans les bois.
\subsubsection[{Chapitre VI. D’un État despotique en cas pareil}]{Chapitre VI. D’un État despotique en cas pareil}
\noindent Ce que je viens de dire est encore plus indispensable dans l’État despotique. Le seigneur qui peut, à tous les instants, être dépouillé de ses terres et de ses esclaves, n’est pas si porté à les conserver.\par
Pierre I\textsuperscript{er}, voulant prendre la pratique d’Allemagne et lever ses tributs en argent, fit un règlement très sage que l’on suit encore en Russie. Le gentilhomme lève la taxe sur les paysans, et la paie au czar. Si le nombre des paysans diminue, il paie tout de même ; si le nombre augmente, il ne paie pas davantage ; il est donc intéressé à ne point vexer ses paysans.
\subsubsection[{Chapitre VII. Des tributs dans les pays où l’esclavage de la glèbe n’est point établi}]{Chapitre VII. Des tributs dans les pays où l’esclavage de la glèbe n’est point établi}
\noindent Lorsque, dans un État, tous les particuliers sont citoyens, que chacun y possède par son domaine ce que le prince y possède par son empire, on peut mettre des impôts sur les personnes, sur les terres, ou sur les marchandises ; sur deux de ces choses, ou sur les trois ensemble.\par
Dans l’impôt de la personne, la proportion injuste serait celle qui suivrait exactement la proportion des biens. On avait divisé à Athènes\footnote{Pollux, liv. VIII, chap. X, art. 130.} les citoyens en quatre classes. Ceux qui retiraient de leurs biens cinq cents mesures de fruits, liquides ou secs, payaient au public un talent ; ceux qui en retiraient trois cents mesures devaient un demi-talent ; ceux qui avaient deux cents mesures payaient dix mines, ou la sixième partie d’un talent ; ceux de la quatrième classe ne donnaient rien. La taxe était juste, quoiqu’elle ne fût point proportionnelle : si elle ne suivait pas la proportion des biens, elle suivait la proportion des besoins. On jugea que chacun avait un {\itshape nécessaire physique} égal ; que ce nécessaire physique ne devait point être taxé ; que l’utile venait ensuite, et qu’il devait être taxé, mais moins que le superflu ; que la grandeur de la taxe sur le superflu empêchait le superflu.\par
Dans la taxe sur les terres, on fait des rôles où l’on met les diverses classes des fonds. Mais il est très difficile de connaître ces différences, et encore plus de trouver des gens qui ne soient point intéressés à les méconnaître. Il y a donc là deux sortes d’injustices : l’injustice de l’homme et l’injustice de la chose. Mais si, en général, la taxe n’est point excessive, si on laisse au peuple un nécessaire abondant, ces injustices particulières ne seront rien. Que si, au contraire, on ne laisse au peuple que ce qu’il lui faut à la rigueur pour vivre, la moindre disproportion sera de la plus grande conséquence.\par
Que quelques citoyens ne paient pas assez, le mal n’est pas grand ; leur aisance revient toujours au public ; que quelques particuliers paient trop, leur ruine se tourne contre le public. Si l’État proportionne sa fortune à celle des particuliers, l’aisance des particuliers fera bientôt monter sa fortune. Tout dépend du moment : l’État commencera-t-il par appauvrir les sujets pour s’enrichir ? ou attendra-t-il que des sujets à leur aise l’enrichissent ? Aura-t-il le premier avantage, ou le second ? Commencera-t-il par être riche, ou finira-t-il par l’être ?\par
Les droits sur les marchandises sont ceux que les peuples sentent le moins, parce qu’on ne leur fait pas une demande formelle. Ils peuvent être si sagement ménagés, que le peuple ignorera presque qu’il les paie. Pour cela, il est d’une grande conséquence que ce soit celui qui vend la marchandise qui paie le droit. Il sait bien qu’il ne paie pas pour lui ; et l’acheteur, qui dans le fond le paie, le confond avec le prix. Quelques auteurs ont dit que Néron avait ôté le droit du vingt-cinquième des esclaves qui se vendaient\footnote{{\itshape Vectigal quoque quintae et vicesimae venalium mancipiorum remissum specie magis quam vi ; quia cum venditor pendere juberetur, in partem pretii emptoribus accrescebat.} Tacite, {\itshape Annales}, liv. XIII.} ; il n’avait pourtant fait qu’ordonner que ce serait le vendeur qui le paierait, au lieu de l’acheteur : ce règlement, qui laissait tout l’impôt, parut l’ôter.\par
Il y a deux royaumes en Europe où l’on a mis des impôts très forts sur les boissons : dans l’un, le brasseur seul paie le droit ; dans l’autre, il est levé indifféremment sur tous les sujets qui consomment. Dans le premier, personne ne sent la rigueur de l’impôt ; dans le second, il est regardé comme onéreux : dans celui-là, le citoyen ne sent que la liberté qu’il a de ne pas payer ; dans celui-ci, il ne sent que la nécessité qui l’y oblige.\par
D’ailleurs, pour que le citoyen paie, il faut des recherches perpétuelles dans sa maison. Rien West plus contraire à la liberté ; et ceux qui établissent ces sortes d’impôts n’ont pas le bonheur d’avoir à cet égard rencontré la meilleure sorte d’administration.
\subsubsection[{Chapitre VIII. Comment on conserve l’illusion}]{Chapitre VIII. Comment on conserve l’illusion}
\noindent Pour que le prix de la chose et le droit puissent se confondre dans la tête de celui qui paie, il faut qu’il y ait quelque rapport entre la marchandise et l’impôt ; et que, sur une denrée de peu de valeur, on ne mette pas un droit excessif. Il y a des pays où le droit excède de dix-sept fois la valeur de la marchandise. Pour lors, le prince ôte l’illusion à ses sujets ; ils voient qu’ils sont conduits d’une manière qui n’est pas raisonnable ; ce qui leur fait sentir leur servitude au dernier point.\par
D’ailleurs, pour que le prince puisse lever un droit si disproportionné à la valeur de la chose, il faut qu’il vende lui-même la marchandise, et que le peuple ne puisse l’aller acheter ailleurs ; ce qui est sujet à mille inconvénients.\par
La fraude étant dans ce cas très lucrative, la peine naturelle, celle que la raison demande, qui est la confiscation de la marchandise, devient incapable de l’arrêter ; d’autant plus que cette marchandise est, pour l’ordinaire, d’un prix très vil. Il faut donc avoir recours à des peines extravagantes, et pareilles à celles que l’on inflige pour les plus grands crimes. Toute la proportion des peines est ôtée. Des gens, qu’on ne saurait regarder comme des hommes méchants, sont punis comme des scélérats ; ce qui est la chose du monde la plus contraire à l’esprit du gouvernement modéré.\par
J’ajoute que plus on met le peuple en occasion de frauder le traitant, plus on enrichit celui-ci, et on appauvrit celui-là. Pour arrêter la fraude, il faut donner au traitant des moyens de vexations extraordinaires, et tout est perdu.
\subsubsection[{Chapitre IX. D’une mauvaise sorte d’impôt}]{Chapitre IX. D’une mauvaise sorte d’impôt}
\noindent Nous parlerons, en passant, d’un impôt établi dans quelques États sur les diverses clauses des contrats civils. Il faut, pour se défendre du traitant, de grandes connaissances, ces choses étant sujettes à des discussions subtiles. Pour lors, le traitant, interprète des règlements du prince, exerce un pouvoir arbitraire sur les fortunes. L’expérience a fait voir qu’un impôt sur le papier sur lequel le contrat doit s’écrire, vaudrait beaucoup mieux.
\subsubsection[{Chapitre X. Que la grandeur des tributs dépend de la nature du gouvernement}]{Chapitre X. Que la grandeur des tributs dépend de la nature du gouvernement}
\noindent Les tributs doivent être très légers dans le gouvernement despotique. Sans cela, qui est-ce qui voudrait prendre la peine d’y cultiver les terres ? Et de plus, comment payer de gros tributs dans un gouvernement qui ne supplée par rien à ce que le sujet a donné ?\par
Dans le pouvoir étonnant du prince, et l’étrange faiblesse du peuple, il faut qu’il ne puisse y avoir d’équivoque sur rien. Les tributs doivent être si faciles à percevoir, et si clairement établis, qu’ils ne puissent être augmentés ni diminués par ceux qui les lèvent. Une portion dans les fruits de la terre, une taxe par tête, un tribut de tant pour cent sur les marchandises, sont les seuls convenables.\par
Il est bon, dans le gouvernement despotique, que les marchands aient une sauvegarde personnelle, et que l’usage les fasse respecter : sans cela, ils seraient trop faibles dans les discussions qu’ils pourraient avoir avec les officiers du prince.
\subsubsection[{Chapitre XI. Des peines fiscales}]{Chapitre XI. Des peines fiscales}
\noindent C’est une chose particulière aux peines fiscales, que, contre la pratique générale, elles sont plus sévères en Europe qu’en Asie. En Europe, on confisque les marchandises, quelquefois même les vaisseaux et les voitures ; en Asie, on ne fait ni l’un ni l’autre. C’est qu’en Europe le marchand a des juges qui peuvent le garantir de l’oppression ; en Asie, les juges despotiques seraient eux-mêmes les oppresseurs. Que ferait le marchand contre un bacha qui aurait résolu de confisquer ses marchandises ?\par
C’est la vexation qui se surmonte elle-même, et se voit contrainte à une certaine douceur. En Turquie, on ne lève qu’un seul droit d’entrée ; après quoi, tout le pays est ouvert aux marchands. Les déclarations fausses n’emportent ni confiscation ni augmentation de droits. On n’ouvre point\footnote{Du Halde, t. II, p. 37.}, à la Chine, les ballots des gens qui ne sont pas marchands. La fraude, chez le Mogol, n’est point punie par la confiscation, mais par le doublement du droit. Les princes\footnote{{\itshape Histoire des Tatars}, III\textsuperscript{e} partie, p. 290.} tartares, qui habitent des villes dans l’Asie, ne lèvent presque rien sur les marchandises qui passent. Que si, au Japon, le crime de fraude dans le commerce est un crime capital, c’est qu’on a des raisons pour défendre toute communication avec les étrangers ; et que la fraude\footnote{Voulant avoir un commerce avec les étrangers, sans se communiquer avec eux, ils ont choisi deux nations : la hollandaise, pour le commerce de l’Europe, et la chinoise, pour celui de l’Asie. Ils tiennent dans une espèce de prison les facteurs et les matelots, et les gênent jusqu’à perdre patience.} y est plutôt une contravention aux lois faites pour la sûreté de l’État, qu’à des lois de commerce.
\subsubsection[{Chapitre XII. Rapport de la grandeur des tributs avec la liberté}]{Chapitre XII. Rapport de la grandeur des tributs avec la liberté}
\noindent Règle générale : on peut lever des tributs plus forts, à proportion de la libellé des sujets ; et l’on est forcé de les modérer, à mesure que la servitude augmente. Cela a toujours été, et cela sera toujours. C’est une règle tirée de la nature, qui ne varie point : on la trouve par tous les pays, en Angleterre, en Hollande et dans tous les États où la liberté va se dégradant jusqu’en Turquie. La Suisse semble y déroger, parce qu’on n’y paie point de tributs : mais on en sait la raison particulière, et même elle confirme ce que je dis. Dans ces montagnes stériles, les vivres sont si chers, et le pays est si peuplé, qu’un Suisse paie quatre fois plus à la nature qu’un Turc ne paie au sultan.\par
Un peuple dominateur, tel qu’étaient les Athéniens et les Romains, peut s’affranchir de tout impôt, parce qu’il règne sur des nations sujettes. Il ne paie pas pour lors à proportion de sa liberté ; parce qu’à cet égard il n’est pas un peuple, mais un monarque.\par
Mais la règle générale reste toujours. Il y a, dans les États modérés, un dédommagement pour la pesanteur des tributs : c’est la liberté. Il y a dans les États\footnote{En Russie, les tributs sont médiocres : on les a augmentés depuis que le despotisme y est plus modéré. Voyez l’{\itshape Histoire des Tatars}, IX\textsuperscript{e} partie.} despotiques un équivalent pour la liberté : c’est la modicité des tributs.\par
Dans de certaines monarchies en Europe, on voit des provinces\footnote{Les pays d’États.} qui, par la nature de leur gouvernement politique, sont dans un meilleur état que les autres. On s’imagine toujours qu’elles ne paient pas assez parce que, par un effet de la bonté de leur gouvernement, elles pourraient payer davantage ; et il vient toujours dans l’esprit de leur ôter ce gouvernement même qui produit ce bien qui se communique, qui se répand au loin, et dont il vaudrait bien mieux jouir.
\subsubsection[{Chapitre XIII. Dans quels gouvernements les tributs sont susceptibles d’augmentation}]{Chapitre XIII. Dans quels gouvernements les tributs sont susceptibles d’augmentation}
\noindent On peut augmenter les tributs dans la plupart des républiques, parce que le citoyen, qui croit payer à lui-même, a la volonté de les payer, et en a ordinairement le pouvoir par l’effet de la nature du gouvernement.\par
Dans la monarchie, on peut augmenter les tributs, parce que la modération du gouvernement y peut procurer des richesses : c’est comme la récompense du prince, à cause du respect qu’il a pour les lois.\par
Dans l’État despotique, on ne peut pas les augmenter, parce qu’on ne peut pas augmenter la servitude extrême.
\subsubsection[{Chapitre XIV. Que la nature des tributs est relative au gouvernement}]{Chapitre XIV. Que la nature des tributs est relative au gouvernement}
\noindent L’impôt par tête est plus naturel à la servitude ; l’impôt sur les marchandises est plus naturel à la liberté, parce qu’il se rapporte d’une manière moins directe à la personne.\par
Il est naturel au gouvernement despotique que le prince ne donne point d’argent à sa milice ou aux gens de sa cour, mais qu’il leur distribue des terres, et par conséquent qu’on y lève peu de tributs. Que si le prince donne de l’argent, le tribut le plus naturel qu’il puisse lever est un tribut par tête. Ce tribut ne peut être que très modique, car, comme on n’y peut pas faire diverses classes de contribuables, à cause des abus qui en résulteraient, vu l’injustice et la violence du gouvernement, il faut nécessairement se régler sur le taux de ce que peuvent payer les plus misérables.\par
Le tribut naturel au gouvernement modéré est l’impôt sur les marchandises. Cet impôt étant réellement payé par l’acheteur, quoique le marchand l’avance, est un prêt que le marchand a déjà fait à l’acheteur : ainsi il faut regarder le négociant, et comme le débiteur général de l’État, et comme le créancier de tous les particuliers. Il avance à l’État le droit que l’acheteur lui paiera quelque jour ; et il a payé pour l’acheteur le droit qu’il a payé pour la marchandise. On sent donc que plus le gouvernement est modéré, que plus l’esprit de liberté règne, que plus les fortunes ont de sûreté, plus il est facile au marchand d’avancer à l’État et de prêter au particulier des droits considérables. En Angleterre, un marchand prête réellement à l’État cinquante ou soixante livres sterling à chaque tonneau de vin qu’il reçoit. Quel est le marchand qui oserait faire une chose de cette espèce dans un pays gouverné comme la Turquie ? Et quand il l’oserait faire, comment le pourrait-il, avec une fortune suspecte, incertaine, ruinée ?
\subsubsection[{Chapitre XV. Abus de la liberté}]{Chapitre XV. Abus de la liberté}
\noindent Ces grands avantages de la liberté ont fait que l’on a abusé de la liberté même. Parce que le gouvernement modéré a produit d’admirables effets, on a quitté cette modération ; parce qu’on a tiré de grands tributs, on en a voulu tirer d’excessifs ; et, méconnaissant la main de la liberté qui faisait ce présent, on s’est adressé à la servitude qui refuse tout.\par
La liberté a produit l’excès des tributs ; mais l’effet de ces tributs excessifs est de produire à leur tour la servitude ; et l’effet de la servitude, de produire la diminution des tributs.\par
Les monarques de l’Asie ne font guère d’édits que pour exempter chaque année de tributs quelque province de leur empire\footnote{C’est l’usage des empereurs de la Chine.} : les manifestations de leur volonté sont des bienfaits. Mais, en Europe, les édits des princes affligent même avant qu’on les ait vus, parce qu’ils y parlent toujours de leurs besoins, et jamais des nôtres.\par
D’une impardonnable nonchalance, que les ministres de ces pays-là tiennent du gouvernement et souvent du climat, les peuples tirent cet avantage qu’ils ne sont point sans cesse accablés par de nouvelles demandes. Les dépenses n’y augmentent point, parce qu’on n’y fait point de projets nouveaux ; et si, par hasard, on y en fait, ce sont des projets dont on voit la fin, et non des projets commencés. Ceux qui gouvernent l’État ne le tourmentent pas, parce qu’ils ne se tourmentent pas sans cesse eux-mêmes. Mais, pour nous, il est impossible que nous ayons jamais de règle dans nos finances, parce que nous savons toujours que nous ferons quelque chose, et jamais ce que nous ferons.\par
On n’appelle plus parmi nous un grand ministre celui qui est le sage dispensateur des revenus publics ; mais celui qui est homme d’industrie, et qui trouve ce qu’on appelle des expédients.\par
 \textbf{Chapitre XVI. {\itshape Des conquêtes des mahométans} }  \par
Ce furent ces tributs\footnote{Voyez, dans l’histoire, la grandeur, la bizarrerie et même la folie de ces tributs. Anastase en imagina un pour respirer l’air : {\itshape ut quisque pro haustu aeris penderet.}} excessifs qui donnèrent lieu à cette étrange facilité que trouvèrent les Mahométans dans leurs conquêtes. Les peuples, au lieu de cette suite continuelle de vexations que l’avarice subtile des empereurs avait imaginées, se virent soumis à un tribut simple, payé aisément, reçu de même : plus heureux d’obéir à une nation barbare qu’à un gouvernement corrompu, dans lequel ils souffraient tous les inconvénients d’une liberté qu’ils n’avaient plus, avec toutes les horreurs d’une servitude présente.
\subsubsection[{Chapitre XVII. De l’augmentation des troupes}]{Chapitre XVII. De l’augmentation des troupes}
\noindent Une maladie nouvelle s’est répandue en Europe ; elle a saisi nos princes, et leur fait entretenir un nombre désordonné de troupes. Elle a ses redoublements, et elle devient nécessairement contagieuse : car, sitôt qu’un État augmente ce qu’il appelle ses troupes, les autres soudain augmentent les leurs, de façon qu’on ne gagne rien par là que la ruine commune. Chaque monarque tient sur pied toutes les armées qu’il pourrait avoir si ses peuples étaient en danger d’être exterminés ; et on nomme paix cet état\footnote{Il est vrai que c’est cet état d’effort qui maintient principalement l’équilibre, parce qu’il éreinte les grandes puissances.} d’effort de tous contre tous. Aussi l’Europe est-elle si ruinée, que les particuliers qui seraient dans la situation où sont les trois puissances de cette partie du monde les plus opulentes, n’auraient pas de quoi vivre. Nous sommes pauvres avec les richesses et le commerce de tout l’univers ; et bientôt, à force d’avoir des soldats, nous n’aurons plus que des soldats, et nous serons comme des Tartares\footnote{Il ne faut, pour cela, que faire valoir la nouvelle invention des milices établies dans presque toute l’Europe, et les porter au même excès que l’on a fait les troupes réglées.}.\par
Les grands princes, non contents d’acheter les troupes des plus petits, cherchent de tous côtés à payer des alliances, c’est-à-dire, presque toujours à perdre leur argent.\par
La suite d’une telle situation est l’augmentation perpétuelle des tributs ; et, ce qui prévient tous les remèdes à venir, on ne compte plus sur les revenus, mais on fait la guerre avec son capital. Il n’est pas inouï de voir des États hypothéquer leurs fonds pendant la paix même, et employer, pour se ruiner, des moyens qu’ils appellent extraordinaires, et qui le sont si fort que le fils de famille le plus dérangé les imagine à peine.
\subsubsection[{Chapitre XVIII. De la remise des tributs}]{Chapitre XVIII. De la remise des tributs}
\noindent La maxime des grands empires d’Orient, de remettre les tributs aux provinces qui ont souffert, devrait bien être portée dans les États monarchiques. Il y en a bien où elle est établie ; mais elle accable plus que si elle n’y était pas, parce que le prince n’en levant ni plus ni moins, tout l’État devient solidaire. Pour soulager un village qui paie mal, on charge un autre qui paie mieux ; on ne rétablit point le premier, on détruit le second. Le peuple est désespéré entre la nécessité de payer de peur des exactions, et le danger de payer crainte des surcharges.\par
Un État bien gouverné doit mettre, pour le premier article de sa dépense, une somme réglée pour les cas fortuits. Il en est du public comme des particuliers, qui se ruinent lorsqu’ils dépensent exactement les revenus de leurs terres.\par
À l’égard de la solidité entre les habitants du même village, on a dit\footnote{Voyez le {\itshape Traité des finances des Romains}, chap. II, imprimé à Paris chez Briasson, en 1740.} qu’elle était raisonnable, parce qu’on pouvait supposer un complot frauduleux de leur part ; mais où a-t-on pris que, sur des suppositions, il faille établir une chose injuste par elle-même et ruineuse pour l’État ?
\subsubsection[{Chapitre XIX. Qu’est-ce qui est plus convenable au prince et au peuple, de la ferme ou de la régie des tributs ?}]{Chapitre XIX. Qu’est-ce qui est plus convenable au prince et au peuple, de la ferme ou de la régie des tributs ?}
\noindent La régie est l’administration d’un bon père de famille, qui lève lui-même, avec économie et avec ordre, ses revenus.\par
Par la régie, le prince est le maître de presser ou de retarder la levée des tributs, ou suivant ses besoins, ou suivant ceux de ses peuples. Par la régie, il épargne à l’État les profits immenses des fermiers, qui l’appauvrissent d’une infinité de manières. Par la régie, il épargne au peuple le spectacle des fortunes subites qui l’affligent. Par la régie, l’argent levé passe par peu de mains ; il va directement au prince, et par conséquent revient plus promptement au peuple. Par la régie, le prince épargne au peuple une infinité de mauvaises lois qu’exige toujours de lui l’avarice importune des fermiers, qui montrent un avantage présent dans des règlements funestes pour l’avenir.\par
Comme celui qui a l’argent est toujours le maître de l’autre, le traitant se rend despotique sur le prince même : il n’est pas législateur, mais il le force à donner des lois.\par
J’avoue qu’il est quelquefois utile de commencer par donner à ferme un droit nouvellement établi. Il y a un art et des inventions pour prévenir les fraudes, que l’intérêt des fermiers leur suggère, et que les régisseurs n’auraient su imaginer : or le système de la levée étant une fois fait par le fermier, on peut avec succès établir la régie. En Angleterre, l’administration de l’accise et du revenu des postes, telle qu’elle est aujourd’hui, a été empruntée des fermiers.\par
Dans les républiques, les revenus de l’État sont presque toujours en régie. L’établissement contraire fut un grand vice du gouvernement de Rome\footnote{César fut obligé d’ôter les publicains de la province d’Asie et d’y établir une autre sorte d’administration, comme nous l’apprenons de Dion. Et Tacite nous dit que la Macédoine et l’Achaïe, provinces qu’Auguste avait laissées au peuple romain, et qui, par conséquent, étaient gouvernées sur l’ancien plan, obtinrent d’être du nombre de celles que l’empereur gouvernait par ses officiers.}. Dans les États despotiques où la régie est établie, les peuples sont infiniment plus heureux : témoin la Perse et la Chine\footnote{Voyez Chardin, {\itshape Voyage de Perse}, t. VI.}. Les plus malheureux sont ceux où le prince donne à ferme ses ports de mer et ses villes de commerce. L’histoire des monarchies est pleine des maux faits par les traitants.\par
Néron, indigné des vexations des publicains, forma le projet impossible et magnanime d’abolir tous les impôts. Il n’imagina point la régie : il fit quatre ordonnances\footnote{Tacite, {\itshape Ann}., liv. XIII.} : que les lois faites contre les publicains, qui avaient été jusque-là tenues secrètes, seraient publiées ; qu’ils ne pourraient plus exiger ce qu’ils avaient négligé de demander dans l’année ; qu’il y aurait un préteur établi pour juger leurs prétentions sans formalité ; que les marchands ne paieraient rien pour les navires. Voilà les beaux jours de cet empereur.
\subsubsection[{Chapitre XX. Des traitants}]{Chapitre XX. {\itshape Des traitants}}
\noindent Tout est perdu lorsque la profession lucrative des traitants parvient encore par ses richesses à être une profession honorée. Cela peut être bon dans les États despotiques, où souvent leur emploi est une partie des fonctions des gouverneurs eux-mêmes. Cela n’est pas bon dans la république ; et une chose pareille détruisit la république romaine. Cela n’est pas meilleur dans la monarchie ; rien n’est plus contraire à l’esprit de ce gouvernement. Un dégoût saisit tous les autres états ; l’honneur y perd toute sa considération ; les moyens lents et naturels de se distinguer ne touchent plus ; et le gouvernement est frappé dans son principe.\par
On vit bien, dans les temps passés, des fortunes scandaleuses ; c’était une des calamités des guerres de cinquante ans : mais pour lors, ces richesses furent regardées comme ridicules, et nous les admirons.\par
Il y a un lot pour chaque profession. Le lot de ceux qui lèvent les tributs est les richesses, et les récompenses de ces richesses sont les richesses mêmes. La gloire et l’honneur sont pour cette noblesse qui ne connaît, qui ne voit, qui ne sent de vrai bien que l’honneur et la gloire. Le respect et la considération sont pour ces ministres et ces magistrats qui, ne trouvant que le travail après le travail, veillent nuit et jour pour le bonheur de l’empire.
\section[{Troisième partie}]{Troisième partie}\renewcommand{\leftmark}{Troisième partie}

\subsection[{Livre quatorzième. Des lois dans le rapport qu’elles ont avec la nature du climat}]{Livre quatorzième. Des lois dans le rapport qu’elles ont avec la nature du climat}
\subsubsection[{Chapitre I. Idée générale}]{Chapitre I. Idée générale}
\noindent S’il est vrai que le caractère de l’esprit et les passions du cœur soient extrêmement différents dans les divers climats, les lois doivent être relatives et à la différence de ces passions, et à la différence de ces caractères.
\subsubsection[{Chapitre II. Combien les hommes sont différents dans les divers climats}]{Chapitre II. Combien les hommes sont différents dans les divers climats}
\noindent L’air froid\footnote{Cela paraît même à la vue : dans le froid on paraît plus maigre.} resserre les extrémités des fibres extérieures de notre corps ; cela augmente leur ressort, et favorise le retour du sang des extrémités vers le cœur. Il diminue la longueur\footnote{On sait qu’il raccourcit le fer.} de ces mêmes fibres ; il augmente donc encore par là leur force. L’air chaud, au contraire, relâche les extrémités des fibres, et les allonge ; il diminue donc leur force et leur ressort.\par
On a donc plus de vigueur dans les climats froids. L’action du cœur et la réaction des extrémités des fibres s’y font mieux, les liqueurs sont mieux en équilibre, le sang est plus déterminé vers le cœur, et réciproquement le cœur a plus de puissance. Cette force plus grande doit produire bien des effets : par exemple, plus de confiance en soi-même, c’est-à-dire plus de courage ; plus de connaissance de sa supériorité, c’est-à-dire moins de désir de la vengeance ; plus d’opinion de sa sûreté, c’est-à-dire plus de franchise, moins de soupçons, de politique et de ruses. Enfin cela doit faire des caractères bien différents. Mettez un homme dans un lieu chaud et enfermé, il souffrira, par les raisons que je viens de dire, une défaillance de cœur très grande. Si, dans cette circonstance, on va lui proposer une action hardie, je crois qu’on l’y trouvera très peu disposé ; sa faiblesse présente mettra un découragement dans son âme ; il craindra tout, parce qu’il sentira qu’il ne peut rien. Les peuples des pays chauds sont timides comme les vieillards le sont ; ceux des pays froids sont courageux comme le sont les jeunes gens. Si nous faisons attention aux dernières\footnote{Celles pour la succession d’Espagne.} guerres, qui sont celles que nous avons le plus sous nos yeux, et dans lesquelles nous pouvons mieux voir de certains effets légers, imperceptibles de loin, nous sentirons bien que les peuples du nord, transportés dans les pays du midi\footnote{En Espagne, par exemple.}, n’y ont pas fait d’aussi belles actions que leurs compatriotes qui, combattant dans leur propre climat, y jouissaient de tout leur courage.\par
La force des fibres des peuples du nord fait que les sucs les plus grossiers sont tirés des aliments. Il en résulte deux choses : l’une, que les parties du chyle, ou de la lymphe, sont plus propres, par leur grande surface, à être appliquées sur les fibres et à les nourrir ; l’autre, qu’elles sont moins propres, par leur grossièreté, à donner une certaine subtilité au suc nerveux. Ces peuples auront donc de grands corps et peu de vivacité.\par
Les nerfs, qui aboutissent de tous côtés au tissu de notre peau, font chacun un faisceau de nerfs. Ordinairement ce n’est pas tout le nerf qui est remué, c’en est une partie infiniment petite. Dans les pays chauds, où le tissu de la peau est relâché, les bouts des nerfs sont épanouis et exposés à la plus petite action des objets les plus faibles. Dans les pays froids, le tissu de la peau est resserré, et les mamelons comprimés ; les petites houppes sont, en quelque façon, paralytiques ; la sensation ne passe guère au cerveau que lorsqu’elle est extrêmement forte, et qu’elle est de tout le nerf ensemble. Mais c’est d’un nombre infini de petites sensations que dépendent l’imagination, le goût, la sensibilité, la vivacité.\par
J’ai observé le tissu extérieur d’une langue de mouton, dans l’endroit où elle paraît, à la simple vue, couverte de mamelons. J’ai vu avec un microscope, sur ces mamelons, de petits poils ou une espèce de duvet ; entre les mamelons étaient des pyramides, qui formaient par le bout comme de petits pinceaux. Il y a grande apparence que ces pyramides sont le principal organe du goût.\par
J’ai fait geler la moitié de cette langue, et j’ai trouvé, à la simple vue, les mamelons considérablement diminués ; quelques rangs même de mamelons s’étaient enfoncés dans leur gaine. J’en ai examiné le tissu avec le microscope, je n’ai plus vu de pyramides. À mesure que la langue s’est dégelée, les mamelons, à la simple vue, ont paru se relever ; et, au microscope, les petites houppes ont commencé à reparaître.\par
Cette observation confirme ce que j’ai dit, que, dans les pays froids, les houppes nerveuses sont moins épanouies : elles s’enfoncent dans leurs gaines, où elles sont à couvert de l’action des objets extérieurs. Les sensations sont donc moins vives.\par
Dans les pays froids, on aura peu de sensibilité pour les plaisirs ; elle sera plus grande dans les pays tempérés ; dans les pays chauds, elle sera extrême. Comme on distingue les climats par les degrés de latitude, on pourrait les distinguer, pour ainsi dire, par les degrés de sensibilité. J’ai vu les opéras d’Angleterre et d’Italie ; ce sont les mêmes pièces et les mêmes acteurs : mais la même musique produit des effets si différents sur les deux nations, l’une est si calme, et l’autre si transportée, que cela paraît inconcevable.\par
Il en sera de même de la douleur : elle est excitée en nous par le déchirement de quelque fibre de notre corps. L’auteur de la nature a établi que cette douleur serait plus forte à mesure que le dérangement serait plus grand : or il est évident que les grands corps et les fibres grossières des peuples du nord sont moins capables de dérangement que les fibres délicates des peuples des pays chauds, l’âme y est donc moins sensible à la douleur. Il faut écorcher un Moscovite pour lui donner du sentiment.\par
Avec cette délicatesse d’organes que l’on a dans les pays chauds, l’âme est souverainement émue par tout ce qui a du rapport à l’union des deux sexes : tout conduit à cet objet.\par
Dans les climats du nord, à peine le physique de l’amour a-t-il la force de se rendre bien sensible ; dans les climats tempérés, l’amour, accompagné de mille accessoires, se rend agréable par des choses qui d’abord semblent être lui-même, et ne sont pas encore lui ; dans les climats plus chauds, on aime l’amour pour lui-même ; il est la cause unique du bonheur ; il est la vie.\par
Dans les pays du midi, une machine délicate, faible, mais sensible, se livre à un amour qui, dans un sérail, naît et se calme sans cesse ; ou bien à un amour qui, laissant les femmes dans une plus grande indépendance, est exposé à mille troubles. Dans les pays du nord, une machine saine et bien constituée, mais lourde, trouve ses plaisirs dans tout ce qui peut remettre les esprits en mouvement : la chasse, les voyages, la guerre, le vin. Vous trouverez dans les climats du nord des peuples qui ont peu de vices, assez de vertus, beaucoup de sincérité et de franchise. Approchez des pays du midi, vous croirez vous éloigner de la morale même : des passions plus vives multiplieront les crimes ; chacun cherchera à prendre sur les autres tous les avantages qui peuvent favoriser ces mêmes passions. Dans les pays tempérés, vous verrez des peuples inconstants dans leurs manières, dans leurs vices même, et dans leurs vertus ; le climat n’y a pas une qualité assez déterminée pour les fixer eux-mêmes.\par
La chaleur du climat peut être si excessive que le corps y sera absolument sans force. Pour lors l’abattement passera à l’esprit même ; aucune curiosité, aucune noble entreprise, aucun sentiment généreux ; les inclinations y seront toutes passives ; la paresse y fera le bonheur ; la plupart des châtiments y seront moins difficiles à soutenir que l’action de l’âme, et la servitude moins insupportable que la force d’esprit qui est nécessaire pour se conduire soi-même.
\subsubsection[{Chapitre III. Contradiction dans les caractères de certains peuples du midi}]{Chapitre III. Contradiction dans les caractères de certains peuples du midi}
\noindent Les Indiens\footnote{« Cent soldats d’Europe, dit Tavernier, n’auraient pas grand-peine à battre mille soldats indiens. »} sont naturellement sans courage ; les enfants\footnote{« Les Persans mêmes qui s’établissent aux Indes prennent, à la troisième génération, la nonchalance et la lâcheté indienne. » Voyez Bernier, {\itshape Sur le Mogol}, t. I, p. 282.} même des Européens nés aux Indes perdent celui de leur climat. Mais comment accorder cela avec leurs actions atroces, leurs coutumes, leurs pénitences barbares ? Les hommes s’y soumettent à des maux incroyables, les femmes s’y brûlent elles-mêmes : voilà bien de la force pour tant de faiblesse.\par
La nature, qui a donné à ces peuples une faiblesse qui les rend timides, leur a donné aussi une imagination si vive que tout les frappe à l’excès. Cette même délicatesse d’organes, qui leur fait craindre la mort, sert aussi à leur faire redouter mille choses plus que la mort. C’est la même sensibilité qui leur fait fuir tous les périls, et les leur fait tous braver.\par
Comme une bonne éducation est plus nécessaire aux enfants qu’à ceux dont l’esprit est dans sa maturité, de même les peuples de ces climats ont plus besoin d’un législateur sage que les peuples du nôtre. Plus on est aisément et fortement frappé, plus il importe de l’être d’une manière convenable, de ne recevoir pas des préjugés, et d’être conduit par la raison.\par
Du temps des Romains, les peuples du nord de l’Europe vivaient sans arts, sans éducation, presque sans lois ; et cependant, par le seul bon sens attaché aux fibres grossières de ces climats, ils se maintinrent avec une sagesse admirable contre la puissance romaine, jusqu’au moment où ils sortirent de leurs forêts pour la détruire.
\subsubsection[{Chapitre IV. Cause de l’immutabilité de la religion, des mœurs, des manières, des lois dans les pays d’Orient}]{Chapitre IV. Cause de l’immutabilité de la religion, des mœurs, des manières, des lois dans les pays d’Orient}
\noindent Si, avec cette faiblesse d’organes qui fait recevoir aux peuples d’Orient les impressions du monde les plus fortes, vous joignez une certaine paresse dans l’esprit, naturellement liée avec celle du corps, qui fasse que cet esprit ne soit capable d’aucune action, d’aucun effort, d’aucune contention, vous comprendrez que l’âme, qui a une fois reçu des impressions, ne peut plus en changer. C’est ce qui fait que les lois, les mœurs\footnote{On voit, par un fragment de Nicolas de Damas, recueilli par Constantin Porphyrogénète, que la coutume était ancienne en Orient d’envoyer étrangler un gouverneur qui déplaisait ; elle était du temps des Mèdes.} et les manières, même celles qui paraissent indifférentes, comme la façon de se vêtir, sont aujourd’hui en Orient comme elles étaient il y a mille ans.
\subsubsection[{Chapitre V. Que les mauvais législateurs sont ceux qui ont favorisé les vices du climat, et les bons sont ceux qui s’y sont opposés}]{Chapitre V. Que les mauvais législateurs sont ceux qui ont favorisé les vices du climat, et les bons sont ceux qui s’y sont opposés}
\noindent Les Indiens croient que le repos et le néant sont le fondement de toutes choses et la fin où elles aboutissent. Ils regardent donc l’entière inaction comme l’état le plus parfait et l’objet de leurs désirs. Ils donnent au souverain être le surnom d’immobile\footnote{Panamanack. Voyez Kircher.}. Les Siamois croient que la félicité\footnote{La Loubère, {\itshape Relation de Siam}, p. 446.} suprême consiste à n’être point obligé d’animer une machine et de faire agir un corps.\par
Dans ces pays, où la chaleur excessive énerve et accable, le repos est si délicieux et le mouvement si pénible, que ce système de métaphysique paraît naturel, et Foë\footnote{Foë veut réduire le cœur au pur vide. « Nous avons des yeux et des oreilles ; mais la perfection est de ne voir ni entendre ; une bouche, des mains, etc., la perfection est que ces membres soient dans l’inaction. » Ceci est tiré du dialogue d’un philosophe chinois, rapporté par le P. Du Halde, t. III.}, législateur des Indes, a suivi ce qu’il sentait, lorsqu’il a mis les hommes dans un état extrêmement passif ; mais sa doctrine, née de la paresse du climat, la favorisant à son tour, a causé mille maux.\par
Les législateurs de la Chine furent plus sensés lorsque, considérant les hommes, non pas dans l’état paisible où ils seront quelque jour, mais dans l’action propre à leur faire remplir les devoirs de la vie, ils firent leur religion, leur philosophie et leurs lois toutes pratiques. Plus les causes physiques portent les hommes au repos, plus les causes morales les en doivent éloigner.
\subsubsection[{Chapitre VI. De la culture des terres dans les climats chauds}]{Chapitre VI. De la culture des terres dans les climats chauds}
\noindent La culture des terres est le plus grand travail des hommes. Plus le climat les porte à fuir ce travail, plus la religion et les lois doivent y exciter. Ainsi les lois des Indes, qui donnent les terres aux princes, et ôtent aux particuliers l’esprit de propriété, augmentent les mauvais effets du climat, c’est-à-dire la paresse naturelle.
\subsubsection[{Chapitre VII. Du monachisme}]{Chapitre VII. {\itshape Du monachisme}}
\noindent Le monachisme y fait les mêmes maux ; il est né dans les pays chauds d’Orient, où l’on est moins porté à l’action qu’à la spéculation.\par
En Asie, le nombre des derviches, ou moines, semble augmenter avec la chaleur du climat ; les Indes, où elle est excessive, en sont remplies : on trouve en Europe cette même différence.\par
Pour vaincre la paresse du climat, il faudrait que les lois cherchassent à ôter tous les moyens de vivre sans travail ; mais dans le midi de l’Europe elles font tout le contraire : elles donnent à ceux qui veulent être oisifs des places propres à la vie spéculative, et y attachent des richesses immenses. Ces gens, qui vivent dans une abondance qui leur est à charge, donnent avec raison leur superflu au bas peuple : il a perdu la propriété des biens ; ils l’en dédommagent par l’oisiveté dont ils le font jouir ; et il parvient à aimer sa misère même.
\subsubsection[{Chapitre VIII. Bonne coutume de la Chine}]{Chapitre VIII. Bonne coutume de la Chine}
\noindent Les relations\footnote{Le P. Du Halde, {\itshape Histoire de la Chine}, t. II, p. 72.} de la Chine nous parlent de la cérémonie d’ouvrir les terres, que l’empereur fait tous les ans\footnote{Plusieurs rois des Indes font de même. {\itshape Relation du royaume de Siam} par La Loubère, p. 69.}. On a voulu exciter\footnote{Venty, troisième empereur de la troisième dynastie, cultiva la terre de ses propres mains, et fit travailler à la soie, dans son palais, l’impératrice et ses femmes. {\itshape Histoire de la Chine}.} les peuples au labourage par cet acte public et solennel.\par
De plus, l’empereur est informé chaque année du laboureur qui s’est le plus distingué dans sa profession ; il le fait mandarin du huitième ordre.\par
Chez les anciens Perses\footnote{M. Hyde, {\itshape Religion des Perses}.}, le huitième jour du mois nommé chorrem-ruz, les rois quittaient leur faste pour manger avec les laboureurs. Ces institutions sont admirables pour encourager l’agriculture.
\subsubsection[{Chapitre IX. Moyens d’encourager l’industrie}]{Chapitre IX. Moyens d’encourager l’industrie}
\noindent Je ferai voir, au livre XIX, que les nations paresseuses sont ordinairement orgueilleuses. On pourrait tourner l’effet contre la cause, et détruire la paresse par l’orgueil. Dans le midi de l’Europe, où les peuples sont si frappés par le point d’honneur, il serait bon de donner des prix aux laboureurs qui auraient le mieux cultivé leurs champs, ou aux ouvriers qui auraient porté plus loin leur industrie. Cette pratique réussira même par tout pays. Elle a servi de nos jours, en Irlande, à l’établissement d’une des plus importantes manufactures de toile qui soit en Europe.
\subsubsection[{Chapitre X. Des lois qui ont rapport à la sobriété des peuples}]{Chapitre X. Des lois qui ont rapport à la sobriété des peuples}
\noindent Dans les pays chauds, la partie aqueuse du sang se dissipe beaucoup par la transpiration\footnote{M. Bernier, faisant un voyage de Lahor à Cachemir, écrivait : « Mon corps est un crible : à peine ai-je avalé une pinte d’eau, que je la vois sortir comme une rosée de tous mes membres jusqu’au bout des doigts ; j’en bois dix pintes par jour, et cela ne me fait point de mal. » {\itshape Voyage} de Bernier, t. II, p. 261.} ; il y faut donc substituer un liquide pareil. L’eau y est d’un usage admirable : les liqueurs fortes y coaguleraient les globules\footnote{Il y a dans le sang des globules rouges, des parties fibreuses, des globules blancs, et de l’eau dans laquelle nage tout cela.} du sang qui restent après la dissipation de la partie aqueuse.\par
Dans les pays froids, la partie aqueuse du sang s’exhale peu par la transpiration ; elle reste en grande abondance : on y peut donc user des liqueurs spiritueuses, sans que le sang se coagule. On y est plein d’humeurs ; les liqueurs fortes, qui donnent du mouvement au sang, y peuvent être convenables.\par
La loi de Mahomet, qui défend de boire du vin, est donc une loi du climat d’Arabie : aussi avant Mahomet, l’eau était-elle la boisson commune des Arabes. La loi\footnote{Platon, liv. II des {\itshape Lois}. Aristote, {\itshape Du soin des affaires domestiques}. Eusèbe, {\itshape Préparations évangéliques}, liv. XII, chap. XVII.} qui défendait aux Carthaginois de boire du vin, était aussi une loi du climat ; effectivement le climat de ces deux pays est à peu près le même.\par
Une pareille loi ne serait pas bonne dans les pays froids, où le climat semble forcer à une certaine ivrognerie de nation, bien différente de celle de la personne. L’ivrognerie se trouve établie par toute la terre, dans la proportion de la froideur et de l’humidité du climat. Passez de l’équateur jusqu’à notre pôle, vous y verrez l’ivrognerie augmenter avec les degrés de latitude. Passez du même équateur au pôle opposé, vous y trouverez l’ivrognerie aller vers le midi\footnote{Cela se voit dans les Hottentots et les peuples de la pointe du Chili, qui sont plus près du sud.}, comme de ce côté-ci elle avait été vers le nord.\par
Il est naturel que, là où le vin est contraire au climat, et par conséquent à la santé, l’excès en soit plus sévèrement puni que dans les pays où l’ivrognerie a peu de mauvais effets pour la personne, où elle en a peu pour la société, où elle ne rend point les hommes furieux, mais seulement stupides. Ainsi les lois\footnote{Comme fit Pittacus, selon Aristote, {\itshape Politique}, liv. II, chap. III. Il vivait dans un climat où l’ivrognerie n’est pas un vice de nation.} qui ont puni un homme ivre, et pour la faute qu’il faisait, et pour l’ivresse, n’étaient applicables qu’à l’ivrognerie de la personne, et non à l’ivrognerie de la nation. Un Allemand boit par coutume, un Espagnol par choix.\par
Dans les pays chauds, le relâchement des fibres produit une grande transpiration des liquides ; mais les parties solides se dissipent moins. Les fibres, qui n’ont qu’une action très faible et peu de ressort, ne s’usent guère ; il faut peu de suc nourricier pour les réparer : on y mange donc très peu.\par
Ce sont les différents besoins dans les différents climats qui ont formé les différentes manières de vivre ; et ces différentes manières de vivre ont formé les diverses sortes de lois. Que, dans une nation, les hommes se communiquent beaucoup, il faut de certaines lois ; il en faut d’autres chez un peuple où l’on ne se communique point.
\subsubsection[{Chapitre XI. Des lois qui ont du rapport aux maladies du climat}]{Chapitre XI. Des lois qui ont du rapport aux maladies du climat}
\noindent Hérodote\footnote{Liv. II.} nous dit que les lois des Juifs sur la lèpre ont été tirées de la pratique des Égyptiens. En effet, les mêmes maladies demandaient les mêmes remèdes. Ces lois furent inconnues aux Grecs et aux premiers Romains, aussi bien que le mal. Le climat de l’Égypte et de la Palestine les rendit nécessaires ; et la facilité qu’a cette maladie à se rendre populaire nous doit bien faire sentir la sagesse et la prévoyance de ces lois.\par
Nous en avons nous-mêmes éprouvé les effets. Les croisades nous avaient apporté la lèpre ; les règlements sages que l’on fit l’empêchèrent de gagner la masse du peuple.\par
On voit, par la loi\footnote{Liv. II, tit. I, § 3 ; et tit. XVIII, § I.} des Lombards, que cette maladie était répandue en Italie avant les croisades, et mérita l’attention des législateurs. Rotharis ordonna qu’un lépreux, chassé de sa maison, et relégué dans un endroit particulier, ne pourrait disposer de ses biens, parce que dès le moment qu’il avait été tiré de sa maison, il était censé mort. Pour empêcher toute communication avec les lépreux, on les rendait incapables des effets civils.\par
Je pense que cette maladie fut apportée en Italie par les conquêtes des empereurs grecs, dans les armées desquels il pouvait y avoir des milices de la Palestine ou de l’Égypte. Quoi qu’il en soit, les progrès en furent arrêtés jusqu’au temps des croisades.\par
On dit que les soldats de Pompée, revenant de Syrie, rapportèrent une maladie à peu près pareille à la lèpre. Aucun règlement fait pour lors n’est venu jusqu’à nous ; mais il y a apparence qu’il y en eut, puisque ce mal fut suspendu jusqu’au temps des Lombards.\par
Il y a deux siècles qu’une maladie, inconnue à nos pères, passa du Nouveau Monde dans celui-ci, et vint attaquer la nature humaine jusque dans la source de la vie et des plaisirs. On vit la plupart des plus grandes familles du midi de l’Europe périr par un mal qui devint trop commun pour être honteux, et ne fut plus que funeste. Ce fut la soif de l’or qui perpétua cette maladie ; on alla sans cesse en Amérique, et on en rapporta toujours de nouveaux levains.\par
Des raisons pieuses voulurent demander qu’on laissât cette punition sur le crime ; mais cette calamité était entrée dans le sein du mariage, et avait déjà corrompu l’enfance même.\par
Comme il est de la sagesse des législateurs de veiller à la santé des citoyens, il eût été très sensé d’arrêter cette communication par des lois faites sur le plan des lois mosaïques.\par
La peste est un mal dont les ravages sont encore plus prompts et plus rapides. Son siège principal est en Égypte, d’où elle se répand par tout l’univers. On a fait, dans la plupart des États de l’Europe, de très bons règlements pour l’empêcher d’y pénétrer ; et on a imaginé de nos jours un moyen admirable de l’arrêter : on forme une ligne de troupes autour du pays infesté, qui empêche toute communication.\par
Les Turcs\footnote{Ricaut, De l’empire ottoman, p. 284.}, qui n’ont à cet égard aucune police, voient les chrétiens dans la même ville échapper au danger, et eux seuls périr. Ils achètent les habits des pestiférés, s’en vêtissent, et vont leur train. La doctrine d’un destin rigide qui règle tout, fait du magistrat un spectateur tranquille : il pense que Dieu a déjà tout fait, et que lui n’a rien à faire.
\subsubsection[{Chapitre XII. Des lois contre ceux qui se tuent eux-mêmes}]{Chapitre XII. Des lois contre ceux qui se tuent\protect\footnotemark  eux-mêmes}
\footnotetext{L’action de ceux qui se tuent eux-mêmes est contraire à la loi naturelle et à la religion révélée.}
\noindent Nous ne voyons point dans les histoires que les Romains se fissent mourir sans sujet ; mais les Anglais se tuent sans qu’on puisse imaginer aucune raison qui les y détermine, ils se tuent dans le sein même du bonheur. Cette action, chez les Romains, était l’effet de l’éducation ; elle tenait à leur manière de penser et à leurs coutumes : chez les Anglais, elle est l’effet d’une maladie\footnote{Elle pourrait bien être compliquée avec le scorbut qui, surtout dans quelques pays, rend un homme bizarre et insupportable à lui-même. Voyage de François Pyrard, part. II, chap. XXI.}, elle tient à l’état physique de la machine, et est indépendante de toute autre cause.\par
Il y a apparence que c’est un défaut de filtration du suc nerveux : la machine, dont les forces motrices se trouvent à tout moment sans action, est lasse d’elle-même ; l’âme ne sent point de douleur, mais une certaine difficulté de l’existence. La douleur est un mal local qui nous porte au désir de voir cesser cette douleur ; le poids de la vie est un mal qui n’a point de lieu particulier, et qui nous porte au désir de voir finir cette vie.\par
Il est clair que les lois civiles de quelques pays ont eu des raisons pour flétrir l’homicide de soi-même ; mais, en Angleterre, on ne peut pas plus le punir qu’on ne punit les effets de la démence.
\subsubsection[{Chapitre XIII. Effets qui résultent du climat d’Angleterre}]{Chapitre XIII. Effets qui résultent du climat d’Angleterre}
\noindent Dans une nation à qui une maladie du climat affecte tellement l’âme, qu’elle pourrait porter le dégoût de toutes choses jusqu’à celui de la vie, on voit bien que le gouvernement qui conviendrait le mieux à des gens à qui tout serait insupportable, serait celui où ils ne pourraient pas se prendre à un seul de ce qui causerait leurs chagrins ; et où les lois gouvernant plutôt que les hommes, il faudrait, pour changer l’État, les renverser elles-mêmes.\par
Que si la même nation avait encore reçu du climat un certain caractère d’impatience qui ne lui permît pas de souffrir longtemps les mêmes choses, on voit bien que le gouvernement dont nous venons de parler serait encore le plus convenable.\par
Ce caractère d’impatience n’est pas grand par lui-même ; mais il peut le devenir beaucoup, quand il est joint avec le courage.\par
Il est différent de la légèreté, qui fait que l’on entreprend sans sujet, et que l’on abandonne de même. Il approche plus de l’opiniâtreté, parce qu’il vient d’un sentiment des maux, si vif, qu’il ne s’affaiblit pas même par l’habitude de les souffrir.\par
Ce caractère, dans une nation libre, serait très propre à déconcerter les projets de la tyrannie\footnote{Je prends ici ce mot pour le dessein de renverser le pouvoir établi, et surtout la démocratie. C’est la signification que lui donnaient les Grecs et les Romains.}, qui est toujours lente et faible dans ses commencements, comme elle est prompte et vive dans sa fin ; qui ne montre d’abord qu’une main pour secourir, et opprime ensuite avec une infinité de bras.\par
La servitude commence toujours par le sommeil. Mais un peuple qui n’a de repos dans aucune situation, qui se tâte sans cesse, et trouve tous les endroits douloureux, ne pourrait guère s’endormir.\par
La politique est une lime sourde, qui use et qui parvient lentement à sa fin. Or les hommes dont nous venons de parler ne pourraient soutenir les lenteurs, les détails, le sang-froid des négociations ; ils y réussiraient souvent moins que toute autre nation ; et ils perdraient, par leurs traités, ce qu’ils auraient obtenu par leurs armes.
\subsubsection[{Chapitre XIV. Autres effets du climat}]{Chapitre XIV. Autres effets du climat}
\noindent Nos pères, les anciens Germains, habitaient un climat où les passions étaient très calmes. Leurs lois ne trouvaient dans les choses que ce qu’elles voyaient, et n’imaginaient rien de plus. Et comme elles jugeaient des insultes faites aux hommes par la grandeur des blessures, elles ne mettaient pas plus de raffinement dans les offenses faites aux femmes. La loi\footnote{Chap. {\itshape LVIII}, § 1 et 2.} des Allemands est là-dessus fort singulière. Si l’on découvre une femme à la tête, on paiera une amende de six sols ; autant si c’est à la jambe jusqu’au genou ; le double depuis le genou. Il semble que la loi mesurait la grandeur des outrages faits à la personne des femmes, comme on mesure une figure de géométrie ; elle ne punissait point le crime de l’imagination, elle punissait celui des yeux. Mais lorsqu’une nation germanique se fut transportée en Espagne, le climat trouva bien d’autres lois. La loi des Wisigoths défendit aux médecins de saigner une femme ingénue qu’en présence de son père et de sa mère, de son frère, de son fils ou de son oncle. L’imagination des peuples s’alluma, celle des législateurs s’échauffa de même ; la loi soupçonna tout pour un peuple qui pouvait tout soupçonner.\par
Ces lois eurent donc une extrême attention sur les deux sexes. Mais il semble que, dans les punitions qu’elles firent, elles songèrent plus à flatter la vengeance particulière qu’à exercer la vengeance publique. Ainsi, dans la plupart des cas, elles réduisaient les deux coupables dans la servitude des parents ou du mari offensé. Une femme\footnote{Loi des Wisigoths, liv. III, tit. IV, § 9.} ingénue, qui s’était livrée à un homme marié, était remise dans la puissance de sa femme, pour en disposer à sa volonté. Elles obligeaient les esclaves\footnote{{\itshape Ibid.}, liv. III, tit. IV, § 6.} de lier et de présenter au mari sa femme qu’ils surprenaient en adultère ; elles permettaient à ses enfants\footnote{{\itshape Ibid.}, liv. III, tit. IV, § 13.} de l’accuser, et de mettre à la question ses esclaves pour la convaincre. Aussi furent-elles plus propres à raffiner à l’excès un certain point d’honneur qu’à former une bonne police. Et il ne faut pas être étonné si le comte Julien crut qu’un outrage de cette espèce demandait la perte de sa patrie et de son roi. On ne doit pas être surpris si les Maures, avec une telle conformité de mœurs, trouvèrent tant de facilité à s’établir en Espagne, à s’y maintenir et à retarder la chute de leur empire.
\subsubsection[{Chapitre XV. De la différente confiance que les lois ont dans le peuple selon les climats}]{Chapitre XV. De la différente confiance que les lois ont dans le peuple selon les climats}
\noindent Le peuple japonais a un caractère si atroce, que ses législateurs et ses magistrats n’ont pu avoir aucune confiance en lui : ils ne lui ont mis devant les yeux que des juges, des menaces et des châtiments ; ils l’ont soumis, pour chaque démarche, à l’inquisition de la police. Ces lois qui, sur cinq chefs de famille, en établissent un comme magistrat sur les quatre autres ; ces lois qui, pour un seul crime, punissent toute une famille ou tout un quartier ; ces lois qui ne trouvent point d’innocents là où il peut y avoir un coupable, sont faites pour que tous les hommes se méfient les uns des autres, pour que chacun recherche la conduite de chacun, et qu’il en soit l’inspecteur, le témoin et le juge.\par
Le peuple des Indes au contraire est doux\footnote{Voyez Bernier, t. I, p. 40.}, tendre, compatissant : aussi ses législateurs ont-ils eu une grande confiance en lui. Ils ont établi peu\footnote{Voyez dans le quatorzième recueil des {\itshape Lettres édifiantes}, p. 403, les principales lois ou coutumes des peuples de l’Inde de la presqu’île deçà le Gange.} de peines, et elles sont peu sévères ; elles ne sont pas même rigoureusement exécutées. Ils ont donné les neveux aux oncles, les orphelins aux tuteurs, comme on les donne ailleurs à leurs pères : ils ont réglé la succession par le mérite reconnu du successeur. Il semble qu’ils ont pensé que chaque citoyen devait se reposer sur le bon naturel des autres.\par
Ils donnent aisément la liberté\footnote{{\itshape Lettres édifiantes}, neuvième recueil, p. 378.} à leurs esclaves ; ils les marient ; ils les traitent comme leurs enfants\footnote{Savais pensé que la douceur de J’esclavage aux Indes avait fait dire à Diodore qu’il n’y avait dans ce pays ni maître ni esclave ; mais Diodore a attribué à toute l’Inde ce qui, selon Strabon, liv. XV, n’était propre qu’à une nation particulière.} : heureux climat, qui fait naître la candeur des mœurs, et produit la douceur des lois !
\subsection[{Livre quinzième. Comment les lois de l’esclavage civil ont du rapport avec la nature du climat}]{Livre quinzième. Comment les lois de l’esclavage civil ont du rapport avec la nature du climat}
\subsubsection[{Chapitre I. De l’esclavage civil}]{Chapitre I. De l’esclavage civil}
\noindent L’esclavage proprement dit est l’établissement d’un droit qui rend un homme tellement propre à un autre homme, qu’il est le maître absolu de sa vie et de ses biens. Il n’est pas bon par sa nature : il n’est utile ni au maître ni à l’esclave ; à celui-ci, parce qu’il ne peut rien faire par vertu ; à celui-là, parce qu’il contracte avec ses esclaves toutes sortes de mauvaises habitudes, qu’il s’accoutume insensiblement à manquer à toutes les vertus morales, qu’il devient fier, prompt, dur, colère, voluptueux, cruel.\par
Dans les pays despotiques, où l’on est déjà sous l’esclavage politique, l’esclavage civil est plus tolérable qu’ailleurs. Chacun y doit être assez content d’y avoir sa subsistance et la vie. Ainsi la condition de l’esclave n’y est guère plus à charge que la condition du sujet.\par
Mais, dans le gouvernement monarchique, où il est souverainement important de ne point abattre ou avilir la nature humaine, il ne faut point d’esclaves. Dans la démocratie, où tout le monde est égal, et dans l’aristocratie, où les lois doivent faire leurs efforts pour que tout le monde soit aussi égal que la nature du gouvernement peut le permettre, des esclaves sont contre l’esprit de la constitution ; ils ne servent qu’à donner aux citoyens une puissance et un luxe qu’ils ne doivent point avoir.
\subsubsection[{Chapitre II. Origine du droit de l’esclavage chez les jurisconsultes romains}]{Chapitre II. Origine du droit de l’esclavage chez les jurisconsultes romains}
\noindent On ne croirait jamais que c’eût été la pitié qui eût établi l’esclavage, et que pour cela elle s’y fût prise de trois manières\footnote{{\itshape Institutes} de Justinien, liv. I.}.\par
Le droit des gens a voulu que les prisonniers fussent esclaves, pour qu’on ne les tuât pas. Le droit civil des Romains permit à des débiteurs que leurs créanciers pouvaient maltraiter, de se vendre eux-mêmes ; et le droit naturel a voulu que des enfants, qu’un père esclave ne pouvait plus nourrir, fussent dans l’esclavage comme leur père.\par
Ces raisons des jurisconsultes ne sont point sensées. Il est faux qu’il soit permis de tuer dans la guerre autrement que dans le cas de nécessité ; mais, dès qu’un homme en a fait un autre esclave, on ne peut pas dire qu’il ait été dans la nécessité de le tuer, puisqu’il ne l’a pas fait. Tout le droit que la guerre peut donner sur les captifs est de s’assurer tellement de leur personne qu’ils ne puissent plus nuire. Les homicides faits de sang-froid par les soldats, et après la chaleur de l’action, sont rejetés de toutes les nations\footnote{Si l’on ne veut citer celles qui mangent leurs prisonniers.} du monde.\par
2{\itshape °} Il n’est pas vrai qu’un homme libre puisse se vendre. La vente suppose un prix : l’esclave se vendant, tous ses biens entreraient dans la propriété du maître ; le maître ne donnerait donc rien, et l’esclave ne recevrait rien. Il aurait un {\itshape pécule}, dira-t-on ; mais le pécule est accessoire à la personne. S’il n’est pas permis de se tuer, parce qu’on se dérobe à sa patrie, il n’est pas plus permis de se vendre. La liberté de chaque citoyen est une partie de la liberté publique. Cette qualité, dans l’État populaire, est même une partie de la souveraineté. Vendre sa qualité de citoyen est un acte\footnote{Je parle de l’esclavage pris à la rigueur, tel qu’il était chez les Romains, et qu’il est établi dans nos colonies.} d’une telle extravagance, qu’on ne peut pas la supposer dans un homme. Si la liberté a un prix pour celui qui l’achète, elle est sans prix pour celui qui la vend. La loi civile, qui a permis aux hommes le partage des biens, n’a pu mettre au nombre des biens une partie des hommes qui devaient faire ce partage. La loi civile, qui restitue sur les contrats qui contiennent quelque lésion, ne peut s’empêcher de restituer contre un accord qui contient la lésion la plus énorme de toutes.\par
La troisième manière, c’est la naissance. Celle-ci tombe avec les deux autres. Car, si un homme n’a pu se vendre, encore moins a-t-il pu vendre son fils qui n’était pas né. Si un prisonnier de guerre ne peut être réduit en servitude, encore moins ses enfants.\par
Ce qui fait que la mort d’un criminel est une chose licite, c’est que la loi qui le punit a été faite en sa faveur. Un meurtrier, par exemple, a joui de la loi qui le condamne ; elle lui a conservé la vie à tous les instants : il ne peut donc pas réclamer contre elle. Il n’en est pas de même de l’esclave : la loi de l’esclavage n’a jamais pu lui être utile ; elle est dans tous les cas contre lui, sans jamais être pour lui ; ce qui est contraire au principe fondamental de toutes les sociétés.\par
On dira qu’elle a pu lui être utile, parce que le maître lui a donné la nourriture. Il faudrait donc réduire l’esclavage aux personnes incapables de gagner leur vie. Mais on ne veut pas de ces esclaves-là. Quant aux enfants, la nature, qui a donné du lait aux mères, a pourvu à leur nourriture ; et le reste de leur enfance est si près de l’âge où est en eux la plus grande capacité de se rendre utiles, qu’on ne pourrait pas dire que celui qui les nourrirait, pour être leur maître, donnât rien.\par
L’esclavage est d’ailleurs aussi oppose au droit civil qu’au droit naturel. Quelle loi civile pourrait empêcher un esclave de fuir, lui qui n’est point dans la société, et que par conséquent aucunes lois civiles ne concernent ? Il ne peut être retenu que par une loi de famille, c’est-à-dire par la loi du maître.
\subsubsection[{Chapitre III. Autre origine du droit de l’esclavage}]{Chapitre III. Autre origine du droit de l’esclavage}
\noindent J’aimerais autant dire que le droit de l’esclavage vient du mépris qu’une nation conçoit pour une autre, fondé sur la différence des coutumes.\par
Lopès de Gamar\footnote{{\itshape Biblioth. angl}., t. XIII, part. II, art. 3.} dit « que les Espagnols trouvèrent, près de Sainte-Marthe, des paniers où les habitants avaient des denrées : c’étaient des cancres, des limaçons, des cigales, des sauterelles. Les vainqueurs en firent un crime aux vaincus ». L’auteur avoue que c’est là-dessus qu’on fonda le droit qui rendait les Américains esclaves des Espagnols ; outre qu’ils fumaient du tabac, et qu’ils ne se faisaient pas la barbe à l’espagnole.\par
Les connaissances rendent les hommes doux ; la raison porte à l’humanité : il n’y a que les préjugés qui y fassent renoncer.
\subsubsection[{Chapitre IV. Autre origine du droit de l’esclavage}]{Chapitre IV. Autre origine du droit de l’esclavage}
\noindent J’aimerais autant dire que la religion donne à ceux qui la professent un droit de réduire en servitude ceux qui ne la professent pas, pour travailler plus aisément à sa propagation.\par
Ce fut cette manière de penser qui encouragea les destructeurs de l’Amérique dans leurs crimes\footnote{Voyez l’{\itshape Histoire de la conquête du Mexique}, par Solis, et celle du Pérou, par Garcilasso de la Vega.}. C’est sur cette idée qu’ils fondèrent le droit de rendre tant de peuples esclaves ; car ces brigands, qui voulaient absolument être brigands et chrétiens, étaient très dévots.\par
Louis XIII\footnote{Le P. Labat, {\itshape Nouveau Voyage aux îles de l’Amérique}, t. IV, p. 114, an. 1722, in-12.} se fit une peine extrême de la loi qui rendait esclaves les nègres de ses colonies ; mais quand on lui eut bien mis dans l’esprit que c’était la voie la plus sûre pour les convertir, il y consentit.
\subsubsection[{Chapitre V. De l’esclavage des nègres}]{Chapitre V. De l’esclavage des nègres}
\noindent Si j’avais à soutenir le droit que nous avons eu de rendre les nègres esclaves, voici ce que je dirais :\par
Les peuples d’Europe ayant exterminé ceux de l’Amérique, ils ont dû mettre en esclavage ceux de l’Afrique, pour s’en servir à défricher tant de terres.\par
Le sucre serait trop cher, si l’on ne faisait travailler la plante qui le produit par des esclaves.\par
Ceux dont il s’agit sont noirs depuis les pieds jusqu’à la tête ; et ils ont le nez si écrasé qu’il est presque impossible de les plaindre.\par
On ne peut se mettre dans l’esprit que Dieu, qui est un être très sage, ait mis une âme, surtout une âme bonne, dans un corps tout noir.\par
Il est si naturel de penser que c’est la couleur qui constitue l’essence de l’humanité, que les peuples d’Asie, qui font des eunuques, privent toujours les noirs du rapport qu’ils ont avec nous d’une façon plus marquée.\par
On peut juger de la couleur de la peau par celle des cheveux, qui, chez les Égyptiens, les meilleurs philosophes du monde, étaient d’une si grande conséquence, qu’ils faisaient mourir tous les hommes roux qui leur tombaient entre les mains.\par
Une preuve que les nègres n’ont pas le sens commun, c’est qu’ils font plus de cas d’un collier de verre que de l’or, qui, chez des nations policées, est d’une si grande conséquence.\par
Il est impossible que nous supposions que ces gens-là soient des hommes ; parce que, si nous les supposions des hommes, on commencerait à croire que nous ne sommes pas nous-mêmes chrétiens.\par
De petits esprits exagèrent trop l’injustice que l’on fait aux Africains. Car, si elle était telle qu’ils le disent, ne serait-il pas venu dans la tête des princes d’Europe, qui font entre eux tant de conventions inutiles, d’en faire une générale en faveur de la miséricorde et de la pitié ?
\subsubsection[{Chapitre VI. Véritable origine du droit de l’esclavage}]{Chapitre VI. Véritable origine du droit de l’esclavage}
\noindent Il est temps de chercher la vraie origine du droit de l’esclavage. Il doit être fondé sur la nature des choses : voyons s’il y a des cas où il en dérive.\par
Dans tout gouvernement despotique, on a une grande facilité à se vendre : l’esclavage politique y anéantit en quelque façon la liberté civile.\par
M. Perry\footnote{{\itshape État présent de la grande Russie}, par Jean Perry, Paris, 1717, in-12.} dit que les Moscovites se vendent très aisément. J’en sais bien la raison : c’est que leur liberté ne vaut rien.\par
À Achim tout le monde cherche à se vendre. Quelques-uns des principaux seigneurs\footnote{{\itshape Nouveau Voyage autour du monde}, par Guillaume Dampierre, t. III, Amsterdam, 1711.} n’ont pas moins de mille esclaves, qui sont des principaux marchands, qui ont aussi beaucoup d’esclaves sous eux, et ceux-ci beaucoup d’autres ; on en hérite, et on les fait trafiquer. Dans ces États, les hommes libres, trop faibles contre le gouvernement, cherchent à devenir les esclaves de ceux qui tyrannisent le gouvernement.\par
C’est là l’origine juste et conforme à la raison de ce droit d’esclavage très doux que l’on trouve dans quelques pays ; et il doit être doux parce qu’il est fondé sur le choix libre qu’un homme, pour son utilité, se fait d’un maître ; ce qui forme une convention réciproque entre les deux parties.
\subsubsection[{Chapitre VII. Autre origine du droit de l’esclavage}]{Chapitre VII. Autre origine du droit de l’esclavage}
\noindent Voici une autre origine du droit de l’esclavage, et même de cet esclavage cruel que l’on voit parmi les hommes.\par
Il y a des pays où la chaleur énerve le corps, et affaiblit si fort le courage, que les hommes ne sont portés à un devoir pénible que par la crainte du châtiment : l’esclavage y choque donc moins la raison ; et le maître y étant aussi lâche à l’égard de son prince, que son esclave l’est à son égard, l’esclavage civil y est encore accompagné de l’esclavage politique.\par
Aristote\footnote{{\itshape Politique}, liv. I, chap. V.} veut prouver qu’il y a des esclaves par nature, et ce qu’il dit ne le prouve guère. Je crois que, s’il y en a de tels, ce sont ceux dont je viens de parler.\par
Mais, comme tous les hommes naissent égaux, il faut dire que l’esclavage est contre la nature, quoique dans certains pays il soit fondé sur une raison naturelle ; et il faut bien distinguer ces pays d’avec ceux où les raisons naturelles mêmes le rejettent, comme les pays d’Europe où il a été si heureusement aboli.\par
Plutarque nous dit, dans la {\itshape Vie de Numa}, que du temps de Saturne il n’y avait ni maître ni esclave. Dans nos climats, le christianisme a ramené cet âge.
\subsubsection[{Chapitre VIII. Inutilité de l’esclavage parmi nous}]{Chapitre VIII. Inutilité de l’esclavage parmi nous}
\noindent Il faut donc borner la servitude naturelle à de certains pays particuliers de la terre. Dans tous les autres, il me semble que, quelque pénibles que soient les travaux que la société y exige, on peut tout faire avec des hommes libres.\par
Ce qui me fait penser ainsi, c’est qu’avant que le christianisme eût aboli en Europe la servitude civile, on regardait les travaux des mines comme si pénibles, qu’on croyait qu’ils ne pouvaient être faits que par des esclaves ou par des criminels. Mais on sait qu’aujourd’hui les hommes qui y sont employés vivent heureux\footnote{On peut se faire instruire de ce qui se passe, à cet égard, dans les mines du Hartz dans la basse Allemagne, et dans celles de Hongrie.}. On a, par de petits privilèges, encouragé cette profession ; on a joint à l’augmentation du travail celle du gain ; et on est parvenu à leur faire aimer leur condition plus que toute autre qu’ils eussent pu prendre.\par
Il n’y a point de travail si pénible qu’on ne puisse proportionner à la force de celui qui le fait, pourvu que ce soit la raison, et non pas l’avarice, qui le règle. On peut, par la commodité des machines que l’art invente ou applique, suppléer au travail forcé qu’ailleurs on fait faire aux esclaves. Les mines des Turcs, dans le banat de Témeswar, étaient plus riches que celles de Hongrie, et elles ne produisaient pas tant, parce qu’ils n’imaginaient jamais que les bras de leurs esclaves.\par
Je ne sais si c’est l’esprit ou le cœur qui me dicte cet article-ci. Il n’y a peut-être pas de climat sur la terre où l’on ne pût engager au travail des hommes libres. Parce que les lois étaient mal faites, on a trouvé des hommes paresseux : parce que ces hommes étaient paresseux, on les a mis dans l’esclavage.
\subsubsection[{Chapitre IX. Des nations chez lesquelles la liberté civile est généralement établie}]{Chapitre IX. Des nations chez lesquelles la liberté civile est généralement établie}
\noindent On entend dire tous les jours qu’il serait bon que parmi nous il y eût des esclaves.\par
Mais, pour bien juger de ceci, il ne faut pas examiner s’ils seraient utiles à la petite partie riche et voluptueuse de chaque nation ; sans doute qu’ils lui seraient utiles ; mais, prenant un autre point de vue, je ne crois pas qu’aucun de ceux qui la composent voulût tirer au sort pour savoir qui devrait former la partie de la nation qui serait libre, et celle qui serait esclave. Ceux qui parlent le plus pour l’esclavage l’auraient le plus en horreur, et les hommes les plus misérables en auraient horreur de même. Le cri pour l’esclavage est donc le cri du luxe et de la volupté, et non pas celui de l’amour de la félicité publique. Qui peut douter que chaque homme, en particulier, ne fût très content d’être le maître des biens, de l’honneur et de la vie des autres ; et que toutes ses passions ne se réveillassent d’abord à cette idée ? Dans ces choses, voulez-vous savoir si les désirs de chacun sont légitimes ? Examinez les désirs de tous.
\subsubsection[{Chapitre X. Diverses espèces d’esclavage}]{Chapitre X. Diverses espèces d’esclavage}
\noindent Il y a deux sortes de servitude : la réelle et la personnelle. La réelle est celle qui attache l’esclave au fonds de terre. C’est ainsi qu’étaient les esclaves chez les Germains, au rapport de Tacite\footnote{{\itshape De moribus Germanorum}, XXV.}. Ils n’avaient point d’office dans la maison ; ils rendaient à leur maître une certaine quantité de blé, de bétail, ou d’étoffe : l’objet de leur esclavage n’allait pas plus loin. Cette espèce de servitude est encore établie en Hongrie, en Bohême et dans plusieurs endroits de la basse Allemagne.\par
La servitude personnelle regarde le ministère de la maison, et se rapporte plus à la personne du maître.\par
L’abus extrême de l’esclavage est lorsqu’il est, en même temps, personnel et réel. Telle était la servitude des Ilotes chez les Lacédémoniens ; ils étaient soumis à tous les travaux hors de la maison, et à toutes sortes d’insultes dans la maison : cette {\itshape ilotie} est contre la nature des choses. Les peuples simples n’ont qu’un esclavage réel\footnote{« Vous ne pourriez, dit Tacite ({\itshape Sur les mœurs des Germains}), distinguer le maître de l’esclave, par les délices de la vie. »}, parce que leurs femmes et leurs enfants font les travaux domestiques. Les peuples voluptueux ont un esclavage personnel, parce que le luxe demande le service des esclaves dans la maison. Or l’ilotie joint, dans les mêmes personnes, l’esclavage établi chez les peuples voluptueux, et celui qui est établi chez les peuples simples.
\subsubsection[{Chapitre XI. Ce que les lois doivent faire par rapport à l’esclavage}]{Chapitre XI. Ce que les lois doivent faire par rapport à l’esclavage}
\noindent Mais, de quelque nature que soit l’esclavage, il faut que les lois civiles cherchent à en ôter, d’un côté, les abus, et, de l’autre, les dangers.
\subsubsection[{Chapitre XII. Abus de l’esclavage}]{Chapitre XII. Abus de l’esclavage}
\noindent Dans les États mahométans\footnote{Voyez Chardin, {\itshape Voyage de Perse.}}, on est non seulement maître de la vie et des biens des femmes esclaves, mais encore de ce qu’on appelle leur vertu ou leur honneur. C’est un des malheurs de ces pays, que la plus grande partie de la nation n’y soit faite que pour servir à la volupté de l’autre.\par
Cette servitude est récompensée par la paresse dont on fait jouir de pareils esclaves ; ce qui est encore pour l’État un nouveau malheur.\par
C’est cette paresse qui rend les sérails d’Orient\footnote{Voyez Chardin, t. II, dans sa description du marché d’Izagour.} des lieux de délices pour ceux mêmes contre qui ils sont faits. Des gens qui ne craignent que le travail peuvent trouver leur bonheur dans ces lieux tranquilles. Mais on voit que par là on choque même l’esprit de l’établissement de l’esclavage.\par
La raison veut que le pouvoir du maître ne s’étende point au-delà des choses qui sont de son service : il faut que l’esclavage soit pour l’utilité, et non pas pour la volupté. Les lois de la pudicité sont du droit naturel, et doivent être senties par toutes les nations du monde.\par
Que si la loi qui conserve la pudicité des esclaves est bonne dans les États où le pouvoir sans bornes se joue de tout, combien le sera-t-elle dans les monarchies ? combien le sera-t-elle dans les États républicains ?\par
Il y a une disposition de la loi\footnote{Liv. I, tit. XXXII, § 5.} des Lombards, qui paraît bonne pour tous les gouvernements : « Si un maître débauche la femme de son esclave, ceux-ci seront tous deux libres. » Tempérament admirable pour prévenir et arrêter, sans trop de rigueur, l’incontinence des maîtres.\par
Je ne vois pas que les Romains aient eu, à cet égard, une bonne police. Ils lâchèrent la bride à l’incontinence des maîtres ; ils privèrent même, en quelque façon, leurs esclaves du droit des mariages. C’était la partie de la nation la plus vile ; mais quelque vile qu’elle fût, il était bon qu’elle eût des mœurs ; et de plus, en lui Ôtant les mariages, on corrompait ceux des citoyens.
\subsubsection[{Chapitre XIII. Danger du grand nombre d’esclaves}]{Chapitre XIII. Danger du grand nombre d’esclaves}
\noindent Le grand nombre d’esclaves a des effets différents dans les divers gouvernements. Il n’est point à charge dans le gouvernement despotique ; l’esclavage politique établi dans le corps de l’État fait que l’on sent peu l’esclavage civil. Ceux que l’on appelle hommes libres ne le sont guère plus que ceux qui n’y ont pas ce titre ; et ceux-ci, en qualité d’eunuques, d’affranchis ou d’esclaves, ayant en main presque toutes les affaires, lu condition d’un homme libre et celle d’un esclave se touchent de fort près. Il est donc presque indifférent que peu ou beaucoup de gens y vivent dans l’esclavage.\par
Mais, dans les États modérés, il est très important qu’il n’y ait point trop d’esclaves. La liberté politique y rend précieuse la liberté civile ; et celui qui est privé de cette dernière est encore privé de l’autre. Il voit une société heureuse dont il n’est pas même partie ; il trouve la sûreté établie pour les autres, et non pas pour lui ; il sent que son maître a une âme qui peut s’agrandir, et que la sienne est contrainte de s’abaisser sans cesse. Rien ne met plus près de la condition des bêtes que de voir toujours des hommes libres, et de ne l’être pas. De telles gens sont des ennemis naturels de la société ; et leur nombre serait dangereux.\par
Il ne faut donc pas être étonné que, dans les gouvernements modérés, l’État ait été si troublé par la révolte des esclaves, et que cela soit arrivé si rarement\footnote{La révolte des mamelucks était un cas particulier : c’était un corps de milice qui usurpa l’empire.} dans les États despotiques.
\subsubsection[{Chapitre XIV. Des esclaves armés}]{Chapitre XIV. Des esclaves armés}
\noindent Il est moins dangereux dans la monarchie d’armer les esclaves que dans les républiques. Là, un peuple guerrier, un corps de noblesse, contiendront assez ces esclaves armés. Dans la république, des hommes uniquement citoyens ne pourront guère contenir des gens qui, ayant les armes à la main, se trouveront égaux aux citoyens.\par
Les Goths qui conquirent l’Espagne se répandirent dans le pays, et bientôt se trouvèrent très faibles. Ils firent trois règlements considérables : ils abolirent l’ancienne coutume qui leur défendait de\footnote{Loi des Wisigoths, liv. III, tit. I, § 1.} s’allier par mariage avec les Romains ; ils établirent que tous les affranchis\footnote{{\itshape Ibid.}, liv. V, tit. VII, § 20.} du fisc iraient à la guerre, sous peine d’être réduits en servitude ; ils ordonnèrent que chaque Goth mènerait à la guerre et armerait la dixième\footnote{{\itshape Ibid.}, liv. IX, tit. II, § 9.} partie de ses esclaves. Ce nombre était peu considérable en comparaison de ceux qui restaient. De plus, ces esclaves, menés à la guerre par leur maître, ne faisaient pas un corps séparé ; ils étaient dans l’armée, et restaient, pour ainsi dire, dans la famille.
\subsubsection[{Chapitre XV. Continuation du même sujet}]{Chapitre XV. Continuation du même sujet}
\noindent Quand toute la nation est guerrière, les esclaves armés sont encore moins à craindre.\par
Par la loi des Allemands, un esclave qui volait\footnote{Loi des Allemands, chap. V, § 3.} une chose qui avait été déposée était soumis à la peine qu’on aurait infligée à un homme libre ; mais s’il l’enlevait\footnote{{\itshape Ibid.}, chap. V, § 5, {\itshape per virtutem}.} par violence, il n’était obligé qu’à la restitution de la chose enlevée. Chez les\par
Allemands, les actions qui avaient pour principe le courage et la force n’étaient point odieuses. Ils se servaient de leurs esclaves dans leurs guerres.\par
Dans la plupart des républiques, on a toujours cherché à abattre le courage des esclaves ; le peuple allemand, sûr de lui-même, songeait à augmenter l’audace des siens ; toujours armé, il ne craignait rien d’eux ; c’étaient des instruments de ses brigandages ou de sa gloire.
\subsubsection[{Chapitre XVI. Précautions à prendre dans le gouvernement modéré}]{Chapitre XVI. Précautions à prendre dans le gouvernement modéré}
\noindent L’humanité que l’on aura pour les esclaves pourra prévenir dans l’État modéré les dangers que l’on pourrait craindre de leur trop grand nombre. Les hommes s’accoutument à tout, et à la servitude même, pourvu que le maître ne soit pas plus dur que la servitude. Les Athéniens traitaient leurs esclaves avec une grande douceur : on ne voit point qu’ils aient troublé l’État à Athènes, comme ils ébranlèrent celui de Lacédémone.\par
On ne voit point que les premiers Romains aient eu des inquiétudes à l’occasion de leurs esclaves. Ce fut lorsqu’ils eurent perdu pour eux tous les sentiments de l’humanité, que l’on vit naître ces guerres civiles qu’on a comparées aux guerres puniques\footnote{« La Sicile, dit Florus, plus cruellement dévastée par la guerre servile que par la guerre punique. » Liv. III.}.\par
Les nations simples, et qui s’attachent elles-mêmes au travail, ont ordinairement plus de douceur pour leurs esclaves que celles qui y ont renoncé. Les premiers Romains vivaient, travaillaient et mangeaient avec leurs esclaves ; ils avaient pour eux beaucoup de douceur et d’équité : la plus grande peine qu’ils leur infligeassent était de les faire passer devant leurs voisins avec un morceau de bois fourchu sur le dos. Les mœurs suffisaient pour maintenir la fidélité des esclaves ; il ne fallait point de lois.\par
Mais, lorsque les Romains se furent agrandis, que leurs esclaves ne furent plus les compagnons de leur travail, mais les instruments de leur luxe et de leur orgueil ; comme il n’y avait point de mœurs, on eut besoin de lois. Il en fallut même de terribles pour établir la sûreté de ces maîtres cruels qui vivaient au milieu de leurs esclaves comme au milieu de leurs ennemis.\par
On fit le sénatus-consulte {\itshape Sillanien} et d’autres lois\footnote{Voyez tout le titre {\itshape De senat. consult. Sillan.} au Digeste.} qui établirent que, lorsqu’un maître serait tué, tous les esclaves qui étaient sous le même toit, ou dans un lieu assez près de la maison pour qu’on pût entendre la voix d’un homme, seraient, sans distinction, condamnés à la mort. Ceux qui, dans ce cas, réfugiaient un esclave pour le sauver étaient punis comme meurtriers\footnote{Leg. {\itshape Si quis}, § 12, ff. {\itshape De senat. consult. Sillan}.}. Celui-là même à qui son maître aurait ordonné\footnote{Quand Antoine commanda à Éros de le tuer, ce n’était point lui commander de le tuer, mais de se tuer lui-même, puisque, s’il lui eût obéi, il aurait été puni comme meurtrier de son maître.} de le tuer, et qui lui aurait obéi, aurait été coupable ; celui qui ne l’aurait point empêché de se tuer lui-même, aurait été puni\footnote{Leg. I, § 22, ff. {\itshape De senat. consult. Sillan.}}. Si un maître avait été tué dans un voyage, on faisait mourir\footnote{Leg. I, § 31, ff. {\itshape ibid.}} ceux qui étaient restés avec lui, et ceux qui s’étaient enfuis.\par
Toutes ces lois avaient lieu contre ceux mêmes dont l’innocence était prouvée ; elles avaient pour objet de donner aux esclaves pour leur maître un respect prodigieux. Elles n’étaient pas dépendantes du gouvernement civil, mais d’un vice ou d’une imperfection du gouvernement civil. Elles ne dérivaient point de l’équité des lois civiles, puisqu’elles étaient contraires aux principes des lois civiles. Elles étaient proprement fondées sur le principe de la guerre, à cela près que c’était dans le sein de l’État qu’étaient les ennemis. Le sénatus-consulte Sillanien dérivait du droit des gens, qui veut qu’une société, même imparfaite, se conserve.\par
CI est un malheur du gouvernement lorsque la magistrature se voit contrainte de faire ainsi des lois cruelles. C’est parce qu’on a rendu l’obéissance difficile que l’on est obligé d’aggraver la peine de la désobéissance, ou de soupçonner la fidélité. Un législateur prudent prévient le malheur de devenir un législateur terrible. C’est parce que les esclaves ne purent avoir, chez les Romains, de confiance dans la loi, que la loi ne put avoir de confiance en eux.
\subsubsection[{Chapitre XVII. Règlements à faire entre le maître et les esclaves}]{Chapitre XVII. Règlements à faire entre le maître et les esclaves}
\noindent Le magistrat doit veiller à ce que l’esclave ait sa nourriture et son vêtement : cela doit être réglé par la loi.\par
Les lois doivent avoir attention qu’ils soient soignés dans leurs maladies et dans leur vieillesse. Claude\footnote{Xiphilin, {\itshape in Claudio}.} ordonna que les esclaves qui auraient été abandonnés par leurs maîtres étant malades, seraient libres s’ils échappaient. Cette loi assurait leur liberté ; il aurait encore fallu assurer leur vie.\par
Quand la loi permet au maître d’ôter la vie à son esclave, c’est un droit qu’il doit exercer comme juge, et non pas comme maître : il faut que la loi ordonne des formalités qui ôtent le soupçon d’une action violente.\par
Lorsqu’à Rome il ne fut plus permis aux pères de faire mourir leurs enfants, les magistrats infligèrent\footnote{Voyez la loi 3 au Code {\itshape de patria potestate}, qui est de l’empereur Alexandre.} la peine que le père voulait prescrire. Un usage pareil entre le maître et les esclaves serait raisonnable dans les pays où les maîtres ont droit de vie et de mort.\par
La loi de Moïse était bien rude. « Si quelqu’un frappe son esclave, et qu’il meure sous sa main, il sera puni ; mais s’il survit un jour ou deux, il ne le sera pas, parce que c’est son argent. » Quel peuple que celui où il fallait que la loi civile se relâchât de la loi naturelle !\par
Par une loi des Grecs\footnote{Plutarque, {\itshape De la superstition}.}, les esclaves trop rudement traités par leurs maîtres pouvaient demander d’être vendus à un autre. Dans les derniers temps, il y eut à Rome une pareille loi\footnote{Voyez la constitution d’Antonin Pie. {\itshape Institut}., liv. I, tit. VII.}. Un maître irrité contre son esclave, et un esclave irrité contre son maître, doivent être séparés.\par
Quand un citoyen maltraite l’esclave d’un autre, il faut que celui-ci puisse aller devant le juge. Les {\itshape Lois}\footnote{Liv. IX.} de Platon et de la plupart des peuples ôtent aux esclaves la défense naturelle : il faut donc leur donner la défense civile.\par
À Lacédémone, les esclaves ne pouvaient avoir aucune justice contre les insultes ni contre les injures. L’excès de leur malheur était tel qu’ils n’étaient pas seulement esclaves d’un citoyen, mais encore du public ; ils appartenaient à tous et à un seul. À Rome, dans le tort fait à un esclave, on ne considérait que l’intérêt du maître\footnote{Ce fut encore souvent l’esprit des lois des peuples qui sortirent de la Germanie, comme on le peut voir par leurs codes.}. On confondait, sous l’action de la loi Aquilienne, la blessure faite à une bête et celle faite à un esclave ; on n’avait attention qu’à la diminution de leur prix. À Athènes\footnote{Démosthène, {\itshape Oratio contra Midiam}, p. 610, éd. de Francfort, de l’an 1604.}, on punissait sévèrement, quelquefois même de mort, celui qui avait maltraité l’esclave d’un autre. La loi d’Athènes, avec raison, ne voulait point ajouter la perte de la sûreté à celle de la liberté.
\subsubsection[{Chapitre XVIII. Des affranchissements}]{Chapitre XVIII. Des affranchissements}
\noindent On sent bien que quand, dans le gouvernement républicain, on a beaucoup d’esclaves, il faut en affranchir beaucoup. Le mal est que, si on a trop d’esclaves, ils ne peuvent être contenus ; si l’on a trop d’affranchis, ils ne peuvent pas vivre, et ils deviennent à charge à la république : outre que celle-ci peut être également en danger de la part d’un trop grand nombre d’affranchis et de la part d’un trop grand nombre d’esclaves. Il faut donc que les lois aient l’œil sur ces deux inconvénients.\par
Les diverses lois et les sénatus-consultes qu’on fit à Rome pour et contre les esclaves, tantôt pour gêner, tantôt pour faciliter les affranchissements, font bien voir l’embarras où l’on se trouva à cet égard. Il y eut même des temps où l’on n’osa pas faire des lois. Lorsque, sous Néron\footnote{Tacite, {\itshape Annales}, liv. XIII.}, on demanda au sénat qu’il fût permis aux patrons de remettre en servitude les affranchis ingrats, l’empereur écrivit qu’il fallait juger les affaires particulières, et ne rien statuer de général.\par
Je ne saurais guère dire quels sont les règlements qu’une bonne république doit faire là-dessus ; cela dépend trop des circonstances. Voici quelques réflexions.\par
Il ne faut pas faire tout à coup, et par une loi générale, un nombre considérable d’affranchissements. On sait que, chez les Volsiniens\footnote{Supplément de Freinshemius, II\textsuperscript{e} décade, liv. V.}, les affranchis, devenus maîtres des suffrages, firent une abominable loi qui leur donnait le droit de coucher les premiers avec les filles qui se mariaient à des ingénus.\par
Il y a diverses manières d’introduire insensiblement de nouveaux citoyens dans la république. Les lois peuvent favoriser le pécule, et mettre les esclaves en état d’acheter leur liberté. Elles peuvent donner un terme à la servitude, comme celles de Moïse, qui avaient borné à six ans celle des esclaves hébreux\footnote{Exode, chap. XXI.}. Il est aisé d’affranchir toutes les années un certain nombre d’esclaves parmi ceux qui, par leur âge, leur santé, leur industrie, auront le moyen de vivre. On peut même guérir le mal dans sa racine : comme le grand nombre d’esclaves est lié aux divers emplois qu’on leur donne, transporter aux ingénus une partie de ces emplois, par exemple le commerce ou la navigation, c’est diminuer le nombre des esclaves.\par
Lorsqu’il y a beaucoup d’affranchis, il faut que les lois civiles fixent ce qu’ils doivent à leur patron, ou que le contrat d’affranchissement fixe ces devoirs pour elles.\par
On sent que leur condition doit être plus favorisée dans l’état civil que dans l’état politique, parce que, dans le gouvernement même populaire, la puissance ne doit point tomber entre les mains du bas peuple.\par
À Rome, où il y avait tant d’affranchis, les lois politiques furent admirables à leur égard. On leur donna peu, et on ne les exclut presque de rien. Ils eurent bien quelque part à la législation, mais ils n’influaient presque point dans les résolutions qu’on pouvait prendre. Ils pouvaient avoir part aux charges et au sacerdoce même\footnote{Tacite, {\itshape Annales}, liv. XIII.} {\itshape ;} mais ce privilège était, en quelque façon, rendu vain par les désavantages qu’ils avaient dans les élections. Ils avaient droit d’entrer dans la milice ; mais, pour être soldat, il fallait un certain cens. Rien n’empêchait les affranchis\footnote{Harangue d’Auguste, dans Dion, liv. LVI.} de s’unir par mariage avec les familles ingénues ; mais il ne leur était pas permis de s’allier avec celles des sénateurs. Enfin leurs enfants étaient ingénus, quoiqu’ils ne le fussent pas eux-mêmes.
\subsubsection[{Chapitre XIX. Des affranchis et des eunuques}]{Chapitre XIX. Des affranchis et des eunuques}
\noindent Ainsi, dans le gouvernement de plusieurs, il est souvent utile que la condition des affranchis soit peu au-dessous de celle des ingénus, et que les lois travaillent à leur ôter le dégoût de leur condition. Mais, dans le gouvernement d’un seul, lorsque le luxe et le pouvoir arbitraire règnent, on n’a rien à faire à cet égard. Les affranchis se trouvent presque toujours au-dessus des hommes libres : ils dominent à la cour du prince et dans les palais des grands ; et, comme ils ont étudié les faiblesses de leur maître, et non pas ses vertus, ils le font régner, non pas par ses vertus, mais par ses faiblesses. Tels étaient à Rome les affranchis du temps des empereurs.\par
Lorsque les principaux esclaves sont eunuques, quelque privilège qu’on leur accorde, on ne peut guère les regarder comme des affranchis. Car, comme ils ne peuvent avoir de famille, ils sont, par leur nature, attachés à une famille ; et ce n’est que par une espèce de fiction qu’on peut les considérer comme citoyens.\par
Cependant il y a des pays où on leur donne toutes les magistratures : « Au Tonquin, dit Dampierre\footnote{T. III p. 91.}, tous les mandarins civils et militaires sont eunuques\footnote{C’était autrefois de même à la Chine. Les deux Arabes mahométans qui y voyagèrent au \textsc{ix}\textsuperscript{e} siècle, disent {\itshape l’eunuque}, quand ils veulent parler du gouverneur d’une ville.}. » Ils n’ont point de famille ; et quoiqu’ils soient naturellement avares, le maître ou le prince profite à la fin de leur avarice même.\par
Le même Dampierre\footnote{T. III, p. 94.} nous dit que, dans ce pays, les eunuques ne peuvent se passer des femmes, et qu’ils se marient. La loi qui leur permet le mariage ne peut être fondée, d’un côté, que sur la considération que l’on y a pour de pareilles gens ; et de l’autre, sur le mépris qu’on y a pour les femmes.\par
Ainsi l’on confie à ces gens-là les magistratures, parce qu’ils n’ont point de famille ; et, d’un autre côté, on leur permet de se marier, parce qu’ils ont les magistratures.\par
C’est pour lors que les sens qui restent veulent obstinément suppléer à ceux que l’on a perdus ; et que les entreprises du désespoir sont une espèce de jouissance. Ainsi, dans Milton, cet esprit à qui il ne reste que des désirs, pénétré de sa dégradation, veut faire usage de son impuissance même.\par
On voit, dans l’histoire de la Chine, un grand nombre de lois pour ôter aux eunuques tous les emplois civils et militaires ; mais ils reviennent toujours. Il semble que les eunuques en Orient soient un mal nécessaire.
\subsection[{Livre seizième. Comment les lois de l’esclavage domestique ont du rapport avec, la nature du climat}]{Livre seizième. Comment les lois de l’esclavage domestique ont du rapport avec \\
la nature du climat}
\subsubsection[{Chapitre I. De la servitude domestique}]{Chapitre I. De la servitude domestique}
\noindent Les esclaves sont plutôt établis pour la famille qu’ils ne sont dans la famille. Ainsi, je distinguerai leur servitude de celle où sont les femmes dans quelques pays, et que j’appellerai proprement la servitude domestique.
\subsubsection[{Chapitre II. Que dans les pays du midi il y a dans les deux sexes une inégalité naturelle}]{Chapitre II. Que dans les pays du midi il y a dans les deux sexes une inégalité naturelle}
\noindent Les femmes sont nubiles, dans les climats chauds, à huit, neuf et dix ans : ainsi l’enfance et le mariage y vont presque toujours ensemble\footnote{Mahomet épousa Cadhisja à cinq ans, coucha avec elle à huit. Dans les pays chauds d’Arabie et des Indes, les filles sont nubiles à huit ans, et accouchent l’année d’après. Prideaux, {\itshape Vie de Mahomet}. On voit des femmes, dans les royaumes d’Alger, enfanter à neuf, dix et onze ans. Laugier de Tassis, {\itshape Histoire du royaume d’Alger}, p. 61.}. Elles sont vieilles à vingt : la raison ne se trouve donc jamais chez elles avec la beauté. Quand la beauté demande l’empire, la raison le fait refuser ; quand la raison pourrait l’obtenir, la beauté n’est plus. Les femmes doivent être dans la dépendance, car la raison ne peut leur procurer dans leur vieillesse un empire que la beauté ne leur avait pas donné dans la jeunesse même. Il est donc très simple qu’un homme, lorsque la religion ne s’y oppose pas, quitte sa femme pour en prendre une autre, et que la polygamie s’introduise.\par
Dans les pays tempérés, où les agréments des femmes se conservent mieux, où elles sont plus tard nubiles, et où elles ont des enfants dans un âge plus avancé, la vieillesse de leur mari suit en quelque façon la leur ; et, comme elles y ont plus de raison et de connaissances quand elles se marient, ne fût-ce que parce qu’elles ont plus longtemps vécu, il a dû naturellement s’introduire une espèce d’égalité dans les deux sexes, et par conséquent la loi d’une seule femme.\par
Dans les pays froids, l’usage presque nécessaire des boissons fortes établit l’intempérance parmi les hommes. Les femmes, qui ont à cet égard une retenue naturelle, parce qu’elles ont toujours à se défendre, ont donc encore l’avantage de la raison sur eux.\par
La nature, qui a distingué les hommes par la force et par la raison, n’a mis à leur pouvoir de terme que celui de cette force et de cette raison. Elle a donné aux femmes les agréments, et a voulu que leur ascendant finît avec ces agréments ; mais dans les pays chauds, ils ne se trouvent que dans les commencements, et jamais dans le cours de leur vie.\par
Ainsi la loi qui ne permet qu’une femme se rapporte plus au physique du climat de l’Europe qu’au physique du climat de l’Asie. C’est une des raisons qui a fait que le mahométisme a trouvé tant de facilité à s’établir en Asie, et tant de difficulté à s’étendre en Europe ; que le christianisme s’est maintenu en Europe, et a été détruit en Asie ; et qu’enfin les mahométans font tant de progrès à la Chine, et les chrétiens si peu. Les raisons humaines sont toujours subordonnées à cette cause suprême, qui fait tout ce qu’elle veut, et se sert de tout ce qu’elle veut.\par
Quelques raisons particulières à Valentinien\footnote{Voyez Jomandès, {\itshape De regnorum et temporum successione} et les historiens ecclésiastiques.} lui firent permettre la polygamie dans l’empire. Cette loi, violente pour nos climats, fut ôtée\footnote{Voyez la loi 7 au code {\itshape De judaeis et coelicolis} ; et la {\itshape Novelle} 18, chap. V.} par Théodose, Arcadius et Honorius.
\subsubsection[{Chapitre III. Que la pluralité des femmes dépend beaucoup de leur entretien}]{Chapitre III. Que la pluralité des femmes dépend beaucoup de leur entretien}
\noindent Quoique, dans les pays où la polygamie est une fois établie, le grand nombre des femmes dépende beaucoup des richesses du mari, cependant on ne peut pas dire que ce soient les richesses qui fassent établir dans un État la polygamie : la pauvreté peut faire le même effet, comme je le dirai en parlant des sauvages.\par
La polygamie est moins un luxe, que l’occasion d’un grand luxe chez des nations puissantes. Dans les climats chauds, on a moins de besoins\footnote{À Ceylan, un homme vit pour dix sols par mois : on n’y mange que du riz et du poisson. {\itshape Recueil des voyages qui ont servi à l’établissement de la Compagnie des Indes}, t. II, part. I.} ; il en coûte moins pour entretenir une femme et des enfants. On y peut donc avoir un plus grand nombre de femmes.
\subsubsection[{Chapitre IV. De la polygamie, ses diverses circonstances}]{Chapitre IV. De la polygamie, ses diverses circonstances}
\noindent Suivant les calculs que l’on fait en divers endroits de l’Europe, il y naît plus de garçons que de filles\footnote{M. Arbuthnot trouve qu’en Angleterre le nombre des garçons excède celui des filles : on a eu tort d’en conclure que ce fût la même chose dans tous les climats.} : au contraire, les relations de l’Asie\footnote{Voyez Kempfer, qui nous rapporte un dénombrement de Méaco, où l’on trouve 182 072 mâles et 223 573 femelles.} et de l’Afrique\footnote{Voyez le {\itshape Voyage de Guinée} de M. Smith, partie seconde, sur le pays d’Ante.} nous disent qu’il y naît beaucoup plus de filles que de garçons. La loi d’une seule femme en Europe, et celle qui en permet plusieurs en Asie et en Afrique, ont donc un certain rapport au climat.\par
Dans les climats froids de l’Asie, il naît, comme en Europe, plus de garçons que de filles. C’est, disent les Lamas\footnote{Du Halde, Mémoires de la Chine, t. IV, p. 461.}, la raison de la loi qui, chez eux, permet à une femme d’avoir plusieurs maris\footnote{Albuzéir-el-Hassen, un des deux mahométans arabes qui allèrent aux Indes et à la Chine au IXe siècle, prend cet usage pour une prostitution. C’est que rien ne choquait tant les idées mahométanes.}.\par
Mais je ne crois pas qu’il y ait beaucoup de pays où la disproportion soit assez grande pour qu’elle exige qu’on y introduise la loi de plusieurs femmes, ou la loi de plusieurs maris, Cela veut dire seulement que la pluralité des femmes, ou même la pluralité des hommes, s’éloigne moins de la nature dans de certains pays que dans d’autres.\par
J’avoue que si ce que les relations nous disent était vrai, qu’à Bantam\footnote{{\itshape Recueil des voyages qui ont servi à l’établissement de la Compagnie des Indes}, t. I.} il y a dix femmes pour un homme, ce serait un cas bien particulier de la polygamie.\par
Dans tout ceci je ne justifie pas les usages, mais j’en rends les raisons.
\subsubsection[{Chapitre V. Raison d’une loi du Malabar}]{Chapitre V. Raison d’une loi du Malabar}
\noindent Sur la côte du Malabar, dans la caste des Naïres\footnote{{\itshape Voyage} de François Pyrard, chap. XXVII. {\itshape Lettres édifiantes}, troisième et dixième recueils, sur le Malléami dans la côte du Malabar. Cela est regardé comme un abus de la profession militaire ; et, comme dit Pyrard, une femme de la caste des bramines n’épouserait jamais plusieurs maris.}, les hommes ne peuvent avoir qu’une femme, et une femme au contraire peut avoir plusieurs maris. Je crois qu’on peut découvrir l’origine de cette coutume. Les Naïres sont la caste des nobles, qui sont les soldats de toutes ces nations. En Europe, on empêche les soldats de se marier. Dans le Malabar, où le climat exige davantage, on s’est contenté de leur rendre le mariage aussi peu embarrassant qu’il est possible : on a donné une femme à plusieurs hommes ; ce qui diminue d’autant l’attachement pour une famille et les soins du ménage, et laisse à ces gens l’esprit militaire.
\subsubsection[{Chapitre VI. De la polygamie en elle-même}]{Chapitre VI. De la polygamie en elle-même}
\noindent À regarder la polygamie en général, indépendamment des circonstances qui peuvent la faire un peu tolérer, elle n’est point utile au genre humain, ni à aucun des deux sexes, soit à celui qui abuse, soit à celui dont on abuse. Elle n’est pas non plus utile aux enfants ; et un de ses grands inconvénients est que le père et la mère ne peuvent avoir la même affection pour leurs enfants ; un père ne peut pas aimer vingt enfants, comme une mère en aime deux. C’est bien pis quand une femme a plusieurs maris ; car, pour lors, l’amour paternel ne tient Plus qu’à cette opinion, qu’un père peut croire, s’il veut, ou que les autres peuvent croire, que de certains enfants lui appartiennent.\par
On dit que le roi de Maroc a dans son sérail des femmes blanches, des femmes noires, des femmes jaunes. Le malheureux ! à peine a-t-il besoin d’une couleur.\par
La possession de beaucoup de femmes ne prévient pas toujours les désirs\footnote{C’est ce qui fait que l’on cache avec tant de soin les femmes en Orient.} pour celle d’un autre ; il en est de la luxure comme de l’avarice : elle augmente sa soif par l’acquisition des trésors.\par
Du temps de Justinien, plusieurs philosophes, gênés par le christianisme, se retirèrent en Perse auprès de Cosroës. Ce qui les frappa le plus, dit Agathias\footnote{De la vie et des actions de Justinien, p. 403.}, ce fut que la polygamie était permise à des gens qui ne s’abstenaient pas même de l’adultère.\par
La pluralité des femmes, qui le dirait ! mène à cet amour que la nature désavoue : c’est qu’une dissolution en entraîne toujours une autre. À la révolution qui arriva à Constantinople, lorsqu’on déposa le sultan Achmet, les relations disaient que le peuple ayant pillé la maison du chiaya, on n’y avait pas trouvé une seule femme. On dit qu’à Alger\footnote{Laugier de Tassis, {\itshape Histoire d’Alger}.} on est parvenu à ce point, qu’on n’en a pas dans la plupart des sérails.
\subsubsection[{Chapitre VII. De l’égalité du traitement dans le cas de la pluralité des femmes}]{Chapitre VII. De l’égalité du traitement dans le cas de la pluralité des femmes}
\noindent De la loi de la pluralité des femmes suit celle de l’égalité du traitement. Mahomet, qui en permet quatre, veut que tout soit égal entre elles : nourriture, habits, devoir conjugal. Cette loi est aussi établie aux Maldives\footnote{Voyage de François Pyrard, chap. XII.}, où on peut épouser trois femmes.\par
La loi de Moïse\footnote{{\itshape Exode}, chap. XXI, vers. 10 et 11.} veut même que, si quelqu’un a marié son fils à une esclave, et qu’ensuite il épouse une femme libre, il ne lui ôte rien des vêtements, de la nourriture et des devoirs. On pouvait donner plus à la nouvelle épouse ; mais il fallait que la première n’eût pas moins.
\subsubsection[{Chapitre VIII. De la séparation des femmes d’avec les hommes}]{Chapitre VIII. De la séparation des femmes d’avec les hommes}
\noindent C’est une conséquence de la polygamie, que, dans les nations voluptueuses et riches, on ait un très grand nombre de femmes. Leur séparation d’avec les hommes, et leur clôture, suivent naturellement de ce grand nombre. L’ordre domestique le demande ainsi : un débiteur insolvable cherche à se mettre à couvert des poursuites de ses créanciers. Il y a de tels climats où le physique a une telle force que la morale n’y peut presque rien. Laissez un homme avec une femme ; les tentations seront des chutes, l’attaque sûre, la résistance nulle. Dans ces pays, au lieu de préceptes, il faut des verrous.\par
Un livre classique\footnote{« Trouver à l’écart un trésor dont on soit le maure, ou une belle femme seule dans un appartement reculé ; entendre la voix de son ennemi qui va périr, si on ne le secourt : admirable pierre de touche. » Traduction d’un ouvrage chinois sur la morale, dans le P. Du Halde, t. III, p. 151.} de la Chine regarde comme un prodige de vertu de se trouver seul dans un appartement reculé avec une femme, sans lui faire violence.
\subsubsection[{Chapitre IX. Liaison du gouvernement domestique avec le politique}]{Chapitre IX. Liaison du gouvernement domestique avec le politique}
\noindent Dans une république, la condition des citoyens est bornée, égale, douce, modérée ; tout s’y ressent de la liberté publique. L’empire sur les femmes n’y pourrait pas être si bien exercé ; et, lorsque le climat a demandé cet empire, le gouvernement d’un seul a été le plus convenable. Voilà une des raisons qui a fait que le gouvernement populaire a toujours été difficile à établir en Orient.\par
Au contraire, la servitude des femmes est très conforme au génie du gouvernement despotique, qui aime à abuser de tout. Aussi a-t-on vu, dans tous les temps, en Asie, marcher d’un pas égal la servitude domestique et le gouvernement despotique.\par
Dans un gouvernement où l’on demande surtout la tranquillité, et où la subordination extrême s’appelle la paix, il faut enfermer les femmes ; leurs intrigues seraient fatales au mari. Un gouvernement qui n’a pas le temps d’examiner la conduite des sujets, la tient pour suspecte, par cela seul qu’elle paraît et qu’elle se fait sentir.\par
Supposons un moment que la légèreté d’esprit et les indiscrétions, les goûts et les dégoûts de nos femmes, leurs passions grandes et petites, se trouvassent transportés dans un gouvernement d’Orient, dans l’activité et dans cette liberté où elles sont parmi nous ; quel est le père de famille qui pourrait être un moment tranquille ? Partout des gens suspects, partout des ennemis ; l’État serait ébranlé, on verrait couler des flots de sang.
\subsubsection[{Chapitre X. Principe de la morale d’Orient}]{Chapitre X. Principe de la morale d’Orient}
\noindent Dans le cas de la multiplicité des femmes, plus la famille cesse d’être une, plus les lois doivent réunir à un centre ces parties détachées ; et plus les intérêts sont divers, plus il est bon que les lois les ramènent à un intérêt.\par
Cela se fait surtout par la clôture. Les femmes ne doivent pas seulement être séparées des hommes par la clôture de la maison, mais elles en doivent encore être séparées dans cette même clôture, en sorte qu’elles y fassent comme une famille particulière dans la famille. De là dérive pour les femmes toute la pratique de la morale : la pudeur, la chasteté, la retenue, le silence, la paix, la dépendance, le respect, l’amour, enfin une direction générale de sentiments à la chose du monde la meilleure par sa nature, qui est l’attachement unique à sa famille.\par
Les femmes ont naturellement à remplir tant de devoirs qui leur sont propres, qu’on ne peut assez les séparer de tout ce qui pourrait leur donner d’autres idées, de tout ce qu’on traite d’amusements et de tout ce qu’on appelle des affaires.\par
On trouve des mœurs plus pures dans les divers États d’Orient, à proportion que la clôture des femmes y est plus exacte. Dans les grands États, il y a nécessairement des grands seigneurs. Plus ils ont de grands moyens, plus ils sont en état de tenir les femmes dans une exacte clôture, et de les empêcher de rentrer dans la société. C’est pour cela que, dans les empires du Turc, de Perse, du Mogol, de la Chine et du Japon, les mœurs des femmes sont admirables.\par
On ne peut pas dire la même chose des Indes, que le nombre infini d’îles et la situation du terrain ont divisées en une infinité de petits États, que le grand nombre des causes, que je n’ai pas le temps de rapporter ici, rendent despotiques.\par
Là, il n’y a que des misérables qui pillent, et des misérables qui sont pillés. Ceux qu’on appelle des grands n’ont que de très petits moyens ; ceux que l’on appelle des gens riches n’ont guère que leur subsistance. La clôture des femmes n’y peut être aussi exacte ; l’on n’y peut pas prendre d’aussi grandes précautions pour les contenir ; la corruption de leurs mœurs y est inconcevable.\par
C’est là qu’on voit jusqu’à quel point les vices du climat, laissés dans une grande liberté, peuvent porter le désordre. C’est là que la nature a une force, et la pudeur une faiblesse qu’on ne peut comprendre. À Patane\footnote{{\itshape Recueil des voyages qui ont servi à l’établissement de la Compagnie des Indes}, t. II, part. II, p. 196.}, la lubricité\footnote{{\itshape Aux} Maldives, les pères marient leurs filles à dix et onze ans, parce que c’est un grand péché, disent-ils, de leur laisser endurer nécessité d’hommes. {\itshape Voyage} de François Pyrard, chap. XII,. À Bantam, sitôt qu’une fille a treize ou quatorze ans, il faut la marier, si l’on ne veut qu’elle mène une vie débordée. {\itshape Recueil des voyages qui ont servi à l’établissement de la Compagnie des Indes}, p. 348.} des femmes est si grande, que les hommes sont contraints de se faire de certaines garnitures pour se mettre à l’abri de leurs entreprises. Selon M. Smith\footnote{{\itshape Voyage de Guinée}, seconde partie, p. 192 de la traduction : « Quand les femmes, dit-il, rencontrent un homme, elles le saisissent et le menacent de le dénoncer à leur mari, s’il les méprise. Elles se glissent dans le lit d’un homme, elles le réveillent, et s’il les refuse, elles le menacent de se laisser prendre sur le fait. »}, les choses ne vont pas mieux dans les petits royaumes de Guinée. Il semble que, dans ces pays-là, les deux sexes perdent jusqu’à leurs propres lois.
\subsubsection[{Chapitre XI. De la servitude domestique indépendante de la polygamie}]{Chapitre XI. De la servitude domestique indépendante de la polygamie}
\noindent Ce n’est pas seulement la pluralité des femmes qui exige leur clôture dans de certains lieux d’Orient ; c’est le climat. Ceux qui liront les horreurs, les crimes, les perfidies, les noirceurs, les poisons, les assassinats, que la liberté des femmes fait faire à Goa et dans les établissements des Portugais dans les Indes, où la religion ne permet qu’une femme, et qui les compareront à l’innocence et à la pureté des mœurs des femmes de Turquie, de Perse, du Mogol, de la Chine et du Japon, verront bien qu’il est souvent aussi nécessaire de les séparer des hommes, lorsqu’on n’en a qu’une, que quand on en a plusieurs.\par
C’est le climat qui doit décider de ces choses. Que servirait d’enfermer les femmes dans nos pays du Nord, où leurs mœurs sont naturellement bonnes ; où toutes leurs passions sont calmes, peu actives, peu raffinées ; où l’amour a sur le cœur un empire si réglé, que la moindre police suffit pour les conduire ?\par
Il est heureux de vivre dans ces climats qui permettent qu’on se communique ; où le sexe qui a le plus d’agréments semble parer la société ; et où les femmes, se réservant aux plaisirs d’un seul, servent encore à l’amusement de tous.
\subsubsection[{Chapitre XII. De la pudeur naturelle}]{Chapitre XII. De la pudeur naturelle}
\noindent Toutes les nations se sont également accordées à attacher du mépris à l’incontinence des femmes : c’est que la nature a parlé à toutes les nations. Elle a établi la défense, elle a établi l’attaque ; et, ayant mis des deux côtés des désirs, elle a placé dans l’un la témérité, et dans l’autre la honte. Elle a donné aux individus, pour se conserver, de longs espaces de temps, et ne leur a donné pour se perpétuer, que des moments.\par
Il n’est donc pas vrai que l’incontinence suive les lois de la nature ; elle les viole au contraire. C’est la modestie et la retenue qui suivent ces lois.\par
D’ailleurs il est de la nature des êtres intelligents de sentir leurs imperfections : la nature a donc mis en nous la pudeur, c’est-à-dire la honte de nos imperfections.\par
Quand donc la puissance physique de certains climats viole la loi naturelle des deux sexes et celle des êtres intelligents, c’est au législateur à faire des lois civiles qui forcent la nature du climat et rétablissent les lois primitives.
\subsubsection[{Chapitre XIII. De la jalousie}]{Chapitre XIII. {\itshape De la jalousie}}
\noindent Il faut bien distinguer, chez les peuples, la jalousie de passion d’avec la jalousie de coutume, de mœurs, de lois. L’une est une fièvre ardente qui dévore ; l’autre, froide, mais quelquefois terrible, peut s’allier avec l’indifférence et le mépris.\par
L’une, qui est un abus de l’amour, tire sa naissance de l’amour même. L’autre tient uniquement aux mœurs, aux manières de la nation, aux lois du pays, à la morale, et quelquefois même à la religion\footnote{Mahomet recommanda à ses sectateurs de garder leurs femmes. Un certain imam dit, en mourant, la même chose, et Confucius n’a pas moins prêché cette doctrine.}.\par
Elle est presque toujours l’effet de la force physique du climat, et elle est le remède de cette force physique.
\subsubsection[{Chapitre XIV. Du gouvernement de la maison en Orient}]{Chapitre XIV. Du gouvernement de la maison en Orient}
\noindent On change si souvent de femmes en Orient qu’elles ne peuvent avoir le gouvernement domestique. On en charge donc les eunuques ; on leur remet toutes les clefs, et ils ont la disposition des affaires de la maison. « En Perse, dit M. Chardin, on donne aux femmes leurs habits, comme on ferait à des enfants. » Ainsi ce soin qui semble leur convenir si bien, ce soin qui, partout ailleurs, est le premier de leurs soins, ne les regarde pas.
\subsubsection[{Chapitre XV. Du divorce et de la répudiation}]{Chapitre XV. Du divorce et de la répudiation}
\noindent Il y a cette différence entre le divorce et la répudiation, que le divorce se fait par un consente ment mutuel à l’occasion d’une incompatibilité mutuelle ; au lieu que la répudiation se fait par la volonté et pour l’avantage d’une des deux parties, indépendamment de la volonté et de l’avantage de l’autre.\par
il est quelquefois si nécessaire aux femmes de répudier, et il leur est toujours si fâcheux de le faire, que la loi est dure, qui donne ce droit aux hommes sans le donner aux femmes. Un mari est le maître de la maison ; il a mille moyens de tenir ou de remettre ses femmes dans le devoir ; et il semble que, dans ses mains, la répudiation ne soit qu’un nouvel abus de sa puissance. Mais une femme qui répudie n’exerce qu’un triste remède. C’est toujours un grand malheur pour elle d’être contrainte d’aller chercher un second mari, lorsqu’elle a perdu la plupart de ses agréments chez un autre. C’est un des avantages des charmes de la jeunesse dans les femmes, que, dans un âge avancé, un mari se porte à la bienveillance par le souvenir de ses plaisirs.\par
C’est donc une règle générale que, dans tous les pays où la loi accorde aux hommes la faculté de répudier, elle doit aussi l’accorder aux femmes. Il y a plus : dans les climats où les femmes vivent sous un esclavage domestique, il semble que la loi doive permettre aux femmes la répudiation, et aux maris seulement le divorce.\par
Lorsque les femmes sont dans un sérail, le mari ne peut répudier pour cause d’incompatibilité de mœurs : c’est la faute du mari, si les mœurs sont incompatibles.\par
La répudiation pour raison de la stérilité de la femme ne saurait avoir lieu que dans le cas d’une femme unique\footnote{Cela ne signifie pas que la répudiation, pour raison de stérilité, soit permise dans le christianisme.} {\itshape : lorsque} l’on a plusieurs femmes, cette raison n’est, pour le mari, d’aucune importance.\par
La loi des Maldives\footnote{{\itshape Voyage} de François Pyrard. On la reprend plutôt qu’une autre, parce que, dans ce cas, il faut moins de dépenses.} permet de reprendre une femme qu’on a répudiée. La loi du Mexique\footnote{{\itshape Histoire de sa conquête}, par Solis, p. 499.} défendait de se réunir, sous peine de la vie. La loi du Mexique était plus sensée que celle des Maldives ; dans le temps même de la dissolution, elle songeait à l’éternité du mariage : au lieu que la loi des Maldives semble se jouer également du mariage et de la répudiation.\par
La loi du Mexique n’accordait que le divorce. C’était une nouvelle raison pour ne point permettre à des gens, qui s’étaient volontairement séparés, de se réunir. La répudiation semble plutôt tenir à la promptitude de l’esprit et à quelque passion de l’âme ; le divorce semble être une affaire de conseil.\par
Le divorce a ordinairement une grande utilité politique ; et quant à l’utilité civile, il est établi pour le mari et pour la femme, et n’est pas toujours favorable aux enfants.
\subsubsection[{Chapitre XVI. De la répudiation et du divorce chez les Romains}]{Chapitre XVI. De la répudiation et du divorce chez les Romains}
\noindent Romulus permit au mari de répudier sa femme si elle avait commis un adultère, préparé du poison, ou falsifié les clefs. Il ne donna point aux femmes le droit de répudier leur mari. Plutarque\footnote{{\itshape Vie de Romulus}.} appelle cette loi, une loi très dure.\par
Comme la loi d’Athènes\footnote{C’était une loi de Solon.} donnait à la femme, aussi bien qu’au mari, la faculté de répudier ; et que l’on voit que les femmes obtinrent ce droit chez les premiers Romains, nonobstant la loi de Romulus, il est clair que cette institution fut une de celles que les députés de Rome rapportèrent d’Athènes, et qu’elle fut mise dans les lois des Douze Tables.\par
Cicéron\footnote{{\itshape Mimam res suas sibi habere jussit, ex duodecim tabulis causam addidit. Philipp. II.}} dit que les causes de répudiation venaient de la loi des Douze Tables. On ne peut donc pas douter que cette loi n’eût augmenté le nombre des causes de répudiation établies par Romulus.\par
La faculté du divorce fut encore une disposition, ou du moins une conséquence de la loi des Douze Tables. Car, dès le moment que la femme ou le mari avait séparément le droit de répudier, à plus forte raison pouvaient-ils se quitter de concert, et par une volonté mutuelle.\par
La loi ne demandait point qu’on donnât des causes pour le divorce\footnote{Justinien changea cela. {\itshape Novelle} 117, chap. X.}. C’est que, par la nature de la chose, il faut des causes pour la répudiation, et qu’il n’en faut point pour le divorce ; parce que là où la loi établit des causes qui peuvent rompre le mariage, l’incompatibilité mutuelle est la plus forte de toutes.\par
Denys d’Halicarnasse\footnote{Liv. II.} Valère-Maxime\footnote{Liv. II, chap. IV.} et Aulu-Gelle\footnote{Liv. IV, chap. III.} rapportent un fait qui ne me paraît pas vraisemblable : ils disent que, quoiqu’on eût à Rome la faculté de répudier sa femme, on eut tant de respect pour les auspices, que personne, pendant cinq cent vingt ans\footnote{Selon Denys d’Halicarnasse et Valère-Maxime, et cinq cent vingt-trois, selon Aulu-Gelle. Aussi ne mettent-ils pas les mêmes consuls.}, n’usa de ce droit jusqu’à Carvilius Ruga, qui répudia la sienne pour cause de stérilité. Mais il suffit de connaître la nature de l’esprit humain pour sentir quel prodige ce serait que, la loi donnant à tout un peuple un droit pareil, personne n’en usât. Coriolan, partant pour son exil, conseilla\footnote{Voyez le discours de Véturie, dans Denys d’Halicarnasse, liv. VIII.} à sa femme de se marier à un homme plus heureux que lui. Nous venons de voir que la loi des Douze Tables et les mœurs des Romains étendirent beaucoup la loi de Romulus. Pourquoi ces extensions, si on n’avait jamais fait usage de la faculté de répudier ?\par
De plus, si les citoyens eurent un tel respect pour les auspices, qu’ils ne répudièrent jamais, pourquoi les législateurs de Rome en eurent-ils moins ? Comment la loi corrompit-elle sans cesse les mœurs ?\par
En rapprochant deux passages de Plutarque, on verra disparaître le merveilleux du fait en question. La loi royale\footnote{Plutarque, {\itshape Vie de Romulus}.} permettait au mari de répudier dans les trois cas dont nous avons parlé. « Et elle voulait, dit Plutarque\footnote{Plutarque, {\itshape Vie de Romulus}.}, que celui qui répudierait dans d’autres cas, fût obligé de donner la moitié de ses biens à sa femme, et que l’autre moitié fût consacrée à Cérès. » On pouvait donc répudier dans tous les cas, en se soumettant à la peine. Personne ne le fit avant Carvilius Ruga\footnote{Effectivement, la cause de stérilité n’est point portée par la loi de Romulus. Il y a apparence qu’il ne fut point sujet à la confiscation, puisqu’il suivait l’ordre des censeurs.}, « qui, comme dit encore Plutarque\footnote{Dans la {\itshape Comparaison de Thésée et de Romulus}.}, répudia sa femme pour cause de stérilité, deux cent trente ans après Romulus » ; c’est-à-dire, qu’il la répudia soixante et onze ans avant la loi des Douze Tables, qui étendit le pouvoir de répudier, et les causes de répudiation.\par
Les auteurs que j’ai cités disent que Carvilius Ruga aimait sa femme ; mais qu’à cause de sa stérilité, les censeurs lui firent faire serment qu’il la répudierait, afin qu’il pût donner des enfants à la république ; et que cela le rendit odieux au peuple. Il faut connaître le génie du peuple romain pour découvrir la vraie cause de la haine qu’il conçut pour Carvilius. Ce n’est point parce que Carvilius répudia sa femme qu’il tomba dans la disgrâce du peuple : c’est une chose dont le peuple ne s’embarrassait pas. Mais Carvilius avait fait un serment aux censeurs, qu’attendu la stérilité de sa femme, il la répudierait pour donner des enfants à la république. C’était un joug que le peuple voyait que les censeurs allaient mettre sur lui. Je ferai voir, dans la suite\footnote{Au liv. XXIII, chap. XXI.} de cet ouvrage, les répugnances qu’il eut toujours pour des règlements pareils. Mais d’où peut venir une telle contradiction entre ces auteurs ? Le voici : Plutarque a examiné un fait, et les autres ont raconté une merveille.
\subsection[{Livre dix-septième. Comment les lois de la servitude politique ont du rapport avec la nature du climat}]{Livre dix-septième. Comment les lois de la servitude politique ont du rapport avec la nature du climat}
\subsubsection[{Chapitre I. De la servitude politique}]{Chapitre I. De la servitude politique}
\noindent La servitude politique ne dépend pas moins de la nature du climat, que la civile et la domestique, comme on va le faire voir.
\subsubsection[{Chapitre II. Différence des peuples par rapport au courage}]{Chapitre II. Différence des peuples par rapport au courage}
\noindent Nous avons déjà dit que la grande chaleur énervait la force et le courage des hommes ; et qu’il y avait dans les climats froids une certaine force de corps et d’esprit qui rendait les hommes capables des actions longues, pénibles, grandes et hardies. Cela se remarque non seulement de nation à nation, mais encore dans le même pays, d’une partie à une autre. Les peuples du nord de la Chine\footnote{Le P. Du Halde, t. I, p. 112.} sont plus courageux que ceux du midi ; les peuples du midi de la Corée\footnote{Les livres chinois le disent ainsi. {\itshape Ibid., t.} IV, p. 448.} ne le sont pas tant que ceux du nord.\par
Il ne faut donc pas être étonné que la lâcheté des peuples des climats chauds les ait presque toujours rendus esclaves, et que le courage des peuples des climats froids les ait maintenus libres. C’est un effet qui dérive de sa cause naturelle.\par
Ceci s’est encore trouvé vrai dans l’Amérique ; les empires despotiques du Mexique et du Pérou étaient vers la ligne, et presque tous les petits peuples libres étaient et sont encore vers les pôles.
\subsubsection[{Chapitre III. Du climat de l’Asie}]{Chapitre III. Du climat de l’Asie}
\noindent Les relations nous disent\footnote{Voyez les {\itshape Voyages du Nord}, t. VIII ; l’{\itshape Histoire des Tatars} et le quatrième volume de la {\itshape Chine} du P. Du Halde.} « que le nord de l’Asie, ce vaste continent qui va du quarantième degré, ou environ, jusques au pôle, et des frontières de la Moscovie jusqu’à la mer Orientale, est dans un climat très froid ; que ce terrain immense est divisé de l’ouest à l’est par une chaîne de montagnes qui laissent au nord la Sibérie, et au midi la grande Tartarie ; que le climat de la Sibérie est si froid, qu’à la réserve de quelques endroits, elle ne peut être cultivée ; et que, quoique les Russes aient des établissements tout le long de l’Irtis, ils n’y cultivent rien ; qu’il ne vient dans ce pays que quelques petits sapins et arbrisseaux ; que les naturels du pays sont divisés en de misérables peuplades, qui sont comme celles du Canada ; que la raison de cette froidure vient, d’un côté, de la hauteur du terrain, et de l’autre, de ce qu’à mesure que l’on va du midi au nord, les montagnes s’aplanissent, de sorte que le vent du nord souffle partout sans trouver d’obstacles ; que ce vent, qui rend la Nouvelle-Zemble inhabitable, soufflant dans la Sibérie, la rend inculte ; qu’en Europe, au contraire, les montagnes de Norvège et de Laponie sont des boulevards admirables qui couvrent de ce vent les pays du Nord ; que cela fait qu’à Stockholm, qui est à cinquante-neuf degrés de latitude ou environ, le terrain produit des fruits, des grains, des plantes ; et qu’autour d’Abo, qui est au soixante-unième degré, de même que vers les soixante-trois et soixante-quatre, il y a des mines d’argent, et que le terrain est assez fertile ».\par
Nous voyons encore dans les relations « que la grande Tartarie, qui est au midi de la Sibérie, est aussi très froide ; que le pays ne se cultive point ; qu’on n’y trouve que des pâturages pour les troupeaux ; qu’il n’y croît point d’arbres, mais quelques broussailles, comme en Islande ; qu’il y a, auprès de la Chine et du Mogol, quelques pays où il croît une espèce de millet, mais que le blé ni le riz n’y peuvent mûrir ; qu’il n’y a guère d’endroits dans la Tartarie chinoise, aux 43\textsuperscript{e}, 44\textsuperscript{e} et 45\textsuperscript{e} degrés, où il ne gèle sept ou huit mois de l’année ; de sorte qu’elle est aussi froide que l’Islande, quoiqu’elle dût être plus chaude que le midi de la France ; qu’il n’y a point de villes, excepté quatre ou cinq vers la mer Orientale, et quelques-unes que les Chinois, par des raisons de politique, ont bâties près de la Chine ; que dans le reste de la grande Tartarie, il n’y en a que quelques-unes placées dans les Boucharies, Turkestan et Charisme ; que la raison de cette extrême froidure vient de la nature du terrain nitreux, plein de salpêtre, et sablonneux, et de plus, de la hauteur du terrain. Le P. Verbiest avait trouvé qu’un certain endroit à quatre-vingts lieues au nord de la grande muraille, vers la source de Kavamhuram, excédait la hauteur du rivage de la mer près de Pékin de trois mille pas géométriques ; que cette hauteur\footnote{La Tartarie est donc comme une espèce de montagne plate.} est cause que, quoique quasi toutes les grandes rivières de l’Asie aient leur source dans le pays, il manque cependant d’eau, de façon qu’il ne peut être habité qu’auprès des rivières et des lacs ».\par
Ces faits posés, je raisonne ainsi : l’Asie n’a point proprement de zone tempérée ; et les lieux situés dans un climat très froid y touchent immédiatement ceux qui sont dans un climat très chaud, c’est-à-dire la Turquie, la Perse, le Mogol, la Chine, la Corée et le Japon.\par
En Europe, au contraire, la zone tempérée est très étendue, quoiqu’elle soit située dans des climats très différents entre eux, n’y ayant point de rapport entre les climats d’Espagne et d’Italie, et ceux de Norvège et de Suède. Mais, comme le climat y devient insensiblement froid en allant du midi au nord, à peu près à proportion de la latitude de chaque pays, il y arrive que chaque pays est à peu près semblable à celui qui en est voisin ; qu’il n’y a pas une notable différence ; et que, comme je viens de le dire, la zone tempérée y est très étendue.\par
De là il suit qu’en Asie, les nations sont opposées aux nations du fort au faible ; les peuples guerriers, braves et actifs touchent immédiatement des peuples efféminés, paresseux, timides : il faut donc que l’un soit conquis, et l’autre conquérant. En Europe, au contraire, les nations sont opposées du fort au fort ; celles qui se touchent ont à peu près le même courage. C’est la grande raison de la faiblesse de l’Asie et de la force de l’Europe, de la liberté de l’Europe et de la servitude de l’Asie : cause que je ne sache pas que l’on ait encore remarquée. C’est ce qui fait qu’en Asie il n’arrive jamais que la liberté augmente ; au lieu qu’en Europe elle augmente ou diminue selon les circonstances.\par
Que la noblesse moscovite ait été réduite en servitude par un de ses princes, on y verra toujours des traits d’impatience que les climats du Midi ne donnent point. N’y avons-nous pas vu le gouvernement aristocratique établi pendant quelques jours ? Qu’un autre royaume du Nord ait perdu ses lois, on peut s’en fier au climat, il ne les a pas perdues d’une manière irrévocable.
\subsubsection[{Chapitre IV. Conséquence de ceci}]{Chapitre IV. Conséquence de ceci}
\noindent Ce que nous venons de dire s’accorde avec les événements de l’histoire. L’Asie a été subjuguée treize fois ; onze fois par les peuples du Nord, deux fois par ceux du Midi. Dans les temps reculés, les Scythes la conquirent trois fois ; ensuite les Mèdes et les Perses chacun une ; les Grecs, les Arabes, les Mogols, les Turcs, les Tartares, les Persans et les Aguans. Je ne parle que de la haute Asie, et je ne dis rien des invasions faites dans le reste du midi de cette partie du monde, qui a continuellement souffert de très grandes révolutions.\par
En Europe, au contraire, nous ne connaissons, depuis l’établissement des colonies grecques et phéniciennes, que quatre grands changements : le premier causé par les conquêtes des Romains ; le second, par les inondations des Barbares qui détruisirent ces mêmes Romains ; le troisième, par les victoires de Charlemagne ; et le dernier, par les invasions des Normands. Et si l’on examine bien ceci, on trouvera, dans ces changements mêmes, une force générale répandue dans toutes les parties de l’Europe. On sait la difficulté que les Romains trouvèrent à conquérir en Europe, et la facilité qu’ils eurent à envahir l’Asie. On connaît les peines que les peuples du Nord eurent à renverser l’empire romain, les guerres et les travaux de Charlemagne, les diverses entreprises des Normands. Les destructeurs étaient sans cesse détruits.
\subsubsection[{Chapitre V. Que, quand les peuples du nord de l’Asie et ceux du nord de l’Europe ont conquis, les effets de la conquête n’étaient pas les mêmes}]{Chapitre V. Que, quand les peuples du nord de l’Asie et ceux du nord de l’Europe ont conquis, les effets de la conquête n’étaient pas les mêmes}
\noindent Les peuples du nord de l’Europe l’ont conquise en hommes libres ; les peuples du nord de l’Asie l’ont conquise en esclaves, et n’ont vaincu que pour un maître.\par
La raison en est que le peuple tartare, conquérant naturel de l’Asie, est devenu esclave lui-même. Il conquiert sans cesse dans le midi de l’Asie, il forme des empires ; mais la partie de la nation qui reste dans le pays se trouve soumise à un grand maître qui, despotique dans le midi, veut encore l’être dans le nord ; et, avec un pouvoir arbitraire sur les sujets conquis, le prétend encore sur les sujets conquérants. Cela se voit bien aujourd’hui dans ce vaste pays qu’on appelle la Tartarie chinoise, que l’empereur gouverne presque aussi despotiquement que la Chine même, et qu’il étend tous les jours par ses conquêtes.\par
On peut voir encore dans {\itshape l’Histoire de la Chine} que les empereurs\footnote{Comme Venti, cinquième empereur de la cinquième dynastie.} ont envoyé des colonies chinoises dans la Tartarie. Ces Chinois sont devenus Tartares et mortels ennemis de la Chine ; mais cela n’empêche pas qu’ils n’aient porté dans la Tartarie l’esprit du gouvernement chinois.\par
Souvent une partie de la nation tartare qui a conquis, est chassée elle-même ; et elle rapporte dans ses déserts un esprit de servitude qu’elle a acquis dans le climat de l’esclavage. L’histoire de la Chine nous en fournit de grands exemples, et notre histoire ancienne aussi\footnote{Les Scythes conquirent trois fois l’Asie, et en furent trois fois chassés. Justin, liv. II.}.\par
C’est ce qui a fait que le génie de la nation tartare ou gétique a toujours été semblable à celui des empires de l’Asie. Les peuples, dans ceux-ci, sont gouvernés par le bâton ; les peuples tartares, par les longs fouets. L’esprit de l’Europe a toujours été contraire à ces mœurs : et, dans tous les temps, ce que les peuples d’Asie ont appelé punition, les peuples d’Europe l’ont appelé outrage\footnote{Ceci n’est point contraire à ce que je dirai au liv. XXIII, chap. XX, sur la manière de penser des peuples germains sur le bâton. Quelque instrument que ce fût, ils regardèrent toujours comme un affront le pouvoir ou l’action arbitraire de battre.}.\par
Les Tartares détruisant l’empire grec établirent dans les pays conquis la servitude et le despotisme ; les Goths conquérant l’empire romain fondèrent partout la monarchie et la liberté.\par
Je ne sais si le fameux Rudbeck, qui, dans son {\itshape Atlantique}, a tant loué la Scandinavie, a parlé de cette grande prérogative qui doit mettre les nations qui l’habitent au-dessus de tous les peuples du monde ; c’est qu’elles ont été la source de la liberté de l’Europe, c’est-à-dire de presque toute celle qui est aujourd’hui parmi les hommes.\par
Le Goth Jornandès a appelé le nord de l’Europe la fabrique du genre humain\footnote{{\itshape Humani generis officinam}.}. Je l’appellerai plutôt la fabrique des instruments qui brisent les fers forgés au midi. C’est là que se forment ces nations vaillantes, qui sortent de leur pays pour détruire les tyrans et les esclaves, et apprendre aux hommes que, la nature les ayant faits égaux, la raison n’a pu les rendre dépendants que pour leur bonheur.
\subsubsection[{Chapitre VI. Nouvelle cause physique de la servitude de l’Asie et de la liberté de l’Europe}]{Chapitre VI. Nouvelle cause physique de la servitude de l’Asie et de la liberté de l’Europe}
\noindent En Asie, on a toujours vu de grands empires ; en Europe, ils n’ont jamais pu subsister. C’est que l’Asie que nous connaissons a de plus grandes plaines ; elle est coupée en plus grands morceaux par les mers ; et, comme elle est plus au midi, les sources y sont plus aisément taries, les montagnes y sont moins couvertes de neiges, et les fleuves\footnote{Les eaux se perdent ou s’évaporent avant de se ramasser, ou après s’être ramassées.} moins grossis y forment de moindres barrières.\par
La puissance doit donc être toujours despotique en Asie. Car, si la servitude n’y était pas extrême, il se ferait d’abord un partage que la nature du pays ne peut pas souffrir.\par
En Europe, le partage naturel forme plusieurs États d’une étendue médiocre, dans lesquels le gouvernement des lois n’est pas incompatible avec le maintien de l’État : au contraire, il y est si favorable que, sans elles, cet État tombe dans la décadence, et devient inférieur à tous les autres.\par
C’est ce qui a formé un génie de liberté, qui rend chaque partie très difficile à être subjuguée et soumise à une force étrangère, autrement que par les lois et l’utilité de son commerce.\par
Au contraire, il règne en Asie un esprit de servitude qui ne l’a jamais quittée ; et, dans toutes les histoires de ce pays, il n’est pas possible de trouver un seul trait qui marque une âme libre : on n’y verra jamais que l’héroïsme de la servitude.
\subsubsection[{Chapitre VII. De l’Afrique et de l’Amérique}]{Chapitre VII. De l’Afrique et de l’Amérique}
\noindent Voilà ce que je puis dire sur l’Asie et sur l’Europe. L’Afrique est dans un climat pareil à celui du midi de l’Asie, et elle est dans une même servitude. L’Amérique\footnote{Les petits peuples barbares de l’Amérique sont appelés {\itshape Indios bravos} par les Espagnols ; bien plus difficiles à soumettre que les grands empires du Mexique et du Pérou.}, détruite et nouvellement repeuplée par les nations de l’Europe et de l’Afrique, ne peut guère aujourd’hui montrer son propre génie : mais ce que nous savons de son ancienne histoire est très conforme à nos principes.
\subsubsection[{Chapitre VIII. De la capitale de l’empire}]{Chapitre VIII. De la capitale de l’empire}
\noindent Une des conséquences de ce que nous venons de dire, c’est qu’il est important à un très grand prince de bien choisir le siège de son empire. Celui qui le placera au midi courra risque de perdre le nord ; et celui qui le placera au nord conservera aisément le midi. Je ne parle pas des cas particuliers : la mécanique a bien ses frottements qui souvent changent ou arrêtent les effets de la théorie : la politique a aussi les siens.
\subsection[{Livre dix-huitième. Des lois dans le rapport qu’elles ont avec la nature du terrain}]{Livre dix-huitième. Des lois dans le rapport qu’elles ont avec la nature du terrain}
\subsubsection[{Chapitre I. Comment la nature du terrain influe sur les lois}]{Chapitre I. Comment la nature du terrain influe sur les lois}
\noindent La bonté des terres d’un pays y établit naturellement la dépendance. Les gens de la campagne, qui y font la principale partie du peuple, ne sont pas si jaloux de leur liberté ; ils sont trop occupés et trop pleins de leurs affaires particulières. Une campagne qui regorge de biens craint le pillage, elle craint une armée. « Qui est-ce qui forme le bon parti, disait Cicéron à Atticus\footnote{Liv. VII.} ? Seront-ce les gens de commerce et de la campagne ? À moins que nous n’imaginions qu’ils sont opposés à la monarchie, eux à qui tous les gouvernements sont égaux, dès lors qu’ils sont tranquilles. »\par
Ainsi le gouvernement d’un seul se trouve plus souvent dans les pays fertiles, et le gouvernement de plusieurs dans les pays qui ne le sont pas ; ce qui est quelquefois un dédommagement.\par
La stérilité du terrain de l’Attique y établit le gouvernement populaire ; et la fertilité de celui de Lacédémone, le gouvernement aristocratique. Car, dans ces temps-là, on ne voulait point dans la Grèce du gouvernement d’un seul : or le gouvernement aristocratique a plus de rapport avec le gouvernement d’un seul.\par
Plutarque\footnote{{\itshape Vie de Solon}.} nous dit « que la sédition Cilonienne ayant été apaisée à Athènes, la ville retomba dans ses anciennes dissensions, et se divisa en autant de partis qu’il y avait de sortes de territoires dans le pays de l’Attique. Les gens de la montagne voulaient à toute force le gouvernement populaire ; ceux de la plaine demandaient le gouvernement des principaux ; ceux qui étaient près de la mer étaient pour un gouvernement mêlé des deux ».
\subsubsection[{Chapitre II. Continuation du même sujet}]{Chapitre II. Continuation du même sujet}
\noindent Ces pays fertiles sont des plaines où l’on ne peut rien disputer au plus fort : on se soumet donc à lui ; et, quand on lui est soumis, l’esprit de liberté n’y saurait revenir ; les biens de la campagne sont un gage de la fidélité. Mais, dans les pays de montagnes, on peut conserver ce que l’on a, et l’on a peu à conserver. La liberté, c’est-à-dire le gouvernement dont on jouit, est le seul bien qui mérite qu’on le défende. Elle règne donc plus dans les pays montagneux et difficiles que dans ceux que la nature semblait avoir plus favorisés.\par
Les montagnards conservent un gouvernement plus modéré, parce qu’ils ne sont pas si fort exposés à la conquête. Ils se défendent aisément, ils sont attaqués difficilement ; les munitions de guerre et de bouche sont assemblées et portées contre eux avec beaucoup de dépense ; le pays n’en fournit point. Il est donc plus difficile de leur faire la guerre, plus dangereux de l’entreprendre ; et toutes les lois que l’on fait pour la sûreté du peuple y ont moins de lieu.
\subsubsection[{Chapitre III. Quels sont les pays les plus cultivés}]{Chapitre III. Quels sont les pays les plus cultivés}
\noindent Les pays ne sont pas cultivés en raison de leur fertilité, mais en raison de leur liberté ; et si l’on divise la terre par la pensée, on sera étonné de voir la plupart du temps des déserts dans ses parties les plus fertiles, et de grands peuples dans celles où le terrain semble refuser tout.\par
Il est naturel qu’un peuple quitte un mauvais pays pour en chercher un meilleur, et non pas qu’il quitte un bon pays pour en chercher un pire. La plupart des invasions se font donc dans les pays que la nature avait faits pour être heureux ; et, comme rien n’est plus près de la dévastation que l’invasion, les meilleurs pays sont le plus souvent dépeuplés, tandis que l’affreux pays du Nord reste toujours habité, par la raison qu’il est presque inhabitable.\par
On voit, par ce que les historiens nous disent du passage des peuples de la Scandinavie sur les bords du Danube, que ce n’était point une conquête, mais seulement une transmigration dans des terres désertes.\par
Ces climats heureux avaient donc été dépeuplés par d’autres transmigrations, et nous ne savons pas les choses tragiques qui s’y sont passées.\par
« Il paraît par plusieurs monuments, dit Aristote\footnote{Ou celui qui a écrit le livre {\itshape De Mirabilibus}.}, que la Sardaigne est une colonie grecque. Elle était autrefois très riche ; et Aristée, dont on a tant vanté l’amour pour l’agriculture, lui donna des lois. Mais elle a bien déchu depuis ; car les Carthaginois s’en étant rendus les maîtres, ils y détruisirent tout ce qui pouvait la rendre propre à la nourriture des hommes et défendirent, sous peine de la vie, d’y cultiver la terre. » La Sardaigne n’était point rétablie du temps d’Aristote ; elle ne l’est point encore aujourd’hui.\par
Les parties les plus tempérées de la Perse, de la Turquie, de la Moscovie et de la Pologne, n’ont pu se rétablir des dévastations des grands et des petits Tartares.
\subsubsection[{Chapitre IV. Nouveaux effets de la fertilité et de la stérilité du pays}]{Chapitre IV. Nouveaux effets de la fertilité et de la stérilité du pays}
\noindent La stérilité des terres rend les hommes industrieux, sobres, endurcis au travail, courageux, propres à la guerre ; il faut bien qu’ils se procurent ce que le terrain leur refuse. La fertilité d’un pays donne, avec l’aisance, la mollesse et un certain amour pour la conservation de la vie.\par
On a remarqué que les troupes d’Allemagne levées dans des lieux où les paysans sont riches, comme en Saxe, ne sont pas si bonnes que les autres. Les lois militaires pourront pourvoir à cet inconvénient par une plus sévère discipline.
\subsubsection[{Chapitre V. Des peuples des îles}]{Chapitre V. Des peuples des îles}
\noindent Les peuples des îles sont plus portés à la liberté que les peuples du continent. Les îles sont ordinairement d’une petite étendue\footnote{Le Japon déroge à ceci par sa grandeur et par sa servitude.} ; une partie du peuple ne peut pas être si bien employée à opprimer l’autre ; la mer les sépare des grands empires, et la tyrannie ne peut pas s’y prêter la main ; les conquérants sont arrêtés par la mer ; les insulaires ne sont pas enveloppés dans la conquête, et ils conservent plus aisément leurs lois.
\subsubsection[{Chapitre VI. Des pays formés par l’industrie des hommes}]{Chapitre VI. Des pays formés par l’industrie des hommes}
\noindent Les pays que l’industrie des hommes a rendus habitables, et qui ont besoin, pour exister, de la même industrie, appellent à eux le gouvernement modéré. Il y en a principalement trois de cette espèce : les deux belles provinces de Kiang-nan et Tche-kiang à la Chine, l’Égypte et la Hollande.\par
Les anciens empereurs de la Chine n’étaient point conquérants. La première chose qu’ils firent pour s’agrandir fut celle qui prouva le plus leur sagesse. On vit sortir de dessous les eaux les deux plus belles provinces de l’empire ; elles furent faites par les hommes. C’est la fertilité inexprimable de ces deux provinces qui a donné à l’Europe les idées de la félicité de cette vaste contrée. Mais un soin continuel et nécessaire pour garantir de la destruction une partie si considérable de l’empire demandait plutôt les mœurs d’un peuple sage que celles d’un peuple voluptueux, plutôt le pouvoir légitime d’un monarque que la puissance tyrannique d’un des pote. Il fallait que le pouvoir y fût modéré, comme il l’était autrefois en Égypte. Il fallait que le pouvoir y fût modéré, comme il l’est en Hollande, que la nature a faite pour avoir attention sur elle-même, et non pas pour être abandonnée à la nonchalance ou au caprice.\par
Ainsi, malgré le climat de la Chine, où l’on est naturellement porté à l’obéissance servile, malgré les horreurs qui suivent la trop grande étendue d’un empire, les premiers législateurs de la Chine furent obligés de faire de très bonnes lois, et le gouvernement fut souvent obligé de les suivre.
\subsubsection[{Chapitre VII. Des ouvrages des hommes}]{Chapitre VII. Des ouvrages des hommes}
\noindent Les hommes, par leurs soins et par de bonnes lois, ont rendu la terre plus propre à être leur demeure. Nous voyons couler les rivières là où étaient des lacs et des marais ; c’est un bien que la nature n’a point fait, mais qui est entretenu par la nature. Lorsque les Perses\footnote{Polybe, liv. X.} étaient les maîtres de l’Asie, ils permettaient à ceux qui amèneraient de l’eau de fontaine en quelque lieu qui n’aurait point été encore arrosé, d’en jouir pendant cinq générations ; et comme il sort quantité de ruisseaux du mont Taurus, ils n’épargnèrent aucune dépense pour en faire venir de l’eau. Aujourd’hui, sans savoir d’où elle peut venir, on la trouve dans ses champs et dans ses jardins.\par
Ainsi, comme les nations destructrices font des maux qui durent plus qu’elles, il y a des nations industrieuses qui font des biens qui ne finissent pas même avec elles.
\subsubsection[{Chapitre VIII. Rapport général des lois}]{Chapitre VIII. Rapport général des lois}
\noindent Les lois ont un très grand rapport avec la façon dont les divers peuples se procurent la subsistance. Il faut un code de lois plus étendu pour un peuple qui s’attache au commerce et à la mer, que pour un peuple qui se contente de cultiver ses terres. Il en faut un plus grand pour celui-ci que pour un peuple qui vit de ses troupeaux. Il en faut un plus grand pour ce dernier que pour un peuple qui vit de sa chasse.
\subsubsection[{Chapitre IX. Du terrain de l’Amérique}]{Chapitre IX. Du terrain de l’Amérique}
\noindent Ce qui fait qu’il y a tant de nations sauvages en Amérique, c’est que la terre y produit d’elle-même beaucoup de fruits dont on peut se nourrir. Si les femmes y cultivent autour de la cabane un morceau de terre, le mais y vient d’abord. La chasse et la pêche achèvent de mettre les hommes dans l’abondance. De plus, les animaux qui paissent, comme les bœufs, les buffles, etc., y réussissent mieux que les bêtes carnassières. Celles-ci ont eu de tout temps l’empire de l’Afrique.\par
Je crois qu’on n’aurait point tous ces avantages en Europe, si l’on y laissait la terre inculte ; il n’y viendrait guère que des forêts, des chênes et autres arbres stériles.
\subsubsection[{Chapitre X. Du nombre des hommes dans le rapport avec la manière dont ils se procurent la subsistance}]{Chapitre X. Du nombre des hommes dans le rapport avec la manière dont ils se procurent la subsistance}
\noindent Quand les nations ne cultivent pas les terres, voici dans quelle proportion le nombre des hommes s’y trouve. Comme le produit d’un terrain inculte est au produit d’un terrain cultivé, de même le nombre des sauvages, dans un pays, est au nombre des laboureurs dans un autre ; et quand le peuple qui cultive les terres cultive aussi les arts, cela suit des proportions qui demanderaient bien des détails.\par
Ils ne peuvent guère former une grande nation. S’ils sont pasteurs, ils ont besoin d’un grand pays pour qu’ils puissent subsister en certain nombre : s’ils sont chasseurs, ils sont encore en plus petit nombre, et forment, pour vivre, une plus petite nation.\par
Leur pays est ordinairement plein de forêts ; et comme les hommes n’y ont point donné de cours aux eaux, il est rempli de marécages, où chaque troupe se cantonne et forme une petite nation.
\subsubsection[{Chapitre XI. Des peuples sauvages et des peuples barbares}]{Chapitre XI. Des peuples sauvages et des peuples barbares}
\noindent Il y a cette différence entre les peuples sauvages et les peuples barbares, que les premiers sont de petites nations dispersées, qui, par quelques raisons particulières, ne peuvent pas se réunir ; au lieu que les barbares sont ordinairement de petites nations qui peuvent se réunir. Les premiers sont ordinairement des peuples chasseurs ; les seconds, des peuples pasteurs. Cela se voit bien dans le nord de l’Asie. Les peuples de la Sibérie ne sauraient vivre en corps, parce qu’ils ne pourraient se nourrir ; les Tartares peuvent vivre en corps pendant quelque temps, parce que leurs troupeaux peuvent être rassemblés pendant quelque temps. Toutes les hordes peuvent donc se réunir-, et cela se fait lorsqu’un chef en a soumis beaucoup d’autres ; après quoi, il faut qu’elles fassent de deux choses l’une : qu’elles se séparent, ou qu’elles aillent faire quelque grande conquête dans quelque empire du Midi.
\subsubsection[{Chapitre XII. Du droit des gens chez les peuples qui ne cultivent point les terres}]{Chapitre XII. Du droit des gens chez les peuples qui ne cultivent point les terres}
\noindent Ces peuples, ne vivant pas dans un terrain limité et circonscrit, auront entre eux bien des sujets de querelle ; ils se disputeront la terre inculte, comme parmi nous les citoyens se disputent les héritages. Ainsi ils trouveront de fréquentes occasions de guerre pour leurs chasses, pour leurs pêches, pour la nourriture de leurs bestiaux, pour l’enlèvement de leurs esclaves ; et, n’ayant point de territoire, ils auront autant de choses à régler par le droit des gens qu’ils en auront peu à décider par le droit civil.
\subsubsection[{Chapitre XIII. Des lois civiles chez les peuples qui ne cultivent point les terres}]{Chapitre XIII. Des lois civiles chez les peuples qui ne cultivent point les terres}
\noindent C’est le partage des terres qui grossit principalement le code civil. Chez les nations où l’on n’aura pas fait ce partage, il y aura très peu de lois civiles.\par
On peut appeler les institutions de ces peuples des mœurs plutôt que des lois.\par
Chez de pareilles nations, les vieillards, qui se souviennent des choses passées, ont une grande autorité ; on n’y peut être distingué par les biens, mais par la main et par les conseils.\par
Ces peuples errent et se dispersent dans les pâturages ou dans les forêts. Le mariage n’y sera pas aussi assuré que parmi nous, où il est fixé par la demeure, et où la femme tient à une maison ; ils peuvent donc plus aisément changer de femmes, en avoir plusieurs, et quelquefois se mêler indifféremment comme les bêtes.\par
Les peuples pasteurs ne peuvent se séparer de leurs troupeaux, qui font leur subsistance ; ils ne sauraient non plus se séparer de leurs femmes, qui en ont soin. Tout cela doit donc marcher ensemble ; d’autant plus que, vivant ordinairement dans de grandes plaines, où il y a peu de lieux forts d’assiette, leurs femmes, leurs enfants, leurs troupeaux deviendraient la proie de leurs ennemis.\par
Leurs lois régleront le partage du butin, et auront, comme nos lois saliques, une attention particulière sur les vols.
\subsubsection[{Chapitre XIV. De l’état politique des peuples qui ne cultivent point les terres}]{Chapitre XIV. De l’état politique des peuples qui ne cultivent point les terres}
\noindent Ces peuples jouissent d’une grande liberté : car, comme ils ne cultivent point les terres, ils n’y sont point attachés ; ils sont errants, vagabonds ; et si un chef voulait leur ôter leur liberté, ils l’iraient d’abord chercher chez un autre, ou se retireraient dans les bois pour y vivre avec leur famille. Chez ces peuples, la liberté de l’homme est si grande, qu’elle entraîne nécessairement la liberté du citoyen.
\subsubsection[{Chapitre XV. Des peuples qui connaissent l’usage de la monnaie}]{Chapitre XV. Des peuples qui connaissent l’usage de la monnaie}
\noindent Aristippe, ayant fait naufrage, nagea et aborda au rivage prochain ; il vit qu’on avait tracé sur le sable des figures de géométrie : il se sentit ému de joie, jugeant qu’il était arrivé chez un peuple grec, et non pas chez un peuple barbare.\par
Soyez seul, et arrivez par quelque accident chez un peuple inconnu ; si vous voyez une pièce de monnaie, comptez que vous êtes arrivé chez une nation policée.\par
La culture des terres demande l’usage de la monnaie. Cette culture suppose beaucoup d’arts et de connaissances ; et l’on voit toujours marcher d’un pas égal les arts, les connaissances et les besoins. Tout cela conduit à l’établissement d’un signe de valeurs.\par
Les torrents et les incendies nous ont fait découvrir que les terres contenaient des métaux\footnote{C’est ainsi que Diodore (liv. V, chap. XXV) nous dit que les bergers trouvèrent l’or des Pyrénées.}. Quand ils en ont été une fois séparés, il a été aisé de les employer.
\subsubsection[{Chapitre XVI. Des lois civiles chez les peuples qui ne connaissent point l’usage de la monnaie}]{Chapitre XVI. Des lois civiles chez les peuples qui ne connaissent point l’usage de la monnaie}
\noindent Quand un peuple n’a pas l’usage de la monnaie, on ne connaît guère chez lui que les injustices qui viennent de la violence ; et les gens faibles, en s’unissant, se défendent contre la violence. Il n’y a guère là que des arrangements politiques. Mais chez un peuple où la monnaie est établie, on est sujet aux injustices qui viennent de la ruse ; et ces injustices peuvent être exercées de mille façons. On y est donc forcé d’avoir de bonnes lois civiles ; elles naissent avec les nouveaux moyens et les diverses manières d’être méchant.\par
Dans les pays où il n’y a point de monnaie, le ravisseur n’enlève que des choses, et les choses ne se ressemblent jamais. Dans les pays où il y a de la monnaie, le ravisseur enlève des signes, et les signes se ressemblent toujours. Dans les premiers pays rien ne peut être caché, parce que le ravisseur porte toujours avec lui des preuves de sa conviction : cela n’est pas de même dans les autres.
\subsubsection[{Chapitre XVII. Dès lors politiques chez les peuples qui n’ont point l’usage de la monnaie}]{Chapitre XVII. Dès lors politiques chez les peuples qui n’ont point l’usage de la monnaie}
\noindent Ce qui assure le plus la liberté des peuples qui ne cultivent point les terres, c’est que la monnaie leur est inconnue. Les fruits de la chasse, de la pêche, ou des troupeaux ne peuvent s’assembler en assez grande quantité, ni se garder assez, pour qu’un homme se trouve en état de corrompre tous les autres : au lieu que, lorsque l’on a des signes de richesses, on peut faire un amas de ces signes, et les distribuer à qui l’on veut.\par
Chez les peuples qui n’ont point de monnaie, chacun a peu de besoins, et les satisfait aisément et également. L’égalité est donc forcée : aussi leurs chefs ne sont-ils point despotiques.
\subsubsection[{Chapitre XVIII. Force de la superstition}]{Chapitre XVIII. Force de la superstition}
\noindent Si ce que les relations nous disent est vrai, la constitution d’un peuple de la Louisiane nommé les {\itshape Natchés}, déroge à ceci. Leur chef\footnote{{\itshape Lettres édifiantes}, vingtième recueil.} dispose des biens de tous ses sujets, et les fait travailler à sa fantaisie : ils ne peuvent lui refuser leur tête ; il est comme le Grand Seigneur. Lorsque l’héritier présomptif vient à naître, on lui donne tous les enfants à la mamelle, pour le servir pendant sa vie. Vous diriez que c’est le grand Sésostris. Ce chef est traité dans sa cabane avec les cérémonies qu’on ferait à un empereur du Japon ou de la Chine.\par
Les préjugés de la superstition sont supérieurs à tous les autres préjugés, et ses raisons à toutes les autres raisons. Ainsi, quoique les peuples sauvages ne connaissent point naturellement le despotisme, ce peuple-ci le connaît. Ils adorent le soleil, et si leur chef n’avait pas imaginé qu’il était le \&ère du soleil, ils n’auraient trouvé en lui qu’un misérable comme eux.
\subsubsection[{Chapitre XIX. De la liberté des Arabes et de la servitude des Tartares}]{Chapitre XIX. De la liberté des Arabes et de la servitude des Tartares}
\noindent Les Arabes et les Tartares sont des peuples pasteurs. Les Arabes se trouvent dans les cas généraux dont nous avons parlé, et sont libres ; au lieu que les Tartares (peuple le plus singulier de la terre) se trouvent dans l’esclavage politique\footnote{Lorsqu’on proclame un kan, tout le peuple s’écrie : {\itshape Que sa parole lui serve de glaive}.}. J’ai déjà\footnote{Liv. XVII, chap. V.} donné quelques raisons de ce dernier fait : en voici de nouvelles.\par
Ils n’ont point de villes, ils n’ont point de forêts, ils ont peu de marais, leurs rivières sont presque toujours glacées, ils habitent une immense plaine, ils ont des pâturages et des troupeaux, et par conséquent des biens : mais ils n’ont aucune espèce de retraite ni de défense. Sitôt qu’un kan est vaincu, on lui coupe la tête\footnote{Ainsi, il ne faut pas être étonné si Mirivéis, s’étant rendu maître d’Ispahan, fit tuer tous les princes du sang.} {\itshape ;} on traite de la même manière ses enfants ; et tous ses sujets appartiennent au vainqueur. On ne les condamne pas à un esclavage civil ; ils seraient à charge à une nation simple, qui n’a point de terres à cultiver, et n’a besoin d’aucun service domestique. Ils augmentent donc la nation. Mais, au lieu de l’esclavage civil, on conçoit que l’esclavage politique a dû s’introduire.\par
En effet, dans un pays où les diverses hordes se font continuellement la guerre et se conquièrent sans cesse les unes les autres ; dans un pays où, par la mort du chef, le corps politique de chaque horde vaincue est toujours détruit, la nation en général ne peut guère être libre : car il n’y en a pas une seule partie qui ne doive avoir été un très grand nombre de fois subjuguée.\par
Les peuples vaincus peuvent conserver quelque liberté, lorsque, par la force de leur situation, ils sont en état de faire des traités après leur défaite. Mais les Tartares, toujours sans défense, vaincus une fois, n’ont jamais pu faire des conditions.\par
J’ai dit, au chapitre II, que les habitants des plaines cultivées n’étaient guère libres : des circonstances font que les Tartares, habitant une terre inculte, sont dans le même cas.
\subsubsection[{Chapitre XX. Du droit des gens des Tartares}]{Chapitre XX. Du droit des gens des Tartares}
\noindent Les Tartares paraissent entre eux doux et humains, et ils font des conquérants très cruels : ils passent au fil de l’épée les habitants des villes qu’ils prennent ; ils croient leur faire grâce lorsqu’ils les vendent ou les distribuent à leurs soldats. Ils ont détruit l’Asie depuis les Indes jusqu’à la Méditerranée ; tout le pays qui forme l’orient de la Perse en est resté désert.\par
Voici ce qui me paraît avoir produit un pareil droit des gens. Ces peuples n’avaient point de villes ; toutes leurs guerres se faisaient avec promptitude et avec impétuosité. Quand ils espéraient de vaincre, ils combattaient ; ils augmentaient l’armée des plus forts quand ils ne l’espéraient pas. Avec de pareilles coutumes, ils trouvaient qu’il était contre leur droit des gens qu’une ville, qui ne pouvait leur résister, les arrêtât. Ils ne regardaient pas les villes comme une assemblée d’habitants, mais comme des lieux propres à se soustraire à leur puissance. Ils n’avaient aucun art pour les assiéger, et ils s’exposaient beaucoup en les assiégeant ; ils vengeaient par le sang tout celui qu’ils venaient de répandre.
\subsubsection[{Chapitre XXI. Loi civile des Tartares}]{Chapitre XXI. Loi civile des Tartares}
\noindent Le P. Du Halde dit que, chez les Tartares, c’est toujours le dernier des mâles qui est l’héritier, par la raison qu’à mesure que les aînés sont en état de mener la vie pastorale, ils sortent de la maison avec une certaine quantité de bétail que le père leur donne, et vont former une nouvelle habitation. Le dernier des mâles, qui reste dans la maison avec son père, est donc son héritier naturel.\par
J’ai ouï dire qu’une pareille coutume était observée dans quelques petits districts d’Angleterre, et on la trouve encore en Bretagne, dans le duché de Rohan, où elle a lieu pour les rotures.\par
C’est sans doute une loi pastorale venue de quelque petit peuple breton, ou portée par quelque peuple germain. On sait, par César et Tacite, que ces derniers cultivaient peu les terres.
\subsubsection[{Chapitre XXII. D’une loi civile des peuples germains}]{Chapitre XXII. D’une loi civile des peuples germains}
\noindent J’expliquerai ici comment ce texte particulier de la loi salique, que l’on appelle ordinairement la loi salique, tient aux institutions d’un peuple qui ne cultivait point les terres, ou du moins qui les cultivait peu.\par
La loi salique\footnote{Tit. LXII.} veut que, lorsqu’un homme laisse des enfants, les mâles succèdent à la terre salique au préjudice des filles.\par
Pour savoir ce que c’était que les terres saliques, il faut chercher ce que c’était que les propriétés ou l’usage des terres chez les Francs, avant qu’ils fussent sortis de la Germanie.\par
M. Échard a très bien prouvé que le mot salique vient du mot sala, qui signifie maison ; et qu’ainsi la terre salique était la terre de la maison. J’irai plus loin, et j’examinerai ce que c’était que la maison, et la terre de la maison, chez les Germains.\par
« Ils n’habitent point de villes, dit Tacite\footnote{{\itshape Nullas Germanorum populis urbes habitari satis notum est, ne pati quidem inter se junctas sedes ; colunt discreti, ut nemus placuit. Vicos locant, non in nostrum morem connexis et cohaerentibus aedificiis : suam quisque domum spatio circumdat.De moribus Germ.}}, et ils ne peuvent souffrir que leurs maisons se touchent les unes les autres ; chacun laisse autour de sa maison un petit terrain ou espace, qui est clos et fermé. » Tacite parlait exactement. Car plusieurs lois des codes\footnote{La loi des Allemands, chap. X, et la loi des Bavarois, tit. X, § 1 et 2.} barbares ont des dispositions différentes contre ceux qui renversaient cette {\itshape enceinte, et ceux} qui pénétraient dans la maison même.\par
Nous savons, par Tacite et César, que les terres que les Germains cultivaient ne leur étaient données que pour un an ; après quoi elles redevenaient publiques. Ils n’avaient de patrimoine que la maison, et un morceau de terre dans l’enceinte autour de la maison\footnote{Cette enceinte s’appelle {\itshape curtis} dans les chartes.}. C’est ce patrimoine particulier qui appartenait aux mâles. En effet, pourquoi aurait-il appartenu aux filles ? Elles passaient dans une autre maison,\par
La terre salique était donc cette enceinte qui dépendait de la maison du Germain ; c’était la seule propriété qu’il eût. Les Francs, après la conquête, acquirent de nouvelles propriétés, et on continua à les appeler des terres saliques.\par
Lorsque les Francs vivaient dans la Germanie, leurs biens étaient des esclaves, des troupeaux, des chevaux, des armes, etc. La maison et la petite portion de terre qui y était jointe étaient naturellement données aux enfants mâles qui devaient y habiter. Mais, lorsque après la conquête, les Francs eurent acquis de grandes terres, on trouva dur que les filles et leurs enfants ne pussent y avoir de part. Il s’introduisit un usage, qui permettait au père de rappeler sa fille et les enfants de sa fille. On fit taire la loi ; et il fallait bien que ces sortes de rappels fussent communs, puisqu’on en fit des formules\footnote{Voyez Marculfe, liv. II, {\itshape formules} 10 et 12 ; {\itshape l’Appendice} de Marculfe, {\itshape formule} 49, et les {\itshape Formules anciennes}, appelées de Sirmond, {\itshape formule} 22.}.\par
Parmi toutes ces formules, j’en trouve une singulière\footnote{{\itshape Formule} 55, dans le recueil de Lindernbroch.}. Un aïeul rappelle ses petits-enfants pour succéder avec ses fils et avec ses filles. Que devenait donc la loi salique ? il fallait que, dans ces temps-là même, elle ne fût plus observée ; ou que l’usage continuel de rappeler les filles eût fait regarder leur capacité de succéder comme le cas le plus ordinaire.\par
La loi salique n’ayant point pour objet une certaine préférence d’un sexe sur un autre, elle avait encore moins celui d’une perpétuité de famille, de nom, ou de transmission de terre : tout cela n’entrait point dans la tête des Germains. C’était une loi purement économique, qui donnait la maison, et la terre dépendante de la maison, aux mâles qui devaient l’habiter, et à qui, par conséquent, elle convenait le mieux.\par
Il n’y a qu’à transcrire ici le titre {\itshape Des Alleux} de la loi salique, ce texte si fameux, dont tant de gens ont parlé, et que si peu de gens ont lu.\par
1° « Si un homme meurt sans enfants, son père ou sa mère lui succéderont. 2° S’il n’a ni père ni mère, son frère ou sa sœur lui succéderont. 3° S’il n’a ni frère ni sœur, la sœur de sa mère lui succédera. 4° Si sa mère n’a point de sœur, la sœur de son père lui succédera. 5° Si son père n’a point de sœur, le plus proche parent par mâle lui succédera. 6° Aucune portion\footnote{{\itshape De terra vero salica in mulierem nulla portio hereditatis transit, sed hoc virilis sexus acquirit, hoc est filii in ipsa hereditate succedunt.} Tit. LXII, § 6.} de la terre salique ne passera aux femelles ; mais elle appartiendra aux mâles, c’est-à-dire que les enfants mâles succéderont à leur père. »\par
Il est clair que les cinq premiers articles concernent la succession de celui qui meurt sans enfants ; et le sixième, la succession de celui qui a des enfants.\par
Lorsqu’un homme mourait sans enfants, la loi voulait qu’un des deux sexes n’eût de préférence sur l’autre que dans de certains cas. Dans les deux premiers degrés de succession, les avantages des mâles et des femelles étaient les mêmes ; dans le troisième et le quatrième, les femmes avaient la préférence ; et les mâles l’avaient dans le cinquième.\par
Je trouve les semences de ces bizarreries dans Tacite. « Les enfants\footnote{{\itshape Sororum filiis idem apud avunculum quam apud patrem honor. Quidam sanctiorem arctioremque hunc nexum sanguinis arbitrantur, et in accipiendis obsidibus magis exigunt, tanquam ii et animum firmius et domum latius teneant.De moribus Germ.}} des sœurs, dit-il, sont chéris de leur oncle comme de leur propre père. Il y a des gens qui regardent ce lien comme plus étroit, et même plus saint ; ils le préfèrent, quand ils reçoivent des otages. » C’est pour cela que nos premiers historiens\footnote{Voyez dans Grégoire de Tours, liv. VIII, chap. XVIII et XX ; liv. IX, chap. XVI et XX, les fureurs de Gontran sur les mauvais traitements faits à Ingunde, sa nièce, par Leuvigilde ; et comme Childebert, son frère, fit la guerre pour la venger.} nous parlent tant de l’amour des rois francs pour leur sœur et pour les enfants de leur sœur. Que si les enfants des sœurs étaient regardés dans la maison comme les enfants mêmes, il était naturel que les enfants regardassent leur tante comme leur propre mère.\par
La sœur de la mère était préférée à la sœur du père ; cela s’explique par d’autres textes de la loi salique : lorsqu’une femme était veuve\footnote{Loi salique, tit. XLVII.}, elle tombait sous la tutelle des parents de son mari ; la loi préférait pour cette tutelle les parents par femmes aux parents par mâles. En effet, une femme qui entrait dans une famille s’unissant avec les personnes de son sexe, elle était plus liée avec les parents par femmes qu’avec les parents par mâles. De plus, quand un homme\footnote{{\itshape Ibid.}, tit. LXI, § 1.} en avait tué un autre, et qu’il n’avait pas de quoi satisfaire à la peine pécuniaire qu’il avait encourue, la loi lui permettait de céder ses biens, et les parents devaient suppléer à ce qui manquait. Après le père, la mère et le frère, c’était la sœur de la mère qui payait, comme si ce lien avait quelque chose de plus tendre ; or, la parenté qui donne les charges devait de même donner les avantages.\par
La loi salique voulait qu’après la sœur du père, le plus proche parent par mâle eût la succession ; mais s’il était parent au-delà du cinquième degré, il ne succédait pas. Ainsi une femme au cinquième degré aurait succédé au préjudice d’un mâle du sixième : et cela se voit dans la loi\footnote{{\itshape Et deinceps usque ad quintum genuculum qui proximus fuerit in hereditatem succedat}, tit. LVI, §6.} des Francs ripuaires, fidèle interprète de la loi salique dans le titre des alleux, où elle suit pas à pas le même titre de la loi salique.\par
Si le père laissait des enfants, la loi salique voulait que les filles fussent exclues de la succession à la terre salique, et qu’elle appartînt aux enfants mâles.\par
il me sera aisé de prouver que la loi salique n’exclut pas indistinctement les filles de la terre salique, mais dans le cas seulement où des frères les excluraient. P Cela se voit dans la loi salique même, qui, après avoir dit que les femmes ne posséderaient rien de la terre salique, mais seulement les mâles, s’interprète et se restreint elle-même ; « c’est-à-dire, dit-elle, que le fils succédera à l’hérédité du père ».\par
2° Le texte de la loi salique est éclairci par la loi des Francs ripuaires, qui a aussi un titre\footnote{Tit. LVI.} des alleux très conforme à celui de la loi salique.\par
3{\itshape °} Les lois de ces peuples barbares, tous originaires de la Germanie, s’interprètent les unes les autres, d’autant plus qu’elles ont toutes à peu près le même esprit. La loi des Saxons\footnote{Tit. VII, § 1. {\itshape Pater aut mater defuncti, filio non filiae hereditatem relinquant}. § 4 : {\itshape Qui defunctus, non filios sed filias reliquerit, ad eas omnis hereditas pertineat.}} veut que le père et la mère laissent leur hérédité à leur fils, et non pas à leur fille ; mais que s’il n’y a que des filles, elles aient toute l’hérédité.\par
4° Nous avons deux anciennes formules\footnote{Dans Marculfe, liv. II, {\itshape formule} 12, et dans {\itshape l’Appendice} de Marculfe, {\itshape formule} 49.} qui posent le cas où, suivant la loi salique, les filles sont exclues par les mâles ; c’est lorsqu’elles concourent avec leur frère.\par
5{\itshape °} Une autre formule\footnote{Dans le recueil de Lindembroch, {\itshape formule} 55.} prouve que la fille succédait au préjudice du petit-fils ; elle n’était donc exclue que par le fils.\par
6° Si les filles, par la loi salique, avaient été généralement exclues de la succession des terres, il serait impossible d’expliquer les histoires, les formules et les chartes, qui parlent continuellement des terres et des biens des femmes dans la première race.\par
On a eu tort de dire\footnote{Du Cange, Pithou, etc.} que les terres saliques étaient des fiefs. 1° Ce titre est intitulé {\itshape Des Alleux}. 2° Dans les commencements, les fiefs n’étaient point héréditaires. 3° Si les terres saliques avaient été des fiefs, comment Marculfe aurait-il traité d’impie la coutume qui excluait les femmes d’y succéder, puisque les mâles mêmes ne succédaient pas aux fiefs ? 4° Les chartes que l’on cite pour prouver que les terres saliques étaient des fiefs, prouvent seulement qu’elles étaient des terres franches. 5° Les fiefs ne furent établis qu’après la conquête, et les usages saliques existaient avant que les Francs partissent de la Germanie. 6° Ce ne fut point la loi salique qui, en bornant la succession des femmes, forma l’établissement des fiefs ; mais ce fut l’établissement des fiefs qui mit des limites à la succession des femmes et aux dispositions de la loi salique.\par
Après ce que nous venons de dire, on ne croirait pas que la succession perpétuelle des mâles à la couronne de France pût venir de la loi salique. Il est pourtant indubitable qu’elle en vient. Je le prouve par les divers codes des peuples barbares.\par
La loi salique\footnote{Tit. LXII.} et la loi des Bourguignons\footnote{Tit. I, § 3 ; XIV, § I ; et tit. LI.} ne donnèrent point aux filles le droit de succéder à la terre avec leurs frères ; elles ne succédèrent pas non plus à la couronne. La loi des Wisigoths\footnote{Liv. IV, tit. II, § I.}, au contraire, admit les filles\footnote{Les nations germaines, dit Tacite, avaient des usages communs : elles en avaient aussi de particuliers.} à succéder aux terres avec leurs frères ; les femmes furent capables de succéder à la couronne. Chez ces peuples, la disposition de la loi civile força\footnote{La couronne, chez les Ostrogoths, passa deux fois par les femmes aux mâles ; l’une par Amalasunthe, dans la personne d’Athalaric, et l’autre par Amalafrède, dans la personne de Théodat. Ce n’est pas que, chez eux, les femmes ne pussent régner par elles-mêmes : Amalasunthe, après la mort d’Athalaric, régna, et régna même après l’élection de Théodat, et concurremment avec lui. Voyez les lettres d’Amalasunthe et de Théodat dans Cassiodore, liv. XI.} la loi politique.\par
Ce ne fut pas le seul cas où la loi politique, chez les Francs, céda à la loi civile. Par la disposition de la loi salique, tous les frères succédaient également à la terre ; et c’était aussi la disposition de la loi des Bourguignons. Aussi, dans la monarchie des Francs, et dans celle des Bourguignons, tous les frères succédèrent-ils à la couronne, à quelques violences, meurtres et usurpations près, chez les Bourguignons.
\subsubsection[{Chapitre XXIII. De la longue chevelure des rois francs}]{Chapitre XXIII. De la longue chevelure des rois francs}
\noindent Les peuples qui ne cultivent point les terres n’ont pas même l’idée du luxe. Il faut voir dans Tacite l’admirable simplicité des peuples germains : les arts ne travaillaient point à leurs ornements, ils les trouvaient dans la nature. Si la famille de leur chef devait être remarquée par quelque signe, c’était dans cette même nature qu’ils devaient le chercher. les rois des Francs, des Bourguignons et des Wisigoths avaient pour diadème leur longue chevelure.
\subsubsection[{Chapitre XXIV. Des mariages des rois francs}]{Chapitre XXIV. Des mariages des rois francs}
\noindent J’ai dit ci-dessus que, chez les peuples qui ne cultivent point les terres, les mariages étaient beaucoup moins fixes, et qu’on y prenait ordinairement plusieurs femmes. « Les Germains étaient presque les seuls\footnote{{\itshape Prope soli barbarorum singulis uxoribus contenti sunt.De moribus Germ.}} de tous les barbares qui se contentassent d’une seule femme, si l’on en excepte\footnote{{\itshape Exceptis admodum paucis qui, non libidine, sed ob nobilitatem, plurimis nuptiis ambiuntur.Ibid.}}, dit Tacite, quelques personnes qui, non par dissolution, mais à cause de leur noblesse, en avaient plusieurs. »\par
Cela explique comment les rois de la première race eurent un si grand nombre de femmes. Ces mariages étaient moins un témoignage d’incontinence qu’un attribut de dignité : c’eût été les blesser dans un endroit bien tendre, que de leur faire perdre une telle prérogative\footnote{Voyez la {\itshape Chronique} de Frédégaire sur l’an 628.}. Cela explique comment l’exemple des rois ne fut pas suivi par les sujets.
\subsubsection[{Chapitre XXV. Childéric}]{Chapitre XXV. {\itshape Childéric}}
\noindent « Les mariages chez les Germains sont sévères\footnote{{\itshape Severa matrimonia… Germanie. Nemo illic vitia ridet ; nec corrumpere et corrumpi saeculum vocatur.De moribus Germ.}}, dit Tacite : les vices n’y sont point un sujet de ridicule : corrompre, ou être corrompu, ne s’appelle point un usage ou une manière de vivre : il y a peu d’exemples\footnote{{\itshape Paucissima in tam numerosa gente adulteria.Ibid.}}, dans une nation si nombreuse, de la violation de la foi conjugale. »\par
Cela explique l’expulsion de Childéric : il choquait des mœurs rigides, que la conquête n’avait pas eu le temps de changer.
\subsubsection[{Chapitre XXVI. De la majorité des rois francs}]{Chapitre XXVI. De la majorité des rois francs}
\noindent Les peuples barbares qui ne cultivent point les terres n’ont point proprement de territoire, et sont, comme nous avons dit, plutôt gouvernés par le droit des gens que par le droit civil. Ils sont donc presque toujours armés. Aussi Tacite dit-il « que les Germains ne faisaient aucune affaire publique ni particulière sans être armés\footnote{{\itshape Nihil, neque publicae, neque privatae rei, nisi armati agunt.} Tacite, {\itshape De moribus Germ.}}. Ils donnaient leur avis par un signe qu’ils faisaient avec leurs armes\footnote{{\itshape Si displicuit sententia, aspernantur ; sin placuit, frameas concutiunt.Ibid.}}. Sitôt qu’ils pouvaient les porter, ils étaient présentés à l’assemblée\footnote{{\itshape Sed arma sumere non ante cuiquam moris quam civitas suffecturum probaverit.Ibid.}} {\itshape ;} on leur mettait dans les mains un javelot\footnote{{\itshape Tum in ipso concilio, vel principum aliquis, vel pater, vel propinquus, scuto frameaque juvenem ornant.}} : dès ce moment ils sortaient de l’enfance\footnote{{\itshape Haec apud illos toga, hic primus juventae honos ; ante hoc domus pars videntur, mox reipublicae.}} ; ils étaient une partie de la famille, ils en devenaient une de la république ».\par
« Les aigles, disait\footnote{Théodoric, dans Cassiodore, liv. I, lett. 38.} le roi des Ostrogoths, cessent de donner la nourriture à leurs petits sitôt que leurs plumes et leurs ongles sont formés ; ceux-ci n’ont plus besoin du secours d’autrui, quand ils vont eux-mêmes chercher une proie. Il serait indigne que nos jeunes gens qui sont dans nos années fussent censés être dans un âge trop faible pour régir leur bien, et pour régler la conduite de leur vie. C’est la vertu qui fait la majorité chez les Goths. »\par
Childebert II avait quinze\footnote{Il avait à peine cinq ans, dit Grégoire de Tours, liv. V, chap. I, lorsqu’il succéda à son père en l’an 575, c’est-à-dire qu’il avait cinq ans. Gontran le déclara majeur en l’an 585 : il avait donc quinze ans.} ans, lorsque Gontran son oncle le déclara majeur et capable de gouverner par lui-même. On voit, dans la loi des Ripuaires, cet âge de quinze ans, la capacité de porter les armes, et la majorité marcher ensemble. « Si un Ripuaire est mort, ou a été tué, y est-il dit\footnote{Tit. LXXXI.}, et qu’il ait laissé un fils, il ne pourra poursuivre, ni être poursuivi en jugement, qu’il n’ait quinze ans complets ; pour lors il répondra lui-même, ou choisira un champion. » Il fallait que l’esprit fût assez formé pour se défendre dans le jugement, et que le corps le fût assez pour se défendre dans le combat. Chez les Bourguignons\footnote{Tit. LXXXVII.}, qui avaient aussi l’usage du combat dans les actions judiciaires, la majorité était encore à quinze ans.\par
Agathias nous dit que les armes des Francs étaient légères : ils pouvaient donc être majeurs à quinze ans. Dans la suite, les armes devinrent pesantes, et elles l’étaient déjà beaucoup du temps de Charlemagne, comme il paraît par nos capitulaires et par nos romans. Ceux qui\footnote{Il n’y eut point de changement pour les roturiers.} avaient des fiefs, et qui par conséquent devaient faire le service militaire, ne furent plus majeurs qu’à vingt-un ans\footnote{Saint Louis ne fut majeur qu’à cet âge. Cela changea par un édit de Charles V, de l’an 1374.}.
\subsubsection[{Chapitre XXVII. Continuation du même sujet}]{Chapitre XXVII. Continuation du même sujet}
\noindent On a vu que, chez les Germains, on n’allait point à l’assemblée avant la majorité ; on était partie de la famille, et non pas de la République. Cela fit que les enfants de Clodomir, roi d’Orléans et conquérant de la Bourgogne, ne furent point déclarés rois, parce que, dans l’âge tendre où ils étaient, ils ne pouvaient pas être présentés à l’assemblée. Ils n’étaient pas rois encore, mais ils devaient l’être lorsqu’ils seraient capables de porter les armes : et cependant Clotilde leur aïeule gouvernait l’État\footnote{Il paraît par Grégoire de Tours, liv. III, qu’elle choisit deux hommes de Bourgogne, qui était une conquête de Clodomir, pour les élever au siège de Tours, qui était aussi du royaume de Clodomir.}. Leurs oncles Clotaire et Childebert les égorgèrent, et partagèrent leur royaume. Cet exemple fut cause que, dans la suite, les princes pupilles furent déclarés rois, d’abord après la mort de leurs pères. Ainsi le duc Gondovald sauva Childebert II de la cruauté de Chilpéric, et le fit déclarer roi\footnote{Grégoire de Tours, liv. V, chap. I : {\itshape Vix lustro aetatis uno jam peracto, qui die dominicae natalis, regnare caepit.}} à l’âge de cinq ans.\par
Mais, dans ce changement même, on suivit le premier esprit de la nation ; de sorte que les actes ne se passaient pas même au nom des rois pupilles. Aussi y eut-il chez les Francs une double administration : l’une qui regardait la personne du roi pupille, et l’autre qui regardait le royaume ; et dans les fiefs, il y eut une différence entre la tutelle et la baillie.
\subsubsection[{Chapitre XXVIII. De l’adoption chez les germains}]{Chapitre XXVIII. De l’adoption chez les germains}
\noindent Comme chez les Germains on devenait majeur en recevant les armes, on était adopté par le même signe. Ainsi, Gontran voulant déclarer majeur son neveu Childebert, et de plus l’adopter, il lui dit : « J’ai mis\footnote{Voyez Grégoire de Tours, liv. VII, chap. XXIII.} ce javelot dans tes mains, comme un signe que je t’ai donné mon royaume. » Et se tournant vers l’assemblée : « Vous voyez que mon fils Childebert est devenu un homme ; obéissez-lui. » Théodoric, roi des Ostrogoths, voulant adopter le roi des Hérules, lui écrivit\footnote{Dans Cassiodore, liv. IV, lettre II.} : « C’est une belle chose parmi nous de pouvoir être adopté par les armes : car les hommes courageux sont les seuls qui méritent de devenir nos enfants. Il y a une telle force dans cet acte, que celui qui en est l’objet, aimera toujours mieux mourir que de souffrir quelque chose de honteux. Ainsi, par la coutume des nations, et parce que vous êtes un homme, nous vous adoptons par ces boucliers, ces épées, ces chevaux, que nous vous envoyons. »
\subsubsection[{Chapitre XXIX. Esprit sanguinaire des rois francs}]{Chapitre XXIX. Esprit sanguinaire des rois francs}
\noindent Clovis n’avait pas été le seul des princes, chez les Francs, qui eût entrepris des expéditions dans les Gaules. Plusieurs de ses parents y avaient mené des tribus particulières ; et comme il eut de plus grands succès, et qu’il put donner des établissements considérables à ceux qui l’avaient suivi, les Francs accoururent à lui de toutes les tribus, et les autres chefs se trouvèrent trop faibles pour lui résister. Il forma le dessein d’exterminer toute sa maison, et il y réussit\footnote{Grégoire de Tours, liv. II.}. Il craignait, dit Grégoire de Tours\footnote{{\itshape Ibid.}}, que les Francs ne prissent un autre chef. Ses enfants et ses successeurs suivirent cette pratique autant qu’ils purent : on vit sans cesse le frère, l’oncle, le neveu, que dis-je ? le fils, le père, conspirer contre toute sa famille. La loi séparaît sans cesse la monarchie ; la crainte, l’ambition et la cruauté voulaient la réunir.
\subsubsection[{Chapitre XXX. Des assemblées de la nation chez les Francs}]{Chapitre XXX. Des assemblées de la nation chez les Francs}
\noindent On a dit ci-dessus que les peuples qui ne cultivent point les terres jouissaient d’une grande liberté. Les Germains furent dans ce cas. Tacite dit qu’ils ne donnaient à leurs rois ou chefs qu’un pouvoir très modéré\footnote{{\itshape Nec regibus libera aut infinita potestas. Caeterum neque animadvertere neque vincire, neque verberare}, etc. {\itshape De moribus Germ.}} ; et César\footnote{{\itshape In pace nullus est communis magistratus ; sed principes regionum atque pagorum inter suos jus dicunt.} D{\itshape e bello gall.}, liv. VI.}, qu’ils n’avaient pas de magistrat commun pendant la paix, mais que dans chaque village les princes rendaient la justice entre les leurs. Aussi les Francs, dans la Germanie, n’avaient-ils point de roi, comme Grégoire de Tours\footnote{Liv. II.} le prouve très bien.\par
« Les princes\footnote{{\itshape De minoribus principes consultant, de majoribus omnes, ita tamen ut ea quorum pertes plebem arbitrium est, apud principes quoque pertractentur.De moribus Germ.}}, dit Tacite, délibèrent sur les petites choses, toute la nation sur les grandes ; de sorte pourtant que les affaires dont le peuple prend connaissance sont portées de même devant les princes. » Cet usage se conserva après la conquête, comme\footnote{{\itshape Lex consensu populi fît et constitutione regis. Capitulaires} de Charles le Chauve, an 864, art. 6.} on le voit dans tous les monuments.\par
Tacite\footnote{{\itshape Licet apud concilium accusare, et discrimen capitis intendere.De moribus Germ.}} dit que les crimes capitaux pouvaient être portés devant l’assemblée. Il en fut de même après la conquête, et les grands vassaux y furent jugés.
\subsubsection[{Chapitre XXXI. De l’autorité du clergé dans la première race}]{Chapitre XXXI. De l’autorité du clergé dans la première race}
\noindent Chez les peuples barbares, les prêtres ont ordinairement du pouvoir, parce qu’ils ont et l’autorité qu’ils doivent tenir de la religion, et la puissance que chez des peuples pareils donne la superstition. Aussi voyons-nous, dans Tacite, que les prêtres étaient fort accrédités chez les Germains, qu’ils mettaient la police\footnote{{\itshape Silentium per sacerdotes, quibus et coercendi jus est, imperatur.De moribus Germ.}} dans l’assemblée du peuple. Il n’était permis qu’à\footnote{{\itshape Nec regibus libera aut infinita potestas. Caeterum neque animadvertere, neque vincire, neque verberare, nisi sacerdotibus est permissum ; non quasi in paenam, nec ducis jussu, sed velut deo imperante, quem adesse bellatoribus credunt.Ibid.}} eux de châtier, de lier, de frapper : ce qu’ils faisaient, non pas par un ordre du prince, ni pour infliger une peine ; mais comme par une inspiration de la divinité, toujours présente à ceux qui font la guerre.\par
Il ne faut pas être étonné si, dès le commencement de la première race, on voit les évêques arbitres\footnote{Voyez la constitution de Clotaire de l’an 560, art. 6.} des jugements, si on les voit paraître dans les assemblées de la nation, s’ils influent si fort dans les résolutions des rois, et si on leur donne tant de biens.
\subsection[{Livre dix-neuvième. Des lois dans le rapport qu’elles ont avec les principes qui forment. l’esprit général, les mœurs et les manières d’une nation}]{Livre dix-neuvième. Des lois dans le rapport qu’elles ont avec les principes qui forment \\
l’esprit général, les mœurs et les manières d’une nation}
\subsubsection[{Chapitre I. Du sujet de ce livre}]{Chapitre I. Du sujet de ce livre}
\noindent Cette matière est d’une grande étendue. Dans cette foule d’idées qui se présentent à mon esprit, je serai plus attentif à l’ordre des choses qu’aux choses mêmes. Il faut que j’écarte à droite et à gauche, que je perce, et que je me fasse jour.
\subsubsection[{Chapitre II. Combien pour les meilleures lois il est nécessaire que les esprits soient préparés}]{Chapitre II. Combien pour les meilleures lois il est nécessaire que les esprits soient préparés}
\noindent Rien ne parut plus insupportable aux Germains\footnote{Ils coupaient la langue aux avocats et disaient : {\itshape Vipère, cesse de siffler.} Tacite.} que le tribunal de Varus. Celui que Justinien érigea\footnote{Agathias, liv. IV.} chez les Laziens, pour faire le procès au meurtrier de leur roi, leur parut une chose horrible et barbare. Mithridate\footnote{Justin, liv. XXXVIII.}, haranguant contre les Romains, leur reproche surtout les formalités\footnote{{\itshape Calumnias litium.Ibid.}} de leur justice. Les Parthes ne purent supporter ce roi qui, ayant été élevé à Rome, se rendit affable\footnote{{\itshape Prompti aditus, nova comitas, ignotae Parthis virtutes, nova vitia.} Tacite.} et accessible à tout le monde. La liberté même a paru insupportable à des peuples qui n’étaient pas accoutumés à en jouir. C’est ainsi qu’un air pur est quelquefois nuisible à ceux qui ont vécu dans les pays marécageux.\par
Un Vénitien nommé Balbi, étant au Pégu\footnote{Il en a fait la description en 1596. {\itshape Recueil des voyages qui ont servi à l’établissement de la Compagnie des Indes}, t. III, part. I, p. 33.}, fut introduit chez le roi. Quand celui-ci apprit qu’il n’y avait point de roi à Venise, il fit un si grand éclat de rire, qu’une toux le prit, et qu’il eut beaucoup de peine à parler à ses courtisans. Quel est le législateur qui pourrait proposer le gouvernement populaire à des peuples pareils ?
\subsubsection[{Chapitre III. De la tyrannie}]{Chapitre III. {\itshape De la tyrannie}}
\noindent Il y a deux sortes de tyrannie : une réelle, qui consiste dans la violence du gouvernement ; et une d’opinion, qui se fait sentir lorsque ceux qui gouvernent établissent des choses qui choquent la manière de penser d’une nation.\par
Dion dit qu’Auguste voulut se faire appeler Romulus ; mais qu’ayant appris que le peuple craignait qu’il ne voulût se faire roi, il changea de dessein. Les premiers Romains ne voulaient point de roi, parce qu’ils n’en pouvaient souffrir la puissance ; les Romains d’alors ne voulaient point de roi, pour n’en point souffrir les manières. Car, quoique César, les triumvirs, Auguste, fussent de véritables rois, ils avaient gardé tout l’extérieur de l’égalité, et leur vie privée contenait une espèce d’opposition avec le faste des rois d’alors ; et quand ils ne voulaient point de roi, cela signifiait qu’ils voulaient garder leurs manières, et ne pas prendre celles des peuples d’Afrique et d’Orient.\par
Dion\footnote{Liv. LIV, p. 532.} nous dit que le peuple romain était indigné contre Auguste, à cause de certaines lois trop dures qu’il avait faites ; mais que sitôt qu’il eut fait revenir le comédien Pylade, que les factions avaient chassé de la ville, le mécontentement cessa. Un peuple pareil sentait plus vivement la tyrannie lorsqu’on chassait un baladin, que lorsqu’on lui ôtait toutes ses lois.
\subsubsection[{Chapitre IV. Ce que c’est que l’esprit général}]{Chapitre IV. Ce que c’est que l’esprit général}
\noindent Plusieurs choses gouvernent les hommes : le climat, la religion, les lois, les maximes du gouvernement, les exemples des choses passées, les mœurs, les manières ; d’où il se forme un esprit général qui en résulte.\par
À mesure que, dans chaque nation, une de ces causes agit avec plus de force, les autres lui cèdent d’autant. La nature et le climat dominent presque seuls sur les sauvages ; les manières gouvernent les Chinois ; les lois tyrannisent le Japon ; les mœurs donnaient autrefois le ton dans Lacédémone ; les maximes du gouvernement et les mœurs anciennes le donnaient dans Rome.
\subsubsection[{Chapitre V. Combien il faut être attentif à ne point changer l’esprit général d’une nation}]{Chapitre V. Combien il faut être attentif à ne point changer l’esprit général d’une nation}
\noindent S’il y avait dans le monde une nation qui eût une humeur sociable, une ouverture de cœur, une joie dans la vie, un goût, une facilité à communiquer ses pensées ; qui fût vive, agréable, enjouée, quelquefois imprudente, souvent indiscrète ; et qui eût avec cela du courage, de la générosité, de la franchise, un certain point d’honneur, il ne faudrait point chercher à gêner par des lois ses manières, pour ne point gêner ses vertus. Si en général le caractère est bon, qu’importe de quelques défauts qui s’y trouvent ?\par
On y pourrait contenir les femmes, faire des lois pour corriger leurs mœurs, et borner leur luxe ; mais qui sait si on n’y perdrait pas un certain goût qui serait la source des richesses de la nation, et une politesse qui attire chez elle les étrangers ?\par
C’est au législateur à suivre l’esprit de la nation, lorsqu’il n’est pas contraire aux principes du gouvernement ; car nous ne faisons rien de mieux que ce que nous faisons librement, et en suivant notre génie naturel.\par
Qu’on donne un esprit de pédanterie à une nation naturellement gaie, l’État n’y gagnera rien, ni pour le dedans, ni pour le dehors. Laissez-lui faire les choses frivoles sérieusement, et gaiement les choses sérieuses.
\subsubsection[{Chapitre VI. Qu’il ne faut pas tout corriger}]{Chapitre VI. Qu’il ne faut pas tout corriger}
\noindent Qu’on nous laisse comme nous sommes, disait un gentilhomme d’une nation qui ressemble beaucoup à celle dont nous venons de donner une idée. La nature répare tout. Elle nous a donné une vivacité capable d’offenser, et propre à nous faire manquer à tous les égards ; cette même vivacité est corrigée par la politesse qu’elle nous procure, en nous inspirant du goût pour le monde, et surtout pour le commerce des femmes.\par
Qu’on nous laisse tels que nous sommes. Nos qualités indiscrètes, jointes à notre peu de malice, font que les lois qui gêneraient l’humeur sociable parmi nous ne seraient point convenables.
\subsubsection[{Chapitre VII. Des athéniens et des lacédémoniens}]{Chapitre VII. Des athéniens et des lacédémoniens}
\noindent Les Athéniens, continuait ce gentilhomme, étaient un peuple qui avait quelque rapport avec le nôtre. Il mettait de la gaieté dans les affaires ; un trait de raillerie lui plaisait sur la tribune comme sur le théâtre. Cette vivacité qu’il mettait dans les conseils, il la portait dans l’exécution. Le caractère des Lacédémoniens était grave, sérieux, sec, taciturne. On n’aurait pas plus tiré parti d’un Athénien en l’ennuyant, que d’un Lacédémonien en le divertissant.
\subsubsection[{Chapitre VIII. Effets de l’humeur sociable}]{Chapitre VIII. Effets de l’humeur sociable}
\noindent Plus les peuples se communiquent, plus ils changent aisément de manières, parce que chacun est plus un spectacle pour un autre ; on voit mieux les singularités des individus. Le climat qui fait qu’une nation aime à se communiquer fait aussi qu’elle aime à changer ; et ce qui fait qu’une nation aime à changer fait aussi qu’elle se forme le goût.\par
La société des femmes gâte les mœurs, et forme le goût : l’envie de plaire plus que les autres établit les parures ; et l’envie de plaire plus que soi-même établit les modes. Les modes sont un objet important : à force de se rendre l’esprit frivole, on augmente sans cesse les branches de son commerce\footnote{Voyez {\itshape La Fable des abeilles}.}.
\subsubsection[{Chapitre IX. De la vanité et de l’orgueil des nations}]{Chapitre IX. De la vanité et de l’orgueil des nations}
\noindent La vanité est un aussi bon ressort pour un gouvernement, que l’orgueil en est un dangereux. Il n’y a pour cela qu’à se représenter, d’un côté, les biens sans nombre qui résultent de la vanité : de là le luxe, l’industrie, les arts, les modes, la politesse, le goût ; et, d’un autre côté, les maux infinis qui naissent de l’orgueil de certaines nations : la paresse, la pauvreté, l’abandon de tout, la destruction des nations que le hasard a fait tomber entre leurs mains, et de la leur même. La paresse\footnote{Les peuples qui suivent le kan de Malacamber, ceux de Carnataca et de Coromandel, sont des peuples orgueilleux et paresseux ; ils consomment peu, parce qu’ils sont misérables ; au lieu que les Mogols et les peuples de l’Hindoustan s’occupent et jouissent des commodités de la vie, comme les Européens. {\itshape Recueil des voyages qui ont servi à l’établissement de la Compagnie des Indes}, t. I, p. 54.} est l’effet de l’orgueil ; le travail est une suite de la vanité : l’orgueil d’un Espagnol le portera à ne pas travailler ; la vanité d’un Français le portera à savoir travailler mieux que les autres.\par
Toute nation paresseuse est grave ; car ceux qui ne travaillent pas se regardent comme souverains de ceux qui travaillent.\par
Examinez toutes les nations, et vous verrez que, dans la plupart, la gravité, l’orgueil et la paresse marchent du même pas.\par
Les peuples d’Achim\footnote{Voyez Dampierre, t. III.} sont fiers et paresseux : ceux qui n’ont point d’esclaves en louent un, ne fût-ce que pour faire cent pas, et porter deux pintes de riz ; ils se croiraient déshonorés s’ils les portaient eux-mêmes.\par
Il y a plusieurs endroits de la terre où l’on se laisse croître les ongles pour marquer que l’on ne travaille point.\par
Les femmes des Indes\footnote{{\itshape Lettres édifiantes}, douzième recueil, p. 80.} croient qu’il est honteux pour elles d’apprendre à lire : c’est l’affaire, disent-elles, des esclaves qui chantent des cantiques dans les pagodes. Dans une caste, elles ne filent point ; dans une autre, elles ne font que des paniers et des nattes, elles ne doivent pas même piler le riz ; dans d’autres, il ne faut pas qu’elles aillent quérir de l’eau, L’orgueil y a établi ses règles, et il les fait suivre. Il n’est pas nécessaire de dire que les qualités morales ont des effets différents selon qu’elles sont unies à d’autres : ainsi l’orgueil, joint à une vaste ambition, à la grandeur des idées, etc., produisit chez les Romains les effets que l’on sait.
\subsubsection[{Chapitre X. Du caractère des Espagnols et de celui des Chinois}]{Chapitre X. Du caractère des Espagnols et de celui des Chinois}
\noindent Les divers caractères des nations sont mêlés de vertus et de vices, de bonnes et de mauvaises qualités. Les heureux mélanges sont ceux dont il résulte de grands biens, et souvent on ne les soupçonnerait pas ; il y en a dont il résulte de grands maux, et qu’on ne soupçonnerait pas non plus.\par
La bonne foi des Espagnols a été fameuse dans tous les temps. Justin\footnote{Liv. XLIII.} nous parle de leur fidélité à garder les dépôts : ils ont souvent souffert la mort pour les tenir secrets. Cette fidélité qu’ils avaient autrefois, ils l’ont encore aujourd’hui. Toutes les nations qui commercent à Cadix confient leur fortune aux Espagnols ; elles ne s’en sont jamais repenties. Mais cette qualité admirable, jointe à leur paresse, forme un mélange dont il résulte des effets qui leur sont pernicieux : les peuples de l’Europe font, sous leurs yeux, tout le commerce de leur monarchie.\par
Le caractère des Chinois forme un autre mélange, qui est en contraste avec le caractère des Espagnols. Leur vie précaire\footnote{Par la nature du climat et du terrain.} fait qu’ils ont une activité prodigieuse et un désir si excessif du gain, qu’aucune nation commerçante ne peut se fier à eux\footnote{Le P. Du Halde, t. II.}. Cette infidélité reconnue leur a conservé le commerce du Japon ; aucun négociant d’Europe n’a osé entreprendre de le faire sous leur nom, quelque facilité qu’il y eût eu à l’entreprendre par leurs provinces maritimes du Nord.
\subsubsection[{Chapitre XI. Réflexion}]{Chapitre XI. {\itshape Réflexion}}
\noindent Je n’ai point dit ceci pour diminuer rien de la distance infinie qu’il y a entre les vices et les vertus : à Dieu ne plaise ! J’ai seulement voulu faire comprendre que tous les vices politiques ne sont pas des vices moraux, et que tous les vices moraux ne sont pas des vices politiques ; et c’est ce que ne doivent point ignorer ceux qui font des lois qui choquent l’esprit général.
\subsubsection[{Chapitre XII. Des manières et des mœurs dans l’état despotique}]{Chapitre XII. Des manières et des mœurs dans l’état despotique}
\noindent C’est une maxime capitale, qu’il ne faut jamais changer les mœurs et les manières dans l’État despotique ; rien ne serait plus promptement suivi d’une révolution. C’est que, dans ces États, il n’y a point de lois, pour ainsi dire ; il n’y a que des mœurs et des manières ; et, si vous renversez cela, vous renversez tout.\par
Les lois sont établies, les mœurs sont inspirées ; celles-ci tiennent plus à l’esprit général, celles-là tiennent plus à une institution particulière : or il est aussi dangereux, et plus, de renverser l’esprit général, que de changer une institution particulière.\par
On se communique moins dans les pays où chacun, et comme supérieur et comme inférieur, exerce et souffre un pouvoir arbitraire, que dans ceux où la liberté règne dans toutes les conditions. On y change donc moins de manières et de mœurs ; les manières plus fixes approchent plus des lois : ainsi, il faut qu’un prince ou un législateur y choque moins les mœurs et les manières que dans aucun pays du monde.\par
Les femmes y sont ordinairement enfermées, et n’ont point de ton à donner. Dans les autres pays où elles vivent avec les hommes, l’envie qu’elles ont de plaire, et le désir que l’on a de leur plaire aussi, font que l’on change continuellement de manières. Les deux sexes se gâtent, ils perdent l’un et l’autre leur qualité distinctive et essentielle ; il se met un arbitraire dans ce qui était absolu, et les manières changent tous les jours.
\subsubsection[{Chapitre XIII. Des manières chez les Chinois}]{Chapitre XIII. Des manières chez les Chinois}
\noindent Mais c’est à la Chine que les manières sont indestructibles. Outre que les femmes y sont absolument séparées des hommes, on enseigne dans les écoles les manières comme les mœurs. On connaît un lettré\footnote{Dit le P. Du Halde.} à la façon aisée dont il fait la révérence. Ces choses, une fois données en préceptes et par de graves docteurs, s’y fixent comme des principes de morale, et ne changent plus.
\subsubsection[{Chapitre XIV. Quels sont les moyens naturels de changer les mœurs et les manières d’une nation}]{Chapitre XIV. Quels sont les moyens naturels de changer les mœurs et les manières d’une nation}
\noindent Nous avons dit que les lois étaient des institutions particulières et précises du législateur, et les mœurs et les manières, des institutions de la nation en général. De là il suit que lorsqu’on veut changer les mœurs et les manières, il ne faut pas les changer par les lois : cela paraîtrait trop tyrannique ; il vaut mieux les changer par d’autres mœurs et d’autres manières.\par
Ainsi, lorsqu’un prince veut faire de grands changements dans sa nation, il faut qu’il réforme par les lois ce qui est établi par les lois, et qu’il change par les manières ce qui est établi par les manières : et c’est une très mauvaise politique de changer par les lois ce qui doit être changé par les manières.\par
La loi qui obligeait les Moscovites à se faire couper la barbe et les habits, et la violence de Pierre I\textsuperscript{er}, qui faisait tailler jusqu’aux genoux les longues robes de ceux qui entraient dans les villes, étaient tyranniques. Il y a des moyens pour empêcher les crimes : ce sont les peines ; il y en a pour faire changer les manières : ce sont les exemples.\par
La facilité et la promptitude avec laquelle cette nation s’est policée ont bien montré que ce prince avait trop mauvaise opinion d’elle, et que ces peuples n’étaient pas des bêtes, comme il le disait. Les moyens violents qu’il employa étaient inutiles ; il serait arrivé tout de même à son but par la douceur.\par
Il éprouva lui-même la facilité de ces changements. Les femmes étaient renfermées, et en quelque façon esclaves ; il les appela à la cour, il les fit habiller à l’allemande, il leur envoyait des étoffes. Ce sexe goûta d’abord une façon de vivre qui flattait si fort son goût, sa vanité et ses passions, et la fit goûter aux hommes.\par
Ce qui rendit le changement plus aisé, c’est que les mœurs d’alors étaient étrangères au climat, et y avaient été apportées par le mélange des nations et par les conquêtes. Pierre I\textsuperscript{er}, donnant les mœurs et les manières de l’Europe à une nation d’Europe, trouva des facilités qu’il n’attendait pas lui-même. L’empire du climat est le premier de tous les empires. Il n’avait donc pas besoin de lois pour changer les mœurs et les manières de sa nation : il lui eût suffi d’inspirer d’autres mœurs et d’autres manières.\par
En général, les peuples sont très attachés à leurs coutumes ; les leur ôter violemment, c’est les rendre malheureux : il ne faut donc pas les changer, mais les engager à les changer eux-mêmes.\par
Toute peine qui ne dérive pas de la nécessité est tyrannique. La loi n’est pas un pur acte de puissance ; les choses indifférentes par leur nature ne sont pas de son ressort.
\subsubsection[{Chapitre XV. Influence du gouvernement domestique sur le politique}]{Chapitre XV. Influence du gouvernement domestique sur le politique}
\noindent Ce changement des mœurs des femmes influera sans doute beaucoup dans le gouvernement de Moscovie. Tout est extrêmement lié : le despotisme du prince s’unit naturellement avec la servitude des femmes ; la liberté des femmes avec l’esprit de la monarchie.
\subsubsection[{Chapitre XVI. Comment quelques législateurs ont confondu les principes qui gouvernent les hommes}]{Chapitre XVI. Comment quelques législateurs ont confondu les principes qui gouvernent les hommes}
\noindent Les mœurs et les manières sont des usages que les lois n’ont point établis, ou n’ont pas pu, ou n’ont pas voulu établir.\par
Il y a cette différence entre les lois et les mœurs, que les lois règlent plus les actions du citoyen, et que les mœurs règlent plus les actions de l’homme. Il y a cette différence entre les mœurs {\itshape et les} manières, que les premières regardent plus la conduite intérieure, les autres l’extérieure.\par
Quelquefois, dans un État, ces choses se confondent\footnote{Moïse fit un même code pour les lois et la religion. Les premiers Romains confondirent les coutumes anciennes avec les lois.}. Lycurgue fit un même code pour les lois, les mœurs et les manières ; et les législateurs de la Chine en firent de même.\par
Il ne faut pas être étonné si les législateurs de Lacédémone et de la Chine confondirent les lois, les mœurs et les manières : c’est que les mœurs représentent les lois, et les manières représentent les mœurs.\par
Les législateurs de la Chine avaient pour principal {\itshape objet de} faire vivre leur peuple tranquille. Ils voulurent que les hommes se respectassent beaucoup ; que chacun sentît à tous les instants qu’il devait beaucoup aux autres, qu’il n’y avait point de citoyen qui ne dépendît, à quelque égard, d’un autre citoyen. Ils donnèrent donc aux règles de la civilité la plus grande étendue.\par
Ainsi, chez les peuples chinois, on vit les gens\footnote{Voyez le P. Du Halde.} de village observer entre eux des cérémonies comme les gens d’une condition relevée : moyen très propre à inspirer la douceur, à maintenir parmi le peuple la paix et le bon ordre, et à ôter tous les vices qui viennent d’un esprit dur. En effet, s’affranchir des règles de la civilité, n’est-ce pas chercher le moyen de mettre ses défauts plus à l’aise ?\par
La civilité vaut mieux, à cet égard, que la politesse. La politesse flatte les vices des autres, et la civilité nous empêche de mettre les nôtres au jour : c’est une barrière que les hommes mettent entre eux pour s’empêcher de se corrompre.\par
Lycurgue, dont les institutions étaient dures, n’eut point la civilité pour objet lorsqu’il forma les manières : il eut en vue cet esprit belliqueux qu’il voulait donner à son peuple. Des gens toujours corrigeant, ou toujours corrigés, qui instruisaient toujours et étaient toujours instruits, également simples et rigides, exerçaient plutôt entre eux des vertus qu’ils n’avaient des égards.
\subsubsection[{Chapitre XVII. Propriété particulière au gouvernement de la Chine}]{Chapitre XVII. Propriété particulière au gouvernement de la Chine}
\noindent Les législateurs de la Chine firent plus\footnote{Voyez les livres classiques dont le P. Du Halde nous a donné de si beaux morceaux.} : ils confondirent la religion, les lois, les mœurs et les manières ; tout cela fut la morale, tout cela fut la vertu. Les préceptes qui regardaient ces quatre points furent ce que l’on appela les rites. Ce fut dans l’observation exacte de ces rites que le gouvernement chinois triompha. On passa toute sa jeunesse à les apprendre, toute sa vie à les pratiquer. Les lettrés les enseignèrent, les magistrats les prêchèrent. Et, comme ils enveloppaient toutes les petites actions de la vie, lorsqu’on trouva le moyen de les faire observer exactement, la Chine fut bien gouvernée.\par
Deux choses ont pu aisément graver les rites dans le cœur et l’esprit des Chinois : l’une, leur manière d’écrire extrêmement composée, qui a fait que, pendant une très grande partie de la vie, l’esprit a été uniquement\footnote{C’est ce qui a établi l’émulation, la fuite de l’oisiveté, et l’estime pour le savoir.} occupé de ces rites, parce qu’il a fallu apprendre à lire dans les livres, et pour les livres qui les contenaient ; l’autre, que les préceptes des rites n’ayant rien de spirituel, mais simplement des règles d’une pratique commune, il est plus aisé d’en convaincre et d’en frapper les esprits que d’une chose intellectuelle.\par
Les princes qui, au lieu de gouverner par les rites gouvernèrent par la force des supplices, voulurent faire faire aux supplices ce qui n’est pas dans leur pouvoir, qui est de donner des mœurs. Les supplices retrancheront bien de la société un citoyen qui, ayant perdu ses mœurs, viole les lois ; mais si tout le monde a perdu ses mœurs, les rétabliront-ils ? Les supplices arrêteront bien plusieurs conséquences du mal général, mais ils ne corrigeront pas ce mal. Aussi, quand on abandonna les principes du gouvernement chinois, quand la morale y fut perdue, l’État tomba-t-il dans l’anarchie, et l’on vit des révolutions.
\subsubsection[{Chapitre XVIII. Conséquence du chapitre précédent}]{Chapitre XVIII. Conséquence du chapitre précédent}
\noindent Il résulte de là que la Chine ne perd point ses lois par la conquête. Les manières, les mœurs, les lois, la religion y étant la même chose, on ne peut changer tout cela à la fois. Et comme il faut que le vainqueur ou le vaincu changent, il a toujours fallu à la Chine que ce fût le vainqueur : car ses mœurs n’étant point ses manières, ses manières ses lois, ses lois sa religion, il a été plus aisé qu’il se pliât peu à peu au peuple vaincu, que le peuple vaincu à lui.\par
Il suit encore de là une chose bien triste : c’est qu’il n’est presque pas possible que le christianisme s’établisse jamais à la Chine\footnote{Voyez les raisons données par les magistrats chinois, dans les décrets par lesquels ils proscrivent la religion chrétienne. {\itshape Lettres édifiantes}, dix-septième recueil.}. Les vœux de virginité, les assemblées des femmes dans les églises, leur communication nécessaire avec les ministres de la religion, leur participation aux sacrements, la confession auriculaire, l’extrême-onction, le mariage d’une seule femme : tout cela renverse les mœurs et les manières du pays, et frappe encore du même coup sur la religion et sur les lois.\par
La religion chrétienne, par l’établissement de la charité, par un culte public, par la participation aux mêmes sacrements, semble demander que tout s’unisse : les rites des Chinois semblent ordonner que tout se sépare.\par
Et, comme on a vu que cette séparation\footnote{Voyez le liv. IV, chap. III, et le liv. XIX, chap. XIII.} tient en général à l’esprit du despotisme, on trouvera dans ceci une des raisons qui font que le gouvernement monarchique et tout gouvernement modéré s’allient mieux\footnote{Voyez ci-après le liv. XXIV, chap. III.} avec la religion chrétienne.
\subsubsection[{Chapitre XIX. Comment s’est faite cette union de la religion, des lois, des mœurs et des manières chez les Chinois}]{Chapitre XIX. Comment s’est faite cette union de la religion, des lois, des mœurs et des manières chez les Chinois}
\noindent Les législateurs de la Chine eurent pour principal objet du gouvernement la tranquillité de l’empire. La subordination leur parut le moyen le plus propre à la maintenir. Dans cette idée, ils crurent devoir inspirer le respect pour les pères, et ils rassemblèrent toutes leurs forces pour cela. Ils établirent une infinité de rites et de cérémonies, pour les honorer pendant leur vie et après leur mort. Il était impossible de tant honorer les pères morts sans être porté à les honorer vivants. Les cérémonies pour les pères morts avaient plus de rapport à la religion, celles pour les pères vivants avaient plus de rapport aux lois, aux mœurs et aux manières : mais ce n’était que les parties d’un même code, et ce code était très étendu.\par
Le respect pour les pères était nécessairement lié avec tout ce qui représentait les pères : les vieillards, les maîtres, les magistrats, l’empereur. Ce respect pour les pères supposait un retour d’amour pour les enfants ; et, par conséquent, le même retour des vieillards aux jeunes gens, des magistrats à ceux qui leur étaient soumis, de l’empereur à ses sujets. Tout cela formait les rites, et ces rites l’esprit général de la nation.\par
On va sentir le rapport que peuvent avoir, avec la constitution fondamentale de la Chine, les choses qui paraissent les plus indifférentes. Cet empire est formé sur l’idée du gouvernement d’une famille. Si vous diminuez l’autorité paternelle, ou même si vous retranchez les cérémonies qui expriment le respect que l’on a pour elle, vous affaiblissez le respect pour les magistrats qu’on regarde comme des pères ; les magistrats n’auront plus le même soin pour les peuples, qu’ils doivent considérer comme des enfants ; ce rapport d’amour qui est entre le prince et les sujets se perdra aussi peu à peu. Retranchez une de ces pratiques, et vous ébranlez l’État. Il est fort indifférent en soi que tous les matins une belle-fille se lève pour aller rendre tels et tels devoirs à sa belle-mère ; mais, si l’on fait attention que ces pratiques extérieures rappellent sans cesse à un sentiment qu’il est nécessaire d’imprimer dans tous les cœurs, et qui va de tous les cœurs former l’esprit qui gouverne l’empire, l’on verra qu’il est nécessaire qu’une telle ou une telle action particulière se fasse.
\subsubsection[{Chapitre XX. Explication d’un paradoxe sur les Chinois}]{Chapitre XX. Explication d’un paradoxe sur les Chinois}
\noindent Ce qu’il y a de singulier, c’est que les Chinois, dont la vie est entièrement dirigée par les rites, sont néanmoins le peuple le plus fourbe de la terre. Cela paraît surtout dans le commerce, qui n’a jamais pu leur inspirer la bonne foi qui lui est naturelle. Celui qui achète doit porter\footnote{Journal de Lange en 1721 et 1722 ; tome VIII des {\itshape Voyages du Nord}, p. 363.} sa propre balance ; chaque marchand en ayant trois, une forte pour acheter, une légère pour vendre, et une juste pour ceux qui sont sur leurs gardes. Je crois pouvoir expliquer cette contradiction.\par
Les législateurs de la Chine ont eu deux objets : ils ont voulu que le peuple fût soumis et tranquille, et qu’il fût laborieux et industrieux. Par la nature du climat et du terrain, il a une vie précaire ; on n’y est assuré de sa vie qu’à force d’industrie et de travail.\par
Quand tout le monde obéit et que tout le monde travaille, l’État est dans une heureuse situation. C’est la nécessité, et peut-être la nature du climat, qui ont donné à tous les Chinois une avidité inconcevable pour le gain ; et les lois n’ont pas songé à l’arrêter. Tout a été défendu, quand il a été question d’acquérir par violence ; tout a été permis, quand il s’est agi d’obtenir par artifice ou par industrie. Ne comparons donc pas la morale des Chinois avec celle de l’Europe. Chacun, à la Chine, a dû être attentif à ce qui lui était utile ; si le fripon a veillé à ses intérêts, celui qui est dupe devait penser aux siens. À Lacédémone, il était permis de voler ; à la Chine, il est permis de tromper.
\subsubsection[{Chapitre XXI. Comment les lois doivent être relatives aux mœurs et aux manières}]{Chapitre XXI. Comment les lois doivent être relatives aux mœurs et aux manières}
\noindent Il n’y a que des institutions singulières qui confondent ainsi des choses naturellement séparées, les lois, les mœurs et les manières ; mais quoiqu’elles soient séparées, elles ne laissent pas d’avoir entre elles de grands rapports.\par
On demanda à Solon si les lois qu’il avait données aux Athéniens étaient les meilleures : « Je leur ai donné, répondit-il, les meilleures de celles qu’ils pouvaient souffrir. » Belle parole, qui devrait être entendue de tous les législateurs. Quand la sagesse divine dit au peuple juif : « Je vous ai donné des préceptes qui ne sont pas bons », cela signifie qu’ils n’avaient qu’une bonté relative ; ce qui est l’éponge de toutes les difficultés que l’on peut faire sur les lois de Moïse.
\subsubsection[{Chapitre XXII. Continuation du même sujet}]{Chapitre XXII. Continuation du même sujet}
\noindent Quand un peuple a de bonnes mœurs, les lois deviennent simples. Platon\footnote{Des lois, liv, XII.} dit que Rhadamanthe, qui gouvernait un peuple extrêmement religieux, expédiait tous les procès avec célérité, déférant seulement le serment sur chaque chef. Mais, dit le même Platon\footnote{{\itshape Ibid.}}, quand un peuple n’est pas religieux, on ne peut faire usage du serment que dans les occasions où celui qui jure est sans intérêt, comme un juge et des témoins.
\subsubsection[{Chapitre XXIII. Comment les lois suivent les mœurs}]{Chapitre XXIII. Comment les lois suivent les mœurs}
\noindent Dans le temps que les mœurs des Romains étaient pures, il n’y avait point de loi particulière contre le péculat. Quand ce crime commença à paraître, il fut trouvé si infâme, que d’être condamné à restituer\footnote{{\itshape In Simplum}.} ce qu’on avait pris, fut regardé comme une grande peine : témoin le jugement de L. Scipion\footnote{Tite-Live, liv. XXXVIII.}.
\subsubsection[{Chapitre XXIV. Continuation du même sujet}]{Chapitre XXIV. Continuation du même sujet}
\noindent Les lois qui donnent la tutelle à la mère ont plus d’attention à la conservation de la personne du pupille ; celles qui la donnent au plus proche héritier ont plus d’attention à la conservation des biens. Chez les peuples dont les mœurs sont corrompues, il vaut mieux donner la tutelle à la mère. Chez ceux où les lois doivent avoir de la confiance dans les mœurs des citoyens, on donne la tutelle à l’héritier des biens, ou à la mère, et quelquefois à tous les deux.\par
Si l’on réfléchit sur les lois romaines, on trouvera que leur esprit est conforme à ce que je dis. Dans le temps où l’on fit la loi des Douze Tables, les mœurs à Rome étaient admirables. On déféra la tutelle au plus proche parent du pupille, pensant que celui-là devait avoir la charge de la tutelle, qui pouvait avoir l’avantage de la succession. On ne crut point la vie du pupille en danger, quoiqu’elle fût mise entre les mains de celui à qui sa mort devait être utile. Mais, lorsque les mœurs changèrent à Rome, on vit les législateurs changer aussi de façon de penser. « Si, dans la substitution pupillaire, disent {\itshape Gaïus}\footnote{{\itshape Institutes}, liv. II, tit. VI, § 2 ; la compilation d’Ozel, à Leyde, 1658.} et Justinien\footnote{{\itshape Institutes}, liv. II, de pupil. substit., § 3.}, le testateur craint que le substitué ne dresse des embûches au pupillaire, il peut laisser à découvert la substitution vulgaire\footnote{La substitution vulgaire est : Si un tel ne prend pas l’hérédité, je lui substitue, etc. La pupillaire est : Si un tel meurt avant sa puberté, je lui substitue, etc.}, et mettre la pupillaire dans une partie du testament qu’on ne pourra ouvrir qu’après un certain temps. » Voilà des craintes et des précautions inconnues aux premiers Romains.
\subsubsection[{Chapitre XXV. Continuation du même sujet}]{Chapitre XXV. Continuation du même sujet}
\noindent La loi romaine donnait la liberté de se faire des dons avant le mariage ; après le mariage elle ne le permettait plus. Cela était fondé sur les mœurs des Romains, qui n’étaient portés au mariage que par la frugalité, la simplicité et la modestie, mais qui pouvaient se laisser séduire par les soins domestiques, les complaisances et le bonheur de toute une vie.\par
La loi des Wisigoths\footnote{Liv. III, tit. I, § 5.} voulait que l’époux ne pût donner à celle qu’il devait épouser au-delà du dixième de ses biens, et qu’il ne pût lui rien donner la première année de son mariage. Cela venait encore des mœurs du pays. Les législateurs voulaient arrêter cette jactance espagnole, uniquement portée à faire des libéralités excessives dans une action d’éclat.\par
Les Romains, par leurs lois, arrêtèrent quelques inconvénients de l’empire du monde le plus durable, qui est celui de la vertu : les Espagnols, par les leurs, voulaient empêcher le mauvais effet de la tyrannie du monde la plus fragile, qui est celle de la beauté.
\subsubsection[{Chapitre XXVI. Continuation du même sujet}]{Chapitre XXVI. Continuation du même sujet}
\noindent La loi de Théodose et de Valentinien\footnote{Leg. 8, Cod. {\itshape de repudiis.}} tira les causes de répudiation des anciennes mœurs\footnote{Et de la loi des Douze Tables. Voyez Cicéron, {\itshape seconde Philippique}.} et des manières de Romains. Elle mit au nombre de ces causes l’action d’un mari\footnote{{\itshape Si verberibus, quae ingenuis aliena sunt, afficientem probaverit.}} qui châtierait sa femme d’une manière indigne d’une personne ingénue. Cette cause fut omise dans les lois suivantes\footnote{Dans la {\itshape Novelle} CXVII, chap. XIV.} : c’est que les mœurs avaient changé à cet égard ; les usages d’Orient avaient pris la place de ceux d’Europe. Le premier eunuque de l’impératrice femme de Justinien second la menaça, dit l’histoire, de ce châtiment dont on punit les enfants dans les écoles. Il n’y a que des mœurs établies, ou des mœurs qui cherchent à s’établir, qui puissent faire imaginer une pareille chose.\par
Nous avons vu comment les lois suivent les mœurs : voyons à présent comment les mœurs suivent les lois.
\subsubsection[{Chapitre XXVII. Comment les lois peuvent contribuer à former les mœurs, les manières et le caractère d’une nation}]{Chapitre XXVII. Comment les lois peuvent contribuer à former les mœurs, les manières et le caractère d’une nation}
\noindent Les coutumes d’un peuple esclave sont une partie de sa servitude : celles d’un peuple libre sont une partie de sa liberté.\par
J’ai parlé au livre XI\footnote{Chap. VI.} d’un peuple libre ; j’ai donné les principes de sa constitution : voyons les effets qui ont dû suivre, le caractère qui a pu s’en former, et les manières qui en résultent.\par
Je ne dis point que le climat n’ait produit, en grande partie, les lois, les mœurs et les manières de cette nation ; mais je dis que les mœurs et les manières de cette nation devraient avoir un grand rapport à ses lois.\par
Comme il y aurait dans cet État deux pouvoirs visibles, la puissance législative et l’exécutrice, et que tout citoyen y aurait sa volonté propre, et ferait valoir à son gré son indépendance, la plupart des gens auraient plus d’affection pour une de ces puissances pour que l’autre, le grand nombre n’ayant pas ordinairement assez d’équité ni de sens pour les affectionner également toutes les deux.\par
Et, comme la puissance exécutrice, disposant de tous les emplois, pourrait donner de grandes espérances et jamais de craintes, tous ceux qui obtiendraient d’elle seraient portés à se tourner de son côté, et elle pourrait être attaquée par tous ceux qui n’en espéreraient rien.\par
Toutes les passions y étant libres, la haine, l’envie, la jalousie, l’ardeur de s’enrichir et de se distinguer, paraîtraient dans toute leur étendue ; et si cela était autrement, l’État serait comme un homme abattu par la maladie, qui n’a point de passions parce qu’il n’a point de forces.\par
La haine qui serait entre les deux partis durerait, parce qu’elle serait toujours impuissante.\par
Ces partis étant composés d’hommes libres, si l’un prenait trop le dessus, l’effet de la liberté ferait que celui-ci serait abaissé, tandis que les citoyens, comme les mains qui secourent le corps, viendraient relever l’autre.\par
Comme chaque particulier, toujours indépendant, suivrait beaucoup ses caprices et ses fantaisies, ou changerait souvent de parti ; on en abandonnerait un où l’on laisserait tous ses amis pour se lier à un autre dans lequel on trouverait tous ses ennemis ; et souvent, dans cette nation, on pourrait oublier les lois de l’amitié et celles de la haine.\par
Le monarque serait dans le cas des particuliers ; et, contre les maximes ordinaires de la prudence, il serait souvent obligé de donner sa confiance à ceux qui l’auraient le plus choqué, et de disgracier ceux qui l’auraient le mieux servi, faisant par nécessité ce que les autres princes font par choix.\par
On craint de voir échapper un bien que l’on sent, que l’on ne connaît guère, et qu’on peut nous déguiser ; et la crainte grossit toujours les objets. Le peuple serait inquiet sur sa situation, et croirait être en danger dans les moments mêmes les plus sûrs.\par
D’autant mieux que ceux qui s’opposeraient le plus vivement à la puissance exécutrice, ne pouvant avouer les motifs intéressés de leur opposition, ils augmenteraient les terreurs du peuple, qui ne saurait jamais au juste s’il serait en danger ou non. Mais cela même contribuerait à lui faire éviter les vrais périls où il pourrait, dans la suite, être exposé.\par
Mais le corps législatif ayant la confiance du peuple, et étant plus éclairé que lui, il pourrait le faire revenir des mauvaises impressions qu’on lui aurait données, et calmer ses mouvements.\par
C’est le grand avantage qu’aurait ce gouvernement sur les démocraties anciennes dans lesquelles le peuple avait une puissance immédiate ; car, lorsque les orateurs l’agitaient, ces agitations avaient toujours leur effet.\par
Ainsi, quand les terreurs imprimées n’auraient point d’objet certain, elles ne produiraient que de vaines clameurs et des injures : et elles auraient même ce bon effet, qu’elles tendraient tous les ressorts du gouvernement, et rendraient tous les citoyens attentifs. Mais si elles naissaient à l’occasion du renversement des lois fondamentales, elles seraient sourdes, funestes, atroces, et produiraient des catastrophes.\par
Bientôt on verrait un calme affreux, pendant lequel tout se réunirait contre la puissance violatrice des lois.\par
Si, dans le cas où les inquiétudes n’ont pas d’objet certain, quelque puissance étrangère menaçait l’État, et le mettait en danger de sa fortune ou de sa gloire ; pour lors, les petits intérêts cédant aux plus grands, tout se réunirait en faveur de la puissance exécutrice.\par
Que si les disputes étaient formées à l’occasion de la violation des lois fondamentales, et qu’une puissance étrangère parût, il y aurait une révolution qui ne changerait pas la forme du gouvernement, ni sa constitution : car les révolutions que forme la liberté ne sont qu’une confirmation de la liberté.\par
Une nation libre peut avoir un libérateur ; une nation subjuguée ne peut avoir qu’un autre oppresseur.\par
Car tout homme qui a assez de force pour chasser celui qui est déjà le maître absolu dans un État, en a assez pour le devenir lui-même.\par
Comme, pour jouir de la liberté, il faut que chacun puisse dire ce qu’il pense ; et que, pour la conserver, il faut encore que chacun puisse dire ce qu’il pense, un citoyen, dans cet État, dirait et écrirait tout ce que les lois ne lui ont pas défendu expressément de dire ou d’écrire.\par
Cette nation, toujours échauffée, pourrait plus aisément être conduite par ses passions que par la raison, qui ne produit jamais de grands effets sur l’esprit des hommes ; et il serait facile à ceux qui la gouverneraient de lui faire faire des entreprises contre ses véritables intérêts.\par
Cette nation aimerait prodigieusement sa liberté, parce que cette liberté serait vraie ; et il pourrait arriver que, pour la défendre, elle sacrifierait son bien, son aisance, ses intérêts ; qu’elle se chargerait des impôts les plus durs, et tels que le prince le plus absolu n’oserait les faire supporter à ses sujets.\par
Mais, comme elle aurait une connaissance certaine de la nécessité de s’y soumettre, qu’elle paierait dans l’espérance bien fondée de ne payer plus ; les charges y seraient plus pesantes que le sentiment de ces charges ; au lieu qu’il y a des États où le sentiment est infiniment au-dessus du mal.\par
Elle aurait un crédit sûr, parce qu’elle emprunterait à elle-même, et se paierait elle-même. Il pourrait arriver qu’elle entreprendrait au-dessus de ses forces naturelles, et ferait valoir contre ses ennemis des immenses richesses de fiction, que la confiance et la nature de son gouvernement rendraient réelles.\par
Pour conserver sa liberté, elle emprunterait de ses sujets ; et ses sujets, qui verraient que son crédit serait perdu si elle était conquise, auraient un nouveau motif de faire des efforts pour défendre sa liberté.\par
Si cette nation habitait une île, elle ne serait point conquérante, parce que des conquêtes séparées l’affaibliraient. Si le terrain de cette île était bon, elle le serait encore moins, parce qu’elle n’aurait pas besoin de la guerre pour s’enrichir. Et, comme aucun citoyen ne dépendrait d’un autre citoyen, chacun ferait plus de cas de sa liberté que de la gloire de quelques citoyens, ou d’un seul.\par
Là, on regarderait les hommes de guerre comme des gens d’un métier qui peut être utile et souvent dangereux, comme des gens dont les services sont laborieux pour la nation même ; et les qualités civiles y seraient plus considérées.\par
Cette nation, que la paix et la liberté rendraient aisée, affranchie des préjugés destructeurs, serait portée à devenir commerçante. Si elle avait quelqu’une de ces marchandises primitives qui servent à faire de ces choses auxquelles la main de l’ouvrier donne un grand prix, elle pourrait faire des établissements propres à se procurer la jouissance de ce don du ciel dans toute son étendue.\par
Si cette nation était située vers le nord, et qu’elle eût un grand nombre de denrées superflues ; comme elle manquerait aussi d’un grand nombre de marchandises que son climat lui refuserait, elle ferait un commerce nécessaire, mais grand, avec les peuples du Midi : et, choisissant les États qu’elle favoriserait d’un commerce avantageux, elle ferait des traités réciproquement utiles avec la nation qu’elle aurait choisie.\par
Dans un État où, d’un côté, l’opulence serait extrême et, de l’autre, les impôts excessifs, on ne pourrait guère vivre sans industrie avec une fortune bornée. Bien des gens, sous prétexte de voyages ou de santé, s’exileraient de chez eux, et iraient chercher l’abondance dans les pays de la servitude même.\par
Une nation commerçante a un nombre prodigieux de petits intérêts particuliers ; elle peut donc choquer et être choquée d’une infinité de manières. Celle-ci deviendrait souverainement jalouse ; et elle s’affligerait plus de la prospérité des autres, qu’elle ne jouirait de la sienne.\par
Et ses lois, d’ailleurs douces et faciles, pourraient être si rigides à l’égard du commerce et de la navigation qu’on ferait chez elle, qu’elle semblerait ne négocier qu’avec des ennemis.\par
Si cette nation envoyait au loin des colonies, elle le ferait plus pour étendre son commerce que sa domination.\par
Comme on aime à établir ailleurs ce qu’on trouve établi chez soi, elle donnerait au peuple de ses colonies la forme de son gouvernement propre : et ce gouvernement portant avec lui la prospérité, on verrait se former de grands peuples dans les forêts mêmes qu’elle enverrait habiter.\par
Il pourrait être qu’elle aurait autrefois subjugué une nation voisine qui, par sa situation, la bonté de ses ports, la nature de ses richesses, lui donnerait de la jalousie : ainsi, quoiqu’elle lui eût donné ses propres lois, elle la tiendrait dans une grande dépendance ; de façon que les citoyens y seraient libres, et que l’État lui-même serait esclave.\par
L’État conquis aurait un très bon gouvernement civil, mais il serait accablé par le droit des gens ; et on lui imposerait des lois de nation à nation, qui seraient telles que sa prospérité ne serait que précaire et seulement en dépôt pour un maître.\par
La nation dominante habitant une grande île, et étant en possession d’un grand commerce, aurait toutes sortes de facilités pour avoir des forces de mer ; et comme la conservation de sa liberté demanderait qu’elle n’eût ni places, ni forteresses, ni armées de terre, elle aurait besoin d’une armée de mer qui la garantît des invasions ; et sa marine serait supérieure à celle de toutes les autres puissances, qui, ayant besoin d’employer leurs finances pour la guerre de terre, n’en auraient plus assez pour la guerre de mer.\par
L’empire de la mer a toujours donné aux peuples qui l’ont possédé une fierté naturelle ; parce que, se sentant capables d’insulter partout, ils croient que leur pouvoir n’a pas plus de bornes que l’Océan.\par
Cette nation pourrait avoir une grande influence dans les affaires de ses voisins. Car, comme elle n’emploierait pas sa puissance à conquérir, on rechercherait plus son amitié, et l’on craindrait plus sa haine que l’inconstance de son gouvernement et son agitation intérieure ne sembleraient le promettre.\par
Ainsi, ce serait le destin de la puissance exécutrice, d’être presque toujours inquiétée au-dedans, et respectée au-dehors.\par
S’il arrivait que cette nation devînt en quelques occasions le centre des négociations de l’Europe, elle y porterait un peu plus de probité et de bonne foi que les autres ; parce que ses ministres étant souvent obligés de justifier leur conduite devant un conseil populaire, leurs négociations ne pourraient être secrètes, et ils seraient forcés d’être, à cet égard, un peu plus honnêtes gens.\par
De plus, comme ils seraient en quelque façon garants des événements qu’une conduite détournée pourrait faire naître, le plus sûr pour eux serait de prendre le plus droit chemin.\par
Si les nobles avaient eu dans de certains temps un pouvoir immodéré dans la nation, et que le monarque eût trouvé le moyen de les abaisser en élevant le peuple, le point de l’extrême servitude aurait été entre le moment de l’abaissement des grands, et celui où le peuple aurait commencé à sentir son pouvoir.\par
Il pourrait être que cette nation ayant été autrefois soumise à un pouvoir arbitraire, en aurait, en plusieurs occasions, conservé le style ; de manière que, sur le fond d’un gouvernement libre, on verrait souvent la forme d’un gouvernement absolu.\par
À l’égard de la religion, comme dans cet État chaque citoyen aurait sa volonté propre, et serait par conséquent conduit par ses propres lumières, ou ses fantaisies, il arriverait, ou que chacun aurait beaucoup d’indifférence pour toutes sortes de religions de quelque espèce qu’elles fussent, moyennant quoi tout le monde serait porté à embrasser la religion dominante ; ou que l’on serait zélé pour la religion en général, moyennant quoi les sectes se multiplieraient.\par
Il ne serait pas impossible qu’il y eût dans cette nation des gens qui n’auraient point de religion, et qui ne voudraient pas cependant souffrir qu’on les obligeât à changer celle qu’ils auraient, s’ils en avaient une : car ils sentiraient d’abord que la vie et les biens ne sont pas plus à eux que leur manière de penser ; et que qui peut ravir l’un, peut encore mieux ôter l’autre.\par
Si, parmi les différentes religions, il y en avait une à l’établissement de laquelle on eût tenté de parvenir par la voie de l’esclavage, elle y serait odieuse ; parce que, comme nous jugeons des choses par les liaisons et les accessoires que nous y mettons, celle-ci ne se présenterait jamais à l’esprit avec l’idée de liberté.\par
Les lois contre ceux qui professeraient cette religion ne seraient point sanguinaires ; car la liberté n’imagine point ces sortes de peines ; mais elles seraient si réprimantes, qu’elles feraient tout le mal qui peut se faire de sang-froid.\par
Il pourrait arriver de mille manières que le clergé aurait si peu de crédit que les autres citoyens en auraient davantage. Ainsi, au lieu de se séparer, il aimerait mieux supporter les mêmes charges que les laïques, et ne faire à cet égard qu’un même corps : mais, comme il chercherait toujours à s’attirer le respect du peuple, il se distinguerait par une vie plus retirée, une conduite plus réservée, et des mœurs plus pures.\par
Ce clergé ne pouvant protéger la religion, ni être protégé par elle, sans force pour contraindre, chercherait à persuader : on verrait sortir de sa plume de très bons ouvrages, pour prouver la révélation et la providence du grand Être.\par
Il pourrait arriver qu’on éluderait ses assemblées, et qu’on ne voudrait pas lui permettre de corriger ses abus mêmes ; et que, par un délire de la liberté, on aimerait mieux laisser sa réforme imparfaite, que de souffrir qu’il fût réformateur.\par
Les dignités, faisant partie de la constitution fondamentale, seraient plus fixes qu’ailleurs ; mais, d’un autre côté, les grands, dans ce pays de liberté, s’approcheraient plus du peuple ; les rangs seraient donc plus séparés, et les personnes plus confondues.\par
Ceux qui gouvernent ayant une puissance qui se remonte, pour ainsi dire, et se refait tous les jours, auraient plus d’égard pour ceux qui leur sont utiles que pour ceux qui les divertissent : ainsi on y verrait peu de courtisans, de flatteurs, de complaisants, enfin de toutes ces sortes de gens qui font payer aux grands le vide même de leur esprit.\par
On n’y estimerait guère les hommes par des talents ou des attributs frivoles, mais par des qualités réelles ; et de ce genre il n’y en a que deux : les richesses et le mérite personnel.\par
Il y aurait un luxe solide, fondé, non pas sur le raffinement de la vanité, mais sur celui des besoins réels ; et l’on ne chercherait guère dans les choses que les plaisirs que la nature y a mis.\par
On y jouirait d’un grand superflu, et cependant les choses frivoles y seraient proscrites : ainsi plusieurs, ayant plus de bien que d’occasions de dépense, l’emploieraient d’une manière bizarre ; et dans cette nation, il y aurait plus d’esprit que de goût.\par
Comme on serait toujours occupé de ses intérêts, on n’aurait point cette politesse qui est fondée sur l’oisiveté ; et réellement on n’en aurait pas le temps.\par
L’époque de la politesse des Romains est la même que celle de l’établissement du pouvoir arbitraire. Le gouvernement absolu produit l’oisiveté ; et l’oisiveté fait naître la politesse.\par
Plus il y a de gens dans une nation qui ont besoin d’avoir des ménagements entre eux et de ne pas déplaire, plus il y a de politesse. Mais c’est plus la politesse des mœurs que celle des manières qui doit nous distinguer des peuples barbares.\par
Dans une nation où tout homme, à sa manière, prendrait part à l’administration de l’État, les femmes ne devraient guère vivre avec les hommes. Elles seraient donc modestes, c’est-à-dire timides : cette timidité ferait leur vertu ; tandis que les hommes, sans galanterie, se jetteraient dans une débauche qui leur laisserait toute leur liberté et leur loisir.\par
Les lois n’y étant pas faites pour un particulier plus que pour un autre, chacun se regarderait comme monarque ; et les hommes, dans cette nation, seraient plutôt des confédérés que des concitoyens.\par
Si le climat avait donné à bien des gens un esprit inquiet et des vues étendues, dans un pays où la constitution donnerait à tout le monde une part au gouvernement et des intérêts politiques, on parlerait beaucoup de politique ; on verrait des gens qui passeraient leur vie à calculer des événements qui, vu la nature des choses et le caprice de la fortune, c’est-à-dire des hommes, ne sont guère soumis au calcul.\par
Dans une nation libre, il est très souvent indifférent que les particuliers raisonnent bien ou mal ; il suffit qu’ils raisonnent : de là sort la liberté, qui garantit des effets de ces mêmes raisonnements.\par
De même, dans un gouvernement despotique, il est également pernicieux qu’on raisonne bien ou mal ; il suffit qu’on raisonne pour que le principe du gouvernement soit choqué.\par
Bien des gens qui ne se soucieraient de plaire à personne s’abandonneraient à leur humeur. La plupart, avec de l’esprit, seraient tourmentés par leur esprit même : dans le dédain ou le dégoût de toutes choses, ils seraient malheureux avec tant de sujets de ne l’être pas.\par
Aucun citoyen ne craignant aucun citoyen, cette nation serait fière ; car la fierté des rois n’est fondée que sur leur indépendance.\par
Les nations libres sont superbes les autres peuvent plus aisément être vaines.\par
Mais ces hommes si fiers, vivant beaucoup avec eux-mêmes, se trouveraient souvent au milieu des gens inconnus ; ils seraient timides, et l’on verrait en eux, la plupart du temps, un mélange bizarre de mauvaise honte et de fierté.\par
Le caractère de la nation paraîtrait surtout dans leurs ouvrages d’esprit, dans lesquels on verrait des gens recueillis, et qui auraient pensé tout seuls.\par
La société nous apprend à sentir les ridicules ; la retraite nous rend plus propres à sentir les vices. Leurs écrits satiriques seraient sanglants ; et l’on verrait bien des Juvénals chez eux, avant d’avoir trouvé un Horace.\par
Dans les monarchies extrêmement absolues, les historiens trahissent la vérité, parce qu’ils n’ont pas la liberté de la dire : dans les États extrêmement libres, ils trahissent la vérité à cause de leur liberté même, qui produisant toujours des divisions, chacun devient aussi esclave des préjugés de sa faction, qu’il le serait d’un despote.\par
Leurs poètes auraient plus souvent cette rudesse originale de l’invention, qu’une certaine délicatesse que donne le goût ; on y trouverait quelque chose qui approcherait plus de la force de Michel-Ange que de la grâce de Raphaël.
\section[{Quatrième partie}]{Quatrième partie}\renewcommand{\leftmark}{Quatrième partie}

\subsection[{Livre vingtième. Des lois dans le rapport qu’elles ont avec le commerce considéré dans sa nature et ses distinctions}]{Livre vingtième. Des lois dans le rapport qu’elles ont avec le commerce considéré dans sa nature et ses distinctions}
\noindent  {\itshape Docuit quae maximus Atlas.} \par
VIRGILE, Énéide.
\subsubsection[{Chapitre I. Du commerce}]{Chapitre I. {\itshape Du commerce}}
\noindent Les matières qui suivent demanderaient d’être traitées avec plus d’étendue ; mais la nature de cet ouvrage ne le permet pas. Je voudrais couler sur une rivière tranquille ; je suis entraîné par un torrent.\par
Le commerce guérit des préjugés destructeurs et c’est presque une règle générale que, partout où il y a des mœurs douces, il y a du commerce ; et que partout où il y a du commerce, il y a des mœurs douces.\par
Qu’on ne s’étonne donc point si nos mœurs sont moins féroces qu’elles ne l’étaient autrefois. Le commerce a fait que la connaissance des mœurs de toutes les nations a pénétré partout : on les a comparées entre elles, et il en a résulté de grands biens.\par
On peut dire que les lois du commerce perfectionnent les mœurs, par la même raison que ces mêmes lois perdent les mœurs. Le commerce corrompt les mœurs pures\footnote{César dit des Gaulois que le voisinage et le commerce de Marseille les avaient gâtés de façon qu’eux, qui autrefois avaient toujours vaincu les Germains, leur étaient devenus inférieurs. {\itshape Guerre des Gaules}, liv. VI.} : c’était le sujet des plaintes de Platon ; il polit et adoucit les mœurs barbares, comme nous le voyons tous les jours.
\subsubsection[{Chapitre II. De l’esprit du commerce}]{Chapitre II. De l’esprit du commerce}
\noindent L’effet naturel du commerce est de porter à la paix. Deux nations qui négocient ensemble se rendent réciproquement dépendantes : si l’une a intérêt d’acheter, l’autre a intérêt de vendre ; et toutes les unions sont fondées sur des besoins mutuels.\par
Mais, si l’esprit de commerce unit les nations, il n’unit pas de même les particuliers. Nous voyons que, dans les pays\footnote{La Hollande.} où l’on n’est affecté que de l’esprit de commerce, on trafique de toutes les actions humaines, et de toutes les vertus morales : les plus petites choses, celles que l’humanité demande, s’y font ou s’y donnent pour de l’argent.\par
L’esprit de commerce produit dans les hommes un certain sentiment de justice exacte, opposé d’un côté au brigandage, et de l’autre à ces vertus morales qui font qu’on ne discute pas toujours ses intérêts avec rigidité, et qu’on peut les négliger pour ceux des autres.\par
La privation totale du commerce produit au contraire le brigandage, qu’Aristote met au nombre des manières d’acquérir. L’esprit n’en est point opposé à de certaines vertus morales : par exemple, l’hospitalité, très rare dans les pays de commerce, se trouve admirablement parmi les peuples brigands.\par
C’est un sacrilège chez les Germains, dit Tacite, de fermer sa maison à quelque homme que ce soit, connu ou inconnu. Celui qui a exercé\footnote{{\itshape Et qui modo hospes fuerat, monstrator hospitii. De moribus Germ.},. Voyez aussi César, {\itshape Guerre des Gaules}, liv. VI.} l’hospitalité envers un étranger va lui montrer une autre maison où on l’exerce encore, et il y est reçu avec la même humanité. Mais, lorsque les Germains eurent fondé des royaumes, l’hospitalité leur devint à charge. Cela paraît par deux lois du code\footnote{Tit. XXXVIII.} des Bourguignons, dont l’une inflige une peine à tout barbare qui irait montrer à un étranger la maison d’un Romain ; et l’autre règle que celui qui recevra un étranger, sera dédommagé par les habitants, chacun pour sa quote-part.
\subsubsection[{Chapitre III. De la pauvreté des peuples}]{Chapitre III. De la pauvreté des peuples}
\noindent Il y a deux sortes de peuples pauvres : ceux que la dureté du gouvernement a rendus tels ; et ces gens-là sont incapables de presque aucune vertu, parce que leur pauvreté fait une partie de leur servitude ; les autres ne sont pauvres que parce qu’ils ont dédaigné, ou parce qu’ils n’ont pas connu les commodités de la vie ; et ceux-ci peuvent faire de grandes choses, parce que cette pauvreté fait une partie de leur liberté.
\subsubsection[{Chapitre IV. Du commerce dans les divers gouvernements}]{Chapitre IV. Du commerce dans les divers gouvernements}
\noindent Le commerce a du rapport avec la constitution. Dans le gouvernement d’un seul, il est ordinairement fondé sur le luxe ; et quoiqu’il le soit aussi sur les besoins réels, son objet principal est de procurer à la nation qui le fait, tout ce qui peut servir à son orgueil, à ses délices, et à ses fantaisies. Dans le gouvernement de plusieurs, il est plus souvent fondé sur l’économie. Les négociants, ayant l’œil sur toutes les nations de la terre, portent à l’une ce qu’ils tirent de l’autre. C’est ainsi que les républiques de Tyr, de Carthage, d’Athènes, de Marseille, de Florence, de Venise et de Hollande ont fait le commerce.\par
Cette espèce de trafic regarde le gouvernement de plusieurs par sa nature, et le monarchique par occasion. Car, comme il n’est fondé que sur la pratique de gagner peu, et même de gagner moins qu’aucune autre nation, et de ne se dédommager qu’en gagnant continuellement, il n’est guère possible qu’il puisse être fait par un peuple chez qui le luxe est établi, qui dépense beaucoup, et qui ne voit que de grands objets.\par
C’est dans ces idées que Cicéron\footnote{{\itshape Nolo eumdem populum, imperatorem et portitorem esse terrarum.}} disait si bien : « Je n’aime point qu’un même peuple soit en même temps le dominateur et le facteur de l’univers. » En effet, il faudrait supposer que chaque particulier dans cet État, et tout l’État même, eussent toujours la tête pleine de grands projets, et cette même tête remplie de petits : ce qui est contradictoire.\par
Ce n’est pas que, dans ces États qui subsistent par le commerce d’économie, on ne fasse aussi les plus grandes entreprises, et que l’on n’y ait une hardiesse qui ne se trouve pas dans les monarchies : en voici la raison.\par
Un commerce mène à l’autre, le petit au médiocre, le médiocre au grand ; et celui qui a eu tant d’envie de gagner peu se met dans une situation où il n’en a pas moins de gagner beaucoup.\par
De plus, les grandes entreprises des négociants sont toujours nécessairement mêlées avec les affaires publiques. Mais, dans les monarchies, les affaires publiques sont, la plupart du temps, aussi suspectes aux marchands qu’elles leur paraissent sûres dans les États républicains. Les grandes entreprises de commerce ne sont donc pas pour les monarchies, mais pour le gouvernement de plusieurs.\par
En un mot, une plus grande certitude de sa propriété, que l’on croit avoir dans ces États, fait tout entreprendre ; et, parce qu’on croit être sûr de ce que l’on a acquis, on ose l’exposer pour acquérir davantage ; on ne court de risque que sur les moyens d’acquérir : or, les hommes espèrent beaucoup de leur fortune.\par
Je ne veux pas dire qu’il y ait aucune monarchie qui soit totalement exclue du commerce d’économie ; mais elle y est moins portée par sa nature. Je ne veux pas dire que les républiques que nous connaissons soient entièrement privées du commerce de luxe ; mais il a moins de rapport à leur constitution.\par
Quant à l’État despotique, il est inutile d’en parler. Règle générale : dans une nation qui est dans la servitude, on travaille plus à conserver qu’à acquérir ; dans une nation libre, on travaille plus à acquérir qu’à conserver.
\subsubsection[{Chapitre V. Des peuples qui ont fait le commerce d’économie}]{Chapitre V. Des peuples qui ont fait le commerce d’économie}
\noindent Marseille, retraite nécessaire au milieu d’une mer orageuse ; Marseille, ce lieu où les vents, les bancs de la mer, la disposition des côtes ordonnent de toucher, fut fréquentée par les gens de mer. La stérilité\footnote{Justin, liv. XLIII, chap. III.} de son territoire détermina ses citoyens au commerce d’économie. il fallut qu’ils fussent laborieux, pour suppléer à la nature qui se refusait ; qu’ils fussent justes, pour vivre parmi les nations barbares qui devaient faire leur prospérité ; qu’ils fussent modérés, pour que leur gouvernement fût toujours tranquille ; enfin qu’ils eussent des mœurs frugales, pour qu’ils pussent toujours vivre d’un commerce qu’ils conserveraient plus sûrement lorsqu’il serait moins avantageux.\par
On a vu partout la violence et la vexation donner naissance au commerce d’économie, lorsque les hommes sont contraints de se réfugier dans les marais, dans les îles, les bas-fonds de la mer, et ses écueils même. C’est ainsi que Tyr, Venise, et les villes de Hollande furent fondées ; les fugitifs y trouvèrent leur sûreté. Il fallut subsister ; ils tirèrent leur subsistance de tout l’univers.
\subsubsection[{Chapitre VI. Quelques effets d’une grande navigation}]{Chapitre VI. Quelques effets d’une grande navigation}
\noindent Il arrive quelquefois qu’une nation qui fait le commerce d’économie, ayant besoin d’une marchandise d’un pays qui lui serve de fonds pour se procurer les marchandises d’un autre, se contente de gagner très peu, et quelquefois rien, sur les unes, dans l’espérance ou la certitude de gagner beaucoup sur les autres. Ainsi, lorsque la Hollande faisait presque seule le commerce du midi au nord de l’Europe, les vins de France, qu’elle portait au nord, ne lui servaient, en quelque manière, que de fonds pour faire son commerce dans le nord.\par
On sait que souvent, en Hollande, de certains genres de marchandise venue de loin ne s’y vendent pas plus cher qu’ils n’ont coûté sur les lieux mêmes. Voici la raison qu’on en donne : un capitaine qui a besoin de lester son vaisseau prendra du marbre ; il a besoin de bois pour l’arrimage, il en achètera : et pourvu qu’il n’y perde rien, il croira avoir beaucoup fait. C’est ainsi que la Hollande a aussi ses carrières et ses forêts.\par
Non seulement un commerce qui ne donne rien peut être utile, un commerce même désavantageux peut l’être. J’ai ouï dire en Hollande que la pêche de la baleine, en général, ne rend presque jamais ce qu’elle coûte : mais ceux qui ont été employés à la construction du vaisseau, ceux qui ont fourni les agrès, les apparaux, les vivres, sont aussi ceux qui prennent le principal intérêt à cette pêche. Perdissent-ils sur la pêche, ils ont gagné sur les fournitures. Ce commerce est une espèce de loterie, et chacun est séduit par l’espérance d’un billet noir. Tout le monde aime à jouer ; et les gens les plus sages jouent volontiers, lorsqu’ils ne voient point les apparences du jeu, ses égarements, ses violences, ses dissipations, la perte du temps, et même de toute la vie.
\subsubsection[{Chapitre VII. Esprit de l’Angleterre sur le commerce}]{Chapitre VII. Esprit de l’Angleterre sur le commerce}
\noindent L’Angleterre n’a guère de tarif réglé avec les autres nations ; son tarif change, pour ainsi dire, à chaque parlement, par les droits particuliers qu’elle ôte, ou qu’elle impose. Elle a voulu encore conserver sur cela son indépendance. Souverainement jalouse du commerce qu’on fait chez elle, elle se lie peu par des traités, et ne dépend que de ses lois.\par
D’autres nations ont fait céder des intérêts du commerce à des intérêts politiques : celle-ci a toujours fait céder ses intérêts politiques aux intérêts de son commerce.\par
C’est le peuple du monde qui a le mieux su se prévaloir à la fois de ces trois grandes choses : la religion, le commerce et la liberté.
\subsubsection[{Chapitre VIII. Comment on a gêné quelquefois le commerce d’économie}]{Chapitre VIII. Comment on a gêné quelquefois le commerce d’économie}
\noindent On a fait, dans de certaines monarchies, des lois très propres à abaisser les États qui font le commerce d’économie. On leur a défendu d’apporter d’autres marchandises que celles du cru de leur pays : on ne leur a permis de venir trafiquer qu’avec des navires de la fabrique du pays où ils viennent.\par
Il faut que l’État qui impose ces lois puisse aisément faire lui-même le commerce : sans cela, il se fera pour le moins un tort égal. Il vaut mieux avoir affaire à une nation qui exige peu, et que les besoins du commerce rendent en quelque façon dépendante ; à une nation qui, par l’étendue de ses vues ou de ses affaires, sait où placer toutes les marchandises superflues ; qui est riche, et peut se charger de beaucoup de denrées ; qui les paiera promptement ; qui a, pour ainsi dire, des nécessités d’être fidèle ; qui est pacifique par principe, qui cherche à gagner, et non pas à conquérir : il vaut mieux, dis-je, avoir affaire à cette nation qu’à d’autres toujours rivales, et qui ne donneraient pas tous ces avantages.
\subsubsection[{Chapitre IX. De l’exclusion en fait de commerce}]{Chapitre IX. De l’exclusion en fait de commerce}
\noindent La vraie maxime est de n’exclure aucune nation de son commerce sans de grandes raisons. Les Japonais ne commercent qu’avec deux nations, la chinoise et la hollandaise. Les Chinois\footnote{Le P. Du Halde, t. II, p. 170.} gagnent mille pour cent sur le sucre, et quelquefois autant sur les retours. Les Hollandais font des profits à peu près pareils. Toute nation qui se conduira sur les maximes japonaises sera nécessairement trompée. C’est la concurrence qui met un prix juste aux marchandises, et qui établit les vrais rapports entre elles.\par
Encore moins un État doit-il s’assujettir à ne vendre ses marchandises qu’à une seule nation, sous prétexte qu’elle les prendra toutes à un certain prix. Les Polonais ont fait pour leur blé ce marché avec la ville de Dantzig ; plusieurs rois des Indes ont de pareils contrats pour les épiceries avec les Hollandais\footnote{Cela fut premièrement établi par les Portugais. Voyages {\itshape de François Pyrard}, chap. XV, part. II.}. Ces conventions ne sont propres qu’à une nation pauvre, qui veut bien perdre l’espérance de s’enrichir, pourvu qu’elle ait une subsistance assurée ; ou à des nations dont la servitude consiste à renoncer à l’usage des choses que la nature leur avait données, ou à faire sur ces choses un commerce désavantageux.
\subsubsection[{Chapitre X. Établissement propre au commerce d’économie}]{Chapitre X. Établissement propre au commerce d’économie}
\noindent Dans les États qui font le commerce d’économie, on a heureusement établi des banques qui, par leur crédit, ont formé de nouveaux signes des valeurs. Mais on aurait tort de les transporter dans les États qui font le commerce de luxe. Les mettre dans les pays gouvernés par un seul, c’est supposer l’argent d’un côté, et de l’autre la puissance : c’est-à-dire, d’un côté, la faculté de tout avoir sans aucun pouvoir ; et de l’autre, le pouvoir avec la faculté de rien du tout. Dans un gouvernement pareil, il n’y a jamais eu que le prince qui ait eu, ou qui ait pu avoir un trésor ; et partout où il y en a un, dès qu’il est excessif, il devient d’abord le trésor du prince.\par
Par la même raison, les compagnies de négociants, qui s’associent pour un certain commerce, conviennent rarement au gouvernement d’un seul. La nature de ces compagnies est de donner aux richesses particulières la force des richesses publiques. Mais, dans ces États, cette force ne peut se trouver que dans les mains du prince. Je dis plus : elles ne conviennent pas toujours dans les États où l’on fait le commerce d’économie ; et, si les affaires ne sont si grandes qu’elles soient au-dessus de la portée des particuliers, on fera encore mieux de ne point gêner, par des privilèges exclusifs, la liberté du commerce.
\subsubsection[{Chapitre XI. Continuation du même sujet}]{Chapitre XI. Continuation du même sujet}
\noindent Dans les États qui font le commerce d’économie, on peut établir un port franc. L’économie de l’État, qui suit toujours la frugalité des particuliers, donne, pour ainsi dire, l’âme à son commerce d’économie. Ce qu’il perd de tributs par l’établissement dont nous parlons est compensé par ce qu’il peut tirer de la richesse industrieuse de la république. Mais, dans le gouvernement monarchique, de pareils établissements seraient contre la raison ; ils n’auraient d’autre effet que de soulager le luxe du poids des impôts. On se priverait de l’unique bien que ce luxe peut procurer, et du seul frein que, dans une constitution pareille, il puisse recevoir.
\subsubsection[{Chapitre XII. De la liberté du commerce}]{Chapitre XII. De la liberté du commerce}
\noindent La liberté du commerce n’est pas une faculté accordée aux négociants de faire ce qu’ils veulent ; ce serait bien plutôt sa servitude. Ce qui gêne le commerçant ne gêne pas pour cela le commerce. C’est dans les pays de la liberté que le négociant trouve des contradictions sans nombre ; et il n’est jamais moins croisé par les lois que dans les pays de la servitude.\par
L’Angleterre défend de faire sortir ses laines ; elle veut que le charbon soit transporté par mer dans la capitale ; elle ne permet point la sortie de ses chevaux, s’ils ne sont coupés ; les vaisseaux\footnote{{\itshape Acte de navigation} de 1660. Ce n’a été qu’en temps de guerre que ceux de Boston et de Philadelphie ont envoyé leurs vaisseaux en droiture jusque dans la Méditerranée porter leurs denrées.} de ses colonies qui commercent en Europe, doivent mouiller en Angleterre. Elle gêne le négociant, mais c’est en faveur du commerce.
\subsubsection[{Chapitre XIII. Ce qui détruit cette liberté}]{Chapitre XIII. Ce qui détruit cette liberté}
\noindent Là où il y a du commerce, il y a des douanes. L’objet du commerce est l’exportation et l’importation des marchandises en faveur de l’État ; et l’objet des douanes est un certain droit sur cette même exportation et importation, aussi en faveur de l’État. Il faut donc que l’État soit neutre entre sa douane et son commerce, et qu’il fasse en sorte que ces deux choses ne se croisent point ; et alors on y jouit de la liberté du commerce.\par
La finance détruit le commerce par ses injustices, par ses vexations, par l’excès de ce qu’elle impose : mais elle le détruit encore, indépendamment de cela, par les difficultés qu’elle fait naître, et les formalités qu’elle exige. En Angleterre, où les douanes sont en régie, il y a une facilité de négocier singulière : un mot d’écriture fait les plus grandes affaires ; il ne faut point que le marchand perde un temps infini et qu’il ait des commis exprès, pour faire cesser toutes les difficultés des fermiers, ou pour s’y soumettre.
\subsubsection[{Chapitre XIV. Des lois de commerce qui emportent la confiscation des marchandises}]{Chapitre XIV. Des lois de commerce qui emportent la confiscation des marchandises}
\noindent La grande charte des Anglais défend de saisir et de confisquer, en cas de guerre, les marchandises des négociants étrangers, à moins que ce ne soit par représailles. il est beau que la nation anglaise ait fait de cela un des articles de sa liberté.\par
Dans la guerre que l’Espagne eut avec les Anglais en 1740, elle fit une loi\footnote{Publiée à Cadix au mois de mars 1740.} qui punissait de mort ceux qui introduiraient dans les États d’Espagne des marchandises d’Angleterre ; elle infligeait la même peine à ceux qui porteraient dans les États d’Angleterre des marchandises d’Espagne. Une ordonnance pareille ne peut, je crois, trouver de modèle que dans les lois du Japon. Elle choque nos mœurs, l’esprit du commerce et l’harmonie qui doit être dans la proportion des peines ; elle confond toutes les idées, faisant un crime d’État de ce qui n’est qu’une violation de police.
\subsubsection[{Chapitre XV. De la contrainte par corps}]{Chapitre XV. De la contrainte par corps}
\noindent Solon\footnote{Plutarque, au traité Qu’il ne faut point emprunter à usure.} ordonna à Athènes qu’on n’obligerait plus le corps pour dettes civiles. Il tira cette loi d’Égypte\footnote{Diodore, liv. I, part. II, chap. III.} ; Bocchoris l’avait faite, et Sésostris l’avait renouvelée.\par
Cette loi est très bonne pour les affaires\footnote{Les législateurs grecs étaient blâmables, qui avaient défendu de prendre en gage les armes et la charrue d’un homme, et permettaient de prendre l’homme même. Diodore, liv. I, part. II, chap. III.} civiles ordinaires ; mais nous avons raison de ne point l’observer dans celles du commerce. Car les négociants étant obligés de confier de grandes sommes pour des temps souvent fort courts, de les donner et de les reprendre, il faut que le débiteur remplisse toujours au temps fixé ses engagements : ce qui suppose la contrainte par corps.\par
Dans les affaires qui dérivent des contrats civils ordinaires, la loi ne doit point donner la contrainte par corps, parce qu’elle fait plus de cas de la liberté d’un citoyen que de l’aisance d’un autre. Mais, dans les conventions qui dérivent du commerce, la loi doit faire plus de cas de l’aisance publique que de la liberté d’un citoyen ; ce qui n’empêche pas les restrictions et les limitations que peuvent demander l’humanité et la bonne police.
\subsubsection[{Chapitre XVI. Belle loi}]{Chapitre XVI. {\itshape Belle loi}}
\noindent La loi de Genève qui exclut des magistratures, et même de l’entrée dans le Grand Conseil, les enfants de ceux qui ont vécu ou qui sont morts insolvables, à moins qu’ils n’acquittent les dettes de leur père, est très bonne. Elle a cet effet, qu’elle donne de la confiance pour les négociants ; elle en donne pour les magistrats ; elle en donne pour la cité même. La foi particulière y a encore la force de la foi publique.
\subsubsection[{Chapitre XVII. Loi de Rhodes}]{Chapitre XVII. {\itshape Loi de Rhodes}}
\noindent Les Rhodiens allèrent plus loin. Sextus Empiricus\footnote{{\itshape Hypotyposes}, liv.{\itshape  I}, chap. XIV.} dit que, chez eux, un fils ne pouvait se dispenser de payer les dettes de son père, en renonçant à sa succession. La loi de Rhodes était donnée à une république fondée sur le commerce : or je crois que la raison du commerce même y devait mettre cette limitation, que les dettes contractées par le père depuis que le fils avait commencé à faire le commerce n’affecteraient point les biens acquis par celui-ci. Un négociant doit toujours connaître ses obligations, et se conduire à chaque instant suivant l’état de sa fortune.
\subsubsection[{Chapitre XVIII. Des juges pour le commerce}]{Chapitre XVIII. Des juges pour le commerce}
\noindent Xénophon, au livre des {\itshape Revenus}, voudrait qu’on donnât des récompenses à ceux des préfets du commerce qui expédient le plus vite les procès. Il sentait le besoin de notre juridiction consulaire.\par
Les affaires du commerce sont très peu susceptibles de formalités. Ce sont des actions de chaque jour, que d’autres de même nature doivent suivre chaque jour. Il faut donc qu’elles puissent être décidées chaque jour. Il en est autrement des actions de la vie qui influent beaucoup sur l’avenir, mais qui arrivent rarement. On ne se marie guère qu’une fois ; on ne fait pas tous les jours des donations ou des testaments ; on n’est majeur qu’une fois.\par
Platon\footnote{{\itshape Des Lois}, liv. VIII.} dit que dans une ville où il n’y a point de commerce maritime, il faut la moitié moins de lois civiles ; et cela est très vrai. Le commerce introduit dans le même pays différentes sortes de peuples, un grand nombre de conventions, d’espèces de biens et de manières d’acquérir.\par
Ainsi, dans une ville commerçante, il y a moins de juges, et plus de lois.
\subsubsection[{Chapitre XIX. Que le prince ne doit point faire de commerce}]{Chapitre XIX. Que le prince ne doit point faire de commerce}
\noindent Théophile\footnote{Zonare.} voyant un vaisseau où il y avait des marchandises pour sa femme Théodora, le fit brûler. « Je suis empereur, lui dit-il, et vous me faites patron de galère. En quoi les pauvres gens pourront-ils gagner leur vie, si nous faisons encore leur métier ? » Il aurait pu ajouter : Qui pourra nous réprimer, si nous faisons des monopoles ? Qui nous obligera de remplir nos engagements ? Ce commerce que nous faisons, les courtisans voudront le faire ; ils seront plus avides et plus injustes que nous. Le peuple a de la confiance en notre justice ; il n’en a point en notre opulence : tant d’impôts qui font sa misère sont des preuves certaines de la nôtre.
\subsubsection[{Chapitre XX. Continuation du même sujet}]{Chapitre XX. Continuation du même sujet}
\noindent Lorsque les Portugais et les Castillans dominaient dans les Indes orientales, le commerce avait des branches si riches, que leurs princes ne manquèrent pas de s’en saisir. Cela ruina leurs établissements dans ces parties-là.\par
Le vice-roi de Goa accordait à des particuliers des privilèges exclusifs. On n’a point de confiance en de pareilles gens ; le commerce est discontinué par le changement perpétuel de ceux à qui on le confie ; personne ne ménage ce commerce, et ne se soucie de le laisser perdu à son successeur ; le profit reste dans des mains particulières, et ne s’étend pas assez.
\subsubsection[{Chapitre XXI. Du commerce de la noblesse dans la monarchie}]{Chapitre XXI. Du commerce de la noblesse dans la monarchie}
\noindent Il est contre l’esprit du commerce que la noblesse le fasse dans la monarchie. « Cela serait pernicieux aux villes, disent\footnote{{\itshape Leg. nobiliores}, cod. {\itshape De commerc.}} les empereurs Honorius et Théodose, et ôterait entre les marchands et les plébéiens la facilité d’acheter et de vendre. »\par
Il est contre l’esprit de la monarchie que la noblesse y fasse le commerce. L’usage qui a permis en Angleterre le commerce à la noblesse est une des choses qui a le plus contribué à y affaiblir le gouvernement monarchique.
\subsubsection[{Chapitre XXII. Réflexion particulière}]{Chapitre XXII. Réflexion particulière}
\noindent Des gens, frappés de ce qui se pratique dans quelques États, pensent qu’il faudrait qu’en France il y eût des lois qui engageassent les nobles à faire le commerce. Ce serait le moyen d’y détruire la noblesse, sans aucune utilité pour le commerce. La pratique de ce pays est très sage : les négociants n’y sont pas nobles, mais ils peuvent le devenir. Ils ont l’espérance d’obtenir la noblesse, sans en avoir l’inconvénient actuel. Ils n’ont pas de moyen plus sûr de sortir de leur profession que de la bien faire, ou de la faire avec bonheur ; chose qui est ordinairement attachée à la suffisance.\par
Les lois qui ordonnent que chacun reste dans sa profession, et la fasse passer à ses enfants, ne sont et ne peuvent être utiles que dans les États\footnote{Effectivement cela y est souvent ainsi établi.} despotiques, où personne ne peut ni ne doit avoir d’émulation.\par
Qu’on ne dise pas que chacun fera mieux sa profession lorsqu’on ne pourra pas la quitter pour une autre. Je dis qu’on fera mieux sa profession, lorsque ceux qui y auront excellé espéreront de parvenir à une autre.\par
L’acquisition qu’on peut faire de la noblesse à prix d’argent encourage beaucoup les négociants à se mettre en état d’y parvenir. Je n’examine pas si l’on fait bien de donner ainsi aux richesses le prix de la vertu : il y a tel gouvernement où cela peut être très utile.\par
En France, cet état de la robe qui se trouve entre la grande noblesse et le peuple ; qui, sans avoir le brillant de celle-là, en a tous les privilèges ; cet état qui laisse les particuliers dans la médiocrité, tandis que le corps dépositaire des lois est dans la gloire ; cet état encore dans lequel on n’a de moyen de se distinguer que par la suffisance et par la vertu ; profession honorable, mais qui en laisse toujours voir une plus distinguée : cette noblesse toute guerrière, qui pense qu’en quelque degré de richesses que l’on soit, il faut faire sa fortune mais qu’il est honteux d’augmenter son bien, si on ne commence par le dissiper ; cette partie de la nation, qui sert toujours avec le capital de son bien ; qui, quand elle est ruinée, donne sa place à une autre qui servira avec son capital encore ; qui va à la guerre pour que personne n’ose dire qu’elle n’y a pas été ; qui, quand elle ne peut espérer les richesses, espère les honneurs ; et lorsqu’elle ne les obtient pas, se console, parce qu’elle a acquis de l’honneur : toutes ces choses ont nécessairement contribué à la grandeur de ce royaume. Et si, depuis deux ou trois siècles, il a augmenté sans cesse sa puissance, il faut attribuer cela à la bonté de ses lois, non pas à la fortune, qui n’a pas ces sortes de constance.
\subsubsection[{Chapitre XXIII. À quelles nations il est désavantageux de faire le commerce}]{Chapitre XXIII. À quelles nations il est désavantageux de faire le commerce}
\noindent Les richesses consistent en fonds de terre ou en effets mobiliers : les fonds de terre de chaque pays sont ordinairement possédés par ses habitants. La plupart des États ont des lois qui dégoûtent les étrangers de l’acquisition de leurs terres ; il n’y a même que la présence du maître qui les fasse valoir : ce genre de richesses appartient donc à chaque État en particulier. Mais les effets mobiliers, comme l’argent, les billets, les lettres de change, les actions sur les compagnies, les vaisseaux, toutes les marchandises, appartiennent au monde entier, qui, dans ce rapport, ne compose qu’un seul État, dont toutes les sociétés sont les membres : le peuple qui possède le plus de ces effets mobiliers de l’univers, est le plus riche. Quelques États en ont une immense quantité : ils les acquièrent chacun par leurs denrées, par le travail de leurs ouvriers, par leur industrie, par leurs découvertes, par le hasard même. L’avarice des nations se dispute les meubles" de tout l’univers. Il peut se trouver un État si malheureux qu’il sera privé des effets des autres pays, et même encore de presque tous les siens : les propriétaires des fonds de terre n’y seront que les colons des étrangers. Cet État manquera de tout, et ne pourra rien acquérir ; il vaudrait bien mieux qu’il n’eût de commerce avec aucune nation du monde : c’est le commerce qui, dans les circonstances où il se trouvait, l’a conduit à la pauvreté.\par
Un pays qui envoie toujours moins de marchandises ou de denrées qu’il n’en reçoit, se met lui-même en équilibre en s’appauvrissant : il recevra toujours moins, jusqu’à ce que, dans une pauvreté extrême, il ne reçoive plus rien.\par
Dans les pays de commerce, l’argent qui s’est tout à coup évanoui, revient, parce que les États qui l’ont reçu le doivent : dans les États dont nous parlons, l’argent ne revient jamais, parce que ceux qui l’ont pris ne doivent rien.\par
La Pologne servira ici d’exemple. Elle n’a presque aucune des choses que nous appelons les effets mobiliers de l’univers, si ce n’est le blé de ses terres. Quelques seigneurs possèdent des provinces entières ; ils pressent le laboureur pour avoir une plus grande quantité de blé qu’ils puissent envoyer aux étrangers, et se procurer les choses que demande leur luxe. Si la Pologne ne commerçait avec aucune nation, ses peuples seraient plus heureux. Ses grands, qui n’auraient que leur blé, le donneraient à leurs paysans pour vivre ; de trop grands domaines leur seraient à charge, ils les partageraient à leurs paysans ; tout le monde trouvant des peaux ou des laines dans ses troupeaux, il n’y aurait plus une dépense immense à faire pour les habits ; les grands, qui aiment toujours le luxe, et qui ne le pourraient trouver que dans leur pays, encourageraient les pauvres au travail. Je dis que cette nation serait plus florissante, à moins qu’elle ne devînt barbare : chose que les lois pourraient prévenir.\par
Considérons à présent le Japon. La quantité excessive de ce qu’il peut recevoir produit la quantité excessive de ce qu’il peut envoyer : les choses seront en équilibre, comme si l’importation et l’exportation étaient modérées ; et d’ailleurs cette espèce d’enflure produira à l’État mille avantages : il y aura plus de consommation, plus de choses sur lesquelles les arts peuvent s’exercer, plus d’hommes employés, plus de moyens d’acquérir de la puissance. Il peut arriver des cas où l’on ait besoin d’un secours prompt, qu’un État si plein peut donner plus tôt qu’un autre. Il est difficile qu’un pays n’ait des choses superflues ; mais c’est la nature du commerce de rendre les choses superflues utiles, et les utiles nécessaires. L’État pourra donc donner les choses nécessaires à un plus grand nombre de sujets.\par
Disons donc que ce ne sont point les nations qui n’ont besoin de rien, qui perdent à faire le {\itshape commerce} ; {\itshape ce} sont {\itshape celles} qui ont besoin de tout, Ce ne sont point les peuples qui se suffisent à eux-mêmes, mais ceux qui n’ont rien chez eux, qui trouvent de l’avantage à ne trafiquer avec personne.
\subsection[{Livre vingt et unième. Des lois dans le rapport qu’elles ont avec le commerce, considéré dans les révolutions qu’il a eues dans le monde}]{Livre vingt et unième. Des lois dans le rapport qu’elles ont avec le commerce, considéré dans les révolutions qu’il a eues dans le monde}
\subsubsection[{Chapitre I. Quelques considérations générales}]{Chapitre I. Quelques considérations générales}
\noindent Quoique le commerce soit sujet à de grandes révolutions, il peut arriver que de certaines causes physiques, la qualité du terrain ou du climat, fixent pour jamais sa nature.\par
Nous ne faisons aujourd’hui le commerce des Indes que par l’argent que nous y envoyons. Les Romains\footnote{Pline, liv. VI, chap. XXIII.} y portaient toutes les années environ cinquante millions de sesterces. Cet argent, comme le nôtre aujourd’hui, était converti en marchandises qu’ils rapportaient en Occident. Tous les peuples qui ont négocié aux Indes y ont toujours porté des métaux, et en ont rapporté des marchandises.\par
C’est la nature même qui produit cet effet. Les Indiens ont leurs arts, qui sont adaptés à leur manière de vivre. Notre luxe ne saurait être le leur, ni nos besoins être leurs besoins. Leur climat ne leur demande ni ne leur permet presque rien de ce qui vient de chez nous. Ils vont en grande partie nus ; les vêtements qu’ils ont, le pays les leur fournit convenables ; et leur religion, qui a sur eux tant d’empire, leur donne de la répugnance pour les choses qui nous servent de nourriture. Ils n’ont donc besoin que de nos métaux, qui sont les signes des valeurs, et pour lesquels ils donnent des marchandises, que leur frugalité et la nature de leur pays leur procurent en grande abondance. Les auteurs anciens qui nous ont parlé des Indes, nous les dépeignent\footnote{Voyez Pline, liv. VI, chap. XIX, et Strabon, liv. XV.} telles que nous les voyons aujourd’hui, quant à la police, aux manières et aux mœurs. Les Indes ont été, les Indes seront ce qu’elles sont à présent ; et, dans tous les temps, ceux qui négocieront aux Indes y porteront de l’argent, et n’en rapporteront pas.
\subsubsection[{Chapitre II. Des peuples d’Afrique}]{Chapitre II. Des peuples d’Afrique}
\noindent La plupart des peuples des côtes de l’Afrique sont sauvages ou barbares. Je crois que cela vient beaucoup de ce que des pays presque inhabitables séparent de petits pays qui peuvent être habités. Ils sont sans industrie {\itshape ;} ils n’ont point d’arts ; ils ont en abondance des métaux précieux qu’ils tiennent immédiatement des mains de la nature. Tous les peuples policés sont donc en état de négocier avec eux avec avantage ; ils peuvent leur faire estimer beaucoup des choses de nulle valeur, et en recevoir un très grand prix.
\subsubsection[{Chapitre III. Que les besoins des peuples du midi sont différents de ceux des peuples du nord}]{Chapitre III. Que les besoins des peuples du midi sont différents de ceux des peuples du nord}
\noindent Il y a dans l’Europe une espèce de balancement entre les nations du midi et celles du nord. Les premières ont toutes sortes de commodités pour la vie, et peu de besoins ; les secondes ont beaucoup de besoins, et peu de commodités pour la vie. Aux unes, la nature a donné beaucoup, et elles ne lui demandent que peu ; aux autres, la nature donne peu, et elles lui demandent beaucoup. L’équilibre se maintient par la paresse qu’elle a donnée aux nations du midi, et par l’industrie et l’activité qu’elle a données à celles du nord. Ces dernières sont obligées de travailler beaucoup, sans quoi elles manqueraient de tout, et deviendraient barbares. C’est ce qui a naturalisé la servitude chez les peuples du midi : comme ils peuvent aisément se passer de richesses, ils peuvent encore mieux se passer de liberté. Mais les peuples du nord ont besoin de la liberté, qui leur procure plus de moyens de satisfaire tous les besoins que la nature leur a donnés. Les peuples du nord sont donc dans un état forcé, s’ils ne sont libres ou barbares : presque tous les peuples du midi sont, en quelque façon, dans un état violent, s’ils ne sont esclaves.
\subsubsection[{Chapitre IV. Principale différence du commerce des anciens d’avec celui d’aujourd’hui}]{Chapitre IV. Principale différence du commerce des anciens d’avec celui d’aujourd’hui}
\noindent Le monde se met de temps en temps dans des situations qui changent le commerce. Aujourd’hui le commerce de l’Europe se fait principalement du nord au midi. Pour lors la différence des climats fait que les peuples ont un grand besoin des marchandises les uns des autres. Par exemple, les boissons du midi poilées au nord forment une espèce de commerce que les anciens n’avaient guère. Aussi la capacité des vaisseaux, qui se mesurait autrefois par muids de blé, se mesure-t-elle aujourd’hui par tonneaux de liqueurs.\par
Le commerce ancien que nous connaissons, se faisant d’un port de la Méditerranée à l’autre, était presque tout dans le midi. Or, les peuples du même climat ayant chez eux à peu près les mêmes choses, n’ont pas tant de besoin de commercer entre eux que ceux d’un climat différent. Le commerce en Europe était donc autrefois moins étendu qu’il ne l’est à présent.\par
Ceci n’est point contradictoire avec ce que j’ai dit de notre commerce des Indes : la différence excessive du climat fait que les besoins relatifs sont nuls.
\subsubsection[{Chapitre V. Autres différences}]{Chapitre V. Autres différences}
\noindent Le commerce, tantôt détruit par les conquérants, tantôt gêné par les monarques, parcourt la terre, fuit d’où il est opprimé, se repose où on le laisse respirer : il règne aujourd’hui où l’on ne voyait que des déserts, des mers et des rochers ; là où il régnait, il n’y a que des déserts.\par
À voir aujourd’hui la Colchide, qui n’est plus qu’une vaste forêt, où le peuple, qui diminue tous les jours, ne défend sa liberté que pour se vendre en détail aux Turcs et aux Persans, on ne dirait jamais que cette contrée eût été, du temps des Romains, pleine de villes où le commerce appelait toutes les nations du monde. On n’en trouve aucun monument dans le pays ; il n’y en a de traces que dans Pline\footnote{Liv. VI.} et Strabon\footnote{Liv. XI.}.\par
L’histoire du commerce est celle de la communication des peuples. Leurs destructions diverses, et de certains flux et reflux de populations et de dévastations, en forment les plus grands événements.
\subsubsection[{Chapitre VI. Du commerce des anciens}]{Chapitre VI. Du commerce des anciens}
\noindent Les trésors immenses de\footnote{Diodore, liv. II.} Sémiramis, qui ne pouvaient avoir été acquis en un jour, nous font penser que les Assyriens avaient eux-mêmes pillé d’autres nations riches, comme les autres nations les pillèrent après.\par
L’effet du commerce sont les richesses ; la suite des richesses, le luxe ; celle du luxe, la perfection des arts. Les arts, portés au point où on les trouve du temps de Sémiramis\footnote{Diodore, liv. II.}, nous marquent un grand commerce déjà établi.\par
Il y avait un grand commerce de luxe dans les empires d’Asie. Ce serait une belle partie de l’histoire du commerce que l’histoire du luxe ; le luxe des Perses était celui des Mèdes, comme celui des Mèdes était celui des Assyriens.\par
Il est arrivé de grands changements en Asie. La partie de la Perse qui est au nord-est, l’Hyrcanie, la Margiane, la Bactriane, etc., étaient autrefois pleines de villes florissantes\footnote{Voyez Pline, liv. VI, chap. XVI ; et Strabon, liv. XI.} qui ne sont plus ; et le nord\footnote{Strabon, liv. XI.} de cet empire, c’est-à-dire l’isthme qui sépare la mer Caspienne du Pont-Euxin, était couvert de villes et de nations qui ne sont plus encore.\par
Ératosthène\footnote{Strabon, liv. XI.} et Aristobule tenaient de Patrocle\footnote{L’autorité de Patrocle est considérable, comme il paraît par un récit de Strabon, liv. II.} que les marchandises des Indes passaient par l’Oxus dans la mer du Pont. Marc Varron\footnote{Dans Pline, liv. VI, chap. XVII. Voyez aussi Strabon, liv. XI, sur le trajet des marchandises du Phase au Cyrus.} nous dit que l’on apprit, du temps de Pompée dans la guerre contre Mithridate, que l’on allait en sept jours de l’Inde dans le pays des Bactriens, et au fleuve Icarus qui se jette dans l’Oxus ; que par là les marchandises de l’Inde pouvaient traverser la mer Caspienne, entrer de là dans l’embouchure du Cyrus ; que de ce fleuve il ne fallait qu’un trajet par terre de cinq jours pour aller au Phase, qui conduisait dans le Pont-Euxin. C’est sans doute par les nations qui peuplaient ces divers pays, que les grands empires des Assyriens, des Mèdes et des Perses, avaient une communication avec les parties de l’Orient et de l’occident les plus reculées.\par
Cette communication n’est plus. Tous ces pays ont été dévastés par les Tartares\footnote{Il faut que, depuis le temps de Ptolomée, qui nous décrit tant de rivières qui se jettent dans la partie orientale de la mer Caspienne, il y ait eu de grands changements dans ce pays. La carte du czar ne met de ce côté-là que la rivière d’Astrabat ; et celle de M. Bathalsi, rien du tout.}, et cette nation destructrice les habite encore pour les infester. L’Oxus ne va plus à la mer Caspienne : les Tartares l’ont détourné pour des raisons particulières\footnote{Voyez la relation de Genkinson, dans le {\itshape Recueil des voyages du} Nord, t. IV.} ; il se perd dans des sables arides.\par
Le Jaxarte, qui formait autrefois une barrière entre les nations policées et les nations barbares, a été tout de même détourné\footnote{Je crois que de là s’est formé le lac Aral.} par les Tartares, et ne va plus jusqu’à la mer.\par
Séleucus Nicator forma le projet \footnote{Claude César, dans Pline, liv. VI, chap. II.} de joindre le Pont-Euxin à la mer Caspienne. Ce dessein, qui eût donné bien des facilités au commerce qui se faisait dans ce temps-là, s’évanouit à sa mort\footnote{Il fut tué par Ptolomée Ceranus.}. On ne sait s’il aurait pu l’exécuter dans l’isthme qui sépare les deux mers. Ce pays est aujourd’hui très peu connu ; il est dépeuplé et plein de forêts. Les eaux n’y manquent pas, car une infinité de rivières y descendent du mont Caucase ; mais ce Caucase, qui forme le nord de l’isthme, et qui étend des espèces de bras\footnote{Voyez Strabon, liv. XI.} au midi, aurait été un grand obstacle, surtout dans ces temps-là, où l’on n’avait point l’art de faire des écluses.\par
On pourrait croire que Séleucus voulait faire la jonction des deux mers dans le lieu même où le czar Pierre I\textsuperscript{er} l’a faite depuis, c’est-à-dire dans cette langue de terre où le Tanaïs s’approche du Volga ; mais le nord de la mer Caspienne n’était pas encore découvert.\par
Pendant que, dans les empires d’Asie, il y avait un commerce de luxe, les Tyriens faisaient par toute la terre un commerce d’économie. Bochard a employé le premier livre de son {\itshape Chanaan} à faire l’énumération des colonies qu’ils envoyèrent dans tous les pays qui sont près de la mer ; ils passèrent les colonnes d’Hercule et firent des établissements\footnote{Ils fondèrent Tartèse, et s’établirent à Cadix.} sur les côtes de l’Océan.\par
Dans ces temps-là, les navigateurs étaient obligés de suivre les côtes, qui étaient, pour ainsi dire, leur boussole. Les voyages étaient longs et pénibles. Les travaux de la navigation d’Ulysse ont été un sujet fertile pour le plus beau poème du monde, après celui qui est le premier de tous.\par
Le peu de connaissance que la plupart des peuples avaient de ceux qui étaient éloignés d’eux favorisait les nations qui faisaient le commerce d’économie. Elles mettaient dans leur négoce les obscurités qu’elles voulaient : elles avaient tous les avantages que les nations intelligentes prennent sur les peuples ignorants.\par
L’Égypte, éloignée par la religion et par les mœurs de toute communication avec les étrangers, ne faisait guère de commerce au-dehors : elle jouissait d’un terrain fertile et d’une extrême abondance. C’était le Japon de ces temps-là ; elle se suffisait à elle-même.\par
Les Égyptiens furent si peu jaloux du commerce du dehors qu’ils laissèrent celui de la mer Rouge à toutes les petites nations qui y eurent quelque port. Ils souffrirent que les Iduméens, les Juifs et les Syriens y eussent des flottes. Salomon\footnote{Liv. III des Rois, chap. IX, V. 26 ; {\itshape Paralip}., liv. II, chap. VIII.} employa à cette navigation des Tyriens qui connaissaient ces mers.\par
Josèphe\footnote{{\itshape Contre Appion}.} dit que sa nation, uniquement occupée de l’agriculture, connaissait peu la mer : aussi ne fut-ce que par occasion que les Juifs négocièrent dans la mer Rouge. Ils conquirent, sur les Iduméens, Elath et Asiongaber, qui leur donnèrent ce commerce : ils perdirent ces deux villes, et perdirent ce commerce aussi.\par
Il n’en fut pas de même des Phéniciens : ils ne faisaient pas un commerce de luxe ; ils ne négociaient point par la conquête : leur frugalité, leur habileté, leur industrie, leurs périls, leurs fatigues, les rendaient nécessaires à toutes les nations du monde.\par
Les nations voisines de la mer Rouge ne négociaient que dans cette mer et celle d’Afrique. L’étonnement de l’univers à la découverte de la mer des Indes, faite sous Alexandre, le prouve assez. Nous avons dit\footnote{Au chap. I de ce livre.} qu’on porte toujours aux Indes des métaux précieux, et que l’on n’en rapporte point\footnote{La proportion établie en Europe entre l’or et l’argent peut quelquefois faire trouver du profit à prendre dans les Indes de l’or pour de l’argent ; mais c’est peu de chose.} : les flottes juives qui rapportaient par la mer Rouge de l’or et de l’argent, revenaient d’Afrique, et non pas des Indes.\par
Je dis plus : cette navigation se faisait sur la côte orientale de l’Afrique ; et l’état où était la marine pour lors prouve assez qu’on n’allait pas dans des lieux bien reculés.\par
Je sais que les flottes de Salomon et de Jozaphat ne revenaient que la troisième année ; mais je ne vois pas que la longueur du voyage prouve la grandeur de l’éloignement.\par
Pline et Strabon nous disent que le chemin qu’un navire des Indes et de la mer Rouge, fabriqué de joncs, faisait en vingt jours, un navire grec ou romain le faisait en sept\footnote{Voyez Pline, liv. VI, chap. XXII ; et Strabon, liv. XV.}. Dans cette proportion, un voyage d’un an pour les flottes grecques et romaines était à peu près de trois pour celles de Salomon.\par
Deux navires d’une vitesse inégale ne font pas leur voyage dans un temps proportionné à leur vitesse : la lenteur produit souvent une plus grande lenteur. Quand il s’agit de suivre les côtes, et qu’on se trouve sans cesse dans une différente position ; qu’il faut attendre un bon vent pour sortir d’un golfe, en avoir un autre pour aller en avant, un navire bon voilier profite de tous les temps favorables, tandis que l’autre reste dans un endroit difficile, et attend plusieurs jours un autre changement.\par
Cette lenteur des navires des Indes qui, dans un temps égal, ne pouvaient faire que le tiers du chemin que faisaient les vaisseaux grecs et romains, peut s’expliquer par ce que nous voyons aujourd’hui dans notre marine. Les navires des Indes, qui étaient de jonc, tiraient moins d’eau que les vaisseaux grecs et romains, qui étaient de bois, et joints avec du fer.\par
On peut comparer ces navires des Indes à ceux de quelques nations d’aujourd’hui, dont les ports ont peu de fond : tels sont ceux de Venise, et même en général de l’Italie\footnote{Elle n’a presque que des rades ; mais la Sicile a de très bons ports.}, de la mer Baltique et de la province de Hollande\footnote{Je dis de la province de Hollande ; car les ports de celle de Zélande sont assez profonds.}. Leurs navires, qui doivent en sortir et y rentrer, sont d’une fabrique ronde et large de fond ; au lieu que les navires d’autres nations qui ont de bons ports, sont, par le bas, d’une forme qui les fait entrer profondément dans l’eau. Cette mécanique fait que ces derniers navires naviguent plus près du vent, et que les premiers ne naviguent presque que quand ils ont le vent en poupe. Un navire qui entre beaucoup dans l’eau, navigue vers le même côté à presque tous les vents ; ce qui vient de la résistance que trouve dans l’eau le vaisseau poussé par le vent, qui fait un point d’appui, et de la forme longue du vaisseau qui est présenté au vent par son côté, pendant que, par l’effet de la figure du gouvernail, on tourne la proue vers le côté que l’on se propose ; en sorte qu’on peut aller très près du vent, c’est-à-dire, très près du côté d’où vient le vent. Mais quand le navire est d’une figure ronde et large de fond, et que par conséquent il enfonce peu dans l’eau, il n’y a plus de point d’appui ; le vent chasse le vaisseau, qui ne peut résister, ni guère aller que du côté opposé au vent. D’où il suit que les vaisseaux d’une construction ronde de fond sont plus lents dans leurs voyages : 1° ils perdent beaucoup de temps à attendre le vent, surtout s’ils sont obligés de changer souvent de direction ; 2° ils vont plus lentement, parce que, n’ayant pas de point d’appui, ils ne sauraient porter autant de voiles que les autres. Que si, dans un temps où la marine s’est si fort perfectionnée, dans un temps où les arts se communiquent, dans un temps où l’on corrige par l’art, et les défauts de la nature, et les défauts de l’art même, on sent ces différences, que devait-ce être dans la marine des anciens ?\par
Je ne saurais quitter ce sujet. Les navires des Indes étaient petits, et ceux des Grecs et des Romains, si l’on en excepte ces machines que l’ostentation fit faire, étaient moins grands que les nôtres. Or, plus un navire est petit, plus il est en danger dans les gros temps. Telle tempête submerge un navire, qui ne ferait que le tourmenter s’il était plus grand. Plus un corps en surpasse un autre en grandeur, plus sa surface est relativement petite : d’où il suit que dans un petit navire il y a une moindre raison, c’est-à-dire, une plus grande différence de la surface du navire au poids ou à la charge qu’il peut porter, que dans un grand. On sait que, par une pratique à peu près générale, on met dans un navire une charge d’un poids égal à celui de la moitié de l’eau qu’il pourrait contenir. Supposons qu’un navire tînt huit cents tonneaux d’eau, sa charge serait de quatre cents tonneaux ; celle d’un navire qui ne tiendrait que quatre cents tonneaux d’eau serait de deux cents tonneaux. Ainsi la grandeur du premier navire serait, au poids qu’il porterait, comme 8 est à 4 ; et celle du second, comme 4 est à 2. Supposons que la surface du grand soit, à la surface du petit, comme 8 est à 6 ; la surface\footnote{C’est-à-dire, pour comparer les grandeurs de même genre : l’action ou la prise du fluide sur le navire sera, à la résistance du même navire, comme, etc.} de celui-ci sera, à son poids, comme 6 est à 2 ; tandis que la surface de celui-là ne sera, à son poids, que comme 8 est à 4 ; et les vents et les flots n’agissant que sur la surface, le grand vaisseau résistera plus par son poids à leur impétuosité que le petit.
\subsubsection[{Chapitre VII. Du commerce des Grecs}]{Chapitre VII. Du commerce des Grecs}
\noindent Les premiers Grecs étaient tous pirates. Minos, qui avait eu l’empire de la mer, n’avait eu peut-être que de plus grands succès dans les brigandages : son empire était borné aux environs de son île. Mais, lorsque les Grecs devinrent un grand peuple, les Athéniens obtinrent le véritable empire de la mer, parce que cette nation commerçante et victorieuse donna la loi au monarque\footnote{Le roi de Perse.} le plus puissant d’alors, et abattit les forces maritimes de la Syrie, de l’île de Chypre et de la Phénicie.\par
Il faut que je parle de cet empire de la mer qu’eut Athènes. « Athènes, dit Xénophon\footnote{{\itshape De republica Atheniensium}.}, a l’empire de la mer ; mais, comme l’Attique tient à la terre, les ennemis la ravagent, tandis qu’elle fait ses expéditions au loin. Les principaux laissent détruire leurs terres, et mettent leurs biens en sûreté dans quelque île : la populace, qui n’a point de terres, vit sans aucune inquiétude. Mais, si les Athéniens habitaient une île et avaient outre cela l’empire de la mer, ils auraient le pouvoir de nuire aux autres sans qu’on pût leur nuire, tandis qu’ils seraient les maîtres de la mer. » Vous diriez que Xénophon a voulu parler de l’Angleterre.\par
Athènes, remplie de projets de gloire, Athènes, qui augmentait la jalousie, au lieu d’augmenter l’influence ; plus attentive à étendre son empire maritime qu’à en jouir ; avec un tel gouvernement politique, que le bas peuple se distribuait les revenus publics, tandis que les riches étaient dans l’oppression, ne fit point ce grand commerce que lui promettaient le travail de ses mines, la multitude de ses esclaves, le nombre de ses gens de mer, son autorité sur les villes grecques, et plus que tout cela, les belles institutions de Solon. Son négoce fut presque borné à la Grèce et au Pont-Euxin, d’où elle tira sa subsistance.\par
Corinthe fut admirablement bien située : elle sépara deux mers, ouvrit et ferma le Péloponnèse, et ouvrit et ferma la Grèce. Elle fut une ville de la plus grande importance, dans un temps où le peuple grec était un monde, et les villes grecques des nations. Elle fit un plus grand commerce qu’Athènes. Elle avait un port pour recevoir les marchandises d’Asie ; elle en avait un autre pour recevoir celles d’Italie ; car, comme il y avait de grandes difficultés à tourner le promontoire Malée, où des vents\footnote{Voyez Strabon, liv. VIII.} opposés se rencontrent et causent des naufrages, on aimait mieux aller à Corinthe, et l’on pouvait même faire passer par terre les vaisseaux d’une mer à l’autre. Dans aucune ville on ne porta si loin les ouvrages de l’art. La religion acheva de corrompre ce que son opulence lui avait laissé de mœurs. Elle érigea un temple à Vénus, où plus de mille courtisanes furent consacrées. C’est de ce séminaire que sortirent la plupart de ces beautés célèbres dont Athénée a osé écrire l’histoire.\par
Il paraît que, du temps d’Homère, l’opulence de la Grèce était à Rhodes, à Corinthe et à Orchomène. « Jupiter, dit-il\footnote{{\itshape Iliade}, liv. II.}, aima les Rhodiens, et leur donna de grandes richesses. » Il donne à Corinthe\footnote{{\itshape Ibid.}} l’épithète de riche.\par
De même, quand il veut parler des villes qui ont beaucoup d’or, il cite Orchomène\footnote{{\itshape Ibid.} V. 381. Voyez Strabon, liv. IX p. 414, éd. de 1620.} qu’il joint à Thèbes d’Égypte. Rhodes et Corinthe conservèrent leur puissance, et Orchomène la perdit. La position d’Orchomène, près de l’Hellespont, de la Propontide et du Pont-Euxin fait naturellement penser qu’elle tirait ses richesses d’un commerce sur les côtes de ces mers, qui avaient donné lieu à la fable de la toison d’or. Et effectivement, le nom de Miniares est donné à Orchomène\footnote{Strabon, liv. IX p. 414.} et encore aux Argonautes. Mais comme, dans la suite, ces mers devinrent plus connues ; que les Grecs y établirent un très grand nombre de colonies ; que ces colonies négocièrent avec les peuples barbares ; qu’elles communiquèrent avec leur métropole ; Orchomène commença à déchoir, et elle rentra dans la foule des autres villes grecques.\par
Les Grecs, avant Homère, n’avaient guère négocié qu’entre eux, et chez quelque peuple barbare ; mais ils étendirent leur domination à mesure qu’ils formèrent de nouveaux peuples. La Grèce était une grande péninsule dont les caps semblaient avoir fait reculer les mers, et les golfes s’ouvrir de tous côtés, comme pour les recevoir encore. Si l’on jette les yeux sur la Grèce, on verra, dans un pays assez resserré, une vaste étendue de côtes. Ses colonies innombrables faisaient une immense circonférence autour d’elle ; et elle y voyait, pour ainsi dire, tout le monde qui n’était pas barbare. Pénétra-t-elle en Sicile et en Italie ? elle y forma des nations. Navigua-t-elle vers les mers du Pont, vers les côtes de l’Asie Mineure, vers celles d’Afrique ? elle en fit de même. Ses villes acquirent de la prospérité, à mesure qu’elles se trouvèrent près de nouveaux peuples. Et, ce qu’il y avait d’admirable, des îles sans nombre, situées comme en première ligne, l’entouraient encore.\par
Quelles causes de prospérité pour la Grèce, que des jeux qu’elle donnait, pour ainsi dire, à l’univers ; des temples, où tous les rois envoyaient des offrandes ; des fêtes, où l’on s’assemblait de toutes parts ; des oracles qui faisaient l’attention de toute la curiosité humaine ; enfin, le goût et les arts portés à un point, que de croire les surpasser sera toujours ne les pas connaître !
\subsubsection[{Chapitre VIII. D’Alexandre. Sa conquête}]{Chapitre VIII. D’Alexandre. Sa conquête}
\noindent Quatre événements arrivés sous Alexandre firent dans le commerce une grande révolution : la prise de Tyr, la conquête de l’Égypte, celle des Indes et la découverte de la mer qui est au midi de ce pays.\par
L’empire des Perses s’étendait jusqu’à l’Indus\footnote{Strabon, liv. XV.}. Longtemps avant Alexandre, Darius\footnote{Hérodote, {\itshape in Melpomene}.} avait envoyé des navigateurs qui descendirent ce fleuve, et allèrent jusqu’à la mer Rouge. Comment donc les Grecs furent-ils les premiers qui firent par le midi le commerce des Indes ? Comment les Perses ne l’avaient-ils pas fait auparavant ? Que leur servaient des mers qui étaient si proches d’eux, des mers qui baignaient leur empire ? Il est vrai qu’Alexandre conquit les Indes : mais faut-il conquérir un pays pour y négocier ? J’examinerai ceci.\par
L’Ariane\footnote{Strabon, liv. XV.}, qui s’étendait depuis le golfe Persique jusqu’à l’Indus, et de la mer du midi jusqu’aux montagnes des Paropamisades, dépendait bien en quelque façon de l’empire des Perses ; mais, dans sa partie méridionale, elle était aride, brûlée, inculte et barbare. La tradition\footnote{{\itshape Ibid.}} portait que les armées de Sémiramis et de Cyrus avaient péri dans ces déserts ; et Alexandre, qui se fit suivre par sa flotte, ne laissa pas d’y perdre une grande partie de son armée. Les Perses laissaient toute la côte au pouvoir des Ichthyophages\footnote{Pline, liv. VI, chap. XXIII ; Strabon, liv. XV.}, des Orittes et autres peuples barbares. D’ailleurs les Perses\footnote{Pour ne point souiller les éléments, ils ne naviguaient pas sur les fleuves. M. Hyde, {\itshape Religion des Perses}. Encore aujourd’hui ils n’ont point de commerce maritime, et ils traitent d’athées ceux qui vont sur mer.} n’étaient pas navigateurs, et leur religion même leur ôtait toute idée de commerce maritime. La navigation que Darius fit faire sur l’Indus et la mer des Indes fut plutôt une fantaisie d’un prince qui veut montrer sa puissance, que le projet réglé d’un monarque qui veut l’employer. Elle n’eut de suite, ni pour le commerce, ni pour la marine ; et si l’on sortit de l’ignorance, ce fut pour y retomber.\par
Il y a plus : il était reçu\footnote{Strabon, liv. XV.}, avant l’expédition d’Alexandre, que la partie méridionale des Indes était inhabitable\footnote{Hérodote, {\itshape in Melpomene}, dit que Darius conquit les Indes. Cela ne peut être entendu que de l’Ariane : encore ne fut-ce qu’une conquête en idée.} : ce qui suivait de la tradition que Sémiramis\footnote{Strabon, liv. XV.} n’en avait ramené que vingt hommes, et Cyrus que sept.\par
Alexandre entra par le nord. Son dessein était de marcher vers l’orient ; mais, ayant trouvé la partie du midi pleine de grandes nations, de villes et de rivières, il en tenta la conquête, et la fit.\par
Pour lors il forma le dessein d’unir les Indes avec l’Occident par un commerce maritime, comme il les avait unis par des colonies qu’il avait établies dans les terres.\par
Il fit construire une flotte sur l’Hydaspe, descendit cette rivière, entra dans l’Indus, et navigua jusqu’à son embouchure. Il laissa son armée et sa flotte à Patale, alla lui-même avec quelques vaisseaux reconnaître la mer, marqua les lieux où il voulut que l’on construisît des ports, des havres, des arsenaux. De retour à Patale il se sépara de sa flotte, et prit la route de terre pour lui donner du secours, et en recevoir. La flotte suivit la côte depuis l’embouchure de l’Indus, le long du rivage des pays des Orittes, des Ichthyophages, de la Caramanie et de la Perse. Il fit creuser des puits, bâtir des villes ; il défendit aux Ichthyophages\footnote{Ceci ne saurait s’entendre de tous les Ichthyophages, qui habitaient une côte de dix mille stades. Comment Alexandre aurait-il pu leur donner la subsistance ? Comment se serait-il fait obéir ? Il ne peut être ici question que de quelques peuples particuliers. Néarque, dans le livre {\itshape Rerum Indicarum}, dit qu’à l’extrémité de cette côte, du côté de la Perse, il avait trouvé les peuples moins ichthyophages. Je croirais que l’ordre d’Alexandre regardait cette contrée, ou quelque autre encore plus voisine de la Perse.} de vivre de poisson ; il voulait que les bords de cette mer fussent habités par des nations civilisées. Néarque et Onésicrite ont fait le journal de cette navigation, qui fut de dix mois. Ils arrivèrent à Suse ; ils y trouvèrent Alexandre qui donnait des fêtes à son année.\par
Ce conquérant avait fondé Alexandrie, dans la vue de s’assurer de l’Égypte : c’était une clef pour l’ouvrir, dans le lieu même où les rois ses prédécesseurs avaient une clef pour la fermer\footnote{Alexandrie fut fondée dans une plage appelée {\itshape Racotis.} Les anciens rois y tenaient une garnison pour défendre l’entrée du pays aux étrangers, et surtout aux Grecs, qui étaient, comme on sait, de grands pirates. Voyez Pline, liv. VI, chap. X ; et Strabon, liv. XVIII.} ; et il ne songeait point à un commerce dont la découverte de la mer des Indes pouvait seule lui faire naître la pensée.\par
Il paraît même qu’après cette découverte, il n’eut aucune vue nouvelle sur Alexandrie. Il avait bien, en général, le projet d’établir un commerce entre les Indes et les parties occidentales de son empire ; mais, pour le projet de faire ce commerce par l’Égypte, il lui manquait trop de connaissances pour pouvoir le former. Il avait vu l’Indus, il avait vu le Nil ; mais il ne connaissait point les mers d’Arabie qui sont entre deux. À peine fut-il arrivé des Indes, qu’il fit construire de nouvelles flottes, et navigua\footnote{Arrien, {\itshape De Expeditione Alexandri}, liv. VII.} sur l’Euléus le Tigre, l’Euphrate et la mer : il ôta les cataractes que les Perses avaient mises sur ces fleuves : il découvrit que le sein Persique était un golfe de l’Océan. Comme il alla reconnaître\footnote{{\itshape Ibid.}} cette mer, ainsi qu’il avait reconnu celle des Indes ; comme il fit construire un port à Babylone pour mille vaisseaux, et des arsenaux ; comme il envoya cinq cents talents en Phénicie et en Syrie, pour en faire venir des nautoniers, qu’il voulait placer dans les colonies qu’il répandait sur les côtes ; comme enfin il fit des travaux immenses sur l’Euphrate et les autres fleuves de l’Assyrie, on ne peut douter que son dessein ne fût de faire le commerce des Indes par Babylone et le golfe Persique.\par
Quelques gens, sous prétexte qu’Alexandre voulait conquérir l’Arabie\footnote{Strabon, liv. XVI, à la fin.}, ont dit qu’il avait formé le dessein d’y mettre le siège de son empire ; mais comment aurait-il choisi un lieu qu’il ne connaissait pas\footnote{Voyant la Babylonie inondée, il regardait l’Arabie, qui en est proche, comme une île. Aristobule, dans Strabon, liv. XVI.} ? D’ailleurs, c’était le pays du monde le plus incommode : il se serait séparé de son empire. Les califes, qui conquirent au loin, quittèrent d’abord l’Arabie pour s’établir ailleurs.
\subsubsection[{Chapitre IX. Du commerce des rois grecs après Alexandre}]{Chapitre IX. Du commerce des rois grecs après Alexandre}
\noindent Lorsque Alexandre conquit l’Égypte, on connaissait très peu la mer Rouge, et rien de cette partie de l’Océan qui se joint à cette mer, et qui baigne d’un côté la côte d’Afrique, et de l’autre celle de l’Arabie : on crut même depuis qu’il était impossible de faire le tour de la presqu’île d’Arabie. Ceux qui l’avaient tenté de chaque côté avaient abandonné leur entreprise. On disait\footnote{Voyez le livre {\itshape Rerum Indicarum}.} : « Comment serait-il possible de naviguer au midi des côtes de l’Arabie, puisque l’armée de Cambyse, qui la traversa du côté du nord, périt presque toute, et que celle que Ptolomée, fils de Lagus, envoya au secours de Séleucus Nicator à Babylone, souffrit des maux incroyables, et, à cause de la chaleur, ne put marcher que la nuit ? »\par
Les Perses n’avaient aucune sorte de navigation. Quand ils conquirent l’Égypte, ils y apportèrent le même esprit qu’ils avaient eu chez eux ; et la négligence fut si extraordinaire, que les rois grecs trouvèrent que non seulement les navigations des Tyriens, des Iduméens et des Juifs dans l’Océan étaient ignorées, mais que celles même de la mer Rouge l’étaient. Je crois que la destruction de la première Tyr par Nabuchodonosor, et celle de plusieurs petites nations et villes voisines de la mer Rouge, firent perdre les connaissances que l’on avait acquises.\par
L’Égypte, du temps des Perses, ne confrontait point à la mer Rouge : elle ne contenait\footnote{Strabon, liv. XVI.} que cette lisière de terre longue et étroite que le Nil couvre par ses inondations, et qui est resserrée des deux côtés par des chaînes de montagnes. Il fallut donc découvrir la mer Rouge une seconde fois, et l’Océan une seconde fois ; et cette découverte appartint à la curiosité des rois grecs.\par
On remonta le Nil ; on fit la chasse des éléphants dans les pays qui sont entre le Nil et la mer ; on découvrit les bords de cette mer par les terres ; et, comme cette découverte se fit sous les Grecs, les noms en sont grecs, et les temples sont consacrés\footnote{Strabon, liv. XVI.} à des divinités grecques.\par
Les Grecs d’Égypte purent faire un commerce très étendu : ils étaient maîtres des ports de la mer Rouge ; Tyr, rivale de toute nation commerçante, n’était plus ; ils n’étaient point gênés par les anciennes\footnote{Elles leur donnaient de l’horreur pour les étrangers.} superstitions du pays ; l’Égypte était devenue le centre de l’univers.\par
Les rois de Syrie laissèrent à ceux d’Égypte le commerce méridional des Indes, et ne s’attachèrent qu’à ce commerce septentrional qui se faisait par l’Oxus et la mer Caspienne. On croyait, dans ce temps-là, que cette mer était une partie de l’Océan septentrional\footnote{Pline, liv. II, chap. LXVIII ; et liv. VI, chap. IX et XII ; Strabon, liv. XI, p. 507 ; Arrien, {\itshape De l’Expédition d’Alexandre}, liv. III, p. 74 ; et liv. V, p. 104.} ; et Alexandre, quelque temps avant sa mort, avait fait construire\footnote{Arrien, {\itshape De l’Expédition d’Alexandre}, liv. VII.} une flotte pour découvrir si elle communiquait à l’Océan par le Pont-Euxin, ou par quelque autre mer orientale vers les Indes. Après lui, Séleucus et Antiochus eurent une attention particulière à la reconnaître. Ils y entretinrent des flottes\footnote{Pline, liv. II, chap. LXIV.}. Ce que Séleucus reconnut fut appelé mer Séleucide : ce qu’Antiochus découvrit fut appelé mer Antiochide. Attentifs aux projets qu’ils pouvaient avoir de ce côté-là, ils négligèrent les mers du midi ; soit que les Ptolomées, par leurs flottes sur la mer Rouge, s’en fussent déjà procuré l’empire ; soit qu’ils eussent découvert dans les Perses un éloignement invincible pour la marine. La côte du midi de la Perse ne fournissait point de matelots ; on n’y en avait vu que dans les derniers moments de la vie d’Alexandre. Mais les rois d’Égypte, maîtres de l’île de Chypre, de la Phénicie et d’un grand nombre de places sur les côtes de l’Asie Mineure, avaient toutes sortes de moyens pour faire des entreprises de mer. Ils n’avaient point à contraindre le génie de leurs sujets ; ils n’avaient qu’à le suivre.\par
On a de la peine à comprendre l’obstination des anciens à croire que la mer Caspienne était une partie de l’Océan. Les expéditions d’Alexandre, des rois de Syrie, des Parthes et des Romains, ne purent leur faire changer de pensée. C’est qu’on revient de ses erreurs le plus tard qu’on peut. D’abord on ne connut que le midi de la mer Caspienne ; on la prit pour l’Océan ; à mesure que l’on avança le long de ses bords du côté du nord, on crut encore que c’était l’Océan qui entrait dans les terres. En suivant les côtes, on n’avait reconnu, du côté de l’est, que jusqu’au Jaxarte ; et, du côté de l’ouest, que jusqu’aux extrémités de l’Albanie. La mer, du côté du nord, était vaseuse\footnote{Voyez la carte du czar.}, et par conséquent très peu propre à la navigation. Tout cela fit que l’on ne vit jamais que l’Océan.\par
L’armée d’Alexandre n’avait été, du côté de l’orient, que jusqu’à l’Hypanis, qui est la dernière des rivières qui se jettent dans l’Indus. Ainsi le premier commerce que les Grecs eurent aux Indes se fit dans une très petite partie du pays. Séleucus Nicator pénétra jusqu’au Gange\footnote{Pline, liv. VI, chap. XVII.} ; et par là on découvrit la mer où ce fleuve se jette, c’est-à-dire le golfe de Bengale. Aujourd’hui l’on découvre les terres par les voyages de mer : autrefois on découvrait les mers par la conquête des terres.\par
Strabon\footnote{Liv. XV.}, malgré le témoignage d’Apollodore, paraît douter que les rois\footnote{Les Macédoniens de la Bactriane, des Indes, et de l’Ariane, s’étant séparés du royaume de Syrie, formèrent un grand État.} grecs de Bactriane soient allés plus loin que Séleucus et Alexandre. Quand il serait vrai qu’ils n’auraient pas été plus loin vers l’orient que Séleucus, ils allèrent plus loin vers le midi : ils découvrirent\footnote{Apollonius Adramittin, dans Strabon, liv. XI.} Siger et des ports dans le Malabar, qui donnèrent lieu à la navigation dont je vais parler.\par
Pline\footnote{Liv. VI, chap. XXIII.} nous apprend qu’on prit successivement trois routes pour faire la navigation des Indes. D’abord, on alla, du promontoire de Siagre, à l’île de Patalène, qui est à l’embouchure de l’Indus : on voit que c’était la route qu’avait tenue la flotte d’Alexandre. On prit ensuite un chemin plus court\footnote{Pline, Liv. VI, chap. XXIII.} et plus sûr ; et on alla du même promontoire à Siger. Ce Siger ne peut être que le royaume de Siger dont parle Strabon\footnote{Liv. XI, {\itshape Sigertidis regnum}.}, que les rois grecs de Bactriane découvrirent. Pline ne peut dire que ce chemin fût plus court, que parce qu’on le faisait en moins de temps ; car Siger devait être plus reculé que l’Indus, puisque les rois de Bactriane le découvrirent. Il fallait donc que l’on évitât par là le détour de certaines côtes, et que l’on profitât de certains vents. Enfin les marchands prirent une troisième route : ils se rendaient à Canes ou à Océlis ports situés à l’embouchure de la mer Rouge, d’où, par un vent d’ouest, on arrivait à Muziris première étape des Indes, et de là à d’autres ports.\par
On voit qu’au lieu d’aller de l’embouchure de la mer Rouge jusqu’à Siagre, en remontant la côte de l’Arabie heureuse au nord-est, on alla directement de l’ouest à l’est, d’un côté à l’autre, par le moyen des moussons, dont on découvrit les changements en naviguant dans ces parages. Les anciens ne quittèrent les côtes que quand ils se servirent des moussons\footnote{Les moussons soufflent une partie de l’année d’un côté, et une partie de l’année de l’autre ; et les vents alizés soufflent du même côté toute l’année.} et des vents alizés, qui étaient une espèce de boussole pour eux.\par
Pline\footnote{Liv. VI, chap. XXIII.} dit qu’on parlait pour les Indes au milieu de l’été, et qu’on en revenait vers la fin de décembre et au commencement de janvier. Ceci est entièrement conforme aux journaux de nos navigateurs. Dans cette partie de la mer des Indes qui est entre la presqu’île d’Afrique et celle de deçà le Gange, il y a deux moussons : la première, pendant laquelle les vents vont de l’ouest à l’est, commence au mois d’août et de septembre ; la deuxième, pendant laquelle les vents vont de l’est à l’ouest, commence en janvier. Ainsi nous partons d’Afrique pour le Malabar dans le temps que partaient les flottes de Ptolomée, et nous en revenons dans le même temps.\par
La flotte d’Alexandre mit sept mois pour aller de Patale à Suse. Elle partit dans le mois de juillet, c’est-à-dire dans un temps où aujourd’hui aucun navire n’ose se mettre en mer pour revenir des Indes. Entre l’une et l’autre mousson, il y a un intervalle de temps pendant lequel les vents varient ; et où un vent de nord, se mêlant avec les vents ordinaires, cause, surtout auprès des côtes, d’horribles tempêtes. Cela dure les mois de juin, de juillet et d’août. La flotte d’Alexandre, partant de Patale au mois de juillet, essuya bien des tempêtes, et le voyage fut long, parce qu’elle navigua dans une mousson contraire.\par
Pline dit qu’on partait pour les Indes à la fin de l’été : ainsi on employait le temps de la variation de la mousson à faire le trajet d’Alexandrie à la mer Rouge.\par
Voyez, je vous prie, comment on se perfectionna peu à peu dans la navigation. Celle que Darius fit faire pour descendre l’Indus et aller à la mer Rouge, fut de deux ans et demi\footnote{Hérodote, {\itshape in Melpomene}.}. La flotte d’Alexandre\footnote{Pline, liv. VI, chap. XXIII.} descendant l’Indus, arriva à Suse dix mois après, ayant navigué trois mois sur l’Indus, et sept sur la mer des Indes. Dans la suite, le trajet de la côte de Malabar à la mer Rouge se fit en quarante jours\footnote{{\itshape Ibid.}}.\par
Strabon, qui rend raison de l’ignorance où l’on était des pays qui sont entre l’Hypanis et le Gange, dit que, parmi les navigateurs qui vont de l’Égypte aux Indes, il y en a peu qui aillent jusqu’au Gange. Effectivement, on voit que les flottes n’y allaient pas ; elles allaient, par les moussons de l’ouest à l’est, de l’embouchure de la mer Rouge à la côte de Malabar. Elles s’arrêtaient dans les étapes qui y étaient, et n’allaient point faire le tour de la presqu’île deçà le Gange par le cap de Comorin et la côte de Coromandel. Le plan de la navigation des rois d’Égypte et des Romains, était de revenir la même année\footnote{{\itshape Ibid.}}.\par
Ainsi il s’en faut bien que le commerce des Grecs et des Romains aux Indes ait été aussi étendu que le nôtre ; nous qui connaissons des pays immenses qu’ils ne connaissaient pas ; nous qui faisons notre commerce avec toutes les nations indiennes, et qui commerçons même pour elles et naviguons pour elles.\par
Mais ils faisaient ce commerce avec plus de facilité que nous ; et, si l’on ne négociait aujourd’hui que sur la côte du Guzarat et du Malabar, et que, sans aller chercher les îles du midi, on se contentât des marchandises que les insulaires viendraient apporter, il faudrait préférer la route de l’Égypte à celle du cap de Bonne-Espérance. Strabon\footnote{Liv. XV.} dit que l’on négociait ainsi avec les peuples de la Taprobane.
\subsubsection[{Chapitre X. Du tour de l’Afrique}]{Chapitre X. Du tour de l’Afrique}
\noindent On trouve dans l’histoire qu’avant la découverte de la boussole, on tenta quatre fois de faire le tour de l’Afrique. Des Phéniciens envoyés par Nécho\footnote{Hérodote, liv. IV. Il voulait conquérir.}, et Eudoxe\footnote{Pline, liv. II, chap. LXVII. Pomponius Mela, liv. III, chap. IX.}, fuyant la colère de Ptolomée Lature, partirent de la mer Rouge et réussirent. Sataspe\footnote{Hérodote, {\itshape in Melpomene}.} sous Xerxès, et Hannon, qui fut envoyé par les Carthaginois, sortirent des colonnes d’Hercule, et ne réussirent pas.\par
Le point capital pour faire le tour de l’Afrique était de découvrir et de doubler le cap de Bonne-Espérance. Mais, si l’on partait de la mer Rouge, on trouvait ce cap de la moitié du chemin plus près qu’en partant de la Méditerranée. La côte qui va de la mer Rouge au Cap est plus saine que\footnote{Joignez à ceci ce que je dis au chap. XI de ce livre, sur la navigation d’Hannon.} celle qui va du Cap aux colonnes d’Hercule. Pour que ceux qui partaient des colonnes d’Hercule aient pu découvrir le Cap, il a fallu l’invention de la boussole, qui a fait que l’on a quitté la côte d’Afrique, et qu’on a navigué dans le vaste Océan\footnote{On trouve dans l’océan Atlantique, aux mois d’octobre, novembre, décembre et janvier un vent de nord-est. On passe la ligne ; et, pour éluder le vent général d’est, on dirige sa route vers le sud ; ou bien on entre dans la zone torride, dans les lieux où le vent souffle de l’ouest à l’est.} pour aller vers l’île de Sainte-Hélène ou vers la côte du Brésil. Il était donc très possible qu’on fût allé de la mer Rouge dans la Méditerranée, sans qu’on fût revenu de la Méditerranée à la mer Rouge.\par
Ainsi, sans faire ce grand circuit, après lequel on ne pouvait plus revenir, il était plus naturel de faire le commerce de l’Afrique orientale par la mer Rouge, et celui de la côte occidentale par les colonnes d’Hercule.\par
Les rois grecs d’Égypte découvrirent d’abord, dans la mer Rouge, la partie de la côte d’Afrique qui va depuis le fond du golfe où est la cité d’Heroum jusqu’à Dira, c’est-à-dire jusqu’au détroit appelé aujourd’hui de Bab-el-Mandel. De là jusqu’au promontoire des Aromates, situé à l’entrée de la mer Rouge\footnote{Ce golfe, auquel nous donnons aujourd’hui ce nom, était appelé, par les anciens, le sein Arabique : ils appelaient mer Rouge la partie de l’Océan voisine de ce golfe.} la côte n’avait point été reconnue par les navigateurs ; et cela est clair par ce que nous dit Artémidore\footnote{Strabon, liv. XVI.} que l’on connais sait les lieux de cette côte, mais qu’on en ignorait les distances ; ce qui venait de ce qu’on avait successivement connu ces ports par les terres, et sans aller de l’un à l’autre.\par
Au-delà de ce promontoire où commence la côte de l’Océan, on ne connaissait rien, comme nous\footnote{{\itshape Ibid.} Artémidore bornait la côte connue au lieu appelé {\itshape Austricoma}, et Ératosthène, {\itshape ad Cinnamomiferam.}} l’apprenons d’Ératosthène et d’Artémidore.\par
Telles étaient les connaissances que l’on avait des côtes d’Afrique du temps de Strabon, c’est-à-dire du temps d’Auguste. Mais, depuis Auguste, les Romains découvrirent le promontoire Raptum et le promontoire Prassum dont Strabon ne parle pas, parce qu’ils n’étaient pas encore connus. On voit que ces deux noms sont romains.\par
Ptolomée le géographe vivait sous Adrien et Antonin Pie ; et l’auteur du {\itshape Périple} de la mer Érythrée, quel qu’il soit, vécut peu de temps après. Cependant le premier borne l’Afrique\footnote{Liv. I, chap. VII ; liv. IV, chap. IX ; table IV de l’Afrique.} connue au promontoire Prassum, qui est environ au quatorzième degré de latitude sud ; et l’auteur du {\itshape Périple}\footnote{On a attribué ce {\itshape Périple} à Arrien.}, au promontoire Raptum, qui est à peu près au dixième degré de cette latitude. Il y a apparence que celui-ci prenait pour limite un lieu où l’on allait, et Ptolomée un lieu où l’on n’allait plus.\par
Ce qui me confirme dans cette idée, c’est que les peuples autour du Prassum étaient anthropophages\footnote{Ptolomée, liv. IV, chap. IX.}. Ptolomée, qui\footnote{Liv. IV, chap. VII et VIII.} nous parle d’un grand nombre de lieux entre le port des Aromates et le promontoire Raptum, laisse un vide total depuis le Raptum jusqu’au Prassum. Les grands profits de la navigation des Indes durent faire négliger celle d’Afrique. Enfin les Romains n’eurent jamais sur cette côte de navigation réglée : ils avaient découvert ces ports par les terres, et par des navires jetés par la tempête ; et comme aujourd’hui on connaît assez bien les côtes de l’Afrique, et très mal l’intérieur\footnote{Voyez avec quelle exactitude Strabon et Ptolomée nous décrivent les diverses parties de l’Afrique. Ces connaissances venaient des diverses guerres que les deux plus puissantes nations du monde, les Carthaginois et les Romains, avaient eues avec les peuples d’Afrique, des alliances qu’ils avaient contractées, du commerce qu’ils avaient fait dans les terres.}, les anciens connaissaient assez bien l’intérieur, et très mal les côtes.\par
J’ai dit que des Phéniciens envoyés par Nécho et Eudoxe sous Ptolomée Lature, avaient fait le tour de l’Afrique : il faut bien que, du temps de Ptolomée le géographe, ces deux navigations fussent regardées comme fabuleuses, puisqu’il place\footnote{Liv. VII, chap. III.}, depuis le {\itshape sinus magnus}, qui est, je crois, le golfe de Siam, une terre inconnue, qui va d’Asie en Afrique aboutir au promontoire {\itshape Prassum} ; de sorte que la mer des Indes n’aurait été qu’un lac. Les anciens, qui reconnurent les Indes par le nord, s’étant avancés vers l’orient, placèrent vers le midi cette terre inconnue.
\subsubsection[{Chapitre XI. Carthage Marseille}]{Chapitre XI. Carthage Marseille}
\noindent Carthage avait un singulier droit des gens ; elle faisait noyer\footnote{Ératosthène, dans Strabon, liv. XVII, p. 802.} tous les étrangers qui trafiquaient en Sardaigne et vers les colonnes d’Hercule. Son droit politique n’était pas moins extraordinaire ; elle défendit aux Sardes de cultiver la terre, sous peine de la vie. Elle accrut sa puissance par ses richesses, et ensuite ses richesses par sa puissance. Maîtresse des côtes d’Afrique que baigne la Méditerranée, elle s’étendit le long de celles de l’Océan. Hannon, par ordre du sénat de Carthage, répandit trente mille Carthaginois depuis les colonnes d’Hercule jusqu’à Cerné. Il dit que ce lieu est aussi éloigné des colonnes d’Hercule que les colonnes d’Hercule le sont de Carthage. Cette position est très remarquable ; elle fait voir qu’Hannon borna ses établissements au vingt-cinquième degré de latitude nord, c’est-à-dire deux ou trois degrés au-delà des îles Canaries, vers le sud.\par
Hannon, étant à Cerné, fit une autre navigation, dont l’objet était de faire des découvertes plus avant vers le midi. Il ne prit presque aucune connaissance du continent. L’étendue des côtes qu’il suivit fut de vingt-six jours de navigation, et il fut obligé de revenir faute de vivres. Il paraît que les Carthaginois ne firent aucun usage de cette entreprise d’Hannon. Scylax\footnote{Voyez son {\itshape Périple}, article de Carthage.} dit qu’au-delà de Cerné la mer n’est pas navigable\footnote{Voyez Hérodote, {\itshape in Melpomene}, sur les obstacles que Sataspe trouva.}, parce qu’elle y est basse, pleine de limon et d’herbes marines : effectivement il y en a beaucoup dans ces parages\footnote{Voyez les cartes et les relations, le premier volume des {\itshape Voyages qui ont servi à l’établissement de la compagnie des Indes}, part. I, p. 201. Cette herbe couvre tellement la surface de la mer, qu’on a de la peine à voir l’eau ; et les vaisseaux ne peuvent passer à travers que par un vent frais.}. Les marchands carthaginois dont parle Scylax pouvaient trouver des obstacles qu’Hannon, qui avait soixante navires de cinquante rames chacun, avait vaincus. Les difficultés sont relatives ; et, de plus, on ne doit pas confondre une entreprise qui a la hardiesse et la témérité pour objet, avec ce qui est l’effet d’une conduite ordinaire.\par
C’est un beau morceau de l’antiquité que la relation d’Hannon : le même homme qui a exécuté a écrit ; il ne met aucune ostentation dans ses récits. Les grands capitaines écrivent leurs actions avec simplicité, parce qu’ils sont plus glorieux de ce qu’ils ont fait que de ce qu’ils ont dit.\par
Les choses sont comme le style. Il ne donne point dans le merveilleux : tout ce qu’il dit du climat, du terrain, des mœurs, des manières des habitants se rapporte à ce qu’on voit aujourd’hui dans cette côte d’Afrique ; il semble que c’est le journal d’un de nos navigateurs.\par
Hannon remarqua\footnote{Pline nous dit la même chose en parlant du mont Atlas : {\itshape Noctibus micare crebris ignibus, tibiarum cantu tympanorumque sonitu strepere, neminem interdiu cerni.}} sur sa flotte que, le jour, il régnait dans le continent un vaste silence ; que, la nuit, on entendait les sons de divers instruments de musique ; et qu’on voyait partout des feux, les uns plus grands, les autres moindres. Nos relations confirment ceci : on y trouve que, le jour, ces sauvages, pour éviter l’ardeur du soleil, se retirent dans les forêts ; que, la nuit, ils font de grands feux pour écarter les bêtes féroces ; et qu’ils aiment passionnément la danse et les instruments de musique.\par
Hannon nous décrit un volcan avec tous les phénomènes que fait voir aujourd’hui le Vésuve ; et le récit qu’il fait de ces deux femmes velues qui se laissèrent plutôt tuer que de suivre les Carthaginois, et dont il fit porter les peaux à Carthage, n’est pas, comme on l’a dit, hors de vraisemblance.\par
Cette relation est d’autant plus précieuse, qu’elle est un monument punique ; et c’est parce qu’elle est un monument punique, qu’elle a été regardée comme fabuleuse. Car les Romains conservèrent leur haine contre les Carthaginois, même après les avoir détruits. Mais ce ne fut que la victoire qui décida s’il fallait dire {\itshape la} foi {\itshape punique ou la} foi {\itshape romaine}.\par
Des modernes\footnote{M. Dodwel. Voyez sa {\itshape Dissertation sur le Périple d’Hannon}.} ont suivi ce préjugé. Que sont devenues, disent-ils, les villes qu’Hannon nous décrit, et dont, même du temps de Pline, il ne restait pas le moindre vestige ? Le merveilleux serait qu’il en fût resté. Était-ce Corinthe ou Athènes qu’Hannon allait bâtir sur ces côtes ? Il laissait, dans les endroits propres au commerce, des familles carthaginoises ; et, à la hâte, il les mettait en sûreté contre les hommes sauvages et les bêtes féroces. Les calamités des Carthaginois firent cesser la navigation d’Afrique ; il fallut bien que ces familles périssent, ou devinssent sauvages. Je dis plus : quand les ruines de ces villes subsisteraient encore, qui est-ce qui aurait été en faire la découverte dans les bois et dans les marais ? On trouve pourtant dans Scylax et dans Polybe, que les Carthaginois avaient de grands établissements sur ces côtes. Voilà les vestiges des villes d’Hannon ; il n’y en a point d’autres, parce qu’à peine y en a-t-il d’autres de Carthage même.\par
Les Carthaginois étaient sur le chemin des richesses : et, s’ils avaient été jusqu’au quatrième degré de latitude nord, et au quinzième de longitude, ils auraient découvert la côte d’Or et les côtes voisines. Ils y auraient fait un commerce de toute autre importance que celui qu’on y fait aujourd’hui que l’Amérique semble avoir avili les richesses de tous les autres pays : ils y auraient trouvé des trésors qui ne pouvaient être enlevés par les Romains.\par
On a dit des choses bien surprenantes des richesses de l’Espagne. Si l’on en croit Aristote\footnote{{\itshape Des choses merveilleuses}.}, les Phéniciens qui abordèrent à Tartèse y trouvèrent tant d’argent que leurs navires ne pouvaient le contenir ; et ils firent faire de ce métal leurs plus vils ustensiles. Les Carthaginois, au rapport de Diodore\footnote{Liv. VI.}. trouvèrent tant d’or et d’argent dans les Pyrénées, qu’ils en mirent aux ancres de leurs navires. Il ne faut point faire de fond sur ces récits populaires : voici des faits précis.\par
On voit, dans un fragment de Polybe cité par Strabon\footnote{Liv. III.}, que les mines d’argent qui étaient à la source du Bétis, où quarante mille hommes étaient employés, donnaient au peuple romain vingt-cinq mille dragmes par jour : cela peut faire environ cinq millions de livres par an, {\itshape à} cinquante francs le marc. On appelait les montagnes où étaient ces mines, les montagnes d’argent\footnote{{\itshape Mons argentarius}.} ; ce qui fait voir que c’était le Potosi de ces temps-là. Aujourd’hui les mines d’Hanovre n’ont pas le quart des ouvriers qu’on employait dans celles d’Espagne, et elles donnent plus : mais les Romains n’ayant guère que des mines de cuivre, et peu de mines d’argent, et les Grecs ne connaissant que les mines d’Attique, très peu riches, ils durent être étonnés de l’abondance de celles-là.\par
Dans la guerre pour la succession d’Espagne, un homme appelé le marquis de Rhodes, de qui on disait qu’il s’était ruiné dans les mines d’or, et enrichi dans les hôpitaux\footnote{Il en avait eu quelque part la direction.}, proposa à la cour de France d’ouvrir les mines des Pyrénées. Il cita les Tyriens, les Carthaginois et les Romains : on lui permit de chercher ; il chercha, il fouilla partout ; il citait toujours, et ne trouvait rien.\par
Les Carthaginois, maîtres du commerce de l’or et de l’argent, voulurent l’être encore de celui du plomb et de l’étain. Ces métaux étaient voiturés par terre, depuis les ports de la Gaule sur l’Océan jusqu’à ceux de la Méditerranée. Les Carthaginois voulurent les recevoir de la première main ; ils envoyèrent Himilcon, pour former\footnote{Voyez Festus Avienus.} des établissements dans les îles Cassitérides, qu’on croit être celles de Silley.\par
Ces voyages de la Bétique en Angleterre ont fait penser à quelques gens que les Carthaginois avaient la boussole, mais il est clair qu’ils suivaient les côtes. Je n’en veux d’autre preuve que ce que dit Himilcon, qui demeura quatre mois à aller de l’embouchure du Bétis en Angleterre : outre que la fameuse histoire\footnote{Strabon, liv. III, sur la fin.} de ce pilote carthaginois, qui, voyant venir un vaisseau romain, se fit échouer pour ne lui pas apprendre la route d’Angleterre\footnote{Il en fut récompensé par le sénat de Carthage.}, fait voir que ces vaisseaux étaient très près des côtes lorsqu’ils se rencontrèrent.\par
Les anciens pourraient avoir fait des voyages de mer qui feraient penser qu’ils avaient la boussole, quoiqu’ils ne l’eussent pas. Si un pilote s’était éloigné des côtes, et que pendant son voyage, il eût eu un temps serein, que, la nuit, il eût toujours vu une étoile polaire, et, le jour, le lever et le coucher du soleil, il est clair qu’il aurait pu se conduire comme on fait aujourd’hui par la boussole ; mais ce serait un cas fortuit, et non pas une navigation réglée.\par
On voit, dans le traité qui finit la première guerre punique, que Carthage fut principalement attentive à se conserver l’empire de la mer, et Rome à garder celui de la terre. Hannon\footnote{Tite-Live, supplément de Freinshemius, seconde décade, liv. VI.}, dans la négociation avec les Romains, déclara qu’il ne souffrirait pas seulement qu’ils se lavassent les mains dans les mers de Sicile ; il ne leur fut pas permis de naviguer au-delà du Beau promontoire, il leur fut défendu\footnote{Polybe, liv. III.} de trafiquer en Sicile\footnote{Dans la partie sujette aux Carthaginois.}, en Sardaigne, en Afrique, excepté à Carthage exception qui fait voir qu’on ne leur y préparaît pas un commerce avantageux.\par
Il y eut, dans les premiers temps, de grandes guerres entre Carthage et Marseille\footnote{Justin, liv. XLIII, chap. V.} au sujet de la pêche. Après la paix, ils firent concurremment le commerce d’économie. Marseille fut d’autant plus jalouse, qu’égalant sa rivale en industrie, elle lui était devenue inférieure en puissance : voilà la raison de cette grande fidélité pour les Romains. La guerre que ceux-ci firent contre les Carthaginois en Espagne, fut une source de richesses pour Marseille, qui servait d’entrepôt. La ruine de Carthage et de Corinthe augmenta encore la gloire de Marseille ; et, sans les guerres civiles, où il fallait fermer les yeux et prendre un parti, elle aurait été heureuse sous la protection des Romains, qui n’avaient aucune jalousie de son commerce.
\subsubsection[{Chapitre XII. Île de Délos. Mithridate}]{Chapitre XII. Île de Délos. Mithridate}
\noindent Corinthe ayant été détruite par les Romains, les marchands se retirèrent à Délos. La religion et la vénération des peuples faisaient regarder cette île comme un lieu de sûreté\footnote{Voyez Strabon, liv. X.} : de plus, elle était très bien située pour le commerce de l’Italie et de l’Asie, qui, depuis l’anéantissement de l’Afrique et l’affaiblissement de la Grèce, était devenu plus important.\par
Dès les premiers temps, les Grecs envoyèrent, comme nous avons dit, des colonies sur la Propontide et le Pont-Euxin : elles conservèrent, sous les Perses, leurs lois et leur liberté. Alexandre, qui n’était parti que contre les Barbares, ne les attaqua pas\footnote{Il confirma la liberté de la ville d’Amise, colonie athénienne, qui avait joui de l’État populaire, même sous les rois de Perse. Lucullus, qui prit Sinope et Amise, leur rendit la liberté, et rappela les habitants qui s’étaient enfuis sur leurs vaisseaux.}. Il ne paraît pas même que les rois de Pont, qui en occupèrent plusieurs, leur eussent\footnote{Voyez ce qu’écrit Appien sur les Phanagoréens, les Amisiens, les Synopiens, dans son livre {\itshape De la guerre contre Mithridate}.} ôté leur gouvernement politique.\par
La puissance de ces rois augmenta sitôt qu’ils les eurent soumises\footnote{Voyez Appien, sur les trésors immenses que Mithridate employa dans ses guerres, ceux qu’il avait cachés, ceux qu’il perdit si souvent par la trahison des siens, ceux qu’on trouva après sa mort.}. Mithridate se trouva en état d’acheter partout des troupes ; de réparer\footnote{Il perdit une fois 170 000 hommes, et de nouvelles armées reparurent d’abord.} continuellement ses pertes ; d’avoir des ouvriers, des vaisseaux, des machines de guerre ; de se procurer des alliés ; de corrompre ceux des Romains, et les Romains même ; de soudoyer\footnote{Voyez Appien, De la guerre contre Mithridate.} les barbares de l’Asie et de l’Europe ; de faire la guerre longtemps, et par conséquent de discipliner ses troupes : il put les armer, et les instruire dans l’art militaire\footnote{{\itshape Ibid.}} des Romains, et former des corps considérables de leurs transfuges ; enfin il put faire de grandes perles et souffrir de grands échecs, sans périr ; et il n’aurait point péri, si, dans les prospérités, le roi voluptueux et barbare n’avait pas détruit ce que, dans la mauvaise fortune, avait fait le grand prince.\par
C’est ainsi que, dans le temps que les Romains étaient au comble de la grandeur, et qu’ils semblaient n’avoir à craindre qu’eux-mêmes, Mithridate remit en question ce que la prise de Carthage, les défaites de Philippe, d’Antiochus et de Persée avaient décidé. Jamais guerre ne fut plus funeste : et les deux partis ayant une grande puissance et des avantages mutuels, les peuples de la Grèce et de l’Asie furent détruits, ou comme amis de Mithridate, ou comme ses ennemis. Délos fut enveloppée dans le malheur commun. Le commerce tomba de toutes parts ; il fallait bien qu’il fût détruit, les peuples l’étaient.\par
Les Romains, suivant un système dont j’ai parlé ailleurs\footnote{Dans les Considérations sur les causes de la grandeur des Romains.}, destructeurs pour ne pas paraître conquérants, ruinèrent Carthage et Corinthe ; et, par une telle pratique, ils se seraient peut-être perdus, s’ils n’avaient pas conquis toute la terre. Quand les rois de Pont se rendirent maîtres des colonies grecques du Pont-Euxin, ils n’eurent garde de détruire ce qui devait être la cause de leur grandeur.
\subsubsection[{Chapitre XIII. Du génie des Romains pour la marine}]{Chapitre XIII. Du génie des Romains pour la marine}
\noindent Les Romains ne faisaient cas que des troupes de terre, dont l’esprit était de rester toujours ferme, de combattre au même lieu, et d’y mourir. Ils ne pouvaient estimer la pratique des gens de mer, qui se présentent au combat, fuient, reviennent, évitent toujours le danger, emploient souvent la ruse, rarement la force. Tout cela n’était point du génie des Grecs\footnote{Comme l’a remarqué Platon, liv. IV des Lois.}, et était encore moins de celui des Romains.\par
Ils ne destinaient donc à la marine que ceux qui n’étaient pas des citoyens assez considérables\footnote{Polybe, liv. V.} pour avoir place dans les légions : les gens de mer étaient ordinairement des affranchis.\par
Nous n’avons aujourd’hui ni la même estime pour les troupes de terre, ni le même mépris pour celles de mer. Chez les premières\footnote{Voyez les Considérations sur les causes de la grandeur des Romains, etc.}, l’art est diminué ; chez les secondes\footnote{{\itshape Ibid.}}, il est augmenté : or on estime les choses à proportion du degré de suffisance qui est requis pour les bien faire.
\subsubsection[{Chapitre XIV. Du génie des Romains pour le commerce}]{Chapitre XIV. Du génie des Romains pour le commerce}
\noindent On n’a jamais remarqué aux Romains de jalousie sur le commerce. Ce fut comme nation rivale, et non comme nation commerçante, qu’ils attaquèrent Carthage. Ils favorisèrent les villes qui faisaient le commerce, quoiqu’elles ne fussent pas sujettes : ainsi ils augmentèrent, par la cession de plusieurs pays, la puissance de Marseille. Ils craignaient tout des barbares, et rien d’un peuple négociant. D’ailleurs, leur génie, leur gloire, leur éducation militaire, la forme de leur gouvernement, les éloignaient du commerce.\par
Dans la ville, on n’était occupé que de guerres, d’élections, de brigues et de procès ; à la campagne, que d’agriculture ; et dans les provinces, un gouvernement dur et tyrannique était incompatible avec le commerce.\par
Que si leur constitution politique y était opposée, leur droit des gens n’y répugnait pas moins. « Les peuples, dit le jurisconsulte Pomponius\footnote{Leg. 5, § 2, ff.{\itshape  De captivis}.}, avec lesquels nous n’avons ni amitié, ni hospitalité, ni alliance, ne sont point nos ennemis : cependant, si une chose qui nous appartient tombe entre leurs mains, ils en sont propriétaires, les hommes libres deviennent leurs esclaves ; et ils sont dans les mêmes termes à notre égard. »\par
Leur droit civil n’était pas moins accablant. La loi de Constantin, après avoir déclaré bâtards les enfants des personnes viles qui se sont mariées avec celles d’une condition relevée, confond les femmes qui ont une boutique\footnote{{\itshape Quae mercimoniis publice praefuit}. Leg. 5, cod. {\itshape De natural. liberis}.} de marchandises avec les esclaves, les cabaretières, les femmes de théâtre, les filles d’un homme qui tient un lieu de prostitution, ou qui a été condamné à combattre sur l’arène. Ceci descendait des anciennes institutions des Romains.\par
Je sais bien que des gens pleins de ces deux idées : l’une, que le commerce est la chose du monde la plus utile à un État, et l’autre, que les Romains avaient la meilleure police du monde, ont cru qu’ils avaient beaucoup encouragé et honoré le commerce ; mais la vérité est qu’ils y ont rarement pensé.
\subsubsection[{Chapitre XV. Commerce des Romains avec les Barbares}]{Chapitre XV. Commerce des Romains avec les Barbares}
\noindent Les Romains avaient fait de l’Europe, de l’Asie et de l’Afrique un vaste empire : la faiblesse des peuples et la tyrannie du commandement unirent toutes les parties de ce corps immense. Pour lors, la politique romaine fut de se séparer de toutes les nations qui n’avaient pas été assujetties : la crainte de leur porter l’art de vaincre fit négliger l’art de s’enrichir. Ils firent des lois pour empêcher tout commerce avec les Barbares. « Que personne, disent Valens et Gratien\footnote{Leg. {\itshape ad Barbaricum}, cod. {\itshape Quae res exportari non debeant.}}, n’envoie du vin, de l’huile ou d’autres liqueurs aux Barbares, même pour en goûter. Qu’on ne leur porte point de l’or, ajoutent Gratien, Valentinien et Théodose\footnote{Leg. 2, cod. {\itshape De commercio et mercateribus}.} ; et que même ce qu’ils en ont, on le leur ôte avec finesse. » Le transport du fer fut défendu sous peine de la vie\footnote{Leg. 2, {\itshape Quae res exportari non debeant}.}.\par
Domitien, prince timide, fit arracher les vignes dans la Gaule\footnote{Procope, {\itshape Guerre des Perses}, liv. I.}, de crainte sans doute que cette liqueur n’y attirât les Barbares, comme elle les avait autrefois attirés en Italie. Probus et Julien, qui ne les redoutèrent jamais, en rétablirent la plantation.\par
Je sais bien que, dans la faiblesse de l’empire, les Barbares obligèrent les Romains d’établir des étapes\footnote{Voyez les {\itshape Considérations sur les causes de la grandeur des Romains et de leur décadence}.}, et de commercer avec eux. Mais cela même prouve que l’esprit des Romains était de ne pas commercer.
\subsubsection[{Chapitre XVI. Du commerce des Romains avec l’Arabie et les Indes}]{Chapitre XVI. Du commerce des Romains avec l’Arabie et les Indes}
\noindent Le négoce de l’Arabie heureuse et celui des Indes furent les deux branches, et presque les seules, du commerce extérieur. Les Arabes avaient de grandes richesses : ils les tiraient de leurs mers et de leurs forêts ; et, comme ils achetaient peu, et vendaient beaucoup, ils attiraient\footnote{Pline, liv. VI, chap. XXVIII ; et Strabon, liv. XVI.} à eux l’or et l’argent de leurs voisins. Auguste\footnote{{\itshape Ibid.}} connut leur opulence, et il résolut de les avoir pour amis, ou pour ennemis. Il fit passer Élius Gallus d’Égypte en Arabie. Celui-ci trouva des peuples oisifs, tranquilles et peu aguerris. Il donna des batailles, fit des sièges, et ne perdit que sept soldats ; mais la perfidie de ses guides, les marches, le climat, la faim, la soif, les maladies, des mesures mal prises, lui firent perdre son armée.\par
Il fallut donc se contenter de négocier avec les Arabes, comme les autres peuples avaient fait, c’est-à-dire de leur porter de l’or et de l’argent pour leurs marchandises. On commerce encore avec eux de la même manière ; la caravane d’Alep et le vaisseau royal de Suez y portent des sommes immenses\footnote{Les caravanes d’Alep et de Suez y portent deux millions de notre monnaie, et il en passe autant en fraude ; le vaisseau royal de Suez y porte aussi deux millions.}.\par
La nature avait destiné les Arabes au commerce ; elle ne les avait pas destinés à la guerre ; mais lorsque ces peuples tranquilles se trouvèrent sur les frontières des Parthes et des Romains, ils devinrent auxiliaires des uns et des autres. Élius Gallus les avait trouvés commerçants ; Mahomet les trouva guerriers : il leur donna de l’enthousiasme, et les voilà conquérants.\par
Le commerce des Romains aux Indes était considérable. Strabon\footnote{Liv. II, p. 81, éd. de l’année 1587.} avait appris en Égypte qu’ils y employaient cent vingt navires : ce commerce ne se soutenait encore que par leur argent. Ils y envoyaient tous les ans cinquante millions de sesterces. Pline\footnote{Liv. VI, chap. XXIII.} dit que les marchandises qu’on en rapportait se vendaient à Rome le centuple. Je crois qu’il parle trop généralement : ce profit fait une fois, tout le monde aura voulu le faire ; et, dès ce moment, personne ne l’aura fait.\par
On peut mettre en question s’il fut avantageux aux Romains de faire le commerce de l’Arabie et des Indes. Il fallait qu’ils y envoyassent leur argent, et ils n’avaient pas, comme nous, la ressource de l’Amérique, qui supplée à ce que nous envoyons. Je suis persuadé qu’une des raisons qui fit augmenter chez eux la valeur numéraire des monnaies, c’est-à-dire établir le billon, fut la rareté de l’argent, causée par le transport continuel qui s’en faisait aux Indes. Que si les marchandises de ce pays se vendaient à Rome le centuple, ce profit des Romains se faisait sur les Romains mêmes, et n’enrichissait point l’empire.\par
On pourra dire, d’un autre côté, que ce commerce procurait aux Romains une grande navigation, c’est-à-dire une grande puissance ; que des marchandises nouvelles augmentaient le commerce intérieur, favorisaient les arts, entretenaient l’industrie ; que le nombre des citoyens se multipliait à proportion des nouveaux moyens qu’on avait de vivre ; que ce nouveau commerce produisait le luxe, que nous avons prouvé être aussi favorable au gouvernement d’un seul que fatal à celui de plusieurs ; que cet établissement fut de même date que la chute de leur république ; que le luxe à Rome était nécessaire ; et qu’il fallait bien qu’une ville qui attirait à elle toutes les richesses de l’univers, les rendit par son luxe.\par
Strabon\footnote{Il dit, au liv. XII, que les Romains y employaient cent vingt navires ; et au liv. XVII, que les rois grecs y en envoyaient à peine vingt.} dit que le commerce des Romains aux Indes était beaucoup plus considérable que celui des rois d’Égypte ; et il est singulier que les Romains, qui connaissaient peu le commerce, aient eu pour celui des Indes plus d’attention que n’en eurent les rois d’Égypte, qui l’avaient, pour ainsi dire, sous les yeux. Il faut expliquer ceci.\par
Après la mort d’Alexandre, les rois d’Égypte établirent aux Indes un commerce maritime ; et les rois de Syrie, qui eurent les provinces les plus orientales de l’empire, et par conséquent les Indes, maintinrent ce commerce, dont nous avons parlé au chapitre VI, qui se faisait par les terres et par les fleuves, et qui avait reçu de nouvelles facilités par l’établissement des colonies macédoniennes ; de sorte que l’Europe communiquait avec les Indes, et par l’Égypte, et par le royaume de Syrie. Le démembrement qui se fit du royaume de Syrie, d’où se forma celui de Bactriane, ne fit aucun tort à ce commerce. Marin, Tyrien cité par Ptolomée\footnote{Liv. I, chap. II.}, parle des découvertes faites aux Indes par le moyen de quelques marchands macédoniens. Celles que les expéditions des rois n’avaient pas faites, les marchands les firent. Nous voyons, dans Ptolomée\footnote{Liv. VI, chap. XIII.} qu’ils allèrent depuis la tour de Pierre\footnote{Nos meilleures cartes placent la tour de Pierre au centième degré de longitude, et environ le quarantième de latitude.} jusqu’à Séra : et la découverte faite par les marchands d’une étape si reculée, située dans la partie orientale et septentrionale de la Chine, fut une espèce de prodige. Ainsi, sous les rois de Syrie et de Bactriane, les marchandises du midi de l’Inde passaient par l’Indus, l’Oxus et la mer Caspienne, en occident ; et celles des contrées plus orientales et plus septentrionales étaient portées depuis Séra, la tour de Pierre et autres étapes, jusqu’à l’Euphrate. Ces marchands faisaient leur route, tenant, à peu près, le quarantième degré de latitude nord, par des pays qui sont au couchant de la Chine, plus policés qu’ils ne sont aujourd’hui, parce que les Tartares ne les avaient pas encore infestés.\par
Or, pendant que l’empire de Syrie étendait si fort son commerce du côté des terres, l’Égypte n’augmenta pas beaucoup son commerce maritime.\par
Les Parthes parurent et fondèrent leur empire ; et, lorsque l’Égypte tomba sous la puissance des Romains, cet empire était dans sa force, et avait reçu son extension.\par
Les Romains et les Parthes furent deux puissances rivales, qui combattirent, non pas pour savoir qui devait régner, mais exister. Entre les deux empires, il se forma des déserts ; entre les deux empires, on fut toujours sous les armes ; bien loin qu’il y eût du commerce, il n’y eut pas même de communication. L’ambition, la jalousie, la religion, la haine, les mœurs, séparèrent tout. Ainsi le commerce entre l’Occident et l’Orient, qui avaient eu plusieurs routes, n’en eut plus qu’une ; et Alexandrie étant devenue la seule étape, cette étape grossit.\par
Je ne dirai qu’un mot du commerce intérieur. Sa branche principale fut celle des blés qu’on faisait venir pour la subsistance du peuple de Rome : ce qui était une matière de police plutôt qu’un objet de commerce. À cette occasion, les nautoniers reçurent quelques privilèges\footnote{Suétone, {\itshape in Claudio}. Leg. 7, Cod. Théodos. {\itshape De naviculariis}.}, parce que le salut de l’empire dépendait de leur vigilance.
\subsubsection[{Chapitre XVII. Du commerce après la destruction des Romains en Occident}]{Chapitre XVII. Du commerce après la destruction des Romains en Occident}
\noindent L’empire romain fut envahi ; et l’un des effets de la calamité générale fut la destruction du commerce. Les Barbares ne le regardèrent d’abord que comme un objet de leurs brigandages ; et, quand ils furent établis, ils ne l’honorèrent pas plus que l’agriculture et les autres professions du peuple vaincu.\par
Bientôt il n’y eut presque plus de commerce en Europe ; la noblesse, qui régnait partout, ne s’en mettait point en peine.\par
La loi des Wisigoths\footnote{Liv. VIII, tit. IV, § 9.} permettait aux particuliers d’occuper la moitié du lit des grands fleuves, pourvu que l’autre restât libre pour les filets et pour les bateaux ; il fallait qu’il y eût bien peu de commerce dans les pays qu’ils avaient conquis.\par
Dans ces temps-là s’établirent les droits insensés d’aubaine et de naufrage : les hommes pensèrent que les étrangers ne leur étant unis par aucune communication du droit civil, ils ne leur devaient, d’un côté, aucune sorte de justice et, de l’autre, aucune sorte de pitié.\par
Dans les bornes étroites où se trouvaient les peuples du nord, tout leur était étranger : dans leur pauvreté, tout était pour eux un objet de richesse. Établis, avant leurs conquêtes, sur les côtes d’une mer resserrée et pleine d’écueils, ils avaient tiré parti de ces écueils mêmes.\par
Mais les Romains, qui faisaient des lois pour tout l’univers, en avaient fait de très humaines sur les naufrages\footnote{{\itshape Toto titulo}, ff.{\itshape  De incend. ruin. naufrag.} et Cod. {\itshape De naufragiis} ; et leg. 3 ff.{\itshape  de leg. Cornel. De sicaris}.} {\itshape : ils} réprimèrent, à cet égard, les brigandages de ceux qui habitaient les côtes et, ce qui était plus encore, la rapacité de leur fisc\footnote{L. 1, Cod {\itshape De naufragiis}.}.
\subsubsection[{Chapitre XVIII. Règlement particulier}]{Chapitre XVIII. Règlement particulier}
\noindent La loi des Wisigoths\footnote{Liv. XI, tit. III, § 2.} fit pourtant une disposition favorable au commerce ; elle ordonna que les marchands qui venaient de delà la mer seraient jugés, dans les différends qui naissaient entre eux, par les lois et par des juges de leur nation. Ceci était fondé sur l’usage établi chez tous ces peuples mêlés, que chaque homme vécût sous sa propre loi : chose dont je parlerai beaucoup dans la suite.
\subsubsection[{Chapitre XIX. Du commerce depuis l’affaiblissement des Romains en Orient}]{Chapitre XIX. Du commerce depuis l’affaiblissement des Romains en Orient}
\noindent Les mahométans parurent, conquirent et se divisèrent. L’Égypte eut ses souverains particuliers ; elle continua de faire le commerce des Indes. Maîtresse des marchandises de ce pays, elle attira les richesses de tous les autres. Ses soudans furent les plus puissants princes de ces temps-là : on peut voir dans l’histoire, comment, avec une force constante et bien ménagée, ils arrêtèrent l’ardeur, la fougue et l’impétuosité des croisés.
\subsubsection[{Chapitre XX. Comment le commerce se fit jour en Europe à travers la barbarie}]{Chapitre XX. Comment le commerce se fit jour en Europe à travers la barbarie}
\noindent La philosophie d’Aristote ayant été portée en Occident, elle plut beaucoup aux esprits subtils qui, dans les temps d’ignorance, sont les beaux esprits. Des scolastiques s’en infatuèrent, et prirent de ce philosophe\footnote{Voyez Aristote, {\itshape Politique}, liv. I, chap. IX et X.} bien des explications sur le prêt à intérêt, au lieu que la source en était si naturelle dans l’Évangile ; ils le condamnèrent indistinctement et dans tous les cas. Par là, le commerce, qui n’était que la profession des gens vils, devint encore celle des malhonnêtes gens : car toutes les fois que l’on défend une chose naturellement permise ou nécessaire, on ne fait que rendre malhonnêtes gens ceux qui la font.\par
Le commerce passa à une nation pour lors couverte d’infamie ; et bientôt il ne fut plus distingué des usures les plus affreuses, des monopoles, de la levée des subsides et de tous les moyens malhonnêtes d’acquérir de l’argent.\par
Les Juifs\footnote{Voyez, dans Marca Hispanica, les constitutions d’Aragon des années 1228 et 1231 ; et dans Brussel, l’accord de l’année 1206, passé entre le roi, la comtesse de Champagne et Guy de Dampierre.}, enrichis par leurs exactions, étaient pillés par les princes avec la même tyrannie : chose qui consolait les peuples, et ne les soulageait pas.\par
Ce qui se passa en Angleterre donnera une idée de ce qu’on fit dans les autres pays. Le roi Jean\footnote{Slowe, {\itshape in his Survey of London}, liv. III, p. 54.} ayant fait emprisonner les Juifs pour avoir leur bien, il y en eut peu qui n’eussent au moins quelque œil crevé : ce roi faisait ainsi sa chambre de justice. Un d’eux, à qui on arracha sept dents, une chaque jour, donna dix mille marcs d’argent à la huitième. Henri III tira d’Aaron, juif d’York, quatorze mille marcs d’argent, et dix mille pour la reine. Dans ces temps-là, on faisait violemment ce qu’on fait aujourd’hui en Pologne avec quelque mesure. Les rois ne pouvant fouiller dans la bourse de leurs sujets, à cause de leurs privilèges, mettaient à la torture les Juifs, qu’on ne regardait pas comme citoyens.\par
Enfin il s’introduisit une coutume qui confisqua tous les biens des Juifs qui embrassaient le christianisme. Cette coutume si bizarre, nous la savons par la loi\footnote{Édit donné à Bâville le 4 avril 1392.} qui l’abroge. On en a donné des raisons bien vaines ; on a dit qu’on voulait les éprouver, et faire en sorte qu’il ne restât rien de l’esclavage du démon. Mais il est visible que cette confiscation était une espèce de droit\footnote{En France, les Juifs étaient serfs, mainmortables, et les seigneurs leur succédaient. M. Brussel rapporte un accord de l’an 1206, entre le roi et Thibaut comte de Champagne, par lequel il était convenu que les Juifs de l’un ne prêteraient point dans les terres de l’autre.} d’amortissement, pour le prince ou pour les seigneurs, des taxes qu’ils levaient sur les Juifs, et dont ils étaient frustrés lorsque ceux-ci embrassaient le christianisme. Dans ces temps-là, on regardait les hommes comme des terres. Et je remarquerai, en passant, combien on s’est joué de cette nation d’un siècle à l’autre. On confisquait leurs biens lorsqu’ils voulaient être chrétiens ; et, bientôt après, on les fit brûler lorsqu’ils ne voulurent pas l’être.\par
Cependant on vit le commerce sortir du sein de la vexation et du désespoir. Les Juifs, proscrits tour à tour de chaque pays, trouvèrent le moyen de sauver leurs effets. Par là ils rendirent pour jamais leurs retraites fixes ; car tel prince, qui voudrait bien se défaire d’eux, ne serait pas pour cela d’humeur à se défaire de leur argent.\par
Ils\footnote{On sait que, sous Philippe Auguste et sous Philippe le Long, les Juifs, chassés de France, se réfugièrent en Lombardie, et que là ils donnèrent aux négociants étrangers et aux voyageurs, des lettres secrètes sur ceux à qui ils avaient confié leurs effets en France, qui furent acquittées.} inventèrent les lettres de change ; et, par ce moyen, le commerce put éluder la violence, et se maintenir partout ; le négociant le plus riche n’ayant que des biens invisibles, qui pouvaient être envoyés partout, et ne laissaient de trace nulle part.\par
Les théologiens furent obligés de restreindre leurs principes ; et le commerce, qu’on avait violemment lié avec la mauvaise foi, rentra, pour ainsi dire, dans le sein de la probité.\par
Ainsi nous devons aux spéculations des scolastiques tous les malheurs\footnote{Voyez dans le corps du droit, la quatre-vingt-troisième {\itshape Novelle} de Léon, qui révoque la loi de Basile son père. Cette loi de Basile est dans Harménopule sous le nom de Léon, livre III, tit. VII, § 27.} qui ont accompagné la destruction du commerce ; et à l’avarice des princes, l’établissement d’une chose qui le met en quelque façon hors de leur pouvoir.\par
Il a fallu, depuis ce temps, que les princes se gouvernassent avec plus de sagesse qu’ils n’auraient eux-mêmes pensé : car, par l’événement, les grands coups d’autorité se sont trouvés si maladroits, que c’est une expérience reconnue, qu’il n’y a plus que la bonté du gouvernement qui donne de la prospérité.\par
On a commencé à se guérir du machiavélisme, et on s’en guérira tous les jours. Il faut plus de modération dans les conseils. Ce qu’on appelait autrefois des coups d’État ne serait aujourd’hui, indépendamment de l’horreur, que des imprudences.\par
Et il est heureux pour les hommes d’être dans une situation où, pendant que leurs passions leur inspirent la pensée d’être méchants, ils ont pourtant intérêt de ne pas l’être.
\subsubsection[{Chapitre XXI. Découverte de deux nouveaux mondes : état de l’Europe à cet égard}]{Chapitre XXI. Découverte de deux nouveaux mondes : état de l’Europe à cet égard}
\noindent La boussole ouvrit, pour ainsi dire, l’univers. On trouva l’Asie et l’Afrique, dont on ne connaissait que quelques bords, et l’Amérique, dont on ne connaissait rien du tout.\par
Les Portugais, naviguant sur l’océan Atlantique, découvrirent la pointe la plus méridionale de l’Afrique : ils virent une vaste mer ; elle les porta aux Indes orientales. Leurs périls sur cette mer, et la découverte de Mozambique, de Mélinde et de Calicut, ont été chantés par le Camoëns dont le poème fait sentir quelque chose des charmes de {\itshape l’Odyssée} et de la magnificence de {\itshape l’Énéide}.\par
Les Vénitiens avaient fait jusque-là le commerce des Indes par les pays des Turcs, et l’avaient poursuivi au milieu des avanies et des outrages. Par la découverte du cap de Bonne-Espérance, et celles qu’on fit quelque temps après, l’Italie ne fut plus au centre du monde commerçant ; elle fut, pour ainsi dire, dans un coin de l’univers, et elle y est encore. Le commerce même du Levant dépendant aujourd’hui de celui que les grandes nations font aux deux Indes, l’Italie ne le fait plus qu’accessoirement.\par
Les Portugais trafiquèrent aux Indes en conquérants. Les lois gênantes\footnote{Voyez la relation de François Pyrard, deuxième partie, chap. XV.} que les Hollandais imposent aujourd’hui aux petits princes indiens sur le commerce, les Portugais les avaient établies avant eux.\par
La fortune de la maison d’Autriche fut prodigieuse. Charles-Quint recueillit la succession de Bourgogne, de Castille et d’Aragon ; il parvint à l’empire ; et, pour lui procurer un nouveau genre de grandeur, l’univers s’étendit, et l’on vit paraître un monde nouveau sous son obéissance.\par
Christophe Colomb découvrit l’Amérique ; et, quoique l’Espagne n’y envoyât point de forces qu’un petit prince de l’Europe n’eût pu y envoyer tout de même, elle soumit deux grands empires et d’autres grands États.\par
Pendant que les Espagnols découvraient et conquéraient du côté de l’Occident, les Portugais poussaient leurs conquêtes et leurs découvertes du côté de l’Orient : ces deux nations se rencontrèrent ; elles eurent recours au pape Alexandre VI, qui fit la célèbre ligne de démarcation, et jugea un grand procès.\par
Mais les autres nations de l’Europe ne les laissèrent pas jouir tranquillement de leur partage : les Hollandais chassèrent les Portugais de presque toutes les Indes orientales, et diverses nations firent en Amérique des établissements.\par
Les Espagnols regardèrent d’abord les terres découvertes comme des objets de conquête : des peuples plus raffinés qu’eux trouvèrent qu’elles étaient des objets de commerce, et c’est là-dessus qu’ils dirigèrent leurs vues. Plusieurs peuples se sont conduits avec tant de sagesse, qu’ils ont donné l’empire à des compagnies de négociants, qui, gouvernant ces États éloignés uniquement pour le négoce, ont fait une grande puissance accessoire, sans embarrasser l’État principal.\par
Les colonies qu’on y a formées sont sous un genre de dépendance dont on ne trouve que peu d’exemples dans les colonies anciennes, soit que celles d’aujourd’hui relèvent de l’État même, ou de quelque compagnie commerçante établie dans cet État.\par
L’objet de ces colonies est de faire le commerce à de meilleures conditions qu’on ne le fait avec les peuples voisins, avec lesquels tous les avantages sont réciproques. On a établi que la métropole seule pourrait négocier dans la colonie ; et cela avec grande raison, parce que le but de l’établissement a été l’extension du commerce, non la fondation d’une ville ou d’un nouvel empire.\par
Ainsi, c’est encore une loi fondamentale de l’Europe, que tout commerce avec une colonie étrangère est regardé comme un pur monopole punissable par les lois du pays : et il ne faut pas juger de cela par les lois et les exemples des anciens\footnote{Excepté les Carthaginois, comme on voit par le truité qui termina la première guerre punique.} peuples, qui n’y sont guère applicables.\par
Il est encore reçu que le commerce établi entre les métropoles n’entraîne point une permission pour les colonies, qui restent toujours en état de prohibition.\par
Le désavantage des colonies, qui perdent la liberté du commerce, est visiblement compensé par la protection de la métropole\footnote{{\itshape Métropole} est, dans le langage des anciens, l’État qui a fondé la colonie.}, qui la défend par ses armes, ou la maintient par ses lois.\par
De là suit une troisième loi de l’Europe, que, quand le commerce étranger est défendu avec la colonie, on ne peut naviguer dans ses mers que dans les cas établis par les traités.\par
Les nations, qui sont à l’égard de tout l’univers ce que les particuliers sont dans un État, se gouvernent comme eux par le droit naturel et par les lois qu’elles se sont faites. Un peuple peut céder à un autre la mer, comme il peut céder la terre. Les Carthaginois exigèrent\footnote{Polybe, liv. III.} des Romains qu’ils ne navigueraient pas au-delà de certaines limites, comme les Grecs avaient exigé du roi de Perse qu’il se tiendrait toujours éloigné des côtes de la mer\footnote{Le roi de Perse s’obligea, par un traité, de ne naviguer avec aucun vaisseau de guerre au-delà des roches Scyanées et des îles Chélidoniennes. Plutarque, {\itshape Vie de Cimon}.} de la carrière d’un cheval.\par
L’extrême éloignement de nos colonies n’est point un inconvénient pour leur sûreté : car, si la métropole est éloignée pour les défendre, les nations rivales de la métropole ne sont pas moins éloignées pour les conquérir.\par
De plus, cet éloignement fait que ceux qui vont s’y établir ne peuvent prendre la manière de vivre d’un climat si différent ; ils sont obligés de tirer toutes les commodités de la vie du pays d’où ils sont venus. Les Carthaginois\footnote{Aristote, {\itshape Des choses merveilleuses}. Tite-Live, liv. VII de la seconde décade.} pour rendre les Sardes et les Corses plus dépendants, leur avaient défendu, sous peine de la vie, de planter, de semer et de faire rien de semblable ; ils leur envoyaient d’Afrique des vivres. Nous sommes parvenus au même point, sans faire des lois si dures. Nos colonies des îles Antilles sont admirables ; elles ont des objets de commerce que nous n’avons ni ne pouvons avoir ; elles manquent de ce qui fait l’objet du nôtre.\par
L’effet de la découverte de l’Amérique fut de lier à l’Europe l’Asie et l’Afrique. L’Amérique fournit à l’Europe la matière de son commerce avec cette vaste partie de l’Asie qu’on appela les Indes orientales. L’argent, ce métal si utile au commerce, comme signe, fut encore la base du plus grand commerce de l’univers, comme marchandise. Enfin la navigation d’Afrique devint nécessaire ; elle fournissait des hommes pour le travail des mines et des terres de l’Amérique.\par
L’Europe est parvenue à un si haut degré de puissance, que l’histoire n’a rien à comparer là-dessus, si l’on considère l’immensité des dépenses, la grandeur des engagements, le nombre des troupes et la continuité de leur entretien, même lorsqu’elles sont le plus inutiles, et qu’on ne les a que pour l’ostentation.\par
Le père Du Halde\footnote{T. II, p. 170.} dit que le commerce intérieur de la Chine est plus grand que celui de toute l’Europe. Cela pourrait être, si notre commerce extérieur n’augmentait pas l’intérieur. L’Europe fait le commerce et la navigation des trois autres parties du monde ; comme la France, l’Angleterre et la Hollande font à peu près la navigation et le commerce de l’Europe.
\subsubsection[{Chapitre XXII. Des richesses que l’Espagne tira de l’Amérique}]{Chapitre XXII. Des richesses que l’Espagne tira de l’Amérique}
\noindent Si l’Europe\footnote{Ceci parut, il y a plus de vingt ans, dans un petit ouvrage manuscrit de l’auteur, qui a été presque tout fondu dans celui-ci.} a trouvé tant d’avantages dans le commerce de l’Amérique, il serait naturel de croire que l’Espagne en aurait reçu de plus grands. Elle tira du monde nouvellement découvert une quantité d’or et d’argent si prodigieuse, que ce que l’on en avait eu jusqu’alors ne pouvait y être comparé.\par
Mais (ce qu’on n’aurait jamais soupçonné) la misère la fit échouer presque partout. Philippe II, qui succéda à Charles-Quint, fut obligé de faire la célèbre banqueroute que tout le monde sait ; et il n’y a guère jamais eu de prince qui ait plus souffert que lui des murmures, de l’insolence et de la révolte de ses troupes toujours mal payées.\par
Depuis ce temps, la monarchie d’Espagne déclina sans cesse. C’est qu’il y avait un vice intérieur et physique dans la nature de ces richesses, qui les rendait vaines ; et ce vice augmenta tous les jours.\par
L’or et l’argent sont une richesse de fiction ou de signe. Ces signes sont très durables et se détruisent peu, comme il convient à leur nature. Plus ils se multiplient, plus ils perdent de leur prix, parce qu’ils représentent moins de choses.\par
Lors de la conquête du Mexique et du Pérou, les Espagnols abandonnèrent les richesses naturelles pour avoir des richesses de signe qui s’avilissaient par elles-mêmes. L’or et l’argent était très rares en Europe ; et l’Espagne, maîtresse tout à coup d’une très grande quantité de ces métaux, conçut des espérances qu’elle n’avait jamais eues. Les richesses que l’on trouva dans les pays conquis n’étaient pourtant pas proportionnées à celles de leurs mines. Les Indiens en cachèrent une partie ; et de plus, ces peuples, qui ne faisaient servir l’or et l’argent qu’à la magnificence des temples des dieux et des palais des rois, ne les cherchaient pas avec la même avarice que nous ; enfin ils n’avaient pas le secret de tirer les métaux de toutes les mines, mais seulement de celles dans lesquelles la séparation se fait par le feu, ne connaissant pas la manière d’employer le mercure, ni peut-être le mercure même.\par
Cependant l’argent ne laissa pas de doubler bientôt en Europe ; ce qui parut en ce que le prix de tout ce qui s’acheta fut environ du double.\par
Les Espagnols fouillèrent les mines, creusèrent les montagnes, inventèrent des machines pour tirer les eaux, briser le minerai et les séparer ; et, comme ils se jouaient de la vie des Indiens, ils les firent travailler sans ménagement. L’argent doubla bientôt en Europe, et le profit diminua toujours de moitié pour l’Espagne, qui n’avait, chaque année, que la même quantité d’un métal qui était devenu la moitié moins précieux.\par
Dans le double du temps, l’argent doubla encore, et le profit diminua encore de la moitié.\par
Il diminua même de plus de la moitié : voici comment.\par
Pour tirer l’or des mines, pour lui donner les préparations requises, et le transporter en Europe, il fallait une dépense quelconque. Je suppose qu’elle fût comme 1 est {\itshape à} 64 : quand l’argent fut doublé une fois, et par conséquent la moitié moins précieux, la dépense fut comme 2 sont {\itshape à} 64. Ainsi les flottes qui portèrent en Espagne la même quantité d’or, portèrent une chose qui réellement valait la moitié moins, et coûtait la moitié plus.\par
Si l’on suit la chose de doublement en doublement, on trouvera la progression de la cause de l’impuissance des richesses de l’Espagne.\par
Il y a environ deux cents ans que l’on travaille les mines des Indes. Je suppose que la quantité d’argent qui est à présent dans le monde qui commerce, soit à celle qui était avant la découverte comme 32 est à 1, c’est-à-dire qu’elle ait doublé cinq fois : dans deux cents ans encore la même quantité sera à celle qui était avant la découverte comme 64 est à 1, c’est-à-dire qu’elle doublera encore. Or, à présent, cinquante\footnote{Voyez les voyages de Frézier.} quintaux de minerai pour l’or, donnent quatre, cinq et six onces d’or ; et quand il n’y en a que deux, le mineur ne retire que ses frais. Dans deux cents ans, lorsqu’il n’y en aura que quatre, le mineur ne tirera aussi que ses frais. Il y aura donc peu de profit à tirer sur l’or. Même raisonnement sur l’argent, excepté que le travail des mines d’argent est un peu plus avantageux que celui des mines d’or.\par
Que si l’on découvre des mines si abondantes qu’elles donnent plus de profit, plus elles seront abondantes, plus tôt le profit finira.\par
Les Portugais ont trouvé tant d’or dans le Brésil\footnote{Suivant milord Anson, l’Europe reçoit du Brésil tous les ans pour deux millions sterling en or, que l’on trouve dans le sable au pied des montagnes, ou dans le lit des rivières. Lorsque je fis le petit ouvrage dont j’ai parlé dans la première note de ce chapitre, il s’en fallait bien que les retours du Brésil fussent un objet aussi important qu’il l’est aujourd’hui.}, qu’il faudra nécessairement que le profit des Espagnols diminue bientôt considérablement, et le leur aussi.\par
J’ai ouï plusieurs fois déplorer l’aveuglement du Conseil de François I\textsuperscript{er} qui rebuta Christophe Colomb, qui lui proposait les Indes. En vérité, on fit, peut-être par imprudence, une chose bien sage. L’Espagne a fait comme ce roi insensé qui demanda que tout ce qu’il toucherait se convertit en or, et qui fut obligé de revenir aux dieux pour les prier de finir sa misère.\par
Les compagnies et les banques que plusieurs nations établirent, achevèrent d’avilir l’or et l’argent dans leur qualité de signe : car, par de nouvelles fictions, ils multiplièrent tellement les signes des denrées, que l’or et l’argent ne firent plus cet office qu’en partie, et en devinrent moins précieux.\par
Ainsi le crédit public leur tint lieu de mines, et diminua encore le profit que les Espagnols tiraient des leurs.\par
Il est vrai que, par le commerce que les Hollandais firent dans les Indes orientales, ils donnèrent quelque prix à la marchandise des Espagnols ; car, comme ils portèrent de l’argent pour troquer contre les marchandises de l’Orient, ils soulagèrent en Europe les Espagnols d’une partie de leurs denrées qui y abondaient trop.\par
Et ce commerce, qui ne semble regarder qu’indirectement l’Espagne, lui est avantageux comme aux nations mêmes qui le font.\par
Par tout ce qui vient d’être dit, on peut juger des ordonnances du Conseil d’Espagne, qui défendent d’employer l’or et l’argent en dorures et autres superfluités : décret pareil à celui que feraient les États de Hollande s’ils défendaient la consommation de la cannelle.\par
Mon raisonnement ne porte pas sur toutes les mines : celles d’Allemagne et de Hongrie, d’où l’on ne retire que peu de chose au-delà des frais, sont très utiles. Elles se trouvent dans l’État principal ; elles y occupent plusieurs milliers d’hommes qui y consomment les denrées surabondantes : elles sont proprement une manufacture du pays.\par
Les mines d’Allemagne et de Hongrie font valoir la culture des terres ; et le travail de celles du Mexique et du Pérou la détruit.\par
Les Indes et l’Espagne sont deux puissances sous un même maître ; mais les Indes sont le principal, l’Espagne n’est que l’accessoire. C’est en vain que la politique veut ramener le principal à l’accessoire ; les Indes attirent toujours l’Espagne à elles.\par
D’environ cinquante millions de marchandises qui vont toutes les années aux Indes, l’Espagne ne fournit que deux millions et demi : les Indes font donc un commerce de cinquante millions, et l’Espagne de deux millions et demi.\par
C’est une mauvaise espèce de richesse qu’un tribut d’accident et qui ne dépend pas de l’industrie de la nation, du nombre de ses habitants, ni de la culture de ses terres. Le roi d’Espagne, qui reçoit de grandes sommes de sa douane de Cadix, n’est, à cet égard, qu’un particulier très riche dans un État très pauvre. Tout se passe des étrangers à lui sans que ses sujets y prennent presque de part ; ce commerce est indépendant de la bonne et de la mauvaise fortune de son royaume.\par
Si quelques provinces dans la Castille lui donnaient une somme pareille à celle de la douane de Cadix, sa puissance serait bien plus grande : ses richesses ne pourraient être que l’effet de celles du pays ; ces provinces animeraient toutes les autres ; et elles seraient toutes ensemble plus en état de soutenir les charges respectives : au lieu d’un grand trésor, on aurait un grand peuple.
\subsubsection[{Chapitre XXIII. Problème}]{Chapitre XXIII. {\itshape Problème}}
\noindent Ce n’est point à moi à prononcer sur la question, si l’Espagne ne pouvant faire le commerce des Indes par elle-même, il ne vaudrait pas mieux qu’elle le rendît libre aux étrangers. Je dirai seulement qu’il lui convient de mettre à ce commerce le moins d’obstacles que sa politique pourra lui permettre. Quand les marchandises que les diverses nations portent aux Indes y sont chères, les Indes donnent beaucoup de leur marchandise, qui est l’or et l’argent, pour peu de marchandises étrangères : le contraire arrive lorsque celles-ci sont à vil prix. Il serait peut-être utile que ces nations se nuisissent les unes aux autres, afin que les marchandises qu’elles portent aux Indes y fussent toujours à bon marché. Voilà des principes qu’il faut examiner, sans les séparer pourtant des autres considérations : la sûreté des Indes, l’utilité d’une douane unique, les dangers d’un grand changement, les inconvénients qu’on prévoit, et qui souvent sont moins dangereux que ceux qu’on ne peut pas prévoir.
\subsection[{Livre vingt-deuxième. Des lois dans le rapport qu’elles ont avec l’usage de la monnaie}]{Livre vingt-deuxième. Des lois dans le rapport qu’elles ont avec l’usage de la monnaie}
\subsubsection[{Chapitre I. Raison de l’usage de la monnaie}]{Chapitre I. Raison de l’usage de la monnaie}
\noindent Les peuples qui ont peu de marchandises pour le commerce, comme les sauvages, et les peuples policés qui n’en ont que de deux ou trois espèces, négocient par échange. Ainsi les caravanes de Maures qui vont à Tombouctou, dans le fond de l’Afrique, troquer du sel contre de l’or, n’ont pas besoin de monnaie. Le Maure met son sel dans un monceau ; le Nègre, sa poudre dans un autre : s’il n’y a pas assez d’or, le Maure retranche de son sel, ou le Nègre ajoute de son or, jusqu’à ce que les parties conviennent.\par
Mais lorsqu’un peuple trafique sur un très grand nombre de marchandises, il faut nécessairement une monnaie, parce qu’un métal facile à transporter épargne bien des frais que l’on serait obligé de faire si l’on procédait toujours par échange.\par
Toutes les nations ayant des besoins réciproques, il an-ive souvent que l’une veut avoir un très grand nombre de marchandises de l’autre, et celle-ci très peu des siennes ; tandis qu’à l’égard d’une autre nation, elle est dans un cas contraire. Mais lorsque les nations ont une monnaie, et qu’elles procèdent par vente et par achat, celles qui prennent plus de marchandises se soldent, ou paient l’excédent avec de l’argent ; et il y a cette différence, que, dans le cas de l’achat, le commerce se fait à proportion des besoins de la nation qui demande le plus ; et que, dans l’échange, le commerce se fait seulement dans l’étendue des besoins de la nation qui demande le moins, sans quoi cette dernière serait dans l’impossibilité de solder son compte.
\subsubsection[{Chapitre II. De la nature de la monnaie}]{Chapitre II. De la nature de la monnaie}
\noindent La monnaie est un signe qui représente la valeur de toutes les marchandises. On prend quelque métal pour que le signe soit durable\footnote{Le sel dont on se sert en Abyssinie a ce défaut, qu’il se consomme continuellement.}, qu’il se consomme peu par l’usage, et que, sans se détruire, il soit capable de beaucoup de divisions. On choisit un métal précieux, pour que le signe puisse aisément se transporter. Un métal est très propre à être une mesure commune, parce qu’on peut aisément le réduire au même titre. Chaque État y met son empreinte, afin que la forme réponde du titre et du poids, et que l’on connaisse l’un et l’autre par la seule inspection.\par
Les Athéniens, n’ayant point l’usage des métaux, se servirent de bœufs\footnote{Hérodote, in Clio, nous dit que les Lydiens trouvèrent l’art de battre la monnaie ; les Grecs le prirent deux ; les monnaies d’Athènes eurent pour empreinte leur ancien bœuf. J’ai vu une de ces monnaies dans le cabinet du comte de Pembrocke.}, et les Romains de brebis ; mais un bœuf n’est pas la même chose qu’un autre bœuf, comme une pièce de métal peut être la même qu’une autre.\par
Comme l’argent est le signe des valeurs des marchandises, le papier est un signe de la valeur de l’argent ; et, lorsqu’il est bon, il le représente tellement, que, quant à l’effet, il n’y a point de différence.\par
De même que l’argent est un signe d’une chose, et la représente, chaque chose est un signe de l’argent, et le représente ; et l’État est dans la prospérité, selon que, d’un côté, l’argent représente bien toutes choses, et que, d’un autre, toutes choses représentent bien l’argent, et qu’ils sont signes les uns des autres ; c’est-à-dire que, dans leur valeur relative, on peut avoir l’un sitôt que l’on a l’autre. Cela n’arrive jamais que dans un gouvernement modéré, mais n’arrive pas toujours dans un gouvernement modéré : par exemple, si les lois favorisent un débiteur injuste, les choses qui lui appartiennent ne représentent point l’argent, et n’en sont point un signe. À l’égard du gouvernement despotique, ce serait un prodige si les choses y représentaient leur signe : la tyrannie et la méfiance font que tout le monde y enterre son argent\footnote{C’est un ancien usage à Alger, que chaque père de famille ait un trésor enterré. Laugier de Tassis, {\itshape Histoire du royaume d’Alger}.} : les choses n’y représentent donc point l’argent.\par
Quelquefois les législateurs ont employé un tel art, que non seulement les choses représentaient l’argent par leur nature, mais qu’elles devenaient monnaie comme l’argent même. César\footnote{Voyez César, {\itshape De la Guerre civile}, liv. III.}, dictateur, permit aux débiteurs de donner en paiement à leurs créanciers des fonds de terre au prix qu’ils valaient avant la guerre civile. Tibère\footnote{Tacite, liv. VI.} ordonna que ceux qui voudraient de l’argent en auraient du trésor public, en obligeant des fonds pour le double. Sous César, les fonds de terre furent la monnaie qui paya toutes les dettes ; sous Tibère, dix mille sesterces en fonds devinrent une monnaie commune, comme cinq mille sesterces en argent.\par
La grande charte d’Angleterre défend de saisir les terres ou les revenus d’un débiteur, lorsque ses biens mobiliers ou personnels suffisent pour le paiement, et qu’il offre de les donner : pour lors, tous les biens d’un Anglais représentaient de l’argent.\par
Les lois des Germains apprécièrent en argent les satisfactions pour les toits que l’on avait faits, et pour les peines des crimes. Mais comme il y avait très peu d’argent dans le pays, elles réapprécièrent l’argent en denrées ou en bétail. Ceci se trouve fixé dans la loi des Saxons, avec de certaines différences, suivant l’aisance et la commodité des divers peuples. D’abord\footnote{Loi des Saxons, chap. XVIII.} la loi déclare la valeur du sou en bétail : le sou de deux trémisses se rapportait à un bœuf de douze mois, ou à une brebis avec son agneau ; celui de trois trémisses valait un bœuf de seize mois. Chez ces peuples, la monnaie devenait bétail, marchandise ou denrée ; et ces choses devenaient monnaie.\par
Non seulement l’argent est un signe des choses, il est encore un signe de l’argent, et représente l’argent, comme nous le verrons au chapitre du change.
\subsubsection[{Chapitre III. Des monnaies idéales}]{Chapitre III. Des monnaies idéales}
\noindent Il y a des monnaies réelles et des monnaies idéales. Les peuples policés, qui se servent presque tous de monnaies idéales, ne le font que parce qu’ils ont converti leurs monnaies réelles en idéales. D’abord, leurs monnaies réelles sont un certain poids et un certain titre de quelque métal. Mais bientôt la mauvaise foi ou le besoin font qu’on retranche une partie du métal de chaque pièce de monnaie, à laquelle on laisse le même nom : par exemple, d’une pièce du poids d’une livre d’argent, on retranche la moitié de l’argent, et on continue de l’appeler livre ; la pièce qui était une vingtième partie de la livre d’argent, on continue de l’appeler sou, quoiqu’elle ne soit plus la vingtième partie de cette livre. Pour lors, la livre est une livre idéale, et le sou, un sou idéal ; ainsi des autres subdivisions ; et cela peut aller au point que ce qu’on appellera livre, ne sera plus qu’une très petite portion de la livre ; ce qui la rendra encore plus idéale. Il peut même arriver que l’on ne fera plus de pièce de monnaie qui vaille précisément une livre, et qu’on ne fera pas non plus de pièce qui vaille un sou : pour lors, la livre et le sou seront des monnaies purement idéales. On donnera à chaque pièce de monnaie la dénomination d’autant de livres et d’autant de sous que l’on voudra ; la variation pourra être continuelle, parce qu’il est aussi aisé de donner un autre nom à une chose, qu’il est difficile de changer la chose même.\par
Pour ôter la source des abus, ce sera une très bonne loi dans tous les pays où l’on voudra faire fleurir le commerce, que celle qui ordonnera qu’on emploiera des monnaies réelles, et que l’on ne fera point d’opération qui puisse les rendre idéales.\par
Rien ne doit être si exempt de variation que ce qui est la mesure commune de tout.\par
Le négoce par lui-même est très incertain ; et c’est un grand mal d’ajouter une nouvelle incertitude à celle qui est fondée sur la nature de la chose.
\subsubsection[{Chapitre IV. De la quantité de l’or et de l’argent}]{Chapitre IV. De la quantité de l’or et de l’argent}
\noindent Lorsque les nations policées sont les maîtresses du monde, l’or et l’argent augmentent tous les jours, soit qu’elles le tirent de chez elles, soit qu’elles l’aillent chercher là où il est. Il diminue, au contraire, lorsque les nations barbares prennent le dessus. On sait quelle fut la rareté de ces métaux, lorsque les Goths et les Vandales d’un côté, les Sarrasins et les Tartares de l’autre, eurent tout envahi.
\subsubsection[{Chapitre V. Continuation du même sujet}]{Chapitre V. Continuation du même sujet}
\noindent L’argent tiré des mines de l’Amérique, transporté en Europe, de là encore envoyé en Orient, a favorisé la navigation de l’Europe : c’est une marchandise de plus que l’Europe reçoit en troc de l’Amérique, et qu’elle envoie en troc aux Indes. Une plus grande quantité d’or et d’argent est donc favorable lorsqu’on regarde ces métaux comme marchandise : elle ne l’est point lorsqu’on les regarde comme signe, parce que leur abondance choque leur qualité de signe, qui est beaucoup fondée sur la rareté.\par
Avant la première guerre punique, le cuivre était à l’argent comme 960 est 1\footnote{Voyez ci-après XXII, chap. XII.} ; il est aujourd’hui à peu près comme 73 1/2 est à 1\footnote{En supposant l’argent à 49 livres le marc, et le cuivre à 20 sols la livre.}. Quand la proportion serait comme elle était autrefois, l’argent n’en ferait que mieux sa fonction de signe.
\subsubsection[{Chapitre VI. Par quelle raison le prix de l’usure diminua de la moitié lors de la découverte des Indes}]{Chapitre VI. Par quelle raison le prix de l’usure diminua de la moitié lors de la découverte des Indes}
\noindent L’Inca Garcilasso\footnote{{\itshape Histoire des guerres civiles des Espagnols dans les Indes}.} dit qu’en Espagne, après la conquête des Indes, les rentes, qui étaient au denier dix, tombèrent au denier vingt. Cela devait être ainsi. Une grande quantité d’argent fut tout à coup portée en Europe : bientôt moins de personnes eurent besoin d’argent ; le prix de toutes choses augmenta, et celui de l’argent diminua : la proportion fut donc rompue, toutes les anciennes dettes furent éteintes. On peut se rappeler le temps du Système\footnote{On appelait ainsi le projet de M. Law en France.}, où toutes les choses avaient une grande valeur, excepté l’argent. Après la conquête des Indes, ceux qui avaient de l’argent furent obligés de diminuer le prix ou le louage de leur marchandise, c’est-à-dire l’intérêt.\par
Depuis ce temps le prêt n’a pu revenir à l’ancien taux, parce que la quantité de l’argent a augmenté toutes les années en Europe. D’ailleurs, les fonds publics de quelques États, fondés sur les richesses que le commerce leur a procurées, donnant un intérêt très modique, il a fallu que les contrats des particuliers se réglassent là-dessus. Enfin, le change ayant donné aux hommes une facilité singulière de transporter l’argent d’un pays à un autre, l’argent n’a pu être rare dans un lieu, qu’il n’en vînt de tous côtés de ceux où il était commun.
\subsubsection[{Chapitre VII. Comment le prix des choses se fixe dans la variation des richesses de signe}]{Chapitre VII. Comment le prix des choses se fixe dans la variation des richesses de signe}
\noindent L’argent est le prix des marchandises ou denrées. Mais comment se fixera ce prix ? C’est-à-dire par quelle portion d’argent chaque chose sera-t-elle représentée ?\par
Si l’on compare la masse de l’or et de l’argent qui est dans le monde, avec la somme des marchandises qui y sont, il est certain que chaque denrée ou marchandise en particulier pourra être comparée à une certaine portion de la masse entière de l’or et de l’argent. Comme le total de l’une est au total de l’autre, la partie de l’une sera à la partie de l’autre. Supposons qu’il n’y ait qu’une seule denrée ou marchandise dans le monde, ou qu’il n’y en ait qu’une seule qui s’achète, et qu’elle se divise comme l’argent ; cette partie de cette marchandise répondra à une partie de la masse de l’argent ; la moitié du total de l’une, à la moitié du total de l’autre ; la dixième, la centième, la millième de l’une, à la dixième, à la centième, à la millième de l’autre. Mais comme ce qui forme la propriété parmi les hommes n’est pas tout à la fois dans le commerce, et que les métaux ou les monnaies, qui en sont les signes, n’y sont pas aussi dans le même temps, les prix se fixeront en raison composée du total des choses avec le total des signes, et de celle du total des choses qui sont dans le commerce, avec le total des signes qui y sont aussi ; et, comme les choses qui ne sont pas dans le commerce aujourd’hui peuvent y être demain, et que les signes qui n’y sont point aujourd’hui peuvent y rentrer tout de même, l’établissement du prix des choses dépend toujours fondamentalement de la raison du total des choses au total des signes.\par
Ainsi le prince ou le magistrat ne peuvent pas plus taxer la valeur des marchandises, qu’établir, par une ordonnance, que le rapport d’un à dix est égal à celui d’un à vingt. Julien\footnote{{\itshape Histoire de l’Église}, par Socrate, liv. II.} ayant baissé les denrées à Antioche, y causa une affreuse famine.
\subsubsection[{Chapitre VIII. Continuation du même sujet}]{Chapitre VIII. Continuation du même sujet}
\noindent Les noirs de la côte d’Afrique ont un signe des valeurs, sans monnaie : c’est un signe purement idéal, fondé sur le degré d’estime qu’ils mettent dans leur esprit à chaque marchandise, à proportion du besoin qu’ils en ont. Une certaine denrée ou marchandise vaut trois macutes ; une autre, six macutes ; une autre, dix macutes : c’est comme s’ils disaient simplement trois, six, dix. Le prix se forme par la comparaison qu’ils font de toutes les marchandises entre elles ; pour lors, il n’y a point de monnaie particulière, mais chaque portion de marchandise est monnaie de l’autre.\par
Transportons pour un moment parmi nous cette manière d’évaluer les choses, et joignons-la avec la nôtre : toutes les marchandises et denrées du monde, ou bien toutes les marchandises ou denrées d’un État en particulier, considéré comme séparé de tous les autres, vaudront un certain nombre de macutes ; et, divisant l’argent de cet État en autant de parties qu’il y a de macutes, une partie divisée de cet argent sera le signe d’une macute.\par
Si l’on suppose que la quantité de l’argent d’un État double, il faudra pour une macule le double de l’argent ; mais si, en doublant l’argent, vous doublez aussi les macutes, la proportion restera telle qu’elle était avant l’un et l’autre doublement.\par
Si, depuis la découverte des Indes, l’or et l’argent ont augmenté en Europe à raison d’un à vingt, le prix des denrées et marchandises aurait dû monter en raison d’un à vingt. Mais si, d’un autre côté, le nombre des marchandises a augmenté comme un à deux, il faudra que le prix de ces marchandises et denrées ait haussé, d’un côté, à raison d’un à vingt, et qu’il ait baissé en raison d’un à deux, et qu’il ne soit par conséquent qu’en raison d’un à dix.\par
La quantité des marchandises et denrées croît par une augmentation de commerce ; l’augmentation de commerce, par une augmentation d’argent qui arrive successivement, et par de nouvelles communications avec de nouvelles terres et de nouvelles mers, qui nous donnent de nouvelles denrées et de nouvelles marchandises.
\subsubsection[{Chapitre IX. De la rareté relative de l’or et de l’argent}]{Chapitre IX. De la rareté relative de l’or et de l’argent}
\noindent Outre l’abondance et la rareté positive de l’or et de l’argent, il y a encore une abondance et une rareté relative d’un de ces métaux à l’autre.\par
L’avarice garde l’or et l’argent, parce que, comme elle ne veut pas consommer, elle aime des signes qui ne se détruisent point. Elle aime mieux garder l’or que l’argent, parce qu’elle craint toujours de perdre, et qu’elle peut mieux cacher ce qui est en plus petit volume. L’or disparaît donc quand l’argent est commun, parce que chacun en a pour le cacher ; il reparaît quand l’argent est rare, parce que l’on est obligé de le retirer de ses retraites.\par
C’est donc une règle : l’or est commun quand l’argent est rare, et l’or est rare quand l’argent est commun. Cela fait sentir la différence de l’abondance et de la rareté réelle ; chose dont je vais beaucoup parler.
\subsubsection[{Chapitre X. Du change}]{Chapitre X. {\itshape Du change}}
\noindent C’est l’abondance et la rareté relative des monnaies des divers pays, qui forment ce qu’on appelle le change.\par
Le change est une fixation de la valeur actuelle et momentanée des monnaies.\par
L’argent, comme métal, a une valeur comme toutes les autres marchandises ; et il a encore une valeur qui vient de ce qu’il est capable de devenir le signe des autres marchandises ; et s’il n’était qu’une simple marchandise, il ne faut pas douter qu’il ne perdît beaucoup de son prix.\par
L’argent, comme monnaie, a une valeur que le prince peut fixer dans quelques rapports, et qu’il ne saurait fixer dans d’autres.\par
Le prince établit une proportion entre une quantité d’argent comme métal, et la même quantité comme monnaie ; 2° il fixe celle qui est entre divers métaux employés à la monnaie ; 3° il établit le poids et le titre de chaque pièce de monnaie. Enfin il donne à chaque pièce cette valeur idéale dont j’ai parlé. J’appellerai la valeur de la monnaie, dans ces quatre rapports, {\itshape valeur positive}, parce qu’elle peut être fixée par une loi.\par
Les monnaies de chaque État ont, de plus, une {\itshape valeur relative}, dans le sens qu’on les compare avec les monnaies des autres pays : c’est cette valeur relative que le change établit. Elle dépend beaucoup de la valeur positive. Elle est fixée par l’estime la plus générale des négociants, et ne peut l’être par l’ordonnance du prince, parce qu’elle varie sans cesse, et dépend de mille circonstances.\par
Pour fixer la valeur relative, les diverses nations se régleront beaucoup sur celle qui a le plus d’argent. Si elle a autant d’argent que toutes les autres ensemble, il faudra bien que chacune aille se mesurer avec elle ; ce qui fera qu’elles se régleront à peu près entre elles comme elles se sont mesurées avec la nation principale.\par
Dans l’état actuel de l’univers, c’est la Hollande\footnote{Les Hollandais règlent le change de presque toute l’Europe par une espèce de délibération entre eux, selon qu’il convient à leurs intérêts.} qui est cette nation dont nous parlons. Examinons le change par rapport à elle.\par
Il y a en Hollande une monnaie qu’on appelle un florin ; le florin vaut vingt sous, ou quarante demi-sous, ou gros. Pour simplifier les idées, imaginons qu’il n’y ait point de florins en Hollande, et qu’il n’y ait que des gros : un homme qui aura mille florins aura quarante mille gros, ainsi du reste. Or le change avec la Hollande consiste à savoir combien vaudra de gros chaque pièce de monnaie des autres pays ; et, comme l’on compte ordinairement en France par écus de trois livres, le change demandera combien un écu de trois livres vaudra de gros. Si le change est à cinquante-quatre, l’écu de trois livres vaudra cinquante-quatre gros ; s’il est à soixante, il vaudra soixante gros ; si l’argent est rare en France, l’écu de trois livres vaudra plus de gros ; s’il est en abondance, il vaudra moins de gros.\par
Cette rareté ou cette abondance, d’où résulte la mutation du change, n’est pas la rareté ou l’abondance réelle ; c’est une rareté ou une abondance relative : par exemple, quand la France a plus besoin d’avoir des fonds en Hollande, que les Hollandais n’ont besoin d’en avoir en France, l’argent est appelé commun en France, et rare en Hollande ; {\itshape et vice versa}.\par
Supposons que le change avec la Hollande soit à cinquante-quatre. Si la France et la Hollande ne composaient qu’une ville, on ferait comme l’on fait quand on donne la monnaie d’un écu : le Français tirerait de sa poche trois livres, et le Hollandais tirerait de la sienne cinquante-quatre gros. Mais, comme il y a de la distance entre Paris et Amsterdam, il faut que celui qui me donne pour mon écu de trois livres cinquante-quatre gros qu’il a en Hollande, me donne une lettre de change de cinquante-quatre gros sur la Hollande. Il West plus ici question de cinquante-quatre gros, mais d’une lettre de cinquante-quatre gros. Ainsi, pour juger\footnote{Il y a beaucoup d’argent dans une place lorsqu’il y a plus d’argent que de papier ; il y en a peu lorsqu’il y a plus de papier que d’argent.} de la rareté ou de l’abondance de l’argent, il faut savoir s’il y a en France plus de lettres de cinquante-quatre gros destinées pour la France, qu’il n’y a d’écus destinés pour la Hollande. S’il y a beaucoup de lettres offertes par les Hollandais, et peu d’écus offerts par les Français, l’argent est rare en France, et commun en Hollande ; et il faut que le change hausse, et que pour mon écu on me donne plus de cinquante-quatre gros ; autrement je ne le donnerais pas ; {\itshape et vice versa}.\par
On voit que les diverses opérations du change forment un compte de recette et de dépense qu’il faut toujours solder ; et qu’un État qui doit ne s’acquitte pas plus avec les autres par le change, qu’un particulier ne paie une dette en changeant de l’argent.\par
Je suppose qu’il n’y ait que trois États dans le monde : la France, l’Espagne et la Hollande ; que divers particuliers d’Espagne dussent en France la valeur de cent mille marcs d’argent, et que divers particuliers de France dussent en Espagne cent dix mille marcs ; et que quelque circonstance fît que chacun, en Espagne et en France, voulût tout {\itshape à} coup retirer son argent : que feraient les opérations du change ? Elles acquitteraient réciproquement ces deux nations de la somme de cent mille marcs ; mais la France devrait toujours dix mille marcs en Espagne, et les Espagnols auraient toujours des lettres sur la France pour dix mille marcs, et la France n’en aurait point du tout sur l’Espagne.\par
Que si la Hollande était dans un cas contraire avec la France, et que, pour solde, elle lui dût dix mille marcs, la France pourrait payer l’Espagne de deux manières : ou en donnant à ses créanciers en Espagne des lettres sur ses débiteurs de Hollande pour dix mille mares, ou bien en envoyant dix mille marcs d’argent en espèces en Espagne.\par
Il suit de là que, quand un État a besoin de remettre une somme d’argent dans un autre pays, il est indifférent, par la nature de la chose, que l’on y voiture de l’argent, ou que l’on prenne des lettres de change. L’avantage de ces deux manières de payer dépend uniquement des circonstances actuelles : il faudra voir ce qui, dans ce moment, donnera plus de gros en Hollande, ou l’argent porté en espèces\footnote{Les frais de la voiture et de l’assurance déduits.}, ou une lettre sur la Hollande de pareille somme.\par
Lorsque même titre et même poids d’argent en France me rendent même poids et même titre d’argent en Hollande, on dit que le change est au pair. Dans l’état actuel des monnaies\footnote{En 1744.}, le pair est à peu près à cinquante-quatre gros par écu : lorsque le change sera au-dessus de cinquante-quatre gros, on dira qu’il est haut ; lorsqu’il sera au-dessous, on dira qu’il est bas.\par
Pour savoir si, dans une certaine situation du change, l’État gagne ou perd, il faut le considérer comme débiteur, comme créancier, comme vendeur, comme acheteur. Lorsque le change est plus bas que le pair, il perd comme débiteur, il gagne comme créancier ; il perd comme acheteur, il gagne comme vendeur. On sent bien qu’il perd comme débiteur : par exemple, la France devant à la Hollande un certain nombre de gros, moins son écu vaudra de gros, plus il lui faudra d’écus pour payer ; au contraire, si la France est créancière d’un certain nombre de gros, moins chaque écu vaudra de gros, plus elle recevra d’écus. L’État perd encore comme acheteur ; car il faut toujours le même nombre de gros pour acheter la même quantité de marchandises ; et, lorsque le change baisse, chaque écu de France donne moins de gros. Par la même raison, l’État gagne comme vendeur : je vends ma marchandise en Hollande le même nombre de gros que je la vendais ; j’aurai donc plus d’écus en France, lorsqu’avec cinquante gros je me procurerai un écu, que lorsqu’il m’en faudra cinquante-quatre pour avoir ce même écu : le contraire de tout ceci arrivera à l’autre État. Si la Hollande doit un certain nombre d’écus, elle gagnera ; et, si on les lui doit, elle perdra ; si elle vend, elle perdra ; si elle achète, elle gagnera.\par
Il faut pourtant suivre ceci. Lorsque le change est au-dessous du pair, par exemple, s’il est à cinquante au lieu d’être à cinquante-quatre, il devrait arriver que la France, envoyant par le change cinquante-quatre mille écus en Hollande, n’achèterait de marchandises que pour cinquante mille ; et que, d’un autre côté, la Hollande, envoyant la valeur de cinquante mille écus en France, en achèterait pour cinquante-quatre mille : ce qui ferait une différence de huit cinquante-quatrièmes, c’est-à-dire de plus d’un septième de perte pour la France ; de sorte qu’il faudrait envoyer en Hollande un septième de plus en argent ou en marchandises qu’on ne faisait lorsque le change était au pair ; et le mal augmentant toujours, parce qu’une pareille dette ferait encore diminuer le change, la France serait, à la fin, ruinée. Il semble, dis-je, que cela devrait être ; et cela n’est pas, à cause du principe que j’ai déjà établi ailleurs\footnote{Voyez le liv. XX, chap. XXIII.}, qui est que les États tendent toujours à se mettre dans la balance, et à se procurer leur libération. Ainsi ils n’empruntent qu’à proportion de ce qu’ils peuvent payer, et n’achètent qu’à mesure qu’ils vendent. Et, en prenant l’exemple ci-dessus, si le change tombe en France de cinquante-quatre à cinquante, le Hollandais, qui achetait des marchandises pour mille écus, et qui les payait cinquante-quatre mille gros, ne les paierait plus que cinquante mille, si le Français y voulait consentir. Mais la marchandise de France haussera insensiblement, le profit se partagera entre le Français et le Hollandais : car, lorsqu’un négociant peut gagner, il partage aisément son profit ; il se fera donc une communication de profit entre le Français et le Hollandais. De la même manière, le Français, qui achetait des marchandises de Hollande pour cinquante-quatre mille gros, et qui les payait avec mille écus lorsque le change était à cinquante-quatre, serait obligé d’ajouter quatre cinquante-quatrièmes de plus en écus de France, pour acheter les mêmes marchandises. Mais le marchand français, qui sentira la perte qu’il ferait, voudra donner moins de la marchandise de Hollande. Il se fera donc une communication de perte entre le marchand français et le marchand hollandais ; l’État se mettra insensiblement dans la balance, et l’abaissement du change n’aura pas tous les inconvénients qu’on devait craindre.\par
Lorsque le change est plus bas que le pair, un négociant peut, sans diminuer sa fortune, remettre ses fonds dans les pays étrangers ; parce qu’en les faisant revenir, il regagne ce qu’il a perdu ; mais un prince qui n’envoie dans les pays étrangers qu’un argent qui ne doit jamais revenir, perd toujours.\par
Lorsque les négociants font beaucoup d’affaires dans un pays, le change y hausse infailliblement. Cela vient de ce qu’on y prend beaucoup d’engagements, et qu’on y achète beaucoup de marchandises ; et l’on tire sur le pays étranger pour les payer.\par
Si un prince fait de grands amas d’argent dans son État, l’argent y pourra être rare réellement, et commun relativement : par exemple, si, dans le même temps, cet État avait à payer beaucoup de marchandises dans le pays étranger, le change baisserait, quoique l’argent fût rare.\par
Le change de toutes les places tend toujours à se mettre à une certaine proportion ; et cela est dans la nature de la chose même. Si le change de l’Irlande à l’Angleterre est plus bas que le pair, et que celui de l’Angleterre à la Hollande soit aussi plus bas que le pair, celui de l’Irlande à la Hollande sera encore plus bas : c’est-à-dire, en raison composée de celui d’Irlande à l’Angleterre, et de celui de l’Angleterre à la Hollande : car un Hollandais, qui peut faire venir ses fonds indirectement d’Irlande par l’Angleterre, ne voudra pas payer plus cher pour les faire venir directement. Je dis que cela devrait être ainsi ; mais cela n’est pourtant pas exactement ainsi ; il y a toujours des circonstances qui font varier ces choses ; et la différence du profit qu’il y a à tirer par une place, ou à tirer par une autre, fait l’art ou l’habileté particulière des banquiers, dont il n’est point question ici.\par
Lorsqu’un État hausse sa monnaie ; par exemple, lorsqu’il appelle six livres ou deux écus, ce qu’il n’appelait que trois livres ou un écu, cette dénomination nouvelle, qui n’ajoute rien de réel à l’écu, ne doit pas procurer un seul gros de plus par le change. On ne devrait avoir, pour les deux écus nouveaux, que la même quantité de gros que l’on recevait pour l’ancien ; et, si cela n’est pas, ce n’est point l’effet de la fixation en elle-même, mais de celui qu’elle produit comme nouvelle, et de celui qu’elle a comme subite. Le change tient à des affaires commencées, et ne se met en règle qu’après un certain temps.\par
Lorsqu’un État, au lieu de hausser simplement sa monnaie par une loi, fait une nouvelle refonte afin de faire d’une monnaie forte une monnaie plus faible, il arrive que, pendant le temps de l’opération, il y a deux sortes de monnaie : la forte, qui est la vieille, et la faible, qui est la nouvelle ; et comme la forte est décriée et ne se reçoit qu’à la Monnaie, et que, par conséquent, les lettres de change doivent se payer en espèces nouvelles, il semble que le change devrait se régler sur l’espèce nouvelle. Si, par exemple, l’affaiblissement en France était de moitié, et que l’ancien écu de trois livres donnât soixante gros en Hollande, le nouvel écu ne devrait donner que trente gros. D’un autre côté, il semble que le change devrait se régler sur la valeur de l’espèce vieille, parce que le banquier qui a de l’argent et qui prend des lettres est obligé d’aller porter à la Monnaie des espèces vieilles, pour en avoir de nouvelles sur lesquelles il perd. Le change se mettra donc entre la valeur de l’espèce nouvelle et celle de l’espèce vieille. La valeur de l’espèce vieille tombe, pour ainsi dire, et parce qu’il y a déjà dans le commerce de l’espèce nouvelle, et parce que le banquier ne peut pas tenir rigueur, ayant intérêt de faire sortir promptement l’argent vieux de sa caisse pour le faire travailler, et y étant même forcé pour faire ses paiements. D’un autre côté, la valeur de l’espèce nouvelle s’élève, pour ainsi dire, parce que le banquier, avec de l’espèce nouvelle, se trouve dans une circonstance où nous allons faire voir qu’il peut, avec un grand avantage, s’en procurer de la vieille. Le change se mettra donc, comme j’ai dit, entre l’espèce nouvelle et l’espèce vieille. Pour lors, les banquiers ont du profit à faire sortir l’espèce vieille de l’État, parce qu’ils se procurent, par là, le même avantage que donnerait un change réglé sur l’espèce vieille, c’est-à-dire beaucoup de gros en Hollande ; et qu’ils ont un retour en change, réglé entre l’espèce nouvelle et l’espèce vieille, c’est-à-dire plus bas ; ce qui procure beaucoup d’écus en France.\par
Je suppose que trois livres d’espèce vieille rendent, par le change actuel, quarante-cinq gros, et qu’en transportant ce même écu en Hollande on en ait soixante ; mais avec une lettre de quarante-cinq gros, on se procurera un écu de trois livres en France, lequel, transporté en espèce vieille en Hollande, donnera encore soixante gros : toute l’espèce vieille sortira donc de l’État qui fait la refonte, et le profit en sera pour les banquiers.\par
Pour remédier à cela, on sera forcé de faire une opération nouvelle. L’État, qui fait la refonte, enverra lui-même une grande quantité d’espèce vieille chez la nation qui règle le change ; et, s’y procurant un crédit, il fera monter le change au point qu’on aura, à peu de chose près, autant de gros par le change d’un écu de trois livres, qu’on en aurait en faisant sortir un écu de trois livres en espèces vieilles hors du pays. Je dis {\itshape à peu de chose} près, par ce que, lorsque le profit sera modique, on ne sera point tenté de faire sortir l’espèce, à cause des frais de la voiture et des risques de la confiscation.\par
Il est bon de donner une idée bien claire de ceci. Le sieur Bernard, ou tout autre banquier que l’État voudra employer, propose ses lettres sur la Hollande, et les donne à un, deux, trois gros plus haut que le change actuel ; il a fait une provision dans les pays étrangers, par le moyen des espèces vieilles qu’il a fait continuellement voiturer ; il a donc fait hausser le change au point que nous venons de dire. Cependant, à force de donner de ses lettres, il se saisit de toutes les espèces nouvelles, et force les autres banquiers, qui ont des paiements à faire, à porter leurs espèces vieilles à la Monnaie ; et de plus, comme il a eu insensiblement tout l’argent, il contraint, à leur tour, les autres banquiers à lui donner des lettres à un change très haut : le profit de la fin l’indemnise en grande partie de la perte du commencement.\par
On sent que, pendant toute cette opération, l’État doit souffrir une violente crise. L’argent y deviendra très rare : 1˚ parce qu’il faut en décrier la plus grande partie ; 2{\itshape °} parce qu’il en faudra transporter une partie dans les pays étrangers ; 3° parce que tout le monde le resserrera, personne ne voulant laisser au prince un profit qu’on espère avoir soi-même. Il est dangereux de la faire avec lenteur : il est dangereux de la faire avec promptitude. Si le gain qu’on suppose est immodéré, les inconvénients augmentent à mesure.\par
On a vu ci-dessus que, quand le change était plus bas que l’espèce, il y avait du profit à faire sortir l’argent : par la même raison, lorsqu’il est plus haut que l’espèce, il y a du profit à le faire revenir.\par
Mais il y a un cas où on trouve du profit à faire sortir l’espèce, quoique le change soit au pair : c’est lorsqu’on l’envoie dans les pays étrangers pour la faire remarquer ou refondre. Quand elle est revenue, on fait, soit qu’on l’emploie dans le pays, soit qu’on prenne des lettres pour l’étranger, le profit de la monnaie.\par
S’il arrivait que, dans un État, on fît une compagnie qui eût un nombre très considérable d’actions, et qu’on eût fait, dans quelques mois de temps, hausser ces actions vingt ou vingt-cinq fois au-delà de la valeur du premier achat ; et que ce même État eût établi une banque dont les billets dussent faire la fonction de monnaie ; et que la valeur numéraire de ces billets fût prodigieuse, pour répondre à la prodigieuse valeur numéraire des actions (c’est le système de M. Law) : il suivrait de la nature de la chose, que ces actions et billets s’anéantiraient de la même manière qu’ils se seraient établis. On n’aurait pu faire monter tout à coup les actions vingt ou vingt-cinq fois plus haut que leur première valeur, sans donner à beaucoup de gens le moyen de se procurer d’immenses richesses en papier : chacun chercherait à assurer sa fortune ; et, comme le change donne la voie la plus facile pour la dénaturer, ou pour la transporter où l’on veut, on remettrait sans cesse une partie de ses effets chez la nation qui règle le change. Un projet continuel de remettre dans les pays étrangers, ferait baisser le change. Supposons que, du temps du Système, dans le rapport du titre et du poids de la monnaie d’argent, le taux du change fût de quarante gros par écu ; lorsqu’un papier innombrable fut devenu monnaie, on n’aura plus voulu donner que trente-neuf gros par écu ; ensuite que trente-huit, trente-sept, etc. Cela alla si loin, que l’on ne donna plus que huit gros, et qu’enfin il n’y eut plus de change.\par
C’était le change qui devait, en ce cas, régler en France la proportion de l’argent avec le papier. Je suppose que, par le poids et le titre de l’argent, l’écu de trois livres d’argent valût quarante gros, et que le change, se faisant en papier, l’écu de trois livres en papier ne valût que huit gros, la différence était de quatre cinquièmes. L’écu de trois livres en papier valait donc quatre cinquièmes de moins que l’écu de trois livres en argent.
\subsubsection[{Chapitre XI. Des opérations que les romains firent sur les monnaies}]{Chapitre XI. Des opérations que les romains firent sur les monnaies}
\noindent Quelques coups d’autorité que l’on ait faits de nos jours en France sur les monnaies dans deux ministères consécutifs, les Romains en firent de plus grands, non pas dans le temps de cette république corrompue, ni dans celui de cette république qui n’était qu’une anarchie ; mais lorsque, dans la force de son institution, par sa sagesse comme par son courage, après avoir vaincu les villes d’Italie, elle disputait l’empire aux Carthaginois.\par
Et je suis bien aise d’approfondir un peu cette matière, afin qu’on ne fasse pas un exemple de ce qui n’en est point un.\par
Dans la première guerre punique\footnote{Pline, {\itshape Hist. nat.}, liv. XXXIII, art. 13.}, l’as, qui devait être de douze onces de cuivre, n’en pesa plus que deux ; et dans la seconde, il ne fut plus que d’une. Ce retranchement répond à ce que nous appelons aujourd’hui augmentation des monnaies. Ôter d’un écu de dix livres la moitié de l’argent pour en faire deux, ou le faire valoir douze livres, c’est précisément la même chose.\par
Il ne nous reste point de monument de la manière dont les Romains firent leur opération dans la première guerre punique ; mais ce qu’ils firent dans la seconde nous marque une sagesse admirable. La république ne se trouvait point en état d’acquitter ses dettes ; l’as pesait deux onces de cuivre ; et le denier, valant dix as, valait vingt onces de cuivre. La république fit des as d’une once de cuivre\footnote{{\itshape Ibid.}} ; elle gagna la moitié sur ses créanciers ; elle paya un denier avec ces dix onces de cuivre. Cette opération donna une grande secousse à l’État, il fallait la donner la moindre qu’il était possible ; elle contenait une injustice, il fallait qu’elle fût la moindre qu’il était possible. Elle avait pour objet la libération de la république envers ses citoyens, il ne fallait pas qu’elle eût celui de la libération des citoyens entre eux. Cela fit faire une seconde opération ; et l’on ordonna que le denier, qui n’avait été jusque-là que de dix as, en contiendrait seize. Il résulta de cette double opération que, pendant que les créanciers de la république perdaient la moitié\footnote{Ils recevaient dix onces de cuivre pour vingt.}, ceux des particuliers ne perdaient qu’un cinquième\footnote{Ils recevaient seize onces de cuivre pour vingt.} ; les marchandises n’augmentaient que d’un cinquième ; le changement réel dans la monnaie n’était que d’un cinquième : on voit les autres conséquences.\par
Les Romains se conduisirent donc mieux que nous, qui, dans nos opérations, avons enveloppé et les fortunes publiques et les fortunes particulières. Ce n’est pas tout : on va voir qu’ils les firent dans des circonstances plus favorables que nous.
\subsubsection[{Chapitre XII. Circonstances dans lesquelles les Romains firent leurs opérations sur la monnaie}]{Chapitre XII. Circonstances dans lesquelles les Romains firent leurs opérations sur la monnaie}
\noindent Il y avait anciennement très peu d’or et d’argent en Italie. Ce pays a peu ou point de mines d’or et d’argent. Lorsque Rome fut prise par les Gaulois, il ne s’y trouva que mille livres d’or\footnote{Pline, liv. XXXIII, art. 5.}. Cependant les Romains avaient saccagé plusieurs villes puissantes, et ils en avaient transporté les richesses chez eux. Ils ne se servirent longtemps que de monnaie de cuivre : ce ne fut qu’après la paix de Pyrrhus qu’ils eurent assez d’argent pour en faire de la monnaie\footnote{Freinshemius, liv. V de la seconde décade.}. Ils firent des deniers de ce métal, qui valaient dix as\footnote{{\itshape Ibid. loco citato.} Ils frappèrent aussi, dit le même auteur, des demis appelés quinaires, et des quarts appelés sesterces.}, ou dix livres de cuivre. Pour lors, la proportion de l’argent au cuivre était comme 1 à 960 : car le denier romain valant dix as ou dix livres de cuivre, il valait cent vingt onces de cuivre ; et le même denier valant un huitième d’once d’argent\footnote{Un huitième, selon Budé, un septième, selon d’autres auteurs.}, cela faisait la proportion que nous venons de dire.\par
Rome, devenue maîtresse de cette partie de l’Italie, la plus voisine de la Grèce et de la Sicile, se trouva peu {\itshape à} peu entre deux peuples riches, les Grecs et les Carthaginois ; l’argent augmenta chez elle ; et la proportion de 1 à 960 entre l’argent et le cuivre ne pouvant plus se soutenir, elle fit diverses opérations sur les monnaies, que nous ne connaissons pas. Nous savons seulement qu’au commencement de la seconde guerre punique, le denier romain ne valait plus que vingt onces de cuivre\footnote{Pline, {\itshape Hist. nat.}, liv. XXXIII, art. 13.} ; et qu’ainsi la proportion entre l’argent et le cuivre n’était plus que comme 1 est à 160. La réduction était bien considérable, puisque la république gagna cinq sixièmes sur toute la monnaie de cuivre. Mais on ne fit que ce que demandait la nature des choses, et rétablir la proportion entre les métaux qui servaient de monnaie.\par
La paix qui termina la première guerre punique, avait laissé les Romains maîtres de la Sicile. Bientôt ils entrèrent en Sardaigne, ils commencèrent à connaître l’Espagne : la masse de l’argent augmenta encore à Rome. On y fit l’opération qui réduisit le denier d’argent de vingt onces à seize\footnote{Pline, {\itshape Hist. nat.}, liv. XXXIII, art. 13.} ; et elle eut cet effet, qu’elle remit en proportion l’argent et le cuivre : cette proportion était comme 1 est à 160 ; elle fut comme 1 est à 128.\par
Examinez les Romains, vous ne les trouverez jamais si supérieurs que dans le choix des circonstances dans lesquelles ils firent les biens et les maux.
\subsubsection[{Chapitre XIII. Opérations sur les monnaies du temps des empereurs}]{Chapitre XIII. Opérations sur les monnaies du temps des empereurs}
\noindent Dans les opérations que l’on fit sur les monnaies du temps de la république, on procéda par voie de retranchement : l’État confiait au peuple ses besoins, et ne prétendait pas le séduire. Sous les empereurs, on procéda par voie d’alliage. Ces princes, réduits au désespoir par leurs libéralités mêmes, se virent obligés d’altérer les monnaies ; voie indirecte, qui diminuait le mal, et semblait ne le pas toucher : on retirait une partie du don, et on cachait la main ; et, sans parler de diminution de la paie ou des largesses, elles se trouvaient diminuées.\par
On voit encore dans les cabinets\footnote{Voyez {\itshape la Science des médailles} du P. Joubert, éd. de Paris, 1739, p. 59.}, des médailles qu’on appelle fourrées, qui n’ont qu’une lame d’argent qui couvre le cuivre. Il est parlé de cette monnaie dans un fragment du livre LXXVII de Dion\footnote{{\itshape Extrait des vertus et des vices}.}.\par
Didius Julien commença l’affaiblissement. On trouve que la monnaie de Caracalla\footnote{Voyez Savotte, part II, chap. XII ; et le {\itshape Journal des savants du} 28 juillet 1681, sur une découverte de 50 000 médailles.} avait plus de la moitié d’alliage, celle d’Alexandre Sévère\footnote{Idem, {\itshape Ibid.}} les deux tiers : l’affaiblissement continua ; et, sous Galien\footnote{Idem, {\itshape Ibid.}}, on ne voyait plus que du cuivre argenté.\par
On sent que ces opérations violentes ne sauraient avoir lieu dans ces temps-ci ; un prince se tromperait lui-même, et ne tromperait personne. Le change a appris au banquier à comparer toutes les monnaies du monde, et à les mettre à leur juste valeur ; le titre des monnaies ne peut plus être un secret. Si un prince commence le billon, tout le monde continue, et le fait pour lui ; les espèces fortes sortent d’abord, et on les lui renvoie faibles. Si, comme les empereurs romains, il affaiblissait l’argent sans affaiblir l’or, il verrait tout à coup disparaître l’or, et il serait réduit à son mauvais argent. Le change, comme j’ai dit au livre précédent\footnote{Chap. XVI.}, a ôté les grands coups d’autorité, ou du moins le succès des grands coups d’autorité.
\subsubsection[{Chapitre XIV. Comment le change gêne les états despotiques}]{Chapitre XIV. Comment le change gêne les états despotiques}
\noindent La Moscovie voudrait descendre de son despotisme, et ne le peut. L’établissement du commerce demande celui du change ; et les opérations du change contredisent toutes ses lois.\par
En 1745, la czarine fit une ordonnance pour chasser les Juifs, parce qu’ils avaient remis dans les pays étrangers l’argent de ceux qui étaient relégués en Sibérie, et celui des étrangers qui étaient au service. Tous les sujets de l’empire, comme des esclaves, n’en peuvent sortir, ni faire sortir leurs biens, sans permission. Le change, qui donne le moyen de transporter l’argent d’un pays à un autre, est donc contradictoire aux lois de Moscovie.\par
Le commerce même contredit ses lois. Le peuple n’est composé que d’esclaves attachés aux terres, et d’esclaves qu’on appelle ecclésiastiques ou gentilshommes, parce qu’ils sont les seigneurs de ces esclaves. Il ne reste donc guère personne pour le tiers-état, qui doit former les ouvriers et les marchands.
\subsubsection[{Chapitre XV. Usage de quelques pays d’Italie}]{Chapitre XV. Usage de quelques pays d’Italie}
\noindent Dans quelques pays d’Italie, on a fait des lois pour empêcher les sujets de vendre des fonds de terre pour transporter leur argent dans les pays étrangers. Ces lois pouvaient être bonnes, lorsque les richesses de chaque État étaient tellement à lui, qu’il y avait beaucoup de difficulté à les faire passer à un autre. Mais depuis que, par l’usage du change, les richesses ne sont, en quelque façon, à aucun État en particulier, et qu’il y a tant de facilité à les transporter d’un pays à un autre, c’est une mauvaise loi que celle qui ne permet pas de disposer, pour ses affaires, de ses fonds de terre, lorsqu’on peut disposer de son argent. Cette loi est mauvaise, parce qu’elle donne de l’avantage aux effets mobiliers sur les fonds de terre, parce qu’elle dégoûte les étrangers de venir s’établir dans le pays, et enfin, parce qu’on peut l’éluder.
\subsubsection[{Chapitre XVI. Du secours que l’état peut tirer des banquiers}]{Chapitre XVI. Du secours que l’état peut tirer des banquiers}
\noindent Les banquiers sont faits pour changer de l’argent, et non pas pour en prêter. Si le prince ne s’en sert que pour changer son argent, comme il ne fait que de grosses affaires, le moindre profit qu’il leur donne pour leurs remises, devient un objet considérable ; et, si on lui demande de gros profits, il peut être sûr que c’est un défaut de l’administration. Quand, au contraire, ils sont employés à faire des avances, leur art consiste à se procurer de gros profits de leur argent, sans qu’on puisse les accuser d’usure.
\subsubsection[{Chapitre XVII. Des dettes publiques}]{Chapitre XVII. Des dettes publiques}
\noindent Quelques gens ont cru qu’il était bon qu’un État dût à lui-même : ils ont pensé que cela multipliait les richesses, en augmentant la circulation.\par
Je crois qu’on a confondu un papier circulant qui représente la monnaie, ou un papier circulant qui est le signe des profits qu’une compagnie a faits ou fera sur le commerce, avec un papier qui représente une dette. Les deux premiers sont très avantageux à l’État ; le dernier ne peut l’être ; et tout ce qu’on peut en attendre, c’est qu’il soit un bon gage pour les particuliers de la dette de la nation, c’est-à-dire qu’il en procure le paiement. Mais voici les inconvénients qui en résultent.\par
l° Si les étrangers possèdent beaucoup de papiers qui représentent une dette, ils tirent, tous les ans, de la nation, une somme considérable pour les intérêts ;\par
2° Dans une nation ainsi perpétuellement débitrice, le change doit être très bas ;\par
3{\itshape °} L’impôt levé pour le paiement des intérêts de la dette, fait tort aux manufactures, en rendant la main de l’ouvrier plus chère ;\par
4° On ôte les revenus véritables de l’État à ceux qui ont de l’activité ou de l’industrie, pour les transporter aux gens oisifs ; c’est-à-dire qu’on donne des commodités pour travailler à ceux qui ne travaillent point, et des difficultés pour travailler à ceux qui travaillent.\par
Voilà les inconvénients ; je n’en connais point les avantages. Dix personnes ont chacune mille écus de revenu en fonds de terre ou en industrie ; cela fait pour la nation, à cinq pour cent, un capital de deux cent mille écus. Si ces dix personnes emploient la moitié de leur revenu, c’est-à-dire cinq mille écus, pour payer les intérêts de cent mille écus qu’elles ont empruntés à d’autres, cela ne fait encore pour l’État que deux cent mille écus : c’est, dans le langage des algébristes : 200 000 écus – 100 000 écus + 100 000 écus = 200 000 écus.\par
Ce qui peut jeter dans l’erreur, c’est qu’un papier qui représente la dette d’une nation est un signe de richesse ; car il n’y a qu’un État riche qui puisse soutenir un tel papier sans tomber dans la décadence. Que s’il n’y tombe pas, il faut que l’État ait de grandes richesses d’ailleurs. On dit qu’il n’y a point de mal, parce qu’il y a des ressources contre ce mal ; et on dit que le mal est un bien, parce que les ressources surpassent le mal.
\subsubsection[{Chapitre XVIII. Du payement des dettes publiques}]{Chapitre XVIII. Du payement des dettes publiques}
\noindent Il faut qu’il y ait une proportion entre l’État créancier et l’État débiteur. L’État peut être créancier à l’infini ; mais il ne peut être débiteur qu’à un certain degré ; et quand on est parvenu à passer ce degré, le titre de créancier s’évanouit.\par
Si cet État a encore un crédit qui n’ait point reçu d’atteinte, il pourra faire ce qu’on a pratiqué si heureusement dans un État d’Europe\footnote{L’Angleterre.} : c’est de se procurer une grande quantité d’espèces, et d’offrir à tous les particuliers leur remboursement, à moins qu’ils ne veuillent réduire l’intérêt. En effet, comme, lorsque l’État emprunte, ce sont les particuliers qui fixent le taux de l’intérêt ; lorsque l’État veut payer, c’est à lui à le fixer.\par
Il ne suffit pas de réduire l’intérêt : il faut que le bénéfice de la réduction forme un fonds d’amortissement pour payer chaque année une partie des capitaux ; opération d’autant plus heureuse que le succès en augmente tous les jours.\par
Lorsque le crédit de l’État n’est pas entier, c’est une nouvelle raison pour chercher à former un fonds d’amortissement ; parce que ce fonds une fois établi rend bientôt la confiance.\par
1˚ Si l’État est une république, dont le gouvernement comporte, par sa nature, que l’on y fasse des projets pour longtemps, le capital du fonds d’amortissement peut être peu considérable : il faut, dans une monarchie, que ce capital soit plus grand ;\par
2˚ Les règlements doivent être tels, que tous les citoyens de l’État portent le poids de l’établissement de ce fonds, parce qu’ils ont tous le poids de l’établissement de la dette ; le créancier de l’État, par les sommes qu’il contribue, payant lui-même à lui-même ;\par
3{\itshape °} Il y a quatre classes de gens qui paient les dettes de l’État : les propriétaires des fonds de terre, ceux qui exercent leur industrie par le négoce, les laboureurs et artisans, enfin les rentiers de l’État ou des particuliers. De ces quatre classes, la dernière, dans un cas de nécessité, semblerait devoir être la moins ménagée, parce que c’est une classe entièrement passive dans l’État, tandis que ce même État est soutenu par la force active des trois autres. Mais, comme on ne peut la charger plus sans détruire la confiance publique, dont l’État en général, et ces trois classes en particulier, ont un souverain besoin ; comme la foi publique ne peut manquer à un certain nombre de citoyens, sans paraître manquer à tous ; comme la classe des créanciers est toujours la plus exposée aux projets des ministres, et qu’elle est toujours sous les yeux et sous la main, il faut que l’État lui accorde une singulière protection, et que la partie débitrice n’ait jamais le moindre avantage sur celle qui est créancière.
\subsubsection[{Chapitre XIX. Des prêts à intérêt}]{Chapitre XIX. Des prêts à intérêt}
\noindent L’argent est le signe des valeurs. Il est clair que celui qui a besoin de ce signe doit le louer, comme il fait toutes les choses dont il peut avoir besoin. Toute la différence est que les autres choses peuvent ou se louer ou s’acheter ; au lieu que l’argent, qui est le prix des choses, se loue et ne s’achète pas\footnote{On ne parle point des cas où l’or et l’argent sont considérés comme marchandises.}.\par
C’est bien une action très bonne de prêter à un autre son argent sans intérêt : mais on sent que ce ne peut être qu’un conseil de religion, et non une loi civile.\par
Pour que le commerce puisse se bien faire, il faut que l’argent ait un prix, mais que ce prix soit peu considérable. S’il est trop haut, le négociant, qui voit qu’il lui en coûterait plus en intérêts qu’il ne pourrait gagner dans son commerce, n’entreprend rien. Si l’argent n’a point de prix, personne n’en prête, et le négociant n’entreprend rien non plus.\par
Je me trompe quand je dis que personne n’en prête. Il faut toujours que les affaires de la société aillent ; l’usure s’établit, mais avec les désordres que l’on a éprouvés dans tous les temps.\par
La loi de Mahomet confond l’usure avec le prêt à intérêt. L’usure augmente dans les pays mahométans à proportion de la sévérité de la défense : le prêteur s’indemnise du péril de la contravention.\par
Dans ces pays d’Orient, la plupart des hommes n’ont rien d’assuré ; il n’y a presque point de rapport entre la possession actuelle d’une somme, et l’espérance de la ravoir après l’avoir prêtée : l’usure y augmente donc à proportion du péril de l’insolvabilité.
\subsubsection[{Chapitre XX. Des usures maritimes}]{Chapitre XX. Des usures maritimes}
\noindent La grandeur de l’usure maritime est fondée sur deux choses : le péril de la mer, qui fait qu’on ne s’expose à prêter son argent que pour en avoir beaucoup davantage ; et la facilité que le commerce donne à l’emprunteur de faire promptement de grandes affaires, et en grand nombre ; au lieu que les usures de terre, n’étant fondées sur aucune de ces deux raisons, sont ou proscrites par les législateurs, ou, ce qui est plus sensé, réduites à de justes bornes.
\subsubsection[{Chapitre XXI. Du prêt par contrat et de l’usure chez les Romains}]{Chapitre XXI. Du prêt par contrat et de l’usure chez les Romains}
\noindent Outre le prêt fait pour le commerce, il y a encore une espèce de prêt fait par un contrat civil, d’où résulte un intérêt ou usure.\par
Le peuple, chez les Romains, augmentant tous les jours sa puissance, les magistrats cherchèrent à le flatter et à lui faire faire les lois qui lui étaient les plus agréables. Il retrancha les capitaux ; il diminua les intérêts ; il défendit d’en prendre ; il ôta les contraintes par corps ; enfin l’abolition des dettes fut mise en question toutes les fois qu’un tribun voulut se rendre populaire.\par
Ces continuels changements, soit par des lois, soit par des plébiscites, naturalisèrent à Rome l’usure ; car les créanciers voyant le peuple leur débiteur, leur législateur et leur juge, n’eurent plus de confiance dans les contrats. Le peuple, comme un débiteur décrédité, ne tentait à lui prêter que par de gros profits, d’autant plus que, si les lois ne venaient que de temps en temps, les plaintes du peuple étaient continuelles, et intimidaient toujours les créanciers. Cela fit que tous les moyens honnêtes de prêter et d’emprunter furent abolis à Rome, et qu’une usure affreuse, toujours foudroyée\footnote{Tacite, {\itshape Annales}, liv. VI.} et toujours renaissante, s’y établit. Le mal venait de ce que les choses n’avaient pas été ménagées. Les lois extrêmes dans le bien font naître le mal extrême. Il fallut payer pour le prêt de l’argent et pour le danger des peines de la loi.
\subsubsection[{Chapitre XXII. Continuation du même sujet}]{Chapitre XXII. Continuation du même sujet}
\noindent Les premiers Romains n’eurent point de lois pour régler le taux de l’usure\footnote{Usure et intérêt signifiaient la même chose chez les Romains.}. Dans les démêlés qui se formèrent là-dessus entre les plébéiens et les patriciens, dans la sédition même du mont Sacré\footnote{Voyez Denys d’Halicarnasse qui l’a si bien décrite.}, on n’allégua d’un côté que la foi, et de l’autre que la dureté des contrats.\par
On suivait donc les conventions particulières ; et je crois que les plus ordinaires étaient de douze pour cent par an. Ma raison est que, dans le langage ancien chez les Romains, l’intérêt à six pour cent était appelé la moitié de l’usure, l’intérêt à trois pour cent le quart de l’usure\footnote{{\itshape Usurae semisses, trientes, quadrantes}. Voyez là-dessus les divers traités du Digeste et du Code {\itshape de usuris ;} et surtout la loi 17, avec sa note, au ff.{\itshape  de usuris.}} : l’usure totale était donc l’intérêt à douze pour cent.\par
Que si l’on demande comment de si grosses usures avaient pu s’établir chez un peuple qui était presque sans commerce, je dirai que ce peuple, très souvent obligé d’aller sans solde à la guerre, avait très souvent besoin d’emprunter ; et que, faisant sans cesse des expéditions heureuses, il avait très souvent la facilité de payer. Et cela se sent bien dans le récit des démêlés qui s’élevèrent à cet égard : on n’y disconvient point de l’avarice de ceux qui prêtaient ; mais on dit que ceux qui se plaignaient auraient pu payer, s’ils avaient eu une conduite réglée\footnote{Voyez les discours d’Appius là-dessus, dans Denys d’Halicarnasse.}.\par
On faisait donc des lois qui n’influaient que sur la situation actuelle : on ordonnait, par exemple, que ceux qui s’enrôleraient pour la guerre que l’on avait à soutenir, ne seraient point poursuivis par leurs créanciers ; que ceux qui étaient dans les fers seraient délivrés ; que les plus indigents seraient menés dans les colonies : quelquefois on ouvrait le trésor public. Le peuple s’apaisait par le soulagement des maux présents ; et, comme il ne demandait rien pour la suite, le sénat n’avait garde de le prévenir.\par
Dans le temps que le sénat défendait avec tant de constance la cause des usures, l’amour de la pauvreté, de la frugalité, de la médiocrité, était extrême chez les Romains : mais telle était la constitution, que les principaux citoyens portaient toutes les charges de l’État, et que le bas peuple ne payait rien. Quel moyen de priver ceux-là du droit de poursuivre leurs débiteurs, et de leur demander d’acquitter leurs charges, et de subvenir aux besoins pressants de la république ?\par
Tacite\footnote{{\itshape Annales}, liv. VI.} dit que la loi des Douze Tables fixa l’intérêt à un pour cent par an. Il est visible qu’il s’est trompé, et qu’il a pris pour la loi des Douze Tables une autre loi dont je vais parler. Si la loi des Douze Tables avait réglé cela, comment, dans les disputes qui s’élevèrent depuis entre les créanciers et les débiteurs, ne se serait-on pas servi de son autorité ? On ne trouve aucun vestige de cette loi sur le prêt à intérêt ; et, pour peu qu’on soit versé dans l’histoire de Rome, on verra qu’une loi pareille ne devait point être l’ouvrage des décemvirs.\par
La loi Licinienne, faite quatre-vingt-cinq ans\footnote{L’an de Rome 388. Tate-Live, liv. VI.} après la loi des Douze Tables, fut une de ces lois passagères dont nous avons parlé. Elle ordonna qu’on retrancherait du capital ce qui avait été payé pour les intérêts, et que le reste serait acquitté en trois paiements égaux.\par
L’an 398 de Rome, les tribuns Duellius et Menenius firent passer une loi qui réduisait les intérêts à un pour cent par an\footnote{{\itshape Unciaria usure}. Tite-Live, liv. VII.}. C’est cette loi que Tacite\footnote{{\itshape Annales}, liv. VI.} confond avec la loi des Douze Tables ; et c’est la première qui ait été faite chez les Romains pour fixer le taux de l’intérêt. Dix ans après\footnote{Sous le consulat de L. Manlius Torquatus et de C. Plautius, selon Tite-Live, liv. VII ; et c’est la loi dont parle Tacite, {\itshape Annales}, liv. VI.}, cette usure fut réduite à la Moitié\footnote{{\itshape Semiunciaria usura}.} ; dans la suite on l’ôta tout à fait\footnote{Comme le dit Tacite, {\itshape Annales}, liv. VI.} {\itshape ;} et, si nous en croyons quelques auteurs qu’avait vus Tite-Live, ce fut sous le consulat de C. Martius Rutilius et de Q. Servilius\footnote{La loi en fut faite à la poursuite de M. Genucius, tribun du peuple. Tite-Live, liv. VII, à la fin.}, l’an 413 de Rome.\par
Il en fut de cette loi comme de toutes celles où le législateur a porté les choses à l’excès : on trouva un moyen de l’éluder. Il en fallut faire beaucoup d’autres pour la confirmer, corriger, tempérer. Tantôt on quitta les lois pour suivre les usages\footnote{{\itshape Veteri jam more foenus receptum erat}. Appien, {\itshape De la Guerre civile}, liv. I.}, tantôt on quitta les usages pour suivre les lois ; mais, dans ce cas, l’usage devait aisément prévaloir. Quand un homme emprunte, il trouve un obstacle dans la loi même qui est faite en sa faveur : cette loi a contre elle, et celui qu’elle secourt, et celui qu’elle condamne. Le préteur Sempronius Asellus ayant permis aux débiteurs d’agir en conséquence des lois\footnote{{\itshape Permisit eos legibus agere}. Appien, {\itshape De la Guerre civile}, liv. I, 54 ; et l’{\itshape Epitome} de Tite-Live, liv. LXIV.}, fut tué par les créanciers\footnote{L’an de Rome 663.} pour avoir voulu rappeler la mémoire d’une rigidité qu’on ne pouvait plus soutenir.\par
Je quitte la ville pour jeter un peu les yeux sur les provinces.\par
J’ai dit ailleurs\footnote{Liv. XI, chap. XIX.} que les provinces romaines étaient désolées par un gouvernement despotique et dur. Ce n’est pas tout : elles l’étaient encore par des usures affreuses.\par
Cicéron dit\footnote{{\itshape Lettres à Atticus}, liv. V, lettre XXI.} que ceux de Salamine voulaient emprunter de l’argent à Rome, et qu’ils ne le pouvaient pas à cause de la loi Gabinienne. il faut que je cherche ce que c’était que cette loi.\par
Lorsque les prêts à intérêt eurent été défendus à Rome, on imagina toutes sortes de moyens pour éluder la loi\footnote{Tite-Live.} ; et, comme les alliés\footnote{{\itshape Ibid.}} et ceux de la nation latine n’étaient point assujettis aux lois civiles des Romains, on se servit d’un Latin ou d’un allié qui prêtait son nom, et paraissait être le créancier. La loi n’avait donc fait que soumettre les créanciers à une formalité, et le peuple n’était pas soulagé.\par
Le peuple se plaignit de cette fraude ; et Marcus Sempronius, tribun du peuple, par l’autorité du sénat, fit faire un plébiscite\footnote{L’an 561 de Rome. Voyez Tite-Live.} qui portait qu’en fait de prêts, les lois qui défendaient les prêts à usure entre un citoyen romain et un autre citoyen romain, auraient également lieu entre un citoyen et un allié, ou un Latin.\par
Dans ces temps-là, on appelait alliés les peuples de l’Italie proprement dite, qui s’étendait jusqu’à l’Arno et le Rubicon, et qui n’était point gouvernée en provinces romaines.\par
Tacite\footnote{{\itshape Annales}, liv. VI.} dit qu’on faisait toujours de nouvelles fraudes aux lois faites pour arrêter les usures. Quand on ne put plus prêter ni emprunter sous le nom d’un allié, il fut aisé de faire paraître un homme des provinces, qui prêtait son nom.\par
Il fallait une nouvelle loi contre ces abus ; et Gabinius\footnote{L’an 615 de Rome.}, faisant la loi fameuse qui avait pour objet d’arrêter la corruption dans les suffrages, dut naturellement penser que le meilleur moyen pour y parvenir était de décourager les emprunts : ces deux choses étaient naturellement liées ; car les usures augmentaient toujours au temps des élections\footnote{Voyez les {\itshape Lettres de Cicéron à Atticus}, liv. IV, lettres XV et XVI.} parce qu’on avait besoin d’argent pour gagner des voix. On voit bien que la loi Gabinienne avait étendu le sénatus-consulte Sempronien aux provinciaux, puisque les Salaminiens ne pouvaient emprunter de l’argent à Rome, à cause de cette loi. Brutus, sous des noms empruntés, leur en prêta\footnote{Cicéron à Atticus, liv. VI, lettre I.} à quatre pour cent par mois\footnote{Pompée, qui avait prêté au roi Ariobarsane six cents talents, se faisait payer trente-trois talents attiques tous les trente jours. Cicéron à Atticus, liv. V, lettre XXI ; liv. VI, lettre 1, 3.}, et obtint pour cela deux sénatus-consultes, dans le premier desquels il était dit que ce prêt ne serait pas regardé comme une fraude faite à la loi, et que le gouverneur de Cilicie jugerait en conformité des conventions portées par le billet des Salaminiens\footnote{{\itshape Ut neque Salaminis, neque cui eis dedisset, fraudi esset.Ibid.}}.\par
Le prêt à intérêt étant interdit par la loi Gabinienne entre les gens des provinces et les citoyens romains, et ceux-ci ayant pour lors tout l’argent de l’univers entre leurs mains, il fallut les tenter par de grosses usures, qui fissent disparaître, aux yeux de l’avarice, le danger de perdre la dette. Et, comme il y avait à Rome des gens puissants qui intimidaient les magistrats, et faisaient taire les lois, ils furent plus hardis à prêter, et plus hardis à exiger de grosses usures. Cela fit que les provinces furent tour à tour ravagées par tous ceux qui avaient du crédit à Rome ; et, comme chaque gouverneur faisait son édit en entrant dans sa province\footnote{L’édit de Cicéron la fixait à un pour cent par mois, avec l’usure de l’usure au bout de l’an. Quant aux fermiers de la république, il les engageait à donner un délai à leurs débiteurs. Si ceux-ci ne payaient pas au temps fixé, il adjugeait l’usure portée par le billet. Cicéron à Atticus, liv. VI, lettre I.}, dans lequel il mettait à l’usure le taux qu’il lui plaisait, l’avarice prêtait la main à la législation, et la législation à l’avarice.\par
Il faut que les affaires aillent ; et un État est perdu si tout y est dans l’inaction. Il y avait des occasions où il fallait que les villes, les corps, les sociétés des villes, les particuliers, empruntassent, et on n’avait que trop besoin d’emprunter, ne fût-ce que pour subvenir aux ravages des armées, aux rapines des magistrats, aux concussions des gens d’affaires, et aux mauvais usages qui s’établissaient tous les jours ; car on ne fut jamais ni si riche, ni si pauvre. Le sénat, qui avait la puissance exécutrice, donnait par nécessité, souvent par faveur, la permission d’emprunter des citoyens romains, et faisait là-dessus des sénatus-consultes. Mais ces sénatus-consultes mêmes étaient décrédités par la loi : ces sénatus-consultes\footnote{Voyez ce que dit Luccéius, lettre XXI à Atticus, liv. V. Il y eut même un sénatus-consulte général pour fixer l’usure à un pour cent par mois. Voyez la même lettre.} pouvaient donner occasion au peuple de demander de nouvelles tables ; ce qui, augmentant le danger de la perte du capital, augmentait encore l’usure. Je le dirai toujours, c’est la modération qui gouverne les hommes, et non pas les excès.\par
Celui-là paie moins, dit Ulpien\footnote{Leg. 12, ff. {\itshape De verborum significatione}.} qui paie plus tard. C’est ce principe qui conduisit les législateurs, après la destruction de la république romaine.
\subsection[{Livre vingt-troisième. Des lois dans le rapport qu’elles ont avec le nombre des habitants}]{Livre vingt-troisième. Des lois dans le rapport qu’elles ont avec le nombre des habitants}
\subsubsection[{Chapitre I. Des hommes et des animaux par rapport à la multiplication de leur espèce}]{Chapitre I. Des hommes et des animaux par rapport à la multiplication de leur espèce}
\noindent Ô Vénus ! ô mère de l’Amour !\par
-------------------------------------------\par
Dès le premier beau jour que ton astre ramène,\par
Les zéphyrs font sentir leur amoureuse haleine ;\par
La terre orne son sein de brillantes couleurs,\par
Et l’air est parfumé du doux esprit des fleurs.\par
On entend les oiseaux, frappés de ta puissance,\par
Par mille tons lascifs célébrer ta présence :\par
Pour la belle génisse, on voit les fiers taureaux,\par
Ou bondir dans la plaine, ou traverser les eaux :\par
Enfin, les habitants des bois et des montagnes,\par
Des fleuves et des mers, et des vertes campagnes,\par
Brûlant à ton aspect d’amour et de désir,\par
S’engagent à peupler par l’attrait du plaisir :\par
Tant on aime à te suivre, et ce charmant empire,\par
Que donne la beauté sur tout ce qui respire\footnote{Traduction du commencement de Lucrèce, par le sieur d’Hesnaut.} !\par
Les femelles des animaux ont à peu près une fécondité constante. Mais, dans l’espèce humaine, la manière de penser, le caractère, les passions, les fantaisies, les caprices, l’idée de conserver sa beauté, l’embarras de la grossesse, celui d’une famille trop nombreuse troublent la propagation de mille manières.
\subsubsection[{Chapitre II. Des mariages}]{Chapitre II. {\itshape Des mariages}}
\noindent L’obligation naturelle qu’a le père de nourrir ses enfants, a fait établir le mariage, qui déclare celui qui doit remplir cette obligation. Les peuples\footnote{Les Garamantes.} dont parle Pomponius Mela\footnote{Liv. I, chap. VIII.} ne le fixaient que par la ressemblance.\par
Chez les peuples bien policés, le père est celui que les lois, par la cérémonie du mariage, ont déclaré devoir être tel\footnote{{\itshape Pater est quem nuptiae demonstrant.}}, parce qu’elles trouvent en lui la personne qu’elles cherchent.\par
Cette obligation, chez les animaux, est telle que la mère peut ordinairement y suffire. Elle a beaucoup plus d’étendue chez les hommes : leurs enfants ont de la raison, mais elle ne leur vient que par degrés : il ne suffit pas de les nourrir, il faut encore les conduire : déjà ils pourraient vivre, et ils ne peuvent pas se gouverner.\par
Les conjonctions illicites contribuent peu à la propagation de l’espèce. Le père, qui a l’obligation naturelle de nourrir et d’élever les enfants, n’y est point fixé ; et la mère, à qui l’obligation reste, trouve mille obstacles, par la honte, les remords, la gêne de son sexe, la rigueur des lois : la plupart du temps elle manque de moyens.\par
Les femmes qui se sont soumises à une prostitution publique, ne peuvent avoir la commodité d’élever leurs enfants. Les peines de cette éducation sont même incompatibles avec leur condition ; et elles sont si corrompues, qu’elles ne sauraient avoir la confiance de la loi.\par
Il suit de tout ceci, que la continence publique est naturellement jointe à la propagation de l’espèce.
\subsubsection[{Chapitre III. De la condition des enfants}]{Chapitre III. De la condition des enfants}
\noindent C’est la raison qui dicte que, quand il y a un mariage, les enfants suivent la condition du père ; et que, quand il n’y en a point, ils ne peuvent concerner que la mère\footnote{C’est pour cela que, chez les nations qui ont des esclaves, l’enfant suit presque toujours la condition de la mère.}.
\subsubsection[{Chapitre IV. Des familles}]{Chapitre IV. {\itshape Des familles}}
\noindent Il est presque reçu partout que la femme passe dans la famille du mari. Le contraire est, sans aucun inconvénient, établi à Formose\footnote{Le P. Du Halde, t. I, p. 156.} où le mari va former celle de la femme.\par
Cette loi, qui fixe la famille dans une suite de personnes du même sexe, contribue beaucoup, indépendamment des premiers motifs, à la propagation de l’espèce humaine. La famille est une sorte de propriété : un homme, qui a des enfants du sexe qui ne la perpétue pas, n’est jamais content qu’il n’en ait de celui qui la perpétue.\par
Les noms, qui donnent aux hommes l’idée d’une chose qui semble ne devoir pas périr, sont très propres à inspirer à chaque famille le désir d’étendre sa durée. Il y a des peuples chez lesquels les noms distinguent les familles : il y en a où ils ne distinguent que les personnes ; ce qui n’est pas si bien.
\subsubsection[{Chapitre V. Des divers ordres de femmes légitimes}]{Chapitre V. Des divers ordres de femmes légitimes}
\noindent Quelquefois les lois et la religion ont établi plusieurs sortes de conjonctions civiles ; et cela est ainsi chez les mahométans, où il y a divers ordres de femmes, dont les enfants se reconnaissent par la naissance dans la maison, ou par des contrats civils, ou même par l’esclavage de la mère et la reconnaissance subséquente du père.\par
Il serait contre la raison que la loi flétrit dans les enfants ce qu’elle a approuvé dans le père : tous ces enfants y doivent donc succéder, à moins que quelque raison particulière ne s’y oppose, comme au Japon, où il n’y a que les enfants de la femme donnée par l’empereur qui succèdent. La politique y exige que les biens que l’empereur donne ne soient pas trop partagés, parce qu’ils sont soumis à un service, comme étaient autrefois nos fiefs.\par
Il y a des pays où une femme légitime jouit, dans la maison, à peu près des honneurs qu’a dans nos climats une femme unique : là, les enfants des concubines sont censés appartenir à la première femme. Cela est ainsi établi à la Chine. Le respect filial\footnote{Le P. Du Halde, t. II, p. 124.} la cérémonie d’un deuil rigoureux ne sont point dus à la mère naturelle, mais à cette mère que donne la loi.\par
À l’aide d’une telle fiction\footnote{On distingue les femmes en grandes et petites, c’est-à-dire en légitimes ou non ; mais il n’y a point une pareille distinction entre les enfants. « C’est la grande doctrine de l’empire », est-il dit dans un ouvrage chinois sur la morale, traduit par le même père, p. 140.}, il n’y a plus d’enfants bâtards ; et dans les pays où cette fiction n’a pas lieu, on voit bien que la loi qui légitime les enfants des concubines est une loi forcée ; car ce serait le gros de la nation qui serait flétri par la loi. Il n’est pas question non plus, dans ces pays, d’enfants adultérins. Les séparations des femmes, la clôture, les eunuques, les verrous, rendent la chose si difficile que la loi la juge impossible. D’ailleurs le même glaive exterminerait la mère et l’enfant.
\subsubsection[{Chapitre VI. Des bâtards dans les divers gouvernements}]{Chapitre VI. Des bâtards dans les divers gouvernements}
\noindent On ne connaît donc guère les bâtards dans les pays où la polygamie est permise ; on les connaît dans ceux où la loi d’une seule femme est établie. Il a fallu, dans ces pays, flétrir le concubinage ; il a donc fallu flétrir les enfants qui en étaient nés.\par
Dans les républiques, où il est nécessaire que les mœurs soient pures, les bâtards doivent être encore plus odieux que dans les monarchies.\par
On fit peut-être à Rome des dispositions trop dures contre eux. Mais les institutions anciennes mettant tous les citoyens dans la nécessité de se marier, les mariages étant d’ailleurs adoucis par la permission de répudier, ou de faire divorce, il n’y avait qu’une très grande corruption de mœurs qui pût porter au concubinage.\par
Il faut remarquer que la qualité de citoyen étant considérable dans les démocraties, où elle emportait avec elle la souveraine puissance, il s’y faisait surtout des lois sur l’état des bâtards, qui avaient moins de rapport à la chose même et à l’honnêteté du mariage qu’à la constitution particulière de la république. Ainsi le peuple a quelquefois reçu pour citoyens les bâtards\footnote{Voyez Aristote, {\itshape Politique}, liv. VI, chap. IV.}, afin d’augmenter sa puissance contre les grands. Ainsi à Athènes, le peuple retrancha les bâtards du nombre des citoyens, pour avoir une plus grande portion du blé que lui avait envoyé le roi d’Égypte. Enfin, Aristote\footnote{{\itshape lbid.}, liv. III, chap. III.} nous apprend que, dans plusieurs villes, lorsqu’il n’y avait point assez de citoyens, les bâtards succédaient, et que, quand il y en avait assez, ils ne succédaient pas.
\subsubsection[{Chapitre VII. Du consentement des pères au mariage}]{Chapitre VII. Du consentement des pères au mariage}
\noindent Le consentement des pères est fondé sur leur puissance, c’est-à-dire sur leur droit de propriété ; il est encore fondé sur leur amour, sur leur raison, et sur l’incertitude de celle de leurs enfants, que l’âge tient dans l’état d’ignorance, et les passions dans l’état d’ivresse.\par
Dans les petites républiques ou institutions singulières dont nous avons parlé, il peut y avoir des lois qui donnent aux magistrats une inspection sur les mariages des enfants des citoyens, que la nature avait déjà donnée aux pères. L’amour du bien public y peut être tel, qu’il égale ou surpasse tout autre amour. Ainsi Platon voulait que les magistrats réglassent les mariages : ainsi les magistrats lacédémoniens les dirigeaient-ils.\par
Mais, dans les institutions ordinaires, c’est aux pères à marier leurs enfants ; leur prudence à cet égard sera toujours au-dessus de toute autre prudence. La nature donne aux pères un désir de procurer à leurs enfants des successeurs, qu’ils sentent à peine pour eux-mêmes. Dans les divers degrés de progéniture, ils se voient avancer insensiblement vers l’avenir. Mais que serait-ce, si la vexation et l’avarice allaient au point d’usurper l’autorité des pères ? Écoutons Thomas Gage\footnote{Relation de Thomas Gage, p. 171.} sur la conduite des Espagnols dans les Indes :\par
« Pour augmenter le nombre des gens qui paient le tribut, il faut que tous les Indiens qui ont quinze ans se marient ; et même on a réglé le temps du mariage des Indiens à quatorze ans pour les mâles, et à treize pour les filles. On se fonde sur un canon qui dit que la malice peut suppléer à l’âge. » Il vit faire un de ces dénombrements : c’était, dit-il, une chose honteuse. Ainsi, dans l’action du monde qui doit être la plus libre, les Indiens sont encore esclaves.
\subsubsection[{Chapitre VIII. Continuation du même sujet}]{Chapitre VIII. Continuation du même sujet}
\noindent En Angleterre, les filles abusent souvent de la loi pour se marier à leur fantaisie, sans consulter leurs parents. Je ne sais pas si cet usage n’y pourrait pas être plus toléré qu’ailleurs, par la raison que les lois n’y ayant point établi un célibat monastique, les filles n’y ont d’état à prendre que celui du mariage, et ne peuvent {\itshape s’y} refuser. En France, au contraire, où le monachisme est établi, les filles ont toujours la ressource du célibat ; et la loi qui leur ordonne d’attendre le consentement des pères, y pourrait être plus convenable. Dans cette idée, l’usage d’Italie et d’Espagne serait le moins raisonnable : le monachisme y est établi, et l’on peut s’y marier sans le consentement des pères.
\subsubsection[{Chapitre IX. Des filles}]{Chapitre IX. {\itshape Des filles}}
\noindent Les filles, que l’on ne conduit que par le mariage aux plaisirs et à la liberté, qui ont un esprit qui n’ose penser, un cœur qui n’ose sentir, des yeux qui n’osent voir, des oreilles qui n’osent entendre, qui ne se présentent que pour se montrer stupides ; condamnées sans relâche à des bagatelles et à des préceptes, sont assez portées au mariage : ce sont les garçons qu’il faut encourager.
\subsubsection[{Chapitre X. Ce qui détermine au mariage}]{Chapitre X. Ce qui détermine au mariage}
\noindent Partout où il se trouve une place où deux personnes peuvent vivre commodément, il se fait un mariage. La nature y porte assez, lorsqu’elle n’est point arrêtée par la difficulté de la subsistance.\par
Les peuples naissants se multiplient et croissent beaucoup. Ce serait chez eux une grande incommodité de vivre dans le célibat : ce n’en est point une d’avoir beaucoup d’enfants. Le contraire arrive lorsque la nation est formée.
\subsubsection[{Chapitre XI. De la dureté du gouvernement}]{Chapitre XI. De la dureté du gouvernement}
\noindent Les gens qui n’ont absolument rien, comme les mendiants, ont beaucoup d’enfants. C’est qu’ils sont dans le cas des peuples naissants : il n’en coûte rien au père pour donner son art à ses enfants, qui même sont, en naissant, des instruments de cet art. Ces gens, dans un pays riche ou superstitieux, se multiplient, parce qu’ils n’ont pas les charges de la société, mais sont eux-mêmes les charges de la société. Mais les gens qui ne sont pauvres que parce qu’ils vivent dans un gouvernement dur, qui regardent leur champ moins comme le fondement de leur subsistance que comme un prétexte à la vexation ; ces gens-là, dis-je, font peu d’enfants. Ils n’ont pas même leur nourriture ; comment pourraient-ils songer à la partager ? Ils ne peuvent se soigner dans leurs maladies ; comment pourraient-ils élever des créatures qui sont dans une maladie continuelle, qui est l’enfance ?\par
C’est la facilité de parler, et l’impuissance d’examiner, qui ont fait dire que plus les sujets étaient pauvres, plus les familles étaient nombreuses ; que plus on était chargé d’impôts, plus on se mettait en état de les payer : deux sophismes qui ont toujours perdu, et qui perdront à jamais les monarchies.\par
La dureté du gouvernement peut aller jusqu’à détruire les sentiments naturels, par les sentiments naturels même. Les femmes de l’Amérique\footnote{{\itshape Relation} de Thomas Gage p. 58.} ne se faisaient-elles pas avorter, pour que leurs enfants n’eussent pas des maîtres aussi cruels ?
\subsubsection[{Chapitre XII. Du nombre des filles et des garçons dans différents pays}]{Chapitre XII. Du nombre des filles et des garçons dans différents pays}
\noindent J’ai déjà dit\footnote{Au liv. XVI, chap. IV.} qu’en Europe il naît un peu plus de garçons que de filles. On a remarqué qu’au Japon\footnote{Voyez Kempfer, qui rapporte un dénombrement de Méaco.}, il naissait un peu plus de filles que de garçons. Toutes choses égales, il y aura plus de femmes fécondes au Japon qu’en Europe, et par conséquent plus de peuple.\par
Des relations\footnote{{\itshape Recueil des voyages qui ont servi à l’établissement de la Compagnie des Indes}, t. I, p. 347.} disent qu’à Bantam il y a dix filles pour un garçon : une disproportion pareille, qui ferait que le nombre des familles y serait au nombre de celle des autres climats comme un est à cinq et demi, serait excessive. Les familles y pourraient être plus grandes à la vérité ; mais il y a peu de gens assez aisés pour pouvoir entretenir une si grande famille.
\subsubsection[{Chapitre XIII. Des ports de mer}]{Chapitre XIII. Des ports de mer}
\noindent Dans les ports de mer, où les hommes s’exposent à mille dangers, et vont mourir ou vivre dans des climats reculés, il y a moins d’hommes que de femmes ; cependant on y voit plus d’enfants qu’ailleurs : cela vient de la facilité de la subsistance. Peut-être même que les parties huileuses du poisson sont plus propres à fournir cette matière qui sert à la génération. Ce serait une des causes de ce nombre infini de peuple qui est au Japon\footnote{Le Japon est composé d’îles, il y a beaucoup de rivages, et la mer est très poissonneuse.} et à la Chine\footnote{La Chine est pleine de ruisseaux.}, où l’on ne vit presque que de poisson\footnote{Voyez le P. Du Halde, t. II, p. 139, 142 et suivantes.}. Si cela était, de certaines règles monastiques, qui obligent de vivre de poisson, seraient contraires à l’esprit du législateur même.
\subsubsection[{Chapitre XIV. Des productions de la terre qui demandent plus ou moins d’hommes}]{Chapitre XIV. Des productions de la terre qui demandent plus ou moins d’hommes}
\noindent Les pays de pâturages sont peu peuplés, parce que peu de gens y trouvent de l’occupation ; les terres à blé occupent plus d’hommes, et les vignobles infiniment davantage.\par
En Angleterre\footnote{La plupart des propriétaires des fonds de terre, dit Burnet, trouvant plus de profit en la vente de leur laine que de leur blé, enfermèrent leurs possessions. Les communes qui mouraient de faim, se soulevèrent : on proposa une loi agraire ; le jeune roi écrivit même là-dessus : on fit des proclamations contre ceux qui avaient renfermé leurs terres. {\itshape Abrégé de l’histoire de} la {\itshape réforme}, p. 44 et 83.}, on s’est souvent plaint que l’augmentation des pâturages diminuait les habitants ; et on observe, en France, que la grande quantité de vignobles y est une des grandes causes de la multitude des hommes.\par
Les pays où des mines de charbon fournissent des matières propres à brûler, ont cet avantage sur les autres, qu’il n’y faut point de forêts, et que toutes les terres peuvent être cultivées.\par
Dans les lieux où croît le riz, il faut de grands travaux pour ménager les eaux : beaucoup de gens y peuvent donc être occupés. Il y a plus : il y faut moins de terre pour fournir à la substance d’une famille, que dans ceux qui produisent d’autres grains ; enfin la terre, qui est employée ailleurs à la nourriture des animaux, y sert immédiatement à la subsistance des hommes ; le travail que font ailleurs les animaux, est fait là par les hommes ; et la culture des terres devient pour les hommes une immense manufacture.
\subsubsection[{Chapitre XV. Du nombre des habitants par rapport aux arts}]{Chapitre XV. Du nombre des habitants par rapport aux arts}
\noindent Lorsqu’il y a une loi agraire, et que les terres sont également partagées, le pays peut être très peuplé, quoiqu’il y ait peu d’arts, parce que chaque citoyen trouve dans le travail de sa terre précisément de quoi se nourrir, et que tous les citoyens ensemble consomment tous les fruits du pays. Cela était ainsi dans quelques anciennes républiques.\par
Mais dans nos États d’aujourd’hui, les fonds de terre sont inégalement distribués ; ils produisent plus de fruits que ceux qui les cultivent n’en peuvent consommer ; et si l’on y néglige les arts, et qu’on ne s’attache qu’à l’agriculture, le pays ne peut être peuplé. Ceux qui cultivent ou font cultiver, ayant des fruits de reste, rien ne les engage à travailler l’année d’ensuite : les fruits ne seraient point consommés par les gens oisifs, car les gens oisifs n’auraient pas de quoi les acheter. Il faut donc que les arts s’établissent, pour que les fruits soient consommés par les laboureurs et les artisans. En un mot, ces États ont besoin que beaucoup de gens cultivent au-delà de ce qui leur est nécessaire. Pour cela, il faut leur donner envie d’avoir le superflu ; mais il n’y a que les artisans qui le donnent.\par
Ces machines, dont l’objet est d’abréger l’art, ne sont pas toujours utiles. Si un ouvrage est à un prix médiocre, et qui convienne également à celui qui l’achète, et à l’ouvrier qui l’a fait, les machines qui en simplifieraient la manufacture, c’est-à-dire, qui diminueraient le nombre des ouvriers, seraient pernicieuses ; et si les moulins à eau n’étaient pas partout établis, je ne les croirais pas aussi utiles qu’on le dit, parce qu’ils ont fait reposer une infinité de bras, qu’ils ont privé bien des gens de l’usage des eaux, et ont fait perdre la fécondité à beaucoup de terres.
\subsubsection[{Chapitre XVI. Des vues du législateur sur la propagation de l’espèce}]{Chapitre XVI. Des vues du législateur sur la propagation de l’espèce}
\noindent Les règlements sur le nombre des citoyens dépendent beaucoup des circonstances. Il y a des pays où la nature a tout fait ; le législateur n’y a donc rien à faire. À quoi bon engager, par des lois, à la propagation, lorsque la fécondité du climat donne assez de peuple ? Quelquefois le climat est plus favorable que le terrain ; le peuple s’y multiplie, et les famines le détruisent : c’est le cas où se trouve la Chine. Aussi un père y vend-il ses filles, et expose-t-il ses enfants. Les mêmes causes opèrent au Tonkin\footnote{{\itshape Voyages} de Dampierre, t. II, p. 41.} les mêmes effets ; et il ne faut pas, comme les voyageurs arabes, dont Renaudot nous a donné la relation\footnote{P. 167.}, aller chercher l’opinion de la métempsycose pour cela.\par
Les mêmes raisons font que dans l’île Formose\footnote{Voyez le {\itshape Recueil des voyages qui ont servi à l’établissement de la Compagnie des Indes}, t. V, part, I, p. 182 et 188.}, la religion ne permet pas aux femmes de mettre des enfants au monde qu’elles n’aient trente-cinq ans : avant cet âge, la prêtresse leur foule le ventre, et les fait avorter.
\subsubsection[{Chapitre XVII. De la Grèce et du nombre de ses habitants}]{Chapitre XVII. De la Grèce et du nombre de ses habitants}
\noindent Cet effet, qui tient à des causes physiques dans de certains pays d’Orient, la nature du gouvernement le produisit dans la Grèce. Les Grecs étaient une grande nation, composée de villes qui avaient chacune leur gouvernement et leurs lois. Elles n’étaient pas plus conquérantes que celles de Suisse de Hollande et d’Allemagne ne le sont aujourd’hui. Dans chaque république, le législateur avait eu pour objet le bonheur des citoyens au-dedans, et une puissance au-dehors qui ne fût pas inférieure à celle des villes voisines\footnote{Par la valeur, la discipline et les exercices militaires.}. Avec un petit territoire et une grande félicité, il était facile que le nombre des citoyens augmentât et leur devînt à charge : aussi firent-ils sans cesse des colonies\footnote{Les Gaulois, qui étaient dans le même cas, firent de même.} {\itshape ;} ils se vendirent pour la guerre, comme les Suisses font aujourd’hui : rien ne fut négligé de ce qui pouvait empêcher la trop grande multiplication des enfants.\par
Il y avait chez eux des républiques dont la constitution était singulière. Des peuples soumis étaient obligés de fournir la subsistance aux citoyens : les Lacédémoniens étaient nourris par les Ilotes ; les Crétois, par les Périéciens ; les Thessaliens, par les Pénestes. Il ne devait y avoir qu’un certain nombre d’hommes libres, pour que les esclaves fussent en état de leur fournir la subsistance. Nous disons aujourd’hui qu’il faut borner le nombre des troupes réglées : or, Lacédémone était une armée entretenue par des paysans ; il fallait donc borner cette armée ; sans cela, les hommes libres, qui avaient tous les avantages de la société, se seraient multipliés sans nombre, et les laboureurs auraient été accablés.\par
Les politiques grecs s’attachèrent donc particulièrement à régler le nombre des citoyens. Platon\footnote{Dans ses lois, liv. V.} le fixe à cinq mille quarante ; et il veut que l’on arrête, ou que l’on encourage la propagation, selon le besoin, par les honneurs, par la honte et par les avertissements des vieillards ; il veut même que l’on règle le nombre des mariages\footnote{République, liv. V.} de manière que le peuple se répare sans que la république soit surchargée.\par
« Si la loi du pays, dit Aristote\footnote{{\itshape Politique}, liv.{\itshape  VII}, chap. XVI.}, défend d’exposer les enfants, il faudra borner le nombre de ceux que chacun doit engendrer. » Si l’on a des enfants au-delà du nombre défini par la loi, il conseille\footnote{{\itshape Ibid.}} de faire avorter la femme avant que le fœtus ait vie.\par
Le moyen infâme qu’employaient les Crétois pour prévenir le trop grand nombre d’enfants, est rapporté par Aristote ; et j’ai senti la pudeur effrayée quand j’ai voulu le rapporter.\par
Il y a des lieux, dit encore Aristote\footnote{{\itshape Politique}, liv. III, chap. III.}, où la loi fait citoyens les étrangers ou les bâtards, ou ceux qui sont seulement nés d’une mère citoyenne ; mais dès qu’ils ont assez de peuple, ils ne le font plus. Les sauvages du Canada font brûler leurs prisonniers ; mais lorsqu’ils ont des cabanes vides à leur donner, ils les reconnaissent de leur nation.\par
Le chevalier Petty a supposé, dans ses calculs, qu’un homme en Angleterre vaut ce qu’on le vendrait à Alger\footnote{Soixante livres sterling.}. Cela ne peut être bon que pour l’Angleterre : il y a des pays où un homme ne vaut rien ; il y en a où il vaut moins que rien.
\subsubsection[{Chapitre XVIII. De l’état des peuples avant les Romains}]{Chapitre XVIII. De l’état des peuples avant les Romains}
\noindent L’Italie, la Sicile, l’Asie Mineure, l’Espagne, la Gaule, la Germanie étaient à peu près comme la Grèce, pleine de petits peuples, et regorgeaient d’habitants : l’on n’y avait pas besoin de lois pour en augmenter le nombre.
\subsubsection[{Chapitre XIX. Dépopulation de l’univers}]{Chapitre XIX. Dépopulation de l’univers}
\noindent Toutes ces petites républiques furent englouties dans une grande, et l’on vit insensiblement l’univers se dépeupler : il n’y a qu’à voir ce qu’étaient l’Italie et la Grèce avant et après les victoires des Romains.\par
« On me demandera, dit Tite-Live\footnote{Liv. VI.}, où les Volsques ont pu trouver assez de soldats pour faire la guerre, après avoir été si souvent vaincus. il fallait qu’il y eût un peuple infini dans ces contrées, qui ne seraient aujourd’hui qu’un désert, sans quelques soldats et quelques esclaves romains. »\par
« Les oracles ont cessé, dit Plutarque\footnote{{\itshape Œuvres morales} : Des oracles qui ont cessé.}, parce que les lieux où ils parlaient sont détruits ; à peine trouverait-on aujourd’hui dans la Grèce trois mille hommes de guerre. »\par
« Je ne décrirai point, dit Strabon\footnote{Liv. VII, p. 496.}, l’Épire et les lieux circonvoisins, parce que ces pays sont entièrement déserts. Cette dépopulation, qui a commencé depuis longtemps, continue tous les jours ; de sorte que les soldats romains ont leur camp dans les maisons abandonnées. » Il trouve la cause de ceci dans Polybe, qui dit que Paul Émile, après sa victoire, détruisit soixante et dix villes de l’Épire, et en emmena cent cinquante mille esclaves.
\subsubsection[{Chapitre XX. Que les romains furent dans la nécessité de faire des lois pour la propagation de l’espèce}]{Chapitre XX. Que les romains furent dans la nécessité de faire des lois pour la propagation de l’espèce}
\noindent Les Romains, en détruisant tous les peuples, se détruisaient eux-mêmes. Sans cesse dans l’action, l’effort et la violence, ils s’usaient, comme une arme dont on se sert toujours.\par
Je ne parlerai point ici de l’attention qu’ils eurent à se donner des citoyens\footnote{J’ai traité ceci dans les {\itshape Considérations sur les causes de la grandeur des Romains et de leur décadence}.} à mesure qu’ils en perdaient, des associations qu’ils firent, des droits de cité qu’ils donnèrent, et de cette pépinière immense de citoyens qu’ils trouvèrent dans leurs esclaves. Je dirai ce qu’ils firent, non pas pour réparer la perte des citoyens, mais celle des hommes ; et, comme ce fut le peuple du monde qui sut le mieux accorder ses lois avec ses projets, il n’est point indifférent d’examiner ce qu’il fit à cet égard.
\subsubsection[{Chapitre XXI. Des lois des romains sur la propagation de l’espèce}]{Chapitre XXI. Des lois des romains sur la propagation de l’espèce}
\noindent Les anciennes lois de Rome cherchèrent beaucoup à déterminer les citoyens au mariage. Le sénat et le peuple firent souvent des règlements là-dessus, comme le dit Auguste dans sa harangue rapportée par Dion\footnote{Liv. LVI.}.\par
Denys d’Halicarnasse\footnote{Liv. Il.} ne peut croire qu’après la mort des trois cent cinq Fabiens extermines par les Véiens, il ne fût resté de cette race qu’un seul enfant ; parce que la loi ancienne, qui ordonnait à chaque citoyen de se marier et d’élever tous ses enfants, était encore dans sa vigueur\footnote{L’an de Rome 277.}.\par
Indépendamment des lois, les censeurs eurent l’œil sur les mariages ; et, selon les besoins de la république, ils y engagèrent\footnote{Voyez sur ce qu’ils firent à cet égard, Tite-Live, liv. XLV ; l’{\itshape Epitome} de Tite-Live, liv. LIX ; Aulu-Gelle, liv. I, chap. VI ; Valère Maxime, liv. II, chap. {\itshape LX.}} et par la honte et par les peines.\par
Les mœurs, qui commencèrent à se corrompre, contribuèrent beaucoup à dégoûter les citoyens du mariage, qui n’a que des peines pour ceux qui n’ont plus de sens pour les plaisirs de l’innocence. C’est l’esprit de cette harangue\footnote{Elle est dans Aulu-Gelle, liv. I, chap. VI.} que Metellus Numidicus fit au peuple dans sa censure. « S’il était possible de n’avoir point de femme, nous nous délivrerions de ce mal ; mais comme la nature a établi que l’on ne peut guère vivre heureux avec elles, ni subsister sans elles, il faut avoir plus d’égard à notre conservation qu’à des satisfactions passagères. »\par
La corruption des mœurs détruisit la censure, établie elle-même pour détruire la corruption des mœurs ; mais lorsque cette corruption devient générale, la censure n’a plus de force\footnote{Voyez ce que j’ai dit au liv. V, chap. XIX.}.\par
Les discordes civiles, les triumvirats, les proscriptions affaiblirent plus Rome qu’aucune guerre qu’elle eût encore faite : il restait peu de citoyens\footnote{César, après la guerre civile, ayant fait faire le cens, il ne s’y trouva que cent cinquante mille chefs de famille. {\itshape Epitome} de Florus sur Tite-Live, douzième décade.}, et la plupart n’étaient pas mariés. Pour remédier à ce dernier mal, César et Auguste rétablirent la censure, et voulurent même être censeurs\footnote{Voyez Dion, liv. XLIII, et Xiphilin, in {\itshape August.}}. Ils firent divers règlements : César donna des récompenses à ceux qui avaient beaucoup d’enfants\footnote{Dion, liv. XLIII, chap. XXV ; Suétone, {\itshape Vie de César}, chap. XX ; Appien, liv. II, {\itshape De la guerre civile.}} ; il défendit aux femmes qui avaient moins de quarante-cinq ans, et qui n’avaient ni maris ni enfants, de porter des pierreries, et de se servir de litières\footnote{Eusèbe, dans sa {\itshape Chronique.}} : méthode excellente d’attaquer le célibat par la vanité. Les lois d’Auguste furent plus pressantes\footnote{Dion, liv. LIV, chap. XVI.}, il imposa\footnote{L’an 736 de Rome.} des peines nouvelles à ceux qui n’étaient point mariés, et augmenta les récompenses de ceux qui l’étaient, et de ceux qui avaient des enfants. Tacite appelle ces lois Juliennes\footnote{{\itshape Julias rogationes, Annales}, liv. III.} ; il y a apparence qu’on y avait fondu les anciens règlements faits par le sénat, le peuple et les censeurs.\par
La loi d’Auguste trouva mille obstacles ; et trente-quatre ans\footnote{L’an 762 de Rome. Dion, liv. LVI, chap. I.} après qu’elle eut été faite, les chevaliers romains lui en demandèrent la révocation. Il fit mettre d’un côté ceux qui étaient mariés, et de l’autre ceux qui ne l’étaient Pas : ces derniers parurent en plus grand nombre, ce qui étonna les citoyens et les confondit. Auguste, avec la gravité des anciens censeurs, leur parla ainsi\footnote{J’ai abrégé cette harangue, qui est d’une longueur accablante : elle est rapportée dans Dion, liv. LVI.}.\par
« Pendant que les maladies et les guerres nous enlèvent tant de citoyens, que deviendra la ville, si on ne contracte plus de mariages ? La cité ne consiste point dans les maisons, les portiques, les places publiques : ce sont les hommes qui font la cité. Vous ne verrez point, comme dans les fables, sortir des hommes de dessous la terre pour prendre soin de vos affaires. Ce n’est point pour vivre seuls que vous restez dans le célibat : chacun de vous a des compagnes de sa table et de son lit, et vous ne cherchez que la paix dans vos dérèglements. Citerez-vous ici l’exemple des vierges Vestales ? Donc si vous ne gardiez pas les lois de la pudicité, il faudrait vous punir comme elles. Vous êtes également mauvais citoyens, soit que tout le monde imite votre exemple, soit que personne ne le suive. Mon unique objet est la perpétuité de la république. J’ai augmenté les peines de ceux qui n’ont point obéi ; et, à l’égard des récompenses, elles sont telles que je ne sache pas que la vertu en ait encore eu de plus grandes : il y en a de moindres qui portent mille gens à exposer leur vie ; et celles-ci ne vous engageraient pas à prendre une femme et à nourrir des enfants ? »\par
Il donna la loi qu’on nomma de son nom Julia, et Papia Poppaea du nom des Consuls\footnote{Marcus Papius Mutilus et Q. Poppaenus Sabinus. Dion, liv. LVI.} d’une partie de cette année-là. La grandeur du mal paraissait dans leur élection même : Dion\footnote{Dion, liv. LVI.} nous dit qu’ils n’étaient point mariés, et qu’ils n’avaient point d’enfants.\par
Cette loi d’Auguste fut proprement un code de lois, et un corps systématique de tous les règlements qu’on pouvait faire sur ce sujet. On y refondit les lois Juliennes\footnote{Le titre XIV des {\itshape Fragments} d’Ulpien distingue fort bien la loi Julienne de la Papienne.}, et on leur donna plus de force ; elles ont tant de vues, elles influent sur tant de choses, qu’elles forment la plus belle partie des lois civiles des Romains.\par
On en trouve les morceaux dispersés dans les précieux fragments d’Ulpien\footnote{Jacques Godefroi en a fait une compilation.}, dans les lois du Digeste tirées des auteurs qui ont écrit sur les lois Papiennes ; dans les historiens et les autres auteurs qui les ont citées ; dans le code Théodosien qui les a abrogées ; dans les Pères qui les ont censurées, sans doute avec un zèle louable pour les choses de l’autre vie, mais avec très peu de connaissance des affaires de celle-ci.\par
Ces lois avaient plusieurs chefs, et l’on en connaît trente-cinq\footnote{Le trente-cinquième est cité dans la loi 19, ff.{\itshape  De ritu nuptiarum.}}. Mais, allant à mon sujet le plus directement qu’il me sera possible, je commencerai par le chef qu’Aulu-Gelle\footnote{Liv. III, chap. XV.} nous dit être le septième, et qui regarde les honneurs et les récompenses accordés par cette loi.\par
Les Romains, sortis pour la plupart des villes latines, qui étaient des colonies lacédémoniennes\footnote{Denys d’Halicarnasse.}, et qui avaient même tiré de ces villes une partie de leurs lois\footnote{Les députés de Rome, qui furent envoyés pour chercher des lois grecques, allèrent à Athènes et dans les villes d’Italie.}, eurent, comme les Lacédémoniens, pour la vieillesse, ce respect qui donne tous les honneurs et toutes les préséances. Lorsque la république manqua de citoyens, on accorda au mariage et au nombre des enfants les prérogatives que l’on avait données à l’âge\footnote{Aulu-Gelle, liv. II, chap. XV.} ; on en attacha quelques-unes au mariage seul, indépendamment des enfants qui en pourraient naître : cela s’appelait le droit des maris. On en donna d’autres à ceux qui avaient des enfants ; de plus grandes à ceux qui avaient trois enfants. Il ne faut pas confondre ces trois choses : il y avait de ces privilèges dont les gens mariés jouissaient toujours : comme, par exemple, une place particulière au théâtre\footnote{Suétone, {\itshape in Augusto}, chap. XLIV.} ; il y en avait dont ils ne jouissaient que lorsque des gens qui avaient des enfants, ou qui en avaient plus qu’eux, ne les leur ôtaient pas.\par
Ces privilèges étaient très étendus. Les gens mariés qui avaient le plus grand nombre d’enfants étaient toujours préférés, soit dans la poursuite des honneurs, soit dans l’exercice de ces honneurs mêmes\footnote{Tacite, liv. II. {\itshape Ut numerus liberorum in candidatis praepolleret, quod lex jubebat.}}. Le consul qui avait le plus d’enfants, prenait le premier les faisceaux\footnote{Aulu-Gelle, liv. II, chap. XV.}, il avait le choix des provinces\footnote{Tacite, {\itshape Annales}, liv. XV.} {\itshape ;} le sénateur qui avait le plus d’enfants était écrit le premier dans le catalogue des sénateurs ; il disait au sénat son avis le premier\footnote{Voyez la loi 6, § 5, ff.{\itshape  De decurionibus.}} L’on pouvait parvenir avant l’âge aux magistratures, parce que chaque enfant donnait dispense d’un an\footnote{Voyez la loi 2, ff. {\itshape De minoribus.}}. Si l’on avait trois enfants, à Rome, on était exempt de toutes charges personnelles\footnote{Loi 1, § 3 ; et 2, § 1, ff. {\itshape De vacatione et excusat. muner.}}. Les femmes ingénues qui avaient trois enfants, et les affranchies qui en avaient quatre sortaient\footnote{{\itshape Fragments} d’Ulpien, tit. XXIX, § 3.} de cette perpétuelle tutelle où les retenaient\footnote{Plutarque, {\itshape Vie de Numa}.} les anciennes lois de Rome.\par
Que s’il y avait des récompenses, il y avait aussi des peines\footnote{Voyez les {\itshape Fragments} d’Ulpien, aux titres XIV, XV, XVI, XVII et XVIII, qui sont un des beaux morceaux de l’ancienne jurisprudence romaine.}. Ceux qui n’étaient point mariés ne pouvaient rien recevoir par le testament des étrangers\footnote{Sozomène, liv. I, chap. IX. On recevait de ses parents ; {\itshape Fragments} d’Ulpien, tit. XVI, § 1.} ; et ceux qui, étant mariés, n’avaient pas d’enfants, n’en recevaient que la moitié\footnote{Sozomène, liv. I, chap. IX, et {\itshape leg. unic. cod. Theodos. De infirmis poenis coelibis et orbitatis}.}. Les Romains, dit Plutarque\footnote{{\itshape Œuvres morales} : De l’amour des pères envers leurs enfants.}, se mariaient pour être héritiers, et non pour avoir des héritiers.\par
Les avantages qu’un mari et une femme pouvaient se faire par testament étaient limités par la loi. Ils pouvaient se donner le tout\footnote{Voyez un plus long détail de ceci dans les {\itshape Fragments} d’Ulpien, tit. XV et XVI.}, s’ils avaient des enfants l’un de l’autre ; s’ils n’en avaient point, ils pouvaient recevoir la dixième partie de la succession, à cause du mariage ; et s’ils avaient des enfants d’un autre mariage, ils pouvaient se donner autant de dixièmes qu’ils avaient d’enfants.\par
Si un mari s’absentait\footnote{{\itshape Fragments} d’Ulpien, tit. XVI, § 1.} d’auprès de sa femme pour autre cause que pour les affaires de la république, il ne pouvait en être l’héritier.\par
La loi donnait à un mari ou à une femme qui survivait, deux ans pour se remarier\footnote{{\itshape Fragments} d’Ulpien, tit. XIV. Il paraît que les premières lois Juliennes donnèrent trois ans. Harangue d’Auguste dans Dion, liv. LVI ; Suétone, {\itshape Vie d’Auguste}, chap. XXXIV. D’autres lois Juliennes n’accordèrent qu’un an ; enfin la loi Papienne en donna deux : {\itshape Fragments} d’Ulpien, tit. XIV. Ces lois n’étaient point agréables au peuple, et Auguste les tempérait ou les raidissait selon qu’on était plus ou moins disposé à les souffrir.}, et un an et demi dans le cas du divorce. Les pères qui ne voulaient pas marier leurs enfants, ou donner de dot à leurs filles, y étaient contraints par les magistrats\footnote{C’était le trente-cinquième chef de la loi Papienne, leg. 19, ff. {\itshape De ritu nuptiarum.}}.\par
On ne pouvait faire des fiançailles lorsque le mariage devait être différé de plus de deux ans\footnote{Voyez Dion, liv. LIV, anno 736 ; Suétone in {\itshape Octavio}, chap. XXXIV.}, et comme on ne pouvait épouser une fille qu’à douze ans, on ne pouvait la fiancer qu’à dix. La loi ne voulait pas que l’on pût jouir inutilement\footnote{Voyez Dion, liv. LIV ; et dans le même Dion, la harangue d’Auguste, liv. LVI.}, et sous prétexte de fiançailles, des privilèges des gens mariés.\par
Il était défendu à un homme qui avait soixante ans d’épouser une femme qui en avait cinquante\footnote{{\itshape Fragments} d’Ulpien, tit. XVI ; et la loi 27, cod. {\itshape De nuptiis.}}. Comme on avait donné de grands privilèges aux gens mariés, la loi ne voulait point qu’il y eût des mariages inutiles. Par la même raison, le sénatus-consulte Calvisien\footnote{{\itshape Fragments} d’Ulpien, tit. XVI, § 3.} déclarait inégal le mariage d’une femme qui avait plus de cinquante ans, avec un homme qui en avait moins de soixante ; de sorte qu’une femme qui avait cinquante ans ne pouvait se marier sans encourir les peines de ces lois. Tibère ajouta à la rigueur de la loi Papienne\footnote{Voyez Suétone, {\itshape in Claudio}, chap. XXIII.}, et défendit à un homme de soixante ans d’épouser une femme qui en avait moins de cinquante ; de sorte qu’un homme de soixante ans ne pouvait se marier, dans aucun cas, sans encourir la peine ; mais Claude\footnote{Voyez Suétone, {\itshape Vie de Claude}, chap. XXIII ; et les {\itshape Fragments} d’Ulpien, tit. XVI, § 3.} abrogea ce qui avait été fait sous Tibère à cet égard.\par
Toutes ces dispositions étaient plus conformes au climat d’Italie qu’à celui du Nord, où un homme de soixante ans a encore de la force, et où les femmes de cinquante ans ne sont pas généralement stériles.\par
Pour que l’on ne fût pas inutilement borné dans le choix que l’on pouvait faire, Auguste permit à tous les ingénus qui n’étaient pas sénateurs\footnote{Dion, liv. LIV ; {\itshape Fragments} d’Ulpien, tit. XIII.} d’épouser des affranchies\footnote{Harangue d’Auguste, dans Dion, liv. LVI.}. La loi Papienne\footnote{{\itshape Fragments} d’Ulpien, chap. XIII ; et la loi 44, ff. {\itshape De ritu nuptiarum}, à la fin.} interdisait aux sénateurs le mariage avec les femmes qui avaient été affranchies, ou qui s’était produites sur le théâtre ; et du temps d’Ulpien\footnote{Voyez les {\itshape Fragments} d’Ulpien, tit. XIII et XVI.}, il était défendu aux ingénus d’épouser des femmes qui avaient mené une mauvaise vie, qui étaient montées sur le théâtre, ou qui avaient été condamnées par un jugement public. Il fallait que ce fût quelque sénatus-consulte qui eût établi cela. Du temps de la république, on n’avait guère fait de ces sortes de lois, parce que les censeurs corrigeaient, à cet égard, les désordres qui naissaient, ou les empêchaient de naître.\par
Constantin\footnote{Voyez la loi 1, au Cod. {\itshape De nat. lib.}} ayant fait une loi par laquelle il comprenait dans la défense de la loi Papienne, non seulement les sénateurs, mais encore ceux qui avaient un rang considérable dans l’État, sans parler de ceux qui étaient d’une condition inférieure, cela forma le droit de ce temps-là : il n’y eut plus que les ingénus, compris dans la loi de Constantin, à qui de tels mariages fussent défendus. Justinien\footnote{{\itshape Novelle} 117.} abrogea encore la loi de Constantin, et permit à toutes sortes de personnes de contracter ces mariages : c’est par là que nous avons acquis une liberté si triste.\par
Il est clair que les peines portées contre ceux qui se mariaient contre la défense de la loi, étaient les mêmes que celles portées contre ceux qui ne se mariaient point du tout. Ces mariages ne leur donnaient aucun avantage civil\footnote{Loi 37, § 7, ff. {\itshape De oper. libert.} ; {\itshape Fragments} d’Ulpien, tit. XVI, § 2} : la dot\footnote{Fragments, {\itshape ibid.}} était caduque après la mort de la femme\footnote{Voyez ci-dessous le chap. XIII du liv. XXVI.}.\par
Auguste ayant adjugé au trésor public les successions et les legs de ceux que ces lois en déclaraient incapables\footnote{Excepté dans de certains cas. Voyez les {\itshape Fragments} d’Ulpien, tit. XVIII ; et la loi unique, au Code {\itshape De caduc. tollend.}}, ces lois parurent plutôt fiscales que politiques et civiles. Le dégoût que l’on avait déjà pour une charge qui paraissait accablante, fut augmenté par celui de se voir continuellement en proie à l’avidité du fisc. Cela fit que, sous Tibère, on fut obligé de modifier ces lois\footnote{{\itshape Relatum de moderanda Papia Poppae}. Tacite, {\itshape Annales}, liv. III, p. 117.} {\itshape ;} que Néron diminua les récompenses des délateurs au fisc\footnote{Il les réduisit à la quatrième partie. Suétone, {\itshape in Nerone}, chap. X.} ; que Trajan arrêta leur brigandage\footnote{Voyez le {\itshape Panégyrique} de Pline.} ; que Sévère modifia ces lois\footnote{Sévère recula jusqu’à vingt-cinq ans pour les mâles, et vingt pour les filles, le temps des dispositions de la loi Papienne, comme on le voit en conférant le {\itshape Fragment} d’Ulpien, tit. XVI, avec ce que dit Tertullien, {\itshape Apologet.}, chap. IV.} ; et que les jurisconsultes les regardèrent comme odieuses, et, dans leurs décisions, en abandonnèrent la rigueur.\par
D’ailleurs les empereurs énervèrent ces lois\footnote{P. Scipion, censeur, dans sa harangue au peuple sur les mœurs, se plaint de l’abus qui déjà s’était introduit, que le fils adoptif donnait le même privilège que le fils naturel. Aulu-Gelle, liv. V, chap. XIX.}, par les privilèges qu’ils donnèrent des droits de maris, d’enfants, et de trois enfants. Ils firent plus : ils dispensèrent les particuliers des peines de ces lois\footnote{Voyez la loi 31, ff. {\itshape De ritu nuptiarum.}}. Mais des règles établies pour l’utilité publique semblaient ne devoir point admettre de dispense.\par
Il avait été raisonnable d’accorder le droit d’enfants aux Vestales\footnote{Auguste, par la loi Papienne, leur donna le même privilège qu’aux mères. Voyez Dion, liv. LVI. Numa leur avait donné l’ancien privilège des femmes qui avaient trois enfants, qui est de n’avoir point de curateur. Plutarque, dans la {\itshape Vie de Numa}.}, que la religion retenait dans une virginité nécessaire : on donna de même le privilège des maris aux soldats\footnote{Claude le leur accorda. Dion, liv. LX.}, parce qu’ils ne pouvaient se marier. C’était la coutume d’exempter les empereurs de la gêne de certaines lois civiles. Ainsi Auguste fut exempté de la gêne de la loi qui limitait la faculté d’affranchir\footnote{Leg. {\itshape Apud eum}, ff. {\itshape De manumissionibus}, § 1.}, et de celle qui bornait la faculté de léguer\footnote{Dion, Liv. LV.}. Tout cela n’était que des cas particuliers ; mais dans la suite les dispenses furent données sans ménagement, et la règle ne fut plus qu’une exception.\par
Des sectes de philosophie avaient déjà introduit dans l’empire un esprit d’éloignement pour les affaires, qui n’aurait pu gagner à ce point dans le temps de la république\footnote{Voyez dans les {\itshape Offices} de Cicéron, liv. I, ses idées sur cet esprit de spéculation.}, où tout le monde était occupé des arts de la guerre et de la paix. De là une idée de perfection attachée à tout ce qui mène à une vie spéculative ; de là l’éloignement pour les soins et les embarras d’une famille. La religion chrétienne, venant après la philosophie, fixa, pour ainsi dire, des idées que celle-ci n’avait fait que préparer.\par
Le christianisme donna son caractère à la jurisprudence ; car l’empire a toujours du rapport avec le sacerdoce. On peut voir le code Théodosien, qui n’est qu’une compilation des ordonnances des empereurs chrétiens.\par
Un panégyriste\footnote{Nazaire, {\itshape in Panegyrico Constantini}, anno 321.} de Constantin dit à cet empereur : « Vos lois n’ont été faites que pour corriger les vices, et régler les mœurs : vous avez ôté l’artifice des anciennes lois, qui semblaient n’avoir d’autres vues que de tendre des pièges à la simplicité. »\par
Il est certain que les changements de Constantin furent faits, ou sur des idées qui se rapportaient à l’établissement du christianisme, ou sur des idées prises de sa perfection. De ce premier objet vinrent ces lois qui donnèrent une telle autorité aux évêques, qu’elles ont été le fondement de la juridiction ecclésiastique : de là ces lois qui affaiblirent l’autorité paternelle\footnote{Voyez la loi 1, 2 et 3, au Code Théodosien, {\itshape De bonis maternis, maternique generis}, etc., et la loi unique, au même Code, {\itshape De bonis quae filiis famil. acquiruntur.}}, en ôtant au père la propriété des biens de ses enfants. Pour étendre une religion nouvelle, il faut ôter l’extrême dépendance des enfants, qui tiennent toujours moins à ce qui est établi.\par
Les lois faites dans l’objet de la perfection chrétienne furent surtout celles par lesquelles il ôta les peines des lois Papiennes\footnote{Leg. {\itshape unica}, Cod., {\itshape De infirm. poen. caelib. et orbit}.} et en exempta tant ceux qui n’étaient point mariés, que ceux qui, étant mariés, n’avaient pas d’enfants.\par
« Ces lois avaient été établies, dit un historien ecclésiastique\footnote{Sozomène, p. 27.}, comme si la multiplication de l’espèce humaine pouvait être un effet de nos soins ; au lieu de voir que ce nombre croît et décroît selon l’ordre de la Providence. »\par
Les principes de la religion ont extrêmement influé sur la propagation de l’espèce humaine : tantôt ils l’ont encouragée, comme chez les Juifs, les Mahométans, les Guèbres, les Chinois : tantôt ils l’ont choquée, comme ils firent chez les Romains devenus chrétiens.\par
On ne cessa de prêcher partout la continence, c’est-à-dire cette vertu qui est plus parfaite, parce que, par sa nature, elle doit être pratiquée par très peu de gens.\par
Constantin n’avait point ôté les lois décimaires, qui donnaient une plus grande extension aux dons que le mari et la femme pouvaient se faire à proportion du nombre de leurs enfants : Théodose le jeune abrogea encore ces lois\footnote{Leg. 2 et 3, Cod. Théod., {\itshape De jure lib.}}.\par
Justinien déclara valables tous les mariages que les lois Papiennes avaient défendus\footnote{Leg. {\itshape Sancimus}, Cod. {\itshape De nuptiis}.}. Ces lois voulaient qu’on se remariât ; Justinien accorda des avantages à ceux qui ne se remarieraient pas\footnote{{\itshape Novelle} 127, chap. III ; {\itshape Novelle} 118, chap. V.}.\par
Par les lois anciennes, la faculté naturelle que chacun a de se marier, et d’avoir des enfants, ne pouvait être ôtée. Ainsi, quand on recevait un legs à condition de ne point se marier\footnote{Leg. 54, ff. {\itshape De condit. et demonst.}}, lorsqu’un patron faisait jurer son affranchi qu’il ne se marierait point, et qu’il n’aurait point d’enfants\footnote{Leg. 5, § 4, {\itshape De jure patronatus}.}, la loi Papienne annulait et cette condition et ce serment\footnote{Paul, dans ses {\itshape Sentences}, liv. III, tit. IV, § 15.}. Les clauses {\itshape en gardant viduité}, établies parmi nous, contredisent donc le droit ancien, et descendent des constitutions des empereurs, faites sur les idées de la perfection.\par
Il n’y a point de loi qui contienne une abrogation expresse des privilèges et des honneurs que les Romains païens avaient accordés aux mariages et au nombre des enfants ; mais là où le célibat avait la prééminence, il ne pouvait plus y avoir d’honneur pour le mariage ; et, puisque l’on put obliger les traitants à renoncer à tant de profits par l’abolition des peines, on sent qu’il fut encore plus aisé d’ôter les récompenses.\par
La même raison de spiritualité, qui avait fait permettre le célibat, imposa bientôt la nécessité du célibat même. À Dieu ne plaise que je parle ici contre le célibat qu’a adopté la religion ; mais qui pourrait se taire contre celui qu’a formé le libertinage ; celui où les deux sexes, se corrompant par les sentiments naturels mêmes, fuient une union qui doit les rendre meilleurs, pour vivre dans celle qui les rend toujours pires ?\par
C’est une règle tirée de la nature que, plus on diminue le nombre des mariages qui pourraient se faire, plus on corrompt ceux qui sont faits ; moins il y a de gens mariés, moins il y a de fidélité dans les mariages ; comme lorsqu’il y a plus de voleurs, il y a plus de vols.
\subsubsection[{Chapitre XXII. De l’exposition des enfants}]{Chapitre XXII. De l’exposition des enfants}
\noindent Les premiers Romains eurent une assez bonne police sur l’exposition des enfants. Romulus, dit Denys d’Halicarnasse\footnote{{\itshape Antiquités romaines}, liv. II.}, imposa à tous les citoyens la nécessité d’élever tous les enfants mâles et les aînées des filles. Si les enfants étaient difformes et monstrueux, il permettait de les exposer, après les avoir montrés à cinq des plus proches voisins.\par
Romulus ne permit de tuer aucun enfant qui eût moins de trois ans\footnote{{\itshape Ibid.}} {\itshape : par} là il conciliait la loi qui donnait aux pères le droit de vie et de mort sur leurs enfants, et celle qui défendait de les exposer.\par
On trouve encore dans Denys d’Halicarnasse\footnote{Liv. IX.} que la loi qui ordonnait aux citoyens de se marier et d’élever tous leurs enfants, était en vigueur l’an 277 de Rome : on voit que l’usage avait restreint la loi de Romulus, qui permettait d’exposer les filles cadettes.\par
Nous n’avons de connaissance de ce que la loi des Douze Tables, donnée l’an de Rome 301, statua sur l’exposition des enfants, que par un passage de Cicéron\footnote{Liv. III {\itshape De legib}.} qui, parlant du tribunat du peuple, dit que d’abord après sa naissance, tel que l’enfant monstrueux de la loi des Douze Tables, il fut étouffé : les enfants qui n’étaient pas monstrueux étaient donc conservés, et la loi des Douze Tables ne changea rien aux institutions précédentes.\par
« Les Germains, dit Tacite\footnote{{\itshape De moribus Germanorum}.}, {\itshape n’exposent} point leurs enfants ; et, chez eux, les bonnes mœurs ont plus de force que n’ont ailleurs les bonnes lois. » Il y avait donc, chez les Romains, des lois contre cet usage, et on ne les suivait plus. On ne trouve aucune loi romaine qui permette d’exposer les enfants\footnote{Il n’y a point de titre là-dessus dans le Digeste : le titre du Code n’en dit rien, non plus que les {\itshape Novelles.}} : ce fut sans doute un abus introduit dans les derniers temps, lorsque le luxe ôta l’aisance, lorsque les richesses partagées furent appelées pauvreté, lorsque le père crut avoir perdu ce qu’il donna à sa famille, et qu’il distingua cette famille de sa propriété.
\subsubsection[{Chapitre XXIIII. De l’état de l’univers après la destruction des Romains}]{Chapitre XXIIII. De l’état de l’univers après la destruction des Romains}
\noindent Les règlements que firent les Romains pour augmenter le nombre de leurs citoyens eurent leur effet pendant que leur république, dans la force de son institution, n’eut à réparer que les pertes qu’elle faisait par son courage, par son audace, par sa fermeté, par son amour pour la gloire, et par sa vertu même. Mais bientôt les lois les plus sages ne purent rétablir ce qu’une république mourante, ce qu’une anarchie générale, ce qu’un gouvernement militaire, ce qu’un empire dur, ce qu’un despotisme superbe, ce qu’une monarchie faible, ce qu’une cour stupide, idiote et superstitieuse, avaient successivement abattu : on eût dit qu’ils n’avaient conquis le monde que pour l’affaiblir, et le livrer sans défense aux barbares. Les nations gothes, gétiques, sarrasines et tartares les accablèrent tour à tour ; bientôt les peuples barbares n’eurent à détruire que des peuples barbares. Ainsi, dans le temps des fables, après les inondations et les déluges, il sortit de la terre des hommes armés qui s’exterminèrent.
\subsubsection[{Chapitre XXIV. Changements arrivés en Europe par rapport au nombre des habitants}]{Chapitre XXIV. Changements arrivés en Europe par rapport au nombre des habitants}
\noindent Dans l’état où était l’Europe, on n’aurait pas cru qu’elle pût se rétablir ; surtout lorsque, sous Charlemagne, elle ne forma plus qu’un vaste empire. Mais, par la nature du gouvernement d’alors, elle se partagea en une infinité de petites souverainetés. Et, comme un seigneur résidait dans son village ou dans sa ville ; qu’il n’était grand, riche, puissant, que dis-je, qu’il n’était en sûreté que par le nombre de ses habitants, chacun s’attacha avec une attention singulière à faire fleurir son petit pays : ce qui réussit tellement que, malgré les irrégularités du gouvernement, le défaut des connaissances qu’on a acquises depuis sur le commerce, le grand nombre de guerres et de querelles qui s’élevèrent sans cesse, il y eut dans la plupart des contrées d’Europe plus de peuple qu’il n’y en a aujourd’hui.\par
Je n’ai pas le temps de traiter à fond cette matière ; mais je citerai les prodigieuses armées des croises, composées de gens de toute espèce. M. Pufendorff dit\footnote{{\itshape Histoire de l’univers}, chap. V, de la France.} que sous Charles IX il y avait vingt millions d’hommes en France.\par
Ce sont les perpétuelles réunions de plusieurs petits États, qui ont produit cette diminution.\par
Autrefois chaque village de France était une capitale ; il n’y en a aujourd’hui qu’une grande : chaque partie de l’État était un centre de puissance ; aujourd’hui tout se rapporte à un centre ; et ce centre est, pour ainsi dire, l’État même.
\subsubsection[{Chapitre XXV. Continuation du même sujet}]{Chapitre XXV. Continuation du même sujet}
\noindent Il est vrai que l’Europe a, depuis deux siècles, beaucoup augmenté sa navigation : cela lui a procuré des habitants, et lui en a fait perdre. La Hollande envoie tous les ans aux Indes un grand nombre de matelots, dont il ne revient que les deux tiers ; le reste périt ou s’établit aux Indes : même chose doit à peu près arriver à toutes les autres nations qui font ce commerce.\par
Il ne faut point juger de l’Europe comme d’un État particulier qui y ferait seul une grande navigation. Cet État augmenterait de peuple, parce que toutes les nations voisines viendraient prendre part à cette navigation ; il y arriverait des matelots de tous côtés. L’Europe, séparée du reste du monde par la religion\footnote{Les pays mahométans l’entourent presque partout.}, par de vastes mers et par des déserts, ne se répare pas ainsi.
\subsubsection[{Chapitre XXVI. Conséquences}]{Chapitre XXVI. {\itshape Conséquences}}
\noindent De tout ceci il faut conclure que l’Europe est encore aujourd’hui dans le cas d’avoir besoin des lois qui favorisent la propagation de l’espèce humaine : aussi, comme les politiques grecs nous parlent toujours de ce grand nombre de citoyens qui travaillent la république, les politiques d’aujourd’hui ne nous parlent que des moyens propres à l’augmenter.
\subsubsection[{Chapitre XXVII. De la loi faite en France pour encourager la propagation de l’espèce}]{Chapitre XXVII. De la loi faite en France pour encourager la propagation de l’espèce}
\noindent XIV ordonna\footnote{Édit de 1666, en faveur des mariages.} de certaines pensions pour ceux qui auraient dix enfants, et de plus fortes pour ceux qui en auraient douze. Mais il n’était pas question de récompenser des prodiges. Pour donner un certain esprit général qui portât à la propagation de l’espèce, il fallait établir, comme les Romains, des récompenses générales ou des peines générales.
\subsubsection[{Chapitre XXVIII. Comment on peut remédier à la dépopulation}]{Chapitre XXVIII. Comment on peut remédier à la dépopulation}
\noindent Lorsqu’un État se trouve dépeuplé par des accidents particuliers, des guerres, des pestes, des famines, il y a des ressources. Les hommes qui restent peuvent conserver l’esprit de travail et d’industrie ; ils peuvent chercher à réparer leurs malheurs, et devenir plus industrieux par leur calamité même. Le mal presque incurable est lorsque la dépopulation vient de longue main, par un vice intérieur et un mauvais gouvernement. Les hommes y ont péri par une maladie insensible et habituelle : nés dans la langueur et dans la misère, dans la violence ou les préjugés du gouvernement, ils se sont vu détruire, souvent sans sentir les causes de leur destruction. Les pays désolés par le despotisme, ou par les avantages excessifs du clergé sur les laïques, en sont deux grands exemples.\par
Pour rétablir un État ainsi dépeuplé, on attendrait en vain des secours des enfants qui pourraient naître. Il n’est plus temps ; les hommes, dans leur désert, sont sans courage et sans industrie. Avec des terres pour nourrir un peuple, on a à peine de quoi nourrir une famille. Le bas peuple, dans ces pays, n’a pas même de part à leur misère, c’est-à-dire aux friches dont ils sont remplis. Le clergé, le prince, les villes, les grands, quelques citoyens principaux sont devenus insensiblement propriétaires de toute la contrée : elle est inculte ; mais les familles détruites leur en ont laissé les pâtures, et l’homme de travail n’a rien.\par
Dans cette situation, il faudrait faire, dans toute J’étendue de l’empire, ce que les Romains faisaient dans une partie du leur : pratiquer dans la disette des habitants ce qu’ils observaient dans l’abondance ; distribuer des terres à toutes les familles qui n’ont rien ; leur procurer les moyens de les défricher et de les cultiver. Cette distribution devrait se faire à mesure qu’il y aurait un homme pour la recevoir ; de sorte qu’il n’y eût point de moment perdu pour le travail.
\subsubsection[{Chapitre XXIX. Des hôpitaux}]{Chapitre XXIX. {\itshape Des hôpitaux}}
\noindent Un homme n’est pas pauvre parce qu’il n’a rien, mais parce qu’il ne travaille pas. Celui qui n’a aucun bien et qui travaille est aussi à son aise que celui qui a cent écus de revenus sans travailler. Celui qui n’a rien, et qui a un métier, n’est pas plus pauvre que celui qui a dix arpents de terre en propre, et qui doit les travailler pour subsister. L’ouvrier qui a donné à ses enfants son art pour héritage, leur a laissé un bien qui s’est multiplié à proportion de leur nombre. Il n’en est pas de même de celui qui a dix arpents de fonds pour vivre, et qui les partage à ses enfants.\par
Dans les pays de commerce, où beaucoup de gens n’ont que leur art, l’État est souvent obligé de pourvoir aux besoins des vieillards, des malades et des orphelins. Un État bien policé tire cette subsistance du fonds des arts même ; il donne aux uns les travaux dont ils sont capables ; il enseigne les autres à travailler, ce qui fait déjà un travail.\par
Quelques aumônes que l’on fait à un homme nu dans les rues, ne remplissent point les obligations de l’État, qui doit à tous les citoyens une subsistance assurée, la nourriture, un vêtement convenable, et un genre de vie qui ne soit point contraire à la santé.\par
Aureng-Zeb\footnote{Voyez Chardin, {\itshape Voyage de Perse}, t. VIII.}, à qui on demandait pourquoi il ne bâtissait point d’hôpitaux, dit : « Je rendrai mon empire si riche qu’il n’aura pas besoin d’hôpitaux. » Il aurait fallu dire : Je commencerai par rendre mon empire riche, et je bâtirai des hôpitaux.\par
Les richesses d’un État supposent beaucoup d’industrie. Il n’est pas possible que dans un si grand nombre de branches de commerce, il n’y en ait toujours quelqu’une qui souffre, et dont par conséquent les ouvriers ne soient dans une nécessité momentanée.\par
C’est pour lors que l’État a besoin d’apporter un prompt secours, soit pour empêcher le peuple de souffrir, soit pour éviter qu’il ne se révolte : c’est dans ce cas qu’il faut des hôpitaux, ou quelque règlement équivalent, qui puisse prévenir cette misère.\par
Mais quand la nation est pauvre, la pauvreté particulière dérive de la misère générale ; et elle est, pour ainsi dire, la misère générale. Tous les hôpitaux du monde ne sauraient guérir cette pauvreté particulière ; au contraire, l’esprit de paresse qu’ils inspirent augmente la pauvreté générale, et par conséquent la particulière.\par
Henri VIII, voulant réformer l’Église d’Angleterre, détruisit les moines\footnote{Voyez l’{\itshape Histoire de la réforme d’Angleterre}, par M. Burnet.}, nation paresseuse elle-même, et qui entretenait la paresse des autres, parce que, pratiquant l’hospitalité, une infinité de gens oisifs, gentilshommes et bourgeois, passaient leur vie à courir de couvent en couvent. Il ôta encore les hôpitaux où le bas peuple trouvait sa subsistance, comme les gentilshommes trouvaient la leur dans les monastères. Depuis ces changements, l’esprit de commerce et d’industrie s’établit en Angleterre.\par
À Rome, les hôpitaux font que tout le monde est à son aise, excepté ceux qui travaillent, excepté ceux qui ont de l’industrie, excepté ceux qui cultivent les arts, excepté ceux qui ont des terres, excepté ceux qui font le commerce.\par
J’ai dit que les nations riches avaient besoin d’hôpitaux, parce que la fortune y était sujette à mille accidents : mais on sent que des secours passagers vaudraient bien mieux que des établissements perpétuels. Le mal est momentané : il faut donc des secours de même nature, et qui soient applicables à l’accident particulier.
\section[{Cinquième partie}]{Cinquième partie}\renewcommand{\leftmark}{Cinquième partie}

\subsection[{Livre vingt-quatrième. Des lois dans le rapport qu’elles ont avec la religion établie dans chaque pays, considérée dans ses pratiques. et en elle-même}]{Livre vingt-quatrième. Des lois dans le rapport qu’elles ont avec la religion établie dans chaque pays, considérée dans ses pratiques \\
et en elle-même}
\subsubsection[{Chapitre I. Des religions en général}]{Chapitre I. Des religions en général}
\noindent Comme on peut juger parmi les ténèbres celles qui sont les moins épaisses, et parmi les abîmes ceux qui sont les moins profonds, ainsi l’on peut chercher entre les religions fausses celles qui sont les plus conformes au bien de la société ; celles qui, quoiqu’elles n’aient pas l’effet de mener les hommes aux félicités de l’autre vie, peuvent le plus contribuer à leur bonheur dans celle-ci.\par
Je n’examinerai donc les diverses religions du monde, que par rapport au bien que l’on en tire dans l’état civil ; soit que je parle de celle qui a sa racine dans le ciel, ou bien de celles qui ont la leur sur la terre.\par
Comme dans cet ouvrage je ne suis point théologien, mais écrivain politique, il pourrait y avoir des choses qui ne seraient entièrement vraies que dans une façon de penser humaine, n’ayant point été considérées dans le rapport avec des vérités plus sublimes.\par
À l’égard de la vraie religion, il ne faudra que très peu d’équité pour voir que je n’ai jamais prétendu faire céder ses intérêts aux intérêts politiques, mais les unir : or, pour les unir, il faut les connaître.\par
La religion chrétienne, qui ordonne aux hommes de s’aimer, veut sans doute que chaque peuple ait les meilleures lois politiques et les meilleures lois civiles, parce qu’elles sont, après elle, le plus grand bien que les hommes puissent donner et recevoir.
\subsubsection[{Chapitre II. Paradoxe de Bayle}]{Chapitre II. Paradoxe de Bayle}
\noindent M. Bayle\footnote{Pensées sur la comète, etc.} a prétendu prouver qu’il valait mieux être athée qu’idolâtre ; c’est-à-dire, en d’autres termes, qu’il est moins dangereux de n’avoir point du tout de religion, que d’en avoir une mauvaise. « J’aimerais mieux, dit-il, que l’on dît de moi que je n’existe pas, que si l’on disait que je suis un méchant homme. » Ce n’est qu’un sophisme, fondé sur ce qu’il n’est d’aucune utilité au genre humain que l’on croie qu’un certain homme existe, au lieu qu’il est très utile que l’on croie que Dieu est. De l’idée qu’il n’est pas, suit l’idée de notre indépendance ; ou, si nous ne pouvons pas avoir cette idée, celle de notre révolte. Dire que la religion n’est pas un motif réprimant, parce qu’elle ne réprime pas toujours, c’est dire que les lois civiles ne sont pas un motif réprimant non plus. C’est mal raisonner contre la religion, de rassembler dans un grand ouvrage une longue énumération des maux qu’elle a produits, si l’on ne fait de même celle des biens qu’elle a faits. Si je voulais raconter tous les maux qu’ont produits dans le monde les lois civiles, la monarchie, le gouvernement républicain, je dirais des choses effroyables. Quand il serait inutile que les sujets eussent une religion, il ne le serait pas que les princes en eussent, et qu’ils blanchissent d’écume le seul frein que ceux qui ne craignent point les lois humaines puissent avoir.\par
Un prince qui aime la religion, et qui la craint, est un lion qui cède à la main qui le flatte, ou à la voix qui l’apaise : celui qui craint la religion, et qui la hait, est comme les bêtes sauvages qui mordent la chaîne qui les empêche de se jeter sur ceux qui passent : celui qui n’a point du tout de religion, est cet animal terrible qui ne sent sa liberté que lorsqu’il déchire et qu’il dévore.\par
La question n’est pas de savoir s’il vaudrait mieux qu’un certain homme ou qu’un certain peuple n’eût point de religion, que d’abuser de celle qu’il a ; mais de savoir quel est le moindre mal, que l’on abuse quelquefois de la religion, ou qu’il n’y en ait point du tout parmi les hommes.\par
Pour diminuer l’horreur de l’athéisme, on charge trop l’idolâtrie. Il West pas vrai que, quand les anciens élevaient des autels à quelque vice, cela signifiât qu’ils aimassent ce vice : cela signifiait au contraire qu’ils le haïssaient. Quand les\par
Lacédémoniens érigèrent une chapelle à la Peur, cela ne signifiait pas que cette nation belliqueuse lui demandât de s’emparer dans les combats des cœurs des Lacédémoniens. Il y avait des divinités à qui on demandait de ne pas inspirer le crime, et d’autres à qui on demandait de le détourner.
\subsubsection[{Chapitre III. Que le gouvernement modéré convient mieux à la religion chrétienne et le gouvernement despotique à la mahométane}]{Chapitre III. Que le gouvernement modéré convient mieux à la religion chrétienne et le gouvernement despotique à la mahométane}
\noindent La religion chrétienne est éloignée du pur despotisme : c’est que la douceur étant si recommandée dans l’Évangile, elle s’oppose à la colère despotique avec laquelle le prince se ferait justice, et exercerait ses cruautés.\par
Cette religion défendant la pluralité des femmes, les princes y sont moins renfermés, moins séparés de leurs sujets, et par conséquent plus hommes ; ils sont plus disposés à se faire des lois, et plus capables de sentir qu’ils ne peuvent pas tout.\par
Pendant que les princes mahométans donnent sans cesse la mort ou la reçoivent, la religion, chez les chrétiens, rend les princes moins timides, et par conséquent moins cruels. Le prince compte sur ses sujets, et les sujets sur le prince. Chose admirable ! la religion chrétienne, qui ne semble avoir d’objet que la félicité de l’autre vie, fait encore notre bonheur dans celle-ci.\par
C’est la religion chrétienne qui, malgré la grandeur de l’empire et le vice du climat, a empêché le despotisme de s’établir en Éthiopie, et a porté au milieu de l’Afrique les mœurs de l’Europe et ses lois.\par
Le prince héritier d’Éthiopie jouit d’une principauté, et donne aux autres sujets l’exemple de l’amour et de l’obéissance. Tout près de là, on voit le mahométisme faire renfermer les enfants du roi de Sennar : à sa mort, le Conseil les envoie égorger, en faveur de celui qui monte sur le trône\footnote{{\itshape Relation d’Éthiopie}, par le sieur Poncet, médecin, au quatrième recueil des {\itshape Lettres édifiantes}, p. 290.}.\par
Que, d’un côté, l’on se mette devant les yeux les massacres continuels des rois et des chefs grecs et romains, et, de l’autre, la destruction des peuples et des villes par ces mêmes chefs, Thimur et Gengiskan, qui ont dévasté l’Asie ; et nous verrons que nous devons au christianisme, et dans le gouvernement un certain droit politique, et dans la guerre un certain droit des gens, que la nature humaine ne saurait assez reconnaître.\par
C’est ce droit des gens qui fait que, parmi nous, la victoire laisse aux peuples vaincus ces grandes choses : la vie, la liberté, les lois, les biens, et toujours la religion, lorsqu’on ne s’aveugle pas soi-même.\par
On peut dire que les peuples de l’Europe ne sont pas aujourd’hui plus désunis que ne l’étaient dans l’empire romain, devenu despotique et militaire, les peuples et les armées, ou que ne l’étaient les armées entre elles : d’un côté, les armées se faisaient la guerre ; et, de l’autre, on leur donnait le pillage des villes et le partage ou la confiscation des terres.
\subsubsection[{Chapitre IV. Conséquences du caractère de la religion chrétienne et de celui de la religion mahométane}]{Chapitre IV. Conséquences du caractère de la religion chrétienne et de celui de la religion mahométane}
\noindent Sur le caractère de la religion chrétienne et celui de la mahométane, on doit, sans autre examen, embrasser l’une et rejeter l’autre : car il nous est bien plus évident qu’une religion doit adoucir les mœurs des hommes, qu’il ne l’est qu’une religion soit vraie.\par
C’est un malheur pour la nature humaine, lorsque la religion est donnée par un conquérant. La religion mahométane, qui ne parle que de glaive, agit encore sur les hommes avec cet esprit destructeur qui l’a fondée.\par
L’histoire de Sabbacon\footnote{Voyez Diodore, liv. II.}, un des rois pasteurs, est admirable. Le dieu de Thèbes lui apparut en songe, et lui ordonna de faire mourir tous les prêtres d’Égypte. Il jugea que les dieux n’avaient plus pour agréable qu’il régnât, puisqu’ils lui ordonnaient des choses si contraires à leur volonté ordinaire ; et il se retira en Éthiopie.
\subsubsection[{Chapitre V. Que la religion catholique convient mieux à une monarchie, et que la protestante s’accommode mieux d’une république}]{Chapitre V. Que la religion catholique convient mieux à une monarchie, et que la protestante s’accommode mieux d’une république}
\noindent Lorsqu’une religion naît et se forme dans un État, elle suit ordinairement le plan du gouvernement où elle est établie : car les hommes qui la reçoivent, et ceux qui la font recevoir, n’ont guère d’autres idées de police que celles de l’État dans lequel ils sont nés.\par
Quand la religion chrétienne souffrit, il y a deux siècles, ce malheureux partage qui la divisa en catholique et en protestante, les peuples du nord embrassèrent la protestante, et ceux du midi gardèrent la catholique.\par
C’est que les peuples du nord ont et auront toujours un esprit d’indépendance et de liberté que n’ont pas les peuples du midi ; et qu’une religion qui n’a point de chef visible, convient mieux à l’indépendance du climat que celle qui en a un.\par
Dans les pays mêmes où la religion protestante s’établit, les révolutions se firent sur le plan de l’État politique. Luther ayant pour lui de grands princes, n’aurait guère pu leur faire goûter une autorité ecclésiastique qui n’aurait point eu de prééminence extérieure ; et Calvin ayant pour lui des peuples qui vivaient dans des républiques, ou des bourgeois obscurcis dans des monarchies, pouvait fort bien ne pas établir des prééminences et des dignités.\par
Chacune de ces deux religions pouvait se croire la plus parfaite ; la calviniste se jugeant plus conforme à ce que Jésus-Christ avait dit, et la luthérienne à ce que les apôtres avaient fait.
\subsubsection[{Chapitre VI. Autre paradoxe de Bayle}]{Chapitre VI. Autre paradoxe de Bayle}
\noindent M. Bayle, après avoir insulté toutes les religions, flétrit la religion chrétienne : il ose avancer que de véritables chrétiens ne formeraient pas un État qui pût subsister. Pourquoi non ? Ce seraient des citoyens infiniment éclairés sur leurs devoirs, et qui auraient un très grand zèle pour les remplir ; ils sentiraient très bien les droits de la défense naturelle ; plus ils croiraient devoir à la religion, plus ils penseraient devoir à la patrie. Les principes du christianisme, bien gravés dans le cœur, seraient infiniment plus forts que ce faux honneur des monarchies, ces vertus humaines des républiques, et cette crainte servile des États despotiques.\par
Il est étonnant qu’on puisse imputer à ce grand homme d’avoir méconnu l’esprit de sa propre religion ; qu’il n’ait pas su distinguer les ordres pour l’établissement du christianisme d’avec le christianisme même, ni les préceptes de l’Évangile d’avec ses conseils. Lorsque le législateur, au lieu de donner des lois, a donné des conseils, c’est qu’il a vu que ses conseils, s’ils étaient ordonnés comme des lois, seraient contraires à l’esprit de ses lois.
\subsubsection[{Chapitre VII. Des lois de perfection dans la religion}]{Chapitre VII. Des lois de perfection dans la religion}
\noindent Les lois humaines, faites pour parler à l’esprit, doivent donner des préceptes, et point de conseils : la religion, faite pour parler au cœur, doit donner beaucoup de conseils, et peu de préceptes.\par
Quand, par exemple, elle donne des règles, non pas pour le bien, mais pour le meilleur ; non pas pour ce qui est bon, mais pour ce qui est parfait, il est convenable que ce soient des conseils et non pas des lois ; car la perfection ne regarde pas l’universalité des hommes ni des choses. De plus, si ce sont des lois, il en faudra une infinité d’autres pour faire observer les premières. Le célibat fut un conseil du christianisme : lorsqu’on en fit une loi pour un certain ordre de gens, il en fallut chaque jour de nouvelles pour réduire les hommes à l’observation de celle-ci\footnote{Voyez la Bibliothèque des auteurs ecclésiastiques du sixième siècle, t. V, par M. Dupin.}. Le législateur se fatigua, il fatigua la société, pour faire exécuter aux hommes par précepte, ce que ceux qui aiment la perfection auraient exécuté comme conseil.
\subsubsection[{Chapitre VIII. De l’accord des lois de la morale avec celles de la religion}]{Chapitre VIII. De l’accord des lois de la morale avec celles de la religion}
\noindent Dans un pays où l’on a le malheur d’avoir une religion que Dieu n’a pas donnée, il est toujours nécessaire qu’elle s’accorde avec la morale ; parce que la religion, même fausse, est le meilleur garant que les hommes puissent avoir de la probité des hommes.\par
Les points principaux de la religion de ceux de Pégu\footnote{{\itshape Recueil des voyages qui ont servi à l’établissement de la Compagnie des Indes}, t. III, part. I, p. 63.} sont de ne point tuer, de ne point voler, d’éviter l’impudicité, de ne faire aucun déplaisir à son prochain, de lui faire, au contraire, tout le bien qu’on peut. Avec cela ils croient qu’on se sauvera dans quelque religion que ce soit ; ce qui fait que ces peuples, quoique fiers et pauvres, ont de la douceur et de la compassion pour les malheureux.
\subsubsection[{Chapitre IX. Des Esséens}]{Chapitre IX. {\itshape Des Esséens}}
\noindent Les Esséens\footnote{{\itshape Histoire des Juifs}, par Prideaux.} faisaient vœu d’observer la justice envers les hommes ; de ne faire de mal à personne, même pour obéir ; de haïr les injustes ; de garder la foi à tout le monde ; de commander avec modestie ; de prendre toujours le parti de la vérité ; de fuir tout gain illicite.
\subsubsection[{Chapitre X. De la secte stoïque}]{Chapitre X. De la secte stoïque}
\noindent Les diverses sectes de philosophie, chez les anciens, pouvaient être considérées comme des espèces de religion. il n’y en a jamais eu dont les principes fussent plus dignes de l’homme, et plus propres à former des gens de bien, que celle des stoïciens ; et, si je ne pouvais un moment cesser de penser que je suis chrétien, je ne pourrais m’empêcher de mettre la destruction de la secte de Zénon au nombre des malheurs du genre humain.\par
Elle n’outrait que les choses dans lesquelles il y a de la grandeur : le mépris des plaisirs et de la douleur.\par
Elle seule savait faire les citoyens ; elle seule faisait les grands hommes ; elle seule faisait les grands empereurs.\par
Faites pour un moment abstraction des vérités révélées ; cherchez dans toute la nature, et vous n’y trouverez pas de plus grand objet que les Antonins ; Julien même, Julien (un suffrage ainsi arraché ne me rendra point complice de son apostasie), non, il n’y a point eu après lui de prince plus digne de gouverner les hommes.\par
Pendant que les stoïciens regardaient comme une chose vaine les richesses, les grandeurs humaines, la douleur, les chagrins, les plaisirs, ils n’étaient occupés qu’à travailler au bonheur des hommes, à exercer les devoirs de la société : il semblait qu’ils regardassent cet esprit sacré qu’ils croyaient être en eux-mêmes, comme une espèce de providence favorable qui veillait sur le genre humain.\par
Nés pour la société, ils croyaient tous que leur destin était de travailler pour elle : d’autant moins à charge, que leurs récompenses étaient toutes dans eux-mêmes ; qu’heureux par leur philosophie seule, il semblait que le seul bonheur des autres pût augmenter le leur.
\subsubsection[{Chapitre XI. De la contemplation}]{Chapitre XI. De la contemplation}
\noindent Les hommes étant faits pour se conserver, pour se nourrir, pour se vêtir, et faire toutes les actions de la société, la religion ne doit pas leur donner une vie trop contemplative\footnote{C’est l’inconvénient de la doctrine de Foë et de Laockium.}.\par
Les mahométans deviennent spéculatifs par habitude ; ils prient cinq fois le jour, et chaque fois il faut qu’ils fassent un acte par lequel ils jettent derrière leur dos tout ce qui appartient à ce monde : cela les forme à la spéculation. Ajoutez à cela cette indifférence pour toutes choses, que donne le dogme d’un destin rigide.\par
Si d’ailleurs d’autres causes concourent à leur inspirer le détachement, comme si la dureté du gouvernement, si les lois concernant la propriété des terres, donnent un esprit précaire : tout est perdu.\par
La religion des Guèbres rendit autrefois le royaume de Perse florissant ; elle corrigea les mauvais effets du despotisme : la religion mahométane détruit aujourd’hui ce même empire.
\subsubsection[{Chapitre XII. Des pénitences}]{Chapitre XII. {\itshape Des pénitences}}
\noindent Il est bon que les pénitences soient jointes avec l’idée de travail, non avec l’idée d’oisiveté ; avec l’idée du bien, non avec l’idée de l’extraordinaire ; avec l’idée de frugalité, non avec l’idée d’avarice.
\subsubsection[{Chapitre XIII. Des crimes inexpiables}]{Chapitre XIII. Des crimes inexpiables}
\noindent Il paraît, par un passage des livres des pontifes, rapporté par Cicéron\footnote{Liv. II des {\itshape Lois}.}, qu’il y avait chez les Romains des crimes inexpiables\footnote{{\itshape Sacrum commissum, quod neque expiari poterit, impie commissum est ; quod expiari poterit, publici sacerdotes expianto.}} {\itshape ;} et c’est là-dessus que Zozime fonde le récit si propre à envenimer les motifs de la conversion de Constantin, et Julien cette raillerie amère qu’il fait de cette même conversion dans ses {\itshape Césars}.\par
La religion païenne, qui ne défendait que quelques crimes grossiers, qui arrêtait la main et abandonnait le cœur, pouvait avoir des crimes inexpiables ; mais une religion qui enveloppe toutes les passions ; qui n’est pas plus jalouse des actions que des désirs et des pensées ; qui ne nous tient point attachés par quelques chaînes, mais par un nombre innombrable de fils ; qui laisse derrière elle la justice humaine, et commence une autre justice ; qui est faite pour mener sans cesse du repentir à l’amour, et de l’amour au repentir ; qui met entre le juge et le criminel un grand médiateur, entre le juste et le médiateur un grand juge ; une telle religion ne doit point avoir de crimes inexpiables. Mais, quoiqu’elle donne des craintes et des espérances à tous, elle fait assez sentir que, s’il n’y a point de crime qui, par sa nature, soit inexpiable, toute une vie peut l’être ; qu’il serait très dangereux de tourmenter sans cesse la miséricorde par de nouveaux crimes et de nouvelles expiations ; qu’inquiets sur les anciennes dettes, jamais quittes envers le Seigneur, nous devons craindre d’en contracter de nouvelles, de combler la mesure, et d’aller jusqu’au terme où la bonté paternelle finit.
\subsubsection[{Chapitre XIV. Comment la force de la religion s’applique à celle des lois civiles}]{Chapitre XIV. Comment la force de la religion s’applique à celle des lois civiles}
\noindent Comme la religion et les lois civiles doivent tendre principalement à rendre les hommes bons citoyens, on voit que lorsqu’une des deux s’écartera de ce but, l’autre y doit tendre davantage : moins la religion sera réprimante, plus les lois civiles doivent réprimer.\par
Ainsi, au Japon, la religion dominante n’ayant presque point de dogmes, et ne proposant point de paradis ni d’enfer, les lois, pour y suppléer, ont été faites avec une sévérité, et exécutées avec une ponctualité extraordinaires.\par
Lorsque la religion établit le dogme de la nécessité des actions humaines, les peines des lois doivent être plus sévères et la police plus vigilante, pour que les hommes, qui, sans cela, s’abandonneraient eux-mêmes, soient déterminés par ces motifs ; mais si la religion établit le dogme de la liberté, c’est autre chose.\par
De la paresse de l’âme naît le dogme de la prédestination mahométane ; et du dogme de cette prédestination naît la paresse de l’âme. On a dit : Cela est dans les décrets de Dieu, il faut donc rester en repos. Dans un cas pareil, on doit exciter par les lois les hommes endormis dans la religion.\par
Lorsque la religion condamne des choses que les lois civiles doivent permettre, il est dangereux que des lois civiles ne permettent de leur côté ce que la religion doit condamner ; une de ces choses marquant toujours un défaut d’harmonie et de justesse dans les idées, qui se répand sur l’autre.\par
Ainsi les Tartares\footnote{Voyez la relation de Frère Jean Duplan Carpin, envoyé en Tartarie par le pape Innocent IV, en l’année 1246.} de Gengiskan, chez lesquels c’était un péché, et même un crime capital, de mettre le couteau dans le feu, de s’appuyer contre un fouet, de battre un cheval avec sa bride, de rompre un os avec un autre, ne croyaient pas qu’il y eût de péché à violer la foi, à ravir le bien d’autrui, à faire injure à un homme, à le tuer. En un mot, les lois qui font regarder comme nécessaire ce qui est indifférent, ont cet inconvénient, qu’elles font considérer comme indifférent ce qui est nécessaire.\par
Ceux de Formose croient une espèce d’enfer\footnote{{\itshape Recueil des voyages qui ont servi à l’établissement de la Compagnie des Indes}, t. V, part. I, p. 192.}. mais c’est pour punir ceux qui ont manqué d’aller nus en certaines saisons, qui ont mis des vêtements de toile et non pas de soie, qui ont été chercher des huîtres, qui ont agi sans consulter le chant des oiseaux ; aussi ne regardent-ils point comme péché l’ivrognerie et le dérèglement avec les femmes ; ils croient même que les débauches de leurs enfants sont agréables à leurs dieux,\par
Lorsque la religion justifie pour une chose d’accident, elle perd inutilement le plus grand ressort qui soit parmi les hommes. On croit, chez les Indiens, que les eaux du Gange ont une vertu sanctifiante\footnote{{\itshape Lettres édifiantes}, quinzième recueil.} ; ceux qui meurent sur ses bords sont réputés exempts des peines de l’autre vie, et devoir habiter une région pleine de délices : on envoie, des lieux les plus reculés, des urnes pleines des cendres des morts, pour les jeter dans le Gange. Qu’importe qu’on vive vertueusement, ou non ? on se fera jeter-dans le Gange.\par
L’idée d’un lieu de récompense emporte nécessairement l’idée d’un séjour de peines ; et quand on espère l’un sans craindre l’autre, les lois civiles n’ont plus de force. Des hommes qui croient des récompenses sures dans l’autre vie échapperont au législateur ; ils auront trop de mépris pour la mort. Quel moyen de contenir par les lois un homme qui croit être sûr que la plus grande peine que les magistrats lui pourront infliger ne finira dans un moment que pour commencer son bonheur ?
\subsubsection[{Chapitre XV. Comment les lois civiles corrigent quelquefois les fausses religions}]{Chapitre XV. Comment les lois civiles corrigent quelquefois les fausses religions}
\noindent Le respect pour les choses anciennes, la simplicité ou la superstition ont quelquefois établi des mystères ou des cérémonies qui pouvaient choquer la pudeur ; et de cela les exemples n’ont pas été rares dans le monde. Aristote\footnote{{\itshape Politique}, liv.{\itshape  VII}, chap. XVII.} dit que, dans ce cas, la loi permet que les pères de famille aillent au temple célébrer ces mystères pour leurs femmes et pour leurs enfants. Loi civile admirable, qui conserve les mœurs contre la religion !\par
Auguste\footnote{Suétone, {\itshape in Augusto}, chap. XXXI.} défendit aux jeunes gens de l’un et de l’autre sexe d’assister à aucune cérémonie nocturne, s’ils n’étaient accompagnés d’un parent plus âgé ; et lorsqu’il rétablit les fêtes lupercales, il ne voulut pas que les jeunes gens courussent nus\footnote{{\itshape Ibid.}}.
\subsubsection[{Chapitre XVI. Comment les lois de la religion corrigent les inconvénients de la constitution politique}]{Chapitre XVI. Comment les lois de la religion corrigent les inconvénients de la constitution politique}
\noindent D’un autre côté, la religion peut soutenir l’État politique, lorsque les lois se trouvent dans l’impuissance.\par
Ainsi, lorsque l’État est souvent agité par des guerres civiles, la religion fera beaucoup, si elle établit que quelque partie de cet État reste toujours en paix. Chez les Grecs, les Éléens, comme prêtres d’Apollon, jouissaient d’une paix éternelle. Au Japon\footnote{{\itshape Recueil des voyages qui ont servi à l’établissement de la Compagnie des Indes}, t. IV, part, I, p. 127.}, on laisse toujours en paix la ville de Méaco, qui est une ville sainte : la religion maintient ce règlement ; et cet empire, qui semble être seul sur la terre, qui n’a et qui ne veut avoir aucune ressource de la part des étrangers, a toujours dans son sein un commerce que la guerre ne ruine pas.\par
Dans les États où les guerres ne se font pas par une délibération commune, et où les lois ne se sont laissé aucun moyen de les terminer ou de les prévenir, la religion établit des temps de paix ou de trêves, pour que le peuple puisse faire les choses sans lesquelles l’État ne pourrait subsister, comme les semailles et les travaux pareils.\par
Chaque année, pendant quatre mois, toute hostilité cessait entre les tribus arabes\footnote{Voyez Prideaux, {\itshape Vie de Mahomet}, p. 64.} {\itshape : le} moindre trouble eût été une impiété. Quand chaque seigneur faisait en France la guerre ou la paix, la religion donna des trêves, qui devaient avoir lieu dans de certaines saisons.
\subsubsection[{Chapitre XVII. Continuation du même sujet}]{Chapitre XVII. Continuation du même sujet}
\noindent Lorsqu’il y a beaucoup de sujets de haine dans un État, il faut que la religion donne beaucoup de moyens de réconciliation. Les Arabes, peuple brigand, se faisaient souvent des injures et des injustices. Mahomet fit cette loi\footnote{Dans l’Alcoran, liv. I, chap. {\itshape de la Vache}.} : « Si quelqu’un pardonne le sang de son frère\footnote{En renonçant à la loi du talion.}, il pourra poursuivre le malfaiteur pour des dommages et intérêts ; mais celui qui fera tort au méchant, après avoir reçu satisfaction de lui, souffrira au jour du jugement des tourments douloureux. »\par
Chez les Germains, on héritait des haines et des inimitiés de ses proches ; mais elles n’étaient pas éternelles. On expiait l’homicide en donnant une certaine quantité de bétail ; et toute la famille recevait la satisfaction : « chose très utile, dit Tacite\footnote{{\itshape De moribus German.}}, parce que les inimitiés sont très dangereuses chez un peuple libre ». Je crois bien que les ministres de la religion, qui avaient tant de crédit parmi eux, entraient dans ces réconciliations.\par
Chez les Malais\footnote{{\itshape Recueil des voyages qui ont servi à l’établissement de la Compagnie des Indes}, t. VII, p. 303. Voyez aussi les Mémoires du comte de Forbin, et ce qu’il dit sur les Macassars.}, où la réconciliation n’est pas établie, celui qui a tué quelqu’un, sûr d’être assassiné par les parents ou les amis du mort, s’abandonne à sa fureur, blesse et tue tout ce qu’il rencontre.
\subsubsection[{Chapitre XVIII. Comment les lois de la religion ont l’effet des lois civiles}]{Chapitre XVIII. Comment les lois de la religion ont l’effet des lois civiles}
\noindent Les premiers Grecs étaient des petits peuples souvent dispersés, pirates sur la mer, injustes sur la terre, sans police et sans lois. Les belles actions d’Hercule et de Thésée font voir l’état où se trouvait ce peuple naissant. Que pouvait faire la religion, que ce qu’elle fit, pour donner de l’horreur du meurtre ? Elle établit qu’un homme tué par violence était d’abord en colère contre le meurtrier, qu’il lui inspirait du trouble et de la terreur, et voulait qu’il lui cédât les lieux qu’il avait fréquentés\footnote{Platon, {\itshape Des Lois}, liv. IX.} ; on ne pouvait toucher le criminel ni converser avec lui, sans être souillé ou intestable\footnote{Voyez la tragédie d’{\itshape Œdipe à Colone}.} {\itshape ; la} présence du meurtrier devait être épargnée à la ville, et il fallait l’expier\footnote{Platon, {\itshape Des Lois}, liv. IX.}.
\subsubsection[{Chapitre XIX. Que c’est moins la vérité ou la fausseté d’un dogme qui le rend utile ou pernicieux aux hommes dans l’état civil, que l’usage ou l’abus que l’on en fait}]{Chapitre XIX. Que c’est moins la vérité ou la fausseté d’un dogme qui le rend utile ou pernicieux aux hommes dans l’état civil, que l’usage ou l’abus que l’on en fait}
\noindent Les dogmes les plus vrais et les plus saints peuvent avoir de très mauvaises conséquences, lorsqu’on ne les lie pas avec les principes de la société ; et, au contraire, les dogmes les plus faux en peuvent avoir d’admirables, lorsqu’on fait qu’ils se rapportent aux mêmes principes.\par
La religion de Confucius nie l’immortalité de l’âme\footnote{Un philosophe chinois argumente ainsi contre la doctrine de Foë : « Il est dit, dans un livre de cette secte, que le corps est notre domicile, et l’âme l’hôtesse immortelle qui y loge ; mais si le corps de nos parents n’est qu’un logement, il est naturel de le regarder avec le même mépris qu’on a pour un amas de boue et de terre. N’est-ce pas vouloir arracher du cœur la vertu de l’amour des parents ? Cela porte de même à négliger le soin du corps, et à lui refuser la compassion et l’affection si nécessaire pour sa conservation : ainsi les disciples de Foë se tuent à milliers. » Ouvrage d’un philosophe chinois, dans le recueil du P. Du Halde, t. III, p. 52.} ; et la secte de Zénon ne la croyait pas. Qui le dirait ? ces deux sectes ont tiré de leurs mauvais principes des conséquences, non pas justes, mais admirables pour la société.\par
La religion des Tao et des Foë croit l’immortalité de l’âme ; mais de ce dogme si saint, ils ont tiré des conséquences affreuses.\par
Presque par tout le monde, et dans tous les temps, l’opinion de l’immortalité de l’âme, mal prise, a engagé les femmes, les esclaves, les sujets, les amis, à se tuer, pour aller servir dans l’autre monde l’objet de leur respect ou de leur amour. Cela était ainsi dans les Indes occidentales ; cela était ainsi chez les Danois\footnote{Voyez Thomas Bartholin, {\itshape Antiquités danoises}.} ; et cela est encore aujourd’hui au Japon\footnote{Relation du Japon, dans le {\itshape Recueil des voyages qui ont servi à l’établissement de la Compagnie des Indes}.}, à Macassar\footnote{{\itshape Mémoires} de Forbin.}, et dans plusieurs autres endroits de la terre.\par
Ces coutumes émanent moins directement du dogme de l’immortalité de l’âme, que de celui de la résurrection des corps ; d’où l’on a tiré cette conséquence, qu’après la mort un même individu aurait les mêmes besoins, les mêmes sentiments, les mêmes passions. Dans ce point de vue, le dogme de l’immortalité de l’âme affecte prodigieusement les hommes, parce que l’idée d’un simple changement de demeure est plus à la portée de notre esprit, et flatte plus notre cœur, que l’idée d’une modification nouvelle.\par
Ce n’est pas assez pour une religion d’établir un dogme ; il faut encore qu’elle le dirige. C’est ce qu’a fait admirablement bien la religion chrétienne à l’égard des dogmes dont nous parlons : elle nous fait espérer un état que nous croyons, non pas un état que nous sentions ou que nous connaissions ; tout, jusqu’à la résurrection des corps, nous mène à des idées spirituelles.
\subsubsection[{Chapitre XX. Continuation du même sujet}]{Chapitre XX. Continuation du même sujet}
\noindent Les livres\footnote{M. Hyde.} sacrés des anciens Perses disaient : « Si vous voulez être saint, instruisez vos enfants, parce que toutes les bonnes actions qu’ils feront vous seront imputées. » Ils conseillaient de se marier de bonne heure ; parce que les enfants seraient comme un pont au jour du jugement, et que ceux qui n’auraient point d’enfants ne pourraient pas passer. Ces dogmes étaient faux, mais ils étaient très utiles.
\subsubsection[{Chapitre XXI. De la métempsycose}]{Chapitre XXI. De la métempsycose}
\noindent Le dogme de l’immortalité de l’âme se divise en trois branches : celui de l’immortalité pure, celui du simple changement de demeure, celui de la métempsycose ; c’est-à-dire le système des chrétiens, le système des Scythes, le système des Indiens. Je viens de parler des deux premiers ; et je dirai du troisième que, comme il a été bien et mal dirigé, il a aux Indes de bons et de mauvais effets. Comme il donne aux hommes une certaine horreur pour verser le sang, il y a aux Indes très peu de meurtres ; et, quoiqu’on n’y punisse guère de mort, tout le monde y est tranquille.\par
D’un autre côté, les femmes s’y brûlent à la mort de leurs maris : il n’y a que les innocents qui y souffrent une mort violente.
\subsubsection[{Chapitre XXII. Combien il est dangereux que la religion inspire de l’horreur pour des choses indifférentes}]{Chapitre XXII. Combien il est dangereux que la religion inspire de l’horreur pour des choses indifférentes}
\noindent Un certain honneur, que des préjugés de religion établissent aux Indes, fait que les diverses castes ont horreur les unes des autres. Cet honneur est uniquement fondé sur la religion ; ces distinctions de famille ne forment pas des distinctions civiles : il y a tel Indien qui se croirait déshonoré s’il mangeait avec son roi.\par
Ces sortes de distinctions sont liées à une certaine aversion pour les autres hommes, bien différente des sentiments que doivent faire naître les différences des rangs, qui parmi nous contiennent l’amour pour les inférieurs.\par
Les lois de la religion éviteront d’inspirer d’autre mépris que celui du vice, et surtout d’éloigner les hommes de l’amour et de la pitié pour les hommes.\par
La religion mahométane et la religion indienne ont, dans leur sein, un nombre infini de peuples : les Indiens haïssent les mahométans, parce qu’ils mangent de la vache ; les mahométans détestent les Indiens, parce qu’ils mangent du cochon.
\subsubsection[{Chapitre XXIII. Des fêtes}]{Chapitre XXIII. {\itshape Des fêtes}}
\noindent Quand une religion ordonne la cessation du travail, elle doit avoir égard aux besoins des hommes, plus qu’à la grandeur de l’être qu’elle honore.\par
C’était à Athènes\footnote{Xénophon, {\itshape De la République d’Athènes}.} un grand inconvénient que le trop grand nombre de fêtes. Chez ce peuple dominateur, devant qui toutes les villes de la Grèce venaient porter leurs différends, on ne pouvait suffire aux affaires.\par
Lorsque Constantin établit que l’on chômerait le dimanche, il fit cette ordonnance pour les villes\footnote{Leg. 3, Cod. {\itshape De feriis.} Cette loi n’était faite sans doute que pour les païens.}, et non pour les peuples de la campagne : il sentait que dans les villes étaient les travaux utiles, et dans les campagnes les travaux nécessaires.\par
Par la même raison, dans les pays qui se maintiennent par le commerce, le nombre des fêtes doit être relatif à ce commerce même. Les pays protestants et les pays catholiques sont situés\footnote{Les catholiques sont plus vers le midi, et les protestants vers le nord.} de manière que l’on a plus besoin de travail dans les premiers que dans les seconds : la suppression des fêtes convenait donc plus aux pays protestants qu’aux pays catholiques.\par
Dampierre\footnote{{\itshape Nouveaux Voyages autour du monde}, t. II.} remarque que les divertissements des peuples varient beaucoup selon les climats. Comme les climats chauds produisent quantité de fruits délicats, les Barbares, qui trouvent d’abord le nécessaire, emploient plus de temps à se divertir : les Indiens des pays froids n’ont pas tant de loisir, il faut qu’ils pêchent et chassent continuellement : il y a donc chez eux moins de danses, de musique et de festins ; et une religion qui s’établirait chez ces peuples, devrait avoir égard à cela dans l’institution des fêtes.
\subsubsection[{Chapitre XXIV. Des lois de religions locales}]{Chapitre XXIV. Des lois de religions locales}
\noindent Il y a beaucoup de lois locales dans les diverses religions. Et quand Montésuma s’obstinait tant à dire que la religion des Espagnols était bonne pour leur pays, et celle du Mexique pour le sien, il ne disait pas une absurdité, parce qu’en effet les législateurs n’ont pu s’empêcher d’avoir égard à ce que la nature avait établi avant eux.\par
L’opinion de la métempsycose est faite pour le climat des Indes. L’excessive chaleur brûle\footnote{{\itshape Voyage} de Bernier, t. II, p. 137.} toutes les campagnes ; on n’y peut nourrir que très peu de bétail ; on est toujours en danger d’en manquer pour le labourage ; les bœufs ne s’y multiplient\footnote{{\itshape Lettres édifiantes}, douzième recueil, p. 95.} que médiocrement, ils sont sujets à beaucoup de maladies : une loi de religion qui les conserve est donc très convenable à la police du pays.\par
Pendant que les prairies sont brûlées, le riz et les légumes y croissent heureusement, par les eaux qu’on y peut employer : une loi de religion qui ne permet que cette nourriture, est donc très utile aux hommes dans ces climats.\par
La chair\footnote{{\itshape Voyage} de Bernier, t. II, p. 137.} des bestiaux n’y a pas de goût ; et le lait et le beurre qu’ils en tirent fait une partie de leur subsistance : la loi qui défend de manger et de tuer des vaches n’est donc pas déraisonnable aux Indes.\par
Athènes avait dans son sein une multitude innombrable de peuple ; son territoire était stérile : ce fut une maxime religieuse, que ceux qui offraient aux dieux de certains petits présents les honoraient\footnote{Euripide dans Athénée, liv. II, p. 40.} plus que ceux qui immolaient des bœufs.
\subsubsection[{Chapitre XXV. Inconvénient du transport d’une religion d’un pays à un autre}]{Chapitre XXV. Inconvénient du transport d’une religion d’un pays à un autre}
\noindent Il suit de là, qu’il y a très souvent beaucoup d’inconvénients à transporter une religion d’un pays dans un autre\footnote{On ne parle point ici de la religion chrétienne, parce que, comme on a dit au liv. XXIV, chap. I à la fin, la religion chrétienne est le premier bien.}.\par
« Le cochon, dit M. de Boulainvilliers\footnote{{\itshape Vie de Mahomet}.}, doit être très rare en Arabie, où il n’y a presque point de bois, et presque rien de propre à la nourriture de ces animaux ; d’ailleurs, la salure des eaux et des aliments rend le peuple très susceptible des maladies de la peau. » La loi locale qui le défend, ne saurait être bonne pour d’autres pays\footnote{Comme à la Chine.}, où le cochon est une nourriture presque universelle, et en quelque façon nécessaire.\par
Je ferai ici une réflexion. Sanctorius a observé que la chair de cochon que l’on mange se transpire peu\footnote{{\itshape Médecine statique}, sect. III, aphor. 22.} {\itshape ;} et que même cette nourriture empêche beaucoup la transpiration des autres aliments : il a trouvé que la diminution allait à un tiers\footnote{Sect. III, aphor. 23.} {\itshape ;} on sait d’ailleurs que le défaut de transpiration forme ou aigrit les maladies de la peau : la nourriture du cochon doit donc être défendue dans les climats où l’on est sujet à ces maladies, comme celui de la Palestine, de l’Arabie, de l’Égypte et de la Libye.
\subsubsection[{Chapitre XXVI. Continuation du même sujet}]{Chapitre XXVI. Continuation du même sujet}
\noindent M. Chardin\footnote{{\itshape Voyage de Perse}, t. II.} dit qu’il n’y a point de fleuve navigable en Perse, si ce n’est le fleuve Kur, qui est aux extrémités de l’empire. L’ancienne loi des Guèbres, qui défendait de naviguer sur les fleuves, n’avait donc aucun inconvénient dans leur pays ; mais elle aurait ruiné le commerce dans un autre.\par
Les continuelles lotions sont très en usage dans les climats chauds. Cela fait que la loi mahométane et la religion indienne les ordonnent. C’est un acte très méritoire aux Indes de prier Dieu dans l’eau courante\footnote{{\itshape Voyage} de Bernier, t. II.} : mais comment exécuter ces choses dans d’autres climats ?\par
Lorsque la religion fondée sur le climat a trop choqué le climat d’un autre pays, elle n’a pu s’y établir ; et quand on l’y a introduite, elle en a été chassée. Il semble, humainement parlant, que ce soit le climat qui a prescrit des bornes à la religion chrétienne et à la religion mahométane.\par
Il suit de là qu’il est presque toujours convenable qu’une religion ait des dogmes particuliers et un culte général. Dans les lois qui concernent les pratiques de culte, il faut peu de détails ; par exemple, des mortifications, et non pas une certaine mortification. Le christianisme est plein de bon sens : l’abstinence est de droit divin ; mais une abstinence particulière est de droit de police, et on peut la changer.
\subsection[{Livre vingt-cinquième. Des lois dans le rapport qu’elles ont avec l’établissement de la religion de chaque pays et sa police extérieure}]{Livre vingt-cinquième. Des lois dans le rapport qu’elles ont avec l’établissement de la religion de chaque pays et sa police extérieure}
\subsubsection[{Chapitre I. Du sentiment pour la religion}]{Chapitre I. Du sentiment pour la religion}
\noindent L’homme pieux et l’athée parlent toujours de religion ; l’un parle de ce qu’il aime, et l’autre de ce qu’il craint.
\subsubsection[{Chapitre II. Du motif d’attachement pour les diverses religions}]{Chapitre II. Du motif d’attachement pour les diverses religions}
\noindent Les diverses religions du monde ne donnent pas à ceux qui les professent des motifs égaux d’attachement pour elles : cela dépend beaucoup de la manière dont elles se concilient avec la façon de penser et de sentir des hommes.\par
Nous sommes extrêmement portés à l’idolâtrie, et cependant nous ne sommes pas fort attachés aux religions idolâtres ; nous ne sommes guère portés aux idées spirituelles, et cependant nous sommes très attachés aux religions qui nous font adorer un Être spirituel. C’est un sentiment heureux qui vient en partie de la satisfaction que nous trouvons en nous-mêmes, d’avoir été assez intelligents pour avoir choisi une religion qui tire la divinité de l’humiliation où les autres l’avaient mise. Nous regardons l’idolâtrie comme la religion des peuples grossiers ; et la religion qui a pour objet un être spirituel, comme celle des peuples éclairés.\par
Quand, avec l’idée d’un Être spirituel suprême, qui forme le dogme, nous pouvons joindre encore des idées sensibles qui entrent dans le culte, cela nous donne un grand attachement pour la religion, parce que les motifs dont nous venons de parler se trouvent joints à notre penchant naturel pour les choses sensibles. Aussi les catholiques, qui ont plus de cette sorte de culte que les protestants, sont-ils plus invinciblement attachés à leur religion que les protestants ne le sont à la leur, et plus zélés pour sa propagation.\par
Lorsque le peuple d’Éphèse eut appris que les pères du concile avaient décidé qu’on pouvait appeler la Vierge Mère {\itshape de Dieu, il} fut transporté de joie ; il baisait les mains des évêques, il embrassait leurs genoux ; tout retentissait d’acclamations\footnote{Lettre de saint Cyrille.}.\par
Quand une religion intellectuelle nous donne encore l’idée d’un choix fait par la divinité, et d’une distinction de ceux qui la professent d’avec ceux qui ne la professent pas, cela nous attache beaucoup à cette religion. Les mahométans ne seraient pas si bons musulmans, si d’un côté il n’y avait pas de peuples idolâtres qui leur font penser qu’ils sont les vengeurs de l’unité de Dieu, et de l’autre des chrétiens, pour leur faire croire qu’ils sont l’objet de ses préférences.\par
Une religion chargée de beaucoup de pratiques\footnote{Ceci n’est point contradictoire avec ce que j’ai dit au chapitre pénultième du livre précédent : ici, je parle des motifs d’attachement pour une religion, et là, des moyens de la rendre plus générale.} attache plus à elle qu’une autre qui l’est moins ; on tient beaucoup aux choses dont on est continuellement occupé : témoin l’obstination tenace des mahométans\footnote{Cela se remarque par toute la terre. Voyez sur les Turcs les {\itshape Missions du Levant}, le {\itshape Recueil des voyages qui ont servi à l’établissement de la Compagnie des Indes}, t. III, part. 1, p. 201, sur les Maures de Batavia ; et le P. Labat, sur les nègres mahométans, etc.} et des Juifs, et la facilité qu’ont de changer de religion les peuples barbares et sauvages, qui, uniquement occupés de la chasse ou de la guerre, ne se chargent guère de pratiques religieuses.\par
Les hommes sont extrêmement portés à espérer et à craindre ; et une religion qui n’aurait ni enfer ni paradis, ne saurait guère leur plaire. Cela se prouve par la facilité qu’ont eue les religions étrangères à s’établir au Japon, et le zèle et l’amour avec lesquels on les y a reçues\footnote{La religion chrétienne et les religions des Indes : celles-ci ont un enfer et un paradis, au lieu que la religion des Sintos n’en a point.}.\par
Pour qu’une religion attache, il faut qu’elle ait une morale pure. Les hommes, fripons en détail, sont en gros de très honnêtes gens ; ils aiment la morale ; et si je ne traitais pas un sujet si grave, je dirais que cela se voit admirablement bien sur les théâtres : on est sûr de plaire au peuple par les sentiments que la morale avoue, et on est sûr de le choquer par ceux qu’elle réprouve.\par
Lorsque le culte extérieur a une grande magnificence, cela nous flatte et nous donne beaucoup d’attachement pour la religion. Les richesses des temples et celles du clergé nous affectent beaucoup. Ainsi la misère même des peuples est un motif qui les attache à cette religion qui a servi de prétexte à ceux qui ont causé leur misère.
\subsubsection[{Chapitre III. Des temples}]{Chapitre III. {\itshape Des temples}}
\noindent Presque tous les peuples policés habitent dans des maisons. De là est venue naturellement l’idée de bâtir à Dieu une maison où ils puissent l’adorer et l’aller chercher dans leurs craintes ou leurs espérances.\par
En effet, rien n’est plus consolant pour les hommes, qu’un lieu où ils trouvent la divinité plus présente, et où tous ensemble ils font parler leur faiblesse et leur misère.\par
Mais cette idée si naturelle ne vient qu’aux peuples qui cultivent les terres ; et on ne verra pas bâtir de temple chez ceux qui n’ont pas de maisons eux-mêmes.\par
C’est ce qui fit que Gengiskan marqua un si grand mépris pour les mosquées\footnote{Entrant dans la mosquée de Buchara, il enleva l’Alcoran, et le jeta sous les pieds de ses chevaux. {\itshape Histoire des Tartares}, part. III, p. 273.}. Ce prince\footnote{{\itshape Ibid.}, p. 342.} interrogea les mahométans ; il approuva tous leurs dogmes, excepté celui qui porte la nécessité d’aller à La Mecque ; il ne pouvait comprendre qu’on ne pût pas adorer Dieu partout. Les Tartares, n’habitant point de maisons, ne connaissaient point de temples.\par
Les peuples qui n’ont point de temples ont peu d’attachement pour leur religion : voilà pourquoi les Tartares ont été de tout temps si tolérants\footnote{Cette disposition d’esprit a passé jusqu’aux Japonais, qui tirent leur origine des Tartares, comme il est aisé de le prouver.} ; pourquoi les peuples barbares qui conquirent l’empire romain ne balancèrent pas un moment à embrasser le christianisme ; pourquoi les sauvages de l’Amérique sont si peu attachés à leur propre religion ; et pourquoi, depuis que nos missionnaires leur ont fait bâtir au Paraguay des églises, ils sont si fort zélés pour la nôtre.\par
Comme la divinité est le refuge des malheureux, et qu’il n’y a pas de gens plus malheureux que les criminels, on a été naturellement porté à penser que les temples étaient un asile pour eux ; et cette idée parut encore plus naturelle chez les Grecs, où les meurtriers, chassés de leur ville et de la présence des hommes, semblaient n’avoir plus de maisons que les temples, ni d’autres protecteurs que les dieux.\par
Ceci ne regarda d’abord que les homicides involontaires ; mais, lorsqu’on y comprit les grands criminels, on tomba dans une contradiction grossière : s’ils avaient offensé les hommes, ils avaient à plus forte raison offensé les dieux.\par
Ces asiles se multiplièrent dans la Grèce : les temples, dit Tacite\footnote{{\itshape Annales}, liv. II.}, étaient remplis de débiteurs insolvables et d’esclaves méchants ; les magistrats avaient de la peine à exercer la police ; le peuple protégeait les crimes des hommes, comme les cérémonies des dieux ; le sénat fut obligé d’en retrancher un grand nombre.\par
Les lois de Moïse furent très sages. Les homicides involontaires étaient innocents, mais ils devaient être ôtés de devant les yeux des parents du mort : il établit donc un asile pour eux\footnote{Nombres, chap. XXXV.}. Les grands criminels ne méritent point d’asile, ils n’en eurent pas\footnote{{\itshape Ibid.}}. Les Juifs n’avaient qu’un tabernacle portatif, et qui changeait continuellement de lieu ; cela excluait l’idée d’asile. Il est vrai qu’ils devaient avoir un temple ; mais les criminels qui y seraient venus de toutes parts, auraient pu troubler le service divin. Si les homicides avaient été chassés hors du pays, comme ils le furent chez les Grecs, il eût été à craindre qu’ils n’adorassent des dieux étrangers. Toutes ces considérations firent établir des villes d’asile, où l’on devait rester jusqu’à la mort du souverain pontife.
\subsubsection[{Chapitre IV. Des ministres de la religion}]{Chapitre IV. Des ministres de la religion}
\noindent Les premiers hommes, dit Porphyre, ne sacrifiaient que de l’herbe. Pour un culte si simple, chacun pouvait être pontife dans sa famille.\par
Le désir naturel de plaire à la divinité multiplia les cérémonies : ce qui fit que les hommes, occupés à l’agriculture, devinrent incapables de les exécuter toutes, et d’en remplir les détails.\par
On consacra aux dieux des lieux particuliers ; il fallut qu’il y eût des ministres pour en prendre soin, comme chaque citoyen prend soin de sa maison et de ses affaires domestiques. Aussi les peuples qui n’ont point de prêtres sont-ils ordinairement barbares. Tels étaient autrefois les Pédaliens\footnote{Lilius Giraldus, p. 726.}, tels sont encore les Wolgusky\footnote{Peuples de la Sibérie. Voy. la {\itshape Relation} de M. Everard Isbrandsides, dans le {\itshape Recueil des voyages du Nord}, t. VIII.}.\par
Des gens consacrés à la divinité devaient être honorés, surtout chez les peuples qui s’étaient formé une certaine idée d’une pureté corporelle, nécessaire pour approcher des lieux les plus agréables aux dieux, et dépendante de certaines pratiques.\par
Le culte des dieux demandant une attention continuelle, la plupart des peuples furent portés à faire du clergé un corps séparé. Ainsi, chez les Égyptiens, les Juifs et les Perses\footnote{Voyez M. Hyde.}, on consacra à la divinité de certaines familles, qui se perpétuaient et faisaient le service. Il y eut même des religions où l’on ne pensa pas seulement à éloigner les ecclésiastiques des affaires, mais encore à leur ôter l’embarras d’une famille ; et c’est la pratique de la principale branche de la loi chrétienne.\par
Je ne parlerai point ici des conséquences de la loi du célibat : on sent qu’elle pourrait devenir nuisible, à proportion que le corps du clergé serait trop étendu, et que par conséquent celui des laïques ne le serait pas assez.\par
Par la nature de l’entendement humain, nous aimons en fait de religion tout ce qui suppose un effort, comme, en matière de morale, nous aimons spéculativement tout ce qui porte le caractère de la sévérité. Le célibat a été plus agréable aux peuples à qui il semblait convenir le moins, et pour lesquels il pouvait avoir de plus fâcheuses suites. Dans les pays du midi de l’Europe où, par la nature du climat, la loi du célibat est plus difficile à observer, elle a été retenue ; dans ceux du nord, où les passions sont moins vives, elle a été proscrite. Il y a plus : dans les pays où il y peu d’habitants, elle a été admise ; dans ceux où il y en a beaucoup, on l’a rejetée. On sent que toutes ces réflexions ne portent que sur la trop grande extension du célibat, et non sur le célibat même.
\subsubsection[{Chapitre V. Des bornes que les lois doivent mettre aux richesses du clergé}]{Chapitre V. Des bornes que les lois doivent mettre aux richesses du clergé}
\noindent Les familles particulières peuvent périr : ainsi les biens n’y ont point une destination perpétuelle. Le clergé est une famille qui ne peut pas périr : les biens y sont donc attachés pour toujours, et n’en peuvent pas sortir.\par
Les familles particulières peuvent s’augmenter : il faut donc que leurs biens puissent croître aussi. Le clergé est une famille qui ne doit point s’augmenter : les biens doivent donc y être bornés.\par
Nous avons retenu les dispositions du {\itshape Lévitique} sur les biens du clergé, excepté celles qui regardent les bornes de ces biens : effectivement, on ignorera toujours parmi nous quel est le terme après lequel il n’est plus permis à une communauté religieuse d’acquérir.\par
Ces acquisitions sans fin paraissent aux peuples si déraisonnables, que celui qui voudrait parler pour elles serait regardé comme imbécile.\par
Les lois civiles trouvent quelquefois des obstacles à changer des abus établis, parce qu’ils sont liés à des choses qu’elles doivent respecter : dans ce cas, une disposition indirecte marque plus le bon esprit du législateur qu’une autre qui frapperait sur la chose même. Au lieu de défendre les acquisitions du clergé, il faut chercher à l’en dégoûter lui-même ; laisser le droit, et ôter le fait.\par
Dans quelques pays de l’Europe, la considération des droits des seigneurs a fait établir en leur faveur un droit d’indemnité sur les immeubles acquis par les gens de mainmorte. L’intérêt du prince lui a fait exiger un droit d’amortissement dans le même cas. En Castille, où il n’y a point de droit pareil, le clergé a tout envahi ; en Aragon, où il y a quelque droit d’amortissement, il a acquis moins ; en France, où ce droit et celui d’indemnité sont établis, il a moins acquis encore ; et l’on peut dire que la prospérité de cet État est due en partie à l’exercice de ces deux droits. Augmentez-les ces droits, et arrêtez la mainmorte, s’il est possible.\par
Rendez sacré et inviolable l’ancien et nécessaire domaine du clergé ; qu’il soit fixe et éternel comme lui : mais laissez sortir de ses mains les nouveaux domaines.\par
Permettez de violer la règle, lorsque la règle est devenue un abus ; souffrez l’abus, lorsqu’il rentre dans la règle.\par
On se souvient toujours à Rome d’un mémoire qui y fut envoyé à l’occasion de quelques démêlés avec le clergé. On y avait mis cette maxime : « Le clergé doit contribuer aux charges de l’État, quoi qu’en dise l’Ancien Testament. » On en conclut que l’auteur du mémoire entendait mieux le langage de la maltôte que celui de la religion.
\subsubsection[{Chapitre VI. Des monastères}]{Chapitre VI. Des monastères}
\noindent Le moindre bon sens fait voir que ces corps qui se perpétuent sans fin, ne doivent pas vendre leurs fonds à vie, ni faire des emprunts à vie, à moins qu’on ne veuille qu’ils se rendent héritiers de tous ceux qui n’ont point de parents, et de tous ceux qui n’en veulent point avoir. Ces gens jouent contre le peuple, mais ils tiennent la banque contre lui.
\subsubsection[{Chapitre VII. Du luxe de la superstition}]{Chapitre VII. Du luxe de la superstition}
\noindent « Ceux-là sont impies envers les dieux, dit Platon\footnote{{\itshape Des Lois}, liv. X.}, qui nient leur existence ; ou qui l’accordent, mais soutiennent qu’ils ne se mêlent point des choses d’ici-bas ; ou enfin qui pensent qu’on les apaise aisément par des sacrifices : trois opinions également pernicieuses. » Platon dit là tout ce que la lumière naturelle a jamais dit de plus sensé en matière de religion.\par
La magnificence du culte extérieur a beaucoup de rapport à la constitution de l’État. Dans les bonnes républiques, on n’a pas seulement réprimé le luxe de la vanité, mais encore celui de la superstition. On a fait dans la religion des lois d’épargne. De ce nombre sont plusieurs lois de Solon, plusieurs lois de Platon sur les funérailles, que Cicéron a adoptées ; enfin quelques lois de Numa\footnote{{\itshape Rogum vino ne respergito}. Loi des Douze Tables.} sur les sacrifices.\par
« Des oiseaux, dit Cicéron, et des peintures faites en un jour, sont des dons très divins. »\par
« Nous offrons des choses communes, disait un Spartiate, afin que nous ayons tous les jours le moyen d’honorer les dieux. »\par
Le soin que les hommes doivent avoir de rendre un culte à la divinité est bien différent de la magnificence de ce culte. Ne lui offrons point nos trésors, si nous ne voulons lui faire voir l’estime que nous faisons des choses qu’elle veut que nous méprisions.\par
« Que doivent penser les dieux des dons des impies, dit admirablement Platon, puisqu’un homme de bien rougirait de recevoir des présents d’un malhonnête homme ? »\par
Il ne faut pas que la religion, sous prétexte de dons, exige des peuples ce que les nécessités de l’État leur ont laissé ; et, comme dit Platon\footnote{{\itshape Des Lois}, liv. IV.}, des hommes chastes et pieux doivent offrir des dons qui leur ressemblent.\par
il ne faudrait pas non plus que la religion encourageât les dépenses des funérailles. Qu’y a-t-il de plus naturel, que d’ôter la différence des fortunes dans une chose et dans les moments qui égalisent toutes les fortunes ?
\subsubsection[{Chapitre VIII. Du pontificat}]{Chapitre VIII. {\itshape Du pontificat}}
\noindent Lorsque la religion a beaucoup de ministres, il est naturel qu’ils aient un chef, et que le pontificat y soit établi. Dans la monarchie, où l’on ne saurait trop séparer les ordres de l’État, et où l’on ne doit point assembler sur une même tête toutes les puissances, il est bon que le pontificat soit séparé de l’empire. La même nécessité ne se rencontre pas dans le gouvernement despotique, dont la nature est de réunir sur une même tête tous les pouvoirs. Mais, dans ce cas, il pourrait arriver que le prince regarderait la religion comme ses lois mêmes, et comme des effets de sa volonté. Pour prévenir cet inconvénient, il faut qu’il y ait des monuments de la religion ; par exemple, des livres sacrés qui la fixent et qui l’établissent. Le roi de Perse est le chef de la religion ; mais l’Alcoran règle la religion : l’empereur de la Chine est le souverain pontife ; mais il y a des livres qui sont entre les mains de tout le monde, auxquels il doit lui-même se conformer. En vain un empereur voulut-il les abolir, ils triomphèrent de la tyrannie.
\subsubsection[{Chapitre IX. De la tolérance en fait de religion}]{Chapitre IX. De la tolérance en fait de religion}
\noindent Nous sommes ici politiques, et non pas théologiens ; et, pour les théologiens mêmes, il y a bien de la différence entre tolérer une religion et l’approuver.\par
Lorsque les lois d’un État ont cru devoir souffrir plusieurs religions, il faut qu’elles les obligent aussi à se tolérer entre elles. C’est un principe, que toute religion qui est réprimée devient elle-même réprimante : car sitôt que, par quelque hasard, elle peut sortir de l’oppression, elle attaque la religion qui l’a réprimée, non pas comme une religion, mais comme une tyrannie.\par
Il est donc utile que les lois exigent de ces diverses religions, non seulement qu’elles ne troublent pas l’État, mais aussi qu’elles ne se troublent pas entre elles. Un citoyen ne satisfait point aux lois, en se contentant de ne pas agiter le corps de l’État ; il faut encore qu’il ne trouble pas quelque citoyen que ce soit.
\subsubsection[{Chapitre X. Continuation du même sujet}]{Chapitre X. Continuation du même sujet}
\noindent Comme il n’y a guère que les religions intolérantes qui aient un grand zèle pour s’établir ailleurs, parce qu’une religion qui peut tolérer les autres, ne songe guère à sa propagation, ce sera une très bonne loi civile, lorsque l’État est satisfait de la religion déjà établie, de ne point souffrir l’établissement d’une autre\footnote{Je ne parle point dans tout ce chapitre de la religion chrétienne, parce que, comme j’ai dit ailleurs, la religion chrétienne est le premier bien. Voyez la fin du chapitre I du livre précédent, et la {\itshape Défense de l’esprit des lois}, seconde partie.}.\par
Voici donc le principe fondamental des lois politiques en fait de religion. Quand on est maître de recevoir dans un État une nouvelle religion, ou de ne la pas recevoir, il ne faut pas l’y établir ; quand elle y est établie, il faut la tolérer.
\subsubsection[{Chapitre XI. Du changement de religion}]{Chapitre XI. Du changement de religion}
\noindent Un prince qui entreprend dans son État de détruire ou de changer la religion dominante, s’expose beaucoup. Si son gouvernement est despotique, il court plus de risque de voir une révolution, que par quelque tyrannie que ce soit, qui n’est jamais dans ces sortes d’États une chose nouvelle. La révolution vient de ce qu’un État ne change pas de religion, de mœurs et de manières dans un instant, et aussi vite que le prince publie l’ordonnance qui établit une religion nouvelle.\par
De plus, la religion ancienne est liée avec la constitution de l’État, et la nouvelle n’y tient point : celle-là s’accorde avec le climat, et souvent la nouvelle s’y refuse. Il y a plus : les citoyens se dégoûtent de leurs lois ; ils prennent du mépris pour le gouvernement déjà établi ; on substitue des soupçons contre les deux religions à une ferme croyance pour une ; en un mot, on donne à l’État, au moins pour quelque temps, et de mauvais citoyens, et de mauvais fidèles.
\subsubsection[{Chapitre XII. Des lois pénales}]{Chapitre XII. Des lois pénales}
\noindent Il faut éviter les lois pénales en fait de religion. Elles impriment de la crainte, il est vrai ; mais comme la religion a ses lois pénales aussi qui inspirent de la crainte, l’une est effacée par l’autre. Entre ces deux craintes différentes, les âmes deviennent atroces.\par
La religion a de si grandes menaces, elle a de si grandes promesses, que lorsqu’elles sont présentes à notre esprit, quelque chose que le magistrat puisse faire pour nous contraindre à la quitter, il semble qu’on ne nous laisse rien quand on nous l’ôte, et qu’on ne nous ôte rien lorsqu’on nous la laisse.\par
Ce n’est donc pas en remplissant l’âme de ce grand objet, en l’approchant du moment où il lui doit être d’une plus grande importance, que l’on parvient à l’en détacher : il est plus sûr d’attaquer une religion par la faveur, par les commodités de la vie, par l’espérance de la fortune ; non pas par ce qui avertit, mais par ce qui fait que l’on oublie ; non pas par ce qui indigne, mais par ce qui jette dans la tiédeur, lorsque d’autres passions agissent sur nos âmes, et que celles que la religion inspire sont dans le silence. Règle générale : en fait de changement de religion, les invitations sont plus fortes que les peines.\par
Le caractère de l’esprit humain a paru dans l’ordre même des peines qu’on a employées. Que l’on se rappelle les persécutions du Japon\footnote{Voyez le {\itshape Recueil des voyages qui ont servi à l’établissement de la Compagnie des Indes}, t. V, part. I, p. 192.} ; on se révolta plus contre les supplices cruels que contre les peines longues, qui lassent plus qu’elles n’effarouchent, qui sont plus difficiles à surmonter, parce qu’elles paraissent moins difficiles.\par
En un mot, l’histoire nous apprend assez que les lois pénales n’ont jamais eu d’effet que comme destruction.
\subsubsection[{Chapitre XIII. Très humble remontrance aux inquisiteurs d’Espagne et de Portugal}]{Chapitre XIII. Très humble remontrance aux inquisiteurs d’Espagne et de Portugal}
\noindent Une Juive de dix-huit ans, brûlée à Lisbonne au dernier autodafé, donna occasion à ce petit ouvrage ; et je crois que c’est le plus inutile qui ait jamais été écrit. Quand il s’agit de prouver des choses si claires, on est sûr de ne pas convaincre.\par
L’auteur déclare que, quoiqu’il soit Juif, il respecte la religion chrétienne, et qu’il l’aime assez pour ôter aux princes qui ne seront pas chrétiens un prétexte plausible pour la persécuter.\par
« Vous vous plaignez, dit-il aux inquisiteurs, de ce que l’empereur du Japon fait brûler à petit feu tous les chrétiens qui sont dans ses États ; mais il vous répondra : Nous vous traitons, vous qui ne croyez pas comme nous, comme vous traitez vous-mêmes ceux qui ne croient pas comme vous : vous ne pouvez vous plaindre que de votre faiblesse, qui vous empêche de nous exterminer, et qui fait que nous vous exterminons.\par
« Mais il faut avouer que vous êtes bien plus cruels que cet empereur. Vous nous faites mourir, nous qui ne croyons que ce que vous croyez, parce que nous ne croyons pas tout ce que vous croyez. Nous suivons une religion que vous savez vous-mêmes avoir été autrefois chérie de Dieu : nous pensons que Dieu l’aime encore, et vous pensez qu’il ne l’aime plus ; et parce que vous jugez ainsi, vous faites passer par le fer et par le feu ceux qui sont dans cette erreur si pardonnable, de croire que Dieu aime encore ce qu’il a aimé\footnote{C’est la source de l’aveuglement des Juifs, de ne pas sentir que l’économie de l’Évangile est dans l’ordre des desseins de Dieu, et qu’ainsi elle est une suite de son immutabilité même.}.\par
« Si vous êtes cruels à notre égard, vous l’êtes bien plus à l’égard de nos enfants ; vous les faites brûler, parce qu’ils suivent les inspirations que leur ont données ceux que la loi naturelle et les lois de tous les peuples leur apprennent à respecter comme des dieux.\par
« Vous vous privez de l’avantage que vous a donné sur les mahométans la manière dont leur religion s’est établie. Quand ils se vantent du nombre de leurs fidèles, vous leur dites que la force les leur a acquis, et qu’ils ont étendu leur religion par le fer : pourquoi donc établissez-vous la vôtre par le feu ?\par
« Quand vous voulez nous faire venir à vous, nous vous objectons une source dont vous vous faites gloire de descendre. Vous nous répondez que votre religion est nouvelle, mais qu’elle est divine ; et vous le prouvez parce qu’elle s’est accrue par la persécution des païens et par le sang de vos martyrs ; mais aujourd’hui vous prenez le rôle des Dioclétiens, et vous nous faites prendre le vôtre.\par
« Nous vous conjurons, non pas par le Dieu puissant que nous servons, vous et nous, mais par le Christ que vous nous dites avoir pris la condition humaine pour vous proposer des exemples que vous puissiez suivre ; nous vous conjurons d’agir avec nous comme il agirait lui-même s’il était encore sur la terre. Vous voulez que nous soyons chrétiens, et vous ne voulez pas l’être.\par
« Mais si vous ne voulez pas être chrétiens, soyez au moins des hommes : traitez-nous comme vous feriez, si, n’ayant que ces faibles lueurs de justice que la nature nous donne, vous n’aviez point une religion pour vous conduire, et une révélation pour vous éclairer.\par
« Si le ciel vous a assez aimés pour vous faire voir la vérité, il vous a fait une grande grâce ; mais est-ce aux enfants qui ont eu l’héritage de leur père, de haïr ceux qui ne l’ont pas eu ?\par
« Que si vous avez cette vérité, ne nous la cachez pas par la manière dont vous nous la proposez. Le caractère de la vérité, c’est son triomphe sur les cœurs et les esprits, et non pas cette impuissance que vous avouez lorsque vous voulez la faire recevoir par des supplices.\par
« Si vous êtes raisonnables, vous ne devez pas nous faire mourir parce que nous ne voulons pas vous tromper. Si votre Christ est le fils de Dieu, nous espérons qu’il nous récompensera de n’avoir pas voulu profaner ses mystères ; et nous croyons que le Dieu que nous servons, vous et nous, ne nous punira pas de ce que nous avons souffert la mort pour une religion qu’il nous a autrefois donnée, parce que nous croyons qu’il nous l’a encore donnée.\par
« Vous vivez dans un siècle où la lumière naturelle est plus vive qu’elle n’a jamais été, où la philosophie a éclairé les esprits, où la morale de votre Évangile a été plus connue, où les droits respectifs des hommes les uns sur les autres, l’empire qu’une conscience a sur une autre conscience, sont mieux établis. Si donc vous ne revenez pas de vos anciens préjugés, qui, si vous n’y prenez garde, sont vos passions, il faut avouer que vous êtes incorrigibles, incapables de toute lumière et de toute instruction ; et une nation est bien malheureuse, qui donne de l’autorité à des hommes tels que vous.\par
« Voulez-vous que nous vous disions naïvement notre pensée ? Vous nous regardez plutôt comme vos ennemis, que comme les ennemis de votre religion ; car, si vous aimiez votre religion, vous ne la laisseriez pas corrompre par une ignorance grossière.\par
« Il faut que nous vous avertissions d’une chose : c’est que, si quelqu’un dans la postérité ose jamais dire que dans le siècle où nous vivons, les peuples d’Europe étaient policés, on vous citera pour prouver qu’ils étaient barbares ; et l’idée que l’on aura de vous sera telle, qu’elle flétrira votre siècle, et portera la haine sur tous vos contemporains. »
\subsubsection[{Chapitre XIV. Pourquoi la religion chrétienne est si odieuse au Japon}]{Chapitre XIV. Pourquoi la religion chrétienne est si odieuse au Japon}
\noindent J’ai parlé\footnote{Liv. VI, chap. XXIV.} du caractère atroce des âmes japonaises. Les magistrats regardèrent la fermeté qu’inspire le christianisme, lorsqu’il s’agit de renoncer à la foi, comme très dangereuse : on crut voir augmenter l’audace. La loi du Japon punit sévèrement la moindre désobéissance. On ordonna de renoncer à la religion chrétienne : n’y pas renoncer, c’était désobéir ; on châtia ce crime, et la continuation de la désobéissance parut mériter un autre châtiment.\par
Les punitions, chez les Japonais, sont regardées comme la vengeance d’une insulte faite au prince. Les chants d’allégresse de nos martyrs parurent être un attentat contre lui : le titre de martyr intimida les magistrats ; dans leur esprit, il signifiait rebelle ; ils firent tout pour empêcher qu’on ne l’obtînt. Ce fut alors que les âmes s’effarouchèrent, et que l’on vit un combat horrible entre les tribunaux qui condamnèrent et les accuses qui souffrirent, entre les lois civiles et celles de la religion.
\subsubsection[{Chapitre XV. De la propagation de la religion}]{Chapitre XV. De la propagation de la religion}
\noindent Tous les peuples d’Orient, excepté les mahométans, croient toutes les religions en elles-mêmes indifférentes. Ce n’est que comme changement dans le gouvernement, qu’ils craignent l’établissement d’une autre religion. Chez les Japonais, où il y a plusieurs sectes, et où l’État a eu si longtemps un chef ecclésiastique, on ne dispute jamais sur la religion\footnote{Voyez Kempfer.}.\par
Il en est de même chez les Siamois\footnote{{\itshape Mémoires} du comte de Forbin.}. Les Calmouks\footnote{{\itshape Histoire des Tartares}, part. V.} font plus : ils se font une affaire de conscience de souffrir toutes sortes de religions. À Calicut, c’est une maxime d’État, que toute religion est bonne\footnote{{\itshape Voyage} de François Pyrard, chap. XXVII.}.\par
Mais il n’en résulte pas qu’une religion apportée d’un pays très éloigné, et totalement différent de climat, de lois, de mœurs et de manières, ait tout le succès que sa sainteté devrait lui promettre. Cela est surtout vrai dans les grands empires despotiques : on tolère d’abord les étrangers, parce qu’on ne fait point d’attention à ce qui ne paraît pas blesser la puissance du prince ; on y est dans une ignorance extrême de tout. Un Européen peut se rendre agréable par de certaines connaissances qu’il procure : cela est bon pour les commencements. Mais, sitôt que l’on a quelque succès, que quelque dispute s’élève, que les gens qui peuvent avoir quelque intérêt sont avertis ; comme cet État, par sa nature, demande surtout la tranquillité, et que le moindre trouble peut le renverser, on proscrit d’abord la religion nouvelle et ceux qui l’annoncent ; les disputes entre ceux qui prêchent, venant à éclater, on commence à se dégoûter d’une religion, dont ceux mêmes qui la proposent ne conviennent pas.
\subsection[{Livre vingt-sixième. Des lois dans le rapport qu’elles doivent avoir avec l’ordre des choses sur lesquelles elles statuent}]{Livre vingt-sixième. Des lois dans le rapport qu’elles doivent avoir avec l’ordre des choses sur lesquelles elles statuent}
\subsubsection[{Chapitre I. Idée de ce livre}]{Chapitre I. Idée de ce livre}
\noindent Les hommes sont gouvernés par diverses sortes de lois : par le droit naturel ; par le droit divin, qui est celui de la religion ; par le droit ecclésiastique, autrement appelé canonique, qui est celui de la police de la religion ; par le droit des gens, qu’on peut considérer comme le droit civil de l’univers, dans le sens que chaque peuple en est un citoyen ; par le droit politique général, qui a pour objet cette sagesse humaine qui a fondé toutes les sociétés ; par le droit politique particulier, qui concerne chaque société ; par le droit de conquête, fondé sur ce qu’un peuple a voulu, a pu, ou a dû faire violence à un autre ; par le droit civil de chaque société, par lequel un citoyen peut défendre ses biens et sa vie contre tout autre citoyen ; enfin, par le droit domestique, qui vient de ce qu’une société est divisée en diverses familles, qui ont besoin d’un gouvernement particulier.\par
Il y a donc différents ordres de lois ; et la sublimité de la raison humaine consiste à savoir bien auquel de ces ordres se rapportent principalement les choses sur lesquelles on doit statuer, et à ne point mettre de confusion dans les principes qui doivent gouverner les hommes.
\subsubsection[{Chapitre II. Des lois divines et des lois humaines}]{Chapitre II. Des lois divines et des lois humaines}
\noindent On ne doit point statuer par les lois divines ce qui doit l’être par les lois humaines, ni régler par les lois humaines ce qui doit l’être par les lois divines.\par
Ces deux sortes de lois diffèrent par leur origine, par leur objet et par leur nature.\par
Tout le monde convient bien que les lois humaines sont d’une autre nature que les lois de la religion, et c’est un grand principe ; mais ce principe lui-même est soumis à d’autres, qu’il faut chercher.\par
1˚ La nature des lois humaines est d’être soumises à tous les accidents qui arrivent, et de varier à mesure que les volontés des hommes changent : au contraire, la nature des lois de la religion est de ne varier jamais. Les lois humaines statuent sur le bien ; la religion sur le meilleur. Le bien peut avoir un autre objet, parce qu’il y a plusieurs biens ; mais le meilleur n’est qu’un, il ne peut donc pas changer. On peut bien changer les lois, parce qu’elles ne sont censées qu’être bonnes ; mais les institutions de la religion sont toujours supposées être les meilleures.\par
2˚ Il y a des États où les lois ne sont rien, ou ne sont qu’une volonté capricieuse et transitoire du souverain. Si, dans ces États, les lois de la religion étaient de la nature des lois humaines, les lois de la religion ne seraient rien non plus : il est pourtant nécessaire à la société qu’il y ait quelque chose de fixe ; et c’est cette religion qui est quelque chose de fixe.\par
3˚ La force principale de la religion vient de ce qu’on la croit ; la force des lois humaines vient de ce qu’on les craint. L’antiquité convient à la religion, parce que souvent nous croyons plus les choses à mesure qu’elles sont plus reculées : car nous n’avons pas dans la tête des idées accessoires tirées de ces temps-là, qui puissent les contredire. Les lois humaines, au contraire, tirent avantage de leur nouveauté, qui annonce une attention particulière et actuelle du législateur, pour les faire observer.
\subsubsection[{Chapitre III. Des lois civiles qui sont contraires à la loi naturelle}]{Chapitre III. Des lois civiles qui sont contraires à la loi naturelle}
\noindent « Si un esclave, dit Platon\footnote{Liv. IX, {\itshape Des Lois}.}, se défend et tue un homme libre, il doit être traité comme un parricide. » Voilà une loi civile qui punit la défense naturelle.\par
La loi qui, sous Henri VIII, condamnait un homme sans que les témoins lui eussent été confrontés, était contraire à la défense naturelle : en effet, pour qu’on puisse condamner, il faut bien que les témoins sachent que l’homme contre qui ils déposent est celui que l’on accuse, et que celui-ci puisse dire : Ce n’est pas moi dont vous parlez.\par
La loi passée sous le même règne, qui condamnait toute fille qui, ayant eu un mauvais commerce avec quelqu’un, ne le déclarerait point au roi avant de l’épouser, violait la défense de la pudeur naturelle : il est aussi déraisonnable d’exiger d’une fille qu’elle fasse cette déclaration, que de demander d’un homme qu’il ne cherche pas à défendre sa vie.\par
La loi d’Henri II, qui condamne à mort une fille dont l’enfant a péri, en cas qu’elle n’ait point déclaré au magistrat sa grossesse, n’est pas moins contraire à la défense naturelle. Il suffisait de l’obliger d’en instruire une de ses plus proches parentes, qui veillât à la conservation de l’enfant.\par
Quel autre aveu pourrait-elle faire dans ce supplice de la pudeur naturelle ? L’éducation a augmenté en elle l’idée de la conservation de cette pudeur ; et à peine, dans ces moments, est-il resté en elle une idée de la perte de la vie.\par
On a beaucoup parlé d’une loi d’Angleterre\footnote{M. Bayle, dans sa {\itshape Critique de l’histoire du calvinisme}, parle de cette loi, p. 293.} qui permettait à une fille de sept ans de se choisir un mari. Cette loi était révoltante de deux manières : elle n’avait aucun égard au temps de la maturité que la nature a donné à l’esprit, ni au temps de la maturité qu’elle a donné au corps.\par
Un père pouvait, chez les Romains, obliger sa fille à répudier son mari, quoiqu’il eût lui-même consenti au mariage\footnote{Voyez la loi 5, au Code {\itshape De repudiis et judicio de moribus sublato}.}. Mais il est contre la nature que le divorce soit mis entre les mains d’un tiers.\par
Si le divorce est conforme à la nature, il ne l’est que lorsque les deux parties, ou au moins une d’elles, y Consentent ; et lorsque ni l’une ni l’autre n’y consentent, c’est un monstre que le divorce. Enfin, la faculté du divorce ne peut être donnée qu’à ceux qui ont les incommodités du mariage, et qui sentent le moment où ils ont intérêt de les faire cesser.
\subsubsection[{Chapitre IV. Continuation du même sujet}]{Chapitre IV. Continuation du même sujet}
\noindent Gondebaud\footnote{Loi des Bourguignons, tit. XLVIII.}, roi de Bourgogne, voulait que, si la femme ou le fils de celui qui avait volé, ne révélait pas le crime, ils fussent réduits en esclavage. Cette loi était contre la nature. Comment une femme pouvait-elle être accusatrice de son mari ? Comment un fils pouvait-il être accusateur de son père ? Pour venger une action criminelle, il en ordonnait une plus criminelle encore.\par
La loi de Recessuinde permettait aux enfants de la femme adultère, ou à ceux de son mari, de l’accuser et de mettre à la question les esclaves de la maison\footnote{Dans le code des Wisigoths, liv. III, tit. IV, § 13.}. Loi inique, qui, pour conserver les mœurs, renversait la nature, d’où tirent leur origine les mœurs.\par
Nous voyons avec plaisir sur nos théâtres un jeune héros montrer autant d’horreur pour découvrir le crime de sa belle-mère, qu’il en avait eu pour le crime même ; il ose à peine, dans sa surprise, accusé, jugé, condamné, proscrit et couvert d’infamie, faire quelques réflexions sur le sang abominable dont Phèdre est sortie. il abandonne ce qu’il a de plus cher, et l’objet le plus tendre, tout ce qui parle à son cœur, tout ce qui peut l’indigner, pour aller se livrer à la vengeance des dieux qu’il n’a point méritée. Ce sont les accents de la nature qui causent ce plaisir ; c’est la plus douce de toutes les voix.
\subsubsection[{Chapitre V. Cas où l’on peut juger par les principes du droit civil, en modifiant les principes du droit naturel}]{Chapitre V. Cas où l’on peut juger par les principes du droit civil, en modifiant les principes du droit naturel}
\noindent Une loi d’Athènes obligeait\footnote{Sous peine d’infamie ; une autre, sous peine de prison.} les enfants de nourrir leurs pères tombés dans l’indigence ; elle exceptait ceux qui étaient nés d’une courtisane, ceux dont le père avait exposé la pudicité par un trafic infâme\footnote{Plutarque, {\itshape Vie de Solon}.}, ceux à qui il n’avait point donné de métier pour gagner leur vie\footnote{Plutarque, {\itshape Vie de Solon} ; et Galien, in {\itshape Exhort. ad Art. cha}p. VIII.}.\par
La loi considérait que, dans le premier cas, le père se trouvant incertain, il avait rendu précaire son obligation naturelle ; que, dans le second, il avait flétri la vie qu’il avait donnée, et que le plus grand mal qu’il pût faire à ses enfants, il l’avait fait, en les privant de leur caractère ; que, dans le troisième, il leur avait rendu insupportable une vie qu’ils trouvaient tant de difficulté à soutenir. La loi n’envisageait plus le père et le fils que comme deux citoyens, ne statuait plus que sur des vues politiques et civiles ; elle considérait que, dans une bonne république, il faut surtout des mœurs.\par
Je crois bien que la loi de Solon était bonne dans les deux premiers cas, soit celui où la nature laisse ignorer au fils quel est son père, soit celui où elle semble même lui ordonner de le méconnaître ; mais on ne saurait l’approuver dans le troisième, où le père n’avait violé qu’un règlement civil.
\subsubsection[{Chapitre VI. Que l’ordre des successions dépend des principes du droit politique ou civil, et non pas des principes du droit naturel}]{Chapitre VI. Que l’ordre des successions dépend des principes du droit politique ou civil, et non pas des principes du droit naturel}
\noindent La loi Voconienne ne permettait point d’instituer une femme héritière, pas même sa fille unique. Il n’y eut jamais, dit saint Augustin\footnote{{\itshape De civitate Dei}, liv. III.}, une loi plus injuste. Une formule de Marculfe\footnote{Liv. II, chap. XII.} traite d’impie la coutume qui prive les filles de la succession de leurs pères. Justinien\footnote{{\itshape Novelle} 21.} appelle barbare le droit de succéder des mâles, au préjudice des filles. Ces idées sont venues de ce que l’on a regardé le droit que les enfants ont de succéder à leurs pères comme une conséquence de la loi naturelle ; ce qui n’est pas.\par
La loi naturelle ordonne aux pères de nourrir leurs enfants, mais elle n’oblige pas de les faire héritiers. Le partage des biens, des lois sur ce partage, les successions après la mort de celui qui a eu ce partage : tout cela ne peut avoir été réglé que par la société, et par conséquent par des lois politiques ou civiles.\par
Il est vrai que l’ordre politique ou civil demande souvent que les enfants succèdent aux pères ; mais il ne l’exige pas toujours.\par
Les lois de nos fiefs ont pu avoir des raisons pour que l’aîné des mâles, ou les plus proches parents par mâles, eussent tout, et que les filles n’eussent rien ; et les lois des Lombards\footnote{Liv. II, tit. XIV, §§ 6, 7 et 8.} ont pu en avoir pour que les sœurs, les enfants naturels, les autres parents et, à leur défaut, le fisc, concourussent avec les filles.\par
il fut réglé, dans quelques dynasties de la Chine, que les frères de l’empereur lui succéderaient, et que ses enfants ne lui succéderaient pas. Si l’on voulait que le prince eût une certaine expérience, si l’on craignait les minorités, s’il fallait prévenir que des eunuques ne plaçassent successivement des enfants sur le trône, on put très bien établir un pareil ordre de succession ; et quand quelques\footnote{Le P. Du Halde, sur la seconde dynastie.} écrivains ont traité ces frères d’usurpateurs, ils ont jugé sur des idées prises des lois de ces pays-ci.\par
Selon la coutume de Numidie\footnote{Tite-Live, Décade II.} Delsace, frère de Géla, succéda au royaume, non pas Massinisse, son fils. Et encore aujourd’hui\footnote{Voyez les Voyages de M. Schaw, t. I, p. 402.}, chez les Arabes de Barbarie, où chaque village a un chef, on choisit, selon cette ancienne coutume, l’oncle, ou quelque autre parent, pour succéder.\par
Il y a des monarchies purement électives ; et, dès qu’il est clair que l’ordre des successions doit dériver des lois politiques ou civiles, c’est à elles à décider dans quels cas la raison veut que cette succession soit déférée aux enfants, et dans quels cas il faut la donner à d’autres.\par
Dans les pays où la polygamie est établie, le prince a beaucoup d’enfants ; le nombre en est plus grand dans des pays que dans d’autres. Il y a des\footnote{Comme à Lovengo en Afrique. Voyez le {\itshape Recueil des voyages qui ont servi à l’établissement de la compagnie des Indes}, t. IV, part. 1, p. 114, et M. Smith, {\itshape Voyage de Guinée}, part. II, p. 150, sur le royaume de Juida.} États où l’entretien des enfants du roi serait impossible au peuple ; on a pu y établir que les enfants du roi ne lui succéderaient pas, mais ceux de sa sœur.\par
Un nombre prodigieux d’enfants exposerait l’État à d’affreuses guerres civiles. L’ordre de succession qui donne la couronne aux enfants de la sœur, dont le nombre n’est pas plus grand que ne serait celui des enfants d’un prince qui n’aurait qu’une seule femme, prévient ces inconvénients.\par
Il y a des nations chez lesquelles des raisons d’État ou quelque maxime de religion ont demandé qu’une certaine famille fût toujours régnante : telle est aux Indes\footnote{Voyez les {\itshape Lettres édifiantes}, quatorzième recueil ; et les {\itshape Voyages qui ont servi à l’établissement de la compagnie des Indes}, t. III, part. II, p. 644.} la jalousie de sa caste, et la crainte de n’en point descendre. On y a pensé que, pour avoir toujours des princes du sang royal, il fallait prendre les enfants de la sœur aînée du roi.\par
Maxime générale : nourrir ses enfants est une obligation du droit naturel ; leur donner sa succession est une obligation du droit civil ou politique. De là dérivent les différentes dispositions sur les bâtards dans les différents pays du monde : elles suivent les lois civiles ou politiques de chaque pays.
\subsubsection[{Chapitre VII. Qu’il ne faut point décider par les préceptes de la religion lorsqu’il s’agit de ceux de la loi naturelle}]{Chapitre VII. Qu’il ne faut point décider par les préceptes de la religion lorsqu’il s’agit de ceux de la loi naturelle}
\noindent Les Abyssins ont un carême de cinquante jours très rude, et qui les affaiblit tellement que de longtemps ils ne peuvent agir : les Turcs ne manquent pas de les attaquer après leur carême\footnote{{\itshape Recueil des voyages qui ont servi à l’établissement de la Compagnie des Indes}, t. IV, part. I, p. 35 et 103.}. La religion devrait, en faveur de la défense naturelle, mettre des bornes à ces pratiques.\par
Le sabbat fut ordonné aux Juifs : mais ce fut une stupidité à cette nation de ne point se défendre\footnote{Comme ils firent, lorsque Pompée assiégea le temple, voyez Dion, liv. XXXVII.}, lorsque ses ennemis choisirent ce jour pour l’attaquer.\par
Cambyse assiégeant Péluze, mit au premier rang un grand nombre d’animaux que les Égyptiens tenaient pour sacrés : les soldats de la garnison n’osèrent tirer. Qui ne voit que la défense naturelle est d’un ordre supérieur à tous les préceptes ?
\subsubsection[{Chapitre VIII. Qu’il ne faut pas régler par les principes du droit qu’on appelle canonique les choses réglées par les principes du droit civil}]{Chapitre VIII. Qu’il ne faut pas régler par les principes du droit qu’on appelle canonique les choses réglées par les principes du droit civil}
\noindent Par le droit civil des Romains\footnote{Leg. 5, ff. {\itshape ad leg. Juliam peculatus}.}, celui qui enlève d’un lieu sacré une chose privée n’est puni que du crime de vol ; par le droit canonique\footnote{Cap. {\itshape quisquis} 17, quaestione 4 ; Cujas, {\itshape Observationes} liv. XIII, chap. XIX, t. III.}, il est puni du crime de sacrilège. Le droit canonique fait attention au lieu, le droit civil à la chose. Mais n’avoir attention qu’au lieu, c’est ne réfléchir ni sur la nature et la définition du vol ni sur la nature et la définition du sacrilège.\par
Comme le mari peut demander la séparation à cause de l’infidélité de sa femme, la femme la demandait autrefois à cause de l’infidélité du mari\footnote{Beaumanoir, Ancienne coutume de Beauvaisis, chap. XVIII.}. Cet usage, contraire à la disposition des lois romaines\footnote{Leg. I, Code {\itshape ad leg. Jul. de adulteriis}.}, s’était introduit dans les cours d’église\footnote{Aujourd’hui, en France, elles ne connaissent point de ces choses.}, où l’on ne voyait que les maximes du droit canonique ; et effectivement, à ne regarder le mariage que dans des idées purement spirituelles et dans le rapport aux choses de l’autre vie, la violation est la même. Mais les lois politiques et civiles de presque tous les peuples ont avec raison distingué ces deux choses. Elles ont demandé des femmes un degré de retenue et de continence qu’elles n’exigent point des hommes, parce que la violation de la pudeur suppose dans les femmes un renoncement à toutes les vertus ; parce que la femme, en violant les lois du mariage, sort de l’état de sa dépendance naturelle ; parce que la nature a marqué l’infidélité des femmes par des signes certains, outre que les enfants adultérins de la femme sont nécessairement au mari et à la charge du mari, au lieu que les enfants adultérins du mari ne sont pas à la femme, ni à la charge de la femme.
\subsubsection[{Chapitre IX. Que les choses qui doivent être réglées par les principes du droit civil peuvent rarement l’être par les principes des lois de la religion}]{Chapitre IX. Que les choses qui doivent être réglées par les principes du droit civil peuvent rarement l’être par les principes des lois de la religion}
\noindent Les lois religieuses ont plus de sublimité, les lois civiles ont plus d’étendue.\par
Les lois de perfection tirées de la religion ont plus pour objet la bonté de l’homme qui les observe, que celle de la société dans laquelle elles sont observées : les lois civiles, au contraire, ont plus pour objet la bonté morale des hommes en général, que celle des individus.\par
Ainsi, quelque respectables que soient les idées qui naissent immédiatement de la religion, elles ne doivent pas toujours servir de principe aux lois civiles, parce que celles-ci en ont un autre, qui est le bien général de la société.\par
Les Romains firent des règlements pour conserver, dans la république, les mœurs des femmes ; c’étaient des institutions politiques. Lorsque la monarchie s’établit, ils firent là-dessus des lois civiles ; et ils les firent sur les principes du gouvernement civil. Lorsque la religion chrétienne eut pris naissance, les lois nouvelles que l’on fit eurent moins de rapport à la bonté générale des mœurs qu’à la sainteté du mariage ; on considéra moins l’union des deux sexes dans l’état civil, que dans un état spirituel.\par
D’abord, par la loi romaine\footnote{Leg. 11, § {\itshape ult.} ff. {\itshape ad leg. Jul. de adulteriis}.}, un mari qui ramenait sa femme dans sa maison après la condamnation d’adultère, fut puni comme complice de ses débauches. Justinien\footnote{{\itshape Novelle} 134, chap. X.}, dans un autre esprit, ordonna qu’il pourrait, pendant deux ans, l’aller reprendre dans le monastère.\par
Lorsqu’une femme qui avait son mari à la guerre n’entendait plus parler de lui, elle pouvait, dans les premiers temps, aisément se remarier, parce qu’elle avait entre ses mains le pouvoir de faire divorce. La loi de Constantin\footnote{Leg. 7, Code {\itshape De repudiis et judicio de moribus sublato}.} voulut qu’elle attendît quatre ans, après quoi elle pouvait envoyer le libelle de divorce au chef ; et, si son mari revenait, il ne pouvait plus l’accuser d’adultère. Mais Justinien\footnote{{\itshape Authentique Hodie quantiscumque} ; Code {\itshape De repudiis}.} établit que, quelque temps qui se fût écoulé depuis le départ du mari, elle ne pouvait se remarier, à moins que, par la déposition et le serment du chef, elle ne prouvât la mort de son mari. Justinien avait en vue l’indissolubilité du mariage ; mais on peut dire qu’il l’avait trop en vue. Il demandait une preuve positive, lorsqu’une preuve négative suffisait ; il exigeait une chose très difficile, de rendre compte de la destinée d’un homme éloigné et exposé à tant d’accidents ; il présumait un crime, c’est-à-dire la désertion du mari, lorsqu’il était si naturel de présumer sa mort. Il choquait le bien public, en laissant une femme sans mariage ; il choquait l’intérêt particulier, en l’exposant à mille dangers.\par
La loi de Justinien\footnote{{\itshape Authentique Quod hodie} ; Code {\itshape De repudiis}.} qui mit parmi les causes de divorce le consentement du mari et de la femme d’entrer dans le monastère, s’éloignait entièrement des principes des lois civiles. Il est naturel que des causes de divorce tirent leur origine de certains empêchements qu’on ne devait pas prévoir avant le mariage ; mais ce désir de garder la chasteté pouvait être prévu, puisqu’il est en nous. Cette loi favorise l’inconstance dans un état qui, de sa nature, est perpétuel ; elle choque le principe fondamental du divorce, qui ne souffre la dissolution d’un mariage que dans l’espérance d’un autre ; enfin, à suivre même les idées religieuses, elle ne fait que donner des victimes à Dieu sans sacrifice.
\subsubsection[{Chapitre X. Dans quel cas il faut suivre la loi civile qui permet, et non pas la loi de la religion qui défend}]{Chapitre X. Dans quel cas il faut suivre la loi civile qui permet, et non pas la loi de la religion qui défend}
\noindent Lorsqu’une religion qui défend la polygamie s’introduit dans un pays où elle est permise, on ne croit pas, à ne parler que politiquement, que la loi du pays doive souffrir qu’un homme qui a plusieurs femmes embrasse cette religion, à moins que le magistrat ou le mari ne les dédommagent, en leur rendant, de quelque manière, leur état civil. Sans cela, leur condition serait déplorable ; elles n’auraient fait qu’obéir aux lois, et elles se trouveraient privées des plus grands avantages de la société.
\subsubsection[{Chapitre XI. Qu’il ne faut point régler les tribunaux humains par les maximes des tribunaux qui regardent l’autre vie}]{Chapitre XI. Qu’il ne faut point régler les tribunaux humains par les maximes des tribunaux qui regardent l’autre vie}
\noindent Le tribunal de l’inquisition, formé par les moines chrétiens sur l’idée du tribunal de la pénitence, est contraire à toute bonne police. Il a trouvé partout un soulèvement général ; et il aurait cédé aux contradictions, si ceux qui voulaient l’établir n’avaient tiré avantage de ces contradictions mêmes.\par
Ce tribunal est insupportable dans tous les gouvernements. Dans la monarchie, il ne peut faire que des délateurs et des traîtres ; dans les républiques, il ne peut former que des malhonnêtes gens ; dans l’État despotique, il est destructeur comme lui.
\subsubsection[{Chapitre XII. Continuation du même sujet}]{Chapitre XII. Continuation du même sujet}
\noindent C’est un des abus de ce tribunal que, de deux personnes qui sont accusées du même crime, celle qui nie est condamnée à la mort, et celle qui avoue évite le supplice. Ceci est tiré des idées monastiques, où celui qui nie paraît être dans l’impénitence et damné, et celui qui avoue semble être dans le repentir et sauvé. Mais une pareille distinction ne peut concerner les tribunaux humains : la justice humaine, qui ne voit que les actions, n’a qu’un pacte avec les hommes, qui est celui de l’innocence ; la justice divine, qui voit les pensées, en a deux, celui de l’innocence et celui du repentir.
\subsubsection[{Chapitre XIII. Dans quel cas il faut suivre, l’égard des mariages, les lois de la religion, et dans quel cas il faut suivre les lois civiles}]{Chapitre XIII. Dans quel cas il faut suivre, l’égard des mariages, les lois de la religion, et dans quel cas il faut suivre les lois civiles}
\noindent Il est arrivé, dans tous les pays et dans tous les temps, que la religion s’est mêlée des mariages. Dès que de certaines choses ont été regardées comme impures ou illicites, et que cependant elles étaient nécessaires, il a bien fallu y appeler la religion, pour les légitimer dans un cas, et les réprouver dans les autres.\par
D’un autre côté, les mariages étant, de toutes les actions humaines, celle qui intéresse le plus la société, il a bien fallu qu’ils fussent réglés par les lois civiles.\par
Tout ce qui regarde le caractère du mariage, sa forme, la manière de le contracter, la fécondité qu’il procure, qui a fait comprendre à tous les peuples qu’il était l’objet d’une bénédiction particulière qui, n’y étant pas toujours attachée, dépendait de certaines grâces supérieures : tout cela est du ressort de la religion.\par
Les conséquences de cette union par rapport aux biens, les avantages réciproques, tout ce qui a du rapport à la famille nouvelle, à celle dont elle est sortie, à celle qui doit naître : tout cela regarde les lois civiles.\par
Comme un des grands objets du mariage est d’ôter toutes les incertitudes des conjonctions illégitimes, la religion y imprime son caractère, et les lois civiles y joignent le leur, afin qu’il ait toute l’authenticité possible. Ainsi, outre les conditions que demande la religion pour que le mariage soit valide, les lois civiles en peuvent encore exiger d’autres.\par
Ce qui fait que les lois civiles ont ce pouvoir, c’est que ce sont des caractères ajoutés, et non pas des caractères contradictoires. La loi de la religion veut de certaines cérémonies, et les lois civiles veulent le consentement des pères ; elles demandent en cela quelque chose de plus, mais elles ne demandent rien qui soit contraire.\par
Il suit de là que c’est à la loi de la religion à décider si le lien sera indissoluble ou non : car si les lois de la religion avaient établi le lien indissoluble, et que les lois civiles eussent réglé qu’il se peut rompre, ce seraient deux choses contradictoires.\par
Quelquefois les caractères imprimés au mariage par les lois civiles ne sont pas d’une absolue nécessité ; tels sont ceux qui sont établis par les lois qui, au lieu de casser le mariage, se sont contentées de punir ceux qui le contractaient.\par
Chez les Romains, les lois Papiennes déclarèrent injustes les mariages qu’elles prohibaient, et les soumirent seulement à des peines\footnote{Voyez ce que j’ai dit ci-dessus, au chap. XXI du liv. {\itshape Des lois}, dans le rapport qu’elles ont avec le nombre des habitants.} {\itshape ;} et le sénatus-consulte rendu sur le discours de l’empereur Marc-Antonin les déclara nuls ; il n’y eut plus de mariage, de femme, de dot, de mari\footnote{Voyez la loi 16, ff. {\itshape De ritu nuptiarum}, et la loi 3, § I, aussi au Digeste, {\itshape De donationibus inter virum et uxorem}.}. La loi civile se détermine selon les circonstances : quelquefois elle est plus attentive à réparer le mal, quelquefois à le prévenir.
\subsubsection[{Chapitre XIV. Dans quels cas, dans les mariages entre parents, il faut se régler par les lois de la nature ; dans quels cas on doit se régler par les lois civiles}]{Chapitre XIV. Dans quels cas, dans les mariages entre parents, il faut se régler par les lois de la nature ; dans quels cas on doit se régler par les lois civiles}
\noindent En fait de prohibition de mariage entre parents, c’est une chose très délicate de bien poser le point auquel les lois de la nature s’arrêtent, et où les lois civiles commencent. Pour cela il faut établir des principes.\par
Le mariage du fils avec la mère confond l’état des choses : le fils doit un respect sans bornes à sa mère, la femme doit un respect sans bornes à son mari ; le mariage d’une mère avec son fils renverserait dans l’un et dans l’autre leur état naturel.\par
Il y a plus : la nature a avancé dans les femmes le temps où elles peuvent avoir des enfants ; elle l’a reculé dans les hommes ; et, par la même raison, la femme cesse plus tôt d’avoir cette faculté, et l’homme plus tard. Si le mariage entre la mère et le fils était permis, il arriverait presque toujours que, lorsque le mari serait capable d’entrer dans les vues de la nature, la femme n’y serait plus.\par
Le mariage entre le père et la fille répugne à la nature comme le précédent ; mais il répugne moins, parce qu’il n’a point ces deux obstacles. Aussi les Tartares, qui peuvent épouser leurs filles\footnote{Cette loi est bien ancienne parmi eux. Attila, dit Priscus dans son {\itshape Ambassade}, s’arrêta dans un certain lieu pour épouser Esca, sa fille : « chose permise, dit-il, par les lois des Scythes » p. 22.}, n’épousent-ils jamais leurs mères, comme nous le voyons dans les {\itshape Relations}\footnote{{\itshape Histoire des Tartares}, part. III, p. 256.}.\par
Il a toujours été naturel aux pères de veiller sur la pudeur de leurs enfants. Chargés du soin de les établir, ils ont dû leur conserver et le corps le plus parfait, et l’âme la moins corrompue ; tout ce qui peut mieux inspirer des désirs, et tout ce qui est le plus propre à donner de la tendresse. Des pères toujours occupés à conserver les mœurs de leurs enfants, ont dû avoir un éloignement naturel pour tout ce qui pourrait les corrompre. Le mariage n’est point une corruption, dira-t-on ; mais avant le mariage, il faut parler, il faut se faire aimer, il faut séduire ; c’est cette séduction qui a dû faire horreur.\par
Il a donc fallu une barrière insurmontable entre ceux qui devaient donner l’éducation et ceux qui devaient la recevoir, et éviter toute sorte de corruption, même pour cause légitime. Pourquoi les pères privent-ils si soigneusement ceux qui doivent épouser leurs filles, de leur compagnie et de leur familiarité ?\par
L’horreur pour l’inceste du frère avec la sœur a dû partir de la même source. Il suffit que les pères et les mères aient voulu conserver les mœurs de leurs enfants et leurs maisons pures, pour avoir inspiré à leurs enfants de l’horreur pour tout ce qui pouvait les porter à l’union des deux sexes.\par
La prohibition du mariage entre cousins germains a la même origine. Dans les premiers temps, c’est-à-dire dans les temps saints, dans les âges où le luxe n’était point connu, tous les enfants restaient dans la maison\footnote{Cela fut ainsi chez les premiers Romains.}, et s’y établissaient : c’est qu’il ne fallait qu’une maison très petite pour une grande famille. Les enfants des deux frères, ou les cousins germains, étaient regardés et se regardaient entre eux comme frères\footnote{En effet, chez les Romains ils avaient le même nom ; les cousins germains étaient nommés frères.}. L’éloignement qui était entre les frères et les sœurs pour le mariage, était donc aussi entre les cousins germains\footnote{Ils le furent à Rome dans les premiers temps, jusqu’à ce que le peuple fit une loi pour les permettre : il voulait favoriser un homme extrêmement populaire, et qui s’était marié avec sa cousine germaine. Plutarque, au traité {\itshape Des demandes des choses romaines}.}.\par
Ces causes sont si fortes et si naturelles, qu’elles ont agi presque par toute la terre, indépendamment d’aucune communication. Ce ne sont point les Romains qui ont appris aux habitants de Formose\footnote{{\itshape Recueil des voyages des Indes}, t. V, part. I, {\itshape Relation de l’état de l’île de Formose}.} que le mariage avec leurs parents au quatrième degré était incestueux ; ce ne sont point les Romains qui l’ont dit aux Arabes\footnote{{\itshape L’Alcoran}, chap.{\itshape  Des femmes}.} ; ils ne l’ont point enseigné aux Maldives\footnote{Voyez François Pyrard.}.\par
Que si quelques peuples n’ont point rejeté les mariages entre les pères et les enfants, les sœurs et les frères, on a vu dans le livre premier, que les êtres intelligents ne suivent pas toujours leurs lois. Qui le dirait ! des idées religieuses ont souvent fait tomber les hommes dans ces égarements. Si les Assyriens, si les Perses ont épouse leurs mères, les premiers l’ont fait par un respect religieux pour Sémiramis ; et les seconds, parce que la religion de Zoroastre donnait la préférence à ces mariages\footnote{Ils étaient regardés comme plus honorables. Voyez Philon, {\itshape De specialibus legibus quae pertinent ad praecepta decalogi.} Paris, 1640, p. 778.}. Si les Égyptiens ont épousé leurs sœurs, ce fut encore un délire de la religion égyptienne, qui consacra ces mariages en l’honneur d’Isis. Comme l’esprit de la religion est de nous porter à faire avec effort des choses grandes et difficiles, il ne faut pas juger qu’une chose soit naturelle, parce qu’une religion fausse l’a consacrée.\par
Le principe que les mariages entre les pères et les enfants, les frères et les sœurs, sont défendus pour la conservation de la pudeur naturelle dans la maison, servira à nous faire découvrir quels sont les mariages défendus par la loi naturelle, et ceux qui ne peuvent l’être que par la loi civile.\par
Comme les enfants habitent, ou sont censés habiter dans la maison de leur père, et par conséquent le beau-fils avec la belle-mère, le beau-père avec la belle-fille ou avec la fille de sa femme, le mariage entre eux est défendu par la loi de la nature. Dans ce cas l’image a le même effet que la réalité, parce qu’elle a la même cause ; la loi civile ne peut ni ne doit permettre ces mariages.\par
Il y a des peuples chez lesquels, comme j’ai dit, les cousins germains sont regardés comme frères, parce qu’ils habitent ordinairement dans la même maison ; il y en a où on ne connaît guère cet usage. Chez ces peuples, le mariage entre cousins germains doit être regardé comme contraire à la nature ; chez les autres, non.\par
Mais les lois de la nature ne peuvent être des lois locales. Ainsi, quand ces mariages sont défendus ou permis, ils sont, selon les circonstances, permis ou défendus par une loi civile.\par
Il n’est point d’un usage nécessaire que le beau-frère et la belle-sœur habitent dans la même maison. Le mariage n’est donc pas défendu entre eux pour conserver la pudicité dans la maison ; et la loi qui le défend ou le permet, n’est point la loi de la nature, mais une loi civile, qui se règle sur les circonstances, et dépend des usages de chaque pays : ce sont des cas où les lois dépendent des mœurs et des manières.\par
Les lois civiles défendent les mariages lorsque, par les usages reçus dans un certain pays, ils se trouvent être dans les mêmes circonstances que ceux qui sont défendus par les lois de la nature ; et elles les permettent lorsque les mariages ne se trouvent point dans ce cas. La défense des lois de la nature est invariable, parce qu’elle dépend d’une chose invariable, le père, la mère et les enfants habitant nécessairement dans la maison. Mais les défenses des lois civiles sont accidentelles, parce qu’elles dépendent d’une circonstance accidentelle, les cousins germains et autres habitant accidentellement dans la maison.\par
Cela explique comment les lois de Moïse, celles des Égyptiens\footnote{Voyez la loi 8, au Code {\itshape De incestis et inutilibus nuptiis}.} et de plusieurs autres peuples, permettent le mariage entre le beau-frère et la belle-sœur, pendant que ces mêmes mariages sont défendus chez d’autres nations.\par
Aux Indes, on a une raison bien naturelle d’admettre ces sortes de mariages. L’oncle y est regardé comme père, et il est obligé d’entretenir et d’établir ses neveux, comme si c’étaient ses propres enfants : ceci vient du caractère de ce peuple, qui est bon et plein d’humanité. Cette loi ou cet usage en a produit un autre : si un mari a perdu sa femme, il ne manque pas d’en épouser la sœur\footnote{{\itshape Lettres édifiantes}, quatorzième recueil, p. 403.}, et cela est très naturel ; car la nouvelle épouse devient la mère des enfants de sa sœur, et il n’y a point d’injuste marâtre.
\subsubsection[{Chapitre XV. Qu’il ne faut point régler par les principes du droit politique les choses qui dépendent des principes du droit civil}]{Chapitre XV. Qu’il ne faut point régler par les principes du droit politique les choses qui dépendent des principes du droit civil}
\noindent Comme les hommes ont renoncé à leur indépendance naturelle pour vivre sous des lois politiques, ils ont renoncé à la communauté naturelle des biens pour vivre sous des lois civiles.\par
Ces premières lois leur acquièrent la liberté les secondes, la propriété. Il ne faut pas décider par les lois de la liberté, qui, comme nous avons dit, n’est que l’empire de la cité, ce qui ne doit être décidé que par les lois qui concernent la propriété. C’est un paralogisme de dire que le bien particulier doit céder au bien public : cela n’a lieu que dans les cas où il s’agit de l’empire de la cité, c’est-à-dire de la liberté du citoyen ; cela n’a pas lieu dans ceux où il est question de la propriété des biens, parce que le bien public est toujours que chacun conserve invariablement la propriété que lui donnent les lois civiles.\par
Cicéron soutenait que les lois agraires étaient funestes, parce que la cité n’était établie que pour que chacun conservât ses biens.\par
Posons donc pour maxime que, lorsqu’il s’agit du bien public, le bien public n’est jamais que l’on prive un particulier de son bien, ou même qu’on lui en retranche la moindre partie par une loi ou un règlement politique. Dans ce cas, il faut suivre à la rigueur la loi civile, qui est le palladium de la propriété.\par
Ainsi, lorsque le public a besoin du fonds d’un particulier, il ne faut jamais agir par la rigueur de la loi politique ; mais c’est là que doit triompher la loi civile, qui, avec des yeux de mère, regarde chaque particulier comme toute la cité même.\par
Si le magistrat politique veut faire quelque édifice public, quelque nouveau chemin, il faut qu’il indemnise ; le public est, à cet égard, comme un particulier qui traite avec un particulier. C’est bien assez qu’il puisse contraindre un citoyen de lui vendre son héritage, et qu’il lui ôte ce grand privilège qu’il tient de la loi civile, de ne pouvoir être forcé d’aliéner son bien.\par
Après que les peuples qui détruisirent les Romains eurent abusé de leurs conquêtes mêmes, l’esprit de liberté les rappela à celui d’équité ; les droits les plus barbares, ils les exercèrent avec modération ; et, si l’on en doutait, il n’y aurait qu’à lire l’admirable ouvrage de Beaumanoir, qui écrivait sur la jurisprudence dans le douzième siècle.\par
On raccommodait de son temps les grands chemins, comme on fait aujourd’hui. Il dit que, quand un grand chemin ne pouvait être rétabli, on en faisait un autre le plus près de l’ancien qu’il était possible ; mais qu’on dédommageait les propriétaires aux frais de ceux qui tiraient quelque avantage du chemin\footnote{Le seigneur nommait des prud’hommes pour faire la levée sur le paysan ; les gentilshommes étaient contraints à la contribution par le comte, l’homme d’église par l’évêque. Beaumanoir, chap. XXII.}. On se déterminait pour lors par la loi civile ; on s’est déterminé de nos jours par la loi politique.
\subsubsection[{Chapitre XVI. Qu’il ne faut point décider par les règles du droit civil quand il s’agit de décider par celles du droit politique}]{Chapitre XVI. Qu’il ne faut point décider par les règles du droit civil quand il s’agit de décider par celles du droit politique}
\noindent On verra le fond de toutes les questions, si l’on ne confond point les règles qui dérivent de la propriété de la cité, avec celles qui naissent de la liberté de la cité.\par
Le domaine d’un État est-il aliénable, ou ne l’est-il pas ? Cette question doit être décidée par la loi politique, et non pas par la loi civile. Elle ne doit pas être décidée par la loi civile, parce qu’il est aussi nécessaire qu’il y ait un domaine pour faire subsister l’État, qu’il est nécessaire qu’il y ait dans l’État des lois civiles qui règlent la disposition des biens.\par
Si donc on aliène le domaine, l’État sera forcé de faire un nouveau fonds pour un autre domaine. Mais cet expédient renverse encore le gouvernement politique, parce que, par la nature de la chose, à chaque domaine qu’on établira, le sujet paiera toujours plus, et le souverain retirera toujours moins ; en un mot, le domaine est nécessaire, et l’aliénation ne l’est pas.\par
L’ordre de succession est fondé, dans les monarchies, sur le bien de l’État, qui demande que cet ordre soit fixé, pour éviter les malheurs que j’ai dit devoir arriver dans le despotisme, où tout est incertain, parce que tout y est arbitraire.\par
Ce n’est pas pour la famille régnante que l’ordre de succession est établi, mais parce qu’il est de l’intérêt de l’État qu’il y ait une famille régnante. La loi qui règle la succession des particuliers est une loi civile, qui a pour objet l’intérêt des particuliers ; celle qui règle la succession à la monarchie est une loi politique, qui a pour objet le bien et la conservation de l’État.\par
Il suit de là que, lorsque la loi politique a établi dans un État un ordre de succession, et que cet ordre vient à finir, il est absurde de réclamer la succession en vertu de la loi civile de quelque peuple que ce soit. Une société particulière ne fait point de lois pour une autre société. Les lois civiles des Romains ne sont pas plus applicables que toutes autres lois civiles ; ils ne les ont point employées eux-mêmes, lorsqu’ils ont jugé les rois : et les maximes par lesquelles ils ont jugé les rois, sont si abominables, qu’il ne faut point les faire revivre.\par
Il suit encore de là que, lorsque la loi politique a fait renoncer quelque famille à la succession, il est absurde de vouloir employer les restitutions tirées de la loi civile. Les restitutions sont dans la loi, et peuvent être bonnes contre ceux qui vivent dans la loi ; mais elles ne sont pas bonnes pour ceux qui ont été établis pour la loi, et qui vivent pour la loi.\par
Il est ridicule de prétendre décider des droits des royaumes, des nations et de l’univers, par les mêmes maximes sur lesquelles on décide entre particuliers d’un droit pour une gouttière, pour me servir de l’expression de Cicéron\footnote{Liv. I des {\itshape Lois}.}.
\subsubsection[{Chapitre XVII. Continuation du même sujet}]{Chapitre XVII. Continuation du même sujet}
\noindent L’ostracisme doit être examiné par les règles de la loi politique, et non par les règles de la loi civile ; et, bien loin que cet usage puisse flétrir le gouvernement populaire, il est au contraire très propre à en prouver la douceur ; et nous aurions senti cela, si l’exil parmi nous étant toujours une peine, nous avions pu séparer l’idée de l’ostracisme d’avec celle de la punition.\par
Aristote nous dit\footnote{{\itshape Politique}, liv. III, chap. XIII.} qu’il est convenu de tout le monde que cette pratique a quelque chose d’humain et de populaire. Si, dans les temps et dans les lieux où l’on exerçait ce jugement, on ne le trouvait point odieux, est-ce à nous qui voyons les choses de si loin, de penser autrement que les accusateurs, les juges, et l’accusé même ?\par
Et si l’on fait attention que ce jugement du peuple comblait de gloire celui contre qui il était rendu ; que lorsqu’on en eut abusé à Athènes contre un homme sans mérite\footnote{Hyperbolus. Voyez Plutarque, {\itshape Vie d’Aristide}.}, on cessa dans ce moment de l’employer\footnote{Il se trouva opposé à l’esprit du législateur.}, on verra bien qu’on en a pris une fausse idée, et que c’était une loi admirable que celle qui prévenait les mauvais effets que pouvait produire la gloire d’un citoyen, en le comblant d’une nouvelle gloire.
\subsubsection[{Chapitre XVIII. Qu’il faut examiner si les lois qui paraissent se contredire sont du même ordre}]{Chapitre XVIII. Qu’il faut examiner si les lois qui paraissent se contredire sont du même ordre}
\noindent Rome il fut permis au mari de prêter sa femme à un autre. Plutarque nous le dit formellement\footnote{Plutarque, dans sa {\itshape Comparaison de Lycurgue et de Numa}.}. On sait que Caton prêta sa femme à Hortensius\footnote{Plutarque, {\itshape Vie de Caton}. Cela se passa de notre temps, dit Strabon, liv. XI,.}, et Caton n’était point homme à violer les lois de son pays.\par
D’un autre côté, un mari qui souffrait les débauches de sa femme, qui ne la mettait pas en jugement, ou qui la reprenait après la condamnation, était puni\footnote{Leg. \textsc{xi}, § {\itshape ult.} ff. {\itshape ad leg. Jul. de adulteriis}.}. Ces lois paraissent se contredire, et ne se contredisent point. La loi qui permettait à un Romain de prêter sa femme est visiblement une institution lacédémonienne, établie pour donner à la république des enfants d’une bonne espèce, si j’ose me servir de ce terme ; l’autre avait pour objet de conserver les mœurs. La première était une loi politique, la seconde une loi civile.
\subsubsection[{Chapitre XIX. Qu’il ne faut pas décider par les lois civiles les choses qui doivent l’être par les lois domestiques}]{Chapitre XIX. Qu’il ne faut pas décider par les lois civiles les choses qui doivent l’être par les lois domestiques}
\noindent La loi des Wisigoths voulait que les esclaves fussent obligés de lier l’homme et la femme qu’ils surprenaient en adultère\footnote{Loi des Wisigoths, liv. III, tit. IV, § 6.}, et de les présenter au mari et au juge : loi terrible, qui mettait entre les mains de ces personnes viles le soin de la vengeance publique, domestique et particulière !\par
Cette loi ne serait bonne que dans les sérails d’Orient, où l’esclave qui est chargé de la clôture a prévariqué sitôt qu’on prévarique. Il arrête les criminels, moins pour les faire juger que pour se faire juger lui-même, et obtenir que l’on cherche dans les circonstances de l’action si l’on peut perdre le soupçon de sa négligence.\par
Mais dans les pays où les femmes ne sont point gardées, il est insensé que la loi civile les soumette, elles qui gouvernent la maison, à l’inquisition de leurs esclaves.\par
Cette inquisition pourrait être, tout au plus dans de certains cas, une loi particulière domestique, et jamais une loi civile.\par
 \textbf{Chapitre XX. {\itshape Qu’il ne faut pas décider par les principes des lois civiles les choses qui appartiennent au droit des gens} }  \par
La liberté consiste principalement à ne pouvoir être forcé à faire une chose que la loi n’ordonne pas ; et on n’est dans cet état que parce qu’on est gouverné par des lois civiles : nous sommes donc libres, parce que nous vivons sous des lois civiles.\par
Il suit de là que les princes, qui ne vivent point entre eux sous des lois civiles, ne sont point libres ; ils sont gouvernés par la force ; ils peuvent continuellement forcer ou être forcés. De là il suit que les traités qu’ils ont faits par force, sont aussi obligatoires que ceux qu’ils auraient faits de bon gré. Quand nous, qui vivons sous des lois civiles, sommes contraints à faire quelque contrat que la loi n’exige pas, nous pouvons, à la faveur de la loi, revenir contre la violence ; mais un prince, qui est toujours dans cet état dans lequel il force ou il est forcé, ne peut pas se plaindre d’un traité qu’on lui a fait faire par violence. C’est comme s’il se plaignait de son état naturel ; c’est comme s’il voulait être prince à l’égard des autres princes, et que les autres princes fussent citoyens à son égard, c’est-à-dire choquer la nature des choses.
\subsubsection[{Chapitre XXI. Qu’il ne faut pas décider par les lois politiques les choses qui appartiennent au droit des gens}]{Chapitre XXI. Qu’il ne faut pas décider par les lois politiques les choses qui appartiennent au droit des gens}
\noindent Les lois politiques demandent que tout homme soit soumis aux tribunaux criminels et civils du pays où il est, et à l’animadversion du souverain.\par
Le droit des gens a voulu que les princes s’envoyassent des ambassadeurs ; et la raison, tirée de la nature de la chose, n’a pas permis que ces ambassadeurs dépendissent du souverain chez qui ils sont envoyés, ni de ses tribunaux. Ils sont la parole du prince qui les envoie, et cette parole doit être libre. Aucun obstacle ne doit les empêcher d’agir. ils peuvent souvent déplaire, parce qu’ils parlent pour un homme indépendant. On pourrait leur imputer des crimes, s’ils pouvaient être punis pour des crimes ; on pourrait leur supposer des dettes, s’ils pouvaient être arrêtés pour des dettes. Un prince qui a une fierté naturelle, parlerait par la bouche d’un homme qui aurait tout à craindre. Il faut donc suivre, à l’égard des ambassadeurs, les raisons tirées du droit des gens, et non pas celles qui dérivent du droit politique. Que s’ils abusent de leur être représentatif, on le fait cesser en les renvoyant chez eux : on peut même les accuser devant leur maître, qui devient par là leur juge ou leur complice.
\subsubsection[{Chapitre XXII. Malheureux sort de l’inca Athualpa}]{Chapitre XXII. Malheureux sort de l’inca Athualpa}
\noindent Les principes que nous venons d’établir furent cruellement violés par les Espagnols. L’inca Athualpa ne pouvait être jugé que par le droit des gens\footnote{Voyez l’inca Garcilasso de la Vega, p. 108.} : ils le jugèrent par des lois politiques et civiles. Ils l’accusèrent d’avoir fait mourir quelques-uns de ses sujets, d’avoir eu plusieurs femmes, etc. Et le comble de la stupidité fut qu’ils ne le condamnèrent pas par les lois politiques et civiles de son pays, mais par les lois politiques et civiles du leur.
\subsubsection[{Chapitre XXIII. Que lorsque, par quelque circonstance, la loi politique détruit l’État, il faut décider par la loi politique qui le conserve, qui devient quelquefois un droit des gens}]{Chapitre XXIII. Que lorsque, par quelque circonstance, la loi politique détruit l’État, il faut décider par la loi politique qui le conserve, qui devient quelquefois un droit des gens}
\noindent Quand la loi politique, qui a établi dans l’État un certain ordre de succession, devient destructrice du corps politique pour lequel elle a été faite, il ne faut pas douter qu’une autre loi politique ne puisse changer cet ordre ; et, bien loin que cette même loi soit opposée à la première, elle y sera dans le fond entièrement conforme, puisqu’elles dépendront toutes deux de ce principe : LE SALUT DU PEUPLE EST LA SUPRÊME LOI.\par
J’ai dit qu’un grand État\footnote{Voyez ci-dessus, liv. V, chap. XIV ; liv. VIII, chap. XVI-XX ; liv. IX, chap. IV-VII, et liv. X, chap. IX et X.} devenu accessoire d’un autre, s’affaiblissait, et même affaiblissait le principal. On sait que l’État a intérêt d’avoir son chef chez lui, que les revenus publics soient bien administrés, que sa monnaie ne sorte point pour enrichir un autre pays. Il est important que celui qui doit gouverner ne soit point imbu de maximes étrangères ; elles conviennent moins que celles qui sont déjà établies : d’ailleurs les hommes tiennent prodigieusement à leurs lois et à leurs coutumes ; elles font la félicité de chaque nation ; il est rare qu’on les change sans de grandes secousses et une grande effusion de sang, comme les histoires de tous les pays le font voir.\par
Il suit de là que, si un grand État a pour héritier le possesseur d’un grand État, le premier peut fort bien l’exclure, parce qu’il est utile à tous les deux États que l’ordre de la succession soit changé. Ainsi la loi de Russie, faite au commencement du règne d’Élisabeth, exclut-elle très prudemment tout héritier qui posséderait une autre monarchie ; ainsi la loi de Portugal rejette-t-elle tout étranger qui serait appelé à la couronne par le droit du sang.\par
Que si une nation peut exclure, elle a, à plus forte raison, le droit de faire renoncer. Si elle craint qu’un certain mariage n’ait des suites qui puissent lui faire perdre son indépendance, ou la jeter dans un partage, elle pourra fort bien faire renoncer les contractants, et ceux qui naîtront d’eux, à tous les droits qu’ils auraient sur elle ; et celui qui renonce, et ceux contre qui on renonce, pourront d’autant moins se plaindre, que l’État aurait pu faire une loi pour les exclure.
\subsubsection[{Chapitre XXIV. Que les règlements de police sont d’un autre ordre que les autres lois civiles}]{Chapitre XXIV. Que les règlements de police sont d’un autre ordre que les autres lois civiles}
\noindent Il y a des criminels que le magistrat punit, il y en a d’autres qu’il corrige. Les premiers sont soumis à la puissance de la loi, les autres à son autorité ; ceux-là sont retranchés de la société, on oblige ceux-ci de vivre selon les règles de la société.\par
Dans l’exercice de la police, c’est plutôt le magistrat qui punit, que la loi ; dans les jugements des crimes, c’est plutôt la loi qui punit, que le magistrat. Les matières de police sont des choses de chaque instant, et où il ne s’agit ordinairement que de peu : il ne faut donc guère de formalités. Les actions de la police sont promptes, et elle s’exerce sur des choses qui reviennent tous les jours : les grandes punitions n’y sont donc pas propres. Elle s’occupe perpétuellement de détails : les grands exemples ne sont donc point faits pour elle. Elle a plutôt des règlements que des lois. Les gens qui relèvent d’elle sont sans cesse sous les yeux du magistrat ; c’est donc la faute du magistrat s’ils tombent dans des excès. Ainsi il ne faut pas confondre les grandes violations des lois avec la violation de la simple police : ces choses sont d’un ordre différent.\par
De là il suit qu’on ne s’est point conformé à la nature des choses dans cette république d’Italie\footnote{Venise.} où le port des armes à feu est puni comme un crime capital, et où il n’est pas plus fatal d’en faire un mauvais usage que de les porter.\par
Il suit encore que l’action tant louée de cet empereur, qui fit empaler un boulanger qu’il avait surpris en fraude, est une action de sultan, qui ne sait être juste qu’en outrant la justice même.
\subsubsection[{Chapitre XXV. Qu’il ne faut pas suivre les dispositions générales du droit civil, lorsqu’il s’agit de choses qui doivent être soumises à des règles particulières tirées de leur propre nature}]{Chapitre XXV. Qu’il ne faut pas suivre les dispositions générales du droit civil, lorsqu’il s’agit de choses qui doivent être soumises à des règles particulières tirées de leur propre nature}
\noindent Est-ce une bonne loi, que toutes les obligations civiles passées dans le cours d’un voyage entre les matelots dans un navire, soient nulles ? François Pyrard nous dit\footnote{Chap. XIV, part. XII.} que de son temps elle n’était point observée par les Portugais, mais qu’elle l’était par les Français. Des gens qui ne sont ensemble que pour peu de temps ; qui n’ont aucuns besoins, puisque le prince y pourvoit ; qui ne peuvent avoir qu’un objet, qui est celui de leur voyage ; qui ne sont plus dans la société, mais citoyens du navire, ne doivent point contracter de ces obligations qui n’ont été introduites que pour soutenir les charges de la société civile.\par
C’est dans ce même esprit que la loi des Rhodiens faite pour un temps où l’on suivait toujours les côtes, voulait que ceux qui, pendant la tempête, restaient dans le vaisseau, eussent le navire et la charge, et que ceux qui l’avaient quitté, n’eussent rien.
\section[{Sixième partie}]{Sixième partie}\renewcommand{\leftmark}{Sixième partie}

\subsection[{Livre vingt-septième}]{Livre vingt-septième}
\subsubsection[{Chapitre unique. De l’origine et des révolutions des lois des romains sur les successions}]{Chapitre unique. De l’origine et des révolutions des lois des romains sur les successions}
\noindent Cette matière tient à des établissements d’une antiquité très reculée ; et, pour la pénétrer à fond, qu’il me soit permis de chercher dans les premières lois des Romains ce que je ne sache pas que l’on y ait vu jusqu’ici.\par
On sait que Romulus partagea les terres de son petit État à ses citoyens\footnote{Denys d’Halicarnasse, liv. II, chap. III, Plutarque dans sa {\itshape Comparaison de Numa et de Lycurgue} 24, 11.} ; il me semble que c’est de là que dérivent les lois de Rome sur les successions.\par
La loi de la division des terres demanda que les biens d’une famille ne passassent pas dans, une autre : de là il suivit qu’il n’y eut que deux ordres d’héritiers établis par la loi\footnote{{\itshape Ast si intestatus moritur, cui suus haeres nec extabit, agnatus proximus familiam habeto.} Fragment de la loi des Douze Tables, dans Ulpien, titre dernier.} {\itshape ;} les enfants et tous les descendants qui vivaient sous la puissance du père, qu’on appela héritiers-siens ; et, à leur défaut, les plus proches parents par mâles, qu’on appela agnats.\par
Il suivit encore que les parents par femmes, qu’on appela cognats, ne devaient point succéder ; ils auraient transporté les biens dans une autre famille ; et cela fut ainsi établi.\par
Il suivit encore de là que les enfants ne devaient point succéder à leur mère, ni la mère à ses enfants ; cela aurait porté les biens d’une famille dans une autre. Aussi les voit-on exclus dans la loi des Douze Tables\footnote{Voy. les {\itshape Fragments} d’Ulpien, § 8, tit. XXVI, {\itshape Institutes}, tit. III, {\itshape In proemio ad Senatus consultum Tertullianum.}} {\itshape ;} elle n’appelait à la succession que les agnats, et le fils et la mère ne l’étaient pas entre eux.\par
Mais il était indifférent que l’héritier-sien ou, à son défaut, le plus proche agnat, fût mâle lui-même ou femelle, parce que les parents du côté maternel ne succédant point, quoiqu’une femme héritière se mariât, les biens rentraient toujours dans la famille dont ils étaient sortis. C’est pour cela qu’on ne distinguait point dans la loi des Douze Tables si la personne qui succédait était mâle ou femelle\footnote{Paul, liv. IV {\itshape De Sententiis}, tit. VIII, § 3.}.\par
Cela fit que, quoique les petits-enfants par le fils succédassent au grand-père, les petits-enfants par la fille ne lui succédèrent point : car, pour que les biens ne passassent pas dans une autre famille, les agnats leur étaient préférés. Ainsi la fille succéda à son père, et non pas ses enfants\footnote{{\itshape Institutes}, liv. III, tit. I, § 15.}.\par
Ainsi, chez les premiers Romains, les femmes succédaient, lorsque cela s’accordait avec la loi de la division des terres ; et elles ne succédaient point, lorsque cela pouvait la choquer.\par
Telles furent les lois des successions chez les premiers Romains ; et, comme elles étaient une dépendance naturelle de la constitution, et qu’elles dérivaient du partage des terres, on voit bien qu’elles n’eurent pas une origine étrangère, et ne furent point du nombre de celles que rapportèrent les députés que l’on envoya dans les villes grecques.\par
Denys d’Halicarnasse\footnote{Liv. IV, p. 276.} nous dit que Servius Tullius trouvant les lois de Romulus et de Numa sur le partage des terres abolies, il les rétablit, et en fit de nouvelles pour donner aux anciennes un nouveau poids. Ainsi, on ne peut douter que les lois dont nous venons de parler, faites en conséquence de ce partage, ne soient l’ouvrage de ces trois législateurs de Rome.\par
L’ordre de succession ayant été établi en conséquence d’une loi politique, un citoyen ne devait pas le troubler par une volonté particulière ; c’est-à-dire que, dans les premiers temps de Rome, il ne devait pas être permis de faire un testament. Cependant il eût été dur qu’on eût été privé dans ses derniers moments du commerce des bienfaits.\par
On trouva un moyen de concilier à cet égard les lois avec la volonté des particuliers. Il fut permis de disposer de ses biens dans une assemblée du peuple ; et chaque testament fut, en quelque façon, un acte de la puissance législative.\par
La loi des Douze Tables permit à celui qui faisait son testament de choisir pour son héritier le citoyen qu’il voulait. La raison qui fit que les lois romaines restreignirent si fort le nombre de ceux qui pouvaient succéder {\itshape ab intestat} fut la loi du partage des terres ; et la raison pourquoi elles étendirent si fort la faculté de tester, fut que le père pouvant vendre ses enfants\footnote{Denys d’Halicarnasse prouve, par une loi de Numa, que la loi qui permettait au père de vendre son fils trois fois, était une loi de Romulus, non pas des décemvirs. Liv. II.}, il pouvait, à plus forte raison, les priver de ses biens. C’étaient donc des effets différents, puisqu’ils coulaient de principes divers ; et c’est l’esprit des lois romaines à cet égard.\par
Les anciennes lois d’Athènes ne permirent point au citoyen de faire de testament. Solon le permit\footnote{Voyez Plutarque, {\itshape Vie de Solon}.}, excepté à ceux qui avaient des enfants ; et les législateurs de Rome, pénétrés de l’idée de la puissance paternelle, permirent de tester au préjudice même des enfants. Il faut avouer que les anciennes lois d’Athènes furent plus conséquentes que les lois de Rome. La permission indéfinie de tester, accordée chez les Romains, ruina peu à peu la disposition politique sur le partage des terres ; elle introduisit, plus que toute autre chose, la funeste différence entre les richesses et la pauvreté ; plusieurs partages furent assemblés sur une même tête ; des citoyens eurent trop, une infinité d’autres n’eurent rien. Aussi le peuple, continuellement privé de son partage, demanda-t-il sans cesse une nouvelle distribution des terres. Il la demanda dans le temps où la frugalité, la parcimonie et la pauvreté faisaient le caractère distinctif des Romains, comme dans les temps où leur luxe fut porté à l’excès.\par
Les testaments étant proprement une loi faite dans l’assemblée du peuple, ceux qui étaient à l’armée se trouvaient privés de la faculté de tester. Le peuple donna aux soldats le pouvoir de faire\footnote{Ce testament, appelé {\itshape in procinctu}, était différent de celui que l’on appela {\itshape militaire}, qui ne fut établi que par les constitutions des empereurs, leg. I, ff. {\itshape De militari testamento : ce} fut une de leurs cajoleries envers les soldats.}, devant quelques-uns de leurs compagnons, les dispositions qu’ils auraient faites devant lui\footnote{Ce testament n’était point écrit, et était sans formalités, {\itshape sins libra et tabulis}, comme dit Cicéron, liv. I {\itshape de l’Orateur}.}.\par
Les grandes assemblées du peuple ne se faisaient que deux fois l’an ; d’ailleurs, le peuple s’était augmenté et les affaires aussi. On jugea qu’il convenait de permettre à tous les citoyens de faire leur testament devant quelques citoyens romains pubères\footnote{{\itshape Institutes}, liv. II, tit. X, § 1 ; Aulu-Gelle, liv. XV, chap. XXVII. On appela cette sorte de testament, {\itshape per aes et libram.}}, qui représentassent le corps du peuple : on prit cinq citoyens\footnote{Ulpien, tit. X, § 2.}, devant lesquels l’héritier achetait du testateur sa famille, c’est-à-dire son hérédité\footnote{Théophile, {\itshape Institutes}, liv. II, tit. X.} ; un autre citoyen portait une balance pour en peser le prix ; car les Romains n’avaient point encore de monnaie\footnote{Ils n’en eurent qu’au temps de la guerre de Pyrrhus. Tite-Live, parlant du siège de Véies, dit : {\itshape nondum argentum signatum erat}, liv. IV.}.\par
Il y a apparence que ces cinq citoyens représentaient les cinq classes du peuple, et qu’on ne comptait pas la sixième, composée de gens qui n’avaient rien.\par
Il ne faut pas dire, avec Justinien, que ces ventes étaient imaginaires : elles le devinrent, mais au commencement elles ne l’étaient pas. La plupart des lois qui réglèrent dans la suite les testaments tirent leur origine de la réalité de ces ventes ; on en trouve bien la preuve dans les fragments d’Ulpien\footnote{Tit. XX, § 13.}. Le sourd, le muet, le prodigue ne pouvaient faire de testament : le sourd, parce qu’il ne pouvait pas entendre les paroles de l’acheteur de la famille ; le muet, parce qu’il ne pouvait pas prononcer les termes de la nomination ; le prodigue, parce que toute gestion d’affaires lui étant interdite, il ne pouvait pas vendre sa famille. Je passe les autres exemples.\par
Les testaments se faisant dans l’assemblée du peuple, ils étaient plutôt des actes du droit politique que du droit civil, du droit public plutôt que du droit privé : de là il suivit que le père ne pouvait permettre à son fils, qui était en sa puissance, de faire un testament.\par
Chez la plupart des peuples, les testaments ne sont pas soumis à de plus grandes formalités que les contrats ordinaires, parce que les uns et les autres ne sont que des expressions de la volonté de celui qui contracte, qui appartiennent également au droit privé. Mais, chez les Romains, où les testaments dérivaient du droit public, ils eurent de plus grandes formalités\footnote{{\itshape Institutes}, liv. II, tit. X, § 1.} que les autres actes ; et cela subsiste encore aujourd’hui dans les pays de France qui se régissent par le droit romain.\par
Les testaments étant, comme je l’ai dit, une loi du peuple, ils devaient être faits avec la force du commandement, et par des paroles que l’on appela {\itshape directes} et {\itshape impératives}. De là il se forma une règle, que l’on ne pourrait donner ni transmettre son hérédité que par des paroles de commandement\footnote{Titius, sois mon héritier.} : d’où il suivit que l’on pouvait bien, dans de certains cas, faire une substitution\footnote{La vulgaire, la pupillaire, l’exemplaire.}, et ordonner que l’hérédité passât à un autre héritier ; mais qu’on ne pouvait jamais faire de fidéicommis\footnote{Auguste, par des raisons particulières, commença à autoriser les fidéicommis. {\itshape Institutes}, liv. II, tit. XXIII, § 1.}, c’est-à-dire charger quelqu’un, en forme de prière, de remettre à un autre l’hérédité, ou une partie de l’hérédité.\par
Lorsque le père n’instituait ni exhérédait son fils, le testament était rompu ; mais il était valable, quoiqu’il n’exhérédât ni instituât sa fille. J’en vois la raison. Quand il n’instituait ni exhérédait son fils, il faisait tort à son petit-fils qui aurait succédé {\itshape ab intestat} à son père ; mais en n’instituant ni exhérédant sa fille, il ne faisait aucun tort aux enfants de sa fille, qui n’auraient point succédé {\itshape ab intestat} à leur mère\footnote{{\itshape Ad liberos matris intestatae haereditas, lege XII tabularum, non pertinebat, quia ferninae suos haeredes non habent.} Ulpien, fragm. tit. XXVI, § 7.}, parce qu’ils n’étaient héritiers-siens ni agnats.\par
Les lois des premiers Romains sur les successions n’ayant pensé qu’à suivre l’esprit du partage des terres, elles ne restreignirent pas assez les richesses des femmes, et laissèrent par là une porte ouverte au luxe, qui est toujours inséparable de ces richesses. Entre la seconde et la troisième guerre Punique, on commença à sentir le mal ; on fit la loi Voconienne\footnote{Quintus Voconius, tribun du peuple, la proposa. Voyez Cicéron, {\itshape Seconde harangue contre Verrès}. Dans l’{\itshape Epitome} de Tite-Live, liv. XLI, il faut lire Voconius, au lieu de Volumnius.}. Et comme de très grandes considérations la firent faire, qu’il ne nous en reste que peu de monuments, et qu’on n’en a jusqu’ici parlé que d’une manière très confuse, je vais l’éclaircir.\par
Cicéron nous en a conservé un fragment, qui défend d’instituer une femme héritière, soit qu’elle fût mariée, soit qu’elle ne le fût pas\footnote{{\itshape Sanxit. ne quis haeredem virginem neve mulierem faceret.} Cicéron, Seconde harangue contre Verrès.}.\par
{\itshape L’Epitome} de Tite-Live, où il est parlé de cette loi, n’en dit pas davantage\footnote{{\itshape Legem tulit, ne quis haeredem mulierem institueret}, liv. XLI.}. Il paraît, par Cicéron\footnote{Seconde harangue contre Verrès.} et par saint Augustin\footnote{Liv. III de la {\itshape Cité de Dieu}.}, que la fille, et même la fille unique, étaient comprises dans la prohibition.\par
Caton l’Ancien contribua de tout son pouvoir à faire recevoir cette loi\footnote{{\itshape Epitome} de Tite-Live, liv. XLI.}. Aulu-Gelle cite un fragment de la harangue qu’il fit dans cette occasion\footnote{Liv. XVII, chap. VI.}. En empêchant les femmes de succéder, il voulut prévenir les causes du luxe, comme en prenant la défense de la loi Oppienne, il voulut arrêter le luxe même.\par
Dans les {\itshape Institutes} de Justinien\footnote{{\itshape Institutes}, liv. II, tit. XXII.} et de Théophile\footnote{Liv. II, tit. XXII.}, on parle d’un chapitre de la loi Voconienne, qui restreignait la faculté de léguer. En lisant ces auteurs, il n’y a personne qui ne pense que ce chapitre fut fait pour éviter que la succession ne fût tellement épuisée par des legs, que l’héritier refusât de l’accepter. Mais ce n’était point là l’esprit de la loi Voconienne. Nous venons de voir qu’elle avait pour objet d’empêcher les femmes de recevoir aucune succession. Le chapitre de cette loi qui mettait des bornes à la faculté de léguer, entrait dans cet objet : car, si on avait pu léguer autant que l’on aurait voulu, les femmes auraient pu recevoir comme legs ce qu’elles ne pouvaient obtenir comme succession.\par
La loi Voconienne fut faite pour prévenir les trop grandes richesses des femmes. Ce fut donc des successions considérables dont il fallut les priver, et non pas de celles qui ne pouvaient entretenir le luxe. La loi fixait une certaine somme qui devait être donnée aux femmes qu’elle privait de la succession. Cicéron\footnote{{\itshape Nemo censuit plus Fadiae dandum, quam posset ad eam lege Voconia pervenire.De finibus bonorum et malorum}, liv. II.}, qui nous apprend ce fait, ne nous dit point quelle était cette somme ; mais Dion\footnote{{\itshape Cum lege Voconia mulieribus prohiberetur ne qua majorem centum millibus nummum haereditatem posset adire}, liv. LVI.} dit qu’elle était de cent mille sesterces.\par
La loi Voconienne était faite pour régler les richesses, et non pas pour régler la pauvreté : aussi Cicéron nous dit-il\footnote{{\itshape Qui census esset.} Seconde harangue contre Verrès.} qu’elle ne statuait que sur ceux qui étaient inscrits dans le cens.\par
Ceci fournit un prétexte pour éluder la loi. On sait que les Romains étaient extrêmement formalistes ; et nous avons dit ci-dessus que l’esprit de la république était de suivre la lettre de la loi. Il y eut des pères qui ne se firent point inscrire dans le cens, pour pouvoir laisser leur succession à leur fille : et les prêteurs jugèrent qu’on ne violait point la loi Voconienne, puisqu’on n’en violait point la lettre.\par
Un certain Anius Asellus avait institué sa fille unique héritière. Il le pouvait, dit Cicéron : la loi Voconienne ne l’en empêchait pas, parce qu’il n’était point dans le cens\footnote{{\itshape Census non erat. Ibid.}}. Verrès, étant préteur, avait privé la fille de la succession : Cicéron soutient que Verrès avait été corrompu, parce que, sans cela, il n’aura point interverti un ordre que les autres préteurs avaient suivi.\par
Qu’étaient donc ces citoyens qui n’étaient point dans le cens qui comprenait tous les citoyens ? Mais, selon l’institution de Servius Tullius, rapportée par Denys d’Halicarnasse\footnote{Liv. IV.}, tout citoyen qui ne se faisait point inscrire dans le cens, était fait esclave : Cicéron lui-même dit qu’un tel homme perdait la liberté\footnote{{\itshape In oratione pro Caecina}.} : Zonare dit la même chose. Il fallait donc qu’il y eût de la différence entre n’être point dans le cens selon l’esprit de la loi Voconienne, et n’être point dans le cens selon l’esprit des institutions de Servius Tullius.\par
Ceux qui ne s’étaient point fait inscrire dans les cinq premières classes, où l’on était placé selon la proportion de ses biens\footnote{Ces cinq premières classes étaient si considérables, que quelquefois les auteurs n’en rapportent que cinq.}, n’étaient point dans le cens selon l’esprit de la loi Voconienne : ceux qui n’étaient point inscrits dans le nombre des six classes, ou qui n’étaient point mis par les censeurs au nombre de ceux que l’on appelait {\itshape aerarii}, n’étaient point dans le cens suivant les institutions de Servius Tullius. Telle était la force de la nature, que des pères, pour éluder la loi Voconienne, consentaient à souffrir la honte d’être confondus dans la sixième classe avec les prolétaires et ceux qui étaient taxés pour leur tête, ou peut-être même à être renvoyés dans les tables des Cérites\footnote{{\itshape In Caeritum tabulas referri ; aerarius fieri}.}.\par
Nous avons dit que la jurisprudence des Romains n’admettait point les fidéicommis. L’espérance d’éluder la loi Voconienne les introduisit : on instituait un héritier capable de recevoir par la loi, et on le priait de remettre la succession à une personne que la loi en avait exclue. Cette nouvelle manière de disposer eut des effets bien différents. Les uns rendirent l’hérédité ; et l’action de Sextus Peduceus\footnote{Cicéron, {\itshape De finibus bonorum et malorum}, liv. III.} fut remarquable. On lui donna une grande succession ; il n’y avait personne dans le monde que lui qui sût qu’il était prié de la remettre : il alla trouver la veuve du testateur, et lui donna tout le bien de son mari.\par
Les autres gardèrent pour eux la succession ; et l’exemple de P. Sextilius Rufus fut célèbre encore, parce que Cicéron l’emploie dans ses disputes contre les Épicuriens\footnote{{\itshape Ibid.}}. « Dans ma jeunesse, dit-il, je fus prié par Sextilius de l’accompagner chez ses amis, pour savoir d’eux s’il devait remettre l’hérédité de Quintus Fadius Gallus à Fadia sa fille. Il avait assemblé plusieurs jeunes gens, avec de très graves personnages ; et aucun ne fut d’avis qu’il donnât plus à Fadia que ce qu’elle devait avoir par la loi Voconienne. Sextilius eut là une grande succession, dont il n’aurait pas retenu un sesterce, s’il avait préféré ce qui était juste et honnête à ce qui était utile. Je puis croire, ajoute-t-il, que vous auriez rendu l’hérédité ; je puis croire même qu’Épicure l’aurait rendue ; mais vous n’auriez pas suivi vos principes. » Je ferai ici quelques réflexions.\par
C’est un malheur de la condition humaine que les législateurs soient obligés de faire des lois qui combattent les sentiments naturels mêmes : telle fut la loi Voconienne. C’est que les législateurs statuent plus sur la société que sur le citoyen, et sur le citoyen que sur l’homme. La loi sacrifiait et le citoyen et l’homme, et ne pensait qu’à la république. Un homme priait son ami de remettre sa succession à sa fille : la loi méprisait dans le testateur les sentiments de la nature ; elle méprisait dans la fille la piété filiale ; elle n’avait aucun égard pour celui qui était chargé de remettre l’hérédité, qui se trouvait dans de terribles circonstances. La remettait-il ? il était un mauvais citoyen ; la gardait-il ? il était un malhonnête homme. Il n’y avait que les gens d’un bon naturel qui pensassent à éluder la loi ; il n’y avait que les honnêtes gens qu’on pût choisir pour l’éluder : car c’est toujours un triomphe à remporter sur l’avarice et les voluptés, et il n’y a que les honnêtes gens qui obtiennent ces sortes de triomphes. Peut-être même y aurait-il de la rigueur à les regarder en cela comme de mauvais citoyens. Il n’est pas impossible que le législateur eût obtenu une grande partie de son objet, lorsque sa loi était telle, qu’elle ne forçait que les honnêtes gens à l’éluder.\par
Dans le temps que l’on fit la loi Voconienne, les mœurs avaient conservé quelque chose de leur ancienne pureté. On intéressa quelquefois la conscience publique en faveur de la loi, et l’on fit jurer qu’on l’observerait\footnote{Sextilius disait qu’il avait juré de l’observer. Cicéron, {\itshape De finibus bonorum et malorum}, liv. Il.} : de sorte que la probité faisait, pour ainsi dire, la guerre à la probité. Mais, dans les derniers temps, les mœurs se corrompirent au point que les fidéicommis durent avoir moins de force pour éluder la loi Voconienne, que cette loi n’en avait pour se faire suivre.\par
Les guerres civiles firent périr un nombre infini de citoyens. Rome, sous Auguste, se trouva presque déserte ; il fallait la repeupler. On fit les lois Papiennes, où l’on n’omit rien de ce qui pouvait encourager les citoyens à se marier et à avoir des enfants\footnote{Voyez ce que j’en ai dit au liv. XXIII, chap. XXI.}. Un des principaux moyens fut d’augmenter, pour ceux qui se prêtaient aux vues de la loi, les espérances de succéder, et de les diminuer pour ceux qui s’y refusaient ; et, comme la loi Voconienne avait rendu les femmes incapables de succéder, la loi Papienne fit, dans de certains cas, cesser cette prohibition.\par
Les femmes\footnote{Voyez sur ceci les {\itshape Fragments} d’Ulpien, tit. XV, § 16.}, surtout celles qui avaient des enfants, furent rendues capables de recevoir en vertu du testament de leurs maris ; elles purent, quand elles avaient des enfants, recevoir en vertu du testament des étrangers ; tout cela contre la disposition de la loi Voconienne ; et il est remarquable qu’on n’abandonna pas entièrement l’esprit de cette loi. Par exemple, la loi Papienne\footnote{La même différence se trouve dans plusieurs dispositions de la loi Papienne. Voyez les {\itshape Fragments} d’Ulpien, §§ 4 et 5, titre dernier ; et le même au même titre § 6.} permettait à un homme qui avait un enfant\footnote{{\itshape Quod tibi filiolus, vel filia, nascitur ex me,/ Jura parentis habes ; propter me scriberis haeres.}/ JUVÉNAL, {\itshape Sat.} IX.} de recevoir toute l’hérédité par le testament d’un étranger ; elle n’accordait la même grâce à la femme, que lorsqu’elle avait trois enfants\footnote{Voyez la loi 9, code Théodosien, {\itshape De bonis proscriptorum ;} et Dion, liv. LV ; voyez les {\itshape Fragments} d’Ulpien, titre dernier § 6 ; et titre XXIX, § 3.}.\par
Il faut remarquer que la loi Papienne ne rendit les femmes qui avaient trois enfants capables de succéder, qu’en vertu du testament des étrangers ; et qu’à l’égard de la succession des parents, elle laissa les anciennes lois et la loi Voconienne\footnote{{\itshape Fragments} d’Ulpien, tit. XVI, § 1 ; Sozomène, liv. I, chap. XIX.} dans toute leur force. Mais cela ne subsista pas.\par
Rome, abîmée par les richesses de toutes les nations, avait changé de mœurs ; il ne fut plus question d’arrêter le luxe des femmes. Aulu-Gelle, qui vivait sous Adrien, nous dit\footnote{Liv. XX, chap. I.} que de son temps la loi Voconienne était presque anéantie ; elle fut couverte par l’opulence de la cité. Aussi trouvons-nous dans les {\itshape Sentences} de Paul\footnote{Liv. IV, tit. VIII, § 3.}, qui vivait sous Niger, et dans les {\itshape Fragments} d’Ulpien\footnote{Tit. XXVI, § 6.}, qui était du temps d’Alexandre Sévère, que les sœurs du côté du père pouvaient succéder, et qu’il n’y avait que les parents d’un degré plus éloigné qui fussent dans le cas de la prohibition de la loi Voconienne.\par
Les anciennes lois de Rome avaient commencé à paraître dures. Les préteurs ne furent plus touchés que des raisons d’équité, de modération et de bienséance.\par
Nous avons vu que, par les anciennes lois de Rome, les mères n’avaient point de part à la succession de leurs enfants. La loi Voconienne fut une nouvelle raison pour les en exclure. Mais l’empereur Claude donna à la mère la succession de ses enfants, comme une consolation de leur perte ; le sénatus-consulte Tertullien, fait sous Adrien\footnote{C’est-à-dire l’empereur Pie, qui prit le nom d’Adrien par adoption.}, la leur donna lorsqu’elles avaient trois enfants, si elles étaient ingénues ; ou quatre, si elles étaient affranchies. Il est clair que ce sénatus-consulte n’était qu’une extension de la loi Papienne, qui, dans le même cas, avait accordé aux femmes les successions qui leur étaient déférées par les étrangers. Enfin Justinien\footnote{Leg. 2, cod. {\itshape De jure liberorum, Institutes}, liv. III, tit. III, § 4, {\itshape de senatusconsulto Tertulliano}.} leur accorda la succession, indépendamment du nombre de leurs enfants.\par
Les mêmes causes qui firent restreindre la loi qui empêchait les femmes de succéder, firent renverser peu à peu celle qui avait gêné la succession des parents par femmes. Ces lois étaient très conformes à l’esprit d’une bonne république, où l’on doit faire en sorte que ce sexe ne puisse se prévaloir pour le luxe, ni de ses richesses ni de l’espérance de ses richesses. Au contraire, le luxe d’une monarchie rendant le mariage à charge et coûteux, il faut y être invité, et par les richesses que les femmes peuvent donner, et par l’espérance des successions qu’elles peuvent procurer. Ainsi, lorsque la monarchie s’établit à Rome, tout le système fut changé sur les successions. Les préteurs appelèrent les parents par femmes au défaut des parents par mâles : au lieu que par les anciennes lois, les parents par femmes n’étaient jamais appelés. Le sénatus-consulte Orphitien appela les enfants à la succession de leur mère ; et les empereurs Valentinien\footnote{Leg. 9, Cod. {\itshape De suis et legitimis liberis}.}, Théodose et Arcadius appelèrent les petits-enfants par la fille à la succession du grand-père. Enfin l’empereur Justinien ôta jusqu’au moindre vestige du droit ancien sur les successions : il établit trois ordres d’héritiers, les descendants, les ascendants, les collatéraux, sans aucune distinction entre les mâles et les femelles, entre les parents par femmes et les parents par mâles, et abrogea toutes celles qui restaient à cet égard\footnote{Leg. 12, Cod. {\itshape ibid.}, et les {\itshape Novelles} 118 et 127.}. Il crut suivre la nature même, en s’écartant de ce qu’il appela les embarras de l’ancienne jurisprudence.
\subsection[{Livre vingt-huitième. De l’origine et des révolutions des lois civiles chez les Français}]{Livre vingt-huitième. De l’origine et des révolutions des lois civiles chez les Français}
\noindent {\itshape In nova fert animus mutatas dicere formas \\
Corpora.................. }\\
{\scshape Ovid}., {\itshape Métam.}, L. I, v. \textsc{i}.
\subsubsection[{Chapitre I. Du différent caractère des lois des peuples germains}]{Chapitre I. Du différent caractère des lois des peuples germains}
\noindent Les Francs étant sortis de leur pays, ils firent rédiger, par les sages de leur nation, les lois saliques\footnote{Voyez le prologue de la loi salique. M. de Leibnitz dit, dans son traité {\itshape De l’origine des Francs}, que cette loi fut faite avant le règne de Clovis ; mais elle ne put l’être avant que les Francs fussent sortis de la Germanie : ils n’entendaient pas pour lors la langue latine.}. La tribu des Francs Ripuaires s’étant jointe, sous Clovis\footnote{Voyez Grégoire de Tours.} à celle des Francs Saliens, elle conserva ses usages ; et Théodoric\footnote{Voyez le prologue de la loi des Bavarois et celui de la loi salique.}, roi d’Austrasie, les fit mettre par écrit. Il recueillit de même les usages des Bavarois et des Allemands\footnote{{\itshape Ibid.}} qui dépendaient de son royaume. Car la Germanie étant affaiblie par la sortie de tant de peuples, les Francs, après avoir conquis devant eux, avaient fait un pas en arrière, et porté leur domination dans les forêts de leurs pères. Il y a apparence que le code des Thuringiens fut donné par le même Théodoric\footnote{Lex Angliorum Werinorum, hoc est, Thutingorum.}, puisque les Thuringiens étaient aussi ses sujets. Les Frisons ayant été soumis par Charles Martel et Pépin, leur loi n’est pas antérieure à ces princes\footnote{Ils ne savaient point écrire.}. Charlemagne, qui, le premier, dompta les Saxons, leur donna la loi que nous avons. Il n’y a qu’à lire ces deux derniers codes pour voir qu’ils sortent des mains des vainqueurs. Les Wisigoths, les Bourguignons et les Lombards ayant fondé des royaumes, firent écrire leurs lois, non pas pour faire suivre leurs usages aux peuples vaincus, mais pour les suivre eux-mêmes.\par
Il y a dans les lois saliques et ripuaires, dans celles des Allemands, des Bavarois, des Thuringiens et des Frisons, une simplicité admirable : on y trouve une rudesse originale et un esprit qui n’avait point été affaibli par un autre esprit. Elles changèrent peu, parce que ces peuples, si on en excepte les Francs, restèrent dans la Germanie. Les Francs même y fondèrent une grande partie de leur empire : ainsi leurs lois furent toutes germaines. Il n’en fut pas de même des lois des Wisigoths, des Lombards et des Bourguignons ; elles perdirent beaucoup de leur caractère, parce que ces peuples, qui se fixèrent dans leurs nouvelles demeures, perdirent beaucoup du leur.\par
Le royaume des Bourguignons ne subsista pas assez longtemps pour que les lois du peuple vainqueur pussent recevoir de grands changements. Gondebaud et Sigismond, qui recueillirent leurs usages, furent presque les derniers de leurs rois. Les lois des Lombards reçurent plutôt des additions que des changements. Celles de Rotharis furent suivies de celles de Grimoald, de Luitprand, de Rachis, d’Aistulphe ; mais elles ne prirent point de nouvelle forme. Il n’en fut pas de même des lois des Wisigoths\footnote{Euric les donna, Leuvigilde les corrigea. Voyez la {\itshape Chronique} d’Isidore. Chaindasuinde et Recessuinde les réformèrent. Egiga fit faire le code que nous avons, et en donna la commission aux évêques : on conserva pourtant les lois de Chaindasuinde et de Recessuinde, comme il panait par le seizième concile de Tolède.} {\itshape ;} leurs rois les refondirent, et les firent refondre par le clergé.\par
Les rois de la première race ôtèrent bien aux lois saliques et ripuaires ce qui ne pouvait absolument s’accorder avec le christianisme ; mais ils en laissèrent tout le fond\footnote{Voyez le prologue de la loi des Bavarois.}. C’est ce qu’on ne peut pas dire des lois des Wisigoths.\par
Les lois des Bourguignons, et surtout celles des Wisigoths, admirent les peines corporelles. Les lois saliques et ripuaires ne les reçurent pas\footnote{On en trouve seulement quelques-unes dans le décret de Childebert.} ; elles conservèrent mieux leur caractère.\par
Les Bourguignons et les Wisigoths, dont les provinces étaient très exposées, cherchèrent à se concilier les anciens habitants, et à leur donner des lois civiles les plus impartiales\footnote{Voyez le prologue du Code des Bourguignons, et le Code même, surtout le tit. XII, § 5, et le tit. XXXVIII. Voyez aussi Grégoire de Tours, liv. II, chap. XXXIII ; et le code des Wisigoths.} ; mais les rois Francs, sûrs de leur puissance, n’eurent pas ces égards\footnote{Voyez ci-dessous le chap. III.}.\par
Les Saxons, qui vivaient sous l’empire des Francs, eurent une humeur indomptable, et s’obstinèrent à se révolter. On trouve dans leurs lois\footnote{Voyez le chap. II, §§ 8 et 9 ; et le chap. IV, §§ 2 et 7.} des duretés du vainqueur, qu’on ne voit point dans les autres codes des lois des Barbares.\par
On y voit l’esprit des lois des Germains dans les peines pécuniaires, et celui du vainqueur dans les peines afflictives.\par
Les crimes qu’ils font dans leur pays sont punis corporellement ; et on ne suit l’esprit des lois germaniques que dans la punition de ceux qu’ils commettent hors de leur territoire.\par
On y déclare que, pour leurs crimes, ils n’auront jamais de paix, et on leur refuse l’asile des églises mêmes.\par
Les évêques eurent une autorité immense à la cour des rois Wisigoths ; les affaires les plus importantes étaient décidées dans les conciles. Nous devons au code des Wisigoths toutes les maximes, tous les principes et toutes les vues de l’Inquisition d’aujourd’hui ; et les moines n’ont fait que copier contre les juifs, des lois faites autrefois par les évêques.\par
Du reste, les lois de Gondebaud pour les Bourguignons paraissent assez judicieuses ; celles de Rotharis et des autres princes lombards le sont encore plus. Mais les lois des Wisigoths, celles de Recessuinde, de Chaindasuinde et d’Egiga, sont puériles, gauches, idiotes ; elles n’atteignent point le but ; pleines de rhétorique, et vides de sens, frivoles dans le fond, et gigantesques dans le style.
\subsubsection[{Chapitre II. Que les lois des barbares furent toutes personnelles}]{Chapitre II. Que les lois des barbares furent toutes personnelles}
\noindent C’est un caractère particulier de ces lois des barbares, qu’elles ne furent point attachées à un certain territoire : le Franc était jugé par la loi des Francs, l’Allemand par la loi des Allemands, le Bourguignon par la loi des Bourguignons, le Romain par la loi romaine ; et, bien loin qu’on songeât dans ces temps-là à rendre uniformes les lois des peuples conquérants, on ne pensa pas même à se faire législateur du peuple vaincu.\par
Je trouve l’origine de cela dans les mœurs des peuples germains. Ces nations étaient partagées par des marais, des lacs et des forêts ; on voit même dans César\footnote{{\itshape De bello Gallico}, liv. VI.} qu’elles aimaient à se séparer. La frayeur qu’elles eurent des Romains fit qu’elles se réunirent : chaque homme, dans ces nations mêlées, dut être jugé par les usages et les coutumes de sa propre nation. Tous ces peuples, dans leur particulier, étaient libres et indépendants ; et, quand ils furent mêlés, l’indépendance resta encore. La patrie était commune, et la république particulière ; le territoire était le même, et les nations diverses. L’esprit des lois personnelles était donc chez ces peuples avant qu’ils partissent de chez eux, et ils le portèrent dans leurs conquêtes.\par
On trouve cet usage établi dans les formules de Marculfe\footnote{Liv. I, formule 8.}, dans les codes des lois des barbares, surtout dans la loi des Ripuaires\footnote{Chap. XXXI.}, dans les décrets des rois de la première race\footnote{Celui de Clotaire de l’an 560, dans l’édition des {\itshape Capitulaires} de Baluze, t. I, art. 4 ; {\itshape ibid., in fine.}}, d’où dérivèrent les capitulaires que l’on fit là-dessus dans la seconde\footnote{{\itshape Capitulaires} ajoutés à la loi des Lombards, liv. I, tit. XXV, chap. LXXI ; liv. II, tit. XLI, chap. VII ; et tit. LVI, chap. I et II.}. Les enfants suivaient la loi de leur père\footnote{{\itshape Ibid.}, liv. II, tit. V.}, les femmes celle de leur mari\footnote{{\itshape Ibid.} liv. II, tit. VII, chap. I.}, les veuves revenaient à leur loi\footnote{{\itshape Ibid.}, chap. II.}, les affranchis avaient celle de leur patron\footnote{{\itshape Ibid.}, liv. II, tit. XXXV, chap. II.}. Ce n’est pas tout : chacun pouvait prendre la loi qu’il voulait ; la constitution de Lothaire I\textsuperscript{er} exigea que ce choix fût rendu public\footnote{Dans la loi des Lombards, liv. II, tit. LXVII.}.
\subsubsection[{Chapitre III. Différence capitale entre les lois saliques et les lois des Wisigoths et des Bourguignons}]{Chapitre III. Différence capitale entre les lois saliques et les lois des Wisigoths et des Bourguignons}
\noindent J’ai dit\footnote{Au chapitre I de ce livre.} que la loi des Bourguignons et celle des Wisigoths étaient impartiales ; mais la loi salique ne le fut pas : elle établit entre les Francs et les Romains les distinctions les plus affligeantes. Quand\footnote{Loi salique, tit. XLIV, § I.} on avait tué un Franc, un barbare, ou un homme qui vivait sous la loi salique, on payait à ses parents une composition de deux cents {\itshape sols} ; on n’en payait qu’une de cent, lorsqu’on avait tué un Romain possesseur\footnote{{\itshape Qui res in pago ubi remanet proprias habet.} Loi salique, tit. XLIV, § 15 ; voyez aussi le § 7.} {\itshape ;} et seulement une de quarante-cinq, quand on avait tué un Romain tributaire : la composition pour le meurtre d’un Franc, vassal\footnote{{\itshape Qui in truste dominica est, ibid.}, tit. XLIV, § 4.} du roi, était de six cents sols ; et celle du meurtre d’un Romain convive\footnote{{\itshape Si Romanus homo conviva regis fuerit, ibid.}, § 6} du roi\footnote{Les principaux Romains s’attachaient à la cour comme on le voit par la vie de plusieurs évêques qui y furent élevés. Il n’y avait guère que les Romains qui sussent écrire.} n’était que de trois cents. Elle mettait donc une cruelle différence entre le seigneur franc et le seigneur romain, et entre le Franc et le Romain qui étaient d’une condition médiocre.\par
Ce n’est pas tout : si l’on assemblait\footnote{{\itshape Ibid.}, tit. XLV.} du monde pour assaillir un Franc dans sa maison, et qu’on le tuât, la loi salique ordonnait une composition de six cents sols ; mais si on avait assailli un Romain ou un affranchi\footnote{{\itshape Lidus}, dont la condition était meilleure que celle du serf. Loi des Allemands, chap. XCV.} on ne payait que la moitié de la composition. Par la même loi\footnote{Tit. XXXV, §§ 3 et 4.}, si un Romain enchaînait un Franc, il devait trente sols de composition ; mais si un Franc enchaînait un Romain, il n’en devait qu’une de quinze. Un Franc dépouillé par un Romain, avait soixante-deux sols et demi de composition ; et un Romain dépouillé par un Franc, n’en recevait qu’une de trente. Tout cela devait être accablant pour les Romains.\par
Cependant un auteur célèbre\footnote{L’abbé Dubos.} forme un système de {\itshape l’Établissement des Francs dans les Gaules}, sur la présupposition qu’ils étaient les meilleurs amis des Romains. Les Francs étaient donc les meilleurs amis des Romains, eux qui leur firent, eux qui en reçurent des maux effroyables\footnote{Témoin l’expédition d’Arbogaste, dans Grégoire de Tours, {\itshape Histoire}, liv. Il.} ? Les Francs étaient amis des Romains, eux qui, après les avoir assujettis par leurs armes, les opprimèrent de sang-froid par leurs lois ? Ils étaient amis des Romains comme les Tartares qui conquirent la Chine étaient amis des Chinois.\par
Si quelques évêques catholiques ont voulu se servir des Francs pour détruire des rois ariens, s’ensuit-il qu’ils aient désiré de vivre sous des peuples barbares ? En peut-on conclure que les Francs eussent des égards particuliers pour les Romains ? J’en tirerais bien d’autres conséquences : plus les Francs furent sûrs des Romains, moins ils les ménagèrent.\par
Mais l’abbé Dubos a puisé dans de mauvaises sources pour un historien, dans les poètes et les orateurs : ce n’est point sur des ouvrages d’ostentation qu’il faut fonder des systèmes.
\subsubsection[{Chapitre IV. Comment le droit romain se perdit dans le pays du domaine des Francs, et se conserva dans le pays du domaine des Goths et des Bourguignons}]{Chapitre IV. Comment le droit romain se perdit dans le pays du domaine des Francs, et se conserva dans le pays du domaine des Goths et des Bourguignons}
\noindent Les choses que j’ai dites donneront du jour à d’autres, qui ont été jusqu’ici pleines d’obscurité.\par
Le pays qu’on appelle aujourd’hui la France fut gouverné, dans la première race, par la loi romaine ou le code Théodosien, et par les diverses lois des barbares\footnote{Les Francs, les Wisigoths et les Bourguignons.} qui y habitaient.\par
Dans le pays du domaine des Francs, la loi salique était établie pour les Francs, et le code\footnote{Il fut fini l’an 438.} Théodosien pour les Romains. Dans celui du domaine des Wisigoths, une compilation du code Théodosien\footnote{La vingtième année du règne de ce prince, et publiée deux ans après par Anian, comme il paraît par la préface de ce code.}, faite par l’ordre d’Alaric, régla les différends des Romains ; les coutumes de la nation, qu’Euric\footnote{L’an 504 de l’ère d’Espagne : {\itshape Chronique} d’Isidore.} fit rédiger par écrit, décidèrent ceux des Wisigoths. Mais pourquoi les lois saliques acquirent-elles une autorité presque générale dans les pays des Francs ? Et pourquoi le droit romain s’y perdit-il peu à peu, pendant que dans le domaine des Wisigoths le droit romain s’étendit, et eut une autorité générale ?\par
Je dis que le droit romain perdit son usage chez les Francs, à cause des grands avantages qu’il y avait à être Franc, barbare, ou homme vivant sous la loi salique\footnote{{\itshape Francum, aut barbarum, aut hominem qui salica lege vivit.} Loi salique, tit. XLV, § 1.} : tout le monde fut porté à quitter le droit romain pour vivre sous la loi salique. Il fut seulement retenu par les ecclésiastiques\footnote{{\itshape Selon la loi romaine sous laquelle l’église vit}, est-il dit dans la loi des Ripuaires, tit. LVIII, § I. Voyez aussi les autorités sans nombre là-dessus, rapportées par M. Ducange, au mot Lex {\itshape romana.}} parce qu’ils n’eurent point d’intérêt à changer. Les différences des conditions et des rangs ne consistaient que dans la grandeur des compositions, comme je le ferai voir ailleurs. Or, des lois\footnote{Voyez les capitulaires ajoutés à la loi salique dans Lindembroch, à la fin de cette loi, et les divers codes des lois des barbares, sur les privilèges des ecclésiastiques à cet égard. Voyez aussi la lettre de Charlemagne à Pépin, son fils, roi d’Italie, de l’an 807, dans l’édition de Baluze, t. I, p. 452, où il est dit qu’un ecclésiastique doit recevoir une composition triple ; et le recueil des capitulaires, liv. V, art 302, 1, édition de Baluze.} particulières leur donnèrent des compositions aussi favorables que celles qu’avaient les Francs : ils gardèrent donc le droit romain. Ils n’en recevaient aucun préjudice ; et il leur convenait d’ailleurs, parce qu’il était l’ouvrage des empereurs chrétiens.\par
D’un autre côté, dans le patrimoine des Wisigoths, la loi wisigothe\footnote{Voyez cette loi.} ne donnant aucun avantage civil aux Wisigoths sur les Romains, les Romains n’eurent aucune raison de cesser de vivre sous leur loi pour vivre sous une autre : ils gardèrent donc leurs lois, et ne prirent point celles des Wisigoths.\par
Ceci se confirme à mesure qu’on va plus avant. La loi de Gondebaud fut très impartiale, et ne fut pas plus favorable aux Bourguignons qu’aux Romains. Il paraît, par le prologue de cette loi, qu’elle fut faite pour les Bourguignons, et qu’elle fut faite encore pour régler les affaires qui pourraient naître entre les Romains et les Bourguignons ; et, dans ce dernier cas, le tribunal fut mi-parti. Cela était nécessaire pour des raisons particulières, tirées de l’arrangement\footnote{J’en parlerai ailleurs, liv. XXX, chap. VI, VII, VII et IX.} politique de ces temps-là. Le droit romain subsista dans la Bourgogne, pour régler les différends que les Romains pourraient avoir entre eux. Ceux-ci n’eurent point de raison pour quitter leur loi, comme ils en eurent dans le pays des Francs ; d’autant mieux que la loi salique n’était point établie en Bourgogne, comme il paraît par la fameuse lettre qu’Agobard écrivit à Louis le Débonnaire.\par
Agobard\footnote{Agobard, {\itshape Opera.}} demandait à ce prince d’établir la loi salique dans la Bourgogne : elle n’y était donc pas établie. Ainsi le droit romain subsista, et subsiste encore dans tant de provinces qui dépendaient autrefois de ce royaume.\par
Le droit romain et la loi gothe se maintinrent de même dans le pays de l’établissement des Goths : la loi salique n’y fut jamais reçue. Quand Pépin et Charles Martel en chassèrent les Sarrazins, les villes et les provinces qui se soumirent à ces princes\footnote{Voyez Gervais de Tilburi, dans le recueil de Duchesne, t. III, p. 366 : {\itshape Facta pactione cum Francis, quod illic Gothi patriis legibus, moribus paternis vivant. Et sic Narbonensis provincia Pippino subjicitur.} Et une chronique de l’an 759, rapportée par Catel, {\itshape Histoire du Languedoc}. Et l’auteur incertain de la {\itshape Vie de Louis le Débonnaire}, sur la demande faite par les peuples de la Septimanie, dans l’assemblée {\itshape in Carisiaco}, dans le recueil de Duchesne, t. II, p. 316.} demandèrent à conserver leurs lois, et l’obtinrent : ce qui, malgré l’usage de ces temps-là où toutes les lois étaient personnelles, fit bientôt regarder le droit romain comme une loi réelle et territoriale dans ces pays.\par
Cela se prouve par l’édit de Charles le Chauve, donné à Pistes l’an 864, qui distingue les pays dans lesquels on jugeait par le droit romain, d’avec ceux où l’on n’y jugeait pas\footnote{{\itshape In illa terra in qua judicia secundum legem Romanam terminantur, secundum ipsam legem judicetur ; et in illa terra in qua}, etc., art. 16. Voyez aussi l’art. 20.}.\par
L’édit de Pistes prouve deux choses : l’une, qu’il y avait des pays où l’on jugeait selon la loi romaine, et qu’il y en avait où l’on ne jugeait point selon cette loi ; l’autre, que ces pays où l’on jugeait par la loi romaine étaient précisément ceux où on la suit encore aujourd’hui, comme il paraît par ce même édit\footnote{Voyez les art. 12 et 16 de l’édit de Pistes, {\itshape in Cavilono, in Narbona}, etc.}. Ainsi la distinction des pays de la France coutumière, et de la France régie par le droit écrit, était déjà établie du temps de l’édit de Pistes.\par
J’ai dit que, dans les commencements de la monarchie, toutes les lois étaient personnelles : ainsi, quand l’édit de Pistes distingue les pays du droit romain d’avec ceux qui ne l’étaient pas, cela signifie que, dans les pays qui n’étaient point pays de droit romain, tant de gens avaient choisi de vivre sous quelqu’une des lois des peuples barbares, qu’il n’y avait presque plus personne dans ces contrées qui choisît de vivre sous la loi romaine ; et que, dans les pays de la loi romaine, il y avait peu de gens qui eussent choisi de vivre sous les lois des peuples barbares.\par
Je sais bien que je dis ici des choses nouvelles ; mais si elles sont vraies, elles sont très anciennes. Qu’importe, après tout, que ce soit moi, les Valois, ou les Bignon qui les aient dites ?
\subsubsection[{Chapitre V. Continuation du même sujet}]{Chapitre V. Continuation du même sujet}
\noindent La loi de Gondebaud subsista longtemps chez les Bourguignons, concurremment avec la loi romaine ; elle y était encore en usage du temps de Louis le Débonnaire ; la lettre d’Agobard ne laisse aucun doute là-dessus. De même, quoique l’édit de Pistes appelle le pays qui avait été occupé par les Wisigoths, le pays de la loi romaine, la loi des Wisigoths y subsistait toujours ; ce qui se prouve par le synode de Troyes, tenu sous Louis le Bègue l’an 878, c’est-à-dire quatorze ans après l’édit de Pistes.\par
Dans la suite, les lois gothes et bourguignonnes périrent dans leur pays même, par les causes générales\footnote{Voyez ci-dessous les chap. IX, X et XI.} qui firent partout disparaître les lois personnelles des peuples barbares.
\subsubsection[{Chapitre VI. Comment le droit romain se conserva dans le domaine des Lombards}]{Chapitre VI. Comment le droit romain se conserva dans le domaine des Lombards}
\noindent Tout se plie à mes principes. La loi des Lombards était impartiale, et les Romains n’eurent aucun intérêt à quitter la leur pour la prendre. Le motif qui engagea les Romains sous les Francs à choisir la loi salique, n’eut point de lieu en Italie ; le droit romain s’y maintint avec la loi des Lombards.\par
Il arriva même que celle-ci céda au droit romain ; elle cessa d’être la loi de la nation dominante ; et, quoiqu’elle continuât d’être celle de la principale noblesse, la plupart des villes s’érigèrent en républiques, et cette noblesse tomba, ou fut exterminée\footnote{Voyez ce que dit Machiavel de la destruction de l’ancienne noblesse de Florence.}. Les citoyens des nouvelles républiques ne furent point portés à prendre une loi qui établissait l’usage du combat judiciaire, et dont les institutions tenaient beaucoup aux coutumes et aux usages de la chevalerie. Le clergé, dès lors si puissant en Italie, vivant presque tout sous la loi romaine, le nombre de ceux qui suivaient la loi des Lombards dut toujours diminuer.\par
D’ailleurs, la loi des Lombards n’avait point cette majesté du droit romain, qui rappelait à l’Italie l’idée de sa domination sur toute la terre ; elle n’en avait pas l’étendue. La loi des Lombards et la loi romaine ne pouvaient plus servir qu’à suppléer aux statuts des villes qui s’étaient érigées en républiques ; or qui pouvait mieux y suppléer, ou la loi des Lombards, qui ne statuait que sur quelques cas, ou la loi romaine, qui les embrassait tous ?
\subsubsection[{Chapitre VII. Comment le droit romain se perdit en Espagne}]{Chapitre VII. Comment le droit romain se perdit en Espagne}
\noindent Les choses allèrent autrement en Espagne. La loi des Wisigoths triompha, et le droit romain s’y perdit. Chaindasuinde\footnote{Il commença à régner en 642.} et Récessuinde\footnote{Nous ne voulons plus être tourmentés par les lois étrangères, ni par les romaines, Loi des Wisigoths, liv. II, tit. I, § 9 et 10.} proscrivirent les lois romaines, et ne permirent pas même de les citer dans les tribunaux. Récessuinde fut encore l’auteur de la loi qui ôtait la prohibition des mariages entre les Goths et les Romains\footnote{{\itshape Ut tam Gotho Romanam quam Romano Gotham matrimonio liceat sociari.} Loi des Wisigoths, liv. III, tit. I, chap. I.}. Il est clair que ces deux lois avaient le même esprit : ce roi voulait enlever les principales causes de séparation qui étaient entre les Goths et les Romains. Or on pensait que rien ne les séparaît plus que la défense de contracter entre eux des mariages, et la permission de vivre sous des lois diverses.\par
Mais, quoique les rois des Wisigoths eussent proscrit le droit romain, il subsista toujours dans les domaines qu’ils possédaient dans la Gaule méridionale. Ces pays, éloignés du centre de la monarchie, vivaient dans une grande indépendance\footnote{Voyez, dans Cassiodore, les condescendances que Théodoric, roi des Ostrogoths, prince le plus accrédité de son temps, eut pour elles (liv. IV, lettres XIX et XXVI).}. On voit par {\itshape l’Histoire de Vamba, qui} monta sur le trône en 672, que les naturels du pays avaient pris le dessus \footnote{La révolte de ces provinces fut une défection générale, comme il paraît par le jugement qui est à la suite de l’{\itshape Histoire.} Paulus et ses adhérents étaient Romains ; ils furent même favorisés par les évêques. Vamba n’osa pas faire mourir les séditieux qu’il avait vaincus. L’auteur de l’{\itshape Histoire} appelle la Gaule Narbonnaise la nourrice de la perfidie.} : ainsi la loi romaine y avait plus d’autorité, et la loi gothe y en avait moins. Les lois espagnoles ne convenaient ni à leurs manières ni à leur situation actuelle : peut-être même que le peuple s’obstina à la loi romaine, parce qu’il y attacha l’idée de sa liberté. Il y a plus : les lois de Chaindasuinde et de Récessuinde contenaient des dispositions effroyables contre les Juifs ; mais ces Juifs étaient puissants dans la Gaule méridionale. L’auteur de {\itshape l’Histoire du roi Vamba} appelle ces provinces le prostibule des Juifs. Lorsque les Sarrasins vinrent dans ces provinces, ils y avaient été appelés : or qui put les y avoir appelés, que les Juifs ou les Romains ? Les Goths furent les premiers opprimés, parce qu’ils étaient la nation dominante. On voit dans Procope\footnote{{\itshape Gothi qui cladi superfuerant, ex Gallia cum uxoribus liberisque egressi, in Hispaniam ad Teudim jam palam tyrannum se receperunt.De bello Gothorum}, liv. I, chap. XIII.} que, dans leurs calamités, ils se retiraient de la Gaule Narbonnaise en Espagne. Sans doute que, dans ce malheur-ci, ils se réfugièrent dans les contrées de l’Espagne qui se défendaient encore ; et le nombre de ceux qui, dans la Gaule méridionale, vivaient sous la loi des Wisigoths, en fut beaucoup diminué.
\subsubsection[{Chapitre VIII. Faux capitulaire}]{Chapitre VIII. Faux capitulaire}
\noindent Ce malheureux compilateur Benoît Lévite n’alla-t-il pas transformer cette loi wisigothe qui défendait l’usage du droit romain, en un capitulaire\footnote{{\itshape Capitulaires}, édition de Baluze, liv. VI, Chap. CCCXLIII, p. 981, t. I.} qu’on attribua depuis à Charlemagne ? Il fit de cette loi particulière une loi générale, comme s’il avait voulu exterminer le droit romain par tout l’univers.
\subsubsection[{Chapitre IX. Comment les codes des lois des Barbares et les Capitulaires se perdirent}]{Chapitre IX. Comment les codes des lois des Barbares et les Capitulaires se perdirent}
\noindent Les lois saliques, ripuaires, bourguignonnes et wisigothes cessèrent peu à peu d’être en usage chez les Français : voici comment.\par
Les fiefs étant devenus héréditaires, et les arrière-fiefs s’étant étendus, il s’introduisit beaucoup d’usages auxquels ces lois n’étaient plus applicables. On en retint bien l’esprit, qui était de régler la plupart des affaires par des amendes. Mais, les valeurs ayant sans doute changé, les amendes changèrent aussi ; et l’on voit beaucoup de chartes\footnote{M. de La Thaumassière en a recueilli plusieurs. Voyez, par exemple, les chapitres LXI, LXVI, et autres.} où les seigneurs fixaient les amendes qui devaient être payées dans leurs petits tribunaux. Ainsi l’on suivit l’esprit de la loi, sans suivre la loi même.\par
D’ailleurs, la France se trouvant divisée en une infinité de petites seigneuries, qui reconnaissaient plutôt une dépendance féodale qu’une dépendance politique, il était bien difficile qu’une seule loi pût être autorisée. En effet, on n’aurait pas pu la faire observer. L’usage n’était guère plus qu’on envoyât des officiers extraordinaires\footnote{Missi dominici.} dans les provinces, qui eussent l’œil sur l’administration de la justice et sur les affaires politiques. Il paraît même, par les chartes, que, lorsque de nouveaux fiefs s’établissaient, les rois se privaient du droit de les y envoyer. Ainsi, lorsque tout à peu près fut devenu fief, ces officiers ne purent plus être employés ; il n’y eut plus de loi commune, parce que personne ne pouvait faire observer la loi commune.\par
Les lois saliques, bourguignonnes et wisigothes furent donc extrêmement négligées à la fin de la seconde race ; et, au commencement de la troisième, on n’en entendit presque plus parler.\par
Sous les deux premières races, on assembla souvent la nation, c’est-à-dire les seigneurs et les évêques : il n’était point encore question des communes. On chercha dans ces assemblées à régler le clergé, qui était un corps qui se formait, pour ainsi dire, sous les conquérants, et qui établissait ses prérogatives. Les lois faites dans ces assemblées sont ce que nous appelons les capitulaires. Il arriva quatre choses : les lois des fiefs s’établirent, et une grande partie des biens de l’Église fut gouvernée par les lois des fiefs ; les ecclésiastiques se séparèrent davantage, et négligèrent des lois de réforme\footnote{« Que les évêques, dit Charles le Chauve, dans le capitulaire de l’an 844, art. 8, sous prétexte qu’ils ont l’autorité de faire des canons, ne s’opposent pas à cette constitution, ni ne la négligent. » Il semble qu’il en prévoyait déjà la chute.} où ils n’avaient pas été les seuls réformateurs ; on recueillit les canons des conciles\footnote{On inséra dans le recueil des canons un nombre infini de décrétales des papes ; il y en avait très peu dans l’ancienne collection. Denys le Petit en mit beaucoup dans la sienne ; mais celle d’Isidore Mercator fut remplie de vraies et de fausses décrétales. L’ancienne collection fut en usage en France jusqu’à Charlemagne. Ce prince reçut des mains du pape Adrien II, la collection de Denys le Petit, et la fit recevoir. La collection d’Isidore Mercator parut en France vers le règne de Charlemagne ; on s’en entêta : ensuite vint ce qu’on appelle le {\itshape Corps du droit canonique.}} et les décrétales des papes ; et le clergé reçut ces lois, comme venant d’une source plus pure. Depuis l’érection des grands fiefs, les rois n’eurent plus, comme j’ai dit, des envoyés dans les provinces pour faire observer des lois émanées d’eux : ainsi, sous la troisième race, on n’entendit plus parler de capitulaires.
\subsubsection[{Chapitre X. Continuation du même sujet}]{Chapitre X. Continuation du même sujet}
\noindent On ajouta plusieurs capitulaires à la loi des Lombards, aux lois saliques, à la loi des Bavarois. On en a cherché la raison ; il faut la prendre dans la chose même. Les capitulaires étaient de plusieurs espèces. Les uns avaient du rapport au gouvernement politique, d’autres au gouvernement économique, la plupart au gouvernement ecclésiastique, quelques-uns au gouvernement civil. Ceux de cette dernière espèce furent ajoutés à la loi civile, c’est-à-dire aux lois personnelles de chaque nation : c’est pour cela qu’il est dit dans les capitulaires qu’on n’y a rien stipulé contre la loi romaine\footnote{Voyez l’édit de Pistes, art. 20.}. En effet, ceux qui regardaient le gouvernement économique, ecclésiastique ou politique n’avaient point de rapport avec cette loi ; et ceux qui regardaient le gouvernement civil n’en eurent qu’aux lois des peuples barbares, que l’on expliquait, corrigeait, augmentait et diminuait. Mais ces capitulaires, ajoutés aux lois personnelles, firent, je crois, négliger le corps même des capitulaires. Dans des temps d’ignorance, l’abrégé d’un ouvrage fait souvent tomber l’ouvrage même.
\subsubsection[{Chapitre XI. Autres causes de la chute des codes des lois des Barbares, du droit romain, et des Capitulaires}]{Chapitre XI. Autres causes de la chute des codes des lois des Barbares, du droit romain, et des Capitulaires}
\noindent Lorsque les nations germaines conquirent l’empire romain, elles y trouvèrent l’usage de l’écriture ; et, à l’imitation des Romains, elles rédigèrent leurs usages par écrit\footnote{Cela est marqué expressément dans quelques prologues de ces codes. On voit même dans les lois des Saxons et des Frisons des dispositions différentes, selon les divers districts. On ajouta à ces usages quelques dispositions particulières que les circonstances exigèrent : telles furent les lois dures contre les Saxons.}, et en firent des codes. Les règnes malheureux qui suivirent celui de Charlemagne, les invasions des Normands, les guerres intestines replongèrent les nations victorieuses dans les ténèbres dont elles étaient sorties ; on ne sut plus lire ni écrire. Cela fit oublier en France et en Allemagne les lois barbares écrites, le droit romain et les capitulaires. L’usage de l’écriture se conserva mieux en Italie, où régnaient les papes et les empereurs grecs, et où il y avait des villes florissantes, et presque le seul commerce qui se fît pour lors. Ce voisinage de l’Italie fit que le droit romain se conserva mieux dans les contrées de la Gaule autrefois soumises aux Goths et aux Bourguignons ; d’autant plus que ce droit y était une loi territoriale et une espèce de privilège. Il y a apparence que c’est l’ignorance de l’écriture qui fit tomber en Espagne les lois wisigothes. Et, par la chute de tant de lois, il se forma partout des coutumes.\par
Les lois personnelles tombèrent. Les compositions et ce que l’on appelait {\itshape freda}\footnote{J’en parlerai ailleurs.} se réglèrent plus par la coutume que par le texte de ces lois. Ainsi, comme dans l’établissement de la monarchie, on avait passé des usages des Germains à des lois écrites ; on revint, quelques siècles après, des lois écrites à des usages non écrits.
\subsubsection[{Chapitre XII. Des coutumes locales ; révolution des lois des peuples barbares et du droit romain}]{Chapitre XII. Des coutumes locales ; révolution des lois des peuples barbares et du droit romain}
\noindent On voit, par plusieurs monuments, qu’il y avait déjà des coutumes locales dans la première et la seconde race. On y parle de la {\itshape coutume du lieu}\footnote{Préface des {\itshape Formules} de Marculfe.}, de {\itshape l’usage ancien}\footnote{Lois des Lombards, liv. II, tit. LVIII, § 3.}, de la {\itshape coutume}\footnote{{\itshape Ibid.}, liv. II, tit. XLI, § 6.}, des lois et des {\itshape coutumes}\footnote{{\itshape Vie de saint Léger}.}. Des auteurs ont cru que ce qu’on nommait des coutumes étaient les lois des peuples barbares, et que ce que l’on appelait la loi était le droit romain. Je prouve que cela ne peut être. Le roi Pépin ordonna que partout où il n’y aurait point de loi, on suivrait la coutume ; mais que la coutume ne serait pas préférée à la loi\footnote{Loi des Lombards, liv. II, tit. XLI, § 6.}. Or, dire que le droit romain eut la préférence sur les codes des lois des barbares, c’est renverser tous les monuments anciens, et surtout ces codes des lois des barbares qui disent perpétuellement le contraire.\par
Bien loin que les lois des peuples barbares fussent ces coutumes, ce furent ces lois mêmes qui, comme lois personnelles, les introduisirent. La loi salique, par exemple, était une loi personnelle ; mais, dans des lieux généralement ou presque généralement habités par des Francs saliens, la loi salique, toute personnelle qu’elle était, devenait, par rapport à ces Francs saliens, une loi territoriale ; et elle n’était personnelle que pour les Francs qui habitaient ailleurs. Or, si dans un lieu où la loi salique était territoriale, il était arrivé que plusieurs Bourguignons, Allemands, ou Romains même, eussent eu souvent des affaires, elles auraient été décidées par les lois de ces peuples ; et un grand nombre de jugements, conformes à quelques-unes de ces lois, aurait dû introduire dans le pays de nouveaux usages. Et cela explique bien la constitution de Pépin. Il était naturel que ces usages pussent affecter les Francs mêmes du lieu, dans les cas qui n’étaient point décidés par la loi salique ; mais il ne l’était pas qu’ils pussent prévaloir sur la loi salique.\par
Ainsi il y avait dans chaque lieu une loi dominante, et des usages reçus qui servaient de supplément à la loi dominante, lorsqu’ils ne la choquaient pas.\par
Il pouvait même arriver qu’ils servissent de supplément à une loi qui n’était point territoriale ; et, pour suivre le même exemple si, dans un lieu où la loi salique était territoriale, un Bourguignon était jugé par la loi des Bourguignons, et que le cas ne se trouvât pas dans le texte de cette loi, il ne faut pas douter que l’on ne jugeât suivant la coutume du lieu.\par
Du temps du roi Pépin, les coutumes qui s’étaient formées avaient moins de force que les lois ; mais bientôt les coutumes détruisirent les lois : et, comme les nouveaux règlements sont toujours des remèdes qui indiquent un mal présent, on peut croire que, du temps de Pépin, on commençait déjà à préférer les coutumes aux lois.\par
Ce que j’ai dit explique comment le droit romain commença, dès les premiers temps, à devenir une loi territoriale, comme on le voit dans l’édit de Pistes ; et comment la loi gothe ne laissa pas d’y être encore en usage, comme il paraît par le synode de Troyes dont j’ai parlé\footnote{Voyez ci-dessus le chap. V.}. La loi romaine était devenue la loi personnelle générale, et la loi gothe la loi personnelle particulière ; et par conséquent la loi romaine était la loi territoriale. Mais comment l’ignorance fit-elle tomber partout les lois personnelles des peuples barbares, tandis que le droit romain subsista, comme loi territoriale, dans les provinces wisigothes et bourguignonnes ? Je réponds que la loi romaine même eut à peu près le sort des autres lois personnelles : sans cela, nous aurions encore le code Théodosien dans les provinces où la loi romaine était loi territoriale, au lieu que nous y avons les lois de Justinien. Il ne resta presque à ces provinces que le nom de pays de droit romain ou de droit écrit, que cet amour que les peuples ont pour leur loi, surtout quand ils la regardent comme un privilège, et quelques dispositions du droit romain retenues pour lors dans la mémoire des hommes. Mais c’en fut assez pour produire cet effet que, quand la compilation de Justinien parut, elle fut reçue dans les provinces du domaine des Goths et des Bourguignons comme loi écrite, au lieu que, dans l’ancien domaine des Francs, elle ne le fut que comme raison écrite.
\subsubsection[{Chapitre XIII. Différence de la loi salique ou des Francs saliens d’avec celle des Francs ripuaires et des autres peuples barbares}]{Chapitre XIII. Différence de la loi salique ou des Francs saliens d’avec celle des Francs ripuaires et des autres peuples barbares}
\noindent La loi salique n’admettait point l’usage des preuves négatives, c’est-à-dire que, par la loi salique, celui qui faisait une demande ou une accusation devait la prouver, et qu’il ne suffisait pas à l’accusé de la nier : ce qui est conforme aux lois de presque toutes les nations du monde.\par
La loi des Francs ripuaires avait tout un autre esprit\footnote{Cela se rapporte à ce que dit Tacite, que les peuples germains avaient des usages communs et des usages particuliers.} ; elle se contentait des preuves négatives ; et celui contre qui on formait une demande ou une accusation, pouvait, dans la plupart des cas, se justifier, en jurant, avec certain nombre de témoins, qu’il n’avait point fait ce qu’on lui imputait. Le nombre des témoins qui devaient jurer\footnote{Loi des Ripuaires, tit. VI, VII, VIII et autres.} augmentait selon l’importance de la chose ; il allait quelquefois à soixante-douze\footnote{{\itshape Ibid.}, tit. XI, XII et XVII.}. Les lois des Allemands, des Bavarois, des Thuringiens, celles des Frisons, des Saxons, des Lombards et des Bourguignons, furent faites sur le même plan que celles des Ripuaires.\par
J’ai dit que la loi salique n’admettait point les preuves négatives. Il y avait pourtant un cas où elle les admettait\footnote{C’est celui où un antrustion, c’est-à-dire un vassal du roi, en qui on supposait une plus grande franchise, était accusé. Voyez le titre LXXVI du {\itshape Pactus legis salicae}.} ; mais, dans ce cas elle ne les admettait point seules et sans le concours des preuves positives. Le demandeur faisait ouïr ses témoins pour établir sa demande\footnote{Voyez le titre LXXVI du {\itshape Pactus legis salicae}.} {\itshape ;} le défendeur faisait ouïr les siens pour se justifier ; et le juge cherchait la vérité dans les uns et dans les autres témoignages\footnote{Comme il se pratique encore aujourd’hui en Angleterre.}. Cette pratique était bien différente de celle des lois ripuaires et des autres lois barbares, où un accusé se justifiait en jurant qu’il n’était point coupable, et en faisant jurer ses parents qu’il avait dit la vérité. Ces lois ne pouvaient convenir qu’à un peuple qui avait de la simplicité et une certaine candeur naturelle. Il fallut même que les législateurs en prévinssent l’abus, comme on le va voir tout à l’heure.
\subsubsection[{Chapitre XIV. Autre différence}]{Chapitre XIV. Autre différence}
\noindent La loi salique ne permettait point la preuve par le combat singulier ; la loi des Ripuaires\footnote{Tit. XXXII ; tit. LVII, § 2 ; tit. LIX, § 4.}, et presque toutes celles des peuples barbares, la recevaient\footnote{Voyez la note suivante.}. Il me paraît que la loi du combat était une suite naturelle, et le remède de la loi qui établissait les preuves négatives. Quand on faisait une demande, et qu’on voyait qu’elle allait être injustement éludée par un serment, que restait-il à un guerrier qui se voyait sur le point d’être confondu, qu’à demander raison du tort qu’on lui faisait, et de l’offre même du parjure\footnote{Cet esprit paraît bien dans la loi des Ripuaires, tit. LIX, § 4 et tit. LXVII, § 5 ; et le capitulaire de Louis le Débonnaire, ajouté à la loi des Ripuaires, de l’an 803, art. 22.} ? La loi salique, qui n’admettait point l’usage des preuves négatives, n’avait pas besoin de la preuve par le combat, et ne la recevait pas ; mais la loi des Ripuaires\footnote{Voyez cette loi.} et celle des autres peuples\footnote{La loi des Frisons, des Lombards, des Bavarois, des Saxons, des Thuringiens et des Bourguignons.} barbares qui admettaient l’usage des preuves négatives, furent forcées d’établir la preuve par le combat.\par
Je prie qu’on lise les deux fameuses dispositions de Gondebaud\footnote{Dans la loi des Bourguignons, tit. VIII, § 1 et 2, sur les affaires criminelles ; et le tit. XLV, qui porte encore sur les affaires civiles. Voyez aussi la loi des Thuringiens, tit. I, § 3 1 ; tit. VII, § 6 et tit. VIII ; et la loi des Allemands, tit. LXXXIX ; la loi des Bavarois, tit. VIII, chap. II, § 6, et chap. III, § 1, et tit. IX, chap. IV, § 4 ; la loi des Frisons, tit. II, § 3, et tit. XIV, § 4 ; la loi des Lombards, liv. I, tit. XXXII, § 3 ; et tit. XXXV, § 1 ; et liv. II, tit. XXXV, § 2.} roi de Bourgogne, sur cette matière ; on verra qu’elles sont tirées de la nature de la chose. Il fallait, selon le langage des lois des barbares, ôter le serment des mains d’un homme qui en voulait abuser.\par
Chez les Lombards, la loi de Rotharis admit des cas où elle voulait que celui qui s’était défendu par un serment, ne pût plus être fatigué par un combat. Cet usage s’étendit\footnote{Voyez ci-dessous le chap. XVIII, à la fin.} : nous verrons dans la suite quels maux il en résulta, et comment il fallut revenir à l’ancienne pratique.
\subsubsection[{Chapitre XV. Réflexion}]{Chapitre XV. {\itshape Réflexion}}
\noindent Je ne dis pas que, dans les changements qui furent faits au code des lois des barbares, dans les dispositions qui y furent ajoutées, et dans le corps des capitulaires, on ne puisse trouver quelque texte où, dans le fait, la preuve du combat ne soit pas une suite de la preuve négative. Des circonstances particulières ont pu, dans le cours de plusieurs siècles, faire établir de certaines lois particulières. Je parle de l’esprit général des lois des Germains, de leur nature et de leur origine ; je parle des anciens usages de ces peuples, indiqués ou établis par ces lois : et il n’est ici question que de cela.
\subsubsection[{Chapitre XVI. De la preuve par l’eau bouillante établie par la loi salique}]{Chapitre XVI. De la preuve par l’eau bouillante établie par la loi salique}
\noindent La loi salique\footnote{Et quelques autres lois des barbares aussi.} admettait l’usage de la preuve par l’eau bouillante ; et comme cette épreuve était fort cruelle, la loi prenait un tempérament pour en adoucir la rigueur\footnote{Tit. LVI.}. Elle permettait à celui qui avait été ajourné pour venir faire la preuve par l’eau bouillante, de racheter sa main, du consentement de sa partie. L’accusateur, moyennant une certaine somme que la loi fixait, pouvait se contenter du serment de quelques témoins, qui déclaraient que l’accusé n’avait pas commis le crime : et c’était un cas particulier de la loi salique, dans lequel elle admettait la preuve négative.\par
Cette preuve était une chose de convention, que la loi souffrait, mais qu’elle n’ordonnait pas. La loi donnait un certain dédommagement à l’accusateur qui voulait permettre que l’accusé se défendît par une preuve négative : il était libre à l’accusateur de s’en rapporter au serment de l’accusé, comme il lui était libre de remettre le tort ou l’injure.\par
La loi donnait un tempérament\footnote{{\itshape Ibid.}, tit. LVI.}, pour qu’avant le jugement, les parties, l’une dans la crainte d’une épreuve terrible, l’autre à la vue d’un petit dédommagement présent, terminassent leurs différends, et finissent leurs haines. On sent bien que cette preuve négative une fois consommée, il n’en fallait plus d’autre, et qu’ainsi la pratique du combat ne pouvait être une suite de cette disposition particulière de la loi salique.
\subsubsection[{Chapitre XVII. Manière de penser de nos pères}]{Chapitre XVII. Manière de penser de nos pères}
\noindent On sera étonné de voir que nos pères fissent ainsi dépendre l’honneur, la fortune et la vie des citoyens, de choses qui étaient moins du ressort de la raison que du hasard, qu’ils employassent sans cesse des preuves qui ne prouvaient point, et qui n’étaient liées ni avec l’innocence, ni avec le crime.\par
Les Germains, qui n’avaient jamais été subjugués\footnote{Cela paraît par ce que dit Tacite : Omnibus idem habitus.}, jouissaient d’une indépendance extrême. Les familles se faisaient la guerre pour des meurtres, des vols, des injures\footnote{Velleius Paterculus, liv. II, chap. CXVIII, dit que les Germains décidaient toutes les affaires par le combat.}. On modifia cette coutume, en mettant ces guerres sous des règles ; elles se firent par ordre et sous les yeux du magistrat\footnote{Voyez les codes des lois des Barbares, et, pour les temps plus modernes, Beaumanoir, sur la {\itshape Coutume de Beauvaisis.}} ; ce qui était préférable à une licence générale de se nuire.\par
Comme aujourd’hui les Turcs, dans leurs guerres civiles, regardent la première victoire comme un jugement de Dieu qui décide ; ainsi, les peuples germains dans leurs affaires particulières, prenaient l’événement du combat pour un arrêt de la Providence, toujours attentive à punir le criminel ou l’usurpateur.\par
Tacite dit que, chez les Germains, lorsqu’une nation voulait entrer en guerre avec une autre, elle cherchait à faire quelque prisonnier qui pût combattre avec un des siens ; et qu’on jugeait par l’événement de ce combat, du succès de la guerre. Des peuples qui croyaient que le combat singulier réglerait les affaires publiques, pouvaient bien penser qu’il pourrait encore régler les différends des particuliers.\par
Gondebaud\footnote{La loi des Bourguignons, chap. XLV.}, roi de Bourgogne, fut de tous les rois celui qui autorisa le plus l’usage du combat. Ce prince rend raison de sa loi dans sa loi même : « C’est, dit-il, afin que nos sujets ne fassent plus de serment sur des faits obscurs, et ne se parjurent point sur des faits certains. » Ainsi, tandis que les ecclésiastiques déclaraient impie la loi qui permettait le combat\footnote{Voyez les {\itshape Œuvres} d’Agobard.}, la loi des Bourguignons regardait comme sacrilège celle qui établissait le serment.\par
La preuve par le combat singulier avait quelque raison fondée sur l’expérience. Dans une nation uniquement guerrière, la poltronnerie suppose d’autres vices ; elle prouve qu’on a résisté à l’éducation qu’on a reçue, et que l’on n’a pas été sensible à l’honneur, ni conduit par les principes qui ont gouverné les autres hommes ; elle fait voir qu’on ne craint point leur mépris, et qu’on ne fait point de cas de leur estime : pour peu qu’on soit bien né, on n’y manquera pas ordinairement de l’adresse qui doit s’allier avec la force, ni de la force qui doit concourir avec le courage ; parce que, faisant cas de l’honneur, on se sera toute sa vie exercé à des choses sans lesquelles on ne peut l’obtenir. De plus, dans une nation guerrière, où la force, le courage et la prouesse sont en honneur, les crimes véritablement odieux sont ceux qui naissent de la fourberie, de la finesse et de la ruse, c’est-à-dire de la poltronnerie.\par
Quant à la preuve par le feu, après que l’accusé avait mis la main sur un fer chaud, ou dans l’eau bouillante, on enveloppait la main dans un sac que l’on cachetait : si, trois jours après, il ne paraissait pas de marque de brûlure, on était déclaré innocent. Qui ne voit que, chez un peuple exercé à manier des armes, la peau rude et calleuse ne devait pas recevoir assez l’impression du fer chaud ou de l’eau bouillante, pour qu’il y parût trois jours après ? Et, s’il y paraissait, c’était une marque que celui qui faisait l’épreuve était un efféminé. Nos paysans, avec leurs mains calleuses, manient le fer chaud comme ils veulent. Et, quant aux femmes, les mains de celles qui travaillaient pouvaient résister au fer chaud. Les dames ne manquaient point de champions pour les défendre\footnote{Voyez Beaumanoir, {\itshape Coutume de Beauvaisis}, chap.{\itshape  LXI.} Voyez aussi la loi des Angles, chap. XIV, où la preuve par l’eau bouillante n’est que subsidiaire.} ; et, dans une nation où il n’y avait point de luxe, il n’y avait guère d’état moyen.\par
Par la loi des Thuringiens\footnote{Tit. XIV.}, une femme accusée d’adultère n’était condamnée à l’épreuve par l’eau bouillante, que lorsqu’il ne se présentait point de champion pour elle ; et la loi des Ripuaires n’admet cette épreuve que lorsqu’on ne trouve pas de témoins pour se justifier\footnote{Chap. XXXI, § 5.}. Mais une femme qu’aucun de ses parents ne voulait défendre, un homme qui ne pouvait alléguer aucun témoignage de sa probité, étaient par cela même déjà convaincus.\par
Je dis donc que, dans les circonstances des temps où la preuve par le combat et la preuve par le fer chaud et l’eau bouillante furent en usage, il y eut un tel accord de ces lois avec les mœurs, que ces lois produisirent moins d’injustices qu’elles ne furent injustes ; que les effets furent plus innocents que les causes ; qu’elles choquèrent plus l’équité qu’elles n’en violèrent les droits ; qu’elles furent plus déraisonnables que tyranniques.
\subsubsection[{Chapitre XVIII. Comment la preuve par le combat s’étendit}]{Chapitre XVIII. Comment la preuve par le combat s’étendit}
\noindent On pourrait conclure de la lettre d’Agobard à Louis le Débonnaire, que la preuve par le combat n’était point en usage chez les Francs, puisque après avoir remontré à ce prince les abus de la loi de Gondebaud, il demande qu’on juge en Bourgogne les affaires par la loi des Francs\footnote{{\itshape Si placeret domino nostro ut eos transferret ad legem Francorum.}}. Mais, comme on sait d’ailleurs que, dans ce temps-là, le combat judiciaire était en usage en France, on a été dans l’embarras. Cela s’explique par ce que j’ai dit : la loi des Francs saliens n’admettait point cette preuve, et celle des Francs ripuaires\footnote{Voyez cette loi, tit. LIX, § 4 ; et tit. LXVII, § 5.} la recevait. Mais, malgré les clameurs des ecclésiastiques, l’usage du combat judiciaire s’étendit tous les jours en France ; et je vais prouver tout à l’heure que ce furent eux-mêmes qui y donnèrent lieu en grande partie.\par
C’est la loi des Lombards qui nous fournit cette preuve. « Il s’était introduit depuis longtemps une détestable coutume (est-il dit dans le préambule de la constitution d’Othon II) ; c’est que, si la charte de quelque héritage était attaquée de faux, celui qui la présentait faisait serment sur les Évangiles qu’elle était vraie ; et, sans aucun jugement préalable, il se rendait propriétaire de l’héritage : ainsi les parjures étaient sûrs d’acquérir\footnote{Loi des Lombards, liv. II, tit. LV, chap. XXXIV.}. » Lorsque l’empereur Othon I\textsuperscript{er} se fit couronner à Rome\footnote{L’an 962.}, le pape Jean XII tenant un concile, tous les seigneurs d’Italie s’écrièrent qu’il fallait que l’empereur fît une loi pour corriger cet indigne abus\footnote{{\itshape Ab Italiae proceribus est proclamatum, ut imperator sanctus, mutata lege, facinus indignum destruere.} Loi des Lombards, liv. II, tit. LV, chap. XXXIV.}. Le pape et l’empereur jugèrent qu’il fallait renvoyer l’affaire au concile qui devait se tenir peu de temps après à Ravenne\footnote{. Il fut tenu en l’an 967, en présence du pape Jean XIII, et de l’empereur Othon I\textsuperscript{er}.}. Là, les seigneurs firent les mêmes demandes, et redoublèrent leurs cris ; mais, sous prétexte de l’absence de quelques personnes, on renvoya encore une fois cette affaire. Lorsque Othon II, et Conrad\footnote{Oncle d’Othon II, fils de Rodolphe, et roi de la Bourgogne Transjurane.} roi de Bourgogne arrivèrent en Italie, ils eurent à Vérone\footnote{L’an 988.} un colloque avec les seigneurs d’Italie\footnote{{\itshape Cum in hoc ab omnibus imperiales aures pulsarentur}. Loi des Lombards, liv. II, tit. LV, chap. XXXIV.} ; et, sur leurs instances réitérées, l’empereur, du consentement de tous, fit une loi qui portait que, quand il y aurait quelque contestation sur des héritages, et qu’une des parties voudrait se servir d’une charte, et que l’autre soutiendrait qu’elle était fausse, l’affaire se déciderait par le combat ; que la même règle s’observerait lorsqu’il s’agirait de matières de fief ; que les églises seraient sujettes à la même loi, et qu’elles combattraient par leurs champions. On voit que la noblesse demanda la preuve par le combat, à cause de l’inconvénient de la preuve introduite dans les églises ; que, malgré les cris de cette noblesse, malgré l’abus qui criait lui-même, et malgré l’autorité d’Othon, qui arriva en Italie pour parler et agir en maître, le clergé tint ferme dans deux conciles ; que le concours de la noblesse et des princes ayant forcé les ecclésiastiques à céder, l’usage du combat judiciaire dut être regardé comme un privilège de la noblesse, comme un rempart contre l’injustice, et une assurance de sa propriété ; et que, dès ce moment, cette pratique dut s’étendre. Et cela se fit dans un temps où les empereurs étaient grands, et les papes petits ; dans un temps où les Othons vinrent rétablir en Italie la dignité de l’empire.\par
Je ferai une réflexion qui confirmera ce que j’ai dit ci-dessus, que l’établissement des preuves négatives entraînait après lui la jurisprudence du combat. L’abus dont on se plaignait devant les Othons, était qu’un homme à qui on objectait que sa charte était fausse, se défendait par une preuve négative, en déclarant sur les Évangiles qu’elle ne l’était pas. Que fit-on pour corriger l’abus d’une loi qui avait été tronquée ? On rétablit l’usage du combat.\par
Je me suis pressé de parler de la constitution d’Othon II, afin de donner une idée claire des démêlés de ces temps-là entre le clergé et les laïques. Il y avait eu auparavant une constitution\footnote{Dans la loi des Lombards, liv. II, tit. LV, § 33. Dans l’exemplaire dont s’est servi M. Muratori, elle est attribuée à l’empereur Guy.} de Lothaire I\textsuperscript{er}, qui, sur les mêmes plaintes et les mêmes démêlés, voulant assurer la propriété des biens, avait ordonné que le notaire jurerait que sa charte n’était pas fausse ; et que, s’il était mort, on ferait jurer les témoins qui l’avaient signée ; mais le mal restait toujours, il fallut en venir au remède dont je viens de parler.\par
Je trouve qu’avant ce temps-là, dans des assemblées générales tenues par Charlemagne, la nation lui représenta\footnote{Dans la loi des Lombards, liv. II, tit. LV, § 23.} que dans l’état des choses, il était très difficile que l’accusateur ou l’accusé ne se parjurassent, et qu’il valait mieux rétablir le combat judiciaire ; ce qu’il fit.\par
L’usage du combat judiciaire s’étendit chez les Bourguignons, et celui du serment y fut borné. Théodoric, roi d’Italie, abolit le combat singulier chez les Ostrogoths\footnote{Voyez Cassiodore, liv. III, lettres XXIII et XXIV.} : les lois de Chaindasuinde et de Recessuinde semblent en avoir voulu ôter jusqu’à l’idée. Mais ces lois furent si peu reçues dans la Narbonnaise, que le combat y était regardé comme une prérogative des Goths\footnote{{\itshape In palatio quoque Bera comes Barcinonensis, cum impeteretur a quodam vocato Sunila, et infidelitatis argueretur, cum eodem secundum legem propriam, utpote quia uterque Gothus erat, equestri proelio congressus est et victus}. L’auteur incertain de la {\itshape Vie de Louis le Débonnaire}.}.\par
Les Lombards, qui conquirent l’Italie après la destruction des Ostrogoths par les Grecs, y rapportèrent l’usage du combat ; mais leurs premières lois le restreignirent\footnote{Voyez, dans la loi des Lombards, le liv. I, tit. IV, et tit. IX, § 23 ; et liv. II, tit. XXXV, § 4 et 5 ; et tit. LV, § 1, 2 et 3 ; les règlements de Rotharis ; et au § 15, celui de Luitprand.}. Charlemagne\footnote{{\itshape Ibid.}, liv. II, tit. IV, § 23.}, Louis le Débonnaire, les Othons, firent diverses constitutions générales, qu’on trouve insérées dans les lois des Lombards, et ajoutées aux lois saliques, qui étendirent le duel, d’abord dans les affaires criminelles, et ensuite dans les civiles. On ne savait comment faire. La preuve négative par le serment avait des inconvénients ; celle par le combat en avait aussi : on changeait suivant qu’on était plus frappé des uns ou des autres.\par
D’un côté, les ecclésiastiques se plaisaient à voir que, dans toutes les affaires séculières, on recourût aux églises et aux autels\footnote{Le serment judiciaire se faisait pour lors dans les églises ; et il y avait dans la première race, dans le palais des rois, une chapelle exprès pour les affaires qui s’y jugeaient. Voyez les Formules de Marculfe, liv. I, chap. XXXVIII ; les lois des Ripuaires, tit. LIX, § 4 et tit. LXV, § 5 ; l’{\itshape Histoire} de Grégoire de Tours ; le capitulaire de l’an 803, ajouté à la loi salique.} ; et, de l’autre, une noblesse fière aimait à soutenir ses droits par son épée.\par
Je ne dis point que ce fût le clergé qui eût introduit l’usage dont la noblesse se plaignait. Cette coutume dérivait de l’esprit des lois des barbares, et de l’établissement des preuves négatives. Mais une pratique qui pouvait procurer l’impunité à tant de criminels, ayant fait penser qu’il fallait se servir de la sainteté des églises pour étonner les coupables et faire pâlir les parjures, les ecclésiastiques soutinrent cet usage et la pratique à laquelle il était joint ; car d’ailleurs ils étaient opposés aux preuves négatives. Nous voyons dans Beaumanoir\footnote{Chap. XXXIX, p. 212.} que ces preuves ne furent jamais admises dans les tribunaux ecclésiastiques ; ce qui contribua sans doute beaucoup à les faire tomber, et à affaiblir la disposition des codes des lois des barbares à cet égard.\par
Ceci fera encore bien sentir la liaison entre l’usage des preuves négatives et celui du combat judiciaire dont j’ai tant parlé. Les tribunaux laïques les admirent l’un et l’autre, et les tribunaux clercs les rejetèrent tous deux.\par
Dans le choix de la preuve par le combat, la nation suivait son génie guerrier ; car pendant qu’on établissait le combat comme un jugement de Dieu, on abolissait les preuves par la croix, l’eau froide et l’eau bouillante, qu’on avait regardées aussi comme des jugements de Dieu.\par
Charlemagne ordonna que, s’il survenait quelque différend entre ses enfants, il fût terminé par le jugement de la croix. Louis le Débonnaire\footnote{On trouve ses constitutions insérées dans la loi des Lombards, et à la suite des lois saliques.} borna ce jugement aux affaires ecclésiastiques ; son fils Lothaire l’abolit dans tous les cas ; il abolit de même la preuve par l’eau froide\footnote{Dans sa constitution insérée dans la loi des Lombards, liv. II, tit. LV, § 31.}.\par
Je ne dis pas que, dans un temps où il y avait si peu d’usages universellement reçus, ces preuves n’aient été reproduites dans quelques églises, d’autant plus qu’une charte\footnote{De l’an 1200.} de Philippe Auguste en fait mention ; mais je dis qu’elles furent de peu d’usage. Beaumanoir\footnote{Coutume de Beauvaisis, chap. XXXIX.}, qui vivait du temps de saint Louis, et un peu après, faisant l’énumération des différents genres de preuves, parle de celle du combat judiciaire, et point du tout de celles-là.
\subsubsection[{Chapitre XIX. Nouvelle raison de l’oubli des lois saliques, des lois romaines et des Capitulaires}]{Chapitre XIX. Nouvelle raison de l’oubli des lois saliques, des lois romaines et des Capitulaires}
\noindent J’ai déjà dit les raisons qui avaient fait perdre aux lois saliques, aux lois romaines et aux capitulaires, leur autorité ; j’ajouterai que la grande extension de la preuve par le combat en fut la principale cause.\par
Les lois saliques, qui n’admettaient point cet usage, devinrent en quelque façon inutiles, et tombèrent : les lois romaines, qui ne l’admettaient pas non plus, périrent de même. On ne songea plus qu’à former la loi du combat judiciaire, et à en faire une bonne jurisprudence. Les dispositions des capitulaires ne devinrent pas moins inutiles. Ainsi tant de lois perdirent leur autorité, sans qu’on puisse citer le moment où elles l’ont perdue ; elles furent oubliées, sans qu’on en trouve d’autres qui aient pris leur place.\par
Une nation pareille n’avait pas besoin de lois écrites, et ses lois écrites pouvaient bien aisément tomber dans l’oubli.\par
Y avait-il quelque discussion entre deux parties ? On ordonnait le combat. Pour cela, il ne fallait pas beaucoup de suffisance.\par
Toutes les actions civiles et criminelles se réduisent en faits. C’est sur ces faits que l’on combattait ; et ce n’était pas seulement le fond de l’affaire qui se jugeait par le combat, mais encore les incidents et les interlocutoires comme le dit Beaumanoir\footnote{Chap. LXI, p. 309 et 310.}, qui en donne des exemples.\par
Je trouve qu’au commencement de la troisième race la jurisprudence était toute en procédés ; tout fut gouverné par le point d’honneur. Si l’on n’avait pas obéi au juge, il poursuivait son offense. À Bourges\footnote{Charte de Louis le Gros, de l’an 1145, dans le recueil des ordonnances.}, si le prévôt avait mandé quelqu’un, et qu’il ne fût pas venu : « Je t’ai envoyé chercher, disait-il ; tu as dédaigné de venir ; fais-moi raison de ce mépris » ; et l’on combattait. Louis le Gros réforma cette coutume\footnote{{\itshape Ibid.}}.\par
Le combat judiciaire était en usage à Orléans dans toutes les demandes de dettes\footnote{Charte de Louis le Jeune, de l’an 1168, dans le recueil des ordonnances.}. Louis le Jeune déclara que cette coutume n’aurait lieu que lorsque la demande excéderait cinq sols. Cette ordonnance était une loi locale ; car, du temps de saint Louis\footnote{Voyez Beaumanoir, chap. LXIII, p. 325.}, il suffisait que la valeur fût de plus de douze deniers. Beaumanoir\footnote{Voyez la {\itshape Coutume de Beauvaisis}, chap. XXVIII, p. 203.} avait ouï dire à un seigneur de loi qu’il y avait autrefois en France cette mauvaise coutume, qu’on pouvait louer pendant un certain temps un champion pour combattre dans ses affaires. Il fallait que l’usage du combat judiciaire eût, pour lors, une prodigieuse extension.
\subsubsection[{Chapitre XX. Origine du point d’honneur}]{Chapitre XX. Origine du point d’honneur}
\noindent On trouve des énigmes dans les codes des lois des barbares. La loi des Frisons ne donne qu’un demi-sol de composition à celui qui a reçu des coups de bâton\footnote{{\itshape Additio sapientium Wilemari}, tit. V.} ; et il n’y a si petite blessure pour laquelle elle n’en donne davantage. Par la loi salique, si un ingénu donnait trois coups de bâton à un ingénu, il payait trois sols ; s’il avait fait couler le sang, il était puni comme s’il avait blessé avec le fer, et il payait quinze sols : la peine se mesurait par la grandeur des blessures. La loi des Lombards\footnote{Liv. I, tit. VI, § 3.} établit différentes compositions pour un coup, pour deux, pour trois, pour quatre. Aujourd’hui un coup en vaut cent mille.\par
La constitution de Charlemagne, insérée dans la loi des Lombards\footnote{Liv. II, tit. V, § 23.}, veut que ceux à qui elle permet le duel combattent avec le bâton. Peut-être que ce fut un ménagement pour le clergé ; peut-être que, comme on étendait l’usage des combats, on voulut les rendre moins sanguinaires. Le capitulaire de Louis le Débonnaire\footnote{Ajouté à la loi salique sur l’an 819.} donne le choix de combattre avec le bâton ou avec les armes. Dans la suite il n’y eut que les serfs qui combattissent avec le bâton\footnote{Voyez Beaumanoir, chap. LXIV, p. 323.}.\par
Déjà je vois naître et se former les articles particuliers de notre point d’honneur. L’accusateur commençait par déclarer devant le juge qu’un tel avait commis une telle action ; et celui-ci répondait qu’il en avait menti\footnote{{\itshape Ibid.}, p. 329.} {\itshape ;} sur cela, le juge ordonnait le duel. La maxime s’établit que, lorsqu’on avait reçu un démenti, il fallait se battre.\par
Quand un homme avait déclaré qu’il combattrait, il ne pouvait plus s’en départir ; et s’il le faisait, il était condamné à une peine\footnote{Voyez Beaumanoir, chap. III, p. 25 et 329.}. De là suivit cette règle que, quand un homme s’était engagé par sa parole, l’honneur ne lui permettait plus de la rétracter.\par
Les gentilshommes se battaient entre eux à cheval et avec leurs armes\footnote{Voyez, sur les armes des combattants, Beaumanoir, chap. LXI, p. 308, et chap. LXIV, p. 328.} ; et les vilains se battaient à pied et avec le bâton\footnote{{\itshape Ibid.}, chap. LXIV, p. 328. Voyez aussi les chartes de saint Aubin d’Anjou, rapportées par Galland, p. 263.}. De là il suivit que le bâton était l’instrument des outrages\footnote{Chez les Romains, les coups de bâton n’étaient point infâmes. {\itshape Lege Ictus fustium}. De iis qui notantur infamia.}, parce qu’un homme qui en avait été battu, avait été traité comme un vilain.\par
Il n’y avait que les vilains qui combattissent à visage découvert\footnote{Ils n’avaient que l’écu et le bâton. Beaumanoir, chap. LXIV, p. 328.} ; ainsi il n’y avait qu’eux qui pussent recevoir des coups sur la face. Un soufflet devint une injure qui devait être lavée par le sang, parce qu’un homme qui l’avait reçu avait été traité comme un vilain.\par
Les peuples germains n’étaient pas moins sensibles que nous au point d’honneur ; ils l’étaient même plus. Ainsi les parents les plus éloignés prenaient une part très vive aux injures ; et tous leurs codes sont fondés là-dessus. La loi des Lombards veut que celui qui, accompagné de ses gens, va battre un homme qui n’est point sur ses gardes, afin de le couvrir de honte et de ridicule, paie la moitié de la composition qu’il aurait due s’il l’avait tué\footnote{Liv. I, tit. VI, § 1.} ; et que si, par le même motif, il le lie, il paie les trois quarts de la même composition\footnote{{\itshape Ibid.}, § 2.}.\par
Disons donc que nos pères étaient extrêmement sensibles aux affronts ; mais que les affronts d’une espèce particulière, de recevoir des coups d’un certain instrument sur une certaine partie du corps, et donnés d’une certaine manière, ne leur étaient pas encore connus. Tout cela était compris dans l’affront d’être battu ; et, dans ce cas, la grandeur des excès faisait la grandeur des outrages.
\subsubsection[{Chapitre XXI. Nouvelle réflexion sur le point d’honneur chez les Germains}]{Chapitre XXI. Nouvelle réflexion sur le point d’honneur chez les Germains}
\noindent « C’était chez les Germains, dit Tacite\footnote{{\itshape De moribus Germanorum}.} une grande infamie d’avoir abandonné son bouclier dans le combat ; et plusieurs, après ce malheur, s’étaient donné la mort. » Aussi l’ancienne loi salique donne-t-elle quinze sols de composition à celui à qui on avait dit par injure qu’il avait abandonné son bouclier\footnote{Dans le {\itshape Pactus legis salicae}.}.\par
Charlemagne, corrigeant la loi salique\footnote{Nous avons l’ancienne loi, et celle qui fut corrigée par ce prince.}, n’établit dans ce cas que trois sols de composition. On ne peut pas soupçonner ce prince d’avoir voulu affaiblir la discipline militaire : il est clair que ce changement vint de celui des armes ; et c’est à ce changement des armes que l’on doit l’origine de bien des usages.
\subsubsection[{Chapitre XXII. Des mœurs relatives aux combats}]{Chapitre XXII. Des mœurs relatives aux combats}
\noindent Notre liaison avec les femmes est fondée sur le bonheur attaché au plaisir des sens, sur le charme d’aimer et d’être aimé, et encore sur le désir de leur plaire, parce que ce sont des juges très éclairés sur une partie des choses qui constituent le mérite personnel. Ce désir général de plaire produit la galanterie, qui n’est point J’amour, mais {\itshape le} délicat, mais le léger, mais {\itshape le} perpétuel mensonge de l’amour.\par
Selon les circonstances différentes dans chaque nation et dans chaque siècle, l’amour se porte plus vers une de ces trois choses que vers les deux autres. Or je dis que, dans le temps de nos combats, ce fut l’esprit de galanterie qui dut prendre des forces.\par
Je trouve, dans la loi des Lombards\footnote{Liv. II, tit. LV, § 11.}, que, si un des champions avait sur lui des herbes propres aux enchantements, le juge les lui faisait ôter, et le faisait jurer qu’il n’en avait plus. Cette loi ne pouvait être fondée que sur l’opinion commune ; c’est la peur, qu’on a dit avoir inventé tant de choses, qui fit imaginer ces sortes de prestiges. Comme, dans les combats particuliers, les champions étaient armés de toutes pièces, et qu’avec des armes pesantes, offensives et défensives, celles d’une certaine trempe et d’une certaine force donnaient des avantages infinis, l’opinion des armes enchantées de quelques combattants dut tourner la tête à bien des gens.\par
De là naquit le système merveilleux de la chevalerie. Tous les esprits s’ouvrirent à ces idées. On vit, dans les romans, des paladins, des nécromants, des fées, des chevaux ailés ou intelligents, des hommes invisibles ou invulnérables, des magiciens qui s’intéressaient à la naissance ou à l’éducation des grands personnages, des palais enchantés et désenchantés ; dans notre monde, un monde nouveau ; et le cours ordinaire de la nature laissé seulement pour les hommes vulgaires.\par
Des paladins, toujours armés dans une partie du monde pleine de châteaux, de forteresses et de brigands, trouvaient de l’honneur à punir l’injustice et à défendre la faiblesse. De là encore, dans nos romans, la galanterie fondée sur l’idée de l’amour, jointe à celle de force et de protection.\par
Ainsi naquit la galanterie, lorsqu’on imagina des hommes extraordinaires, qui, voyant la vertu jointe à la beauté et à la faiblesse, furent portés à s’exposer pour elle dans les dangers, et à lui plaire dans les actions ordinaires de la vie.\par
Nos romans de chevalerie flattèrent ce désir de plaire, et donnèrent à une partie de l’Europe cet esprit de galanterie que l’on peut dire avoir été peu connu par les anciens.\par
Le luxe prodigieux de cette immense ville de Rome flatta l’idée des plaisirs des sens. Une certaine idée de tranquillité dans les campagnes de la Grèce fit décrire les sentiments de l’amour\footnote{On peut voir les romans grecs du moyen âge.}.\par
L’idée des paladins, protecteurs de la vertu et de la beauté des femmes, conduisit à celle de la galanterie.\par
Cet esprit se perpétua par l’usage des tournois, qui, unissant ensemble les droits de la valeur et de l’amour, donnèrent encore à la galanterie une grande importance.
\subsubsection[{Chapitre XXIII. De la jurisprudence du combat judiciaire}]{Chapitre XXIII. De la jurisprudence du combat judiciaire}
\noindent On aura peut-être de la curiosité à voir cet usage monstrueux du combat judiciaire réduit en principes, et à trouver le corps d’une jurisprudence si singulière. Les hommes, dans le fond raisonnables, mettent sous des règles leurs préjugés mêmes. Rien n’était plus contraire au bon sens que le combat judiciaire ; mais, ce point une fois posé, l’exécution s’en fit avec une certaine prudence.\par
Pour se mettre bien au fait de la jurisprudence de ces temps-là, il faut lire avec attention les règlements de saint Louis, qui fit de si grands changements dans l’ordre judiciaire. Desfontaines était contemporain de ce prince ; Beaumanoir écrivait après lui\footnote{En l’an 1283.} ; les autres ont vécu depuis lui. il faut donc chercher l’ancienne pratique dans les corrections qu’on en a faites.
\subsubsection[{Chapitre XXIV. Règles établies dans le combat judiciaire}]{Chapitre XXIV. Règles établies dans le combat judiciaire}
\noindent Lorsqu’il y avait plusieurs accusateurs\footnote{Beaumanoir, chap. VI, p. 40 et 41.} il fallait qu’ils s’accordassent pour que l’affaire fût poursuivie par un seul ; et s’ils ne pouvaient convenir, celui devant qui se faisait le plaid nommait un d’entre eux qui poursuivait la querelle.\par
Quand un gentilhomme appelait un vilain\footnote{{\itshape Ibid.}, chap. LXIV, p. 328.}, il devait se présenter à pied, et avec l’écu et le bâton ; et, s’il venait à cheval, et avec les armes d’un gentilhomme, on lui ôtait son cheval et ses armes ; il restait en chemise, et était obligé de combattre en cet état contre le vilain.\par
Avant le combat, la justice faisait publier trois bans\footnote{{\itshape Ibid.}, p. 330.}. Par l’un, il était ordonné aux parents des parties de se retirer ; par l’autre, on avertissait le peuple de garder le silence ; par le troisième, il était défendu de donner du secours à une des parties, sous de grosses peines, et même celle de mort, si, par ce secours, un des combattants avait été vaincu.\par
Les gens de justice gardaient le parc\footnote{{\itshape Ibid.}} ; et dans le cas où une des parties aurait parlé de paix, ils avaient grande attention à l’état où elles se trouvaient toutes les deux dans ce moment, pour qu’elles fussent remises dans la même situation, si la paix ne se faisait pas\footnote{{\itshape Ibid.}}.\par
Quand les gages étaient reçus pour crime ou pour faux jugement, la paix ne pouvait se faire sans le consentement du seigneur ; et, quand une des parties avait été vaincue, il ne pouvait plus y avoir de paix que de l’aveu du comte\footnote{Les grands vassaux avaient des droits particuliers.} ; ce qui avait du rapport à nos lettres de grâce.\par
Mais si le crime était capital, et que le seigneur, corrompu par des présents, consentît à la paix, il payait une amende de soixante livres, et le droit qu’il avait de faire punir le malfaiteur, était dévolu au comte\footnote{Beaumanoir, chap. LXIV, p. 330, dit : il {\itshape perdrait sa justice.} Ces paroles, dans les auteurs de ces temps-là, n’ont pas une signification générale, mais restreinte à l’affaire dont il s’agit : Desfontaines, chap. XXI, art 29.}.\par
Il y avait bien des gens qui n’étaient en état d’offrir le combat, ni de le recevoir. On permettait, en connaissance de cause, de prendre un champion ; et pour qu’il eût le plus grand intérêt à défendre sa partie, il avait le poing coupé s’il était vaincu\footnote{Cet usage, que l’on trouve dans les capitulaires, subsistait du temps de Beaumanoir : voyez le chap. {\itshape LXI}, p. 315.}.\par
Quand on a fait dans le siècle passé des lois capitales contre les duels, peut-être aurait-il suffi d’ôter à un guerrier sa qualité de guerrier par la perte de la main, n’y ayant rien ordinairement de plus triste pour les hommes que de survivre à la perte de leur caractère.\par
Lorsque, dans un crime capital\footnote{Beaumanoir, chap. XLIV, p. 330.}, le combat se faisait par champions, on mettait les parties dans un lieu d’où elles ne pouvaient voir la bataille : chacune d’elles était ceinte de la corde qui devait servir à son supplice, si son champion était vaincu.\par
Celui qui succombait dans le combat ne perdait pas toujours la chose contestée. Si, par exemple, l’on combattait sur un interlocutoire, l’on ne perdait que l’interlocutoire\footnote{{\itshape Ibid.}, chap. LXI, p. 309.}.
\subsubsection[{Chapitre XXV. Des bornes que l’on mettait à l’usage du combat judiciaire}]{Chapitre XXV. Des bornes que l’on mettait à l’usage du combat judiciaire}
\noindent Quand les gages de bataille avaient été reçus sur une affaire civile de peu d’importance, le seigneur obligeait les parties à les retirer.\par
Si un fait était notoire\footnote{Beaumanoir, chap. LXI, p. 308. {\itshape Ibid.}, chap. XLIII, p. 239.} ; par exemple, si un homme avait été assassiné en plein marché, on n’ordonnait ni la preuve par témoins ni la preuve par le combat ; le juge prononçait sur la publicité.\par
Quand, dans la cour du seigneur, on avait souvent jugé de la même manière, et qu’ainsi l’usage était connu\footnote{{\itshape Ibid.}, chap. LXI, p. 314. Voyez aussi Desfontaines, chap. XXII, art. 24.}, le seigneur refusait le combat aux parties, afin que les coutumes ne fussent pas changées par les divers événements des combats.\par
On ne pouvait demander le combat que pour soi, ou pour quelqu’un de son lignage, ou pour son seigneur-lige\footnote{Beaumanoir, chap. LXIII, p. 322.}.\par
Quand un accusé avait été absous\footnote{{\itshape Ibid.}}, un autre parent ne pouvait demander le combat ; autrement les affaires n’auraient point eu de fin.\par
Si celui dont les parents voulaient venger la mort venait à reparaître, il n’était plus question du combat : il en était de même si, par une absence notoire, le fait se trouvait impossible\footnote{{\itshape Ibid.}}.\par
Si un homme qui avait été tué\footnote{{\itshape Ibid.}, p. 323.} avait, avant de mourir, disculpé celui qui était accusé, et qu’il eût nommé un autre, on ne procédait point au combat ; mais s’il n’avait nommé personne, on ne regardait sa déclaration que comme un pardon de sa mort : on continuait les poursuites ; et même, entre gentilshommes, on pouvait faire la guerre.\par
Quand il y avait une guerre, et qu’un des parents donnait ou recevait les gages de bataille, le droit de la guerre cessait ; on pensait que les parties voulaient suivre le cours ordinaire de la justice ; et celle qui aurait continué la guerre aurait été condamnée à réparer les dommages.\par
Ainsi la pratique du combat judiciaire avait cet avantage, qu’elle pouvait changer une querelle générale en une querelle particulière, rendre la force aux tribunaux, et remettre dans l’état civil ceux qui n’étaient plus gouvernés que par le droit des gens.\par
Comme il y a une infinité de choses sages qui sont menées d’une manière très folle, il y a aussi des folies qui sont conduites d’une manière très sage.\par
Quand un homme appelé pour un crime\footnote{Beaumanoir, chap. LXIII, p. 324.} montrait visiblement que c’était l’appelant même qui l’avait commis, il n’y avait plus de gages de bataille ; car il n’y a point de coupable qui n’eût préféré un combat douteux à une punition certaine.\par
Il n’y avait point de combat dans les affaires qui se décidaient par des arbitres ou par les cours ecclésiastiques\footnote{{\itshape Ibid.}, p. 325.} ; il n’y en avait pas non plus lorsqu’il s’agissait du douaire des femmes.\par
{\itshape Femme}, dit Beaumanoir, {\itshape ne se peut combattre}. Si une femme appelait quelqu’un sans nommer son champion, on ne recevait point les gages de bataille. Il fallait encore qu’une femme fût autorisée par son baron\footnote{{\itshape Ibid.}}, c’est-à-dire son mari, pour appeler ; mais sans cette autorité elle pouvait être appelée.\par
Si l’appelant ou l’appelé avaient moins de quinze ans\footnote{{\itshape Ibid.}, p. 323. Voyez aussi ce que j’ai dit au liv. XVIII.}, il n’y avait point de combat. On pouvait pourtant l’ordonner dans les affaires de pupilles, lorsque le tuteur ou celui qui avait la baillie, voulait courir les risques de cette procédure.\par
Il me semble que voici les cas où il était permis au serf de combattre. Il combattait contre un autre serf ; il combattait contre une personne franche, et même contre un gentilhomme, s’il était appelé ; mais s’il l’appelait\footnote{{\itshape Ibid.}, chap. LXIII, p. 322.}, celui-ci pouvait refuser le combat ; et même le seigneur du serf était en droit de le retirer de la cour. Le serf pouvait, par une charte du seigneur\footnote{Desfontaines, chap. XXII, art. 7.}, ou par usage, combattre contre toutes personnes franches ; et l’Église prétendait ce même droit pour ses serfs\footnote{{\itshape Habeant bellandi et testificandi licentiam}. Charte de Louis le Gros, de l’an 1118.}, comme une marque de respect pour elle\footnote{{\itshape Ibid.}}.
\subsubsection[{Chapitre XXVI. Du combat judiciaire entre une des parties et un des témoins}]{Chapitre XXVI. Du combat judiciaire entre une des parties et un des témoins}
\noindent BEAUMANOIR\footnote{Chap. LXI, p. 315.} dit qu’un homme qui voyait qu’un témoin allait déposer contre lui pouvait éluder le second, en disant aux juges que sa partie produisait un témoin faux et calomniateur\footnote{Leur doit-on demander, avant qu’ils fassent nul serment, pour qui ils veulent témoigner ; {\itshape car l’enques gist li point d’aus lever de faux témoignage}. Beaumanoir, chap. XXXIX, p. 218.} ; et, si le témoin voulait soutenir la querelle, il donnait les gages de bataille. Il n’était plus question de l’enquête : car, si le témoin était vaincu, il était décidé que la partie avait produit un faux témoin, et elle perdait son procès.\par
Il ne fallait pas laisser jurer le second témoin ; car il aurait prononcé son témoignage, et l’affaire aurait été finie par la déposition des deux témoins. Mais en arrêtant le second, la déposition du premier devenait inutile.\par
Le second témoin étant ainsi rejeté, la partie n’en pouvait faire ouïr d’autres, et elle perdait son procès ; mais, dans le cas où il n’y avait point de gages de bataille\footnote{Beaumanoir, chap. LXI, p. 316.}, on pouvait produire d’autres témoins.\par
Beaumanoir dit\footnote{Chap, VI, p. 39 et 40.} que le témoin pouvait dire à sa partie avant de déposer : « Je ne me bée pas à combattre pour votre querelle, ni à entrer en plaid au mien ; mais si vous me voulez défendre, volontiers dirai ma vérité. » La partie se trouvait obligée à combattre pour le témoin ; et, si elle était vaincue, elle ne perdait point le corps\footnote{Mais si le combat se faisait par champions, le champion vaincu avait le poing coupé.}. mais le témoin était rejeté.\par
Je crois que ceci était une modification de l’ancienne coutume ; et ce qui me le fait penser, c’est que cet usage d’appeler les témoins se trouve établi dans la loi des Bavarois\footnote{Tit. XVI, § 2.} et dans celle des Bourguignons\footnote{Tit. XLV.}, sans aucune restriction.\par
J’ai déjà parlé de la constitution de Gondebaud, contre laquelle Agobard\footnote{Lettres à Louis de Débonnaire.} et saint Avit\footnote{{\itshape Vie de saint Avit}.} se récrièrent tant. « Quand l’accusé, dit ce prince, présente des témoins pour jurer qu’il n’a pas commis le crime, l’accusateur pourra appeler au combat un des témoins ; car il est juste que celui qui a offert de jurer, et qui a déclaré qu’il savait la vérité, ne fasse point de difficulté de combattre pour la soutenir. » Ce roi ne laissait aux témoins aucun subterfuge pour éviter le combat.
\subsubsection[{Chapitre XXVII. Du combat judiciaire entre une partie et un des pairs du seigneur. Appel de faux jugement}]{Chapitre XXVII. Du combat judiciaire entre une partie et un des pairs du seigneur. Appel de faux jugement}
\noindent La nature de la décision par le combat étant de terminer l’affaire pour toujours, et n’étant point compatible avec un nouveau jugement et de nouvelles poursuites\footnote{« {\itshape Car en la cour où l’on va par la raison de l’appel pour les gages maintenir, se bataille est faite, la querelle est venue à fin, si que il n’y a métier de plus d’apiaux.} » Beaumanoir, chap. II, p. 22.}, l’appel, tel qu’il est établi par les lois romaines et par les lois canoniques, c’est-à-dire à un tribunal supérieur, pour faire réformer le jugement d’un autre, était inconnu en France.\par
Une nation guerrière, uniquement gouvernée par le point d’honneur, ne connaissait pas cette forme de procéder ; et, suivant toujours le même esprit, elle prenait contre les juges les voies qu’elle aurait pu employer contre les parties\footnote{{\itshape Ibid.}, chap. LXI, p. 312, et chap. {\itshape LXVII}, p. 338.}.\par
L’appel, chez cette nation, était un défi à un combat par armes, qui devait se terminer par le sang ; et non pas cette invitation à une querelle de plume qu’on ne connut qu’après.\par
Aussi saint Louis dit-il, dans ses {\itshape Établissements}\footnote{Liv. II, chap. XV.}, que l’appel contient félonie et iniquité. Aussi Beaumanoir nous dit-il que, si un homme voulait se plaindre de quelque attentat commis contre lui par son seigneur\footnote{Beaumanoir, chap. LXI, p. 310 et 311 ; et chap. LXVII, p. 337.}, il devait lui dénoncer qu’il abandonnait son fief ; après quoi il l’appelait devant son seigneur suzerain, et offrait les gages de bataille. De même, le seigneur renonçait à l’hommage, s’il appelait son homme devant le comte.\par
Appeler son seigneur de faux jugement, c’était dire que son jugement avait été faussement et méchamment rendu : or, avancer de telles paroles contre son seigneur, c’était commettre une espèce de crime de félonie.\par
Ainsi, au lieu d’appeler pour faux jugement le seigneur qui établissait et réglait le tribunal, on appelait les pairs qui formaient le tribunal même ; on évitait par là le crime de félonie ; on n’insultait que ses pairs, à qui on pouvait toujours faire raison de l’insulte.\par
On s’exposait beaucoup en faussant le jugement des pairs\footnote{Beaumanoir, chap. LXI, p. 313.}. Si l’on attendait que le jugement fût fait et prononcé, on était obligé de les combattre tous, lorsqu’ils offraient de faire le jugement bon\footnote{{\itshape Ibid.}, p. 314.}. Si l’on appelait avant que tous les juges eussent donné leur avis, il fallait combattre tous ceux qui étaient convenus du même avis\footnote{Qui s’étaient accordés au jugement.}. Pour éviter ce danger, on suppliait le seigneur\footnote{Beaumanoir, chap. LXI, p. 314.} d’ordonner que chaque pair dît tout haut son avis ; et, lorsque le premier avait prononcé, et que le second allait en faire de même, on lui disait qu’il était faux, méchant et calomniateur ; et ce n’était plus que contre lui qu’on devait se battre.\par
Desfontaines\footnote{Chap. XXII, art. 1, 10 et 11. Il dit seulement qu’on leur payait à chacun une amende.} voulait qu’avant de fausser\footnote{Appeler de faux jugement.}, on laissât prononcer trois juges ; et il ne dit point qu’il fallût les combattre tous trois, et encore moins qu’il y eût des cas où il fallût combattre tous ceux qui s’étaient déclarés pour leur avis. Ces différences viennent de ce que, dans ces temps-là, il n’y avait guère d’usages qui fussent précisément les mêmes. Beaumanoir rendait compte de ce qui se passait dans le comté de Clermont, Desfontaines de ce qui se pratiquait en Vermandois.\par
Lorsqu’un des pairs ou homme de fief avait déclaré qu’il soutiendrait le jugement\footnote{Beaumanoir, chap. LXI, p. 314.}, le juge faisait donner les gages de bataille, et de plus prenait sûreté de l’appelant qu’il soutiendrait son appel. Mais le pair qui était appelé ne donnait point de sûretés, parce qu’il était homme du seigneur, et devait défendre l’appel, ou payer au seigneur une amende de soixante livres.\par
Si celui qui appelait ne prouvait pas que le jugement fût mauvais, il payait au seigneur une amende de soixante livres\footnote{{\itshape Id., ibid.} Desfontaines, chap. XXII, art. 9.}, la même amende au pair qu’il avait appelé\footnote{Desfontaines, {\itshape ibid.}}, autant à chacun de ceux qui avaient ouvertement consenti au jugement.\par
Quand un homme violemment soupçonné d’un crime qui méritait la mort, avait été pris et condamné, il ne pouvait appeler de faux jugement\footnote{Beaumanoir, chap. LXI, p. 316 ; et Desfontaines, chap. XXII, art. 21.} : car il aurait toujours appelé, ou pour prolonger sa vie, ou pour faire la paix.\par
Si quelqu’un disait que le jugement était faux et mauvais\footnote{Beaumanoir, chap. LXI, p. 314.}, et n’offrait pas de le faire tel, c’est-à-dire de combattre, il était condamné à dix sols d’amende s’il était gentilhomme, et à cinq sols s’il était serf, pour les vilaines paroles qu’il avait dites.\par
Les juges ou pairs qui avaient été vaincus\footnote{Desfontaines, chap. XXII, art. 7.} ne devaient perdre ni la vie ni les membres ; mais celui qui les appelait était puni de mort, lorsque l’affaire était capitale\footnote{Voyez Desfontaines, chap. XXI, art. 11, 12 et suivants, qui distingue les cas où le fausseur perdait la vie, la chose contestée, ou seulement l’interlocutoire.}.\par
Cette manière d’appeler les hommes de fief pour faux jugement était pour éviter d’appeler le seigneur même. Mais si le seigneur n’avait point de pair\footnote{Beaumanoir, chap. LXII, p. 322. Desfontaines, chap. XXII, art. 3.}, ou n’en avait pas assez, il pouvait, à ses frais emprunter des pairs de son seigneur suzerain\footnote{Le comte n’était pas obligé d’en prêter. Beaumanoir, chap. LXVII, p. 337.} ; mais ces pairs n’étaient point obligés de juger, s’ils ne le voulaient ; ils pouvaient déclarer qu’ils n’étaient venus que pour donner leur conseil ; et, dans ce cas particulier\footnote{« Nul ne peut faire jugement en sa cour », dit Beaumanoir, chap. LXVII, p. 336 et 337.}, le seigneur jugeant et prononçant lui-même le jugement, si on appelait contre lui de faux jugement, c’était à lui à soutenir l’appel.\par
Si le seigneur était si pauvre\footnote{{\itshape Ibid.}, chap. LXII, p. 322.} qu’il ne fût pas en état de prendre des pairs de son seigneur suzerain, ou qu’il négligeât de lui en demander, ou que celui-ci refusât de lui en donner, le seigneur ne pouvant pas juger seul, et personne n’étant obligé de plaider devant un tribunal où l’on ne peut faire jugement, l’affaire était portée à la cour du seigneur suzerain.\par
Je crois que ceci fut une des grandes causes de la séparation de la justice d’avec le fief, d’où s’est formée la règle des jurisconsultes français : {\itshape Autre chose est le fief, autre chose est la justice}. Car y ayant une infinité d’hommes de fief qui n’avaient point d’hommes sous eux, ils ne furent point en état de tenir leur cour ; toutes les affaires furent portées à la cour de leur seigneur suzerain ; ils perdirent le droit de justice, parce qu’ils n’eurent ni le pouvoir ni la volonté de le réclamer.\par
Tous les juges qui avaient été du jugement\footnote{Desfontaines, chap. XXI, art. 27 et 28.} devaient être présents quand on le rendait, afin qu’ils pussent ensuivre et dire {\itshape oïl} à celui qui, voulant fausser, leur demandait s’ils ensuivaient ; car, dit Desfontaines\footnote{{\itshape Ibid.}, art. 28.}, « c’est une affaire de courtoisie et de loyauté, et il n’y a point là de fuite ni de remise ». Je crois que c’est de cette manière de penser qu’est venu l’usage que l’on suit encore aujourd’hui en Angleterre, que tous les jurés soient de même avis pour condamner à mort.\par
Il fallait donc se déclarer pour l’avis de la plus grande partie ; et, s’il y avait partage, on prononçait, en cas de crime, pour l’accusé ; en cas de dettes, pour le débiteur ; en cas d’héritages, pour le défendeur.\par
Un pair, dit Desfontaines\footnote{Chap. XXI, art. 37.}, ne pouvait pas dire qu’il ne jugerait pas s’ils n’étaient que quatre\footnote{Il fallait ce nombre au moins. Desfontaines, chap. XXI, art. 36.}, {\itshape ou} s’ils n’y étaient tous, ou si les plus sages n’y étaient ; c’est comme s’il avait dit, dans la mêlée, qu’il ne secourrait pas son seigneur, parce qu’il n’avait auprès de lui qu’une partie de ses hommes. Mais c’était au seigneur à faire honneur à sa cour, et à prendre ses plus vaillants hommes et les plus sages. Je cite ceci pour faire sentir le devoir des vassaux, combattre et juger ; et ce devoir était même tel, que juger c’était combattre.\par
Un seigneur qui plaidait à sa cour contre son vassal\footnote{Voyez Beaumanoir, chap. LXVII, p. 337.}, et qui y était condamné, pouvait appeler un de ses hommes de faux jugement. Mais, à cause du respect que celui-ci devait à son seigneur pour la foi donnée, et la bienveillance que le seigneur devait à son vassal pour la foi reçue, on faisait une distinction : ou le seigneur disait en général que le jugement était faux et mauvais\footnote{« {\itshape Chi jugement est faus et mauvès}. » {\itshape Ibid.}, chap. LXVII, p. 337.} {\itshape ;} ou il imputait à son homme des prévarications personnelles\footnote{« {\itshape Vous avez fait ce jugement faux et mauvais, comme mauvais que vous êtes, ou par lovier, ou par promesse.} » Beaumanoir, chap. LXVII, p. 337.}. Dans le premier cas, il offensait sa propre cour, et en quelque façon lui-même, et il ne pouvait y avoir de gages de bataille ; il y en avait dans le second, parce qu’il attaquait l’honneur de son vassal ; et celui des deux qui était vaincu perdait la vie et les biens, pour maintenir la paix publique.\par
Cette distinction, nécessaire dans ce cas particulier, fut étendue. Beaumanoir dit que, lorsque celui qui appelait de faux jugement, attaquait un des hommes par des imputations personnelles, il y avait bataille ; mais que, s’il n’attaquait que le jugement, il était libre à celui des pairs qui était appelé de faire juger l’affaire par bataille ou par droit\footnote{{\itshape Ibid.}, p. 337 et 338.}. Mais, comme l’esprit qui régnait du temps de Beaumanoir était de restreindre l’usage du combat judiciaire, et que cette liberté donnée au pair appelé, de défendre par le combat le jugement, ou non, est également contraire aux idées de l’honneur établi dans ces temps-là, et à l’engagement où l’on était envers son seigneur de défendre sa cour, je crois que cette distinction de Beaumanoir était une jurisprudence nouvelle chez les Français.\par
Je ne dis pas que tous les appels de faux jugement se décidassent par bataille ; il en était de cet appel comme de tous les autres. On se souvient des exceptions dont j’ai parlé au chapitre XXV. Ici, c’était au tribunal suzerain à voir s’il fallait ôter, ou non, les gages de bataille.\par
On ne pouvait point fausser les jugements rendus dans la cour du roi ; car le roi n’ayant personne qui lui fût égal, il n’y avait personne qui pût l’appeler ; et le roi n’ayant point de supérieur, il n’y avait personne qui pût appeler de sa cour.\par
Cette loi fondamentale, nécessaire comme loi politique, diminuait encore, comme loi civile, les abus de la pratique judiciaire de ces temps-là. Quand un seigneur craignait qu’on ne faussât sa cour\footnote{Desfontaines, chap. XXII, art. 14.}, ou voyait qu’on se présentait pour la fausser, s’il était du bien de la justice qu’on ne la faussât pas, il pouvait demander des hommes de la cour du roi, dont on ne pouvait fausser le jugement ; et le roi Philippe, dit Desfontaines\footnote{{\itshape Ibid.}}, envoya tout son conseil pour juger une affaire dans la cour de l’abbé de Corbie.\par
Mais, si le seigneur ne pouvait avoir des juges du roi, il pouvait mettre sa cour dans celle du roi, s’il relevait nuement de lui ; et, s’il y avait des seigneurs intermédiaires, il s’adressait à son seigneur suzerain, allant de seigneur en seigneur jusqu’au roi.\par
Ainsi, quoiqu’on n’eût pas dans ces temps-là la pratique ni l’idée même des appels d’aujourd’hui, on avait recours au roi, qui était toujours la source d’où tous les fleuves partaient, et la mer où ils revenaient.
\subsubsection[{Chapitre XXVIII. De l’appel de défaute de droit}]{Chapitre XXVIII. De l’appel de défaute de droit}
\noindent On appelait de défaute de droit quand, dans la cour d’un seigneur, on différait, on évitait, ou l’on refusait de rendre la justice aux parties.\par
Dans la seconde race, quoique le comte eût plusieurs officiers sous lui, la personne de ceux-ci était subordonnée, mais la juridiction ne l’était pas. Ces officiers, dans leurs plaids, assises ou placites jugeaient en dernier ressort comme le comte même. Toute la différence était dans le partage de la juridiction : par exemple, le comte pouvait condamner à mort, juger de la liberté et de la restitution des biens\footnote{Capitulaire III, de l’an 812, art. 3, édition de Baluze, p. 497, et de Charles le Chauve, ajouté à la loi des Lombards, liv. II, art. 3.}, et le centenier ne le pouvait pas.\par
Par la même raison, il y avait des causes majeures qui étaient réservées au roi\footnote{Capitulaire III, de l’an 812, art. 2, édition de Baluze, p. 497.} ; c’étaient celles qui intéressaient directement l’ordre politique. Telles étaient les discussions qui étaient entre les évêques, les abbés, les comtes et autres grands, que les rois jugeaient avec les grands vassaux\footnote{{\itshape Cum fidelibus} : Capitulaire de Louis le Débonnaire, édition de Baluze, p. 667.}.\par
Ce qu’ont dit quelques auteurs, qu’on appelait du comte à l’envoyé du roi, ou {\itshape missus dominicus}, n’est pas fondé. Le comte et le {\itshape missus} avaient une juridiction égale et indépendante l’une de l’autre\footnote{Voyez le capitulaire de Charles le Chauve, ajouté à la loi des Lombards, liv. II, art. 3.} {\itshape ;} toute la différence était que le {\itshape missus} tenait ses placites quatre mois de l’année, et le comte les huit autres\footnote{Capitulaire III, de l’an 812, art. 8.}.\par
Si quelqu’un\footnote{Capitulaire ajouté à la loi des Lombards, liv. II, tit. {\itshape LIX.}} condamné dans une assise\footnote{{\itshape Placitum}.}, {\itshape y} demandait qu’on le rejugeât, et succombait encore, il payait une amende de quinze sols, ou recevait quinze coups de la main des juges qui avaient décidé l’affaire.\par
Lorsque les comtes ou les envoyés du roi ne se sentaient pas assez de force pour réduire les grands à la raison, ils leur faisaient donner caution qu’ils se présenteraient devant le tribunal du roi\footnote{Cela paraît par les formules, les chartes et les capitulaires.} : c’était pour juger l’affaire, et non pour la rejuger. Je trouve dans le capitulaire de Metz\footnote{De l’an 757, édition de Baluze, p. 180, art. 9 et 10 ; et le synode {\itshape apud Vernas}, de l’an 755, art. 29, édition de Baluze, p. 175. Ces deux capitulaires furent faits sous le roi Pépin.} l’appel de faux jugement à la cour du roi établi, et toutes autres sortes d’appels proscrits et punis.\par
Si l’on n’acquiesçait pas\footnote{Capitulaire XI de Charlemagne, de l’an 805, édition de Baluze, p. 423 ; et loi de Lothaire, dans la loi des Lombards, liv. II, tit. {\itshape LII}, art. 23.} au jugement des échevins\footnote{Officiers sous le comte : {\itshape scabini.}}, et qu’on ne réclamât pas, on était mis en prison jusqu’à ce qu’on eût acquiescé ; et si l’on réclamait, on était conduit sous une sûre garde devant le roi, et l’affaire se discutait à sa cour.\par
Il ne pouvait guère être question de l’appel de défaute de droit. Car, bien loin que, dans ces temps-là, on eût coutume de se plaindre que les comtes et autres gens qui avaient droit de tenir des assises, ne fussent pas exacts à tenir leur cour, on se plaignait au contraire qu’ils l’étaient trop\footnote{Voyez la loi des Lombards, liv. II, tit. LII, art. 22.} {\itshape ;} et tout est plein d’ordonnances qui défendent aux comtes et autres officiers de justice quelconques, de tenir plus de trois assises par an. Il fallait moins corriger leur négligence, qu’arrêter leur activité.\par
Mais lorsqu’un nombre innombrable de petites seigneuries se formèrent, que différents degrés de vasselage furent établis, la négligence de certains vassaux à tenir leur cour donna naissance à ces sortes d’appels\footnote{On voit des appels de défaute de droit dès le temps de Philippe Auguste.} {\itshape ;} d’autant plus qu’il en revenait au seigneur suzerain des amendes considérables.\par
L’usage du combat judiciaire s’étendant de plus en plus, il y eut des lieux, des cas, des temps, où il fut difficile d’assembler des pairs, et où par conséquent on négligea de rendre la justice. L’appel de défaute de droit s’introduisit ; et ces sortes d’appels ont été souvent des points remarquables de notre histoire, parce que la plupart des guerres de ces temps-là avaient pour motif la violation du droit politique, comme nos guerres d’aujourd’hui ont ordinairement pour cause, ou pour prétexte, celle du droit des gens.\par
Beaumanoir\footnote{Chap. LXI, p. 315.} dit que, dans le cas de défaute de droit, il n’y avait jamais de bataille : en voici les raisons. On ne pouvait pas appeler au combat le seigneur lui-même, à cause du respect dû à sa personne : on ne pouvait pas appeler les pairs du seigneur, parce que la chose était claire, et qu’il n’y avait qu’à compter les jours des ajournements ou des autres délais : il n’y avait point de jugement, et on ne faussait que sur un jugement. Enfin le délit des pairs offensait le seigneur comme la partie ; et il était contre l’ordre qu’il y eût un combat entre le seigneur et ses pairs.\par
Mais comme devant le tribunal suzerain on prouvait la défaute par témoins, on pouvait appeler au combat les témoins\footnote{Beaumanoir, {\itshape ibid.}} ; et par là on n’offensait ni le seigneur ni son tribunal.\par
1° Dans les cas où la défaute venait de la part des hommes ou pairs du seigneur qui avaient différé de rendre la justice, ou évité de faire le jugement après les délais passés, c’étaient les pairs du seigneur qu’on appelait de défaute de droit devant le suzerain ; et, s’ils succombaient, ils payaient une amende à leur seigneur\footnote{Desfontaines, chap. XXI, art. 24.}. Celui-ci ne pouvait porter aucun secours à ses hommes ; au contraire, il saisissait leur fief, jusqu’à ce qu’ils eussent payé chacun une amende de soixante livres.\par
2° Lorsque la défaute venait de la part du seigneur, ce qui arrivait lorsqu’il n’y avait pas assez d’hommes à sa cour pour faire le jugement, ou lorsqu’il n’avait pas assemblé ses hommes, ou mis quelqu’un à sa place pour les assembler, on demandait la défaute devant le seigneur suzerain ; mais, à cause du respect dû au seigneur, on faisait ajourner la partie\footnote{{\itshape Ibid.}, chap. XXI, art. 32.}, et non pas le seigneur.\par
Le seigneur demandait sa cour devant le tribunal suzerain ; et s’il gagnait la défaute, on lui renvoyait l’affaire, et on lui payait une amende de soixante livres\footnote{Beaumanoir, chap. LXI, p. 312.} ; mais, si la défaute était prouvée, la peine contre lui était de perdre le jugement de la chose contestée ; le fond était jugé dans le tribunal suzerain\footnote{Desfontaines, chap. XXI, art. 29.} ; en effet, on n’avait demandé la défaute que pour cela.\par
3° Si l’on plaidait à la cour de son seigneur contre lui\footnote{Sous le règne de Louis VIII, le sire de Nesle plaidait contre Jeanne, comtesse de Flandre ; il la somma de le faire juger dans quarante jours, et il l’appela ensuite de défaute de droit à la cour du roi. Elle répondit qu’elle le ferait juger par ses pairs en Flandre. La cour du roi prononça qu’il n’y serait point renvoyé, et que la comtesse serait ajournée.}, ce qui n’avait lieu que pour les affaires qui concernaient le fief ; après avoir laissé passer tous les délais, on sommait le seigneur même devant bonnes gens\footnote{Desfontaines, chap. XXI, art. 34.}, et on le faisait sommer par le souverain, dont on devait avoir permission. On n’ajournait point par pairs, parce que les pairs ne pouvaient ajourner leur seigneur ; mais ils pouvaient ajourner pour leur seigneur\footnote{{\itshape Ibid.}, art. 9.}.\par
Quelquefois l’appel de défaute de droit était suivi d’un appel de faux jugement\footnote{Beaumanoir, chap. LXI, p. 311.}, lorsque le seigneur, malgré la défaute, avait fait rendre le jugement.\par
Le vassal qui appelait à tort son seigneur de défaute de droit\footnote{{\itshape Ibid.}, p. 312. Mais celui qui n’aurait été homme, ni tenant du seigneur, ne lui payait qu’une amende de 60 livres.} était condamné à lui payer une amende à sa volonté.\par
Les Gantois\footnote{{\itshape Ibid.}, p. 318.} avaient appelé de défaute de droit le comte de Flandre devant le roi, sur ce qu’il avait différé de leur faire rendre jugement en sa cour. Il se trouva qu’il avait pris encore moins de délais que n’en donnait la coutume du pays. Les Gantois lui furent renvoyés ; il fit saisir de leurs biens jusqu’à la valeur de soixante mille livres. Ils revinrent à la cour du roi, pour que cette amende fût modérée : il fut décidé que le comte pouvait prendre cette amende, et même plus, s’il voulait. Beaumanoir avait assisté à ces jugements.\par
4° Dans les affaires que le seigneur pouvait avoir contre le vassal pour raison du corps ou de l’honneur de celui-ci, ou des biens qui n’étaient pas du fief, il n’était point question d’appel de défaute de droit, puisqu’on ne jugeait point à la cour du seigneur, mais à la cour de celui de qui il tenait ; les hommes, dit Desfontaines\footnote{Chap. XXI, art. 35.}, n’ayant pas droit de faire jugement sur le corps de leur seigneur.\par
J’ai travaillé à donner une idée claire de ces choses qui, dans les auteurs de ces temps-là, sont si confuses et si obscures, qu’en vérité les tirer du chaos où elles sont, c’est les découvrir.
\subsubsection[{Chapitre XXIX. Époque du règne de Saint Louis}]{Chapitre XXIX. Époque du règne de Saint Louis}
\noindent Saint Louis abolit le combat judiciaire dans les tribunaux de ses domaines, comme il paraît par l’ordonnance qu’il fit là-dessus\footnote{En 1260.} et par les {\itshape Établissements}\footnote{Liv. I, chap. II et VII ; liv. II, chap. X et XI.}.\par
Mais il ne l’ôta point dans les cours de ses barons\footnote{Comme il paraît partout dans les {\itshape Établissements ;} et Beaumanoir, chap. LXI, p. 309.}, excepté dans le cas d’appel de faux jugement.\par
On ne pouvait fausser la cour de son seigneur\footnote{C’est-à-dire, appeler de faux jugements.} sans demander le combat judiciaire contre les juges qui avaient prononcé le jugement. Mais saint Louis introduisit l’usage de fausser sans combattre\footnote{{\itshape Établissements}, liv. I, chap. VI ; et liv. II, chap. XV.} : changement qui fut une espèce de révolution.\par
Il déclara\footnote{{\itshape Ibid.}, liv. II, chap. XV.} qu’on ne pourrait point fausser les jugements rendus dans les seigneuries de ses domaines, parce que c’était un crime de félonie. Effectivement, si c’était une espèce de crime de félonie contre le seigneur, à plus forte raison en était-ce un contre le roi. Mais il voulut que l’on pût demander amendement\footnote{{\itshape Ibid.}, liv. I, chap. LXXVIII ; et liv. II, chap. XV.} des jugements rendus dans ses cours ; non pas parce qu’ils étaient faussement ou méchamment rendus, mais parce qu’ils faisaient quelque préjudice\footnote{{\itshape Ibid.}, liv. I, chap. LXXVII.}. Il voulut au contraire, qu’on fût contraint de fausser les jugements des cours des barons, si l’on voulait s’en plaindre\footnote{{\itshape Ibid.}, liv. II, chap. XV.}.\par
On ne pouvait point, suivant les {\itshape Établissements}, fausser les cours du domaine du roi, comme on vient de le dire. Il fallait demander amendement devant le même tribunal ; et, en cas que le bailli ne voulût pas faire l’amendement requis, le roi permettait de faire appel à sa cour\footnote{{\itshape Ibid.}, liv. I, chap. LXXVIII.} ; ou plutôt, en interprétant les {\itshape Établissements} par eux-mêmes, de lui présenter une requête ou supplication\footnote{{\itshape Ibid.}, liv. II, chap. XV.}.\par
À l’égard des cours des seigneurs, saint Louis, en permettant de les fausser, voulut que l’affaire fût portée au tribunal du roi ou du seigneur suzerain\footnote{Mais si on ne faussait pas, et qu’on voulût appeler, on n’était point reçu. {\itshape Établissements}, liv. II, chap. XV. Li {\itshape sire en auroit le recort de sa cour, droit faisant}}, non pas pour y être décidée par le combat\footnote{{\itshape Ibid.}, liv. I, chap. VI et LXVII ; et liv. II, chap. XV ; et Beaumanoir, chap. XI, p. 58.}, mais par témoins, suivant une forme de procéder dont il donna des règles\footnote{{\itshape Établissements}, liv. I, chap. I, II et III.}.\par
Ainsi, soit qu’on pût fausser, comme dans les cours des seigneurs, soit qu’on ne le pût pas, comme dans les cours de ses domaines, il établit qu’on pourrait appeler sans courir le hasard d’un combat.\par
Desfontaines\footnote{Chap. XXII, art. 16 et 17.} nous rapporte les deux premiers exemples qu’il ait vus, où l’on ait ainsi procédé sans combat judiciaire. l’un, dans une affaire jugée à la cour de Saint-Quentin, qui était du domaine du roi ; et l’autre, dans la cour de Ponthieu, où le comte, qui était présent, opposa l’ancienne jurisprudence ; mais ces deux affaires furent jugées par droit.\par
On demandera peut-être pourquoi saint Louis ordonna pour les cours de ses barons une manière de procéder différente de celle qu’il établissait dans les tribunaux de ses domaines : en voici la raison. Saint Louis, statuant pour les cours de ses domaines, ne fut point gêné dans ses vues ; mais il eut des ménagements à garder avec les seigneurs qui jouissaient de cette ancienne prérogative, que les affaires n’étaient jamais tirées de leurs cours, à moins qu’on ne s’exposât au danger de les fausser. Saint Louis maintint cet usage de fausser ; mais il voulut qu’on pût fausser sans combattre : c’est-à-dire que, pour que le changement se fit moins sentir, il ôta la chose, et laissa subsister les termes.\par
Ceci ne fut pas universellement reçu dans les cours des seigneurs. Beaumanoir\footnote{Chap. LXI, p. 309.} dit que, de son temps, il y avait deux manières de juger : l’une suivant {\itshape l’Établissement-le-roi}, et l’autre suivant la pratique ancienne ; que les seigneurs avaient droit de suivre l’une ou l’autre de ces pratiques ; mais que quand, dans une affaire, on en avait choisi une, on ne pouvait plus revenir à l’autre. Il ajoute que le comte de Clermont suivait la nouvelle pratique\footnote{{\itshape Ibid.}} tandis que ses vassaux se tenaient à l’ancienne ; mais qu’il pourrait, quand il voudrait, rétablir l’ancienne, sans quoi il aurait moins d’autorité que ses vassaux.\par
Il faut savoir que la France était pour lors divisée en pays du domaine du roi\footnote{Voyez Beaumanoir, Desfontaines, et les {\itshape Établissements}, liv. II, chap. X, XI, XV et autres.}, et en ce qu’on appelait pays des barons, ou en baronnies ; et, pour me servir des termes des{\itshape  Établissements} de saint Louis, en pays de l’obéissance-le-roi, et en pays hors l’obéissance-le-roi. Quand les rois faisaient des ordonnances pour les pays de leurs domaines, ils n’employaient que leur seule autorité ; mais, quand ils en faisaient qui regardaient aussi les pays de leurs barons, elles étaient faites de concert avec eux, ou scellées ou souscrites d’eux\footnote{Voyez les {\itshape Ordonnances} du commencement de la troisième race, dans le recueil de Laurière, surtout celles de Philippe Auguste sur la juridiction ecclésiastique, et celle de Louis VIII sur les Juifs ; et les chartes rapportées par M. Brussel, notamment celle de saint Louis sur le bail et le rachat des terres, et la majorité féodale des filles, t. II, liv. III, p. 35 ; et {\itshape ibid.}, l’ordonnance de Philippe Auguste, p. 7.} ; sans cela, les barons les recevaient, ou ne les recevaient pas, suivant qu’elles leur paraissaient convenir ou non au bien de leurs seigneuries. Les arrière-vassaux étaient dans les mêmes termes avec les grands vassaux. Or les {\itshape Établissements} ne furent pas donnés du consentement des seigneurs, quoiqu’ils statuassent sur des choses qui étaient pour eux d’une grande importance : ainsi ils ne furent reçus que par ceux qui crurent qu’il leur était avantageux de les recevoir. Robert, fils de saint Louis, les admit dans sa comté de Clermont ; et ses vassaux ne crurent pas qu’il leur convînt de les faire pratiquer chez eux.
\subsubsection[{Chapitre XXX. Observation sur les appels}]{Chapitre XXX. Observation sur les appels}
\noindent On conçoit que des appels, qui étaient des provocations à un combat, devaient se faire sur-le-champ. « S’il se part de court sans appeler, dit Beaumanoir\footnote{Chap. LXIII, p. 327 ; {\itshape ibid.}, chap. LXI, p. 312.}, il perd son appel, et tient le jugement pour bon. » Ceci subsista, même après qu’on eut restreint l’usage du combat judiciaire\footnote{Voyez les {\itshape Établissements} de saint Louis, liv. II, chap. XV ; l’Ordonnance de Charles VII, de 1453.}.
\subsubsection[{Chapitre XXXI. Continuation du même sujet}]{Chapitre XXXI. Continuation du même sujet}
\noindent Le vilain ne pouvait pas fausser la cour de son seigneur : nous l’apprenons de Desfontaines\footnote{Chap. XXI, art. 21 et 22.} ; et cela est confirmé par les {\itshape Établissements}\footnote{Liv. I, chap. CXXXVI.}. « Aussi, dit encore Desfontaines\footnote{Chap. II, art. 8.}, n’y a-t-il entre toi seigneur et ton vilain autre juge fors Dieu. »\par
C’était l’usage du combat judiciaire qui avait exclu les vilains de pouvoir fausser la cour de leur seigneur ; et cela est si vrai, que les vilains qui, par charte ou par usage\footnote{Desfontaines, chap. XXII, art. 7. Cet article et le 21, du chap. XXII du même auteur ont été jusqu’ici très mal expliqués. Desfontaines ne met point en opposition le jugement du seigneur avec celui du chevalier, puisque c’était le même ; mais il oppose le vilain ordinaire à celui qui avait le privilège de combattre.} avaient droit de combattre, avaient aussi droit de fausser la cour de leur seigneur, quand même les hommes qui avaient jugé, auraient été chevaliers\footnote{Les chevaliers peuvent toujours être du nombre des juges. Desfontaines, chap. XXI, art. 48.} {\itshape ;} et Desfontaines\footnote{Chap. XXII, art. 14.} donne des expédients pour que ce scandale du vilain qui, en faussant le jugement, combattrait contre un chevalier, n’arrivât pas.\par
La pratique des combats judiciaires commençant à s’abolir, et l’usage des nouveaux appels à s’introduire, on pensa qu’il était déraisonnable que les personnes franches eussent un remède contre l’injustice de la cour de leurs seigneurs, et que les vilains ne l’eussent pas ; et le parlement reçut leurs appels comme ceux des personnes franches.
\subsubsection[{Chapitre XXXII. Continuation du même sujet}]{Chapitre XXXII. Continuation du même sujet}
\noindent Lorsqu’on faussait la cour de son seigneur, il venait en personne devant le seigneur suzerain, pour défendre le jugement de sa cour. De même\footnote{Desfontaines, chap. XXI, art. 33.}, dans le cas d’appel de défaute de droit, la partie ajournée devant le seigneur suzerain menait son seigneur avec elle, afin que si la défaute n’était pas prouvée, il pût ravoir sa cour.\par
Dans la suite, ce qui n’était que deux cas particuliers étant devenu général pour toutes les affaires, par l’introduction de toutes sortes d’appels, il parut extraordinaire que le seigneur fût obligé de passer sa vie dans d’autres tribunaux que les siens, et pour d’autres affaires que les siennes. Philippe de Valois ordonna que les baillis seuls seraient ajournés\footnote{En 1332.}. Et, quand l’usage des appels devint encore plus fréquent, ce fut aux parties à défendre à l’appel ; le fait du juge devint le fait de la partie\footnote{Voyez quel était l’état des choses du temps de Boutillier, qui vivait en l’an 1402. {\itshape Somme rurale}, liv. I, p. 19 et 20.}.\par
J’ai dit\footnote{Ci-dessus, chap. XXX.} que, dans l’appel de défaute de droit, le seigneur ne perdait que le droit de faire juger l’affaire en sa cour. Mais, si le seigneur était attaqué lui-même comme partie\footnote{Beaumanoir, chap. LXI, p. 312 et 318.}, ce qui devint très fréquent\footnote{{\itshape Ibid.}}, il payait au roi, ou au seigneur suzerain devant qui on avait appelé, une amende de soixante livres. De là vint cet usage, lorsque les appels furent universellement reçus, de faire payer l’amende au seigneur lorsqu’on réformait la sentence de son juge : usage qui subsista longtemps, qui fut confirmé par l’ordonnance de Roussillon, et que son absurdité a fait périr.
\subsubsection[{Chapitre XXXIII. Continuation du même sujet}]{Chapitre XXXIII. Continuation du même sujet}
\noindent Dans la pratique du combat judiciaire, le fausseur qui avait appelé un des juges, pouvait perdre par le combat son procès\footnote{Desfontaines, chap. XXI, art. 14.}, et ne pouvait pas le gagner. En effet, la partie qui avait un jugement pour elle, n’en devait pas être privée par le fait d’autrui. Il fallait donc que le fausseur qui avait vaincu, combattît encore contre la partie, non pas pour savoir si le jugement était bon ou mauvais ; il ne s’agissait plus de ce jugement puisque le combat l’avait anéanti ; mais pour décider si la demande était légitime ou non ; et c’est sur ce nouveau point que l’on combattait. De là doit être venue notre manière de prononcer les arrêts : {\itshape La cour met l’appel au néant ; la cour met l’appel et ce dont a été appelé au néant}. En effet, quand celui qui avait appelé de faux jugements était vaincu, l’appel était anéanti ; quand il avait vaincu, le jugement était anéanti, et l’appel même : il fallait procéder à un nouveau jugement.\par
Cela est si vrai que, lorsque l’affaire se jugeait par enquêtes, cette manière de prononcer n’avait pas lieu. M. de La Roche-Flavin\footnote{Des parlements de France, liv. I, chap. XVI.} nous dit que la chambre des enquêtes ne pouvait user de cette forme dans les premiers temps de sa création.
\subsubsection[{Chapitre XXXIV. Comment la procédure devint secrète}]{Chapitre XXXIV. Comment la procédure devint secrète}
\noindent Les duels avaient introduit une forme de procédure publique ; l’attaque et la défense étaient également connues. « Les témoins, dit Beaumanoir\footnote{Chap. LXI, p. 315.}, doivent dire leur témoignage devant tous. »\par
Le commentateur de Boutillier dit avoir appris d’anciens praticiens, et de quelques vieux procès écrits à la main, qu’anciennement, en France, les procès criminels se faisaient publiquement, et en une forme non guère différente des jugements publics des Romains. Ceci était lié avec l’ignorance de l’écriture, commune dans ces temps-là. L’usage de l’écriture arrête les idées, et peut faire établir le secret ; mais, quand on n’a point cet usage, il n’y a que la publicité de la procédure qui puisse fixer ces mêmes idées.\par
Et, comme il pouvait y avoir de l’incertitude sur ce qui avait été jugé par hommes\footnote{Comme dit Beaumanoir, chap. XXXIX, p. 209.}, ou plaidé devant hommes, on pouvait en rappeler la mémoire toutes les fois qu’on tenait la cour, par ce qui s’appelait la procédure {\itshape par record}\footnote{On prouvait par témoins ce qui s’était déjà passé, dit, ou ordonné en justice.} ; et, dans ce cas, il n’était pas permis d’appeler les témoins au combat ; car les affaires n’auraient jamais eu de fin.\par
Dans la suite, il s’introduisit une forme de procéder secrète. Tout était public : tout devint caché, les interrogatoires, les informations, le récolement, la confrontation, les conclusions de la partie publique ; et c’est l’usage d’aujourd’hui. La première forme de procéder convenait au gouvernement d’alors, comme la nouvelle était propre au gouvernement qui fut établi depuis.\par
Le commentateur de Boutillier fixe à l’ordonnance de 1539 l’époque de ce changement. Je crois qu’il se fit peu à peu, et qu’il passa de seigneurie en seigneurie, à mesure que les seigneurs renoncèrent à l’ancienne pratique de juger, et que celle tirée des {\itshape Établissements} de saint Louis vint à se perfectionner. En effet, Beaumanoir\footnote{Chap. XXXIX, p. 218.} dit que ce n’était que dans les cas où l’on pouvait donner des gages de bataille, qu’on entendait publiquement les témoins ; dans les autres, on les oyait en secret, et on rédigeait leurs dépositions par écrit. Les procédures devinrent donc secrètes, lorsqu’il n’y eut plus de gages de bataille.
\subsubsection[{Chapitre XXXV. Des dépens}]{Chapitre XXXV. {\itshape Des dépens}}
\noindent Anciennement en France, il n’y avait point de condamnation de dépens en cour laie\footnote{Desfontaines, dans son {\itshape Conseil}, chap. XXII, art. 3 et 8 ; et Beaumanoir, chap. XXXIII ; {\itshape Établissements}, liv. I, chap. XC.}. La partie qui succombait était assez punie par des condamnations d’amende envers le seigneur et ses pairs. La manière de procéder par le combat judiciaire faisait que, dans les crimes, la partie qui succombait, et qui perdait la vie et les biens, était punie autant qu’elle pouvait l’être ; et, dans les autres cas du combat judiciaire, il y avait des amendes quelquefois fixes, quelquefois dépendantes de la volonté du seigneur, qui faisaient assez craindre les événements des procès. Il en était de même dans les affaires qui ne se décidaient que par le combat. Comme c’était le seigneur qui avait les profits principaux, c’était lui aussi qui faisait les principales dépenses, soit pour assembler ses pairs, soit pour les mettre en état de procéder au jugement. D’ailleurs, les affaires finissant sur le lieu même, et toujours presque sur-le-champ, et sans ce nombre infini d’écritures qu’on vit depuis, il n’était pas nécessaire de donner des dépens aux parties.\par
C’est l’usage des appels qui doit naturellement introduire celui de donner des dépens. Aussi Desfontaines\footnote{Chap. XXII, art. 8.} dit-il que, lorsqu’on appelait par loi écrite, c’est-à-dire quand on suivait les nouvelles lois de saint Louis, on donnait des dépens ; mais que, dans l’usage ordinaire, qui ne permettait point d’appeler sans fausser, il n’y en avait point ; on n’obtenait qu’une amende, et la possession d’an et jour de la chose contestée, si l’affaire était renvoyée au seigneur.\par
Mais, lorsque de nouvelles facilités d’appeler augmentèrent le nombre des appels\footnote{« À présent que l’on est si enclin à appeler » dit Boutillier, {\itshape Somme rurale}, liv. I, tit. III, p. 16.} ; que, par le fréquent usage de ces appels d’un tribunal à un autre, les parties furent sans cesse transportées hors du lieu de leur séjour ; quand l’art nouveau de la procédure multiplia et éternisa les procès ; lorsque la science d’éluder les demandes les plus justes se fut raffinée ; quand un plaideur sut fuir, uniquement pour se faire suivre ; lorsque la demande fut ruineuse, et la défense tranquille ; que les raisons se perdirent dans des volumes de paroles et d’écrits ; que tout fut plein de suppôts de justice qui ne devaient point rendre la justice ; que la mauvaise foi trouva des conseils, là où elle ne trouva pas des appuis ; il fallut bien arrêter les plaideurs par la crainte des dépens. Ils durent les payer pour la décision, et pour les moyens qu’ils avaient employés pour l’éluder. Charles le Bel fit là-dessus une ordonnance générale\footnote{En 1324.}.
\subsubsection[{Chapitre XXXVI. De la partie publique}]{Chapitre XXXVI. De la partie publique}
\noindent Comme, par les lois saliques et ripuaires, et par les autres lois des peuples barbares, les peines des crimes étaient pécuniaires, il n’y avait point pour lors, comme aujourd’hui parmi nous, de partie publique qui fût chargée de la poursuite des crimes. En effet, tout se réduisait en réparations de dommages ; toute poursuite était, en quelque façon, civile, et chaque particulier pouvait la faire. D’un autre côté, le droit romain avait des formes populaires pour la poursuite des crimes, qui ne pouvaient s’accorder avec le ministère d’une partie publique.\par
L’usage des combats judiciaires ne répugnait pas moins à cette idée ; car, qui aurait voulu être la partie publique, et se faire champion de tous contre tous ?\par
Je trouve, dans un recueil de formules que M. Muratori a insérées dans les lois des Lombards, qu’il y avait dans la seconde race, un avoué de la partie publique\footnote{{\itshape Advocatus de parte publica}.}. Mais si on lit le recueil entier de ces formules, on verra qu’il y avait une différence totale entre ces officiers, et ce que nous appelons aujourd’hui la partie publique, nos procureurs généraux, nos procureurs du roi ou des seigneurs. Les premiers étaient plutôt les agents du public pour la manutention politique et domestique, que pour la manutention civile. En effet, on ne voit point dans ces formules qu’ils fussent chargés de la poursuite des crimes et des affaires qui concernaient les mineurs, les églises, ou l’état des personnes.\par
J’ai dit que l’établissement d’une partie publique répugnait à l’usage du combat judiciaire. Je trouve pourtant dans une de ces formules un avoué de la partie publique qui a la liberté de combattre. M. Muratori l’a mise à la suite de la constitution d’Henri I\textsuperscript{er}\footnote{Voyez cette constitution et cette formule dans le second volume des {\itshape Historiens d’Italie}, p. 175.} pour laquelle elle a été faite. Il est dit dans cette constitution que « si quelqu’un tue son père, son frère, son neveu, ou quelque autre de ses parents, il perdra leur succession, qui passera aux autres parents, et que la sienne propre appartiendra au fisc ». Or c’est pour la poursuite de cette succession dévolue au fisc que l’avoué de la partie publique, qui en soutenait les droits, avait la liberté de combattre : ce cas rentrait dans la règle générale.\par
Nous voyons dans ces formules l’avoué de la partie publique agir contre celui qui avait pris un voleur, et ne l’avait pas mené au comte\footnote{Recueil de Muratori, p. 104, sur la loi 88 de Charlemagne, liv. I, tit. XXVI, § 78.} {\itshape ;} contre celui qui avait fait un soulèvement ou une assemblée contre le comte\footnote{Autre formule {\itshape ibid.}, p. 87.} {\itshape ;} contre celui qui avait sauvé la vie à un homme que le comte lui avait donné pour le faire mourir\footnote{{\itshape Ibid.}, p. 104.} {\itshape ;} contre l’avoué des églises\footnote{{\itshape Ibid.}, p. 95.} à qui le comte avait ordonné de lui présenter un voleur, et qui n’avait point obéi ; contre celui qui avait révélé le secret du roi aux étrangers\footnote{{\itshape Ibid.}, p. 88.} {\itshape ;} contre celui qui, à main armée, avait poursuivi l’envoyé de l’empereur\footnote{{\itshape Ibid.}, p. 98.} ; contre celui qui avait méprisé les lettres de l’empereur\footnote{{\itshape Ibid.}, p. 132.}, et il était poursuivi par l’avoué de l’empereur, ou par l’empereur lui-même ; contre celui qui n’avait pas voulu recevoir la monnaie du prince\footnote{{\itshape Ibid.}, p. 132.} ; enfin, cet avoué demandait les choses que la loi adjugeait au fisc\footnote{{\itshape Ibid.}, p. 137.}.\par
Mais, dans la poursuite des crimes, on ne voit point d’avoué de la partie publique ; même quand on emploie les duels\footnote{{\itshape Ibid.}, p. 147.} ; même quand il s’agit d’incendie\footnote{{\itshape Ibid.}} ; même lorsque le juge est tué sur son tribunal\footnote{{\itshape Ibid.}, p. 168.} ; même lorsqu’il s’agit de l’état des personnes\footnote{{\itshape Ibid.}, p. 134.}, de la liberté et de la servitude\footnote{{\itshape Ibid.}, p. 107.}.\par
Ces formules sont faites, non seulement pour les lois des Lombards, mais pour les capitulaires ajoutés : ainsi il ne faut pas douter que, sur cette matière, elles ne nous donnent la pratique de la seconde race.\par
Il est clair que ces avoués de la partie publique durent s’éteindre avec la seconde race, comme les envoyés du roi dans les provinces ; par la raison qu’il n’y eut plus de loi générale, ni de fisc général ; et par la raison qu’il n’y eut plus de comte dans les provinces pour tenir les plaids ; et par conséquent plus de ces sortes d’officiers dont la principale fonction était de maintenir l’autorité du comte.\par
L’usage des combats, devenu plus fréquent dans la troisième race, ne permit pas d’établir une partie publique. Aussi Boutillier, dans sa {\itshape Somme rurale}, parlant des officiers de justice, ne cite-t-il que les baillis, hommes féodaux et sergents. Voyez les {\itshape Établissements}\footnote{Liv. {\itshape I}, chap. I ; et liv. II, chap. XI et XIII.}, et Beaumanoir\footnote{Chap. I, et chap. LXI.}, sur la manière dont on faisait les poursuites dans ces temps-là.\par
Je trouve dans les lois de Jacques II, roi de Majorque\footnote{Voyez ces lois dans les {\itshape Vies des Saints} du mois de juin, t. III, p. 26.}, une création à l’emploi de procureur du roi, avec les fonctions qu’ont aujourd’hui les nôtres\footnote{{\itshape Qui continue nostram sacram curiam sequi teneatur, instituatur qui facta et causas, in ipsa curia promoveat atque prosequatur.}}. Il est visible qu’ils ne vinrent qu’après que la forme judiciaire eut changé parmi nous.
\subsubsection[{Chapitre XXXVII. Comment les établissements de Saint Louis tombèrent dans l’oubli}]{Chapitre XXXVII. {\itshape Comment les} établissements de Saint Louis{\itshape  tombèrent dans l’oubli}}
\noindent Ce fut le destin des établissements qu’ils naquirent, vieillirent et moururent en très peu de temps.\par
Je ferai là-dessus quelques réflexions. Le code que nous avons sous le nom d’Établissements de saint Louis n’a jamais été fait pour servir de loi à tout le royaume, quoique cela soit dit dans la préface de ce code. Cette compilation est un code général qui statue sur toutes les affaires civiles, les dispositions des biens par testament ou entre vifs, les dots et les avantages des femmes, les profits et les prérogatives des fiefs, les affaires de police, etc. Or, dans un temps où chaque ville, bourg ou village avait sa coutume, donner un corps général de lois civiles, c’était vouloir renverser dans un moment toutes les lois particulières sous lesquelles on vivait dans chaque lieu du royaume. Faire une coutume générale de toutes les coutumes particulières serait une chose inconsidérée, même dans ce temps-ci, où les princes ne trouvent partout que de l’obéissance. Car, s’il est vrai qu’il ne faut pas changer lorsque les inconvénients égalent les avantages, encore moins le faut-il lorsque les avantages sont petits, et les inconvénients immenses. Or, si l’on fait attention à l’état où était pour lors le royaume, où chacun s’enivrait de l’idée de sa souveraineté et de sa puissance, on voit bien qu’entreprendre de changer partout les lois et les usages reçus, c’était une chose qui ne pouvait venir dans l’esprit de ceux qui gouvernaient.\par
Ce que je viens de dire prouve encore que ce code des Établissements ne fut pas confirmé en parlement par les barons et gens de loi du royaume, comme il est dit dans un manuscrit de l’hôtel de ville d’Amiens, cité par M. Ducange\footnote{Préface sur les {\itshape Établissements}.}.\par
On voit, dans les autres manuscrits, que ce code fut donné par saint Louis en l’année 1270, avant qu’il partît pour Tunis. Ce fait n’est pas plus vrai ; car saint Louis est parti en 1269, comme l’a remarqué M. Ducange ; d’où il conclut que ce code aurait été publié en son absence. Mais je dis que cela ne peut pas être. Comment saint Louis aurait-il pris le temps de son absence pour faire une chose qui aurait été une semence de troubles, et qui eût pu produire, non pas des changements, mais des révolutions ? Une pareille entreprise avait besoin, plus qu’une autre, d’être suivie de près, et n’était point l’ouvrage d’une régence faible, et même composée de seigneurs qui avaient intérêt que la chose ne réussît pas. C’était Matthieu, abbé de Saint-Denis, Simon de Clermont, comte de Nesle ; et, en cas de mort, Philippe, évêque d’Évreux ; et Jean, comte de Ponthieu. On a vu ci-dessus\footnote{Chap. XXIX.}, que le comte de Ponthieu s’opposa dans sa seigneurie à l’exécution d’un nouvel ordre judiciaire.\par
Je dis, en troisième lieu, qu’il y a grande apparence que le code que nous avons est une chose différente des établissements de saint Louis sur l’ordre judiciaire. Ce code cite les établissements : il est donc un ouvrage sur les établissements, et non pas les établissements. De plus, Beaumanoir, qui parle souvent des établissements de saint Louis, ne cite que des établissements particuliers de ce prince, et non pas cette compilation des établissements. Desfontaines\footnote{Voyez ci-dessus le chap. XXIX.}, qui écrivait sous ce prince, nous parle des deux premières fois que l’on exécuta ses établissements sur l’ordre judiciaire, comme d’une chose reculée. Les établissements de saint Louis étaient donc antérieurs à la compilation dont je parle, qui, à la rigueur, et en adoptant les prologues erronés mis par quelques ignorants à la tête de cet ouvrage, n’aurait paru que la dernière année de la vie de saint Louis, ou même après la mort de ce prince.
\subsubsection[{Chapitre XXXVIII. Continuation du même sujet}]{Chapitre XXXVIII. Continuation du même sujet}
\noindent Qu’est-ce donc que cette compilation que nous avons sous le nom d’Établissements de saint Louis ? Qu’est-ce que ce code obscur, confus et ambigu, où l’on mêle sans cesse la jurisprudence française avec la loi romaine ; où l’on parle comme un législateur, et où l’on voit un jurisconsulte ; où l’on trouve un corps entier de jurisprudence sur tous les cas, sur tous les points du droit civil ? Il faut se transporter dans ces temps-là.\par
Saint Louis, voyant les abus de la jurisprudence de son temps, chercha à en dégoûter les peuples : il fit plusieurs règlements pour les tribunaux de ses domaines, et pour ceux de ses barons ; et il eut un tel succès, que Beaumanoir\footnote{Chap. LXI, p. 309.}, qui écrivait très peu de temps après la mort de ce prince, nous dit que la manière de juger établie par saint Louis était pratiquée dans un grand nombre de cours des seigneurs.\par
Ainsi ce prince remplit son objet, quoique ses règlements pour les tribunaux des seigneurs n’eussent pas été faits pour être une loi générale du royaume, mais comme un exemple que chacun pourrait suivre, et que chacun même aurait intérêt de suivre. Il ôta le mal, en faisant sentir le meilleur. Quand on vit dans ses tribunaux, quand on vit dans ceux des seigneurs, une manière de procéder plus naturelle, plus raisonnable, plus conforme à la morale, à la religion, à la tranquillité publique, à la sûreté de la personne et des biens, on la prit, et on abandonna l’autre.\par
Inviter, quand il ne faut pas contraindre ; conduire, quand il ne faut pas commander, c’est l’habileté suprême. La raison a un empire naturel ; elle a même un empire tyrannique : on lui résiste, mais cette résistance est son triomphe ; encore un peu de temps, et l’on sera forcé de revenir à elle.\par
Saint Louis, pour dégoûter de la jurisprudence française, fit traduire les livres du droit romain, afin qu’ils fussent connus des hommes de loi de ces temps-là. Desfontaines, qui est le premier\footnote{Il dit lui-même dans son prologue : {\itshape Nus luy en prit onques mais cette chose dont j’ay}.} auteur de pratique que nous ayons, fit un grand usage de ces lois romaines : son ouvrage est, en quelque façon, un résultat de l’ancienne jurisprudence française, des lois ou établissements de saint Louis, et de la loi romaine. Beaumanoir fit peu d’usage de la loi romaine ; mais il concilia l’ancienne jurisprudence française avec les règlements de saint Louis.\par
C’est dans l’esprit de ces deux ouvrages, et surtout de celui de Desfontaines, que quelque bailli, je crois, fit l’ouvrage de jurisprudence que nous appelons les {\itshape Établissements}. Il est dit, dans le titre de cet ouvrage, qu’il est fait selon l’usage de Paris et d’Orléans, et de cour de baronnie ; et, dans le prologue, qu’il y est traité des usages de tout le royaume, et d’Anjou, et de cour de baronnie. Il est visible que cet ouvrage fut fait pour Paris, Orléans et Anjou, comme les ouvrages de Beaumanoir et de Desfontaines furent faits pour les comtés de Clermont et de Vermandois : et, comme il paraît par Beaumanoir que plusieurs lois de saint Louis avaient pénétré dans les cours de baronnie, le compilateur a eu quelque raison de dire que son ouvrage regardait aussi les cours de baronnie\footnote{Il n’y a rien de si vague que le titre et le prologue. D’abord ce sont les usages de Paris et d’Orléans, et de cour de baronnie ; ensuite ce sont les usages de toutes les cours laies du royaume, et de la prévôté de France ; ensuite ce sont les usages de tout le royaume, et d’Anjou, et de cour de baronnie.}.\par
Il est clair que celui qui fit cet ouvrage compila les coutumes du pays avec les lois et les établissements de saint Louis. Cet ouvrage est très précieux, parce qu’il contient les anciennes coutumes d’Anjou et les établissements de saint Louis, tels qu’ils étaient alors pratiqués, et enfin ce qu’on y pratiquait de l’ancienne jurisprudence française.\par
La différence de cet ouvrage d’avec ceux de Desfontaines et de Beaumanoir, c’est qu’on y parle en termes de commandement, comme les législateurs ; et cela pouvait être ainsi, parce qu’il était une compilation de coutumes écrites et de lois.\par
Il y avait un vice intérieur dans cette compilation : elle formait un code amphibie, où l’on avait mêlé la jurisprudence française avec la loi romaine ; on rapprochait des choses qui n’avaient jamais de rapport, et qui souvent étaient contradictoires.\par
Je sais bien que les tribunaux français des hommes ou des pairs, les jugements sans appel à un autre tribunal, la manière de prononcer par ces mots : {\itshape je condamne ou j’absous}\footnote{{\itshape Établissements}, liv. II, chap. XV.}, avaient de la conformité avec les jugements populaires des Romains. Mais on fit peu d’usage de cette ancienne jurisprudence ; on se servit plutôt de celle qui fut introduite depuis par les empereurs, qu’on employa partout dans cette compilation, pour régler, limiter, corriger étendre la jurisprudence française.
\subsubsection[{Chapitre XXXIX. Continuation du même sujet}]{Chapitre XXXIX. Continuation du même sujet}
\noindent Les formes judiciaires introduites par saint Louis cessèrent d’être en usage. Ce prince avait eu moins en vue la chose même, c’est-à-dire la meilleure manière de juger, que la meilleure manière de suppléer à l’ancienne pratique de juger. Le premier objet était de dégoûter de l’ancienne jurisprudence, et le second d’en former une nouvelle. Mais les inconvénients de celle-ci ayant paru, on en vit bientôt succéder une autre.\par
Ainsi les lois de saint Louis changèrent moins la jurisprudence française, qu’elles ne donnèrent des moyens pour la changer : elles ouvrirent de nouveaux tribunaux, ou plutôt des voies pour y arriver ; et, quand on put parvenir aisément à celui qui avait une autorité générale, les jugements, qui auparavant ne faisaient que les usages d’une seigneurie particulière, formèrent une jurisprudence universelle. On était parvenu, par la force des établissements, à avoir des décisions générales, qui manquaient entièrement dans le royaume ; quand le bâtiment fut construit, on laissa tomber l’échafaud.\par
Ainsi les lois que fit saint Louis eurent des effets qu’on n’aurait pas dû attendre du chef-d’œuvre de la législation. Il faut quelquefois bien des siècles pour préparer les changements ; les événements mûrissent, et voilà les révolutions.\par
Le parlement jugea en dernier ressort de presque toutes les affaires du royaume. Auparavant il ne jugeait que de celles qui étaient entre les ducs, comtes, barons, évêques, abbés\footnote{Voyez Dutillet, sur la cour des pairs. Voyez aussi La Roche-Flavin, liv. I, chap. III ; Budé, et Paul Émile.}, ou entre le roi et ses vassaux\footnote{Les autres affaires étaient décidées par les tribunaux ordinaires.}, plutôt dans le rapport qu’elles avaient avec l’ordre politique, qu’avec l’ordre civil. Dans la suite, on fut obligé de le rendre sédentaire, et de le tenir toujours assemblé ; et enfin, on en créa plusieurs, pour qu’ils pussent suffire à toutes les affaires.\par
À peine le parlement fut-il un corps fixe, qu’on commença à compiler ses arrêts. Jean de Monluc, sous le règne de Philippe le Bel, fit le recueil qu’on appelle aujourd’hui les registres Olim\footnote{Voyez l’excellent ouvrage de M. le président Hénault sur l’an 1313.}.
\subsubsection[{Chapitre XL. Comment on prit les formes judiciaires des décrétales}]{Chapitre XL. Comment on prit les formes judiciaires des décrétales}
\noindent Mais d’où vient qu’en abandonnant les formes judiciaires établies, on prit celles du droit canonique plutôt que celles du droit romain ? C’est qu’on avait toujours devant les yeux les tribunaux clercs, qui suivaient les formes du droit canonique, et que l’on ne connaissait aucun tribunal qui suivît celles du droit romain. De plus, les bornes de la juridiction ecclésiastique et de la séculière étaient, dans ces temps-là, très peu connues : il y avait des gens\footnote{Beaumanoir, chap. XI, p. 58.} qui plaidaient indifféremment dans les deux cours\footnote{Les femmes veuves, les croisés, ceux qui tenaient les biens des églises, pour raison de ces biens. {\itshape Ibid.}} {\itshape ;} il y avait des matières pour lesquelles on plaidait de même. Il semble\footnote{Voyez tout le chapitre XI de Beaumanoir.} que la juridiction laie ne se fût gardé, privativement à l’autre, que le jugement des matières féodales, et des crimes commis par les laïques dans les cas qui ne choquaient pas la religion\footnote{Les tribunaux clercs, sous prétexte du serment, s’en étaient même saisis, comme on le voit par le fameux concordat passé entre Philippe Auguste, les clercs et les barons, qui se trouve dans les {\itshape Ordonnances} de Laurière.}. Car si, pour raison des conventions et des contrats, il fallait aller à la justice laie, les parties pouvaient volontairement procéder devant les tribunaux clercs, qui, n’étant pas en droit d’obliger la justice laie à faire exécuter la sentence, contraignaient d’y obéir par voie d’excommunication\footnote{Beaumanoir, chap. XI, p. 60.}. Dans ces circonstances, lorsque, dans les tribunaux laïques, on voulut changer de pratique, on prit celle des clercs, parce qu’on la savait ; et on ne prit pas celle du droit romain, parce qu’on ne la savait point : car, en fait de pratique, on ne sait que ce que l’on pratique.
\subsubsection[{Chapitre XLI. Flux et reflux de la juridiction ecclésiastique et de la juridiction laie}]{Chapitre XLI. Flux et reflux de la juridiction ecclésiastique et de la juridiction laie}
\noindent La puissance civile étant entre les mains d’une infinité de seigneurs, il avait été aisé à la juridiction ecclésiastique de se donner tous les jours plus d’étendue : mais, comme la juridiction ecclésiastique énerva la juridiction des seigneurs, et contribua par là à donner des forces à la juridiction royale, la juridiction royale restreignit peu à peu la juridiction ecclésiastique, et celle-ci recula devant la première. Le parlement, qui avait pris dans sa forme de procéder tout ce qu’il y avait de bon et d’utile dans celle des tribunaux des clercs, ne vit bientôt plus que ses abus ; et la juridiction royale se fortifiant tous les jours, elle fut toujours plus en état de corriger ces mêmes abus. En effet, ils étaient intolérables ; et, sans en faire l’énumération, je renverrai à Beaumanoir, à Boutillier, aux ordonnances de nos rois\footnote{Voyez Boutillier, {\itshape Somme rurale}, tit. IX, quelles personnes ne peuvent faire demande en cour laie ; et Beaumanoir, chap. XI, p. 56 ; et les règlements de Philippe Auguste à ce sujet ; et l’établissement de Philippe Auguste fait entre les clercs, le roi et les barons.}. Je ne parlerai que de ceux qui intéressaient plus directement la fortune publique. Nous connaissons ces abus par les arrêts qui les réformèrent. L’épaisse ignorance les avait introduits ; une espèce de clarté parut, et ils ne furent plus. On peut juger, par le silence du clergé, qu’il alla lui-même au-devant de la correction ; ce qui, vu la nature de l’esprit humain, mérite des louanges. Tout homme qui mourait sans donner une partie de ses biens à l’église, ce qui s’appelait mourir {\itshape déconfés}, était privé de la communion et de la sépulture. Si l’on mourait sans faire de testament, il fallait que les parents obtinssent de l’évêque qu’il nommât, concurremment avec eux, des arbitres, pour fixer ce que le défunt aurait dû donner en cas qu’il eût fait un testament. On ne pouvait pas coucher ensemble la première nuit des noces, ni même les deux suivantes, sans en avoir acheté la permission ; c’était bien ces trois nuits-là qu’il fallait choisir, car, pour les autres on n’aurait pas donné beaucoup d’argent. Le parlement corrigea tout cela. On trouve, dans le {\itshape Glossaire du droit français} de Ragueau\footnote{Au mot Exécuteurs testamentaires.} l’arrêt qu’il rendit contre l’évêque d’Amiens\footnote{Du 19 mars 1409.}.\par
Je reviens au commencement de mon chapitre. Lorsque, dans un siècle, ou dans un gouvernement, on voit les divers corps de l’État chercher à augmenter leur autorité, et à prendre les uns sur les autres de certains avantages, on se tromperait souvent si l’on regardait leurs entreprises comme une marque certaine de leur corruption. Par un malheur attaché à la condition humaine, les grands hommes modérés sont rares ; et, comme il est toujours plus aisé de suivre sa force que de l’arrêter, peut-être, dans la classe des gens supérieurs, est-il plus facile de trouver des gens extrêmement vertueux, que des hommes extrêmement sages.\par
L’âme goûte tant de délices à dominer les autres âmes ; ceux mêmes qui aiment le bien s’aiment si fort eux-mêmes, qu’il n’y a personne qui ne soit assez malheureux pour avoir encore à se défier de ses bonnes intentions : et, en vérité, nos actions tiennent à tant de choses, qu’il est mille fois plus aisé de faire le bien, que de le bien faire.
\subsubsection[{Chapitre XLII. Renaissance du droit romain et ce qui en résulta. Changements dans les tribunaux}]{Chapitre XLII. Renaissance du droit romain et ce qui en résulta. Changements dans les tribunaux}
\noindent Le {\itshape Digeste} de Justinien ayant été retrouvé vers l’an 1137, le droit romain sembla prendre une seconde naissance. On établit des écoles en Italie, où on l’enseignait : on avait déjà le Code Justinien et les {\itshape Novelles}. J’ai déjà dit que ce droit y prit une telle faveur, qu’il fit éclipser la loi des Lombards.\par
Des docteurs italiens portèrent le droit de Justinien en France, où l’on n’avait connu que le Code Théodosien\footnote{On suivait en Italie le code de Justinien. C’est pour cela que le pape Jean VIII, dans sa constitution donnée après le synode de Troyes, parle de ce code, non pas parce qu’il était connu en France, mais parce qu’il le connaissait lui-même ; et sa constitution était générale.}, parce que ce ne fut qu’après l’établissement des barbares dans les Gaules, que les lois de Justinien furent faites\footnote{Le code de cet empereur fut publié vers l’an 530.}. Ce droit reçut quelques oppositions ; mais il se maintint, malgré les excommunications des papes, qui protégeaient leurs canons\footnote{Décrétales, liv. V, tit. {\itshape De privilegiis, capite super specula}.}. Saint Louis chercha à l’accréditer, par les traductions qu’il fit faire des ouvrages de Justinien, que nous avons encore manuscrites dans nos bibliothèques ; et j’ai déjà dit qu’on en fit un grand usage dans les {\itshape Établissements}. Philippe le Bel\footnote{Par une charte de l’an 1312, en faveur de l’université d’Orléans, rapportée par Dutillet.} fit enseigner les lois de Justinien, seulement comme raison écrite, dans les pays de la France qui se gouvernaient par les coutumes ; et elles furent adoptées comme loi, dans les pays où le droit romain était la loi.\par
J’ai dit ci-dessus que la manière de procéder par le combat judiciaire demandait, dans ceux qui jugeaient, très peu de suffisance ; on décidait les affaires dans chaque lieu, selon l’usage de chaque lieu, et suivant quelques coutumes simples, qui se recevaient par tradition. Il y avait, du temps de Beaumanoir\footnote{{\itshape Coutume de Beauvaisis}, chap. I, de l’office des baillis.}, deux différentes manières de rendre la justice : dans des lieux, on jugeait par pairs ; dans d’autres, on jugeait par baillis\footnote{Dans la commune, les bourgeois étaient jugés par d’autres bourgeois, comme les hommes de fief se jugeaient entre eux. Voyez La Thaumassière, chap. XIX.}. Quand on suivait la première forme, les pairs jugeaient selon l’usage de leur juridiction\footnote{Aussi toutes les requêtes commençaient-elles par ces mots : « Sire juge, il est d’usage qu’en votre juridiction, etc. », comme il paraît par la formule rapportée dans Boutillier, {\itshape Somme rurale}, liv. I, tit. XXI.} {\itshape ;} dans la seconde, c’étaient des prud’hommes ou vieillards qui indiquaient au bailli le même usage. Tout ceci ne demandait aucunes lettres, aucune capacité, aucune étude. Mais, lorsque le code obscur des {\itshape Établissements} et d’autres ouvrages de jurisprudence parurent ; lorsque le droit romain fut traduit ; lorsqu’il commença à être enseigné dans les écoles ; lorsqu’un certain art de la procédure et qu’un certain art de la jurisprudence commencèrent à se former ; lorsqu’on vit naître des praticiens et des jurisconsultes, les pairs et les prud’hommes ne furent plus en état de juger ; les pairs commencèrent à se retirer des tribunaux du seigneur ; les seigneurs furent peu portés à les assembler : d’autant mieux que les jugements, au lieu d’être une action éclatante, agréable à la noblesse, intéressante pour les gens de guerre, n’étaient plus qu’une pratique qu’ils ne savaient, ni ne voulaient savoir. La pratique de juger par pairs devint moins en usage\footnote{Le changement fut insensible. On trouve encore les pairs employés du temps de Boutillier, qui vivait en 1402, date de son testament, qui rapporte cette formule au liv. I, tit. XXI : « Sire juge, en ma justice haute, moyenne, et basse, que j’ai en tel lieu, cour, plaids, baillis, hommes féodaux et sergents. » Mais il n’y avait plus que les matières féodales qui se jugeassent par pairs. {\itshape Ibid.}, liv. I, tit. I, p. 16.} ; celle de juger par baillis s’étendit. Les baillis ne jugeaient pas\footnote{Comme il paraît par la formule des lettres que le seigneur leur donnait, rapportée par Boutillier, {\itshape Somme rurale}, liv.{\itshape  I}, tit. XIV. Ce qui se prouve encore par Beaumanoir, {\itshape Coutume de Beauvaisis}, chap. I, Des baillis. Ils ne faisaient que la procédure. « {\itshape Le bailli est tenu en la présence des hommes à penre les paroles de chaux qui plaident, et doit demander as parties se ils veulent avoir droit selon les raisons que ils ont dites ; et se ils disent}, Sire, oïl{\itshape , le bailli doit contraindre les hommes que ils fassent le jugement.} » Voyez aussi les {\itshape Établissements de saint Louis}, liv. I, chap. CV ; et liv. II, chap. XV : « {\itshape Li juge, si ne doit pas faire le jugement.} »} : ils faisaient l’instruction, et prononçaient le jugement des prud’hommes ; mais les prud’hommes n’étant plus en état de juger, les baillis jugèrent eux-mêmes.\par
Cela se fit d’autant plus aisément qu’on avait devant les yeux la pratique des juges d’église : le droit canonique et le nouveau droit civil concoururent également à abolir les pairs.\par
Ainsi se perdit l’usage, constamment observé dans la monarchie, qu’un juge ne jugeait jamais seul, comme on le voit par les lois saliques, les capitulaires, et par les premiers écrivains de pratique de la troisième race\footnote{Beaumanoir, chap. LXVII, p. 336 ; et chap. LXI, p. 315 et 316 ; les {\itshape Établissements}, liv. II, chap. XV.}. L’abus contraire, qui n’a lieu que dans les justices locales, a été modéré, et en quelque façon corrigé, par l’introduction en plusieurs lieux d’un lieutenant du juge, que celui-ci consulte, et qui représente les anciens prud’hommes ; par l’obligation où est le juge de prendre deux gradués dans les cas qui peuvent mériter une peine afflictive ; et enfin il est devenu nul par l’extrême facilité des appels.
\subsubsection[{Chapitre XLIII. Continuation du même sujet}]{Chapitre XLIII. Continuation du même sujet}
\noindent Ainsi ce ne fut point une loi qui défendit aux seigneurs de tenir eux-mêmes leur cour ; ce ne fut point une loi qui abolit les fonctions que leurs pairs y avaient ; il n’y eut point de loi qui ordonnât de créer des baillis ; ce ne fut point par une loi qu’ils eurent le droit de juger. Tout cela se fit peu à peu, et par la force de la chose. La connaissance du droit romain, des arrêts des cours, des corps de coutumes nouvellement écrites, demandait une étude dont les nobles et le peuple sans lettres n’étaient point capables.\par
La seule ordonnance que nous ayons sur cette matière\footnote{Elle est de l’an 1287.}, est celle qui obligea les seigneurs de choisir leurs baillis dans l’ordre des laïques. C’est mal à propos qu’on l’a regardée comme la loi de leur création ; mais elle ne dit que ce qu’elle dit. De plus, elle fixe ce qu’elle prescrit par les raisons qu’elle en donne : « C’est afin, est-il dit, que les baillis puissent être punis de leurs prévarications, qu’il faut qu’ils soient pris dans l’ordre des laïques\footnote{{\itshape Ut, si ibi delinquant, superiores sui possint animadvertere in eosdem.}}. » On sait les privilèges des ecclésiastiques dans ces temps-là.\par
Il ne faut pas croire que les droits dont les seigneurs jouissaient autrefois, et dont ils ne jouissent plus aujourd’hui, leur aient été ôtés comme des usurpations : plusieurs de ces droits ont été perdus par négligence ; et d’autres ont été abandonnés, parce que divers changements s’étant introduits dans le cours de plusieurs siècles, ils ne pouvaient subsister avec ces changements.
\subsubsection[{Chapitre XLIV. De la preuve par témoins}]{Chapitre XLIV. De la preuve par témoins}
\noindent Les juges, qui n’avaient d’autres règles que les usages, s’en enquéraient ordinairement par témoins, dans chaque question qui se présentait.\par
Le combat judiciaire devenant moins en usage, on fit les enquêtes par écrit. Mais une preuve vocale mise par écrit n’est jamais qu’une preuve vocale ; cela ne faisait qu’augmenter les frais de la procédure. On fit des règlements qui rendirent la plupart de ces enquêtes inutiles\footnote{Voyez comment on prouvait l’âge et la parenté : {\itshape Établissements}, liv. I, chap. LXXXI et LXXII.} ; on établit des registres publics, dans lesquels la plupart des faits se trouvaient prouvés : la noblesse, l’âge, la légitimité, le mariage. L’écriture est un témoin qui est difficilement corrompu. On fit rédiger par écrit les coutumes. Tout cela était bien raisonnable : il est plus aisé d’aller chercher dans les registres de baptême si Pierre est fils de Paul, que d’aller prouver ce fait par une longue enquête. Quand, dans un pays, il y a un très grand nombre d’usages, il est plus aisé de les écrire tous dans un code que d’obliger les particuliers à prouver chaque usage. Enfin, on fit la fameuse ordonnance qui défendit de recevoir la {\itshape preuve} par témoins pour une dette au-dessus de cent livres, à moins qu’il n’y eût un commencement de preuve par écrit.
\subsubsection[{Chapitre XLV. Des coutumes de France}]{Chapitre XLV. Des coutumes de France}
\noindent La France était régie, comme j’ai dit, par des coutumes non écrites ; et les usages particuliers de chaque seigneurie formaient le droit civil. Chaque seigneurie avait son droit civil, comme le dit Beaumanoir\footnote{Prologue sur la Coutume de Beauvaisis.} ; et un droit si particulier, que cet auteur, qu’on doit regarder comme la lumière de ce temps-là, et une grande lumière, dit qu’il ne croit pas que dans tout le royaume il y eût deux seigneuries qui fussent gouvernées de tout point par la même loi.\par
Cette prodigieuse diversité avait une première origine, et elle en avait une seconde. Pour la première, on peut se souvenir de ce que j’ai dit ci-dessus au chapitre des coutumes locales\footnote{Chap. XII.} ; et, quant à la seconde, on la trouve dans les divers événements des combats judiciaires ; des cas continuellement fortuits devant introduire naturellement de nouveaux usages.\par
Ces coutumes-là étaient conservées dans la mémoire des vieillards ; mais il se forma peu à peu des lois ou des coutumes écrites.\par
1˚ Dans le commencement de la troisième race\footnote{Voyez le recueil des ordonnances de Laurière.}, les rois donnèrent des chartes particulières, et en donnèrent même de générales, de la manière dont je l’ai expliqué ci-dessus : tels sont les établissements de Philippe Auguste, et ceux que fit saint Louis. De même, les grands vassaux, de concert avec les seigneurs qui tenaient d’eux, donnèrent, dans les assises de leurs duchés ou comtés, de certaines chartes ou établissements, selon les circonstances : telles furent l’assise de Geoffroi, comte de Bretagne, sur le partage des nobles ; les coutumes de Normandie, accordées par le duc Raoul ; les coutumes de Champagne, données par le roi Thibaut, les lois de Simon, comte de Montfort, et autres. Cela produisit quelques lois écrites, et même plus générales que celles que l’on avait.\par
2˚ Dans le commencement de la troisième race, presque tout {\itshape le} bas {\itshape peuple} était serf. Plusieurs raisons obligèrent les rois et les seigneurs de les affranchir.\par
Les seigneurs, en affranchissant leurs serfs, leur donnèrent des biens ; il fallut leur donner des lois civiles pour régler la disposition de ces biens. Les seigneurs, en affranchissant leurs serfs, se privèrent de leurs biens ; il fallut donc régler les droits que les seigneurs se réservaient pour l’équivalent de leur bien. L’une et l’autre de ces choses furent réglées par les chartes d’affranchissement ; ces chartes formèrent une partie de nos coutumes, et cette partie se trouva rédigée par écrit.\par
3˚ Sous le règne de saint Louis et les suivants, des praticiens habiles, tels que Desfontaines, Beaumanoir, et autres, rédigèrent par écrit les coutumes de leurs bailliages. Leur objet était plutôt de donner une pratique judiciaire, que les usages de leur temps sur la disposition des biens. Mais tout s’y trouve ; et, quoique ces auteurs particuliers n’eussent d’autorité que par la vérité et la publicité des choses qu’ils disaient, on ne peut douter qu’elles n’aient beaucoup servi à la renaissance de notre droit français. Tel était, dans ces temps-là, notre droit coutumier écrit.\par
Voici la grande époque. Charles VII et ses successeurs firent rédiger par écrit, dans tout le royaume, les diverses coutumes locales, et prescrivirent des formalités qui devaient être observées à leur rédaction. Or, comme cette rédaction se fit par provinces, et que, de chaque seigneurie, on venait déposer dans l’assemblée générale de la province les usages écrits ou non écrits de chaque lieu, on chercha à rendre les coutumes plus générales, autant que cela se put faire sans blesser les intérêts des particuliers qui furent réservés\footnote{Cela se fit ainsi lors de la rédaction des coutumes de Berry et de Paris. Voyez La Thaumassière, chap. III.}. Ainsi nos coutumes prirent trois caractères : elles furent écrites, elles furent plus générales, elles reçurent le sceau de l’autorité royale.\par
Plusieurs de ces coutumes ayant été de nouveau rédigées, on y fit plusieurs changements, soit en ôtant tout ce qui ne pouvait compatir avec la jurisprudence actuelle, soit en ajoutant plusieurs choses tirées de cette jurisprudence.\par
Quoique le droit coutumier soit regardé parmi nous comme contenant une espèce d’opposition avec le droit romain, de sorte que ces deux droits divisent les territoires, il est pourtant vrai que plusieurs dispositions du droit romain sont entrées dans nos coutumes, surtout lorsqu’on en fit de nouvelles rédactions, dans des temps qui ne sont pas fort éloignés des nôtres, où ce droit était l’objet des connaissances de tous ceux qui se destinaient aux emplois civils ; dans des temps où l’on ne faisait pas gloire d’ignorer ce que l’on doit savoir, et de savoir ce que l’on doit ignorer ; où la facilité de l’esprit servait plus à apprendre sa profession qu’à la faire ; et où les amusements continuels n’étaient pas même l’attribut des femmes.\par
Il aurait fallu que je m’étendisse davantage à la fin de ce livre ; et qu’entrant dans de plus grands détails, j’eusse suivi tous les changements insensibles qui, depuis l’ouverture des appels, ont formé le grand corps de notre jurisprudence française. Mais j’aurais mis un grand ouvrage dans un grand ouvrage. Je suis comme cet antiquaire qui partit de son pays, arriva en Égypte, jeta un coup d’œil sur les Pyramides, et s’en retourna\footnote{Dans le {\itshape Spectateur anglais}.}.
\subsection[{Livre vingt-neuvième. De la manière de composer les lois}]{Livre vingt-neuvième. De la manière de composer les lois}
\subsubsection[{Chapitre I. De l’esprit du législateur}]{Chapitre I. De l’esprit du législateur}
\noindent Je le dis, et il me semble que je n’ai fait cet ouvrage que pour le prouver : l’esprit de modération doit être celui du législateur ; le bien politique, comme le bien moral, se trouve toujours entre deux limites. En voici un exemple.\par
Les formalités de la justice sont nécessaires à la liberté. Mais le nombre en pourrait être si grand qu’il choquerait le but des lois mêmes qui les auraient établies : les affaires n’auraient point de fin ; la propriété des biens resterait incertaine ; on donnerait à l’une des parties le bien de l’autre sans examen, ou on les ruinerait toutes les deux à force d’examiner.\par
Les citoyens perdraient leur liberté et leur sûreté, les accusateurs n’auraient plus les moyens de convaincre, ni les accusés le moyen de se justifier.
\subsubsection[{Chapitre II. Continuation du même sujet}]{Chapitre II. Continuation du même sujet}
\noindent Cécilius, dans Aulu-Gelle\footnote{Liv. XX, chap. I.}, discourant sur la loi des Douze Tables, qui permettait au créancier de couper en morceaux le débiteur insolvable, la justifie par son atrocité même, qui empêchait qu’on n’empruntât au-delà de ses facultés\footnote{Cécilius dit qu’il n’a jamais vu ni lu que cette peine eût été infligée ; mais il y a apparence qu’elle n’a jamais été établie. L’opinion de quelques jurisconsultes, que la loi des Douze Tables ne parlait que de la division du prix du débiteur vendu, est très vraisemblable.}. Les lois les plus cruelles seront donc les meilleures ? Le bien sera l’excès, et tous les rapports des choses seront détruits ?
\subsubsection[{Chapitre III. Qui, les lois qui paraissent s’éloigner des vues du législateur y sont souvent conformes}]{Chapitre III. Qui, les lois qui paraissent s’éloigner des vues du législateur y sont souvent conformes}
\noindent La loi de Solon, qui déclarait infâmes tous ceux qui, dans une sédition, ne prendraient aucun parti, a paru bien extraordinaire : mais il faut faire attention aux circonstances dans lesquelles la Grèce se trouvait pour lors. Elle était partagée en de très petits États : il était à craindre que, dans une république travaillée par des dissensions civiles, les gens les plus prudents ne se missent à couvert, et que par là les choses ne fussent portées à l’extrémité.\par
Dans les séditions qui arrivaient dans ces petits États, le gros de la cité entrait dans la querelle, ou la faisait. Dans nos grandes monarchies, les partis sont formés par peu de gens, et le peuple voudrait vivre dans l’inaction. Dans ce cas, il est naturel de rappeler les séditieux au gros des citoyens, non pas le gros des citoyens aux séditieux : dans l’autre, il faut faire rentrer le petit nombre de gens sages et tranquilles parmi les séditieux : c’est ainsi que la fermentation d’une liqueur peut être arrêtée par une seule goutte d’une autre.
\subsubsection[{Chapitre IV. Des lois qui choquent les vues du législateur}]{Chapitre IV. Des lois qui choquent les vues du législateur}
\noindent Il y a des lois que le législateur a si peu connues, qu’elles sont contraires au but même qu’il s’est proposé. Ceux qui ont établi chez les Français que, lorsqu’un des deux prétendants à un bénéfice meurt, le bénéfice reste à celui qui survit, ont cherché sans doute à éteindre les affaires. Mais il en résulte un effet contraire : on voit les ecclésiastiques s’attaquer et se battre, comme des dogues anglais, jusqu’à la mort.
\subsubsection[{Chapitre V. Continuation du même sujet}]{Chapitre V. Continuation du même sujet}
\noindent La loi dont je vais parler se trouve dans ce serment, qui nous a été conservé par Eschine\footnote{{\itshape De falsa legatione}.} : « Je jure que je ne détruirai jamais une ville des Amphictyons, et que je ne détournerai point ses eaux courantes : si quelque peuple ose faire quelque chose de pareil, je lui déclarerai la guerre, et je détruirai ses villes. » Le dernier article de cette loi, qui paraît confirmer le premier, lui est réellement contraire. Amphictyon veut qu’on ne détruise jamais les villes grecques, et sa loi ouvre la porte à la destruction de ces villes. Pour établir un bon droit des gens parmi les Grecs, il fallait les accoutumer à penser que c’était une chose atroce de détruire une ville grecque ; il ne devait donc pas détruire même les destructeurs. La loi d’Amphictyon était juste, mais elle n’était pas prudente. Cela se prouve par l’abus même que l’on en fit. Philippe ne se fit-il pas donner le pouvoir de détruire les villes, sous prétexte qu’elles avaient violé les lois des Grecs ? Amphictyon aurait pu infliger d’autres peines : ordonner, par exemple, qu’un certain nombre de magistrats de la ville destructrice, ou de chefs de l’armée violatrice, seraient punis de morts ; que le peuple destructeur cesserait, pour un temps, de jouir des privilèges des Grecs ; qu’il paierait une amende jusqu’au rétablissement de la ville. La loi devait surtout porter sur la réparation du dommage.
\subsubsection[{Chapitre VI. Que les lois qui paraissent les mêmes n’ont pas toujours le même effet}]{Chapitre VI. Que les lois qui paraissent les mêmes n’ont pas toujours le même effet}
\noindent César\footnote{Dion, liv. XLI.} défendit de garder chez soi plus de soixante sesterces. Cette loi fut regardée à Rome comme très propre à concilier les débiteurs avec les créanciers ; parce qu’en obligeant les riches à prêter aux pauvres, elle mettait ceux-ci en état de satisfaire les riches. Une même loi, faite en France, du temps du Système, fut très funeste : c’est que la circonstance dans laquelle on la fit était affreuse. Après avoir ôté tous les moyens de placer son argent, on ôta même la ressource de le garder chez soi ; ce qui était égal à un enlèvement fait par violence. César fit sa loi pour que l’argent circulât parmi le peuple ; le ministre de France fit la sienne pour que l’argent fût mis dans une seule main. Le premier donna pour de l’argent des fonds de terre, ou des hypothèques sur des particuliers ; le second proposa pour de l’argent des effets qui n’avaient point de valeur, et qui n’en pouvaient avoir par leur nature, par la raison que sa loi obligeait de les prendre.
\subsubsection[{Chapitre VII. Continuation du même sujet. Nécessité de bien composer les lois}]{Chapitre VII. Continuation du même sujet. Nécessité de bien composer les lois}
\noindent La loi de l’ostracisme fut établie à Athènes, à Argos et à Syracuse\footnote{Aristote, {\itshape République}, liv. V, chap. III.}. À Syracuse elle fit mille maux, parce qu’elle fut faite sans prudence. Les principaux citoyens se bannissaient les uns les autres, en se mettant une feuille de figuier à la main\footnote{Plutarque, {\itshape Vie de Denys.}} ; de sorte que ceux qui avaient quelque mérite quittèrent les affaires. À Athènes, où le législateur avait senti l’extension et les bornes qu’il devait donner à sa loi, l’ostracisme fut une chose admirable : on n’y soumettait jamais qu’une seule personne ; il fallait un si grand nombre de suffrages, qu’il était difficile qu’on exilât quelqu’un dont l’absence ne fût pas nécessaire.\par
On ne pouvait bannir que tous les cinq ans : en effet, dès que l’ostracisme ne devait s’exercer que contre un grand personnage qui donnerait de la crainte à ses concitoyens, ce ne devait pas être une affaire de tous les jours.
\subsubsection[{Chapitre VIII. Que les lois qui paraissent les mêmes n’ont pas toujours eu le même motif}]{Chapitre VIII. Que les lois qui paraissent les mêmes n’ont pas toujours eu le même motif}
\noindent On reçoit en France la plupart des lois des Romains sur les substitutions ; mais les substitutions y ont tout un autre motif que chez les Romains. Chez ceux-ci, l’hérédité était jointe à de certains sacrifices qui devaient être faits par l’héritier, et qui étaient réglés par le droit des pontifes\footnote{Lorsque l’hérédité était trop chargée, on éludait le droit des pontifes par de certaines ventes, d’où vint le mot {\itshape sine sacris haereditas.}}. Cela fit qu’ils tinrent à déshonneur de mourir sans héritier, qu’ils prirent pour héritiers leurs esclaves, et qu’ils inventèrent les substitutions. La substitution vulgaire, qui fut la première inventée, et qui n’avait lieu que dans le cas où l’héritier institué n’accepterait pas l’hérédité, en est une grande preuve : elle n’avait point pour objet de perpétuer l’héritage dans une famille du même nom, mais de trouver quelqu’un qui acceptât l’héritage.
\subsubsection[{Chapitre IX. Que les lois grecques et romaines ont puni l’homicide de soi-même, sans avoir le même motif}]{Chapitre IX. Que les lois grecques et romaines ont puni l’homicide de soi-même, sans avoir le même motif}
\noindent Un homme, dit Platon\footnote{Liv. IX des {\itshape Lois}.}, qui a tué celui qui lui est étroitement lié, c’est-à-dire lui-même, non par ordre du magistrat, ni pour éviter l’ignominie, mais par faiblesse, sera puni. La loi romaine punissait cette action, lorsqu’elle n’avait pas été faite par faiblesse d’âme, par ennui de la vie, par impuissance de souffrir la douleur, mais par le désespoir de quelque crime. La loi romaine absolvait dans le cas où la grecque condamnait, et condamnait dans le cas où l’autre absolvait.\par
La loi de Platon était formée sur les institutions lacédémoniennes, où les ordres du magistrat étaient totalement absolus, où l’ignominie était le plus grand des malheurs, et la faiblesse le plus grand des crimes. La loi romaine abandonnait toutes ces belles idées ; elle n’était qu’une loi fiscale.\par
Du temps de la République, il n’y avait point de loi à Rome qui punît ceux qui se tuaient eux-mêmes : cette action, chez les historiens, est toujours prise en bonne part, et l’on n’y voit jamais de punition contre ceux qui l’ont faite.\par
Du temps des premiers empereurs, les grandes familles de Rome furent sans cesse exterminées par des jugements. La coutume s’introduisit de prévenir la condamnation par une mort volontaire. On y trouvait un grand avantage. On obtenait l’honneur de la sépulture, et les testaments étaient exécutés\footnote{{\itshape Eorum qui de se statuebant, humabantur corpora, manebant testamenta, pretium festinandi.} Tacite.} {\itshape ;} cela venait de ce qu’il n’y avait point de loi civile à Rome contre ceux qui se tuaient eux-mêmes. Mais lorsque les empereurs devinrent aussi avares qu’ils avaient été cruels, ils ne laissèrent plus à ceux dont ils voulaient se défaire le moyen de conserver leurs biens, et ils déclarèrent que ce serait un crime de s’ôter la vie par les remords d’un autre crime.\par
Ce que je dis du motif des empereurs est si vrai, qu’ils consentirent que les biens\footnote{Rescrit de l’empereur Pie, dans la loi 3, § 1 et 2, ff. {\itshape De bonis eorum qui ante sententiam mortem sibi consciverunt.}} de ceux qui se seraient tués eux-mêmes ne fussent pas confisqués, lorsque le crime pour lequel ils s’étaient tués n’assujettissait point à la confiscation.
\subsubsection[{Chapitre X. Que les lois qui paraissent contraires dérivent quelquefois du même esprit}]{Chapitre X. Que les lois qui paraissent contraires dérivent quelquefois du même esprit}
\noindent On va aujourd’hui dans la maison d’un homme pour l’appeler en jugement ; cela ne pouvait se faire chez les Romains\footnote{Leg. 18, ff. {\itshape De in jus vocando}.}.\par
L’appel en jugement était une action violente\footnote{Voyez la loi des Douze Tables.}, et comme une espèce de contrainte par corps\footnote{{\itshape Rapit in jus}, Horace, {\itshape Satire IX.} C’est pour cela qu’on ne pouvait appeler en jugement ceux à qui on devait un certain respect.} {\itshape ;} et on ne pouvait pas plus aller dans la maison d’un homme pour l’appeler en jugement, qu’on ne peut aller aujourd’hui contraindre par corps dans sa maison un homme qui n’est condamné que pour des dettes civiles.\par
Les lois romaines\footnote{Voyez la loi 18, ff. {\itshape De in jus vocando}.} et les nôtres admettent également ce principe, que chaque citoyen a sa maison pour asile, et qu’il n’y doit recevoir aucune violence.
\subsubsection[{Chapitre XI. De quelle manière deux lois diverses peuvent être comparées}]{Chapitre XI. De quelle manière deux lois diverses peuvent être comparées}
\noindent En France, la peine contre les faux témoins est capitale ; en Angleterre, elle ne l’est point. Pour juger laquelle de ces deux lois est la meilleure, il faut ajouter : en France, la question contre les criminels est pratiquée ; en Angleterre elle ne l’est point ; et dire encore : en France, l’accusé ne produit point ses témoins, et il est très rare qu’on y admette ce que l’on appelle les faits justificatifs ; en Angleterre, l’on reçoit les témoignages de part et d’autre. Les trois lois françaises forment un système très lié et très suivi ; les trois lois anglaises en forment un qui ne l’est pas moins. La loi d’Angleterre, qui ne connaît point la question contre les criminels, n’a que peu d’espérance de tirer de l’accusé la confession de son crime ; elle appelle donc de tous côtés les témoignages étrangers, et elle n’ose les décourager par la crainte d’une peine capitale. La loi française, qui a une ressource de plus, ne craint pas tant d’intimider les témoins ; au contraire, la raison demande qu’elle les intimide : elle n’écoute que les témoins d’une part\footnote{Par l’ancienne jurisprudence française, les témoins étaient ouïs des deux parts. Aussi voit-on, dans les {\itshape Établissements} de saint Louis, liv. I, chap. VII, que la peine contre les faux témoins en justice était pécuniaire.} ; ce sont ceux que produit la partie publique ; et le destin de l’accusé dépend de leur seul témoignage. Mais, en Angleterre, on reçoit les témoins des deux parts, et l’affaire est, pour ainsi dire, discutée entre eux. Le faux témoignage y peut donc être moins dangereux ; l’accusé y a une ressource contre le faux témoignage, au lieu que la loi française n’en donne point. Ainsi, pour juger lesquelles de ces lois sont les plus conformes à la raison, il ne faut pas comparer chacune de ces lois à chacune ; il faut les prendre toutes ensemble, et les comparer toutes ensemble.
\subsubsection[{Chapitre XII. Que les lois qui paraissent les mêmes sont quelquefois réellement différentes}]{Chapitre XII. Que les lois qui paraissent les mêmes sont quelquefois réellement différentes}
\noindent Les lois grecques et romaines punissaient le receleur du vol comme le voleur\footnote{Leg. 1, ff. {\itshape De receptatoribus}.} : la loi française fait de même. Celles-là étaient raisonnables, celle-ci ne l’est pas. Chez les Grecs et chez les Romains, le voleur étant condamné à une peine pécuniaire, il fallait punir le receleur de la même peine : car tout homme qui contribue de quelque façon que ce soit à un dommage, doit le réparer. Mais, parmi nous, la peine du vol étant capitale, on n’a pas pu, sans outrer les choses, punir le receleur comme le voleur. Celui qui reçoit le vol peut en mille occasions le recevoir innocemment ; celui qui vole est toujours coupable : l’un empêche la conviction d’un crime déjà commis, l’autre commet ce crime : tout est passif dans l’un, il y a une action dans l’autre : il faut que le voleur surmonte plus d’obstacles, et que son âme se raidisse plus longtemps contre les lois.\par
Les jurisconsultes ont été plus loin : ils ont regardé le receleur comme plus odieux que le voleur\footnote{Leg. 1, ff. {\itshape De receptatoribus}.} ; car sans eux, disent-ils, le vol ne pourrait être caché longtemps. Cela, encore une fois, pouvait être bon, quand la peine était pécuniaire ; il s’agissait d’un dommage, et le receleur était ordinairement plus en état de le réparer ; mais la peine devenue capitale, il aurait fallu se régler sur d’autres principes.
\subsubsection[{Chapitre XIII. Qu’il ne faut point séparer les lois de l’objet pour lequel elles sont faites. Des lois romaines sur le vol}]{Chapitre XIII. Qu’il ne faut point séparer les lois de l’objet pour lequel elles sont faites. Des lois romaines sur le vol}
\noindent Lorsque le voleur était surpris avec la chose volée, avant qu’il l’eût portée dans le lieu où il avait résolu de la cacher, cela était appelé chez les Romains un vol manifeste ; quand le voleur n’était découvert qu’après, c’était un vol non manifeste.\par
La loi des Douze Tables ordonnait que le voleur manifeste fût battu de verges et réduit en servitude, s’il était pubère ; ou seulement battu de verges, s’il était impubère : elle ne condamnait le voleur non manifeste qu’au payement du double de la chose volée.\par
Lorsque la loi Porcia eut aboli l’usage de battre de verges les citoyens, et de les réduire en servitude, le voleur manifeste fut condamné au quadruple\footnote{Voyez ce que dit Favorinus sur Aulu-Gelle, liv. XX, chap. I.}, et on continua à punir du double le voleur non manifeste.\par
Il paraît bizarre que ces lois missent une telle différence dans la qualité de ces deux crimes, et dans la peine qu’elles infligeaient : en effet, que le voleur fût surpris avant ou après avoir porté le vol dans le lieu de sa destination, c’était une circonstance qui ne changeait point la nature du crime. Je ne saurais douter que toute la théorie des lois romaines sur le vol ne fût tirée des institutions lacédémoniennes. Lycurgue, dans la vue de donner à ses citoyens de l’adresse, de la ruse et de l’activité, voulut qu’on exerçât les enfants au larcin, et qu’on fouettât rudement ceux qui s’y laisseraient surprendre : cela établit chez les Grecs, et ensuite chez les Romains, une grande différence entre le vol manifeste et le vol non manifeste\footnote{Conférez ce que dit Plutarque, {\itshape Vie de Lycurgue}, avec les lois du {\itshape Digeste}, au titre {\itshape De furtis} et les {\itshape Institutes}, liv. IV, tit. I, § 1, 2 et 3.}.\par
Chez les Romains, l’esclave qui avait volé était précipité de la roche Tarpéienne. Là, il n’était point question des institutions lacédémoniennes ; les lois de Lycurgue sur le vol n’avaient point été faites pour les esclaves ; c’était les suivre que de s’en écarter en ce point.\par
À Rome, lorsqu’un impubère avait été surpris dans le vol, le préteur le faisait battre de verges à sa volonté, comme on faisait à Lacédémone. Tout ceci venait de plus loin. Les Lacédémoniens avaient tiré ces usages des Crétois ; et Platon\footnote{{\itshape Des Lois}, liv. I.}, qui veut prouver que les institutions des Crétois étaient faites pour la guerre, cite celle-ci : « La faculté de supporter la douleur dans les combats particuliers, et dans les larcins qui obligent de se cacher. »\par
Comme les lois civiles dépendent des lois politiques, parce que c’est toujours pour une société qu’elles sont faites, il serait bon que, quand on veut porter une loi civile d’une nation chez une autre, on examinât auparavant si elles ont toutes les deux les mêmes institutions et le même droit politique.\par
Ainsi, lorsque les lois sur le vol passèrent des Crétois aux Lacédémoniens, comme elles y passèrent avec le gouvernement et la constitution même, ces lois furent aussi sensées chez un de ces peuples qu’elles l’étaient chez l’autre. Mais, lorsque de Lacédémone elles furent portées à Rome, comme elles n’y trouvèrent pas la même constitution, elles y furent toujours étrangères, et n’eurent aucune liaison avec les autres lois civiles des Romains.
\subsubsection[{Chapitre XIV. Qu’il ne faut point séparer les lois des circonstances dans lesquelles elles ont été faites}]{Chapitre XIV. Qu’il ne faut point séparer les lois des circonstances dans lesquelles elles ont été faites}
\noindent Une loi d’Athènes voulait que, lorsque la ville était assiégée, on fît mourir tous les gens inutiles\footnote{{\itshape Inutilis aetas occidatur}, Syrian, {\itshape in Hermogenem}.}. C’était une abominable loi politique, qui était une suite d’un abominable droit des gens. Chez les Grecs, les habitants d’une ville prise perdaient la liberté civile, et étaient vendus comme esclaves : la prise d’une ville emportait son entière destruction ; et c’est l’origine non seulement de ces défenses opiniâtres et de ces actions dénaturées, mais encore de ces lois atroces que l’on fit quelquefois.\par
Les lois romaines\footnote{La loi Cornelia, {\itshape De sicariis, Institutes} liv. IV, tit. III ; {\itshape De lege Aquilia}, § 7.} voulaient que les médecins pussent être punis pour leur négligence ou pour leur impéritie. Dans ce cas, elles condamnaient à la déportation le médecin d’une condition un peu relevée, et à la mort celui qui était d’une condition plus basse. Par nos lois, il en est autrement. Les lois de Rome n’avaient pas été faites dans les mêmes circonstances que les nôtres : à Rome, s’ingérait de la médecine qui voulait ; mais, parmi nous, les médecins sont obligés de faire des études et de prendre certains grades ; ils sont donc censés connaître leur art.
\subsubsection[{Chapitre XV. Qu’il est bon quelquefois qu’une loi se corrige elle-même}]{Chapitre XV. Qu’il est bon quelquefois qu’une loi se corrige elle-même}
\noindent La loi des Douze Tables permettait de tuer le voleur de nuit\footnote{Voyez la loi 4, ff. {\itshape Ad leg. Aquil.}}, aussi bien que le voleur de jour qui, étant poursuivi, se mettait en défense ; mais elle voulait que celui qui tuait le voleur criât et appelât les citoyens\footnote{{\itshape Ibid.} Voyez le décret de Tassillon, ajouté à la loi des Bavarois, {\itshape De popularibus legibus}, art. 4.} ; et c’est une chose que les lois qui permettent de se faire justice soi-même doivent toujours exiger. C’est le cri de l’innocence, qui, dans le moment de l’action, appelle des témoins, appelle des juges. Il faut que le peuple prenne connaissance de l’action, et qu’il en prenne connaissance dans le moment qu’elle a été faite ; dans un temps où tout parle, l’air, le visage, les passions, le silence, et où chaque parole condamne ou justifie. Une loi qui peut devenir si contraire à la sûreté et à la liberté des citoyens doit être exécutée dans la présence des citoyens.
\subsubsection[{Chapitre XVI. Choses à observer dans la composition des lois}]{Chapitre XVI. Choses à observer dans la composition des lois}
\noindent Ceux qui ont un génie assez étendu pour pouvoir donner des lois à leur nation ou à une autre doivent faire de certaines attentions sur la manière de les former.\par
Le style en doit être concis. Les lois des Douze Tables sont un modèle de précision : les enfants les apprenaient par cœur\footnote{{\itshape Ut carmen necessarium}. Cicéron, {\itshape De legibus}, liv. II.}. Les {\itshape Novelles} de Justinien sont si diffuses, qu’il fallut les abréger\footnote{C’est l’ouvrage d’Irnerius.}.\par
Le style des lois doit être simple ; l’expression directe s’entend toujours mieux que l’expression réfléchie. Il n’y a point de majesté dans les lois du bas-empire ; on y fait parler les princes comme des rhéteurs. Quand le style des lois est enflé, on ne les regarde que comme un ouvrage d’ostentation.\par
Il est essentiel que les paroles des lois réveillent chez tous les hommes les mêmes idées. Le cardinal de Richelieu convenait que l’on pouvait accuser un ministre devant le roi\footnote{{\itshape Testament politique}.} ; mais il voulait que l’on fût puni si les choses qu’on prouvait n’étaient pas considérables : ce qui devait empêcher tout le monde de dire quelque vérité que ce fût contre lui, puisqu’une chose considérable est entièrement relative, et que ce qui est considérable pour quelqu’un ne l’est pas pour un autre.\par
La loi d’Honorius punissait de mort celui qui achetait comme serf un affranchi, ou qui aurait voulu l’inquiéter\footnote{{\itshape Aut qualibet manumissione donatum inquietare voluerit}. Appendice au code Théodosien, dans le premier tome des œuvres du P. Sirmond, p. 737.}. Il ne fallait point se servir d’une expression si vague : l’inquiétude que l’on cause à un homme dépend entièrement du degré de sa sensibilité.\par
Lorsque la loi doit faire quelque fixation, il faut, autant qu’on le peut, éviter de la faire à prix \&argent. Mille causes changent la valeur de la monnaie ; et avec la même dénomination on n’a plus la même chose. On sait l’histoire de cet impertinent de Rome\footnote{Aulu-Gelle, liv. XX, chap. I.}, qui donnait des soufflets à tous ceux qu’il rencontrait, et leur faisait présenter les vingt-cinq sous de la loi des Douze Tables.\par
Lorsque, dans une loi, l’on a bien fixé les idées des choses, il ne faut point revenir à des expressions vagues. Dans l’ordonnance criminelle de Louis XIV\footnote{On trouve dans le procès-verbal de cette ordonnance les motifs que l’on eut pour cela.}, après qu’on a fait l’énumération exacte des cas royaux, on ajoute ces mots : « Et ceux dont de tout temps les juges royaux ont jugé » ; ce qui fait rentrer dans l’arbitraire dont on venait de sortir.\par
Charles VII\footnote{Dans son ordonnance de Montel-lès-Tours, l’an 1453.} dit qu’il apprend que des parties font appel, trois, quatre et six mois après le jugement, contre la coutume du royaume en pays coutumier : il ordonne qu’on appellera incontinent, à moins qu’il n’y ait fraude ou dol du procureur\footnote{On pouvait punir le procureur, sans qu’il fût nécessaire de troubler l’ordre publie.}, ou qu’il n’y ait grande et évidente cause de relever l’appelant. La fin de cette loi détruit le commencement ; et elle le détruisit si bien que, dans la suite, on a appelé pendant trente ans\footnote{L’ordonnance de 1667 a fait des règlements là-dessus.}.\par
La loi des Lombards ne veut pas qu’une femme qui a pris un habit de religieuse, quoiqu’elle ne soit pas consacrée, puisse se marier\footnote{Liv. II, tit. XXXVII.} : « car, dit-elle, si un époux, qui a engagé à lui une femme seulement par un anneau, ne peut pas sans crime en épouser une autre, à plus forte raison l’épouse de Dieu ou de la sainte Vierge... » Je dis que dans les lois il faut raisonner de la réalité à la réalité, et non pas de la réalité à la figure, ou de la figure à la réalité.\par
Une loi de Constantin veut que le témoignage seul de l’évêque suffise, sans ouïr d’autres témoins\footnote{Dans l’appendice du P. Sirmond au code Théodosien, t. I.}. Ce prince prenait un chemin bien court ; il jugeait des affaires par les personnes, et des personnes par les dignités.\par
Les lois ne doivent point être subtiles ; elles sont faites pour des gens de médiocre entendement : elles ne sont point un art de logique, mais la raison simple d’un père de famille.\par
Lorsque, dans une loi, les exceptions, limitations, modifications, ne sont point nécessaires, il vaut beaucoup mieux n’en point mettre. De pareils détails jettent dans de nouveaux détails.\par
Il ne faut point faire de changement dans une loi sans une raison suffisante. Justinien ordonna qu’un mari pourrait être répudié, sans que la femme perdît sa dot, si pendant deux ans il n’avait pu consommer le mariage\footnote{Leg. I, code {\itshape De repudiis.}}. Il changea sa loi, et donna trois ans au pauvre malheureux\footnote{Voyez l’authentique {\itshape sed hodie}, au code {\itshape De repudiis.}}. Mais, dans un cas pareil, deux ans en valent trois, et trois n’en valent pas plus que deux.\par
Lorsqu’on fait tant que de rendre raison d’une loi, il faut que cette raison soit digne d’elle. Une loi romaine décide qu’un aveugle ne peut pas plaider, parce qu’il ne voit pas les ornements de la magistrature\footnote{Leg. I, ff. {\itshape De postulando}.}. Il faut l’avoir fait exprès, pour donner une si mauvaise raison, quand il s’en présentait tant de bonnes.\par
Le jurisconsulte Paul dit que l’enfant naît parfait au septième mois, et que la raison des nombres de Pythagore semble le prouver\footnote{Dans ses {\itshape Sentences}, liv. IV, tit. IX.}. il est singulier qu’on juge ces choses sur la raison des nombres de Pythagore.\par
Quelques jurisconsultes français ont dit que, lorsque le roi acquérait quelque pays, les églises y devenaient sujettes au droit de régale parce que la couronne du roi est ronde. Je ne discuterai point ici les droits du roi, et si, dans ce cas, la raison de la loi civile ou ecclésiastique doit céder à la raison de la loi politique ; mais je dirai que des droits si respectables doivent être défendus par des maximes graves. Qui a jamais vu fonder sur la figure d’un signe d’une dignité, les droits réels de cette dignité ?\par
Davila\footnote{{\itshape Della guerra civile di Francia}, p. 96.} dit que Charles IX fut déclaré majeur au parlement de Rouen à quatorze ans commencés, parce que les lois veulent qu’on compte le temps du moment au moment, lorsqu’il s’agit de la restitution et de l’administration des biens du pupille : au lieu qu’elle regarde l’année commencée comme une année complète, lorsqu’il s’agit d’acquérir des honneurs. Je n’ai garde de censurer une disposition qui ne paraît pas avoir eu jusqu’ici d’inconvénient ; je dirai seulement que la raison alléguée par le chancelier de l’Hôpital n’était pas la vraie : il s’en faut bien que le gouvernement des peuples ne soit qu’un honneur.\par
En fait de présomption, celle de la loi vaut mieux que celle de l’homme. La loi française\footnote{Elle est du mois de novembre 1702.} regarde comme frauduleux tous les actes faits par un marchand dans les dix jours qui ont précédé sa banqueroute : c’est la présomption de la loi. La loi romaine infligeait des peines au mari qui gardait sa femme après l’adultère, à moins qu’il n’y fût déterminé par la crainte de l’événement d’un procès, ou par la négligence de sa propre honte ; et c’est la présomption de l’homme. Il fallait que le juge présumât les motifs de la conduite du mari, et qu’il se déterminât sur une manière de penser très obscure. Lorsque le juge présume, les jugements deviennent arbitraires ; lorsque la loi présume, elle donne au juge une règle fixe.\par
La loi de Platon\footnote{Liv. IX des {\itshape Lois}.}, comme j’ai dit, voulait qu’on punît celui qui se tuerait, non pas pour éviter l’ignominie, mais par faiblesse. Cette loi était vicieuse, en ce que dans le seul cas où l’on ne pouvait pas tirer du criminel l’aveu du motif qui l’avait fait agir, elle voulait que le juge se déterminât sur ces motifs.\par
Comme les lois inutiles affaiblissent les lois nécessaires, celles qu’on peut éluder affaiblissent la législation. Une loi doit avoir son effet, et il ne faut pas permettre d’y déroger par une convention particulière.\par
La loi Falcidie ordonnait, chez les Romains, que l’héritier eût toujours la quatrième partie de l’hérédité : une autre loi\footnote{C’est l’authentique {\itshape Sed cum testator.}} permit au testateur de défendre à l’héritier de retenir cette quatrième partie : c’est se jouer des lois. La loi Falcidie devenait inutile : car, si le testateur voulait favoriser son héritier, celui-ci n’avait pas besoin de la loi Falcidie ; et s’il ne voulait pas le favoriser, il lui défendait de se servir de la loi Falcidie.\par
Il faut prendre garde que les lois soient conçues de manière qu’elles ne choquent point la nature des choses. Dans la proscription du prince d’Orange, Philippe II promet à celui qui le tuera de donner à lui, ou à ses héritiers, vingt-cinq mille écus et la noblesse ; et cela en parole de roi, et comme serviteur de Dieu. La noblesse promise pour une telle action ! une telle action ordonnée en qualité de serviteur de Dieu ! Tout cela renverse également les idées de l’honneur, celles de la morale, et celles de la religion.\par
il est rare qu’il faille défendre une chose qui n’est pas mauvaise, sous prétexte de quelque perfection qu’on imagine.\par
Il faut dans les lois une certaine candeur. Faites pour punir la méchanceté des hommes, elles doivent avoir elles-mêmes la plus grande innocence. On peut voir dans la loi des Wisigoths\footnote{Liv. XII, tit. II, § 16.} cette requête ridicule, par laquelle on fit obliger les juifs à manger toutes les choses apprêtées avec du cochon, pourvu qu’ils ne mangeassent pas du cochon même. C’était une grande cruauté : on les soumettait à une loi contraire à la leur ; on ne leur laissait garder de la leur que ce qui pouvait être un signe pour les reconnaître.
\subsubsection[{Chapitre XVII. Mauvaise manière de donner des lois}]{Chapitre XVII. Mauvaise manière de donner des lois}
\noindent Les empereurs romains manifestaient, comme nos princes, leurs volontés par des décrets et des édits ; mais ce que nos princes ne font pas, ils permirent que les juges ou les particuliers, dans leurs différends, les interrogeassent par lettres ; et leurs\par
réponses étaient appelées des rescrits. Les décrétales des papes sont, à proprement parler, des rescrits. On sent que c’est une mauvaise sorte de législation. Ceux qui demandent ainsi des lois sont de mauvais guides pour le législateur ; les faits sont toujours mal exposés. Trajan, dit Jules Capitolin\footnote{Voyez Jules Capitolin, {\itshape in Macrino.}}, refusa souvent de donner de ces sortes de rescrits, afin qu’on n’étendît pas à tous les cas une décision, et souvent une faveur particulière. Macrin avait résolu d’abolir tous ces rescrits\footnote{{\itshape Ibid.}} ; il ne pouvait souffrir qu’on regardât comme des lois les réponses de Commode, de Caracalla, et de tous ces autres princes pleins d’impéritie. Justinien pensa autrement, et il en remplit sa compilation.\par
Je voudrais que ceux qui lisent les lois romaines distinguassent bien ces sortes d’hypothèses d’avec les sénatus-consultes, les plébiscites, les constitutions générales des empereurs, et toutes les lois fondées sur la nature des choses, sur la fragilité des femmes, la faiblesse des mineurs et l’utilité publique.
\subsubsection[{Chapitre XVIII. Des idées d’uniformité}]{Chapitre XVIII. Des idées d’uniformité}
\noindent Il y a de certaines idées d’uniformité qui saisissent quelquefois les grands esprits (car elles ont touché Charlemagne), mais qui frappent infailliblement les petits. Ils y trouvent un genre de perfection qu’ils reconnaissent, parce qu’il est impossible de ne le pas découvrir : les mêmes poids dans la police, les mêmes mesures dans le commerce, les mêmes lois dans l’État, la même religion dans toutes ses parties. Mais cela est-il toujours à propos, sans exception ? Le mal de changer est-il toujours moins grand que le mal de souffrir ? Et la grandeur du génie ne consisterait-elle pas mieux à savoir dans cas il faut l’uniformité, et dans quel cas il faut des différences ? À la Chine, les Chinois sont gouvernés par le cérémonial chinois, et les Tartares par le cérémonial tartare : c’est pourtant le peuple du monde qui a le Plus la tranquillité pour objet. Lorsque les citoyens suivent les lois, qu’importe qu’ils suivent la même ?
\subsubsection[{Chapitre XIX. Des législateurs}]{Chapitre XIX. Des législateurs}
\noindent Aristote voulait satisfaire, tantôt sa jalousie contre Platon, tantôt sa passion pour Alexandre. Platon était indigné contre la tyrannie du peuple d’Athènes. Machiavel était plein de son idole, le due de Valentinois. Thomas More, qui parlait plutôt de ce qu’il avait lu que de ce qu’il avait pensé, voulait gouverner tous les États avec la simplicité d’une ville grecque\footnote{Dans son {\itshape Utopie}.}. Harrington ne voyait que la république d’Angleterre, pendant qu’une foule d’écrivains trouvaient le désordre partout où ils ne voyaient point de couronne, Les lois rencontrent toujours les passions et les préjugés du législateur. Quelquefois elles passent au travers, et s’y teignent ; quelquefois elles y restent, et s’y incorporent.
\subsection[{Livre trentième. Théorie des lois féodales, chez les Francs dans le rapport, qu’elles ont avec l’établissement, de la monarchie}]{Livre trentième. Théorie des lois féodales \\
chez les Francs dans le rapport \\
qu’elles ont avec l’établissement \\
de la monarchie}
\subsubsection[{Chapitre I. Des lois féodales}]{Chapitre I. Des lois féodales}
\noindent Je croirais qu’il y aurait une imperfection dans mon ouvrage, si je passais sous silence un événement arrivé une fois dans le monde, et qui n’arrivera peut-être jamais ; si je ne parlais de ces lois que l’on vit paraître en un moment dans toute l’Europe, sans qu’elles tinssent à celles que l’on avait jusques alors connues ; de ces lois qui ont fait des biens et des maux infinis ; qui ont laissé des droits quand on a cédé le domaine ; qui, en donnant à plusieurs personnes divers genres de seigneurie sur la même chose ou sur les mêmes personnes, ont diminué le poids de la seigneurie entière ; qui ont posé diverses limites dans des empires trop étendus ; qui ont produit la règle avec une inclinaison à l’anarchie, et l’anarchie avec une tendance à l’ordre et à l’harmonie.\par
Ceci demanderait un ouvrage exprès ; mais, vu la nature de celui-ci, on y trouvera plutôt ces lois comme je les ai envisagées, que comme je les ai traitées.\par
C’est un beau spectacle que celui des lois féodales. Un chêne antique s’élève\footnote{{\itshape … Quantum vertice ad auras / Aethereas, tantum radice ad Tartara tendit. /} VIRGILE.} ; l’œil en voit de loin les feuillages ; il approche, il en voit la tige ; mais il n’en aperçoit point les racines : il faut percer la terre pour les trouver.
\subsubsection[{Chapitre II. Des sources des lois féodales}]{Chapitre II. Des sources des lois féodales}
\noindent Les peuples qui conquirent l’empire romain étaient sortis de la Germanie. Quoique peu d’auteurs anciens nous aient décrit leurs mœurs, nous en avons deux qui sont d’un très grand poids. César, faisant la guerre aux Germains, décrit les mœurs des Germains\footnote{Liv. VI.} ; et c’est sur ces mœurs qu’il a réglé quelques-unes de ses entreprises\footnote{Par exemple, sa retraite d’Allemagne, {\itshape ibid.}}. Quelques pages de César sur cette matière sont des volumes.\par
Tacite fait un ouvrage exprès sur les mœurs des Germains. Il est court, cet ouvrage ; mais c’est l’ouvrage de Tacite, qui abrégeait tout, parce qu’il voyait tout.\par
Ces deux auteurs se trouvent dans un tel concert avec les codes des lois des peuples barbares que nous avons, qu’en lisant César et Tacite on trouve partout ces codes, et qu’en lisant ces codes on trouve partout César et Tacite.\par
Que si, dans la recherche des lois féodales, je me vois dans un labyrinthe obscur, plein de routes et de détours, je crois que je tiens le bout du fil, et que je puis marcher.
\subsubsection[{Chapitre III. Origine du vasselage}]{Chapitre III. Origine du vasselage}
\noindent César\footnote{Liv. VI de la {\itshape Guerre des Gaules}. Tacite ajoute : {\itshape Nulli domus, aut ager, aut aliqua cura ; prout ad quem venere aluntur. De moribus Germanorum}.} dit « que les Germains ne s’attachaient point à l’agriculture ; que la plupart vivaient de lait, de fromage et de chair ; que personne n’avait de terres ni de limites qui lui fussent propres ; que les princes et les magistrats de chaque nation donnaient aux particuliers la portion de terre qu’ils voulaient, et dans le lieu qu’ils voulaient, et les obligeaient l’année suivante de passer ailleurs ». Tacite dit\footnote{{\itshape De moribus Germanorum}.} « que chaque prince avait une troupe de gens qui s’attachaient à lui et le suivaient ». Cet auteur, qui, dans sa langue, leur donne un nom qui a du rapport avec leur état, les nomme compagnons\footnote{{\itshape Comites}.}. Il y avait entre eux une émulation\footnote{{\itshape De moribus Germanorum}.} singulière pour obtenir quelque distinction auprès du prince, et une même émulation entre les princes sur le nombre et la bravoure de leurs compagnons. « C’est, ajoute Tacite, la dignité, c’est la puissance d’être toujours entouré d’une foule de jeunes gens que l’on a choisis ; c’est un ornement dans la paix, c’est un rempart dans la guerre. On se rend célèbre dans sa nation et chez les peuples voisins, si l’on surpasse les autres par le nombre et le courage de ses compagnons : on reçoit des présents ; les ambassades viennent de toutes parts. Souvent la réputation décide de la guerre. Dans le combat, il est honteux au prince d’être inférieur en courage ; il est honteux à la troupe de ne point égaler la vertu du prince ; c’est une infamie éternelle de lui avoir survécu. L’engagement le plus sacré, c’est de le défendre. Si une cité est en paix, les princes vont chez celles qui font la guerre ; c’est par là qu’ils conservent un grand nombre d’amis. Ceux-ci reçoivent d’eux le cheval du combat et le javelot terrible. Les repas peu délicats, mais grands, sont une espèce de solde pour eux. Le prince ne soutient ses libéralités que par les guerres et les rapines. Vous leur persuaderiez bien moins de labourer la terre et d’attendre l’année, que d’appeler l’ennemi et de recevoir des blessures ; ils n’acquerront pas par la sueur ce qu’ils peuvent obtenir par le sang. »\par
Ainsi, chez les Germains, il y avait des vassaux, et non pas des fiefs. Il n’y avait point de fiefs, parce que les princes n’avaient point de terres à donner ; ou plutôt les fiefs étaient des chevaux de bataille, des armes, des repas. Il y avait des vassaux, parce qu’il y avait des hommes fidèles qui étaient liés par leur parole, qui étaient engagés pour la guerre, et qui faisaient à peu près le même service que l’on fit depuis pour les fiefs.
\subsubsection[{Chapitre IV. Continuation du même sujet}]{Chapitre IV. Continuation du même sujet}
\noindent César\footnote{{\itshape De bello Gallico}, liv. VI.} dit que « quand un des princes déclarait à l’assemblée qu’il avait formé le projet de quelque expédition, et demandait qu’on le suivît, ceux qui approuvaient le chef et l’entreprise se levaient et offraient leur secours. Ils étaient loués par la multitude. Mais s’ils ne remplissaient par leur engagement, ils perdaient la confiance publique, et on les regardait comme des déserteurs et des traîtres ».\par
Ce que dit ici César, et ce que nous avons dit dans le chapitre précédent, après Tacite, est le germe de l’histoire de la première race.\par
Il ne faut pas être étonné que les rois aient toujours eu à chaque expédition de nouvelles armées à refaire, d’autres troupes à persuader, de nouvelles gens à engager ; qu’il ait fallu, pour acquérir beaucoup, qu’ils répandissent beaucoup ; qu’ils acquissent sans cesse, par le partage, des terres et des dépouilles, et qu’ils donnassent sans cesse ces terres et ces dépouilles ; que leur domaine grossît continuellement, et qu’il diminuât sans cesse ; qu’un père qui donnait à un de ses enfants un royaume, y joignit toujours un trésor\footnote{Voyez la {\itshape Vie de Dagobert}.} {\itshape ; que} le trésor du roi fût regardé comme nécessaire à la monarchie ; et qu’un roi ne pût, même pour la dot de sa fille, en faire part aux étrangers sans le consentement des autres rois\footnote{Voyez Grégoire de Tours, liv. VI, sur le mariage de la fille de Chilpéric. Childebert lui envoie des ambassadeurs pour lui dire qu’il n’ait point à donner des villes du royaume de son père à sa fille, ni de ses trésors, ni des serfs, ni des chevaux, ni des cavaliers, ni des attelages de bœufs, etc.}. La monarchie avait son allure par des ressorts qu’il fallait toujours remonter.
\subsubsection[{Chapitre V. De la conquête des Francs}]{Chapitre V. De la conquête des Francs}
\noindent Il n’est pas vrai que les Francs, entrant dans la Gaule, aient occupé toutes les terres du pays pour en faire des fiefs. Quelques gens ont pensé ainsi, parce qu’ils ont vu sur la fin de la seconde race presque toutes les terres devenues des fiefs, des arrière-fiefs ou des dépendances de l’un ou de l’autre ; mais cela a eu des causes particulières qu’on expliquera dans la suite.\par
La conséquence qu’on en voudrait tirer, que les Barbares firent un règlement général pour établir partout la servitude de la glèbe, n’est pas moins fausse que le principe. Si, dans un temps où les fiefs étaient amovibles, toutes les terres du royaume avaient été des fiefs, ou des dépendances des fiefs, et tous les hommes du royaume des vassaux ou des serfs qui dépendaient d’eux ; comme celui qui a les biens a toujours aussi la puissance, le roi, qui aurait disposé continuellement des fiefs, c’est-à-dire de l’unique propriété, aurait eu une puissance aussi arbitraire que celle du sultan l’est en Turquie : ce qui renverse toute l’histoire.
\subsubsection[{Chapitre VI. Des Goths, des Bourguignons et des Francs}]{Chapitre VI. Des Goths, des Bourguignons et des Francs}
\noindent Les Gaules furent envahies par les nations germaines. Les Wisigoths occupèrent la Narbonnaise, et presque tout le Midi ; les Bourguignons s’établirent dans la partie qui regarde l’Orient ; et les Francs conquirent à peu près le reste.\par
Il ne faut pas douter que ces barbares n’aient conservé dans leurs conquêtes les mœurs, les inclinations et les usages qu’ils avaient dans leur pays, parce qu’une nation ne change pas dans un instant de manière de penser et d’agir. Ces peuples, dans la Germanie, cultivaient peu les terres. Il paraît par Tacite et César, qu’ils s’appliquaient beaucoup à la vie pastorale : aussi les dispositions des codes des lois des Barbares roulent-elles presque toutes sur les troupeaux. Roricon, qui écrivait l’histoire chez les Francs, était pasteur.
\subsubsection[{Chapitre VII. Différentes manières de partager les terres}]{Chapitre VII. Différentes manières de partager les terres}
\noindent Les Goths et les Bourguignons ayant pénétré, sous divers prétextes, dans l’intérieur de l’empire, les Romains, pour arrêter leurs dévastations, furent obligés de pourvoir à leur subsistance. D’abord ils leur donnaient du blé\footnote{Voyez Zozime, liv. V, sur la distribution du blé demandée par Alaric.} ; dans la suite ils aimèrent mieux leur donner des terres. Les empereurs, ou, sous leur nom, les magistrats romains firent des conventions avec eux sur le partage du pays\footnote{{\itshape Burgundiones partem Galliae occupaverunt, terrasque cum Gallicis senatoribus diviserunt. Chronique} de Marius, sur l’an 456.}, comme on le voit dans les chroniques et dans les codes des Wisigoths\footnote{Liv. X, tit. I, §§ 8, 9 et 16.} et des Bourguignons\footnote{Chap. LIV, §§ 1 et 2 ; et ce partage subsistait du temps de Louis le Débonnaire, comme il paraît par son capitulaire de l’an 829, qui a été inséré dans la loi des Bourguignons, tit. LXXIX, § 1.}.\par
Les Francs ne suivirent pas le même plan. On ne trouve dans les lois saliques et ripuaires aucune trace d’un tel partage de terres. Ils avaient conquis, ils prirent ce qu’ils voulurent, et ne firent de règlements qu’entre eux.\par
Distinguons donc le procédé des Bourguignons et des Wisigoths dans la Gaule, celui de ces mêmes Wisigoths en Espagne, des soldats auxiliaires sous Augustule et Odoacer en Italie\footnote{Voyez Procope, {\itshape Guerre des Goths.}}, d’avec celui des Francs dans les Gaules, et des Vandales en Afrique\footnote{{\itshape Guerre des Vandales}.}. Les premiers firent des conventions avec les anciens habitants, et en conséquence un partage de terres avec eux ; les seconds ne firent rien de tout cela.
\subsubsection[{Chapitre VIII. Continuation du même sujet}]{Chapitre VIII. Continuation du même sujet}
\noindent Ce qui donne l’idée d’une grande usurpation des terres des Romains par les Barbares, c’est qu’on trouve, dans les lois des Wisigoths et des Bourguignons, que ces deux peuples eurent les deux tiers des terres : mais ces deux tiers ne furent pris que dans de certains quartiers qu’on leur assigna.\par
Gondebaud dit, dans la loi des Bourguignons, que son peuple, dans son établissement, reçut les deux tiers des terres\footnote{{\itshape Licet eo tempore quo populus noster mancipiorum tertiam et duas terrarum partes accepit}, etc. Loi des Bourguignons, tit. LIV, § 1.} ; et il est dit, dans le second supplément à cette loi, qu’on n’en donnerait plus que la moitié à ceux qui viendraient dans le pays\footnote{{\itshape Ut non amplius a Burgundionibus, qui infra venerunt, requiratur, quam ad praesens necessitas fuerit, medietas terrae}, art. II.}. Toutes les terres n’avaient donc pas d’abord été partagées entre les Romains et les Bourguignons.\par
On trouve dans les textes de ces deux règlements les mêmes expressions ; ils s’expliquent donc l’un et l’autre. Et, comme on ne peut pas entendre le second d’un partage universel des terres, on ne peut pas non plus donner cette signification au premier.\par
Les Francs agirent avec la même modération que les Bourguignons ; ils ne dépouillèrent pas les Romains dans toute l’étendue de leurs conquêtes. Qu’auraient-ils fait de tant de terres ? Ils prirent celles qui leur convinrent, et laissèrent le reste.
\subsubsection[{Chapitre IX. Juste application de la loi des Bourguignons et de celle des Wisigoths sur le partage des terres}]{Chapitre IX. Juste application de la loi des Bourguignons et de celle des Wisigoths sur le partage des terres}
\noindent Il faut considérer que ces partages ne furent point faits par un esprit tyrannique, mais dans l’idée de subvenir aux besoins mutuels des deux peuples qui devaient habiter le même pays.\par
La loi des Bourguignons veut que chaque Bourguignon soit reçu en qualité d’hôte chez un Romain. Cela est conforme aux mœurs des Germains qui, au rapport de Tacite\footnote{{\itshape De moribus Germanorum}.}, étaient le peuple de la terre qui aimait le plus à exercer l’hospitalité.\par
La loi veut que le Bourguignon ait les deux tiers des terres, et le tiers des serfs. Elle suivait le génie des deux peuples, et se conformait à la manière dont ils se procuraient la subsistance. Le Bourguignon, qui faisait paître des troupeaux, avait besoin de beaucoup de terres et de peu de serfs ; et le grand travail de la culture de la terre exigeait que le Romain eût moins de glèbe, et un plus grand nombre de serfs. Les bois étaient partagés par moitié, parce que les besoins à cet égard étaient les mêmes.\par
On voit dans le code des Bourguignons\footnote{Et dans celui des Wisigoths.} que chaque Barbare fut placé chez chaque Romain. Le partage ne fut donc pas général ; mais le nombre des Romains qui donnèrent le partage, fut égal à celui des Bourguignons qui le reçurent. Le Romain fut lésé le moins qu’il fut possible. Le Bourguignon, guerrier, chasseur et pasteur, ne dédaignait pas de prendre des friches ; le Romain gardait les terres les plus propres à la culture ; les troupeaux du Bourguignon engraissaient le champ du Romain.
\subsubsection[{Chapitre X. Des servitudes}]{Chapitre X. Des servitudes}
\noindent Il est dit dans la loi des Bourguignons\footnote{Tit. LIV.} que quand ces peuples s’établirent dans les Gaules, ils reçurent les deux tiers des terres et le tiers des serfs. La servitude de la glèbe était donc établie dans cette partie de la Gaule avant l’entrée des Bourguignons\footnote{Cela est confirmé par tout le titre du code {\itshape de agricolis et censitis et colonis.}}.\par
La loi des Bourguignons, statuant sur les deux nations, distingue formellement, dans l’une et dans l’autre, les nobles, les ingénus et les serfs\footnote{{\itshape Si dentem optimati Burgundioni vel Romano nobili excusserit}, tit. XXVI, § 1 ; et {\itshape Si mediocribus personis ingenuis, tam Burgundionibus quam Romanis, ibid.}, § 2.}. La servitude n’était donc point une chose particulière aux Romains, ni la liberté et la noblesse une chose particulière aux Barbares.\par
Cette même loi dit\footnote{Tit. LVII.} que, si un affranchi bourguignon n’avait point donné une certaine somme à son maître, ni reçu une portion tierce d’un Romain, il était toujours censé de la famille de son maître. Le Romain propriétaire était donc libre, puisqu’il n’était point dans la famille d’un autre ; il était libre, puisque sa portion tierce était un signe de liberté.\par
Il n’y a qu’à ouvrir les lois saliques et ripuaires pour voir que les Romains ne vivaient pas plus dans la servitude chez les Francs que chez les autres conquérants de la Gaule.\par
M. le comte de Boulainvilliers a manqué le point capital de son système ; il n’a point prouvé que les Francs aient fait un règlement général qui mît les Romains dans une espèce de servitude.\par
Comme son ouvrage est écrit sans aucun art, et qu’il y parle avec cette simplicité, cette franchise et cette ingénuité de l’ancienne noblesse dont il était sorti, tout le monde est capable de juger et des belles choses qu’il dit, et des erreurs dans lesquelles il tombe. Ainsi je ne l’examinerai point. Je dirai seulement qu’il avait plus d’esprit que de lumières, plus de lumières que de savoir ; mais ce savoir n’était point méprisable, parce que, de notre histoire et de nos lois, il savait très bien les grandes choses.\par
M. le comte de Boulainvilliers et M. l’abbé Dubos ont fait chacun un système, dont l’un semble être une conjuration contre le Tiers-État, et l’autre une conjuration contre la noblesse. Lorsque le Soleil donna à Phaéton son char à conduire, il lui dit : « Si vous montez trop haut, vous brûlerez la demeure céleste ; si vous descendez trop bas, vous réduirez en cendres la terre. N’allez point trop à droite, vous tomberiez dans la constellation du Serpent ; n’allez point trop à gauche, vous iriez dans celle de l’Autel : tenez-vous entre les deux\footnote{{\itshape Nec preme, nec summum molire per aethera currum. / Altius egressus, caelestia tecta cremabis ; / Inferius, terras : medio tutissimus ibis. / Neu te dexterior tortum declinet ad Anguem, / Neve sinisterior pressam rota ducat ad Aram, / Inter utrumque tene…} / OVIDE, {\itshape Métamorphoses}, liv. II.}. »
\subsubsection[{Chapitre XI. Continuation du même sujet}]{Chapitre XI. Continuation du même sujet}
\noindent Ce qui a donné l’idée d’un règlement général fait dans le temps de la conquête, c’est qu’on a vu en France un prodigieux nombre de servitudes vers le commencement de la troisième race ; et, comme on ne s’est pas aperçu de la progression continuelle qui se fit de ces servitudes, on a imaginé dans un temps obscur une loi générale qui ne fut jamais.\par
Dans le commencement de la première race, on voit un nombre infini d’hommes libres, soit parmi les Francs soit parmi les Romains ; mais le nombre des serfs augmenta tellement, qu’au commencement de la troisième tous les laboureurs et presque tous les habitants des villes se trouvèrent serfs\footnote{Pendant que la Gaule était sous la domination des Romains, ils formaient des corps particuliers : c’étaient ordinairement des affranchis ou descendants d’affranchis.} {\itshape ;} et, au lieu que dans le commencement de la première, il y avait dans les villes à peu près la même administration que chez les Romains, des corps de bourgeoisie, un sénat, des cours de judicature, on ne trouve guère, vers le commencement de la troisième, qu’un seigneur et des serfs.\par
Lorsque les Francs, les Bourguignons et les Goths faisaient leurs invasions, ils prenaient l’or, l’argent, les meubles, les vêtements, les hommes, les femmes, les garçons, dont l’armée pouvait se charger ; le tout se rapportait en commun, et l’armée le partageait\footnote{Voyez Grégoire de Tours, liv. II, chap. XXVII ; Aimoin, liv. I, chap. XII.}. Le corps entier de l’histoire prouve qu’après le premier établissement, c’est-à-dire après les premiers ravages, ils reçurent à composition les habitants, et leur laissèrent tous leurs droits politiques et civils. C’était le droit des gens de ces temps-là ; on enlevait tout dans la guerre, on accordait tout dans la paix. Si cela n’avait pas été ainsi, comment trouverions-nous dans les lois saliques et bourguignonnes tant de dispositions contradictoires à la servitude générale des hommes ?\par
Mais ce que la conquête ne fit pas, le même droit des gens\footnote{Voyez les {\itshape Vies des saints}.}, qui subsista après la conquête, le fit. La résistance, la révolte, la prise des villes, emportaient avec elles la servitude des habitants. Et comme, outre les guerres que les différentes nations conquérantes firent entre elles, il y eut cela de particulier chez les Francs, que les divers partages de la monarchie firent naître sans cesse des guerres civiles entre les frères ou neveux, dans lesquelles ce droit des gens fut toujours pratiqué ; les servitudes devinrent plus générales en France que dans les autres pays : et c’est, je crois, une des causes de la différence qui est entre nos lois françaises et celles d’Italie et d’Espagne, sur les droits des seigneurs.\par
La conquête ne fut que l’affaire d’un moment ; et le droit des gens que l’on y employa, produisit quelques servitudes. L’usage du même droit des gens, pendant plusieurs siècles, fit que les servitudes s’étendirent prodigieusement.\par
Theuderic\footnote{Grégoire de Tours, liv. III.}, croyant que les peuples d’Auvergne ne lui étaient pas fidèles, dit aux Francs de son partage : « Suivez-moi, je vous mènerai dans un pays où vous aurez de l’or, de l’argent, des captifs, des vêtements, des troupeaux en abondance ; et vous en transférerez tous les hommes dans votre pays. »\par
Après la paix qui se fit entre Gontran et Chilpéric\footnote{{\itshape Ibid.}, liv. VI, chap. XXXI.}, ceux qui assiégeaient Bourges ayant eu ordre de revenir, ils amenèrent tant de butin, qu’ils ne laissèrent presque dans le pays ni hommes ni troupeaux.\par
Théodoric, roi d’Italie, dont l’esprit et la politique étaient de se distinguer toujours des autres rois barbares, envoyant son armée dans la Gaule, écrit au général\footnote{Lettre XLIII, liv. III dans Cassiodore.} : « Je veux qu’on suive les lois romaines, et que vous rendiez les esclaves fugitifs à leurs maîtres : le défenseur de la liberté ne doit point favoriser l’abandon de la servitude. Que les autres rois se plaisent dans le pillage et la ruine des villes qu’ils ont prises : nous voulons vaincre de manière que nos sujets se plaignent d’avoir acquis trop tard la sujétion. » Il est clair qu’il voulait rendre odieux les rois des Francs et des Bourguignons, et qu’il faisait allusion à leur droit des gens.\par
Ce droit subsista dans la seconde race. L’armée de Pépin étant entrée en Aquitaine, revint en France chargée d’un nombre infini de dépouilles et de serfs, disent les {\itshape Annales} de Metz\footnote{Sur l’an 763. {\itshape Innumerabilibus spoliis et captivis totus ille exercitus ditatus, in Franciam reversus est.}}.\par
Je pourrais citer des autorités sans nombre\footnote{{\itshape Annales de Fulde}, année 739 ; Paul Diacre, {\itshape De gestis Langobardorum}, liv. III, chap. XXX ; et liv. IV, chap. I ; et les {\itshape Vies des saints} citées note suivante.}. Et comme, dans ces malheurs, les entrailles de la charité s’émurent ; comme plusieurs saints évêques, voyant les captifs attachés deux à deux, employèrent l’argent des églises, et vendirent même les vases sacrés pour en racheter ce qu’ils purent ; que de saints moines s’y employèrent ; c’est dans les {\itshape Vies des saints} que l’on trouve les plus grands éclaircissements sur cette matière\footnote{Voyez les vies de saint Épiphane, de saint Eptadius, de saint Césaire, de saint Fidole, de saint Porcien, de saint Trévérius, de saint Eusichius, et de saint Léger ; les miracles de saint Julien.}. Quoiqu’on puisse reprocher aux auteurs de ces vies d’avoir été quelquefois un peu trop crédules sur des choses que Dieu a certainement faites si elles ont été dans l’ordre de ses desseins, on ne laisse pas d’en tirer de grandes lumières sur les mœurs et les usages de ces temps-là.\par
Quand on jette les yeux sur les monuments de notre histoire et de nos lois, il semble que tout est mer, et que les rivages même manquent à la mer\footnote{… {\itshape Deerant quoque littora ponto.} Ovide., liv. I.}. Tous ces écrits froids, secs, insipides et durs, il faut les lire, il faut les dévorer, comme la fable dit que Saturne dévorait les pierres.\par
Une infinité de terres que des hommes libres faisaient valoir, se changèrent en mainmortables\footnote{Les colons même n’étaient pas tous serfs : voyez les lois 18 et 23, au Code {\itshape de agricolis et censitis et colonis}, et la 20\textsuperscript{e} du même titre.}. Quand un pays se trouva privé des hommes libres qui l’habitaient, ceux qui avaient beaucoup de serfs prirent ou se firent céder de grands territoires, et y bâtirent des villages, comme on le voit dans diverses chartes. D’un autre côté, les hommes libres qui cultivaient les arts, se trouvèrent être des serfs qui devaient les exercer ; les servitudes rendaient aux arts et au labourage ce qu’on leur avait ôté.\par
Ce fut une chose usitée, que les propriétaires des terres les donnèrent aux églises pour les tenir eux-mêmes à cens croyant participer par leur servitude à la sainteté des églises.
\subsubsection[{Chapitre XII. Que les terres du partage des barbares ne payaient point de tributs}]{Chapitre XII. Que les terres du partage des barbares ne payaient point de tributs}
\noindent Des peuples simples, pauvres, libres, guerriers, pasteurs, qui vivaient sans industrie, et ne tenaient à leurs terres que par des cases de jonc\footnote{Voyez Grégoire de Tours, liv. II.}, suivaient des chefs pour faire du butin, et non pas pour payer ou lever des tributs. L’art de la maltôte est toujours inventé après coup, et lorsque les hommes commencent à jouir de la félicité des autres arts.\par
Le tribut passager d’une cruche de vin par arpent\footnote{{\itshape Ibid.}, liv. V.}, qui fut une des vexations de Chilpéric et de Frédégonde, ne concerna que les Romains. En effet, ce ne furent pas les Francs qui déchirèrent les rôles de ces taxes, mais les ecclésiastiques, qui, dans ces temps-là, étaient tous Romains\footnote{Cela paraît par toute {\itshape l’Histoire} de Grégoire de Tours. Le même Grégoire demande à un certain Valfiliacus comment il avait pu parvenir à la cléricature, lui qui était Lombard d’origine. Grégoire de Tours, liv. VIII.}. Ce tribut affligea principalement les habitants des villes\footnote{{\itshape Quae conditio universis urbibus per Galliam constitutis summopere est adhibita.Vie de saint Aridius}.} : or, les villes étaient presque toutes habitées par des Romains.\par
Grégoire de Tours\footnote{Liv. VII.} dit qu’un certain juge fut obligé, après la mort de Chilpéric, de se réfugier dans une église, pour avoir, sous le règne de ce prince, assujetti à des tributs des Francs, qui, du temps de Childebert, étaient ingénus {\itshape : Multos de Francis, qui, tempore Childebeiti regis, ingenui fuerant, publico tributo subegit}. Les Francs qui n’étaient point serfs ne payaient donc point de tributs.\par
Il n’y a point de grammairien qui ne pâlisse en voyant comment ce passage a été interprété par M. l’abbé Dubos\footnote{{\itshape Établissement de la monarchie française}, t. III, chap. XIV. p. 515.}. Il remarque que, dans ces temps-là, les affranchis étaient aussi appelés ingénus. Sur cela, il interprète le mot latin {\itshape ingenui} par ces mots : {\itshape affranchis de tributs} ; expression dont on peut se servir dans la langue française, comme on dit {\itshape affranchis de soins, affranchis de peines} ; mais dans la langue latine, {\itshape ingenui a tributis, libertini a tributis, manumissi tributorum}, seraient des expressions monstrueuses.\par
Parthenius, dit Grégoire de Tours\footnote{Liv. III, chap. XXXVI.}, pensa être mis à mort par les Francs, pour leur avoir imposé des tributs. M. l’abbé Dubos, pressé par ce passage, suppose froidement ce qui est en question : c’était, dit-il, une surcharge\footnote{T. III, p. 514.}.\par
On voit, dans la loi des Wisigoths\footnote{{\itshape Judices atque praepositi tertias Romanorum, ab illis qui occupatas tenent, auferant, et Romanis sua exactione sine aliqua dilatione restituant, ut nihil fisco debeat deperire.} Liv. X, tit. I, chap. XIV.}, que, quand un barbare occupait le fonds d’un Romain, le juge l’obligeait de le vendre, pour que ce fonds continuât à être tributaire : les barbares ne payaient donc pas de tributs sur les terres\footnote{Les Vandales n’en payaient point en Afrique. Procope, {\itshape Guerre des Vandales}, liv. I et II ; {\itshape Historia miscella}, liv. XVI, p. 106. Remarquez que les conquérants de l’Afrique étaient un composé de Vandales, d’Alains et de Francs. {\itshape Historia miscella}, liv. XIV, p. 94.}.\par
M. l’abbé Dubos \footnote{{\itshape Établissement des Francs dans les Gaules}, t. III, chap. XIV, p. 510.}, qui avait besoin que les Wisigoths payassent des tributs\footnote{Il s’appuie sur une autre loi des Wisigoths, liv. X, tit. I, art. 2, qui ne prouve absolument rien : elle dit seulement que celui qui a reçu d’un seigneur une terre, sous condition d’une redevance, doit la payer.}, quitte le sens littéral et spirituel de la loi ; et imagine, uniquement parce qu’il imagine, qu’il y avait eu, entre l’établissement des Goths et cette loi, une augmentation de tributs qui ne concernait que les Romains. Mais il n’est permis qu’au P. Hardouin d’exercer ainsi sur les faits un pouvoir arbitraire.\par
M. l’abbé Dubos\footnote{T. III, p. 511.} va chercher, dans le code de Justinien\footnote{Leg. 3, tit. LXIV, liv. XI.}, des lois pour prouver que les bénéfices militaires, chez les Romains, étaient sujets aux tributs : d’où il conclut qu’il en était de même des fiefs ou bénéfices chez les Francs. Mais l’opinion que nos fiefs tirent leur origine de cet établissement des Romains, est aujourd’hui proscrite : elle n’a eu de crédit que dans les temps où l’on connaissait l’histoire romaine et très peu la nôtre, et où nos monuments anciens étaient ensevelis dans la poussière.\par
M. l’abbé Dubos a tort de citer Cassiodore, et d’employer ce qui se passait en Italie et dans la partie de la Gaule soumise à Théodoric, pour nous apprendre ce qui était en usage chez les Francs ; ce sont des choses qu’il ne faut point confondre. Je ferai voir quelque jour, dans un ouvrage particulier, que le plan de la monarchie des Ostrogoths était entièrement différent du plan de toutes celles qui furent fondées dans ces temps-là par les autres peuples barbares : et que, bien loin qu’on puisse dire qu’une chose était en usage chez les Francs, parce qu’elle l’était chez les Ostrogoths, on a au contraire un juste sujet de penser qu’une chose qui se pratiquait chez les Ostrogoths, ne se pratiquait pas chez les Francs.\par
Ce qui coûte le plus à ceux dont l’esprit flotte dans une vaste érudition, c’est de chercher leurs preuves là où elles ne sont point étrangères au sujet, et de trouver, pour parler comme les astronomes, le lieu du soleil.\par
M. l’abbé Dubos abuse des capitulaires comme de l’histoire, et comme des lois des peuples barbares. Quand il veut que les Francs aient payé des tributs, il applique à des hommes libres ce qui ne peut être entendu que des serfs\footnote{{\itshape Établissement de la monarchie française}, t. III, chap. XIV, p. 513, où il cite l’art. 28 de l’édit de Pistes. Voyez ci-dessous le chap. XVIII.} ; quand il veut parler de leur milice, il applique à des serfs ce qui ne pouvait concerner que des hommes libres\footnote{{\itshape Ibid., t.} III, chap. IV, p. 298.}.
\subsubsection[{Chapitre XIII. Quelles étaient les charges des Romains et des Gaulois dans la monarchie des Francs}]{Chapitre XIII. Quelles étaient les charges des Romains et des Gaulois dans la monarchie des Francs}
\noindent Je pourrais examiner si les Romains et les Gaulois vaincus continuèrent de payer les charges auxquelles ils étaient assujettis sous les empereurs. Mais, pour aller plus vite, je me contenterai de dire que, s’ils les payèrent d’abord, ils en furent bientôt exemptés, et que ces tributs furent changés en un service militaire ; et j’avoue que je ne conçois guère comment les Francs auraient été d’abord si amis de la maltôte, et en auraient paru tout à coup si éloignés.\par
Un capitulaire\footnote{De l’an 815, chap. I. Ce qui est conforme au capitulaire de Charles le Chauve, de l’an 844, art. 1 et 2.} de Louis le Débonnaire nous explique très bien l’état où étaient les hommes libres dans la monarchie des Francs. Quelques bandes de Goths ou d’Ibères, fuyant l’oppression des Maures, furent reçus dans les terres de Louis\footnote{{\itshape Pro Hispanis in partibus Aquitaniae, Septimaniae et Provinciae consistentibus.Ibid.}}. La convention qui fut faite avec eux porte que, comme les autres hommes libres, ils iraient à l’armée avec leur comte ; que, dans la marche, ils feraient la garde et les patrouilles sous les ordres du même comte\footnote{{\itshape Excubias et explorationes quas wactas dicunt.Ibid.}}, et qu’ils donneraient aux envoyés du roi, et aux ambassadeurs qui partiraient de sa cour ou iraient vers lui, des chevaux et des chariots pour les voitures\footnote{Ils n’étaient pas obligés d’en donner au comte. {\itshape Ibid.}, art. 5.} ; que d’ailleurs ils ne pourraient être contraints à payer d’autre cens, et qu’ils seraient traités comme les autres hommes libres.\par
On ne peut pas dire que ce fussent de nouveaux usages introduits dans les commencements de la seconde race ; cela devait appartenir au moins au milieu ou à la fin de la première. Un capitulaire de l’an 864 dit expressément que c’était une coutume ancienne que les hommes libres fissent le service militaire, et payassent de plus les chevaux et les voitures dont nous avons parlé\footnote{{\itshape Ut pagenses Franci, qui caballos habent, cum suis comitibus in hostem pergant.} Il est défendu aux comtes de les priver de leurs chevaux ; {\itshape ut hostem facere, et debitos paraveredos secundum antiquam consuetudinem exsolvere possint}. Édit de Pistes, dans Baluze, p. 186.} {\itshape ;} charges qui leur étaient particulières, et dont ceux qui possédaient les fiefs étaient exempts, comme je le prouverai dans la suite.\par
Ce n’est pas tout : il y avait un règlement qui ne permettait guère de soumettre ces hommes libres à des tributs\footnote{Capitulaire de Charlemagne, de l’an 812, chap. I ; Édit de Pistes, de l’an 864, art. 27.}. Celui qui avait quatre manoirs\footnote{{\itshape Quatuor mansos.} Il me semble que ce qu’on appelait {\itshape mansus} était une certaine portion de terre attachée à une cense où il y avait des esclaves ; témoin le capitulaire de l’an 853, {\itshape apud Sylvacum}, tit. XIV, contre ceux qui chassaient les esclaves de leur {\itshape mansus.}} était toujours obligé de marcher à la guerre ; celui qui n’en avait que trois était joint à un homme libre qui n’en avait qu’un ; celui-ci le défrayait pour un quart, et restait chez lui. On joignait de même deux hommes libres qui avaient chacun deux manoirs ; celui des deux qui marchait était défrayé de la moitié par celui qui restait.\par
Il y a plus : nous avons une infinité de chartes où l’on donne les privilèges des fiefs à des terres ou districts possédés par des hommes libres, et dont je parlerai beaucoup dans la suite\footnote{Voyez ci-dessous le chap. XX de ce livre.}. On exempte ces terres de toutes les charges qu’exigeaient sur elles les comtes et autres officiers du roi ; et, comme on énumère en particulier toutes ces charges, et qu’il n’y est point question de tributs, il est visible qu’on n’en levait pas.\par
Il était aisé que la maltôte romaine tombât d’elle-même dans la monarchie des Francs : c’était un art très compliqué, et qui n’entrait ni dans les idées ni dans le plan de ces peuples simples. Si les Tartares inondaient aujourd’hui l’Europe, il faudrait bien des affaires pour leur faire entendre ce que c’est qu’un financier parmi nous.\par
L’auteur\footnote{Dans Duchesne, t. II, p. 287.} incertain de la {\itshape Vie de Louis le Débonnaire}, parlant des comtes et autres officiers de la nation des Francs que Charlemagne établit en Aquitaine, dit qu’il leur donna la garde de la frontière, le pouvoir militaire, et l’intendance des domaines qui appartenaient à la couronne. Cela fait voir l’état des revenus du prince dans la seconde race. Le prince avait gardé des domaines, qu’il faisait valoir par ses esclaves. Mais les indictions, la capitation et autres impôts levés du temps des empereurs sur la personne ou les biens des hommes libres avaient été changés en une obligation de garder la frontière, ou d’aller à la guerre.\par
On voit, dans la même histoire\footnote{Dans Duchesne, t. II, p. 89.}, que Louis le Débonnaire ayant été trouver son père en Allemagne, ce prince lui demanda comment il pouvait être si pauvre, lui qui était roi : que Louis lui répondit qu’il n’était roi que de nom, et que les seigneurs tenaient presque tous ses domaines : que Charlemagne, craignant que ce jeune prince ne perdit leur affection s’il reprenait de lui-même ce qu’il avait inconsidérément donné, il envoya des commissaires pour rétablir les choses.\par
Les évêques écrivant à Louis\footnote{Voyez le capitulaire de l’an 858, art. 14.}, frère de Charles le Chauve, lui disaient : « Ayez soin de vos terres, afin que vous ne soyez pas obligé de voyager sans cesse par les maisons des ecclésiastiques, et de fatiguer leurs serfs par des voitures. Faites en sorte, disaient-ils encore, que vous ayez de quoi vivre et recevoir des ambassades. » Il est visible que les revenus des rois consistaient alors dans leurs domaines\footnote{Ils levaient encore quelques droits sur les rivières, lorsqu’il y avait un pont ou un passage.}.
\subsubsection[{Chapitre XIV. De ce qu’on appelait census}]{Chapitre XIV. De ce qu’on appelait {\itshape census}}
\noindent Lorsque les Barbares sortirent de leur pays, ils voulurent rédiger par écrit leurs usages ; mais comme on trouva de la difficulté à écrire des mots germains avec des lettres romaines, on donna ces lois en latin.\par
Dans la confusion de la conquête et de ses progrès, la plupart des choses changèrent de nature ; il fallut, pour les exprimer, se servir des anciens mots latins qui avaient le plus de rapport aux nouveaux usages. Ainsi, ce qui pouvait réveiller l’idée de l’ancien cens des Romains\footnote{Le {\itshape census} était un mot si générique, qu’on s’en servit pour exprimer les péages des rivières, lorsqu’il y avait un pont ou un bac à passer. Voyez le capitulaire III de l’an 803, édition de Baluze, p. 395, art. 1, et le Ve de l’an 819, p. 616. On appela encore de ce nom les voitures fournies par les hommes libres au roi ou à ses envoyés, comme il paraît par le Capitulaire de Charles le Chauve de l’an 865, art. 8.}, on le nomma {\itshape census, tributum} ; et, quand les choses n’y eurent aucun rapport quelconque, on exprima, comme on put, les mots germains avec des lettres romaines : ainsi on forma le mot {\itshape fredum}, dont je parlerai beaucoup dans les chapitres suivants.\par
Les mots {\itshape census} et {\itshape tributum} ayant été ainsi employés d’une manière arbitraire, cela a jeté quelque obscurité dans la signification qu’avaient ces mots dans la première et dans la seconde race : et des auteurs modernes\footnote{M. l’abbé Dubos, et ceux qui l’ont suivi.}, qui avaient des systèmes particuliers, ayant trouvé ce mot dans les écrits de ces temps-là, ils ont jugé que ce qu’on appelait {\itshape census} était précisément le cens des Romains ; et ils en ont tiré cette conséquence, que nos rois des deux premières races s’étaient mis à la place des empereurs romains, et n’avaient rien changé à leur administration\footnote{Voyez la faiblesse des raisons de M. l’abbé Dubos, {\itshape Établissement de la monarchie française}, t. III, liv. VI, chap. XIV ; surtout l’induction qu’il tire d’un passage de Grégoire de Tours sur un démêlé de son église avec le roi Charibert.}. Et, comme de certains droits levés dans la seconde race ont été, par quelques hasards et par de certaines modifications, convertis en d’autres\footnote{Par exemple, par les affranchissements.}, ils en ont conclu que ces droits étaient le cens des Romains : et comme, depuis les règlements modernes, ils ont vu que le domaine de la couronne était absolument inaliénable, ils ont dit que ces droits, qui représentaient le cens des Romains, et qui ne forment pas une partie de ce domaine, étaient de pures usurpations. Je laisse les autres conséquences.\par
Transporter dans des siècles reculés toutes les idées du siècle où l’on vit, c’est des sources de l’erreur celle qui est la plus féconde. À ces gens qui veulent rendre modernes tous les siècles anciens, je dirai ce que les prêtres d’Égypte dirent à Solon : « Ô Athéniens ! vous n’êtes que des enfants »
\subsubsection[{Chapitre XV. Que ce qu’on appelait census ne se levait que sur les serfs, et non pas sur les hommes libres}]{Chapitre XV. Que ce qu’on appelait census ne se levait que sur les serfs, et non pas sur les hommes libres}
\noindent Le roi, les ecclésiastiques et les seigneurs levaient des tributs réglés, chacun sur les serfs de ses domaines. Je le prouve, à l’égard du roi, par le capitulaire {\itshape de Villis} à l’égard des ecclésiastiques, par les codes des lois des Barbares\footnote{Loi des Allemands, chap. XXII ; et la loi des Bavarois, tit. I, chap. XIV, où l’on trouve les règlements que les ecclésiastiques firent sur leur état.} ; à l’égard des seigneurs, par les règlements que Charlemagne fit là-dessus\footnote{Liv. V des {\itshape Capitulaires}, chap. CCCIII.}.\par
Ces tributs étaient appelés {\itshape census} : c’étaient des droits économiques, et non pas fiscaux ; des redevances uniquement privées, et non pas des charges publiques.\par
Je dis que ce qu’on appelait {\itshape census} était un tribut levé sur les serfs. Je le prouve par une formule de Marculfe, qui contient une permission du roi de se faire clerc, pourvu qu’on soit ingénu, et qu’on ne soit point inscrit dans le registre du cens\footnote{{\itshape Si ille de capite suo bene ingenuus sit, et in puletico publico censitus non est.} Liv. I, form. 19.}. Je le prouve encore par une commission que Charlemagne donna à un comte qu’il envoya dans les contrées de Saxe\footnote{De l’an 789, édition des {\itshape Capitulaires} de Baluze, t. I, p. 250.} ; elle contient l’affranchissement des Saxons, à cause qu’ils avaient embrassé le christianisme ; et c’est proprement une charte d’ingénuité\footnote{{\itshape Et ut ista ingenuitatis pagina firma stabilisque consistat. Ibid.}}. Ce prince les rétablit dans leur première liberté civile, et les exempte de payer le cens\footnote{{\itshape Pristinaeque libertati donatos, et omni nobis debito censu Solutos.Ibid.}}. C’était donc une même chose d’être serf et de payer le cens, d’être libre et de ne le payer pas.\par
Par une espèce de lettres patentes du même prince en faveur des Espagnols qui avaient été reçus dans la monarchie\footnote{{\itshape Praeceptum pro Hispanis}, de l’an 812, édition de Baluze, t. I, p. 500.}, il est défendu aux comtes d’exiger d’eux aucun cens, et de leur ôter leurs terres. On sait que les étrangers qui arrivaient en France étaient traités comme des serfs ; et Charlemagne, voulant qu’on les regardât comme des hommes libres, puisqu’il voulait qu’ils eussent la propriété de leurs terres, défendait d’exiger d’eux le cens.\par
Un capitulaire\footnote{De l’an 844, édition de Baluze, t. II, art. 1 et 2, p. 27.} de Charles le Chauve, donné en faveur des mêmes Espagnols, veut qu’on les traite comme on traitait les autres Francs, et défend d’exiger d’eux le cens : les hommes libres ne le payaient donc pas.\par
L’article 30 de l’édit de Pistes réforme l’abus par lequel plusieurs colons du roi ou de l’église vendaient les terres dépendantes de leurs manoirs à des ecclésiastiques ou à des gens de leur condition, et ne se réservaient qu’une petite case : de sorte qu’on ne pouvait plus être payé du cens ; et il y est ordonné de rétablir les choses dans leur premier état : le cens était donc un tribut d’esclaves.\par
Il résulte encore de là qu’il n’y avait point de cens général dans la monarchie ; et cela est clair par un grand nombre de textes. Car que signifierait ce capitulaire\footnote{Capitulaire III, de l’an 805, art 20 et 22, inséré dans le recueil d’Anzegise, liv. III, art. 15. Cela est conforme à celui de Charles le Chauve, de l’an 854, {\itshape apud Attiniacum}, art. 6.} : « Nous voulons qu’on exige le cens royal dans tous les lieux où autrefois on l’exigeait légitimement\footnote{{\itshape Undecumque legitime exigebatur. Ibid.}} » ? Que voudrait dire celui\footnote{De l’an 812, art. 10 et 11, édition de Baluze, t. I, p. 498.} où Charlemagne ordonne à ses envoyés dans les provinces de faire une recherche exacte de tous les cens qui avaient anciennement été du domaine du roi\footnote{{\itshape Undecumque antiquitus ad partem regis venire solebant}. Capitulaire de l’an 812, art. 10 et 11.} ? et celui\footnote{De l’an 813, art. 6, édition de Baluze, t. I, p. 508.} où il dispose des cens payés par ceux dont on les exige\footnote{{\itshape De illis unde censa exigunt}. Capitulaire de l’an 813, art. 6.} ? Quelle signification donner à cet autre\footnote{Liv. IV des {\itshape Capitulaires}, art. 37, et inséré dans la loi des Lombards.} où on lit : « Si quelqu’un a acquis une terre tributaire sur laquelle nous avions accoutumé de lever le cens\footnote{{\itshape Si quis terram tributatiam, unde census ad partem nostram exire solebat, susceperit} : liv. IV des {\itshape Capitulaires}, art. 37.} » ? à cet autre enfin\footnote{De l’an 805, art. 8.} où Charles le Chauve parle des terres censuelles dont le cens avait de toute antiquité appartenu au roi\footnote{{\itshape Unde census ad partem regis exivit antiquitus}. Capitulaire de l’an 805, art. 8.} ?\par
Remarquez qu’il y a quelques textes qui paraissent d’abord contraires à ce que j’ai dit, et qui cependant le confirment. On a vu ci-dessus que les hommes libres, dans la monarchie, n’étaient obligés qu’à fournir de certaines voitures. Le capitulaire que je viens de citer appelle cela {\itshape census}\footnote{{\itshape Censibus vel paraveredis quos Franci homines ad regiam potestatem exsolvere debent.}}, et il l’oppose au cens qui était payé par les serfs.\par
De plus, l’édit de Pistes\footnote{De l’an 864, art. 34, édition de Baluze, p. 192.} parle de ces hommes francs qui devaient payer le cens royal pour leur tête et pour leurs cases, et qui s’étaient vendus pendant la famine\footnote{{\itshape De illis Francis hominibus qui censum regium de suo capite et de suis recellis debeant. Ibid.}}. Le roi veut qu’ils soient rachetés. C’est que ceux qui étaient affranchis par lettres du roi\footnote{L’article 28 du même édit explique bien tout cela. Il met même une distinction entre l’affranchi romain et l’affranchi franc ; et on y voit que le cens n’était pas général. Il faut le lire.}, n’acquéraient point ordinairement une pleine et entière liberté\footnote{Comme il paraît par un capitulaire de Charlemagne, de l’an 813, déjà cité.} ; mais ils payaient {\itshape censum in capite}, et c’est de cette sorte de gens dont il est ici parlé.\par
Il faut donc se défaire de l’idée d’un cens général et universel, dérivé de la police des Romains, duquel on suppose que les droits des seigneurs ont dérivé de même par des usurpations. Ce qu’on appelait {\itshape cens} dans la monarchie française, indépendamment de l’abus qu’on a fait de ce mot, était un droit particulier levé sur les serfs par les maîtres.\par
Je supplie le lecteur de me pardonner l’ennui mortel que tant de citations doivent lui donner : je serais plus court, si je ne trouvais toujours devant moi le livre de {\itshape l’Établissement de la monarchie française dans les Gaules}, de M. l’abbé Dubos. Rien ne recule plus le progrès des connaissances qu’un mauvais ouvrage d’un auteur célèbre, parce qu’avant d’instruire il faut commencer par détromper.
\subsubsection[{Chapitre XVI. Des leudes ou vassaux}]{Chapitre XVI. Des leudes ou vassaux}
\noindent J’ai parlé de ces volontaires qui, chez les Germains, suivaient les princes dans leurs entreprises. Le même usage se conserva après la conquête. Tacite les désigne par le nom de compagnons\footnote{{\itshape Comites}.} ; la loi salique par celui d’hommes qui sont sous la foi du roi\footnote{{\itshape Qui sunt in truste regis}, tit. XLIV, art. 4.} {\itshape ;} les formules de Marculfe\footnote{Liv. I, formule 18.} par celui d’antrustions du roi\footnote{Du mot {\itshape trew}, qui signifie {\itshape fidèle} chez les Allemands, et chez les Anglais {\itshape true}, vrai.} {\itshape ;} nos premiers historiens par celui de leudes, de fidèles\footnote{{\itshape Leudes, fideles}.} {\itshape ;} et les suivants par celui de vassaux et seigneurs\footnote{{\itshape Vassali, seniores}.}.\par
On trouve dans les lois saliques et ripuaires un nombre infini de dispositions pour les Francs, et quelques-unes seulement pour les antrustions. Les dispositions sur ces antrustions sont différentes de celles faites pour les autres Francs ; on y règle partout les biens des Francs, et on ne dit rien de ceux des antrustions : ce qui vient de ce que les biens de ceux-ci se réglaient plutôt par la loi politique que par la loi civile, et qu’ils étaient le sort d’une armée, et non le patrimoine d’une famille.\par
Les biens réservés pour les leudes furent appelés des biens fiscaux\footnote{{\itshape Fiscalia.} Voyez la formule 14 de {\itshape Marculfe}, liv. I. Il est dit dans la {\itshape Vie de saint Maur, dedit fiscum unum ;} et dans les {\itshape Annales de Metz} sur l’an 747, {\itshape dedit illi comitatus et fiscos plurimos}. Les biens destinés à l’entretien de la famille royale étaient appelés {\itshape regalia.}}, des bénéfices, des honneurs, des fiefs, dans les divers auteurs et dans les divers temps.\par
On ne peut pas douter que d’abord les fiefs ne fussent amovibles\footnote{Voyez le liv. I, tit. I, {\itshape Des fiefs ;} et Cujas sur ce livre.}. On voit, dans Grégoire de Tours\footnote{Liv. IX, chap. XXXVIII.}, que l’on ôte à Sunégisile et à Galloman tout ce qu’ils tenaient du fisc, et qu’on ne leur laisse que ce qu’ils avaient en propriété. Gontran, élevant au trône son neveu Childebert, eut une conférence secrète avec lui, et lui indiqua ceux à qui il devait donner des fiefs, et ceux à qui il devait les ôter\footnote{{\itshape Quos honoraret muneribus, quos ab honore depelleret.Ibid.}, liv. VII.}. Dans une formule de Marculfe, le roi donne en échange, non seulement des bénéfices que son fisc tenait, mais encore ceux qu’un autre avait tenus \footnote{{\itshape Vel reliquis quibuscumque beneficiis, quodcumque ille, vel fiscus noster, in ipsis locis tenuisse noscitur.} Liv. I, formule 30.}. La loi des Lombards oppose les bénéfices à la propriété\footnote{Liv. III, tit. VIII, § 3.}. Les historiens, les formules, les codes des différents peuples barbares, tous les monuments qui nous restent, sont unanimes. Enfin, ceux qui ont écrit le {\itshape Livre des fiefs}\footnote{{\itshape Feudorum}, liv. I, tit. I.} nous apprennent que d’abord les seigneurs purent les ôter à leur volonté, qu’ensuite ils les assurèrent pour un an\footnote{C’était une espèce de précaire que le seigneur renouvelait ou ne renouvelait pas l’année d’ensuite, comme Cujas l’a remarqué.}, et après les donnèrent pour la vie.
\subsubsection[{Chapitre XVII. Du service militaire des hommes libres}]{Chapitre XVII. Du service militaire des hommes libres}
\noindent Deux sortes de gens étaient tenus au service militaire : les leudes vassaux ou arrière-vassaux, qui y étaient obligés en conséquence de leur fief ; et les hommes libres, Francs, Romains et Gaulois, qui servaient sous le comte, et étaient menés par lui et ses officiers.\par
On appelait hommes libres ceux qui, d’un côté, n’avaient point de bénéfices ou fiefs, et qui, de l’autre, n’étaient point soumis à la servitude de la glèbe ; les terres qu’ils possédaient étaient ce qu’on appelait des terres allodiales.\par
Les comtes assemblaient les hommes libres, et les menaient à la guerre\footnote{Voyez le capitulaire de Charlemagne, de l’an 812, art. 3 et 4, édition de Baluze, t. I, p. 491 ; et l’édit de Pistes, de l’an 864, art. 26, t. II, p. 186.} {\itshape ; ils} avaient sous eux des officiers qu’ils appelaient vicaires\footnote{{\itshape Et habebat unusquique comes vicarios et centenarios secum.} Liv. II des {\itshape Capitulaires}, art. 28.} ; et, comme tous les hommes libres étaient divisés en centaines, qui formaient ce que l’on appelait un bourg, les comtes avaient encore sous eux des officiers qu’on appelait centeniers, qui menaient les hommes libres du bourg\footnote{On les appelait {\itshape compagenses.}}, ou leurs centaines, à la guerre.\par
Cette division par centaines est postérieure à l’établissement des Francs dans les Gaules. Elle fut faite par Clotaire et Childebert, dans la vue d’obliger chaque district à répondre des vols qui s’y feraient : on voit cela dans les décrets de ces princes\footnote{Donnés vers l’an 595, art. I. Voyez les {\itshape Capitulaires}, édition de Baluze, p. 20. Ces règlements furent sans doute faits de concert.}. Une pareille police s’observe encore aujourd’hui en Angleterre.\par
Comme les comtes menaient les hommes libres à la guerre, les leudes y menaient aussi leurs vassaux ou arrière-vassaux ; et les évêques, abbés, ou leurs avoués\footnote{{\itshape Advocati}.}, y menaient les leurs\footnote{Capitulaire de Charlemagne, de l’an 812, art. 1 et 5, édition de Baluze, t. I, p. 490.}.\par
Les évêques étaient assez embarrassés : ils ne convenaient pas bien eux-mêmes de leurs faits\footnote{Voyez le capitulaire de l’an 803, donné à Worms, édition de Baluze, p. 408 et 410.}. Ils demandèrent à Charlemagne de ne plus les obliger d’aller à la guerre ; et, quand ils l’eurent obtenu, ils se plaignirent de ce qu’on leur faisait perdre la considération publique : et ce prince fut obligé de justifier là-dessus ses intentions. Quoi qu’il en soit, dans les temps où ils n’allèrent plus à la guerre, je ne vois pas que leurs vassaux y aient été menés par les comtes ; on voit au contraire que les rois ou les évêques choisissaient un des fidèles pour les y conduire\footnote{Capitulaire de Worms, de l’an 803, édition de Baluze, p. 409 ; et le concile de l’an 845, sous Charles le Chauve, {\itshape in Verno palatio}, édition de Baluze, t. II, p. 17, art. 8.}.\par
Dans un capitulaire de Louis le Débonnaire\footnote{{\itshape Capitulare quintum anni 819}, art. 27, édition de Baluze, p. 618.}, le roi distingue trois sortes de vassaux : ceux du roi, ceux des évêques, ceux du comte. Les vassaux d’un leude ou seigneur n’étaient menés à la guerre par le comte, que lorsque quelque emploi dans la maison du roi empêchait ces leudes de les mener eux-mêmes\footnote{{\itshape De vassis dominicis qui adhuc intra casam serviunt, et tamen beneficia habere noscuntur, statutum est ut quicumque ex eis cum domino imperatore domi remanserint, vassallos suos casatos secum non retineant ; sed cum comite, cujus pagenses sunt, ire permittant.} Capitulaire II, de l’an 812, art. 7, édition de Baluze, t. I, p. 494.}.\par
Mais qui est-ce qui menait les leudes à la guerre ? On ne peut douter que ce ne fût le roi, qui était toujours à la tête de ses fidèles. C’est pour cela que, dans les capitulaires, on voit toujours une opposition entre les vassaux du roi et ceux des évêques\footnote{Capitulaire I de l’an 812, art. 5. {\itshape De hominibus nostris, et episcoporum et abbatum qui vel beneficia, vel talia propria habent}, etc. Édition de Baluze, t. I, p. 490.}. Nos rois, courageux, fiers et magnanimes, n’étaient point dans l’armée pour se mettre à la tête de cette milice ecclésiastique ; ce n’était point ces gens-là qu’ils choisissaient pour vaincre ou mourir avec eux.\par
Mais ces leudes menaient de même leurs vassaux et arrière-vassaux ; et cela paraît bien par ce capitulaire\footnote{De l’an 812, chap. I, édition de Baluze, p. 490. {\itshape Ut omnis homo liber qui quatuor mansos vestitos de proprio suo, sive de alicujus beneficio, habet, ipse se praeparet, et ipse in hostem pergat, sive cum seniore suo.}} où Charlemagne ordonne que tout homme libre qui aura quatre manoirs, soit dans sa propriété, soit dans le bénéfice de quelqu’un, aille contre l’ennemi, ou suive son seigneur. Il est visible que Charlemagne veut dire que celui qui n’avait qu’une terre en propre entrait dans la milice du comte, et que celui qui tenait un bénéfice du seigneur partait avec lui.\par
Cependant M. l’abbé Dubos\footnote{T. III, liv. VI, chap. IV, p. 299, {\itshape Établissement de la monarchie française}.} prétend que, quand il est parlé dans les capitulaires des hommes qui dépendaient d’un seigneur particulier, il n’est question que des serfs ; et il se fonde sur la loi des Wisigoths et la pratique de ce peuple. Il vaudrait mieux se fonder sur les capitulaires mêmes. Celui que je viens de citer dit formellement le contraire. Le traité entre Charles le Chauve et ses frères parle de même des hommes libres, qui peuvent prendre à leur choix un seigneur ou le roi ; et cette disposition est conforme à beaucoup d’autres.\par
On peut donc dire qu’il y avait trois sortes de milices : celle des leudes ou fidèles du roi, qui avaient eux-mêmes sous leur dépendance d’autres fidèles ; celle des évêques ou autres ecclésiastiques, et de leurs vassaux ; et enfin celle du comte, qui menait les hommes libres.\par
Je ne dis point que les vassaux ne pussent être soumis au comte, comme ceux qui ont un commandement particulier dépendent de celui qui a un commandement plus général,\par
On voit même que le comte et les envoyés du roi pouvaient leur faire payer le ban, c’est-à-dire une amende, lorsqu’ils n’avaient pas rempli les engagements de leur fief.\par
De même, si les vassaux du roi faisaient des rapines\footnote{Capitulaire de l’an 882, art. II, {\itshape apud Vernis palatium}, édition de Baluze, t. II, p. 17.}, ils étaient soumis à la correction du comte, s’ils n’aimaient mieux se soumettre à celle du roi.
\subsubsection[{Chapitre XVIII. Du double service}]{Chapitre XVIII. Du double service}
\noindent C’était un principe fondamental de la monarchie, que ceux qui étaient sous la puissance militaire de quelqu’un, étaient aussi sous sa juridiction civile : aussi le capitulaire\footnote{Art. 1 et 2 ; et le concile {\itshape in Verno palatio}, de l’an 845, art. 8, édition de Baluze, t. II, p. 17.} de Louis le Débonnaire de l’an 815 fait-il marcher d’un pas égal la puissance militaire du comte et sa juridiction civile sur les hommes libres ; aussi les placites\footnote{Plaids ou assises.} du comte, qui menait à la guerre les hommes libres, étaient-ils appelés les placites des hommes libres\footnote{{\itshape Capitulaires}, liv. IV de la collection d’Anzegise, art. 57 ; et le capitulaire V de Louis le Débonnaire, de l’an 819, art. 14, édition de Baluze, t. I, p. 615.} ; d’où résulta sans doute cette maxime, que ce n’était que dans les placites du comte, et non dans ceux de ses officiers, qu’on pouvait juger les questions sur la liberté. Aussi le comte ne menait-il pas à la guerre les vassaux des évêques ou abbés\footnote{Voyez ci-dessus p. 1074 la note f ; et p. 1075 la note 1.}, parce qu’ils n’étaient pas sous sa juridiction civile ; aussi n’y menait-il pas les arrière-vassaux des leudes ; aussi le glossaire des lois anglaises\footnote{Que l’on trouve dans le recueil de Guillaume Lambard : {\itshape De priscis Anglorum legibus.}} nous dit-il\footnote{Au mot {\itshape satrapia.}} que ceux que les Saxons appelaient {\itshape coples}, furent nommés par les Normands {\itshape comtes, compagnons}, parce qu’ils partageaient avec le roi les amendes judiciaires : aussi voyons-nous, dans tous les temps, que l’obligation de tout vassal envers son seigneur\footnote{Les {\itshape Assises de Jérusalem}, chap. CCXXI et CCXXII, expliquent bien ceci.} fut de poiler les armes et de juger ses pairs dans sa cour\footnote{Les avoués de l’Église ({\itshape advocati}) étaient également à la tête de leurs plaids et de leur milice.}.\par
Une des raisons qui attachait ainsi ce droit de justice au droit de mener à la guerre était que celui qui menait à la guerre faisait en même temps payer les droits du fisc, qui consistaient en quelques services de voiture dus par les hommes libres, et en général en de certains profits judiciaires dont je parlerai ci-après.\par
Les seigneurs eurent le droit de rendre la justice dans leur fief, par le même principe qui fit que les comtes eurent le droit de la rendre dans leur comté ; et, pour bien dire, les comtés, dans les variations arrivées dans les divers temps, suivirent toujours les variations arrivées dans les fiefs : les uns et les autres étaient gouvernés sur le même plan et sur les mêmes idées. En un mot, les comtes dans leurs comtés étaient des leudes ; les leudes dans leurs seigneuries étaient des comtes.\par
On n’a pas eu des idées justes, lorsqu’on a regardé les comtes comme des officiers de justice, et les ducs comme des officiers militaires. Les uns et les autres étaient également des officiers militaires et civils\footnote{Voyez la {\itshape formule} 8 de Marculfe, liv. I, qui contient les lettres accordées à un due, patrice, {\itshape ou} comte, qui leur donnent la juridiction civile et l’administration fiscale.} : toute la différence était que le duc avait sous lui plusieurs comtes, quoiqu’il y eût des comtes qui n’avaient point de duc sur eux, comme nous l’apprenons de Frédégaire\footnote{{\itshape Chronique}, chap. LXXVIII, sur l’an 636.}.\par
On croira peut-être que le gouvernement des Francs était pour lors bien dur, puisque les mêmes officiers avaient en même temps sur les sujets la puissance militaire et la puissance civile, et même la puissance fiscale : chose que j’ai dit, dans les livres précédents, être une des marques distinctives du despotisme.\par
Mais il ne faut pas penser que les comtes jugeassent seuls, et rendissent la justice comme les bachas la rendent en Turquie\footnote{Voyez Grégoire de Tours, liv. V, {\itshape ad annum 580.}} : ils assemblaient, pour juger les affaires, des espèces de plaids ou d’assises\footnote{{\itshape Mallum}.}, où les notables étaient convoqués.\par
Pour qu’on puisse bien entendre ce qui concerne les jugements, dans les formules, les lois des Barbares et les capitulaires, je dirai que les fonctions de comte, du gravion et du centenier étaient les mêmes\footnote{Joignez ici ce que j’ai dit au livre XXVIII, chap. XXVIII ; et au livre XXXI, chap. VIII.} ; que les juges, les rathimburges et les échevins étaient, sous différents noms, les mêmes personnes. C’étaient les adjoints du comte et ordinairement il en avait sept : et, comme il ne lui fallait pas moins de douze personnes pour juger\footnote{Voyez sur tout ceci les capitulaires de Louis le Débonnaire, ajoutés à la loi salique, art. 2 ; et la formule des jugements, donnée par du Cange, au mot {\itshape boni homines}.}, il remplissait le nombre par des notables\footnote{{\itshape Per bonos homines.} Quelquefois il n’y avait que des notables. Voyez l’Appendice aux {\itshape Formules} de Marculfe, chap. LI.}.\par
Mais, qui que ce fût qui eût la juridiction, le roi, le comte, le gravion, le centenier, les seigneurs, les ecclésiastiques, ils ne jugèrent jamais seuls : et cet usage, qui tirait son origine des forêts de la Germanie, se maintint encore lorsque les fiefs prirent une forme nouvelle.\par
Quant au pouvoir fiscal, il était tel, que le comte ne pouvait guère en abuser. Les droits du prince à l’égard des hommes libres, étaient si simples, qu’ils ne consistaient, comme j’ai dit, qu’en de certaines voitures exigées dans de certaines occasions publiques\footnote{Et quelques droits sur les rivières dont j’ai parlé.} ; et, quant aux droits judiciaires, il y avait des lois qui prévenaient les malversations\footnote{Voyez la loi des Ripuaires, tit. LXXXIX ; et la loi des Lombards, liv. II, tit. LII, § 9.}.
\subsubsection[{Chapitre XIX. Des compositions chez les peuples barbares}]{Chapitre XIX. Des compositions chez les peuples barbares}
\noindent Comme il est impossible d’entrer un peu avant dans notre droit politique, si l’on ne connaît parfaitement les lois et les mœurs des peuples germains, je m’arrêtera ! un moment, pour faire la recherche de ces mœurs et de ces lois.\par
Il paraît par Tacite que les Germains ne connaissaient que deux crimes capitaux : ils pendaient les traîtres, et noyaient les poltrons : c’étaient chez eux les seuls crimes qui fussent publics. Lorsqu’un homme avait fait quelque tort à un autre, les parents de la personne offensée ou lésée entraient dans la querelle ; et la haine s’apaisait par une satisfaction\footnote{{\itshape Suscipere tant inimicitias, seu patris, seu propinqui, quant amicitias, necesse est : nec implacabiles durant ; luitur enim etam homicidium certo armentorum ac pecorum numero, recipitque satisfactionem universa domus.} Tacite, {\itshape De moribus Germanorum}.}. Cette satisfaction regardait celui qui avait été offensé, s’il pouvait la recevoir ; et les parents, si l’injure ou le tort leur était commun ; ou si, par la mort de celui qui avait été offensé ou lésé, la satisfaction leur était dévolue.\par
De la manière dont parle Tacite, ces satisfactions se faisaient par une convention réciproque entre les parties : aussi, dans les codes des peuples barbares, ces satisfactions s’appellent-elles des compositions.\par
Je ne trouve que la loi des Frisons\footnote{Voyez cette loi, tit. II, sur les meurtres ; et l’addition de Vulemar sur les vols.} qui ait laissé le peuple dans cette situation où chaque famille ennemie était, pour ainsi dire, dans l’état de nature ; et où, sans être retenue par quelque loi politique ou civile, elle pouvait à sa fantaisie exercer sa vengeance, jusqu’à ce qu’elle eût été satisfaite. Cette loi même fut tempérée : on établit\footnote{{\itshape Additio sapientium}, tit. I, § 1.} que celui dont on demandait la vie, aurait la paix dans sa maison, qu’il l’aurait en allant et en revenant de l’Église, et du lieu où l’on rendait les jugements.\par
Les compilateurs des lois saliques citent un ancien usage des Francs\footnote{Loi salique, tit. LVIII, § 1 ; tit. XVII, § 3.}, par lequel celui qui avait exhumé un cadavre pour le dépouiller était banni de la société des hommes, jusqu’à ce que les parents consentissent à l’y faire rentrer ; et comme, avant ce temps, il était défendu à tout le monde, et à sa femme même, de lui donner du pain ou de le recevoir dans sa maison, un tel homme était à l’égard des autres, et les autres étaient à son égard, dans l’état de nature, jusqu’à ce que cet état eût cessé par la composition.\par
À cela près, on voit que les sages des diverses nations barbares songèrent à faire par eux-mêmes ce qu’il était trop long et trop dangereux d’attendre de la convention réciproque des parties. Ils furent attentifs à mettre un prix juste à la composition que devait recevoir celui à qui on avait fait quelque tort ou quelque injure. Toutes ces lois barbares ont là-dessus une précision admirable : on y distingue avec finesse les cas\footnote{Voyez surtout les titres III, IV, V, VI et VII de la loi salique, qui regardent les vols des animaux.}, on y pèse les circonstances ; la loi se met à la place de celui qui est offensé, et demande pour lui la satisfaction que, dans un moment de sang-froid, il aurait demandée lui-même.\par
Ce fut par l’établissement de ces lois que les peuples germains sortirent de cet état de nature où il semble qu’ils étaient encore du temps de Tacite.\par
Rotharis déclara, dans la loi des Lombards, qu’il avait augmenté les compositions de la coutume ancienne pour les blessures, afin que, le blessé étant satisfait, les inimitiés pussent cesser\footnote{Liv. I, tit. VII, § 15.}, En effet, les Lombards, peuple pauvre, s’étant enrichis par la conquête de l’Italie, les compositions anciennes devenaient frivoles, et les réconciliations ne se faisaient plus. Je ne doute pas que cette considération n’ait obligé les autres chefs des nations conquérantes à faire les divers codes de lois que nous avons aujourd’hui.\par
La principale composition était celle que le meurtrier devait payer aux parents du mort. La différence des conditions en mettait une dans les compositions\footnote{Voyez la loi des Angles, tit. I, § 1, 2, 4 ;{\itshape  ibid.}, tit. V, § 6 ; la loi des Bavarois, tit. I, chap. VIII et IX, et la loi des Frisons, tit. XV.} : ainsi, dans la loi des Angles, la composition était de six cents sous pour la mort d’un adalingue, de deux cents pour celle d’un homme libre, de trente pour celle d’un serf. La grandeur de la composition, établie sur la tête d’un homme, faisait donc une de ses grandes prérogatives ; car, outre la distinction qu’elle faisait de sa personne, elle établissait pour lui, parmi des nations violentes, une plus grande sûreté.\par
La loi des Bavarois nous fait bien sentir ceci\footnote{Tit. II, chap. XX.} : elle donne le nom des familles bavaroises qui recevaient une composition double, parce qu’elles étaient les premières après les Agilolfingues\footnote{Hozidra, Ozza, Sagana, Habilingua, Anniena. {\itshape Ibid.}}. Les Agilolfingues étaient de la race ducale, et on choisissait le duc parmi eux : ils avaient une composition quadruple. La composition pour le duc excédait d’un tiers celle qui était établie pour les Agilolfingues. « Parce qu’il est duc, dit la loi, on lui rend un plus grand honneur qu’à ses parents. »\par
Toutes ces compositions étaient fixées à prix d’argent. Mais, comme ces peuples, surtout pendant qu’ils se tinrent dans la Germanie, n’en avaient guère, on pouvait donner du bétail, du blé, des meubles, des armes, des chiens, des oiseaux de chasse, des terres, etc.\footnote{Ainsi la loi d’Ina estimait la vie une certaine somme d’argent, ou une certaine portion de terre. {\itshape Leges Inae regis}, tit. {\itshape de Villico regio. De priscis Anglorum Legibus}, Cambridge, 1644.}. Souvent même la loi fixait la valeur de ces choses\footnote{Voyez la loi des Saxons, qui fait même cette fixation pour plusieurs peuples, chap. XVIII. Voyez aussi la loi des Ripuaires, tit. XXXVI, § 11 ; la loi des Bavarois, tit. I. § 10 et 11. {\itshape Si aurum non habet, donet aliam pecuniam, mancipia, terram}, etc.}, ce qui explique comment, avec si peu d’argent, il y eut chez eux tant de peines pécuniaires.\par
Ces lois s’attachèrent donc à marquer avec précision la différence des torts, des injures, des crimes, afin que chacun connût au juste jusqu’à quel point il était lésé ou offensé ; qu’il sût exactement la réparation qu’il devait recevoir, et surtout qu’il n’en devait pas recevoir davantage.\par
Dans ce point de vue, on conçoit que celui qui se vengeait après avoir reçu la satisfaction commettait un grand crime. Ce crime ne contenait pas moins une offense publique qu’une offense particulière : c’était un mépris de la loi même. C’est ce crime que les législateurs ne manquèrent pas de punir\footnote{Voyez la loi des Lombards, liv. I, tit. XXV, § 21 ; {\itshape ibid.}, liv. I, tit. IX, § 8 et 34 ; {\itshape ibid.}, § 38 ; et le capitulaire de Charlemagne de l’an 802, chap. XXXII, contenant une instruction donnée à ceux qu’il envoyait dans les provinces}.\par
Il y avait un autre crime, qui fut surtout regardé comme dangereux\footnote{Voyez dans Grégoire de Tours, liv. VII, chap. XLVII, le détail d’un procès, où une partie perd la moitié de la composition qui lui avait été adjugée, pour s’être fait justice elle-même, au lieu de recevoir la satisfaction, quelques excès qu’elle eût soufferts depuis.}, lorsque ces peuples perdirent dans le gouvernement civil quelque chose de leur esprit d’indépendance, et que les rois s’attachèrent à mettre dans l’État une meilleure police : ce crime était de ne vouloir point faire, ou de ne vouloir pas recevoir la satisfaction. Nous voyons, dans divers codes des lois des Barbares, que les législateurs y obligeaient\footnote{Voyez la loi des Saxons, chap. III, § 4 ; la loi des Lombards, liv. I, tit. XXXVII, § 1 et 2 ; et la loi des Allemands, tit. XLV, § 1 et 2. Cette dernière loi permettait de se faire justice soi-même, sur-le-champ, et dans le premier mouvement. Voyez aussi les capitulaires de Charlemagne, de l’an 779, chap. XXII ; de l’an 802, chap. XXXII ; et celui du même, de l’an 805, chap. V.}. En effet, celui qui refusait de recevoir la satisfaction voulait conserver son droit de vengeance ; celui qui refusait de la faire laissait à l’offensé son droit de vengeance : et c’est ce que les gens sages avaient réformé dans les institutions des Germains, qui invitaient à la composition, mais n’y obligeaient pas.\par
Je viens de parler d’un texte de la loi salique, où le législateur laissait à la liberté de l’offensé de recevoir ou de ne recevoir pas la satisfaction : c’est cette loi qui interdisait à celui qui avait dépouillé un cadavre le commerce des hommes, jusqu’à ce que les parents, acceptant la satisfaction, eussent demandé qu’il pût vivre parmi les hommes\footnote{Les compilateurs des lois des Ripuaires paraissent avoir modifié ceci. Voyez le titre LXXXV de ces lois.}. Le respect pour les choses saintes fit que ceux qui rédigèrent les lois saliques ne touchèrent point à l’ancien usage.\par
Il aurait été injuste d’accorder une composition aux parents d’un voleur tué dans l’action du vol, ou à ceux d’une femme qui avait été renvoyée après une séparation pour crime d’adultère. La loi des Bavarois ne donnait point de composition dans des cas pareils, et punissait les parents qui en poursuivaient la vengeance\footnote{Voyez le décret de Tassilon, {\itshape de popularibus legibus}, art. 3, 4, 10, 16, 19 ; la loi des Angles, tit. VII, § 4.}.\par
Il n’est pas rare de trouver dans les codes des lois des Barbares des compositions pour des actions involontaires. La loi des Lombards est presque toujours sensée ; elle voulait que, dans ce cas, on composât suivant sa générosité, et que les parents ne pussent plus poursuivre la vengeance\footnote{Liv. I, tit. IX, § 4.}.\par
Clotaire II fit un décret très sage ; il défendit à celui qui avait été volé de recevoir sa composition en secret\footnote{{\itshape Pactus pro tenore pacis inter Childebertum et Clotarium, anno 593} ; et {\itshape decretio Clotarii II regis, circa annum 595}, chap. XI.}, et sans l’ordonnance du juge. On va voir tout à l’heure le motif de cette loi.
\subsubsection[{Chapitre XX. De ce qu’on a appelé depuis la justice des seigneurs}]{Chapitre XX. De ce qu’on a appelé depuis la justice des seigneurs}
\noindent Outre la composition qu’on devait payer aux parents pour les meurtres, les torts et les injures, il fallait encore payer un certain droit que les codes des lois des Barbares appellent {\itshape fredum}\footnote{Lorsque la loi ne le fixait pas, il était ordinairement le tiers de ce qu’on donnait pour la composition, comme il paraît dans la loi des Ripuaires, chap. LXXXIX, qui est expliquée par le troisième capitulaire de l’an 813, édition de Baluze, t. I, p. 512.}. J’en parlerai beaucoup ; et, pour en donner l’idée, je dirai que c’est la récompense de la protection accordée contre le droit de vengeance. Encore aujourd’hui, dans la langue suédoise, {\itshape fred} veut dire la paix.\par
Chez ces nations violentes, rendre la justice n’était autre chose qu’accorder à celui qui avait fait une offense sa protection contre la vengeance de celui qui l’avait reçue, et obliger ce dernier à recevoir la satisfaction qui lui était due : de sorte que, chez les Germains, à la différence de tous les autres peuples, la justice se rendait pour protéger le criminel contre celui qu’il avait offensé.\par
Les codes des lois des Barbares nous donnent les cas où ces {\itshape freda} devaient être exigés. Dans ceux où les parents ne pouvaient pas prendre de vengeance, ils ne donnent point de {\itshape fredum} : en effet, là où il n’y avait point de vengeance, il ne pouvait y avoir de droit de protection contre la vengeance. Ainsi, dans la loi des Lombards\footnote{Liv. I, tit. IX, § 17, édition de Lindembrock.}, si quelqu’un tuait par hasard un homme libre, il payait la valeur de l’homme mort, sans le {\itshape fredum} ; parce que, l’ayant tué involontairement, ce n’était pas le cas où les parents eussent un droit de vengeance. Ainsi, dans la loi des Ripuaires\footnote{Tit. LXX.}, quand un homme était tué par un morceau de bois ou un ouvrage fait de main d’homme, l’ouvrage ou le bois étaient censés coupables, et les parents les prenaient pour leur usage, sans pouvoir exiger de {\itshape fredum}.\par
De même, quand une bête avait tué un homme, la même loi\footnote{Tit. XLVI. Voyez aussi la loi des Lombards, liv. I, chap. XXI, § 3, édition de Lindembrock : Si {\itshape caballus cum pede}, etc.} établissait une composition sans le {\itshape fredum}, parce que les parents du mort n’étaient point offensés.\par
Enfin, par la loi salique\footnote{Tit. XXVIII, § 6.}, un enfant qui avait commis quelque faute avant l’âge de douze ans payait la composition sans le {\itshape fredum} : comme il ne pouvait porter encore les armes, il n’était point dans le cas où la partie lésée ou ses parents pussent demander la vengeance.\par
C’était le coupable qui payait le {\itshape fredum}, pour la paix et la sécurité que les excès qu’il avait commis lui avaient fait perdre, et qu’il pouvait recouvrer par la protection ; mais un enfant ne perdait point cette sécurité ; il n’était point un homme, et ne pouvait être mis hors de la société des hommes.\par
Ce {\itshape fredum} était un droit local pour celui qui jugeait dans le territoire\footnote{Comme il paraît par le décret de Clotaire II, de l’an 595. {\itshape Fredus tamen judicis, in cujus pago est, reservetur.}}. La loi des Ripuaires\footnote{Tit. LXXXIX.} lui défendait pourtant de l’exiger lui-même ; elle voulait que la partie qui avait obtenu gain de cause, le reçût et le portât au fisc, pour que la paix, dit la loi, fût éternelle entre les Ripuaires.\par
La grandeur du {\itshape fredum} se proportionna à la grandeur de la protection\footnote{{\itshape Capitulare incerti anni cha}p. LVII, dans Baluze, t. I, p. 515, Et il faut remarquer que ce qu’on appelle {\itshape fredum ou faida} dans les monuments de la première race, s’appelle {\itshape bannum} dans ceux de la seconde, comme il paraît par le capitulaire {\itshape de partibus Saxoniae}, de l’an 789.} : ainsi le {\itshape fredum} pour la protection du roi fut plus grand que celui accordé pour la protection du comte et des autres juges.\par
Je vois déjà naître la justice des seigneurs. Les fiefs comprenaient de grands territoires, comme il paraît par une infinité de monuments. J’ai déjà prouvé que les rois ne levaient rien sur les terres qui étaient du partage des Francs ; encore moins pouvaient-ils se réserver des droits sur les fiefs. Ceux qui les obtinrent eurent à cet égard la jouissance la plus étendue ; ils en tirèrent tous les fruits et tous les émoluments ; et, comme un des plus considérables était les profits judiciaires {\itshape (freda)} que l’on recevait par les usages des Francs\footnote{Voyez le capitulaire de Charlemagne, {\itshape de Villi}s, où il met ces {\itshape freda} au nombre des grands revenus de ce qu’on appelait {\itshape villae}, ou domaines du roi.}, il suivait que celui qui avait le fief avait aussi la justice, qui ne s’exerçait que par des compositions aux parents et des profits au seigneur. Elle n’était autre chose que le droit de faire payer les compositions de la loi, et celui d’exiger les amendes de la loi.\par
On voit, par les formules qui portent la confirmation ou la translation à perpétuité d’un fief en faveur d’un leude ou fidèle\footnote{Voyez les formules 3, 4 et 17, liv. I, de Marculfe.}, ou des privilèges des fiefs en faveur des églises\footnote{{\itshape Ibid.}, Formule 2, 3 et 4.}, que les fiefs avaient ce droit. Cela paraît encore par une infinité de chartes\footnote{Voyez les recueils de ces chartes, surtout celui qui est à la fin du cinquième volume des {\itshape Historiens de France} des PP. bénédictins.} qui contiennent une défense aux juges ou officiers du roi d’entrer dans le territoire, pour y exercer quelque acte de justice que ce fût, et y exiger quelque émolument de justice que ce fût. Dès que les juges royaux ne pouvaient plus rien exiger dans un district, ils n’entraient plus dans ce district ; et ceux à qui restait ce district y faisaient les fonctions que ceux-là y avaient faites.\par
Il est défendu aux juges royaux d’obliger les parties de donner des cautions pour comparaître devant eux : c’était donc à celui qui recevait le territoire à les exiger. Il est dit que les envoyés du roi ne pourraient plus demander de logement ; en effet, ils n’y avaient plus aucune fonction.\par
La justice fut donc, dans les fiefs anciens et dans les fiefs nouveaux, un droit inhérent au fief même, un droit lucratif qui en faisait partie. C’est pour cela que, dans tous les temps, elle a été regardée ainsi ; d’où est né ce principe, que les justices sont patrimoniales en France.\par
Quelques-uns ont cru que les justices tiraient leur origine des affranchissements que les rois et les seigneurs firent de leurs serfs. Mais les nations germaines, et celles qui en sont descendues, ne sont pas les seules qui aient affranchi des esclaves, et ce sont les seules qui aient établi des justices patrimoniales. D’ailleurs, les {\itshape Formules} de Marculfe\footnote{Voyez la 3, 4 et 14 du liv. I ; et la charte de Charlemagne, de l’an 771, dans Martène t. I, {\itshape Anecdot. collect.} 11. {\itshape Praecipientes jubemus ut ullus judex publicus… homines ipsius ecclesiae et monasterii ipsius Morbacensis, tam ingenuos quam et servos, et qui super eorum terras manere}, etc.} nous font voir des hommes libres dépendant de ces justices dans les premiers temps : les serfs ont donc été justiciables, parce qu’ils se sont trouvés dans le territoire ; et ils n’ont pas donné l’origine aux fiefs, pour avoir été englobés dans le fief.\par
D’autres gens ont pris une voie plus courte : les seigneurs ont usurpé les justices, ont-ils dit ; et tout a été dit. Mais n’y a-t-il eu sur la terre que les peuples descendus de la Germanie, qui aient usurpé les droits des princes ? L’histoire nous apprend assez que d’autres peuples ont fait des entreprises sur leurs souverains ; mais on n’en voit pas naître ce que l’on a appelé les justices des seigneurs. C’était donc dans le fond des usages et des coutumes des Germains qu’il en fallait chercher l’origine.\par
Je prie de voir, dans Loyseau\footnote{{\itshape Traité des justices de village}.}, quelle est la manière dont il suppose que les seigneurs procédèrent pour former et usurper leurs diverses justices. Il faudrait qu’ils eussent été les gens du monde les plus raffinés, et qu’ils eussent volé, non pas comme les guerriers pillent, mais comme des juges de village et des procureurs se volent entre eux. Il faudrait dire que ces guerriers, dans toutes les provinces particulières du royaume et dans tant de royaumes, auraient fait un système général de politique. Loyseau les fait raisonner comme dans son cabinet il raisonnait lui-même.\par
Je le dirai encore : si la justice n’était point une dépendance du fief, pourquoi voit-on partout\footnote{Voyez M. Du Cange, au mot {\itshape hominium.}} que le service du fief était de servir le roi ou le seigneur, et dans leurs cours et dans leurs guerres ?
\subsubsection[{Chapitre XXI. De la justice territoriale des églises}]{Chapitre XXI. De la justice territoriale des églises}
\noindent Les églises acquirent des biens très considérables. Nous voyons que les rois leur donnèrent de grands fiscs, c’est-à-dire de grands fiefs ; et nous trouvons d’abord les justices établies dans les domaines de ces églises. D’où aurait pris son origine un privilège si extraordinaire ? Il était dans la nature de la chose donnée ; le bien des ecclésiastiques avait ce privilège, parce qu’on ne le lui ôtait pas. On donnait un fisc à l’Église, et on lui laissait les prérogatives qu’il aurait eues, si on l’avait donné à un leude ; aussi fut-il soumis au service que l’État en aurait tiré, s’il avait été accordé au laïque, comme on l’a déjà vu.\par
Les églises eurent donc le droit de faire payer les compositions dans leur territoire, et d’en exiger le {\itshape fredum} ; et, comme ces droits emportaient nécessairement celui d’empêcher les officiers royaux d’entrer dans le territoire pour exiger ces {\itshape freda}, et y exercer tous actes de justice, le droit qu’eurent les ecclésiastiques de rendre la justice dans leur territoire fut appelé {\itshape immunité}, dans le style des formules\footnote{Voyez les {\itshape formules} 3 et 4 de Marculfe, liv. I.}, des chartes et des capitulaires.\par
La loi des Ripuaires\footnote{{\itshape Ne aliubi nisi ad ecclesiam, ubi relaxati sunt, mallum teneant}, tit. LVIII, § 1. Voyez aussi le § 19, édition de Lindembrock.} défend aux affranchis des églises\footnote{{\itshape Tabulariis}.} de tenir l’assemblée où la justice se rend\footnote{{\itshape Mallum}.} ailleurs que dans l’église où ils ont été affranchis. Les églises avaient donc des justices, même sur les hommes libres, et tenaient leurs plaids dès les premiers temps de la monarchie.\par
Je trouve dans la {\itshape Vie des saints}\footnote{{\itshape Vita sancti Germerii, episcopi Tolosani, apud Bollandianos, 16 maii.}}, que Clovis donna à un saint personnage la puissance sur un territoire de six lieues de pays, et qu’il voulut qu’il fût libre de toute juridiction quelconque. Je crois bien que c’est une fausseté, mais c’est une fausseté très ancienne ; le fond de la vie et les mensonges se rapportent aux mœurs et aux lois du temps ; et ce sont ces mœurs et ces lois que l’on cherche ici\footnote{Voyez aussi la {\itshape Vie de saint Melanius} et celle {\itshape de saint Déicole}.}.\par
Clotaire II ordonne\footnote{Dans le concile de Paris, l’an 615. {\itshape Episcopi, vel potentes, qui in aliis possident regionibus, judices vel missos discussores de aliis provinciis non instituant, nisi de loco, qui justitiam percipiant et aliis reddant} : art. 19. Voyez aussi l’article 12.} aux évêques, ou aux grands, qui possèdent des terres dans des pays éloignés, de choisir dans le lieu même ceux qui doivent rendre la justice ou en recevoir les émoluments.\par
Le même prince\footnote{Dans le concile de Paris, l’an 615, art. 5.} règle la compétence entre les juges des églises et ses officiers. Le capitulaire de Charlemagne, de l’an 802, prescrit aux évêques et aux abbés les qualités que doivent avoir leurs officiers de justice. Un autre\footnote{Dans la loi des Lombards, liv. II, tit. XLIV, chap. II, édition de Lindembrock.} du même prince défend aux officiers royaux d’exercer aucune juridiction sur ceux qui cultivent les terres ecclésiastiques \footnote{{\itshape Servi aldiones, libellarii antiqui, vel alii noviter facti.Ibid.}}, à moins qu’ils n’aient pris cette condition en fraude, et pour se soustraire aux charges publiques. Les évêques assemblés à Reims déclarèrent\footnote{Lettre de l’an 858, art. 7, dans les {\itshape Capitulaires}, p. 108. {\itshape Sicut illae res et facultates in quibus vivunt clerici, ita et illae sub consecratione immunitatis sunt de quibus debent militare vassalli.}} que les vassaux des églises sont dans leur immunité. Le capitulaire de Charlemagne, de l’an 806\footnote{Il est ajouté à la loi des Bavarois, art. 7 ; voyez aussi l’art. 3 de l’édition de Lindembrock, p. 444{\itshape . Imprimis omnium jubendum est ut habeant ecclesiae earum justitias, et in vita illorum qui habitant in ipsis ecclesiis et post, tain in pecuniis quam et in substantiis earum.}}, veut que les églises aient la justice criminelle et civile sur tous ceux qui habitent dans leur territoire. Enfin, le capitulaire de Charles le Chauve\footnote{De l’an 857, {\itshape in synodo apud Carisiacum}, art. 4, édition de Baluze, p. 96.} distingue les juridictions du roi, celles des seigneurs et celles des églises ; et je n’en dirai pas davantage.\par
 \textbf{Chapitre XXII} \par
 \textbf{ {\itshape Que les justices étaient établies avant la fin de la seconde race} }  \par
On a dit que ce fut dans le désordre de la seconde race que les vassaux s’attribuèrent la justice dans leurs fiscs : on a mieux aimé faire une proposition générale que de l’examiner : il a été plus facile de dire que les vassaux ne possédaient pas, que de découvrir comment ils possédaient. Mais les justices ne doivent point leur origine aux usurpations ; elles dérivent du premier établissement, et non pas de sa corruption.\par
« Celui qui tue un homme libre, est-il dit dans la loi des Bavarois\footnote{Tit. III, chap. XIII, édition de Lindembrock.}, paiera la composition à ses parents, s’il en a ; et s’il n’en a point, il la paiera au duc, ou à celui à qui il s’était recommandé pendant sa vie. » On sait ce que c’était que se recommander pour un bénéfice.\par
« Celui à qui on a enlevé son esclave, dit la loi des Allemands\footnote{Tit. LXXXV.}, ira au prince auquel est soumis le ravisseur, afin qu’il en puisse obtenir la composition. »\par
« Si un centenier, est-il dit dans le décret de Childebert\footnote{De l’an 595, art. 11 et 12, édition des capitulaires de Baluze, p. 19. {\itshape Pari conditione convenit ut si una centena in alia centena vestigium secuta fuerit et invenerit, vel in quibuscumque fidelium nostrorum terminis vestigium miserit, et ipsum in aliam centenam minime expellere potuerit, aut convietus reddat latronem}, etc.}, trouve un voleur dans une autre centaine que la sienne, ou dans les limites de nos fidèles, et qu’il ne l’en chasse pas, il représentera le voleur, ou se purgera par serment. » Il y avait donc de la différence entre le territoire des centeniers et celui des fidèles.\par
Ce décret de Childebert explique la constitution de Clotaire\footnote{{\itshape Si vestigius comprobatur latronis, tamen praesentia nihil longe mulctando ; aut si persequens latronem suum comprehenderit, integram sibi compositionem accipiat. Quod si in truste invenitur, medietatem compositionis trustis adquirat, et capitale exigat a latrone}, art. 2 et 3.} de la même année, qui, donnée pour le même cas et sur le même fait, ne diffère que dans les termes ; la constitution appelant {\itshape in truste} ce que le décret appelle {\itshape in terminis fidelium nostrorum}. MM. Bignon et Du Cange\footnote{Voyez le Glossaire, au mot {\itshape trustis}.}, qui ont cru que {\itshape in truste} signifiait domaine d’un autre roi, n’ont pas bien rencontré.\par
Dans une constitution\footnote{Insérée dans la loi des Lombards, liv. II, tit. LII, § 14. C’est le capitulaire de l’an 793, dans Baluze, p. 544, art. 10.} de Pépin, roi d’Italie, faite tant pour les Francs que pour les Lombards, ce prince, après avoir imposé des peines aux comtes et autres officiers royaux qui prévariquent dans l’exercice de la justice, ou qui diffèrent de la rendre, ordonne que\footnote{{\itshape Et si forsitan Francus aut Longobardus habens beneficium justitiam facere noluerit, ille judex in cujus ministerio fuerit, contradicat illi beneficium suum, interim dum ipse aut missus ejus justitiam faciat.} Voyez encore la même loi des Lombards, liv. II, tit. LII, § 2, qui se rapporte au capitulaire de Charlemagne de l’an 779, art. 21.}, s’il arrive qu’un Franc ou un Lombard ayant un fief ne veuille pas rendre la justice, le juge dans le district duquel il sera, suspendra l’exercice de son fief ; et que, dans cet intervalle, lui ou son envoyé rendront la justice.\par
Un capitulaire de Charlemagne\footnote{Le troisième de l’an 812, art. 10.} prouve que les rois ne levaient point partout les {\itshape freda}. Un autre du même prince\footnote{Second capitulaire de l’an 813, art. 14 et 20, p. 509.} nous fait voir les règles féodales et la cour féodale déjà établies. Un autre de Louis le Débonnaire veut que, lorsque celui qui a un fief ne rend pas la justice, ou empêche qu’on ne la rende, on vive à discrétion dans sa maison jusqu’à ce que la justice soit rendue\footnote{{\itshape Capitulare quintum anni 819}, art. 23, édition de Baluze, p. 617. {\itshape Ut ubicumque missi, aut episcopum, aut abbatem, aut alium quemlibet, honore praeditum invenerint, qui justitiam facere noluit vel prohibuit, de ipsius rebus vivant quandiu in eo loco justitias facere debent.}}. Je citerai encore deux capitulaires de Charles le Chauve, l’un de l’an 861\footnote{{\itshape Edictum in Carisiaco}, dans Baluze, t. II, p. 152. {\itshape Unusquisque advocatus pro omnibus de sua advocatione… in convenientia ut cum ministerialibus de sua advocatione quos invenerit contra hune bannum nostrum fecisse… castiget.}}, où l’on voit des juridictions particulières établies, des juges et des officiers sous eux ; l’autre de l’an 864\footnote{{\itshape Edictum Pistense}, art. 18, édition de Baluze, t. II, p. 181. {\itshape Si in fiscum nostrum, vel in quamcumque immunitatem, aut alicujus potentis potestatem vel proprietatem confugerit}, etc.}, où il fait la distinction de ses propres seigneuries d’avec celles des particuliers.\par
On n’a point de concessions originaires des fiefs, parce qu’ils furent établis par le partage qu’on sait avoir été fait entre les vainqueurs. On ne peut donc pas prouver par des contrats originaires, que les justices, dans les commencements, aient été attachées aux fiefs. Mais si, dans les formules des confirmations, ou des translations à perpétuité de ces fiefs, on trouve, comme on a dit, que la justice y était établie, il fallait bien que ce droit de justice fût de la nature du fief et une de ses principales prérogatives.\par
Nous avons un plus grand nombre de monuments qui établissent la justice patrimoniale des églises dans leur territoire, que nous n’en avons pour prouver celle des bénéfices ou fiefs des leudes ou fidèles, par deux raisons : la première, que la plupart des monuments qui nous restent ont été conservés ou recueillis par les moines pour l’utilité de leurs monastères ; la seconde, que le patrimoine des églises ayant été formé par des concessions particulières, et une espèce de dérogation à l’ordre établi, il fallait des chartes pour cela ; au lieu que les concessions faites aux leudes étant des conséquences de l’ordre politique, on n’avait pas besoin d’avoir, et encore moins de conserver une charte particulière. Souvent même les rois se contentaient de faire une simple tradition par le sceptre, comme il paraît par la {\itshape Vie de saint Maur}.\par
Mais la troisième formule de Marculfe\footnote{Liv. I. {\itshape Maximum regni nostri augere credimus monimentum, si beneficia opportuna locis ecclesiarum, aut cui volueris dicere, benevola deliberatione concedimus.}} nous prouve assez que le privilège d’immunité, et par conséquent celui de la justice, étaient communs aux ecclésiastiques et aux séculiers, puisqu’elle est faite pour les uns et pour les autres. Il en est de même de la constitution de Clotaire II\footnote{Je l’ai citée dans le chapitre précédent {\itshape : Episcopi vel potentes.}}.
\subsubsection[{Chapitre XXIII. Idée générale du livre de l’établissement de la monarchie française dans les gaules, par M. l’abbé Dubos}]{Chapitre XXIII. Idée générale du livre de l’établissement de la monarchie française dans les gaules, par M. l’abbé Dubos}
\noindent Il est bon qu’avant de finir ce livre, j’examine un peu l’ouvrage de M. l’abbé Dubos, parce que mes idées sont perpétuellement contraires aux siennes ; et que, s’il a trouvé la vérité, je ne l’ai pas trouvée.\par
Cet ouvrage a séduit beaucoup de gens, parce qu’il est écrit avec beaucoup d’art ; parce qu’on y suppose éternellement ce qui est en question ; parce que, plus on y manque de preuves, plus on y multiplie les probabilités ; parce qu’une infinité de conjectures sont mises en principe, et qu’on en tire comme conséquences d’autres conjectures. Le lecteur oublie qu’il a douté pour commencer à croire. Et, comme une érudition sans fin est placée, non pas dans le système, mais à côté du système, l’esprit est distrait par des accessoires, et ne s’occupe plus du principal. D’ailleurs, tant de recherches ne permettent pas d’imaginer qu’on n’ait rien trouvé ; la longueur du voyage fait croire qu’on est enfin arrivé.\par
Mais quand on examine bien, on trouve un colosse immense qui a des pieds d’argile ; et c’est parce que les pieds sont d’argile, que le colosse est immense. Si le système de M. l’abbé Dubos avait eu de bons fondements, il n’aurait pas été obligé de faire trois mortels volumes pour le prouver ; il aurait tout trouvé dans son sujet ; et, sans aller chercher de toutes parts ce qui en était très loin, la raison elle-même se serait chargée de placer cette vérité dans la chaîne des autres vérités. L’histoire et nos lois lui auraient dit : « Ne prenez pas tant de peine : nous rendrons témoignage de vous. »
\subsubsection[{Chapitre XXIV. Continuation du même sujet. Réflexion sur le fond du système}]{Chapitre XXIV. Continuation du même sujet. Réflexion sur le fond du système}
\noindent M. l’abbé Dubos veut ôter toute espèce d’idée que les Francs soient entrés dans les Gaules en conquérants : selon lui, nos rois, appelés par les peuples, n’ont fait que se mettre à la place, et succéder aux droits des empereurs romains.\par
Cette prétention ne peut pas s’appliquer au temps où Clovis, entrant dans les Gaules, saccagea et prit les villes ; elle ne peut pas s’appliquer non plus au temps où il défit Syagrius, officier romain, et conquit le pays qu’il tenait : elle ne peut donc se rapporter qu’à celui où Clovis, devenu maître d’une grande partie des Gaules par la violence, aurait été appelé par le choix et l’amour des peuples à la domination du reste du pays. Et il ne suffit pas que Clovis ait été reçu, il faut qu’il ait été appelé ; il faut que M. l’abbé Dubos prouve que les peuples ont mieux aimé vivre sous la domination de Clovis, que de vivre sous la domination des Romains, ou sous leurs propres lois. Or, les Romains de cette partie des Gaules qui n’avait point encore été envahie par les Barbares, étaient, selon M. l’abbé Dubos, de deux sortes : les uns étaient de la confédération armorique, et avaient chassé les officiers de l’empereur, pour se défendre eux-mêmes contre les Barbares, et se gouverner par leurs propres lois ; les autres obéissaient aux officiers romains. Or, M. l’abbé Dubos prouve-t-il que les Romains, qui étaient encore soumis à l’empire, aient appelé Clovis ? point du tout. Prouve-t-il que la république des Armoriques ait appelé Clovis, et fait même quelque traité avec lui ? point du tout encore. Bien loin qu’il puisse nous dire quelle fut la destinée de cette république, il n’en saurait pas même montrer l’existence ; et, quoiqu’il la suive depuis le temps d’Honorius jusqu’à la conquête de Clovis, quoiqu’il y rapporte avec un art admirable tous les événements de ces temps-là, elle est restée invisible dans les auteurs. Car il y a bien de la différence entre prouver, par un passage de Zozime\footnote{{\itshape Histoire}, liv. VI.}, que, sous l’empire d’Honorius, la contrée armorique et les autres provinces des Gaules se révoltèrent, et formèrent une espèce de république\footnote{{\itshape Totusque tractus armoricus, aliaeque Galliarum provinciae.Ibid.}} ; et faire voir que, malgré les diverses pacifications des Gaules, les Armoriques formèrent toujours une république particulière, qui subsista jusqu’à la conquête de Clovis. Cependant il aurait besoin, pour établir son système, de preuves bien fortes et bien précises. Car, quand on voit un conquérant entrer dans un État, et en soumettre une grande partie par la force et par la violence, et qu’on voit quelque temps après l’État entier soumis, sans que l’histoire dise comment il l’a été, on a un très juste sujet de croire que l’affaire a fini comme elle a commencé.\par
Ce point une fois manqué, il est aisé de voir que tout le système de M. l’abbé Dubos croule de fond en comble ; et, toutes les fois qu’il tirera quelque conséquence de ce principe, que les Gaules n’ont pas été conquises par les Francs, mais que les Francs ont été appelés par les Romains, on pourra toujours la lui nier.\par
M. l’abbé Dubos prouve son principe par les dignités romaines dont Clovis fut revêtu ; il veut que Clovis ait succédé à Childéric son père, dans l’emploi de maître de la milice. Mais ces deux charges sont purement de sa création. La lettre de saint Remi à Clovis, sur laquelle il se fonde\footnote{T. II, liv. III, chap. XVIII, p. 270.}, n’est qu’une félicitation sur son avènement à la couronne. Quand l’objet d’un écrit est connu, pourquoi lui en donner un qui ne l’est pas ?\par
Clovis, sur la fin de son règne, fut fait consul par l’empereur Anastase : mais quel droit pouvait lui donner une autorité simplement annale ? Il y a apparence, dit M. l’abbé Dubos, que, dans le même diplôme, l’empereur Anastase fit Clovis proconsul. Et moi, je dirai qu’il y a apparence qu’il ne le fit pas. Sur un fait qui n’est fondé sur rien, l’autorité de celui qui le nie est égale à l’autorité de celui qui l’allègue. J’ai même une raison pour cela. Grégoire de Tours, qui parle du consulat, ne dit rien du proconsulat. Ce proconsulat n’aurait été même que d’environ six mois. Clovis mourut un an et demi après avoir été fait consul ; il n’est pas possible de faire du proconsulat une charge héréditaire. Enfin, quand le consulat, et, si l’on veut, le proconsulat lui furent donnés, il était déjà le maître de la monarchie, et tous ses droits étaient établis.\par
La seconde preuve que M. l’abbé Dubos allègue, c’est la cession faite par l’empereur Justinien aux enfants et aux petits-enfants de Clovis, de tous les droits de l’empire sur les Gaules. J’aurais bien des choses à dire sur cette cession. On peut juger de l’importance que les rois des Francs y mirent, par la manière dont ils en exécutèrent les conditions. D’ailleurs, les rois des Francs étaient maîtres des Gaules ; ils étaient souverains paisibles ; Justinien n’y possédait pas un pouce de terre ; l’empire d’Occident était détruit depuis longtemps, et l’empereur d’Orient n’avait de droit sur les Gaules que comme représentant l’empereur d’Occident ; c’étaient des droits sur des droits. La monarchie des Francs était déjà fondée ; le règlement de leur établissement était fait ; les droits réciproques des personnes et des diverses nations qui vivaient dans la monarchie étaient convenus ; les lois de chaque nation étaient données, et même rédigées par écrit. Que faisait cette cession étrangère à un établissement déjà formé ?\par
Que veut dire M. l’abbé Dubos avec les déclamations de tous ces évêques, qui, dans le désordre, la confusion, la chute totale de l’État, les ravages de la conquête, cherchent à flatter le vainqueur ? Que suppose la flatterie, que la faiblesse de celui qui est obligé de flatter ? Que prouve la rhétorique et la poésie, que l’emploi même de ces arts ? Qui ne serait étonné de voir Grégoire de Tours, qui, après avoir parlé des assassinats de Clovis, dit que cependant Dieu prosternait tous les jours ses ennemis, parce qu’il marchait dans ses voies ? Qui peut douter que le clergé n’ait été bien aise de la conversion de Clovis, et qu’il n’en ait même tiré de grands avantages ? Mais qui peut douter en même temps que les peuples n’aient essuyé tous les malheurs de la conquête, et que le gouvernement romain n’ait cédé au gouvernement germanique ? Les Francs n’ont point voulu, et n’ont pas même pu tout changer ; et même peu de vainqueurs ont eu cette manie. Mais, pour que toutes les conséquences de M. l’abbé Dubos fussent vraies, il aurait fallu que non seulement ils n’eussent rien changé chez les Romains, mais encore qu’ils se fussent changés eux-mêmes.\par
Je m’engagerais bien, en suivant la méthode de M. l’abbé Dubos, à prouver de même que les Grecs ne conquirent pas la Perse. D’abord, je parlerais des traités que quelques-unes de leurs villes firent avec les Perses : je parlerais des Grecs qui furent à la solde des Perses, comme les Francs furent à la solde des Romains. Que si Alexandre entra dans le pays des Perses, assiégea, prit et détruisit la ville de Tyr, c’était une affaire particulière, comme celle de Syagrius. Mais voyez comment le pontife des Juifs vient au-devant de lui ; écoutez l’oracle de Jupiter Ammon ; ressouvenez-vous comment il avait été prédit à Gordium ; voyez comment toutes les villes courent, pour ainsi dire, au-devant de lui ; comment les satrapes et les grands arrivent en foule. Il s’habille à la manière des Perses ; c’est la robe consulaire de Clovis. Darius ne lui offrit-il pas la moitié de son royaume ? Darius n’est-il pas assassiné comme un tyran ? La mère et la femme de Darius ne pleurent-elles pas la mort d’Alexandre ? Quinte-Curce, Arrien, Plutarque, étaient-ils contemporains d’Alexandre ? L’imprimerie ne nous a-t-elle pas donné des lumières qui manquaient à ces auteurs\footnote{Voyez le {\itshape Discours préliminaire} de M. l’abbé Dubos.} ? Voilà l’histoire de {\itshape l’Établissement de la monarchie française dans les Gaules}.
\subsubsection[{Chapitre XXV. De la noblesse française}]{Chapitre XXV. De la noblesse française}
\noindent M. l’abbé Dubos soutient que, dans les premiers temps de notre monarchie, il n’y avait qu’un seul ordre de citoyens parmi les Francs. Cette prétention, injurieuse au sang de nos premières familles, ne le serait pas moins aux trois grandes maisons qui ont successivement régné sur nous. L’origine de leur grandeur n’irait donc point se perdre dans l’oubli, la nuit et le temps ? L’histoire éclairerait des siècles où elles auraient été des familles communes ; et, pour que Chilpéric, Pépin et Hugues Capet fussent gentilshommes, il faudrait aller chercher leur origine parmi les Romains ou les Saxons, c’est-à-dire parmi les nations subjuguées ?\par
M. l’abbé Dubos fonde son opinion sur la loi salique\footnote{Voyez l’{\itshape Établissement de la monarchie française}, t. III, liv. VI, chap. IV, p. 304.}. Il est clair, dit-il, par cette loi, qu’il n’y avait point deux ordres de citoyens chez les Francs. Elle donnait deux cents sous de composition pour la mort de quelque Franc que ce fût\footnote{Il cite le titre XLIV de cette loi, et la loi des Ripuaires, tit. VII et XXXVI.} {\itshape ;} mais elle distinguait, chez les Romains, le convive du roi, pour la mort duquel elle donnait trois cents sous de composition, du Romain possesseur, à qui elle en donnait cent, et du Romain tributaire, à qui elle n’en donnait que quarante-cinq. Et, comme la différence des compositions faisait la distinction principale, il conclut que, chez les Francs, il n’y avait qu’un ordre de citoyens, et qu’il y en avait trois chez les Romains.\par
Il est surprenant que son erreur même ne lui ait pas fait découvrir son erreur. En effet, il eût été bien extraordinaire que les nobles Romains, qui vivaient sous la domination des Francs, y eussent eu une composition plus grande, et y eussent été des personnages plus importants que les plus illustres des Francs, et leurs plus grands capitaines. Quelle apparence que le peuple vainqueur eût eu si peu de respect pour lui-même, et qu’il en eût eu tant pour le peuple vaincu ? De plus, M. l’abbé Dubos cite les lois des autres nations barbares, qui prouvent qu’il y avait parmi eux divers ordres de citoyens. Il serait bien extraordinaire que cette règle générale eût précisément manqué chez les Francs. Cela aurait dû lui faire penser qu’il entendait mal, ou qu’il appliquait mal les textes de la loi salique : ce qui lui est effectivement arrivé.\par
On trouve, en ouvrant cette loi, que la composition pour la mort d’un antrustion\footnote{{\itshape Qui in truste dominica est}, tit. XLIV, § 4 ; et cela se rapporte à la formule XIII de Marculfe, {\itshape de regis antrustione}. Voyez aussi le tit. LXVI de la loi salique, § 3 et 4 ; et le tit. LXXIV ; et la loi des Ripuaires, tit. XI ; et le capitulaire de Charles le Chauve, {\itshape apud Carisiacum}, de l’an 877, chap. XX.}, c’est-à-dire, d’un fidèle ou vassal du roi, était de six cents sous, et que celle pour la mort d’un Romain, convive du roi, n’était que de trois cents\footnote{Loi salique, tit. XLIV, § 6.}. On y trouve\footnote{{\itshape Ibid.}, § 4.} que la composition pour la mort d’un simple Franc était de deux cents sous\footnote{{\itshape Ibid.}, § 1.} et que celle pour la mort d’un Romain d’une condition ordinaire, n’était que de cent\footnote{{\itshape Ibid.}, tit. XLIV, § 15.}. On payait encore pour la mort d’un Romain tributaire, espèce de serf ou d’affranchi, une composition de quarante-cinq {\itshape sous}\footnote{{\itshape Ibid.}, § 7.} {\itshape ;} mais je n’en parlerai point, non plus que de celle pour la mort du serf franc, ou de l’affranchi franc : il n’est point ici question de ce troisième ordre de personnes.\par
Que fait M. l’abbé Dubos ? Il passe sous silence le premier ordre de personnes chez les Francs, c’est-à-dire, l’article qui concerne les antrustions ; et ensuite, comparant le Franc ordinaire pour la mort duquel on payait deux cents sous de composition, avec ceux qu’il appelle des trois ordres chez les Romains, et pour la mort desquels on payait des compositions différentes, il trouve qu’il n’y avait qu’un seul ordre de citoyens chez les Francs, et qu’il y en avait trois chez les Romains.\par
Comme, selon lui, il n’y avait qu’un seul ordre de personnes chez les Francs, il eût été bon qu’il n’y en eût eu qu’un aussi chez les Bourguignons, parce que leur royaume forma une des principales pièces de notre monarchie. Mais il y a dans leurs codes trois sortes de compositions\footnote{{\itshape Si quis, quolibet casu, dentem optimati Burgundioni vel Romano nobili cxcusserit, solidos viginti quinque cogatur exselvere ; de mediocribus personis ingenuis, tam Burgundionibus quam Romanis, si dens excussus fuerit, decem solidis componatur ; de inferioribus personis, quinque solidos}, art. 1, 2 et 3 du tit. XXVI de la loi des Bourguignons.} ; l’une pour le noble Bourguignon ou Romain, l’autre pour le Bourguignon ou Romain d’une condition médiocre, la troisième pour ceux qui étaient d’une condition inférieure dans les deux nations. M. l’abbé Dubos n’a point cité cette loi.\par
Il est singulier de voir comment il échappe aux passages qui le pressent de toutes parts\footnote{{\itshape Établissement de la monarchie française}, t. III, liv. VI, chap. IV et V.}. Lui parle-t-on des grands, des seigneurs, des nobles ? Ce sont, dit-il, de simples distinctions, et non pas des distinctions d’ordre ; ce sont des choses de courtoisie, et non pas des prérogatives de la loi : ou bien, dit-il, les gens dont on parle étaient du conseil du roi ; ils pouvaient même être des Romains ; mais il n’y avait toujours qu’un seul ordre de citoyens chez les Francs. D’un autre côté, s’il est parlé de quelque Franc d’un rang inférieur, ce sont des serfs\footnote{{\itshape Établissement de la monarchie française}, t. III, chap. V, p. 319 et 320.} ; et c’est de cette manière qu’il interprète le décret de Childebert. Il est nécessaire que je m’arrête sur ce décret. M. l’abbé Dubos l’a rendu fameux, parce qu’il s’en est servi pour prouver deux choses : l’une\footnote{{\itshape Ibid.}, liv. VI, chap. IV, p. 307 et 308.}, que toutes les compositions que l’on trouve dans les lois des Barbares n’étaient que des intérêts civils ajoutés aux peines corporelles, ce qui renverse de fond en comble tous les anciens monuments ; l’autre, que tous les hommes libres étaient jugés directement et immédiatement par le roi\footnote{{\itshape Ibid.}, p. 309 ; et au chap. suiv., p. 319 et 320.}, ce qui est contredit par une infinité de passages et d’autorités qui nous font connaître l’ordre judiciaire de ces temps-là\footnote{Voyez le liv. XXVIII de cet ouvrage, chap. XXVIII{\itshape  ;} et le liv. XXXI, chap. VIII.}.\par
Il est dit dans ce décret, fait dans une assemblée de la nation, que si le juge trouve un voleur fameux, il le fera lier pour être envoyé devant le roi, si c’est un Franc {\itshape (Francus)} ; mais si c’est une personne plus faible {\itshape (debilior persona), il} sera pendu sur le lieu\footnote{{\itshape Itaque Colonia convertit et ita bannivimus, ut unusquisque judex criminosum latronem ut audierit, ad casam suam ambulet, et ipsum ligare faciat : ita ut, si Francus fuerit, ad nostram praesentiam dirigatur et, si debilior persona fuerit, in loco pendatur.} Capitulaire de l’édition de Baluze, t. I, p. 19.}. Selon M. l’abbé Dubos, {\itshape Francus} est un homme libre, {\itshape debilior persona} est un serf. J’ignorerai pour un moment ce que peut signifier ici le mot {\itshape Francus} ; et je commencerai par examiner ce qu’on peut entendre par ces mots {\itshape une personne plus faible}. Je dis que, dans quelque langue que ce soit, tout comparatif suppose nécessairement trois termes, le plus grand, le moindre, et le plus petit. S’il n’était ici question que des hommes libres et des serfs, on aurait dit {\itshape un serf}, et non pas {\itshape un homme d’une moindre puissance}. Ainsi {\itshape debilior persona} ne signifie point là un serf, mais une personne au-dessous de laquelle doit être le serf. Cela supposé, {\itshape Francus} ne signifiera pas un homme libre, mais un homme puissant : et {\itshape Francus} est pris ici dans cette acception, parce que, parmi les Francs, étaient toujours ceux qui avaient dans l’État une plus grande puissance, et qu’il était plus difficile au juge ou au comte de corriger. Cette explication s’accorde avec un grand nombre de capitulaires\footnote{Voyez le livre XXVIII de cet ouvrage, chap. XXVIII ; et le livre XXXI, chap. VIII.} qui donnent les cas dans lesquels les criminels pouvaient être renvoyés devant le roi, et ceux où ils ne le pouvaient pas.\par
On trouve dans la {\itshape Vie de Louis le Débonnaire}, écrite par Tégan\footnote{Chap. XLIII et XLIV.} que les évêques furent les principaux auteurs de l’humiliation de cet empereur, surtout ceux qui avaient été serfs, et ceux qui étaient nés parmi les Barbares. Tégan apostrophe ainsi Hébon, que ce prince avait tiré de la servitude, et avait fait archevêque de Reims : « Quelle récompense l’empereur a-t-il reçue de tant de bienfaits\footnote{{\itshape O qualem remunerationem reddidisti ei ! Fecit te liberum, non nobilem, quod impossibile est post libertatem.Ibid.}} ! Il t’a fait libre, et non pas noble ; il ne pouvait pas te faire noble après t’avoir donné la liberté. »\par
Ce discours, qui prouve si formellement deux ordres de citoyens, n’embarrasse point M. l’abbé Dubos. Il répond ainsi\footnote{{\itshape Établissement de la monarchie française}, t. III, liv. VI, chap. IV, p. 316.} : « Ce passage ne veut point dire que Louis le Débonnaire n’eût pas pu faire entrer Hébon dans l’ordre des nobles. Hébon, comme archevêque de Reims, eût été du premier ordre, supérieur à celui de la noblesse. » Je laisse au lecteur à décider si ce passage ne le veut point dire ; je lui laisse à juger s’il est ici question d’une préséance du clergé sur la noblesse. « Ce passage prouve seulement, continue M. l’abbé Dubos\footnote{{\itshape Ibid.}}, que les citoyens nés libres étaient qualifiés de nobles hommes : dans l’usage du monde, noble homme, et homme né libre ont signifié longtemps la même chose. » Quoi ! sur ce que, dans nos temps modernes, quelques bourgeois ont pris la qualité de nobles hommes, un passage de la vie de Louis le Débonnaire s’appliquera à ces sortes de gens ! « Peut-être aussi, ajoute-t-il encore\footnote{{\itshape Ibid.}}, qu’Hébon n’avait point été esclave dans la nation des Francs, mais dans la nation saxonne, ou dans une autre nation germanique, où les citoyens étaient divisés en plusieurs ordres. » Donc, à cause du {\itshape peut-être} de M. l’abbé Dubos, il n’y aura point eu de noblesse dans la nation des Francs. Mais il n’a jamais plus mal appliqué de {\itshape peut-être}. On vient de voir que Tégan\footnote{{\itshape Omnes episcopi molesti fuerunt Ludovico, et maxime ii quos e servili conditione honoratos habebat, cum his qui ex barbaris nationibus ad hoc fastigium perducti sunt.De gestis Ludovici Pii}, chap. XLIII et XLIV.} distingue les évêques qui avaient été opposés à Louis le Débonnaire, dont les uns avaient été serfs, et les autres étaient d’une nation barbare. Hébon était des premiers, et non pas des seconds. D’ailleurs, je ne sais comment on peut dire qu’un serf tel qu’Hébon aurait été Saxon ou Germain : un serf da point de famille, ni par conséquent de nation. Louis le Débonnaire affranchit Hébon ; et, comme les serfs affranchis prenaient la loi de leur maître, Hébon devint Franc, et non pas Saxon ou Germain.\par
Je viens d’attaquer, il faut que je me défende. On me dira que le corps des antrustions formait bien dans l’État un ordre distingué de celui des hommes libres ; mais que, comme les fiefs furent d’abord amovibles, et ensuite à vie, cela ne pouvait pas former une noblesse d’origine, puisque les prérogatives n’étaient point attachées à un fief héréditaire. C’est cette objection qui a sans doute fait penser à M. de Valois qu’il n’y avait qu’un seul ordre de citoyens chez les Francs : sentiment que M. l’abbé Dubos a pris de lui, et qu’il a absolument gâté à force de mauvaises preuves. Quoi qu’il en soit, ce n’est point M. l’abbé Dubos qui aurait pu faire cette objection. Car, ayant donné trois ordres de noblesse romaine, et la qualité de convive du roi pour le premier, il n’aurait pas pu dire que ce titre marquât plus une noblesse d’origine que celui d’antrustion. Mais il faut une réponse directe. Les antrustions ou fidèles n’étaient pas tels, parce qu’ils avaient un fief ; mais on leur donnait un fief, parce qu’ils étaient antrustions ou fidèles. On se ressouvient de ce que j’ai dit dans les premiers chapitres de ce livre : ils n’avaient pas pour lors, comme ils eurent dans la suite, le même fief ; mais s’ils n’avaient pas celui-là, ils en avaient un autre, et parce que les fiefs se donnaient à la naissance, et parce qu’ils se donnaient souvent dans les assemblées de la nation, et enfin parce que, comme il était de l’intérêt des nobles d’en avoir, il était aussi de l’intérêt du roi de leur en donner. Ces familles étaient distinguées par leur dignité de fidèles, et par la prérogative de pouvoir se recommander pour un fief. Je ferai voir dans le livre suivant\footnote{Chap. XXIII.} comment, par les circonstances des temps, il y eut des hommes libres qui furent admis à jouir de cette grande prérogative, et par conséquent à entrer dans l’ordre de la noblesse. Cela n’était point ainsi du temps de Gontran et de Childebert, son neveu ; et cela était ainsi du temps de Charlemagne. Mais quoique, dès le temps de ce prince, les hommes libres ne fussent pas incapables de posséder des fiefs, il paraît, par le passage de Tégan rapporté ci-dessus, que les serfs affranchis en étaient absolument exclus. M. l’abbé Dubos\footnote{{\itshape Histoire de l’Établissement de la monarchie française}, t. III, liv. VI, chap. IV, p. 302.}, qui va en Turquie pour nous donner une idée de ce qu’était l’ancienne noblesse française, nous dira-t-il qu’on se soit jamais plaint en Turquie de ce qu’on y élevait aux honneurs et aux dignités des gens de basse naissance, comme on s’en plaignait sous les règnes de Louis le Débonnaire et de Charles le Chauve ? On ne s’en plaignait pas du temps de Charlemagne, parce que ce prince distingua toujours les anciennes familles d’avec les nouvelles ; ce que Louis le Débonnaire et Charles le Chauve ne firent pas.\par
Le public ne doit pas oublier qu’il est redevable à M. l’abbé Dubos de plusieurs compositions excellentes. C’est sur ces beaux ouvrages qu’il doit le juger, et non pas sur celui-ci. M. l’abbé Dubos y est tombé dans de grandes fautes, parce qu’il a plus eu devant les yeux M. le comte de Boulainvilliers, que son sujet. Je ne tirerai de toutes mes critiques que cette réflexion : Si ce grand homme a erré, que ne dois-je pas craindre ?
\subsection[{Livre trente et unième. Théorie des lois féodales chez les Francs, dans le rapport qu’elles ont avec les révolutions de leur monarchie}]{Livre trente et unième. Théorie des lois féodales chez les Francs, dans le rapport qu’elles ont avec les révolutions de leur monarchie}
\subsubsection[{Chapitre I. Changements dans les offices et les fiefs}]{Chapitre I. Changements dans les offices et les fiefs}
\noindent D’abord les comtes n’étaient envoyés dans leurs districts que pour un an ; bientôt ils achetèrent la continuation de leurs offices. On en trouve un exemple dès le règne des petits-enfants de Clovis. Un certain Peonius était comte dans la ville d’Auxerre\footnote{Grégoire de Tours, liv. IV, chap. XLII.} ; il envoya son fils Mummolus porter de l’argent à Gontran pour être continué dans son emploi ; le fils donna de l’argent pour lui-même, et obtint la place du père. Les rois avaient déjà commencé à corrompre leurs propres grâces.\par
Quoique, par la loi du royaume, les fiefs fussent amovibles, ils ne se donnaient pourtant, ni ne s’ôtaient d’une manière capricieuse et arbitraire ; et c’était ordinairement une des principales choses qui se traitaient dans les assemblées de la nation. On peut bien penser que la corruption se glissa dans ce point, comme elle s’était glissée dans l’autre ; et que l’on continua la possession des fiefs pour de l’argent, comme on continuait la possession des comtés.\par
Je ferai voir, dans la suite de ce livre\footnote{Chap. VII.}, qu’indépendamment des dons que les princes firent pour un temps, il y en eut d’autres qu’ils firent pour toujours. Il arriva que la cour voulut révoquer les dons qui avaient été faits : cela mit un mécontentement général dans la nation, et l’on en vit bientôt naître cette révolution fameuse dans l’histoire de France, dont la première époque fut le spectacle étonnant du supplice de Brunehault.\par
Il paraît d’abord extraordinaire que cette reine, fille, sœur, mère de tant de rois, fameuse encore aujourd’hui par des ouvrages dignes d’un édile ou d’un proconsul romain, née avec un génie admirable pour les affaires, douée de qualités qui avaient été si longtemps respectées, se soit vue tout à coup exposée à des supplices si longs, si honteux, si cruels\footnote{{\itshape Chronique} de Frédégaire, chap. XLII.}, par un roi\footnote{Clotaire II, fils de Chilpéric, et père de Dagobert.} dont l’autorité était assez mal affermie dans sa nation, si elle n’était tombée, par quelque cause particulière, dans la disgrâce de cette nation. Clotaire lui reprocha la mort de dix rois\footnote{{\itshape Chronique} de Frédégaire, chap. {\itshape XLII.}} {\itshape ;} mais il y en avait deux qu’il fit lui-même mourir {\itshape ; la} mort de quelques autres fut le crime du sort ou de la méchanceté d’une autre reine ; et une nation qui avait laissé mourir Frédégonde dans son lit, qui s’était même opposée à la punition de ses épouvantables crimes\footnote{Voyez Grégoire de Tours, liv. VIII, chap. XXXI.}, devait être bien froide sur ceux de Brunehault.\par
Elle fut mise sur un chameau, et on la promena dans toute l’armée ; marque certaine qu’elle était tombée dans la disgrâce de cette armée. Frédégaire dit que Protaire, favori de Brunehault, prenait le bien des seigneurs, et en gorgeait le fisc, qu’il humiliait la noblesse, et que personne ne pouvait être sûr de garder le poste qu’il avait\footnote{{\itshape Saeva illi fuit contra personas iniquitas, fisco nimium tribuens, de rebus personarum ingeniose fiscum vellens implere… ut nullus reperiretur qui gradum quem arripuerat potuisset adsumere. Chronique} de Frédégaire, chap. XXVII, sur l’an 605.}. L’armée conjura contre lui, on le poignarda dans sa tente ; et Brunehault, soit par les vengeances qu’elle tira de cette mort\footnote{{\itshape Ibid.}, chap. XXVIII, sur l’an 607.}, soit par la poursuite du même plan, devint tous les jours plus odieuse à la nation\footnote{{\itshape Ibid.}, chap. XLI, sur l’an 613. {\itshape Burgundiae farones, tam episcopi quam ceteri leudes, timentes Brunichildem, et odium in eam habentes, consilium inientes}, etc.}.\par
Clotaire, ambitieux de régner seul, et plein de la plus affreuse vengeance, sûr de périr si les enfants de Brunehault avaient le dessus, entra dans une conjuration contre lui-même ; et, soit qu’il fût malhabile, ou qu’il fût forcé par les circonstances, il se rendit accusateur de Brunehault, et fit faire de cette reine un exemple terrible.\par
Warnachaire avait été l’âme de la conjuration contre Brunehault ; il fut fait maire de Bourgogne ; il exigea de Clotaire qu’il ne serait jamais déplacé pendant sa vie\footnote{{\itshape Chronique} de Frédégaire, chap. XLII, sur l’an 613. {\itshape Sacramento a Clotario, accepto ne unquam vitae suae temporibus degradaretur.}}. Par là le maire ne put plus être dans le cas où avaient été les seigneurs français ; et cette autorité commença à se rendre indépendante de l’autorité royale.\par
C’était la funeste régence de Brunehault qui avait surtout effarouché la nation. Tandis que les lois subsistèrent dans leur force, personne ne put se plaindre de ce qu’on lui ôtait un fief, puisque la loi ne le lui donnait pas pour toujours ; mais quand l’avarice, les mauvaises pratiques, la corruption firent donner des fiefs, on se plaignit de ce qu’on était privé par de mauvaises voies des choses que souvent on avait acquises de même. Peut-être que si le bien publie avait été le motif de la révocation des dons, on n’aurait rien dit ; mais on montrait l’ordre, sans cacher la corruption ; on réclamait le droit du fisc, pour prodiguer les biens du fisc à sa fantaisie ; les dons ne furent plus la récompense ou l’espérance des services. Brunehault, par un esprit corrompu, voulut corriger les abus de la corruption ancienne. Ses caprices n’étaient point ceux d’un esprit faible : les leudes et les grands officiers se crurent perdus ; ils la perdirent.\par
Il s’en faut bien que nous ayons tous les actes qui furent passés dans ces temps-là ; et les faiseurs de chroniques, qui savaient à peu près de l’histoire de leur temps, ce que les villageois savent aujourd’hui de celle du nôtre, sont très stériles. Cependant nous avons une constitution de Clotaire, donnée dans le concile de Paris\footnote{Quelque temps après le supplice de Brunehault, l’an 615. Voyez l’édition des {\itshape Capitulaires} de Baluze, p. 21.} pour la réformation des abus, qui fait voir que ce prince fit cesser les plaintes qui avaient donné lieu à la révolution\footnote{{\itshape Quae contra rationis ordinem acta vel ordinata sunt, ne in antea, quod avertat divinitas, contingant, disposuerimus, Christo praesule, per hujus edicti tenorem generaliter emendare.In proemio, ibid.}, art. 16.}. D’un côté, il y confirme tous les dons qui avaient été faits ou confirmés par les rois ses prédécesseurs\footnote{{\itshape Ibid.}, art. 16.} ; et il ordonne, de l’autre, que tout ce qui a été ôté à ses leudes ou fidèles leur soit rendu\footnote{{\itshape Ibid.}, art. 17.}.\par
Ce ne fut pas la seule concession que le roi fit dans ce concile. Il voulut que ce qui avait été fait contre les privilèges des ecclésiastiques fût corrigé\footnote{{\itshape Et quod per tempora ex hoc praetermissum est, vel dehinc, perpetualiter observetur.}} : il modéra l’influence de la cour dans les élections aux évêchés\footnote{{\itshape Ita ut episcopo decedente, in loco ipsius qui a metropolitano ordinari debet cum principalibus, a clero et populo eligatur ; et si persona condigna fuerit, per ordinationem principis ordinetur ; vel certe si de palatio eligitur, per meritum personae et doctrinae ordinetur.Ibid.}, art. I.}. Le roi réforma de même les affaires fiscales : il voulut que tous les nouveaux cens fussent ôtés\footnote{{\itshape Ut ubicumque census novus impie additus est,… emendetur}, art. 8.} ; qu’on ne levât aucun droit de passage établi depuis la mort de Gontran, Sigebert et Chilpéric\footnote{{\itshape Ibid.}, art. 9.} ; c’est-à-dire, qu’il supprimait tout ce qui avait été fait pendant les régences de Frédégonde et de Brunehault : il défendit que ses troupeaux fussent menés dans les forêts des particuliers\footnote{{\itshape Ibid.}, art. 21.} : et nous allons voir tout à l’heure que la réforme fut encore plus générale, et s’étendit aux affaires civiles.
\subsubsection[{Chapitre II. Comment le gouvernement civil fut réformé}]{Chapitre II. Comment le gouvernement civil fut réformé}
\noindent On avait vu jusqu’ici la nation donner des marques d’impatience et de légèreté sur le choix, ou sur la conduite de ses maîtres ; on l’avait vue régler les différends de ses maîtres entre eux, et leur imposer la nécessité de la paix. Mais, ce qu’on n’avait pas encore vu, la nation le fit pour lors : elle jeta les yeux sur sa situation actuelle ; elle examina ses lois de sang-froid ; elle pourvut à leur insuffisance ; elle arrêta la violence ; elle régla le pouvoir.\par
Les régences mâles, hardies et insolentes de Frédégonde et de Brunehault avaient moins étonné cette nation, qu’elles ne l’avaient avertie. Frédégonde avait défendu ses méchancetés par ses méchancetés mêmes ; elle avait justifié le poison et les assassinats par le poison et les assassinats ; elle s’était conduite de manière que ses attentats étaient encore plus particuliers que publics. Frédégonde fit plus de maux, Brunehault en fit craindre davantage. Dans cette crise, la nation ne se contenta pas de mettre ordre au gouvernement féodal, elle voulut aussi assurer son gouvernement civil : car celui-ci était encore plus corrompu que l’autre ; et cette corruption était d’autant plus dangereuse qu’elle était plus ancienne, et tenait plus, en quelque sorte, à l’abus des mœurs qu’à l’abus des lois.\par
L’Histoire de Grégoire de Tours et les autres monuments nous font voir, d’un côté, une nation féroce et barbare ; et, de l’autre, des rois qui ne l’étaient pas moins. Ces princes étaient meurtriers, injustes et cruels, parce que toute la nation l’était. Si le christianisme parut quelquefois les adoucir, ce ne fut que par les terreurs que le christianisme donne aux coupables. Les églises se défendirent contre eux par les miracles et les prodiges de leurs saints. Les rois n’étaient point sacrilèges, parce qu’ils redoutaient les peines des sacrilèges ; mais d’ailleurs ils commirent, ou par colère, ou de sang-froid, toutes sortes de crimes et d’injustices, parce que ces crimes et ces injustices ne leur montraient pas la main de la divinité si présente. Les Francs, comme j’ai dit, souffraient des rois meurtriers, parce qu’ils étaient meurtriers eux-mêmes ; ils n’étaient point frappés des injustices et des rapines de leurs rois, parce qu’ils étaient ravisseurs et injustes comme eux. Il y avait bien des lois établies ; mais les rois les rendaient inutiles par de certaines lettres appelées {\itshape Préceptions}\footnote{C’étaient des ordres que le roi envoyait aux juges, pour faire ou souffrir de certaines choses contre la loi.}, qui renversaient ces mêmes lois : c’était à peu près comme les rescrits des empereurs romains, soit que les rois eussent pris d’eux cet usage, soit qu’ils l’eussent tiré du fond même de leur naturel. On voit dans Grégoire de Tours qu’ils faisaient des meurtres de sang-froid, et faisaient mourir des accusés qui n’avaient pas seulement été entendus ; ils donnaient des préceptions pour faire des mariages illicites \footnote{Voyez Grégoire de Tours, liv. IV, p. 227. L’histoire et les chartes sont pleines de ceci ; et l’étendue de ces abus paraît surtout dans l’édit de Clotaire II, de l’an 615, donné pour les réformer. Voyez les {\itshape Capitulaires}, édition de Baluze, t. I, p. 22.} {\itshape ; ils} en donnaient pour transporter les successions ; ils en donnaient pour ôter le droit des parents ; ils en donnaient pour épouser les religieuses. Ils ne faisaient point, à la vérité, des lois de leur seul mouvement ; mais ils suspendaient la pratique de celles qui étaient faites.\par
L’édit de Clotaire redressa tous les griefs. Personne ne put plus être condamné sans être entendu\footnote{Art. 22.} {\itshape ;} les parents durent toujours succéder selon l’ordre établi par la loi\footnote{{\itshape Ibid.}, art. 6.} {\itshape ;} toutes préceptions pour épouser des filles, des veuves ou des religieuses, furent nulles, et on punit sévèrement ceux qui les obtinrent et en firent usage\footnote{{\itshape Ibid.}, art. 18.}. Nous saurions peut-être plus exactement ce qu’il statuait sur ces préceptions, si l’article 13 de ce décret et les deux suivants n’avaient péri par le temps. Nous n’avons que les premiers mots de cet article 13, qui ordonne que les préceptions seront observées ; ce qui ne peut pas s’entendre de celles qu’il venait d’abolir par la même loi. Nous avons une autre constitution du même prince\footnote{Dans l’édition des {\itshape Capitulaires} de Baluze, t. I, p. 7.}, qui se rapporte à son édit, et corrige de même, de point en point, tous les abus des préceptions.\par
Il est vrai que M. Baluze, trouvant cette constitution sans date, et sans le nom du lieu où elle a été donnée, l’a attribuée à Clotaire I\textsuperscript{er}. Elle est de Clotaire II. J’en donnerai trois raisons :\par
1° Il y est dit que le roi conservera les immunités accordées aux églises par son père et son aïeul\footnote{J’ai parlé au livre précédent de ces immunités, qui étaient des concessions de droits de justice, et qui contenaient des défenses aux juges royaux de faire aucune fonction dans le territoire, et étaient équivalentes à l’érection ou concession d’un fief.}. Quelles immunités aurait pu accorder aux églises Childéric, aïeul de Clotaire I\textsuperscript{er}, lui qui n’était pas chrétien, et qui vivait avant que la monarchie eût été fondée ? Mais si l’on attribue ce décret à Clotaire II, on lui trouvera pour aïeul Clotaire I\textsuperscript{er} lui-même, qui fit des dons immenses aux églises pour expier la mort de son fils Cramne, qu’il avait fait brûler avec sa femme et ses enfants.\par
2{\itshape °} Les abus que cette constitution corrige subsistèrent après la mort de Clotaire I\textsuperscript{er}, et furent même portés à leur comble pendant la faiblesse du règne de Gontran, la cruauté de celui de Chilpéric, et les détestables régences de Frédégonde et de Brunehault. Or, comment la nation aurait-elle pu souffrir des griefs si solennellement proscrits, sans s’être jamais récriée sur le retour continuel de ces griefs ? Comment n’aurait-elle pas fait pour lors ce qu’elle fit lorsque Chilpéric II\footnote{Il commença à régner vers l’an 670.} ayant repris les anciennes violences, elle le pressa d’ordonner que, dans les jugements, on suivit la loi et les coutumes, comme on faisait anciennement\footnote{Voyez la {\itshape Vie de} saint {\itshape Léger.}} ?\par
3{\itshape °} Enfin, cette constitution, faite pour redresser les griefs, ne peut point concerner Clotaire I\textsuperscript{er}, puisqu’il n’y avait point sous son règne de plaintes dans le royaume à cet égard, et que son autorité y était très affermie, surtout dans le temps où {\itshape l’on} place cette constitution ; au lieu qu’elle convient très bien aux événements qui arrivèrent sous le règne de Clotaire II, qui causèrent une révolution dans l’état politique du royaume. Il faut éclairer l’histoire par les lois, et les lois par l’histoire.
\subsubsection[{Chapitre III. Autorité des maires du palais}]{Chapitre III. Autorité des maires du palais}
\noindent J’ai dit que Clotaire II s’était engagé à ne point ôter à Warnachaire la place de maire pendant sa vie. La révolution eut un autre effet. Avant ce temps, le maire était le maire du roi : il devint le maire du royaume ; le roi le choisissait, la nation le choisit. Protaire, avant la révolution, avait été fait maire par Théodéric\footnote{{\itshape Instigante Brunichilde, Theoderico jubente}, etc. Frédégaire, chap, XXVII, sur l’an 605.}, et Landéric par Frédégonde\footnote{{\itshape Gesta regum Francorum}, chap. XXXVI.} ; mais depuis, la nation fut en possession d’élire\footnote{Voyez Frédégaire, {\itshape Chronique}, chap. LIV, sur l’an 626 ; et son continuateur anonyme, chap. CI, sur l’an 695 ; et chap. XV, sur l’an 715. Aimoin, liv. IV, chap. XV. Eginhard, {\itshape Vie de Charlemagne}, chap. XLVIII. {\itshape Gesta regum Francorum}, chap. XLV.}.\par
Ainsi il ne faut pas confondre, comme ont fait quelques auteurs, ces maires du palais avec ceux qui avaient cette dignité avant la mort de Brunehault, les maires du roi avec les maires du royaume. On voit, par la loi des Bourguignons, que chez eux la charge de maire n’était point une des premières de l’État\footnote{Voyez la loi des Bourguignons, {\itshape in praefat}., et le second supplément de cette loi, tit. XIII.} ; elle ne fut pas non plus une des plus éminentes chez les premiers rois francs\footnote{Voyez Grégoire de Tours, liv. IX, chap. XXXVI.}.\par
Clotaire rassura ceux qui possédaient des charges et des fiefs ; et, après la mort de Warnachaire, ce prince ayant demandé aux seigneurs assemblés à Troyes qui ils voulaient mettre en sa place, ils s’écrièrent tous qu’ils n’éliraient point ; et, lui demandant sa faveur, ils se mirent entre ses mains\footnote{{\itshape Eo anno, Clotarius cum proceribus et leudibus Burgundiae Trecassinis conjungitur, cum eorum esset sollicitus, si vellent jam, Warnachario discesso, alium in ejus honoris gradum sublimare ; sed omnes unanimiter denegantes se nequaquam velle majorem-domus eligere, regis gratiam obnixe petentes, cum rege transegere.Chronique} de Frédégaire, chap. LIV, sur l’an 626.}.\par
Dagobert réunit, comme son père, toute la monarchie : la nation se reposa sur lui, et ne lui donna point de maire. Ce prince se sentit en liberté ; et, rassuré d’ailleurs par ses victoires, il reprit le plan de Brunehault. Mais cela lui réussit si mal, que les leudes d’Austrasie se laissèrent battre par les Sclavons\footnote{{\itshape Istam victoriam quam Vinidi contra Francos meruerunt, non tantum Sclavinorum fortitudo obtinuit, quantum dementatio Austrasiorum, dum se cernebant cum Dagoberto odium incurrisse, et assidue expoliarentur.Chronique} de Frégédaire, chap. LXVIII, sur l’an 630.}, s’en retournèrent chez eux, et les marches de l’Austrasie furent en proie aux Barbares.\par
Il prit le parti d’offrir aux Austrasiens de céder l’Austrasie à son fils Sigebert, avec un trésor, et de mettre le gouvernement du royaume et du palais entre les mains de Cunibert, évêque de Cologne, et du duc Adalgise. Frédégaire n’entre point dans le détail des conventions qui furent faites pour lors ; mais le roi les confirma toutes par ses chartes, et d’abord l’Austrasie fut mise hors de danger\footnote{{\itshape Deinceps Austrasii eorum studio limitem et regnum Francorum contra Vinidos utiliter défensasse noscuntur.Chronique} de Frédégaire, chap. LXXV, sur l’an 632.}.\par
Dagobert, se sentant mourir, recommanda à Aega sa femme Nentechilde et son fils Clovis. Les leudes de Neustrie et de Bourgogne choisirent ce jeune prince pour leur roi\footnote{{\itshape Chronique} de Frédégaire, chap. LXXIX, sur l’an 638.}. Aega et Nentechilde gouvernèrent le palais\footnote{{\itshape Ibid.}} ; ils rendirent tous les biens que Dagobert avait pris\footnote{{\itshape Ibid.}, chap. LXXX, sur l’an 639.}, et les plaintes cessèrent en Neustrie et en Bourgogne, comme elles avaient cessé en Austrasie.\par
Après la mort d’Aega, la reine Nentechilde engagea les seigneurs de Bourgogne à élire Floachatus pour leur maire\footnote{{\itshape Ibid.}, chap. LXXXIX, sur l’an 641.}. Celui-ci envoya aux évêques et aux principaux seigneurs du royaume de Bourgogne des lettres, par lesquelles il leur promettait de leur conserver pour toujours, c’est-à-dire pendant leur vie, leurs honneurs et leurs dignités\footnote{{\itshape Ibid.Floachatus cunctis ducibus a regno Burgundiae, seu et pontificibus, per epistolam etiam et sacramentis firmavit unicuique gradum honoris et dignitatem, seu et amicitiam, perpetuo conservare.}}. Il confirma sa parole par un serment. C’est ici que l’auteur du {\itshape Livre des maires de la maison royale} met le commencement de l’administration du royaume par des maires du palais\footnote{{\itshape Deinceps a temporibus Clodovei, qui fuit filius Dagoberti inclyti regis, pater vero Theoderici, regnum Francorum decidens per majores-domus coepit ordinari.De majoribus domus regiae}.}.\par
Frédégaire, qui était Bourguignon, est entré dans de plus grands détails sur ce qui regarde les maires de Bourgogne dans le temps de la révolution dont nous parlons, que sur les maires d’Austrasie et de Neustrie ; mais les conventions qui furent faites en Bourgogne furent, par les mêmes raisons, faites en Neustrie et en Austrasie.\par
La nation crut qu’il était plus sûr de mettre la puissance entre les mains d’un maire qu’elle élisait, et à qui elle pouvait imposer des conditions, qu’entre celles d’un roi dont le pouvoir était héréditaire.
\subsubsection[{Chapitre IV. Quel était, à l’égard des maires, le génie de la nation}]{Chapitre IV. Quel était, à l’égard des maires, le génie de la nation}
\noindent Un gouvernement dans lequel une nation qui avait un roi élisait celui qui devait exercer la puissance royale, paraît bien extraordinaire ; mais, indépendamment des circonstances où l’on se trouvait, je crois que les Francs tiraient à cet égard leurs idées de bien loin.\par
Ils étaient descendus des Germains, dont Tacite dit que, dans le choix de leur roi, ils se déterminaient par sa noblesse ; et dans le choix de leur chef, par sa vertu\footnote{{\itshape Reges ex nobilitate, duces ex virtute sumunt.De morib. Germ.}}. Voilà les rois de la première race, et les maires du palais ; les premiers étaient héréditaires, et les seconds étaient électifs.\par
On ne peut douter que ces princes, qui, dans l’assemblée de la nation, se levaient, et se proposaient pour chefs de quelque entreprise à tous ceux qui voudraient les suivre, ne réunissent pour la plupart, dans leur personne, et l’autorité du roi et la puissance du maire. Leur noblesse leur avait donné la royauté ; et leur vertu, les faisant suivre par plusieurs volontaires qui les prenaient pour chefs, leur donnait la puissance du maire. C’est par la dignité royale que nos premiers rois furent à la tête des tribunaux et des assemblées, et donnèrent des lois du consentement de ces assemblées : c’est par la dignité de duc ou de chef qu’ils firent leurs expéditions, et commandèrent leurs armées.\par
Pour connaître le génie des premiers Francs {\itshape à} cet égard, il n’y a qu’à jeter les yeux sur la conduite que tint Arbogaste, Franc de nation, à qui Valentinien avait donné le commandement de l’armée\footnote{Voyez Sulpicius Alexander, dans Grégoire de Tours, liv. II.}. Il enferma l’empereur dans le palais ; il ne permit à qui que ce fût de lui parler d’aucune affaire civile ou militaire. Arbogaste fit pour lors ce que les Pépins firent depuis.
\subsubsection[{Chapitre V. Comment les maires obtinrent le commandement des armées}]{Chapitre V. Comment les maires obtinrent le commandement des armées}
\noindent Pendant que les rois commandèrent les armées, la nation ne pensa point à se choisir un chef. Clovis et ses quatre fils furent à la tête des Français, et les menèrent de victoire en victoire. Thibault, fils de Théodebert, prince jeune, faible et malade, fut le premier des rois qui resta dans son palais\footnote{L’an 552.}. Il refusa de faire une expédition en Italie contre Narsès, et il eut le chagrin de voir les Francs se choisir deux chefs qui les y menèrent\footnote{{\itshape Leutheris vero et Butilinus, tametsi id regi ipsorum minime placebat, belli cum eis societatem inierunt.} Agathias, liv. I. Grégoire de Tours, liv. IV, chap. IX.}. Des quatre enfants de Clotaire I\textsuperscript{er}, Gontran fut celui qui négligea le plus de commander les armées\footnote{Gontran ne fit pas même l’expédition contre Gondovalde, qui se disait fils de Clotaire, et demandait sa part du royaume.} ; d’autres rois suivirent cet exemple : et pour remettre sans péril le commandement en d’autres mains, ils le donnèrent à plusieurs chefs ou ducs\footnote{Quelquefois au nombre de vingt. Voyez Grégoire de Tours, liv. V, chap. XXVII ; liv. VIII, chap. XVIII et XXX, liv. X, chap. III. Dagobert, qui n’avait point de maire en Bourgogne, eut la même politique, et envoya contre les Gascons dix ducs, et plusieurs comtes qui n’avaient point de ducs sur eux. {\itshape Chronique} de Frédégaire, chap. LXXVIII, sur l’an 636.}\par
On en vit naître des inconvénients sans nombre : il n’y eut plus de discipline, on ne sut plus obéir ; les armées ne furent plus funestes qu’à leur propre pays ; elles étaient chargées de dépouilles avant d’arriver chez les ennemis. On trouve dans Grégoire de Tours une vive peinture de tous ces maux\footnote{Grégoire de Tours, liv. VIII, chap. XXX ; et liv. X, chap. III.}. « Comment pourrons-nous obtenir la victoire, disait Gontran, nous qui ne conservons pas ce que nos pères ont acquis ? Notre nation n’est plus la même\footnote{{\itshape Ibid.}}… » Chose singulière ! elle était dans la décadence dès le temps des petits-fils de Clovis.\par
Il était donc naturel qu’on en vînt à faire un duc unique ; un duc qui eût de l’autorité sur cette multitude infinie de seigneurs et de leudes qui ne connaissaient plus leurs engagements ; un duc qui rétablît la discipline militaire, et qui menât contre l’ennemi une nation qui ne savait plus faire la guerre qu’à elle-même. On donna la puissance aux maires du palais.\par
La première fonction des maires du palais fut le gouvernement économique des maisons royales. Ils eurent, concurremment avec d’autres officiers, le gouvernement politique des fiefs\footnote{Voyez le second supplément à la loi des Bourguignons, titre XIII, et Grégoire de Tours, liv. IX, chap. XXXVI.} {\itshape ;} et, à la fin, ils en disposèrent seuls. Ils eurent aussi l’administration des affaires de la guerre et le commandement des armées ; et ces deux fonctions se trouvèrent nécessairement liées avec les deux autres. Dans ces temps-là, il était plus difficile d’assembler les armées que de les commander : et quel autre que celui qui disposait des grâces, pouvait avoir cette autorité ? Dans cette nation indépendante et guerrière, il fallait plutôt inviter que contraindre ; il fallait donner ou faire espérer les fiefs qui vaquaient par la mort du possesseur, récompenser sans cesse, faire craindre les préférences : celui qui avait la surintendance du palais devait donc être le général de l’armée.
\subsubsection[{Chapitre VI. Seconde époque de l’abaissement des rois de la première race}]{Chapitre VI. Seconde époque de l’abaissement des rois de la première race}
\noindent Depuis le supplice de Brunehault, les maires avaient été administrateurs du royaume sous les rois ; et, quoiqu’ils eussent la conduite de la guerre, les rois étaient pourtant à la tête des armées, et le maire et la nation combattaient sous eux. Mais la victoire du duc Pépin sur Théodoric et son maire\footnote{Voyez les {\itshape Annales de Metz} sur les années 687 et 688.} acheva de dégrader les rois\footnote{{\itshape Mis quidem nomina regum imponens, ipse totius regni habens privilegium}, etc. {\itshape Ibid.}, sur l’an 695.} {\itshape ;} celle que remporta Charles Martel sur Chilpéric et son maire Rainfroy\footnote{{\itshape Ibid.}, sur l’an 719.}, confirma cette dégradation. L’Austrasie triompha deux fois de la Neustrie et de la Bourgogne ; et la mairerie d’Austrasie étant comme attachée à la famille des Pépins, cette mairerie s’éleva sur toutes les autres maireries, et cette maison sur toutes les autres maisons. Les vainqueurs craignirent que quelque homme accrédité ne se saisît de la personne des rois pour exciter des troubles. Ils les tinrent dans une maison royale, comme dans une espèce de prison\footnote{{\itshape Sedemque illi regalem sub sua ditione concessit. Annales} de Metz sur l’an 719.}. Une fois chaque année ils étaient montrés au peuple. Là ils faisaient des ordonnances\footnote{{\itshape Ex Chronico Centulensi}, liv. II. {\itshape Ut responsa quae erat edoctus, vel potius jussus, ex sua velut potestate redderet.}}, mais c’étaient celles du maire ; ils répondaient aux ambassadeurs, mais c’étaient les réponses du maire. C’est dans ce temps que les historiens nous parlent du gouvernement des maires sur les rois qui leur étaient assujettis\footnote{{\itshape Annales} de Metz, sur l’an 691. {\itshape Anno principatus Pippini super Theodericum… Annales} de Fulde ou de Laurishan : {\itshape Pippinus dux Francorum obtinuit regnum Francorum per annos 27, cum regibus sibi subjectis.}}.\par
Le délire de la nation pour la famille de Pépin alla si loin, qu’elle élut pour maire un de ses petits-fils qui était encore dans l’enfance\footnote{{\itshape Posthaec Theudoaldus, filius ejus (Grimoaldi) parvulus, in loco ipsius, cum praedicto rege Dagoberto, major-domus palatii effectus est.} Le continuateur anonyme de Frédégaire, sur l’an 714, chap. CIV.} ; elle l’établit sur un certain Dagobert, et mit un fantôme sur un fantôme.
\subsubsection[{Chapitre VII. Des grands offices et des fiefs sous les maires du palais}]{Chapitre VII. Des grands offices et des fiefs sous les maires du palais}
\noindent Les maires du palais n’eurent garde de rétablir l’amovibilité des charges et des offices ; ils ne régnaient que par la protection qu’ils accordaient à cet égard à la noblesse : ainsi les grands offices continuèrent à être donnés pour la vie, et cet usage se confirma de plus en plus.\par
Mais j’ai des réflexions particulières à faire sur les fiefs. Je ne puis douter que, dès ce temps-là, la plupart n’eussent été rendus héréditaires.\par
Dans le traité d’Andely\footnote{Rapporté par Grégoire de Tours, liv. IX. Voyez aussi l’édit de Clotaire II, de l’an 615, art. 16.}, Gontran et son neveu Childebert s’obligent de maintenir les libéralités faites aux leudes et aux églises par les rois leurs prédécesseurs ; et il est permis aux reines, aux filles, aux veuves des rois, de disposer, par testament, et pour toujours, des choses qu’elles tiennent du fisc\footnote{{\itshape Ut si quid de agris fiscalibus vel speciebus atque praesidio, pro arbitrii sui voluntate, facere, aut cuiquam conferre voluerint, fixa stabilitate perpetuo conservetur.}}.\par
Marculfe écrivait ses {\itshape Formules} du temps des maires\footnote{Voyez la 24\textsuperscript{e} et la 341\textsuperscript{e} du liv. I.}. On en voit plusieurs où les rois donnent et à la personne et aux héritiers\footnote{Voyez la formule 14 du liv. I, qui s’applique également à des biens fiscaux donnés directement pour toujours, ou donnés d’abord en bénéfice, et ensuite pour toujours : S{\itshape icut ab illo, aut a fisco nostro fuit possessa.} Voyez aussi la formule 17, {\itshape ibid.}} : et, comme les formules sont les images des actions ordinaires de la vie, elles prouvent que, sur la fin de la première race, une partie des fiefs passait déjà aux héritiers. Il s’en fallait bien que l’on eût, dans ce temps-là, l’idée d’un domaine inaliénable ; c’est une chose très moderne, et qu’on ne connaissait alors ni dans la théorie, ni dans la pratique.\par
On verra bientôt sur cela des preuves de fait : et, si je montre un temps où il ne se trouva plus de bénéfices pour l’année, ni aucun fonds pour son entretien, il faudra bien convenir que les anciens bénéfices avaient été aliénés. Ce temps est celui de Charles Martel, qui fonda de nouveaux fiefs, qu’il faut bien distinguer des premiers.\par
Lorsque les rois commencèrent à donner pour toujours, soit par la corruption qui se glissa dans le gouvernement, soit par la constitution même qui faisait que les rois étaient obligés de récompenser sans cesse, il était naturel qu’ils commençassent plutôt à donner à perpétuité les fiefs que les comtés. Se priver de quelques terres était peu de chose ; renoncer aux grands offices, c’était perdre la puissance même.
\subsubsection[{Chapitre VIII. Comment les Alleux furent changés en fiefs}]{Chapitre VIII. Comment les Alleux furent changés en fiefs}
\noindent La manière de changer un alleu en fief se trouve dans une formule de Marculfe\footnote{Liv. I, form. 13.}. On donnait sa terre au roi ; il la rendait au donateur en usufruit ou bénéfice, et celui-ci désignait au roi ses héritiers.\par
Pour découvrir les raisons que l’on eut de dénaturer ainsi son alleu, il faut que je cherche, comme dans des abîmes, les anciennes prérogatives de cette noblesse qui, depuis onze siècles, est couverte de poussière, de sang et de sueur.\par
Ceux qui tenaient des fiefs avaient de très grands avantages. La composition pour les torts qu’on leur faisait, était plus forte que celle des hommes libres. il paraît par les {\itshape Formules} de Marculfe, que c’était un privilège du vassal du roi, que celui qui le tuerait paierait six cents sous de composition. Ce privilège était établi par la loi salique\footnote{Tit. XLIV. Voyez aussi les titres LXVI, §§ 3 et 4 ; et le titre LXXIV.} et par celle des Ripuaires\footnote{Tit. XI.} ; et pendant que ces deux lois ordonnaient six cents sous pour la mort du vassal du roi, elles n’en donnaient que deux cents pour la mort d’un ingénu, Franc, Barbare, ou homme vivant sous la loi salique ; et que cent pour celle d’un Romain\footnote{Voyez la loi des Ripuaires, tit. VII ; et la loi salique, tit. XLIV, art. 1 et 4.}.\par
Ce n’était pas le seul privilège qu’eussent les vassaux du roi. Il faut savoir que quand un homme était cité en jugement, et qu’il ne se présentait point, ou n’obéissait pas aux ordonnances des juges, il était appelé devant le roi\footnote{Loi salique, tit. LIX et LXXVI.} ; et s’il persistait dans sa contumace, il était mis hors de la protection du roi, et personne ne pouvait le recevoir chez soi, ni même lui donner du pain\footnote{{\itshape Extra sermonem regis}. Loi salique, tit. LIX et LXXVI.} : or, s’il était d’une condition ordinaire, ses biens étaient confisqués\footnote{{\itshape Ibid.}, tit. LIX, § 1.} ; mais s’il était vassal du roi, ils ne l’étaient pas\footnote{{\itshape Ibid.}, tit. LXXVI, § 1.}. Le premier, par sa contumace, était censé convaincu du crime, et non pas le second. Celui-là, dans les moindres crimes, était soumis à la preuve par l’eau bouillante\footnote{{\itshape Ibid.}, tit. LVI et LIX.} ; celui-ci n’y était condamné que dans le cas du meurtre\footnote{{\itshape Ibid.}, tit. LXXVI, § 1.}. Enfin un vassal du roi ne pouvait être contraint de jurer en justice contre un autre vassal\footnote{{\itshape Ibid.}, tit. LXXVI, § 2.}. Ces privilèges augmentèrent toujours ; et le capitulaire de Carloman fait cet honneur aux vassaux du roi, qu’on ne peut les obliger de jurer eux-mêmes, mais seulement par la bouche de leurs propres vassaux\footnote{{\itshape Apud Vernis palatium}, de l’an 883, art. 4 et 11.}. De plus, lorsque celui qui avait les honneurs ne s’était pas rendu à l’armée, sa peine était de s’abstenir de chair et de vin, autant de temps qu’il avait manqué au service ; mais l’homme libre qui n’avait pas suivi le comte\footnote{{\itshape Capitulaire} de Charlemagne, qui est le second de l’an 812, art. 1 et 3.}, payait une composition de soixante sous\footnote{{\itshape Heribannum}.}, et était mis en servitude jusqu’à ce qu’il l’eût payée.\par
Il est donc aisé de penser que les Francs qui n’étaient point vassaux du roi, et encore plus les Romains, cherchèrent à le devenir ; et qu’afin qu’ils ne fussent pas privés de leurs domaines, on imagina l’usage de donner son alleu au roi, de le recevoir de lui en fief, et de lui désigner ses héritiers. Cet usage continua toujours ; et il eut surtout lieu dans les désordres de la seconde race, où tout le monde avait besoin d’un protecteur, et voulait faire corps avec d’autres seigneurs\footnote{{\itshape Non infirmis reliquit haeredibus}, dit Lambert d’Ardres, dans du Cange, au mot {\itshape alodis.}}, et entrer, pour ainsi dire, dans la monarchie féodale, parce qu’on n’avait plus la monarchie politique.\par
Ceci continua dans la troisième race, comme on le voit par plusieurs chartes\footnote{Voyez celles que du Cange cite au mot {\itshape alodis ;} et celles que rapporte Galland, {\itshape Traité du franc-alleu}, p. 14 et suiv.} ; soit qu’on donnât son alleu, et qu’on le reprit par le même acte ; soit qu’on le déclarât alleu, et qu’on le reconnût en fief. on appelait ces fiefs, {\itshape fiefs de reprise}.\par
Cela ne signifie pas que ceux qui avaient des fiefs les gouvernassent en bons pères de famille ; et, quoique les hommes libres cherchassent beaucoup à avoir des fiefs, ils traitaient ce genre de biens comme on administre aujourd’hui les usufruits. C’est ce qui fit faire à Charlemagne, prince le plus vigilant et le plus attentif que nous ayons eu, bien des règlements\footnote{Capitulaire II de l’an 802, art. 10 ; et le Capitulaire VII de l’an 803, art. 3 ; et le Capitulaire I,{\itshape  incerti anni}, art. 49 ; et le Capitulaire de l’an 806, art. 7.} pour empêcher qu’on ne dégradât les fiefs en faveur de ses propriétés. Cela prouve seulement que de son temps la plupart des bénéfices étaient encore à vie, et que par conséquent on prenait plus de soin des alleux que des bénéfices ; mais cela n’empêche pas que l’on n’aimât encore mieux être vassal du roi qu’homme libre. On pouvait avoir des raisons pour disposer d’une certaine portion particulière d’un fief ; mais on ne voulait pas perdre sa dignité même.\par
Je sais bien encore que Charlemagne se plaint, dans un capitulaire\footnote{Le cinquième de l’an 806, art. 8.}, que, dans quelques lieux, il y avait des gens qui donnaient leurs fiefs en propriété, et les rachetaient ensuite en propriété. Mais je ne dis point qu’on n’aimât mieux une propriété qu’un usufruit : je dis seulement que, lorsqu’on pouvait faire d’un alleu un fief qui passât aux héritiers, ce qui est le cas de la formule dont j’ai parlé, on avait de grands avantages à le faire.
\subsubsection[{Chapitre IX. Comment les biens ecclésiastiques furent convertis en fiefs}]{Chapitre IX. Comment les biens ecclésiastiques furent convertis en fiefs}
\noindent Les biens fiscaux n’auraient dû avoir d’autre destination que de servir aux dons que les rois pouvaient faire pour inviter les Francs à de nouvelles entreprises, lesquelles augmentaient d’un autre côté les biens fiscaux ; et cela était, comme j’ai dit, l’esprit de la nation ; mais les dons prirent un autre cours. Nous avons un discours de Chilpéric\footnote{Dans Grégoire de Tours, liv. VI, chap. XLVI.}, petit-fils de Clovis, qui se plaignait déjà que ses biens avaient été presque tous donnés aux églises. « Notre fisc est devenu pauvre, disait-il ; nos richesses ont été transportées aux églises\footnote{Cela fit qu’il annula les testaments faits en faveur des églises, et même les dons faits par son père : Gontran les rétablit, et fit même de nouveaux dons. Grégoire de Tours, liv. VII, chap. VII.}. Il n’y a plus que les évêques qui règnent ; ils sont dans la grandeur, et nous n’y sommes plus. »\par
Cela fit que les maires, qui n’osaient attaquer les seigneurs, dépouillèrent les églises : et une des raisons qu’allégua Pépin pour entrer en Neustrie\footnote{Voyez les {\itshape Annales} de Metz sur l’an 687 : {\itshape Excitor imprimis querelis sacerdotum et servorum dei, qui me saepius adierunt ut pro sublatis injuste patrimoniis}, etc.}, fut qu’il y avait été invité par les ecclésiastiques, pour arrêter les entreprises des rois, c’est-à-dire des maires, qui privaient l’Église de tous ses biens.\par
Les maires d’Austrasie, c’est-à-dire la maison des Pépins, avaient traité l’Église avec plus de modération qu’on n’avait fait en Neustrie et en Bourgogne ; et cela est bien clair par nos chroniques\footnote{{\itshape Ibid.}}, où les moines ne peuvent se lasser d’admirer la dévotion et la libéralité des Pépins. Ils avaient occupé eux-mêmes les premières places de l’Église. « Un corbeau ne crève pas les yeux à un corbeau », comme disait Chilpéric aux évêques\footnote{Dans Grégoire de Tours.}.\par
Pépin soumit la Neustrie et la Bourgogne ; mais ayant pris, pour détruire les maires et les rois, le prétexte de l’oppression des églises, il ne pouvait plus les dépouiller sans contredire son titre, et faire voir qu’il se jouait de la nation. Mais la conquête de deux grands royaumes, et la destruction du parti opposé, lui fournirent assez de moyens de contenter ses capitaines.\par
Pépin se rendit maître de la monarchie en protégeant le clergé : Charles Martel, son fils, ne put se maintenir qu’en l’opprimant. Ce prince, voyant qu’une partie des biens royaux et des biens fiscaux avait été donnée à vie ou en propriété à la noblesse, et que le clergé, recevant des mains des riches et des pauvres, avait acquis une grande partie des allodiaux mêmes, il dépouilla les églises : et les fiefs du premier partage ne subsistant plus, il forma une seconde fois des fiefs\footnote{{\itshape Karolus plurima juri ecclesiastico detrahens, praedia fisco sociavit, ac deinde militibus dispertivit.Ex Chronico Centulensi}, liv. II.}. Il prit, pour lui et pour ses capitaines, les biens des églises et les églises mêmes ; et fit cesser un abus qui, à la différence des maux ordinaires, était d’autant plus facile à guérir, qu’il était extrême.
\subsubsection[{Chapitre X. Richesses du clergé}]{Chapitre X. Richesses du clergé}
\noindent Le clergé recevait tant, qu’il faut que, dans les trois races, on lui ait donné plusieurs fois tous les biens du royaume. Mais si les rois, la noblesse et le peuple trouvèrent le moyen de leur donner tous leurs biens, ils ne trouvèrent pas moins celui de les leur ôter. La piété fit fonder les églises dans la première race ; mais l’esprit militaire les fit donner aux gens de guerre, qui les partagèrent à leurs enfants. Combien ne sortit-il pas de terres de la mense du clergé ! Les rois de la seconde race ouvrirent leurs mains, et firent encore d’immenses libéralités ; les Normands arrivent, pillent et ravagent, persécutent surtout les prêtres et les moines, cherchent les abbayes, regardent où ils trouveront quelque lieu religieux : car ils attribuaient aux ecclésiastiques la destruction de leurs idoles, et toutes les violences de Charlemagne, qui les avait obligés les uns après les autres de se réfugier dans le Nord. C’étaient des haines que quarante ou cinquante années n’avaient pu leur faire oublier. Dans cet état des choses, combien le clergé perdit-il de biens ! À peine y avait-il des ecclésiastiques pour les redemander. Il resta donc encore à la piété de la troisième race assez de fondations à faire et de terres à donner : les opinions répandues et crues dans ces temps-là auraient privé les laïques de tout leur bien, s’ils avaient été assez honnêtes gens. Mais si les ecclésiastiques avaient de l’ambition, les laïques en avaient aussi : si le mourant donnait, le successeur voulait reprendre. On ne voit que querelles entre les seigneurs et les évêques, les gentilshommes et les abbés ; et il fallait qu’on pressât vivement les ecclésiastiques, puisqu’ils furent obligés de se mettre sous la protection de certains seigneurs, qui les défendaient pour un moment, et les opprimaient après.\par
Déjà une meilleure police, qui s’établissait dans le cours de la troisième race, permettait aux ecclésiastiques d’augmenter leur bien. Les calvinistes parurent, et firent battre de la monnaie de tout ce qui se trouva d’or et d’argent dans les églises. Comment le clergé aurait-il été assuré de sa fortune ? il ne l’était pas de son existence. Il traitait des matières de controverse, et l’on brûlait ses archives. Que servit-il de redemander à une noblesse toujours ruinée ce qu’elle n’avait plus, ou ce qu’elle avait hypothéqué de mille manières ? Le clergé a toujours acquis, il a toujours rendu, et il acquiert encore.
\subsubsection[{Chapitre XI. État de l’Europe du temps de Charles Martel}]{Chapitre XI. État de l’Europe du temps de Charles Martel}
\noindent Charles Martel, qui entreprit de dépouiller le clergé, se trouva dans les circonstances les plus heureuses : il était craint et aimé des gens de guerre, et il travaillait pour eux ; il avait le prétexte de ses guerres contre les Sarrasins\footnote{Voyez les {\itshape Annales} de Metz.} ; quelque haï qu’il fût du clergé, il n’en avait aucun besoin ; le pape, à qui il était nécessaire, lui tendait les bras : on sait la célèbre ambassade\footnote{{\itshape Epistolam quoque, decreto Romanorum principum, sibi praedictus praesul Gregorius miserat, quod sese populus Romanus, relicta imperatoris dominatione, ad suam defensionem et invictam clementiam convertere voluisset : Annales} de Metz, sur l’an 741… {\itshape Eo pacto patrato, ut a partibus imperatoris recederet} : Frédégaire.} que lui envoya Grégoire III. Ces deux puissances furent fort unies, parce qu’elles ne pouvaient se passer l’une de l’autre : le pape avait besoin des Francs pour le soutenir contre les Lombards et contre les Grecs ; Charles Martel avait besoin du pape pour humilier les Grecs, embarrasser les Lombards, se rendre plus respectable chez lui, et accréditer les titres qu’il avait, et ceux que lui ou ses enfants pourraient prendre\footnote{On peut voir, dans les auteurs de ce temps-là, l’impression que l’autorité de tant de papes fit sur l’esprit des Français. Quoique le roi Pépin eût déjà été couronné par l’archevêque de Mayence, il regarda l’onction qu’il reçut du pape Étienne comme une chose qui le confirmait dans tous ses droits.}. Il ne pouvait donc manquer son entreprise.\par
Saint Eucher, évêque d’Orléans, eut une vision qui étonna les princes. Il faut que je rapporte à ce sujet la lettre\footnote{Anno 858, {\itshape apud Catisiacum}, édition de Baluze, t. II, p. 101.} que les évêques assemblés à Reims écrivirent à Louis le Germanique, qui était entré dans les terres de Charles le Chauve, parce qu’elle est très propre à nous faire voir quel était, dans ces temps-là, l’état des choses, et la situation des esprits. Ils disent\footnote{{\itshape Ibid.}, t. II, art. 7, p. 109.} que « saint Eucher ayant été ravi dans le ciel, il vit Charles Martel tourmenté dans l’enfer inférieur, par l’ordre des saints qui doivent assister avec Jésus-Christ au jugement dernier ; qu’il avait été condamné à cette peine avant le temps, pour avoir dépouillé les églises de leurs biens, et s’être par là rendu coupable des péchés de tous ceux qui les avaient dotées ; que le roi Pépin fit tenir à ce sujet un concile ; qu’il fit rendre aux églises tout ce qu’il put retirer des biens ecclésiastiques ; que, comme il n’en put ravoir qu’une partie à cause de ses démêlés avec Vaifre, duc d’Aquitaine, il fit faire, en faveur des églises, des lettres précaires du reste\footnote{{\itshape Precaria, quod precibus utendum conceditur}, dit Cujas, dans ses notes sur le Livre I des fiefs. Je trouve dans un diplôme du roi Pépin, daté de la troisième année de son règne, que ce prince n’établit pas le premier ces lettres précaires ; il en cite une faite par le maire Ébroin et continuée depuis. Voyez le diplôme de ce roi, dans le tome V des {\itshape Historiens de France} des bénédictins, art. 6.} ; et régla que les laïques paieraient une dîme des biens qu’ils tenaient des églises, et douze deniers pour chaque maison ; que Charlemagne ne donna point les biens de l’Église ; qu’il fit au contraire un capitulaire par lequel il s’engagea, pour lui et ses successeurs, de ne les donner jamais ; que tout ce qu’ils avancent est écrit, et que même plusieurs d’entre eux l’avaient entendu raconter à Louis le Débonnaire, père des deux rois ».\par
Le règlement du roi Pépin dont parlent les évêques fut fait dans le concile tenu à Leptines\footnote{L’an 743. Voyez le livre V des {\itshape Capitulaires}, art. 3, édition de Baluze, p. 825.}. L’Église y trouvait cet avantage, que ceux qui avaient reçu de ses biens ne les tenaient plus que d’une manière précaire ; et que d’ailleurs elle en recevait la dîme, et douze deniers pour chaque case qui lui avait appartenu. Mais c’était un remède palliatif, et le mal restait toujours.\par
Cela même trouva de la contradiction, et Pépin fut obligé de faire un autre capitulaire\footnote{Celui de Metz, de l’an 756, art. 4.}, où il enjoignit à ceux qui tenaient de ces bénéfices de payer cette dîme et cette redevance, et même d’entretenir les maisons de l’évêché ou du monastère, sous peine de perdre les biens donnés. Charlemagne renouvela les règlements de Pépin\footnote{Voyez son capitulaire de l’an 803, donné à Worms, édition de Baluze, p. 411, où il règle le contrat précaire ; et celui de Francfort, de l’an 794, p. 267, art. 24, sur les réparations des maisons ; et celui de l’an 800, p. 330.}.\par
Ce que les évêques disent dans la même lettre, que Charlemagne promit, pour lui et ses successeurs, de ne plus partager les biens des églises aux gens de guerre, est conforme au capitulaire de ce prince, donné à Aix-la-Chapelle l’an 803, fait pour calmer les terreurs des ecclésiastiques à cet égard ; mais les donations déjà faites subsistèrent toujours\footnote{Comme il paraît par la note précédente et par le capitulaire de Pépin, roi d’Italie, où il est dit que le roi donnerait en fief les monastères à ceux qui se recommanderaient pour des fiefs. Il est ajouté à la loi des Lombards, liv. III, tit. I, § 30 et aux lois saliques, recueil des lois de Pépin, dans Échard, p. 195, tit. XXVI, art. 4.}. Les évêques ajoutent, et avec raison, que Louis le Débonnaire suivit la conduite de Charlemagne, et ne donna point les biens de l’Église aux soldats.\par
Cependant les anciens abus allèrent si loin, que, sous les enfants de Louis le Débonnaire, les laïques établissaient des prêtres dans leurs églises, ou les chassaient, sans le consentement des évêques\footnote{Voyez la constitution de Lothaire I\textsuperscript{er}, dans la loi des Lombards, liv. III, loi I, § 43.}. Les églises se partageaient entre les héritiers\footnote{{\itshape Ibid.}, § 44.} ; et quand elles étaient tenues d’une manière indécente, les évêques n’avaient d’autre ressource que d’en retirer les reliques\footnote{{\itshape Ibid.}}.\par
Le capitulaire de Compiègne\footnote{Donné la vingt-huitième année du règne de Charles le Chauve, l’an 868, édition de Baluze, p. 203.} établit que l’envoyé du roi pourrait faire la visite de tous les monastères avec l’évêque, de l’avis et en présence de celui qui le tenait\footnote{{\itshape Cum concilio et consensu ipsius qui locum retinet.}} {\itshape ;} et cette règle générale prouve que l’abus était général.\par
Ce n’est pas qu’on manquât de lois pour la restitution des biens des églises. Le pape ayant reproché aux évêques leur négligence sur le rétablissement des monastères, ils écrivirent\footnote{{\itshape Concilium apud Bonoilum}, seizième année de Charles le Chauve, l’an 856, édition de Baluze, p. 78.} à Charles le Chauve qu’ils n’avaient point été touchés de ce reproche, parce qu’ils n’en étaient pas coupables, et ils l’avertirent de ce qui avait été promis, résolu et statué dans tant d’assemblées de la nation. Effectivement ils en citent neuf.\par
On disputait toujours. Les Normands arrivèrent, et mirent tout le monde d’accord.
\subsubsection[{Chapitre XII. Établissement des dîmes}]{Chapitre XII. Établissement des dîmes}
\noindent Les règlements faits sous le roi Pépin avaient plutôt donné à l’Église l’espérance d’un soulagement qu’un soulagement effectif ; et, comme Charles Martel trouva tout le patrimoine public entre les mains des ecclésiastiques, Charlemagne trouva les biens des ecclésiastiques entre les mains des gens de guerre. On ne pouvait faire restituer à ceux-ci ce qu’on leur avait donné ; et les circonstances où l’on était pour lors rendaient la chose encore plus impraticable qu’elle n’était de sa nature. D’un autre côté, le christianisme ne devait pas périr, faute de ministres, de temples et d’instructions\footnote{Dans les guerres civiles qui s’élevèrent du temps de Charles Martel, les biens de l’église de Reims furent donnés aux laïques. On laissa le clergé subsister comme il pourrait, est-il dit dans la vie de saint Remi. Surius, t. I, p. 279.}.\par
Cela fit que Charlemagne établit les dîmes\footnote{Loi des Lombards, liv. III, tit. III, §§ 1 et 2.}, nouveau genre de bien, qui eut cet avantage pour le clergé, qu’étant singulièrement donné à l’Église, il fut plus aisé dans la suite d’en reconnaître les usurpations.\par
On a voulu donner à cet établissement des dates bien plus reculées : mais les autorités que l’on cite me semblent être des témoins contre ceux qui les allèguent. La constitution\footnote{C’est celle dont j’ai tant parlé au chapitre IV ci-dessus, que l’on trouve dans l’édition des {\itshape Capitulaires} de Baluze, t. I, art. II, p. 9.} de Clotaire dit seulement qu’on ne lèverait point de certaines dîmes sur les biens de l’Église\footnote{{\itshape Agraria et pascuaria, vel decimas porcorum, ecclesiae concedimus ; ita ut actor aut decimator in reluis ecclesiae nullus accedat.} Le capitulaire de Charlemagne, de l’an 800, édition de Baluze, p. 336, explique très bien ce que c’était que cette sorte de dîme dont Clotaire exempte l’Église : c’était le dixième des cochons que l’on mettait dans les forêts du roi pour engraisser : et Charlemagne veut que ses juges le paient comme les autres, afin de donner l’exemple. On voit que c’était un droit seigneurial ou économique.}. Bien loin donc que l’Église levât les dîmes dans ces temps-là, toute sa prétention était de s’en faire exempter. Le second concile de Mâcon\footnote{Canone V, ex tomo I {\itshape Conciliorum antiquorum Galliae}, opera Jacobi Sirmundi.}, tenu l’an 585, qui ordonne que l’on paie les dîmes, dit, à la vérité, qu’on les avait payées dans les temps anciens ; mais il dit aussi que, de son temps, on ne les payait plus.\par
Qui doute qu’avant Charlemagne on n’eût ouvert la Bible, et prêché les dons et les offrandes du {\itshape Lévitique} ? Mais je dis qu’avant ce prince les dîmes pouvaient être prêchées, mais qu’elles n’étaient point établies.\par
J’ai dit que les règlements faits sous le roi Pépin avaient soumis au paiement des dîmes et aux réparations des églises, ceux qui possédaient en fief les biens ecclésiastiques. C’était beaucoup d’obliger par une loi dont on ne pouvait disputer la justice, les principaux de la nation à donner l’exemple.\par
Charlemagne fit plus : et on voit, par le capitulaire {\itshape de Villis}\footnote{Art. 6, édition de Baluze, p. 332. Il fut donné l’an 800.}, qu’il obligea ses propres fonds au paiement des dîmes : c’était encore un grand exemple.\par
Mais le bas peuple n’est guère capable d’abandonner ses intérêts par des exemples. Le synode de Francfort\footnote{Tenu sous Charlemagne, l’an 794.} lui présenta un motif plus pressant pour payer les dîmes. On y fit un capitulaire dans lequel il est dit que, dans la dernière famine, on avait trouvé les épis de blé vides ; qu’ils avaient été dévorés par les démons, et qu’on avait entendu leurs voix qui reprochaient de n’avoir pas payé la dîme\footnote{{\itshape Experimento enim didicimus in anno quo illa valida faines irrepsit, ebullire vacuas annonas a daemonibus devoratas, et voces exprobrationis auditas}, etc., édition de Baluze, p. 267, art. 23.} : et, en conséquence, il fut ordonné à tous ceux qui tenaient les biens ecclésiastiques, de payer la dîme ; et, en conséquence encore, on l’ordonna à tous.\par
Le projet de Charlemagne ne réussit pas d’abord : cette charge parut accablante\footnote{Voyez entre autres le capitulaire de Louis le Débonnaire, de l’an 829, édition de Baluze, p. 663, contre ceux qui, dans la vue de ne pas payer la dîme, ne cultivaient point leurs terres ; et art. 5 : {\itshape Nonis quidem et decimis, unde et genitor} noster {\itshape et} nos {\itshape frequenter in diversis placitis admonitionem} fecimus.}. Le paiement des dîmes chez les Juifs était entré dans le plan de la fondation de leur république ; mais ici le paiement des dîmes était une charge indépendante de celles de l’établissement de la monarchie. On peut voir, dans les dispositions ajoutées à la loi des Lombards\footnote{Entre autres, celle de Lothaire, liv. III, tit. III, chap. VI.}, la difficulté qu’il y eut à faire recevoir les dîmes par les lois civiles : on peut juger, par les différents canons des conciles, de celle qu’il y eut à les faire recevoir par les lois ecclésiastiques.\par
Le peuple consentit enfin à payer les dîmes, à condition qu’il pourrait les racheter. La constitution de Louis le Débonnaire\footnote{De l’an 829, art. 7, dans Baluze, t. I, p. 663.}, et celle de l’empereur Lothaire\footnote{Loi des Lombards, liv. III, tit. III, § 8.} son fils, ne le permirent pas.\par
Les lois de Charlemagne sur l’établissement des dîmes étaient l’ouvrage de la nécessité ; la religion seule y eut part, et la superstition n’en eut aucune.\par
La fameuse division\footnote{Loi des Lombards, liv. III, tit. III, § 4.} qu’il fit des dîmes en quatre parties, pour la fabrique des églises, pour les pauvres, pour l’évêque, pour les clercs, prouve bien qu’il voulait donner à l’Église cet état fixe et permanent qu’elle avait perdu.\par
Son testament\footnote{C’est une espèce de codicille rapporté par Éginhard, et qui est différent du testament même qu’on trouve dans Goldaste et Baluze.} fait voir qu’il voulut achever de réparer les maux que Charles Martel, son aïeul, avait faits. Il fit trois parties égales de ses biens mobiliers : il voulut que deux de ces parties fussent divisées en vingt-une, pour les vingt-une métropoles de son empire ; chaque partie devait être subdivisée entre la métropole et les évêchés qui en dépendaient. Il partagea le tiers qui restait en quatre parties ; il en donna une à ses enfants et ses petits-enfants, une autre fut ajoutée aux deux tiers déjà donnés, les deux autres furent employées en œuvres pies. Il semblait qu’il regardât le don immense qu’il venait de faire aux églises, moins comme une action religieuse, que comme une dispensation politique.
\subsubsection[{Chapitre XIII. Des élections aux évêchés et abbayes}]{Chapitre XIII. Des élections aux évêchés et abbayes}
\noindent Les églises étant devenues pauvres, les rois abandonnèrent les élections aux évêchés et autres bénéfices ecclésiastiques\footnote{Voyez le capitulaire de Charlemagne de l’an 803, art. 2, édition de Baluze, p. 379 ; et l’édit de Louis le Débonnaire, de l’an 834, dans Goldaste, {\itshape Constitutions impériales}, t. I.}. Les princes s’embarrassèrent moins d’en nommer les ministres, et les compétiteurs réclamèrent moins leur autorité. Ainsi, l’Église recevait une espèce de compensation pour les biens qu’on lui avait ôtés.\par
Et si Louis le Débonnaire laissa au peuple romain le droit d’élire les papes\footnote{Cela est dit dans le fameux canon {\itshape Ego Ludovicus}, qui est visiblement supposé. Il est dans l’édition de Baluze, p. 591, sur l’an 817.}, ce fut un effet de l’esprit général de son temps : on se gouverna à l’égard du siège de Rome comme on faisait à l’égard des autres.
\subsubsection[{Chapitre XIV. Des fiefs de Charles Martel}]{Chapitre XIV. Des fiefs de Charles Martel}
\noindent Je ne dirai point si Charles Martel donnant les biens de l’Église en fief, il les donna à vie, ou à perpétuité. Tout ce que je sais, c’est que, du temps de Charlemagne\footnote{Comme il paraît par son capitulaire de l’an 801, art. 17, dans Baluze, t. I, p. 360.} et de Lothaire I\textsuperscript{er}\footnote{Voyez sa constitution insérée dans le code des Lombards, liv. III, tit. I, § 44.}, il y avait de ces sortes de biens qui passaient aux héritiers et se partageaient entre eux.\par
Je trouve de plus qu’une partie fut donnée en alleu, et l’autre partie en fief\footnote{Voyez la constitution ci-dessus et le capitulaire de Charles le Chauve, de l’an 846, chap. XX, {\itshape in villa Sparnaco}, édition de Baluze, t. II, p. 31 ; et celui de l’an 85 3, chap. III et V, dans le synode de Soissons, édition de Baluze, t. II, p. 54 ; et celui de l’an 854, {\itshape apud Attiniacum}, chap. X, édition de Baluze t. II, p. 70. Voyez aussi le capitulaire premier de Charlemagne, {\itshape incerti anni}, art. 49 et 56, édition de Baluze, t. I, p. 519.}.\par
J’ai dit que les propriétaires des alleux étaient soumis au service comme les possesseurs des fiefs. Cela fut sans doute en partie cause que Charles Martel donna en alleu aussi bien qu’en fief.
\subsubsection[{Chapitre XV. Continuation du même sujet}]{Chapitre XV. Continuation du même sujet}
\noindent Il faut remarquer que les fiefs ayant été changés en biens d’Église, et les biens d’Église ayant été changés en fiefs, les fiefs et les biens d’Église prirent réciproquement quelque chose de la nature de l’un et de l’autre. Ainsi les biens d’Église eurent les privilèges des fiefs, et les fiefs eurent les privilèges des biens d’Église : tels furent les droits honorifiques dans les églises, qu’on vit naître dans ces temps-là\footnote{Voyez les {\itshape Capitulaires}, liv. V, art. 44 ; et l’édit de Pistes de l’an 866, art. 8 et 9, où l’on voit les droits honorifiques des seigneurs établis tels qu’ils sont aujourd’hui.}. Et, comme ces droits ont toujours été attachés à la haute justice, préférablement à ce que nous appelons aujourd’hui le fief, il suit que les justices patrimoniales étaient établies dans le temps même de ces droits.
\subsubsection[{Chapitre XVI. Confusion de la royauté et de la mairerie. Seconde race}]{Chapitre XVI. Confusion de la royauté et de la mairerie. Seconde race}
\noindent L’ordre des matières a fait que j’ai troublé l’ordre des temps ; de sorte que j’ai parlé de Charlemagne avant d’avoir parlé de cette époque fameuse de la translation de la couronne aux Carlovingiens faite sous le roi Pépin : chose qui, à la différence des événements ordinaires, est peut-être plus remarquée aujourd’hui qu’elle ne le fut dans le temps même qu’elle arriva.\par
Les rois n’avaient point d’autorité, mais ils avaient un nom ; le titre de roi était héréditaire, et celui de maire était électif. Quoique les maires, dans les derniers temps, eussent mis sur le trône celui des Mérovingiens qu’ils voulaient, ils n’avaient point pris de roi dans une autre famille ; et l’ancienne loi qui donnait la couronne à une certaine famille, n’était point effacée du cœur des Francs. La personne du roi était presque inconnue dans la monarchie ; mais la royauté ne l’était pas. Pépin, fils de Charles Martel, crut qu’il était à propos de confondre ces deux titres ; confusion qui laisserait toujours de l’incertitude si la royauté nouvelle était héréditaire, ou non : et cela suffisait à celui qui joignait à la royauté une grande puissance. Pour lors, l’autorité du maire fut jointe à l’autorité royale. Dans le mélange de ces deux autorités, il se fit une espèce de conciliation. Le maire avait été électif, et le roi héréditaire : la couronne, au commencement de la seconde race, fut élective, parce que le peuple choisit ; elle fut héréditaire, parce qu’il choisit toujours dans la même famille\footnote{Voyez le testament de Charlemagne ; et le partage que Louis le Débonnaire fit à ses enfants dans l’assemblée des États tenue à Quierzy, rapportée par Goldaste : {\itshape Quem populus eligere velit, ut patri suo succedat in regni haereditate}.}.\par
Le père Le Cointe, malgré la foi de tous les monuments\footnote{L’anonyme, sur l’an 752 ; et {\itshape Chron. Centul.} sur l’an 754.}, nie que le pape ait autorisé ce grand changement\footnote{{\itshape Fabella quae post Pippini mortem excogitata est, aequitati ac sanctitati Zachariae papae plurimum adversatur…Annales ecclésiastiques des Français}, t. II, p. 319.} : une de ses raisons est qu’il aurait fait une injustice. Et il est admirable de voir un historien juger de ce que les hommes ont fait, par ce qu’ils auraient dû faire ! Avec cette manière de raisonner, il n’y aurait plus d’histoire.\par
Quoi qu’il en soit, il est certain que, dès le moment de la victoire du duc Pépin, sa famille fut régnante, et que celle des Mérovingiens ne le fut plus. Quand son petit-fils Pépin fut couronné roi ce ne fut qu’une cérémonie de plus, et un fantôme de moins : il n’acquit rien par là que les ornements royaux ; il n’y eut rien de changé dans la nation.\par
J’ai dit ceci pour fixer le moment de la révolution, afin qu’on ne se trompe pas, en regardant comme une révolution ce qui n’était qu’une conséquence de la révolution.\par
Quand Hugues Capet fut couronné roi au commencement de la troisième race, il y eut un plus grand changement, parce que l’État passa de l’anarchie à un gouvernement quelconque ; mais, quand Pépin prit la couronne, on passa d’un gouvernement au même gouvernement.\par
Quand Pépin fut couronné roi, il ne fit que changer de nom ; mais, quand Hugues Capet fut couronné roi, la chose changea, parce qu’un grand fief, uni à la couronne, fit cesser l’anarchie.\par
Quand Pépin fut couronné roi, le titre de roi fut uni au plus grand office ; quand Hugues Capet fut couronné roi, le titre de roi fut uni au plus grand fief.
\subsubsection[{Chapitre XVII. Chose particulière dans l’élection des rois de la seconde race}]{Chapitre XVII. Chose particulière dans l’élection des rois de la seconde race}
\noindent On voit, dans la formule de la consécration de Pépin\footnote{Tome V des {\itshape Historiens de France}, par les PP. bénédictins, p. 9.}, que Charles et Carloman furent aussi oints et bénis ; et que les seigneurs français s’obligèrent, sous peine d’interdiction et d’excommunication, de n’élire jamais personne d’une autre race\footnote{{\itshape Ut nunquam de alterius lumbis regem in aevo praesumant eligere, sed ex ipsorum. Ibid.}, p. 10.}.\par
Il paraît, par les testaments de Charlemagne et de Louis le Débonnaire, que les Francs choisissaient entre les enfants des rois ; ce qui se rapporte très bien à la clause ci-dessus. Et, lorsque l’empire passa dans une autre maison que celle de Charlemagne, la faculté d’élire, qui était restreinte et conditionnelle, devint pure et simple ; et on s’éloigna de l’ancienne constitution.\par
Pépin, se sentant près de sa fin, convoqua les seigneurs ecclésiastiques et laïques à Saint-Denis\footnote{L’an 768.} ; et partagea son royaume à ses deux fils Charles et Carloman. Nous n’avons point les actes de cette assemblée ; mais on trouve ce qui s’y passa dans l’auteur de l’ancienne collection historique mise au jour par Canisius\footnote{T. II, {\itshape Lectionis antiquae}.}, et celui des {\itshape Annales} de Metz, comme l’a remarqué M. Baluze\footnote{Édition des {\itshape Capitulaires}, t. I, p. 188.}. Et j’y vois deux choses en quelque façon contraires : qu’il fit le partage du consentement des grands ; et ensuite, qu’il le fit par un droit paternel. Cela prouve ce que j’ai dit, que le droit du peuple, dans cette race, était d’élire dans la famille : c’était, à proprement parler, plutôt un droit d’exclure qu’un droit d’élire.\par
Cette espèce de droit d’élection se trouve confirmée par les monuments de la seconde race. Tel est ce capitulaire de la division de l’empire que Charlemagne fait entre ses trois enfants, où, après avoir formé leur partage, il dit\footnote{Dans le capitulaire premier de l’an 806, édition de Baluze, p. 439, art. 5.} que, « si un des trois frères a un fils, tel que le peuple veuille l’élire pour qu’il succède au royaume de son père, ses oncles y consentiront ».\par
Cette même disposition se trouve dans le partage que Louis le Débonnaire fit entre ses trois enfants\footnote{Dans Goldaste, {\itshape Constitutions impériales}, t. II, p. 19.}, Pépin, Louis et Charles, l’an 837, dans l’assemblée d’Aix-la-Chapelle ; et encore dans un autre partage du même empereur, fait vingt ans auparavant, entre Lothaire, Pépin et Louis\footnote{Édition de Baluze, p. 574, art. 14. {\itshape Si vero aliquis illorum decedens, legitimos filios reliquerit, non inter eos potestas ipsa dividatur ; sed potius populus, pariter conveniens, unum ex eis, quem dominus voluerit, eligat ; et hunc senior frater in loco fratris et filii suscipiat.}}. On peut voir encore le serment que Louis le Bègue fit à Compiègne, lorsqu’il y fut couronné. « Moi, Louis\footnote{Capitulaire de l’an 877, édition de Baluze, p. 272.}, constitué roi par la miséricorde de Dieu et l’élection du peuple, je promets… » Ce que je dis est confirmé par les actes du concile de Valence\footnote{Dans Dumont, {\itshape Corps diplomatique}, t. I, art. 36.}, tenu l’an 890, pour l’élection de Louis, fils de Boson, au royaume d’Arles. On y élit Louis ; et on donne pour principales raisons de son élection, qu’il était de la famille impériale\footnote{Par femmes.}, que Charles le Gras lui avait donné la dignité de roi, et que l’empereur Arnoul l’avait investi par le sceptre et par le ministère de ses ambassadeurs. Le royaume d’Arles, comme les autres, démembrés ou dépendant de l’empire de Charlemagne, était électif et héréditaire.
\subsubsection[{Chapitre XVIII. Charlemagne}]{Chapitre XVIII. {\itshape Charlemagne}}
\noindent Charlemagne songea à tenir le pouvoir de la noblesse dans ses limites, et à empêcher l’oppression du clergé et des hommes libres. Il mit un tel tempérament dans les ordres de l’État, qu’ils furent contrebalancés, et qu’il resta le maître. Tout fut uni par la force de son génie. Il mena continuellement la noblesse d’expédition en expédition ; il ne lui laissa pas le temps de former des desseins, et l’occupa tout entière à suivre les siens. L’empire se maintint par la grandeur du chef : le prince était grand, l’homme l’était davantage. Les rois ses enfants furent ses premiers sujets, les instruments de son pouvoir, et les modèles de l’obéissance. il fit d’admirables règlements ; il fit plus, il les fit exécuter. Son génie se répandit sur toutes les parties de l’empire. On voit, dans les lois de ce prince, un esprit de prévoyance qui comprend tout, et une certaine force qui entraîne tout. Les prétextes pour éluder les devoirs sont ôtés ; les négligences corrigées, les abus réformés ou prévenus\footnote{Voyez son capitulaire III de l’an 811, p. 486, art. 1, 2, 3, 4, 5, 6, 7 et 8 ; et le capitulaire premier de l’an 812, p. 490, art. 1 ; et le capitulaire de la même année, p. 494, art. 9 et 11 ; et autres.}. Il savait punir ; il savait encore mieux pardonner. Vaste dans ses desseins, simple dans l’exécution, personne n’eut à un plus haut degré l’art de faire les plus grandes choses avec facilité, et les difficiles avec promptitude. Il parcourait sans cesse son vaste empire, portant la main partout où il allait tomber. Les affaires renaissaient de toutes parts, il les finissait de toutes parts. Jamais prince ne sut mieux braver les dangers ; jamais prince ne les sut mieux éviter. Il se joua de tous les périls, et particulièrement de ceux qu’éprouvent presque toujours les grands conquérants : je veux dire les conspirations. Ce prince prodigieux était extrêmement modéré ; son caractère était doux, ses manières simples ; il aimait à vivre avec les gens de sa cour. Il fut peut-être trop sensible au plaisir des femmes ; mais un prince qui gouverna toujours par lui-même, et qui passa sa vie dans les travaux, peut mériter plus d’excuses. Il mit une règle admirable dans sa dépense : il fit valoir ses domaines avec sagesse, avec attention, avec économie ; un père de famille pourrait apprendre dans ses lois à gouverner sa maison\footnote{Voyez le capitulaire {\itshape de Villis}, de l’an 800 ; son capitulaire II de l’an 813, art. 6 et 19 ; et le liv. V des {\itshape Capitulaires}, art. 303.}. On voit dans ses {\itshape Capitulaires} la source pure et sacrée d’où il tira ses richesses. Je ne dirai plus qu’un mot : il ordonnait qu’on vendît les œufs des basses-cours de ses domaines, et les herbes inutiles de ses jardins\footnote{Capitulaire {\itshape de Villis}, art. 39. Voyez tout ce capitulaire qui est un chef-d’œuvre de prudence, de bonne administration et d’économie.} ; et il avait distribué à ses peuples toutes les richesses des Lombards, et les immenses trésors de ces Huns qui avaient dépouillé l’univers.
\subsubsection[{Chapitre XIX. Continuation du même sujet}]{Chapitre XIX. Continuation du même sujet}
\noindent Charlemagne et ses premiers successeurs craignirent que ceux qu’ils placeraient dans des lieux éloignés ne fussent portés à la révolte ; ils crurent qu’ils trouveraient plus de docilité dans les ecclésiastiques : ainsi ils érigèrent en Allemagne un grand nombre d’évêchés\footnote{Voyez entre autres la fondation de l’archevêché de Brême, dans le capitulaire de 789, édition de Baluze, p. 245.}, et y joignirent de grands fiefs. Il paraît, par quelques chartes, que les clauses qui contenaient les prérogatives de ces fiefs n’étaient pas différentes de celles qu’on met tait ordinairement dans ces concessions\footnote{Par exemple, la défense aux juges royaux d’entrer dans le territoire pour exiger les {\itshape freda} et autres droits. J’en ai beaucoup parlé au livre précédent.}, quoiqu’on voie aujourd’hui les principaux ecclésiastiques d’Allemagne revêtus de la puissance souveraine. Quoi qu’il en soit, c’étaient des pièces qu’ils mettaient en avant contre les Saxons. Ce qu’ils ne pouvaient attendre de l’indolence ou des négligences d’un leude, ils crurent qu’ils devaient l’attendre du zèle et de l’attention agissante d’un évêque : outre qu’un tel vassal, bien loin de se servir contre eux des peuples assujettis, aurait au contraire besoin d’eux pour se soutenir contre ses peuples.
\subsubsection[{Chapitre XX. Louis le débonnaire}]{Chapitre XX. Louis le débonnaire}
\noindent Auguste, étant en Égypte, fit ouvrir le tombeau d’Alexandre. On lui demanda s’il voulait qu’on ouvrit ceux des Ptolomées ; il dit qu’il avait voulu voir le roi, et non pas les morts. Ainsi, dans l’histoire de cette seconde race, on cherche Pépin et Charlemagne ; on voudrait voir les rois, et non pas les morts.\par
Un prince, jouet de ses passions, et dupe de ses vertus mêmes ; un prince qui ne connut jamais sa force ni sa faiblesse ; qui ne sut se concilier ni la crainte ni l’amour ; qui, avec peu de vices dans le cœur, avait toutes sortes de défauts dans l’esprit, prit en main les rênes de l’empire que Charlemagne avait tenues.\par
Dans le temps que l’univers est en larmes pour la mort de son père ; dans cet instant d’étonnement où tout le monde demande Charles, et ne le trouve plus ; dans le temps qu’il hâte ses pas pour aller remplir sa place, il envoie devant lui des gens affidés pour arrêter ceux qui avaient contribué au désordre de la conduite de ses sœurs. Cela causa de sanglantes tragédies\footnote{L’auteur incertain de la {\itshape Vie de Louis le Débonnaire}, dans le recueil de Duchesne, t. II, p. 295.} : c’étaient des imprudences bien précipitées. Il commença à venger les crimes domestiques, avant d’être arrivé au palais, et à révolter les esprits, avant d’être le maître.\par
Il fit crever les yeux à Bernard, roi d’Italie, son neveu, qui était venu implorer sa clémence, et qui mourut quelques jours après : cela multiplia ses ennemis. La crainte qu’il en eut le détermina à faire tondre ses frères : cela en augmenta encore le nombre. Ces deux derniers articles lui furent bien reprochés\footnote{Voyez le procès-verbal de sa dégradation, dans le recueil de Duchesne, t. II, p. 333.} : on ne manqua pas de dire qu’il avait violé son serment, et les promesses solennelles qu’il avait faites à son père le jour de son couronnement\footnote{Il lui ordonna d’avoir pour ses sœurs, ses frères et ses neveux une clémence sans bornes, {\itshape indeficientem misericordiam.} Tégan, dans le recueil de Duchesne, t. II, p. 276.}.\par
Après la mort de l’impératrice Hirmengarde, dont il avait trois enfants, il épousa Judith ; il en eu un fils ; et bientôt, mêlant les complaisances d’un vieux mari avec toutes les faiblesses d’un vieux roi, il mit un désordre dans sa famille, qui entraîna la chute de la monarchie.\par
Il changea sans cesse les partages qu’il avait faits à ses enfants. Cependant ces partages avaient été confirmés tour à tour par ses serments, ceux de ses enfants et ceux des seigneurs. C’était vouloir tenter la fidélité de ses sujets ; c’était chercher à mettre de la confusion, des scrupules et des équivoques dans l’obéissance ; c’était confondre les droits divers des princes, dans un temps surtout où les forteresses étant rares, le premier rempart de l’autorité était la foi promise et la foi reçue.\par
Les enfants de l’empereur, pour maintenir leurs partages, sollicitèrent le clergé, et lui donnèrent des droits inouïs jusqu’alors. Ces droits étaient spécieux ; on faisait entrer le clergé en garantie d’une chose qu’on avait voulu qu’il autorisât. Agobard\footnote{Voyez ses lettres.} représenta à Louis le Débonnaire qu’il avait envoyé Lothaire à Rome pour le faire déclarer empereur ; qu’il avait fait des partages à ses enfants, après avoir consulté le ciel par trois jours de jeûnes et de prières. Que pouvait faire un prince superstitieux, attaqué d’ailleurs par la superstition même ? On sent quel échec l’autorité souveraine reçut deux fois, par la prison de ce prince et sa pénitence publique. On avait voulu dégrader le roi, on dégrada la royauté.\par
On a d’abord de la peine à comprendre comment un prince, qui avait plusieurs bonnes qualités, qui ne manquait pas de lumières, qui aimait naturellement le bien, et, pour tout dire enfin, le fils de Charlemagne, put avoir des ennemis si nombreux\footnote{Voyez le procès-verbal de sa dégradation dans le recueil de Duchesne, t. II, p. 331. Voyez aussi sa {\itshape Vie} écrite par Tégan. {\itshape Tante enim odio laborabat, ut taederet eos vita ipsius}, dit l’auteur incertain, dans Duchesne, t. II, p. 307.}, si violents, si irréconciliables, si ardents à l’offenser, si insolents dans son humiliation, si déterminés à le perdre ; et ils l’auraient perdu deux fois sans retour, si ses enfants, dans le fond plus honnêtes gens qu’eux, eussent pu suivre un projet, et convenir de quelque chose.
\subsubsection[{Chapitre XXI. Continuation du même sujet}]{Chapitre XXI. Continuation du même sujet}
\noindent La force que Charlemagne avait mise dans la nation subsista assez sous Louis le Débonnaire, pour que l’État pût se maintenir dans sa grandeur, et être respecté des étrangers. Le prince avait l’esprit faible ; mais la nation était guerrière. L’autorité se perdait au-dedans, sans que la puissance parût diminuer au-dehors.\par
Charles Martel, Pépin et Charlemagne gouvernèrent l’un après l’autre la monarchie. Le premier flatta l’avarice des gens de guerre ; les deux autres celle du clergé ; Louis le Débonnaire mécontenta tous les deux.\par
Dans la constitution française, le roi, la noblesse et le clergé avaient dans leurs mains toute la puissance de l’État. Charles Martel, Pépin et Charlemagne se joignirent quelquefois d’intérêts avec l’une des deux parties pour contenir l’autre, et presque toujours avec toutes les deux : mais Louis le Débonnaire détacha de lui l’un et l’autre de ces corps. Il indisposa les évêques par des règlements qui leur parurent rigides, parce qu’il allait plus loin qu’ils ne voulaient aller eux-mêmes. Il y a de très bonnes lois faites mal à propos. Les évêques, accoutumés dans ces temps-là à aller à la guerre contre les Sarrasins et les Saxons, étaient bien éloignés de l’esprit monastique\footnote{« Pour lors les évêques et les clercs commencèrent à quitter les ceintures et les baudriers d’or, les couteaux enrichis de pierreries qui y étaient suspendus, les habillements d’un goût exquis, les éperons, dont la richesse accablait leurs talons. Mais l’ennemi du genre humain ne souffrit point une telle dévotion, qui souleva contre elle les ecclésiastiques de tous les ordres, et se fit à elle-même la guerre. » L’auteur incertain de la {\itshape Vie de Louis le Débonnaire}, dans le recueil de Duchesne, t. II, p. 298.}. D’un autre côté, ayant perdu toute sorte de confiance pour sa noblesse, il éleva des gens de néant\footnote{Tégan dit que ce qui se faisait très rarement sous Charlemagne se fit communément sous Louis.}. Il la priva de ses emplois, la renvoya du palais, appela des étrangers\footnote{Voulant contenir la noblesse, il prit pour son chambrier un certain Bénard, qui acheva de la désespérer.}. Il s’était séparé de ces deux corps, il en fut abandonné.
\subsubsection[{Chapitre XXII. Continuation du même sujet}]{Chapitre XXII. Continuation du même sujet}
\noindent Mais ce qui affaiblit surtout la monarchie, c’est que ce prince en dissipa les domaines\footnote{{\itshape Villas regias, quae erant sui et avi et tritavi, fidelibus suis tradidit eas in possessiones sempiternas : fecit enim hoc diu tempore.} Tegan, {\itshape De gestis Ludovici pii}.}. C’est ici que Nitard, un des plus judicieux historiens que nous ayons ; Nitard, petit-fils de Charlemagne, qui était attaché au parti de Louis le Débonnaire, et qui écrivait l’histoire par ordre de Charles le Chauve, doit être écouté.\par
Il dit « qu’un certain Adelhard avait eu pendant un temps un tel empire sur l’esprit de l’empereur, que ce prince suivait sa volonté en toutes choses ; qu’à l’instigation de ce favori, il avait donné les biens fiscaux à tous ceux qui en avaient voulu\footnote{{\itshape Hinc libertates, hinc publica in propriis usibus distribuere suasit.} Nitard, liv. IV, à la fin.} ; et par là avait anéanti la république\footnote{{\itshape Rempublicam penitus annulavit. Ibid.}} ». Ainsi, il fit dans tout l’empire ce que j’ai dit\footnote{Voyez le liv. XXX, chap. XIII.} qu’il avait fait en Aquitaine : chose que Charlemagne répara, et que personne ne répara plus.\par
L’État fut mis dans cet épuisement où Charles Martel le trouva lorsqu’il parvint à la mairerie ; et l’on était dans ces circonstances, qu’il n’était plus question d’un coup d’autorité pour le rétablir.\par
Le fisc se trouva si pauvre que, sous Charles le Chauve, on ne maintenait personne dans les honneurs\footnote{Hincmar, lettre I à Louis le Bègue.}, on n’accordait la sûreté à personne, que pour de l’argent : quand on pouvait détruire les Normands, on les laissait échapper pour de l’argent\footnote{Voyez le fragment de la {\itshape Chronique du monastère de Saint-Serge d’Angers}, dans Duchesne, t. II, p. 401.} ; et le premier conseil qu’Hincmar donne à Louis le Bègue, c’est de demander dans une assemblée de quoi soutenir les dépenses de sa maison.
\subsubsection[{Chapitre XXIII. Continuation du même sujet}]{Chapitre XXIII. Continuation du même sujet}
\noindent Le clergé eut sujet de se repentir de la protection qu’il avait accordée aux enfants de Louis le Débonnaire. Ce prince, comme j’ai dit, n’avait jamais donné de préceptions des biens de l’Église aux laïques\footnote{Voyez ce que disent les évêques dans le synode de l’an 845, {\itshape apud Teudonis villam}, art. 4.} ; mais bientôt Lothaire en Italie, et Pépin en Aquitaine, quittèrent le plan de Charlemagne, et reprirent celui de Charles Martel. Les ecclésiastiques eurent recours à l’empereur contre ses enfants ; mais ils avaient affaibli eux-mêmes l’autorité qu’ils réclamaient. En Aquitaine, on eut quelque condescendance ; en Italie, on n’obéit pas.\par
Les guerres civiles, qui avaient troublé la vie de Louis le Débonnaire, furent le germe de celles qui suivirent sa mort. Les trois frères, Lothaire, Louis et Charles, cherchèrent, chacun de leur côté, à attirer les grands dans leur parti, et à se faire des créatures. Ils donnèrent à ceux qui voulurent les suivre, des préceptions des biens de l’Église ; et, pour gagner la noblesse, ils lui livrèrent le clergé.\par
On voit, dans les {\itshape Capitulaires}\footnote{Voyez le synode, de l’an 845, {\itshape apud Teudonis villam}, art. 3 et 4, qui décrit très bien l’état des choses ; aussi bien que celui de la même année, tenu au palais de Vernes, art. 12 ; et le synode de Beauvais, encore de la même année, art. 3, 4 et 6 ; et le capitulaire {\itshape in villa Sparnaco}, de l’an 846, art. 20 ; et la lettre que les évêques assemblés à Reims écrivirent, l’an 858, à Louis le Germanique, art. 8.}, que ces princes furent obligés de céder à l’importunité des demandes, et qu’on leur arracha souvent ce qu’ils n’auraient pas voulu donner : on y voit que le clergé se croyait plus opprimé par la noblesse que par les rois. Il paraît encore que Charles le Chauve fut celui qui attaqua le plus le patrimoine du clergé\footnote{Voyez le capitulaire {\itshape in villa Sparnaco}, de l’an 846. La noblesse avait irrité le roi contre les évêques, de sorte qu’il les chassa de l’assemblée : on choisit quelques canons des synodes, et on leur déclara que ce seraient les seuls qu’on observerait ; on ne leur accorda que ce qu’il était impossible de leur refuser. Voyez les art. 20, 21 et 22. Voyez aussi la lettre que les évêques assemblés écrivirent l’an 858 à Louis le Germanique, art. 8 ; et l’édit de Pistes, de 864, art. 5.}, soit qu’il fût le plus irrité contre lui, parce qu’il avait dégradé son père à son occasion, soit qu’il fût le plus timide. Quoi qu’il en soit, on voit dans les {\itshape Capitulaires}\footnote{Voyez le même capitulaire de l’an 846, {\itshape in villa Sparnaco}. Voyez aussi le capitulaire de l’assemblée tenue {\itshape apud Marsnam}, de l’an 847, art. 4, dans laquelle le clergé se retrancha à demander qu’on le remît en possession de tout ce dont il avait joui sous le règne de Louis le Débonnaire. Voyez aussi le capitulaire de l’an 851, {\itshape apud Marsnam}, art. 6 et 7, qui maintient la noblesse et le clergé dans leurs possessions ; et celui {\itshape apud Bonoilum}, de l’an 856, qui est une remontrance des évêques au roi, sur ce que les maux, après tant de lois faites, n’avaient pas été réparés ; et enfin la lettre que les évêques assemblés à Reims écrivirent, l’an 858, à Louis le Germanique, art. 8.} des querelles continuelles entre le clergé qui demandait ses biens, et la noblesse qui refusait, qui éludait, ou qui différait de les rendre ; et les rois entre deux.\par
C’est un spectacle digne de pitié, de voir l’état des choses en ces temps-là. Pendant que Louis le Débonnaire faisait aux églises des dons immenses de ses domaines, ses enfants distribuaient les biens du clergé aux laïques. Souvent la même main qui fondait des abbayes nouvelles, dépouillait les anciennes. Le clergé n’avait point un état fixe. On lui ôtait ; il regagnait ; mais la couronne perdait toujours.\par
Vers la fin du règne de Charles le Chauve, et depuis ce règne, il ne fut plus guère question des démêlés du clergé et des laïques sur la restitution des biens de l’Église. Les évêques jetèrent bien encore quelques soupirs dans leurs remontrances à Charles le Chauve, que l’on trouve dans le capitulaire de l’an 856, et dans la lettre\footnote{Art. 8.} qu’ils écrivirent à Louis le Germanique l’an 858 ; mais ils proposaient des choses, et ils réclamaient des promesses tant de fois éludées, que l’on voit qu’ils n’avaient aucune espérance de les obtenir.\par
Il ne fut plus question que de réparer en général les torts faits dans l’Église et dans l’État\footnote{Voyez le capitulaire de l’an 851, art. 6 et 7.}. Les rois s’engageaient de ne point ôter aux leudes leurs hommes libres, et de ne plus donner les biens ecclésiastiques par des préceptions\footnote{Charles le Chauve, dans le synode de Soissons, dit qu’il avait promis aux évêques de ne plus donner de préceptions des biens de l’Église. Capitulaire de l’an 853, art. II, édition de Baluze, t. II, p. 56.} {\itshape ;} de sorte que le clergé et la noblesse parurent s’unir d’intérêts.\par
Les étranges ravages des Normands, comme j’ai dit, contribuèrent beaucoup à mettre fin à ces querelles.\par
Les rois, tous les jours moins accrédités, et par les causes que j’ai dites, et par celles que je dirai, crurent n’avoir d’autre parti à prendre que de se mettre entre les mains des ecclésiastiques. Mais le clergé avait affaibli les rois, et les rois avaient affaibli le clergé.\par
En vain Charles le Chauve et ses successeurs appelèrent-ils le clergé pour soutenir l’État, et en empêcher la chute\footnote{Voyez dans Nitard, liv. IV, comment, après la fuite de Lothaire, les rois Louis et Charles consultèrent les évêques pour savoir s’ils pourraient prendre et partager le royaume qu’il avait abandonné. En effet, comme les évêques formaient entre eux un corps plus uni que les leudes, il convenait à ces princes d’assurer leurs droits par une résolution des évêques, qui pourraient engager tous les autres seigneurs à les suivre.} {\itshape ;} en vain se servirent-ils du respect que les peuples avaient pour ce corps, pour maintenir celui qu’on devait avoir pour eux\footnote{Voyez le capitulaire de Charles le Chauve, {\itshape apud Saponarias}, de l’an 859, art. 3 : « Venilon que j’avais fait archevêque de Sens, m’a sacré ; et je ne devais être chassé du royaume par personne, {\itshape saltem sine audientia et judicio episcoporum, quorum ministerio in regem sum consecratus, et qui throni Dei sunt dicti, in quibus Deus sedet, et per quos sua decernit judicia ; quorum paternis correctionibus et castigatoriis judiciis me subdere fui paratus, et in praesenti sum subditus}. »} ; en vain cherchèrent-ils à donner de l’autorité à leurs lois par l’autorité des canons\footnote{Voyez le capitulaire de Charles le Chauve, {\itshape de Carisiaco}, de l’an 857, édition de Baluze, t. II, p. 88, art. 1, 2, 3, 4 et 7.} {\itshape ;} en vain joignirent-ils les peines ecclésiastiques aux peines civiles\footnote{Voyez le synode de Pistes, de l’an 862, art. 4 ; et le capitulaire de Carloman et de Louis II, {\itshape apud Vernis palatium}, de l’an 883, art. 4 et 5.} ; en vain, pour contrebalancer l’autorité du comte, donnèrent-ils à chaque évêque la qualité de leur envoyé dans les provinces\footnote{Capitulaire de l’an 876, sous Charles le Chauve, {\itshape in synodo Pontigonensi}, édition de Baluze, art. 12.} : il fut impossible au clergé de réparer le mal qu’il avait fait ; et un étrange malheur, dont je parlerai bientôt, fit tomber la couronne à terre.
\subsubsection[{Chapitre XXIV. Que les hommes libres furent rendus capables de posséder des fiefs}]{Chapitre XXIV. Que les hommes libres furent rendus capables de posséder des fiefs}
\noindent J’ai dit que les hommes libres allaient à la guerre sous leur comte, et les vassaux sous leur seigneur. Cela faisait que les ordres de l’État se balançaient les uns les autres ; et, quoique les leudes eussent des vassaux sous eux, ils pouvaient être contenus par le comte, qui était à la tête de tous les hommes libres de la monarchie.\par
D’abord\footnote{Voyez ce que j’ai dit ci-dessus au livre XXX, chapitre dernier, vers la fin.}, ces hommes libres ne purent pas se recommander pour un fief, mais ils le purent dans la suite ; et je trouve que ce changement se fit dans le temps qui s’écoula depuis le règne de Gontran jusqu’à celui de Charlemagne. Je le prouve par la comparaison qu’on peut faire du traité d’Andely\footnote{De l’an 587, dans Grégoire de Tours, liv. IX.} passé entre Gontran, Childebert et la reine Brunehault, et le partage fait par Charlemagne à ses enfants, et un partage pareil fait par Louis le Débonnaire\footnote{Voyez le chapitre suivant, où je parle plus au long de ces partages, et les notes où ils sont cités.}. Ces trois actes contiennent des dispositions à peu près pareilles à l’égard des vassaux ; et, comme on y règle les mêmes points, et à peu près dans les mêmes circonstances, l’esprit et la lettre de ces trois traités se trouvent à peu près les mêmes à cet égard.\par
Mais, pour ce qui concerne les hommes libres, il s’y trouve une différence capitale. Le traité d’Andely ne dit point qu’ils pussent se recommander pour un fief ; au lieu qu’on trouve, dans les partages de Charlemagne et de Louis le Débonnaire, des clauses expresses pour qu’ils pussent s’y recommander : ce qui fait voir que, depuis le traité d’Andely, un nouvel usage s’introduisit, par lequel les hommes libres étaient devenus capables de cette grande prérogative.\par
Cela dut arriver lorsque Charles Martel ayant distribué les biens de l’Église à ses soldats, et les ayant donnés, partie en fief, partie en alleu, il se fit une espèce de révolution dans les lois féodales. Il est vraisemblable que les nobles, qui avaient déjà des fiefs, trouvèrent plus avantageux de recevoir les nouveaux dons en alleu, et que les hommes libres se trouvèrent encore trop heureux de les recevoir en fief.
\subsubsection[{Chapitre XXV. Cause principale de l’affaiblissement de la seconde race. Changement dans les Alleux}]{Chapitre XXV. Cause principale de l’affaiblissement de la seconde race. Changement dans les Alleux}
\noindent Charlemagne, dans le partage\footnote{De l’an 806, entre Charles, Pépin et Louis. Il est rapporté par Goldaste et par Baluze, t. I, p. 439.} dont j’ai parlé au chapitre précédent, régla qu’après sa mort les hommes de chaque roi recevraient des bénéfices dans le royaume de leur roi, et non dans le royaume d’un autre\footnote{Art. 9, p. 443. Ce qui est conforme au traité d’Andely, dans Grégoire de Tours, liv. IX.} ; au lieu qu’on conserverait ses alleux dans quelque royaume que ce fût. Mais il ajoute que tout homme libre pourrait, après la mort de son seigneur, se recommander pour un fief dans les trois royaumes à qui il voudrait, de même que celui qui n’avait jamais eu de seigneur\footnote{Art. 10. Et il n’est point parlé de ceci dans le traité d’Andely.}. On trouve les mêmes dispositions dans le partage que fit Louis le Débonnaire à ses enfants l’an 817\footnote{Dans Baluze, t. I, p. 174. {\itshape Licentiam habeat unusquisque liber homo qui seniorem non habuerit, cuicumque ex his tribus fratribus voluerit, se commendandi}, art. 9. Voyez aussi le partage que fit le même empereur, l’an 837, art. 6, édition de Baluze, p. 686.}.\par
Mais, quoique les hommes libres se recommandassent pour un fief, la milice du comte n’en était point affaiblie : il fallait toujours que l’homme libre contribuât pour son alleu, et préparât des gens qui en fissent le service, à raison d’un homme pour quatre manoirs ; ou bien qu’il préparât un homme qui servit pour lui le fief ; et quelques abus s’étant introduits là-dessus, ils furent corrigés, comme il paraît par les constitutions de Charlemagne\footnote{De l’an 811, édition de Baluze, t. I, p. 486, art. 7 et 8 ; et celle de l’an 812, {\itshape ibid.}, p. 490, art. 1. {\itshape Ut omnis liber homo qui quatuor mansos vestitos de proprio suo, sive de alicujus beneficio habet, ipse se praeparet, et ipse in hostem pergat, sive cum seniore suo}, etc. Voyez le capitulaire de l’an 807, édition de Baluze, t. I, p. 458.}, et par celle de Pépin, roi d’Italie\footnote{De l’an 793, insérée dans la loi des Lombards, liv. III, tit. IX, chap. IX.}, qui s’expliquent l’une l’autre.\par
Ce que les historiens ont dit, que la bataille de Fontenay causa la ruine de la monarchie, est très vrai ; mais qu’il me soit permis de jeter un coup d’œil sur les funestes conséquences de cette journée.\par
Quelque temps après cette bataille, les trois frères, Lothaire, Louis et Charles, firent un traité\footnote{En l’an 847, rapporté par Aubert le Mire et Baluze, t. II, p. 42. {\itshape Conventus apud Marsnam}.}, dans lequel je trouve des clauses qui durent changer tout l’état politique chez les Français.\par
Dans l’annonciation\footnote{{\itshape Adnunciatio}.} que Charles fit au peuple de la partie de ce traité qui le concernait, il dit que tout homme libre pourrait choisir pour seigneur qui il voudrait, du roi ou des autres seigneurs\footnote{{\itshape Ut unusquisque liber homo in nostro regno seniorem quem voluerit, in nobis et in nostris fidelibus, accipiat.} Art. 2 de l’annonciation de Charles.}.\par
Avant ce traité, l’homme libre pouvait se recommander pour un fief, mais son alleu restait toujours sous la puissance immédiate du roi, c’est-à-dire sous la juridiction du comte ; et il ne dépendait du seigneur auquel il s’était recommandé, qu’à raison du fief qu’il en avait obtenu. Depuis ce traité, tout homme libre put soumettre son alleu au roi, ou à un autre seigneur, à son choix. Il n’est point question de ceux qui se recommandaient pour un fief, mais de ceux qui changeaient leur alleu en fief, et sortaient, pour ainsi dire, de la juridiction civile, pour entrer dans la puissance du roi ou du seigneur qu’ils voulaient choisir.\par
Ainsi ceux qui étaient autrefois nuement sous la puissance du roi, en qualité d’hommes libres sous le comte, devinrent insensiblement vassaux les uns des autres, puisque chaque homme libre pouvait choisir pour seigneur qui il voulait, ou du roi, ou des autres seigneurs ;\par
2° Qu’un homme changeant en fief une terre qu’il possédait à perpétuité, ces nouveaux fiefs ne pouvaient plus être à vie. Aussi voyons-nous, un moment après, une loi générale pour donner les fiefs aux enfants du possesseur : elle est de Charles le Chauve, un des trois princes qui contractèrent\footnote{Capitulaire de l’an 877, tit. LIII, art. 9 et 10, {\itshape apud Carisiacum. Similiter et de nostris vassallis faciendum est}, etc. Ce capitulaire se rapporte à un autre de la même année et du même lieu, art. 3.}.\par
Ce que j’ai dit de la liberté qu’eurent tous les hommes de la monarchie, depuis le traité des trois frères, de choisir pour seigneur qui ils voulaient, du roi ou des autres seigneurs, se confirme par les actes passés depuis ce temps-là.\par
Du temps de Charlemagne, lorsqu’un vassal avait reçu d’un seigneur une chose, ne valût-elle qu’un sou, il ne pouvait plus le quitter\footnote{Capitulaire d’Aix-la-Chapelle, de l’an 813, art. 16. {\itshape Quod nullus seniorem suum dimittat, postquam ab eo acceperit valente solidum unum.} Et le capitulaire de Pépin, de l’an 783, art. 5.}. Mais, sous Charles le Chauve, les vassaux purent impunément suivre leurs intérêts ou leur caprice ; et ce prince s’exprime si fortement là-dessus, qu’il semble plutôt les inviter à jouir de cette liberté, qu’à la restreindre\footnote{Voyez le capitulaire {\itshape de Carisiaco}, de l’an 856, art. 10 et 13, édition de Baluze, t. II, p. 83, dans lequel le roi et les seigneurs ecclésiastiques et laïques convinrent de ceci : {\itshape Et si aliquis de vobis sit cui suus senioratus non placet, et illi simulat ut ad alium seniorem melius quam ad illum acaptare possit, veniat ad illum, et ipse tranquille et pacifico animo donet illi commeatum… et quod deus illi cupierit et ad alium seniorem acaptare potuerit, pacifice habeat}.}. Du temps de Charlemagne, les bénéfices étaient plus personnels que réels ; dans la suite ils devinrent plus réels que personnels.
\subsubsection[{Chapitre XXVI. Changement dans les fiefs}]{Chapitre XXVI. Changement dans les fiefs}
\noindent Il n’arriva pas de moindres changements dans les fiefs que dans les alleux. On voit par le capitulaire\footnote{De l’an 757, art. 6, édition de Baluze, p. 181.} de Compiègne, fait sous le roi Pépin, que ceux à qui le roi donnait un bénéfice, donnaient eux-mêmes une partie de ce bénéfice à divers vassaux ; mais ces parties n’étaient point distinguées du tout. Le roi les ôtait lorsqu’il ôtait le tout ; et, à la mort du leude, le vassal perdait aussi son arrière-fief ; un nouveau bénéficiaire venait, qui établissait aussi de nouveaux arrière-vassaux. Ainsi l’arrière-fief ne dépendait point du fief ; c’était la personne qui dépendait. D’un côté, l’arrière-vassal revenait au roi, parce qu’il n’était pas attaché pour toujours au vassal ; et l’arrière-fief revenait de même au roi, parce qu’il était le fief même, et non pas une dépendance du fief.\par
Tel était l’arrière-vasselage, lorsque les fiefs étaient amovibles ; tel il était encore, pendant que les fiefs furent à vie. Cela changea lorsque les fiefs passèrent aux héritiers, et que les arrière-fiefs y passèrent de même. Ce qui relevait du roi immédiatement, n’en releva plus que médiatement ; et la puissance royale se trouva, pour ainsi dire, reculée d’un degré, quelquefois de deux, et souvent davantage.\par
On voit, dans les {\itshape Livres des Fiefs}\footnote{Liv. I, chap. {\itshape I.}}, que quoique les vassaux du roi pussent donner en fief, c’est-à-dire en arrière-fief du roi, cependant ces arrière-vassaux ou petits vavasseurs ne pouvaient pas de même donner en fief ; de sorte que ce qu’ils avaient donné, ils pouvaient toujours le reprendre. D’ailleurs une telle concession ne passait point aux enfants comme les fiefs, parce qu’elle n’était point censée faite selon la loi des fiefs.\par
Si l’on compare l’état où était l’arrière-vasselage du temps que les deux sénateurs de Milan écrivaient ces {\itshape Livres}, avec celui où il était du temps du roi Pépin, on trouvera que les arrière-fiefs conservèrent plus longtemps leur nature primitive que les fiefs\footnote{Au moins en Italie et en Allemagne.}.\par
Mais lorsque ces sénateurs écrivirent, on avait mis des exceptions si générales à cette règle, qu’elles l’avaient presque anéantie. Car, si celui qui avait reçu un fief du petit vavasseur l’avait suivi à Rome dans une expédition, il acquérait tous les droits de vassal ; de même, s’il avait donné de l’argent au petit vavasseur pour obtenir le fief, celui-ci ne pouvait le lui ôter, ni l’empêcher de le transmettre à son fils, jusqu’à ce qu’il lui eût rendu son argent\footnote{Liv. I {\itshape Des Fiefs}, chap. I.}. Enfin, cette règle n’était plus suivie dans le sénat de Milan\footnote{{\itshape Ibid.}}.
\subsubsection[{Chapitre XXVII. Autre changement arrivé dans les fiefs}]{Chapitre XXVII. Autre changement arrivé dans les fiefs}
\noindent Du temps de Charlemagne\footnote{Capitulaire de l’an 802, art. 7, édition de Baluze, p. 365.}, on était obligé, sous de grandes peines, de se rendre à la convocation, pour quelque guerre que ce fût ; on ne recevait point d’excuses ; et le comte qui aurait exempté quelqu’un, aurait été puni lui-même. Mais le traité des trois frères\footnote{{\itshape Apud Marsnam}, l’an 847, édition de Baluze, p. 42.} mit là-dessus une restriction qui tira, pour ainsi dire, la noblesse de la main du roi\footnote{{\itshape Volumus ut cujuscumque nostrum homo, in cujuscumque regno sit, cum seniore suo in hostem, vel aliis suis utilitatibus pergat ; nisi talis regni invasio quam Lamtuveri dicunt, quod absit, acciderit, ut omnis populus illius regni ad eam repellendam communiter pergat.} Art. 5, {\itshape ibid.}, p. 44.} : on ne fut plus tenu de suivre le roi à la guerre, que quand cette guerre était défensive. Il fut libre, dans les autres, de suivre son seigneur, ou de vaquer à ses affaires. Ce traité se rapporte à un autre, fait cinq ans auparavant\footnote{{\itshape Apud Argentoratum}, dans Baluze, {\itshape Capitulaires}, t. II, p. 39.} entre les deux frères Charles le Chauve et Louis roi de Germanie, par lequel ces deux frères dispensèrent leurs vassaux de les suivre à la guerre, en cas qu’ils fissent quelque entreprise l’un contre l’autre ; chose que les deux princes jurèrent, et qu’ils firent jurer aux deux armées.\par
La mort de cent mille Français à la bataille de Fontenay fit penser à ce qui restait encore de noblesse que, par les querelles particulières de ses rois sur leur partage, elle serait enfin exterminée\footnote{Effectivement, ce fut la noblesse qui fit ce traité. Voyez Nitard, liv. IV.} {\itshape ;} et que leur ambition et leur jalousie ferait verser tout ce qu’il y avait encore de sang à répandre. On fit cette loi, que la noblesse ne serait contrainte de suivre les princes à la guerre, que lorsqu’il s’agirait de défendre l’État contre une invasion étrangère. Elle fut en usage pendant plusieurs siècles\footnote{Voyez la loi de Guy, roi des Romains, parmi celles qui ont été ajoutées à la loi salique et à celle des Lombards, tit. VI, § 2, dans Échard.}.
\subsubsection[{Chapitre XXVIII. Changements arrivés dans les grands offices et dans les fiefs}]{Chapitre XXVIII. Changements arrivés dans les grands offices et dans les fiefs}
\noindent Il semblait que tout prît un vice particulier, et se corrompît en même temps. J’ai dit que, dans les premiers temps, plusieurs fiefs étaient aliénés à perpétuité : mais c’étaient des cas particuliers, et les fiefs en général conservaient toujours leur propre nature ; et si la couronne avait perdu les fiefs, elle en avait substitué d’autres. J’ai dit encore que la couronne n’avait jamais aliéné les grands offices à perpétuité\footnote{Des auteurs ont dit que la comté de Toulouse avait été donnée par Charles Martel, et passa d’héritier en héritier jusqu’au dernier Raymond, mais si cela est, ce fut l’effet de quelques circonstances qui purent engager à choisir les comtes de Toulouse parmi les enfants du dernier possesseur.}.\par
Mais Charles le Chauve fit un règlement général, qui affecta également et les grands offices et les fiefs : il établit, dans ses {\itshape Capitulaires}, que les comtés seraient données aux enfants du comte ; et il voulut que ce règlement eût encore lieu pour les fiefs\footnote{Voyez son capitulaire de l’an 877, tit. LIII, art. 9 et 10, {\itshape apud Carisiacum.} Ce capitulaire se rapporte à un autre de la même année et du même lieu, art. 3.}.\par
On verra tout à l’heure que ce règlement reçut une plus grande extension ; de sorte que les grands offices et les fiefs passèrent à des parents plus éloignés. Il suivit de là que la plupart des seigneurs, qui relevaient immédiatement de la couronne, n’en relevèrent plus que médiatement. Ces comtes, qui rendaient autrefois la justice dans les plaids du roi ; ces comtes, qui menaient les hommes libres à la guerre, se trouvèrent entre le roi et ses hommes libres ; et la puissance se trouva encore reculée d’un degré.\par
Il y a plus : il paraît par les capitulaires que les comtes avaient des bénéfices attachés à leur comté, et des vassaux sous eux\footnote{Le capitulaire III de l’an 812, art. 7 ; et celui de l’an 815, art. 6, sur les Espagnols ; le recueil des {\itshape Capitulaires}, liv. V, art. 288 ; et le capitulaire de l’an 869, art. 2 ; et celui de l’an 877, art. 13, édition de Baluze.}. Quand les comtés furent héréditaires, ces vassaux du comte ne furent plus les vassaux immédiats du roi ; les bénéfices attachés aux comtés ne furent plus les bénéfices du roi ; les comtes devinrent plus puissants, parce que les vassaux qu’ils avaient déjà les mirent en état de s’en procurer d’autres.\par
Pour bien sentir l’affaiblissement qui en résulta à la fin de la seconde race, il n’y a qu’à voir ce qui arriva au commencement de la troisième, où la multiplication des arrière-fiefs mit les grands vassaux au désespoir.\par
C’était une coutume du royaume que, quand les aînés avaient donné des partages à leurs cadets, ceux-ci en faisaient hommage à l’aîné\footnote{Comme il paraît par Othon de Frissingue, {\itshape Des Gestes de Frédéric}, liv. II, chap. XXIX.} ; de manière que le seigneur dominant ne les tenait plus qu’en arrière-fief. Philippe Auguste, le duc de Bourgogne, les comtes de Nevers, de Boulogne, de Saint-Paul, de Dampierre, et autres seigneurs, déclarèrent que dorénavant, soit que le fief fût divisé par succession ou autrement, le tout relèverait toujours du même seigneur, sans aucun seigneur moyen\footnote{Voyez l’ordonnance de Philippe Auguste, de l’an 1209, dans le nouveau recueil.}. Cette ordonnance ne fut pas généralement suivie, car, comme j’ai dit ailleurs, il était impossible de faire dans ces temps-là des ordonnances générales ; mais plusieurs de nos coutumes se réglèrent là-dessus.
\subsubsection[{Chapitre XXIX. De la nature des fiefs depuis le règne de Charles le chauve}]{Chapitre XXIX. De la nature des fiefs depuis le règne de Charles le chauve}
\noindent J’ai dit que Charles le Chauve voulut que, quand le possesseur d’un grand office ou d’un fief laisserait en mourant un fils, l’office ou le fief lui fût donné. Il serait difficile de suivre le progrès des abus qui en résultèrent, et de l’extension qu’on donna à cette loi dans chaque pays. Je trouve dans les {\itshape Livres des Fiefs}\footnote{Liv. I, tit. I.}, qu’au commencement du règne de l’empereur Conrad II, les fiefs, dans les pays de sa domination, ne passaient point aux petits-fils ; ils passaient seulement à celui des enfants du dernier possesseur que le seigneur avait choisi\footnote{{\itshape Sic progressum est, ut ad filios deveniret in quem dominus hoc vellet beneficium confirmare. Ibid.}} : ainsi les fiefs furent donnés par une espèce d’élection que le seigneur fit entre ses enfants.\par
J’ai expliqué, au chapitre XVII de ce livre, comment, dans la seconde race, la couronne se trouvait à certains égards élective, et à certains égards héréditaire. Elle était héréditaire, parce qu’on prenait toujours les rois dans cette race ; elle l’était encore, parce que les enfants succédaient ; elle était élective, parce que le peuple choisissait entre les enfants. Comme les choses vont toujours de proche en proche, et qu’une loi politique a toujours du rapport à une autre loi politique, on suivit pour la succession des fiefs le même esprit que l’on avait suivi pour la succession à la couronne\footnote{Au moins en Italie et en Allemagne.}. Ainsi les fiefs passèrent aux enfants, et par droit de succession et par droit d’élection ; et chaque fief se trouva, comme la couronne, électif et héréditaire.\par
Ce droit d’élection dans la personne du seigneur ne subsistait pas\footnote{{\itshape Quod hodie ita stabilitum est, ut ad omnes aequaliter veniat.} Liv. I {\itshape Des Fiefs}, tit. I.} du temps des auteurs des {\itshape Livres des Fiefs}\footnote{Gerardus Niger et Aubertus de Orto.}, c’est-à-dire sous le règne de l’empereur Frédéric {\itshape I\textsuperscript{er}}.
\subsubsection[{Chapitre XXX. Continuation du même sujet}]{Chapitre XXX. Continuation du même sujet}
\noindent Il est dit dans les {\itshape Livres des Fiefs}\footnote{Liv. I {\itshape Des Fiefs}, tit. I.} que, quand l’empereur Conrad partit pour Rome, les fidèles qui étaient à son service lui demandèrent de faire une loi pour que les fiefs, qui passaient aux enfants, passassent aussi aux petits-enfants ; et que celui dont le frère était mort sans héritiers légitimes pût succéder au fief qui avait appartenu à leur père commun : cela fut accordé.\par
On y ajoute, et il faut se souvenir que ceux qui parlent vivaient\footnote{Cujas l’a très bien prouvé.} du temps de l’empereur Frédéric I\textsuperscript{er}, « que les anciens jurisconsultes avaient toujours tenu que la succession des fiefs en ligne collatérale ne passait point au-delà des frères germains ; quoique, dans des temps modernes, on l’eût portée jusqu’au septième degré, comme, par le droit nouveau, on l’avait portée en ligne directe jusqu’à l’infini\footnote{Liv. I {\itshape Des Fiefs}, tit. I.} ». C’est ainsi que la loi de Conrad reçut peu à peu des extensions.\par
Toutes ces choses supposées, la simple lecture de l’histoire de France fera voir que la perpétuité des fiefs s’établit plus tôt en France qu’en Allemagne. Lorsque l’empereur Conrad II commença à régner en 1024, les choses se trouvèrent encore en Allemagne comme elles étaient déjà en France sous le règne de Charles le Chauve, qui mourut en 877. Mais en France, depuis le règne de Charles le Chauve, il se fit de tels changements, que Charles le Simple se trouva hors d’état de disputer à une maison étrangère ses droits incontestables à l’empire ; et qu’enfin, du temps de Hugues Capet, la maison régnante, dépouillée de tous ses domaines, ne put pas même soutenir la couronne.\par
La faiblesse d’esprit de Charles le Chauve mit en France une égale faiblesse dans l’État. Mais comme Louis le Germanique son frère, et quelques-uns de ceux qui lui succédèrent, eurent de plus grandes qualités, la force de leur État se soutint plus longtemps.\par
Que dis-je ? Peut-être que l’humeur flegmatique, et, si j’ose le dire, l’immutabilité de l’esprit de la nation allemande, résista plus longtemps que celui de la nation française à cette disposition des choses, qui faisait que les fiefs, comme par une tendance naturelle, se perpétuaient dans les familles.\par
J’ajoute que le royaume d’Allemagne ne fut pas dévasté, et, pour ainsi dire, anéanti, comme le fut celui de France, par ce genre particulier de guerre que lui firent les Normands et les Sarrasins. Il y avait moins de richesses en Allemagne, moins de villes à saccager, moins de côtes à parcourir, plus de marais à franchir, plus de forêts à pénétrer. Les princes, qui ne virent pas à chaque instant l’État prêt à tomber, eurent moins besoin de leurs vassaux, c’est-à-dire en dépendirent moins. Et il y a apparence que si les empereurs d’Allemagne n’avaient été obligés de s’aller faire couronner à Rome, et de faire des expéditions continuelles en Italie, les fiefs auraient conservé plus longtemps chez eux leur nature primitive.
\subsubsection[{Chapitre XXXI. Comment l’empire sortit de la maison de Charlemagne}]{Chapitre XXXI. Comment l’empire sortit de la maison de Charlemagne}
\noindent L’empire, qui, au préjudice de la branche de Charles le Chauve, avait déjà été donné aux bâtards de celle de Louis le Germanique\footnote{Arnoul et son fils Louis IV.}, passa encore dans une maison étrangère, par l’élection de Conrad, duc de Franconie, l’an 912. La branche qui régnait en France, et qui pouvait à peine disputer des villages, était encore moins en état de disputer l’empire. Nous avons un accord passé entre Charles le Simple et l’empereur Henri I\textsuperscript{er}, qui avait succédé à Conrad. On l’appelle le pacte de Bonn\footnote{De l’an 926, rapporté par Aubert le Mire, {\itshape Cod. donationum piarum}, chap. XXVII.}. Les deux princes se rendirent dans un navire qu’on avait placé au milieu du Rhin, et se jurèrent une amitié éternelle. On employa un {\itshape mezzo termine} assez bon. Charles prit le titre de roi de la France occidentale, et Henri celui de roi de la France orientale. Charles contracta avec le roi de Germanie, et non avec l’empereur.
\subsubsection[{Chapitre XXXII. Comment la couronne de France passa dans la maison de Hugues Capet}]{Chapitre XXXII. Comment la couronne de France passa dans la maison de Hugues Capet}
\noindent L’hérédité des fiefs et l’établissement général des arrière-fiefs éteignirent le gouvernement politique, et formèrent le gouvernement féodal. Au lieu de cette multitude innombrable de vassaux que les rois avaient eus, ils n’en eurent plus que quelques-uns, dont les autres dépendirent. Les rois n’eurent presque plus d’autorité directe : un pouvoir qui devait passer par tant d’autres pouvoirs, et par de si grands pouvoirs, s’arrêta ou se perdit avant d’arriver à son terme. De si grands vassaux n’obéirent plus ; et ils se servirent même de leurs arrière-vassaux pour ne plus obéir. Les rois, privés de leurs domaines, réduits aux villes de Reims et de Laon, restèrent à leur merci. L’arbre étendit trop loin ses branches, et la tête se sécha. Le royaume se trouva sans domaine, comme est aujourd’hui l’empire. On donna la couronne à un des plus puissants vassaux.\par
Les Normands ravageaient le royaume ; ils venaient sur des espèces de radeaux ou de petits bâtiments, entraient par l’embouchure des rivières, les remontaient, et dévastaient le pays des deux côtés. Les villes d’Orléans et de Paris arrêtaient ces brigands\footnote{Voyez le capitulaire de Charles le Chauve, de l’an 877, {\itshape apud Carisiacum}, sur l’importance de Paris, de Saint-Denis, et des châteaux sur la Loire, dans ces temps-là.} ; et ils ne pouvaient avancer ni sur la Seine ni sur la Loire. Hugues Capet, qui possédait ces deux villes, tenait dans ses mains les deux clefs des malheureux restes du royaume ; on lui déféra une couronne qu’il était seul en état de défendre. C’est ainsi que depuis on a donné l’empire à la maison qui tient immobiles les frontières des Turcs.\par
L’empire était sorti de la maison de Charlemagne dans le temps que l’hérédité des fiefs ne s’établissait que comme une condescendance. Elle fut même plus tard en usage chez les Allemands que chez les Français\footnote{Voyez ci-dessus le chap. XXX, p. 1183.} : cela fit que l’empire, considéré comme un fief, fut électif. Au contraire, quand la couronne de France sortit de la maison de Charlemagne, les fiefs étaient réellement héréditaires dans ce royaume : la couronne, comme un grand fief, le fut aussi.\par
Du reste, on a eu grand tort de rejeter sur le moment de cette révolution tous les changements qui étaient arrivés, ou qui arrivèrent depuis. Tout se réduisit à deux événements : la famille régnante changea, et la couronne fut unie à un grand fief.
\subsubsection[{Chapitre XXXIII. Quelques conséquences de la perpétuité des fiefs}]{Chapitre XXXIII. Quelques conséquences de la perpétuité des fiefs}
\noindent Il suivit de la perpétuité des fiefs que le droit d’aînesse ou de primogéniture s’établit parmi les Français. On ne le connaissait point dans la première race\footnote{Voyez la loi salique et la loi des Ripuaires, au titre {\itshape Des alleux.}} : la couronne se partageait entre les frères ; les alleux se divisaient de même ; et les fiefs, amovibles ou à vie, n’étant pas un objet de succession, ne pouvaient pas être un objet de partage.\par
Dans la seconde race, le titre d’empereur qu’avait Louis le Débonnaire, et dont il honora Lothaire son fils aîné, lui fit imaginer de donner à ce prince une espèce de primauté sur ses cadets. Les deux rois\footnote{Voyez le capitulaire de l’an 817, qui contient le premier partage que Louis le Débonnaire fit entre ses enfants.} devaient aller trouver l’empereur chaque année, lui porter des présents, et en recevoir de lui de plus grands ; ils devaient conférer avec lui sur les affaires communes. C’est ce qui donna à Lothaire ces prétentions qui lui réussirent si mal. Quand Agobard écrivit pour ce prince\footnote{Voyez ses deux lettres à ce sujet, dont l’une a pour titre {\itshape de divisione imperii.}}, il allégua la disposition de l’empereur même, qui avait associé Lothaire à l’empire, après que, par trois jours de jeûne et par la célébration des saints sacrifices, par des prières et des aumônes, Dieu avait été consulté ; que la nation lui avait prêté serment, qu’elle ne pouvait point se parjurer ; qu’il avait envoyé Lothaire à Rome, pour être confirmé par le pape. Il pèse sur tout ceci, et non pas sur le droit d’aînesse. Il dit bien que l’empereur avait désigné un partage aux cadets, et qu’il avait préféré l’aîné ; mais en disant qu’il avait préféré l’aîné, c’était dire en même temps qu’il aurait pu préférer les cadets.\par
Mais quand les fiefs furent héréditaires, le droit d’aînesse s’établit dans la succession des fiefs, et, par la même raison, dans celle de la couronne, qui était le grand fief. La loi ancienne, qui formait des partages, ne subsista plus : les fiefs étant chargés d’un service, il fallait que le possesseur fût en état de le remplir. On établit un droit de primogéniture ; et la raison de la loi féodale força celle de la loi politique ou civile.\par
Les fiefs passant aux enfants du possesseur, les seigneurs perdaient la liberté d’en disposer ; et, pour s’en dédommager, ils établirent un droit qu’on appela le droit de rachat, dont parlent nos coutumes, qui se paya d’abord en ligne directe, et qui, par usage, ne se paya plus qu’en ligne collatérale.\par
Bientôt les fiefs purent être transportés aux étrangers, comme un bien patrimonial. Cela fit naître le droit de lods et ventes, établi dans presque tout le royaume. Ces droits furent d’abord arbitraires ; mais quand la pratique d’accorder ces permissions devint générale, on les fixa dans chaque contrée.\par
Le droit de rachat devait se payer à chaque mutation d’héritier, et se paya même d’abord en ligne directe\footnote{Voyez l’ordonnance de Philippe Auguste, de l’an 1209, sur les fiefs.}. La coutume la plus générale l’avait fixé à une année du revenu. Cela était onéreux et incommode au vassal, et affectait, pour ainsi dire, le fief. Il obtint souvent, dans l’acte d’hommage, que le seigneur ne demanderait plus pour le rachat qu’une certaine somme d’argent\footnote{On trouve dans les chartes plusieurs de ces conventions, comme dans le capitulaire de Vendôme et celui de l’abbaye de Saint-Cyprien en Poitou, dont M. Galland, p. 55, a donné des extraits.}, laquelle, par les changements arrivés aux monnaies, est devenue de nulle importance : ainsi le droit de rachat se trouve aujourd’hui presque réduit à rien, tandis que celui de lods et ventes a subsisté dans toute son étendue. Ce droit-ci ne concernant ni le vassal ni ses héritiers, mais étant un cas fortuit qu’on ne devait ni prévoir ni attendre, on ne fit point ces sortes de stipulations, et on continua à payer une certaine portion du prix.\par
Lorsque les fiefs étaient à vie, on ne pouvait pas donner une partie de son fief, pour le tenir pour toujours en arrière-fief ; il eût été absurde qu’un simple usufruitier eût disposé de la propriété de la chose. Mais, lorsqu’ils devinrent perpétuels, cela fut permis\footnote{Mais on ne pouvait pas abréger le fief, c’est-à-dire en éteindre une portion.}, avec de certaines restrictions que mirent les coutumes\footnote{Elles fixèrent la portion dont on pouvait se jouer.} : ce qu’on appela {\itshape se jouer de son fief}.\par
La perpétuité des fiefs ayant fait établir le droit de rachat, les filles purent succéder à un fief, au défaut des mâles. Car le seigneur donnant le fief à la fille, il multipliait les cas de son droit de rachat, parce que le mari devait le payer comme la femme\footnote{C’est pour cela que le seigneur contraignait la veuve de se remarier.}. Cette disposition ne pouvait avoir lieu pour la couronne ; car, comme elle ne relevait de personne, il ne pouvait point y avoir de droit de rachat sur elle.\par
La fille de Guillaume V, comte de Toulouse, ne succéda pas à la comté. Dans la suite, Aliénor succéda à l’Aquitaine, et Mathilde à la Normandie ; et le droit de la succession des filles parut dans ces temps-là si bien établi, que Louis le Jeune, après la dissolution de son mariage avec Aliénor, ne fit aucune difficulté de lui rendre la Guyenne. Comme ces deux derniers exemples suivirent de très près le premier, il faut que la loi générale qui appelait les femmes à la succession des fiefs se soit introduite plus tard dans la comté de Toulouse que dans les autres provinces du royaume\footnote{La plupart des grandes maisons avaient leurs lois de succession particulières. Voyez ce que M. de La Thaumassière nous dit sur les maisons du Berri.}.\par
La constitution de divers royaumes de l’Europe a suivi l’état actuel où étaient les fiefs dans les temps que ces royaumes ont été fondés. Les femmes ne succédèrent ni à la couronne de France ni à l’empire, parce que, dans l’établissement de ces deux monarchies, les femmes ne pouvaient succéder aux fiefs ; mais elles succédèrent dans les royaumes dont l’établissement suivit celui de la perpétuité des fiefs, tels que ceux qui furent fondés par les conquêtes des Normands, ceux qui furent fondés par les conquêtes faites sur les Maures ; d’autres enfin, qui, au-delà des limites de l’Allemagne, et dans des temps assez modernes, prirent, en quelque façon, une seconde naissance par l’établissement du christianisme.\par
Quand les fiefs étaient amovibles, on les donnait à des gens qui étaient en état de les servir, et il n’était point question des mineurs. Mais, quand ils furent perpétuels, les seigneurs prirent le fief jusqu’à la majorité, soit pour augmenter leurs profits, soit pour faire élever le pupille dans l’exercice des armes\footnote{On voit dans le capitulaire de l’année 877, {\itshape apud Carisiacum}, art. 3, édition de Baluze, t. II, p. 269, le moment où les rois firent administrer les fiefs pour les conserver aux mineurs : exemple qui fut suivi par les seigneurs, et donna l’origine à ce que nous appelons la garde-noble.}. C’est ce que nos coutumes appellent la {\itshape garde-noble}, laquelle est fondée sur d’autres principes que ceux de la tutelle, et en est entièrement distincte.\par
Quand les fiefs étaient à vie, on se recommandait pour un fief ; et la tradition réelle, qui se faisait par le sceptre, constatait le fief, comme fait aujourd’hui l’hommage. Nous ne voyons pas que les comtes, ou même les envoyés du roi, reçussent les hommages dans les provinces ; et cette fonction ne se trouve pas dans les commissions de ces officiers qui nous ont été conservées dans les capitulaires. Ils faisaient bien quelquefois prêter le serment de fidélité à tous les sujets\footnote{On en trouve la formule dans le capitulaire II de l’an 802. Voyez aussi celui de l’an 854, art. 13 et autres.} ; mais ce serment était si peu un hommage de la nature de ceux qu’on établit depuis, que, dans ces derniers, le serinent de fidélité était une action jointe à l’hommage, qui tantôt suivait et tantôt précédait l’hommage, qui n’avait point lieu dans tous les hommages, qui fut moins solennelle que l’hommage, et en était entièrement distincte\footnote{M. Du Cange, au mot {\itshape Hominium}, p. 1163, et au mot {\itshape Fidelitas}, p. 474, cite les chartes des anciens hommages, où ces différences se trouvent, et grand nombre d’autorités qu’on peut voir. Dans l’hommage, le vassal mettait sa main dans celle du seigneur, et jurait : le serment de fidélité se faisait en jurant sur les évangiles. L’hommage se faisait à genoux ; le serment de fidélité debout. Il n’y avait que le seigneur qui pût recevoir l’hommage ; mais ses officiers pouvaient prendre le serment de fidélité. Voyez Litleton, sect. 91 et 92. Foi {\itshape et hommage}, c’est fidélité et hommage.}.\par
Les comtes et les envoyés du roi faisaient encore, dans les occasions, donner aux vassaux dont la fidélité était suspecte, une assurance qu’on appelait {\itshape firmitas}\footnote{Capitulaire de Charles le Chauve, de l’an 860, {\itshape post reditum a Confluentibus}, art. 3, édition de Baluze, p. 145.} ; mais cette assurance ne pouvait être un hommage, puisque les rois se la donnaient entre eux\footnote{{\itshape Ibid.}, art. 1.}.\par
Que si l’abbé Suger\footnote{{\itshape Liber de administratione sua}.} parle d’une chaire de Dagobert, où, selon le rapport de l’antiquité, les rois de France avaient coutume de recevoir les hommages des seigneurs, il est clair qu’il emploie ici les idées et le langage de son temps.\par
Lorsque les fiefs passèrent aux héritiers, la reconnaissance du vassal, qui n’était dans les premiers temps qu’une chose occasionnelle, devint une action réglée : elle fut faite d’une manière plus éclatante, elle fut remplie de plus de formalités, parce qu’elle devait porter la mémoire des devoirs réciproques du seigneur et du vassal, dans tous les âges.\par
Je pourrais croire que les hommages commencèrent à s’établir du temps du roi Pépin, qui est le temps où j’ai dit que plusieurs bénéfices furent donnés à perpétuité : mais je le croirais avec précaution, et dans la supposition seule que les auteurs des anciennes Annales des Francs n’aient pas été des ignorants, qui, décrivant les cérémonies de l’acte de fidélité que Tassillon, duc de Bavière, fit à Pépin\footnote{{\itshape Anno} 757, chap. XVII.}, aient parlé suivant les usages qu’ils voyaient pratiquer de leur temps\footnote{{\itshape Tassillo venit in vassatico se commendans, per manus sacramenta juravit multa et innumerabilia, reliquiis sanctorum manus imponens, et fidelitatem promisit Pippino.} Il semblerait qu’il y aurait là un hommage et un serment de fidélité.}.
\subsubsection[{Chapitre XXXIV. Continuation du même sujet}]{Chapitre XXXIV. Continuation du même sujet}
\noindent Quand les fiefs étaient amovibles ou à vie, ils n’appartenaient guère qu’aux lois politiques ; c’est pour cela que, dans les lois civiles de ces temps-là, il est fait si peu de mention des lois des fiefs. Mais lorsqu’ils devinrent héréditaires, qu’ils purent se donner, se vendre, se léguer, ils appartinrent et aux lois politiques et aux lois civiles. Le fief, considéré comme une obligation au service militaire, tenait au droit politique ; considéré comme un genre de bien qui était dans le commerce, il tenait au droit civil. Cela donna naissance aux lois civiles sur les fiefs.\par
Les fiefs étant devenus héréditaires, les lois concernant l’ordre des successions durent être relatives à la perpétuité des fiefs. Ainsi s’établit, malgré la disposition du droit romain et de la loi salique\footnote{Au titre {\itshape Des alleux.}}, cette règle du droit français : {\itshape Propres ne remontent point}\footnote{Liv. IV, {\itshape De feudis}, tit. LIX.}. Il fallait que le fief fût servi ; mais un aïeul, un grand-oncle auraient été de mauvais vassaux à donner au seigneur : aussi cette règle n’eut-elle d’abord lieu que pour les fiefs, comme nous l’apprenons de Boutillier\footnote{{\itshape Somme rurale}, liv. I, tir. LXXVI, p. 447.}.\par
Les fiefs étant devenus héréditaires, les seigneurs, qui devaient veiller à ce que le fief fût servi, exigèrent que les filles qui devaient succéder au fief\footnote{Suivant une ordonnance de saint Louis, de l’an 1246, pour constater les coutumes d’Anjou et du Maine, ceux qui auront le bail d’une fille héritière d’un fief, donneront assurance au seigneur qu’elle ne sera mariée que de son consentement.}, et, je crois, quelquefois les mâles, ne pussent se marier sans leur consentement ; de sorte que les contrats de mariage devinrent pour les nobles une disposition féodale et une disposition civile. Dans un acte pareil, fait sous les yeux du seigneur, on fit des dispositions pour la succession future, dans la vue que le fief pût être servi par les héritiers : aussi les seuls nobles eurent-ils d’abord la liberté de disposer des successions futures par contrat de mariage, comme l’ont remarqué Boyer\footnote{Décision 155, n° 8 ; et 204, n° 38.} et Aufrerius\footnote{{\itshape In Capella Tholosana}, décision 453.}.\par
Il est inutile de dire que le retrait lignager, fondé sur l’ancien droit des parents, qui est un mystère de notre ancienne jurisprudence française que je n’ai pas le temps de développer, ne put avoir lieu à l’égard des fiefs, que lorsqu’ils devinrent perpétuels.\par
{\itshape Italiam, Italiam}\footnote{{\itshape Énéide}, liv. III, vers 523.}… Je finis le traité des fiefs où la plupart des auteurs l’ont commencé.
 


% at least one empty page at end (for booklet couv)
\ifbooklet
  \pagestyle{empty}
  \clearpage
  % 2 empty pages maybe needed for 4e cover
  \ifnum\modulo{\value{page}}{4}=0 \hbox{}\newpage\hbox{}\newpage\fi
  \ifnum\modulo{\value{page}}{4}=1 \hbox{}\newpage\hbox{}\newpage\fi


  \hbox{}\newpage
  \ifodd\value{page}\hbox{}\newpage\fi
  {\centering\color{rubric}\bfseries\noindent\large
    Hurlus ? Qu’est-ce.\par
    \bigskip
  }
  \noindent Des bouquinistes électroniques, pour du texte libre à participation libre,
  téléchargeable gratuitement sur \href{https://hurlus.fr}{\dotuline{hurlus.fr}}.\par
  \bigskip
  \noindent Cette brochure a été produite par des éditeurs bénévoles.
  Elle n’est pas faîte pour être possédée, mais pour être lue, et puis donnée.
  Que circule le texte !
  En page de garde, on peut ajouter une date, un lieu, un nom ; pour suivre le voyage des idées.
  \par

  Ce texte a été choisi parce qu’une personne l’a aimé,
  ou haï, elle a en tous cas pensé qu’il partipait à la formation de notre présent ;
  sans le souci de plaire, vendre, ou militer pour une cause.
  \par

  L’édition électronique est soigneuse, tant sur la technique
  que sur l’établissement du texte ; mais sans aucune prétention scolaire, au contraire.
  Le but est de s’adresser à tous, sans distinction de science ou de diplôme.
  Au plus direct ! (possible)
  \par

  Cet exemplaire en papier a été tiré sur une imprimante personnelle
   ou une photocopieuse. Tout le monde peut le faire.
  Il suffit de
  télécharger un fichier sur \href{https://hurlus.fr}{\dotuline{hurlus.fr}},
  d’imprimer, et agrafer ; puis de lire et donner.\par

  \bigskip

  \noindent PS : Les hurlus furent aussi des rebelles protestants qui cassaient les statues dans les églises catholiques. En 1566 démarra la révolte des gueux dans le pays de Lille. L’insurrection enflamma la région jusqu’à Anvers où les gueux de mer bloquèrent les bateaux espagnols.
  Ce fut une rare guerre de libération dont naquit un pays toujours libre : les Pays-Bas.
  En plat pays francophone, par contre, restèrent des bandes de huguenots, les hurlus, progressivement réprimés par la très catholique Espagne.
  Cette mémoire d’une défaite est éteinte, rallumons-la. Sortons les livres du culte universitaire, cherchons les idoles de l’époque, pour les briser.
\fi

\ifdev % autotext in dev mode
\fontname\font — \textsc{Les règles du jeu}\par
(\hyperref[utopie]{\underline{Lien}})\par
\noindent \initialiv{A}{lors là}\blindtext\par
\noindent \initialiv{À}{ la bonheur des dames}\blindtext\par
\noindent \initialiv{É}{tonnez-le}\blindtext\par
\noindent \initialiv{Q}{ualitativement}\blindtext\par
\noindent \initialiv{V}{aloriser}\blindtext\par
\Blindtext
\phantomsection
\label{utopie}
\Blinddocument
\fi
\end{document}
