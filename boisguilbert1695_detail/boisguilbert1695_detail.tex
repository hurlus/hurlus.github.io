%%%%%%%%%%%%%%%%%%%%%%%%%%%%%%%%%
% LaTeX model https://hurlus.fr %
%%%%%%%%%%%%%%%%%%%%%%%%%%%%%%%%%

% Needed before document class
\RequirePackage{pdftexcmds} % needed for tests expressions
\RequirePackage{fix-cm} % correct units

% Define mode
\def\mode{a4}

\newif\ifaiv % a4
\newif\ifav % a5
\newif\ifbooklet % booklet
\newif\ifcover % cover for booklet

\ifnum \strcmp{\mode}{cover}=0
  \covertrue
\else\ifnum \strcmp{\mode}{booklet}=0
  \booklettrue
\else\ifnum \strcmp{\mode}{a5}=0
  \avtrue
\else
  \aivtrue
\fi\fi\fi

\ifbooklet % do not enclose with {}
  \documentclass[french,twoside]{book} % ,notitlepage
  \usepackage[%
    papersize={105mm, 297mm},
    inner=12mm,
    outer=12mm,
    top=20mm,
    bottom=15mm,
    marginparsep=0pt,
  ]{geometry}
  \usepackage[fontsize=9.5pt]{scrextend} % for Roboto
\else\ifav
  \documentclass[french,twoside]{book} % ,notitlepage
  \usepackage[%
    a5paper,
    inner=25mm,
    outer=15mm,
    top=15mm,
    bottom=15mm,
    marginparsep=0pt,
  ]{geometry}
  \usepackage[fontsize=12pt]{scrextend}
\else% A4 2 cols
  \documentclass[twocolumn]{report}
  \usepackage[%
    a4paper,
    inner=15mm,
    outer=10mm,
    top=25mm,
    bottom=18mm,
    marginparsep=0pt,
  ]{geometry}
  \setlength{\columnsep}{20mm}
  \usepackage[fontsize=9.5pt]{scrextend}
\fi\fi

%%%%%%%%%%%%%%
% Alignments %
%%%%%%%%%%%%%%
% before teinte macros

\setlength{\arrayrulewidth}{0.2pt}
\setlength{\columnseprule}{\arrayrulewidth} % twocol
\setlength{\parskip}{0pt} % classical para with no margin
\setlength{\parindent}{1.5em}

%%%%%%%%%%
% Colors %
%%%%%%%%%%
% before Teinte macros

\usepackage[dvipsnames]{xcolor}
\definecolor{rubric}{HTML}{800000} % the tonic 0c71c3
\def\columnseprulecolor{\color{rubric}}
\colorlet{borderline}{rubric!30!} % definecolor need exact code
\definecolor{shadecolor}{gray}{0.95}
\definecolor{bghi}{gray}{0.5}

%%%%%%%%%%%%%%%%%
% Teinte macros %
%%%%%%%%%%%%%%%%%
%%%%%%%%%%%%%%%%%%%%%%%%%%%%%%%%%%%%%%%%%%%%%%%%%%%
% <TEI> generic (LaTeX names generated by Teinte) %
%%%%%%%%%%%%%%%%%%%%%%%%%%%%%%%%%%%%%%%%%%%%%%%%%%%
% This template is inserted in a specific design
% It is XeLaTeX and otf fonts

\makeatletter % <@@@


\usepackage{blindtext} % generate text for testing
\usepackage[strict]{changepage} % for modulo 4
\usepackage{contour} % rounding words
\usepackage[nodayofweek]{datetime}
% \usepackage{DejaVuSans} % seems buggy for sffont font for symbols
\usepackage{enumitem} % <list>
\usepackage{etoolbox} % patch commands
\usepackage{fancyvrb}
\usepackage{fancyhdr}
\usepackage{float}
\usepackage{fontspec} % XeLaTeX mandatory for fonts
\usepackage{footnote} % used to capture notes in minipage (ex: quote)
\usepackage{framed} % bordering correct with footnote hack
\usepackage{graphicx}
\usepackage{lettrine} % drop caps
\usepackage{lipsum} % generate text for testing
\usepackage[framemethod=tikz,]{mdframed} % maybe used for frame with footnotes inside
\usepackage{pdftexcmds} % needed for tests expressions
\usepackage{polyglossia} % non-break space french punct, bug Warning: "Failed to patch part"
\usepackage[%
  indentfirst=false,
  vskip=1em,
  noorphanfirst=true,
  noorphanafter=true,
  leftmargin=\parindent,
  rightmargin=0pt,
]{quoting}
\usepackage{ragged2e}
\usepackage{setspace} % \setstretch for <quote>
\usepackage{tabularx} % <table>
\usepackage[explicit]{titlesec} % wear titles, !NO implicit
\usepackage{tikz} % ornaments
\usepackage{tocloft} % styling tocs
\usepackage[fit]{truncate} % used im runing titles
\usepackage{unicode-math}
\usepackage[normalem]{ulem} % breakable \uline, normalem is absolutely necessary to keep \emph
\usepackage{verse} % <l>
\usepackage{xcolor} % named colors
\usepackage{xparse} % @ifundefined
\XeTeXdefaultencoding "iso-8859-1" % bad encoding of xstring
\usepackage{xstring} % string tests
\XeTeXdefaultencoding "utf-8"
\PassOptionsToPackage{hyphens}{url} % before hyperref, which load url package

% TOTEST
% \usepackage{hypcap} % links in caption ?
% \usepackage{marginnote}
% TESTED
% \usepackage{background} % doesn’t work with xetek
% \usepackage{bookmark} % prefers the hyperref hack \phantomsection
% \usepackage[color, leftbars]{changebar} % 2 cols doc, impossible to keep bar left
% \usepackage[utf8x]{inputenc} % inputenc package ignored with utf8 based engines
% \usepackage[sfdefault,medium]{inter} % no small caps
% \usepackage{firamath} % choose firasans instead, firamath unavailable in Ubuntu 21-04
% \usepackage{flushend} % bad for last notes, supposed flush end of columns
% \usepackage[stable]{footmisc} % BAD for complex notes https://texfaq.org/FAQ-ftnsect
% \usepackage{helvet} % not for XeLaTeX
% \usepackage{multicol} % not compatible with too much packages (longtable, framed, memoir…)
% \usepackage[default,oldstyle,scale=0.95]{opensans} % no small caps
% \usepackage{sectsty} % \chapterfont OBSOLETE
% \usepackage{soul} % \ul for underline, OBSOLETE with XeTeX
% \usepackage[breakable]{tcolorbox} % text styling gone, footnote hack not kept with breakable


% Metadata inserted by a program, from the TEI source, for title page and runing heads
\title{\textbf{ Détail de la France, Factum de la France, opuscules divers }}
\date{1695}
\author{Boisguilbert, Pierre Le Pesant de (1646, 1714)}
\def\elbibl{Boisguilbert, Pierre Le Pesant de (1646, 1714). 1695. \emph{Détail de la France, Factum de la France, opuscules divers}}
\def\elsource{Pierre Le Pesant Boisguilbert, {\itshape Détail de la France, Factum de la France, opuscules divers}, in {\itshape Économistes-financiers du \textsc{xviii}\textsuperscript{e} siècle, précédés de Notices historiques sur chaque auteur, et accompagnés de commentaires et de notes explicatives, par M. Eugène Daire}, Paris, Guillaumin, 1843, p. 171-424. PDF : \href{http://gallica.bnf.fr/ark:/12148/bpt6k54449}{\dotuline{Gallica}}\footnote{\href{http://gallica.bnf.fr/ark:/12148/bpt6k54449}{\url{http://gallica.bnf.fr/ark:/12148/bpt6k54449}}}.}

% Default metas
\newcommand{\colorprovide}[2]{\@ifundefinedcolor{#1}{\colorlet{#1}{#2}}{}}
\colorprovide{rubric}{red}
\colorprovide{silver}{lightgray}
\@ifundefined{syms}{\newfontfamily\syms{DejaVu Sans}}{}
\newif\ifdev
\@ifundefined{elbibl}{% No meta defined, maybe dev mode
  \newcommand{\elbibl}{Titre court ?}
  \newcommand{\elbook}{Titre du livre source ?}
  \newcommand{\elabstract}{Résumé\par}
  \newcommand{\elurl}{http://oeuvres.github.io/elbook/2}
  \author{Éric Lœchien}
  \title{Un titre de test assez long pour vérifier le comportement d’une maquette}
  \date{1566}
  \devtrue
}{}
\let\eltitle\@title
\let\elauthor\@author
\let\eldate\@date


\defaultfontfeatures{
  % Mapping=tex-text, % no effect seen
  Scale=MatchLowercase,
  Ligatures={TeX,Common},
}


% generic typo commands
\newcommand{\astermono}{\medskip\centerline{\color{rubric}\large\selectfont{\syms ✻}}\medskip\par}%
\newcommand{\astertri}{\medskip\par\centerline{\color{rubric}\large\selectfont{\syms ✻\,✻\,✻}}\medskip\par}%
\newcommand{\asterism}{\bigskip\par\noindent\parbox{\linewidth}{\centering\color{rubric}\large{\syms ✻}\\{\syms ✻}\hskip 0.75em{\syms ✻}}\bigskip\par}%

% lists
\newlength{\listmod}
\setlength{\listmod}{\parindent}
\setlist{
  itemindent=!,
  listparindent=\listmod,
  labelsep=0.2\listmod,
  parsep=0pt,
  % topsep=0.2em, % default topsep is best
}
\setlist[itemize]{
  label=—,
  leftmargin=0pt,
  labelindent=1.2em,
  labelwidth=0pt,
}
\setlist[enumerate]{
  label={\bf\color{rubric}\arabic*.},
  labelindent=0.8\listmod,
  leftmargin=\listmod,
  labelwidth=0pt,
}
\newlist{listalpha}{enumerate}{1}
\setlist[listalpha]{
  label={\bf\color{rubric}\alph*.},
  leftmargin=0pt,
  labelindent=0.8\listmod,
  labelwidth=0pt,
}
\newcommand{\listhead}[1]{\hspace{-1\listmod}\emph{#1}}

\renewcommand{\hrulefill}{%
  \leavevmode\leaders\hrule height 0.2pt\hfill\kern\z@}

% General typo
\DeclareTextFontCommand{\textlarge}{\large}
\DeclareTextFontCommand{\textsmall}{\small}

% commands, inlines
\newcommand{\anchor}[1]{\Hy@raisedlink{\hypertarget{#1}{}}} % link to top of an anchor (not baseline)
\newcommand\abbr[1]{#1}
\newcommand{\autour}[1]{\tikz[baseline=(X.base)]\node [draw=rubric,thin,rectangle,inner sep=1.5pt, rounded corners=3pt] (X) {\color{rubric}#1};}
\newcommand\corr[1]{#1}
\newcommand{\ed}[1]{ {\color{silver}\sffamily\footnotesize (#1)} } % <milestone ed="1688"/>
\newcommand\expan[1]{#1}
\newcommand\foreign[1]{\emph{#1}}
\newcommand\gap[1]{#1}
\renewcommand{\LettrineFontHook}{\color{rubric}}
\newcommand{\initial}[2]{\lettrine[lines=2, loversize=0.3, lhang=0.3]{#1}{#2}}
\newcommand{\initialiv}[2]{%
  \let\oldLFH\LettrineFontHook
  % \renewcommand{\LettrineFontHook}{\color{rubric}\ttfamily}
  \IfSubStr{QJ’}{#1}{
    \lettrine[lines=4, lhang=0.2, loversize=-0.1, lraise=0.2]{\smash{#1}}{#2}
  }{\IfSubStr{É}{#1}{
    \lettrine[lines=4, lhang=0.2, loversize=-0, lraise=0]{\smash{#1}}{#2}
  }{\IfSubStr{ÀÂ}{#1}{
    \lettrine[lines=4, lhang=0.2, loversize=-0, lraise=0, slope=0.6em]{\smash{#1}}{#2}
  }{\IfSubStr{A}{#1}{
    \lettrine[lines=4, lhang=0.2, loversize=0.2, slope=0.6em]{\smash{#1}}{#2}
  }{\IfSubStr{V}{#1}{
    \lettrine[lines=4, lhang=0.2, loversize=0.2, slope=-0.5em]{\smash{#1}}{#2}
  }{
    \lettrine[lines=4, lhang=0.2, loversize=0.2]{\smash{#1}}{#2}
  }}}}}
  \let\LettrineFontHook\oldLFH
}
\newcommand{\labelchar}[1]{\textbf{\color{rubric} #1}}
\newcommand{\milestone}[1]{\autour{\footnotesize\color{rubric} #1}} % <milestone n="4"/>
\newcommand\name[1]{#1}
\newcommand\orig[1]{#1}
\newcommand\orgName[1]{#1}
\newcommand\persName[1]{#1}
\newcommand\placeName[1]{#1}
\newcommand{\pn}[1]{\IfSubStr{-—–¶}{#1}% <p n="3"/>
  {\noindent{\bfseries\color{rubric}   ¶  }}
  {{\footnotesize\autour{ #1}  }}}
\newcommand\reg{}
% \newcommand\ref{} % already defined
\newcommand\sic[1]{#1}
\newcommand\surname[1]{\textsc{#1}}
\newcommand\term[1]{\textbf{#1}}

\def\mednobreak{\ifdim\lastskip<\medskipamount
  \removelastskip\nopagebreak\medskip\fi}
\def\bignobreak{\ifdim\lastskip<\bigskipamount
  \removelastskip\nopagebreak\bigskip\fi}

% commands, blocks
\newcommand{\byline}[1]{\bigskip{\RaggedLeft{#1}\par}\bigskip}
\newcommand{\bibl}[1]{{\RaggedLeft{#1}\par\bigskip}}
\newcommand{\biblitem}[1]{{\noindent\hangindent=\parindent   #1\par}}
\newcommand{\dateline}[1]{\medskip{\RaggedLeft{#1}\par}\bigskip}
\newcommand{\labelblock}[1]{\medbreak{\noindent\color{rubric}\bfseries #1}\par\mednobreak}
\newcommand{\salute}[1]{\bigbreak{#1}\par\medbreak}
\newcommand{\signed}[1]{\bigbreak\filbreak{\raggedleft #1\par}\medskip}

% environments for blocks (some may become commands)
\newenvironment{borderbox}{}{} % framing content
\newenvironment{citbibl}{\ifvmode\hfill\fi}{\ifvmode\par\fi }
\newenvironment{docAuthor}{\ifvmode\vskip4pt\fontsize{16pt}{18pt}\selectfont\fi\itshape}{\ifvmode\par\fi }
\newenvironment{docDate}{}{\ifvmode\par\fi }
\newenvironment{docImprint}{\vskip6pt}{\ifvmode\par\fi }
\newenvironment{docTitle}{\vskip6pt\bfseries\fontsize{18pt}{22pt}\selectfont}{\par }
\newenvironment{msHead}{\vskip6pt}{\par}
\newenvironment{msItem}{\vskip6pt}{\par}
\newenvironment{titlePart}{}{\par }


% environments for block containers
\newenvironment{argument}{\itshape\parindent0pt}{\vskip1.5em}
\newenvironment{biblfree}{}{\ifvmode\par\fi }
\newenvironment{bibitemlist}[1]{%
  \list{\@biblabel{\@arabic\c@enumiv}}%
  {%
    \settowidth\labelwidth{\@biblabel{#1}}%
    \leftmargin\labelwidth
    \advance\leftmargin\labelsep
    \@openbib@code
    \usecounter{enumiv}%
    \let\p@enumiv\@empty
    \renewcommand\theenumiv{\@arabic\c@enumiv}%
  }
  \sloppy
  \clubpenalty4000
  \@clubpenalty \clubpenalty
  \widowpenalty4000%
  \sfcode`\.\@m
}%
{\def\@noitemerr
  {\@latex@warning{Empty `bibitemlist' environment}}%
\endlist}
\newenvironment{quoteblock}% may be used for ornaments
  {\begin{quoting}}
  {\end{quoting}}

% table () is preceded and finished by custom command
\newcommand{\tableopen}[1]{%
  \ifnum\strcmp{#1}{wide}=0{%
    \begin{center}
  }
  \else\ifnum\strcmp{#1}{long}=0{%
    \begin{center}
  }
  \else{%
    \begin{center}
  }
  \fi\fi
}
\newcommand{\tableclose}[1]{%
  \ifnum\strcmp{#1}{wide}=0{%
    \end{center}
  }
  \else\ifnum\strcmp{#1}{long}=0{%
    \end{center}
  }
  \else{%
    \end{center}
  }
  \fi\fi
}


% text structure
\newcommand\chapteropen{} % before chapter title
\newcommand\chaptercont{} % after title, argument, epigraph…
\newcommand\chapterclose{} % maybe useful for multicol settings
\setcounter{secnumdepth}{-2} % no counters for hierarchy titles
\setcounter{tocdepth}{5} % deep toc
\markright{\@title} % ???
\markboth{\@title}{\@author} % ???
\renewcommand\tableofcontents{\@starttoc{toc}}
% toclof format
% \renewcommand{\@tocrmarg}{0.1em} % Useless command?
% \renewcommand{\@pnumwidth}{0.5em} % {1.75em}
\renewcommand{\@cftmaketoctitle}{}
\setlength{\cftbeforesecskip}{\z@ \@plus.2\p@}
\renewcommand{\cftchapfont}{}
\renewcommand{\cftchapdotsep}{\cftdotsep}
\renewcommand{\cftchapleader}{\normalfont\cftdotfill{\cftchapdotsep}}
\renewcommand{\cftchappagefont}{\bfseries}
\setlength{\cftbeforechapskip}{0em \@plus\p@}
% \renewcommand{\cftsecfont}{\small\relax}
\renewcommand{\cftsecpagefont}{\normalfont}
% \renewcommand{\cftsubsecfont}{\small\relax}
\renewcommand{\cftsecdotsep}{\cftdotsep}
\renewcommand{\cftsecpagefont}{\normalfont}
\renewcommand{\cftsecleader}{\normalfont\cftdotfill{\cftsecdotsep}}
\setlength{\cftsecindent}{1em}
\setlength{\cftsubsecindent}{2em}
\setlength{\cftsubsubsecindent}{3em}
\setlength{\cftchapnumwidth}{1em}
\setlength{\cftsecnumwidth}{1em}
\setlength{\cftsubsecnumwidth}{1em}
\setlength{\cftsubsubsecnumwidth}{1em}

% footnotes
\newif\ifheading
\newcommand*{\fnmarkscale}{\ifheading 0.70 \else 1 \fi}
\renewcommand\footnoterule{\vspace*{0.3cm}\hrule height \arrayrulewidth width 3cm \vspace*{0.3cm}}
\setlength\footnotesep{1.5\footnotesep} % footnote separator
\renewcommand\@makefntext[1]{\parindent 1.5em \noindent \hb@xt@1.8em{\hss{\normalfont\@thefnmark . }}#1} % no superscipt in foot
\patchcmd{\@footnotetext}{\footnotesize}{\footnotesize\sffamily}{}{} % before scrextend, hyperref


%   see https://tex.stackexchange.com/a/34449/5049
\def\truncdiv#1#2{((#1-(#2-1)/2)/#2)}
\def\moduloop#1#2{(#1-\truncdiv{#1}{#2}*#2)}
\def\modulo#1#2{\number\numexpr\moduloop{#1}{#2}\relax}

% orphans and widows
\clubpenalty=9996
\widowpenalty=9999
\brokenpenalty=4991
\predisplaypenalty=10000
\postdisplaypenalty=1549
\displaywidowpenalty=1602
\hyphenpenalty=400
% Copied from Rahtz but not understood
\def\@pnumwidth{1.55em}
\def\@tocrmarg {2.55em}
\def\@dotsep{4.5}
\emergencystretch 3em
\hbadness=4000
\pretolerance=750
\tolerance=2000
\vbadness=4000
\def\Gin@extensions{.pdf,.png,.jpg,.mps,.tif}
% \renewcommand{\@cite}[1]{#1} % biblio

\usepackage{hyperref} % supposed to be the last one, :o) except for the ones to follow
\urlstyle{same} % after hyperref
\hypersetup{
  % pdftex, % no effect
  pdftitle={\elbibl},
  % pdfauthor={Your name here},
  % pdfsubject={Your subject here},
  % pdfkeywords={keyword1, keyword2},
  bookmarksnumbered=true,
  bookmarksopen=true,
  bookmarksopenlevel=1,
  pdfstartview=Fit,
  breaklinks=true, % avoid long links
  pdfpagemode=UseOutlines,    % pdf toc
  hyperfootnotes=true,
  colorlinks=false,
  pdfborder=0 0 0,
  % pdfpagelayout=TwoPageRight,
  % linktocpage=true, % NO, toc, link only on page no
}

\makeatother % /@@@>
%%%%%%%%%%%%%%
% </TEI> end %
%%%%%%%%%%%%%%


%%%%%%%%%%%%%
% footnotes %
%%%%%%%%%%%%%
\renewcommand{\thefootnote}{\bfseries\textcolor{rubric}{\arabic{footnote}}} % color for footnote marks

%%%%%%%%%
% Fonts %
%%%%%%%%%
\usepackage[]{roboto} % SmallCaps, Regular is a bit bold
% \linespread{0.90} % too compact, keep font natural
\newfontfamily\fontrun[]{Roboto Condensed Light} % condensed runing heads
\ifav
  \setmainfont[
    ItalicFont={Roboto Light Italic},
  ]{Roboto}
\else\ifbooklet
  \setmainfont[
    ItalicFont={Roboto Light Italic},
  ]{Roboto}
\else
\setmainfont[
  ItalicFont={Roboto Italic},
]{Roboto Light}
\fi\fi
\renewcommand{\LettrineFontHook}{\bfseries\color{rubric}}
% \renewenvironment{labelblock}{\begin{center}\bfseries\color{rubric}}{\end{center}}

%%%%%%%%
% MISC %
%%%%%%%%

\setdefaultlanguage[frenchpart=false]{french} % bug on part


\newenvironment{quotebar}{%
    \def\FrameCommand{{\color{rubric!10!}\vrule width 0.5em} \hspace{0.9em}}%
    \def\OuterFrameSep{\itemsep} % séparateur vertical
    \MakeFramed {\advance\hsize-\width \FrameRestore}
  }%
  {%
    \endMakeFramed
  }
\renewenvironment{quoteblock}% may be used for ornaments
  {%
    \savenotes
    \setstretch{0.9}
    \normalfont
    \begin{quotebar}
  }
  {%
    \end{quotebar}
    \spewnotes
  }


\renewcommand{\headrulewidth}{\arrayrulewidth}
\renewcommand{\headrule}{{\color{rubric}\hrule}}

% delicate tuning, image has produce line-height problems in title on 2 lines
\titleformat{name=\chapter} % command
  [display] % shape
  {\vspace{1.5em}\centering} % format
  {} % label
  {0pt} % separator between n
  {}
[{\color{rubric}\huge\textbf{#1}}\bigskip] % after code
% \titlespacing{command}{left spacing}{before spacing}{after spacing}[right]
\titlespacing*{\chapter}{0pt}{-2em}{0pt}[0pt]

\titleformat{name=\section}
  [block]{}{}{}{}
  [\vbox{\color{rubric}\large\raggedleft\textbf{#1}}]
\titlespacing{\section}{0pt}{0pt plus 4pt minus 2pt}{\baselineskip}

\titleformat{name=\subsection}
  [block]
  {}
  {} % \thesection
  {} % separator \arrayrulewidth
  {}
[\vbox{\large\textbf{#1}}]
% \titlespacing{\subsection}{0pt}{0pt plus 4pt minus 2pt}{\baselineskip}

\ifaiv
  \fancypagestyle{main}{%
    \fancyhf{}
    \setlength{\headheight}{1.5em}
    \fancyhead{} % reset head
    \fancyfoot{} % reset foot
    \fancyhead[L]{\truncate{0.45\headwidth}{\fontrun\elbibl}} % book ref
    \fancyhead[R]{\truncate{0.45\headwidth}{ \fontrun\nouppercase\leftmark}} % Chapter title
    \fancyhead[C]{\thepage}
  }
  \fancypagestyle{plain}{% apply to chapter
    \fancyhf{}% clear all header and footer fields
    \setlength{\headheight}{1.5em}
    \fancyhead[L]{\truncate{0.9\headwidth}{\fontrun\elbibl}}
    \fancyhead[R]{\thepage}
  }
\else
  \fancypagestyle{main}{%
    \fancyhf{}
    \setlength{\headheight}{1.5em}
    \fancyhead{} % reset head
    \fancyfoot{} % reset foot
    \fancyhead[RE]{\truncate{0.9\headwidth}{\fontrun\elbibl}} % book ref
    \fancyhead[LO]{\truncate{0.9\headwidth}{\fontrun\nouppercase\leftmark}} % Chapter title, \nouppercase needed
    \fancyhead[RO,LE]{\thepage}
  }
  \fancypagestyle{plain}{% apply to chapter
    \fancyhf{}% clear all header and footer fields
    \setlength{\headheight}{1.5em}
    \fancyhead[L]{\truncate{0.9\headwidth}{\fontrun\elbibl}}
    \fancyhead[R]{\thepage}
  }
\fi

\ifav % a5 only
  \titleclass{\section}{top}
\fi

\newcommand\chapo{{%
  \vspace*{-3em}
  \centering % no vskip ()
  {\Large\addfontfeature{LetterSpace=25}\bfseries{\elauthor}}\par
  \smallskip
  {\large\eldate}\par
  \bigskip
  {\Large\selectfont{\eltitle}}\par
  \bigskip
  {\color{rubric}\hline\par}
  \bigskip
  {\Large TEXTE LIBRE À PARTICPATION LIBRE\par}
  \centerline{\small\color{rubric} {hurlus.fr, tiré le \today}}\par
  \bigskip
}}

\newcommand\cover{{%
  \thispagestyle{empty}
  \centering
  {\LARGE\bfseries{\elauthor}}\par
  \bigskip
  {\Large\eldate}\par
  \bigskip
  \bigskip
  {\LARGE\selectfont{\eltitle}}\par
  \vfill\null
  {\color{rubric}\setlength{\arrayrulewidth}{2pt}\hline\par}
  \vfill\null
  {\Large TEXTE LIBRE À PARTICPATION LIBRE\par}
  \centerline{{\href{https://hurlus.fr}{\dotuline{hurlus.fr}}, tiré le \today}}\par
}}

\begin{document}
\pagestyle{empty}
\ifbooklet{
  \cover\newpage
  \thispagestyle{empty}\hbox{}\newpage
  \cover\newpage\noindent Les voyages de la brochure\par
  \bigskip
  \begin{tabularx}{\textwidth}{l|X|X}
    \textbf{Date} & \textbf{Lieu}& \textbf{Nom/pseudo} \\ \hline
    \rule{0pt}{25cm} &  &   \\
  \end{tabularx}
  \newpage
  \addtocounter{page}{-4}
}\fi

\thispagestyle{empty}
\ifaiv
  \twocolumn[\chapo]
\else
  \chapo
\fi
{\it\elabstract}
\bigskip
\makeatletter\@starttoc{toc}\makeatother % toc without new page
\bigskip

\pagestyle{main} % after style

  
\chapteropen
\chapter[{Le Détail de la France, la cause de la diminution de ses biens, et la facilité du remède, en fournissant en un mois tout l’argent dont le roi a besoin, et enrichissant tout le monde.}]{Le Détail de la France, la cause de la diminution de ses biens, et la facilité du remède, en fournissant en un mois tout l’argent dont le roi a besoin, et enrichissant tout le monde.}\renewcommand{\leftmark}{Le Détail de la France, la cause de la diminution de ses biens, et la facilité du remède, en fournissant en un mois tout l’argent dont le roi a besoin, et enrichissant tout le monde.}


\chaptercont
\section[{Première partie. De la diminution de la richesse nationale.}]{Première partie. De la diminution de la richesse nationale.}
\subsection[{Chapitre I.}]{Chapitre I.}
\noindent De tous les pays du monde dont les peuples ne sont pas tout à fait barbares, il n’y en a presque aucun dont la richesse, ou l’indigence, ne soit l’effet de la situation naturelle, participant à ces deux états, selon que son climat et sa terre se rencontrent plus ou moins propres à produire les choses nécessaires à la vie, ou avec lesquelles on se les peut procurer. Il n’y a que l’Espagne et la Hollande qui dérogent absolument à une règle si générale d’une manière bien opposée : celle-ci, ne produisant presque aucunes commodités, les a en abondance et à meilleur marché que dans les lieux où elles croissent, ainsi que les peuples les plus riches de la terre ; et l’autre, avec un excellent terroir et un climat heureux, ne peut subsister sans des secours étrangers.\par
Bien que la France soit le plus riche royaume du monde, on peut dire, toutefois, qu’elle n’est pas tout à fait exempte des désordres de l’Espagne, et qu’elle ne répond pas autant qu’elle le pourrait aux avances que la nature semble avoir faites en sa faveur ; puisque, sans parler de ce qui pourrait être, mais seulement de ce qui a été, on maintient que le produit en est aujourd’hui à 5 ou 600 millions moins par an dans ses revenus, tant en fonds qu’en industrie, qu’il n’était il y a trente ans ; que le mal augmente tous les jours, c’est-à-dire la diminution, parce que les mêmes causes subsistent toujours, et reçoivent même de l’accroissement, sans qu’on en puisse accuser celui des revenus du roi, lesquels n’ont jamais si peu haussé qu’ils ont fait depuis 1660, qu’ils n’ont augmenté que d’environ un tiers, au lieu que depuis deux cents ans ils avaient toujours doublé tous les trente ans.\par
Ce fait va être établi dans la première partie de ces Mémoires, ainsi que la diminution présente des biens de la France. Dans la seconde, on découvrira les causes de ces désordres ; et dans la troisième, on établira la facilité du remède, en fournissant quantité d’argent comptant au roi, et lui augmentant ses revenus ordinaires ; parce qu’on en fera autant de ceux de ses sujets, qui en sont le principe, en leur faisant racheter la cause de la diminution de leurs biens : ce qui produira tous ces effets à l’égard de Sa Majesté et de ses peuples, et cela sans nul mouvement extraordinaire, qui puisse troubler la certitude du présent, pour un avenir incertain ; mais en remettant seulement les choses dans un état naturel, qui est celui où elles étaient autrefois, et où elles seraient encore, si un mécompte presque continuel, causé par des intérêts indirects, ne les en avait tirées, en causant à tous moments des surprises à MM. les premiers ministres qui n’avaient que de bonnes intentions.
\subsection[{Chapitre II.}]{Chapitre II.}
\noindent Quelque surprenants que soient les efforts de la France dans cette présente guerre, l’étonnement sera encore plus grand de voir, par ces Mémoires, qu’elle produit tous ces prodiges avec la moitié de ses forces, l’autre étant suspendue par une puissance supérieure, qui arrête d’une manière indirecte des causes qui sembleraient devoir aller trop loin.\par
Sa puissance vient de ce que, produisant toutes sortes de choses nécessaires à la vie en assez grande abondance, non-seulement pour nourrir une grande quantité d’habitants qu’elle renferme, mais encore pour en faire part à ceux qui en manquent, elle se trouve en même temps environnée de voisins qui, n’ayant pas le même avantage, épuisent leurs contrées pour trouver quelque chose de propre aux délices et au superflu, afin de changer avec elle contre le nécessaire ; et cela ne suffisant pas encore à leurs besoins, ils se voient contraints de se faire ses voituriers, et de lui aller chercher, dans les contrées les plus éloignées, de ce même superflu pour en tirer le même nécessaire.\par
Comme les quatre éléments sont les principes de tous les êtres, et que c’est d’eux dont ils se forment tous, de même, tout le fondement et la cause de toutes les richesses de l’Europe sont le blé, le vin, le sel et la toile, qui abondent en France ; et on ne se procure les autres choses qu’à proportion que l’on a plus qu’il ne faut de celles-là. Et ainsi tous les biens de la France étant divisés en deux espèces, en biens fonds et en biens de revenu d’industrie, cette dernière, qui renferme trois fois plus de monde que l’autre, hausse ou baisse à proportion de la première. En sorte que la croissance des fruits de la terre fait travailler les avocats, les médecins, les spectacles et les moindres artisans, de quelque art qu’ils puissent être ; de manière qu’on voit très-peu de ces sortes de gens dans les pays stériles, au lieu qu’ils abondent dans les autres.
\subsection[{Chapitre III.}]{Chapitre III.}
\noindent Par tout ce qu’on vient de dire de la France, on aurait peine à comprendre de quelle façon les revenus en peuvent être diminués d’une aussi grande somme que 500 millions par an, tant ceux en fonds que ceux d’industrie, la même terre, le même climat et les mêmes habitants (à fort peu près) y étant encore, et n’y ayant ni avocat, ni médecin, ni artisan qui ne soit disposé à gagner tout autant comme il faisait il y a trente ans. Cependant toutes ces choses ne sont pas à la moitié de notoriété publique, et leur diminution, qui a commencé en 1660, ou environ, continue tous les jours avec augmentation, parce que la cause en est la même, qui est la diminution du revenu des fonds, qui ne sont pas, l’un portant l’autre, à la moitié de ce qu’ils étaient en ce temps-là. Et si quelques-uns n’ont pas souffert un si puissant déchet, c’est parce qu’appartenant à des personnes élevées en dignité, des receveurs riches d’ailleurs les ont pris à ferme avec perte de leur part, pour acheter en quelque manière une protection qu’ils destinaient à d’autres usages. D’autres fonds d’ailleurs ont beaucoup plus baissé, y en ayant plusieurs qui ne sont pas au quart de ce qu’ils étaient autrefois. Ainsi ceux qui avaient 1 000 livres de rentes en fonds, n’en ayant plus que 500, n’emploient plus des ouvriers que pour la moitié de ce qu’ils faisaient autrefois, lesquels en usent de même à leur tour à l’égard de ceux desquels ils se procuraient leurs besoins, par une circulation naturelle qui fait que les fonds commençant le mouvement, il faut que l’argent qu’ils forment pour faire sortir les denrées qu’ils produisent, passe par une infinité de mains avant que, son circuit achevé, il revienne à eux ; de manière que ne faisant ces passages que, pour autant qu’il en est sorti la première fois, on peut dire qu’une diminution de 500 livres par an en pure perte dans un fonds en produit une de plus de 3 000 livres par an au corps de la république, et par conséquent préjudicie extrêmement au roi, qui ne peut jamais tirer autant d’impôts de sujets pauvres comme de riches.
\subsection[{Chapitre IV.}]{Chapitre IV.}
\noindent Si la diminution du revenu des fonds, qui a causé celle des revenus de l’industrie, est une chose si certaine que personne n’en doute, la cause ne l’est pas moins, quoiqu’on n’y fasse point de réflexion, et que l’on mette sur le compte de l’augmentation des revenus du roi Ce qui n’en est point du tout l’effet.\par
Les fonds sont diminués de moitié pour le moins, parce que le prix de toutes les denrées est à la moitié de ce qu’il était il y a trente ans ; et les denrées souffrent cette diminution, parce qu’il s’en consomme beaucoup moins. Par exemple, les boucheries donnent bien moins ; les foires des villes où il se débitait des boissons ne sont pas au quart, pour la quantité, de ce qu’elles étaient, et le prix même en est bien moindre. Ainsi, il faut que les fonds qui les produisaient souffrent une pareille diminution, provenant non seulement de celle du prix dans la vente des denrées, mais encore dans leur croissance ; parce que qu’y ayant aucuns fruits de la terre qui ne demandent de la dépense pour la culture, qui produit plus ou moins que l’on fait des avances pour mettre les choses dans leur perfection, lesquelles sont toujours les mêmes indépendamment du débit que l’on en aura, ce débit venant à ne pas répondre à ce qu’on a mis, fait que l’on néglige ces mêmes avances dans la suite, et réduit le produit non seulement à la moitié de ce qu’il était, mais même à rien, y ayant des terres entièrement abandonnées, qui étaient autrefois en grande valeur, qui est une perte qui se répand sur tout le corps de l’État : en sorte qu’un pareil destin arrivé à un village d’auprès Cherbourg en fait ressentir des effets jusqu’à Bayonne, par une liaison imperceptible, mais très réelle, que toutes les parties d’un État ont les unes avec les autres.
\subsection[{Chapitre V.}]{Chapitre V.}
\noindent La perte de la moitié des biens en général de la France étant constante, par les raisons qu’on vient de traiter ; quoique la réduction de cette perte ou estimation à un prix certain soit une chose indifférente en elle-même, cependant on en a bien voulu faire la supputation, afin d’en tirer deux avantages : le premier, de la rendre plus sensible, et le second, de faire toucher au doigt et à l’œil quel intérêt le roi a, indépendamment de celui du public, à changer la situation des choses, puisque, s’il est vrai, comme on le va montrer, qu’il y ait 500 millions moins de revenu qu’il n’y avait il y a trente ans, il est certain qu’étant rétabli (ce qui est très aisé), Sa Majesté fera une des plus grandes conquêtes qu’elle puisse jamais faire, non seulement sans répandre de sang ni sans sortir de ses États, mais même en enrichissant tout le monde, ce dont il aura nécessairement sa part.\par
On maintient donc que la diminution est de 500 millions par an, parce qu’elle est de la moitié des biens du royaume, et que ces mêmes biens seulement en fonds, tant réels, comme les terres, que par accident, comme les charges, les greffes, les péages et les moulins, allaient autrefois à 700 millions par an : ainsi ces mêmes biens, quand ils ne seraient que doublés par les biens d’industrie, feraient plus de 1 400 millions par an ; de sorte que, tout étant diminué de moitié, s’il y a de l’erreur dans cette supputation, c’est de ne pas porter le déchet assez loin.
\subsection[{Chapitre VI.}]{Chapitre VI.}
\noindent Il reste à faire voir que cette perte n’est point l’effet de l’augmentation des revenus du roi depuis trente ans, puisqu’ils n’ont jamais reçu si peu de hausse en pareil espace de temps, et que depuis deux siècles environ, les revenus des peuples, au lieu de diminuer comme ils ont fait, doublaient au contraire dans le même période de temps, ce qui était cause de l’augmentation de ceux du roi ; et l’un et l’autre étaient causés par l’abondance des espèces d’or et d’argent, que la découverte du Nouveau-Monde avait rendues et rend tous les jours plus communes. Tout ceci n’est qu’une question de fait, que l’on va établir, en commençant à la mort de Charles VII, arrivée en 1461.\par
Philippe de Commines, qui passe pour l’auteur le plus assuré du siècle passé, et qui ne parle que des choses qu’il a vues, dit que tout le revenu du roi, à la mort de ce monarque, n’allait qu’à 1 800 000 livres par an, et que quand Louis XI mourut, en 1483, la France produisait au roi 4 700 000 livres.\par
La minorité de Charles VIII, qui lui succéda, adoucit un peu les choses ; et Louis XII, appelé père du peuple, qui le suivit, les continua à peu près sur le même pied. Mais François I\textsuperscript{er} étant arrivé à la couronne, en 1515, les guerres qu’il eut à soutenir lui ayant fait mettre les affaires sur le même pied que du temps de Louis XI, son revenu, en 1525, allait à près de 9 000 000, ce qui est le double de ce qu’il était trente-cinq ans auparavant. Cela continua à peu près jusqu’à la mort de Henri II, et sous la minorité de ses enfants il se trouva que les revenus de la couronne allaient à 16 000 000, c’est-à-dire qu’ils avaient pareillement doublé dans le même espace de temps.\par
Enfin sous Henri III, en 1582, ces mêmes revenus vont à 32 millions, comme on peut voir dans l’histoire de Mézeray. Les guerres civiles vinrent ensuite, qui suspendirent l’état des choses. Henri IV commençait à les rétablir quand sa mort imprévue donna lieu à une minorité peu propre à augmenter les affaires du royaume, de manière que les revenus de la couronne n’allaient qu’à 35 000 000 à l’arrivée du cardinal de Richelieu au ministère, qui les laissa à sa mort à 70 000 000, en sorte qu’ils doublèrent de tout point ; et il semble qu’ils auraient suivi cette gradation, puisqu’en 1660, qui est l’année où les biens des particuliers, tant en fonds qu’en industrie, étaient au plus haut point où ils furent jamais (et depuis lequel temps ils ont toujours diminué), ceux du roi avaient encore augmenté, quoique l’on fût en guerre au-dehors et assez souvent au-dedans. Depuis ce temps-là on ne trouvera pas que les revenus du roi aient augmenté que d’environ un tiers, même en y comprenant les conquêtes du roi, qui sont un dixième sur tout le royaume ; et ceux des peuples sont diminués au moins de la moitié.
\subsection[{Chapitre VII.}]{Chapitre VII.}
\noindent Bien que la France soit plus remplie d’argent qu’elle n’a jamais été, que la magnificence et l’abondance y soient extrêmes ; comme ce n’est qu’en quelques particuliers, et que la plus grande partie est dans la dernière indigence, cela ne peut pas compenser la perte que fait l’État dans le plus grand nombre. Ou plutôt, à parler proprement, comme la richesse d’un royaume consiste en son terroir et en son commerce, on peut dire que l’un et l’autre n’ont jamais été dans un si grand désordre, c’est-à-dire les terres si mal cultivées et les denrées si mal vendues, parce que la consommation en a été entièrement anéantie à l’égard des étrangers, et beaucoup diminuée au-dedans par des intérêts personnels, qui ont fait que l’on a surpris MM. les ministres, en obtenant des édits également dommageables au roi et au peuple, comme on fera voir dans la seconde partie de ces Mémoires.\par
Mais, pour ne rien anticiper et finir ce premier point de la diminution présente des biens de la France, on dira que, bien que les revenus de Sa Majesté, quant à la somme, soient au plus haut point qu’ils ont jamais été, cependant il y a deux choses incontestables à remarquer : la première, qu’il s’en faut beaucoup, ainsi que l’on a dit, que cette augmentation soit proportionnée à celle des espèces d’or et d’argent, et à la hausse qu’elle apporte tous les jours au prix de toutes choses, dans l’Europe et dans les autres parties du monde ; et la seconde, que, lorsqu’en 1582 la France rapportait au roi 32 000 000, il était bien plus riche qu’il n’est aujourd’hui, parce que, comme il y a un dixième d’augmentation au domaine de la France, c’était sur le pied de 35 000 000, lesquels, eu égard au prix des choses de ce temps-là et à celui de présent, répondent à 175 000 000 d’aujourd’hui ; attendu que, comme l’or et l’argent ne sont et n’ont jamais été une richesse en eux-mêmes, ne valent que par relation, et qu’autant qu’ils peuvent procurer les choses nécessaires à la vie, auxquelles ils servent seulement de gage et d’appréciation, il est indifférent d’en avoir plus ou moins, pourvu qu’ils puissent produire les mêmes effets\par
Ainsi, comme en 1250, qu’on trouve, par des anciens registres, qu’un ouvrier dans Paris, qui gagne aujourd’hui 40 ou 50 sous par jour, ne gagnait en ce temps-là que 4 deniers, c’est-à-dire la centième partie de ce qu’il fait à présent ; toutefois il vivait avec autant de commodité, parce que toutes choses y étaient proportionnées : il avait ses besoins avec ses 4 deniers comme font ceux du même métier aujourd’hui avec leurs 50 sous. Et il s’ensuit qu’un homme qui avait mille livres de rente dans ce siècle était plus riche qu’un qui en a cent mille à présent. Or, bien que sous Henri III les choses ne fussent pas en cet état et que les denrées eussent beaucoup haussé de prix, cependant ce n’était pas en un point qui pût faire que le roi, avec ses revenus de ce temps-là, ne s’en procurât pas beaucoup davantage qu’il ne ferait aujourd’hui. En effet, les trente-cinq millions de Henri III, étant environ le tiers des revenus de la couronne de ce temps, les denrées n’étaient qu’en un cinquième du prix d’à présent ; et la mesure du blé, qui donne le prix à tout, qui vaut maintenant 40 sous, n’en valait que 8 en ce temps-là, comme cela se justifie par les appréciations qui en restent. Ce qui montre incontestablement que les revenus de la couronne étaient sur le pied de 175 000 000 d’aujourd’hui ; cependant la France n’était pas ruinée comme elle est, toutes ses terres étant cultivées autant bien qu’elles le pouvaient être et ses denrées au plus haut prix qu’elles eussent été, sans qu’on les vît devenir inutiles comme à présent, tandis que ses voisins ne demanderaient pas mieux que de les acheter et de les consommer.\par
Les particuliers se pouvaient ruiner, ou par trop de dépenses, ou par d’autres causes ordinaires ; mais le corps de l’État n’en souffrait point, et les terres, qui sont le principe de tous les biens, tant réels que d’industrie, changeant de maître, c’était sans aucune diminution de leur juste et première valeur ; parce qu’il n’y en avait aucune, ni dans la quantité des denrées qu’elles produisent, ni dans le prix, ni dans la facilité du débit. De manière qu’on peut dire que, bien que le roi tirât de la France sur le pied de 175 000 000, et que ces mêmes revenus ne soient guère qu’à 112 ou 115 000 000 à présent, cependant il levait beaucoup moins sur les peuples que l’on ne fait, parce que toute la France contribuait au paiement des impôts autant qu’il était en son pouvoir, au lieu que présentement il n’y a que la moitié qui soit utile, l’autre étant entièrement ou abandonnée, ou beaucoup moins cultivée qu’elle ne le pourrait être, ou plutôt qu’elle ne l’a été, par des causes qui ne sont rien moins que l’effet du hasard, ainsi que l’on va faire voir.
\section[{Seconde partie. Des causes de la diminution de la richesse nationale.}]{Seconde partie. Des causes de la diminution de la richesse nationale.}
\subsection[{Chapitre I.}]{Chapitre I.}
\noindent Bien que la cause de la diminution des biens de la France doive être une chose aussi constante que la diminution même, cependant, quoique tout le monde convienne de l’une, il s’en faut beaucoup que ce soit la même chose de l’autre. Les commissaires du premier ordre envoyés par tout le royaume pour trouver les moyens de rétablir ce qui était défectueux, étaient une marque certaine qu’on n’était pas persuadé que tout fût dans la perfection ; et comme cette tentative a été sans suite, on veut croire que c’est que l’on ne convint pas aisément de la cause du mal, et par conséquent du remède. Les uns ont prétendu dire que c’était qu’il n’y avait plus de commerce ; mais c’était apporter pour cause du désordre le désordre même. Les autres ont avancé qu’il n’y avait plus d’argent ; mais on vient de voir dans le changement des espèces combien ils se sont mécomptés ; et les autres, enfin, ont allégué l’augmentation des revenus du roi, pour ne pas dire des impôts, ce qui eût ôté toute espérance de changement, étant difficile de diminuer une chose dont les causes demandent de l’augmentation et jamais de diminution. On a assez fait voir, dans la première partie de ces Mémoires, le peu de fondement d’un pareil raisonnement ; c’est pourquoi on n’en parlera pas davantage, pour passer aux véritables causes de ces désordres.
\subsection[{Chapitre II.}]{Chapitre II.}
\noindent On a prouvé la diminution de tous les revenus de la France par celle du produit des fonds, tant dans le prix de la vente des denrées, que dans la quantité de leur croissance, et que l’un et l’autre étaient l’effet du défaut de la consommation, qui était pareillement diminuée de moitié, tous les biens du monde étant inutiles, à moins qu’ils ne soient consommés. Ainsi, pour trouver les causes de la ruine de la France, il ne faut que découvrir celles de la ruine de la consommation : il y en a deux essentielles, qui, bien loin d’être l’effet de quelque intérêt public, ne sont au contraire produites que par quelques intérêts particuliers, très aisés à faire cesser ou changer, sans presque aucune perte de leur part.\par
La consommation a cessé, parce qu’elle est devenue absolument défendue et absolument impossible. Elle est {\itshape défendue}, par l’incertitude de la Taille, qui étant entièrement arbitraire, n’a point de tarif plus certain que d’être payée plus haut plus on est pauvre, et plus on fait valoir des fonds appartenant à des personnes indéfendues ; et plus bas plus on est riche, et plus on a des recettes considérables, qui portent avec elles le pouvoir de faire payer sa Taille aux malheureux, parce que l’on tient les terres à plus haut prix, pour acheter en quelque manière cette licence, par la protection de ceux à qui elles appartiennent : en sorte qu’il n’est point extraordinaire de voir, dans une même paroisse, une recette de 3 ou 4000 livres de rente ne contribuer que pour dix ou douze écus à la Taille, pendant qu’un autre, qui ne tient que pour 3 à 400 livres de fermage, en paiera cent pour sa part ; et comme l’un et l’autre n’ont point de titre pour souffrir ou faire ce désordre, ils n’y sont maintenus que par une infinité de circonstances, dont on parlera dans la suite, infiniment plus dommageables à tout le corps de l’État que la Taille même. Enfin, la consommation est devenue {\itshape impossible} par les Aides et par les Douanes sur les sorties et passages du royaume, qui ont mis toutes les denrées à un point, que non seulement elles ne se transportent plus au-dehors au quart de ce qu’elles faisaient autrefois, mais quelles périssent même dans les lieux où elles croissent, pendant qu’en d’autres lieux tout proches elles valent un prix exorbitant ; ce qui ruine également les deux contrées, parce que tout pays qui ne vend point ses denrées ne tire point celles des autres : c’est ce que l’on traitera en particulier, après avoir parlé des Tailles.
\subsection[{Chapitre III.}]{Chapitre III.}
\noindent La Taille, qui n’a commencé en France à être ordinaire que depuis que l’Église (sous prétexte de dévotions et de fondations pieuses) a si fort surpris les rois et les princes, qu’elle s’est fait donner généralement tous leurs Domaines, qui étaient si considérables, qu’ils se passaient aisément de rien lever sur leur peuple, hors les occasions extraordinaires, a toujours doublé tous les trente ans (ainsi qu’il a été dit) depuis son institution, qui est environ le règne de Charles VII, jusqu’en 1651. Et bien que depuis ce temps-là elle ait toujours diminué, cependant elle a cent fois plus ruiné le monde qu’elle n’avait fait auparavant. Car, bien qu’elle ne soit qu’à 36 millions par an, et qu’on l’ait vue à 48 millions en 1650 et 1651, on peut dire toutefois que la misère est trois fois plus grande dans les campagnes qu’elle n’a jamais été. Et, avec tout cela, on soutient, comme on le va faire voir présentement, qu’elle pourrait doubler, non seulement sans incommoder personne, mais même sans empêcher que chacun ne s’enrichît. En effet, on peut dire qu’il n’y a pas le tiers de la France qui y contribue, n’y ayant que les plus faibles et les plus misérables, et ceux qui ont le moins de fonds. En sorte qu’étant trop forte à leur égard, elle les ruine absolument ; et après qu’ils sont devenus inutiles aux contributions publiques, elle en va ruiner d’autres à leur tour : outre qu’une personne ruinée ne consommant plus rien, les denrées de ceux qui se sont exemptés leur devenant inutiles par ce moyen, ils sont bien plus ruinés que s’ils avaient trois fois payé la Taille de ceux qui ne sont accablés que par leur crédit, ou par celui de leurs maîtres ; et c’est ce qui se comprendra bien mieux par la description que l’on va faire de la manière dont les Tailles se départissent ; d’abord par Élection et par paroisses, par MM. les Commissaires départis dans les généralités ; ensuite la façon dont les collecteurs qui sont élus par les paroisses les asseyent sur chaque particulier, les moyens dont ils se servent pour se les faire payer, et les autres pour s’en défendre ; et enfin, les divers intérêts des receveurs, des juges et des sergents, et comment le tout se fait d’une manière ruineuse : en sorte que l’on va demeurer d’accord qu’une guerre continuelle serait bien moins à charge au peuple qu’un impôt exigé d’une pareille façon.
\subsection[{Chapitre IV.}]{Chapitre IV.}
\noindent La Taille, qui était d’abord départie par les Élus, puis par les Trésoriers de France, et enfin par les Commissaires envoyés du Conseil, ne produisait d’abord aucuns des pernicieux effets que l’on voit à présent. Au contraire, la tradition porte que, comme la plus haute Taille était une marque d’opulence et de distinction, les particuliers se piquaient d’en payer davantage que leurs voisins, pour être préférés aux honneurs, comme on voit arriver aux rétributions de l’église, où les riches veulent se signaler par-dessus les pauvres. Mais aujourd’hui c’est justement le contraire, et lorsque la somme à laquelle une généralité est arrêtée, est venue du Conseil, tout le monde fait sa cour à MM. les intendants, afin que leurs paroisses soient favorablement traitées, indépendamment du pouvoir où elles peuvent être de payer plus ou moins de Taille. En sorte qu’il n’est pas extraordinaire de voir une paroisse de cent feux, et du contenu de 1 500 arpents de terre, payer beaucoup moins que la paroisse qui n’en contiendra que la moitié. Mais celui qui cause ce soulagement, qu’on peut appeler une ruine, a pour sa récompense l’exemption de ses fermiers ou receveurs, qui sont taxés à rien ou très peu de chose, mais qui, par une espèce de contre-échange, lui paient la Taille : et si les autres fermiers ou détenteurs de fonds à louage tiennent les terres à huit livres l’arpent, ceux des seigneurs les prennent à dix et onze livres. Quoique quelques intendants bien intentionnés aient voulu arrêter ce désordre, cependant, comme il était impossible que ce fût d’une manière générale, et qui ôtât toute jalousie, parce que de très grands seigneurs se trouvant dans cette espèce, on ne pouvait pas commencer par eux, comme il eût été de nécessité pour montrer l’exemple, ils ont tous abandonné ce projet dès les commencements ; et cette conduite a passé et passe imperceptiblement d’une condition à l’autre, jusqu’aux personnes qui sembleraient être les moins privilégiées, parce qu’il n’a jamais été constant à quel degré il fallait commencer d’arrêter un si grand mal. En sorte qu’aujourd’hui, une des plus agréables fonctions de MM. les intendants des provinces est cette répartition ; parce que comme l’usage n’est pas que la justice seule en décide, on a recours à tous les moyens qui peuvent servir à se faire considérer, un homme étant respecté dans le pays à proportion que ses paroisses sont favorablement traitées par MM. les intendants. Ce mauvais exemple dans le département des paroisses autorise en quelque façon une pareille conduite dans l’assiette particulière des contribuables de chaque lieu, d’une manière surprenante, en quoi les autres collecteurs ou asséeurs, outre la pente naturelle qu’on a à suivre les mauvais exemples, se trouvent merveilleusement secondés, ou plutôt forcés, par des intérêts indirects des receveurs des Tailles, tant généraux que particuliers, comme on le justifiera par la suite.
\subsection[{Chapitre V.}]{Chapitre V.}
\noindent Les départements étant envoyés dans chaque paroisse, elle élit aussitôt des personnes pour asseoir et cueillir l’impôt, que l’on appelle communément collecteurs ; sur quoi il sera dit en passant, ou plutôt par avance, que cette seule fonction, dont il ne revient pas un denier au roi, coûte plus au peuple, et par conséquent à l’État, que la Taille même. Les collecteurs élus en plus ou moins grande quantité, suivant que la Taille de la paroisse est forte, y en ayant jusqu’à sept dans les lieux considérables, se font faire la cour à leur tour, pour l’asseoir sur leurs concitoyens. Mais c’est de la manière que des gens qui croient, que la misère autorise tout, peuvent faire ; c’est-à-dire qu’on commence par se venger de ceux de qui on croit être blessé en pareille occasion, ce qui se substitue jusqu’à la troisième génération ; après quoi on a soin de ses parents et amis, riches ou pauvres, ce qui n’est presque d’aucune considération. Ajoutez aussi que les moindres collecteurs (parce qu’on en fait de tous les degrés) ont un intérêt plus fort que tous ceux-là, qui est le soulagement de leur pauvreté, à laquelle cette commission donne quelque remise pour l’aggraver d’une manière plus violente. Car la Taille s’asseyant à la pluralité des voix, ils prennent de l’argent des riches pour leur vendre leurs suffrages ; et la moindre corruption est d’en recevoir des repas. En sorte que, ces collecteurs ayant peine quelquefois à convenir, ils sont des trois mois de temps à s’assembler tous les jours sans rien déterminer ; ce qui est autant de temps perdu pour des personnes en qui il compose le principal revenu, outre les autres dépenses, toutes les assemblées ne se faisant d’ordinaire qu’au cabaret. D’ailleurs, la collecte étant en retardement, et par conséquent l’apport des deniers en recette, les receveurs des Tailles, qui ont érigé en revenus ordinaires les courses d’huissiers, et les contraintes qu’ils exercent contre les paroissiens faute de paiement dans les temps prescrits, ne manquent pas de jouer leur rôle. De façon qu’autrefois dans les grands lieux, par où les collecteurs commençaient, c’était de prendre de l’argent en rente en leur propre et privé nom, un seul pour le tout, pour payer le premier quartier de la Taille, sauf à acquitter à la fin de la recette. Mais comme la plus grande partie ne s’assied plus maintenant que sur les misérables, ainsi qu’il a été dit, et qu’on en va encore toucher un mot, il se trouve extrêmement de mauvais deniers, et le recours sur la paroisse étant une chose d’une trop longue discussion, et dont on ne peut jamais retirer le tiers de ce qu’on y met et de ce qu’il faut avancer pour y parvenir, ils aiment mieux perdre ce qui leur est dû, et l’on en a vu plusieurs avoir été décrétés pour ces sortes de dettes.\par
Mais, pour continuer dans la manière de l’assiette, après avoir fait ce que l’on vient de dire, on épargne ou l’on considère (ce qui est le mot en usage) les fermiers du seigneur de la paroisse, à proportion que l’on croit qu’il s’est employé lui-même auprès de MM. les intendants pour faire considérer la paroisse ; on a le même égard pour les gentilshommes qui sont de quelque considération, pour ceux qui appartiennent à des personnes de justice, jusqu’à des procureurs et des sergents. En sorte que tout le fardeau tombe sur les artisans ou marchands qui n’ont d’autre fonds que leur industrie, à proportion que l’on croit que l’on en pourra être payé. De manière que c’est à ces sortes de gens, qui font toute la richesse d’un État, à se tenir le plus couverts qu’ils peuvent ; et même, comme ils aiment mieux tout abandonner que de se voir exposés en proie à leurs ennemis ou à leurs envieux, ou bien ils se retirent avec le bien qu’ils peuvent avoir amassé dans les lieux francs, où n’étant pas faits au commerce du pays, ils n’ont plus d’autres ressources que de vivre d’épargne, et de réduire toutes leurs consommations ; au lieu que s’ils avaient demeuré dans les endroits de leur naissance, ils auraient continué à s’enrichir et enrichir les autres, ce qui est inséparable l’un de l’autre ; ou bien, enfin, ils font leur retraite en des pays étrangers. Il n’y a pas cinquante ans qu’au bourg de Fécamp, sur la côte de Normandie, il y avait cinquante bâtiments terre-neuviens, c’est-à-dire qui allaient à la pêche des morues en Terre-Neuve, et faisaient par conséquent, chacun sur le lieu, pour sept à huit mille livres de consommation : ils n’avaient d’autre occupation qu’une simple maison pour leurs femmes et leurs enfants, et pour eux lorsqu’ils n’étaient point en mer ; cependant, on les a si bien fatigués par des Tailles exorbitantes, qu’on leur faisait payer aussi fortes que s’ils avaient eu des recettes de dix mille livres, sans nulle protection, qu’ils se sont tous retirés, et il n’en restait pas trois avant le commencement de la guerre : les uns ont tout à fait quitté le commerce ; quelques-uns se sont établis ailleurs ; et la plus grande partie étant de la nouvelle religion, a passé en Hollande, où ils ont acquis des richesses immenses.\par
Le rôle étant enfin achevé de la manière que l’on vient de dire, il en faut faire la collecte ; et c’est où les désordres ne sont pas moindres que dans l’assiette.
\subsection[{Chapitre VI.}]{Chapitre VI.}
\noindent Comme ce recouvrement est une corvée des plus désagréables qu’on puisse imaginer, les collecteurs, en quelque nombre qu’ils soient, ne la veulent faire que tous unis ensemble, et marchant par les rues conjointement. De manière qu’aux endroits où il y en a sept, on voit sept personnes, au lieu de se relever, marcher continuellement par les rues ; et comme la Taille ne se tire pas dans une année à beaucoup près, on voit les collecteurs de l’année présente marcher, ou plutôt saccager d’un côté, pendant que ceux de la précédente en usent de même d’un autre ; et lorsqu’il y a. quelque étape ou quelque ustensile à cueillir, comme il faut de nouveaux collecteurs, cela forme une nouvelle brigade sur le modèle des autres, lesquelles jointes ensemble, sans parler de la collecte du sel, qui se fait de la même manière en plusieurs endroits, composent une espèce d’armée qui, pendant une année entière, perd son temps à battre le pavé, sans presque rien recevoir que mille injures et mille imprécations. Et cela parce que, comme lors de l’assiette, l’intérêt des particuliers imposables, et qui ne comptent sur aucune protection, est de cacher toute sorte de montre d’aisance par une cessation entière de commerce et de consommation ; de même lors de la collecte ils en ont un autre, qui est de ne payer que sou à sou, après mille contraintes et mille exécutions, soit pour se venger des collecteurs de les avoir imposés à une somme trop forte, en retardant par là leur apport en recette, et leur faisant souffrir des courses d’huissiers, ou pour rebuter ceux de l’année suivante de les mettre en une pareille somme, par les difficultés des paiements ; — de manière qu’après avoir marché une semaine tout entière, ils ne remportent souvent que des malédictions, pendant que d’un autre côté ils sont accablés de frais par les receveurs des Tailles, qui ont érigé ces sortes de contraintes en revenant-bon de leurs charges. Et s’il arrive que des paroisses, à l’aide de quelques personnes qui leur peuvent prêter de l’argent, paient à jour nommé sans souffrir de courses, elles sont assurées d’avoir de la hausse l’année suivante ; parce qu’aux départements les receveurs sont assez les maîtres, sous prétexte qu’ils sont garants du recouvrement. Ainsi il faut que toute l’année tous les collecteurs soient chaque jour sur pied ; et tel les fait venir cent fois en sa maison pour avoir le paiement de sa Taille, qui a de l’argent caché. Et, comme on s’est engagé de montrer que la collecte coûte plus au peuple que ce qui revient de la Taille au roi, attendu la manière dont les choses se font, on continuera le détail dont on vient de parler.\par
Lorsqu’après les injures et les imprécations par lesquelles les contribuables ont jeté une partie de leur bile et de leur colère, il faut enfin venir au paiement, voici comme les choses se traitent : les collecteurs n’oseraient trop pousser les Taillables, de peur de souffrir un pareil traitement à leur tour. Ainsi, bien qu’ils puissent exécuter eux-mêmes les meubles et les emporter faute de paiement, il faut néanmoins qu’ils aient souffert eux-mêmes force contraintes de la part des receveurs, avant que d’en venir à ces extrémités ; c’est-à-dire plusieurs courses d’huissiers et de sergents, lesquels il faut, d’abord qu’ils sont arrivés, régaler dans des cabarets, afin qu’ils ne fassent qu’une simple course et non une {\itshape exécution}, et leur donner de l’argent indépendamment de celui qu’il leur faut pour leur course, et auquel ils n’ont que la moindre part ; — tout cela pourtant dans les commencements, car dans les fins ce sont toutes exécutions.\par
On amène alors les bestiaux de la paroisse en général, sans s’informer si ceux à qui ils appartiennent en particulier ont payé tout à fait leur taille ou non, ce qui est fort indifférent. Il faut encore de l’argent à l’huissier afin qu’il n’amène point les bêtes saisies bien loin, et qu’il ne les fasse pas vendre sans délai ; et puis, quand l’année va expirer, il n’est plus question de courses ni d’exécutions, mais ce sont des emprisonnements ; et il faut encore de l’argent aux huissiers, afin qu’au lieu de mener les collecteurs dans les prisons, qui sont souvent éloignées, ils les mettent en arrêt dans une hôtellerie voisine, où ils vivent aux dépens de leurs confrères. Que si le geôlier les réclame, ou a mérité les bonnes grâces du receveur par son savoir-faire, il les faut mener en prison, où il coûte trois sous quatre deniers par tête chaque jour pour coucher sur la paille ; et il faut que leurs femmes ou leurs enfants, éloignés quelquefois de trois ou quatre lieues, leur portent à manger ; et comme c’est souvent dans les temps froids, et que les prisons de campagne sont mal conditionnées, ils reviennent presque toujours malades de fatigue et de misère. De plus, chaque fois que les collecteurs vont en recette, il ne faut pas oublier un présent à M. le receveur, des fruits du terroir, quoi qu’il puisse coûter ; autrement, quelque mal que l’on souffre, ce serait encore davantage. Enfin, considérant la manière dont la Taille se départit, s’impose et se paie, et comme la vengeance du trop à quoi l’on croit avoir été imposé se perpétue de père en fils ; il faut demeurer d’accord qu’elle est également la ruine des biens, des corps et des âmes.\par
On oubliait encore un article, qui est les procès qu’elle cause : il s’est trouvé des paroisses où, dans le premier mois de la Taille, il s’était donné jusqu’à cent exploits ; c’est-à-dire que deux cents personnes avaient été occupées à aller plaider l’une contre l’autre en des lieux éloignés, en quittant leur travail et leur commerce par une pure animosité, leur intérêt au fond n’étant pas le plus souvent d’un écu, pour lequel ils en perdent plus de cinquante.\par
Ainsi, toutes ces choses jointes ensemble, on répète encore que la moindre incommodité que la Taille apporte au peuple consiste dans les sommes qui en reviennent au roi ; et la perfection est que, tant ceux qui en sont accablés par l’injustice de leurs sommes, que ceux qui exemptent leurs terres, sont également ruinés ; parce que, outre la raison générale, que ceux qui peuvent aider à porter la Taille sont ruinés à chaque moment faute de protection, et surtout par la collecte lorsqu’ils y passent à leur tour, le nombre des taillables diminue tous les jours ; en sorte qu’il faut aujourd’hui paiera trente ce que l’on était soixante à payer autrefois. D’ailleurs, la consommation ne se fait point, et parce que l’on ruine les consommants, et parce que aussi ceux qui auraient le pouvoir n’oseraient, à cause de la conséquence et l’envie que cela leur attirerait dans la répartition. De manière que tous les biens étant diminués de moitié par cette seule raison et non par la quotité de l’impôt, les personnes qui s’exemptent ont bien plus perdu que les autres, y ayant une infinité de grandes recettes, comme de 20 à 30 000 livres par an, qui sont diminuées de moitié sans qu’on en puisse accuser la Taille, dont elles n’ont jamais rien payé. Cependant ces personnes, qui n’eussent pas voulu contribuer d’un vingtième pour un impôt général, et dont l’institution est d’être porté également par tout le monde à proportion de ses facultés, ne font nulle réflexion qu’elles sont punies de leur injustice par la perte de plus de la moitié de ces mêmes biens quelles voulaient exempter tout à fait : loin de là, ceci ne les empêche pas de persévérer dans la même conduite par ce raisonnement, qu’à moins que le contraire ne soit général, il ne produirait aucun effet à leur égard ; si bien que ce sera leur rendre un très grand service que de les obliger à faire prendre par leurs receveurs leur véritable part de la Taille. Et il n’y a pas de doute que la seule cause de la diminution étant ôtée, leurs terres ne reprennent leur ancien prix ; elles y gagneraient donc au quadruple, et le roi et le peuple de même, comme l’on montrera dans la troisième partie de ces Mémoires.
\subsection[{Chapitre VII.}]{Chapitre VII.}
\noindent Quoique le chapitre précédent n’ait que trop fait voir les sinistres effets de la Taille arbitraire, et du pouvoir où chacun est par son moyen de ruiner son ennemi ou celui à qui il porte envie lorsqu’il se trouve sans défense, cependant il ne sera pas hors de propos d’en faire encore remarquer quelques-uns qui, venant comme en sous-ordre, ne sont pas moins déplorables.\par
Premièrement, tous habitants de campagnes, taillables, ne doivent plus posséder aucun fonds, depuis que tous ceux qui en avaient de cette espèce les vendirent en 1648 et les années suivantes, parce que les Tailles ayant alors doublé, les riches commencèrent à faire pratiquer l’injustice dans la répartition, en la renvoyant presque tout entière sur les pauvres ; ce qui mit donc ceux-ci dans l’obligation et dans la nécessité de vendre tout ce qu’ils avaient de bien. Quoique l’augmentation des Tailles eût une cause très juste, qui était celle des biens tant en fonds qu’en industrie, qui avaient doublé le prix où ils étaient trente ans auparavant, on vit alors beaucoup de personnes de campagne vouloir payer autant de Taille comme elles avaient de revenu, et se restreindre à leur simple industrie pour vivre elles et leur famille, sans pouvoir être écoutées, ce qui se pratique encore aujourd’hui quand l’occasion s’en présente : — En sorte qu’il n’y a point d’autre ressource pour ces gens-là que de vendre leur bien à vil prix, le plus souvent au seigneur de la paroisse, qui, le réunissant à ses autres biens du même lieu, et le couvrant du commun manteau de sa protection, empêche que ses receveurs ne paient plus de Taille, pour cette nouvelle augmentation, qu’ils faisaient auparavant ; et cela retourne en pure perte sur toute la paroisse, et par contrecoup sur le seigneur, par les raisons qu’on a dites tant de fois. Ainsi les petits fonds ne pouvant plus être ni achetés ni possédés par des particuliers taillables, ils sont baillés dans l’occasion pour rien, faute de marchands, qui est une perte à la masse de l’État qui se communique insensiblement aux grandes terres, lesquelles autour de Paris comme ailleurs ne se vendent que la moitié de ce qu’elles faisaient autrefois : d’où suit encore la ruine d’une infinité de monde, parce que les hypothèques contractées sur l’ancien prix, comme les partages et autres semblables, qui se payaient aisément dans la première valeur des terres, ne pouvant plus être acquittées à cause du déchet, il en faut venir à des licitations où, la diminution, et les frais de justice et de déchet, emportant tout, les créanciers et les débiteurs se trouvent également ruinés. — L’autre pernicieux effet est qu’un particulier qui possède un petit fonds y applique ses soins et y fait des améliorations, soit à planter ou à engraisser les terres, bien plus considérables que lorsque ce même fonds est confondu dans une grande recette, où à peine le fait-on valoir la moitié, et rien du tout à l’égard de la Taille. Et cela est si véritable, qu’un fonds de quatre ou six arpents sera baillé aisément à 50 livres et paiera 20 livres de Taille ; et lorsque, par le sort commun, il vient aux mains du seigneur ou de quelque puissant, on ne le compte que sur le pied de la moitié, et il ne fait point augmenter la Taille du receveur. — Et enfin le troisième et dernier effet de cette incertitude d’impôt est que, comme il faut éviter toute montre de richesse par les raisons ci-devant traitées ; et que l’âme de l’agriculture et du labourage est l’engrais des terres, qu’on n’obtient pas sans bestiaux, on n’oserait presque en avoir la quantité nécessaire quand même on le pourrait, de peur de le payer au double par l’envie des voisins. Aussi est-il ordinaire de voir des paroisses où il y avait autrefois des 1 000 ou 1 200 bêtes à laine, n’en avoir pas le quart présentement ; ce qui oblige d’abandonner une partie des terres dont les fonds ne sont pas très bons naturellement, parce qu’ayant besoin d’améliorations, on ne peut ou on n’oserait les y faire ; ce qui est une perte générale pour l’État, qui n’a pas d’autres biens que la culture de ces mêmes terres.
\subsection[{Chapitre VIII.}]{Chapitre VIII.}
\noindent De si grands désordres auraient cessé il y a longtemps si personne n’avait intérêt à leur maintien. Mais, comme les receveurs des Tailles, tant généraux que particuliers, se trouvent dans cette situation, ils se sont toujours opposés indirectement au remède qu’on y a voulu apporter ; car si cette incertitude est le principe de tout le mal, c’est elle précisément qui fait une partie de leurs revenus et qui les fait agir de la sorte, en quoi ils se trouvent secondés par les Élus et les Cours des aides. — En effet, les receveurs particuliers, outre cet intérêt de frais et de courses d’huissiers et d’exécutions, dont on a parlé ci-dessus et dont ils ont une partie, et les présents que cela leur attire, en ont encore un, qui leur est commun avec les receveurs généraux, qui est la remise que le roi leur fait pour le recouvrement de la Taille, laquelle est présentement de 9 deniers pour livre, et qui était autrefois bien plus considérable, ayant été jusqu’à 6 sous pour livre. Le principe, la cause de cette remise, est la difficulté de faire le recouvrement de la Taille dans les temps qu’il est nécessaire de la fournir à S. M. On suppose donc que cette gratification leur est faite pour les dédommager des sommes qu’ils sont obligés d’avancer de leurs propres deniers, ce qu’ils ne font assurément point présentement ; mais, lorsque les particuliers taillables ne sont pas en état de s’acquitter, les collecteurs le font pour eux, ou il leur faudrait périr dans la prison. — De manière qu’anciennement, lorsque les Tailles se payaient aisément et à l’envi par les peuples, les receveurs, tant généraux que particuliers, n’avaient que leurs gages, qui sont très considérables. Mais ensuite l’injustice s’étant introduite avec la hausse dans la répartition des tailles, lorsqu’on accabla les pauvres pour soulager les riches, cela produisit la difficulté des paiements et l’occasion aux receveurs de demander des remises pour les dédommager de leurs avances. Ainsi il est de leur intérêt que la taille ait toujours une montre de difficulté de paiement, ce qui ne serait pas, étant justement répartie ; car bien loin de ruiner personne, dans ce cas, elle serait alors beaucoup au-dessous de ce qu’elle pourrait être, sans faire la moindre peine. — Il n’en faut point d’autre marque que les lieux taillables, comme les petites villes, qui ont obtenu du roi le pouvoir de mettre leur Taille en {\itshape tarif}, c’est-à-dire, au lieu d’une capitation très injuste et telle qu’on l’a décrite ci-devant, la faculté de la mettre sur les denrées qui se consomment sur le lieu, par où toute injustice est évitée. Car, bien que de cette manière elle double le prix précédent, parce que, outre qu’il faut que celui qui prend ce droit à ferme y gagne, et qu’il lui coûte des frais pour opérer ce recouvrement qui se fait aux portes, et qui nécessite des commis, c’est que cette permission, qui est très difficile à obtenir, ne s’accorde qu’à des conditions onéreuses, comme de faire quelque ouvrage considérable, outre le prix de la Taille, ainsi qu’à Honfleur et au Pont-Audemer, qui n’ont obtenu le tarif qu’à condition de bâtir chacun un port. Cependant, avec tout cela, cette concession n’a pas sitôt été faite, que ces lieux très misérables, où on laissait tomber les maisons, ont recouvré tout d’un coup la richesse et l’abondance, et l’on y a plus rebâti et réparé en quatre ans qu’on n’avait fait les trente années précédentes.\par
Ce qui est aisé à croire, puisque quoiqu’il se lève le double régulièrement de ce qui se payait au roi, toutefois, comme cela fait cesser tous les désordres dont on a parlé, le peuple y gagne vingt pour un. Mais il s’en faut bien que ce soit la même chose des receveurs ni des juges des Tailles. En effet, bien que par une maxime générale la campagne ne vaille qu’autant que les villes tirent et consomment, et que ceux qui se retirent des champs pour les habiter le fassent pour faire plus de consommation, on ne laisse pas de mettre toujours dans la concession des tarifs, que nul de la campagne ne se pourra retirer dans les dits lieux dont la taille est mise en tarif, pas même ceux qui, en étant originaires, n’en seraient sortis qu’un an auparavant ; et cela dans l’intérêt prétendu de la campagne, parce que, dit-on, les tarifs les ruinent. Mais ceux qui tiennent ce langage savent fort bien le contraire, et il ne faut, pour en demeurer d’accord, que comparer les lieux voisins de ceux qui sont en tarif, à ceux qui en sont éloignés. Cependant le manque de bonne foi sur cet article, dans les personnes intéressées, a été si loin, que l’on a vu des officiers de la cour des Aides rapporter à leurs confrères, qu’entre autres bonnes affaires qu’ils avaient faites pour le bien de la compagnie, ils avaient empêché plusieurs lieux qui demandaient cette concession, de l’obtenir, quoiqu’ils fissent des offres très avantageuses à Sa Majesté, offres qu’ils avaient fait rejeter par MM. les ministres, toujours en alléguant l’intérêt de la campagne. Ce qu’il y a d’épouvantable dans cette conduite, est que ces personnes, en agissant ainsi, causent au peuple mille fois plus de mal qu’elles ne se font de bien à elles-mêmes, et que ce mal finit encore par retomber sur elles si elles possèdent des fonds d’héritages, comme il est facile de s’en convaincre en réfléchissant sur le contenu de ces Mémoires. Ainsi, des lieux où il se ferait un très grand commerce, s’il ne leur était pas absolument défendu par la Taille arbitraire, sont contraints de demeurer dans la dernière misère, et ne peuvent obtenir une grâce qui semble être de droit naturel, qui est que tout débiteur se puisse libérer en la manière qui lui est plus commode, sans faire de tort à personne. C’est ce qu’on traitera plus amplement dans la suite en parlant de la facilité des remèdes du désordre.\par
On finit l’article de la Taille, dans lequel on croit avoir assez fait voir ce qu’on avait avancé d’abord, que la consommation était anéantie, parce qu’elle était absolument défendue par la manière dont la Taille est imposée et cueillie. Il reste à montrer que si la consommation est défendue, elle n’est pas moins impossible, par les raisons que l’on va dire. En sorte qu’on croirait que les désordres dont on vient de parler seraient sans exemple et plus que suffisants pour réduire les choses au point où elles sont aujourd’hui, c’est-à-dire à une perte de la moitié de tous les biens, sans que personne en ait profité ; si ceux qui vont suivre, dans ces Mémoires, n’étaient encore plus surprenants et plus ruineux, étant en quelque manière la cause des premiers, et le principe qui a contraint les peuples d’user d’injustice dans la répartition des tailles.
\subsection[{Chapitre IX.}]{Chapitre IX.}
\noindent Le meilleur terroir du monde ne diffère en rien du plus mauvais lorsqu’il n’est pas cultivé, comme il arrive à l’Espagne ; mais on peut dire en même temps que, quelque gras et quelque cultivé qu’il soit, lorsque la consommation des denrées qu’il produit ne se fait point, non seulement il n’est pas plus utile au propriétaire que s’il n’y croissait rien, mais même qu’il le met dans une plus mauvaise situation, parce que n’y ayant point de culture qui ne demande des frais, ces frais tournent en pure perte avec les fruits lorsque la consommation n’en a pas lieu. C’est là l’état où les Aides, et les Douanes sur les sorties et passages du royaume, ont réduit les meilleures contrées de la France, à tel point qu’on ne craint pas de dire qu’elles ont fait et font tous les jours vingt fois plus de tort aux biens en général qu’il n’en revient au roi ; ce qui se justifiera parfaitement par la description du détail de la perception de ces deux droits, et ne laissera qu’un étonnement que le mal ne soit plus grand encore, ayant des causes si pernicieuses. Mais, avant que de passer plus avant, on établit pour {\itshape principe}, que {\itshape consommation et revenu sont une seule et même chose} ; et que la ruine de la consommation est la ruine du revenu ; de manière, donc, que lorsque dans la suite on dira que tel impôt, ne rapportant au roi que 100 000 livres, diminue la consommation sur le prix ou sur la quantité de deux millions, cela signifiera réellement, et de fait, deux millions de diminution dans le revenu.\par
On parlera d’abord des Aides, et ensuite des Douanes sur les sorties.
\subsection[{Chapitre X.}]{Chapitre X.}
\noindent Ce qu’on appelle Aides est un droit qui se perçoit tant sur le vin qui se vend en détail que sur celui qui entre en des lieux clos. Il est fort ancien, et a succédé au vingtième, qui se prenait sur toutes sortes de denrées vendues par le propriétaire après sa provision prise ; et ce droit de vingtième avait succédé à la dîme royale de tous les fruits de la terre, qui faisait autrefois tout le revenu des princes, ayant été de tout temps la redevance la plus certaine de la royauté, car l’Écriture sainte et l’Histoire romaine font mention également que les rois la percevaient.\par
Ce droit d’Aide n’a pas toujours été égal, mais s’est perçu tantôt dans un pays sur le pied du 16\textsuperscript{e}, du 12\textsuperscript{e} et du 8\textsuperscript{e}, et tantôt dans un autre sur le pied du 4\textsuperscript{e} denier de la vente en détail des liqueurs, comme en Normandie, où il est partout à ce taux. À quoi, si l’on ajoute quelques nouveaux droits, tels que le quart en sus, le droit de jauge, cela va presque au tiers ; et comme le principal débit se fait dans les villes et lieux clos, les droits d’entrées pour le roi, pour les hôpitaux et pour les villes mêmes à cause des charges publiques, composent des sommes qui, jointes avec tous ces droits de débit, font un capital excédant de beaucoup le prix de la marchandise, surtout dans les petits crûs. Il s’est trouvé, en effet, des années où les droits ont été vingt fois plus forts dans le détail que le prix en gros de la denrée, ce qui anéantit si fort la consommation, qu’il faut que les pauvres ouvriers boivent de l’eau, les liqueurs dans le débit étant en un prix exorbitant ; ou qu’ils vendent leurs manufactures beaucoup plus chères, ce qui anéantit le commerce étranger, parce que les horsains, trouvant les marchandises trop chères, ont établi des manufactures dans d’autres royaumes où les ouvriers ont passé et passent tous les jours, ce qui se justifierait par une infinité d’exemples.\par
Ainsi, par une conséquence nécessaire, les fruits de la terre deviennent à rien, et l’on en abandonne absolument la culture. Il y a une infinité d’arpents de vignes vendus autrefois des mille livres, qui sont aujourd’hui laissés en friche, ce qui, après avoir ruiné les propriétaires et leurs créanciers, ruine ensuite, par le raisonnement traité dans la première partie, tous les revenus d’industrie, qui n’ont d’être et de mouvement qu’autant qu’ils en reçoivent des revenus en fonds, de sorte qu’une pareille diminution se multiplie dix fois sur tout le corps de l’État ; jusque-là que, bien qu’en Normandie le naturel du pays rende la plaidoirie la dernière chose susceptible des effets de la misère, cependant, aux lieux dont la principale richesse consistait en vins et en boissons, toutes les charges de judicature et leurs dépendances ne sont pas à la sixième partie de ce qu’elles étaient autrefois ; ce qui, diminuant encore la part que le roi prend dans ces sortes de fonctions, comme le papier timbré, les amendes et les contrôles d’exploits, amène à dire qu’il rachète au triple l’augmentation qu’on a prétendu lui procurer dans celle des droits d’Aides, qui sont presque seuls cause de la ruine générale.
\subsection[{Chapitre XI.}]{Chapitre XI.}
\noindent Les Aides, se recevant autrefois comme les Tailles et par les receveurs généraux, n’étaient point en parti, et le premier bail général qui s’en trouve est fait en 1604, pour 510 000 livres. Quoiqu’il fût pour dix ans, au bout de deux ou trois seulement, le fermier se fit bailler une hausse sous main, avec une prolongation de trois à quatre ans, ce qui ayant continué de la même manière, parce que ceux qui les tenaient trouvaient par ce jeu le moyen de dissimuler la trace de leurs profits, en moins de quinze ans la ferme monta à 1 400 000 livres ; et a si bien haussé par cette même méthode, que les Aides sont à 19 millions, ou environ, aujourd’hui.\par
On a fait ce détail pour établir deux choses, savoir : que, depuis 1604 jusqu’en 1619, les fermiers de ces droits gagnèrent des sommes exorbitantes ; et que depuis ce temps-là jusqu’en 1670, il n’y en a eu presque aucun qui n’ait profité considérablement, ce qui est la cause de tout le mal, parce que les hausses de baux n’étant point sans l’addition de quelques nouveaux droits, quoique ceux qui étaient établis produisissent déjà une grande diminution à la consommation, et par conséquent au revenu de la France, la quantité de fortunes que cela produisait (avec l’aide indispensable des hautes protections) ôtait toute espérance que le mal pût jamais recevoir de remède. Et ce qu’il y a de plus merveilleux est que, tandis que d’un côté l’on diminuait les Tailles, dont la quotité n’était point du tout la cause de la misère des peuples, on haussait les Aides, qui faisaient tout le désordre, et cela parce que la Taille n’est point un principe de grande fortune pour ceux qui s’en mêlent, et que les Aides, au contraire, ont toujours produit les étonnantes élévations que l’on a vues jusqu’à présent. En effet, les douze millions de diminution sur les Tailles depuis l’année 1651, ne sont justement que ce que les Aides ont souffert d’augmentation depuis cette même époque ; et ce qu’il y a de fâcheux, c’est que lorsque le produit des fermes n’a pu enrichir les fermiers d’une façon directe par la consommation ordinaire et qui se pouvait faire, ils ont eu recours à des moyens indirects que l’on ne pourrait pas croire si on ne les voyait tous les jours de ses yeux.
\subsection[{Chapitre XII.}]{Chapitre XII.}
\noindent Les droits des Aides ayant été mis sur un pied exorbitant, il a fallu de deux choses l’une : ou abandonner tout à fait le commerce des liqueurs en détail, ou tromper les fermiers sur la quantité du débit. On a fait l’un et l’autre en partie, c’est-à-dire que cette sorte de consommation a été réduite au quart de ce qu’elle était auparavant, ce qui est déjà une perte inestimable pour l’État ; et que, pour le peu que l’on n’a pu se dispenser de vendre, il a été nécessaire d’user de fraude, ce qui se fait par le moyen de caves inconnues dans lesquelles on dépose des liqueurs sous des noms empruntés, et d’où l’on tire la nuit pour remplir les futailles que l’on a déclarées en vente, ce qui en est sorti pendant le jour, à quelque chose près, sans quoi le cabaretier perdrait considérablement sur la marchandise, quand même il donnerait sa peine pour rien.\par
Et, comme il était impossible aux fermiers des Aides d’empêcher ce désordre par les voies ordinaires, en vérifiant la fraude par témoins, ils ont obtenu des édits et déclarations, qui portent que les procès-verbaux de leurs commis, quels qu’ils soient, feront foi dans tout leur énoncé ; et comme il ne s’en fait aucune enquête de vie et de mœurs lors de leur réception, et qu’ils ont d’ailleurs pour profit particulier le tiers des amendes et confiscations prononcées en conséquence de leurs procès-verbaux, ils sont absolument juges et parties, et ont en leur disposition les biens de tous les hôteliers de leurs districts ; et s’ils ne les font pas périr tous dès l’entrée de leur bail, c’est qu’il n’est de leur intérêt de le faire qu’à la fin. Mais ils usent d’une autre manière pour faire leur compte, également dommageable au corps de l’État, qui est que comme, par le moyen de leurs procès-verbaux, ils sont maîtres de tous les biens des hôteliers, ils ne souillent vendre qu’à ceux qu’il leur plaît, c’est-à-dire à ceux qui achètent des liqueurs d’eux seuls, à tel prix qu’ils y mettent, tous les commis en faisant marchandise, ce qui était anciennement défendu par les ordonnances. En outre, comme ils mettent à ces liqueurs un prix exorbitant, qu’ils les vendent trois fois ce qu’elles leur coûtent, il faut bien, pour que les hôteliers les puissent débiter d’une façon proportionnée, ce qui ne serait pas si chacun était en pouvoir ou de vendre ou de faire sa provision, qu’ils aient grand soin d’empêcher l’un et l’autre par les moyens que l’on vient de dire, et auxquels on va encore en ajouter d’autres.\par
Attendu qu’ils ne pourraient pas aisément avoir des commis dans tous les lieux écartés, pour tenir l’œil qu’il ne se fasse point de fraudes dans le débit, en visitant trois ou quatre fois le jour les caves, afin de voir de combien les futailles sont diminuées, ce qui consommerait tout le produit de la ferme, ils ont coutume de faire périr dans les lieux éloignés autant d’hôtelleries ou de cabarets qu’il s’en élève, ce qui a si bien banni cette sorte de consommation dans les campagnes, que lorsque ce n’est pas dans une grande route, on fait des sept à huit lieues de chemin sans trouver où apaiser sa soif ; de manière que tous les cabarets étant dans les villes et gros lieux, les commis sont maîtres de toute la consommation en détail, dont ils ne peuvent tirer aucune utilité en leur particulier, qu’en la réduisant à la sixième partie de ce qu’elle était autrefois, comme on peut dire qu’elle est aujourd’hui, non seulement à l’égard des hôteliers, mais même en ce qui regarde les particuliers.\par
En effet, comme il faut le plus souvent aller quérir le vin par charroi dans les lieux où on le récole, il y a des édits qui portent qu’il faudra faire des déclarations avant que d’entrer dans les lieux clos du passage et payer de certains droits, et à d’autres montrer seulement les congés de passer que l’on a pris au premier bureau ; et comme ce sont presque toujours les mêmes fermiers qui font valoir les droits, l’intérêt des commis étant que personne qu’eux ne fasse le commerce des vins, et qu’il y ait le moins de monde possible qui en fasse sa provision, afin de réduire dans la nécessité d’aller au cabaret, ils font les choses d’une manière que quand on a une fois fait cette route, il ne prend point d’envie d’y retourner. Car, premièrement, avant de se mettre en chemin, il faut aller faire sa déclaration au bureau prochain, prendre une attestation de la quantité de vin qu’on voiture ; et si l’on est éloigné du bureau, perdre une journée à attendre la commodité de M. le commis, qui n’est jamais le temps de l’arrivée des voituriers : ainsi il faut que ceux-ci jeûnent ou qu’ils aillent manger au cabaret. Ensuite, s’étant mis en chemin, il faut au premier lieu clos s’arrêter à la porte, pour aller pareillement porter sa déclaration, et voir si elle est conforme, et si les futailles sont de la jauge déclarée. M. le commis n’est souvent pas au logis, ou n’y veut être, ni le jaugeur non plus, pendant lequel temps il faut que les chevaux soient au vent et à la pluie, n’y ayant hôtelier assez hardi pour leur donner le couvert que le tout ne soit fait. Que si les jaugeurs ne se rapportent pas, comme cela peut arriver, il n’y va pas moins que de la confiscation de la marchandise et des chevaux ; ou bien il faut se racheter par une honnêteté à M. le commis, qui excède trois fois le profit, que l’on peut faire sur sa voiture. Que si encore les chevaux se sont déferrés en chemin, et qu’on n’ait pu atteindre le lieu de déclaration qu’un peu tard, on dit que l’on n’en reçoit point après soleil couché ; de sorte qu’il est nécessaire d’employer une fois plus de journées pour faire ce chemin, qu’il ne faudrait sans ce désordre. Et comme les hôtelleries sont d’une cherté effroyable, à cause du prix exorbitant des boissons, les hôteliers déclarant qu’à quelque prix qu’ils mettent le vin, ils y perdent encore, attendu les grands droits, et qu’ainsi il faut qu’ils se sauvent sur les autres denrées, qu’ils vendent quatre fois leur prix ordinaire ; il s’ensuit qu’une seule couchée dehors de plus emporte tout le profit, quand même tous les inconvénients qu’on vient de dire n’y seraient pas. De plus, comme il y a des droits à payer par {\itshape avance}, soit que le vin que l’on voiture se conserve ou se gâte, comme cela arrive fort souvent, cela retarde encore extrêmement cette sorte de commerce, et rompt celui qui se pouvait faire par échange de marchandise à marchandise, attendu qu’il faut de l’argent comptant. D’ailleurs, les droits se prenant sur tout le contenu en la futaille sans aucune déduction pour la lie, et ces droits étant ce qu’il y a de plus cher, puisqu’ils excèdent de beaucoup ce qui peut revenir au propriétaire ; pour les sauver en partie, on tire les liqueurs à clair, en sorte que n’étant plus nourries par leur lie, surtout les cidres en Normandie, elles s’aigrissent aisément et causent des maladies à ceux qui sont dans la nécessité d’en boire, comme font tous les pauvres ; outre que cela diminue encore extrêmement cette sorte de consommation.
\subsection[{Chapitre XIII.}]{Chapitre XIII.}
\noindent Quelque évident que soit tout ce qu’on a dit dans le chapitre précédent, pour peu que l’on ait l’usage du monde, il ne sera pas néanmoins mal à propos de le fortifier de quelques preuves nouvelles, afin de montrer jusqu’à quel point les Aides ont poussé cet intérêt de ruiner la consommation et par conséquent le pays, pour une utilité particulière qui ne va pas à la millième partie du mal qu’elles font au corps de l’État ; qui est la source générale dont le roi tire tous ses revenus.\par
Bien que la Normandie, généralement parlant, ne soit pas un pays de vins, cependant le voisinage de la mer du Nord, où il est tout à fait inconnu, fait que le peu qui y croît, ou qui y croissait, les trois quarts des vignes ayant été arrachées depuis trente ans, se vendait parfaitement bien ; et c’est dans ce même canton qu’il y a eu des arpents de vignes vendus des mille livres (ainsi que l’on a dit), et depuis entièrement abandonnés, le terroir ordinairement caillouteux n’étant bon à rien, après que la vigne est arrachée : c’est tout le canton qui se trouve depuis Mantes jusqu’à Pont-de-l’Arche, qui pouvait faire autrefois environ 20 000 arpents en vignes seulement. Bien que ce soit un fort petit crû, eu égard aux vins de Champagne, et même de ceux qui sont au-dessus de Mantes, cependant c’était un revenu très certain pour les propriétaires, qui prenaient très grand soin à faire ménager leurs vignes, y ayant différence de plus de moitié entre les bien accommoder ou les négliger. Mais depuis qu’on a mis le droit de sept francs par muid sur les vins de toute espèce qui passeraient les rivières d’Eure, Seine, Andelle et Iton, pour aller aux provinces de Normandie et Picardie où il n’en croît point, cet établissement, qui n’eut (à ce que porte la tradition) depuis trente ans qu’un principe d’intérêt particulier, comme de faire valoir quelques cantons de la Champagne, en mettant la Picardie dans l’obligation de ne se fournir de vins que dans cette province, coûte, depuis ce temps-là, plus de 15 millions par an aux provinces de Picardie, Normandie et Île-de-France ; et à l’égard du roi, pour 80 000 liv. que cela lui porte, qu’on est bien assuré qu’il ne voudrait pas avoir à ce prix, quand même son intérêt ne se rencontrerait pas contraire, on a été dans l’obligation de diminuer les Tailles de\par
150 000 livres sur la seule élection de Mantes ; et ce qui en reste est payé avec bien plus de difficulté que n’était le total autrefois, sans qu’on en puisse coter d’autres raisons que la naissance de ce droit. En effet, depuis ce temps, les vignes sont venues en non-valeur ; et ç’a été un très bon ménage en quantité d’endroits de les arracher, puisqu’après avoir fait les frais de la culture et de la récolte, et que les vignerons s’étaient endettés pour ce sujet, on avait le malheur de voir gâter le vin dans les caves sans en pouvoir trouver le débit, par les raisons traitées ci-dessus. En sorte qu’on montrera des procès dans lesquels des marchands de futailles, les ayant vendues à crédit avant la récolte, n’ont pas voulu pour leur paiement les reprendre avec le vin dont elles étaient remplies, dont néanmoins on ne leur demandait rien, quoique ce même vin à dix ou douze lieues de là valût un prix exorbitant. Mais, par les circonstances traitées ci-dessus, il y a moins à perdre le vin qu’à risquer des charrettes et des chevaux, en entreprenant d’en faire le transport ; et le grand préjudice qu’une pareille disposition fait au corps de l’État, est que ces mêmes pays où le vin est si cher, parce que l’on n’y en récolte point et qu’on n’ose y en mener, ne sauraient plus se défaire des denrées qu’ils donnaient en échange, comme les salines et les avoines également rares dans les pays vignobles, lesquelles étaient enlevées par les mêmes voitures qui amenaient les vins, ce qui faisait un commerce fort considérable, et enrichissait les uns et les autres. Au lieu qu’il faut présentement que la plupart des terres des pays vignobles demeurent à labourer ; qu’on y manque d’avoine parce qu’elle est très chère ; et que les contrées maritimes se perdent entièrement, parce que les grains pèsent trop eu égard au prix, qui ne peut plus couvrir les frais de voiturage par terre, les hôtelleries étant aussi chères qu’elles sont, et étant impossible de rapporter du vin comme on faisait autrefois. Ainsi chaque contrée périt, faute de pouvoir échanger les denrées qu’elle recueille contre celles qu’elle ne produit pas, ce qui prouve évidemment que la consommation est devenue impossible.
\subsection[{Chapitre XIV.}]{Chapitre XIV.}
\noindent Bien que ce désordre des Aides ne soit pas en un si haut point dans toute la France, cependant, outre qu’il y a peu de contrées qui en soient tout à fait exemptes, on peut dire qu’il suffit qu’une diminution considérable se fasse ressentir sur quelque partie des denrées que ce soit, pour communiquer ce mal à toutes les espèces, par une participation nécessaire de cherté ou d’avilissement de prix que toutes les marchandises de même sorte ont les unes avec les autres à l’égard du prix du marchand, surtout dans un même État. C’est ainsi, par exemple, qu’il suffit qu’il se rencontre deux sacs de blé plus qu’il ne faut pour la consommation ordinaire, et que le marchand est obligé de vendre à quelque prix que ce soit, pour apporter une extrême diminution au prix des blés dans un marché ; et s’il en arrive de même dans les marchés suivants, ce mal va toujours en augmentant ; et après s’être communiqué à la contrée, il gagne les pays les plus éloignés. Le vin, qui se consommait autrefois par le transport qui s’en faisait aux pays où il manquait, et les autres marchandises qu’on en rapportait en contre-échange, pour faire au moins valoir la voiture du retour, ne pouvant plus passer, par les raisons traitées ci-dessus, non seulement deviennent en pure perte à leurs propriétaires respectifs, mais deviennent encore la cause de la ruine des autres propriétaires (qui les eussent pu faire consommer sur le lieu), parce que le prix en étant avili par cette grosse abondance, il ne peut pas même suffire pour les frais des façons, qui sont toujours les mêmes, comme les journées d’ouvriers, gages des valets, qui ne baissent jamais lorsqu’ils ont une fois gagné un prix certain, attendu qu’il y a une espèce de pacte tacite parmi ces sortes de gens, d’aimer mieux mendier ou jeûner, que de rien rabattre de leur prix ordinaire ; fière prétention que l’abondance est très propre à maintenir, parce que l’avilissement des denrées leur fait gagner en une journée ou deux leur nourriture de toute la semaine, et qu’ils tirent de là avantage pour contraindre leurs maîtres de ne leur rien diminuer, dans la nécessité où sont ceux-ci de tout abandonner ou de faire faire leur besogne à quelque prix que ce soit. De là, donc, la ruine des fermiers des terres, qui entraîne celle de leurs maîtres et de leurs créanciers, par une gradation qui va jusqu’à l’infini, et qui doit tout sou principe à la cessation de la consommation ; en sorte que les terres, venant à être licitées, sont données presque pour rien, ce qui se communique aux autres provinces et fait qu’en Bretagne, où ce désordre d’Aides et de Tailles est inconnu, les terres ne laissent pas d’être diminuées de la moitié de leur ancien prix, par la contagion de la proximité de la Normandie. Et il en va de même à plus forte raison des autres provinces qui ne jouissent pas de si grands privilèges que la Bretagne. Cependant, c’est un si grand coup d’État de ne laisser pas baisser le prix une fois contracté par les marchandises, que les Hollandais, à qui la pratique a appris tout ce qui se pouvait sur le commerce, bien loin de les avilir pour tout un État, par un intérêt particulier, ont soin, au contraire, lorsqu’il s’en rencontre trop, comme du poivre, parce que l’année a été trop abondante, ou que la consommation n’a pas répondu, de jeter ces denrées à la mer : — par ce premier principe, que, pour conserver l’harmonie d’un État, il faut que toutes ses parties contribuent à sa richesse ; ce qui ne se peut dès lors que les proportions sont dérangées, et ce qui arrive dans la situation dont on vient de parler.
\subsection[{Chapitre XV.}]{Chapitre XV.}
\noindent Il reste à traiter des Douanes qui se paient sur ce qui sort le royaume, qui causent à peu près les mêmes effets que les Aides, avec cette différence que les désordres en sont d’autant plus déplorables, qu’au lieu que le plus grand mal des Aides tombe sur le dedans du royaume, ce qui est aisé à rétablir quand on voudra ne pas sacrifier l’intérêt général à celui de quelques particuliers ; le désordre des Douanes, au contraire, en diminuant absolument le revenu du roi, a banni les étrangers de nos ports, et les a obligés d’aller chercher dans d’autres pays, à meilleur compte, des denrées qu’ils venaient autrefois quérir chez nous ; et cela, pour enrichir les commis et directeurs de ces droits, les principaux fermiers y perdant aussi bien que le roi ; — en sorte qu’un si petit intérêt a causé tous les désordres que souffre un État qui ne trouve plus le débit de ses marchandises.\par
On appelle communément Douane le droit qui se tire des denrées qui s’enlèvent hors le royaume, ou qui sont apportées du dehors, ou même de celles qui ne font que passer d’une province à l’autre, quoique souvent le chemin qu’elles font ne soit que très peu considérable. Tant qu’elles ont été modérées, elles n’ont fait aucun désordre ; mais aussitôt qu’elles ont été portées à un prix exorbitant, elles ont été également dommageables et au roi et à l’État, puisqu’elles ont banni tout commerce étranger ; les peuples du dehors ayant été contraints d’apprendre nos manufactures en attirant nos ouvriers, et d’aller chercher à meilleur compte nos denrées naturelles, comme nos blés et nos vins, en d’autres pays qui se sont enrichis à nos dépens et ont appris à devenir bons ménagers depuis que nous avons cessé de l’être. Et il semble pourtant qu’on aurait dû éviter ce désordre encore plus que tous les autres, après ce qui était arrivé du temps d’Henri IV au sujet des Douanes, dont le récit, qui se trouve dans un historien contemporain, prouve plus que tout ce qu’on pourrait rapporter sur ce sujet. — À la paix de Vervins, bien qu’un des articles du traité portât que les droits d’entrée et de sortie des marchandises dans les États des rois de France et d’Espagne demeureraient dans la situation où ils avaient toujours été, sans pouvoir être haussés réciproquement ; cependant Philippe III, nouvellement arrivé à la couronne, étant peut-être mécontent de la paix, voulut y donner atteinte par quelque infraction : il haussa dans ses ports extrêmement tous les droits d’entrée et de sortie, et la France en ayant fait autant, comme par représailles, bien qu’on n’eût point augmenté le prix de la ferme, les fermiers firent banqueroute entièrement, et ne purent satisfaire à leur bail, à cause de la grande diminution que cela apporta à la consommation et au commerce. Et il n’y a pas longtemps que la même chose arriva en une ville de France, où l’impôt sur l’enlèvement des eaux-de-vie pour l’Angleterre étant excessif, celui qui avait sous-fermé les Aides de cette ville (comme cela arrive quelquefois) n’eut aucun produit de cet article la première année de son bail, à cause du prix exorbitant, parce que les étrangers prirent un autre style, qui était d’envoyer de très petites barques au bas des rochers de la côte, au haut desquels les pauvres gens transportaient de nuit des barriques d’eau-de-vie, et puis avec des cordes les descendaient dans ces barques, en sorte que le fermier n’en recevait rien du tout. Pour parer à cet inconvénient, il fit savoir l’année suivante qu’il se contenterait de la moitié du droit permis par son bail, ce qui lui fit un profit considérable et remit l’abondance dans le pays, le commerce n’étant jamais le même, lorsqu’il se conduit en cachette, comme quand il se fait ouvertement.\par
Mais pour venir davantage aux causes du désordre, il faut descendre au détail. — Tous les édits faits au sujet des Douanes et passages portent, par un style général, obligation de déclarer, avant l’ouverture des ballots, la qualité, quantité, poids, mesure et diversité des marchandises que l’on veut transporter, ou qui arrivent, le tout à peine de confiscation et de grosses amendes. Si, après l’ouverture, la vérification qui s’en fait ne se trouve conforme à la déclaration qui a été mise par écrit, article par article, le tout est confisqué, sans qu’on soit reçu, pour éviter cet inconvénient, d’abandonner la marchandise à la visite, pour payer tels droits qu’on voudra demander ; et ces confiscations se partagent en trois parts, savoir : le tiers aux moindres commis qui agissent à la garde, le tiers au directeur ou receveur, et le troisième tiers au fermier, avec cette différence que ce dernier est à la discrétion du directeur, qui se met peu en peine de lui, pourvu qu’il fasse sa fortune, qui lui est immanquable du moment que les droits de Douane sont en un point si exorbitant que toute la consommation et le commerce en soient ruinés. Car, si ce qu’on paie sur les denrées est une chose aisée qui n’interrompe point le trafic, et par conséquent la richesse du pays, le roi en tire à la vérité bien davantage de cette sorte ; mais jamais le directeur ne fera de fortune, ni tous ceux qui sont employés à la levée de cet impôt. C’est ce qu’on va faire voir par des faits si certains et si constants, qu’il sera impossible de ne pas convenir de cette vérité ; mais auparavant on dira que ces places de receveurs ou directeurs sont les premières commissions, que les princes ne méprisent pas de demander pour leurs créatures, en sorte que ce sont gens d’une haute protection ; et lorsque la main dont ils tiennent leurs emplois n’est pas publiquement visible, c’est marque qu’ils ne prêtent que leur ministère à d’autres personnes puissantes qui en tirent ce qu’il y a de plus utile. Il est encore à remarquer que ceux qui nomment à ces conditions, pour faire valoir l’obligation qu’ils veulent qu’on leur en ait, disent une chose assez extravagante, si tout le monde n’en était témoin, qui est que cet emploi rapportera 5 ou 6 000 liv. de rente, quoique les gages ne soient bien souvent que de 1 200 liv. ; sur quoi il faut payer le bureau, les lettres et autres menus frais. C’est par où ceux de ces commis qui ont quelque conscience sauvent leur scrupule, en prétendant recevoir par là une permission tacite de tromper le roi, le public et leurs maîtres.
\subsection[{Chapitre XVI.}]{Chapitre XVI.}
\noindent Les droits de Douane, principalement sur les sorties du royaume, étant une fois mis sur un pied exorbitant, après que le commerce des denrées qui se transportent en est extrêmement diminué, la partie qui reste ne peut subsister que de la manière que l’on va dire : ou il faut frauder tout à fait la Douane, par des transports secrets pendant la nuit, ou s’accommoder avec les directeurs pour tromper les maîtres. Dans l’un et l’autre cas, les premiers font leur compte ; car, si on hasarde en tâchant de frauder (comme il est impossible de n’être pas quelquefois pris), de plein droit appartient le tiers de la confiscation aux directeurs. Mais bien souvent ils ne font point éclater la chose, et traitent de la part de leurs maîtres, le marchand y gagnant encore assez, quand il la perdrait tout entière, de sauver les autres suites d’une confiscation. L’autre manière leur est pour le moins aussi avantageuse, qui est de s’adresser d’abord à eux, et de traiter de bonne foi de la remise qu’ils veulent faire, moyennant une honnêteté à leur profit des droits de leurs maîtres, et par conséquent du roi, en quoi ils se montrent honnêtes gens, et de bonne composition. — Ainsi, d’une manière ou d’autre, il faut que les droits soient grands ; c’est à quoi leurs protecteurs ont soin de veiller, et de faire périr plutôt tout un pays, que de souffrir les Douanes à un point que les marchandises les puissent supporter, sans obliger de recourir à un de ces deux expédients. Et, dans la crainte que l’excès des droits ne suffît pas pour arriver à leurs fins, ils ont surpris des édits de MM. les ministres, qui mettent les biens du marchand à leur discrétion, puisque, bien que par toutes les lois du monde ce soit au demandeur à établir sa demande, dans la Douane c’est tout le contraire, ainsi qu’on a montré au chapitre précédent. Le marchand doit enseigner aux receveurs ce qu’il leur faut, article par article, et tout ce que doit rédiger par écrit une partie qui a intérêt qu’on se méprenne. Que si cela arrive par mégarde, étant presque impossible que cela soit autrement, ils disent pour raisons d’un procédé si injuste, que s’ils se méprenaient on ne les redresserait point. — Mais pour montrer que c’est un piège qu’ils veulent tendre, en faisant naître un procès où ils sont juges et parties, il ne faut que répondre que c’est à eux à savoir leurs édits et leurs attributions, et par conséquent ce qui leur appartient, et non pas au marchand, qui n’en peut rien apprendre que par eux. — En second lieu, s’ils appréhendaient si fort de se méprendre, ils n’ont qu’à faire comme tous les vendeurs, à demander beaucoup plus qu’il ne faut ; assurément, le marchand les redressera, ou ils n’y perdront pas. Mais, de vouloir faire établir une diminution par le défendeur qui la doit moins savoir, sous peine de tout perdre s’il se méprend, au lieu que l’erreur dans le demandeur ne serait que très peu de chose, supposé même qu’il s’y en rencontrât ; c’est la dernière des injustices, qui n’a d’exemple que dans l’inquisition d’Espagne, qui passe pour le tribunal le plus violent du monde.\par
On passe sous silence les autres manières qu’ils apportent pour fatiguer les marchands, étant quelquefois six ou sept jours sans trouver le temps de recevoir la livraison des marchandises, soit pour tirer une contribution de leur diligence, ou même, quoiqu’ils aient déjà été salariés, pour apporter du retardement au transport. De quelque manière que les choses se passent, on n’en peut avoir aucune justice, parce qu’ayant de fortes protections, ils ne reconnaissent aucuns des juges ordinaires, mais en ont de particuliers qu’ils nomment eux-mêmes : c’est de cette sorte que les directeurs des Douanes se sont enrichis, à mesure que le commerce, tant du dedans que du dehors du royaume, s’est diminué ; les mêmes désordres se pratiquent dans le transport des marchandises tant d’une province à l’autre, qu’au sortir du royaume.
\subsection[{Chapitre XVII.}]{Chapitre XVII.}
\noindent Il s’enlevait autrefois une quantité de blés en France, surtout en Normandie, pour les pays qui en manquaient ; et comme elle en produit plus (étant bien cultivée) qu’elle n’en peut consommer, elle est ruinée du moment que le transport ne s’en fait plus. C’est ce qui est arrivé par l’impôt de 66 livres sur chaque muid qui sortait du royaume : de sorte que les étrangers sont allés s’en pourvoir à Dantzick et à Hambourg ; et la trop grande quantité qui en est demeurée dans le pays a fait cesser de labourer les médiocres terres, et négliger en plusieurs endroits les meilleures ; et par ce moyen mettre une famine à l’argent, non moins préjudiciable au corps de l’État que celle qui arrive au blé. Car, comme quand cela advient, c’est que la proportion étant ôtée entre ce qu’on veut avoir, qui est le blé, et ce qu’on baille en contre-échange, qui est l’argent, tout le commerce demeure ; le même désordre se rencontre lorsque, les blés étant à vil prix, il en faut beaucoup plus pour avoir de l’argent : — : ce qui produit le même effet à l’égard de la république, qui, ne pouvant s’entretenir que par un commerce et une circulation continuelle, où les proportions sont absolument nécessaires, tout cesse en même temps qu’elles ne se rencontrent plus, quoi que ce soit qui en soit cause. De manière que, comme au Pérou on meurt de faim au milieu de l’argent, on est très misérable en France dans l’abondance de toutes les choses nécessaires à la vie. Et ce qui est plus déplorable, c’est que ces malheurs, qui arrivent souvent ailleurs par nécessité, ne se trouvent en France que par une forte méprise, ou plutôt par des intérêts indirects, dont il ne revient rien au roi ; outre que les années stériles ne pouvant être secourues par les abondantes, qui ne sont plus d’un rapport à l’accoutumé, on a vu, depuis trente ans, le blé hors de raison, ce qui faisait périr les pauvres ; ou à vil prix, ce qui ruinait également et les riches et les pauvres : ces premiers ne pouvant fournir de travail à ceux-ci, qui ne peuvent cependant subsister que de ce seul revenu. On ne doit donc pas objecter que cette obligation de laisser les grains dans un pays soit un remède certain contre la famine, puisque, outre que l’expérience a fait voir le contraire, les blés ayant été à un prix excessif {\itshape quatre fois} depuis trente ans, au lieu que dans l’espace de cent ans auparavant la même chose n’était pas arrivée ; c’est qu’une année stérile n’est jamais guère secourue que par la précédente, ou au plus par celle d’auparavant, les blés en France n’étant pas, en général, gardés plus longtemps, et le surplus étant consommé à vil prix par des engrais, ou par l’impatience des maîtres qui veulent être payés de leurs fermiers, ou parce qu’on n’a pas de lieu propre pour les garder et remuer souvent comme il serait nécessaire ; et bien loin qu’un impôt qui a causé une ruine si générale ait apporté quelque utilité au roi, c’est tout le contraire, puisque n’en ayant jamais reçu un sou, il a perdu les droits d’entrée sur les marchandises que les étrangers apportaient en venant quérir nos blés.\par
Il y avait autrefois une fort bonne manufacture de chapeaux fins en Normandie, qui valait une très grande somme au roi, soit pour droits d’entrée des matières qui venaient du dehors, ou pour la sortie lorsqu’elles étaient ouvragées : on doubla ces droits, et aussitôt les ouvriers passèrent aux pays étrangers, où ayant établi des manufactures de chapeaux fins, à eux jusqu’alors inconnues, les droits du roi furent réduits à la sixième partie de ce qu’ils étaient auparavant.\par
Les cartes à jouer se fabriquaient en France, surtout à Rouen, pour toute l’Europe, et même pour tout le Nouveau-Monde des Espagnols : un impôt de rien, qui servait seulement d’occasion aux directeurs de fatiguer les marchands, a fait pareillement transporter cette manufacture en une infinité d’endroits.\par
Le papier s’enlevait pareillement en une très grande quantité, et il a reçu le même sort, des mêmes causes.\par
Les pipes de tabac, qui se fabriquaient en quantité, ont pris la même route par de pareilles raisons.\par
Les baleines à accommoder les habillements ont été longtemps uniquement apprêtées à Rouen pour toute la terre où l’on en use ; et comme les Douanes pour l’entrée de la matière haussaient à tous moments, pour les éviter on faisait faire à cette sorte de marchandise 4 ou 500 lieues dans les terres plus qu’il n’eût été nécessaire, afin d’esquiver les entrées de Rouen. Mais enfin la subtilité de MM. les directeurs, en donnant leurs avis propres à ruiner tout pour s’enrichir, a triomphé de celle des commerçants, de sorte qu’ils ont surpris tant d’édits de MM. les ministres, qu’ils ont contraint ce trafic de prendre le chemin des autres ; et on ajoutera en faveur de ceux qui leur donnaient leur protection, qu’on est fort persuadé qu’il s’en fallait beaucoup qu’ils sussent au juste ce qu’elle devait coûter au roi et au peuple.\par
Les vins se levaient aussi en quantité aux foires de Rouen pour les pays étrangers, qui fournissaient au roi des sommes considérables pouf la sortie même des moindres crûs : on a haussé l’impôt, et ces mêmes étrangers ont été s’en fournir ailleurs.\par
En effet, ce qui coûte pour la sortie des plus petits vins allant à 25 livres par muid, qui n’est pas souvent vendu 20 livres sur le lieu distant d’une journée ou deux, il n’est pas étonnant qu’un pareil droit en ait entièrement anéanti le commerce ; et ce qu’il y a de merveilleux est que, pendant qu’on haussait tous ces droits, qui ruinaient également et le roi et les particuliers, sans que la découverte de l’erreur en l’un pût faire changer de conduite à l’égard des autres, on diminuait les Tailles de trois fois plus que n’étaient ces impôts, bien que ce ne fût pas la quantité des Tailles qui incommodât les peuples, ainsi qu’on a dit, et que l’on fera encore remarquer davantage lorsqu’on parlera des remèdes.
\subsection[{Chapitre XVIII.}]{Chapitre XVIII.}
\noindent On est persuadé que la simple narration de tous ces faits aura amplement satisfait à l’obligation contractée au commencement de ces Mémoires, de découvrir la cause de la grande diminution des revenus de la France, sans que l’augmentation de ceux du roi y ait aucune part, ni qu’on puisse en accuser le manque des espèces d’or et d’argent, qui sont en bien plus grande abondance dans le royaume que lorsque les revenus en étaient plus considérables. Et, quoique cette vérité soit très constante, comme elle pourrait passer pour paradoxe à l’égard de ceux qui ont accoutumé de dire, lorsqu’ils voient l’opulence diminuer dans un pays, qu’il n’y a plus d’argent ; il est à propos, pour l’éclaircissement de ces Mémoires, de dire un mot de la nature et des qualités de l’or et de l’argent, tant monnayés qu’en essence, et de faire connaître quel rang l’argent tient dans le monde.\par
Il est très certain qu’il n’est point un bien de lui-même, et que la quantité ne fait rien pour l’opulence d’un pays en général, pourvu qu’il y en ait assez pour soutenir les prix contractés par les denrées nécessaires à la vie ; de façon qu’il ne peut empêcher les lieux d’où on le tire d’être très misérables, et qu’un homme qui a deux écus, en ces contrées-là, à dépenser par jour, passe sa vie avec plus de peine qu’un autre qui, étant en Languedoc, n’a que six sous pour son entretien : et même on peut dire que plus un pays est riche, plus il est en état de se passer d’espèces, puisque alors il y a plus de monde à l’égard de qui elles peuvent être représentées par un morceau de papier sous le nom de billets de change.\par
L’argent est donc un gage incorruptible que tous les hommes sont convenus de se bailler, et de se prendre les uns des autres réciproquement sur le pied courant, afin de se procurer pour autant de denrées dont ils ont besoin ; parce que celui qui reçoit l’argent est certain qu’il produira le même effet, à son égard, pour les choses dont il a besoin ; personne au monde ne le recevant pour le consommer ou en faire magasin, à moins que ce ne soit pour en attendre une plus grande quantité, et en produire un plus grand effet tout à la fois. De manière que si toutes les denrées nécessaires à la vie avaient, comme l’argent, un prix certain, et que le temps ne les altérât pas, ou que les divers degrés plus ou moins considérables de perfection qu’elles ont chacune en particulier n’en dérobassent pas la véritable estimation, si bien qu’elles eussent un prix courant toutes les fois qu’on aurait besoin de s’en servir, on pourrait dire que l’or et l’argent ne seraient pas plus recherchés que tous les autres métaux les plus communs, et qu’ils leur céderaient même, étant moins propres aux autres usages de la vie ; parce que l’échange se ferait immédiatement comme il se faisait au commencement du monde, et qu’il se fait encore à l’égard de quelques marchandises en gros après qu’elles sont appréciées.\par
De ces principes il s’ensuit la conséquence, que dans la richesse, qui n’est autre chose que le pouvoir de se procurer l’entretien commode de la vie, tant pour le nécessaire que pour le superflu (étant indifférent au bout de l’année, à celui qui l’a passée dans l’abondance, de songer s’il s’est procuré ses commodités avec peu ou beaucoup d’argent), l’argent n’est que le moyen et l’acheminement, au lieu que les denrées utiles à la vie sont la fin et le but ; et qu’ainsi un pays peut être riche sans beaucoup d’argent, et celui qui n’a que de l’argent, très misérable, s’il ne le peut échanger que difficilement avec ces mêmes denrées. De manière que les flottes d’Espagne ne sont pas sitôt venues en Europe, qu’il faut porter presque tout l’argent aux pays d’où on a tiré les denrées pour les porter en celui où les mines sont situées ; et cet argent, y étant arrivé, produit par une révolution continuelle les mêmes effets qu’il a produits dans sa naissance, faisant plus ou moins de tours et retours qu’il change plus ou moins souvent de maître, c’est-à-dire qu’il se fait plus ou moins de commerce ou de consommation. Mais les pays comme la France, qui produisent les denrées nécessaires à la vie, ont cet avantage sur ceux d’où on tire l’argent, que l’échange se fait d’une manière bien avantageuse, attendu que l’argent ne se consommant point par l’usage, produit des utilités sans bornes et sans fin aux pays où on le porte ; tandis que les denrées que l’on donne en contre-échange ne sont utiles qu’une seule fois, périssant par l’usage. Et pendant que l’argent a une qualité d’être inaltérable par le temps et les accidents, il a en même temps celle de ne point augmenter par la garde, comme les autres marchandises ; et quand il produit de l’utilité, ce n’est point dans le coffre, mais en le gardant le moins qu’il est possible ; et comme c’est la consommation, dont il n’est que l’esclave, qui mène sa marche, du moment qu’elle cesse, il s’arrête aussitôt, et demeure comme immobile dans les mains où il se trouve lorsque le désordre commence à se faire sentir. De façon que, si la plus mauvaise situation d’un marchand, lorsque le commerce va, est d’avoir son argent inutile dans son coffre, parce qu’il ne lui produit rien, c’est son avantage, lorsqu’il ne va pas, qu’il ne soit pas dehors, attendu que s’il ne gagne rien, il ne perd rien ; ce qu’il courrait risque de faire par les banqueroutes, inséparables de la cessation du commerce. — Et ce qui est dit du marchand l’est également de toutes les personnes qui vivent de leurs rentes, soit en fonds de terre ou rentes constituées, lesquelles, recevant des racquits, ne les peuvent reconstituer faute de sûreté, parce que les affectations les plus ordinaires étant sur les terres, le produit en diminue tous les jours à vue d’œil par l’anéantissement de la consommation : aussi elles aiment mieux perdre l’intérêt que de hasarder le capital, se réduisant à faire moins de dépense, ce qui est un surcroît de mal pour le corps de la république. De façon que tous les revenus d’industrie cessent tout à fait, et l’argent, qui forme pour autant de revenu qu’il fait de pas, ne sortant point des fortes mains, arrête entièrement son cours ordinaire ; ce qui met le pays dans une paralysie de tous ses membres, et fait qu’un État est misérable au milieu de l’abondance de toutes sortes de biens. Ce sont là des effets que les pauvres ressentent les premiers, mais qui se communiquent ensuite imperceptiblement à tous les autres membres de l’État, même aux plus relevés, ainsi que l’on a fait voir par ces Mémoires ; ce qui devrait bien les intéresser aux moyens d’arrêter un si grand désordre, où le roi participe assurément à proportion du rang qu’il tient dans l’État.
\subsection[{Chapitre XIX.}]{Chapitre XIX.}
\noindent Il est aisé de voir, par tout ce qu’on vient de dire, que pour faire beaucoup de revenu dans un pays riche en denrées, il n’est pas nécessaire qu’il y ait beaucoup d’argent, mais seulement beaucoup de consommation, un million faisant plus d’effet de cette sorte que dix millions lorsqu’il n’y a point de consommation ; parce que ce million se renouvelle mille fois, et fera pour autant de revenu à chaque pas, tandis que les dix millions restés dans un coffre ne sont pas plus utiles à un État que si c’étaient des pierres ; et ce qui fait plus de mal au corps de la France, est que c’est le menu peuple sur qui le désordre des Tailles et l’excès du prix des liqueurs en détail agissent davantage, parce que c’est lui qui a le moins de défense et qui fait le moins de provisions, et cependant c’est lui en même temps qui fait le plus de consommation, parce qu’il est en plus grand nombre. — En effet, un journalier n’a pas plutôt reçu le prix de sa journée, qu’il va boire une pinte de vin, étant à un prix raisonnable ; le cabaretier en vendant son vin en rachète du fermier ou du vigneron ; le vigneron en paie son maître, qui fait travailler l’ouvrier, et satisfait sa passion ou à bâtir, ou à acheter des charges, ou à consommer de quelque manière que ce puisse être, à proportion qu’il est payé de ceux qui font valoir ses fonds. Que si ce même vin, qui valait 4 sous la mesure, vient tout d’un coup, par une augmentation d’impôt, à en valoir 10, ainsi que nous l’avons vu arriver de nos jours, le journalier, voyant que ce qui lui resterait de sa journée ne pourrait pas suffire pour nourrir sa femme et ses enfants, se réduit à boire de l’eau, comme ils font presque tous dans les villes considérables, et fait cesser par là la circulation que lui fournissait sa journée, et est réduit à l’aumône, non sans blesser les intérêts du roi, qui avait sa part à tous les pas de cette circulation anéantie. Il en va de même des autres denrées, n’y en ayant aucune dont l’anéantissement de consommation causé par les désordres marqués ci-devant ne fasse d’abord cesser dix ou douze sortes de métiers, qui roulaient tous sur ce premier principe, et ne rejaillisse ensuite par contrecoup et sur le roi, et sur tout le reste des professions du corps de l’État ; et alors, bien que l’argent demeure, il cesse, faute de circulation, de fortifier aucun revenu, et est comme s’il était mort à l’égard du pays. En sorte que, s’il y a 500 millions de rente moins en France qu’il n’y avait il y a trente ans, ce n’est pas qu’il y ait moins d’argent, mais c’est qu’y ayant pour beaucoup moins de denrées excrues, vendues et consommées, cela a communiqué le même mal à toutes les autres sortes de biens qui tirent leur être des fruits de la terre. Il n’en faut donc point accuser le manque d’argent, mais s’en prendre seulement à ce qu’il ne fait pas son cours ordinaire ; et la vaisselle d’argent réduite en monnaie ces jours passés n’a pas apporté plus de remède à ce mal que ne fait une flotte du Pérou à la misère de l’Espagne, qui, depuis qu’elle en reçoit, n’en devient pas plus riche, parce que l’argent ne fait qu’y passer, et qu’elle ne le voit que dans sa naissance. Ainsi, celui de la vaisselle, après son premier cours, a gagné les forts dont on vient de parler et dont il est impossible de le tirer. Et il aurait été cent fois plus avantageux à la France d’ôter quelques-uns de ces édits qui ruinent la consommation pour des quantités de millions par an, quoique le produit à l’égard du roi soit fort médiocre, et de reporter le montant des droits sur les Tailles, afin que Sa Majesté ne perdît rien, ce qui n’aurait pas été à un sou pour livre, que de réduire de la vaisselle en monnaie, l’utilité qui en est venue à Sa Majesté pouvant aisément être compensée d’ailleurs.\par
Enfin, le corps de la France souffre lorsque l’argent n’est pas dans un mouvement continuel, ce qui ne peut être que tant qu’il est {\itshape meuble}, et entre les mains du peuple ; mais sitôt qu’il devient {\itshape immeuble}, ne pouvant cesser de l’être, parce qu’on ne trouve aucune sûreté à le reconstituer sur une terre, ou à le prêter pour acheter une charge qui peut être supprimée ou anéantie par la création de pareilles qui la tireront hors du commerce, ou enfin à rejeter ce même argent dans le trafic, par les raisons qu’on vient de marquer, on peut dire que tout est perdu. Or, quand tout l’argent serait entre les mains du menu peuple, où il est toujours meuble, il faut qu’il retourne aussitôt entre les mains des puissants, qui le refont immeuble en la plus grande partie, parce que l’harmonie de la république, qu’une puissance supérieure régit invisiblement, subsistant du mélange de bons et de mauvais ménagers, toutes choses, tant meubles qu’immeubles, sont dans une révolution continuelle, et le riche devient pauvre afin que le pauvre puisse devenir riche. En effet, un dissipateur de ses fonds et de son argent-immeuble, comme le rachat d’une rente constituée et le prix d’une terre, en fait un meuble en le consommant en sa dépense journalière, qui ne devrait être tirée que du produit de ces mêmes fonds ; tandis qu’un bon ménager, ne consommant pas ses revenus ordinaires, soit de fonds de terre ou d’industrie, en forme un argent-immeuble, c’est-à-dire dont il a dessein de se former un immeuble, comme une terre, une maison, ou une partie de rente ; ce que ne pouvant faire comme on vient de dire, cet argent ne retourne plus chez le peuple, en passant par les mains du dissipateur qui le refait meuble. Ainsi le corps de l’État fait une très grande perte, parce que c’est le menu peuple qui lui forme le plus de revenu ; un écu faisant plus de chemin et par conséquent de consommation en une journée chez les pauvres, qu’en trois mois chez les riches, qui, ne faisant que de grosses affaires, attendent longtemps que leur somme soit fournie, même dans les meilleurs temps, pour faire sortir leur argent, ce qui est toujours préjudiciable à un État. De manière que Philippe de Commines remarque que, si le roi Louis XI tripla son revenu en quinze années, personne ne fut ruiné, parce qu’il dépensait aussi tôt tout ce qu’il recevait ; ce qui montre assez l’intérêt qu’un pays a que ses habitants ne soient pas dans l’obligation de dépenser moins d’argent qu’ils n’en reçoivent.
\subsection[{Chapitre XX.}]{Chapitre XX.}
\noindent Il ne faut point de preuves plus certaines de tout ce qu’on vient de dire, que l’exemple des marchandes de menues denrées de Paris, lesquelles s’enrichissent à emprunter de l’argent à cinq sous d’intérêt par semaine pour un écu, c’est-à-dire à plus de 400 pour cent par an, le produit excédant quatre fois le capital ; car, bien qu’une pareille conduite, quand l’intérêt serait infiniment au-dessous de celui-là, ruinerait le plus riche homme du monde, cependant elle enrichit et fait vivre ces pauvres gens ; et la manière dont cela se fait est aisée à concevoir. C’est parce que cette marchande, ayant vendu pour quatre ou cinq écus de marchandise en une journée, sur laquelle elle a quelquefois gagné la moitié, elle retourne le lendemain de grand matin à l’emplette, et, faisant cette manœuvre cinq à six fois la semaine, il lui est aisé de trouver et sa vie et de quoi satisfaire à ceux qui lui ont prêté ; et ce genre de commerce ne cesse que lorsque les pauvres journaliers, qui se fournissent uniquement chez elle, cessent de le faire, pour ne plus trouver leur journée, qui est anéantie à Paris comme ailleurs par des causes traitées une infinité de fois.
\subsection[{Chapitre XXI.}]{Chapitre XXI.}
\noindent Quoiqu’on ait assez montré l’intérêt que le roi a à la ruine de la consommation, qui attire toutes les pernicieuses conséquences dont on vient de parler, on va mettre ce même intérêt dans un nouveau jour, pour le rendre encore plus sensible à ceux qui en voudraient douter. — Il est certain que le roi entretient ses armées et sa dépense ordinaire, non avec de l’argent à proprement parler, mais avec du blé, de la viande, du linge, des habits, et enfin avec toutes les autres choses nécessaires à l’entretien de la vie, lesquelles, croissant en ses États, sont consommées pour la plus grande quantité par ses sujets, et une partie lui est baillée par redevance ; et si ce n’est pas immédiatement, c’est la même chose, parce que les dix écus qu’un chapelier baille au roi pour sa Taille, après les avoir tirés du profit qu’il a fait sur mille chapeaux qu’il a fabriqués et vendus, la nourriture et entretien de sa famille prélevés, est une obligation et un gage qu’il donne au roi de lui fournir dix chapeaux à lui ou à son ordre, en quoi faisant son gage lui sera restitué, comme il arrive infailliblement ; — car Sa Majesté n’a pas sitôt reçu ce gage, qu’elle le rebaille à un capitaine de chevau-légers, qui le reporte avec la même diligence au chapelier pour en tirer les dix chapeaux, lequel refait faire aux dix écus la même circulation, à moins que le canal n’en soit interrompu, c’est-à-dire que la boutique du chapelier ne soit démontée parce que les chapeaux ne se peuvent plus vendre, comme nous avons vu arriver, par les raisons traitées ci-dessus ; et ainsi de toutes les autres marchandises dont on peut faire le même raisonnement : — ce qui montre évidemment le grand préjudice que le roi reçoit de la ruine de la consommation, et que c’est le surprendre que de dire qu’on la ruine pour l’enrichir.\par
Et, pour conclusion entière de la seconde partie de ces Mémoires, on dira qu’il n’y a qu’à comparer ce qui se passe chez nos voisins avec ce qui se fait en France à l’égard des impôts. On a déjà montré dans la première partie que, bien qu’il n’y ait jamais eu une pareille diminution de biens, cependant le roi lève moins à présent sur ses sujets que plusieurs de ses ancêtres : on dira maintenant, et on le maintient, qu’il n’y a point de prince dans l’Europe qui ne tire à proportion beaucoup davantage, et où cependant il en coûte tant à ses peuples ; et bien que cela paraisse un paradoxe, c’est pourtant une vérité constante. En effet, une vigne arrachée pour ne pouvoir supporter l’impôt qu’on a mis dessus (comme cela arrive tous les jours), ne va point au profit du roi, et ne ruine pas moins le propriétaire ; et comme ce mécompte s’est rencontré dans une infinité de denrées, ainsi qu’on a fait voir, on en peut tirer les mêmes conclusions. Dans tous les autres États on proportionne les impôts aux choses sur lesquelles on les lève ; et de cette manière le prince et les peuples y trouvent également leur compte ; et c’est ainsi que, pour descendre davantage dans le détail, il est certain que l’Angleterre ne vaut point le quart de la France, soit par le nombre du peuple, qui est une partie essentielle à la bonté du pays, à cause que la consommation ne se saurait faire sans lui ; soit pour la fertilité du terroir (et si la conquête des Gaules coûta huit années à Jules César, celle de toute l’Angleterre ne fut l’effet que d’une seule campagne) ; cependant l’Angleterre vient de rapporter depuis trois ou quatre ans près de quatre-vingts millions par an au prince d’Orange, et cela sans réduire les peuples à la mendicité, ni les mettre dans l’obligation d’abandonner la culture des terres ; et si la guerre n’avait point interrompu son commerce, c’eût été encore tout autre chose. Que l’on considère encore tous les princes d’Allemagne, jusqu’au moindre ; que l’on considère leurs États, qui ne sont pas un atome en comparaison de la France, et toutefois ce qu’ils en tirent va à un trentième ou environ, et même encore à plus. La Savoye en tout son contenu, sans le Piémont, ne vaut point la moindre des Élections de Normandie, au nombre de trente-deux. Son terroir, très mauvais et très stérile, ne peut nourrir qu’une partie de ses habitants, et encore très misérablement ; il n’y a ni rivières, ni villes considérables où l’on fasse quelques manufactures ; cependant elle rapportait 500 000 écus à son prince par an avant la guerre ; et cela, parce que les choses se faisaient comme en Angleterre, en Allemagne et dans tous les pays du monde, c’est-à-dire qu’on faisait rapporter à la terre tout ce que son climat et son terroir, aidés de secours humains, pouvaient produire ; on y consommait tout ce qu’on y pouvait consommer, et on y vendait tout ce qu’on y pouvait vendre, qui est une situation qui devrait être sacrée aux ministres de tous les princes du monde, leur étant permis de pousser les droits de leurs maîtres jusqu’à tel point qu’ils peuvent aller, tant qu’ils ne donneront point atteinte à ces deux mamelles de toute la république, l’agriculture et le commerce. Mais de croire mieux servir un monarque par une conduite contraire, comme on ne peut pas nier qu’il arrive présentement en France, cela se réfute si fort de soi-même par la simple narration des choses rapportées dans ces Mémoires, que l’on n’en dira rien davantage. Mais cette même doctrine peut être établie, sans aller chez les Étrangers, par ce qui se passe en France aux lieux où la Taille n’est point arbitraire et sujette aux pernicieux effets dont on a parlé, et où pareillement les Aides et Droits sur les passages n’ont point encore eu lieu : on verra la différence de ces contrées avec les autres. — La généralité de Montauban ne vaut pas la sixième partie de la généralité de Rouen, soit pour la situation, qui n’a ni mer ni rivière pour voisines ; au lieu que la généralité de Rouen a Paris d’un côté et la mer de l’autre, qui est la plus avantageuse situation du monde ; son terroir n’a point son pareil en fécondité ; les villes et bourgs y sont sans nombre, et peuplés à proportion ; et cependant, avec tous ces avantages, elle ne rapporte au roi qu’un tiers de plus que celle de Montauban, qui, en Taille seule, qui est réelle, rapporte 3 400 000 livres ; tandis que tout ce que le roi a jamais tiré de la généralité de Rouen, en revenus ordinaires, n’a jamais été à plus de six à sept millions tout compris. Mais la différence à l’égard des peuples est encore bien plus grande : dans la généralité de Montauban, il est impossible de trouver un pied de terre auquel on ne fasse rapporter tout ce qu’il peut produire ; il n’y a point d’homme, quelque pauvre qu’il soit, qui ne soit couvert d’un habit de laine d’une manière honnête ; qui ne mange du pain et ne boive de la boisson autant qu’il lui en faut ; et presque tous usent de viande, tous ont des maisons couvertes en tuiles, et on les répare quand elles en ont besoin. Mais dans la généralité de Rouen, les terres qui ne sont pas du premier degré d’excellence sont abandonnées, ou si mal cultivées qu’elles causent plus de perte que de profit à leurs maîtres ; la viande est une denrée inconnue par les campagnes, ainsi qu’aucune sorte de liqueur pour le commun peuple ; la plupart des maisons sont presque en totale ruine, sans qu’on prenne la peine de les réparer, bien qu’on les bâtisse à peu de frais, puisqu’elles ne sont que de chaume et de terre ; et avec tout cela, les peuples s’estimeraient heureux s’ils pouvaient avoir du pain et de l’eau à peu près leur nécessaire, ce qu’on ne voit presque jamais ; et tous ces désordres arrivent pendant que le pays pourrait non seulement faire subsister parfaitement bien les habitants d’une manière fort heureuse, mais même aider ses voisins, comme il faisait autrefois, si les proportions absolument nécessaires pour une pareille harmonie n’étaient ruinées par des intérêts indirects, ce qui retombe également sur Sa Majesté, puisqu’il est aussi impossible que des terroirs, incultes et des peuples qui meurent de faim lui soient utiles à quelque chose, qu’il est difficile qu’une situation contraire ne lui soit pas très avantageuse. Mais comme ceux qui fournissent les mémoires à MM. les ministres n’ont pas les mêmes intérêts, qu’ils en ont même de tout opposés, il ne faut pas s’étonner qu’ils sacrifient ceux et du roi et des peuples à leurs avantages personnels ; et bien qu’ils ne profitent pas en leur particulier pour la cinquantième partie du mal qu’ils font au corps de l’État, leur intérêt, quelque petit qu’il soit en comparaison de ce mal, prévaut à l’utilité publique, ce qui est aujourd’hui érigé en profession ordinaire, remplie de personnes de la plus haute protection. De manière que, quoique les désordres sautent aux yeux, et que le roi ait un intérêt très grand, sans parler de celui des peuples, de les faire cesser, personne jusqu’ici n’a été assez osé pour leur déclarer la guerre, ou plutôt à leur manœuvre.\par
C’est pourtant sur ces principes qu’on va passer à la troisième partie de ces Mémoires, qui traiteront des remèdes de ces désordres, dont on établira la facilité et l’utilité d’une manière si constante, qu’il n’y a que ceux qui en attendent ou leur doivent leur fortune, qui y pourraient apporter de l’opposition par leurs actions ou par leurs paroles. Leur principale objection sera le délai qu’ils demanderont, ou le prétendu bouleversement des affaires qu’ils opposeront ; mais l’un et l’autre sont ridicules, attendu que ce sont les peuples mêmes qui parlent dans ces Mémoires, au nombre de quinze millions, contre trois cents personnes au plus, qui s’enrichissent de la ruine du roi et des peuples, lesquels ne demandent que la simple publication de {\itshape deux édits} pour être au bout de deux heures en état de labourer leurs terres en friche, et de vendre leurs denrées perdues, ce qui doublerait sur-le-champ et le revenu de leurs terres, et celui du roi. Or, on ne peut, sans renoncer à la raison, dire à des gens qui offrent de payer, qu’il leur est impossible de le faire, surtout quand on est aussi suspect que doivent être ces trois cents contredisants.
\section[{Troisième partie. Des moyens de rétablir la richesse nationale.}]{Troisième partie. Des moyens de rétablir la richesse nationale.}
\subsection[{Chapitre I.}]{Chapitre I.}
\noindent Pour venir donc aux remèdes de si grands désordres, on dira d’abord qu’il n’y a rien de si aisé du côté de la chose, et rien de si difficile de la part de ceux à qui il s’en faut beaucoup qu’ils soient indifférents. En effet, il semblerait que les seules personnes qui devraient être intéressées dans les impôts qui se lèvent, ainsi que dans toutes autres dettes, ne seraient que le roi et ses peuples, Sa Majesté pour recevoir, et ses peuples pour payer ; et par conséquent, qu’on devrait être certain de l’acceptation d’une proposition qui ferait recevoir le double à Sa Majesté, pendant qu’il n’en coûterait pas le tiers à ses peuples. Cependant, bien que dans tout ceci il n’y ait rien que de très véritable et de très sensible par tout ce qui se passe et chez l’étranger, et en France même, on ne laisse pas de n’avoir qu’une légère espérance du succès. Quoi qu’il en puisse arriver, on dira qu’on ne veut apporter aucun trouble à la disposition présente pour un si grand bien ; qu’il n’est nécessaire de congédier ni fermiers, ni receveurs ; qu’on aura un extrême respect pour le fait de Sa Majesté, bien qu’on ne puisse pas dire que l’on en ait toujours usé de même, parce qu’il est absolument nécessaire de ne pas ruiner le commerce entre le roi et ses peuples, en rescindant d’autorité absolue des actes qu’on a cru faire de bonne foi. Car une pareille conduite fait que, dans le trafic particulier, une charge de nouvelle création, ou des gages ou rentes sur le fait de Sa Majesté, ne se vendent et achètent que sur le pied de la moitié d’un autre effet de pareil revenu, qui aurait un particulier pour garant. Ainsi nulle objection de ce côté-là : si on fait payer davantage à Sa Majesté, et moins par ses peuples, c’est parce que toutes sortes de paiements, et surtout les tributs, tirant leurs qualités, ou leurs degrés d’excès ou de justice, du pouvoir ou de l’incapacité de de ceux qui les paient, il est constant qu’un particulier qui payait 100 francs de Taille sur une ferme de 1 000 livres, sera bien moins chargé en en payant 200, si la ferme peut revenir à 2 000 livres, puisque ce sera 800 francs qu’on lui donnera à pur profit, et qu’il sera entièrement déchargé de son impôt sur ces premières mille livres. Or, sa ferme reprendra ce premier prix qu’elle avait autrefois, lorsqu’il lui sera permis de la labourer, cultiver, et en vendre les denrées qui y croîtront ; parce que les causes des défenses et de l’impossibilité de faire ces choses seront levées, ainsi qu’il est très facile, comme on va le faire voir.
\subsection[{Chapitre II.}]{Chapitre II.}
\noindent Pour commencer à lever les défenses de la consommation, marquées dans la première partie de ces Mémoires, qui sont l’incertitude de la Taille arbitraire, qui attire après elle les désordres de la collecte, l’un et l’autre faisant un déchet à la consommation de plus de 150 millions par an, sans qu’il en revienne un denier au roi ; il n’est pas nécessaire d’opérer le moindre bouleversement, tant à l’égard des personnes que des choses, mais seulement d’ôter l’injustice de la répartition, et de faire observer toutes les ordonnances, tant anciennes que modernes, qui ne portent rien moins que ce qui se pratique. Et comme cette injustice est aujourd’hui établie si généralement, que plus un homme est puissant, et moins ses fermiers doivent payer de Taille, ce qui est sa ruine, ainsi qu’à tout le reste de l’État, il est à propos que Sa Majesté ait la bonté d’expliquer elle-même à toutes les personnes de sa cour, que, pour leur propre intérêt, elles en doivent user envers lui, afin que le commerce soit réciproque, comme il en use envers elles, et comme elles-mêmes en usent envers tout le monde, et surtout envers l’Église.\par
Il est certain que plus un homme est élevé en dignité et en naissance, plus Sa Majesté lui marque de distinction dans la répartition tant des bénéfices, que des charges de la cour. Il est pareillement certain que plus ces mêmes gens sont dans l’élévation, plus ils se veulent distinguer dans les rétributions qu’ils font à l’Église, dans les spectacles, et enfin dans toutes les autres occasions, à l’exception des droits du roi ; et bien qu’il y ait longtemps que les personnes de vertu, même de cette profession, conviennent que la véritable piété n’a ni part ni obligation au bien que l’on fait à l’Église, cependant, ses ministres ont eu l’adresse de mettre les choses sur le pied qu’on les voit aujourd’hui. En sorte qu’un grand seigneur, après avoir dépensé des sommes immenses pour l’enterrement ou de son père, ou de sa femme, soutiendra son receveur ou fermier dans trente procès qu’il fera pour s’exempter de payer une pistole, à laquelle il aura été mis plus que l’année précédente, bien que son imposition ne soit pas à la trentième partie de ce qu’elle devrait être si la répartition était juste ; parce qu’il y a un si grand abus, qu’on regarde comme une espèce d’infamie de payer cette juste proportion. Ainsi, ces désordres subsistent par un double intérêt, qui n’est pas, à proprement parler, un véritable intérêt, mais une ruine générale, réellement et de fait, par une contravention continuelle que l’on fait aux lois divines et humaines ; et il n’en faut point d’autre marque que les propres termes de l’ordonnance de Charles VII, de l’année 1445, lorsque les Tailles commencèrent d’être ordinaires ; elle porte ces mots : « Voulons égalité estre gardée entre nos sujets ès charges et faix qu’ils ont à supporter, sans que l’un porte ou soit contraint à porter les faix et charges de l’autre, sous ombre de privilège et de cléricature, ny autrement : et voulons les instructions et ordonnances royaux estre gardées selon leur forme et teneur. »\par
On peut dire que la richesse ou la diminution de la France a été à proportion que ces ordonnances ont été observées, de même que dans tous les pays du monde, comme on peut voir par l’exemple de la Hollande, qui, étant gouvernée par un peuple qui ne souffre point d’injustice dans la répartition des impôts, ne laisse pas d’être le plus riche État de l’Europe, eu égard à sa situation. Et quoique les impôts y soient excessifs, de manière qu’on ne craint point de dire qu’il contribue six fois plus pour les charges publiques que ne fait à proportion la France à Sa Majesté, cependant il ne se trouve point un seul pauvre dans tout cet État : et c’est cette importante maxime qui faisait dire à Mécenas, en parlant à Auguste, « qu’aucunes personnes, non pas même les pupilles, ne devaient être exemptes des Tailles et impositions publiques ; d’autant, disait-il, que l’utilité des choses à quoi elles sont destinées tourne également au profit et conservation de ceux qui les paient ». Et quand Dieu a commandé de payer les tributs aux princes, il a prétendu parler à tout le monde, et non pas aux misérables et aux indéfendus seulement, qui ne s’en pouvaient exempter ; ou bien ce précepte aurait été inutile, puisqu’il n’aurait eu lieu qu’à l’égard de ceux qui n’auraient pu faire autrement, ce qui ne se peut dire sans impiété.
\subsection[{Chapitre III.}]{Chapitre III.}
\noindent Ceci donc supposé, que le roi veuille et entende que la Taille soit désormais répartie avec justice, c’est-à-dire que les riches paient comme riches, et les pauvres comme pauvres, tant pour l’intérêt de Sa Majesté que pour celui de ceux mêmes qui s’exemptaient, il n’y a rien de si aisé que l’exécution. — Il ne faut qu’ordonner qu’environ trois ou quatre mois avant le département, tous les particuliers, tant exempts que non exempts, des lieux taillables, apporteront au greffe de leur Élection une déclaration au juste de tout ce qu’ils font valoir, soit comme propriétaires, soit comme fermiers ; le prix qu’ils en tiennent, avec copie de leurs baux qu’ils signeront véritables, à peine de confiscation ; ensemble le prix que pourraient valoir les terres ou biens qui ne sont point baillés à ferme, et qu’on fait valoir par ses mains, eu égard aux biens et aux terres voisines. On mettra que les trésoriers ou marguilliers de la paroisse apporteront pareillement un état de tous ceux qui, ne faisant rien valoir, vivent de leur travail manuel, et n’ont qu’une simple habitation ; ils marqueront leur métier, leur âge, leur nombre d’enfants demeurant avec eux, leur âge pareillement, et ce à quoi ils sont imposés de Taille. — Le tout étant remis au greffe, sera enliassé par paroisse, et sera marqué au bas de tous les baux pareillement combien chaque fermier paie de taille ; et le tout sera émargé à côté de chaque cote du rôle de l’année, dont il y a toujours copie au greffe de chaque Élection. — Ceci fait, les officiers de l’Élection, à commencer par le président jusqu’au procureur du roi, se partageront les paroisses de leur dite Élection, en en prenant chacun vingt ou trente, à proportion de leur nombre, dont le dernier reçu fera les partages, et les autres les choisiront suivant leur rang et degré. Il sera nécessaire que, dans le lot de chacun, il ne tombe aucune paroisse où celui à qui elle sera échue ait du bien, ou ses parents au premier degré ; et dans ce cas il la faudrait échanger contre une autre paroisse d’un autre lot. — Chaque officier ayant ainsi son département, il fera une estimation, premièrement de tout ce que les occupants des fonds non privilégiés font valoir, soit comme fermiers ou comme propriétaires, sans nulle distinction ; et après en avoir fait un arrêté à combien cela revient sur les fonds au marc la livre, si c’est un sou et demi, deux sous ou davantage pour livre, sans rien encore arrêter, ils conféreront tous ensemble de la même Élection, pour voir si les choses sont sur le même pied dans chaque lot ; et au cas que cela ne fût pas, ils feront une seconde estimation, pour voir combien il faudrait qu’un lot contribuât à la décharge de l’autre afin de rendre les choses égales, dont ils feront pareillement un arrêté au bas de chaque rôle, sur lequel ils feront la répartition de chaque contribuable occupant des fonds, sur le pied de toute l’Élection, et le marqueront à chaque cote du même rôle. Ils en useront de même à l’égard des Taillables à cause de leur seule industrie, à la réserve de ceux qui se trouveront dans les villes taillables ou gros bourgs, parce que, comme dans les simples villages il se voit peu de négociants considérables, la simple industrie n’est pas sujette à de grandes Tailles. Mais il n’en va pas de même dans les gros lieux, ce qui fait qu’il en faut user autrement. Premièrement on a pu voir, par ce qui a été dit des endroits taillables qui ont obtenu permission de mettre leur impôt en tarif, l’avantage qui leur en revient, ainsi qu’à Sa Majesté : c’est pourquoi elle gagnerait extrêmement de l’accorder à tous ceux qui le demanderaient ; et bien que cette concession paraisse du droit des gens, n’y ayant rien ce semble de si juste que de permettre à un débiteur de s’acquitter en la manière qui lui soit plus commode, ils ne laisseront pas de fournir une bonne somme d’argent pour cette concession. Mais jusqu’à ce que cela soit fait, comme il y a peu de ces gros lieux taillables qui n’aient de la campagne et du labourage, outre les habitants qui sont dans l’enceinte de leurs murailles, on observera la même conduite à l’égard des laboureurs et de ceux qui font valoir ces fonds, que dans les simples villages ; et pour les gens de métier qui gagnent leur vie de leur art, ou de leur travail manuel, on les divisera par classes, suivant leur degré et rang, qui est assez connu de tout le monde, ou même suivant les classes qui viennent d’être faites dans la répartition de la contribution des arts et métiers, et l’on mettra à côté de chaque cote du rôle ce qui reviendra à chacun de sa quote-part de la Taille, en la répartissant également entre ceux d’une même profession, dont ils seraient également prenables, dans les villes et bourgs seulement. On en usera de même à l’égard de ceux qui sont simples journaliers dans la campagne, les mettant à une simple somme, qui ne pourra être plus basse qu’un écu, ni plus haute que 6 livres, suivant et à proportion de la qualité de leur métier et de leur âge, lorsqu’il serait au-dessus de soixante-dix ans, outre encore les 2 sous pour livre de leur {\itshape occupation}, même pour simple habitation, tant aux champs qu’aux villes et bourgs, afin de laisser une entière liberté de prendre avec leur travail manuel telles fermes qu’ils aviseront bien être, sans que cela attirât de la confusion. — Les choses ainsi réglées par chaque Élu dans son district, il en ferait son rapport au Commissaire départi lors du département des Tailles, qui n’aurait qu’à confirmer dans l’assiette de chaque paroisse ce qui aurait été fait par les Élus, en donnant au marc la livre, suivant la même répartition, ce qu’il y aurait de hausse ou de rabais dans l’Élection, ou plutôt dans la Généralité. Les rôles ainsi arrêtés seraient envoyés dans les paroisses, l’assiette étant faite, ce qui épargnerait dès ce moment bien du temps et du mal. Les collecteurs anciens auraient ordre de mettre chez les trésoriers ou marguilliers une liste par ordre de tous ceux à qui il écherrait d’être collecteurs année par année, en commençant par la présente, qui y demeurerait un mois ; pendant lequel temps tous les Taillables pourraient aller voir la somme à laquelle ils seraient imposés, et s’il y avait erreur au fait, comme s’ils avaient plus que le marc la livre de leur {\itshape occupation}, à proportion du reste de la paroisse, ils feraient leur protestation à côté de leur {\itshape taux}, en mettant simplement le mot de {\itshape protestation} écrit de leur main ou de celle d’un autre, avec leur marque, pour en faire répondre l’Élu, ou ceux qui auraient baillé de fausses déclarations, sans que néanmoins cela les empêchât de payer l’année, parce qu’il leur serait pourvu de récompense dans la suite. Dans le même mois, tous ceux qui ne voudraient point être collecteurs à l’avenir, ni garants des mauvais deniers, déclareraient à côté de leur imposition, pareillement, qu’ils se soumettent de porter toute leur année dans le mois chez le receveur des Tailles, qui serait obligé d’avoir de plus grands registres, afin de laisser plus de blanc pour chaque paroisse, et que le nom de chaque particulier y trouvât place. Le mois passé, le premier de ceux qui n’aurait point fait sa soumission d’apporter son impôt dans le mois, serait obligé de faire la collecte à la garantie seulement de ses semblables qui n’auraient point fait de soumission, et aurait les 2 sous pour livre, parce qu’il ne pourrait demander aucune récompense des frais et mises. Mais on est assuré qu’il n’y en aurait point, et que tous les laboureurs et gens un peu accommodés satisferaient dans le mois, afin de s’exempter de la garantie de la collecte et des 2 sous pour livre. Et à l’égard des manouvriers, outre qu’il faudrait ordonner que l’année delà Taille se prendrait avant toutes dettes et charges, même les louages de maisons, il n’y en aurait aucuns qui ne trouvassent à emprunter une légère somme à quoi irait leur imposition, d’autant plus que la consommation étant rétablie, il n’y aurait aucuns de ces gens-là qui ne trouvassent amplement leur journée, le manque de laquelle est ce qui les ruinait, et non 30 sous, plus ou moins, de Taille, ce qui ne va qu’à un denier par jour, c’est-à-dire rien. Enfin, comme les plus grands désordres de la Taille n’ont jamais été, à beaucoup près, dans sa quotité, ainsi qu’on a fait voir, mais dans ses suites fâcheuses, comme son incertitude et sa collecte, il est indubitable que le bien qui reviendrait de ces règlements serait infiniment au-dessus de toutes les objections que l’on pourrait faire ; et la Taille étant justement répartie, il n’y a que les mendiants qui ne seraient pas en état de la payer facilement. — Et, comme les espèces sont beaucoup plus fécondes que l’imagination, on ne doute pas qu’il ne puisse arriver tel incident, dans un cas particulier, où une Déclaration sur le modèle de ces Mémoires n’aurait pas pourvu ; mais dans ces occasions-là, ou les Élus, ou les commissaires départis, y remédieraient aisément, suivant ce même style. Tout le travail de l’assiette tombant sur les Élus, et de la recette particulière sur les receveurs des Tailles, il serait juste de leur partager moitié par moitié les 6 deniers pour livre que l’on impose ordinairement pour ce sujet, le papier et les frais de l’écriture étant fournis par les greffiers des rôles nouvellement créés.\par
On est persuadé que, de cette sorte, la consommation deviendra permise, que le roi et les particuliers y trouveront extrêmement leur compte, et qu’à en consulter les plus apparents et les plus raisonnables, on les fera convenir qu’une pareille disposition procurerait autant de bénédictions et de repos, que la situation contraire, qui est celle d’aujourd’hui, attire de misères et de troubles, outre la haine implacable qui cause la perte des âmes, ce qui se perpétue jusqu’à la troisième génération. — Cette première cause de la diminution des biens de la France, savoir la défense de la consommation, étant levée par une Déclaration de deux ou trois pages, qui ne troublera en rien la situation présente des choses, il faut passer à la seconde cause de cette même diminution, qui est l’impossibilité de la consommation, que l’on va montrer, dans le chapitre suivant, être aussi facile à faire cesser, sans produire davantage de mouvement ; à la réserve que, pour tout le reste des baux des droits d’Aides, Passages et Sorties du royaume, on donnera pour commis aux fermiers généraux et particuliers les receveurs des Tailles, après que tous les lieux sujets aux dits droits auront été abonnés d’une manière fort juste suivant le prix du bail, qui est une méthode que les mêmes fermiers pratiquent dans toutes les occasions, lorsqu’ils le peuvent aisément, en gagnant par là les frais des bureaux, des commis et des quêtes, et les peuples se rédimant d’une vexation effroyable.
\subsection[{Chapitre IV.}]{Chapitre IV.}
\noindent On peut dire, en général, que les impôts que le roi tire de la France sont infiniment au-dessous de son pouvoir, parce que les causes dont on a parlé diminuent plus de la moitié de ses forces. En effet, y a-t-il rien de plus étonnant que de voir des fonds de vignobles, autrefois d’une très grande valeur, entièrement abandonnés ? Ce sont ces désordres que l’on veut faire cesser ; et pour y parvenir, il faut évaluer ce qui revient au roi des causes qui y donnent lieu, et voir si on ne peut point donner un autre cours à ces sortes de revenus. — Tous les droits d’aides, entrées et sorties des grosses villes, passages et travers, y compris une partie des Domaines, ne sont qu’à 31 millions par an présentement, sur quoi il en faut lever environ six à sept millions pour les Domaines, auxquels on ne touche point : ainsi reste à vingt-quatre, sur quoi on en tire encore le {\itshape convoi de Bordeaux}, qui va à près de cinq millions : ainsi reste à dix-neuf. On n’apporte aucun changement aux droits d’entrée dans le royaume, se réservant à mettre quelque règle qui rende les choses moins fâcheuses aux négociants, ce qui va encore à plus de deux millions : ainsi reste à {\itshape dix-sept}, qui font tous les désordres dont on a parlé, et à qui il faut donner un autre cours. Il est certain qu’en en remettant douze sur les Tailles, on ne fera que rétablir les choses comme elles étaient il y a quarante ans, pendant que tous les fonds étaient au double prix qu’ils sont aujourd’hui, et les revenus d’industrie dans la même situation, par une conséquence infaillible. De manière qu’on doit conclure avec certitude que ce changement d’impôt sera reçu avec mille actions de grâces de la part des peuples, comme une chose qui leur donne la vie en remettant leurs fonds en valeur. Jusqu’ici on ne peut pas dire qu’il faille aucun mouvement dans l’État pour un si grand bien, ni que les revenus ordinaires du roi courent aucun hasard, sur l’incertitude que l’on ne manquera jamais d’objecter dans les succès qu’on promet, ni qu’il faille attendre la fin de la guerre, qui n’a rien de commun avec ce qui se passe dans le milieu du royaume. Ainsi, il n’est plus question que de trouver où replacer cinq millions qui restent des dix-sept, auxquels on fait changer de cours, comme étant par leur manière, et non par leur quotité, cause de l’impossibilité de la consommation, c’est-à-dire d’une diminution de plus de 250 millions par an, en pure perte, dans le corps de l’État. — Pour replacer donc ces cinq millions, il reste toutes les villes franches qui ne paient point de tailles, comme Paris, Rouen et autres ; lesquelles étant sujettes à des droits d’aides effroyables, ainsi qu’on a marqué, et qui ont causé la ruine entière de plusieurs, en seront déchargées à l’avenir. Il reste encore les Ecclésiastiques, Nobles et privilégiés de la campagne, des pays d’Aides, qui ne contribuent point au rachat, ne payant point de Taille, où la plus grande partie serait rejetée, et n’y étant pas moins sujets, consentiront volontiers et avec justice d’acheter un si grand bien au prix de quelque chose du leur. Il n’y a pas d’apparence de rejeter tant les uns que les autres dans l’incertitude d’un impôt personnel, sujet au désordre dont on a parlé, et qui l’a si fort décrié. Il est donc plus juste de l’affecter sur les maisons tant des villes que de la campagne, en supposant deux conséquences infaillibles : la première, que qui dit un homme, dit un homme buvant et mangeant ; et la seconde, que plus un homme est riche, et plus il a de suite ; que plus il a de suite, et plus il habite une grande maison ; et enfin, que plus une maison est grande, et plus elle a de cheminées. De manière que ce tarif, qui a été celui de toutes les nations où les peuples ont choisi le genre d’impôt le plus commode, est assurément le plus juste, et celui où il est le moins possible de prévariquer sans qu’on s’en aperçoive aussitôt. Et quand à Paris on a fait une imposition pour les boues, les lumières de nuit et les pauvres, on l’a mise sur les maisons ; et cela n’a pas causé le moindre désordre ni aucun procès, quoiqu’on prétende qu’elle monte à 800 000 livres. Mais comme ce genre d’impôt fait passer l’argent immédiatement de la main de celui qui paie en celle de celui qui reçoit, sans qu’il soit possible que cent millions de pareil impôt fassent la fortune de personne, c’est là le plus grand obstacle qu’il pourra recevoir dans son exécution. Cependant, on maintient qu’en mettant toutes les cheminées de la ville et faubourgs de Paris à une pistole chacune, et celles des villes franches à demi-pistole chacune ; celles de tous les Nobles et privilégiés de campagne possédant des fonds, à une demi-pistole pareillement, et celles des villes closes où, quoique taillables, il y avait des droits d’entrée, à 40 sous chacune, et celles des bourgs où il se payait pareillement des droits, à 20 sous chacune ; les contribuables ne paieraient pas la moitié de ce qu’ils faisaient auparavant, outre tous les désordres dont ils seraient déchargés ; et le roi recevrait beaucoup davantage, puisqu’on croit que pour les cinq millions cela irait à plus de douze. — Chaque Élu, dans son district, en userait comme on a marqué à l’égard de la Taille ; il ferait un état de ce qu’il y aurait de maisons et de cheminées : l’impôt se prendrait en privilège avant les louages, et il serait portable à la recette des Tailles par chaque contribuable, qui, le faisant dans le {\itshape premier mois}, serait déchargé des deux sous pour livre auxquels il serait sujet dans le cas contraire, et qui iraient alors au profit de celui qui en ferait la collecte, et qui serait établi par les contribuables, ou par l’Élu à leur défaut ; mais on est bien assuré que tout le monde satisferait à cette obligation. Ainsi, Sa Majesté, outre l’augmentation en ses revenus et en ceux des peuples, le repos de leurs biens et de leurs consciences, recevrait en un mois, et par avance, ce qu’elle est toujours plus de quinze mois à percevoir. On a omis de marquer que les receveurs des Tailles et les Élus auraient la même rétribution, chacun par moitié, des six deniers pour livre, ce qui ne va à rien.
\subsection[{Chapitre V.}]{Chapitre V.}
\noindent Pour savoir la facilité de ce recouvrement tant des Tailles augmentées de ce supplément pour les Aides, que de cet excédent rejeté sur les maisons et sur les cheminées, ainsi que l’on a dit, il ne faut pas examiner les choses en général, ce qui est toujours sujet à confusion, mais descendre dans le particulier ; et ce qui se conclura d’une seule personne contribuable à cet impôt, de la manière qu’on l’établit, prouvera pour tout le reste. — Tous les revenus du roi, à quelque somme qu’ils puissent aller, n’étant qu’un assemblage de plusieurs sommes payées par divers particuliers, qui n’ont tous qu’un même intérêt de faire valoir chacun leur profession le plus qu’il est possible, ce que l’on prouvera pour l’un sera une conviction certaine pour tous les autres. Il y a quatre sortes de personnes intéressées à la situation que l’on propose, savoir les laboureurs, les artisans ou ceux qui vivent de leur industrie, les bourgeois des villes franches, et enfin les nobles et privilégiés de la campagne dans les pays d’Aides. Il est indubitable que toutes les quatre y trouveront également leur compte, et que ceux qui contrediront les dispositions proposées par ces Mémoires n’ont assurément pas procuration d’elles pour stipuler leurs intérêts. Car, pour commencer par les laboureurs, comme le corps le plus étendu, on peut considérer toutes les fermes à 1 000 livres l’une portant l’autre, le plus ou le moins n’y faisant rien en cette occasion, puisque le tout sera proportionné à la valeur des choses. Il est constant qu’elles consistent toutes en labourage pour recueillir des grains, en culture de vigne ou de plant, pour avoir des boissons, et en nourriture et engrais, pour vendre des bestiaux. Or, on ne peut pas douter, et on l’a assez montré dans la première partie de ces Mémoires, que toutes ces choses sont à la moitié, et de prix et de quantité, de ce qu’elles étaient il y a trente ans ; en sorte qu’une ferme baillée aujourd’hui à 1 000 livres, et dont on est même souvent mal payé, et le fermier obligé de faire banqueroute, était autrefois à 2 000 livres. Or, c’est la cause d’un si grand mal, marquée dans la seconde partie de cet ouvrage, que l’on met en vente à ce fermier, et à son maître en même temps, et à quel prix ? À 30 ou 40 francs au plus, puisque sur le pied de deux sous pour livre de la Taille, l’addition environ d’un tiers pour le rachat ou la réunion des Aides, et Douanes sur les sorties et passages, aux Tailles, ne va qu’à ce prix ; et pour une si petite somme payée d’avance, il fera le double prix de la vente de ses marchandises ; et comme pour faire 1 000 livres de fermage au profit du maître il faut que le laboureur en forme plus de 2 000 livres, tant pouf fournir à son entretien et celui de sa famille qu’aux frais du labourage, ce sera plus de 2 000 livres d’augmentation sur cette même ferme, dont le roi ne manquera pas d’avoir sa part, lorsque ses revenus auront pour principe d’augmentation l’accroissement de la fortune de ses sujets, ainsi qu’ils avaient eu depuis le roi Charles VII jusqu’à l’année 1660. Il n’en faut pas davantage pour montrer, ainsi que l’on a dit, que ceux qui s’opposeront à la situation proposée par ces Mémoires ont assurément d’autres intérêts à ménager que ceux des propriétaires de fonds et des laboureurs. — À l’égard des manouvriers, comme ce sont les plus misérables qui doivent faire la règle des autres, tout le monde sait qu’outre que leurs intérêts sont les mêmes que ceux des maîtres des fonds et des laboureurs, qui leur donnent leur journée, ou plutôt leur vie à gagner, étant presque tous, l’un portant l’autre, à cent sous ou six livres de Taille, leur ruine provenait de ce que ne trouvant point de travail, par les causes qu’on a marquées, ils ne pouvaient d’ailleurs avoir de boisson qu’à un prix excessif, et souvent même n’en trouvaient pas, à cause du dépérissement des cabarets, ces sortes de gens ne faisant point de provision : or, ce désordre cessera pareillement, à leur égard, moyennant quarante ou cinquante sous par an, c’est-à-dire quelque chose plus qu’un denier par jour, et le tout leur sera aisément avancé par ceux qui ont accoutumé de les mettre en besogne. — Pour les bourgeois des grandes villes, on ne pourra pas dire qu’on les met à la Taille : au contraire, ils se rédimeront pour le moins de la moitié de la somme qu’ils payaient par la plus effroyable servitude qui fut jamais, sans parler de l’intérêt que les habitants des villes ont à la valeur des fonds de la campagne, comme les possédant presque tous, et qu’ainsi ils ne devraient pas refuser de contribuer de quelque chose pour les rétablir. Cependant, on maintient qu’indépendamment de cette raison, ils y gagneront le double. En effet, qu’on regarde à Paris un marchand tenant une maison de 7 à 800 livres, il n’en habitera environ que quatre chambres, ayant quatre cheminées. Néanmoins, sa famille étant composée pour l’ordinaire, de huit ou neuf personnes, tant enfants que garçons de boutique ; à mettre le tout l’un portant l’autre à cinq sous par tête, à un demi-muid de vin par an, ce qui ne fait pas deux demi-setiers par jour, il paiera 80 francs pour les Aides, avec mille sortes d’embarras, de périls et de pertes de journées aux bureaux et aux portes, s’il les fait venir de quelque bien qu’il ait à la campagne. Et par la réduction par cheminées, comme elle s’est faite et se fait encore dans tous les pays du monde, il ne lui en coûtera que 40 francs d’une façon commode, et le roi sera payé par avance. — Il reste les gentilshommes et privilégiés de la campagne des pays d’Aides, dont on peut faire le même raisonnement que des Tailles, puisque la ruine de la consommation leur est également préjudiciable, étant tous possesseurs de fonds ; mais, indépendamment de cette raison générale, ils y gagneront encore le double, en considérant l’argent qui sortait de leur bourse, puisque n’y en ayant aucun qui n’achetât ou qui ne vendît des boissons, il est impossible que, dans l’un ou l’autre cas, il ne leur en coûtât 40 ou 50 francs par an ; tandis que, par la réduction par cheminées, mettant les choses sur le pied d’une consommation qui attirât une pareille somme pour les droits d’Aides, cela n’irait qu’à 25 ou 30 francs. — Ainsi, il est aisé de voir de tous points que ceux qui contrediront ces propositions n’ont nullement procuration des personnes intéressées, savoir celles qui paient, pour tenir un pareil langage, non plus que pour dire qu’il faut attendre que la paix ait lieu, qui est assurément une défaite pour faire manquer une chose qui, causant la félicité générale des peuples et la richesse du roi, ne produirait pas, à beaucoup près, le même effet à l’égard de quelques autres personnes, dont le nombre n’étant pas à la millième partie de ceux que cela enrichirait ne doit pas, toutefois, entrer en considération pour arrêter un si grand bien ; outre l’intérêt du roi, qui est du double plus fort dans l’un que dans l’autre. Car il est fort indifférent à un fermier ruiné par l’incertitude de la Taille et par le désordre des Aides et des Douanes, qu’il y ait paix ou guerre, pour se racheter à forfait, par un prix fort médiocre, des causes de sa ruine ; et quand quelques hôteliers ont demandé aux fermiers des Aides de s’{\itshape abonner}, ou de traiter pour une somme certaine par an, moyennant laquelle ils fussent exempts d’avoir tous les jours des commis qui les tourmentassent dans leurs caves, jamais ces fermiers n’ont considéré, pour le leur accorder, s’il y avait paix ou guerre ; ils ne l’auraient même pu faire sans se rendre ridicules ; et ce qui conclut sous ce rapport, conclut également sous l’autre. — Il y a encore une objection que l’on peut faire, qui est l’erreur qui a pu se rencontrer dans la réduction des sommes qui sont la cause de la ruine, en sorte que le rejet est peut-être plus fort que l’on n’a marqué. Mais on répond que, comme les causes de la misère publique ne consistent pas dans l’importance des sommes qui se paient au roi, mais bien dans la manière de lever ces sommes, quand même il y aurait cinq à six millions d’erreur dans ce calcul, le roi y gagnerait encore dès la première année ; puisqu’on prétend que n’y ayant point d’erreur, il en aurait six ou sept de surcroît. Et il est aisé de soutenir les choses sur ce même pied, par l’exemple d’une seule ferme ou d’un seul particulier, puisque, dans le premier cas, le propriétaire d’un fonds autrefois de 2 000 livres de rente, et présentement de la moitié mal payée, au lieu de payer 140 livres, pour le remettre dans la première opulence en paiera 145 ou 150 au plus ; et ainsi de tous les autres, et même des particuliers qui ne font rien valoir. Pour Sa Majesté, il est inconcevable l’utilité qu’elle en retirera, puisque la plus grande partie de ses revenus étant attachée, au pied de la lettre, à ceux de ses sujets, les uns haussant, nécessairement il en sera de même des autres ; et le roi aura 200 millions de rente, parce que les terres qui étaient baillées à 1 000 livres seront affermées 2 000 ; et elles souffriront cette augmentation, parce qu’on leur fera porter, en n’y épargnant rien pour la culture, tout ce qu’elles seront capables de produire, attendu que la consommation de ce qui y croissait, revenant permise et possible, rien ne deviendra inutile, mais tournera à l’avantage du roi et du public ; ce qui ne se faisait pas ci-devant à beaucoup près, et ce qui est la seule cause de la ruine des peuples, et non les impôts, n’y ayant prince sur la terre qui lève moins sur ses États, que celui qui produit les plus grands effets.
\subsection[{Chapitre VI.}]{Chapitre VI.}
\noindent On peut dire que tout ce qu’on doit résumer de ces Mémoires est que, quelque essentielles que soient à la bonne ou mauvaise disposition d’un pays les qualités du climat et du terroir, cependant l’exemple de l’Espagne et de la Hollande montre évidemment que l’habileté ou la méprise de ceux qui gouvernent y contribue pour le moins autant que la nature. En effet, comme tout consiste dans la croissance des denrées aux pays fertiles, leur production dépend d’une infinité de circonstances, entre lesquelles il est absolument nécessaire de conserver l’harmonie ; en sorte que, manquant à une seule, leur liaison réciproque fait que tout l’édifice est détruit. C’est ainsi qu’on a vu en Allemagne les mines d’argent, qui en fournissaient tout le monde avant la découverte des Indes, s’anéantir elles-mêmes du moment que ce métal étant devenu plus commun, il ne put plus supporter les frais qu’il fallait faire en Europe pour le tirer des entrailles de la terre. Mais ce que la nécessité a fait en Allemagne, la méprise l’a produit en France à l’égard des marchandises dont elle fournissait les Étrangers, et même de celles qui se consomment au-dedans, comme on n’a que trop fait voir dans ces Mémoires. Cette diminution de 5 à 600 millions par an dans ses revenus, tant en fonds qu’en industrie, n’est que l’effet d’une pareille conduite ; en sorte que si on voit une terre, autrefois bien cultivée, entièrement en friche, c’est que les fruits ne pouvant supporter quelque impôt nouveau, il a fallu en abandonner la culture, et anéantir par là tous ceux que le produit en faisait vivre, n’y ayant aucune profession dans la république qui n’attende son maintien et sa subsistance des fruits de la terre. De manière que, lorsqu’il arrive quelqu’un de ces nouveaux impôts, qui ne vont souvent qu’à très peu de chose à l’égard du roi, si toutes les professions du monde entendaient leur intérêt, elles se cotiseraient par tête pour racheter cette nouveauté, et y gagneraient cent pour un, et le roi la même chose. — Mais pour suivre les conséquences de cette ruine de proportion dans l’économie du commerce, on maintient que la Provence a des denrées que l’on ne prend pas presque la peine de ramasser de terre sur le lieu, lesquelles sont vendues un très grand prix à Paris, en Normandie, et autres contrées éloignées ; cependant on n’en fait venir que pour l’extrême nécessité, et la raison est évidente : c’est que dans ce trajet, qui est de 200 lieues, il faut passer par une infinité de villes et lieux fermés, où les voituriers étant obligés de faire les stations marquées ci-devant aux articles des Douanes et des Aides, cela emporte tant de temps, et met les choses sur un pied tel, qu’il faut trois mois et demi pour faire ce voyage, qui ne demanderait pas plus d’un mois ou cinq semaines sans ces obstacles ; ce qui ne pouvant être porté par la marchandise, à cause des frais qui accompagnent une si longue voiture, en fait abandonner le commerce, et par conséquent celui du retour. La Normandie a semblablement des denrées, comme des toiles, très rares et très chères en Provence, que la certitude d’un pareil sort empêche de se mettre en chemin. Cependant on n’oserait presque envisager les suites d’une pareille disposition, puisque cette cessation intéresse, outre les deux contrées d’où les marchandises sortent et arrivent réciproquement, toutes celles où elles passent, à cause de la consommation inséparable des voitures ; et que, rejaillissant ensuite sur toutes les professions du monde, ainsi que l’on vient de dire, il se trouve que toute la république souffre un dommage inestimable d’une cause dont (quand même tous ses autres revenus ordinaires n’en seraient pas altérés) le roi ne tire que très peu de chose, qui, étant réparti par un autre canal sur tous les peuples intéressés, n’irait pas à un sou par tête, au lieu que bien souvent cela leur coûte leur ruine entière. — Ainsi, c’est en vain que le terroir et le climat ; secondés de l’industrie des peuples, sont propres aux productions les plus nécessaires et les plus recherchées de la nature, puisque le manque de proportion dans un édit, surpris par un intérêt indirect secondé d’une recommandation qu’on veut croire innocemment trompée, détruit plus de biens en une heure que toutes ces causes n’en pouvaient produire en plusieurs années. De sorte que ce manque de proportion fait que les terres sont entièrement abandonnées faute de gens qui les cultivent, et que les hommes périssent de faim, manque des biens qui croîtraient sur ces terres s’il leur était permis de les cultiver, bien que ces hommes et ces terres aient réciproquement de quoi se payer l’utilité qu’ils tireraient les uns des autres. En effet, ces hommes paieraient de leur travail manuel les blés qu’ils recevraient de ces terres pour se nourrir, et ces terres donneraient ces blés pour la peine que ces hommes emploieraient à leur culture ; et ainsi de toutes les autres professions de la république, qui par un enchaînement mutuel sont nécessaires les unes aux autres. On peut dire la même chose des années stériles et des abondantes, qui doivent être dans un commerce perpétuel, se fournissant les unes aux autres ce qu’elles ont de trop, pour avoir ce qu’elles ont de moins et qui leur est nécessaire. Mais, comme ce commercé a été interrompu, les proportions dans le prix des denrées ont été entièrement ruinées, et l’on a vu toujours depuis trente ans où une cherté extraordinaire au blé, et autres denrées nécessaires à la vie, qui n’étaient estimées à rien quelques années auparavant, ou une cherté pareille à l’argent, en sorte qu’on ne se le pouvait procurer qu’avec beaucoup plus de denrées que de coutume ; ce qui mettant l’État dans une maladie continuelle, oh ne doit pas s’étonner qu’il ait perdu la moitié de ses forces, comme on maintient qu’il a fait depuis ce temps. Et tout ce manque de correspondance n’arrive, tant entre ces années stériles et abondantes, qu’entre ces terres incultes et ces hommes oiseux et autres semblables, que parce que les deux mouvements pour le change ne se faisant pas immédiatement, mais bien avec la rencontre d’une infinité de circonstances intermédiaires, le désordre qui arrive à une seule, par les causes marquées ci-dessus, en empêche absolument le trajet, comme celui de Provence en Normandie. En effet, les fruits de la terre ne se vendant plus un prix qui puisse supporter les servitudes contractées pour leur culture, ainsi que l’on a dit, le maître n’emploie plus les ouvriers nécessaires à cultiver son fonds, et la terre étant moins cultivée dans les années abondantes, est moins en état de secourir les années stériles. — Outre ce manque de proportion, il y en a encore un autre qui n’est pas moins essentiel, savoir la juste répartition des impôts, à laquelle dérogeant presque continuellement, comme on fait en France, ils deviennent ruineux à l’État, non par leur quotité, mais par leur inégalité, ainsi que l’on a montré dans l’article des Tailles ; et on n’en parlerait pas davantage sans cette grande quantité de créations de nouvelles Charges, dans lesquelles, après que le roi et le peuple, qui ne sont qu’une seule et même chose, quelque fondé jusqu’ici qu’ait été l’usage sur une maxime toute contraire, ont été {\itshape constitués} à un très gros intérêt (y en ayant eu quelqu’unes dont le revenu a presque égalé le prix de l’achat dès la première année), on compte pour rien un article général qu’on a toujours mis à chaque création, exemption de tutelle, curatelle, collecte, logement de gens de guerre, et autres charges publiques, et souvent même exemption de Taille, en renvoyant toutes ces choses sur le reste du peuple, comme si c’était sur un pays ennemi. Et comme ce sont tous les plus riches qui achètent ces Charges, il s’ensuit que tout le fardeau tombe sur les misérables. Ainsi, cette ruine de proportion, entre des personnes qui doivent contribuer également aux charges publiques, fait le même effet dans un État qu’une voiture de 100 000 pesant, qu’on pourrait faire porter à quarante chevaux de Paris à Lyon, mais qu’on chargerait tout entière sur trois seulement : si, après que ceux-ci auraient succombé à la première journée, on les remplaçait successivement par trois autres, il est certain que tous périraient à moitié chemin, sans qu’on en pût accuser l’excès du fardeau à l’égard des quarante chevaux, mais seulement la disproportion à le partager à ces bêtes de somme suivant leur nombre.
\subsection[{Chapitre VII.}]{Chapitre VII.}
\noindent L’autre maxime générale qu’il faut tirer de ces Mémoires, est que la première et principale cause de la diminution des biens de la France vient de ce que dans les moyens, tant {\itshape ordinaires} qu’{\itshape extraordinaires}, que l’on emploie pour faire trouver de l’argent au roi, on considère la France à l’égard du prince comme un pays ennemi, ou qu’on ne reverra jamais, dans lequel on ne trouve point extraordinaire que l’on abatte et ruine une maison de dix mille écus, pour vendre pour vingt ou trente pistoles de plomb pu de bois. Comme cet anéantissement de cent fois davantage que le profit qu’on y fait ne regarde qu’un pays où l’on ne prend nul intérêt, cette conduite, qui, sans cette circonstance, passerait pour une extravagance entière, est un coup d’habileté. Mais, dans un royaume tranquille et entièrement dévoué au service de son prince, il s’en faut beaucoup qu’il faille rien faire d’approchant. Comme les peuples ne le peuvent aider que de ce qui croît dans leurs domaines, et à proportion qu’il y croît, il ne doit point considérer ses États autrement que si tout le terrain lui appartenait en propre, comme en Turquie, et que ses sujets n’en fussent que de simples fermiers. Cependant, outre la raison qu’on vient de dire, qu’on ne le peut payer que de ce qui croît dans le pays, il est constant qu’il y a bien des provinces dont il tire en plusieurs lieux bien plus que le propriétaire ; et pour faire voir combien on déroge à une maxime qui lui serait si avantageuse, il ne faut que considérer comme les choses se passent, et si les terres étant à lui réellement et de fait, on en userait de même à l’égard des fermiers, comme on fait envers les propriétaires. Commençons par les impôts ordinaires, comme les Tailles, les Aides et les Douanes, et puis nous parlerons des extraordinaires.\par
Si toute la généralité de Rouen était au roi en propre, comme il y en avait autrefois une très grande partie, dont se sont formées ces grandes abbayes fondées par les anciens ducs, et que la baillant par contrat à ferme à plusieurs particuliers, il ne leur demandât aucun prix certain, mais qu’il leur dît : — « Quand vous voudrez un muid de vin, il faudra payer dix-sept droits à sept ou huit bureaux séparés qui n’ouvrent qu’à certaines heures et à certains jours ; et si vous manquez de payer au moindre de ces bureaux, quoique vous l’ayez trouvé fermé à votre arrivée, et que vous ne puissiez retarder sans de grands frais, votre marchandise, charrette et chevaux, seront entièrement confisqués au profit des maîtres du bureau, dont la déposition fera foi contre vous quand vous ne conviendrez pas de la contravention. En allant par pays porter votre marchandise, il faudra pareillement faire des déclarations à tous les lieux fermés où vous passerez, et y tarder tant qu’il plaira aux commis de vous faire attendre pour les recevoir, quand vous devriez y employer quatre fois plus de temps qu’il ne serait nécessaire pour faire un tel voyage. De plus, quand vous voudrez vendre votre marchandise aux étrangers, qui ne demanderaient pas mieux que de l’acheter à un prix raisonnable, il me sera permis d’y mettre un impôt si exorbitant, qu’ils seront obligés d’aller s’en pourvoir ailleurs. Ainsi, bien qu’il ne m’en revienne rien du tout, vos denrées vous demeureront en pure perte, avec tous les frais que vous aurez pu faire pour les approfiter ; vous pourrez même souvent les voir périr, surtout vos liqueurs, n’en pouvant trouver un denier, quoiqu’à une journée au plus de votre demeure elles valent un prix exorbitant ; mais c’est que si vous hasardiez d’en porter là, vous pourriez perdre votre peine et votre marchandise, parce que j’ai baillé à ferme de certains droits à prendre sur le passage, pour lesquels il faut beaucoup de formalités fort difficiles à observer, et dans lesquelles les intéressés sont juges et parties ; et pour peu qu’on y manque tout est perdu ; et bien qu’il ne me revienne pas la dixième partie du tort que cela vous fait et à votre marchandise, cependant on me fait entendre qu’il est de mon intérêt que les choses aillent comme cela. De plus, il me faut payer par an une certaine somme ou quantité d’argent, qui ne sera point à proportion des terres que vous tiendrez de moi, de manière que vous payerez souvent le double, en tenant seulement cinq arpents, de ce qu’un autre, dans la même paroisse, paie en en faisant valoir trente. Mais il vous faut acheter la protection de ceux qui font la répartition, tant en général qu’en particulier, lesquels sont dans une entière possession de ne garder aucune justice en ce rencontre. Outre cela, il faut que vous vous gardiez bien de me payer régulièrement à l’échéance du terme, car ce serait le moyen de vous ruiner, attendu que ceux à qui je baille ces sortes de soins ont intérêt qu’il se fasse des frais pour recouvrer les paiements ; de façon que bien que ce soit un mal que ces sortes de frais, c’en est toutefois un moindre que d’être sujet toutes les années à une augmentation du prix de la ferme, qui est inséparable de la facilité du paiement. Il est encore nécessaire de vous tenir clos et couvert, et, si vous avez de l’argent, de le cacher ou l’enterrer, au lieu de trafiquer, de peur de tomber dans ces inconvénients d’augmentation de ferme ; et même il est nécessaire de ne pas mettre sur votre terre les bestiaux qui la pourraient engraisser. Il en faut user de même à l’égard de la consommation ; c’est-à-dire que dans la dépense, tant pour la bouche que pour les habits de vous et de votre famille, il est besoin d’affecter une grande montre de pauvreté. Enfin, comme ce fermage est très mal réparti et plus mal payé, et par nécessité et par affectation, il vous faut tous les quatre à cinq ans en faire la collecte, dans laquelle, si vous n’êtes pas tout à fait ruinés (comme il arrive en une infinité de cas semblables), vous en serez très incommodés ; car ni vous, ni vos confrères, n’êtes point quittes en abandonnant la ferme et tout ce que vous pouvez avoir vaillant, et souvent il faut périr dans une prison pour ne pouvoir payer un fermage quatre fois trop fort, pendant que vous avez des voisins qui ne paient pas la vingtième partie de ce qu’ils devraient porter. »\par
Quelques obligations qu’une infinité de personnes assez connues dans le monde aient à la situation présente, il est pourtant nécessaire que pour la défendre ils fassent de deux choses l’une, ou qu’ils nient que ce soit là l’état d’aujourd’hui, ou bien qu’ils disent que c’est la meilleure manière de faire valoir les biens d’un souverain, et que c’est entendre parfaitement bien ses intérêts que d’en user de la sorte. Mais comme, pour parler sérieusement, il est impossible de tenir aucun de ces deux langages, à moins d’entreprendre de renverser le sens commun, ou d’imposer à la foi publique, on continuera encore un peu cette peinture de l’état présent, et l’on ajoutera qu’un prince qui ferait valoir ses États de cette manière serait assurément très mal servi, et que ses sujets lui pourraient dire avec raison : « — « Sire, quoique vous ne vouliez qu’être payé, et recevoir le plus d’argent qu’il est possible, la manière dont vous en usez semble être inventée pour nous ruiner et vous aussi ; car, comme toute notre richesse et la vôtre ne peuvent provenir que de la vente des biens qui croîtront sur votre terre, ce que vous proposez ferait tout périr. Mais que Votre Majesté compte ce qui lui en viendrait de la façon qu’elle l’entend, et nous le lui doublerons, pourvu qu’elle nous laisse la liberté de vendre et de consommer ce que bon nous semblera ; ce qui nous sera bien facile, puisque nous ferons trois fois plus de débit de cette sorte que de l’autre. » — Quelque ridicule que soit cette description, il est pourtant vrai que c’est justement l’état présent des choses ; et que, quoique extrêmement dommageable au roi et au peuple, on préfère tous les jours ce parti à l’autre, par des raisons qui ne sont que trop connues : et ce qu’il y a d’effroyable, c’est qu’il n’y a pas jusqu’à la moindre denrée à qui on ne fasse souffrir le même sort, d’en ruiner absolument la consommation ; de manière qu’on n’a pas poussé cette peinture aussi loin qu’est l’original, à beaucoup près. Et pour comble de désordre, on fait entendre au roi et à MM. les premiers ministres, qui sont les premiers surpris, que c’est par une pareille manœuvre qu’on augmente les revenus de Sa Majesté, en supposant un impossible, que pour enrichir un prince il faut ruiner les peuples, en leur causant vingt fois autant de perte qu’on fait passer de profit dans les coffres du prince, qui est l’état des choses d’aujourd’hui, comme on a pu voir par tout ce qui a été dit précédemment. Le déchet que la manière de lever les revenus du roi cause au peuple, n’allant au profit de personne (sans quoi on ne lui déclarerait pas une si forte guerre, puisque, si le prince ou ceux qui se mêlent dans la levée de ses revenus, faisaient passer entièrement sur sa tête ou sur la leur la diminution qu’ils causent, l’État ne ferait aucune perte, lui étant indifférent, de même qu’au roi, par qui et comment les biens soient possédés, pourvu qu’ils existent, attendu que dans ce cas il pourrait toujours s’en aider dans les occasions pressantes comme est celle d’aujourd’hui), il n’est donc pas question de faire miracle pour former au roi cent millions de rente plus qu’il n’a, en rétablissant à ses sujets le double de leurs biens, tels qu’ils les avaient autrefois ; il est seulement nécessaire de laisser agir la nature en cessant de lui faire une perpétuelle violence par des intérêts indirects, qui, se couvrant d’une confusion continuelle, dérobent le point de vue de la cause des misères, et bouchent par de hautes protections toutes les avenues aux remèdes : si bien que, quoique les maux soient constants, et qu’il soit même permis de les déplorer, il n’est pas moins criminel de vouloir remonter jusqu’à leur source, et d’en parler, qu’il n’est en Turquie de disputer de la religion du pays. Voilà pour les revenus {\itshape ordinaires}. — Et pour les {\itshape extraordinaires}, on peut dire que l’on garde encore une conduite opposée à celle que l’on observerait si toute la France était au roi. En effet, il est arrivé que pour une somme très modique qu’il a reçue, on a permis à l’acquéreur d’une nouvelle Charge de prendre sur le peuple, qui est le propre bien du roi, son intérêt au denier quatre ou cinq. Or, il est certain que ce même peuple étant le fonds du roi, c’est la même erreur que si le propriétaire d’un héritage assignait sur son fermier une rente au denier quatre, et crût par là ne rien devoir : il est constant qu’il gagnerait bien davantage à prendre la constitution sur lui au denier dix-huit. De plus, une nouvelle Charge ne pouvant être créée sans diminuer, les anciennes, le corps de l’État, qui n’est composé que de particuliers qui les possèdent, en souffre encore extrêmement. De façon qu’il se trouve que, pour 10 000 écus que le roi reçoit d’une nouvelle création, qui amène trois articles, savoir : les droits à prendre sur le peuple, la décharge des impôts publics sur le reste du peuple, à cause des privilèges attachés à tous les nouveaux offices, et le tort enfin que cela fait aux anciennes charges ; il se trouve, dis-je, que pour les 10 000 écus que le roi reçoit ainsi, le royaume souffre une diminution de plus de cent mille écus en sa totalité. Par exemple, la collecte de la Taille étant un fardeau de la conséquence qu’on a représenté, un nouvel office du plus vil prix, acquis par un homme riche, renvoie, par son privilège, cette servitude sur un pauvre qu’elle ruine tout à fait. Or, il en va de la pauvreté comme des diamants ; il y a de certains degrés où tout nouveau surcroît double et triple son effet, tant pour celui qui les souffre, que pour l’État. En effet, un laboureur qui n’a que cent écus pour acheter des bestiaux, pour charger sa terre d’un fermage de mille livres, ne peut en être privé sans se ruiner, ainsi que son maître, ses créanciers et leurs créanciers jusqu’à l’infini, parce que tout le produit d’une terre dépendant de l’engrais, du moment qu’il cesse, on n’en tire pas les frais : en sorte que l’enlèvement de ces cent écus à ce pauvre laboureur, pour les frais d’une collecte, cause une perte de cinq ou six mille livres au corps de l’État ; et cela non seulement pour une année, mais pour plusieurs de suite, puisqu’une terre délaissée est longtemps à se remettre, quand même ces désordres cesseraient, loin de recevoir de l’augmentation, comme ils font tous les jours ; au lieu que cent écus payés par un homme riche ne font pas le moindre mouvement dans l’État. Cependant, la maxime d’aujourd’hui, par la création de nouvelles Charges, fait si bien régner la disproportion dans les impôts, que l’on peut conclure qu’il est certain que dans tout l’argent que le roi reçoit, tant à l’ordinaire qu’à l’extraordinaire, le peuple ou l’État, qui est le propre bien du roi, est constitué en autant de revenu, et souvent davantage, que le roi reçoit de capital, le déchet ou le surplus n’allant au profit de personne, mais étant entièrement anéanti, ainsi qu’on a fait voir.
\subsection[{Chapitre VIII.}]{Chapitre VIII.}
\noindent Enfin l’on conclut tous ces Mémoires par l’article le plus important, qui est de fournir au roi, présentement et sans délai, tout l’argent nécessaire pour mettre fin à une guerre que l’envie de sa gloire lui a seule attirée, et qui n’est soutenue avec tant d’obstination par ses ennemis, que parce que les mémoires qu’ils ont de ce qui se passe dans le détail des affaires du royaume, leur apprennent que les fonds dont on tire les moyens extraordinaires pour la soutenir, ne peuvent pas durer longtemps. En effet, que l’on compte l’intérêt que le roi fait, celui dont a chargé les peuples la diminution que la création des nouvelles charges a apportée aux anciennes, le désordre de leurs exemptions, qui a renvoyé tous les impôts sur les misérables, et, par conséquent, ruinant les proportions, a anéanti pour beaucoup plus de biens que le roi n’en pouvait recevoir, ainsi que l’on a fait voir aux chapitres précédents, il se trouvera que Sa Majesté, ne faisant qu’un seul et même corps avec son État, n’a pas reçu un denier qui n’ait autant d’intérêt constitué sur elle ou sur le peuple, ou même anéanti entièrement, qu’elle a reçu de capital. Et quand un pareil mécompte ne serait qu’au quart de ce qu’il est effectivement, il est impossible qu’il puisse être de durée.\par
Pour revenir donc aux manières de fournir de l’argent comptant au roi, on maintient que l’exécution du projet traité dans ces Mémoires en est un moyen très certain. En effet, quel plus court chemin pour être payé de son débiteur, que de lui faire venir du bien, ou de lui aider à liquider une succession embarrassée ? Et il ne faut pas dire que cela demande quelque délai, et que quelque utilité qu’il vienne au peuple de la certitude morale des Tailles et de la liberté entière des chemins, ce qui serait par la réunion d’une partie des Aides et Douanes comme elles étaient il n’y a que trente-cinq ans, et le surplus comme dans tous les autres royaumes du monde, ce ne peut être que dans un an au plus tôt que l’on en verrait les effets. Car ou soutient formellement qu’il ne faut que vingt-quatre heures, et que l’{\itshape édit} qui porterait que chaque Élu prendrait un certain nombre de paroisses à asseoir la Taille suivant l’{\itshape occupation} de chacun, soit fermier ou propriétaire, eu égard à la somme répartie sur toute l’Élection, sans nulle considération de qualité, et que quiconque porterait la somme dès le premier mois à la Recette, serait exempt de la collecte, ferait le même effet que si on venait annoncer à divers particuliers très misérables qu’il leur vient d’échoir une succession d’immeubles très opulente : car bien qu’il ne fût dû aucun fermage qu’un an après, cependant ils ne laisseraient pas de s’en sentir dès le même moment, parce que tout le monde leur prêterait très volontiers, voyant la certitude d’être remboursé, et du capital et des intérêts, tout au plus après l’année échue. Tout de même, la crainte étant levée, par cet édit, d’être exposé en proie à ses ennemis ou envieux par toute montre d’opulence, qui est néanmoins inséparable et du commerce et du labourage, on verrait un fermier de terres emprunter de tous côtés pour charger sa ferme de bestiaux, qu’on lui prêterait très volontiers, voyant qu’il ne pourrait plus être saisi pour la Taille de ses voisins, ni la sienne être augmentée d’une façon exorbitante parce qu’il mettrait ses terres en valeur. Cependant, comme cela produirait un engrais qui est toujours suivi d’une bonne levée, il serait en état d’en partager le profit avec ceux qui lui auraient aidé. L’artisan qui n’ose se découvrir, mettrait aussitôt un cheval sur pied pour faire son commerce, moitié à crédit, comme ils font tous, et moitié autrement, sans craindre que cela le fît accabler de Taille, comme c’est l’ordinaire, ni qu’il fût obligé tous les quatre ans de se voir ruiné par la collecte, qui lui emporterait, par la perte de son temps et les autres misères attachées à cet emploi, tout ce qu’il aurait pu gagner les années précédentes ; et les uns et les autres, ayant fait quelque profit, ne craindraient plus de se nourrir et vêtir suivant leurs facultés, parce que c’est une chose fort naturelle ; ce qui, faisant gagner le marchand et l’artisan des villes, les mettrait en état de consommer les denrées provenantes du labourage, et rétablirait par là cette circulation qui fait le maintien des États dont le terroir est fécond, mais d’une fécondité tout à fait inutile lorsqu’il est impossible ou défendu de le faire valoir, comme on soutient que c’est aujourd’hui le cas de plus de la moitié de la France ; ce qui fait sa misère, et non les impôts, qui sont moindres à proportion (ainsi que l’on a dit) qu’en nul État de l’Europe. — Et l’autre édit qui joindrait les Douanes sur les sorties, et les Aides aux Tailles, c’est-à-dire qui ordonnerait que celui qui payait six livres de Taille en paierait huit ou neuf, et que le laboureur qui en payait 100 livres serait à 140, ce qui l’exempterait de toutes les circonstances et de tous les effets de ces deux impôts, dont on a assez parlé, lesquels coûtaient à l’un et à l’autre vingt fois, voire trente fois davantage, ferait aussitôt sortir tous les vignerons et tous les autres artisans de la dépendance des vins du fond de leurs tanières, pour rétablir les vignes ; en quoi ils seraient aidés par tout le monde, tant maîtres qu’autres, qui seraient assurés d’être remboursés par la récolte, les chemins étant devenus libres pour pouvoir porter les vins où il n’en croît point, et où il ne s’en consommait point, que la vingtième partie de ce qui y eût été possible si les abords n’en eussent pas été absolument défendus ; et les propriétaires recommenceraient à compter dans leur bien chaque arpent de vigne pour 1 000 livres, comme ils faisaient autrefois, et non pour rien, comme ils font présentement, et contracteraient sur ce pied, tant en vendant qu’en achetant ; plus de cent mille cabarets paraîtraient en moins de huit jours, y en ayant eu deux ou trois fois davantage d’anéantis depuis trente ans ; et comme il n’y a point de cabaret qui ne mène dix ou douze professions après lui, comme le boucher, le boulanger et autres, ce serait plus d’un million de familles que ce seul article remettrait en mouvement, et par conséquent tirerait de misère ; et ainsi de tous les autres héritages à proportion, et des professions qui en attendent leur subsistance. Voilà donc tout le monde riche en vingt-quatre heures, et tout l’argent en mouvement. Il n’est plus question que de faire voir comme le roi y peut participer avec autant de diligence, qui est la chose du monde la plus aisée, parce qu’elle est très naturelle, et comme une conséquence nécessaire de ce premier mouvement.\par
On crie de tout temps en France contre les impôts, et les riches bien plus que les pauvres, à cause de cette malheureuse coutume qui s’est introduite, de n’avoir aucune justice dans la répartition des charges publiques ; ce qui, mettant les choses sur un pied, que s’en défend qui peut, plus un homme est puissant, moins il en paie, parce qu’il est plus en état de s’en exempter. Et comme entre les moyens dont on se sert pour se procurer ce privilège, le bruit et les plaintes sont un des plus considérables, elles se font bien mieux entendre dans la bouche des riches que dans celle des pauvres, ce qui fait que ces derniers sont toujours accablés ; ce qui, retombant par contrecoup sur les riches (ainsi que l’on a fait voir), ruine enfin les uns et les autres. Un premier ministre ne doit donc pas se mettre beaucoup en peine si on crie, mais seulement si on a sujet de crier. Or, il est constant que lorsqu’on prend tout le bien d’un homme, comme on peut dire qu’on a fait ces années dernières, quand, ou par des suppressions, ou par des taxes, on a enlevé tout le vaillant d’un officier en le privant d’une charge qu’il avait achetée de bonne foi, et sans qu’il y eût aucun cas particulier qui le distinguât de toutes les autres personnes revêtues de dignités bien plus considérables, à qui on n’a rien demandé ou peu de chose ; il est constant, dis-je, que cet homme a très grand sujet de déplorer son malheur, les besoins de l’État demandant que les peuples aident de leurs biens et de leurs personnes, mais jamais que les uns contribuent de tout leur vaillant, pendant qu’il en coûte beaucoup moins aux autres ; ce qui, étant un monstre dans la justice distributive, ruine absolument un État par les raisons tracées ci-dessus : — à quoi on peut encore ajouter que cette conduite, établissant pour principe qu’il n’y a aucune règle certaine pour la contribution des Charges, cela les rend toutes susceptibles à tous moments d’un entier anéantissement ; ce qui, les jetant dans une juste crainte de cette destinée, les diminue extrêmement de prix, sans que le roi, ni personne, en profite. Lorsque le cardinal de Richelieu eut doublé en dix ans tous les revenus de la couronne, on cria très fort contre lui ; mais c’était avec la dernière injustice que l’on faisait ces plaintes, car cette augmentation était l’effet de celle de tous les biens du royaume, qui avaient plus que doublé pareillement : il fut vendu sous son ministère des Charges dix fois ce qu’elles avaient coûté aux personnes mêmes qui en étaient revêtues. L’on se plaint extrêmement présentement, et il n’y a rien de si commun dans la bouche du peuple, tant riches que pauvres, que de parler du malheur du temps ; mais c’est avec fondement, puisque depuis trente ans c’est justement le contrepied de ce qui arriva sous le cardinal de Richelieu, y ayant des charges, sans parler des terres, qui ne sont pas à la dixième partie de ce qu’elles étaient en 1660. Ceci donc posé, c’est une grande avance pour Sa Majesté que ses peuples soient riches, pour en tirer du secours, comme on maintient qu’ils peuvent être en vingt-quatre heures, par la simple publication de deux ou trois édits qui, ne congédiant ni fermiers ni receveurs, rendront seulement {\itshape les chemins libres} et {\itshape les impôts justement répartis} ; ce qui, étant de droit divin et naturel, est observé chez toutes les nations, même les plus barbares, hormis en France, le plus poli royaume du monde, et y a causé seul tous les malheurs dont on se plaint.\par
À l’égard des moyens de tirer tous ces secours, quand il n’y en aurait point d’autres que ceux dont on s’est servi jusqu’ici, comme de créer des charges et autres semblables, que l’on soutient et que l’on a montré être très contraires aux intérêts de l’État, on peut assurer que ce serait beaucoup de chemin fait de mettre les peuples en pouvoir de les acheter, puisque, rétablissant ces mêmes peuples en possession de leurs biens que l’on peut dire être anéantis, les conséquences en sont naturelles, savoir l’achat des choses qui font plaisir, entre lesquelles les dignités tiennent le premier lieu. Or, comme la vanité y a plus de part qu’autre chose, on ne la satisfait qu’à proportion qu’on est en état de le faire, c’est-à-dire que le revenu et la valeur des fonds, qui donnent l’être à tous les autres biens, mettent en pouvoir de le faire : c’est ce qui fait que les Charges ont haussé et baissé, depuis que la création de la {\itshape paulette} les a rendues immeubles, conformément à tous les fonds.\par
Mais ce n’est pas de ces moyens dont on prétend se servir ; on n’en veut point employer aucun qui ne soit utile de lui-même à l’État, en sorte que le peuple, après avoir payé ce qu’on lui demandera, se trouvera dans une situation plus avantageuse qu’il n’était auparavant ; et cela, jusqu’à ce que les revenus ordinaires aient gagné un pied qui suffise à toutes les dépenses extraordinaires d’aujourd’hui, ce que l’on soutient devoir arriver avant deux ou trois ans, parce que ces revenus ordinaires, étant mis sur le pied de ceux des peuples, ils hausseront avec eux comme ils avaient fait depuis deux cents ans jusqu’en 1660.\par
Mais pour revenir à ces moyens extraordinaires d’aujourd’hui, c’est qu’entre les causes qui ont produit cette grande diminution de biens de toute la France, outre celles que l’on a marquées par l’incertitude des Tailles et la vexation des Aides et des Douanes, qui seront levées de la manière que l’on a dit, il y en a de particulières, qui, ne faisant pas moins de mal, seraient rachetées sans presque nul mouvement par les peuples, argent comptant, le plus volontiers du monde ; en sorte qu’ils n’auraient pas sitôt donné une pistole, que cela leur en fournirait deux ou trois de revenu, sans qu’il fût besoin de venir à des emprisonnements et à des violences pour de pareils recouvrements, comme on a vu pour tous les autres. Par exemple, dans les villes taillables, étant nécessaire que l’industrie porte une partie des charges, comme elle n’a point d’autre arbitration que la fantaisie ou la vengeance de ceux qui asseyent la Taille, il s’y fait des désordres effroyables : cette conduite, ruinant tout l’un après l’autre, il n’y a rien qu’elles ne donnassent pour se rédimer de cette vexation, en obtenant permission de labourer par une somme certaine qui se prendrait en autre assiette, et celles qui l’ont pu obtenir par des soumissions, excédant de beaucoup leur Taille, pour des travaux publics, se sont relevées entièrement de leurs misères. Il ne faudrait qu’écouter celles qui se voudraient mettre en {\itshape Tarif}, et les offres qu’elles feraient pour cette obtention : on est assuré qu’il s’en présenterait une grande quantité, pourvu que les cours des Aides et les receveurs des Tailles ne fussent pas écoutés, à cause de la fin que cela met à toutes les vexations ci-devant marquées, dont il leur revenait environ un pour cent du tort que cela faisait au peuple. Cet article produirait plus d’un million, qui n’est rien, comme on en convient, pour les besoins présents, mais qui mettrait ces lieux-là, par l’abondance que cela y porterait, en état de fournir d’autres secours sur-le-champ ; de façon qu’on ne cite pas ceci pour la somme, mais seulement pour l’exemple, et pour montrer qu’il est possible de mettre le peuple, après avoir donné de l’argent, en une meilleure situation qu’il n’était auparavant, en tirant cette amélioration des trésors de la terre, où ils étaient anéantis par les méprises dont on a tant parlé, qui ont été si loin, que l’on a souvent mis en vente ces anéantissements à un pour cent, ainsi qu’on est obligé d’en convenir. Or, comme il y a pour 500 millions et davantage de diminution en France dans ses revenus depuis quarante ans, par de pareilles causes, il s’en faut beaucoup que cet article des Tailles en soit l’unique principe ; de façon qu’il y a bien des sommes à recevoir au roi pour former le capital d’un rachat si considérable et si utile au peuple. De plus, il y a une infinité d’impôts dont le roi ne tire presque rien, qui causent un mal extraordinaire au commerce, dont les commerçants rachèteraient l’exemption à un denier très haut, et y gagneraient encore ; l’on en indiquera pour plus de 40 millions payables en moins de six mois, pourvu que l’on voulut cesser les nouvelles créations, qui mettent toutes les familles dans la dernière extrémité : car comme les charges forment un effet considérable dans l’État, étant tirées hors du commerce, par la création des nouvelles, cela ruine tous ceux qui en sont revêtus, lorsqu’ils sont dans l’obligation de les vendre, ainsi que leurs créanciers, jusqu’à l’infini.\par
Et enfin, outre toutes ces ressources, pourquoi le roi n’en userait-il pas dans ses besoins comme tous les hommes du monde ? Qu’il prenne de l’argent en rente au plus bas denier que faire se pourra. — Les deux édits dont on a tant parlé, une fois publiés, feraient que tout le monde s’empresserait de lui en donner ; parce que, outre que c’est une suite nécessaire de la richesse du peuple qui augmenterait considérablement, c’est que l’augmentation certaine des biens du roi assurerait dans l’esprit de ces mêmes peuples, et le capital, et les arrérages. Et supposé qu’il lui fallût 50 millions par an d’extraordinaire jusqu’à la fin de la guerre, et qu’il fût dans l’obligation de tout prendre en rente, de quoi on ne convient pas, quand elle durerait encore quatre ans, ce ne serait que de 10 millions de rente qu’il se serait endetté, et les peuples ou l’État de rien du tout, sans parler du rétablissement de leurs richesses. Or, on demande si, depuis quatre ans que la guerre est commencée, c’est là la situation des choses. On est bien assuré qu’il en coûte plus de cent millions de rente au roi ou à l’État. — Le lendemain de la publication de ces édits, les denrées, reprenant leur ancien prix, reformeront les revenus dont se tirent les capitaux des parties de rente ; et la création des nouvelles Charges qui sera cessée, ôtant d’un côté le commerce de l’argent au denier dix, les traitants le faisant valoir sur ce pied (dont tout le déchet du prix ordinaire retombait sur le roi), et de l’autre remettant toutes les charges dans le trafic ordinaire, cela rétablira les choses dans l’ancien cours, qui est de faire empresser les peuples à constituer sur le roi. Mais il est nécessaire, pour maintenir ce commerce, d’y conserver la bonne foi, pour l’intérêt même de Sa Majesté, sans que l’autorité souveraine y puisse introduire aucune jurisprudence singulière lors du racquit, ainsi qu’on a vu autrefois, qui ne fût reçue entre deux particuliers, de même que dans les armées il faut absolument payer les vivres sur le pied courant, si on veut qu’elles puissent subsister ; car bien qu’il n’y eût rien de si aisé que de les avoir pour rien une première fois, comme de cette manière les pourvoyeurs n’y reviendraient plus, cela ferait tout périr. Il serait encore nécessaire qu’il y eût un bureau particulier pour le rachat de ces sortes de rentes par le roi même, en perdant, par les propriétaires, trois mois de leur intérêt : ce serait le moyen d’y faire apporter tous les dépôts de France, ainsi que de l’argent des mineurs, voyant qu’on serait assuré d’avoir son intérêt et de retirer son capital sans nul risque quand on voudrait. Il serait encore à propos que ces sortes de rentes ne pussent jamais être saisies pour la dette des transportants, ne conservant ni suite ni hypothèque, non plus que l’argent même ; en sorte que tout paiement fait et endossé sur le premier instrument serait bon et valable, soit pour le capital ou les intérêts, hormis en cas de stellionat ou de larcin, lorsqu’il y aurait une dénonciation précédente. On est certain qu’on en apporterait plus qu’on ne voudrait ; et le roi, dès la première année, par le moyen des édits dont on a parlé, aurait plus qu’il ne faudrait d’augmentations pour payer l’intérêt de 50 millions ; dans la seconde, pour payer celui de plus de 100 millions ; et dans la troisième, ses revenus ordinaires iraient à plus de 150 millions ; cette augmentation continuant jusques à ce qu’ils eussent doublé, même en temps de guerre. Et tout cela, parce que la consommation redûment permise et possible par la liberté des chemins et la certitude et juste répartition des Tailles, une ferme de 1 000 livres, qui ne paiera cette année à Sa Majesté que 100 livres de Taille, et 40 livres pour sa quote-part du rachat des Aides, et Douanes sur les sorties et passages, reprendra son prix d’autrefois de 2 000 livres : ainsi ce sera sur le même pied d’impôt 280 livres, sans que le propriétaire se puisse plaindre de cette augmentation, qui ne sera que reflet de celle de sa richesse. Cet article seul va à plus de 50 millions d’augmentation par an, et les Gabelles et Domaines, qui marchent comme les richesses du pays, recevront un même accroissement, puisque la dépense de bouche étant un des premiers effets de l’opulence principalement chez les pauvres, qui font la plus considérable consommation de la Gabelle, il est nécessaire qu’elle ressente les effets de ce changement de scène.\par
Pour les Domaines, le papier de formule et le contrôle y tenant une place essentielle, ils augmenteront à proportion des fonds qui seront contestés en justice, dans les occasions, suivant qu’ils seront en valeur ; au lieu que la plupart, bien loin de faire naître des procès pour la propriété, étaient presque à l’abandon. Et quand le roi aura 100 millions de rente plus qu’il n’avait, ce sera parce que ses sujets auront 500 millions plus qu’ils n’ont présentement, et qu’ils avaient autrefois, dont ils n’ont été privés, sans que personne en ait profité, qu’à cause qu’on a quitté les manières usitées de lever les droits du prince dans tous les États du monde, tant anciens que modernes, pour en prendre de toutes particulières et inconnues à toute la terre, dont le récit fait horreur ainsi que les effets, qui ne sont rien autre chose que de faire périr de faim et de misère un peuple très laborieux, dans le plus fertile pays du monde, et sous le meilleur prince qui fut jamais ; et ce qu’il y a de plus surprenant, ces malheureux effets étant produits par de très habiles et de très intègres ministres. Mais, c’est que le gouvernement d’un État, à l’égard des finances, n’étant autre chose que la régie du commerce, tant du dedans que du dehors du royaume, ainsi que de l’agriculture, pour en tirer les droits du prince, cela ne se peut faire que par une parfaite connaissance du détail, et une infinité de circonstances qu’il leur est impossible de connaître par eux-mêmes. Ainsi toutes les mesures qu’ils peuvent prendre dépendant absolument des faits particuliers, s’ils n’arrivent chez eux que très corrompus, c’est une situation dont on peut tirer toutes les conséquences. Et comme il y a longtemps que ce mal a commencé, s’étant facilement introduit, parce que les effets n’en étaient pas à beaucoup près si pernicieux dans son principe, ce qui l’a fait recevoir plus aisément ; il s’est tellement enraciné, et s’est formé tant de créatures, que tout le monde concourt tous les jours auprès d’un premier ministre pour les augmenter, et pour s’opposer à leur cessation. En effet, on maintient qu’on a établi des impôts, et on l’a assez fait voir, qui ont fait quatre fois plus de tort au roi qu’ils ne lui ont profité, et cent fois plus de perte au peuple en général qu’il n’en revenait d’utilité aux entrepreneurs. Cependant, il est presque impossible qu’une ruine si générale ne soit pas la conséquence d’intérêts si peu considérables ; et cela parce que l’intérêt particulier étant toujours beaucoup plus sensible et bien mieux ménagé que le général, on emploie toutes sortes de moyens pour le soutenir, et que le peuple n’a personne pour se faire entendre, l’habileté consistant à cacher le point de vue qui peut faire connaître d’une manière évidente que ce profit que l’on fait est cela même qui ruine le roi et le peuple, Ainsi donc, voilà la malheureuse situation d’un premier ministre, c’est de voir toute la terre en mouvement et toute la faveur en action, non seulement pour le tromper, mais pour l’obliger à immoler et son prince et le peuple à des intérêts particuliers, n’étant applaudi, par tous ceux qui prétendent former seuls le monde, qu’à proportion qu’il donne dans cette surprise ; et il ne pourrait même entreprendre de faire le moindre pas en arrière sans s’attirer tous ceux qu’on vient de dire sur les bras ! Car, en suivant les routes tracées, de quelques dérèglements qu’elles soient accompagnées, il n’est garant de rien, et les agréments qui accompagnent la place qu’il remplit, auxquels il est très naturel d’être sensible, ne courent aucun risque ni pour lui ni pour les siens, quelques désordres qui arrivent ; au lieu que dans la moindre nouveauté, ayant tous ceux dont on vient de parler déchaînés contre lui, il prendrait tous les accidents sur son compte, et il est bien difficile qu’il les pût ou prévoir ou conjurer, parce que, ne pouvant faire un pas dans cette occasion sans une parfaite connaissance du détail de tout le royaume, il ne la saurait avoir sans la pratique de tous les états et de toutes les conditions, ce que l’on n’a jamais vu dans aucun ministre ; de façon que, ne l’ayant point par lui-même, il est pareillement dans l’obligation de ne s’en rapporter à personne, par les raisons qu’on vient de dire.\par
Ce qui fait espérer le succès de ces Mémoires est qu’ils découvrent sincèrement ce détail, dont la parfaite connaissance est si avantageuse au roi et au public, et qu’on prenait tant de peine à cacher à ceux qui pouvaient arrêter le désordre, — dont le premier pas du remède est de faire connaître, comme l’on fait, qu’il n’est point besoin de mouvement extraordinaire, ni de rien mettre au hasard, mais seulement de permettre au peuple d’être riche, de labourer et de commercer, en en faisant part au roi, — sans qu’il soit nécessaire d’autre chose que d’arrêter ceux qui avaient intérêt à ruiner tout, et que d’obliger les fermiers de Sa Majesté à recevoir en un seul paiement, sans nuls frais, des receveurs des tailles, le prix de leurs fermes, avec tel profit qu’il plairait au roi de leur donner, et pour lequel, après avoir accablé les peuples, ils étaient souvent obligés de faire banqueroute eux-mêmes. Ou plutôt, comme toutes les fermes ne se tiennent plus à forfait, à cause des diminutions prétendues par les fermiers, il n’est point nécessaire de mouvement pour changer la nature des impôts qui les composent, ce qui sert encore de réponse à l’objection de ceux qui prétendent qu’il faut attendre la paix pour faire ces changements.\par
Ainsi, pour faire avoir au roi tout l’argent nécessaire pour la dépense, tant ordinaire qu’extraordinaire, il est seulement besoin de tirer du néant, en faveur de ses peuples, tous les biens anéantis depuis trente ans. Et comme depuis ce temps on maintient que pour une pistole d’augmentation que le roi reçoit il en coûte dix-neuf en pure perte au peuple, ce sont ces dix-neuf que l’on veut faire revivre en vingt-quatre heures ; et si, lorsque Sa Majesté crée ou des rentes sur la maison de ville de Paris, ou des Charges qui donnent du revenu, elle ne doute pas qu’elle ne reçoive de l’argent de ceux qui les veulent posséder, avec combien plus de raison doit-elle espérer, en donnant plus de 500 millions de rente à ses peuples, d’en recevoir bien davantage, avec encore cette différence que c’est, dans le premier cas, toujours sur ce même peuple que se forme le fonds en l’état qu’il est, avec même souvent la méprise traitée ci-dessus, c’est-à-dire que la demande même de l’argent porte avec elle la diminution des fonds, au lieu que dans l’espèce que l’on propose, c’est justement tout le contraire ; — et que, comme par ci-devant plus le peuple payait d’argent à l’extraordinaire, plus il augmentait sa ruine, en achetant en quelque manière sa destruction ; dans cette occasion, à chaque somme que le roi recevra à l’avenir de la façon proposée par ces Mémoires, ce sera autant de diminution que la misère souffrira ; — parce que comme la cause en était augmentée dans l’un, elle sera anéantie dans l’autre. — Et à l’égard des recouvrements pour les avances que l’on pourra faire au roi sur de pareils fonds, au lieu de venir mettre la désolation partout, comme ci-devant, parce que les sommes demandées portaient avec elles l’impossibilité de payer, en ruinant les principes d’où se forme l’argent chez le peuple ; tout au contraire, l’argent que l’on demandera en ouvrira la source, qui était tarie chez ce même peuple. Et pour l’avance des revenus ordinaires, elle est d’autant plus aisée qu’elle n’était ci-devant, qu’il est d’autant plus facile à un fermier ou propriétaire d’une terre de 1 000 livres, dont les meubles, fruits ou levées étant sur la terre, valent pour l’ordinaire 3 ou 4 000 livres, d’avancer environ 100 livres huit mois devant qu’il les dût, qu’à un Traitant d’avancer plusieurs fois plus qu’il n’a vaillant.\par
Pour finir et réduire ces Mémoires, on demeure d’accord qu’il est ridicule d’avancer que le roi puisse tirer le double de ce qu’il lève à présent, les choses demeurant en l’état qu’elles sont ; mais il est également opposé à la vérité de nier que le propriétaire d’un arpent de vigne, autrefois de valeur de 100 livres de rente, et présentement abandonné, ne veuille ou ne puisse pas donner une pistole, voire deux, à Sa Majesté, du moment que la cause de cet anéantissement sera levée, en quoi il recevra bien plus d’utilité que Sa Majesté même. Ainsi, pour nier ce qui est contenu dans ces réflexions, savoir, que la France est diminuée de plus de moitié dans ses revenus depuis trente ans, sans que personne en ait profité ; que, bien loin que l’augmentation des revenus du roi en soit cause, ils ont bien moins haussé depuis 1660 qu’ils n’avaient fait depuis deux cents ans en pareil espace de temps ; que même cette augmentation coûte au peuple dix pour un de ce qu’il en revient au roi, ce qui n’a jamais eu d’exemple ; qu’il n’y a point de prince sur la terre qui ne tire beaucoup davantage à proportion de ses sujets, et qu’il n’y a point pareillement de peuple à qui il en coûte le quart à proportion, pour les subsides du prince, de ce qu’il en coûte à celui de France ; et qu’enfin le roi peut, en quinze jours, se mettre lui et ses peuples sur le pied de tous ses voisins, c’est-à-dire doubler ses revenus en doublant ceux de ses sujets ; pour nier, dis-je, toutes ces choses ou plutôt tous ces faits, il faut soutenir que la France est autant cultivée et en valeur, à l’égard du commerce et du labourage, qu’elle peut être ou qu’elle a jamais été ; ou que, quand elle le serait davantage, les peuples n’en seraient pas plus riches, et par conséquent Sa Majesté. Or, l’un ne peut être soutenu sans imposer aux yeux de toute la Terre, et l’autre sans renoncer à la raison. À l’égard du délai, qui est où se retranchent les défenseurs, ou plutôt les favoris de la situation présente, si préjudiciable au roi et au peuple, en prétendant que le temps n’est pas propre, il faut renoncer pareillement au sens commun, pour dire qu’un homme qui voit périr plein ses caves de vin, faute de trouver à qui les vendre, a besoin que la paix soit faite pour les porter à douze ou quinze lieues de chez lui, où ce vin vaut un prix excessif, et en rapporter en contre-échange les marchandises du lieu, dont le manque de débit faisait souffrir le même sort aux gens de cette autre contrée. Et à l’égard de la Taille, il ne s’agit d’autre chose que de faire observer les ordonnances, c’est-à-dire empêcher la prévarication. Or, on n’a jamais dit qu’il fallait que la paix fût faite pour être en pouvoir de rendre justice : ainsi ces sortes de raisons ne peuvent être alléguées que par des parties intéressées au maintien de ce désordre.
\subsection[{Chapitre IX.}]{Chapitre IX.}
\noindent I. La Suède et le Danemarck, unis ensemble comme ils étaient il y a cent cinquante ans, sont beaucoup plus étendus que n’est la France ; cependant le produit, tant à l’égard du prince que des peuples, ne va pas à la dixième partie de celui de la France.\par
II. La raison de cette différence est que le terroir de la France est excellent pour produire les denrées nécessaires à la vie, et que celui du Danemarck et de la Suède ne vaut rien du tout.\par
III. Quelque bonne que soit une terre, quand elle n’est pas cultivée, elle est la même à l’égard du propriétaire et du prince, comme si elle ne valait rien du tout.\par
IV. C’est un fait qui ne peut être contesté, que plus de la moitié de la France est ou en friche ou mal cultivée, c’est-à-dire beaucoup moins qu’elle ne le pourrait être, et même qu’elle n’était autrefois, ce qui est encore plus ruineux que si le terroir était entièrement abandonné, parce que le produit ne peut répondre aux frais de la culture.\par
V. Il est certain que cette diminution a une estimation et un prix fixe, comme celui de tous les revenus du monde, n’y ayant rien qu’on ne puisse estimer.\par
VI. Après une exacte recherche, on trouve que cette diminution va à plus de 500 millions par an, dont il ne faut point d’autre marque, que tous les immeubles ne sont pas, l’un portant l’autre, à la moitié du prix qu’ils étaient autrefois.\par
VII. Il est encore certain qu’un si grand désordre, qui n’a jamais eu d’exemple depuis la création du monde, qu’un royaume opulent ait perdu la moitié de ses richesses en trente ou quarante années, et cela sans peste, tremblement de terre, guerre civile et étrangère, ou autres de ces grands accidents qui ruinent les monarchies ; il est certain, dis-je, que cela a une cause, et que ce n’est point l’effet du hasard.\par
VIII. Il est indubitable que qui pourrait trouver cette cause, et l’exposer en vente au peuple, il n’y a point de marché au monde où le roi et ses sujets gagnassent davantage.\par
IX. Quoi que ce soit qu’ils donnassent, pourvu qu’il fût au-dessous de la somme qu’ils gagneraient, il est certain que ce serait un édit qui serait profitable au peuple, puisqu’ils entreraient en possession d’une chose qu’ils n’avaient pas, et qui leur serait très avantageuse, le roi payé.\par
X. Il est encore hors de doute qu’un homme qui laisse son bien en friche souffre une plus grande violence que celui dont les héritages sont saisis, et comme il ne faut qu’un quart d’heure pour remettre ce dernier en possession, par la mainlevée qu’on lui signifierait, il n’en faut pas davantage pour remettre le premier en état de cultiver sa terre.\par
XI. Tout consiste donc à trouver la cause de cet abandonnement, pour pouvoir, en vingt-quatre heures, rendre le roi et ses peuples très riches.\par
XII. Il ne peut y avoir que deux causes qui empêchent un homme de cultiver sa terre, ou parce qu’il faut une certaine opulence, qu’il n’est point en état de se procurer, ni par lui, ni par emprunt, ou à cause qu’après l’avoir cultivée, il ne pourrait pas avoir le débit de sa production, comme il faisait autrefois, ce qui lui ferait perdre toutes ses avances, et le jetterait dans le malheureux intérêt de laisser son bien en friche.\par
XIII. C’est justement ce qui se passe par la Taille arbitraire pour le premier empêchement ; car il est très ordinaire qu’une grande recette ne paie rien (ou peu de chose) de taille, pendant qu’un misérable, qui n’a que ses bras pour la subsistance de lui et de sa famille, est accablé : la raison même pour laquelle il ne l’est pas davantage, est que si on l’imposait encore à une plus haute somme, on n’en pourrait recouvrer le paiement. Ainsi, s’il entreprenait de labourer la terre qui est en friche, la récolte ne serait pas pour lui, et il perdrait encore les frais, qui sont considérables.\par
XIV. Et pour le second obstacle, de ne point cultiver la terre à cause qu’après la récolte on ne pourrait avoir le débit des denrées, les droits d’Aides et de Douanes sur les sorties et passages du royaume, quatre fois plus forts que la marchandise ne peut porter, ont mis les choses sur un pied qu’il ne se consomme pas la quatrième partie qu’il se faisait il y a trente ou quarante ans ; et il n’est point surprenant de voir toute une contrée ne boire que de l’eau, pendant qu’on arrache les vignes et les arbres dans une contrée voisine ; et bien loin que les droits du roi en soient augmentés, cela a empêché qu’ils n’aient doublé depuis 1660, comme ils avaient fait tous les trente ans, depuis 1447 jusqu’en ladite année 1660.\par
XV. Le remède à tout cela est aisé, pourvu qu’on ne veuille avoir égard qu’aux intérêts du roi et des peuples, dans le genre des subsides : il faut voir s’il n’y en a aucun qui, faisant passer l’argent immédiatement de la main du peuple en celle du roi, ait d’ailleurs une règle et un niveau si certain de proportion avec chaque état, que le pauvre paie comme pauvre, et le riche comme riche, et cela sans ministère de juge ni d’autorité, auquel on ne peut avoir recours sans qu’il en coûte en frais et en perte de temps une fois davantage qu’il ne faut pour satisfaire à l’impôt.\par
XVI. Dans l’édit de la Capitation, on a eu l’intention de remédier à tous ces désordres, mais on peut dire que l’on n’a satisfait qu’à un point, qui est de faire passer l’argent immédiatement dans les mains du roi sans ministère de Traitants. Mais premièrement la cause de l’abandonnement des terres n’en est point levée ; en second lieu, bien loin qu’on y ait gardé partout cette règle de proportion qui fait payer chaque particulier suivant son pouvoir, il se trouve des classes où un homme qui a une charge de 100 000 écus, et du bien à proportion, paie la même chose qu’un autre dont l’emploi ne coûte que 500 liv. Ainsi, comme pour les mettre à une même somme il a fallu faire descendre le puissant, étant impossible de faire monter l’autre, il se trouve que le roi ne tire pas, à beaucoup près, d’un de ses sujets le secours proportionné à ses forces, pendant que l’autre en est peut-être accablé ; ce qui est cause que la suite de cette nouvelle découverte ne répond pas à ce qu’on s’en est promis.\par
XVII. Pour revenir donc au premier article de ces Mémoires, et satisfaire à tous les besoins de l’État, et remettre tous les peuples dans leur ancienne opulence, il n’est point nécessaire de faire de miracles, mais seulement de cesser de faire une continuelle violence à la nature, en imitant et nos voisins et nos ancêtres, qui n’ont jamais connu que deux manières d’impôts, savoir, les {\itshape feux}, c’est-à-dire les cheminées, et la {\itshape Dîme des terres}, qui a été la première redevance des rois de France, jusqu’à ce que, par les donations qu’ils ont eu la faiblesse d’en faire à l’Église, ils s’en soient laissé dépouiller.\par
XVIII. De cette manière, on satisfait à tout ce qui manque à la Capitation : il y a autant de classes que de degrés de richesse, sans que cela puisse former la moindre contestation ; le commerce et la consommation n’en reçoivent pas la moindre atteinte ; et partout où les peuples ont pu choisir le genre d’impôt le plus commode, ils s’en sont tenus à ceux-là.\par
XIX. Au lieu de la Dîme, afin de faire moins de mouvement, il ne faut qu’ordonner que la Taille sera assise suivant l’{\itshape occupation}, et qu’un homme qui n’a que son industrie ne pourra payer que depuis 3 livres jusqu’à 6 : de cette sorte, à 2 sous pour livre, elle remplira plus que la somme où elle est aujourd’hui, parce que les villes Taillables, où l’industrie paie la plus grande partie de la Taille, seront mises au {\itshape Tarif}, ce qu’elles demandent toutes avec empressement. Et à l’égard des Aides, des Douanes, et autres impôts des passages, qui ruinent la consommation, en remettant sur la Taille, jusqu’à la concurrence du tiers de la Taille, comme ils étaient autrefois, et le surplus sur les cheminées, il se trouvera que les peuples ne paieront pas la sixième partie de ce qu’ils paient aujourd’hui, et que le roi recevra le double de ses revenus d’à présent, parce que la Taille, jointe à une partie des Aides, ayant pour Tarif la valeur des héritages, ils reprendront leur prix d’autrefois, qui était le double de celui d’aujourd’hui, et par conséquent la Taille doublera pareillement, sans que le propriétaire s’en puisse plaindre, puisque l’augmentation des revenus du roi ne sera qu’une suite de celle de son opulence.\par
XX. Il ne faut point dire qu’il faut du temps pour cela, puisque entre la permission de vendre sa marchandise, quand il se trouve des personnes en état de l’acheter, et la vendre, il n’y a que vingt-quatre heures d’intervalle ; et entre l’avoir vendue, et être plus riche que l’on n’était, il n’y a aucun intervalle ; et entre être plus riche que l’on n’était, et faire plus de dépenses, ou à acheter des fonds, ou à les cultiver mieux, il n’y a pareillement encore aucun intervalle ; et entre faire ces mouvements et jeter de l’argent parmi le peuple, il n’y a point non plus d’intervalle. Et du moment que le peuple a de l’argent, il consomme les fruits qu’il fait venir par son travail, et est en état de payer le roi à proportion. Ainsi donc, tout dépend de la culture de la terre, qui ne peut marcher tant que l’on ôte le pouvoir aux laboureurs de faire les avances que cette culture réclame, et de débiter les denrées qui croissent sur son fonds.\par
XXI. Et pour dire un mot de la forte méprise qui est arrivée dans la création des nouvelles charges, on soutient qu’il n’y a point encore eu de manière qui ait si fort ruiné la culture de la terre ; parce qu’ayant presque toutes porté avec elles une exemption des impôts publics, comme c’étaient des personnes puissantes qui les acquéraient, elles se déchargeaient du poids de leurs impôts sur une infinité de malheureux, que cela mettait tout à fait hors d’état de labourer la terre. En outre, ces nouvelles créations anéantissant une infinité d’anciennes charges achetées à la bonne foi, et qui faisaient presque tout le bien des familles, cela a établi pour principe qu’il n’en fallait plus compter aucune à l’avenir pour un bien certain, parce qu’étant susceptibles à tous moments d’anéantissement, il y avait danger perpétuel de perdre leur argent pour tous ceux qui les achetaient, ou prêtaient des fonds dans ce but. En sorte que le roi a anéanti pour dix fois davantage de biens qu’il n’a reçu de secours de ces nouvelles créations, et fait que l’argent ne peut plus passer d’une main à l’autre, comme il faisait autrefois, parce qu’on ne peut point dire qu’il y ait aucune acquisition assurée, n’y ayant rien de si pernicieux que de prendre le capital du bien d’un particulier pour les besoins du prince. Et comme dans les taxes qu’on a imposées sur les officiers il y en avait plusieurs beaucoup au-dessus de leurs forces, les Traitants en étant venus à des exécutions, ils en ont été entièrement ruinés, bien que le roi n’en ait rien reçu.\par
XXII. Il ne faut pas espérer que les Traitants proposent jamais d’autres affaires, parce que leur intention étant d’avoir de fortes remises, ils ne les peuvent espérer que de recouvrements difficiles, et par conséquent ruineux, leur étant avantageux à mesure qu’ils sont dommageables au peuple ; parce que les frais des exécutions où il en faut venir sont partagés entre eux, les huissiers et les recors, qui leur font de fortes remises de ce qui leur est taxé.\par
XXIII. Toutes ces vérités, qui seront niées par les Traitants et par ceux qui les protègent, qui sont en bien plus grand nombre qu’on ne croit, seront attestées par toutes les personnes des provinces, qui sont de quelque considération, soit dans les charges ou dans le commerce : qu’importe, toutefois, si ceux qui ont intérêt de tout ruiner, étant seuls écoutés, on ne donne aucune audience aux personnes qui voudraient tout sauver, mais qui ne pourraient pas même la demander trop fortement, sans courir risque à leur particulier ?\par
XXIV. On a réduit ces Mémoires par articles, afin de rendre la mauvaise foi de ceux qui en voudraient nier la conséquence plus sensible, parce que n’en pouvant contester aucun en particulier sans découvrir leur manque de lumières ou de bonne foi, il faut qu’ils conviennent, malgré qu’ils en aient, que le roi peut s’enrichir, lui et ses peuples, en quinze jours, lorsqu’il ne voudra plus souffrir que quelques particuliers fassent leur fortune à le ruiner, lui et ses sujets ; et recouvrer par conséquent tout l’argent nécessaire pour cette présente guerre, sans mettre ses peuples au désespoir, comme on peut dire qu’est un homme qui se voit exécuté et vendu en ses biens pour des sommes dix fois plus fortes qu’il n’a vaillant, ce qui le met à l’aumône, lui et sa famille, sans donner un denier au roi, ainsi qu’il arrive tous les jours. — Tout cela sans nul plus grand mouvement, que de faire exécuter les mandements de la Taille, qui portent qu’elle sera assise suivant les facultés de chacun, et d’y joindre une partie des Aides, comme on fait les Étapes, et comme cela était il y a trente ans, ce qui demande quatre fois moins de mouvement que la Capitation.\par
XXV. De cette manière, on maintient que les peuples auraient deux cents millions de rente en quinze jours, plus qu’ils n’avaient, par cette mainlevée de leurs biens auparavant saisis. Et comme il faut au roi soixante millions par an d’extraordinaire, il y a mille façons de les avoir de ceux à qui on viendrait d’en rétablir quatre fois davantage, outre l’avenir qui doublerait encore avant deux ou trois ans, qui seraient nécessaires pour remettre les fonds.
\subsection[{Chapitre X.}]{Chapitre X.}
\noindent L’état où la France est réduite présentement, de ne pouvoir fournir au roi, que par des emprisonnements, et vente entière de biens, les sommes nécessaires, ne vient point de leur excès, mais de ce que tous les biens des peuples sont saisis depuis trente ans, et qu’ils n’en ont aucune disposition.\par
En effet, la Taille arbitraire contraint un marchand de cacher son argent, et un laboureur de laisser la terre en friche ; parce que si l’un voulait faire commerce, et l’autre labourer, ils seraient tous deux accablés de Taille par les personnes puissantes, qui sont en possession de ne rien payer, ou peu de chose.\par
Et les Aides, les Douanes, et les impôts sur les passages et sorties du royaume, quatre fois plus forts que la marchandise ne peut porter, font qu’un homme voit périr plein ses caves de boissons, pendant qu’elles sont très chères dans son voisinage, ce qui fait plus de 500 millions de rente de diminution dans le revenu du royaume.\par
Si le roi veut bien exposer en vente la cause qui produit cette perte, qui va toujours en augmentant, puisqu’on maintient qu’il ne reçoit point une pistole qu’il n’en coûte dix en pure perte à son Royaume, il aura cent mille marchands en vingt-quatre heures, qui ne l’auront pas sitôt payé, qu’ils seront plus riches qu’ils n’étaient ; parce que des causes contraires les effets sont contraires ; c’est-à-dire, que le roi veuille bien revendre à ses peuples la jouissance de leurs biens, sans qu’il soit besoin de congédier ni fermiers ni Traitants.
\section[{Supplément au Détail de la France.}]{Supplément au Détail de la France.}
\noindent Il est surprenant que dans les grands besoins qu’a présentement l’État de secours extraordinaires, les peuples faisant offre de les fournir dans le moment, au moyen de quelques accommodements, lesquels, sans rien déranger, n’exigent qu’un simple acte de volonté des personnes en place, et mettront ces mêmes peuples au même instant en état d’y satisfaire avec profit de leur part ; il est étonnant, dis-je, qu’on ne veuille accepter ces offres qu’après la conclusion de la paix, bien que ce soit l’unique moyen d’en procurer une très avantageuse. En sorte que, par une destinée jusqu’ici inouïe, ceux à qui il tombe en charge de payer, se soumettent de le faire sans demander de délai, et les personnes qui ne doivent avoir d’autres fonctions que de recevoir, exigent un terme et un délai, fort incertains, pour l’accepter. Outre cette situation monstrueuse, on peut assurer que la guerre étrangère coûte dix et vingt fois moins au royaume que les désordres intestins causés par les manières que l’on pratique pour recouvrer les fonds afin d’y subvenir ; si bien que, mettant pour ainsi dire l’incendie dans toutes les contrées de la France, il est plus opportun de l’arrêter que la guerre du dehors, dont, encore une fois, la conclusion avantageuse dépendra absolument de cette paix du dedans, qui se peut terminer à moins d’un mois ; et l’allégation de la guerre étrangère comme un obstacle au rétablissement de la félicité générale est la même erreur que si, le feu étant aux quatre coins d’une maison, on soutenait qu’il ne faut pas l’éteindre qu’un procès que l’on aurait pour la propriété en un tribunal éloigné ne fût jugé ; et c’est ce qui se verra mieux par un petit détail de cette guerre intestine, ou de cet embrasement du royaume, article par article.\par
Faut-il attendre la paix pour faire labourer les terres dans toutes les provinces, où la plupart demeurent en friche par le bas prix du blé, qui n’en peut supporter les frais, et où l’on néglige pareillement l’engrais de toutes les autres, ce qui fait un tort de plus de 500 000 muids de blé par an à la France, et 500 millions de perte dans le revenu des peuples, par la cessation de la circulation de ce premier produit, qui mène à sa suite toutes les professions d’industrie, lesquelles vivent et meurent avec lui ?\par
Faut-il attendre la paix pour un autre article, qui est une suite du précédent, savoir : pour faire payer les propriétaires des fonds par ceux qui les font valoir, desquels nul maître ne recevant rien, ou il ne fait nul achat dans les boutiques, ou ne satisfaisant pas aux crédits précédents, les marchands sont obligés de faire banqueroute ?\par
Faut-il attendre la paix pour faire cesser d’arracher les vignes, comme on fait tous les jours, pendant que les trois quarts des peuples ne boivent que de l’eau, à cause des impôts effroyables sur les liqueurs, qui excèdent de quatre ou cinq fois le prix de la marchandise ; et quand le produit qui donne lieu à une pareille destruction est offert d’être payé au double à l’égard du roi d’une autre manière par les peuples, ce qui serait un quadruple profit de leur part, ne peuvent-ils être écoutés, et doit-on les renvoyer à un autre temps, en soutenant qu’il faut attendre que toutes les vignes soient arrachées pour donner permission aux peuples de les cultiver ; ce qui serait entièrement inutile, et ne vaudrait guère mieux que d’appeler un médecin pour guérir un mort ?\par
Faut-il attendre la paix pour ordonner que les Tailles seront justement réparties dans tout le royaume, et que l’on ne mettra pas de grandes recettes à rien ou peu de chose, pendant qu’un misérable qui n’a que ses bras pour vivre lui et toute une famille, voit, après la vente de ses chétifs meubles ou instruments dont il gagne sa vie, comme on fait pour l’ustensile qui se règle sur le niveau de la Taille, enlever les portes et les sommiers de sa maison pour satisfaire au surplus d’un impôt excédant quatre fois ses forces ? M. de Sully, qui rétablit la France, l’ayant trouvée au point où elle peut être aujourd’hui, n’était pas persuadé que la guerre eût rien de commun avec ces règlements, puisqu’il lit une ordonnance en 1597 pour régler la juste répartition de la Taille, ainsi que tous les autres désordres, qu’il arrêta au milieu de deux guerres, l’une civile et l’autre étrangère, qui désolaient le dedans et le dehors du royaume d’une bien plus cruelle manière que ne peut être la conjoncture d’aujourd’hui ; et le tout fut si ponctuellement exécuté, que le roi et les peuples devinrent très riches, de très mal dans leurs affaires qu’ils étaient auparavant.\par
Faut-il attendre la paix pour sauver la vie à deux ou trois cent mille créatures qui périssent au moins toutes les années de misère, surtout dans l’enfance, n’y en ayant pas la moitié qui puisse parvenir à l’âge de gagner leur vie, parce que les mères manquent de lait, faute de nourriture ou par excès de travail ; tandis que dans un âge plus avancé, n’ayant que du pain et de l’eau, sans lits, vêtements, ni aucuns remèdes dans leurs maladies, et dépourvues de forces suffisantes pour le travail, qui est leur unique revenu, elles périssent avant même d’avoir atteint le milieu de leur carrière ?\par
Faut-il attendre la paix pour la donner aux immeubles, ce qui se peut en un instant, le roi déclarant qu’il se contentera désormais de subsides réglés proportionnés aux forces de chacun des contribuables, ainsi qu’il se fait présentement en Angleterre, en Hollande, et dans tous les pays du monde, et qu’il s’est fait même en France durant onze cents ans ; et que l’on ne bombardera plus rien, surtout les charges, comme il est arrivé à une infinité de personnes ; ce qui faisant tout le vaillant d’un homme, le réduit à l’aumône, et mettant tous les autres possesseurs de semblables biens dans l’attente d’un pareil sort, les ruine presque également sans que le roi reçoive rien ? N’est-ce pas, en effet, leur ôter tout crédit, puisque le crédit ne roulant que sur la solvabilité du sujet qui s’en sert, cette solvabilité s’anéantit par la destruction du prix des fonds qu’il possède ; tout comme dans une ville menacée de bombardement, quoique les maisons ne ressentent actuellement aucun mal, elles perdent neuf parts sur dix de leur valeur ordinaire, qu’elles reprennent aussitôt que cette crainte est passée. Ainsi on peut en un instant, par l’établissement d’une paix intestine, doubler et tripler le prix de tous les immeubles, et par conséquent le crédit, qui est la moitié, encore une fois, du revenu des peuples.\par
Faut-il attendre la paix pour mettre le roi en état de payer les officiers à point nommé, afin que ceux-ci soient en pouvoir de faire leurs recrues dans les temps commodes, et de bonne heure ?\par
Faut-il attendre la paix pour donner assez de secours au roi afin que par un engagement considérable on fasse des soldats volontairement, et que l’on ne mène pas des forçats liés et garrottés à l’armée, comme on fait aux galères et même au gibet ; ce qui, au rapport de M. de Sully, dans ses Mémoires, ne sert qu’à décourager les autres, décrier le métier et la nation, parce qu’ils désertent tous à la première occasion, ou meurent de chagrin ?\par
Faut-il attendre la paix pour cesser de constituer l’État sous le nom du roi, en sorte qu’après la fin de la guerre le paiement des intérêts de l’argent pris en rente coûtera plus aux peuples que l’entretien de la guerre, de façon que c’en sera une perpétuelle qu’ils auront à soutenir ?\par
Faut-il attendre la paix pour purger l’État des billets de monnaie qui, par le déconcertement qu’ils apportent dans le commerce, coûtent quatre fois plus par an que la valeur de toutes les sommes pour lesquelles on en a créé, c’est-à-dire quatre fois plus que la guerre étrangère ? Que le royaume s’en recharge par un juste partage sur la tête des particuliers et Communautés. L’{\itshape endos} qu’ils y mettront, payable en quatre ans par quatre paiements différents, avec intérêts, les fera circuler dans le trafic sans aucune perte du transportant ; et le rétablissement de la consommation, possible en trois heures par la simple cessation d’une très grosse violence faite à la nature, dédommagera au quadruple tous ces endosseurs, de cette prétendue nouvelle charge, ainsi que la crue ou la hausse de la fourniture des besoins du roi.\par
Faut-il enfin attendre la paix pour cesser de vendre tous les jours des immeubles, surtout des Charges, avec promesse qu’on en jouira tranquillement, et que ceux qui auront prêté leur argent pour cet achat auront un privilège spécial, et puis, quelque temps après, revendre ce nouvel effet à un autre, sans nul dédommagement au premier acquéreur non plus qu’au prêteur ; ce qui ôtant la confiance, qui est l’âme du trafic, rompt tout commerce entre le prince et ses sujets, fait que l’argent seul, pouvant être à l’abri de pareils orages, est estimé l’unique bien, et comme tel resserré dans les cachettes les plus obscures qu’on peut trouver, avec une cessation entière de toutes sortes de consommations, dont cet argent est uniquement le très humble valet ? C’est une très grande absurdité de chercher d’autre cause de la rareté que l’on en voit régner, que cette même destruction de consommation, comme de nier qu’en la rétablissant, comme cela se peut en un moment, on le verra aussi commun que jamais ; bien que depuis un très long temps on ne l’ait cherchée que dans la destruction de la seule cause qui le fait marcher, savoir, encore une fois, la ruine de la consommation.\par
L’esprit le plus borné et le plus rempli de ténèbres qui fut jamais ne peut être assez aveuglé pour produire de pareils soutiens : il n’y a que le cœur ; car, au témoignage de l’Écriture sainte, lorsqu’il est une fois corrompu, un saint revenant exprès de l’autre monde, ne le changerait pas. Aussi, quoiqu’on va montrer qu’il est aussi certain que les peuples peuvent par trois heures de travail de MM. les ministres, et un mois d’exécution de leur part, sans rien déconcerter, ni mettre aucun établissement précédent au hasard, qu’ils peuvent, dis-je, fournir cent millions de hausse au roi pour ses besoins présents, avec quadruple profit de leur part, et que l’on fasse cette preuve avec autant de certitude que si un ange la venait apporter du ciel ; on ne prétend pas néanmoins convertir un seul de ces cœurs corrompus, c’est-à-dire ceux en qui la destruction publique est le principe de la haute fortune : on ne s’adresse qu’aux esprits qui pourraient se laisser gâter par la contagion de sujets dépravés, et par conséquent suspects sur une pareille matière.\par
Voici comment on fait cette preuve : ce qui est constamment vrai, ne serait pas plus certain quand tous les saints du paradis le viendraient attester, et il est à coup sûr aussi indubitable que la Seine passe dans Paris, que si les anges en venaient rendre témoignage.\par
Il y a une seconde chose incontestable, savoir, que tous les faits sur lesquels plusieurs s’accordent sans aucune convenance précédente entre eux, sont aussi certains que si nos propres yeux nous en portaient témoignage.\par
Tous les hommes raisonnables qui n’ont jamais été à Rome parieraient tout leur bien, contre une pièce de trente sous, qu’il existe au monde une ville de ce nom, parce que trop de gens l’ont dit et écrit sans avoir concerté de mentir, pour que cela ne soit pas véritable ; et même si quelqu’un voulait contredire ce fait, on le traiterait de fou et d’extravagant.\par
Or, on maintient que l’établissement de cent millions de hausse de la part des peuples, avec quadruple profit de leur part, possible en trois heures de travail et un mois d’exécution, a le même degré de certitude que cet exemple de Rome, attendu que tous les peuples non suspects sont prêts à en signer la proposition aux conditions marquées ; et l’on soutient en même temps que si le roi ordonnait à quelqu’un de mettre par écrit des raisons qui fissent voir l’impossibilité d’un pareil recouvrement, outre qu’il ne saurait par où commencer ni par où finir, il serait en horreur et à Dieu et aux hommes. Et la demande du délai jusqu’après la paix est un aveu pur et simple que la chose est très aisée, ou la contradiction impossible, puisque la paix ou la guerre étrangère n’ont nulle relation avec ce qui se passe au-dedans du royaume à l’égard des tributs : c’est donc montrer grossièrement que, ne pouvant nier que les manières pratiquées mettent le feu aux quatre coins de la France, on souhaite seulement que l’on remette à l’éteindre jusqu’à la paix ; non, encore une fois, qu’elle ait aucun rapport à ces désordres, mais parce que l’on espère par là obtenir un délai, et que l’embrasement soit continué, attendu qu’on y trouve son compte, et que l’on est au nombre des incendiaires qui se font bien payer pour de pareils services.\par
De si cruelles dispositions et de semblables énoncés ne doivent pas surprendre de la part des Traitants, puisque c’est à l’aide d’une pareille politique qu’ils se procurent ces fortunes immenses qui font la ruine de l’État, et qu’ils se sont fait donner, depuis 1689, 200 millions pour leur part, sans celle du néant, qui croissant sous leurs pieds, excède de dix à vingt fois ce que tant le roi qu’eux reçoivent par un si funeste canal ; et même de pareilles objections n’auraient pas également surpris dans la bouche des ministres avant 1661, parce que ou ils étaient Traitants eux-mêmes, ou ils prenaient part dans tous les partis, comme il fut vérifié contradictoirement à la chambre de justice ; — ce qui était la même chose à l’arrivée de M. de Sully au ministère, lequel dit au roi Henri IV que les Traitants, qui sont la ruine d’un État, n’avaient été inventés par les ministres que pour prévariquer, leur étant impossible de rien prendre dans les tributs réglés passant droit des mains des peuples en celle du prince, comme il se pratique dans tous les pays du monde ; au lieu que par les Partisans ils sont les maîtres absolus des biens de tout le monde, mettant un homme riche sur le carreau, et le dernier des misérables dans l’opulence quand il leur plaît, et ne sont privés pour leur particulier de recevoir quelques sommes que ce puisse être, qu’autant qu’ils les veulent refuser, n’y ayant d’autres bornes que celles que l’on peut attendre de leur modération ; — comme, dis-je, c’était là la situation des ministres avant 1661, la demande de délai pour changer des manières si déplorables n’eût pas surpris, parce qu’on l’eût regardée comme des {\itshape lettres-d’État} de leur part pour se maintenir dans une si agréable situation à leur égard, quoique si funeste au roi et aux peuples ; — mais aujourd’hui et depuis 1661, que l’intégrité tout entière a succédé tout à coup dans le ministère, et sans aucun milieu, à une extrême prévarication, on ne peut qu’être surpris d’avoir vu trois fois un quadruplement de Partisans et de manières désolantes, ainsi que la demande actuelle d’un délai pour éteindre le feu qui est aux quatre coins du royaume, avec un refus de recevoir de la part des peuples tous les besoins du roi, dans un temps qu’ils sont absolument nécessaires à la monarchie, parce qu’on ose appeler un renversement d’État la cessation du plus grand bouleversement qui fut jamais, qui fait une très grande violence à la nature, et qui peut être arrêté en un moment avec beaucoup moins de dérangement qu’il n’y en eut lors de la Capitation établie en 1695, au milieu de la guerre.\par
Et si, quant à cette Capitation, qui avait promis la cessation des Affaires extraordinaires, elle n’a eu d’autre résultat, grâce à ceux qui trompèrent MM. les ministres dans la répartition, que de rendre l’impôt ridicule, et par suite insuffisant à atteindre aux besoins du roi, il n’est pas à craindre qu’il en arrive de même dans celle qu’on propose, puisqu’elle ira à plus de cent millions avec quadruple profit de ceux qui paieront six fois leur cote précédente, et cela par la simple attention à ces quatre articles, savoir : les blés et liqueurs, la juste répartition des Tailles, et la cessation des affaires extraordinaires ; ce qui n’exige qu’un simple acte de volonté du roi et de MM. les ministres, pour finir une très grande violence qu’on fait à la nature, bien que la négligence de cette attention coûte, de compte fait, plus de quinze cents millions de perte par an au royaume depuis 1661, que l’intégrité est dans le ministère, les prévarications précédentes n’ayant rien produit de si funeste ; mais bien le contraire, et tous les biens se trouvant doublés en 1661, ainsi que ceux du roi, du prix qu’ils étaient trente ans auparavant.\par
Que si ce nombre de 1 500 millions étonne, on le prend d’une autre manière, et on maintient que sur quarante mille villes, bourgs et villages qu’il peut y avoir dans le royaume, il n’y en a aucun, l’un portant l’autre, qui n’ait perdu cinquante mille livres de revenu tant en fonds qu’en industrie, ou plutôt dix et vingt fois davantage que ce que le roi en tire par toutes sortes d’impôts, à le vérifier sur tel lieu que le parti contraire voudra choisir, sans qu’on en puisse accuser le manque d’espèces, qui sont aujourd’hui au double dans la France, comptant exactement ce qui est entré et sorti, de ce qu’il y en avait en 1661, que les quinze cents millions de rente existaient. Mais c’est que l’argent est devenu paralytique, et qu’il avait au contraire des jambes de cerf en ce temps-là, ce qui est le seul principe de la richesse des peuples, et par conséquent de la fourniture des besoins du roi. Car les tributs, comme toutes sortes de redevances, tirent leur qualité d’excès ou de modicité, non de la quotité absolue des sommes que l’on demande, mais de la valeur des fonds dont on les exige, et la vigueur de ceux-ci n’est qu’à proportion de la vente des denrées qu’ils produisent ; d’où il suit que cette production pouvant être doublée en un moment, il n’en faudrait pas davantage pour rendre au cours des espèces la même rapidité qu’imprime à l’eau d’un torrent la levée de la digue qui la retenait sur le bord d’une pente ; et la même absurdité qui se rencontrerait dans l’objection que cette eau ne pourrait couler dans la vallée, après l’enlèvement de la digue, qu’une guerre étrangère ne fût terminée, se trouve encore dans l’allégation des personnes qui prétendent qu’il faut attendre la fin de cette même guerre pour voir marcher la consommation, bien que les causes violentes qui l’arrêtent puissent être ôtées en un moment, en quelque temps que ce soit.\par
Quand on dit cent millions d’augmentation dans les revenus du roi en un instant, ce n’est pas 100 millions d’espèces de nouvelle fabrique, comme au Pérou, c’est cent millions de pain, de vin, de viande, ou autres denrées, qui étant le seul soutien de la vie, le sont pareillement des armées, lesquelles seront fournies au moyen de dix millions seulement, et même moins, qui faisant dix voyages et dix retours des mains des peuples en celles du prince, enfanteront cette livraison de denrées dont il se perd tous les jours dix fois davantage, tant produites qu’à produire ; pendant que d’un autre côté ces dix millions, qui ne marchent jamais que par l’ordre de la consommation, résident des années entières dans des retraites dont toutes les machines du monde ne les peuvent tirer : loin de là, toutes les mesures que l’on prend ne servent qu’à les y enfoncer davantage, au lieu qu’en un instant on les peut mettre, ainsi que tout le reste, en mouvement ; ce qu’on offre à la garantie des peuples, qui vaut beaucoup mieux que celle des Traitants, n’y ayant qui que ce soit, non intéressé à la cause des désordres, qui ne donne avec plaisir et profit les deux sous pour livre de son revenu pour être payé du surplus avec exactitude, ce qui n’est pas à beaucoup près présentement, et ce qui est immanquable par le système proposé, beaucoup plus propre au soutien de la guerre que toutes les pratiques employées jusqu’à ce jour.\par


\begin{raggedleft}FIN DU DÉTAIL DE LA FRANCE.\end{raggedleft}
\chapterclose


\chapteropen
\chapter[{Factum de la France}]{Factum de la France}\renewcommand{\leftmark}{Factum de la France}

\begin{center}\emph{Ou Moyens très faciles de faire recevoir au roi quatre-vingts millions par-dessus la capitation, praticables par deux heures de travail de MM. Les ministres et un mois d’exécution de la part des peuples, sans congédier aucun fermier général ni particulier, ni autre mouvement que de rétablir quatre ou cinq fois davantage de revenu à la France, c’est-à-dire, plus de cinq cents millions sur plus de mille cinq cents anéantis depuis 1661, parce qu’on fait voir clairement, en même temps, que l’on ne peut faire d’objection contre cette proposition, soit par rapport au temps et à la conjoncture, comme n’étant pas propres à aucun changement, soit au prétendu péril, risque, ou quelques autres causes que ce puisse être, sans renoncer à la raison et au sens commun ; en sorte que l’on maintient qu’il n’y a point d’homme sur la terre qui ose mettre sur le papier une pareille contradiction, et là souscrire de son nom, sans se perdre d’honneur ; et que l’on montre en même temps l’impossibilité de sortir autrement de la conjoncture présente.}\end{center}

\chaptercont
\section[{Chapitre I.}]{Chapitre I.}
\noindent Il parut il y a dix ans, autant par hasard que de dessein prémédité, au moins à l’égard du public, un Mémoire ou Traité intitulé, {\itshape le Détail de la France}. Bien qu’il fît voir la facilité que le roi avait, sans rien déconcerter, de lever toutes les sommes nécessaires dans la conjoncture du temps, en procurant même l’utilité de ses peuples, il n’eut aucune réussite, et on n’y fit pas même la moindre attention.\par
L’auteur n’en espérait pas davantage, et il l’avait marqué en termes exprès. La raison de cela était qu’il y avait encore, pour ainsi dire, de l’huile dans la lampe : le motif ou les causes de la ruine de la France, par les surprises que l’on faisait à MM. les ministres, avaient encore par devers eux de quoi payer amplement les entrepreneurs, comme eux pareillement assez de profit pour acheter de la protection. Mais aujourd’hui que tout a pris fin faute de matière, on doit présumer un succès moins traversé, parce qu’il y aura moins d’intérêt à contredire les propositions passées, ou plutôt une nécessité absolue de les admettre. C’est pourquoi on offre de la part des peuples, sans crainte d’être désavoué, tous les besoins du royaume, à quelque somme qu’ils puissent monter, tant sur terre que sur mer, pour mettre ses ennemis dans la nécessité de n’attendre la paix que de la justice et de la modération de Louis le Grand, comme par le passé.\par
On maintient encore une fois que s’il ne tient qu’à 80 millions par an par-dessus les tributs ordinaires, et même davantage, sans compter la Capitation en l’état qu’elle est, la chose sera bientôt faite, et cela sans nul déconcertement, ni rupture d’aucun traité que le roi ait fait avec qui que ce soit, et faisant même beaucoup moins de mouvement qu’il n’y en eût, bien qu’il ne s’en trouvât aucun lors du premier établissement de la Capitation.\par
On parle avec d’autant plus de hardiesse et de certitude, dans toutes les circonstances qui accompagnent cette proposition, que ces 80 millions ne seront que l’effet de plus de 500 que Sa Majesté aura rétablis à ses peuples par deux heures d’attention de MM. ses ministres, et quinze jours d’exécution chez les peuples, ainsi que l’on a dit, aux conditions marquées.\par
Que l’on suspende un peu l’idée de ridicule et d’extravagance que peut jeter une pareille proposition dans l’esprit d’une infinité de monde. Que l’on songe que le grand saint Augustin et Lactance, célèbres auteurs, n’ont pas acquis bien de l’honneur à traiter de fou et d’insensé un évêque nommé Virgile, qui, de leur siècle, vint annoncer les antipodes. Christophe Colomb reçut le même traitement en presque toutes les cours de l’Europe, avant que d’être écouté et aidé par quelque particulier en Espagne. Copernic, du dernier siècle, fut menacé du feu par toute la Théologie, sur l’exposition de son système, quoique aujourd’hui le plus universellement reçu.\par
L’auteur des 80 millions est dans une bien plus heureuse situation que n’étaient tous ces grands hommes : non seulement il n’est pas seul de son avis comme eux, mais il maintient qu’il n’est que l’avocat de tout ce qu’il y a de laboureurs et de commerçants dans le royaume, c’est-à-dire de tous ceux qui sont la source et principe de toutes les richesses de l’État, tant à l’égard du roi que des peuples. En sorte que, pour tempérer d’abord la grande vocation qu’on aurait à traiter ces discours de vision, et en rejeter même une grande dose, dès l’abord, sur les contredisants, le procès va rouler entre les laboureurs et marchands, de qui seuls partent toutes sortes de paiements, tant envers le prince que les propriétaires, et ceux qui n’ont d’autre fonction que de recevoir.\par
Ces premiers disent et publient hautement qu’ils sont prêts de payer les sommes marquées au titre de ce Mémoire, aux conditions mentionnées, qui ne tiennent à rien, puisqu’il ne s’agit que d’un simple acte de volonté de la part de personnes que l’on sait bien être en pouvoir de faire ce qui leur plaît ; et les parties adverses sont ceux à qui on ne demande autre chose que de recevoir, mais qui disent, et croient même marquer par là leur sagesse et leurs lumières, que ces paiements sont impossibles.\par
Or, on peut voir sur qui de ces deux personnages le ridicule doit tomber, par l’exemple des lettres de change. Un sujet qui serait porteur d’un papier de cette nature pour la valeur de mille livres sur un riche marchand, pourrait-il sans extravagance lui en faire signifier la protestation, après que l’autre lui aurait dit qu’il est prêt d’en faire le paiement, et l’aurait même sommé de le recevoir ?\par
Voilà les lois et le point de droit sur quoi va rouler la question. L’auteur de ces Mémoires ne veut passer que pour un extravagant achevé, s’il se méprend ; et s’il n’est pas avoué par tous les peuples dans ses propositions, il consent d’encourir cette peine, et même d’être mis aux lieux où l’on renferme les insensés, au cas qu’il ne rencontre pas juste. Et pour l’en convaincre il n’exige pas de forts raisonnements, et qui aient à peu près autant d’apparence que les siens ; mais il déclare d’abord qu’au cas que tout ce qu’on lui pourra objecter contre ses offres, ou plutôt celles des peuples, soit par l’impossibilité absolue, soit pour le temps, comme n’étant pas propre à aucun changement, soit pour le péril, soit pour le déconcertement ; au cas, dis-je, que ces objections ne soient pas une extravagance achevée étant mises par écrit, à faire horreur au ciel et à la terre, et qu’elles puissent trouver quelqu’un pour les signer, d’être lui-même traité de la manière qu’il vient de consentir, ce qu’il réitérera presque à chaque page de cet ouvrage, de peur que l’on ne l’oublie.\par
Comme le mot d’{\itshape extravagance} va souvent être employé dans ce Mémoire, bien que ce ne soit pas une expression que la politesse et la civilité souffrent d’ordinaire ni dans les discours ni dans les écrits entre les honnêtes gens, on est obligé, avant que d’entrer en matière, de faire une petite digression, pour marquer la nécessité de son usage dans cette occasion, et purger aussi l’idée d’injure que l’on y voudrait supposer, à l’égard de ceux envers lesquels on pourra s’en servir.\par
Pour le premier, comme la France a actuellement la gangrène, ou si on veut la pierre dans les reins, il faut, pour la guérison, user d’incisions dans le vif, et d’opérations très violentes dans les parties les plus nobles, les remèdes ordinaires n’étant plus de saison, et se trouvant beaucoup au-dessous de la force du mal.\par
Or, toute autre expression pouvant laisser l’idée, sinon d’une vision, au moins d’un problème, dans ce que l’auteur de ces Mémoires propose, à l’égard de tout ce qui n’est pas laboureur ou marchand, c’est-à-dire le beau monde, il serait difficile que qui que ce soit de ce genre s’embarquât à pénétrer dans ses raisons, et à en porter un jugement certain, pour faire le procès à de si illustres préjugés et à de si prétendus grands hommes, dans la pensée qu’après beaucoup de peine et de travail on ne trouverait que de l’obscurité, qui est plus qu’il n’en faut pour faire traiter l’auteur de visionnaire. — C’est dans ces occasions que l’on se fait un plaisir de croire que les faits les plus évidents sont des faussetés, où l’on se ferme les yeux dessus ; et après les avoir en quelque manière brûlés, on contredit les conséquences les plus certaines qui s’en tirent, pour se persuader à soi-même, et vouloir le faire croire aux autres, qu’il n’est pas à présumer que des gens si éclairés et si zélés pour le service du roi et du public aient commis de si lourdes fautes ; qu’ils avaient des raisons à eux seuls connues ; que si on les savait, on ne les calomnierait pas de la sorte ; qu’il est de la justice de ne pas condamner des gens sans les entendre, surtout quand ils sont morts, ce qui les met hors d’état de défendre leurs intérêts et d’apprendre les motifs particuliers de leur conduite. — La situation présente, ou plutôt le désordre de la France, a pourvu à se procurer de pareils défenseurs ; c’est pourquoi ce langage, quelque dépravé qu’il soit, ne manquera pas de sujets qui s’en serviront dans l’occasion présente ; ils ne se convertiraient même pas quand un mort viendrait de l’autre monde attester la vérité de ces Mémoires ; et cela au sentiment de l’Écriture Sainte, parce que le cœur est pris ; ce qui étant, ni l’esprit, ni l’honneur, ni la conscience, n’ont plus de voix au chapitre.\par
Mais lorsque l’on parle d’extravagance, et que l’on maintient, comme l’on fera dans ces Mémoires, que telle et telle affaire n’a pu être faite sans de deux choses l’une, ou que les auteurs eussent tout à fait perdu l’esprit, ce qui n’est pas assurément, ni même présumable, ou qu’ils eussent si fort erré au fait, qu’ils ont produit autant d’extravagances que s’ils avaient eu la cervelle entièrement démontée, il faut absolument prendre un parti, et il n’y a pas moyen d’user de subterfuge, ni de prétexter de son ignorance sur de pareilles matières. — Tout le monde, pour qu’il ait le sens commun, est juge compétent, et ne peut s’abstenir de prononcer sans mauvaise foi, sous prétexte de son manque de lumière.\par
C’est par de pareils raisonnements, ou de semblables principes, qu’on soutient qu’on peut rétablir la France en deux heures, et l’on passe carrière d’abord, en répétant ce qu’on a déjà dit, savoir que l’auteur de cette proposition veut bien passer pour un extravagant lui-même, et le plus grand qui fut jamais, si on peut lui faire aucune objection, soit pour la brièveté du temps, le péril ou quelques autres raisons que ce puisse être, qui ait la moindre apparence, et qui ne soit pas une extravagance achevée, pourvu qu’elle soit mise par écrit ; car c’est ce qui arrive toujours dans tous les faits que l’on affirme et que l’on contredit ; l’erreur est cause qu’il y a un des deux assurément qui commet la même extravagance que s’il avait perdu l’esprit. — Et qui que ce soit ne se doit formaliser d’être tombé dans cette faiblesse : tous les plus grands hommes et les plus célèbres auteurs y ont été surpris : il n’y a point d’absurdités qu’ils n’aient dites et écrites sur la foi de mauvais Mémoires, dans des ouvrages d’ailleurs très beaux, et qui les ont rendus très célèbres. — Saint Augustin et Lactance, comme l’on a marqué, ont traité d’extravagant le premier auteur des antipodes ; la suite leur a fait voir que l’extravagance était de leur côté. — Ainsi, il doit être permis à l’auteur de ce discours d’user, pour défendre la vérité, et les intérêts du roi et des peuples, des mêmes termes que de si grands hommes n’ont pas craint d’employer pour la combattre.\par
Ce préambule posé, que l’on a cru nécessaire pour qu’on ne fît pas un procès à l’auteur sur la forme d’un ouvrage dont le fonds est inattaquable, on va entrer en matière, déclarant que l’on a un très grand respect pour les personnes que l’on va montrer avoir toujours erré en fait ; — ce qui ne préjudicie point à leur intégrité, de laquelle on est très convaincu, — et qu’on se serait même servi d’expressions plus douces, si on avait cru le pouvoir faire sans trahir la cause du roi et des peuples, qu’on a entrepris de défendre. La justice même oblige de dire que, bien loin que MM. les ministres soient répréhensibles de s’être si fort mépris en fait, ils ne pouvaient sans miracle faire autrement, succédant à des sujets qui leur avaient montré de très mauvais exemples, et tracé des routes très défectueuses ; et bien loin d’être en état de s’en détourner, on peut dire que tout le monde conspirait à les y maintenir, y ayant plus de fortune à faire à tromper un ministre en France, en ruinant le roi et les peuples, qu’à conquérir un royaume entier pour le monarque, en quelque pays que ce soit.
\section[{Chapitre II.}]{Chapitre II.}
\noindent On promet {\itshape quatre-vingts} millions et plus par-dessus les impôts ordinaires, même la Capitation, par deux heures de travail et quinze jours d’exécution ; on promet, de plus, de payer toutes les dettes du roi et de l’État en dix ans de paix, et on promet enfin un doublement des revenus du roi, en supprimant la Capitation, avant quatre ou cinq ans ; le tout sans rien risquer, ni déconcerter, ni user de pouvoir absolu. — Voilà la plus grande extravagance qui puisse jamais tomber dans l’esprit, ni être proposée, si l’auteur ne rencontre pas juste dans la moindre de ses parties : mais que l’on suspende son jugement jusqu’à l’entière lecture de cet ouvrage, et que l’idée de ridicule, encore une fois, qui se présente avec violence à l’esprit, tempère un peu son ardeur, et l’on verra invinciblement que c’est le même procès qu’eurent les grands hommes qu’on a cités, au sujet des antipodes.\par
Personne ne doute que le principe et la base des revenus de tous les princes du monde ne soient ceux de leurs sujets, qui ne sont à proprement parler que leurs fermiers, les souverains n’étant en pouvoir de rien recevoir plus ou moins, qu’à proportion que ceux qui font valoir les terres sont en état, par le produit qu’ils en tirent, de leur payer des tributs. Cette maxime, qui se pratique également par tous les États, avait été en usage en France jusqu’à la mort du roi François I\textsuperscript{er}, n’y ayant été dérogé que médiocrement depuis ce temps, jusqu’en 1660. Mais on peut dire que depuis cette année on a pris le contrepied, et l’on a cru ne pouvoir faire plus utilement et plus diligemment recevoir de l’argent au monarque, surtout dans les besoins extraordinaires, que, non pas en augmentant le revenu et les biens des peuples, mais en les diminuant partout, et les détruisant en plusieurs endroits presque entièrement, à un taux certain l’un portant l’autre, savoir : vingt de perte par pur anéantissement à l’égard du propriétaire pour un de profit au roi, partagé même avec l’entrepreneur et ses protecteurs, lesquels faisaient une fortune de prince pour un si déplorable service. — Comme voilà le Héros de la pièce, et que c’est sur ce fondement que tout va rouler, on maintient ce fait incontestable, et aussi public qu’il est constant que la Seine passe dans Paris : en sorte que quiconque le voudrait nier, se rendrait aussi ridicule que celui qui ne voudrait pas convenir d’une vérité semblable. — La perte de la moitié des biens de la France, tant en fonds qu’en industrie, qui suivent nécessairement le sort de ces premiers, a autant de témoins qu’il y a d’hommes dans le royaume, sans parler des registres, baux et contrats qui font cette preuve par écrit, comme les peuples par témoins. — On maintient encore que cette diminution depuis 1660 va à plus de {\itshape quinze cents} millions par an : que ce mot de centaines de millions n’étonne point et ne cause point de surprise ! Comme on compte le revenu d’une maison, d’une ferme et d’un village, tant dans les diminutions que dans les hausses, il est aisé, à qui est rompu dans ces matières, de supputer celui de tout un royaume. On a fait celui de l’Angleterre, qui ne vaut pas le quart de la France, à le prendre de toutes les manières, quand ces deux États seront gouvernés par les mêmes maximes, et on prétend qu’il va à près de 700 millions par an. — Et pour la France, ceux qui se formaliseront de ces expressions ou de ces calculs, trouveront bon, s’il leur plaît, que l’on compte par plusieurs centaines de millions les revenus d’un État qui fournit souvent à son prince, dans des années, plus de cent cinquante millions, et à l’Église ordinairement plus de trois cents millions, tant de revenu en fonds que de casuel, qui surpasse de beaucoup le premier, dans la religion comme ailleurs. — Dans la seule Élection de Mantes le revenu des vignes, tant par un abandon entier de la plus grande partie, quoique autrefois d’un très grand produit aux propriétaires, que par la diminution sur celles qui subsistent encore, va de perte à {\itshape deux millions quatre cent mille} livres de compte fait, par un calcul juste et certain, vérifié sur les lieux ; et comme les revenus en fonds, bien que menant ceux d’industrie, n’en font pas la quatrième partie, ces derniers les excédant beaucoup davantage, c’est plus de dix millions de perte en pur anéantissement sur une seule Élection ; et bien loin que le roi ait rien gagné à ce beau ménage, il a perdu plus de cinq cent mille livres sur les Tailles, qu’il a fallu diminuer, tant dans cette Élection que dans les circonvoisines, à cause du déchet des biens ; et tant s’en faut encore que l’augmentation des Aides ait remplacé cette perte sur les Tailles ; elles n’ont pas atteint la dixième partie de ce dommage. Et comme ce sort est arrivé à l’Élection de Mantes par une cause générale à tout le royaume, on en peut tirer les mêmes conséquences, et supposer certainement la même perte pour toute la France.\par
Que l’on commence donc à aller bride en main, en prétendant revêtir l’auteur de ces Mémoires de l’idée d’extravagance, sur cette diminution de quinze cents millions de rente arrivée au royaume depuis 1660 ; d’autant que, quoique les Aides tiennent constamment le principal personnage dans un pareil désastre, y comprenant les Droits de sortie, passage et Douanes du royaume, qui ne sont ni moins criminels, ni moins outrageants pour la raison et le sens commun, que ces mêmes Aides, cause de tant de malheurs ; cependant ces prétendus droits du prince ont en outre pour consorts, dans la destruction de ses peuples, deux camarades qui les ont fort bien secondés, s’ils ne les ont pas égalés, dans l’anéantissement de ces quinze cents millions de rente, savoir, l’{\itshape injustice} et l’{\itshape incertitude} dans la répartition de la Taille, autre point où, bien qu’il n’y ait eu que de la négligence et du manque d’attention de la part de ceux qui gouvernaient, ou tout au plus un mauvais exemple personnel, en ce qui touchait leurs propres fonds, le désastre a cependant été si terrible par la ruine de la consommation, et par conséquent du revenu, que l’on peut assurer que si les démons avaient tenu conseil pour aviser au moyen de damner et de détruire tous les peuples du royaume, ils n’auraient pu rien établir de plus propre à arriver à une pareille fin. — On en fera un détail plus particulier dans la suite, lorsqu’il sera question de sa cessation ; ce qui n’exige point assurément une demi-heure d’attention de la part de MM. les ministres, et quinze jours d’exécution dans les provinces, quand cette commission sera donnée à des sujets versés en de pareilles matières, et surtout du pays comme autrefois, les Élus n’étant autre chose dans leur institution que des répartiteurs nommés par le peuple.\par
L’autre adjoint dans la ruine de la France est quelque chose de bien plus pitoyable encore : non seulement ce n’est point l’effet d’un intérêt indirect, comme dans les Aides, qui ait aveuglé les entrepreneurs pour se procurer de l’utilité aux dépens de la ruine publique, ni la faute du manque d’attention au bien général, comme dans la répartition des Tailles ; mais c’est au contraire une production de réflexions très sages et très pieuses à ce qu’on s’imagine, savoir : le soutien de l’{\itshape avilissement} des grains, que l’on a cru devoir établir et maintenir, par des efforts continuels d’une prétendue très fine politique, à être en perte au laboureur, le prix ne pouvant atteindre aux frais de la culture en quantité d’endroits, bien loin de satisfaire au paiement du propriétaire et des impôts ; ce qui a attiré, outre plus de 500 millions de diminution de rente dans le royaume, comme cela est aujourd’hui, l’abandon d’une infinité de terres de difficile exploitation, et la prodigalité des grains à des usages étrangers, comme nourriture de bestiaux et confection de manufactures ; ce qui ne menace rien moins que d’une cherté extraordinaire à la première stérilité. — En un mot, on a cru qu’afin que tout le monde fût à son aise, il fallait que les grains fussent à si bas prix, que les fermiers ne pussent rien bailler à leurs maîtres, et ceux-ci aucun travail, aux ouvriers ; ce qui étant tout leur revenu, la privation en excède dix fois le prétendu bas prix du pain. — Et l’on a pensé pareillement que pour éviter les horreurs d’une cherté extraordinaire, il est avantageux de faire abandonner la culture d’une infinité de terres, et l’engrais de presque toutes en général, le prix de la récolte n’en pouvant supporter les frais, et qu’il fallait aussi prodiguer les grains à ces usages étrangers que l’on vient de marquer. — Quelque horreur que doive inspirer une pareille conduite, qui a été un enfant de la spéculation, qui ne peut jamais produire que des monstres dans les arts, que l’on n’apprend jamais que par la pratique, jusqu’à celui de faire un soulier, que le plus grand génie du monde ne pourrait construire sur un mémoire dressé par l’ouvrier le plus habile, sans exhiber un objet ridicule ; il n’en est pas moins vrai que cette conduite a cru mériter des applaudissements, et que ses auteurs ont pensé qu’on devait les appeler les {\itshape Josephs} de leur pays. — Il y a un chapitre entier à la fin de cet ouvrage, et même, si l’on est curieux, on trouvera un petit volume où l’on fait voir, clair comme le jour, et sans aucune crainte de répartie, qui ne soit une extravagance achevée, que plus les grains sont à vil prix, plus les pauvres sont misérables, et surtout les ouvriers ; et, en même temps, que plus il sort de blés de la France, et plus on se garantit d’une cherté extraordinaire dans les années stériles.
\section[{Chapitre III.}]{Chapitre III.}
\noindent Voici le premier acte de la pièce, et sur lequel il faut faire une pause, pour commencer à soutenir, aux termes du cartel établi, que les revenus de la France sont diminués de quinze cents millions depuis 1660, et que les trois causes que l’on vient de marquer ont produit ce malheureux effet ; et que comme l’auteur se soumet d’être traité en insensé s’il ne rencontre pas juste, il maintient en même temps qu’il ne peut être démenti dans l’un et l’autre de ces deux faits, sans une extravagance achevée.\par
Or, pour revenir au premier dessein de cet ouvrage, on ne peut contester sur les principes établis au commencement, qui sont ceux de tous les États de la terre, que, les revenus du prince n’ayant d’autre source que ceux des peuples, quiconque pourrait rétablir en un instant les quinze cents millions de rente dont les peuples ont joui jusqu’en 1660, prouverait que tout ce qu’on a proposé pour le roi, savoir, les {\itshape quatre-vingts} millions de hausse dans la conjoncture présente, et le paiement de toutes les dettes de l’État sous son nom, ainsi que le doublement de tous ses revenus, au lieu d’être une extravagance, se trouve une chose fort naturelle et fort aisée ; puisque, bien loin d’être l’effet de vision ou de violence, ce ne serait qu’une suite, ou plutôt qu’une très petite partie d’une opulence générale répandue en quelque façon gratuitement ; et c’est de cette manière qu’on l’entend, comme on va voir bientôt, après qu’on aura montré dans un chapitre ce que c’est que la richesse suivant les lois de la nature, car la fausse idée qu’on s’en est faite dans ces derniers temps ayant produit tout le désordre, la simple reconnaissance de la cause du mal le fera cesser, et rétablira l’opulence.
\section[{Chapitre IV.}]{Chapitre IV.}
\noindent La richesse, au commencement du monde, et par la destination de la nature et l’ordre du Créateur, n’était autre chose qu’une ample jouissance des besoins de la vie : comme ils se réduisaient uniquement à la simple nourriture et au vêtement nécessaire pour se garantir des rigueurs du temps, le tout se terminait presque en deux seuls genres de métiers, savoir le laboureur et le pasteur, les troupeaux, avant le déluge, n’ayant point d’autre usage que d’habiller les hommes de leur dépouille ; et ce furent là les deux professions que se partagèrent les deux enfants d’Adam, après la création de l’univers. — À leur exemple, ceux qui les suivirent furent longtemps maîtres et valets, et les propres constructeurs de leurs besoins ; la vente n’était qu’un troc ou un échange, qui se faisait de la main à la main, sans nul ministère d’argent, lequel ne fut connu que longtemps après. — Mais, depuis, la corruption, la violence et la volupté s’étant mises de la partie, après les besoins on voulut le délicieux et le superflu ; ce qui ayant multiplié les métiers, de deux qu’ils étaient d’abord, degré par degré, en plus de deux cents qu’ils sont aujourd’hui en France, cet échange immédiat ne put plus subsister. — Le vendeur d’une denrée ne trafiquant presque jamais avec un sujet qui fût possesseur de celle qu’il avait dessein de se procurer en se défaisant de la sienne, et ne la pouvant même recouvrer qu’après un long trajet et une infinité de ventes et de reventes, par le moyen des deux cents mains ou professions qui composent aujourd’hui l’harmonie des États polis et magnifiques, il a fallu une garantie et une sorte de procuration, pour ainsi dire, de ce premier acheteur, que l’intention du vendeur serait effectuée par le recouvrement de la denrée qu’il voulait avoir en se dessaisissant de la sienne. — C’est par là que le ministère de l’argent est devenu nécessaire, par une convention et un consentement général de tous les hommes, qu’en quelque pays que ce soit, à moins de quelque grand éloignement, ou d’une violence qui dérange les choses, celui qui est porteur d’argent est assuré de se procurer pour autant de la denrée dont il a besoin, qu’il s’est défait de la sienne, et certain que l’objet de son désir lui sera livré avec autant de diligence et d’exactitude que si l’échange ou le troc s’en étaient faits immédiatement et de la main à la main, comme au commencement du monde. — Il y a là-dessus une attention à faire, qui est que l’argent, malgré la corruption qui en a fait une idole, ne peut fournir aucun des besoins de la vie étant réduit en monnaie, mais est seulement garant que le vendeur d’une denrée ne la perdra pas, et que celle dont il a besoin en troc de la sienne lui sera livrée, ne se trouvant pas chez son acheteur. — Il faut faire encore une réflexion, savoir, que cette fonction est si peu singulière à l’argent, quelque idée qui règne au contraire, qu’il n’en fait pas la dixième partie, et même la cinquantième dans les temps d’opulence, qui n’est autre chose qu’une grande consommation, c’est-à-dire une très grande richesse. — Le papier, le parchemin et même la parole en font, encore une fois, cinquante fois plus que lui : ainsi on a grand tort, dans les occasions de misère, de mettre la cause des désordres sur son compte, et d’alléguer pitoyablement qu’il a passé en la plus grande partie dans les pays étrangers. Pourquoi ne dit-on pas que le papier et le parchemin y sont également allés, et que c’est faute de matière que le trafic a cessé, et que l’on ne vend et n’achète plus ? — On ne le dit point, parce qu’on sait bien que cela serait ridicule. Or, de tenir le même discours de l’argent, est de la même absurdité, puisque, quand cette éclipse d’espèces serait véritable, comme non, on ne lui pourrait imputer que son sou la livre de la cessation du commerce, dans lequel n’ayant que la cinquantième partie des fonctions, on ne pourrait pas le rendre criminel pour un plus haut degré. Or, tout étant diminué depuis 1660 de plus de la moitié, on voit l’erreur de cette pitoyable raison, le manque d’argent. — Ces allégations seraient véritables au Pérou si les mines tarissaient, parce qu’étant uniquement le fruit du pays, il faudrait que les peuples y mourussent de faim s’ils n’en faisaient pas sortir toutes les années une très grande quantité du pays, pour l’échanger contre les denrées nécessaires à la subsistance. — Sans parler des îles Maldives, où, par une convention unanime, de certaines coquilles font la fonction de l’argent monnayé ; ni de celles de l’Amérique, où les colons de l’Europe qui les habitaient ne manquaient d’aucune chose nécessaire à leurs besoins, sans presque jamais voir un denier d’argent, parce que le tabac seul, tant en gros qu’en détail, en remplaçait toutes les fonctions ; et que, si on voulait avoir pour un sou de pain, et même moins, on donnait pour un sou de tabac, et ainsi du reste, ceux qui le recevaient étant assurés d’en tirer le même avantage, en se procurant leurs nécessités ; sans citer, dis-je, tous ces exemples, les foires de Lyon en France, qui forment un commerce par an de plus de 80 millions, n’ont jamais connu ni vu un sou d’argent dans ce trafic : tout se fait par échange immédiat de denrée à denrée, ou par billets, lesquels, après une infinité de mains, retournent au premier tireur, où il n’échoit qu’une compensation. — L’argent n’est donc rien moins qu’un principe de richesse dans les contrées où il n’est point le fruit du pays : il n’est que le lien du commerce, et le gage de la tradition future des échanges, quand la livraison ne se fait pas sur-le-champ à l’égard d’un des contractants ; et il partage même cette fonction avec tant d’autres choses, comme la simple parole, le papier, le parchemin et les denrées mêmes, qu’il est dispensé de la plus grande partie de ce personnage, qu’on lui suppose faussement être singulier. Il est même indifférent, pour ce qui lui reste d’emploi dans cet usage, dont on n’a jamais besoin que lorsqu’il n’apparaît pas assez de solvabilité dans l’un des contractants pour s’en fier à sa parole, au papier ou au parchemin ; il est indifférent, dis-je, qu’il y en ait peu ou beaucoup dans une contrée pour lui procurer de l’opulence, c’est-à-dire une entière jouissance, non seulement des besoins de la vie, mais même de tout ce que l’esprit humain a pu inventer pour les délices. — Il n’y a qu’une clause indispensable, à savoir que, s’il est indifférent que les choses soient à haut ou à bas prix, il est d’une nécessité absolue que le tout soit réciproque : autrement plus de proportion, et par conséquent plus de commerce ; et ainsi, plus de richesse, ou plutôt beaucoup de misère, qui est aujourd’hui la situation de la France. — Un homme qui recevait mille francs par an sous le roi François I\textsuperscript{er} était aussi riche, et passait sa vie aussi commodément et magnifiquement, que celui qui reçoit aujourd’hui quinze mille francs toutes les années, parce que le blé ne valait que vingt sous le setier à Paris, qui doit valoir aujourd’hui, année commune, quinze ou seize francs, et que les souliers ne se vendaient pas plus de cinq sous, par appréciation imprimée dans les ordonnances, comme on l’y peut voir. Le laboureur qui ne vendait son blé que vingt sous, et le cordonnier ses souliers que cinq sous, y trouvaient pareillement leur compte, parce que les proportions s’y rencontraient. — Mais si, comme aujourd’hui, le blé avait valu quinze francs, le cordonnier serait mort de faim avec ses souliers vendus cinq sous : comme par réciproque le laboureur eût tout quitté si, vendant son blé vingt sous, lui ou son maître eussent été obligés d’acheter les souliers quatre francs.\par
Ce sont donc les proportions qui font toute la richesse, parce que c’est par leur seul moyen que les échanges, et par conséquent le commerce, se peuvent faire : il serait ridicule de faire de la différence entre deux repas également bons, parce que l’un aurait coûté beaucoup et l’autre bien moins, en prétendant établir un plus haut degré de félicité dans celui pour lequel on aurait déboursé davantage. Et c’est par le déconcertement de cette harmonie que les 1 500 millions de rente, éclipsés en France depuis 1660, se sont évanouis. — Comme cette justice qui doit être entre deux commerçants qui ne trafiquent uniquement que l’un avec l’autre se doit étendre en plus de deux cents professions que renferme aujourd’hui la France, et qu’elles ont toutes un intérêt solidaire de l’entretenir, parce que ce n’est que d’elle seule qu’elles peuvent obtenir leur subsistance et leur maintien, il ne faut pas qu’elle soit déconcertée en la moindre de ses parties, c’est-à-dire que le plus chétif ouvrier vende à perte : autrement sa destruction, comme un levain contagieux, corrompt aussitôt toute la masse. Il faut que cela se fasse, non seulement d’homme à homme, mais aussi de pays à pays, de province à province, de royaume à royaume, et même d’année à année, en s’aidant et se fournissant réciproquement de ce qu’elles ont de trop, et recevant en contre-échange les choses dont elles sont en disette. — Cependant, par une corruption du cœur effroyable, il n’y a point de particulier, bien qu’il ne doive attendre sa félicité que du maintien de cette harmonie, qui ne travaille depuis le matin jusqu’au soir et ne fasse tous ses efforts pour la ruiner. Il n’y a point d’ouvrier qui ne tâche, de toutes ses forces, de vendre sa marchandise trois fois plus qu’elle ne vaut, et d’avoir celle de son voisin pour trois fois moins qu’elle ne coûte à établir. — Ce n’est qu’à la pointe de l’épée que la justice se maintient dans ces rencontres : c’est néanmoins de quoi la nature ou la Providence se sont chargées. Et, comme elle a ménagé des retraites et des moyens aux animaux faibles pour ne devenir pas tous la proie de ceux qui, étant forts, et naissant en quelque manière armés, vivent de carnage ; de même, dans le commerce de la vie, elle a mis un tel ordre que, pourvu qu’on la laisse faire, il n’est point au pouvoir du plus puissant, en achetant la denrée d’un misérable, d’empêcher que cette vente ne procure la subsistance à ce dernier, ce qui maintient l’opulence, à laquelle l’un et l’autre sont redevables également de la subsistance proportionnée à leur état. On a dit, {\itshape pourvu qu’on laisse faire la nature}, c’est-à-dire qu’on lui donne sa liberté, et que qui que ce soit ne se mêle à ce commerce que pour y départir protection à tous, et empêcher la violence. — C’est néanmoins de quoi on a pris le contrepied, n’y ayant point de moyens, quelque épouvantables qu’ils fussent, qu’on n’ait crus non seulement légitimes, mais qu’on n’ait même réputés l’enseigne de la plus fine politique pour ruiner cette harmonie, en attaquant ou accablant singulièrement toutes les denrées, les unes après les autres, par le moyen des partisans. Quand on avait détruit un genre de biens, en sorte qu’il n’y avait plus rien à faire pour les entrepreneurs, qui causaient cette désolation sous prétexte de faire venir de l’argent au roi, bien qu’il ne reçût pas la centième partie du mal que cela causait, on transportait les mêmes mesures aux autres genres de biens qui n’étaient pas encore anéantis, en surprenant toujours MM. les ministres ; en sorte que celui qui a le plus ruiné de pays, et par conséquent le roi, est celui qui a le mieux fait ses affaires.\par
Les grands profits attachés à de pareilles entreprises, et qui donnaient moyen de partager avec des protecteurs du premier degré, que l’on veut croire que l’on trompait également, mais qui étaient néanmoins les premiers ministres jusqu’en 1661, comme il sera justifié, faisaient qu’on se mettait l’esprit à l’alambic pour maintenir et augmenter cette manœuvre, et empêcher en même temps toutes sortes de remèdes et d’obstacles que les peuples y auraient pu apporter. Mais on ne laisse pas de croire que, depuis cette époque, il n’y ait encore eu que de la surprise, bien que ces manières aient sextuplé, et qu’on ait englouti jusqu’aux immeubles qui avaient toujours paru sacrés. Du reste, ceci est trop public pour passer pour calomnie, ou être révoqué en doute : les 1 500 millions de rente constamment éclipsés, les terres en friche, plus de la moitié des vignes du royaume arrachées, pendant que les trois quarts des peuples ne boivent que de l’eau, arrêtent la grande vocation que les intéressés pourraient avoir à nier des faits aussi certains, et dont on leur est uniquement redevable ; et voici comme cela est arrivé.\par
C’est, par le moyen des Traitants, {\itshape trop peu d’attention} à la répartition des Tailles, et {\itshape trop d’attention} au commerce des blés et des liqueurs, dont il fallait absolument laisser l’économie à la nature, comme partout ailleurs. — Il convient de faire un court détail de ces trois causes, et l’on verra que ce n’est pas sans raison qu’on maintient qu’elles ont fait plus de destruction dans la France que jamais les plus grands ennemis, et même tous les fléaux de Dieu dans leur plus grande violence ; le ravage de ces manières ayant regagné par leur durée, depuis 1660, ce qui pourrait paraître de plus violent dans ces marques extraordinaires de la colère du Ciel.
\section[{Chapitre V.}]{Chapitre V.}
\noindent Pour commencer par les Tailles, dont on ne dira que peu de chose, parce qu’on en a assez parlé dans le livre intitulé le {\itshape Détail de la France}, auquel ceux qui sont curieux d’en apprendre parfaitement l’anatomie pourront avoir recours, et dont ce qu’on va toucher ne sera qu’un abrégé, il y a, avant que d’en parler, une attention à faire, qui servira également pour cet article et pour les deux autres.\par
Tous les revenus ou plutôt toutes les richesses du monde, tant d’un prince que de ses sujets, ne consistent que dans la consommation ; tous les fruits de la terre les plus exquis et les denrées les plus précieuses n’étant que du fumier d’abord qu’elles ne sont pas consommées. Ce qui fait que les pays les plus féconds non habités et par conséquent cultivés, à cause du petit nombre d’hommes, sont presque entièrement inutiles, à leur prince. — Or, du moment que, quoique ces contrées se rencontrent très remplies de sujets propres à faire valoir les présents de la nature, il est de leur intérêt de ne rien consommer, et qu’ils sont même mis dans l’impossibilité de le faire, le pays ni le prince n’en sont pas plus riches que s’il n’y avait qui que ce soit ou peu de monde. La terre devient alors comme un herbage du plus grand produit, qui ne rapporte rien à son maître lorsque les bêtes que l’on met dessus sont emmuselées et empêchées de pâturer par cette violence, ce qui ruine entièrement l’herbage et les propriétaires des bêtes, qui meurent par cette force majeure, bien loin d’engraisser.\par
Voilà le portrait en raccourci de la Taille dans les provinces où elle est arbitraire, c’est-à-dire dans presque les trois quarts du royaume, sans qu’il y ait en aucune façon la moindre différence. Et cela, par le moyen de trois circonstances qui l’accompagnent, et ne la quittent jamais un moment : — la première, son incertitude, tant dans l’assiette des paroisses que sur la tête de tous les. particuliers ; — la seconde, son injustice d’être haute et violente, non par rapport aux facultés des contribuables, ce qui est néanmoins l’esprit de son institution, comme dans tous les pays de la terre, même les plus barbares et les plus grossiers, mais eu égard seulement au plus ou moins de protection et d’élévation qu’un homme peut avoir pour s’en défendre, lui ou ses fermiers ; — et la troisième enfin, la collecte de cet impôt, dont, à cause de la mauvaise répartition, une grande partie demeure en perte à ceux qui sont chargés de ce malheureux recouvrement ; et comme chacun y passe à son tour, il échoit à tout le monde, par conséquent, d’être à tour de rôle ruiné tout à fait.\par
Pour reprendre chaque article, et montrer qu’il n’y eut jamais de plus grands bourreaux de la consommation : d’abord, l’incertitude, qui commence la danse, met dans l’obligation tous les sujets qui y sont exposés de s’abstenir de toutes sortes de dépenses, et même de trafic qui fasse bruit : il n’y a qu’un ordinaire de pain et d’eau qui puisse faire vivre un homme en sûreté de n’être pas la victime de son voisin, s’il lui voyait acheter un morceau de viande ou un habit neuf ; s’il a de l’argent par hasard, il faut qu’il le tienne caché, parce que, pour peu qu’on en ait le vent, c’est un homme perdu. — Par l’injustice, qui est le second article, il est fort naturel et fort ordinaire de voir une grande recette ne pas contribuer d’un liard pour livre, pendant qu’un malheureux qui n’a que ses bras pour vivre, lui et toute sa famille, est à un taux qui excède tout ce qu’il a vaillant ; en sorte qu’après la vente de quelques chétifs meubles, comme paillasse, couverture et ustensiles propres seulement au travail manuel, on procède à la vente des portes, des sommiers et de la charpente des maisons. Ce qui ruine ce prétendu privilégié, et le roi par conséquent, bien plus que si ce fonds presque exempt avait payé six fois la Taille où il est imposé, et qu’il en eût déchargé tout à fait ce malheureux ; parce que toutes les terres n’ayant du produit, ainsi qu’on a dit, qu’à proportion que les fruits qui y croissent trouvent de la consommation, et ceux qui la pourraient faire en étant empêchés par ces manières, ces fruits tombent en pure perte, et les maîtres n’en tirent pas les frais de la culture. Et pour le faire voir sans crainte de nulle répartie, il n’y a qu’à jeter les yeux sur une infinité de grands domaines appartenant à des gens de la plus haute considération, on les trouvera diminués depuis 1660, qu’on a entièrement abandonné l’attention à la juste répartition des Tailles, sans renouveler ni faire observer les anciennes ordonnances, qui ne parlaient d’autre chose que d’y veiller continuellement ; on verra, dis-je, que ces terres sont diminuées de moitié l’une portant l’autre, et quelques-unes davantage, pour servir de soulte aux autres afin que le tout soit sous le même niveau, sans qu’on en puisse accuser sans fausseté l’excès de la Taille, dont ces terres n’ont jamais presque rien payé, et ce sera rendre un très grand service à leurs maîtres que de leur en faire prendre leur juste part, pour décharger les misérables, puisque par là, la cause de la ruine de ces fonds étant ôtée, ils reprendront incontinent leur ancienne valeur. Et ceux qui ont quelque connaissance du {\itshape Détail} en conviennent ; mais ils marquent en même temps qu’il faut que la chose soit générale, sans quoi une justice particulière qu’on pourrait faire ne produirait qu’une hausse de paiement, sans nulle utilité singulière. — Et la collecte enfin, venant en surtaux sur des sujets déjà accablés, et les constituant en quelque manière cautions et garants de paiements dont le recouvrement d’une partie ne se pourra jamais faire, achève de les ruiner et met le comble à leur désolation, ou plutôt à leur désespoir ; ce qui, sans parler des emprisonnements, dont le nombre est tel qu’une infinité de collecteurs de Tailles font plus de séjour dans les geôles que dans leurs maisons mêmes, est le dernier degré de destruction de la consommation, par la perte de leur temps, qui est tout leur revenu, ainsi que celui du roi et du royaume. — Ce désordre, qui coûte plus de 500 millions de perte par an à la France, et la vie à tant de malheureux qui périssent, tant en santé qu’en maladie, faute de nourriture et de secours, ainsi que de bâtiments qui les puissent défendre des injures du temps, ayant été en la plus grande partie détruits par cette belle économie de la Taille ; ce désordre, dis-je, quelque grand et quelque effroyable qu’il soit, peut être arrêté en une demi-heure de travail et quinze jours d’exécution, puisqu’il n’est question que d’un simple acte de la volonté du roi et de MM. les ministres, comme on expliquera mieux et plus particulièrement dans le chapitre du remède.\par
Il faut passer à la seconde cause de la destruction de 1 500 millions de rente, qui sont les blés, à l’égard desquels il faut rappeler ce qu’on a dit ci-devant, que la richesse n’est autre chose qu’une jouissance entière, non seulement de tous les besoins de la vie, mais même de tout ce qui forme les délices et la magnificence, pour lequel il faut avoir affaire avec plus de deux cents professions, qui composent aujourd’hui les États polis et opulents. À cet effet, il est nécessaire que tous ces deux cents métiers fassent un échange continuel entre eux, pour s’aider réciproquement de ce qu’ils ont de trop, et recevoir en contre-échange les choses dont ils manquent ; et cela non seulement d’homme à homme, mais même de pays à pays et de royaume à royaume ; autrement l’un périt par l’abondance d’une denrée ou sa disette, pendant qu’un autre homme, ou une autre contrée, sont dans la même misère d’une façon tout opposée. C’est ce divorce qui forme la misère générale, tandis que le commerce réciproque qui aurait pu se faire aurait formé deux perfections de deux très grandes défectuosités.\par
Il y a encore une attention à faire, qui est que ce désordre durera éternellement, si ce trafic, ou cet échange, si nécessaire et si utile, ne se fait avec un profit réciproque de toutes les parties, c’est-à-dire tant des vendeurs que des acheteurs, soit que le commerce se fasse par le canal de l’argent, ou par troc de denrée à denrée ; et celui qui prétend faire autrement non seulement ruine son correspondant, mais se détruit aussi lui-même. Si le premier laboureur, trafiquant uniquement avec le pasteur, ne lui avait pas voulu donner assez de blé pour se nourrir, pendant qu’il eût exigé de lui tout son vêtement nécessaire, tiré des dépouilles des bêtes, non seulement il l’aurait fait mourir de faim, mais il aurait lui-même péri dans la suite de froid, en détruisant le seul ouvrier de ce besoin si pressant, savoir le vêtement. Et cette harmonie, d’une nécessité si indispensable alors entre ces deux hommes, est de la même obligation entre plus de deux cents professions qui composent aujourd’hui le maintien de la France. Le bien et le mal qui arrivent à toutes en particulier est solidaire à toutes les autres, comme la moindre indisposition survenue à l’un des membres du corps humain attaque bientôt tous les autres, et fait par suite périr le sujet, si on n’y met ordre incontinent.\par
Le dépérissement qui arrive à une de ces deux cents professions n’est pas d’abord aussi sensible que celui qui aurait pu se rencontrer entre les deux premiers et uniques ouvriers de la terre ; mais avec le temps, et en augmentant à vue d’œil, il produit le même effet qu’aurait fait l’autre. Le vendeur n’est donc que le commissionnaire de l’acheteur, comme l’acheteur est mis dans le pouvoir d’acheter par le vendeur, qui en doit faire autant de la denrée de ce premier acheteur, ou immédiatement, ou par une plus longue circulation au moyen de l’argent, toujours aux conditions marquées, c’est-à-dire avec une utilité perpétuelle de tous ceux qui jouent un personnage sur ce théâtre, c’est-à-dire de tous les hommes du monde.\par
On a fait ce préambule, parce que la dérogeance à cette règle à l’égard des blés coûte à la France, depuis 1660, près de trois à quatre cents millions de rente. Comme cette denrée mène toutes les autres, qui la suivent pour ainsi dire pied à pied, le mécompte qui s’y rencontre ne fait aucun crédit, et embrassant aussitôt toutes les professions, il les coule à fond sur-le-champ.\par
Si le laboureur, qui est leur commissionnaire pour les faire subsister, vend son blé trop cher, par un prix qui n’ait pas de proportion avec le prix du travail de ces deux cents métiers, voilà une famine qui fait périr une infinité de monde, dont on n’a que trop fait d’expérience ; et par fait contraire, le blé étant à vil prix comme aujourd’hui, ne pouvant atteindre non seulement au paiement du propriétaire, mais même aux frais de la culture, le canal nécessaire pour faire passer cette manne aux mains des ouvriers, qui n’ont d’autre revenu que leurs bras, est coupé, savoir le maître, qui n’est point payé. Et voilà toutes ces deux cents professions à sec ; leur travail leur devient infructueux, comme les grains en perte à ce laboureur : en sorte qu’il est par là mis hors de pouvoir, non seulement de payer son propriétaire, mais même de continuer à cultiver la terre ; ce qui en fait demeurer quantité en friche, négliger les engrais des meilleures, et prodiguer les grains à des usages étrangers, comme nourriture de bestiaux, surtout les chevaux, et confections de manufactures, savoir les bières et amidons ; ce qui encore, à la première année stérile, ne manque pas de produire une cherté extraordinaire ; par où ces deux cents professions ressentent la même misère par un excès tout opposé, pendant que la compensation de ces deux désordres en eût formé deux grands biens, comme on a déjà dit, si un zèle mal fondé n’avait pas procuré ce mal d’avilissement de grains, qui enfante lui seul l’autre extrémité, savoir le prix exorbitant. Le remède est aisé, et en la main de MM. les ministres ; mais comme le manque de lumière a fait tomber dans ce désordre, dont la connaissance, la plus grossière et la plus imparfaite, ne peut être acquise que par la pratique du labourage, il s’en faut beaucoup que ce soit l’espèce de ceux qui se sont mêlés, depuis 1660, de cette direction. Ils ont cru que cette manne coûtait aussi peu à percevoir et faire venir que celle que Dieu envoya dans le désert aux Israélites, ou tout au plus qu’elle était comme des champignons, ou comme des truffes ; quelle croissait en tout son contenu à pur profit au laboureur, et qu’à quelque bas prix qu’elle pût être, il gagnait moins, mais ne pouvait jamais {\itshape perdre} ; et qu’ainsi il fallait qu’une autorité supérieure empêchât que les pauvres ne fussent la victime de son avidité. C’est néanmoins cette autorité qui a tout gâté, ayant également ruiné les riches et les pauvres, dans l’une et dans l’autre extrémité de cherté et d’avilissement des grains, qui se sont enfantées et s’enfantent même toujours réciproquement, comme on verra plus particulièrement par le chapitre qui est à la fin de cet ouvrage.\par
Ainsi, ces deux articles du désordre des Tailles et des blés coûtent la moitié des 1 500 millions de perte arrivés au royaume depuis 1660, d’autant plus aisée à rétablir, que ce n’a été l’effet d’aucun intérêt particulier, mais seulement manque d’attention dans l’un, et suite de trop d’attention dans l’autre, savoir les grains. Il n’y avait qu’à laisser faire la nature, comme partout ailleurs, et la liberté, qui est la commissionnaire de cette même nature, n’aurait pas manqué de faire une compensation avantageuse, qui aurait formé un très grand bien de deux très grandes misères. Le surplus des 1 500 millions de déchet, allant à environ 800 millions, est l’unique ouvrage des Traitants, tant ordinaires qu’extraordinaires. Mais, quoique le rétablissement soit beaucoup plus aisé du côté de la nature, il est beaucoup plus difficile de la part des personnes intéressées au maintien de ce mal, quelque effroyable qu’il soit ; et il en arrive comme dans les maladies du corps humain, qui sont d’autant plus dangereuses qu’elles attaquent les parties les plus nobles.\par
C’est une chose aujourd’hui si publique, bien que ce fût un crime autrefois d’être de part, et de recevoir des gratifications de gens d’affaires, que personne ne s’en cache plus ; et quoiqu’un savant théologien ait imprimé, il y a trente ans, que c’est risquer sa damnation que de se faire Partisan, les choses ont si fort changé depuis, que les personnes aujourd’hui de la plus haute piété ne se font plus aucun scrupule, non seulement de prendre part à ce métier, mais même de l’avouer publiquement.\par
Apparemment que l’ignorance où elles sont des maux qu’un pareil canal des revenus du prince fait au roi et au royaume, les entretient dans cette tranquillité ; ce qui ne serait pas si elles savaient que le souverain ne reçoit pas un sou par de semblables moyens, qu’il n’en coûte dix-neuf sur vingt en pure perte aux peuples, par la ruine de la consommation, et par conséquent de leurs biens, ainsi que la vie à une infinité de misérables, qui périssent manque de leurs besoins.\par
Que l’on jette les yeux sur une contrée désolée, comme sur l’Élection de Mantes, puisqu’on en a parlé ; ce qui prouve également pour le reste du royaume, attendu que c’est par une cause générale : elle a perdu 2 400 000 livres sur les seules vignes, ce qui fait plus de dix millions de dommage par an sur les biens, tant en fonds qu’en industrie, par contrecoup ; et que l’on en demande la raison même aux enfants qui ne font que quitter la mamelle, ils ne bégayeront point pour dire que c’est l’ouvrage des Traitants, apprenant par là à parler de leurs parents. Cependant la haute protection que ces messieurs ont, et qu’ils savent se procurer, fait qu’on les respecte si fort, que pour leur contribution, pour la quote-part de la cessation de leur ministère, au rétablissement en deux heures de 500 millions, dans la destruction desquels, et même beaucoup davantage, ils jouent un si grand rôle, on n’en veut pas congédier un seul, ni leur ôter un cheveu de la tête, comme si c’étaient les gens du monde les plus nécessaires à l’État, loin d’être ses plus grands ennemis, au témoignage de M. de Sully parlant à Henri IV. Ce qui n’empêche pas qu’on ne montre, comme l’on va faire voir dans le chapitre suivant, que le crime les a établis et maintenus jusqu’en 1660, depuis lequel temps, encore qu’ils aient quadruplé et sextuplé, ce n’a été que par surprise à l’égard de MM. les ministres, qui n’avaient que de bonnes intentions, bien que les malheurs opérés par le crime de leurs prédécesseurs aient reçu la même hausse que leur nombre et leurs fonctions.
\section[{Chapitre VI.}]{Chapitre VI.}
\noindent Les princes les plus riches et les peuples les moins chargés sont ceux chez qui les impôts passent droit des mains des contribuables en celles du monarque, et où il y a le moins de genres de tributs, et par suite de personnes employées à leur recouvrement ; ou plutôt toutes les nations du monde, tant anciennes que nouvelles, n’ont jamais connu que ces manières, ainsi que la France, pareillement, jusqu’au règne de François I\textsuperscript{er}.\par
Les Romains n’avaient pas sitôt conquis un pays qu’ils y imposaient un tribut. Quel était ce tribut ? C’était ou une somme par feu, c’est-à-dire par cheminée, ou un dixième du revenu, ce qui se levait par des receveurs ou questeurs, sans autres frais que des appointements réglés à ceux qui faisaient cette recette ; et cette redevance de cheminées et de dixième a été longtemps l’unique impôt en France, ainsi que dans les autres provinces qui y ont été jointes ; ce qui est encore en Angleterre et serait toujours en France, si cela n’enrichissait pas, seulement, le prince et les peuples. Ainsi nul déconcertement dans le commerce, nul embarras dans le trafic des peuples, et par conséquent ni juges, ni ordonnances pour ce sujet, dont on ne trouve pas la moindre trace chez tous les écrivains qui nous ont laissé l’histoire de ces maîtres du monde. — Le monarque ottoman administre aujourd’hui une domination de douze cents lieues d’étendue, à la prendre presque de tous les côtés, de la même façon. Soixante et dix receveurs répandus dans les diverses contrées qui composent cet empire font toute la recette, et en comptent tous les trois mois à un receveur général résidant dans la capitale, qui rapporte ensuite aux ministres, sans que cela prenne plus d’une heure ou deux la semaine de tout le temps des uns ou des autres.\par
Tous les tributs de ce grand empire sont de deux espèces uniquement, savoir : une légère capitation, qui se paie depuis les enfants à la mamelle jusqu’au plus grand âge, et les douanes sur les sorties et entrées des États du prince principalement. Ce qui a un taux certain, savoir, trois, cinq ou dix pour cent, qui est le plus haut degré : ainsi nuls juges, nulles ordonnances, parce qu’il n’y a nul procès sur de pareilles matières, non plus que dans l’empire romain, ou plutôt dans tous les États du monde. — Le Mogol a 68 millions de revenus, administrés de pareille façon, ce qui fait qu’on en a une connaissance parfaite ; cette douane, dis-je, est affermée 68 millions par un bail de deux lignes, savoir que tout ce qui sort et entre doit la dixième partie en argent ou en nature, au choix du marchand, de façon qu’il ne faut pareillement ni juges ni ordonnances pour les impôts, parce qu’il ne peut y avoir de procès. — En Angleterre, présentement, le peuple que l’on sait être le moins souple de la terre, paie tranquillement le cinquième de tous ses revenus, dont l’assiette se fait par les habitants de chaque paroisse, et la perception par les ministres ou curés, qui en portent le montant en recette, sans frais et sans procès. Cependant, ce peuple, si jaloux de sa liberté, se porte volontiers à de si hautes contributions, non pour défendre son pays que l’on voudrait envahir, mais par pure jalousie et envie de la gloire du premier prince du monde, parce que le ciel le comble de bénédictions, ainsi que sa famille royale. — En Hollande, la contribution des peuples, pour une guerre qui a le même objet, va à la troisième partie des revenus. Cependant, là non plus qu’en Angleterre, on n’y voit aucuns pauvres, quoique ces pays soient beaucoup moins bien partagés par la nature que n’est la France. C’est-à-dire, que qui que ce soit n’y demande l’aumône en titre d’office, et il n’y a point de sujet, si dépourvu qu’il puisse être, qui, loin d’être réduit au pain et à l’eau, n’use de viande et de liqueur, ou de nourriture équivalente, ne soit vêtu de drap et chaussé de souliers, la chaussure de bois y étant tout à fait inconnue.\par
Cependant ce cinquième en Angleterre, et même plus, et ce troisième en Hollande, de tous les revenus, s’exige et se perçoit non seulement sans procès et sans questions, mais même sans contrainte, exécutions ni emprisonnements ; bien que dans l’un et dans l’autre de ces deux États ce degré d’impôts aille à plus de 100 millions par an, c’est-à-dire sur le pied de plus de 300 millions en France, par rapport à la différence de la richesse naturelle de ces contrées avec celle de ce dernier royaume. Et c’est aussi, d’ailleurs, ce qu’il a payé, tant qu’il a été administré par les mêmes principes que l’Angleterre et la Hollande, c’est-à-dire quand le nombre des impôts se réduisait à trois ou quatre genres, qu’ils étaient justement répartis, et passaient droit des mains des peuples en celles du prince.\par
Que ce discours ne surprenne ni ne soulève point les esprits ; la preuve et la vérification en vont être faites en parlant du règne de François I\textsuperscript{er}. Mais, pour l’anticiper en quelque manière, on dira que cela est aisé à supposer dans une disposition où il n’y avait que trois ou quatre genres de tributs, et cent ou six-vingts personnes au plus payées par le prince pour les percevoir, et nuls juges, parce qu’il n’y avait point de procès, nulles terres en friches, ni aucunes denrées en perte au marchand. Au lieu qu’à présent il n’y a pas moins de dix mille genres de tributs, y en ayant plus de cent cinquante sur la seule administration de la justice, tous venus depuis 1660 ; dix mille juges pareillement, au moins, qui n’ont d’autre fonction que de décider les procès, inséparables de pareilles manières, et cent mille hommes employés à la perception ou aux poursuites qu’elle entraîne, se payant presque tous par leurs mains avec la libéralité que tout le monde connaît, c’est-à-dire que le dernier des hommes croit pouvoir faire légitimement et fait pour l’ordinaire une fortune de prince. Le tout sans parler de la part du néant qui, naissant, comme on a déjà dit, sous les pieds de pareils entrepreneurs, en absorbe sur vingt parts dix-neuf, et ne laisse passer aux mains du roi que cette vingtième partie, sur laquelle il leur faut encore les préciputs marqués, en sorte que plus de la moitié du royaume est inutile tant au prince qu’à ses peuples. Que l’on ne quitte jamais de vue les vignes de Mantes, car elles sont véritablement la mesure dont il faut se servir pour évaluer les désastres de tout le royaume ; et ceux qui se trouveront choqués par un pareil énoncé, n’auront d’autre parti à prendre qu’un profond silence ; ou bien ils s’attireraient plus que le soupçon de n’avoir pas participé à de pareils désordres par l’effet seulement d’une simple surprise.\par
Mais, pour revenir à la gestion et au gouvernement de la France durant onze cents ans, on peut assurer qu’elle a été régie, depuis son établissement jusqu’à la mort de François I\textsuperscript{er}, arrivée en 1547, comme l’Angleterre et la Hollande, ou plutôt comme tous les États du monde. Les rois vivaient et subsistaient magnifiquement de leurs seuls domaines, hors les occasions extraordinaires, comme des guerres qui pouvaient survenir, que leurs sujets donnaient tous les secours nécessaires par les voies susdites de dixième ou de cheminées. — Mais la Religion, par des surprises assez connues, se fit donner la plus grande partie de ces domaines (ce qui l’a entièrement perdue, au rapport de Gerson), parce qu’alors l’ignorance était si grande, qu’on ne connaissait presque point d’autre piété que de donner ses terres et ses fonds à l’Église, jusque-là que l’on voit celle-ci accorder l’absolution aux mourants de les avoir volées et enlevées de force aux légitimes possesseurs, lorsqu’on en donnait une partie à ses ministres. Outre que ces faits se trouvent attestés par des écrits originaux, Mézeray, auteur célèbre, en fait une ample mention avec des circonstances encore plus affreuses, en sorte qu’on n’a cru rien faire d’extraordinaire d’en toucher quelques mots, pour obliger à faire attention aux acquisitions que font les mainmortes tous les jours avec applaudissement en France, bien qu’elles soient défendues dans tous autres États chrétiens, et que le prince des Pays-Bas fasse serment, en prenant possession, que l’Église, n’acquerra rien de son règne, et que la république de Venise crut autrefois pouvoir et devoir entreprendre une guerre contre Rome, jusqu’à se faire excommunier pour ce sujet.\par
Ces manières qui firent bannir la religion catholique de Suède dans les siècles passés, pour retirer presque tous les biens du royaume, dont elle s’était emparée, et les réunir à la Couronne, dont ils sont presque seuls l’entretien aujourd’hui, obligèrent les rois de France de mettre d’abord sur les peuples les Tailles, qui se percevaient par les peuples mêmes, sans aucun ministère étranger relies n’étaient pas perpétuelles, mais suivant et à proportion des occasions. — On y ajouta ensuite les Aides dans les villes franches, pour y tenir lieu de Tailles, dont la perception se faisait également par les peuples, presque uniquement sur les cabarets, tous les Nobles et Privilégiés en étant exempts, n’y ayant alors aucuns droits d’entrée ni de passage, mais seulement quelques droits de sortie hors le royaume, ce qui se pratique partout. — La Gabelle ou l’impôt sur le sel vint ensuite, c’est-à-dire que les rois achetaient toute cette denrée des propriétaires qui la faisaient fabriquer, et la faisaient revendre dans des greniers, avec obligation aux peuples de n’en point prendre ailleurs ; et quoique ce fût à un prix très modéré, et qui était quatre fois moindre que celui d’aujourd’hui, le prince en tirait beaucoup davantage, par proportion et par rapport au taux où toutes choses étaient dans ce temps-là.\par
Ainsi tout se réduisait à ces quatre sortes de revenus, presque administrés sans aucune main étrangère que celle des peuples. Il n’y avait ni ministres, ni conseil des finances : la cour des Aides de Paris se réduisait à quatre officiers ; les trésoriers de France à deux, et l’Élection de même, qui étaient plutôt des directeurs, que des juges de procès qui ne pouvaient jamais naître.\par
Et les ministres du prince n’avaient d’autre fonction que la dispensation de l’impôt, sans en avoir aucune relative à sa perception, quoiqu’à présent, quand les journées seraient six fois plus longues à leur égard qu’à celui des autres hommes, ils n’auraient pas la moitié du temps nécessaire pour suffire à cette dernière besogne, malgré le grand nombre d’autres personnes qu’ils appellent et s’associent tous les jours à cet effet : bien loin alors d’être accablés et de succomber presque comme aujourd’hui sous le faix, il était indifférent qu’ils fussent dans le royaume pour ce sujet, ou absents à deux ou trois cents lieues. La levée des deniers du prince, qui était uniquement l’affaire des peuples, n’en était pas retardée d’un moment, témoins Brissonnet et Devers, les deux premiers ministres des finances du roi Charles VIII, qui purent accompagner ce prince à la conquête du royaume de Naples, pendant vingt-deux mois, sans que la recette de ses deniers en souffrît le moindre inconvénient. Voilà comme les affaires étaient administrées, c’est-à-dire sans nul emploi, ni occupation pour la perception des finances, de la part de ceux qui gouvernaient.\par
Il faut voir maintenant quel en était le produit, et si, les choses ayant véritablement changé en France depuis ce temps du tout au tout, on peut soutenir, sans renoncer à la raison, que ç’a été pour l’avantage du royaume, tant par rapport à la quotité que le prince reçoit, qu’à la facilité que les peuples ont à lui fournir ses redevances et ses besoins, tant à l’ordinaire que dans les conjonctures importantes, comme est celle d’aujourd’hui. — Le roi François I\textsuperscript{er}, qui fut le dernier règne où cette heureuse situation ne reçut point d’atteinte, savoir où les peuples seuls se mêlaient des impôts, qui se réduisaient à trois ou quatre genres, ainsi qu’on a dit, et non pas à dix mille comme aujourd’hui, sans aucun ministère étranger, à plus forte raison sans donner de l’emploi à plus de cent mille hommes qui ont présentement cette fonction, avec une forte espérance, à l’exemple de leurs semblables, d’y faire une très grande fortune par la destruction du commerce et du labourage, pour ne pas dire par la ruine du roi et de ses peuples, quoique ce soit la même chose ; François I\textsuperscript{er}, dis-je, levait seize millions de tribut réglé dans son royaume, qu’il laissa tranquillement à son successeur, quoiqu’il possédât un cinquième moins d’États que ne fait à présent le grand monarque qui règne. Cela se voit dans les Mémoires imprimés de M. de Sully, lequel avait vu et vécu avec les contemporains. Or, on maintient que les 16 millions de ce temps fournissaient au roi François I\textsuperscript{er} sur le pied de 240 millions, en sorte que s’il avait joui de ce qui a été réuni à la France depuis, il aurait eu 300 millions de rente sans qu’il y eût rien manqué. — Que l’on marche encore une fois bride en main sur le prétendu ridicule de ce fait : il est véritable dans tout son contenu, et ce qui va suivre en va faire convenir ceux même qui auront plus de désagrément à passer un pareil aveu, par rapport à l’intérêt et à la part qu’ils ont aux manières que l’on pratique. Les peuples, sous François I\textsuperscript{er}, payaient 240 millions d’aujourd’hui, parce que pour fournir la somme de 16 millions il leur fallait vendre la même quantité de denrées qu’il serait nécessaire pour payer à présent 240 millions ; et le roi jouissait de 240 millions, parce qu’avec cette somme ceux à qui il les distribuait se procuraient le même degré de leurs besoins qu’ils pourraient faire à présent avec 240 millions. Toutes choses n’étaient qu’à la quinzième partie du prix qu’elles sont aujourd’hui. Pour en convenir, il n’y a qu’à jeter les yeux sur les ordonnances de police imprimées en ce temps-là ; on verra que le blé est apprécié à vingt sous le setier, mesure de Paris, qui doit être et a même été depuis trente ans, l’un portant l’autre, à quinze ou seize francs, quoique le partage en ait été très mal fait, ayant été tantôt une fois plus haut, et tantôt une fois plus bas, qui est une des principales causes de la misère de la France, bien que ce ne soit rien moins que l’effet du hasard, mais d’un zèle aveugle et d’une piété mal comprise ; ce qui étant aisé à rétablir, sera la principale ressource dans la conjoncture présente pour la fourniture des 80 millions.\par
Mais pour revenir à la parité des 16 millions du roi François I\textsuperscript{er} avec 240 millions d’à présent, on soutient que de dire que ce n’est pas la même chose sans aucune différence, c’est prétendre que le roi saint Louis, qui ne donnait que 6 000 livres à sa fille en la mariant à un roi de Castille, n’était pas plus riche qu’un médiocre homme de boutique aujourd’hui dans Paris, qui donne souvent plus que cette quantité d’argent à un gendre de même métier que lui Il faudrait pareillement dire qu’un maître maçon, qui gagnait quatre deniers par jour il y a trois cents ans dans Paris, comme l’on voit par des registres publics de ce temps-là, donnait tout son temps et toute sa peine pour moins que demi-livre de pain par jour ; et comme il n’y eût pas eu seulement assez pour déjeuner, il aurait fallu que, pour le surplus, lui et toute sa famille demandassent l’aumône, si ces quatre deniers n’avaient pas suffi pour avoir autant de denrées que l’on s’en procurerait à présent avec trente sous. On ne poussera pas plus loin le ridicule de ceux qui voudraient soutenir qu’il y ait de la disparité entre les 16 millions du roi François I\textsuperscript{er}, en revenu réglé, tant dans la cause que les effets, et 240 millions d’à présent. Mais pour faire voir que la suite et la dépendance de son règne répondaient à une pareille richesse, il n’y a qu’à jeter les yeux sur ce qui se passa de son temps.\par
Personne n’ignore que, presque durant tout le temps qu’il régna, c’est-à-dire pendant plus de trente ans, il eut précisément à défendre le royaume contre les mêmes nations qui ont aujourd’hui conjuré la ruine de la France. L’on sait encore que ces peuples, au lieu d’obéir à différents princes, comme à présent, étaient soumis à une ou deux têtes couronnées seulement, savoir l’empereur Charles-Quint et son frère Ferdinand, roi de Hongrie ; que l’Angleterre se mit souvent de la partie, le pape et les Vénitiens de même, et qu’il n’est pas jusqu’aux Suisses qui ne lui déclarèrent la guerre, sur laquelle nation très belliqueuse il obtint l’unique et la plus grande victoire qu’aucun prince ait jamais remportée. Avec tout cela, non seulement il ne perdit pas un pouce de terre, augmenta considérablement son domaine, surtout en Italie, mais même on peut dire qu’il aurait conquis tous les pays de ses ennemis, qui ne pouvaient lui résister à force ouverte, s’ils ne lui eussent pas corrompu, non seulement ses princes, ses principaux officiers, mais même jusqu’à son conseil, ce qui seul lui fit perdre la liberté à la bataille de Pavie, le duché de Milan, le royaume de Naples, et même l’empire. Et, bien loin que tant d’ennemis lui fissent retrancher sur ses autres dépenses, jamais prince n’avait été plus magnifique avant lui, soit en achats de meubles précieux, puisqu’il donna d’une seule tapisserie 22 000 écus, revenant à près d’un million d’aujourd’hui, que Charles-Quint son adversaire ne put payer quoiqu’il en eût envie, et que le marchand, comme Flamand, fût son sujet ; soit en constructions de palais superbes. De plus, il rétablit les lettres dans son royaume et même dans l’Europe, ayant fait venir tous les habiles gens en toutes sortes de sciences par de grands frais, et les entretenant de grosses pensions. Comme l’imprimerie ne faisait alors que de commencer, les exemplaires des meilleurs et plus rares auteurs étaient en manuscrits, dont l’ignorance des siècles précédents avait très mal pourvu la France ; c’est ce qui l’obligea à faire encore une dépense effroyable, tant par l’envoi des gens expédiés pour leur recherche dans les contrées les plus reculées du Levant, que pour l’achat de ces mêmes manuscrits, qui coûtèrent souvent des sommes considérables.\par
Deux ans avant sa mort, bien loin que tant de guerres, dans lesquelles il avait bien souvent éprouvé de très mauvais succès, l’eussent épuisé, et mis son royaume à bout, il équipa une flotte de deux cents voiles, aussi bien fournie de monde et d’armements qu’elle pourrait être aujourd’hui en n’y épargnant rien, avec laquelle il ravagea les côtes d’Angleterre, et conquit l’île de Wight, sous le règne de Henri VIII, le prince le plus riche, le plus puissant et le plus accrédité et autorisé que jamais cette île ait vu dominer sur elle, qui fut obligé de battre en retraite, ne lui ayant pu opposer un pareil nombre de voiles. Les armées n’étaient pas à la vérité, à beaucoup près, si nombreuses qu’aujourd’hui ; mais elles ne coûtaient pas moins : chacun des gendarmes, qui étaient en bien plus grand nombre qu’aujourd’hui, recevait assez pour nourrir quatre hommes et quatre chevaux, qui étaient autant d’aides dans les combats, et la paie d’un fantassin revenait à plus de quarante sous d’aujourd’hui. Ne l’était pas qui voulait ; on choisissait, et tous avaient un goujat ou un valet ; cela se voit dans les Mémoires imprimés d’un nommé Boivin, courrier de cabinet, qui a fait imprimer le détail des guerres de Piémont. — Et le roi François I\textsuperscript{er} en mourant, en 1547, loin d’être accablé de dettes, dont il n’avait que très peu, laissa quatre millions d’argent comptant, quelques-uns même disent huit ; mais en s’en tenant au premier, c’est plus de soixante millions par rapport aux prix d’aujourd’hui.\par
Toutes ces magnificences et toutes ces dépenses furent-elles opérées en foulant les peuples, et par le moyen de contraintes, d’exécutions et d’emprisonnements ? — Rien moins que cela ; et pour en convenir, il ne faut que l’écouter parler sur son lit de mort. Voici ses dernières paroles, rapportées par un contemporain à Henri II, son fils et son successeur :\par
« Sache, mon fils, que je te laisse un beau royaume, rempli des meilleurs peuples qui soient sur la terre ; non seulement ils ne m’ont jamais rien refusé, mais même ils ont toujours prévenu mes besoins : mais sache aussi, en même temps, que je ne leur ai rien demandé que de juste, et de ma connaissance, je n’ai jamais fait violence à personne ; car sache, mon fils, que ce ne sera point le grand nombre de troupes, ni les armées formidables qui te feront craindre à tes ennemis, mais seulement l’amour que tes sujets auront pour toi ; outre cet avantage, ce te sera une grande consolation, quand tu auras à comparaître devant Dieu, comme je vais faire dans peu d’heures, de n’avoir rien fait que de juste. »\par
Ce testament était véritable au pied de la lettre, vu les sommes et les manières dont on usait en France, pour tirer sur le pied de trois cents millions d’aujourd’hui. Quelque différence qu’il y ait assurément dans la réussite, il s’en trouve encore mille fois davantage dans le cérémonial du recouvrement d’à présent. — Par le premier, il n’y avait que trois ou quatre sortes d’impôts, et dans le second il y en a plus de dix mille ; et s’il ne s’en trouve pas davantage, c’est parce qu’il ne se rencontre plus personne pour les établir, parce que n’y ayant plus rien à détruire, il n’y a par conséquent plus rien à gagner. Tout passait droit sans embarras de province à autre, et même des deux extrémités du royaume ; et à présent, il y a 3 à 400 pour cent d’impôt à payer pour le passage des choses d’une contrée limitrophe dans la voisine, et même l’on fait périr tout, qui est un tribut que les nations les plus barbares n’ont jamais demandé à leurs plus grands ennemis ; sans parler de la multiplicité de bureaux, qui est un redoublement et triplement de mal. Les corsaires d’Alger et de Maroc, ayant pris un vaisseau chrétien, le rendent au propriétaire pour le tiers de sa valeur, afin de ne pas le ruiner, et de le reprendre une autre fois, s’il est possible ; mais en France, un Traitant ne se soucie guère que tout périsse après lui, pourvu qu’il fasse sa fortune. — Sous François I\textsuperscript{er}, il n’y avait que les peuples qui se mêlaient du recouvrement, et cela sans frais ; et à présent il y a plus de cent mille personnes qui vivent et s’enrichissent dessus, c’est-à-dire aux dépens du roi et des peuples. Et ce qu’ils tirent même pour leur subsistance est dix-neuf fois moins violent que ce qu’ils anéantissent de biens, puisqu’il est constant qu’ils ne lèvent pas plus de huit cents millions, que leur seul ministère a abîmés, et dont plus de cinq cents peuvent ressusciter en un moment, quand on voudra bien ouvrir les yeux sur un pareil ménage ; et afin de ne pas gendarmer les acteurs, on répète encore ce que l’on a déjà dit, que l’on ne congédiera pas un seul des entrepreneurs ordinaires ; on traitera avec eux pour quelques seuls adoucissements, de leur consentement.\par
On va voir, dans le chapitre suivant, par quels degrés cette heureuse situation du règne de François I\textsuperscript{er} a commencé à décliner, et est enfin arrivée à son comble, comme on peut dire qu’elle l’est aujourd’hui : la seule reconnaissance de la cause du mal fera tout le remède par sa cessation, ces deux choses étant inséparables dans un art comme est le gouvernement des peuples, c’est-à-dire que le remède d’un mal n’est jamais que la cessation de sa cause, quoiqu’on ait allégué, pitoyablement, que l’auteur du premier ouvrage sur ce sujet avait trouvé le {\itshape principe} du désordre, mais n’avait pas trouvé le {\itshape remède}, ce qui est une impertinence achevée, puisque l’un ne va jamais sans l’autre, non plus qu’il ne peut y avoir de montagne sans vallée.
\section[{Chapitre VII.}]{Chapitre VII.}
\noindent On est obligé de dire un mot, avant de parler de la première atteinte que reçut l’heureuse situation du règne de François I\textsuperscript{er} et de ses prédécesseurs, de la manière dont la dispensation des revenus du prince se faisait.\par
Chaque année portait nécessairement ses charges, parce que chaque fonds avait sa destination, à laquelle on ne touchait jamais, et la levée était plus ou moins grande, au pied de la lettre, suivant les besoins de l’État. Il n’y avait point de renvoi de la charge d’une année, ce qui a fait depuis une confusion effroyable, parce que, par ces renvois d’année sur autre, tout étant consommé souvent deux ou trois ans avant qu’il soit dû et échu, et survenant des besoins nécessaires et inopinés, il faut avoir recours à des manières ruineuses pour le prince et pour ses peuples, comme des emprunts à gros intérêt, et autres choses encore plus désolantes. — Voilà la première brèche par où les Traitants se donnèrent entrée pour offrir leur malheureux ministère, lequel, comme une pelote de neige, a toujours grossi, jusqu’à ce qu’enfin il soit parvenu à son comble, comme on peut dire qu’il est aujourd’hui. — Ce qui néanmoins ne serait pas arrivé si des personnes puissantes, comme on va dire, ne s’étaient mises de la partie, pour participer au gain effroyable que faisaient de pareils entrepreneurs de la ruine du roi et de ses peuples.\par
M. Fouquet, dans ses défenses imprimées et signifiées au conspect du célèbre tribunal devant qui il avait à répondre, atteste cette vérité, {\itshape qu’il n’y avait jamais de renvoi de charges d’une année à l’autre}, dont la pratique cessée a fait toute la confusion des finances, ayant établi le pouvoir de pêcher en eau trouble, par l’impossibilité où l’on était de découvrir les fraudes et les surprises parmi de si grandes ténèbres.\par
Lors de la prison du roi François I\textsuperscript{er}, les Enfants de France ayant été donnés en otage, pour les retirer il fallut payer leur rançon, estimée à douze cent mille écus d’or, valant quatre millions de ce temps-là, c’est-à-dire plus de cinquante millions d’aujourd’hui. On ne s’avisa point d’avoir recours aux traitants, aux partisans, encore moins à des constitutions de rentes sur le prince, qui est la même chose que si les peuples se constituaient eux-mêmes, puisqu’il leur tombe également en charge de payer le capital et les intérêts, quoiqu’on s’aveugle assez aujourd’hui pour croire le contraire, et que l’on regarde fort indifféremment les dettes que le prince contracte ; en sorte qu’on aime mieux que le monarque constitue sur lui un million de rentes à un denier ou intérêt effroyable, que non pas qu’il demandât un écu à chaque particulier, qui serait bien fâché, toutefois, s’il est sage, de se constituer lui-même pour le paiement des arrérages de ses dettes, ou pour sa dépense ordinaire, puisque cette conduite l’enverrait bientôt à l’aumône. Cependant, que le roi ou lui en usent de la sorte, c’est également la même chose, quoique, encore une fois, qui que ce soit n’y fasse pas la moindre réflexion. — Mais, pour revenir à la rançon des Enfants de France, cette somme effroyable ne se pouvant trouver dans les revenus ordinaires, les peuples ne balancèrent pas un moment à se cotiser à un dixième de tout le revenu. Ce fut chaque lieu, c’est-à-dire chaque ville ou village, qui fit l’imposition, la répartition, la collecte et l’apport en recette, après que la masse avait été partagée par tous les députés des provinces, au niveau des précédents impôts qui en faisaient la règle. — On en usa de même en plusieurs autres rencontres, et ce dixième avait été payé plus d’une fois, ainsi que sous le roi Jean, ce qui est l’usage de toutes les nations du monde, le tout sans ministère étranger, autorité supérieure, ni aucuns frais.\par
Mais il faut enfin venir à la fatale époque où ces heureuses manières prirent fin, pour donner naissance à celles les qui ont réduit la France en l’état où elle est, et non pas tous ses ennemis, dont elle se rira toujours, étant plus puissante à elle seule que toute l’Europe ensemble, lorsqu’elle emploiera toutes ses forces, c’est-à-dire quand elles ne seront pas énervées par des mesures qui lui font plus de dommage que ses plus redoutables adversaires ; ce qui peut être opéré par deux heures de travail, et cela au sentiment de Tacite, qui a dit et publié il y a plus de quinze siècles : {\itshape Galli si non dissenserint, vix vinci possunt} (la France est invincible lorsqu’elle ne se fera point la guerre à elle-même), comme on peut dire qu’elle se fait, depuis 1660, d’une manière effroyable ; et pour en convenir, il n’y a qu’à jeter les yeux sur ses campagnes désolées, ou plutôt sur la perte de la moitié de ses richesses, et il faudra reconnaître que ses plus grands ennemis n’auraient jamais pu lui produire un pareil ravage, ni lui causer tant de dommage par les plus grandes victoires.\par
Pour entrer donc en matière sur la naissance de la cause de la ruine, ce fut sous le règne du roi Henri II, successeur de François I\textsuperscript{er}, que les premiers fondements en furent jetés. — Catherine de Médicis, qu’il avait épousée fort jeune, et n’étant encore que duc d’Orléans, était une princesse qui aimait la magnificence et la très grande profusion, c’est-à-dire qu’elle se plaisait à dépenser plus que ne portaient ses revenus ordinaires ; ainsi il lui fallut avoir recours à des moyens étrangers. Sa beauté, son esprit et sa fécondité la faisant extrêmement considérer par le roi son époux, et lui laisser par conséquent un degré d’autorité nécessaire à changer l’état des choses, ce fut alors que les Italiens qui étaient à sa cour, et dont quelques-uns étaient ses proches parents, lui offrirent leur service pour ce sujet, c’est-à-dire d’avancer de l’argent sur de nouveaux impôts ou créations, traitant à forfait d’une nouvelle affaire, dont ils savaient bien que le roi aurait la moindre partie et eux le reste, qu’ils partageraient avec elle, comme l’on verra dans la suite. — La création des présidiaux, que l’on éclipsa des parlements sans aucun dédommagement, et des lieutenants criminels, dont on ôta les fonctions aux lieutenants civils, se trouve en première date, et voilà la première graine d’une semence qui a tant provigné par la suite. Comme il fallut donner des gages à tous ces nouveaux officiers, et même aux lieutenants civils, pour les dédommager en quelque manière de cette nouvelle érection, ce fut plus de 50 000 écus de rente, desquels le roi se trouva constitué. Il se fit encore beaucoup d’autres nouveautés, trop longues à détailler ; et s’il n’y en eut pas davantage, ce ne fut pas manque de bonne volonté du côté de la reine. Le connétable de Montmorency, qui avait la principale part au conseil, ne lui permettait pas de tailler en plein drap.\par
Après la mort du roi Henri II son mari, ce fut à peu près la même chose ; l’intention ne manqua pas à la reine, mais elle trouva un obstacle dans les princes de Guise, qui avaient grande part au gouvernement, à cause de Marie Stuart, leur nièce, épouse du roi régnant François II ; et ces princes étant d’ailleurs très populaires, et par conséquent très ennemis des nouveautés, quelque grande vocation que Catherine de Médicis eût pour de pareilles affaires, qui lui étaient pareillement inspirées par les Italiens, il fallut qu’elle en prît où elle pouvait, et non pas suivant sa volonté. Mais ayant enfin été délivrée de cette entrave par la mort du roi François II, qui arriva bientôt après, elle n’eut ni repos ni patience qu’elle n’eût renvoyé Marie Stuart, sa veuve, dans son île. Et cela, par une dérogeance à la plus grossière politique, puisque ayant encore trois fils à marier, et ces sortes de dispenses étant aisées à obtenir entre souverains, il était des intérêts de la France de se conserver une reine qui possédait actuellement le royaume d’Écosse, et était héritière présomptive des deux autres monarchies d’Angleterre et d’Irlande, qui était la raison pour laquelle on avait pris tant de peine et fait de très grands armements pour la faire venir dans sa plus grande jeunesse. — On marque cette chasse pour montrer ce que l’on doit attendre du zèle pour l’intérêt public, lorsqu’il se trouve en compromis avec l’utilité particulière et personnelle, comme le cas est arrivé une infinité de fois depuis ce temps : il n’est pas étonnant que cette dernière ait toujours eu la préférence, puisqu’une reine et une mère y succomba dans une occasion si importante, et que l’envie de gouverner et de dépenser l’emporta sur l’établissement de ses enfants, contre la gloire et l’agrandissement d’un royaume dont elle avait l’honneur de porter la couronne, bien qu’à ne consulter que les apparences, elle n’eût jamais dû espérer un si haut degré de grandeur ; ce qui devait l’exciter à en marquer encore plus de reconnaissance. Comme ce sacrifice, encore une fois, du bien public à l’intérêt particulier est la principale et peut-être l’unique cause de la ruine de la France, on s’est étendu sur cet article, afin que l’on ne s’étonne point si l’on s’est laissé aller tant de fois à une pareille faiblesse, puisqu’une personne qui semblait avoir par devers elle un bien plus violent préservatif pour l’empêcher d’y tomber, ne laissa pas d’y être prise dans une si importante occasion, et voilà la clef de la diminution ou de la perte des biens de la France. Toutes les couronnes du monde sur la tête d’un des fils de Catherine de Médicis ne l’eussent pas dédommagée de la privation d’une partie du gouvernement que MM. de Guise se seraient retenue au moyen de leur nièce, comme par le passé ; il la fallut renvoyer au plus tôt ; après quoi la régence lui fut donc accordée sous le règne du roi Charles IX.\par
Ce fut à ce coup que cette reine, se trouvant en quelque manière émancipée, donna pleine carrière à ses profusions, et par conséquent à des affaires nouvelles, par le moyen de MM. les Italiens. — Les États généraux qui se tinrent dans ce temps, comme c’était la coutume, firent assurément leur devoir : les députés de tous les ordres furent chargés, par toutes les provinces, de représenter que les Traitants et les Partisans étaient des voleurs publics qui ruinaient le roi et les peuples. Comme ces assemblées n’étaient ordinairement convoquées que pour avoir des secours extraordinaires, tous les députés unanimement marquaient qu’il n’y avait point de moyen plus court et plus certain de recouvrer de l’argent, que de reprendre le bien des Italiens et de leurs consorts, ceux-ci l’ayant volé au prince et au royaume, et de les renvoyer aussi gueux dans leurs pays qu’ils en étaient venus, n’ayant tous rien vaillant, de notoriété publique, à leur arrivée. Un auditeur des comptes, qui fut entendu dans les États, fit voir que, de chaque écu que le roi recevait par un pareil canal, il n’y en allait que quatorze sous à son profit. Comme tout ceci se trouve imprimé et peut être vu de tout le monde, on n’avance rien que de très certain, ni qui puisse être soupçonné de calomnie ou de discours séditieux.\par
Mais, pour revenir à Catherine de Médicis, toutes ces remontrances n’opérèrent rien ; elle continua son même genre de vie, et même après que le roi Charles IX fut déclaré majeur, elle se retint par son adresse la principale part au gouvernement ; pour à quoi parvenir, les historiens l’accusent d’avoir fomenté les dissensions du royaume, ou plutôt les guerres civiles, afin de se rendre nécessaire, mettant un jeune monarque hors de pouvoir démêler, par son peu d’expérience, de pareilles difficultés. Ce qui est un surcroît de preuves de ce que peut l’intérêt particulier sur celui du public ; et comme l’occasion s’est souvent présentée, et que ce dernier a toujours eu le dessous, on ne doit pas s’étonner de la ruine de la France, ni que l’on en mette la principale cause sur ce compte.\par
Le roi Charles IX étant mort en 1574, Henri quitta la Pologne pour venir prendre la couronne. Par malheur il se rencontra pour la dépense, et même la plus superflue, d’un semblable caractère que la reine Catherine de Médicis, si même il ne la surpassa pas, puisqu’aux seules noces du duc de Joyeuse il en coûta douze cent mille écus, qui reviennent à plus de dix millions d’aujourd’hui. Comme cette disposition se trouva jointe avec une bien plus grande autorité que celle d’une régence, et que les mêmes Italiens subsistaient, pour lui fournir les mêmes moyens d’y donner cours comme par le passé, on peut dire qu’alors les choses furent poussées dans l’excès.\par
Et cela alla à un si haut degré, que les pourvoyeurs de sa maison, n’étant point du tout payés, refusèrent absolument de rien fournir davantage ; en sorte qu’elle eût été tout à fait sans ordinaire, si le tiers État ne s’était obligé à payer personnellement les intéressés. Et ce fut toujours la même confusion et le même désordre jusqu’à sa mort.\par
Le roi Henri IV étant venu à la couronne, comme il s’y introduisait de la manière qu’il pouvait, ainsi qu’il déclarait souvent lui-même, c’est-à-dire avec mille peines et mille embarras, le royaume étant plutôt une conquête à son égard qu’une succession, il n’était point du tout en état de réformer, ni de trouver à redire, dans tout ce que ceux qui étaient chargés du soin des finances faisaient, quoique très défectueux et très rempli de prévarication. Mais en 1594, ne sachant plus où donner de la tête seulement pour vivre, et étant obligé d’aller manger chez le tiers et le quart, comme on voit par les lettres, imprimées, qu’il écrivait à M. de Sully, ce même M. de Sully, lors âgé de trente-huit ans, et ayant passé toute sa vie à la guerre, et non dans les finances, ne balança point à prendre son parti. — Il fit remarquer à ce monarque que c’étaient les Traitants et les Partisans qui le réduisaient en ce pitoyable état, sur quoi le roi lui ayant reparti, par quelle raison donc le surintendant et son conseil les souffraient et admettaient-ils ? M. de Sully lui dit que c’était parce que le même surintendant et tout son Conseil étaient de moitié avec tous ceux qui le désolaient ainsi que ses peuples. Et pour lui justifier une si violente accusation, il lui fit voir un catalogue de tous les intéressés dans les fermes générales, où le surintendant d’O, les intendants des finances et les conseillers d’État étaient à la tête, ainsi que dans les autres affaires particulières, les unes et les autres s’adjugeant également devant eux, ce qui les rendait juges et parties. Le grand-duc de Toscane, parent de Catherine de Médicis, avait trouvé le métier si bon, qu’il s’était mis de la partie, ce qui est une certitude que la reine y avait eu sa part. Le duc de Sully ajouta qu’il y avait un moyen de l’enrichir, savoir que tous les tributs passassent droit des mains des peuples en celles du prince. Le roi ayant fait voir ce projet à son Conseil, tous lui repartirent que c’étaient des fous qui lui inspiraient de pareilles manières. À quoi il repartit sur-le-champ qu’eux qui étaient très sages l’ayant ruiné, il voulait voir si les fous ne l’enrichiraient pas, ce qui ne manqua pas d’arriver, et lui de le publier par la suite ; savoir, que {\itshape les sages l’avaient appauvri}, et {\itshape les fous rendu opulent}.\par
En effet, ayant chargé M. de Sully du soin de ses finances, quoique très inexpérimenté dans cette science, à parler le langage d’aujourd’hui, son ignorance fut si heureuse, qu’en dix ans il paya 200 millions de dettes sur trente-cinq millions de revenu qu’avait seulement le roi alors, et en amassa trente, d’argent fait, sur ces trente-cinq millions de revenu, qui furent déposés dans la Bastille, et s’y trouvèrent à la mort de Henri IV.\par
Mais les Italiens ou les habiles financiers étant remontés sur le théâtre à l’aide de Marie de Médicis, déclarée régente sous la minorité du roi Louis XIII, et à peu près du même caractère que Catherine pour la dépense, les trente millions furent consommés, sans qu’il y eût aucune guerre étrangère ni autres occasions extraordinaires ; au lieu qu’ils avaient été amassés, par M. de Sully, en partie pendant qu’on avait la guerre avec l’Espagne, qui s’empara, comme l’on sait, tant par surprise qu’autrement, de plusieurs places considérables presque aux portes de Paris, sans qu’on alléguât, lors de son entrée dans le ministère par des manières nouvelles, la pitoyable raison qu’on apporte aujourd’hui, que la guerre n’est pas propre à aucun changement, l’administration du dedans du royaume n’ayant absolument rien de commun, non plus que celle de la justice, avec ce que les armées font au-dehors. Et, comme il serait ridicule de dire que l’on ne peut pas faire gagner la cause à un homme qui a l’équité de son côté, par la raison de la guerre qui est en Italie et en Espagne, il est de la même absurdité de se dispenser par cette raison de partager justement les tributs tant sur les personnes que sur les denrées, dont le dérangement coûte au royaume vingt fois plus que le roi n’en tire, et par conséquent beaucoup davantage qu’il ne faudrait pour faire finir la même guerre. Ainsi, ces objections sont le contraire de ce que la raison la plus grossière devrait dicter ; mais il en va de ces allégations comme dans tous les mauvais procès, celui qui a tort n’a d’autre ressource que chicaner pour reculer le jugement. On a fait cette digression parce que de pareilles objections sont aujourd’hui le cheval de bataille ordinaire dont on combat le rétablissement de la France, en se retranchant sur le délai pour arrêter des manières qui font horreur au ciel et à la terre, pendant qu’absolument il ne faut que deux heures, M. de Sully n’en ayant pas employé davantage, pour établir son projet au milieu de la guerre.\par
Mais, pour revenir à la chronique du ministère de Marie de Médicis, les Italiens ayant replongé le royaume en l’état d’où M. de Sully l’avait tiré, il leur fut ôté de la façon que tout le monde sait, c’est-à-dire un peu violente, quoique très juste au fond. Le cardinal de Richelieu vint peu de temps après sur les rangs ; et, sans entrer dans le détail de son ministère, on dira seulement que tous les revenus du royaume doublèrent de son temps, ainsi que ceux du roi, auquel n’ayant trouvé que trente-cinq millions de rente, il en laissa soixante et dix à sa mort.\par
Les Italiens revinrent à la charge, et recommencèrent leurs manières sous une régence, par de pareilles pratiques que sous Marie et Catherine de Médicis. Ils y trouvèrent des oppositions sans nombre, et toutes constamment pour le service du roi durant sa minorité. Il ne faut point dire, quoiqu’on ait donné un autre jour et une autre interprétation à ce qui se passa alors, que c’était par un esprit de rébellion ; puisque outre le témoignage du roi François I\textsuperscript{er}, qui marque qu’{\itshape il n’y eut jamais de peuple plus soumis} ; de celui de Guichardin, historien italien, qui raconte, en parlant de la bataille de Fornoue, où la personne du roi Charles VIII se trouva en péril, que toutes les troupes se rassemblèrent aussitôt autour de lui, « parce que, dit-il, cette nation aime son roi jusqu’à l’adoration » ; outre, dis-je, ces preuves authentiques, on ne pouvait pas accuser les contemporains de vouloir fermer leur bourse au souverain, puisqu’ils avaient vu tranquillement tripler les Tailles en moins de trente ans, parce que c’étaient des sommes qui passaient droit des mains des peuples en celles du prince. C’était aux Traitants et aux Partisans à qui ils en voulaient, qui ruinaient tout pour leur profit particulier, étant appuyés des ministres avec qui ils partageaient.\par
Ce sont les propres termes de la harangue de M. Amelot, premier président de la cour des aides de Paris, concertée avec toutes les compagnies, ou plutôt avec tous les peuples. Comme elle se trouve imprimée dans les recueils de ce temps-là, et qu’il y a peu de bibliothèques qui n’aient donné place à ces sortes de livres, l’on ne se fera aucun scrupule de la rapporter, quelque forte qu’elle soit, ne faisant que citer ce qui est déjà public ; d’autant plus que l’on croirait trahir les intérêts de la cause que l’on défend, si on omettait la moindre des raisons qui viennent à l’appui.\par
Il dit donc, en parlant à la reine régente, « que les affaires extraordinaires et les Partisans n’avaient été inventés et mis en pratique que pour ruiner le roi et les peuples, et former des profits indirects aux ministres, parce qu’ils ne pouvaient rien prendre sur les tributs réglés, sans qu’on s’en aperçût ; qu’il ne fallait point néanmoins employer d’autre moyen dans les nécessités de l’État, et imposer sur les peuples tous les besoins du roi dans les occasions, et puis les ôter quand elles étaient passées ».\par
En un mot, il fit voir par les termes de sa harangue, que les Partisans étant constamment la cause de la ruine du commerce et du labourage, qui est un mérite que personne ne leur contestera jamais, et dont ceux qui sont sincères parmi eux ne disconviennent pas, il était certain que le champ et la vigne des ministres de ce temps-là étaient la destruction des champs et des vignes. Quoique le mal ait toujours augmenté depuis, en sorte qu’on peut dire sans contredit qu’il est enfin arrivé à son comble, comme il n’y a eu que de la surprise de la part de MM. les ministres qui sont venus depuis 1660, ces faits très certains, bien loin de les offenser, leur feront un sensible plaisir, en leur faisant quitter une route qu’ils croient très innocente, et par conséquent avantageuse au roi ; et cela, sur la foi d’auteurs qu’ils pensaient remplis d’intégrité, bien que ce fût justement le contraire.\par
Mais pour vérifier, ou plutôt fortifier, la harangue de M. Amelot, ce qui se passa à la chambre de justice, au conspect de toute la France, et pour ainsi dire contradictoirement avec les parties intéressées, montre qu’il n’en dit pas encore assez. Un des chefs d’accusation contre ce ministre était qu’il avait pris part dans les affaires du roi, soit par des pensions des fermiers généraux et particuliers, soit par des parts qu’il se retenait dans les partis, l’un et l’autre étant un crime, suivant les lois de toutes les nations du monde. — Mais quand il vit qu’on le prenait sur ce ton-là, bien loin de demeurer muet, non seulement on ne l’en put convaincre bien clairement, mais même rétorquant en quelque manière l’argument contre les parties, à proprement parler il fit voir que le ministre, dont il n’était en quelque sorte que le commis, avait eu part dans toutes les affaires extraordinaires qui s’étaient faites de son temps ; qu’il avait une pension de 40 000 écus sur les fermes générales, et que dans toutes les affaires particulières, qui que ce soit ne lui en avait jamais proposé aucune que l’argent à la main ou par avance, ou dans la suite : il en nomme quantité de cette sorte, et même quelques-unes dont ce ministre s’était fait seul Traitant, La perfection est que l’accusateur ou plutôt l’accusé déclare qu’il n’en disait qu’une partie, et que l’on n’eût pas à réchauffer davantage, autrement qu’il dirait bien d’autres choses, ou plutôt ferait l’histoire de la vie du cardinal Mazarin, ce qui ne lui causerait pas beaucoup d’honneur, quoique ses parties en voulussent faire un saint en matière d’intégrité. Tout ceci se signifiait et s’imprimait publiquement aux yeux de tout le royaume, et demeura néanmoins sans repartie ; ce qui s’appelle un acquiescement en justice, puisque cela se passait devant un tribunal où étaient actuellement les parties en procès pour cette seule question. Les vingt millions que ce ministre avait laissés pour porter son nom, ne furent point battants pour obliger à en défendre l’honneur, comme cela n’eût pas manqué s’il ne s’était pas agi de combattre une vérité connue de tout le monde.\par
Ce n’est pas tout, M. Fouquet maintient, que sous tel maître tels disciples ; qu’ainsi toutes les personnes considérables, tant de la cour et du Conseil, qu’employées dans l’administration des finances, menaient le même genre de vie ; et pour ne laisser aucun doute, il les nomme toutes l’une après l’autre, ainsi que les sortes d’{\itshape Affaires} où elles avaient pris part. On s’abstient de les déclarer plus précisément, pour des considérations ; mais ceux qui seront curieux de le savoir l’apprendront facilement par la lecture du procès de M. Fouquet, dont il y a peut-être plus de deux mille exemplaires imprimés en France, et qui se vendent publiquement chez les libraires ; en sorte qu’il n’y a point de reprise à faire contre l’auteur de ces Mémoires, puisqu’il n’apprend rien, mais ne fait que citer ce qui est connu de tout le monde. — Et on aurait d’ailleurs grand tort de se formaliser, après la mort de ces messieurs, de ce discours, puisque eux, de leur vivant, qui voyaient et entendaient tout, et même à quelques-uns desquels on le signifiait {\itshape en forme}, n’en firent aucune reprise, ayant toujours conservé la même tranquillité ou prudence qui avait paru dans les héritiers du maître, sur de semblables allégations. — Enfin, M. Fouquet termine son catalogue ou son plaidoyer par déclarer qu’il n’y avait rien de {\itshape nouveau} en tout cela ; que tous les ministres et toutes les personnes employées dans l’administration eu avaient toujours usé de la sorte ; que les rois mêmes le trouvaient bon, sous prétexte que cela leur fournissait les moyens de soutenir la dignité de leurs emplois.\par
Voilà les fondateurs de la préférence donnée aux {\itshape Affaires extraordinaires} et aux Partisans, sur les tributs réglés passant droit des mains des peuples en celles du prince, comme la France avait été régie durant onze cents ans, et comme le sont tous les États du monde, tant anciens que nouveaux. La certitude de ce changement coûte la perte de la moitié des biens du royaume en pur anéantissement, n’y ayant point de traité qui n’abîme vingt fois autant de denrées qu’il fait passer de profit dans les coffres du prince ; cette certitude, dis-je, ou plutôt la cause du souverain et des peuples, qui ne sont point deux choses séparées, était dans de mauvais termes, d’avoir à défendre leurs intérêts devant des gens qui étaient juges et parties, contre toutes les règles de la justice et de la raison. — Et le prétendu zèle pour le bien de l’État, que l’on voudrait supposer avoir été assez grand dans leur personne pour préférer le bien général à leur utilité particulière, lorsqu’ils se trouvaient en compromis devant eux, et qu’il s’agissait de donner leur jugement, ne peut être pensé ni allégué raisonnablement, après Catherine de Médicis, qui succomba à la tentation, comme on l’a dit, dans une occasion bien plus importante, quoiqu’elle eut de bien plus forts intérêts, et personnels et publics, de n’avoir pas cette faiblesse. — Outre que ce qui s’est passé en plusieurs autres rencontres, ne montre que trop lequel des deux, en pareils procès, a toujours perdu sa cause.\par
Mais enfin, quelque forte vocation qu’eussent ces messieurs de faire leurs affaires aux dépens du roi et des peuples, il s’en fallait beaucoup qu’ils taillassent en plein drap ; la volonté y était toujours tout entière, mais le pouvoir souvent y manquait. — Les parlements et les compagnies s’étaient conservé l’autorité de faire des remontrances lors des établissements qui, ayant pour principes ceux qu’on vient de marquer, eussent fait un trop notable préjudice au roi et aux peuples. Voilà le palladium ou Dieu tutélaire qui avait conservé la France depuis la suppression des États généraux qui avaient cette fonction auparavant, et qui s’en étaient si bien acquittés, que jamais monarchie, depuis la création du monde, n’a été de si longue durée ni si florissante, ayant fourni au monarque, dans ses besoins, trois fois plus que les manières opposées, savoir les Partisans, n’ont jamais fait dans les nécessités les plus urgentes, comme peut être celle d’aujourd’hui. Il ne faut que le règne de François I\textsuperscript{er} pour fermer la bouche aux contredisants et à leurs protecteurs. Ces États avaient si bien fait, et les compagnies supérieures après eux, qu’ils avaient fait doubler tous les trente à quarante ans les biens du royaume, ainsi que ceux du roi, et cela jusqu’en 1660, malgré les traverses qui leur étaient données par ceux dont on vient de faire l’histoire, et qui commencèrent il y a déjà plus d’un siècle à faire supprimer les États généraux. — Outre les raisons que ce détail fait assez présumer pour en user de la sorte, on n’a qu’à jeter les yeux sur les harangues prononcées publiquement, au conspect du roi et de tout le royaume, pour voir comme les Traitants et leurs fauteurs sont accommodés, pour convenir par quel intérêt ces assemblées conservatrices du royaume ont été anéanties.\par
Mais enfin les compagnies supérieures y avaient suppléé, et avaient produit à peu près la même utilité, en sorte que la France se trouvait, en 1660, en l’état le plus florissant qu’elle se fût jamais vue : le même sort qu’on leur a fait subir à l’égard du droit de remontrance, en a fondé la décadence, que l’on peut dire aujourd’hui être arrivée à sa perfection du côté des facultés des peuples seulement, mais non de leur zèle, ni même du pouvoir naturel du commerce et de la culture des terres, puisque la plus grande partie peut être rétablie en deux ou trois heures, par la simple cessation de la plus grande violence que la nature ait jamais soufferte depuis la création du monde ; et cette proposition est faite de la part des peuples mêmes, aux conditions déjà tant de fois marquées, que si toute objection que l’on pourra faire, soit pour le temps, soit pour le péril, n’est pas une preuve et une montre évidente d’une extravagance ou d’une prévarication achevée, l’avocat consent d’être lui-même traité comme un insensé ; et c’est ce qu’on verra dans la suite invinciblement, ainsi que l’impossibilité de sortir autrement de la conjoncture présente, après qu’on aura dit un mot de cette suppression de remontrance, et des circonstances qui ont réduit la France, depuis 1660, au malheureux état où elle se trouve, de ne pouvoir plus fournir les besoins du roi, quoique beaucoup au-dessous de ce qu’elle avait contribué autrefois, et de ce qu’elle peut faire, encore une fois, par deux heures d’attention seulement.
\section[{Chapitre VIII.}]{Chapitre VIII.}
\noindent Voici, en 1660 ou 1661, l’assemblage des deux plus grands contradictoires unis ensemble qui se rencontrèrent jamais, savoir une très grande intégrité dans le ministre, et un très grand désordre dans l’administration.\par
On vit les tributs réglés comme les Tailles, passant droit des mains des peuples en celles du prince, très négligés, ce qui avait déjà été commencé sous le ministère précédent ; et les {\itshape affaires extraordinaires}, ou plutôt les traités et les partis portés au comble de leurs vœux : cette négligence des Tailles de dessein prémédité, afin que le désordre les rendant insuffisantes à atteindre aux besoins de l’État, cela donnât lieu aux affaires extraordinaires, par pure surprise du ministre, qui était très intègre. — Aucune denrée ne devint exempte ; nul lieu, nul passage ne se put plus rencontrer sur une route, qu’il ne fallût donner des déclarations et payer des redevances qui n’étaient que le résultat des pratiques usitées par des commis pour tout faire consommer en frais encore trois fois plus ruineux que les sommes mêmes. — Ce n’est pas tout, on vit plusieurs Traitants d’impôts sur une même denrée, principalement les liqueurs, dans un même lieu et pour un même prince, ce qui semblait devoir porter sa réprobation avec soi, puisqu’ils avaient leur fortune, telle qu’on l’a vue, à prendre par préciput, ainsi que les frais de bureaux et de commis ; et ceux-ci, chacun, les embarras et les séjours des voituriers à employer à leur profit, les ayant érigés en revenus par l’exigence de contributions particulières pour échapper à de pareils inconvénients ; outre que ces préciputs, dis-je, étaient autant d’enlèvements ou larcins qu’on faisait au roi, tout ce qui se lève sur les peuples et ne passe point directement entre ses mains ne pouvant être appelé autrement.\par
Mais c’est là le moindre désordre de pareilles manières, parce qu’au moins, si cela n’avait point eu d’autre mal, il n’y aurait rien eu d’anéanti, et la seule justice se serait trouvée blessée ; mais les suites d’une pareille conduite sont et ont été quelque chose de bien plus effroyable. — Comme la richesse consiste dans un échange continuel de ce que l’un a de trop avec un autre, pour prendre en contre-échange les choses dont celui avec qui il traite abonde ; du moment que cette facilité manque, ou plutôt ce commerce, un pays devient aussitôt misérable au milieu de l’abondance. — Or, il faut que cette heureuse situation s’arrête du moment que les proportions en sont ôtées, et qu’un commerçant, sans qu’il importe lequel des deux, ne pourrait faire l’échange ou le troc qu’à perte, par rapport aux frais qu’il a fallu faire pour établir le produit qu’il a dessein de vendre ; auquel cas voilà aussitôt le marché rompu, ce qui désole également l’une et l’autre partie, et a incontinent après une suite effroyable de misère, parce que l’opulence d’un État, surtout de la France, consistant dans le maintien de toutes les professions, au nombre aujourd’hui de plus de deux cents, leur existence est réciproquement solidaire, se donnant à tous moments et recevant pareillement la vie les unes des autres.\par
Ce sont les fruits de la terre, et en premier lieu les grains et les liqueurs, qui commencent le mouvement, et qui passant par le canal des maîtres et propriétaires aux mains des ouvriers, font que ceux-ci donnent en contre-échange le fruit de leur travail, toujours aux conditions marquées de proportion qui permettent à tous de trouver leur compte, sans quoi le moindre déconcertement devient aussitôt contagieux et corrompt toute la masse. C’est la crainte d’un pareil désordre qui fait jeter aux Hollandais le poivre dans la mer, et qui fait donner aux Anglais de l’argent, aux dépens du public, à ceux qui viennent du dehors enlever les blés dans l’abondance. Et c’est néanmoins le contraire, par une surprise effroyable, que l’on bâtit et fomente tous les jours en France par toutes sortes d’efforts, depuis 1660, qui est uniquement la cause des 1 500 millions de perte arrivée au royaume depuis ce temps. — Les blés ont éprouvé et éprouvent à chaque moment ce sort : mais comme il n’en est pas question présentement, et que l’on en a déjà parlé, comme l’on en fera encore mention lorsqu’il s’agira du rétablissement possible en deux heures, on vient aux liqueurs, qui sont la seconde manne primitive du royaume, tant pour la subsistance des peuples que pour leur former du revenu ; l’excédent de la consommation personnelle dans les propriétaires leur fournissant le moyen de se procurer le surplus de leurs besoins, comme pareillement aux ouvriers de ces mêmes besoins, le canal pour se pourvoir de liqueurs. — Or, ce qui s’est fait depuis 1600 a condamné les deux tiers des peuples à ne boire que de l’eau, parce que la plupart des propriétaires des vignes ont été obligés de les arracher, et réduits par là à la dernière misère.\par
Voici comme la chose est arrivée. Ces liqueurs, tant vins, cidres et eaux-de-vie, qui passaient avec profit réciproque des mains des maîtres en celles des ouvriers et acheteurs, furent obligés tout à coup de recevoir une hausse effroyable de prix pour porter le profit des Traitants, ainsi que ce qu’on donnait au roi, qu’on a toujours augmenté presque à tous les baux ; les frais des bureaux et commis, les séjours ruineux que les voitures étaient obligées d’endurer pour acquitter ces droits en divers lieux, ou bien pour racheter ce même séjour : tout cela devant être porté par la marchandise, la mit à un taux exorbitant, et ceux qui en faisaient leur provision auparavant n’y pouvant plus atteindre par le fruit de leur travail, ce fut une nécessité ou de s’en passer, ou de l’avoir du marchand à une perte considérable de sa part, ce qui est toujours égal pour l’un et pour l’autre par les raisons marquées, et par conséquent la ruine d’un État, ce qu’on ne peut nier être aujourd’hui la situation de la France, non plus que ce ne soit de pareilles causes qu’elle est provenue. — Enfin, les choses vinrent dans un si grand excès en 1677, qui fut une année très abondante, que les vignerons ou marchands ayant voituré des vins par une rivière en une foire d’une ville considérable, et la quantité excédant la consommation (quoique dans les temps précédents elle eût été six fois plus forte avec profit), il arriva que ces marchands, qui ne trouvaient pas à beaucoup près le prix de l’impôt qu’il avait fallu garantir et promettre par avance, voulurent quitter aux Traitants leur denrée en pure perte, ne demandant qu’à s’en retourner libres de toute obligation ; mais ceux-ci déclarèrent que ce marché leur serait trop préjudiciable, et que tout ce qu’ils pouvaient faire de plus favorable était que les bateaux répondissent pareillement du droit, et qu’ils s’abtiendraient d’exercer leur contrainte sur les personnes.\par
Il ne faut pas consulter l’oracle pour convenir que c’est à de pareilles manières que la France est redevable de sa ruine ; mais afin qu’on ne révoque point de pareils faits en doute, qui sont néanmoins très constants, ce qui se passe tous les jours en France dans plusieurs de ses provinces est d’un pareil degré d’horreur, bien que, par la plus grande des surprises ce soit l’autorité du roi et de MM. les ministres qui soit employée nuit et jour à maintenir un pareil état de choses.\par
L’on saura que toutes les denrées du Japon et de la Chine, étant arrivées en France, n’augmentent du prix qu’elles ont coûté sur le lieu, que de trois parts sur une, ne faisant que quadrupler, et même souvent moins. Les droits des princes d’où elles sortent, et qui n’ont point d’autres revenus que ces douanes, trois à quatre mille lieues de trajet, les tempêtes et les pirates, ne coûtent que cette somme à conjurer.\par
Mais les liqueurs qui passent en France d’une province à l’autre, quoique souvent limitrophes, augmentent de dix-neuf parts sur vingt, et même davantage. Les vins que l’on donne dans l’Anjou et l’Orléanais souvent à un sou la mesure et même moins, c’est-à-dire avec perte du vigneron, se vendent 20 et 24 sous dans la Picardie et la Normandie, et il n’y a pas encore trop à gagner pour les marchands ; c’est-à-dire que les commis et Traitants qui empêchent ce trajet sont six fois plus formidables et plus destructeurs du commerce que ne sont les pirates, les tempêtes, et trois à quatre mille lieues de route ; en sorte que les liqueurs croissant aux portes de ceux qui ne boivent que de l’eau, ils sont obligés d’être dans cette misère, ou d’acheter ces liqueurs six fois plus que si elles venaient de la Chine et du Japon ; ce qui ruine également les marchands et les acheteurs par les raisons marquées, et par conséquent le roi.\par
Comme le premier mobile de tout ce beau ménage, ce sont ceux qu’on appelle les {\itshape fermiers du roi}, on peut apercevoir par tout ce narré, qui ne fait mention que d’une partie du désordre, dont on peut voir le surplus dans le livre qui porte pour titre {\itshape le Détail de la France}, ou plutôt par ce qui est public aux yeux de tout le monde ; on peut voir, dis-je, comme ce nom de fermiers du prince convient peu à ces messieurs, puisque le devoir et la fonction d’un homme qui tient une recette, étant de cultiver et de faire valoir le fonds le plus qu’il est possible, eux, au contraire, ont cru ne pouvoir mieux faire le profit du maître qu’en détruisant tout, et causant plus de ravages que des armées ennemies qui auraient entrepris de tout désoler. Car ces excès ou ces fléaux de Dieu n’ont jamais qu’une courte durée, après quoi un pays saccagé se remet incontinent, souvent mieux qu’auparavant, ainsi que l’on a déjà dit plusieurs fois. — Mais il n’en va pas de même de ceux-ci ; après que dans un bail le plus apparent ou le plus grossier a été détruit, les successeurs n’y peuvent faire leur compte que par un rehaussement de droits qui, diminuant encore la consommation, augmente par conséquent la ruine et des peuples et du roi, qui n’a d’autre bien que les fonds de ses sujets, lesquels ne le peuvent payer qu’à proportion des fruits qui croissent dessus, et qui peuvent être consommés, sans quoi ils demeurent en perte, et font abandonner la terre, comme il n’est que trop connu. Et pour un si important service, ces messieurs font des fortunes de prince ; et, pour anéantir cent fois plus de biens qu’ils n’en font passer aux coffres du prince, ils méritent d’avoir mille fois plus de facultés qu’ils ne possédaient en se mettant en besogne. — Voilà pour les Aides que l’on sait jouer un si grand rôle dans la ruine de la France, et dont la cessation, sans nuls risques et périls, aura une si grande part dans le rétablissement des 500 millions de biens aux peuples, sans qu’il soit besoin de plus d’une demi-heure d’attention, comme on fera voir dans la suite.\par
On vient aux Douanes, Droits de passage et sortie du royaume, sur lesquels on peut dire d’abord que c’est à peu près le même cérémonial, même désolation et même extravagance, par erreur au fait dans MM. les ministres, qu’à l’égard des Aides. — Il est à remarquer encore que celles qui se paient dans le milieu du royaume de provinces à autres, comme réputées étrangères, sont indignes, et font honte à la raison.\par
Elles avaient été établies lorsque ces contrées appartenaient à des princes autres que les rois de France ; mais, ces provinces appartenant maintenant à la couronne, et n’y ayant aucune de ces Douanes qui ne cause des vexations effroyables par les séjours ruineux des voituriers, et qui ne désole par conséquent le commerce et la consommation, elles devraient être ôtées, et le produit tout au plus remis avec les autres tributs, comme la Taille ; ce qui fait étant, comme cela est possible, le pays y gagnerait cent pour un, dont le roi aura amplement sa part, c’est-à-dire trois fois plus qu’il ne reçoit.\par
La Douane de Valence doit sa naissance à un crime que le malheur des temps fit tolérer, et que par conséquent le rétablissement de l’ordre devait abolir. Lors des guerres civiles de la religion, le connétable de Lesdiguières s’étant rendu chef du parti des huguenots dans cette contrée, établit cet impôt par la force majeure sans aucune autorité de prince, pour l’entretien de ses troupes ; et après que les choses furent pacifiées, des intérêts personnels, contraires à ceux de l’État, l’ont maintenu jusqu’à présent. Ce sont ces mêmes abus qui ont fait maintenir les autres douanes, et augmenter tous les jours, par conséquent, la ruine du royaume ; ce qui a été si loin pour les droits de sortie, quoiqu’on sache que la richesse d’un État consiste dans les envois au-dehors, qu’il s’en trouve jusqu’à 26 dans un seul port de mer, c’est-à-dire vingt-six droits ou déclarations à passer à diverses personnes ou différents bureaux, avant qu’un seul vaisseau puisse décharger ou mettre à la voile, et emporter ou débarquer les marchandises chargées.\par
Il n’y a pas un de ces receveurs de droits ou déclarations qui ne veuille faire sa fortune : ils savent bien tous que ce ne peut être par le moyen de leurs gages, qui sont souvent très médiocres ; ce n’est donc que par les vexations telles et semblables que l’on a marquées à l’article des Aides. Ce qui va si loin, qu’un célèbre négociant, pour être quitte d’un {\itshape coup de chapeau} que doit le vendeur de certaines denrées avant de les livrer, par une ancienne ordonnance, fondée on ne sait sur quoi ; pour être quitte, dis-je, de cette servitude, ou plutôt de ces accompagnements qu’on avait soin de cultiver comme le reste, donnait 1 500 livres par an en pure perte, qui n’allaient point assurément au profit du roi, non pas même de ses prétendus fermiers ; encore voulait-on lui persuader qu’on lui faisait grâce. Ainsi, on peut juger du reste par cet échantillon. C’est par de pareilles manières, dont ceci n’est que la moindre partie, que les étrangers, lesquels, de compte fait, avant 1660, prenaient une fois plus de marchandises du royaume qu’ils n’en apportaient, en ont depuis ce temps introduit deux fois plus qu’ils n’en ont enlevé, c’est-à-dire que la France est devenue redevable, de créancière qu’elle était.\par
Mais comme les peuples qui voyaient qu’on les minait peu à peu, et qu’ils étaient comme brûlés à petit feu, ne marquaient pas une entière complaisance pour des manières qui les désolaient, et qu’ils faisaient agir les compagnies supérieures par des remontrances sur de pareils établissements, en faisant voir qu’ils portaient un très grand préjudice au roi, et n’étaient utiles qu’aux entrepreneurs ; quelque intègre et quelque éclairé que fût le ministre, il crut que c’était une atteinte à l’autorité du roi, et une dérogeance au respect dû par des sujets à leur souverain. Il fit abroger les remontrances par l’ordonnance de 1667, qui établissait que tout édit qui serait présenté serait accepté et exécuté par provision, sauf à en remontrer après la surprise ; ce qui était tout à fait inutile, parce que chaque nouveauté se fortifiant de patrons, personne ne s’en voulait rendre ennemi, outre que les longueurs, pendant que le mal faisait son cours, rendaient vaines toutes les poursuites. Cette même ordonnance fut encore renouvelée en 1673. Voilà la fondation et le couronnement des 1 500 millions de rente perdus dans le royaume depuis environ quarante ans. Et la ruine de la France, qui avait été tentée inutilement pendant plus d’un siècle et demi, comme on l’a fait voir, ne put avoir sa perfection qu’en y employant l’autorité du roi tout entière, sans quoi on n’en fût jamais venu à bout.\par
En effet, si après l’établissement d’un premier droit sur l’entrée des boissons et liqueurs dans une ville de grande consommation, sur la présentation d’un second par un nouveau Traitant, avec nouveau bureau et nouveaux commis, on avait, avant d’en souffrir l’introduction, remontré que cela était contraire aux intérêts du roi, parce qu’outre que ces nouveaux frais n’allaient point à son profit, c’était un surcroît d’empêchement à la consommation, qui était détruite par ces manières, sans nulle utilité à personne ; et que si Sa Majesté voulait hausser la levée, il fallait qu’il n’y eût qu’un enchérisseur, savoir celui qui en dirait le plus, qu’un bureau, qu’une recette, et par conséquent qu’un embarras au commerce ; sur de pareilles remontrances, dis-je, aurait-on pu dire, sans renoncer à la raison, que c’était l’intérêt du prince que tous ces préciputs, que tant de frais d’anéantissement, fussent portés par la marchandise ? — Ce degré d’horreur se renforce au troisième, au quatrième et au cinquième, et enfin, au onzième établissement, comme il se trouve en quelques villes du royaume, sur une même denrée, dans un même lieu, toujours avec les mêmes circonstances, ou plutôt les mêmes vexations, qui ont réduit la consommation d’une des villes où cette malheureuse scène se passe, de 60 000 pièces devin qu’elle était autrefois, présentement à peine à 4 000, et fait par conséquent arracher les vignes, et diminuer la Taille de six fois plus que le roi ne recevait de cette hausse des Aides. Que l’on ne s’étonne donc plus des dix millions de rente perdus sur la seule Élection de Mantes, et à proportion autant dans le reste du royaume, par un intérêt solidaire que toutes les provinces ont les unes avec les autres. — Tout de même à l’égard des vingt-six droits ou déclarations sur la charge d’un vaisseau : la simple exposition du fait, dès la première addition au premier droit, bien loin d’attendre le vingt-sixième, eût formé un degré d’horreur, qui n’eût pas permis d’opiner autrement dans le Conseil du roi, que par des exécrations contre les auteurs de pareilles impositions.\par
Qui est-ce qui n’eût point pensé que c’est la même chose, sans aucune différence, que si un prince ayant à recevoir 100 000 livres par an sur quelques particuliers très disposés et très en état de les payer, son intendant commettait dix personnes, avec chacune 1 000 livres de gages, pour percevoir 10 000 livres chacune, bien qu’une seule, faisant toute la recette, n’eût pas de quoi s’employer en ne donnant que la vingtième partie de son temps ? Ne dirait-on pas que l’intendant partage ses gages moitié par moitié avec les commis, et qu’il fait son compte aux dépens de celui de son maître ?\par
Cela est justement arrivé depuis 1660, par l’abrogation des remontrances des peuples, non de la part du ministre qui était très intègre, mais du côté de la cour, et de toutes les personnes considérables du royaume, qui ont érigé ces désordres, ou plutôt la ruine de la France, en revenu réglé. — Premièrement, on ne parvient à la place de receveur ou de fermier général, qu’en prenant des recettes à plus haut prix que leur juste valeur, des personnes d’élévation, qui font cela fort innocemment, ne sachant pas ce que doit coûter un pareil profit au roi et au royaume. Toutes les commissions sont autant de bénéfices brigués par toutes les personnes de condition, soit pour servir de récompense à leurs domestiques, et épargner leur propre bourse, ou pour en tirer des contributions personnelles. — C’est ce que M. Fouquet déclare dans ses défenses, où il nomme tous les demandeurs en de pareilles occasions, savoir toutes les personnes de la cour et du Conseil actuellement vivantes. — Ainsi, quelques bonnes intentions qu’ait un ministre, il n’est applaudi et on ne chante ses louanges qu’à proportion qu’il contente tant de demandeurs : ce que ne pouvant faire non seulement en ne levant que des tributs réglés, mais même par un petit nombre d’{\itshape affaires}, qui ne pourraient pas contenter la vingtième partie des prétendants, il faut qu’il donne les mains comme malgré lui à toutes ces horreurs.\par
Voilà les manières et la nation qui ont réduit le royaume en l’état où il se trouve, d’une façon d’autant plus déplorable, que ceux qui auraient été à portée de signaler au roi et à MM. les ministres le désordre et ses causes, étaient engagés par leur intérêt à le maintenir. Et c’était leur méthode, lorsqu’on se déclarait contre ces manières d’une façon sourde et à paroles perdues, de publier que c’étaient des esprits inquiets et visionnaires qui tenaient ce langage, et qui voulaient même renverser le royaume, appelant renversement la cessation du plus grand bouleversement qui fut jamais. En effet, si la France n’avait consisté qu’en quatre ou cinq cents personnes, dont tout au plus un pareil cortège était composé, c’est-à-dire de sujets qui méritent du ménagement, ils auraient eu raison de parler de la sorte ; mais comme c’est au contraire le royaume, qui consiste en quinze millions d’âmes, et le roi à la tête, qui sont ruinés par ces manières, de semblables allégations ne peuvent être considérées que comme une horrible extravagance.\par
Ce genre de gouvernement ayant ruiné tous les revenus, et les Traitants et les Partisans n’ayant plus de fortune à faire par l’addition de nouveaux droits sur les denrées, ce qui n’était plus possible, la guerre de 1689 survint, et MM. les ministres, quoique personnellement très intègres, ne supposèrent point qu’il y eût d’autres mesures pour trouver les fonds nécessaires, que les canaux qu’on vient de coter, savoir le service des Traitants et Partisans, qu’ils acceptèrent à l’égard des fonds et immeubles, pour leur faire souffrir le même sort qu’avaient éprouvé les revenus et denrées, sur lesquels il n’y avait plus rien à faire, qui sont les termes dont ils se servent ; ce qui signifie en langage clair et net, qu’il n’y a plus rien à gagner pour eux, quand il n’y a plus rien à détruire. Ce qui saute aux yeux de tout le monde est trop public, savoir, une désolation générale, qui est leur ouvrage, pour laisser le moindre soupçon que cette expression soit trop forte et trop violente. — Ils attaquèrent donc les charges et dignités de la robe, ainsi que les emplois de leur dépendance, que l’on sait composer ou qui composaient une si grande masse dans le royaume, et en quinze ou seize ans ils leur ont fait subir le sort qu’avaient éprouvé les revenus, au même compte de la destruction des denrées et produit des terres, savoir vingt de perte en pur anéantissement, pour un de profit au roi. Ce qu’il y a de plus cruel, est que cela a coupé l’arbre par le pied, et anéanti toutes les fabriques de monnaie en papier et parchemin, parce que ces sortes de fabriques ne roulent que sur la solvabilité des propriétaires des immeubles, et que ces derniers ont vu s’évanouir tout leur crédit, qu’il a fallu remplacer par l’argent en personne, du moment où leurs fonds ont été exposés à un anéantissement continuel. Sans que, toutefois, on puisse se plaindre en aucune façon de MM. les ministres, qui pratiquaient ces manières avec la dernière douleur, mais auxquels il était aussi impossible d’en user autrement, qu’il le serait à un sujet né dans l’erreur, d’embrasser et de professer la religion catholique, dans un pays où il n’y aurait que des hérétiques.\par
Mais enfin ce moyen étant absorbé, et ayant pris fin comme l’autre, et aucun Partisan ne se présentant plus aujourd’hui pour traiter de nouveautés, parce qu’il est assuré qu’il ne s’en pourrait pas défaire, ceux qui s’étaient accommodés de presque toutes, ne s’en trouvant pas bien, et se voyant exposés sous ce rapport à souffrir le sort de leurs prédécesseurs, c’est-à-dire à payer une seconde fois, ou bien à n’avoir rien acheté, et à avoir perdu leur argent ; on espère que le rétablissement de la France, dans une conjoncture si importante, n’aura plus tant d’ennemis à combattre, d’autant plus que l’on déclare que ce qui est fait est fait, et que l’on ne prétend faire rendre gorge à qui que ce soit, contre l’usage ordinaire. — Que si l’on s’est étendu sur cette troisième cause des désordres de la France, c’est pour couper pied à toutes les objections que l’on pourrait faire au rétablissement du royaume. Outre que d’ailleurs, bien qu’il ne soit pas indispensable de supprimer les fermes ni les fermiers du roi, quoique ce fût le plus grand service que l’on pourrait jamais rendre à l’État, témoin le ménage qu’ils ont fait depuis 1660, cependant il est nécessaire que leurs fonctions soient réduites à un cérémonial moins désolant, ce qui leur sera utile, loin d’être dommageable. Or, comme jusqu’ici ils ont été regardés comme des gens sacrés jusqu’à la moindre partie de leur ministère, quelque effroyables et quelque désolantes qu’elles soient toutes, il a été à propos d’en faire un crayon, et de montrer en même temps qu’il s’en fallait beaucoup que les fondateurs et protecteurs de l’Ordre fussent gens à canoniser, n’ayant eu rien moins pour objet, dans de pareils établissements, que l’intérêt du roi.\par
Cet éclaircissement procurera un peu plus de tranquillité au salut du royaume, en faisant examiner par quel motif on y fera des objections, ainsi que les personnes qui les mettront en avant. C’est de cette manière qu’on prétend s’acquitter en deux heures de la promesse contenue dans le titre et au commencement de ce Mémoire, c’est-à-dire par la cessation de la plus grande violence que la nature ait jamais éprouvée depuis la création du monde, n’y ayant pas un des trois établissements dont il est question, qui ne soit une extravagance achevée, commise innocemment depuis 1660, par erreur au fait, sur la foi de la probité des premiers auteurs, mais qui ne peut être soutenue après connaissance de cause, sans renoncer à la raison, comme l’on verra invinciblement par la suite.
\section[{Chapitre IX.}]{Chapitre IX.}
\noindent Personne ne peut douter, après ce qui vient d’être rapporté, que l’on ne fait aucune injustice aux Aides, Droits de passage et de sortie du royaume, en mettant sur leur compte la cause de 800 millions de perte, dans celle de 1 500 qu’éprouve le royaume depuis 1660. Or, quoique cette cause soit encore plus violente que les deux autres, il ne faut constamment qu’un instant pour la faire cesser, avec d’autant moins d’inconvénients et de crainte, qu’il est certain que ce n’a jamais été que l’intérêt des entrepreneurs qui a mis les choses sur ce pied.\par
Pour se résumer donc, l’État est présentement, à l’égard de ces trois causes de sa ruine, comme un particulier et même une contrée qui se trouveraient dans la dernière désolation par un principe très violent, agissant sur eux immédiatement, et dont la simple cessation pourrait en un moment les remettre dans une très grande félicité. Un homme condamné à mort pour un crime d’État, avec une confiscation de tous ses biens, qui seraient fort considérables, recevant sa grâce du roi, passerait dans un instant du dernier malheur à une très heureuse situation. La ville de La Rochelle, qui éprouva les rigueurs que l’on sait, lors de sa prise par le roi Louis XIII, ne fut qu’un moment à acheter le pain cent sous la livre, c’est-à-dire à voir tous les jours cent ou cent vingt de ses habitants mourir de faim ; et puis, les portes ouvertes par sa reddition, se procurer ce même pain à moins d’un sou la livre.\par
Si quelqu’un, dans l’un et l’autre de ces deux cas, proposant le remède qui les aurait tirés d’affaire, eût eu pour objection que l’on ne pourrait prendre ses mesures sans déconcerter leur situation naturelle, ou tout au moins qu’ils n’auraient pu jouir des fruits de ces grâces après qu’elles auraient été faites, qu’une guerre qui se passait à deux cents lieues n’eût été finie, n \textsuperscript{}!aurait-on pas estimé que ceux qui tenaient un pareil langage méritaient les petites-maisons ? ou plutôt aurait-on daigné leur répondre ?\par
On maintient, encore une fois, que de tout point c’est là la situation de la France à l’égard des 500 millions de rente, partie des quinze cents perdus, qu’on peut lui rétablir en deux heures, sans risquer davantage qu’à l’égard de ce particulier condamné, et de La Rochelle assiégée ; et que les allégations de prétendu déconcertement, de péril, ou de conjoncture de la guerre, sont d’un pareil degré d’extravagance qu’il aurait été dans les deux cas qu’on vient de marquer. Ainsi, pour entrer d’abord en matière, et prendre les trois causes l’une après l’autre pour leur cessation, comme on a fait pour leur découverte, on va voir, en particulier comme en général, qu’il n’y a pas moyen de tenir pied sur la contradiction, sans renoncer à la raison.\par
La Taille qui se trouve la première à la tête, comme ennemie jurée de la consommation, par son {\itshape incertitude}, qui met tout le monde sur le qui-vive ; par son {\itshape injustice}, qui fauche tous les sujets les uns après les autres, sans les quitter qu’ils ne soient sans pain, sans meubles et sans maison ; et sa {\itshape collecte}, qui oblige ceux qui ont quelque chose, de payer de temps en temps pour les insolvables, ou de périr à la peine, comme il arrive souvent ; la Taille, dis-je, peut être dépouillée de ces trois effroyables désordres en un moment, par une simple injonction de MM. les ministres aux intendants des provinces, de faire observer les anciennes ordonnances dans la dernière exactitude, sans nulle acception de personnes. Les descentes de MM. les maîtres des requêtes dans les provinces, qui n’étaient qu’en une certaine saison de l’année, n’avaient été anciennement ordonnées que pour ce sujet. Il est marqué en termes exprès qu’ils imposeront sur-le-champ, et même les Élus, tous ceux qui n’ont pas un taux proportionné à leur exploitation, soit en propre ou par fermage, et qu’ils déchargeront pareillement ceux qui se trouveront dans une situation opposée. Les mandements des Tailles, envoyés toutes les années dans les paroisses, l’ordonnent Semblablement ; cependant on peut assurer qu’il n’y eut jamais rien de plus mal exécuté ; et il est même presque impossible que cela soit autrement, par rapport aux sujets qui ont cette fonction. Anciennement ce n’étaient que des personnes du pays ; mais depuis quarante ou cinquante ans, il a fallu absolument n’en point être ; en sorte que, quelques bonnes intentions qu’ils aient, il est impossible qu’ils fassent jamais rien de bien, arrivant dans une contrée où tout leur est nouveau, et où tout le monde se trouve payé pour leur faire de faux rapports, et qui que ce soit pour leur dire la vérité.\par
Cependant l’exécution des anciennes ordonnances et la justice sont, aisées à mettre en pratique, après que MM. les ministres l’auront commandé, qui est par où il faut commencer. — Il n’est question que d’ordonner que chaque intendant partagera le soin des Élections à trois ou quatre officiers de ces compagnies, choisissant ceux qui sont entendus, non seulement dans le commerce et dans le labourage, mais même qui connaissent les contrées et les facultés des particuliers qui y ont du bien ; ce qu’il est aisé de savoir, quand on voudra s’y employer fidèlement, jusqu’à un cep de vigne, un arbre, un pouce de terre, et la moindre bête de nourriture. — Cette connaissance acquise par eux, ou en prenant des mémoires de sujets entendus, comme il s’en trouve dans toutes les paroisses, moyennant quelque légère rétribution, il faut qu’ils fassent une estimation des facultés de chaque village, en marquant sur un rôle à chaque cote : Celui-là a tant de terres en fermage ou à lui, de tant de valeur ; tant en labour, tant en simple pâture, tant d’excellente, tant de médiocre ; tant de bestiaux, et tant de vin ou de cidre, année commune ; et son fermage va à tant par an. Quelque surprenant que cela paraisse en gros, il n’y a rien de plus facile dans le particulier, lorsque ce sont des gens du métier ; et quand une Élection serait composée de cent cinquante ou deux cents paroisses, trois ou quatre sujets, dans chacune, en viendraient facilement à bout en quinze jours ou trois semaines ; c’est-à-dire que tout le bien d’une Généralité serait constant et connu en aussi peu de temps, tous travaillant ensemble, et ainsi celui de tout le royaume par la même raison. — Il faudrait marquer aussi le nombre des privilégiés, nobles ou ecclésiastiques, ou par leur emploi ; si c’est par ancienne ou nouvelle création, et s’ils n’excèdent point la quantité d’exploitation portée par leurs privilèges. Tout de même des misérables, n’ayant que leurs bras pour leur subsistance, sans nulle occupation que leur simple demeure.\par
Les choses en cet état, un intendant ferait faire la balance des biens de toute sa Généralité, Élection par Élection, pour imposer la Taille sur chacune, à proportion des biens ; et puis par subdivision par paroisse, et les préposés ensuite sur chaque particulier, sans se rapporter aux habitants que pour en prendre les mémoires, n’y ayant aucun d’eux qui ose et qui soit en état de mettre les receveurs ou fermiers des personnes considérables à leur juste taux. — Ainsi, du premier abord, voilà l’{\itshape incertitude} et l’{\itshape injustice}, qui coûtent plus de trois à quatre cents millions de rente au royaume, sauvées, et même les procès, puisque n’y ayant plus que des questions de fait, le subdélégué ou l’intendant les pourrait vider sur-le-champ.\par
Mais il faut encore faire disparaître la {\itshape Collecte}, et cela est aisé, même de l’agrément des peuples. — Il faut ordonner que quiconque portera, dans les trois premiers mois de l’échéance de la Taille, toute son année droit en recette, sera exempt d’être collecteur, et garant du recouvrement de la paroisse : il n’y a qui que ce soit, jusqu’au plus misérable, qui ne vende sa chemise pour être exempt de cette servitude ; et qui, lorsqu’elle lui viendra à son tour, par l’acceptation que ne manqueront pas de faire les riches de ce parti, ne donnera tout pour avoir le même avantage. — Il faut ordonner pareillement que la Taille, et les autres impôts qui l’accompagnent pendant la guerre, se prendront par privilège comme une rente foncière, c’est-à-dire avant le prix du louage des terres et maisons.\par
L’usage était, ci-devant, que le maître précédait pour une année sur la Taille, mais c’était à cause de son injustice, qui eût souvent tout emporté ; cette injustice étant ôtée, et l’équité rétablie, comme la cause cesse, l’effet doit cesser pareillement. De cette manière, le receveur des Tailles décernera contrainte contre chaque particulier, lors du premier envoi des mandements, dès qu’il aura passé sa soumission au greffe de l’Élection, qu’il entend payer toute son imposition dans les trois mois, pour être exempt d’être collecteur. — Que si ce dernier ne l’effectuait pas, il n’y aurait rien de gâté, puisque cette redevance précédant le paiement du maître, ce serait au receveur à y donner ordre.\par
À l’égard des villes taillables et gros bourgs, où la seule industrie paie une grosse Taille, il les faut absolument mettre {\itshape en Tarif} ; il n’y a aucun de ces lieux qui ne le demande à mains jointes, et ceux qui l’ont pu obtenir ont acquis un degré de richesse qui devrait porter à ne refuser jamais une grâce pareille. Le seul obstacle qui l’a empêché jusqu’ici, est que les juges et les receveurs s’y sont tous opposés. En effet, cela met fin aux procès, ainsi qu’aux frais et contraintes que les receveurs ont érigés en revenus réglés, et dont il faut qu’une paroisse souffre une certaine quantité ; autrement elle serait haussée au premier département, dont ils sont presque toujours les maîtres, sous prétexte qu’ils ne pourront faire le recouvrement si on ne suit pas leur idée.\par
Comme voilà bien du monde nouvellement mis en besogne, il les faut payer tous, autrement on sera mal servi, comme il arrive d’ordinaire, et surtout à la guerre, où, si l’on veut que les troupes fassent leur devoir, et ne pillent point, il leur faut faire toucher leur solde. Par bonheur, dans cette nouvelle fonction il y a un fonds certain et naturel, sans qu’il en coûte rien au roi et au peuple. Les six deniers pour livre qui se donnaient aux collecteurs des paroisses pour le recouvrement de la Taille demeurent entièrement inutiles, et il ne reste plus que les frais du papier et confection des rôles ; et comme ce sera l’affaire des subdélégués et de ceux qui seront chargés de chaque contrée, il faut sur ce fonds que l’intendant leur fasse départir à chacun 4 à 500 francs par an plus ou moins, suivant le travail et l’étendue du district ; ils en donneront quittance aux receveurs des Tailles, qui en compteront aux Chambres des comptes comme du reste, parce que l’ordre de l’intendant sera attaché avec les quittances. Il faut aussi une somme, comme de 1 000 livres ou à peu près, aux receveurs particuliers, pour augmentation d’un commis qui sera nécessaire pour la perception de tous ces impôts singuliers. Il faut enfin qu’il en reste une somme aux intendants, comme de 2 ou 3 000 livres, pour payer les espions qui avertiront que les préposés par lui commis ne font pas leur devoir, ayant favorisé dans l’assiette leurs parents et amis ; auquel cas il les faudra destituer avec infamie, et leur faire payer le dommage de ceux qui auront été lésés, sans nul rejet, parce que ce sera leur faute. Tout ceci se trouve marqué par le règlement des Tailles de 1604, du temps de M. de Sully, que l’on n’a fait que copier en cela comme en tout le reste, surtout les blés ; ce qui est conforme à tous les gouvernements du monde. Il faudra encore que les intendants soient souvent en campagne pour partir au pied levé, sans avertir personne, pour vérifier sur les lieux si les avis qu’on leur a donnés sont véritables, ce qui demande des frais. Enfin, il est nécessaire que tout le monde conçoive qu’il sera impossible d’user de supercherie sans s’exposer à une punition exemplaire.\par
Mais comme le principe de toutes sortes de paiements, et par conséquent de la Taille comme du reste, est la vente des denrées, ce recouvrement sera extrêmement facilité, par la valeur que l’on va y mettre, surtout aux blés, qui, menant la cadence, sont présentement en perte aux laboureurs, le prix n’atteignant pas même les frais de la culture, comme on va voir dans le chapitre suivant.
\section[{Chapitre X.}]{Chapitre X.}
\noindent Le dérangement qui se rencontre dans le prix des blés par leur avilissement, qui, ruinant les proportions qui doivent être entre les frais de leur culture, ensemble le paiement du fermage, et le prix qu’on l’achète, empêche ce premier commerce, par lequel cette manne primitive passe uniquement aux mains de ceux qui n’ont que leur travail pour se la procurer ; ce qui est encore la ruine des uns et des autres, n’étant pas moins préjudiciable à un État, s’il ne l’est pas même davantage, que la situation opposée, qui ne produit des horreurs que par ce même manque de proportion, tous les excès étant également dommageables, quoique diamétralement opposés ; ce dérangement, dis-je, n’est ni l’effet du hasard ni de la nature, qui par sa destination entend et fait toujours si bien, qu’il n’y a point de métier ni de profession qui ne nourrisse toujours son maître, comme elle ne met point d’animaux au monde qu’elle ne les assure de leur pâture en même temps.\par
Cette malheureuse disposition, qui coûte au royaume présentement plus que quatre fois les besoins du roi, rendent tout le monde très misérable, et les ouvriers plus que qui que ce soit, est la suite d’une volonté déterminée, que depuis six à sept ans on met à exécution avec la dernière rigueur, et même de très grands frais, par cette cruelle et fausse idée, que les grains sont de la nature des truffes et des champignons ; par la continuation, dis-je, de cette pensée, comme en 1660, que le blé est un présent gratuit de la nature, et qu’ainsi l’intérêt de l’État, surtout des pauvres, est de forcer les propriétaires de le donner à meilleur marché qu’il serait possible. On ne persiste, après la reconnaissance de l’erreur, dans cette conduite, que parce que des sujets couverts d’applaudissements ne veulent point convenir qu’ils aient été capables d’une pareille méprise, leur obstination à maintenir le mal leur étant moins préjudiciable, à ce qu’ils croient, qu’un désaveu de leur conduite passée, quelque bien qu’il en vînt au royaume ; ils ont cru que l’État ne pouvait éviter un excès, savoir une extrême cherté, qu’en se jetant dans l’autre, qui est l’avilissement, quoique n’étant pas moins préjudiciable par lui-même : c’est lui seul qui produit les chertés, comme on peut voir par le chapitre qui est à la fin de cet ouvrage. Cependant, comme l’on ne doute point que ceux qui n’ont pas de si déplorables intérêts ouvriront enfin les yeux, on passe avec confiance au remède.\par
On dira d’abord que le roi et MM. les ministres sont absolument maîtres du prix des grains, les pouvant faire baisser et hausser à leur volonté, en quelque temps et en quelque saison que ce soit : comme l’état d’avilissement où il se trouve est l’effet d’une main étrangère autre que celle de la nature, de même, par des manières contraires qui coûteront beaucoup moins, on peut mettre cette denrée au prix et en l’état qu’elle doit être pour supporter ses charges, c’est-à-dire les frais de la culture, et couler tranquillement aux mains de ceux qui n’ont d’autre fonds que leurs bras. L’on ne s’explique pas plus précisément sur ce sujet, parce que quoique cela se pratique en une infinité d’endroits, comme à Rome, en Angleterre, en Hollande et en Turquie, et qu’on ait agi de même en France en 1679, sans quoi cette année aurait été aussi cruelle que 1693 et 1694 ; cependant il est de l’intérêt de cette démarche qu’elle ne soit pas absolument publique, étant de la nature du secret, qui perd la vie aussitôt qu’il voit le jour.\par
Tout ce qu’on peut déclarer, est que la cherté ou l’avilissement, surtout dans un pays fécond comme la France, n’est rien moins, à la rigueur, que l’effet du manque ou de l’abondance des blés pour la subsistance de tous les peuples ; le dernier a toujours été l’ouvrage d’attentions déterminées comme aujourd’hui, et l’autre de la folie et de l’aveuglement du peuple, qui se forme lui-même le monstre qui le dévore. En un mot, le peuple est assurément comme un troupeau de moutons que l’on voudrait faire entrer par une très petite porte, et très embarrassée ; il n’y a qu’à en prendre un ou deux par les oreilles, et les tirer par force, aussitôt tous les autres s’y poussent avec la même violence dont il avait fallu user pour y conduire les deux premiers. Et quand il y aurait une très grande porte tout contre, exposée à leur vue, qui, les conduisant au même lieu, leur donnerait un passage bien plus aisé, il ne serait pas possible à force de coups de leur faire prendre ce parti, mais ils continueraient de s’étouffer les uns les autres pour suivre les premiers. Voilà le portrait du peuple, et sa conduite dans ses démarches tumultueuses, surtout à l’égard des blés. — Ainsi, en un moment ce fonds étant rétabli, on maintient que c’est plus de 300 millions de rente au royaume remis en un instant, parce que les proportions, dont le déconcertement est la ruine du commerce, commenceront à reparaître, et à fournir par conséquent la subsistance à toutes les deux cents professions, qui attendent uniquement leur nourriture du laboureur. C’est pourquoi on passe aux Douanes, sorties et passages du royaume, ainsi qu’aux droits d’Aides sur les liqueurs, qui prennent pour leur part, ainsi qu’on a dit, plus de 800 millions par an dans la perte des biens du royaume.\par
Le rétablissement en est d’autant plus aisé, que quoiqu’on les soutienne nuit et jour par des efforts continuels ; qu’il y ait plus de vingt mille hommes, et peut-être plus de trente, qui n’ont d’autre emploi que cette occupation, c’est-à-dire de ruiner les peuples, et par conséquent le roi ; cependant il n’y a qui que ce soit qui ne les déteste dans le particulier, et qui ne convienne que, si on avait eu intention de détruire le royaume, on n’aurait pas pu prendre d’autres mesures. Le cadavre que nous avons sous les yeux par la désolation de la culture des terres et du commerce, purge cet énoncé de tout soupçon de calomnie.\par
En effet, si un marchand, ayant ses magasins remplis d’excellentes denrées, et propres à l’usage de tout le monde, ne les voulait point livrer, après en avoir fait la vente dans sa maison, qu’après que l’on en aurait fait déclaration à vingt-six de ses facteurs et commis dispersés en divers quartiers de la ville, et souvent absents de leur demeure, en sorte qu’il fallut un temps infini pour s’acquitter de ces servitudes, n’estimerait-on pas aussitôt qu’il aurait perdu l’esprit, et tout le monde ne le quitterait-il pas ? Or, une contrée commerce avec une autre tout comme un marchand avec un autre marchand ; les mêmes mesures et les mêmes facilités doivent être observées dans ce commerce, et le même degré d’extravagance qu’on impute à l’un serait pareillement applicable à l’autre. Car, si quelque ami de ce négociant qui exigerait vingt-six déclarations avant que de se dessaisir de sa denrée, lui représentait qu’il eût à quitter cette manière, autrement qu’il se ruinerait et passerait pour un fou, et que le commerçant lui repartît qu’il convient de l’extravagance de cette conduite, mais qu’il ne la peut abandonner dans le moment, de peur de troubler l’ordre de ses affaires, et qu’au moins il faut attendre qu’un procès qu’il a à deux cents lieues de sa demeure soit terminé ; ne serait-ce pas pour le coup qu’on le ferait enfermer, et qu’on lui ôterait absolument l’administration de ses biens ? Voilà néanmoins, en cet article de Douanes, la situation de la France, tant dans les sorties du royaume que les passages de contrée à contrée ; et les raisons que l’on apporte pour ne pas faire cesser le désordre, sans perdre un moment, sont d’un pareil métier et valeur que celles qu’on vient de mettre dans la bouche de ce marchand particulier.\par
Les Aides sont à peu près de même nature, surtout dans quatre Généralités, savoir Rouen, Caen, Amiens et Alençon, où le droit de {\itshape quatrième denier} de tout ce qui se vend de liqueurs en détail s’exige non au quatrième, mais au troisième, parce qu’on n’a point d’égard aux lies et diminutions journalières, mais seulement au volume de la futaille, ce qui, joint à des droits d’entrées effroyables, surtout dans les villes non taillables de ces contrées, fait que cette exigence de tous points n’est et ne se doit point appeler une contribution, mais une confiscation, comme l’effet qu’elle a produit n’a que trop justifié. La seule Élection de Mantes, comme l’on a dit, y est pour 2 400 000 liv. par an sur les vignes, ce qui n’est qu’un baromètre du reste du royaume, puisque cela procède d’une cause générale. Les cidres en Normandie, qui tiennent lieu de vins, ont été pareillement mis, par ce même principe, dans un si grand désarroi, que dans les années abondantes il s’en perd plus de la moitié que l’on néglige absolument de mettre à profit, ou qui périt, se gâtant par la garde, pendant que les trois quarts des peuples, non seulement de la Normandie, mais même de la Bretagne, Picardie et Beauce, qui sont limitrophes, ne boivent que de l’eau à ordinaire réglée. — C’est en vain que la Bourgogne, comme un pays d’États, jouit de cette exemption des Aides ; sa manne nourricière, savoir les vins, à l’aide de laquelle et de l’excédent elle se peut procurer ses autres besoins particuliers, est également coulée à fond, de même que si elle avait ces droits dans ses entrailles. Ainsi ce sont ses intérêts que l’on défend pour le moins autant que ceux de ces quatre Généralités : c’est pourquoi elle doit contribuer, en comprenant ses avantages, à lever la cause de l’avilissement où elle voit souvent cette denrée lors d’une récolte abondante ; et quoi que ce soit qu’elle paie, c’est-à-dire le double de ce que le roi reçoit présentement, elle y gagnera encore quatre pour un, et ainsi des autres contrées du royaume, qui suivent toutes le sort les unes des autres, quelque éloignées qu’elles soient de celles où le désordre qui les dévore a pris naissance ; et, par la raison des contraires, le rétablissement ou la cessation du mal produira incontinent le même effet à leur égard. Le vin qu’on donne souvent à un sou la mesure en Bourgogne, en Orléanais, dans la petite Champagne et en Anjou, n’est à ce misérable prix au-dessous des frais du vigneron que parce qu’il est à 24 sous dans la Picardie et la Normandie ; et il est à cet excès dans ces provinces, par les mêmes raisons que le pain était à 100 sous la livre lors du siège de La Rochelle.\par
Dix mille commis arrêtent les avenues de ces liqueurs, tout comme l’armée du roi empêchait le passage des grains dans cette ville ; et lorsque les portes furent ouvertes, la même extravagance qui se serait rencontrée dans ceux qui auraient allégué que ces habitants affamés n’auraient pu soulager leur misère en se procurant du pain à {\itshape un sou la livre, puisqu’il ne valait pas davantage hors les portes}, qu’une guerre qui se faisait à deux cents lieues de ces quartiers ne fût terminée ; la même folie, dis-je, se trouve dans ceux qui prétendent que ces dix mille commis, qui font périr une moitié du royaume par l’abondance des liqueurs, et l’autre par l’excès du prix, ne peuvent être congédiés sans renverser l’État, ou tout au moins qu’il faut attendre que la guerre soit finie en Allemagne, en Italie et en Espagne.\par
Pour commencer par les Douanes, sorties et passages du royaume, c’est un Pérou pour le roi et pour ses peuples de les supprimer toutes à l’égard du dedans de l’État ; la raison des divers princes qui les avaient établies étant cessée, il en doit être de même de l’effet, par les effroyables suites qui les accompagnent toutes. À l’égard des entrées de la France, il les faut conserver en l’état qu’elles sont pour les sommes seulement, en aplanissant les difficultés, dont il ne revient rien au roi, mais qui rebutent les étrangers. Pour les droits de sortie, il ne leur faut faire aucun quartier, mais les supprimer entièrement, puisque ce sont les plus grands ennemis du roi et du royaume qu’il puisse jamais y avoir.\par
En effet, la misère étant le plus grand mal qui puisse arriver à un État, et l’avilissement des fruits, dont on ne peut trouver les frais de la culture, étant le plus grand principe de la désolation, il en faut user comme à l’égard d’un ennemi déclaré, qui vient pour envahir un pays : lorsqu’on le voit dans le dessein de faire retraite, il lui faut faire un pont d’or. Or, est-ce faire ce pont d’or à cet avilissement, le plus grand destructeur de biens qu’il y eut jamais, que de lui former jusqu’à vingt-six obstacles sur le même lieu, par autant de gens à gages, et dont la fortune consiste à le faire rester dans le pays pour continuer ses ravages, comme on vient de marquer à l’égard des Douanes sur les sorties et passages de la France ? C’est la même conduite à l’égard des blés et l’économie des Tailles. Tous ces monstres que l’on a décrits ne travaillent nuit et jour que pour maintenir cet avilissement : ainsi, pour continuer à faire la guerre à cette effroyable manière, il faut absolument réduire le droit de {\itshape quatrième} au {\itshape huitième} dans ces quatre Généralités, comme partout ailleurs où les Aides ont lieu.\par
Lorsque ce droit fut établi pour la campagne, où il n’était point, environ vers l’année 1640, à ce que l’on croit, toutes les contrées donnèrent une somme pour en être exemptes ; mais dans les seules quatre généralités mentionnées, les gentilshommes et personnes notables eurent l’indiscrétion de l’acheter presque pour rien ; et concevant bien qu’il n’était pas exigible au pied de la lettre, sans tout ruiner, ils n’en tiraient pas le tiers, et sous-fermaient aux cabaretiers à très grand marché. Mais après 1660, ceux qui gouvernaient, croyant le roi lésé dans cette vente, comme il l’était effectivement, le retirèrent sans remboursement aux acquéreurs, estimant que la jouissance leur en tenait lieu, ce qui était véritable ; et il n’y aurait eu rien de gâté, s’ils avaient continué à le faire valoir comme les premiers acquéreurs. Mais, l’ayant voulu exiger à la dernière rigueur, ce fut une confiscation des vignes et des liqueurs, et une condamnation aux deux tiers des peuples du royaume de ne boire que de l’eau, d’autant plus qu’on quadrupla les droits d’entrée en même temps, dans les villes non taillables de ces quatre généralités, par l’établissement de divers Traitants et bureaux, qui triplaient, par ce cérémonial, et l’embarras ou séjour des voitures, le mal déjà causé par l’excès des sommes. Ce qui réduisit la consommation de ces villes à la dixième ou douzième partie de ce qu’elle était auparavant ; et encore davantage à la campagne, puisque n’y ayant point constamment de village autrefois où il n’y eût jusqu’à deux ou trois cabarets, présentement c’est un hasard si dans dix il s’en trouve un seul pour toute la contrée. Par où on peut voir le profit que les Traitants ont fait en ruinant le roi et les peuples.\par
Ainsi on ne renverse point l’État, ni on ne les congédie point, en réduisant le {\itshape quatrième} au {\itshape huitième}, et on ne délivre point la France tout à coup, comme on fit à La Rochelle : on les ménage, au contraire, et l’on veut vivre avec eux, en les priant de souffrir seulement qu’on ouvre une porte pour que ces provinces de vignobles qui périssent par l’abondance deviennent riches tout à coup. Sur ce même compte, il faut réduire les droits d’entrée des villes non taillables, dans ces quatre généralités, à la juste moitié de ce qu’ils sont à présent ; et comme il y a plusieurs Traitants, il faut que la réduction soit au sou la livre du prix de leurs baux, et ils y gagneront considérablement, puisqu’ils pratiquent eux-mêmes cette remise tous les jours dans les occasions, lorsqu’ils sont habiles, sachant bien que sans cela on ne vendrait rien et qu’ils perdraient tout.\par
Il faut encore que tous ces divers droits soient réduits à une seule et même somme certaine, d’un nom de monnaie d’argent, et nullement revêtus d’un nom de guerre, comme par ci-devant, c’est-à-dire {\itshape parisis, sou-denier, travers, resve, haut passage, grand, petit} et {\itshape nouveau droit}, qui, se trouvant souvent combinés ensemble, sont autant de pièges tendus à des gens qui ne savent ni lire ni écrire, comme sont tous les voituriers, pour tout confisquer ou les ruiner en séjours, quand ils ne veulent pas les racheter à prix d’argent.\par
La {\itshape jauge} est le comble de la vexation : outre qu’il est impossible naturellement de construire une futaille d’une justesse mathématique, en sorte qu’il n’y ait point un verre ou un setier plus ou moins, il est de la même impossibilité à un jaugeur de garder une pareille exactitude dans son calcul, et jamais deux pareilles gens ne se rencontrent dans leurs mesures, même à beaucoup près, comme on a quelquefois fait expérience. Ils en usent même si bien, qu’ils crient leurs suffrages à l’encan à qui en donnera le plus du commis ou du voiturier, pour rendre un procès-verbal favorable à l’un ou à l’autre sur la contenance de la futaille. Il les faut absolument supprimer, et les contrées gagneront cent pour un en les remboursant. On peut ordonner que l’on fasse les vaisseaux le plus justes que faire se pourra, en marquant la mesure ; et lorsque dans les entrées on croira apercevoir, à vue d’œil, que les futailles sont défectueuses, il faudra, sans les pouvoir arrêter, dénoncer les propriétaires aux juges, pour être condamnés en amende, comme on fait un cabaretier lorsque ses vaisseaux ne sont pas justes ; ce qui ne pourra être fait à moins que le mal ne soit considérable, et sans frais, devant l’intendant ou son subdélégué ; autrement le remède serait pire que le mal.\par
Il y a encore un monstre à conjurer, c’est-à-dire les déclarations, droits de passages, qui s’exigent sur ce qui {\itshape passe debout} à chaque endroit, et qui causent les mêmes vexations dont on a parlé. Il faut de la liberté dans les chemins, si l’on veut voir de la consommation, et par conséquent du revenu : ce qui ne peut être tant qu’il y aura à chaque pas des gens payés, et qui attendent leur fortune à empêcher qu’un pays ne commerce avec l’autre, en s’aidant réciproquement des denrées dont l’abondance les ruine, pour recouvrer celles dont la disette les rend pareillement misérables. Pour ce sujet, il faut ordonner que tout voiturier, soit par eau ou par charroi, qui voudra conduire des liqueurs en quelque lieu, si éloigné qu’il puisse être, sera obligé d’en prendre un {\itshape passe-avant} du plus prochain bureau des aides, s’il y en a, sinon du juge de police, qui ne pourra coûter que dix sous, tout compris : cet acte portera la déclaration de la quantité de la voiture, et du lieu où on l’expédie ; et avec ce viatique, il se mettra en chemin, sans que qui que ce soit le puisse arrêter dans sa route, soit bourgs ou villes murées, ni aucun bureau exiger autre chose que la simple vue de son acte, sans s’en dessaisir, ni le retarder un moment, lui ni sa voiture. Dans les lieux, comme villes et bourgs d’aides, où il passera la nuit, il ne pourra décharger ni toucher à sa denrée, à moins de quelque inconvénient auquel il faudrait donner ordre, auquel cas il serait tenu d’aller avertir le receveur des droits du lieu ; le tout, à peine de confiscation de la marchandise, charrettes et chevaux, et de mille livres d’amende contre l’hôtel où les contrevenants seraient logés. Que si le voiturier en chemin trouve à vendre sa marchandise plus commodément qu’aux lieux où il la destinait, il le pourra faire en payant les droits du lieu ; si c’est dans un village où il ne soit rien dû, il ne paiera rien.\par
De cette sorte, non seulement on ne renverse pas l’État, mais au contraire, étant tout bouleversé, on le remet dans une entière félicité ; en un mot, en cet article comme aux deux autres, c’est la {\itshape levée du siège de} La Rochelle ; et la même extravagance qui se serait rencontrée dans les objections qu’on aurait pu faire, en soutenant qu’il aurait fallu du temps, après les portes ouvertes, pour avoir le pain à un sou de cent fois autant qu’il était, se trouve dans cette occasion, si quelqu’un prétendait qu’une Déclaration publiée sur ce style ne mettrait pas aussitôt toutes choses en valeur, et par conséquent tous les peuples dans la félicité, et en état de fournir avec profit tous les besoins du roi.\par
Cette modération qu’on apporte aux fonctions et aux bases du revenu des Traitants, on maintient, comme on l’a déjà dit, qu’elle ne sera nullement préjudiciable à leurs intérêts, et qu’ils regagneront en gros, par la hausse de consommation, ce qu’ils allégueraient aujourd’hui devoir perdre par l’altération du détail. Cela n’a jamais manqué toutes les fois que le cas est arrivé, et récemment dans la distribution, du tabac, où la recette a augmenté après qu’on a eu baissé le prix. Et le contraire à l’égard des lettres, et l’on sait des bureaux notablement diminués par la hausse des droits. Enfin on maintient que la réduction dans les quatre Généralités, dont le saccagement qui s’y commet par les Aides ruine également tout le reste du royaume, ne doit point diminuer d’un sou le prix des baux, par cette modération du quatrième au huitième, et des droits d’entrée dans les villes non taillables.\par
Que si les fermiers d’aujourd’hui ne le veulent pas comprendre, cela ne fera aucun dérangement, parce que, comme aucun n’est à forfait, et que tous demandent chaque année des dédommagements à cause du malheur des temps, il y a du monde tout prêt à prendre leur place à cette condition de ne rien diminuer, et on est assuré qu’ils y feront leur compte.\par
Il reste les droits de passage et de sortie, tant du royaume que des provinces réputées étrangères, établis par une surprise effroyable : il est assuré que le roi n’en reçoit point présentement quinze cent mille livres, non compris le {\itshape convoi de Bordeaux}, auquel on ne touche point, n’y ayant presque que le pont de Joigny dont le produit soit considérable. Or, outre que cette somme de quinze cent mille livres sera bien plus que gagnée dans la masse de tout le royaume par une opulence générale, quand le roi la remettrait à ses peuples en pure perte sur lui, n’y vouloir pas entendre, c’est la même chose que de ne vouloir pas semer pour recueillir vingt pour un, en regardant le blé qu’on jette dans la terre comme perdu. Les 80 millions de hausse de tributs dont on va faire fonds sur les peuples, avec des applaudissements et des actions de grâces de la part de tous ceux qui ne sont point suspects sur cette matière, ce qui répond que c’est de l’argent comptant ; cette somme, dis-je, est une récolte assez abondante pour n’y pas épargner une pareille semence.\par
Et pour montrer invinciblement qu’il n’y a rien que de très réel dans les suites d’une Déclaration qui ne coûtera point trois heures à construire sur ce modèle, en rectifiant les trois articles, seuls principes de la misère des peuples, il n’y a qu’à en faire un essai en la publiant seulement, parce qu’on en suspendra l’exécution d’un mois ou deux : on maintient que dans le moment tous les biens seront considérablement augmentés ; on pourra alors juger, par cet échantillon, de l’effet qu’on doit attendre de la pièce, et qui est visionnaire, de l’auteur de ces Mémoires, ou des contredisants.\par
Comptant donc sur 5 à 600 millions de hausse dans la consommation par un effet subit, et une violence cessée comme à La Rochelle, il faut venir à la part du roi, qu’il y aurait autant d’injustice au peuple de refuser au prince, par suite de cette augmentation de biens, qu’il y avait de surprise ci-devant à ériger la confiscation entière, tant des meubles que des immeubles, en contribution réglée ; ce qui ayant commis le prince et ses sujets par des refus d’une part, que la seule impossibilité d’exécuter empêchait d’être criminels, et de vaines contraintes, quoique des plus violentes, de l’autre, a plus détruit de biens et fait de ravages que jamais les plus grands ennemis du royaume dans leurs victoires les plus complètes, depuis l’établissement de la monarchie.\par
Il faut que les tributs coulent aux mains du prince comme les rivières coulent dans la mer, c’est-à-dire tranquillement, ce qui ne manquera jamais d’arriver, lorsqu’ils seront proportionnés au pouvoir des contribuables, tant sur les choses que sûr les personnes : la dérogeance qu’on a apportée à cette règle est seule cause de tout le désordre. Un monarque en doit user envers ses peuples comme Dieu déclare qu’il fera envers les chrétiens ; savoir, qu’il demandera beaucoup à qui aura beaucoup, et peu à qui aura peu. Et sur le même style, un père de l’Église atteste que, de quelque grand prix que soit le paradis, Dieu ne le vend aux fidèles, quelque misérables qu’ils soient, que le prix qu’ils le peuvent acheter : voilà l’unique niveau des tributs, et celui des quatre-vingts millions de hausse que l’on va établir dans le chapitre suivant.
\section[{Chapitre XI.}]{Chapitre XI.}
\noindent On a dit, au commencement de ces Mémoires, que les princes les plus riches étaient ceux qui avaient le moins de genres de tributs, et qui passaient le plus droit en leurs mains sans poser nulle part au sortir de celles de leurs peuples.\par
Or, pour en former un de ce genre, il n’est point nécessaire de faire rien de nouveau : il n’y a qu’à s’adresser à la Capitation, qui a d’abord ces deux qualités de passer droit, sans frais, des mains des peuples en celles du monarque ; et, pour lui faire atteindre jusqu’au niveau de ses besoins dans la conjoncture présente, ce qu’elle ne fait pas à beaucoup près, quoique ce fût l’intention des fondateurs portée par le titre même de son établissement, il n’est pas si nécessaire de la perfectionner, que de la faire cesser d’être ridicule. En effet, le principe de qualités ou d’emplois que l’on y a marqué, pour régler le degré de contribution dans chaque particulier, indépendamment de sa très grande richesse ou de son extrême misère, ce niveau, dis-je, n’en faisant aucune différence, est une mesure aussi absurde que serait une loi qui ordonnerait que l’on paierait le drap chez un marchand, et la dépense au cabaret, non à proportion de ce qu’on aurait pris chez l’un et chez l’autre, mais suivant la qualité et la dignité du sujet qui se serait pourvu de ses besoins. Les tributs sont une redevance aussi légitime, commandée par la bouche de Dieu même, que peut être le paiement de quelque dette que ce soit, et cela au sou la livre des biens que l’on possède dans un État ; et c’est bailler le change que d’y avoir mis un niveau qui fasse payer aux uns quatre fois plus qu’ils ne tirent, et ne doivent par conséquent, et aux autres la cinquantième partie moins qu’ils ne sont tenus par cette même règle de justice.\par
Il est certain, et public, que les qualités et dignités ne dénotent non plus les facultés d’un homme, que sa taille ou la couleur de ses cheveux. Il est donc du même ridicule d’avoir établi qu’un avocat ou marchand, ou un seigneur de paroisse et un officier paieront la même somme, qu’il le serait de régler que tous les boiteux contribueraient pour la même part, et que ceux qui marcheraient droit en fourniraient une autre : la raison de l’extravagance de cette dernière disposition se trouve, en ce qu’il se rencontrerait en l’une et l’autre de ces deux classes des sujets très riches, et d’autres qui n’auraient rien du tout, l’opulence ou la misère n’étant nécessairement attachée à aucune profession, non plus qu’à aucun genre de taille, ou couleur de poil. Cette diversité se trouvant donc chez les avocats, les marchands, les officiers, les seigneurs de paroisses, on ne peut nier que la parité de méprise ou de ridicule ne se rencontre également dans la disposition qui se pratique, et dans celle que l’on vient de marquer.\par
On ne peut présumer autre chose dans ceux que MM. les ministres avaient chargés de cette économie, sinon qu’ils ont eu dessein de rendre illusoire l’intention portée à la tête, savoir la suppression des {\itshape affaires extraordinaires} en rendant le produit de cet impôt insuffisant à atteindre aux besoins du roi ; ce qui n’eût pas été s’ils s’y fussent pris d’une autre manière. Et cela, par le même esprit que l’on avait eu en laissant déconcerter les Tailles par la souffrance de la mauvaise répartition, afin de donner ouverture aux partis ; de sorte que, de 56 millions qu’elles étaient, il les a fallu réduire à 32, pendant que l’on triplait les Aides, qui ne remplaçaient pas à beaucoup près ce déficit à l’égard du roi, et coûtaient dix fois la Taille au peuple ; et il ne faut pas dire qu’il demeurait une partie des Tailles en pertes, parce que c’était un jeu fait à la main, les répartiteurs traitant de ce regrat, où ils gagnaient des sommes immenses ; car, aujourd’hui que la Taille, accompagnée de la Capitation et de l’Ustensile, va à plus de cinquante-six millions, on n’y perd rien, quoique la campagne soit quatre fois plus pauvre. Ou tout au plus que, se trouvant bien partagés du côté des biens, ils n’ont pas voulu que les facultés fissent le niveau de ce tribut, mais les dignités ; ce qui, exigeant une possibilité générale, et les plus dénués faisant par conséquent la règle, c’était une sauvegarde à leur opulence de ne payer que très peu de chose par rapport à leurs possessions. En quoi ils se sont bien plus trompés que le prince, puisque les {\itshape affaires extraordinaires} ayant recommencé mieux que jamais, le dépérissement que cela a causé à la masse de l’État leur coûte trois fois plus que n’aurait fait une quadruple Capitation, qui n’aurait pas même été nécessaire pour les garantir de cet orage. On en prend à témoin toute la Robe, les Marchands et les Seigneurs des paroisses ; et il faut qu’ils conviennent, pour peu qu’ils veuillent dire la vérité, qu’il en est arrivé comme aux Tailles ; la décharge que les riches ont faite de leur juste contribution, pour en accabler les pauvres, ayant mis ceux-ci hors d’état de consumer l’herbage dont on a parlé, qui signifie généralement tous les biens, il est devenu entièrement en perte aux propriétaires, qui ont été tout à fait ruinés par ce prétendu privilège.\par
Parce qu’il y a une attention à faire, à laquelle qui que ce soit n’a jamais réfléchi, savoir, que le corps d’État est comme le corps humain, dont toutes les parties et tous les membres doivent également concourir au commun maintien, attendu que la désolation de l’un devient aussitôt solidaire, et fait périr tout le sujet. C’est ce qui fait que toutes ces parties n’étant pas d’une égale force et vigueur, les plus robustes s’exposent et se présentent même pour recevoir les coups que l’on porterait aux plus faibles et plus délicates, qui ne sont point à l’épreuve de la moindre atteinte ; sans parler du serpent à qui l’Écriture sainte fait servir de symbole de prudence, à cause qu’étant assailli, il couvre sa tête de tout son corps : la nature n’apprend-elle pas de même aux hommes, en semblable occasion, à présenter les mains et les bras pour parer ou recevoir les coups que l’on porte aux yeux et à la tête ?\par
Les pauvres, dans le corps de l’État, sont les yeux et le crâne, et par conséquent les parties délicates et faibles ; et les riches sont les bras et le reste du corps : les coups que l’on y porte pour les besoins de l’État sont presque imperceptibles tombant sur ces parties fortes et robustes, mais mortels quand ils atteignent les endroits faibles, qui sont les misérables, ce qui par contrecoup désole ceux qui leur avaient refusé leur secours.\par
L’on sait comme le ménage d’un pauvre se mène ; toute sa fortune roule assez souvent sur un écu ou deux qui, par un renouvellement continuel, le font subsister lui et toute sa famille, et consommer par conséquent les denrées qui croissent sur le fonds des riches, sans quoi elles leur demeurent en perte, ce qui est la situation d’aujourd’hui. — S’ils sont privés de cet écu ou deux tout à coup, par une injuste répartition d’impôt, ou quelque Affaire extraordinaire causée par l’insuffisance des tributs réglés d’atteindre aux besoins du roi, à cause que les puissants n’ont pas à beaucoup près voulu fournir leur contingent, voilà ce crâne et ces yeux blessés mortellement, qui font périr tous ces membres robustes qui n’ont pas voulu leur parer les coups ; ce qu’ils auraient pu faire aisément, sans en recevoir que de très légères atteintes. Pour l’intérêt donc des riches, il faut payer la Capitation au {\itshape dixième} de tous les biens, tant en fonds qu’en industrie ; et ce sera à titre lucratif de leur part, tant par le rétablissement des trois articles ci-dessus mentionnés, que par cette dernière raison ; et on ne craint point de répartie ou de contradiction, qui ne soit absolument une extravagance, en soutenant, comme on fait, qu’il n’y a aucun de ces contribuables qui ne gagne dix pour un de ce qu’ils paieront.\par
Il y a eu en tout temps, et dans tous les États du monde, des Capitations ; autrefois en France, sous les rois Jean et François I\textsuperscript{er}, et présentement en Angleterre et en Hollande ; et toutes, n’ayant d’autres règles que la quotité de biens, n’ont jamais fait le moindre fracas ni le moindre dérangement tant dans leur levée que dans leur paiement. La surprise l’a pu établir autrement en l’état qu’elle se trouve aujourd’hui en France ; mais, après ces éclaircissements, il n’y a que le crime qui la puisse refuser de la manière qu’on la propose, qui est celle de toutes les nations du monde.\par
L’allégation qu’il est difficile de trouver la quotité des biens des particuliers, ou cruel à eux d’en rendre compte, est tout à fait impertinente, puisque, dans le premier cas, elle suppose, en quelque sorte, qu’autrefois les peuples en France, ainsi qu’en Angleterre et en Hollande, étaient sorciers, pour avoir de pareilles révélations, et que ceux d’aujourd’hui ont, au contraire, perdu le sens ; et que, dans l’autre, on traite de cruauté une méthode qui, étant le salut de l’État dans la conjoncture actuelle, se pratique tous les jours tranquillement dans cent autres occasions bien moins importantes.\par
Faut-il, en effet, rebâtir une église ou un presbytère, les frais s’imposent et se répartissent au sou la livre de ce qu’on a de bien dans la paroisse. Est-il besoin de régler le mariage ou la légitime d’une fille avec ses frères après la mort du père et de la mère, cela se fait tous les jours devant les parents, ou par la justice, sur vue des pièces. La même chose des dettes qui surviennent longtemps après sur Une succession partagée entre plusieurs collatéraux.\par
Depuis le plus grand seigneur jusqu’au dernier ouvrier, il y a des baromètres certains d’opulence, et évidents pour ceux qui ont la pratique de la vie privée, mais qui sont lettres closes pour tout ce qui n’en a que la simple spéculation, comme sont tous MM. les intendants de provinces, quelque bien intentionnés qu’ils soient. Le crû de Paris, dont ils sont originaires, ce qui n’était pas autrefois, à beaucoup près, est fort peu propre à donner la connaissance d’un État, puisqu’on y peut posséder de très grandes richesses sans avoir un pied de terre, que l’on compte pour le dernier des biens, quoiqu’elle donne le principe à tous les autres ; l’on renferme ordinairement toutes ses attentions à l’égard de la campagne, en ces quartiers-là, à des embellissements et décorations de maisons de plaisance.\par
Ce dixième, encore une fois, est aussi aisé à trouver en ce royaume qu’ailleurs, quand on y emploiera les mêmes sujets qui agissent en ces contrées, et qui travailleront à leurs périls et risques, en sorte que MM. les ministres n’auront point la tête rompue des injustices qu’on y pourrait commettre. C’est un dixième en {\itshape argent} qu’il faut payer, et non point en {\itshape essence} ou {\itshape dîme royale} comme une personne de la première considération, tant par son mérite personnel que par l’élévation de ses emplois, a voulu proposer au roi, sur la foi d’un particulier qui en avait composé le projet, sans avoir jamais pratiqué ni le commerce ni l’agriculture, ce qui ne peut qu’enfanter des monstres.\par
En effet, il est inouï que l’on puisse établir ni trouver à donner à ferme une levée du dixième de toutes les denrées d’un village, sans donner un lieu pour les déposer, n’y ayant nul endroit du monde où il s’en trouve d’inutiles, puisqu’on n’a pas souvent moyen d’entretenir les plus nécessaires. De plus, l’obligation de bailler caution, comme pour les deniers du roi, de payer de trois mois en trois mois comme on fait la Taille, et de percevoir cette dîme sur les nobles et privilégiés qui en étaient auparavant exempts, sont des clauses qui font qu’il n’y a point d’habitant de la campagne qui n’aimât mieux donner de l’argent en pure perte, que de se rendre adjudicataire d’un pareil fermage, à la quatrième partie de sa juste valeur. De quoi on peut voir un exemple lors de la saisie des terres appartenant à des gentilshommes, puisque la régie est donnée souvent pour la dixième partie de sa juste valeur, sans que les créanciers puissent faire autrement, et sans que le saisi même use de violence pour ce sujet. Tous ceux même qui possèdent des dîmes dans des villages éloignés savent bien que, s’ils les proclamaient sans fournir de bâtiments, en ayant tous lorsqu’elles sont un peu considérables, et à condition de donner caution et de payer de trois mois en trois mois, sans nul quartier, ils n’en trouveraient quoi que ce soit, ou tout au plus que la dixième partie de la valeur précédente ; puisque, dispensant de toutes ces clauses, ils en perdent encore souvent la meilleure partie lors du dépérissement du prix des denrées comme aujourd’hui ; ce qu’un remplacement de tailles et d’autres impôts ne peut souffrir, puisque le paiement à l’échéance du terme est de rigueur, attendu que le maintien de l’État, qui ne souffre point de retardement, roule uniquement sur la levée des impôts.\par
On a fait cette reprise pour montrer que le rétablissement de la France n’a point deux manières, et qu’il n’y a uniquement que celle qui a été pratiquée en France dans tous les siècles, et dont l’usage a été reçu et l’est présentement dans tous les États du monde, qui est celui qu’on propose à titre, encore une fois, lucratif de la part des peuples ; car, bien que la Capitation, payée régulièrement à ce dixième par une fidèle exécution de ce système, atteindrait d’une manière constante à plus de cent millions, elle ne prendrait point assurément la cinquième ou la sixième partie des biens que le roi aura rétablis à ses peuples en un instant, sans que l’on craigne aucune objection à l’égard du déconcertement, et encore moins de la conjoncture ni de la brièveté du temps, qu’on ne fasse voir aussitôt être un renoncement à la raison et au sens commun ; en sorte qu’on maintient, comme on a déjà fait plusieurs fois, qu’il n’y a point d’homme assez abandonné de Dieu et de ses semblables pour oser mettre par écrit et souscrire de son nom des objections pareilles.\par
La réprobation des établissements que l’on combat, et l’exécration de leurs effets, qui sont publics, purgent ces expressions de tout soupçon de témérité et même d’extravagance ; ce qui serait, et l’auteur punissable corporellement, s’il n’avait pas tout un royaume pour témoin des vérités qu’il énonce : c’est le seul intérêt du roi et des peuples qui l’a conduit à les mettre au grand jour, avec d’autant plus de confiance, que l’intégrité de MM. les ministres, qui est aussi connue que les désordres que l’on combat, l’assure qu’il ne risque rien à leur égard, mais qu’il leur rend un très grand service.\par
Mais, pour anticiper les objections et pour épargner la peine de les faire à ceux qui voudraient y avoir recours, on soutient, d’abord, que l’on ne peut impugner tout le contenu de ce Mémoire, qu’en soutenant le mérite des trois articles combattus, et par conséquent leur maintien. Or, pour faire voir l’horreur d’un pareil rôle, il n’y a qu’à penser si on pourrait trouver un homme sur la terre assez dépourvu de sens et de raison, ou plutôt assez ennemi de Dieu et des hommes, pour qu’il osât dire publiquement qu’il est auteur d’aucune de ces trois dispositions.\par
En effet, quelqu’un aurait-il bien l’impudeur de tenir ce langage : « C’est moi qui suis cause de la mauvaise répartition des Tailles, en sorte que l’on ruine tout à fait les misérables, ce qui les met entièrement hors d’état de commercer et de consommer, par où les riches perdent six fois plus qu’il ne leur aurait coûté en prenant leur juste part de cet impôt, dont le désordre rejaillit sur les revenus du roi ? »Des blés, la même chose. Un homme bien sensé aurait-il le courage de dire : « C’est moi qui ai statué et établi qu’il faut que les grains soient à si bas prix, afin que tout le monde soit à son aise, que les fermiers ne puissent pas donner un sou à leurs maîtres, lesquels, par conséquent, ne font travailler aucuns ouvriers ? Et aussi, comme ce bas prix empêche de labourer les mauvaises terres pour n’en pouvoir supporter les frais, que cet abandon est un excellent moyen pour éviter les chertés extraordinaires dans les années stériles, et faire consommer les grains par les bestiaux, comme il arrive maintenant ? » À l’égard des Aides, Douanes et passages, ne faudrait-il pas renforcer d’effronterie ou d’extravagance pour se dire auteur de toute la manœuvre qui s’y fait, et publier qu’on a eu raison d’établir vingt-six déclarations à passer, ou droits à payer, sur un même lieu et pour un même prince, avant qu’une marchandise puisse être embarquée ; et qu’à l’égard des liqueurs, on a un juste sujet de payer dix mille personnes aux dépens du roi et du public, pour faire arracher la moitié des vignes du royaume, et obliger les deux tiers des peuples à ne boire que de l’eau ?\par
Voilà pour l’aveu de l’établissement : qu’on ne croie pas qu’il y ait personne qui puisse en réclamer l’honneur.\par
Pour le délai, sous prétexte de la conjoncture, qui est la ressource la plus ordinaire de la part des personnes intéressées au maintien de cet état de choses, l’extravagance et le renoncement à la raison n’y sont pas en un degré moindre, puisque chacun de ces articles, pris à part, apporte au royaume plus de préjudice qu’il n’en éprouve de tous les ennemis du roi, et que le principe qui produit tous ces désordres n’a, d’ailleurs, pas plus de rapport à la paix ou à la guerre, qu’à la vie ou à la mort du roi de la Chine : on ne peut donc user de pareils raisonnements pour retarder le remède, sans montrer qu’on ne craint ni Dieu ni les hommes.\par
D’autre côté, comme pour sortir de la conjoncture présente il faut des sommes très considérables, on maintient qu’il n’y a pas maintenant d’homme, si habile qu’il soit, dans le royaume, qui, mettant d’une part les charges ordinaires et indispensables de l’État, ainsi que le paiement des arrérages de tout ce qui est dû sous le nom du roi, et de l’autre ce que les revenus ordinaires peuvent fournir, puisse, non pas trouver les moyens de faire la balance égale, mais seulement ceux de parer à la quatrième partie du déficit que présentent les ressources de l’État, et qui voulût, surtout, hypothéquer sa fortune à la garantie du succès de ses expédients. En sorte donc que le combat est entre ces deux situations : l’auteur de ce Mémoire propose au nom des peuples, dont il n’est que l’avocat, des manières qui sont celles de toute la terre, que l’on ne peut contredire sans renoncer à la raison et se rendre ridicule ; et il a pour adversaires des gens qui veulent qu’on préfère une espérance fondée sur des moyens qu’ils auraient honte de proposer par écrit, et du succès desquels ils seraient très fâchés qu’on fît dépendre leur propre fortune.\par
Le seul et plus cruel ennemi enfin que ces dispositions ont à combattre, est que la base de ce grand rétablissement de biens aux peuples, qui les mettra en état d’en faire part au roi, roulant uniquement sur la cessation de manières établies et pratiquées avec applaudissement envers les auteurs, de la part seule, néanmoins, de sujets intéressés, flatteurs ou ignorants, il s’ensuit une conséquence très fâcheuse, savoir, que cette destruction ne peut être un grand bien qu’autant que l’admission de ce qu’on renverse était un très grand mal, et également la ruine du roi et des peuples. Or, un pareil énoncé n’est guère un langage de courtisan. Mais, comme MM. les ministres d’aujourd’hui n’y sont pour autre chose que pour avoir trop agi sur la foi de leurs prédécesseurs, ayant jugé d’autrui par eux-mêmes, et supposé autant d’intégrité dans les autres que celle qui les caractérise, la reconnaissance de la surprise, loin d’intéresser leur réputation, leur pourra, au contraire, procurer beaucoup d’honneur aux dépens de ceux qui leur ont légué un si déplorable système.\par
Et tout compté, c’est un marché bien avantageux de se tirer de l’état actuel par un rétablissement entier de la richesse des peuples, qui attire celle du roi après elle, et par conséquent le paiement de ses dettes, comme du temps de M. de Sully. Mais quelque utilité qu’il en vienne au royaume, et quelque modique prix que l’on exige pour un si grand bien, on n’obtiendra jamais le consentement de gens à qui un bouleversement général est bien moins sensible qu’une ruine singulière de l’espoir d’amasser de la fortune, ou la crainte de perdre une réputation très mal acquise, dont ils tiraient le même profit que s’ils l’avaient très bien méritée. Comme ce n’est pas là, encore une fois, à beaucoup près, l’espèce de MM. les ministres, on est persuadé qu’ils regarderont avec bonté un travail qui n’a eu d’autre objet que de rendre service au roi, au public et à eux-mêmes, d’autant plus qu’ils ajouteront, par leurs grandes lumières, ce qui pourrait manquer à la perfection de ces Mémoires ; par où on les finit, avec une forte persuasion, fondée sur l’idée générale de tous ceux qui en ont pris connaissance, que l’auteur s’est acquitté de ce qui était porté dans le titre de son ouvrage.\par
Et pour dernière preuve, physique et incontestable, de la vérité de tout ce Détail, c’est que celui qui l’a composé se dit publiquement auteur de quatre-vingts millions de hausse d’exigences sur les peuples, et en attend des remerciements, à cause des conditions qui l’accompagnent ; pendant que ceux qui le voudraient contredire, ou proposer de bien moindres sommes, exigibles par les méthodes usitées, n’oseraient ni se découvrir, ni se déclarer les auteurs de pareils projets. La raison de l’une et de l’autre conduite est très sensible, puisque par la première l’auteur ne se propose que de faire payer la cinquième partie de ce qu’on aura rétabli de biens aux contribuables ; et que, par l’autre, il faudrait exiger l’impossible, ce qui n’est pas sans exemple, ou plutôt ce qui n’en a que trop eu par le passé.\par
Et comme il est inouï de demander aux peuples ce qu’ils ne sauraient payer, il leur serait également criminel de refuser à leur prince, pour ses besoins, une partie des facultés dont il les aurait remis en possession. Pour à quoi parvenir, on maintient à la face de toute la terre, sons crainte encore une fois d’être contredit par écrit, qu’il ne faut point trois heures de travail de la part de MM. les ministres, et quinze jours d’exécution de celle des peuples, parce qu’il ne s’agit que de cessation d’une très grande violence, comme au siège de La Rochelle.\par
Les blés de Barbarie, dès qu’ils seront exclus de la Provence, redonneront au Languedoc six fois cette hausse d’impôt, et à la Provence même. Si cette province achète les grains plus cher, n’en croissant que très peu chez elle, elle y regagnera au triple par la vente, augmentée et de prix et de quantité, de ses huiles, olives, raisins et figues sèches, que l’on sait souvent y être à rebut, et qui ne sont en ce misérable état que parce que les provinces où les blés servent de contre-échange pour se procurer le reste, sont mises hors de ce pouvoir par leur avilissement. Cet établissement des blés de Barbarie ne peut tout au plus être bon que dans des temps de stérilité ; mais, par la continuation ordinaire, il n’y a rien de si préjudiciable ; et ce maintien continuel n’est même que l’effet de l’intérêt singulier et personnel des munitionnaires qui, pour gagner sur leurs marchés, en faisant leurs magasins à meilleur compte, se mettent peu en peine du bien général du roi et des peuples : joint à cela l’utilité particulière des entrepreneurs, qui se conservent dans ce commerce par de la protection achetée à prix d’argent.\par
Et cette faute contre la politique, d’admission de blés étrangers hors le temps de stérilité, surtout dans un pays fécond comme la France, est si grossière, que, outre l’exemple de l’Angleterre, qui achète le contraire à prix d’argent, c’est-à-dire la sortie des grains, l’Espagne, qui, par l’abandon presque continuel de la culture de la plus grande partie de ses meilleures terres, semblerait être fort excusable de la commettre, attendu que la cherté y est plus ordinaire que le prix raisonnable, connaît toutefois si bien, dans les années d’abondance, l’horrible inconvénient d’avilir une denrée de cette nature, que, depuis l’union des deux monarchies en la royale maison de Bourbon, elle a prié qu’on ne lui en apportât pas dans ces occasions, quoiqu’il y eût à gagner pour le menu peuple, à parler le langage erroné qui règne en France depuis si longtemps. Ainsi, on maintient qu’il n’y a point de muid de blé, refusé de la Barbarie, qui n’en fasse croître cent d’augmentation dans le royaume, par les raisons marquées et connues de tous les laboureurs, mais qui sont lettres closes pour la spéculation, seule cause de cette surprise ; et, indépendamment encore de cette augmentation de cent pour un dans la production, ce sera la même crue dans le revenu, n’y ayant pareillement aucun de ces muids, bannis de la Provence, et par suite de la France, qui ne procure pour sa part quatre mille livres de revenu, par les mêmes principes.\par
Enfin, pour dernière période de ce Mémoire, on soutient que les peuples ne pouvant payer rien au roi que par la vente de leurs denrées, et le prince étant en état de doubler en un moment cette même vente, par la cessation d’une violence qui en a anéanti ou suspendu plus de la moitié ; il est de la dernière extravagance de traiter de visionnaires ceux qui viennent annoncer que le roi peut également doubler les tributs, non seulement sans ruiner personne, mais en enrichissant tout le monde. Car l’augmentation du prix des denrées fait celui des terres, qui seules font vivre tous les états, depuis le plus élevé jusqu’au plus abject. Et le laboureur, enfin, cultive pour lui et toutes les autres conditions ; et il leur fait part au sou la livre du bien et du mal qu’il souffre dans son commerce ou sa vente ; quoique ce soit là précisément la chose du monde que les pauvres conçoivent le moins, ainsi que les personnes de spéculation, remplies de charité, qui se laissent abuser par la voix de gens qui raisonnent moins bien que des bêtes, lorsqu’ils opinent par emportement, comme l’on a déjà dit, et sans connaissance de cause, de quoi ils ne sont pas capables.\par
Les quatre généralités, soulagées du côté des Aides, feront revivre sur-le-champ les provinces mitoyennes du royaume, qui recommuniqueront incontinent le même bien aux contrées les plus éloignées ; en sorte que la Capitation au dixième des biens ne sera pas le quart de ce qu’elles auront gagné à ce marché.
\section[{Chapitre XII.}]{Chapitre XII.}
\noindent Pour récapituler tout ce Mémoire, on maintient que le roi est en pouvoir de se rétablir, quand il lui plaira, trois cents millions de revenu réglé, comme du temps du roi François I\textsuperscript{er}, non en usant de contraintes, ni d’exécutions contre les peuples, comme il a été fait ; mais en les remettant en possession de leurs facultés tout entières, de la moitié desquelles, s’élevant à plus de quinze cents millions, ils ont été privés par des manières enfantées uniquement par le crime, ainsi qu’on a fait voir, et continuées par surprise depuis 1660.\par
Pour ce sujet, il est nécessaire que le roi regarde la France et toutes ses richesses comme à lui uniquement appartenantes, et qu’il considère tous les possesseurs comme ses propres fermiers ; enfin, qu’il se persuade que tout ce qui les incommode dans leur labourage, dans leur commerce et dans leur trafic, est la même chose que si le dommage lui était fait personnellement dans quelques fonds qu’il peut posséder en certaines provinces du royaume.\par
Or, du moment qu’il y a une infinité d’établissements pour tirer des peuples ses diverses redevances, dont les frais se prennent avant tout, pendant que l’embarras qui accompagne la levée anéantit vingt fois autant de biens que l’on en fait toucher au prince ; n’est-il pas constant que c’est comme si le mal était fait à lui-même, et que par conséquent la cessation de ce mal, qui peut avoir lieu sans délai, enrichissant ses peuples, c’est une opulence personnelle que l’on lui procure ?\par
On demande volontiers à tous les contredisants, qui ne peuvent être que la nation qui vit et qui s’enrichit de la ruine du roi et des peuples, si des dix mille genres d’impôts qu’il y a aujourd’hui en France, levés par le canal des Traitants et Partisans, avec les circonstances connues et marquées, il y en a un seul dont le fonds ne soit fait et ne s’exige pas d’un {\itshape Taillable} ou d’un homme {\itshape sujet à la Capitation} ; ces deux impôts enfermant également les Nobles, Bourgeois et Roturiers, c’est-à-dire tous les hommes du royaume. — De manière qu’évaluant tout ce que le roi reçoit par ces dix mille canaux, qui donnent de l’emploi à plus de 100 000 hommes, et le remettant sur la Taille et sur la Capitation, voilà tout d’un coup cent mille paies à 1 000 liv. chacune par an, qui est bon marché ; c’est-à-dire 100 millions de gagnés pour le roi et ses peuples. Ce qui n’est que la moindre partie de l’utilité, puisque la plupart des anéantissements de biens causés par ce ministère, revivraient sur-le-champ, au profit de ces peuples, et par conséquent du prince. Car de croire que le canal d’un Partisan fasse trouver du bien où il n’y en a point, lui n’ayant rien ordinairement, c’est renoncer à la raison, et imposer à la foi publique, qui sait que c’est justement le contraire ; et que sa main seule, comme le feu, consume l’objet auquel elle s’attache.\par
Pour montrer cette vérité plus clairement en un seul article, il n’y a que des taillables qui nourrissent les bestiaux, dont les boucheries des villes sont fournies. Or, n’y a-t-il pas des Traitants, bureaux et commis pour leur entrée dans ces mêmes villes ? ne s’en trouve-t-il pas sur le débit de la viande et du suif ? n’y en a-t-il pas pareillement sur les laines qui en proviennent, sur les draps qui en sont construits, sur leurs passages et sorties ? Ce n’est pas tout : les cuirs, qui partent du même principe, n’ont-ils pas semblablement leurs impôts à part, et jusqu’à quatre ou cinq, pour peu qu’ils fassent du chemin en se mettant en route ? — Tous ces frais et préciputs doivent être payés et portés par le maître du mouton, savoir, un taillable ou payeur de capitation, qui l’a nourri et élevé ; lequel étant le fermier du roi, c’est la même chose, par contrecoup, que si on faisait sortir ces sommes de la bourse ou de la libéralité du prince ; qui est le moindre désordre, ce que l’on ne saurait assez répéter, puisque le Néant en tire encore dix-neuf fois davantage que ces appointements ; et pour le faire voir, on maintient qu’il n’y a pas aujourd’hui dans le royaume la quatrième partie des bestiaux qu’il s’y trouvait il y a quarante-cinq ans ; ce qui apporte un pareil déchet à la culture des terres, qui n’est bonne et mauvaise qu’à proportion des troupeaux qui paissent dessus.\par
La même chose des vins : les dix ou onze bureaux qui se rencontrent sur les liqueurs, avec la paie et la fortune des Traitants, doivent être portés avant tout par un homme taillable ou sujet à la capitation. Or, en recevant immédiatement des mains de ceux-ci ce qui revient au prince par ce malheureux cérémonial, c’est une richesse immense pour eux comme pour le monarque, et une cessation de misère pour tous les peuples. Car, la conséquence de cette effroyable économie, c’est tout simplement que, outre la ruine des laboureurs et vignerons, plus de la moitié des peuples des grandes villes, surtout de deçà la Loire, et même de Paris et des campagnes, ne mange point de viande, et ne boit que de l’eau ; ce qui diminue la plus grande partie de leurs forces, et par conséquent leur travail.\par
C’est le même raisonnement sur l’impôt des bois, sur le charbon, sur le foin, sur la volaille, sur les œufs, sur le beurre, sur le poisson, sur le tabac, et enfin sur toutes les autres denrées, n’y en ayant presque aucune d’exempte ; on trouvera mêmes bureaux, mêmes commis, mêmes Traitants, même paie, ou plutôt même fortune, et mêmes anéantissements à essuyer par des taillables ou payeurs de capitation, qui seraient prêts de racheter au triple ce qui revient au roi de ces horribles manières, et même avec quadruple profit de leur part.\par
Que l’on ne traite point ceci de vision, c’est une pure réalité, et le contraire ne peut être soutenu sans extravagance, et sans montrer que l’on ne craint ni Dieu, ni les hommes ; tandis que ce qu’on propose se réduit à demander qu’on administre la France comme le sont tous les autres États du monde, et comme elle l’a été, même, jusqu’à la mort du roi François I\textsuperscript{er}. On se borne, en un mot, à supplier MM. les ministres de vouloir bien enrichir, du même coup, les peuples et le roi.\par
Il n’y a point, en effet, de fermier tenant des terres à louage, qui ne soit content de hausser le prix de son fermage, si on lui augmente le produit du terrain de sa ferme. Que l’on fasse une convocation de cent laboureurs, bourgeois ou marchands, de toutes les contrées du royaume, il n’y en a pas un qui ne convienne, pourvu qu’on ne les ait pas corrompus pour les faire parler contre leur conscience, de payer quatre fois sa capitation, et même par avance, pourvu qu’ils soient déchargés de tous ces malheureux préciputs, qui n’ont été inventés que pour ruiner le roi et les peuples, et enrichir les entrepreneurs.\par
Et pourtant, ce qu’il y a d’effroyable est que, dans la conjoncture présente, où la France a besoin de toutes ses forces pour se défendre de tant d’ennemis, on a pris justement le contrepied, entassant tous les jours Traitant sur Traitant, avec les circonstances marquées, c’est-à-dire vingt de perte sur le fonds, pour un de profit au roi. Quoiqu’il n’y ait que de la surprise de la part de MM. les ministres, depuis 1660 seulement, on ne laisse pas de dire, lorsqu’on propose de cesser de pareilles manières, qu’on veut renverser l’État, comme si l’État consistait, ainsi qu’on l’a déjà dit, dans ceux qui ruinent ses terres et le commerce, par conséquent le roi et ses peuples ; mais comme c’est justement le contraire, et que la nation que l’on combat est la plus grande ennemie du royaume, on doit regarder avec horreur les effroyables allégations que l’on veut renverser l’État, lorsqu’on parle de faire cesser la plus grande désolation qui fut jamais. Ne faudrait-il pas, au contraire, tomber d’accord qu’on, veut procurer un grand loisir à MM. les ministres et au Conseil des finances, qui ne sont occupés aujourd’hui, depuis le matin jusqu’au soir, qu’à diriger et combattre des monstres qu’on n’aurait jamais dû établir ? Et, bien que cela se fasse avec la dernière intégrité de leur part, il s’en faut beaucoup que ce soit la même chose dans le sous-ordre et les secondes mains, dont le nombre est infini ; car il n’y a de parti, quelque borné qu’il soit, qui ne forme des profits indirects à plus de cent personnes, lesquelles, sans être Traitants, joignent leur voix pour dire qu’on veut renverser l’État.\par
Comme les maux se guérissent par le contraire de ce qui les avait produits, à mesure que le roi aura besoin de secours, il n’aura qu’à en user avec ses peuples comme le propriétaire de ferme dont nous avons parlé tout à l’heure, qui hausse sans difficulté le fermage de son locataire, parce qu’il augmente, en même temps, l’étendue de son exploitation. Le roi peut en toute sûreté dire à ses peuples : « Vous me paierez tant de hausse de taille et de capitation, parce que je vous supprime tel et tel parti qui vous coûtait dix fois davantage ; ainsi vous gagnerez quatre fois plus que moi à ce marché. » Mais on ne prendra pas ce parti tant que l’on consultera la nation dont on vient de parler, à qui la destruction du royaume serait bien moins sensible que celle de sa fortune, comme cela s’est vérifié toutes les fois que le cas est advenu. Toutefois, comme ce n’est pas là le caractère de MM. les ministres, qui sont très intègres, quoique très surpris, on espère quelque succès de la nécessité des conjonctures qui ne permettent pas d’employer tout autre remède pour le salut de l’État. D’autant plus que l’on fait une espèce de transaction avec les destructeurs du royaume, en se contentant de leur demander quelques adoucissements, qui rétabliront sans délai assez de facultés aux peuples, avec profit de leur part, pour fournir au roi les 80 millions de hausse dont il a besoin ; et qui seront encore une preuve certaine que la destruction entière du mal mettra plus tard le royaume en état de donner au roi trois cents millions, comme du temps de François I\textsuperscript{er}.\par
L’erreur où l’on a été jusqu’ici à l’égard de l’argent, le regardant comme le principe de richesse, ce qui n’est qu’au Pérou, ne peut être alléguée après la lecture du chapitre qu’on en a fait, où l’on montre qu’il est uniquement l’esclave de la consommation, suivant pas à pas sa destinée, et marchant ou s’arrêtant avec elle, un écu faisant cent mains en une journée, lorsqu’il y a beaucoup de ventes et de reventes, et demeurant des mois entiers en un seul endroit, lorsque la consommation est ruinée, comme il arrive à présent ; d’où il suit qu’étant possible de rétablir cette consommation pour plus de cinq cents millions en un instant, ce sera autant de marche d’argent, et non point de nouvelles espèces remises sur pied ; par où le prétendu ridicule d’une hausse si subite de revenus est amplement purgé et rejeté sur les contredisants, qui ne pourront pas tenir, lorsque l’autorité, qu’ils ne doivent qu’à l’erreur de MM. les ministres, leur manquera, leur système n’ayant pu se maintenir jusqu’à ce jour que comme celui de l’{\itshape Alcoran}, c’est-à-dire par la défense de parler contre, et la menace d’être empalé sans rémission si l’on désobéissait. Du reste, en tout ceci l’on n’a été que l’organe ou l’avocat des peuples ; et on craint si peu d’en être désavoué, que l’on se soumet d’apporter la signature de cent mille hommes, ayant tous chacun dix mille écus de bien l’un portant l’autre ; c’est donc un marché sans peur et sans péril, qui ne peut être refusé que par ceux dont on a parlé.\par
Et pour finir, comme l’a marqué le titre de ce Mémoire, on maintient qu’il n’y a point d’homme sur la terre qui puisse faire une objection, sous quelque prétexte que ce soit, à la levée de quatre-vingts millions, qui ne sera que la cinquième partie de ce qu’on aura rétabli par trois heures de travail au peuple, sans un ridicule complet, et sans être en horreur à Dieu et aux hommes, pendant que cette offre est, au contraire, comblée de bénédictions. Comme aussi, l’on prétend toujours qu’il est pareillement impossible d’établir d’une autre façon le quart de la fourniture des besoins du roi dans la conjoncture actuelle, et qu’il n’y a personne au monde qui voulût être garant de la réussite de la moindre partie ; par où l’on peut voir, avec quel fondement on peut rejeter le parti qu’on offre, pour tabler sur un autre si dépourvu de certitude, dans une occasion où il ne se faut pas méprendre.\par
Enfin, l’auteur de ces Mémoires les présente au public à une condition, qui ne lui sera point enviée par les contredisants, savoir, celle qui était pratiquée par les Athéniens. Ce peuple avait établi que tout porteur de nouveaux règlements serait tranquillement écouté, quel qu’il fût ; mais qu’il fallait commencer par avoir une corde au cou, afin que, si l’exécution, loin de se trouver avantageuse, se trouvait dommageable à l’État, l’auteur fût étranglé immédiatement. Si la France en avait usé de la sorte il y a cent cinquante ans, lorsque les Italiens jetèrent la première semence des manières qui l’ont réduite en l’état où elle se trouve aujourd’hui, le roi aurait, certainement, deux cents millions de revenu réglé plus qu’il n’a aujourd’hui, et ne devrait pas un sou, parce qu’il y aurait deux cents mille édits ou déclarations et dix mille genres d’impôts de moins, tous venus depuis ce temps : le sort porté par les lois des Athéniens, arrivé au premier inventeur avec justice, aurait tari tout à fait la source de pareilles entreprises. Mais, loin de cette destinée, il y a eu deux cent mille fortunes obtenues par où il n’échoyait qu’une corde à Athènes, ce qui a produit au gouvernement un sort tout contraire ; sa destruction, par le défaut de cette sauvegarde, a été érigée en plus court moyen de se procurer la plus haute opulence. La France a vu ruiner entièrement, par ces porteurs de nouveautés, son commerce et la culture de ses terres ; et plus de la moitié du royaume devenir inutile au peuple, et par conséquent au prince ; sans parler de la destruction des sujets et de la fécondité des familles, suite nécessaire de la désolation de l’agriculture.\par
Et pour faire voir, par un parallèle, ce que serait la France si ce système n’avait pas enrayé, en quelque sorte, le progrès de la force et de la richesse de l’État, on rappellera que la Judée, du temps de la plus grande puissance de ses rois, n’a jamais possédé qu’un territoire de 70 lieues de long sur 25 de large, c’est-à-dire dix fois moins grand que celui de la France : cependant ses monarques, au rapport de l’Écriture sainte, mettaient sur pied des armées de seize cent soixante-dix mille hommes. Et, comme les gens propres à porter les armes ne font pas la cinquième partie de la population d’une contrée, les vieillards, les indisposés dans leur corps, les femmes et les enfants, formant au moins les quatre autres, c’est près de neuf millions de créatures que ce pays contenait et nourrissait ; c’est-à-dire, sur le pied de cent millions en France, qui pourraient y subsister, si les circonstances étaient égales. Et il ne faut point faire de reprise sur la fécondité de la Judée, qui n’était autre chose que le nombre et le travail de ses habitants, puisque aujourd’hui, que les choses ont bien changé, par les ravages qu’elle a soufferts, il n’y a pas cent mille âmes dans cette contrée, et que le terroir y paraît naturellement très mauvais ; et sa fertilité, vantée dans l’Écriture, n’était donc que l’effet de ce nombre et de ce travail, de même que l’habitation commode des Barbets dans les Alpes.\par
On a fait cette observation pour montrer la possibilité où était la France de fournir au roi François I\textsuperscript{er} sur le pied de trois cents millions de rente, n’ayant point les entraves qu’elle a souffertes depuis, et qui l’ont énervée de plus de la moitié ; ce qui est une garantie certaine, pareillement, de la facilité qu’elle aura de se rétablir dans son état naturel, lorsque les causes violentes qui la réduisent en cette pitoyable situation auront disparu, comme cela peut avoir lieu en un moment, ainsi que dans toutes les occasions où la nature souffre violence, suivant le principe des philosophes : que {\itshape tout ce qui est violent ne peut durer}. Ce qui forme une espèce de certitude de voir bientôt rétablir le royaume, les maux comme les biens ayant leur période, après l’expiration duquel il faut une révolution qui remette les choses au premier état, surtout les biens ; et les cœurs des peuples étant toujours disposés à bien faire, du moment qu’on les met en pouvoir de donner cours à leur bonne volonté, ce qui est, par malheur, le contraire de la marche suivie, depuis bien longtemps, jusqu’à ce jour.\par


\begin{raggedleft}FIN DU FACTUM DE LA FRANCE.\end{raggedleft}
\chapterclose


\chapteropen
\chapter[{Traité de la nature, culture, commerce et intérêt des grains}]{Traité de la nature, culture, commerce et intérêt des grains}\renewcommand{\leftmark}{Traité de la nature, culture, commerce et intérêt des grains}

\begin{center}\emph{Tant par rapport au public, qu’à toutes les conditions d’un État ; divisé en deux parties, dont la première fait voir que plus les grains sont à vil prix, plus les pauvres, surtout les ouvriers, sont misérables ; et la seconde, que plus il sort des blés d’un royaume, et plus il se garantit des funestes effets d’une extrême disette.}\end{center}

\chaptercont
\section[{Considérations préliminaires.}]{Considérations préliminaires.}
\noindent Bien que l’agriculture ait été dans les premiers temps l’occupation des personnes les plus élevées, puisque les enfants de David, au rapport de Josèphe, invitaient leurs amis à la toison de leurs troupeaux, et que Tite-Live raconte que, dans l’ancienne Rome, on allait prendre les sénateurs à côté de leur charrue\footnote{{\itshape A villâ in senatum senatores accersebantur.}} ; les choses ont bien changé depuis, car ce qui était un honneur est devenu une espèce de dérogeance à toutes sortes de mérites ; et on peut dire aujourd’hui, en France, qu’on laisse aux derniers des hommes la commission de nourrir et de faire subsister tous les autres.\par
Bien qu’il se rencontre des laboureurs dans toutes les conditions, il faut qu’un homme, avant que de s’y appliquer, soit estimé, et de lui et de tout le monde, incapable de rien faire de plus relevé que cette profession, qui passe pour la dernière de toutes, pendant qu’elle aurait besoin d’un mérite distingué, et qui se composât de l’assemblage d’une longue pratique jointe à une étude très sérieuse, pour porter les choses à la perfection nécessaire à la commune utilité de tous les peuples.\par
Il y a même quelque chose de plus : non seulement la spéculation et la pratique ne se sont jamais trouvées réunies pour ce fait en aucun sujet, mais on peut dire même qu’elles en ont toujours été séparées par une si grande distance, qu’il y a plus de commerce entre les peuples d’un hémisphère à l’autre, qu’il ne s’en rencontre aujourd’hui entre les personnes qui ont la simple spéculation du labourage et celles qui le pratiquent actuellement. Cependant, la dispensation des fruits qui en viennent étant entièrement entre les mains de ceux qui n’en ont que la théorie, c’est-à-dire qui en ignorent absolument les véritables intérêts, sans que les autres y aient aucune part, quand même il se rencontrerait des sujets propres à réfléchir sur la pratique (ce qui est très rare), il en est arrivé le même désordre que lors de la construction de la fameuse tour de Babel : les ouvriers ne savaient plus ce qu’ils faisaient, ou plutôt pratiquaient le contraire de ce qui eût été nécessaire pour la perfection de l’ouvrage ; non qu’ils eussent perdu le sens, mais parce que, par un effet de la Providence, étant venus en un moment à parler différents langages, ils ne s’entrentendaient plus, ce qui produisait nécessairement une confusion inexprimable.\par
On maintient donc que la même chose est arrivée en France depuis quarante ans, à l’égard des blés, et que si on les a vus, depuis ce temps, ou à un prix excessif plusieurs fois, ce qui a fait périr une infinité de monde, ou dans un avilissement effroyable, ce qui ruinait également et les riches et les pauvres, c’a été par un malentendu, ou une mésintelligence continuelle entre la pratique et la spéculation à leur égard, puisque la réunion de ces deux genres de connaissances n’eût pas manqué d’empêcher ces deux extrémités, et de les compenser l’une par l’autre, comme il se pratique dans tous les États de l’Europe, et comme on a fait même en France pendant plusieurs siècles avant 1660. Il se rencontre à la vérité des ordonnances contraires, mais elles avaient été faites dans des temps durs et de nécessité, et la pratique en avait été négligée dans la suite, comme il est aisé de s’en convaincre ; et si l’on s’en servait, ce n’étaient guère que des gouverneurs, qui en tiraient sous main des rétributions, pour faire semblant de ne pas voir les enlèvements.\par
C’est pour faire cette paix et cette réunion, que l’on a cru plusieurs années bien employées à la pratique, ainsi qu’à la spéculation du labourage et du commerce, qui en est une suite nécessaire, dont l’effet a été de comprendre invinciblement, et de se mettre même en état de le persuader aux autres, qu’il n’y a qu’un moyen d’éviter les deux extrémités dont on vient de parler, également dommageables à un État, qui est de maintenir si fort la balance égale entre ces deux inconvénients, que se remplaçant, ou se compensant continuellement l’un l’autre, il s’en forme un tout permanent, qui partage également les blés à toutes les années, comme fait un père équitable le pain à ses enfants.\par
Or, il n’y a qu’un moyen, qui est celui que l’on a marqué au commencement de ce Mémoire, savoir qu’on ne peut éviter les désordres d’une extrême cherté qu’en laissant libre en tout temps, sans aucun impôt, hors les cas extraordinaires, l’enlèvement des blés aux pays étrangers ; pendant que de l’autre côté l’excès de l’avilissement de cette même denrée, qui n’est guère moins dommageable, s’il ne l’est pas autant, quoique l’on pense le contraire, parce qu’il fait moins de bruit, ne peut être garanti qu’en ne souffrant jamais l’anéantissement des grains, qui est une suite certaine du bas prix, et par conséquent une marque évidente d’une cherté future et prochaine, ainsi que l’expérience n’a que trop fait voir, et que l’on montrera encore plus dans la suite.\par
Pour se résumer donc après ce préambule, que l’on a cru nécessaire, on soutient, comme on l’a fait au commencement de ce Mémoire, que le peuple ne sera jamais moins riche ni plus misérable que lorsqu’il achètera le blé à {\itshape vil prix}. Ce sera la première partie ; et la seconde, que l’on ne peut éviter une extrême cherté de temps à autre, pour ne se pas servir d’un mot plus violent, qu’en vendant {\itshape toujours} des blés aux étrangers.\par
Ces deux propositions feront peut-être traiter d’abord l’auteur, comme le fut Christophe Colomb, et peut-être d’une manière plus violente, puisque si celui-ci passa pour un extravagant quand il exposa ses idées sur l’existence d’un nouveau-monde, l’auteur de ces Mémoires mériterait les noms de bourreau et de traître à la patrie s’il était dans l’erreur ; mais on espère que l’on ne courra aucun risque jusqu’à l’entière lecture de cet ouvrage ; et même, pour ne pas s’exposer un seul instant à un sort semblable, on dira, par anticipation, qu’on ne fait autre chose que de proposer de suivre l’exemple de la Hollande et de l’Angleterre, où le peuple disposant de son destin, au moins à l’égard de la subsistance, pratique exactement les conseils que l’on vient aujourd’hui donner à la France.
\section[{Première partie,}]{Première partie,}
\noindent Où l’on fait voir que plus les grains sont à vil prix, plus les pauvres, surtout les ouvriers, sont misérables.\par
\subsection[{Chapitre I.}]{Chapitre I.}
\noindent Tous les biens de la France, ainsi que de tous les autres pays, et dont elle est mieux partagée qu’eux, consistent, généralement parlant, en deux genres, savoir : les fruits de la terre, qui étaient les seuls dans la naissance, ou plutôt l’innocence du monde, et les biens d’industrie, ce qui se réduit encore aux quatre sortes d’espèces suivantes : ces mannes de la terre ; et la propriété des fonds qui les font naître, et qui en partage le profit entre le maître et les fermiers, qui est la seconde espèce ; la troisième est formée par le louage des maisons de villes, les rentes hypothéquées, les charges de robe, d’épée et de finance, l’argent et les billets de change ; et la quatrième, enfin, consiste dans le travail manuel, et le commerce tant en gros qu’en détail. Ces trois dernières espèces tirent, d’abord, leur naissance et leur maintien des fruits de la terre, puisque où il n’en croît point, comme sur les sables ou sur les rochers, ils y sont tout à fait inconnus ; mais ce n’est que la première fois qu’ils lui ont gratuitement cette obligation ; car, incontinent après, il faut que ces trois autres sortes de biens redonnent l’être à ces mêmes fruits dont ils tirent leur origine, et que cette circulation ne soit jamais interrompue d’un seul moment, parce que la moindre cessation devient aussitôt mortelle à toutes les deux parties, de quelque part que cela arrive.\par
En effet, les fruits essentiels, et comme capitaux, que produit la France, consistant en blés, ce qui en fait la première et plus considérable partie ; en liqueurs, comme vins, cidres et eaux-de-vie ; en bestiaux, qui forment les chairs et les laines, et en toiles, jamais le laboureur n’élèvera et ne nourrira sur la terre ces quatre denrées, et toutes les autres en très grand nombre qui en sont une suite, si les trois autres états de biens dont on a parlé ne les lui achètent à un prix qui soit au-dessus des frais qu’il lui a fallu faire pour les mener en leur perfection ; comme en même temps, il faut absolument que le laboureur et son maître, qui ne sont qu’une seule et même chose, et ne forment qu’un intérêt commun, achètent de toutes les professions de la vie, ainsi que de tous ceux qui vivent du travail manuel ou du commerce, au nombre de deux cents de compte fait, une partie au sou la livre de ce qu’ils leur peuvent fournir ; et à un prix, pareillement, qui les mette hors de perte, afin que le tout soit réciproque. Ce n’est pas tout : il est encore nécessaire que toutes ces deux cents professions trafiquent aussi mutuellement, par un commerce continuel, du produit de leur art, le tout au niveau des fruits de la terre, et surtout des blés, auxquels elles doivent toutes leur naissance, comme on l’a dit ; parce qu’aucune ne peut être démontée, sans faire aussitôt part de son mal à toutes les autres professions, quelles qu’elles soient, ou immédiatement, ou par contrecoup, attendu qu’elles forment toutes comme une chaîne d’Opulence, qui n’a de prix que par l’assemblage des anneaux dont elle se compose, et qui perd sa valeur, ou, du moins la plus grande partie de sa valeur, dès qu’on en a détaché un seul.\par
De manière que, pour entretenir l’harmonie sur laquelle roule toute la consistance des peuples et des États, et par conséquent les revenus du prince, il ne faut point qu’une partie paisse l’autre ; c’est-à-dire qu’il est nécessaire que la balance soit si égale dans tous ces commerces, que tout le monde y trouve pareillement son compte ; ou bien, il arrivera infailliblement, comme lorsqu’on vend à faux poids ou fausse mesure, que c’est une nécessité qu’un des commerçants soit bientôt ruiné.\par
Par tous ces raisonnements, il est aisé de voir que, pendant que chaque homme privé travaille à son utilité particulière, il ne doit pas perdre l’attention de l’équité et du bien général, puisque c’est de cela qu’il doit avoir sa subsistance ; et qu’en les détruisant un moment à l’égard d’un commerçant avec qui il trafique, quoique par l’erreur commune, et par la corruption du cœur, il croie avoir tout gagné, il doit au contraire s’attendre, si cette conduite devenait générale, comme il arrive quelquefois, à en payer la folle enchère par sa propre destruction qu’il se bâtit par là dans la suite, ainsi qu’on le va faire voir. Cependant, tout le travail des hommes, depuis le matin jusqu’au soir, est de pratiquer justement le contraire ; et il n’y en a aucun qui ne fût content, en achetant la marchandise d’un autre, de l’avoir non seulement à perte de la part du vendeur, mais encore tout ce qu’il a vaillant par-dessus le marché, tant l’intérêt aveugle les hommes ; en sorte que, si une autorité supérieure, et générale, n’intervenait pour arrêter cette avidité à l’égard des denrées absolument nécessaires, comme les grains, en y mettant le taux, il y a des hommes assez inhumains pour ne vouloir sauver la vie à leurs semblables, dans des occasions pressantes, qu’au prix de tout leur bien. Et, comme cette police ne peut pas être égale dans le détail, il faut y suppléer, d’une façon indirecte, en empêchant, par une autorité puissante, qu’une marchandise ne devienne la proie et la victime de l’avidité d’un commerçant, lequel serait content, si cela était à sa disposition, de sacrifier tout à son intérêt individuel, sans souci aucun de la religion et de l’humanité, qui sont bannies de toutes opérations de ventes et d’achats, parce qu’on croit avoir satisfait à Dieu et aux hommes en n’usant point de fraude et de supercherie, et ne faisant que profiter de la nécessité des circonstances.
\subsection[{Chapitre II.}]{Chapitre II.}
\noindent Ce que l’on vient de marquer dans le chapitre précédent se vérifie avec certitude, à l’égard des blés, de deux manières opposées, quoique le faux zèle n’en reconnaisse qu’une, savoir le prix excessif des grains, qui fait constamment périr une infinité de misérables, comme on n’en a que trop fait expérience, ayant toujours été regardée comme un fléau dont Dieu se sert pour punir les péchés des hommes. Mais, de soutenir que l’excès qui lui est opposé, savoir le grand avilissement de ces grains par rapport au prix des autres denrées, ne soit pas un mal aussi violent, et qui n’ait pas d’aussi funestes résultats, quoiqu’il ne fasse pas tant de bruit ; c’est ignorer absolument ce qui se passe dans le monde, et n’avoir qu’une simple spéculation du détail du labourage, et du commerce de l’agriculture.\par
Pour venir d’abord au fait, on demande à ceux que le zèle aveugle, et met dans la disposition de souhaiter toujours des blés à bas prix en faveur des pauvres, s’ils croiraient leurs vœux accomplis dans toute leur plénitude, au cas que l’on pût revoir cette denrée de grains au même taux qu’elle était en 1550, savoir, le setier de Paris pesant 240 livres ou environ, à 20 sous ou 21 sous année commune. Comme il n’y a point d’ouvrier de campagne qui gagne moins de sept à huit sous par jour, ce qui double dans les mois de récolte, et qu’une ferme ou terre, du rapport de 200 setiers de blé, a besoin de cinq ou six de ces ouvriers pendant tout le cours de l’année pour la faire valoir ; chacun de ces gens-là, en prenant plus que la valeur d’un cent pour leur part, ce serait une nécessité que le maître laboureur leur donnât non seulement toute sa récolte, mais même qu’il eût une mine d’argent, pour payer trois ou quatre fois davantage, afin de les satisfaire, et pour semer et se nourrir lui et toute sa famille. On ne poussera pas plus loin le ridicule de cette situation par rapport à l’état présent, qui ne l’était pas à ces temps-là, parce que cet ouvrier de huit et de seize sous par jour ne gagnait en 1550 qu’un pareil nombre de deniers, et que les souliers, qu’on vend aujourd’hui cent sous et six francs à Paris, furent évalués et appréciés à cinq sous par les ordonnances de Henri II en 1549, et les perdreaux et les levrauts à six deniers.\par
Ainsi, on n’a pas besoin de plus grand discours pour faire voir l’horreur du faux zèle, à prendre les choses absolument, et sans les approfondir ; mais, pour ne pas remonter si haut, ou descendre moins loin, en ne parlant que de l’année 1600, c’est-à-dire d’un temps dont plusieurs de nos contemporains ont connaissance, ce même setier de Paris valait trois livres dix sous ou environ, année commune, les souliers quinze sous, et le reste à proportion ; et bien que le blé eût triplé le prix auquel il était cinquante ans auparavant, on ne lui fit point de querelle comme on fait aujourd’hui, quoiqu’à le prendre depuis 1650, il n’ait pas reçu une si forte hausse, hors le temps de cherté extraordinaire, que l’on ne doit pas compter ; et cela, attendu que toutes choses avaient pris le même surcroît, et qu’il n’y avait pas lieu pour l’ouvrier de se plaindre d’acheter son blé trois fois davantage, alors que le cordonnier vendait quinze sous les mêmes souliers qu’il avait donnés pour cinq dans le temps que le blé valait trois fois moins.\par
Les prétendus protecteurs des pauvres ne peuvent point encore, sans renoncer à la raison, réclamer ce prix des grains ; car, quoique les conséquences perdraient les deux tiers du ridicule marqué ci-devant dans la réclamation du prix de vingt sous le setier, qui subsistait raisonnablement en 1550, la dose qui en resterait serait encore assez forte pour tout ruiner, sur le niveau d’aujourd’hui. En effet, s’il eût fallu que le laboureur eût acheté, dans la première supposition, trois fois plus de blé qu’il n’en eût recueilli ; pour satisfaire ses ouvriers, il ne pourrait encore, dans la seconde, les payer avec toute sa récolte. Ainsi, il n’y a pas encore moyen de tenir, puisque, pour qu’une chose soit impertinente et ridicule, il n’est pas besoin que le désordre soit dans le dernier excès, il suffit que la raison soit tant soit peu blessée : or, elle le serait encore, dans ce cas, d’une façon effroyable.\par
Sur ce principe, il faut venir hardiment en l’année 1650, c’est-à-dire de nos jours, où le blé, setier de Paris, fut à dix et onze francs année commune, sans que personne criât à la famine, ni manifestât même aucune surprise, et sans qu’on lui fit pareillement de peine de ce qu’il avait triplé le prix auquel il était cinquante ans auparavant, par les mêmes raisons qui lui avaient procuré ce repos en 1600, savoir que les souliers qui valaient quinze sous en ce temps-là, étaient vendus en 1650 quarante-cinq et cinquante sous, et tout le reste à proportion. Et cependant, lorsqu’en l’année 1700, et suivantes, où nous sommes, toutes ces mêmes denrées, hormis les blés, ont {\itshape doublé} par des causes très naturelles, dont on fera un chapitre à part (qui ne sont autres que les crues d’argent qui arrivent tous les jours dans l’Europe), on souffre tranquillement que toutes sortes de marchandises prennent leur quote-part de hausse de prix, comme elles ont toujours fait depuis la découverte du Nouveau-Monde ; mais on refuse cette justice aux grains seuls, et l’on croit avoir tout gagné en obligeant un laboureur ou son maître, qui ne sont qu’une seule et même chose ou un même intérêt, à donner ses grains au même prix qu’ils faisaient il y a cinquante ans, pendant qu’ils sont contraints d’acheter toutes les denrées au double, tant pour leurs besoins personnels que pour les choses nécessaires à l’agriculture, fait qui, les obligeant en tout temps d’en partager les profits avec une infinité de monde, les ruine lorsque les proportions n’y sont plus gardées. Il y a plus même, dira-t-on, car cela les met dans l’impossibilité absolue de continuer ce commerce avec la perfection nécessaire au maintien de l’État ; ce qui se recommuniquant dans la suite à toutes les autres conditions, qui veulent injustement vendre leurs denrées cher, et acheter le grain à bon marché, a pour conséquence infaillible de les détruire elles-mêmes. En effet, le principe de toutes les richesses de la France étant la culture des terres, ce désordre de manque de proportion la rend d’abord imparfaite par l’épargne qu’on est obligé d’y apporter, et la ruine même entièrement en beaucoup d’endroits ; ce qui fait payer la folle enchère de l’injustice des premiers auteurs de tout le désordre, savoir ceux qui prétendent acheter toujours à bon marché, et vendre toujours très cher.
\subsection[{Chapitre III.}]{Chapitre III.}
\noindent Il est aisé de voir, par tout ce qu’on vient de dire au chapitre précédent, qu’on ne pourrait pas souhaiter, sans extravagance, que le setier de Paris ne valut encore que vingt sous comme en 1550, ou trois livres dix sous comme en 1600. Or, sur ce même pied, on maintient que, de le vouloir à peu près à neuf ou dix francs, ainsi qu’on prétend aujourd’hui, et comme il était sans aucunes réclamations en 1650, c’est laisser un degré d’irrégularité capable de tout perdre, en ruinant tous les états, et par conséquent les pauvres, qui n’ont d’autre subsistance que le travail que leur fournissent les personnes riches et propriétaires des fonds, en sorte qu’un homme qui n’a que ses bras ou sa journée pour vivre, est perdu dès lors qu’il ne la peut trouver, quand même le blé ne vaudrait que vingt sous le setier, comme en 1550. D’où il suit qu’il suffit de dire que, le blé étant sur le pied de neuf à dix livres le setier, mesure de Paris, comme il est à présent, et même moins, il est impossible à la plupart des fermiers de rien payer à leurs maîtres, ce qui ruine également les uns et les autres, pour montrer invinciblement que tous les ouvriers perdent les trois quarts de leur revenu, s’ils ne sont pas entièrement réduits à la mendicité, ainsi qu’on en voit tous les jours.\par
La Providence a voulu qu’en France les riches et les pauvres se fussent mutuellement nécessaires pour subsister, puisque les premiers périraient avec toutes leurs facultés et possessions, qui ne sont originairement autre chose que les terres (tout le surplus, comme rentes, charges et redevances, n’étant proprement biens que par fiction, et par rapport à cette première cause qui leur donne l’être), si les seconds, c’est-à-dire les pauvres, ne leur prêtaient le secours de leurs bras pour mettre ces biens en valeur ; comme, par réciproque, si la terre donnait ses richesses d’elle-même, sans aucune contrainte, au lieu de ne nourrir et de ne payer les hommes, comme elle fait, qu’à proportion de leur travail, et selon la sentence prononcée de la bouche de Dieu même après le péché d’Adam, il arriverait que tous ceux qui n’auraient aucun fonds seraient absolument hors d’état de subsister ; et ainsi l’intérêt de ces deux conditions, le riche et le pauvre, est d’être dans un perpétuel commerce : et comme la première loi du trafic est que l’une et l’autre partie y trouvent leur compte, sans quoi il cesse entièrement parce qu’il détruit son sujet, il faut absolument tenir la balance égale, afin de partager l’utilité, et qu’un des bassins ne venant pas à pencher trop d’un côté par la survenue de quelque poids extraordinaire, il n’emporte pas tout le profit de l’autre, ce qui le mettrait hors d’état de continuer à l’avenir. C’est le prix des blés qui fait la balance pour l’agriculture entre le fermier et son maître, et l’ouvrier qui aide à le faire valoir. Or, pour montrer que la balance est trop penchée du côté de l’ouvrier, le blé étant à neuf et dix francs le setier à Paris, il faut nécessairement descendre dans la qualité et les divers genres de perfection des terres de la France.\par
Il est certain qu’il y a plus de cent degrés de différence entre les plus fécondes et les mieux partagées de la nature, et les moindres, qui semblent n’avoir été créées que pour former la contenance du monde, ne fournissant rien, ni pour le labourage ni pour la pâture. En effet, si l’on en voit, quoiqu’en très petite quantité, où deux mauvais chevaux seulement peuvent exploiter jusqu’à cent arpents par an, et renfouir ou tourner deux arpents par jour, sans aucun besoin d’engrais, qui ferait tout périr par un trop grand produit, et qui ne laissent pas de payer l’usure de la semence à vingt pour un ; et cela, toutes les années, sans reposer jamais, contre l’usage presque de toutes les autres ; il s’en trouve d’un autre côté, et en bien plus grand nombre, qu’il faut comme forcer de produire, et cela par un travail continuel, tant d’engrais que d’augmentation de chevaux, le terrain résistant à chaque pas ; et avec tout cela, il leur faut donner du repos au moins de trois années une, et même plus souvent, comme des sept à huit années de suite, et quelquefois encore jusqu’à quinze à vingt ans, à proportion que le prix des blés permet de croire que la culture en pourrait supporter les frais.\par
Aussi un arpent de terre du moindre degré de perfection, affermé trois livres, comme il s’en rencontre plusieurs, et même au-dessous, ce qui fait six livres, attendu l’année du repos, ne peut être exploité sans une forte semence, c’est-à-dire un setier de la valeur d’environ huit livres : il faut quatre labours au moins, et assez souvent cinq, qu’on ne paie jamais moins que trois livres dix sous chacun, et même plus pour les mauvaises terres, qui sont ordinairement pierreuses, et qui obligent par conséquent, par le dépérissement qu’elles causent au soc, de le porter souvent à la forge pour le recharger ; ainsi voilà encore quatorze francs de frais au moins ; il faut le fumier, qui ne peut être au-dessous de douze chariotées, ou d’autres mesures à proportion, ce qui fait encore douze francs ; il y a les frais de la récolte pour l’approfiter sur le champ, qui allant à trois livres, voilà plus de trente-huit francs semés en terre, et quand le rapport est de quatre setiers, ce qui n’arrive presque jamais dans de pareil terroir, on se tient bien heureux ; et si le blé qu’on a semé a coûté huit francs le setier, comme les mauvaises terres le détériorent toujours et lui font perdre sa perfection, au contraire des excellentes, comme en Hongrie, où le seigle devient froment au bout de trois ans ; le grain de ce mauvais terroir n’est vendu au plus que six francs. Ainsi, voilà le laboureur et le maître dans une perte considérable, qui les oblige de laisser la terre en friche, comme il arrive tous les jours ; y en ayant quantité d’incultes, autrefois labourées, ce qui n’arrive pas sans réduire et le maître et le laboureur dans une extrême indigence ; que si le blé avait valu onze à douze livres le setier, comme il le peut aisément, le maître et le laboureur, les valets et les ouvriers, y auraient également trouvé leur compte, et c’aurait été une garantie formelle et une défense certaine contre les horreurs d’une année stérile, qui ne manque jamais d’arriver de temps en temps.\par
Voilà donc de bien des façons la prétendue pitié et charité de ceux qui veulent, en faveur des pauvres, le blé à bas prix, loin de leur compte ; puisque ce premier pauvre, qui est ouvrier, est non seulement réduit à la mendicité par le congé qu’il reçoit en même temps que l’on cesse d’exploiter la terre, mais qu’en outre le fermier et le maître sont jetés dans la dernière misère ; et que toutes les conditions de l’État, qui attendent leur subsistance de ce premier mobile, reçoivent le même destin au sou la livre de la nécessité que l’on a de leur profession, sans préjudice de la certitude d’une générale, lorsque la disposition du ciel ne se rencontrera pas favorable aux biens de la terre.\par
Ainsi, on voit que les Anglais n’ont pas perdu le sens, de donner de l’argent à ceux qui font l’enlèvement de leurs blés pour les pays étrangers, afin d’obliger les habitants de faire valoir les mauvaises terres, de quoi ils ont quantité, et l’on a vu même pratiquer cette conduite une année après que les grains y avaient été d’une cherté extraordinaire, sans alléguer cette pitoyable raison, qu’il faut craindre de retomber dans la misère d’une stérilité quand on ne fait que d’en sortir, et fournir un royaume de blé amplement avant que d’en faire part aux étrangers, puisque c’est justement le contraire, comme on a fait voir, et qu’on montrera encore mieux dans la seconde partie.\par
Ce que l’on a dit du sort des mauvaises terres, d’être en perte au laboureur et au maître, le blé étant à bas prix, est commun au sou la livre à celles du premier degré d’excellence ; parce que, si les charges de la culture sont moindres, le profit est pour le maître, qui afferme son bien un prix proportionné, lequel ne pouvant être atteint par la récolte, le blé étant à bas prix, produit tous les mêmes effets que l’on vient de marquer, et envers autant de personnes.
\subsection[{Chapitre IV.}]{Chapitre IV.}
\noindent Quoique l’erreur du raisonnement de ceux qui veulent le blé à bas prix en faveur des pauvres ne soit que trop vérifiée par tout ce qu’on vient de dire, il est à propos de descendre dans le détail de toutes les conditions, et de montrer que toutes leurs richesses consistent dans la culture de la terre ; que c’est pour elles tout ce que le laboureur sème et recueille ; que, quand il sème beaucoup, elles recueillent beaucoup, et peu également quand c’est le contraire ; qu’ainsi c’est leur intérêt de le mettre sans cesse en état de faire une récolte abondante, de quoi étant empêché aujourd’hui par le bas prix des grains, tous leurs vœux et tous leurs souhaits doivent tendre à ce qu’ils reprennent un taux qui l’oblige à cultiver autant qu’il est possible.\par
Toutes les professions, arts et métiers qui composent un État, et surtout en France, où il s’en rencontre beaucoup plus de genres et d’espèces qu’en nul lieu du monde, ont pour objet leur subsistance, en procurant ou fournissant celle des autres, ce qui les oblige d’avoir recours à eux, et de se donner de l’emploi réciproquement les uns aux autres : néanmoins, tous n’ont pas une fonction d’égale nécessité, et dont le monde ne se puisse passer absolument. Les uns fournissent le nécessaire, comme la première et la plus grossière subsistance, c’est-à-dire le pain et les liqueurs ; ceux-ci quelque chose de plus, comme les moindres mets ; ceux-là, les viandes, entre lesquelles il se rencontre quantité de différents degrés, comme le délicat, le sensuel, le superflu, et enfin le fantasque et absolument inutile ; et tous ces divers degrés, qui se rencontrent non seulement dans le manger, mais aussi dans les habits, dans les meubles, dans les équipages, dans les spectacles, et enfin dans tout le reste de ce qui s’appelle magnificence, et qui donne l’être à plus de deux cents professions, arts et métiers qui se trouvent en France, prennent, comme on a dit, leur naissance des fruits de la terre, laquelle, si elle devenait aussi stérile que les sables d’Afrique, congédierait ou ferait périr plus de cent soixante et dix de ces deux cents professions : ainsi, encore une fois, leur intérêt est de maintenir le laboureur, et de l’empêcher de périr. Or, c’est une maxime constante dans la mécanique, que tout métier doit nourrir son maître, ou que ce maître doit fermer incontinent sa boutique, de façon que, du moment où le laboureur ne vendra pas son blé, comme il arrive assez souvent, un prix qui puisse porter les frais de la culture et toutes les charges accessoires, comme les impôts et les autres paiements divers du fermage, il est certain que ce fermier abandonnera tout, ou ne satisfera pas à ce qu’il doit rapporter au propriétaire. Voilà, dès ce moment, toutes ces deux cents professions en péril, et si le sort de ce fermier lui est commun avec quantité d’autres, comme il est impossible que cela ne soit autrement, puisque le mal procède d’une cause générale, tous les états souffrent un déchet considérable.\par
En effet, un propriétaire de fonds qui n’est point payé, ne peut rien acheter, puisqu’on n’a rien sans argent. La première grêle tombe sur les choses superflues ; après cela, si le désordre continue, on se retranche peu à peu, de degré en degré, suivant l’échelle que l’on vient de marquer. Et comme c’est l’opulence qui les avait fait naître, qui n’est ordinairement autre que les fruits de la terre, leur chute les entraîne toutes avec elle.\par
Il y a encore une attention à faire, qui est, que cette réforme ne s’en tient pas seulement au superflu, et même au commode et à l’utile, mais qu’elle attaque jusqu’au plus nécessaire de plusieurs conditions ou métiers, par un contrecoup qui devient aussitôt contagieux, et embrasse toutes les professions. En effet, s’il n’y avait que le superflu et le magnifique qui souffrissent, le désordre ne serait pas tant à déplorer ; mais comme l’ouvrier du superflu et du magnifique, n’exerce cet art et cette profession que pour se procurer le nécessaire, l’un ne peut être retranché, sans que la perte de l’autre ne s’ensuive aussitôt, ce qui cause un nouveau déchet dans l’État, parce que chaque particulier doit soutenir sa dépense ordinaire, sur laquelle les denrées nécessaires ont contracté un prix, lequel venant à baisser, elles deviennent toutes en perte au marchand ou à l’ouvrier.\par
Dans ces occasions, un homme vivant de ses rentes, qui a cent écus dans sa poche, et qui les aurait dépensés pour des besoins utiles et commodes seulement, si son fermier ne l’avait pas assuré qu’il ne lui peut bailler d’argent à l’échéance du terme qui approche, les garde bien soigneusement, afin de les faire filer pour le simple nécessaire, et cette trop longue garde maintient l’argent dans un trop long repos contre sa nature, qui est de toujours marcher, et de produire du revenu à chaque pas qu’il fait. Or, sans ce déchet arrivé à la cause primitive, qui est le blé, les cent écus dont on vient de parler auraient fait cent, voire deux cents mains, dans le temps de leur résidence, s’ils avaient toujours été en route ; et cette forte garde, qui a si longtemps arrêté cette somme dans son premier gîte, ne se peut faire sans intéresser tous les passages qui ne subsistaient que de la coutume où ils étaient de la voir ordinairement à l’aide de leurs denrées ou de leurs services, car la mévente des grains rend dans ce cas les unes et les autres complètement inutiles.\par
Et, comme il y a de l’ordre dans l’augmentation de la dépense, à proportion qu’on augmente de facultés ; que, dès qu’on a plus que le nécessaire, on se procure le commode ; qu’ensuite de cela, on passe au délicat, au superflu, au magnifique, et enfin, dans tous les excès que la vanité a inventés pour ruiner les riches, et enrichir ceux qui n’avaient rien de leur origine ; de même, lorsqu’il faut déchanter par la cessation des revenus en fonds, causée par l’avilissement des blés, la réforme refait le même chemin en rétrogradant, ce qui ruine d’abord tous les ouvriers de magnificence et de superflu, et jette un levain qui, gâtant tout l’État, produit les banqueroutes que l’on ne manque jamais de voir dans ces occasions, et fait dire, aux aveugles en pareille matière, que c’est qu’il n’y a plus d’argent : il en est autant et plus que jamais, mais c’est qu’il devient paralytique, comme on a fait voir.\par
Et pour montrer encore plus clairement cette vérité, on n’a qu’à jeter les yeux sur les banqueroutes qui se sont faites à Paris depuis que le blé est à vil prix : il y en a plus qu’il ne s’en était rencontré dix ans auparavant, qu’il avait été au double de ce qu’il est aujourd’hui. En effet, un propriétaire qui n’est point payé, ne donne point trente pistoles d’une perruque, cinquante pistoles d’une écharpe, quatre mille francs d’un carrosse ; ainsi, il faut que les marchands de pareilles magnificences, qui ont fait de grandes avances, et se sont constitués en de grands crédits, pour fournir leur magasin de pareilles superfluités, du moment qu’ils n’en trouvent pas le débit, périssent entièrement en prenant la fuite, et abandonnant tout à leurs créanciers, ce qui devient si contagieux, qu’une seule banqueroute en attire une infinité d’autres.\par
Il y a encore un autre désordre, qui est pareillement un enfant de la première cause, c’est que lorsqu’un ouvrier ou marchand voit ses affaires en désordre, et qu’il ne pourra satisfaire ceux à qui il doit à l’échéance des termes, manque de débit, il {\itshape fait finance}, comme on appelle, pour échapper à la mendicité ; c’est-à-dire qu’il donne tout à vil prix et à perte, non de lui, mais de ses créanciers, et met ensuite l’argent dans sa poche, et la clef sous la porte de sa maison, en prenant congé de la compagnie, pour ne plus reparaître du tout, ou qu’après qu’il aura obtenu des remises considérables de ceux à qui il doit ; ce qui, outre le désordre que cela cause à tout l’État, en forme encore un effroyable, en ce que cette vente à vil prix et à perte de marchandises qui devraient être bien plus chères par leur nature, réduit au néant celles de tous les autres vendeurs, qui ne peuvent jamais espérer de la libéralité du chaland la préférence de leurs denrées à un prix plus haut que celui auquel on peut les avoir autre part ; et ce premier commerçant n’est obligé de donner sa marchandise à perte, que parce qu’il a eu le blé du laboureur à la même condition.
\subsection[{Chapitre V.}]{Chapitre V.}
\noindent On sera peut-être surpris, à cause de l’erreur si généralement établie sur la nature ou le prix des grains, de ce que l’on ose avancer, que tous ces sujets dont la fortune va en déroute, qui endurent une si grande perte, et la causent à tant d’autres, comme de 10, 20, 30, 40 et 50 mille francs, et même davantage, ne souffrent ce malheureux destin, que pour avoir prétendu gagner les uns cinquante francs, cent francs ou trois cents francs au plus par an, sur le pain qu’ils mangeaient, et qui se consommait dans leur maison à Paris. Le pain du commun ne revient pas à présent à plus de quinze deniers la livre, sur le pied de dix livres le setier : or, le mettre à une moitié davantage, comme environ deux sous, ce qui n’augmente la dépense sur une famille d’environ dix ou douze personnes, comme elles sont toutes à peu près, que de cinq ou six sous par jour, cela ne formerait que cent francs par an ; et ce ménage, ou prétendu profit de ces cent francs, fait perdre plus de dix mille livres, et réduit toute la famille à l’aumône.\par
Quoique ce fait soit constant, le peuple, qui ne diffère en rien des bêtes dans ses raisonnements généraux, et qui n’étend point ses vues au-delà de son intérêt personnel et singulier du moment, aura peine à comprendre ces principes, savoir : qu’il ne peut être riche et à son aise tant que le blé est à vil prix, et qu’il faut au contraire qu’il conserve le niveau et les proportions de hausse contractées par toutes les denrées, au moins depuis cent cinquante ans, afin que, la balance étant toujours dans son équilibre, le commerce se puisse faire avec justice, à faute de quoi tout périt. Mais cela n’est pas moins incontestable : tout ce qui se passe, tout ce que l’on voit, et que l’on vient de vérifier, ne le montre que trop. Tous les états ensemencent les terres, et ce n’est point le laboureur seul qui a cette commission, quoiqu’on le suppose grossièrement ; et, comme lorsqu’on sème peu, on recueille peu, et qu’au contraire la moisson est abondante quand on cultive quantité de terres ; tous les états et toutes les conditions doivent faire ce raisonnement, chacun pour leur particulier, quand ils achètent le blé ou le pain un prix considérable, qui ne soit point exorbitant, dont il n’est, point nécessaire de faire d’exception, puisque tous les excès sont défectueux, et n’entrent point dans le raisonnement : quand, dis-je, ils se fournissent de ce premier besoin de la vie à un prix raisonnable, qui ne constitue pas le laboureur, qui n’est que leur commissionnaire, en perte comme aujourd’hui ; c’est un nombre de semences qu’ils jettent sur la terre, et qui leur rapportera avec usure une récolte abondante, et les cinq ou six sous par jour ou cent francs par an, produiront souvent plus de deux ou trois mille livres ; au lieu que n’ayant semé que pour les frais de la récolte, qui est le fort aujourd’hui des laboureurs, ils doivent s’attendre que le maître ne recevant rien, il ne leur formera aucun, profit, par nulle action de leur marchandise, ce qui les fera périr avec ce même laboureur.\par
Quoique tout ceci n’ait l’idée que d’une spéculation très abstraite pour tous ceux qui ne sont point actuellement laboureurs, on peut assurer, néanmoins, que c’est réellement et de fait une pure pratique, et que les choses se passent journellement de la sorte ; que l’excédent du nécessaire s’érige en commode ; que le surplus du commode se transmue en délicat, et que l’abondance pareillement de ce dernier enfante le magnifique, qui se divise encore en de nouvelles branches, qui s’étendent aussi loin que la vivacité de l’esprit ou la corruption du cœur peuvent imaginer.\par
Et comme cette abondance de nécessaire est le premier mobile et la première cause de toute cette génération, du moment qu’elle cesse par l’avilissement du prix des grains, toute la postérité périt aussitôt, par la raison fournie par la philosophie ou par la nature, que quand la cause cesse, les effets ont incontinent le même sort.\par
Bien que, par tout ce qu’on vient de dire, il soit impossible de ne pas donner les mains à un raisonnement si sensible et si naturel, appuyé sur deux faits si incontestables, qui se passent aux yeux de tout le monde, quoique sans nulle attention qui puisse faire revenir des préjugés qu’engendre l’erreur du peuple, ainsi qu’une compassion aveugle, causée par l’ignorance de toutes les personnes en place, sur la nature et les véritables intérêts des blés ; cependant, comme l’exemple de ce qui s’est vu dans la découverte de la figure de la terre, n’a que trop appris le destin que doivent attendre tous les porteurs de nouveautés surprenantes, il est à propos de fortifier encore ce raisonnement par un parallèle du sort des peuples, dans toutes les conditions, pendant ces dernières années que les grains ont toujours été à bas prix, avec l’état où ces mêmes peuples se trouvaient durant les trois précédentes, que les blés étaient constamment à un taux beaucoup plus élevé qu’ils ne le sont aujourd’hui, et c’est ce que l’on va voir dans le chapitre suivant.
\subsection[{Chapitre VI.}]{Chapitre VI.}
\noindent La certitude du fait, que depuis 1690 jusqu’à 1700, et même quelque chose de plus, le blé a toujours été à dix-huit livres le setier, et que, depuis 1700, il a toujours baissé jusqu’à aujourd’hui, qu’il n’est qu’à neuf ou dix livres, n’a pas besoin d’être établie : ainsi, il n’est question que de faire la comparaison qu’on vient de marquer.\par
Toutes les conditions ont des baromètres, ou des pierres de touche de leur aisance ou de leur incommodité, exposées au grand jour, qui ne permettent pas de douter un moment de la situation où elles se trouvent.\par
Si l’on voulait soutenir qu’en l’année 1660, et autour de ce temps, des peuples, qui achetaient des charges de robe sans nul produit, jusqu’à des cent mille francs et quarante mille écus, et les moindres à proportion ; et cela, dans toutes les contrées du royaume, sans en souffrir jamais de vacantes un seul moment, que la préférence ne formât des espèces de combats ; si l’on prétendait, dis-je, avancer que cette situation n’eût pas une montre et une supériorité de richesse d’une infinité de degrés, sur l’état d’aujourd’hui, que ces mêmes charges vaquent par douzaines plusieurs années, sans qu’on en puisse trouver le quart de ce prix précédent, pendant que plus des deux tiers des inférieures sont abandonnées aux {\itshape parties casuelles} par les propriétaires, ou bien qu’on n’en veut qu’à un prix moindre qu’auraient coûté les {\itshape provisions} en 1660 ; il faudrait assurément que l’auteur d’une pareille doctrine commençât par établir le pyrrhonisme, et à douter qu’il fit jour en plein soleil. Tout comme de dire, que cette opulence était singulière aux gens de robe ; car elle était assurément générale, et toutes les conditions avaient une pareille montre d’opulence, qui ne permettait pas de douter qu’elle ne fût réelle et effective dans tous les états.\par
Depuis ce temps-là, ou environ, toutes choses ont toujours été en dépérissant, hormis quelques époques, où la stérilité, venant au secours des peuples, quoique quelquefois trop fort, relevait le prix des grains, redressait la balance, et rétablissait les proportions nécessaires dans le commerce général : en effet, sans ce secours, on peut dire que tous les laboureurs auraient péri, comme avait déjà fait une infinité ; et, quoique le remède soit violent, il peut néanmoins se comparer à tous ceux qu’on emploie pour la guérison du corps humain ; leur opération n’agit jamais, même avec le plus de succès, sans altérer le sujet qui les subit, et sans qu’il en coûte du sang, ainsi qu’une diminution ou suspension momentanée des forces vitales.\par
C’est de cette sorte que, quelque effroyables qu’aient été les désastres qu’entraînèrent à leur suite les années 1693 et 1694, les cinq ou six années suivantes compensèrent avantageusement le mal, ce que l’on ose avancer sur un principe qui est certain, et que l’on établira sans crainte de repartie dans le chapitre suivant, savoir qu’un long avilissement du prix des grains fait plus de dommage à un État, et fait même périr plus de monde qu’une excessive cherté, qui ne dure au moins qu’une année ; et qu’ainsi, si on la veut réprouver absolument, il faut prétendre qu’au lieu de se réjouir d’une victoire obtenue sur un ennemi puissant qui, venant pour envahir et ruiner un royaume, aurait été vaincu, et l’avantage même suivi de conquêtes faites sur lui ; qu’au lieu donc, dis-je, de faire des feux de joie de ce succès, il faudrait le déplorer et en prendre le deuil, comme d’une calamité publique, parce que la victoire aurait coûté la vie à un nombre considérable d’hommes.\par
Les six années consécutives depuis 1694, virent le blé presque toujours au double prix de ce qu’il est aujourd’hui ; et par conséquent toutes les terres, tant bonnes que mauvaises, bien cultivées ; le blé bien ménagé, et non pas détourné à des usages étrangers, comme il arrive dans les temps d’avilissement ; les propriétaires bien payés, et toutes choses en valeur ; et il n’y avait point de professions dans l’État qui ne tirât son sou la livré de cette opulence, par la vigueur de ce premier être, qui leur donne la naissance à toutes, ainsi qu’on a montré. Les laines, les toiles, toutes les manufactures se vendaient une moitié plus de ce qu’elles sont aujourd’hui, et les charges de robe presque le double, ce qui étant le comble de la perfection de cette situation, est un baromètre certain de l’opulence générale : le tout est trop récent pour qu’on le puisse révoquer en doute. Et, pour répondre par avance à l’objection, que la guerre seule a changé cette disposition, on a vu les choses en cet état, non seulement durant trois années de cette dernière guerre, mais même durant toutes celles qui précédèrent la paix des Pyrénées, ainsi que pendant toutes les autres ; et même, à parler sainement, si les guerres se soutenaient avec les revenus ordinaires du prince, comme il ne serait pas impossible, si tous les commerces étaient dans leur perfection, on peut dire qu’elles seraient plus avantageuses à la France qu’une tranquillité entière : la guerre met toutes choses en mouvement ; elle purge les humeurs peccantes, et elle charme en quelque manière la vivacité d’une nation qui n’aime pas naturellement le repos, et à qui même il est souvent dommageable. Mais, pour revenir aux marques sensibles d’opulence de ces trois, ou six dernières années, qui ont terminé le siècle qui vient de finir, outre celles qu’on vient de coter et qui sont incontestables ; il y en a d’enregistrées, dont la preuve se peut faire aisément par écrit, puisqu’il n’y a qu’à représenter les rôles ou les comptes des commis des Aides.\par
Comme la richesse et l’opulence des personnes élevées se marquent par l’achat des Charges, les bâtiments, et tout l’attirail d’une magnificence complète, qui est produite par la possession d’une très grande abondance du nécessaire, ainsi qu’on a dit ; de même, le peuple qui prend sa quote-part au sou la livre de son état à cette situation, a également le cabaret par-devers lui, surtout les ouvriers, pour singulier baromètre de ses facultés : c’est là que souvent fêtes et dimanches, hors les heures du service divin, si les juges de police font leur devoir, et souvent même les jours ouvriers, plus de la moitié du prix du travail de la semaine se consomme, et souvent même tout à fait. Cela hausse et baisse, au niveau, et à proportion de ce travail ; si on a beaucoup gagné, on dépense beaucoup, et peu à proportion ; et la cessation de cette conduite est une marque certaine que l’on n’a point trouvé de travail ou très peu, faute de commerce ou de vente, causée par l’anéantissement du premier principe.\par
Or, il est certain, et MM. les ministres ne le savent que trop par les défalcations que les fermiers, tant généraux que particuliers, leur ont demandées depuis trois ans, que le produit des Aides est diminué de plus de moitié ; il y a des lieux même où cela a été jusqu’aux deux tiers, et même aux trois quarts.\par
Les livres ou registres de tous les marchands, qui font foi en justice, n’en feraient que trop encore d’une pareille diminution, si l’on ne s’en veut pas rapporter à leurs discours, bien qu’ils n’aient autre chose à la bouche ; et c’est dans cette conjoncture que l’argent, bien loin de produire continuellement une espèce de représentation avec du papier et des billets de change, lorsqu’il ne peut suffire par sa volubilité ou par sa quantité à celle de la consommation, est réduit lui-même à la dixième partie de ses fonctions ou de sa marche ordinaire, faisant des années entières de résidence dans des mains où il serait à peine resté un moment, si, la cessation de la consommation, par la ruine de la proportion des prix, sans laquelle elle ne se peut faire, ne le retenait pas immobile par force : ce qui fait dire dans ces circonstances, mais seulement par le peuple, qu’il n’y a plus d’argent, parce qu’on ne le voit plus marcher, comme si l’on pouvait prétendre qu’un homme endormi en quelque lieu secret, fût mort, parce qu’on ne le verrait plus toujours par voie et par chemin, ainsi qu’il se montrait auparavant.
\subsection[{Chapitre VII.}]{Chapitre VII.}
\noindent Pour mettre le comble enfin au soutien que l’on fait, que rien n’est si préjudiciable à un État, que l’avilissement du prix des grains, par rapport à celui qui est contracté antérieurement par les autres denrées, et par les grains mêmes ; il faut prouver, comme c’est la vérité, que cette situation fait périr beaucoup plus de monde de mort violente ou non naturelle, que quelque stérilité que ce soit.\par
Quoique cette proposition doive causer un très grand degré de hausse de surprise, parce qu’elle renchérit très fort sur tout ce discours, elle n’est pas pour cela moins véritable ; et, quelque prévention qui règne pour croire le contraire, on sera obligé d’y donner les mains, pour peu d’attention que l’on fasse au détail des faits qu’on va exposer hardiment aux yeux du public, parce qu’ils sont incontestables, quoique beaucoup trop ignorés, malheureusement, à cause de la grande distance qui se trouve entre ceux qui souffrent ce malheureux destin, et les personnes qui pourraient le faire changer en un moment, s’il n’y avait pas une infinité de ressorts, tendus depuis le matin jusqu’au soir, pour les faire errer au fait, malgré les lumières de leur esprit, et la sincérité de leurs intentions.\par
L’on sait, et personne ne le conteste, que les deux extrémités, quoique très opposées, étant presque toujours vicieuses, produisent également les mêmes pernicieux effets ; le trop de froid, comme le trop de chaleur, détruit également le sujet sur lequel ils agissent ; le trop d’aliments pris sans mesure fait mourir un homme, aussi bien qu’une abstinence totale prolongée trop longtemps.\par
Il y a même plus : quoique les guerres, surtout celles qui sont trop violentes, aient toujours été regardées comme le plus grand et le plus terrible des fléaux de Dieu, parce qu’elles font plus de destruction et périr davantage de monde, et qu’ainsi elles aient un degré d’horreur au-dessus des effets de la stérilité ou de la famine ; cependant, Sénèque ose soutenir, et personne n’a encore jusqu’ici entrepris de le contredire, que la gourmandise fait plus périr de monde que la guerre ou l’épée ; et enfin, après le siège de La Rochelle, il mourut autant de personnes pour avoir trop mangé, l’estomac ayant perdu l’habitude de digérer, qu’il en avait péri par la famine.\par
Sur ce compte, on maintient que l’avilissement du prix des grains, qui est une espèce d’indigestion d’État, causée par la trop grande abondance, attaquant toutes les conditions, est un ver ou un chancre qui les ronge ou les mine peu à peu ; et quoiqu’on se retranche continuellement par une diminution de dépense, ce qui s’augmente à vue d’œil, le mal est souvent si violent, qu’il ne prend fin qu’avec celle d’une infinité de personnes et de familles.\par
C’est dans ces occasions que l’abondance dans un royaume est aussi préjudiciable que le trop d’aliments pris en même temps par un homme : comme l’excès empêche les fonctions de la nature, et que tout se tourne en corruption, ce qui détruit le sujet, il en va de même du trop de grains, dont on ne peut faire l’évacuation nécessaire, pour satisfaire aux obligations qui accompagnent toutes sortes de commerces, et surtout le labourage.\par
En effet, un laboureur accoutumé à vivre commodément, lui et toute sa famille, ainsi que son maître, lorsqu’il était en état de payer celui-ci, est fait vendre par ce même maître et avec perte par l’avilissement du prix des grains, et par là réduit à l’aumône, et bien souvent le maître même, ou à gagner leur vie par le travail des mains ; à quoi n’étant pas faits, ainsi qu’aux mauvais aliments, qui en sont une suite nécessaire, on peut dire avec assurance que les personnes ne tardent guère à souffrir le même sort que les biens : le chagrin d’esprit, la honte, la désolation générale, les font périr à vue d’œil, eux et toute leur famille ; le mal commence par les enfants, car, comme ils ont besoin de secours pour être élevés jusqu’à l’âge où ils seraient en état de gagner leur vie, et qu’ils ne peuvent en recevoir de parents qui se trouvent dénués de toutes choses, on peut dire avec certitude qu’il en périt plus de la moitié manque de leurs besoins, tant à la mamelle que dans la première enfance, toutes les maladies devenant mortelles dans ces occasions, faute de soins, de remèdes et de nourriture convenable. Et, comme ce désastre des laboureurs devient aussitôt contagieux, et embrasse tous les états, ainsi qu’on a fait voir, ce sort devient commun ; et si les riches sont obligés de retrancher leur superflu, comme il produit le nécessaire à beaucoup d’arts et de professions, c’est un congé entier, une désolation générale, que ce retranchement leur cause : les familles nombreuses n’ont plus d’autre ressource que d’en espérer la diminution de la bonté du ciel, et on peut dire que leur extrême misère concourt extrêmement à fournir les moyens pour en obtenir cette grâce. C’est alors qu’il serait excellent d’entendre ces gens charitables, qui veulent en faveur des pauvres les grains au plus bas prix qu’ils puissent être, en leur demandant s’ils croient leurs vœux pleinement exaucés par cette situation, et si leur intention était de faire devenir les riches très misérables, pour après cela étendre le mal à toutes les conditions.\par
Il n’y a que l’expérience et une forte attention, en descendant personnellement dans une très grande discussion de tous les faits singuliers, qui puissent rendre tout ceci vraisemblable, mais il n’en est pas moins certain : une extrême nécessité, non seulement tarit toutes les tendresses de la nature, mais fait même outrager cette nature dans les occasions pressantes. C’est ainsi qu’on a vu, dans les villes assiégées et poussées par la famine, la mère arracher l’aliment de la bouche de son enfant pour soutenir sa propre vie, et lors du siège de Jérusalem sous Tite Vespasien, une mère dévorer son propre enfant faute de pouvoir se procurer une autre nourriture. Comme la nécessité ne connaît point de lois, elle transgresse même les plus sacrées, à proportion de l’excès où elle se trouve. Que l’on ne s’étonne donc point de ce qu’on avance, que l’extrême misère fait regarder, comme une grâce, la diminution des familles, et que cette situation apporte avec elle les moyens de se la procurer : ce mal, à la vérité, fait moins de bruit et de fracas, que celui qui est causé par une extrême stérilité ; mais, s’il est moins violent dans les apparences, il est plus pernicieux dans les effets ; et il en va comme du poignard et du poison, dont on se sert pour faire périr les hommes. Deux sujets poignardés causeront plus de bruit et d’horreur, et attireront plus de poursuites violentes, que vingt autres qui auront péri par un poison lent, administré en secret : les doutes sur la véritable cause de la mort, et le degré plus grand d’incertitude sur l’auteur du crime diminuent de moitié, dans cette circonstance, tout le fracas qui suit ordinairement l’autre manière de faire périr les hommes ; mais, avec tout cela, celle-ci ne fait pas moins de mal ; au contraire, elle renchérit sur l’autre, en ce qu’elle fait plus longtemps souffrir son sujet, et que le dehors moins violent qu’elle jette diminue les mesures nécessaires pour la conjurer, ce qui n’arrive pas en l’autre, où le ciel et la terre semblent s’armer dans ces occasions pour tirer vengeance du passé, et prévenir le mal dans l’avenir.\par
On s’est étendu sur ce parallèle, parce qu’on peut dire la même chose de la misère causée par la trop grande cherté, et de celle que produit l’avilissement des grains : si l’une poignarde, l’autre empoisonne, et toutes deux ont les mêmes suites, tant dans leur naissance et leur progrès, que dans leur fin, comme on vient de marquer, en rappelant que, si de temps en temps cette maladie d’avilissement de grains, ne recevait du soulagement par une cherté trop violente, et qui n’arrive pas sans qu’il en coûte du sang au corps de l’État, les suites d’un grand avilissement auraient porté les choses dans la dernière désolation, comme d’un abandon entier de la culture de la plupart des terres, qui reçoivent leur sort et leur ordre de porter du prix des blés, ainsi que l’on montre par tout ce qui a été dit ci-dessus, et qui a plus qu’acquitté l’auteur de ce qu’il avait promis dans cette première partie, savoir que plus les grains sont à vil prix, et plus le menu peuple, ainsi que les riches, sont misérables : c’est pourquoi on passe à la seconde, dans laquelle on espère également tenir parole.
\section[{Seconde partie. Où l’on fait voir que plus l’on enlèvera de blés en France, et moins on aura à craindre les extrêmes chertés.}]{Seconde partie. Où l’on fait voir que plus l’on enlèvera de blés en France, et moins on aura à craindre les extrêmes chertés.}
\subsection[{Chapitre I.}]{Chapitre I.}
\noindent L’on n’évitera jamais en France les malheurs d’une extrême cherté, qu’en laissant une entière liberté aux étrangers d’enlever des blés en tout temps, et en telle quantité qu’il leur plaira, hors les occasions de prix exorbitant, qui portent leur défense avec elles, par ces règles du commerce qui ne permettent point qu’on le fasse avec perte, ainsi qu’il arriverait dans ces rencontres. Dans l’espérance donc que l’on a d’un lecteur moins farouche, et plus revenu des préjugés que les gens du commun, on va entrer en matière, et on est assuré que cette seconde proposition sera également hors de crainte de toute repartie, comme on maintient qu’est la première.\par
Quelque effroyable et quelque horrible qu’ait paru le portrait de l’avilissement du prix des blés, en sorte qu’il y en a plus qu’il n’en faut pour lui faire son procès, malgré l’idée du vulgaire qui le canonise en France, au contraire de ce qui se pratique en Angleterre, où le peuple décide du sort de sa subsistance, voici bien une autre pièce qui le rend encore plus criminel, et qui doit par conséquent presser sa condamnation.\par
C’est la cherté extraordinaire des grains qu’il mène nécessairement à sa suite, et qu’il ne manque jamais de faire ressentir au même degré d’horreur qu’il s’est rencontré lui-même dans une situation tout opposée, cet avilissement étant la semence unique d’où s’enfante cet excès de prix, qui passe pour un des fléaux de Dieu ; par ce principe certain, qu’il n’y a rien de modéré chez le peuple, qui, ne connaissant point de milieu, passe en un moment d’une extrémité à l’autre : on en conviendra, pour peu d’attention que l’on veuille faire à ce qui va suivre.\par
Les grains en France ont deux intérêts et deux faces, bien que tous deux se rencontrent toujours dans un combat continuel, ne cherchant qu’à se détruire, parce que chaque parti est persuadé qu’il ne peut être heureux que par la destruction de son ennemi. Ces deux partis se forment des deux effets que produisent les grains, le premier, de nourrir les hommes dans l’Europe, en sorte que le défaut de cette manne les fait périr ; et l’autre est que la possession où se trouvent les propriétaires des fonds, d’en avoir une plus grande quantité qu’ils n’ont besoin pour leur usage personnel et singulier, leur sert de moyen pour se procurer, par la vente de ce surplus, toutes les autres choses que demandent les nécessités, les délices, ou la magnificence de la vie.\par
Le premier intérêt exige que les grains existent en la plus grande quantité qu’il est possible, et à bon marché, et s’en tient là ; et l’autre serait bien du même sentiment sur la quantité, si l’excès ne les avilissait pas, ce qui étant impossible, comme l’expérience le montre assez, il ne balance pas à prendre son parti, à les souhaiter, et faire tous ses efforts pour les voir à haut prix, quand même il devrait y en avoir moins : le procès donc est entre les vendeurs de blé et ceux qui l’achètent. Mais, s’il est constant que, sous un certain rapport, le commerce des grains ressemble au trafic de toutes les autres denrées, où l’acheteur voudrait avoir la marchandise pour rien, et le vendeur en tirer quatre fois plus que son prix ; il n’est pas moins vrai qu’il en diffère beaucoup sous un autre, en ce que dans le négoce ordinaire la cupidité du marchand se trouve restreinte, d’abord par la certitude où il est que son voisin, dont la boutique est aussi bien fournie que la sienne, vendra toujours à des conditions raisonnables, et ensuite par la circonstance que le chaland n’est pas dans la nécessité absolue d’acquérir les choses dont il fait commerce. Mais ces deux circonstances, qui mettent la police dans le trafic de toutes les autres denrées, ne sauraient la mettre dans le trafic des grains. Le laboureur ne peut pas plus se passer de vendre ses blés, que celui qui veut manger n’a le pouvoir de ne pas s’en fournir ; et ce sont ces deux obligations qui causent le désordre et font que, dans ce trafic, les deux partis dont on vient de parler sont continuellement en guerre. Il y a même plus, c’est qu’un degré d’avantage que l’un a sur l’autre, est un levain qui multiplie aussitôt à vue d’œil, et met les choses dans un tel excès, qu’un parti terrasse tout à fait l’autre, ce qui est la ruine de l’État, de quelque côté que tourne la victoire.\par
On vient de marquer que l’intérêt de tout acheteur est qu’il y ait quantité de marchands, ainsi que beaucoup de marchandises, afin que la concurrence leur fasse réciproquement donner la denrée au rabais, pour avoir la préférence du débit ; et qu’au contraire, le marchand ne vend jamais mieux que lorsqu’il est assuré, par la rareté de la denrée, qu’il n’a pas beaucoup de concurrents, et que l’acheteur est presque dans l’obligation de le payer à sou mot.\par
Or, dans le commerce des blés, quand il se rencontre par une année abondante à bas prix, la vente d’une partie ne suffisant point pour satisfaire aux besoins du ménage et payer le maître, il faut que le fermier fasse main basse sur tout, ce qui rengrége son mal, de manière qu’il est presque obligé de remporter ses sacs du marché sans délier, ce qui augmente et le vil prix et la nécessité de vendre : en sorte que ne s’en pouvant défaire, mérite à perte, par rapport aux frais du labourage, par les voies ordinaires, il le prodigue à l’engrais des bestiaux, et même à la confection des manufactures, comme amidons et bières, contre sa destruction naturelle, à cause des frais que le prix de la marchandise ne peut porter. Ainsi, voilà le parti de la grande existence des blés victorieux, et qui a entièrement détruit son ennemi : on appelle cet avantage, qui est le bon marché des grains, très faussement celui du menu peuple ; et c’est une victoire dont il paie dans la suite la folle enchère au triple, sans parler du mal présent, qui est la cessation de toute sorte de travail.\par
En effet, cette dissipation de blés dans une année abondante, causée par la nécessité du laboureur, et cette négligence de la culture, qui ont supprimé les provisions et les précautions contre les effets d’une année stérile, qui ne manque jamais d’arriver de temps en temps, font qu’on est pris au dépourvu par cette année stérile : c’est alors que la chance tourne du tout au tout, et que la première cherté qui l’accompagne nécessairement reçoit les mêmes degrés de hausse, des mêmes causes qui avaient produit l’avilissement dans l’abondance.\par
Il ne faut qu’une petite quantité de vente du laboureur pour satisfaire aux obligations journalières du ménage : ainsi il croit être en droit, comme il est en pouvoir, de tenir ferme avec le surplus dans sa maison, et bien loin de rapporter le grain sans le délier du marché, il ne se donne pas la peine de l’y voiturer. Ainsi, beaucoup moins de vendeurs, et bien moins d’obligation de vendre ; et par conséquent les excès de cherté dont on n’a que trop fait expérience depuis quarante ans en France.\par
Bien que tout ce qui s’est dit dans ce chapitre prouve assez cette naissance réciproque que se donnent la cherté et l’avilissement, par rapport seulement à la simple attention du commerce et de la venté des grains, cette vérité paraîtra encore bien plus constante quand oh viendra à descendre dans le détail de l’agriculture, qui donne le premier sort à cette situation différente du prix des grains ; en sorte qu’on peut dire, comme dans la musique, que c’est lui qui bat la mesure, et qui assigne à chacun sa partie, comme on va faire voir dans le chapitre suivant.
\subsection[{Chapitre II.}]{Chapitre II.}
\noindent Si la terre en France produisait le blé comme elle fait les truffes et les champignons ; que ce fût un pur effet de sa libéralité, qui n’exigeât aucuns frais ni soins pour la culture, en sorte qu’étant nécessaire de tout attendre de sa bonté purement gratuite, les attentions ou les travaux n’auraient aucune part au plus ou moins d’abondance de la récolte, la raison dicterait d’elle-même de ménager avec la dernière rigueur la seule ressource de la garde, qui resterait pour empêcher la disette dans les années que la terre et le ciel ne seraient pas favorables à la production.\par
Il en pourrait être encore comme l’on vient de le dire, si la culture, ou l’acquisition de ces mêmes blés, coûtait aussi peu dans ce royaume qu’elle fait en Égypte, où l’on prétend que c’est le Nil qui prend pour son compte et les frais des quatre labours qui sont nécessaires presque partout ailleurs pour préparer les terres, et ceux des engrais et améliorations que l’on est obligé d’y apporter, de telle sorte qu’il ne reste plus qu’à jeter la semence sur sa vase, et à attendre, sans aucune crainte de froid, gelée et orage, qu’elle ait rendu sa valeur avec usure ; ce qui a fait appeler ce pays autrefois {\itshape le grenier des Romains}, et fait que les dispositions du ciel, qui font presque tout ailleurs, sont comptées pour rien en cette contrée\footnote{{\itshape Pastor ægyptius nunquam respicit cælum.}}.\par
Ce système pourrait encore passer en Moscovie, où la neige, restée sur la terre huit à neuf mois de temps, laisse dans le sol, après être tout à fait fondue, un sel qui, à l’aide d’un simple labour, très facile, remplace toutes sortes d’engrais, et donne, après deux mois seulement de résidence des grains dans le champ, une récolte très abondante.\par
Si les choses avaient lieu de la sorte en France, on aurait assurément tort de vouloir capituler avec les blés, c’est-à-dire exiger ou stipuler un prix certain, afin de labourer les terres, surtout les mauvaises, sans perte de ses frais. On oserait dire, cependant, que c’est sur cette supposition que le peuple raisonne en France, quoiqu’il faille précisément partir d’une toute contraire. Car, bien loin que les terres y soient à beaucoup près d’une pareille libéralité, on peut assurer qu’elles sont toutes ou la plus grande partie très rebelles à la main du laboureur, et avec cela très intéressées, ne donnant rien pour rien, et qu’à proportion des soins et des engrais qu’on leur a prêtés ; et que souvent même lorsque le ciel n’est pas favorable, il s’en rencontre quantité qui font banqueroute, laissant expirer le terme fatal, ou la saison de la récolte, sans rendre ni intérêt ni capital, c’est-à-dire la semence.\par
Comme elles se divisent en plus de cent classes différentes de mérite, elles sont exposées plus ou moins à voir décider leur sort pour la culture uniquement par le prix des grains. Comme toutes choses ne peuvent être portées dans leur perfection si l’intérêt de l’ouvrier ou de l’entrepreneur ne s’y rencontre, il y en a plus de la moitié que l’on ne saurait ménager avec les engrais nécessaires, proportionnés à l’ingratitude du terroir, le bon blé étant à neuf à dix francs dans Paris, c’est-à-dire cinq à six francs le petit grain dans les provinces. Il est donc impossible, quand le mal continue, que le laboureur ne souffre le sort marqué dans la première partie.\par
Ainsi, on ne peut contester que le prix des blés est un baromètre immanquable qui fait hausser et baisser la culture des terres à mesure qu’il augmente ou qu’il diminue. On en use de la sorte à leur égard, d’abord sur l’article des engrais, et enfin par un abandon entier lorsque le mal est extrême, et que les prétendus vœux des personnes pitoyables sont exaucés, c’est-à-dire le blé en perte au laboureur.\par
Ce n’est pas tout : cet abandon, ou des engrais ou de la culture entière d’une quantité de terres, n’est qu’une partie du mal que cause l’avilissement du prix du blé ; puisque, si d’un côté l’intérêt particulier fait prendre ce parti, il cause encore un autre effet non moins dommageable, savoir de prodiguer la consommation des blés à des usages tout à fait étrangers, comme nourriture de chevaux, engrais de bestiaux, et confection de manufactures, ainsi qu’on a dit ; pour après, par un sort tout contraire, lorsque cet avilissement a causé la disette, à la première année stérile, comme cela est impossible autrement, obliger les hommes à avoir recours à la nourriture des bêtes, savoir les avoines, la chair des animaux, comme chevaux, et même l’herbe ; ce qui n’est pas sans exemple, parce que ces mêmes bêtes, dans le trop grand avilissement des grains, avaient usurpé une pâture seulement destinée à l’usage des hommes.\par
L’on voit par tout ce raisonnement ou cette exposition de faits incontestables, que ces deux grands ennemis, l’avilissement des grains et leur excessive cherté, qui sont choses contraires, se trouvent dans une guerre continuelle, et qu’ils n’ont ni repos ni patience, qu’ils ne se soient terrassés réciproquement, pour renaître après cela comme des phénix de leur propre cendre, et reparaître plus violents que jamais.\par
En effet, sans traiter la question de savoir qui commence la querelle, n’est-ce pas ainsi que les choses se passent ? — Une cherté extraordinaire fait labourer avec attention et profit les plus mauvaises terres et ne rien négliger pour augmenter la levée des meilleures, ce qui, joint à une attention et un ménagement continuel de l’usage de toutes sortes de grains, comme d’une marchandise très précieuse, forme une abondance dans le royaume plus que suffisante à ses besoins ordinaires ; mais parce que cet excédent ne trouve pas l’évacuation au-dehors qui serait nécessaire, comme il arrive dans ce qui se passe à l’égard du corps humain, ce superflu est un levain contagieux à l’avènement d’une année fertile, qui corrompt, par un avilissement effroyable, toutes les matières naguère si précieuses, et produit les résultats désastreux tant de fois marqués.\par
Puis le bas prix, à son tour, a sa revanche ; et par l’abandon ou négligence de culture et prodigalité d’usage des grains, une année stérile faisant pencher la balance de l’autre côté, voilà une cherté effroyable, et ses suites monstrueuses qui paraissent tout à coup, et que tout le monde déplore, sans que personne jusqu’ici se soit avisé ou ait pu comprendre que c’est l’effet uniquement des vœux des gens charitables et des mesures aveugles prises pour seconder un zèle si mal fondé.\par
On voit donc qu’il est absolument nécessaire, pour éviter ces deux extrémités, de faire la paix entre elles, ou plutôt de ne leur donner pas continuellement une semence de guerre : il y a même longtemps qu’elles ne se seraient pas donné de si rudes secousses ni livré de si furieux combats, si une main étrangère, par des opérations tout à fait hors-d’œuvre, n’avait pas marqué se défier de la nature, et qu’il n’était pas à propos de s’en rapporter uniquement à elle pour la dispensation de ses faveurs, bien qu’on lui soit redevable en partie de tout ce que produit la terre. C’est ce qu’on va montrer encore mieux dans le chapitre suivant.
\subsection[{Chapitre III.}]{Chapitre III.}
\noindent On est persuadé que qui que ce soit ne peut révoquer en doute, après ce qu’on vient de dire, que l’avilissement des grains ne produise la cherté extraordinaire, comme celle-ci à son tour donne la naissance à celle qui l’avait enfantée ; ainsi, il est constant qu’il ne faut qu’arrêter une de ces deux situations pour les faire cesser toutes deux à jamais.\par
D’abord qu’il paraît la moindre crainte d’un haussement de prix des grains, on écrit dans les pays étrangers et on tâche d’en faire venir de tous côtés, et ces mesures sont très naturelles ; et même, quelque soin qu’on prenne, on se trouve souvent court dans toutes ces précautions ; de façon qu’en venant annoncer comme on a fait, et dont on conviendra assurément pour peu qu’on fasse réflexion à ces Mémoires, qu’il y a un moyen certain de se garantir de cette extrémité, qui passe pour un des fléaux de Dieu ; savoir d’en éviter un autre, qui est l’extrême avilissement de ces mêmes grains, on maintient que c’est rendre à la France le plus grand service qu’elle puisse jamais recevoir, tant par la comparaison du passé que par rapport à l’avenir, et par le mal que l’on fera cesser et par le bien que l’on attirera.\par
Pour contrepied au désordre de l’avilissement causé par tant de maux, il faut vendre du blé aux étrangers, ce qui, outre le mal que cela bannira pour jamais, savoir et l’anéantissement et la famine, également dommageables, changera la situation de la Franco à l’égard des étrangers en les rendant redevables, de créanciers qu’ils étaient auparavant, ainsi qu’il est constant.\par
Du moment que l’on parle d’enlèvement de blés, aussitôt le monde se soulève, tant le peuple, qui est aveugle, que les personnes les plus éclairées ; et l’on croit que l’avarice insatiable des propriétaires des grains veut sacrifier la vie des misérables à leur avidité. Cette erreur est si profondément enracinée dans l’esprit, par la faute marquée au commencement de ce Mémoire, savoir le manque d’union de la pratique et de la spéculative du labourage, ce qui, en cette occasion comme partout ailleurs, n’enfante que des idées monstrueuses, que l’on ose dire qu’un homme ressuscité aurait peine à faire revenir la plupart des gens de cette prévention. Cependant le faible d’une pareille disposition sera de beaucoup augmenté par le détail qu’on va faire de la quantité pitoyable qu’il est nécessaire de faire sortir au-dehors, afin d’empêcher les pernicieux effets des deux extrémités de cherté et d’avilissement de grains si opposées, et en même temps si unies à ruiner également un État.\par
On sera bien honteux lorsqu’il paraîtra clair comme le jour, comme il va arriver, qu’il est seulement question de semer, non pour recevoir vingt pour un, qui est la plus forte usure que donnent les terres les plus abondantes, ni même cinquante, mais plus de cent pour un, ce que l’agriculture ne connaît point. En sorte que l’on maintient que le même ridicule qui se rencontrerait dans un homme qui soutiendrait qu’il ne faudrait pas semer la terre lorsqu’on craindrait la cherté, de peur que l’État ne se trouvât dépourvu de blés pour la nourriture des hommes pendant l’année courante, se trouve dans le raisonnement de ceux qui veulent qu’on ne laisse point sortir de grains hors le royaume qu’après plusieurs récoltes consécutives très abondantes ; c’est-à-dire que, outre les malheurs ci-devant marqués, on ne pourra, dans cette disposition, mettre cette marchandise à profit qu’après qu’on en aura perdu une très grande partie, et cessé d’en faire produire à la terre encore une plus considérable.
\subsection[{Chapitre IV.}]{Chapitre IV.}
\noindent Les auteurs de la conduite ou du raisonnement que l’on combat dans ce Mémoire ne tombent en une erreur si grossière que parce qu’ils raisonnent à l’égard des blés comme un gouverneur de place frontière qui craint un siège, ou comme un maître d’arithmétique qui sait et qui est assuré que, quand de cinq on ôte deux, il reste trois ; tout comme l’homme de guerre est certain que plus il sortira de blé de sa place, moins il en restera ; et qu’ainsi c’est autant de renfort qu’il donne à son ennemi, pouvant être pressé par la disette, si la place vient à être bloquée.\par
Voilà donc les idées qui se présentent à la spéculation, laquelle ne peut s’empêcher de traiter d’extravagance tout ce qu’on peut rapporter au contraire. Mais outre tout ce qu’on a dit ci-dessus, qui montre assez le faible ou l’erreur pitoyable de ce raisonnement, on va faire voir un détail de la quantité de blés et de grains qui peuvent croître en France, ainsi que du nombre dont le royaume a besoin pour sa consommation ordinaire ; et l’on verra que c’est leur prix seul qui ensemence les terres, depuis les plus mauvaises, où de mémoire d’homme on n’a jamais vu rien croître, jusqu’aux mieux partagées de la nature ; et puis il y a encore un sous-ordre, ou une subdivision de divers degrés de fécondité, de stérilité, ou d’abondance dans la récolte, qui reçoit le taux ou ses ordres de ce même prix, qui met plus ou moins en état de faire les frais nécessaires dans le ménagement, d’où dépend absolument le sort d’une bonne ou mauvaise levée.\par
L’empire même que le prix des grains se donne dans ce commerce ne s’en tient pas là : il étend également ses ordres et son pouvoir sur la consommation, ainsi qu’on a dit ; il la suit pas à pas, et la hausse ou baisse de moitié à autre, ou plutôt du tout au tout, ainsi qu’il fait le labourage, sans perdre jamais l’un et l’autre de vue ; et c’est ce qui justifie les Anglais de n’avoir pas perdu le sens, comme il faudrait supposer, si le raisonnement contraire n’était pas erroné, de donner de l’argent à pur profit à ceux qui vendent les blés du pays aux étrangers, et même à leurs plus grands ennemis, attendu qu’il en faudrait donner même aux démons s’ils en faisaient la demande, puisque c’est pour éviter un très grand mal et se procurer en même temps un très grand bien.\par
C’est par là qu’ils font défricher tous les jours une infinité de terres qui ne l’avaient encore jamais été, en soutenant les blés à un prix qui puisse satisfaire aux frais nécessaires pour y parvenir ; et ainsi recueillant assurément cent pour un qu’ils ont fait sortir, ils évitent et les horreurs de la stérilité, et ceux de l’avilissement.\par
Sur ces principes, on maintient qu’année commune, il croit presque toujours en France une moitié plus de blé qu’il n’est nécessaire pour sa consommation ordinaire ; cela peut aller à dix-huit cent mille muids ou à deux millions, ou trois millions mesure de Paris ; dont il en faut à peu près les deux tiers pour le dedans du royaume : ainsi, sur le pied de quatorze à quinze millions de créatures qu’il peut y avoir en France, à cinq quarterons par jour par tête, c’est douze cent mille grands muids de consommation, et six ou huit cent mille d’excédent qu’il faut absolument perdre, si après plusieurs années consécutives d’abondance, qui soutiennent les choses à peu près sur ce pied, il n’y a aucune sortie permise, ni liberté d’en donner aux étrangers, qui, bien loin d’être une garantie contre les accidents d’une stérilité ou d’une cherté extraordinaire, est, au contraire, ce qui l’avance et ce qui la produit, ainsi qu’on a montré d’une façon invincible. On ne répétera donc point ce que l’on n’a que trop détaillé ; mais on fera seulement remarquer que la culture et la production de ces six à huit cent mille muids, qui excèdent la consommation ordinaire du royaume, ne rendent pas leurs frais, la tête de blé étant à neuf ou dix francs le setier à Paris, c’est-à-dire le petit blé à cinq ou six livres dans les provinces. Et, si les maîtres, dans ces occasions, ne faisaient crédit à leurs fermiers des quatre ou cinq années de suite, en attendant une stérilité, après laquelle ils ne soupirent pas moins ardemment que les juifs après le Messie, il est constant qu’ils périraient tous, et que presque toute la France demeurerait en friche.\par
Car enfin, ainsi que l’on a dit, toutes les terres n’étant pas d’un pareil degré, à beaucoup près, de fécondité ou de facilité d’exploitation, y ayant même plus de cent degrés de différence entre elles ; dans cette rencontre, c’est uniquement le prix du blé qui décide de leur sort, et de celui du laboureur, à l’égard du profit ou de la perte qu’il y a à les faire valoir.\par
En effet, si le prix ne manquait point de garantie, non seulement il n’en proviendrait pas deux millions de muids, comme il arrive ordinairement, mais même ce nombre pourrait doubler, et même tripler naturellement, sans rien supposer en cela que de très possible.\par
Il est très assuré qu’il y a des terres qu’on ne laboure jamais, par le manque qu’on vient de marquer ; d’autres que de quinze années une ou deux ; d’autres que tous les sept ou huit ans, et presque toutes se reposent, au moins de trois années une ; pendant qu’il s’en rencontre de plus mal partagées que celles-là, à qui naturellement on ne devrait rien demander, qu’on laboure toutes les années, et qui rapportent même jusqu’à deux récoltes dans un même été.\par
La raison de cette différence est, que n’y en ayant aucune qui soit à l’épreuve, et qui puisse résister à la quantité d’engrais possible et nécessaire à les rendre fécondes, du moment que celles de ce genre se trouvent situées dans des lieux où on leur peut procurer cet avantage à un prix qui ne soit pas au-dessus de celui des fruits de la récolte, on ne manque jamais de prendre ces mesures à leur égard : ce sont celles qui se trouvent aux portes et environs des grandes villes, lesquelles, nonobstant leur défaut d’être pierreuses ou sablonneuses, sont toutes érigées en potagers, et même à porter des blés toutes les années, sans avoir jamais un moment de repos. La raison de cela est que les fumiers des villes n’ayant point d’autre intérêt que d’en être enlevés au plus tôt, le terrain limitrophe a la préférence du transport à cause de la proximité, laquelle produit encore la faculté du débit des fruits de ce terroir abonni malgré la nature ; et cette violence qu’on lui fait s’éloigne et gagne le pays au-dehors à proportion du prix des grains, jusque-là qu’on a vu des laboureurs à deux lieues d’une ville maritime, entretenir deux chevaux et un valet tout le long de l’année, pour aller quérir seulement deux charges par jour de certains immondices, arrosées d’épanchement d’eaux salées, qui ont la vertu de tripler les effets de toutes autres sortes d’engrais ; c’est-à-dire que ces laboureurs dépensent huit cents francs par an, en faisant faire tous les jours huit lieues à leurs chevaux, pour abonnir seulement quinze ou seize arpents de terre ; et c’était avec profit, les blés étant à 16 ou 18 francs à Paris ; comme c’était avec perte, ou plutôt qu’on laisse cette manœuvre, dès qu’ils ne sont plus qu’à 9 ou 10 francs.\par
C’est sur ce compte que les Maures, ayant été chassés d’Espagne au commencement du siècle passé, se présentèrent à la France, et lui offrirent, si on leur voulait donner à habiter la contrée la plus stérile et la plus inculte qui se rencontrât, comme la grande Provence, ou les landes de Bordeaux, de la rendre la plus fertile du royaume. Quoique cela paraisse surprenant, cela est pourtant très certain, et ils en seraient venus à bout. Voici comment : comme ils avaient emporté des effets mobiliers, c’est-à-dire beaucoup d’argent, ils l’auraient tout employé à faire souffrir à ces lieux stériles le sort des terroirs semblables qui se rencontrent aux portes des grandes villes : comme il n’y aurait eu aucune différence du côté de la nature, mais seulement des frais, la récolte, soutenue de la frugalité de ces peuples, les aurait dédommagés, ce qui ne se rencontre pas chez ceux du Septentrion, qui mangent beaucoup davantage, et veulent faire meilleure chère ; et si ces Maures avaient été en perte dans la première et seconde année, ils ne l’auraient assurément pas été dans la suite, et se seraient même récompensés du passé, et enrichis pour toujours : la raison de cela est que, dans le labourage, ce sont les premières années qui coûtent le plus ; que c’est d’elles que le laboureur reçoit sa destinée pour toute son exploitation ; s’il est assez fort pour n’y rien épargner, il est riche pour toute sa vie ; sinon, il y perdra assurément tout ce qu’il y aura mis.\par
En effet, c’est une vérité connue de tous ceux qui ont jamais fait ce commerce, qu’en matière de labourage, l’abondance produit l’abondance, et la misère de même : un fermier qui a fait des frais infinis d’acheter des fumiers et des pailles, qui ne sont qu’une seule et même chose lorsqu’on a des bestiaux, se procure une heureuse récolte, c’est-à-dire une grande abondance de ces mêmes fourrages, qui lui donne le moyen de reformer les fumiers sur le lieu ; il n’est plus obligé de les acheter, ni de les aller quérir au loin, mais entretient cette circulation toute sa vie, à moins qu’un trop long avilissement des grains, produisant sa ruine, ne l’oblige à tout quitter ; qui est une perte pour tout l’État, d’autant plus grande, que la cause étant générale, elle porte cette même destinée en une infinité d’endroits.\par
On voit donc, par tout ce qu’on vient de dire, que c’est uniquement le prix des grains, quoique cette vérité ait été jusqu’ici si peu connue, qui décide et de l’abondance et de la richesse du royaume. Mais la surprise sera encore bien plus grande lorsqu’on viendra à approfondir, comme on va faire dans le chapitre suivant, la grandeur de la méprise dans laquelle on a vécu jusqu’ici en France sur cet article, puisqu’on va faire voir que tous les malheurs de l’une et l’autre situation d’avilissement ou de cherté de grains, ne sont arrivés que parce qu’on a cru s’en garantir en empêchant trois ou quatre mille muids de blé de sortir du royaume par an, bien qu’il n’y eût aucun muid de cette réserve qui n’en ait fait périr plus de cent pour sa part, toutes les années l’une portant l’autre, et fort souvent trois cents ; sans parler de près de cinq cents millions de rente que cette conduite coûte en pure perte au royaume, et la vie à une infinité de monde, et la ruine de toutes les conditions, qui n’ont du bien au sou la livre, depuis la plus élevée jusqu’à la plus abjecte, qu’à proportion que les fruits de la terre, et surtout les blés, sont non en existence, mais en valeur, dont l’antipode est lorsqu’ils ne peuvent porter les frais de la culture.
\subsection[{Chapitre V.}]{Chapitre V.}
\noindent L’avilissement du prix des grains, comme leur extrême cherté, qui en est une suite nécessaire, étant le plus grand mal qui puisse arriver au royaume, tout ce qui y donne lieu doit être regardé avec le même degré d’horreur. Or, la défense de faire sortir des blés étant cela même qui produit cet avilissement, c’est à elle seule qu’il faut déclarer la guerre ; mais avant de le faire, il est à propos de purger l’erreur publique, et qui est la première idée qui se présente à l’esprit lorsqu’on n’est pas rompu dans ce commerce, savoir que l’on ne peut ôter du blé d’un tas ou d’une quantité, sans diminution ou sans perte sur le nombre : outre que cela n’est pas absolument vrai, puisque sur ce principe on ne sèmerait jamais ; de la même sorte, si une diminution augmente le prix du restant, et que l’enlèvement d’une petite quantité procure des soins pour la conservation du surplus, qui ne se peut faire sans frais, il sera certain de dire que l’enlèvement d’une partie augmente, loin d’amoindrir la masse dans la suite.\par
Mais il y a plus, cette sortie de blés, quelle qu’elle soit dans la plus grande liberté aux étrangers d’y en venir prendre, a si peu de rapport à la quantité nécessaire pour la subsistance du royaume, qu’elle n’est non plus considérable par la crainte de la diminuer, que si un munitionnaire d’armée ayant fait marché de fournir le pain à vingt onces de poids chacun, on viendrait dire qu’il aurait affamé l’armée parce qu’il aurait manqué la pesanteur d’un demi-gros ou environ dans la livraison, d’autant plus que cette justesse ne s’est jamais rencontrée dans le débit de cette denrée.\par
En effet, on ne ravitaille point un grand royaume naturellement fécond, comme on fait une ville ou un vaisseau, où il ne croît aucuns grains : cependant il est vrai de dire que, si dans les extrêmes chertés on n’en apportait de dehors, la moitié du peuple périrait, bien que cet apport ne soit pas capable de lui-même de nourrir la cinquième partie du monde à qui il sauve la vie ; mais voici comme les choses se passent. On a fait voir ci-dessus, que les grains ont deux faces, et produisent deux effets fort opposés l’un à l’autre, qui se font une guerre continuelle, savoir l’un de nourrir l’homme, et l’autre de fournir au propriétaire de quoi avoir le surplus de ses besoins, de quelque nature qu’ils soient. Le premier fait ce qu’il peut, et n’a d’autre but que de l’avoir à très vil prix, indépendamment de toute sorte de justice et d’équité, et même des conséquences, quelque terribles qu’elles puissent être, comme on l’a montré ; et l’autre, tout au contraire, ne respire qu’à le voir dans l’excès avec aussi peu de raison : les années stériles ou abondantes font gagner la cause à l’un ou à l’autre. On a parlé des suites de ces premières, ainsi que de celles de l’autre parti ; mais il est à propos de faire encore mention de celles-là, par rapport à ce qu’on s’est engagé de prouver, savoir que ce qu’on apporte en France de blés, ou qu’on y enlève, n’intéresse non plus par sa quantité la nourriture des peuples, que la diminution marquée ci-dessus au pain de munition.\par
Pour le montrer, il est nécessaire de descendre dans le détail de la manière dont les chertés désolantes, pour ne pas dire famines, arrivent : c’est un pur effet de la brutalité et de la bêtise du peuple, et non absolument de la stérilité de la terre, dans un pays comme la France, quoiqu’elle y donne lieu ; c’est cette foule confuse de gens sans tête, sans cervelle, qui se filent le cordeau dont ils sont étranglés.\par
On sait les effets de la terreur panique, lorsqu’elle s’empare des esprits de toute une armée, puisqu’on a vu quelquefois deux ou trois cents hommes en mettre plus de dix mille en fuite, lesquels pour garantir leur vie, sans même être poursuivis, se précipitaient dans les fleuves, et se noyaient presque tous. On a vu dans des bateaux de passage remplis de monde, au moindre trou qui paraissait par où l’eau entrait, et qui eut été aisé à étouper, tous se jeter en foule sur l’autre côté, et par là renverser le bateau et se noyer tous.\par
C’est par la même conduite que ces chertés extraordinaires arrivent, puisqu’on n’en a jamais vu aucune, quelque grande qu’elle fût, qu’il n’y eût encore plus de blé en France ou de l’année, ou des précédentes, qu’il n’en fallait pour nourrir tous les peuples. Et, pour le faire voir, il n’y a qu’à considérer que si, en 1693 et 1694, on avait réduit en monnaie tout l’or et l’argent du royaume qui est en vaisselle, même celui des sacristies, comme portent les canons dans ces occasions, cela aurait assurément formé plus de deux cents millions ; et que l’on eût donné quatre ou cinq pistoles à chacune des trois ou quatre millions de personnes seulement exposées aux effets de la disette, non seulement, aucune n’aurait péri, mais même n’aurait jeûné un seul moment : cependant, tout cet argent n’aurait pas été du blé, et ne l’aurait pu former s’il ne l’avait pas déjà été ; mais il l’aurait forcé de sortir des réduits où l’inhumanité des possesseurs le détenait, par le malentendu de la conduite des peuples.\par
Ce qui fait donc la balance entre ces deux partis ci-devant marqués, et qui sont si fort ennemis l’un de l’autre, quoiqu’ils doivent être toujours en équilibre, autrement l’État souffre de quelque côté que soit l’avantage ; ce sont les marchés où l’on vend publiquement les grains ; ce sont eux qui décident du sort des peuples, de façon ou d’autre, à l’égard du prix des blés. En effet, un marché ou étape publique, où il se vend ordinairement cinq cents setters de blé toutes les semaines, n’en peut voir l’altération dessus ou dessous de vingt seulement, sans que ces mêmes grains ne reçoivent une hausse ou une diminution très considérable, qui s’augmente à vue d’œil, et qui double et qui triple par le moindre surcroît tous les effets précédents ; de même qu’une balance suspendue en équilibre, parce que le poids est égal dans chacun des deux bassins, comme de cent livres de quelque matière que ce soit, ne peut recevoir une augmentation de deux livres seulement en un de ses plateaux, sans que l’autre ne soit emporté entièrement, et ne descende aussi bas, en faisant remonter celui qui a perdu le contrepoids aussi haut, que s’il n’y avait rien du tout, et que toute la charge fût en un seul. Voilà justement ce qui se passe dans les marchés à l’égard du prix des blés : une surcharge ou une diminution de vingt sacs sur la fourniture ordinaire, encore une fois, du marché ou étape de cinq cents sacs d’apport chaque semaine, emporte la balance et la fait pencher tout à fait d’un côté ; et comme du mal en ces occasions il vient le mal, l’avilissement des blés produit l’avilissement, et la cherté le haussement continuel de prix. Il arrive à l’égard de cette balance de marchés, que lorsqu’un côté a emporté l’autre par l’altération que l’on vient de marquer, la surcharge qui arrive à toute heure porte les choses à un excès, de façon ou d’autre, également préjudiciable à l’État.\par
Et comme entre la très grande cherté des grains et leur plus fort avilissement il y a sept ou huit degrés au moins de différence, et qu’il vaut dans ces occasions sept fois plus ou sept fois moins que dans la situation opposée ; ce serait aussi mal raisonner de dire dans la cherté, qu’il y a sept fois moins de blé qu’il ne faut pour la nourriture de la France, parce qu’on l’a vu dans les années précédentes à sept fois meilleur marché ; tout comme, dans l’avilissement, d’avancer qu’il s’en trouve sept fois plus qu’il n’est nécessaire pour la consommation habituelle ; et enfin, c’est la même extravagance que si on disait, dans cet exemple de balance, mise d’abord en équilibre par une égalité de poids, et puis tirée de cette situation par une surcharge de deux ou trois livres, qui fait qu’un côté emporte tout à fait l’autre ; que si on avançait, dis-je, qu’il n’y a rien du tout dans un plateau et que tout est dans l’autre, parce que la situation n’en est point différente que si cela était effectivement : cependant il n’y a rien de plus faux, puisque, faisant le même parti de deux ou trois livres de surcharge au côté emporté, on rétablirait l’équilibre. Cette différence de sept degrés du prix des blés est que, dans la cherté, le laboureur est sept fois moins pressé de vendre, et, dans l’avilissement, sept fois plus dans l’obligation de se défaire de sa denrée, poussé par le maître ou par l’intérêt, ce qui forme le contrepoids.\par
Il faut faire trêve pour un moment avec cette parité de balance, pour faire une digression sur la manière dont les chertés extraordinaires arrivent, leur naissance, leur progrès, et comme elles reçoivent leur excès de désolation ; et on sera surpris de voir que ce n’est qu’un malentendu, et le plus souvent une terreur panique du peuple, qui l’oblige à se précipiter la tête la première dans un fleuve très profond et très rapide, pour fuir un ennemi qui n’a ni pieds ni jambes pour l’atteindre, ni armes pour l’offenser.\par
On ne peut pas dire que le ciel, qui n’est pas toujours également favorable à la terre pour concourir à la perfection de ses fruits, ou plutôt qui ne l’est jamais d’une égale manière, ne donne pas le premier lieu à cette disposition : une longue sécheresse, une grande abondance de pluie, un hiver rude et fâcheux, sans neige, qui est une excellente couverture aux blés contre les rigueurs du froid, et enfin une petite pluie emniellée qui attaque ordinairement le tuyau un peu avant sa maturité, et le met absolument hors d’état de nourrir davantage le grain dans l’épi, sont autant d’ennemis que cette manne primitive des hommes dans l’Europe doit essuyer, et non pas combattre, ou du moins autrement que par des vœux. Du moment que quelqu’un de ces dérangements a produit son effet, un peu plus tôt, ou un peu plus tard, de suite l’alarme se répand parmi le peuple, que l’année ne sera pas opulente, et que les blés ont manqué en quantité de contrées ; et il en arrive comme dans toutes les rumeurs publiques, on fait le mal beaucoup plus grand qu’il n’est. Le désordre commence par la campagne, dont les habitants ont un double intérêt de répandre ce bruit : le premier afin de faire hausser le prix des grains, et le second pour se dispenser de payer leurs maîtres, alléguant, le plus souvent contre vérité, qu’ils n’ont pas recueilli de quoi ensemencer leurs terres, et se nourrir eux et leurs familles : tout le reste du menu monde, qui est extrêmement disposé à prendre le ton plaintif, soit par un chagrin naturel, ou par dépit de n’être pas dans une meilleure fortune, donne encore une rehausse à la commune renommée, sans connaissance de cause et plus grand approfondissement, de quoi même il n’est pas capable.\par
Ainsi, voilà aussitôt deux effets qui suivent le premier, savoir que tous les vendeurs de blés, dans l’espérance que le mal augmentera, s’abstiennent de fournir les marchés à leur ordinaire, n’oubliant rien pour obtenir de leurs créanciers un délai de paiement, dans la promesse de leur en faire de bien plus considérables avec le temps ; et l’autre, que ceux qui font leur provision de blés ordinairement de semaine en semaine ou de mois en mois, se hâtent au plus tôt de se fournir pour toute l’année, et même davantage, le tout sur une terreur panique d’un mal qui n’est grand que parce que la fantaisie et l’erreur font croire ce qui n’est pas.\par
Cependant, il advient de ces deux effets d’une stérilité, qui n’est souvent que factice en la plus grande partie, une suite aussi réelle que si elle était véritable : savoir un rehaussement de prix des grains, attendu que, pendant que les marchés sont moins fournis d’un côté que par le passé, ils sont plus dépouillés qu’à l’ordinaire ; ces dispositions augmentent suivant et à proportion de la renommée.\par
Ce n’est pas tout : quand l’année se trouverait très abondante, et que le peuple se serait mépris dans ses conjectures ou ses idées, le mal ou le rehaussement qui a pris racine ne s’arrête pas pour cela, au moins en partie, attendu que, comme lorsque les grains sont à vil prix aucun laboureur ou marchand ne vendrait, si la nécessité de payer ses dettes ne le talonnait de près, ce qui fait que, dans l’avilissement, il est obligé de faire main-basse sur tout à cause qu’il faut beaucoup de blés pour faire peu d’argent ; de même, il est tiré de cette situation par le haut prix, qui le met en pouvoir de moins vendre pour satisfaire à ses obligations, et ainsi de moins fournir les marchés.\par
Voilà donc la balance, pour y revenir, qui a perdu son équilibre, car ce sont les marchés seuls qui décident souverainement en cette occasion, et non la quantité des blés, quelle qu’elle soit, qu’il peut y avoir, ou dans les greniers, ou dans les granges des métairies : vingt sacs dessus ou dessous dans un marché font le sort des grains, pendant qu’une fois plus ou moins, repostés dans les lieux qu’on vient de marquer, ne change en rien leur destinée. Même, toutes les fois que la police a voulu y mettre la main, pour obliger les propriétaires des grains de fournir régulièrement les marchés, avec défense de trop garder de blés dans les étapes publiques, y ayant une infinité d’ordonnances imprimées et publiées sur ce sujet, on peut assurer que cela n’a fait qu’augmenter l’alarme, ainsi que le mal, bien loin de le diminuer.\par
C’est donc dans ces rencontres que les blés étrangers font des merveilles, et ont sauvé la vie à une infinité de monde dans plusieurs occasions, non par leur quantité, qui ne va pas à plus gros qu’un pois de pain pour chaque personne, par rapport à la quantité d’hommes qu’il y a dans la France, mais parce qu’ils remettent l’équilibre dans la balance ; et tout comme il serait ridicule de dire qu’un plateau d’un poids de cent livres, et qui aurait absolument emporté l’autre dans lequel il n’y aurait rien, pourrait être établi en équilibre en remettant seulement deux livres dans le plateau vide, il serait de la même absurdité de prétendre que vingt ou trente mille muids de blé sauvent la vie au peuple d’un royaume, à qui il faut plus de douze cent mille muids par an ; mais qu’au contraire, comme on a remarqué ci-dessus, ce côté de balance que l’on croyait absolument vide parce qu’on le voyait tout à fait emporté en haut, ayant déjà cent livres pesant et venant à recevoir deux livres d’augmentation, reprend l’équilibre que l’autre bassin avait gagné sur lui par la surcharge d’un pareil avantage.
\subsection[{Chapitre VI.}]{Chapitre VI.}
\noindent Pour expliquer encore plus nettement le rôle du commerce des blés à l’égard de l’étranger tant dans l’envoi au-dehors que pour la réception au-dedans, on peut dire que tout y est violent et extrême, parce que tout y est exposé à la fougue d’un public, ou plutôt d’une troupe aveugle et tumultueuse qui ne sait ni ce qui lui convient ni ce qui lui est préjudiciable : c’est assez que le peuple se trouve assemblé pour former une sédition, et comme il prend l’alarme jusqu’à se soulever de la sortie d’une très petite quantité de grains, mille fois au-dessous de celle que le bas prix en fait anéantir ou par négligence de labourer, ou par prodigalité à consommer, il croit tout à fait être tiré d’une crainte de disette par l’arrivée d’une petite quantité de grains étrangers.\par
L’année 1679 aurait vu les mêmes désastres que celles de 1693 et 1694, sans vingt-cinq ou trente mille muids de blés étrangers au plus qui conjurèrent assurément le mal, parce qu’ils étaient arrivés avant que le prix eût gagné un taux trop violent ; ce qui, n’ayant pas eu lieu en 1693 et 1694, une quantité plus forte ne put se rendre maîtresse du désastre, et il advint ce que l’on voit tous les jours dans les incendies, que le feu s’éteint aisément dans le principe, mais non pas quand il a gagné beaucoup de terrain : la balance donc est la nécessité de vendre et d’acheter ; voilà les deux bassins où le moindre poids, soit d’un côté, soit de l’autre, produit une baisse ou une hausse qui va toujours en augmentant.\par
Tout ceci montre évidemment, encore une fois, que la réception ou sortie des blés étrangers n’est d’aucune considération pour le royaume par rapport à la subsistance, mais seulement à l’équilibre de la balance et au prix : comme l’excès de cherté n’est ordinairement fondé que sur des bruits ou terreurs paniques, ne provenant que de la possibilité qui existe à un degré plus ou moins étendu, pour les laboureurs, de ne pas vendre leurs grains, l’arrivée d’un vaisseau chargé de cette denrée fait une espèce de miracle, parce qu’on ne manque jamais de dire que c’est l’avancement d’une bien plus grande quantité, et cela fort sagement, qui va arriver au premier jour.\par
De plus, comme on a marqué ci-dessus, et c’est la vérité, que la fourniture des marchés seule, se trouvant forte ou légère, fait le sort du prix des blés, indépendamment de quelque abondance qu’il puisse y avoir dans les greniers ou dans les granges, un seul vaisseau de trois à quatre cents muids de blé seulement est comme si l’on portait ce nombre tout d’un coup à un marché qui n’en eût ordinairement que trente à quarante muids aux jours de vente, comme sont tout au plus les mieux accrédités. Il est constant qu’à moins que la cherté ne fût extrême et que les acheteurs ne se fournissent pour plus que leur provision ordinaire ou pour revendre aux autres, le prix tomberait tout d’un coup ; et si cette manœuvre continuait, on pourrait dire que tout serait perdu, comme on a marqué dans la première partie de ces Mémoires.\par
C’est la même chose dans la situation contraire, par la sortie de quelques blés, lors de l’anéantissement du prix ; le peuple, qui ne raisonne pas plus dans ce dernier cas que dans le précédent, pour passer sans nul motif en un instant d’un excès à l’excès contraire, croit que tout est perdu du moment qu’on permet d’enlever des blés, quelque quantité qu’il y en ait de superflu. Il ne faut pas supposer qu’il puisse songer que c’est le prix qui sème et engraisse la terre et qui produit, par conséquent, l’abondance qui entretient la magnificence dans les riches et donne le nécessaire aux ouvriers. Cette attention excède de beaucoup les lumières de gens, lesquels, quoique doués de raison, en ont moins que les bêtes lorsqu’ils opinent tumultueusement ; et comme ils croient tout sauvé par l’arrivée de dix ou douze mille muids de blé et même bien moins, ils pensent tout perdu par la simple permission d’en enlever qui ne pourrait jamais, dans la plus grande liberté, atteindre jusqu’à ce nombre, et qui ne serait pas la cinquantième partie de ce que cet enlèvement conserverait ou ferait produire à la terre de surcroît dans le royaume, par les engrais que cela mettrait en état de n’y pas épargner.\par
Il s’imagine, d’abord qu’il voit cette licence de sortie, qu’on va le prendre à la gorge, et que l’on ne peut pas enlever moins que la moitié des blés du royaume, et peut-être tout ; toutes les réflexions précédentes ou toutes ces vérités, qui sont d’une certitude incontestable, n’entreront jamais dans son esprit ; et ce qu’il y a de plus merveilleux est qu’il communique ce raisonnement, tout dépravé qu’il soit, aux personnes les plus éclairées, mais qui n’ont pas la pratique, parce qu’elles sont dans l’élévation.\par
La piété et la charité chrétiennes viennent encore de surcroît, et l’on se persuade avoir mérité le paradis en disant qu’il faut que les blés soient à bas prix, afin que le pauvre monde puisse subsister. Mais, pour résumer le tout, il est incontestable que la sortie ou l’arrivée des blés en France ne produit point d’autre effet que de redresser la balance lorsqu’elle déroge trop à l’équilibre ; et, comme on prend avec avidité le parti d’en faire venir lorsqu’il est trop cher, c’est une méprise effroyable de n’en vouloir pas user de même pour la sortie quand ils se rencontrent dans une situation opposée, c’est-à-dire dans un grand avilissement.\par
Il se trouve même par cette conduite autant de dérogeance et à la politique, et à la justice, et même à la religion, qu’il s’en rencontrerait dans un juge de police qui, baissant le prix du pain aux boulangers lors de la diminution de celui du blé, ne voudrait point, lorsqu’il hausserait, leur rendre la même justice, et s’aveuglerait assez pour croire que ces malheureux pourraient servir le public et tenir leurs boutiques fournies à leur perte, puisque assurément le parti qu’ils prendraient serait de tout abandonner, de fermer leurs maisons et de prendre la fuite, ce qui attire aussitôt une mutinerie ou sédition, bien loin de procurer l’utilité publique : c’est la même chose des laboureurs, et on tombe dans la même erreur à leur égard.\par
On peut même assurer que l’on n’a pas toujours été dans cette surprise. La liberté a été autrefois entière, hors les temps tout à fait extraordinaires, et on n’avait prétendu en 1650 faire une querelle aux blés par la suppression de cette libre sortie, que pour les obliger de regagner le prix de cinquante ans auparavant, qui était trois fois moindre, quoiqu’ils fussent bien plus criminels qu’ils ne le sont aujourd’hui, de vouloir seulement excéder de moitié le prix de 1650, et cela par les raisons traitées dans la première partie de ce Mémoire. En 1600, ce fut la même chose, une même gradation de prix se rencontrant à remonter cinquante ans auparavant, et les blés, en reconnaissance de cette grâce, avaient triplé tous les revenus en triplant leur valeur, tant en 1600 qu’en 1650, tant pour les ouvriers que pour les propriétaires ; mais on souffre aujourd’hui à peu près cette gradation pour les premiers, et on crie à l’horreur lorsque les seconds demandent la même justice, ce qui est la ruine de tous les deux, ne pouvant point subsister l’un sans l’autre, et leur sort, bon ou mauvais, étant toujours réciproquement solidaire.\par
Il paraît par les Mémoires de M. de Sully, que toutes ses attentions ne tendaient qu’à favoriser la sortie des grains, qu’on croit maintenant devoir presque toujours empêcher par un trait de la plus fine politique, quoiqu’il n’y eût pas disparité pareille dans la situation de ces temps-là, par rapport à la hausse des blés, avec celle d’aujourd’hui, puisqu’il ne s’agit présentement que de leur laisser prendre une moitié de surcroît de ce qu’ils étaient vendus il y a cinquante ans ; et que dans les deux époques marquées ils avaient triplé en pareil espace de temps, ainsi qu’on vient de le dire.\par
Cependant, pour revenir à ce qui se fit en 1600, le parlement de Toulouse ayant voulu, par un zèle très mal fondé, empêcher la libre sortie des blés, M. de Sully en donna aussitôt avis au roi Henri IV, lors éloigné, et lui manda que si cette conduite avait lieu, il ne fallait pas qu’il s’attendît que les peuples pussent payer les subsides ordinaires, et que par conséquent les recettes seraient stériles : ce qui fit que Sa Majesté manda au parlement de Toulouse de se tenir en repos, et d’employer son zèle à quelque autre usage moins préjudiciable à l’État.\par
Néanmoins, le raisonnement du peuple et des gens charitables d’à présent a pour base une idée toute contraire, quand ils se révoltent contre la sortie des blés. Mais, pour abréger matière, on leur demanderait volontiers aux uns et aux autres qu’ils missent eux-mêmes le prix aux blés : si ce doit être au plus bas prix qu’ils aient jamais été, ils n’ont qu’à les mettre à vingt sous le setier à Paris, puisqu’il y était en 1550 ; s’ils trouvent ce prix ridicule, comme il l’est effectivement, et même quelque chose de plus, ils conviennent donc qu’il faut une proportion : or, il n’y en aura pas tant que le prix ne pourra pas porter les frais de la culture à beaucoup près, comme il se rencontre dans la situation actuelle.\par
Sur ce principe ou sur ce raisonnement, le peuple, ainsi que les gens pitoyables qui se récrient contre la sortie d’une très petite quantité de blés, c’est-à-dire la centième partie ou même la millième de ce qu’il faudrait pour la subsistance ordinaire de la population, alors même qu’il ne s’en rencontrerait pas toujours le double, tant de celui excru dans l’année que de celui qui est en garde ; le peuple, dis-je, aurait bien meilleure grâce, et serait bien mieux fondé, d’attaquer les propriétaires des terres qui demeurent en friche pour ne pouvoir supporter les frais du labourage ; tout de même que ceux qui ne font pas les engrais nécessaires aux terres exploitées, parce que cette négligence diminue la récolte de plus de moitié. Ce n’est pas tout, et sa colère ne s’en doit pas tenir là ; il faut qu’il assaille encore tous ceux qui prodiguent les grains à des usages étrangers, comme nourriture et engrais de bestiaux et confection de manufactures. Or, bien que tous ces articles apportent un déchet à la nourriture des hommes de cinquante fois plus fort et plus violent, voire bien souvent de mille, ainsi qu’on fera voir dans le chapitre suivant, que celui qui aurait pu arriver par la sortie de quelque quantité de blés que les étrangers eussent enlevée, et qui aurait empêché cet autre désordre, cependant le peuple, si attentif à ses intérêts, voit tout ce mécompte très tranquillement, il n’y fait pas même la moindre réflexion ; et quoique l’on ne s’en étonne pas, parce qu’il n’en est point capable, il y a lieu d’être surpris que des gens en qui la raison semble avoir établi son principal siège tiennent le même langage. La cause en a été marquée dans la première partie de ce Mémoire, et c’est la même qui avait rempli de fort grands hommes d’une si grossière erreur à l’égard de la figure du monde : quelque effroyable qu’elle soit en cette occasion, elle va recevoir un degré de hausse dans le chapitre suivant, qui donnera lieu de s’étonner que l’esprit humain ait jamais été capable d’une faute si effroyable.
\subsection[{Chapitre VII.}]{Chapitre VII.}
\noindent Toute la cause du désordre marqué dans ce Mémoire consiste en ce que jamais qui que ce soit n’a fait un moment d’attention à la quantité de blés qui pouvait sortir du royaume dans les temps d’une pleine liberté : on a cru qu’il n’y avait nulle différence entre réduire le peuple à la famine et cette licence ; et tout le monde est si bien persuadé de cette maxime, que le moindre enlèvement produit presque les mêmes effets et cause une aussi grande alarme qu’une forte stérilité. De manière qu’on est honteux de dire qu’au lieu de vingt-cinq ou trente mille muids de blé qu’il est possible d’apporter dans le royaume dans le temps de cherté, et que les étrangers voient sortir de leurs ports tranquillement et même avec joie, dans l’idée qu’ils ont avec vérité que cette sortie leur procure la richesse et l’abondance, il serait à peine possible, dans les temps même des plus grands avilissements, d’en tirer dix mille de la France, voire moins, avec grand bruit encore, et sans tomber presque aussitôt dans l’excès tout opposé ; en sorte que tous les malheurs de l’une et l’autre extrémité dont on n’a que trop fait expérience, auraient pu être aisément conjurés par la sortie seulement de mille muids de blé, dans la plus grande partie des années abondantes.\par
Que l’on ne s’étonne point de cette différence de situation ou de remuement d’esprit entre la France et les autres États, les causes ne produisent leurs effets que suivant et à proportion des dispositions des sujets sur qui elles agissent ; et comme, parmi les corps, les uns sont très aisés à émouvoir et les autres très difficiles, de même en France la fausse idée que l’on a sur la sortie des grains a mis les choses sur un pied, que cinquante mille muids de blé, et même cent mille tirés de Hambourg, de Dantzick ou de l’Angleterre, étonneraient moins les peuples que seulement cinquante muids enlevés de France.\par
C’est sur ce compte que l’on maintient que, faute d’avoir vendu mille muids de blés toutes les années, l’une portant l’autre, aux étrangers et peut-être bien moins, la France a perdu plus de cinq cents millions de rente, avec l’obligation de laisser quantité de ses terres en friche et de mal labourer les autres, ainsi que de consommer une énorme quantité de grains à des usages étrangers ; ce qui, joint à l’abandonnement ou négligence des terres, a causé plus de cinq cent mille muids de perte, d’où sont provenus les horreurs de la stérilité, et tous les malheurs qui accompagnent l’extrême cherté et le grand avilissement des grains.\par
Ces effets épouvantables d’une terreur panique répandue sans raison et sans fondement se vérifient tous les jours par une infinité d’exemples, sans parler de ceux qu’on a ci-devant marqués. On sait qu’à la conquête du Nouveau-Monde par les Espagnols, leurs armées les plus nombreuses n’étant composées que de trois ou quatre cents soldats, ils battirent et défirent souvent trois à quatre cent mille hommes, et en assujettirent enfin presque autant de millions qu’ils étaient de têtes. Et de nos jours, l’entreprise qui se fit dans l’île de Madagascar fit à peu près voir la même chose ; celui qui en a fait imprimer la relation, remarque que l’on ne pouvait voir sans surprise trois ou quatre cents Européens avoir assujetti plus de trois cents lieues de pays, en obligeant quatre cent mille hommes, tous portant les armes, de leur payer des redevances et des contributions dans la crainte d’en être punis en cas qu’ils y eussent manqué, comme il arrivait dans ces occasions. Voilà les effets de la prudence et de la raison, lorsqu’elle se trouve divisée en trop de parties, ce qui la réduisant comme en poussière, est cause qu’elle n’a non plus d’effet que tous les autres corps lorsqu’ils souffrent ce sort. Qu’on ne s’étonne donc plus que la France ait souffert de si grands malheurs, et une si forte diminution dans ses biens et dans ses hommes, d’une si petite cause : il était impossible que cela fût autrement.\par
Et il faut croire que l’on n’était pas tombé dans cette erreur du temps de l’empire romain, quoiqu’il ne fût rien moins que barbare, puisque Sénèque le philosophe, qui avait une parfaite connaissance de l’état de toutes les contrées de la terre tant par rapport au présent qu’au passé, marque dans ses écrits que jamais la nature, dans sa plus grande colère\footnote{{\itshape Etiam irata natura}.}, n’avait refusé le nécessaire à qui que ce fût. Puisque donc il y a un si grand avantage à suivre les lois de la nature en ces occasions, il ne sera pas hors de sujet d’expliquer plus clairement en quoi consiste l’effet de ses ordonnances dans le détail, comme on va faire dans le chapitre suivant, après qu’on aura dit un mot de la différence d’intérêt et de délicatesse à l’égard des grains qui se rencontre entre les peuples de France et ceux des autres contrées ; et pourquoi tout le Septentrion voit sortir avec plaisir ses grains en une très grande quantité, et que l’Angleterre même donne de l’argent à pur profit pour fomenter ce commerce, pendant que l’enlèvement du moindre nombre en France, quelque abondance qu’il se rencontre, ne se peut faire sans une espèce de soulèvement.\par
Outre les raisons d’État dont on a parlé, que l’on connaît ailleurs et qu’on n’a jamais pénétrées dans ce royaume au moins depuis quelque temps, savoir que c’est un moyen certain d’éviter la famine ; il y a une cause sensible, particulièrement à la France, qui, se présentant d’abord à l’esprit, est embrassée aveuglément par le peuple qui s’en tient toujours dans sa conduite à la première idée, sans percer plus avant.\par
Cette différence donc vient de la nourriture des peuples. Il est constant, et personne ne le conteste, qu’en France les seuls grains forment presque tout l’aliment du menu peuple, sans même aucun secours ni de boissons ni de légumes, comme partout ailleurs, et encore bien moins de viande et de poisson ; au lieu qu’en Angleterre on peut dire que c’est le pain qui tient la moindre place dans la pitance ordinaire des habitants. La viande et le poisson, qui y sont en très grande abondance, et par conséquent à vil prix, relèvent les grains de plus de trois quarts, et souvent même de tout, des fonctions qu’ils ont en France d’y nourrir presque seuls les peuples. Il n’y a si malheureux homme de campagne qui n’ait sa provision de viande salée et de bière, qui est un second aliment ; et cela va si loin qu’ils ne font aucun usage du bouillon dans lequel on fait cuire les viandes, quoique le plus délicieux mets du menu peuple en France : ils le jettent dans la rue avec le reste des immondices, ainsi que les extrémités des bêtes, qu’ils ne mettent point à profit, comme partout ailleurs.\par
Ainsi les deux partis ou les deux intérêts des blés, dont on a ci-devant parlé, s’y trouvent dans une situation bien différente de ce qu’ils sont en France : celui de faire subsister uniquement le peuple n’est pas à beaucoup près dans un si haut degré, ce qui fortifiant l’autre, savoir de former du revenu aux propriétaires des fonds ou plutôt au pays, on ne doit pas s’étonner de voir en Angleterre et dans les pays du Nord une conduite si opposée à celle qui se pratique en France, et si, pendant qu’on regarde avec plaisir un enlèvement de cinquante mille muids de blés dans ces contrées, on se soulève en France à la sortie de huit ou dix muids seulement, quoique ce soit autant de semence pour en faire renaître cent fois davantage, par les raisons qu’on n’a que trop montrées, mais dans lesquelles le peuple n’est point capable d’entrer dans ce royaume.\par
Ce qu’il y a encore à remarquer est que cette décharge de fonctions du pain dans la nourriture des peuples prend son taux et hausse à proportion que l’intérêt opposé, qui est le haut prix des grains, ou plutôt le revenu des propriétaires et des maîtres, se fortifie, parce que le seul et unique usage des richesses étant de se procurer toutes sortes de commodités jusqu’au dernier degré de magnificence, cela ne se peut faire sans communiquer à toutes sortes d’arts et professions, chacun au sou la livre, une partie de cette aisance qui met en état de se procurer tout ce qu’on désire : ainsi voilà bien du monde relevé de la condamnation de ne manger que du pain et de ne boire que de l’eau par une ample fonction de son art, qui règle seule son ordinaire ; ce qui fait que dans le bon prix des grains la consommation de viande est triplée, et les blés par conséquent dispensés de tenir lieu de toutes sortes de mets, ainsi que de liqueurs à l’égard du peuple : c’est pourquoi aussi, dans les temps de stérilité, il s’en fait une bien plus grande consommation, parce que si le taux fait que les misérables en mangent moins, ceux d’une fortune mitoyenne en absorbent beaucoup davantage, attendu que le pain leur tenant lieu de viande, à laquelle ils étaient accoutumés, et dont ils sont privés par le haut prix du blé, ils en mangent beaucoup plus, sans néanmoins presque jamais rassasier.
\subsection[{Chapitre VIII.}]{Chapitre VIII.}
\noindent On a déjà remarqué que la nature, qui n’est autre que la Providence, ne traite pas les hommes d’une manière moins favorable qu’elle ne fait les bêtes ; et que comme il n’y en a aucune à qui elle n’apprête la nourriture en la mettant au monde, elle en userait assurément de même envers tous les peuples, si, par des défiances outrées, sous prétexte de mesures prudentes, ils ne lui faisaient une espèce d’outrage qu’elle se croit engagée de punir, en les mettant souvent, après tous leurs efforts, dans une situation plus fâcheuse que n’est jamais celle des nations que la grossièreté et la barbarie obligent uniquement de s’en rapporter à elle.\par
Il y a assurément de l’ingratitude de la part de la France envers la nature, en tenant cette conduite : elle l’a mieux partagée de ses faveurs qu’aucune contrée de l’Europe ; et si cette disposition s’est souvent vue altérée, comme on ne peut pas dire que cela soit autrement, c’est par la même raison que les Israélites virent la suppression de la manne dans le désert. Comme cette défiance est bien plus criminelle en ce royaume qu’ailleurs, on ne doit pas s’étonner qu’il en ait été puni plus rigoureusement. On n’avait qu’à laisser agir la nature en ce qui concerne les blés, comme on fait à l’égard des fontaines, et on peut dire qu’ils n’auraient jamais plus manqué ni fait de désordre, soit par la sécheresse ou par l’inondation, que l’on ne voit arriver aux eaux vives, et qui ne sont pas naturellement malfaisantes comme pourraient être celles des torrents.\par
Les blés sortent de la terre par le travail de l’homme et les influences du ciel, de la même manière que les eaux coulent des sources ; ils ne tarissent jamais tant que le cours est libre ; la nature s’est chargée du soin de leur dispensation, pourvu qu’on s’en rapporte à elle, et qu’on ne fasse pas des digues et des chaussées pour retenir tout sur le lieu de leur naissance ; parce qu’en ce cas il en arrive comme aux eaux, l’avarice cause une très grande perte, outre que l’eau d’un réservoir n’est jamais si naturelle ni si bonne que celle d’un ruisseau : de même des blés retenus par une violence se corrompent aisément, pendant que les lieux limitrophes périssent par une situation contraire, savoir la disette, ainsi qu’on a montré ci-devant ; et d’ailleurs, la source se tarit, parce que l’étang ou le réservoir a gagné le niveau et la hauteur de son origine ; ainsi il n’y a plus d’écoulement, et voilà une sécheresse générale pour toutes les contrées voisines. On a assez montré, sans le répéter, que la plupart des terres ne pouvant s’exploiter, les grains étant à bas prix, et les magasins forcés les avilissant tout à fait, c’est leur donner leur congé, et prononcer une interdiction générale de jamais ensemencer, que de les retenir malgré leur nature.\par
Il faut des réservoirs, mais c’est à la nature à les faire, et non pas à l’autorité et à la violence. Et pour reprendre l’exemple des sources, les étangs, et les lacs qu’elles forment naturellement, et sans aucun ministère étranger, causent une très grande utilité, sans aucun des fâcheux accidents marqués ci-dessus ; témoin le lac de Genève, qui, loin de tarir la source du Rhône lors qu’il y est entré, ou qu’il l’a formé, en ressort plus auguste et plus majestueux qu’il n’était auparavant.\par
Il en va de même des réservoirs des blés faits par la nature, et voici quels ils sont : c’est quand ils sont formés par l’intérêt général de tous les peuples, sans intervention d’aucune autorité supérieure, qui doit être bannie de toutes les productions de la terre, parce que la nature, loin d’obéir à l’autorité des hommes, s’y montre toujours rebelle, et ne manque jamais de punir l’outrage qu’on lui fait, par disettes et désolations qui ne sont que trop connues. Ces réservoirs sont créés dès que les laboureurs, pouvant avec partie de leur récolte payer leurs maîtres, gardent leur surplus pour les années stériles, ce qui les enrichit de fournir l’État, au lieu que de l’autre manière l’un et l’autre manquent tout à fait.
\subsection[{Chapitre IX.}]{Chapitre IX.}
\noindent Pour résumer tout ce que l’on a dit en ce Mémoire, dans lequel on n’a été que l’organe ou l’orateur des laboureurs et habitants des champs, ou plutôt de toute la terre, on ne croit pas que qui que ce soit puisse douter des vérités qui y sont contenues, quelque surprenantes qu’elles aient paru d’abord. Et l’on ne peut dire que, dans cette espèce de procès criminel, l’accusation n’ait pas satisfait à son obligation première, qui est de prouver, par la représentation du corps de délit, que le crime est constant. Les terres en friche ou mal cultivées, exposées à la vue de tout le monde, voilà le cadavre de la France, et le fait qui met l’auteur hors de toute crainte de passer pour mauvais citoyen, en venant annoncer, comme il a dit et répète encore, que le peuple ne sera jamais plus misérable que lorsque le blé sera à vil prix, c’est-à-dire lorsqu’il n’aura pas de proportion avec celui qui est contracté par les autres denrées, parce qu’alors le commerce continuel, qui doit être entre toutes les conditions, cesse entièrement, n’étant fondé que sur un équilibre naturel qui se trouve rompu dès qu’une partie vend à perte, comme l’on maintient qu’il faut que cela soit aussitôt que la tête du blé est à neuf ou dix francs dans Paris.\par
La seconde proposition, que l’on n’évitera jamais les sinistres effets des années stériles qu’en laissant libre la sortie des blés hors du royaume, est de pareille nature : l’horreur de l’énoncé se tourne en maxime de la plus grande utilité qui puisse être dans un État, quand la discussion en est faite. Outre les raisons marquées ci-dessus, qui laissent peu de doute, outre l’exemple de l’Angleterre, où le peuple, décidant immédiatement, de son sort, regarde cette liberté de sortie comme la garantie la plus certaine contre la famine, on n’a qu’à jeter les yeux sur ce qui se passe en Hollande à l’égard de toutes sortes de marchandises, et même des blés : la maxime générale de ces rois du commerce est de regarder l’abondance de quelques sortes de denrées que ce puisse être, non-seulement comme la ruine de l’espèce qui est dans l’avilissement, mais même de toutes les autres, par le rapport nécessaire et la communication réciproque de bien et de mal qu’elles doivent avoir continuellement ensemble, sans quoi tout est perdu. Ainsi il n’y a rien que ces peuples ne fassent pour conjurer ce désordre dans ces occasions, et ils croient n’avoir pas moins d’obligation à la mer d’engloutir ce qu’ils jugent avoir d’excédant, et qu’ils y jettent par une sage folie en pure perte, que de leur avoir apporté le restant par une infinité de travaux et au péril de leurs vies.\par
Les denrées les plus précieuses du Nouveau-Monde, comme les épiceries du plus grand prix, ne sont point exemptes de ce sort. À l’égard des blés, comme il n’en croit pas à beaucoup près la quantité nécessaire au pays, ils ont en quelque manière forcé la nature, par une maxime presque semblable à ces précédentes, pour faire en sorte que, dans les stérilités de l’Europe, bien loin d’avoir besoin de tirer des secours extraordinaires des autres contrées, c’est chez eux que les pays les plus fertiles et les plus féconds viennent chercher les moyens de conjurer la violence du mal qu’ils souffrent. Par une maxime fondamentale et à laquelle on ne déroge jamais, il est établi que la source des blés qui s’y trouvent repostés comme dans un magasin, est et sera toujours libre en tout temps, quelque cause qu’il puisse y avoir de pratiquer le contraire : de cette façon, et sur la foi de cette politique, tout le Septentrion en fait son entrepôt pour fournir dans les occasions, avec la facilité de la mer, les contrées qui se trouvent dans le besoin de cette manne primitive.\par
De cette manière, ils ont une garantie certaine, quelque malheur qu’il arrive, de n’avoir qu’à se défendre du prix et non pas du manque de l’espèce, ce qui serait sans ressource dans un pays qui ne produit pas de grains. Mais il y a encore plus : dans la concurrence, ils ont non-seulement la préférence, mais même avec diminution, parce qu’ils gagnent les frais du transport, à quoi le marchand n’étant point obligé, il trouve son compte de leur donner sa marchandise à bien meilleur marché, vendant sur le lieu, que s’il était obligé d’essuyer les frais et les risques d’une longue voiture.\par
On voit par là que la nature ne respire que la liberté, puisque c’est par l’entière jouissance d’une chose dont elle est si jalouse, qu’elle fournit abondamment une nourriture dans un pays où elle ne croît point, pendant qu’elle la refuse souvent aux contrées qui la produisent en plus grande quantité.\par
Il est aisé de voir, par tout ce qu’on vient de dire, de quelle conséquence est dans un pays, pour y entretenir l’abondance, d’empêcher qu’aucune marchandise n’y soit à rebut, qui est le moyen de la faire tarir ; parce que constituant les entrepreneurs en perte, ils cessent entièrement leur trafic, qui fait payer la folle enchère de l’avilissement précédent de la denrée. Comme on porte trop de respect aux grains pour les jeter dans la mer, au moins il ne faut pas refuser la ressource, dans les occasions d’abondance, d’en faire part aux voisins dans la crainte de tomber dans la situation opposée ; puisqu’au contraire c’est le moyen de tomber dans cette extrémité que l’on appréhende si fort, et qui est une suite de cet avilissement, ainsi qu’on a montré.
\subsection[{Chapitre X.}]{Chapitre X.}
\noindent Pour terminer enfin cet ouvrage, dans lequel on pense s’être amplement acquitté des deux obligations contractées en chacune des deux parties, on croit et on maintient que le seul et unique intérêt de la France, ainsi que de tous les royaumes du monde, est que toutes les terres y soient parfaitement cultivées, avec tous les engrais nécessaires ; que toutes sortes de commerces se portent dans la plus grande valeur qu’ils puissent être ; que tous les hommes dont le travail est la seule ressource pour leur subsistance ne perdent pas un moment de temps, et ne soient jamais dans l’oisiveté. Si les choses se trouvaient dans cette situation, que l’on peut beaucoup plus souhaiter qu’espérer de voir jamais dans la dernière perfection, ce qui n’est guère qu’en Hollande et dans la Chine, ce serait un extrême aveuglement de craindre jamais les sinistres effets d’aucune stérilité, quelque violente qu’elle pût être : plus de six millions de muids de blé que cette disposition produirait, pendant que la consommation ordinaire n’en exigerait que la moitié au plus, supposé que les hommes même eussent doublé, ce qui est très possible, feraient une si forte garantie, que rien d’approchant d’une terreur panique ne pourrait jamais tomber dans l’esprit.\par
Il faut donc faire comme la nature ; lorsqu’elle ne peut pas produire un sujet tout à fait accompli, elle en forme un moins parfait : il n’est donc point nécessaire que les landes de Bordeaux et la Crau de Provence soient rendues aussi fécondes et aussi abondantes que les terres qui sont aux portes de Paris, comme promettaient les Maures lors de leur sortie d’Espagne ; il est seulement besoin que ce qui se labourait il y a quarante ans, et qui avait toujours été cultivé à remonter tous les siècles de la monarchie, le soit encore. Or, il est impossible que cela arrive jamais, tant que l’entrepreneur est constitué en perte, comme il le sera toujours tant que la marchandise ne pourra porter ses frais.\par
Il y a une police nécessaire que la nature seule peut mettre, et jamais l’autorité, dans les divers personnages ou représentations qui entrent toutes, au sou la livre de leur art ou profession, dans la perfection de toutes sortes d’ouvrages et de commerce, et surtout de l’agriculture.\par
Quoiqu’elles se donnent également et réciproquement la naissance les unes aux autres, ainsi que l’on a remarqué, au lieu de conspirer conjointement à leur commun maintien, comme elles devraient faire, elles ne travaillent depuis le matin jusqu’au soir qu’à se détruire, et à se revêtir des dépouilles l’une de l’autre. L’ouvrier voudrait avoir tout le prix des fruits d’une récolte pour sa peine, sans s’embarrasser de quoi celui qui le met en besogne paie son maître et les impôts, non plus que de l’impuissance où il sera de recharger sa terre pour lui redonner une autre fois sa vie à gagner ; et le fermier à son tour désirerait avoir la peine de tous ceux dont il se sert pour emménager ses fonds, pour beaucoup moins qu’il ne faut à ces artisans, afin de s’entretenir eux et leurs familles.\par
Lequel des deux qui gagne sa cause, l’État souffre, parce que les terres demeurent, et que le commerce ne se fait point. Il n’y a donc que l’équilibre qui puisse tout sauver ; et la nature seule, encore une fois, l’y peut mettre ; mais il ne faut pas l’empêcher d’agir. C’est néanmoins ce que l’on fait, lorsqu’on défend aux laboureurs de vendre leurs blés à ceux qui en offrent de l’argent, car voilà la cause de l’ouvrier gagnée, quoique perdue dans la suite.\par
La nécessité seule, qui mène ces sortes de gens-là, a perdu l’empire qu’elle avait sur eux : s’ils gagnent la dépense de toute la semaine en une seule journée de travail, parce que le blé est à rebut, loin d’en suivre le niveau pour leur salaire, cette situation les fortifie à rengréger la misère du maître en exigeant un plus haut prix, par la possibilité où ils sont, en cas de refus, de se passer de travail un temps considérable. Et comme la culture de la terre n’a point de moment qui ne soit fatal, c’est-à-dire que, si tout n’est fait au jour et à l’heure marquée par les saisons, tout est perdu, le laboureur n’a que le choix ou de périr en laissant tout, ou de faire une dépense dont il ne sera jamais remboursé. Cette situation gagne aussitôt tous les arts et professions, où l’on voit la même rébellion de la part de l’ouvrier à l’égard de l’entrepreneur, et jusqu’aux domestiques envers leurs maîtres, lesquels au moindre mot leur mettent le marché à la main, sentant le pain à bas prix ; pour après, tant les ouvriers que les valets, en payer la folle enchère, lorsque leur provision ayant pris fin, et revenant de leur révolte, ils ne trouvent plus le marché, à beaucoup près, qu’ils ont refusé ; parce que la misère s’étant puissamment établie, tout le monde est dans l’intérêt de congédier les gens, et non pas d’en prendre de nouveaux.\par
Cette proportion d’intérêt est donc nécessaire entre toutes sortes de commerçants, et que l’on ne tire pas une double utilité en s’emparant de la part de l’autre ; autrement, toute l’harmonie sur laquelle roule le maintien de l’État est entièrement détruite.\par
C’est néanmoins ce qui arrive entre ces ouvriers et leur maître dans le bas prix du blé ; parce que cette denrée étant sujette à révolution, par des causes qui ne sont point au pouvoir des hommes, comme les dispositions du ciel, l’artisan qui prétend suivre sa destinée en cas de hausse, comme il fait effectivement, ne veut point faire cette justice dans le rabais, ce qui est cause de tous les malheurs dont on vient de parler, et dont on n’a que trop fait d’expérience.\par
En effet, il est juste de hausser le prix des ouvriers lorsque leurs ouvrages, ainsi que leurs besoins, reçoivent un pareil sort ; et même en ces occasions ils ne s’en rapportent pas à la libéralité de leurs maîtres, qui ne seraient pas plus raisonnables qu’eux, si tout dépendait de leur bonne volonté ; mais dans ces rencontres ils se font faire justice d’une manière qu’eux ni leurs maîtres, non plus que l’État, ne souffrent aucune perte. Comme l’abondance du commerce que mène toujours après soi le haut prix des denrées, et surtout des blés, ainsi que les crues d’argent qui arrivent toutes les années en Europe, mettent la presse à recouvrer des ouvriers, ils capitulent pour la hausse, non en menaçant de ne rien faire, mais d’aller d’un autre côté où on leur accordera leurs prétentions : c’est de cette sorte que ceux qui gagnaient quinze deniers par jour il y a cent cinquante ans, se sont fait accorder et ont aujourd’hui quinze et vingt sous pour le même travail, parce que les blés, qui valaient vingt sous le setier à Paris en ce temps, comme l’on a dit, ont valu et devaient valoir seize à dix-huit livres ; ainsi des autres denrées.\par
Et ils ne manquent jamais de se procurer cette situation de surcroît toutes les fois que les grains renchérissent, quand ce n’est point dans l’excès ; puis, quand ils viennent à baisser, on peut dire que les laboureurs sont ruinés, ainsi que toutes les professions qui en attendent leur destinée, et qu’ils perdent dans la suite ce qu’ils ont gagné dans les précédentes années ; y ayant un esprit de rébellion si fort établi contre la justice dans ces occasions entre les ouvriers, en prenant le parti que l’on vient de marquer, que l’on voit, dans les villes de commerce, des sept à huit cents ouvriers d’une seule manufacture s’absenter tout à coup et en un moment, en quittant les ouvrages imparfaits, parce qu’on leur voulait diminuer d’un sou leur journée, le prix de leurs ouvrages étant baissé quatre fois davantage ; les plus mutins usant de violence envers ceux qui auraient pu être raisonnables.\par
Il y a même des statuts parmi eux, dont quelques-uns sont par écrit, et qu’ils se remettent de main en main, quoique la plupart forains et étrangers, par lesquels il est porté que si l’un d’eux entreprend de diminuer le prix ordinaire, il soit aussitôt interdit de faire le métier ; et outre la voie de fait dont ils usent en ces occasions, le maître même s’en ressent, par une défense générale à tous les ouvriers de travailler jamais chez lui : on a vu des marchands considérables faire banqueroute par cette seule raison, qu’ils avaient été deux ou trois ans sans pouvoir trouver personne pour faire leurs ouvrages, quoiqu’il y en eût quantité sur le lieu, du même art, qui ne trouvaient point de maîtres.\par
Cet entêtement de maintenir le prix contracté n’est point singulier aux simples journaliers ; tous les arts et métiers le regardent comme la sauvegarde et le seul maintien de leur profession, et ils aiment mieux ne vendre qu’une seule pièce au prix marqué, que d’en débiter dix à quelque chose de rabais, quoique le profit sur le nombre excédât de beaucoup la diminution ou la perte sur le singulier ; le contraire est une chose sur laquelle ils sont incapables d’entendre raison.\par
Pour en faire demeurer d’accord, il n’y a qu’à marchander durant un mois tous les jours écu à écu, ou pistole à pistole, une perruque ou un carrosse ; le vendeur a refusé vingt fois le marché pour une pistole ou deux de moins, en faisant des serments que c’est tout ce qu’il y gagnait, lesquels sont de pareil mérite et valeur dans le trafic qu’en amour ; et puis quand le marché est conclu, et la chose livrée et payée, qu’on la lui rapporte un moment après, il ne la voudra pas reprendre à la moitié de perte.\par
On a fait ce détail par rapport au prix que doivent être les blés, parce que comme la richesse d’un État consiste dans un commerce continuel, en sorte que ni terre, ni ouvriers, ni ouvrages, ne soient jamais dans un moment de repos, ce qui produit le même effet à l’égard de l’argent ; cette interruption ou ce déconcertement ne vient que de leur avilissement, après que l’on a mis un taux aux denrées dans leur hausse, qui ne les peut point suivre quand ils changent de situation.\par
Or, comme il est impossible de faire entendre raison à toutes les nations dont on vient de parler, et de les faire baisser quand les blés haussent, il faut nécessairement soutenir le prix que le grain a une fois contracté, et non pas le détruire de gaieté de cœur, comme on peut dire qu’on a fait depuis quarante ans sous prétexte de faire plaisir aux pauvres, bien que cela les ruine entièrement, ainsi que l’on a fait voir.\par
Enfin, le commerce ne se fait que par une utilité réciproque ; et il faut que chacune des parties, tant les acheteurs que les vendeurs, soit dans un égal intérêt ou nécessité de vendre ou d’acheter ; autrement, si cet équilibre cesse, celui qui a l’avantage se sert de l’occasion pour faire capituler l’autre en lui faisant subir la loi qu’il lui veut imposer.\par
En effet, un homme qui se peut passer de vendre, ayant affaire à un autre qui est dans la nécessité d’acheter, ou bien le contraire, le marché ne se conclura point sans destruction d’un des deux.\par
Or, dans la liberté qu’on ôte aux laboureurs de soutenir le prix de leurs blés par un enlèvement au-dehors, de nulle considération à l’égard de la subsistance nécessaire du royaume, quand il n’en doublerait pas et l’excroissance et la garde, ainsi qu’on a fait voir, est la même chose que si pendant que deux hommes se battraient l’épée à la main, et seraient fort acharnés l’un contre l’autre, quelqu’un pour mettre la paix, ou les séparer, en saisissait entièrement un au corps et le mettait hors de défense : le combat serait assurément fini, parce que l’autre se servirait de l’occasion pour tuer tout à fait son ennemi, ce qui n’est pas sans exemple.\par
Les blés avec le reste du commerce se défendent vaillamment, ce qui fait voir un combat dans lequel on remarque bien de la bravoure ; mais lorsqu’on les a saisis au corps, leur ennemi les perce d’outre en outre : c’est la raison de la différence des deux situations si opposées, dont on a parlé entre les commerçants, de ne vouloir vendre qu’à leur mot, et puis quand la nécessité les a gagnés et qu’on les a saisis par le corps, ils donnent à très grande perte.\par
On croit avoir convaincu les plus incrédules, par ce Mémoire, des deux propositions qui avaient semblé d’abord révolter le ciel et la terre. La raison de cette erreur si commune, ainsi qu’on a dit au commencement de cet ouvrage, est que la véritable connaissance des grains étant une suite nécessaire d’un assemblage continuel de pratique et de spéculation à leur égard, on peut dire que ces deux dispositions ont été séparées depuis quarante ans par une si grande distance, que la possession de l’une par la situation du sujet a été une exclusion formelle à avoir jamais l’autre : ceux qui pouvaient s’énoncer n’en avaient nulle pratique, et les sujets qui y sont destinés par leur condition ne sont plus en état d’en expliquer les intérêts, qu’un cheval qui boite, de marquer son mal.\par
Pour dernière période de ce Mémoire, la première partie se réduit à faire voir que l’on a cru, afin que tout le monde fût à son aise, qu’il fallait qu’aucun laboureur ne pût payer son maître ; et l’autre, que pour éviter les horreurs d’une extrême cherté, il était à propos que l’on cessât de labourer les terres de difficile exploitation, ainsi que d’engraisser les meilleures, et qu’on consumât les grains à la nourriture des bestiaux et confection des manufactures ; ce qui étant également la désolation d’un État, on s’est cru comptable au ciel et à la terre de travailler à faire revenir d’une si grande erreur, qui a fait plus de maux en France que tous les fléaux de Dieu, regagnant par sa durée ce qui pourrait paraître de plus violent dans de pareils malheurs, qui n’ont jamais qu’un temps limité : en quoi on peut dire que la Providence a voulu en quelque façon enrayer la France, laquelle sans cela est elle seule plus puissante que toute l’Europe ensemble ; et c’était le sentiment de Corneille Tacite, quand il a marqué qu’elle est invincible lorsqu’elle n’a pas à se défendre d’elle-même. C’est avec bien plus de sujet que l’on doit faire aujourd’hui le même raisonnement, puisque, outre que la valeur de la nation a toujours été en augmentant, elle se trouve un monarque à la tête, qui, n’ayant point eu de pareil par le passé, pourrait lui seul faire dire aujourd’hui ce qu’on a publié de toute la nation ; et comme le rétablissement de l’erreur est possible en peu de temps, on laisse aux lecteurs d’en tirer les conséquences dans la conjoncture présente, surtout y ayant des ministres aussi intègres et aussi éclairés que ceux qui se trouvent en place.\par


\begin{raggedleft}FIN DU TRAITÉ DES GRAINS.\end{raggedleft}
\chapterclose


\chapteropen
\chapter[{Dissertation sur la nature des richesses, de l’argent et des tributs, où l’on découvre la fausse idée qui règne dans le monde à l’égard de ces trois articles.}]{Dissertation sur la nature des richesses, de l’argent et des tributs, où l’on découvre la fausse idée qui règne dans le monde à l’égard de ces trois articles.}\renewcommand{\leftmark}{Dissertation sur la nature des richesses, de l’argent et des tributs, où l’on découvre la fausse idée qui règne dans le monde à l’égard de ces trois articles.}


\chaptercont
\section[{Chapitre I.}]{Chapitre I.}
\noindent Tout le monde veut être riche, et la plupart ne travaillent nuit et jour que pour le devenir ; mais on se méprend pour l’ordinaire dans la route que l’on prend pour y réussir.\par
L’erreur, dans la véritable acquisition de richesses qui puissent être permanentes, vient premièrement de ce que l’on s’abuse dans l’idée que l’on se fait de l’opulence, ainsi qu’à l’égard de celle de l’argent.\par
On croit que c’est une matière où l’on ne peut point pécher par l’excès, ni jamais, en quelque condition que l’on se trouve, en trop posséder ou acquérir ; l’attention aux intérêts des autres est une pure vision, ou des réflexions de religion, qui ne passent point la théorie. Mais, pour montrer que l’on s’abuse grossièrement, qui mettrait ceux qui y sont dévoués si singulièrement en possession de toute la terre avec toutes ses richesses, sans en rien excepter ni diminuer, n’en ferait-il pas les derniers des malheureux, s’ils ne pouvaient disposer du labeur de leurs semblables ? Et ne préféreraient-ils pas la condition d’un mendiant dans un monde habité ? Car premièrement, outre qu’il leur faudrait être eux-mêmes les fabricateurs de tous leurs besoins, bien loin de servir par là leur sensualité, ce serait un chef-d’œuvre si, par un travail continuel, ils pouvaient atteindre jusqu’à se procurer le nécessaire ; et puis, dans la moindre, indisposition, il faudrait périr manque de secours, ou plutôt de désespoir.\par
Et même sans supposer les choses dans cet excès, un très petit nombre d’hommes en possession d’un très grand pays, comme il est arrivé quelquefois par des naufrages, n’ont-ils pas été autant de malheureux, bien loin d’être autant de monarques ? Et il n’est que trop certain, par les relations espagnoles de la découverte du Nouveau-Monde, que les premiers conquérants, quoique maîtres absolus d’un pays où l’on mesurait l’or et l’argent par pipes, passèrent plusieurs années si misérablement leur vie, que, outre que plusieurs moururent de faim, presque tous ne se garantirent de cette extrémité que par les aliments les plus vils et les plus répugnants de la nature.\par
Ce n’est donc ni l’étendue du pays que l’on possède, ni la quantité de l’or et de l’argent, que la corruption du cœur a érigé en idoles, qui font absolument un homme riche et opulent : elles n’en forment qu’un misérable, comme l’on peut voir par les exemples que l’on vient de citer ; ce qui se vérifie tous les jours encore par le parallèle de ce qui se passe au pays des mines, où cinquante écus à dépenser par jour font vivre un homme moins commodément qu’il ne ferait en Hongrie avec huit ou dix sous, qui suffisent presque pour jouir abondamment de tous les besoins nécessaires et agréables. On voit par cette vérité, qui est incontestable, qu’il s’en faut beaucoup qu’il suffise pour être riche de posséder un grand domaine et une très grande quantité de métaux précieux, qui ne peuvent que laisser périr misérablement leur possesseur, quand l’un n’est point cultivé ; et l’autre ne se peut échanger contre les besoins immédiats de la vie, comme la nourriture et les vêtements, desquels personne ne saurait se passer. Ce sont donc eux seuls qu’il faut appeler richesses ; et c’est le nom que leur donna le créateur lorsqu’il mit le premier homme en possession de la terre après l’avoir formé : ce ne furent point l’or ni l’argent qui reçurent ce titre d’opulence, puisqu’ils ne furent en usage que longtemps après, c’est-à-dire tant que l’innocence, au moins suivant les lois de la nature, subsista parmi les habitants du globe, et les degrés de dérogeance à cette disposition ont été ceux de l’augmentation de la misère générale. On a fait, encore une fois, une idole de ces métaux ; et laissant là l’objet et l’intention pour lesquels ils avaient été appelés dans le commerce, savoir pour y servir de gages dans l’échange et la tradition réciproque des denrées, lorsqu’elle ne se put plus faire immédiatement à cause de leur multiplication, on les a presque quittés de ce service pour en former des divinités auxquelles on a sacrifié et sacrifie tous les jours plus de biens et de besoins précieux, et même d’hommes, que jamais l’aveugle antiquité n’en immola à ces fausses divinités qui ont si longtemps formé tout le culte et toute la religion de la plus grande partie des peuples. Ainsi, il est à propos de faire un chapitre particulier de l’or et l’argent, pour montrer par où ce désordre est entré dans le monde, où il a fait un si grand ravage, surtout dans ces derniers temps, que jamais ceux des nations les plus barbares dans leurs plus grandes inondations n’en approchèrent, quelque description épouvantable que l’on en trouve chez les historiens. On espère qu’après la découverte de la source du mal, il y aura moins de chemin à faire pour arriver au remède, et que cela pourra porter les hommes à revenir de leur aveuglement, d’anéantir tous les jours une infinité de biens, de fruits de la terre, et de commodités de la vie, seules propres à faire subsister l’homme, pour recouvrer une denrée qui, n’étant absolument d’aucun usage par elle-même, n’avait été appelée au service des hommes que pour faciliter l’échange et le trafic, ainsi qu’on a déjà dit. On espère qu’après vérification de ce fait incontestable, que la misère des peuples ne vient que de ce qu’on a fait un maître, ou plutôt un tyran, de ce qui était un esclave, on quittera cette erreur, et que, rétablissant les choses dans leur état naturel, la fin de cette révolte sera celle de la désolation publique.
\section[{Chapitre II.}]{Chapitre II.}
\noindent Le ciel n’est pas si éloigné de la terre qu’il se trouve de distance entre la véritable idée que l’on doit avoir de l’argent, et celle que la corruption en a établie dans le monde, et qui est presque reçue si généralement, qu’à peine l’autre est-elle connue, quoique cet oubli soit une si grande dépravation, qu’elle cause la ruine des États, et fait plus de destruction que les plus grands ennemis étrangers pourraient jamais causer par leurs ravages.\par
En effet, l’argent, dont on fait une idole depuis le matin jusqu’au soir, avec les circonstances que l’on a marquées, et qui sont trop connues pour être révoquées en doute, n’est absolument d’aucun usage par lui-même, n’étant propre ni à se nourrir, ni à se vêtir ; et nul de tous ceux qui le recherchent avec tant d’avidité, et à qui, pour y parvenir, le bien et le mal sont également indifférents, n’est porté dans cette poursuite qu’afin de s’en dessaisir aussitôt, pour se procurer les besoins de son état ou de sa subsistance.\par
Il n’est donc tout au plus, et n’a jamais été, qu’un moyen de recouvrer les denrées, parce que lui-même n’est acquis que par une vente précédente de denrées, cette intention étant généralement tant dans ceux qui le reçoivent que dans ceux qui s’en dessaisissent ; en sorte que si tous les besoins de la vie se réduisaient à trois ou quatre espèces, comme au commencement du monde, l’échange se faisant immédiatement et troc pour troc, ce qui se pratique même encore en bien des contrées, les métaux aujourd’hui si précieux ne seraient d’aucune utilité.\par
Il n’y a même aucune denrée si abjecte, propre à nourrir l’homme, qui ne lui fût préférée, en quelque quantité qu’elle se rencontrât, s’il était absolument défendu ou impossible au possesseur de l’argent de s’en dessaisir, ce qui le réduirait bientôt au même état que le Midas de la fable.\par
Ce n’est donc que comme garant tout au plus des échanges, et de la tradition réciproque, qu’il a été appelé dans le monde, lorsque la corruption et la politesse ayant multiplié les besoins de la vie, de trois ou quatre espèces qu’ils étaient dans son enfance, jusqu’à plus de deux cents où ils se trouvent aujourd’hui ; ce qui fait que n’y ayant pas moyen que le commerce et le troc s’en fassent de main à main, comme dans ces temps d’innocence ; et le vendeur d’une denrée ne trafiquant pas le plus souvent avec le marchand de celle dont il a actuellement besoin, et pour le recouvrement de laquelle il se dessaisit de la sienne, l’argent alors vient au secours, et la recette qu’il en fait de son acheteur lui est une procuration, avec garantie, que son intention sera effectuée en quelque lieu que se trouve le marchand, et cela pour autant, et sur un prix courant et proportionné à ce qu’il s’est dessaisi les mains de la denrée dont il était propriétaire : voilà donc l’unique fonction de l’argent ; et chaque degré de dérogeance qu’on y admet, quoiqu’elle se voie aujourd’hui à un excès effroyable, est autant de déchet à la félicité d’un État.\par
En effet, tant qu’il s’en tient là, non seulement il n’y a rien de gâté, mais bien loin d’être obligé de lui sacrifier tous les jours tant de victimes afin de le recouvrer, pour peu qu’il fit le rebelle, si les hommes s’entr’entendaient, il serait aisé de lui donner son congé ; ce qui lui arrive même à chaque moment en une infinit\phantomsection
\label{\_GoBack}é d’occasions, quoiqu’on n’y prenne pas garde.\par
Comme il n’est tout au plus, ainsi qu’on vient de dire, qu’une garantie de la livraison future d’une denrée, qu’on ne reçoit pas immédiatement en vendant celle que l’on possède, du moment qu’elle se peut procurer sans son ministère, il sera obligé de renfermer tout son orgueil à demeurer absolument inutile et immobile.\par
Le cuivre et le bronze, dont on fait de la monnaie pour des sommes considérables, ne le remplacent-ils pas ? N’en a-t-on pas fait souvent de cuir, dans certaines occasions, qui, avec la marque du prince, qui ne coûte rien, a la même vertu, et même davantage, puisqu’elle a procuré les besoins de la vie plus que n’ont jamais fait les piles d’argent au Pérou et au Nouveau-Monde ?\par
Aux îles Maldives, où les peuples ne sont point du tout barbares, étant même polis et magnifiques, comme on peut voir par les relations, de certaines coquilles, qui se donnent par petits sacs, ont le même pouvoir, et procurent la même certitude de livraison future de ce qu’on veut ou voudra avoir, que font l’or et l’argent partout ailleurs où ils sont en vogue, bien que ces îles n’en soient pas même destituées, et qu’elles ne laissent pas d’en souffrir tranquillement la concurrence avec des matières aussi abjectes que sont des coquilles.\par
Les îles de l’Amérique ont été longtemps, quoique abondantes en argent, sans en connaître l’usage dans le trafic journalier, même parmi les nations de l’Europe qui les habitaient, bien que les peuples ne manquassent d’aucun de leurs besoins qu’ils construisaient dessus le lieu, ou qu’on leur apportait abondamment de l’ancien monde.\par
Le tabac seul faisait tout le trafic, ainsi que la fonction de l’argent, tant en gros qu’en détail : si l’on voulait avoir pour un sou de pain et même moins, on donnait pour autant de ce fruit de la terre, qui avait un prix fixe et certain, sur lequel il n’y avait non plus de contestation que sur la monnaie courante, en quelque pays que ce soit ; et cependant avec tout cela, le nécessaire, le commode et le magnifique n’y manquaient non plus qu’ailleurs.\par
Mais qu’est-il nécessaire d’aller si loin chercher des exemples pour vérifier cette doctrine, que c’est une erreur grossière de regarder l’or et l’argent comme le principe unique de la richesse, et de la félicité de la vie ?\par
Nous avons dans l’Europe, et on le pratique même tous les jours, un moyen bien plus facile et à bien meilleur marché pour mettre ces métaux à la raison, et, détruisant leur usurpation, les renfermer dans leurs véritables bornes, qui sont d’être valets et esclaves du commerce uniquement, et non ses tyrans, et cela en leur donnant pour concurrents non du cuivre, non des coquilles, non du tabac, comme dans les lieux mentionnés, qui coûtent de la peine et du travail à recouvrer, mais un simple morceau de papier qui ne coûte rien, et remplace néanmoins toutes les fonctions de l’argent pour des quantités de millions, une infinité de fois, c’est-à-dire par autant de mains qu’il passe, tant que ces métaux ne sortent point de leur état naturel, et des principes qui les ont fait appeler dans le monde.\par
On demande donc à toute la nation polie, si prévenue des maximes régnantes, et qui ignore absolument la pratique et l’usage du commerce qui fait subsister tous les hommes, sans vouloir même jamais s’en instruire de peur que la reconnaissance de son erreur ne lui fût préjudiciable ; on demande, dis-je, si les billets d’un célèbre négociant dont le crédit est puissamment établi par une opulence certaine connue, ce dont il existe plus d’un exemple en Europe, ne valent et ne prévalent pas à l’argent comptant ; et si en ayant toute la vertu et toute l’efficace, ils n’ont pas des avantages particuliers sur les métaux, par la facilité de la garde et du transport, sans crainte d’enlèvements violents ?\par
Il y a bien plus : c’est que ces billets ne seront jamais acquittés tant qu’ils ne se trouveront qu’en des mains sages et innocentes, et qui n’en veulent faire qu’un usage de conduite prudente, soit par rapport au passé ou au présent, qui est de ne se dessaisir de son bien, surtout d’une somme considérable, que pour se procurer l’équivalent soit en immeubles ou en meubles, si l’on est négociant, et non le consommer en dépense ordinaire, soit faite ou à faire, qui est le seul cas où le billet n’est plus d’usage ; sans quoi, après une infinité de mains qu’il aurait toutes enrichies, en garantissant la livraison future de ce qu’on ne pouvait fournir sur-le-champ, il serait retourné à son premier tireur, ou il n’y aurait échu qu’une compensation.\par
De cette manière, voilà une opulence générale, c’est-à-dire une jouissance et une consommation effroyable de biens, sans le ministère de la moindre somme d’argent. Voilà donc encore une fois les prêtres de cette idole bien loin de leur compte, d’en faire un dieu tutélaire de la vie, et de soutenir que les hommes ne sont heureux ou malheureux qu’à proportion qu’ils possèdent plus ou moins de ce métal si recherché.\par
Les foires de Lyon prouvent l’erreur du sentiment contraire toutes les années, lesquelles étant tantôt bonnes et tantôt mauvaises, on n’en peut nullement attribuer la cause à l’abondance ou au défaut de l’argent, puisque sur un commerce de vente et de revente de plus de quatre-vingts millions qui les compose, on n’y a jamais vu un sou marqué d’argent comptant ; tout se fait par échange et par billets, lesquels, après une infinité de mains, retournent enfin au premier tireur, ainsi qu’on a déjà dit.\par
En voilà plus qu’il n’en faut pour montrer que la quantité plus ou moins considérable d’or et d’argent, surtout dans un pays rempli de denrées nécessaires et commodes à la vie, est absolument indifférente pour en faire jouir abondamment les habitants ; mais ce n’est que lorsque ces métaux demeurent dans leurs limites naturelles, car du moment qu’ils en sortent, comme l’on n’a que trop fait l’expérience en plus d’un endroit, ils deviennent nécessaires, parce qu’ils s’érigent en tyrans, ne voulant pas souffrir qu’autres qu’eux s’appellent richesses ; et c’est ce qu’on va voir dans les chapitres suivants, où l’on montrera les deux issues par où l’argent a quitté son ministère ; dont la première est l’ambition, le luxe, l’avarice, l’oisiveté et la paresse ; et l’autre, le crime formel, tant celui qui est puni par les lois, qu’un autre genre que l’ignorance fait couronner tous les jours.
\section[{Chapitre III.}]{Chapitre III.}
\noindent La condamnation que Dieu prononça contre les hommes en la personne du premier de tous, de ne pouvoir à l’avenir, après son péché, vivre ni subsister que par le travail et à la sueur de leur corps, ne fut ponctuellement exécutée que tant que l’innocence du monde dura, c’est-à-dire tant qu’il n’y eut aucune différence de conditions et d’états : chaque sujet était alors son valet et son maître, et jouissait des richesses et des trésors de la terre à proportion qu’il avait personnellement le talent de les faire valoir ; toute l’ambition et tout le luxe se réduisaient à se procurer la nourriture et le vêtement. Les deux premiers ouvriers du monde, qui en étaient en même temps les deux monarques, se partagèrent ces deux métiers : l’un laboura la terre pour avoir des grains, et l’autre nourrit des troupeaux pour se couvrir, et l’échange mutuel qu’ils pouvaient faire les faisait jouir réciproquement du travail l’un de l’autre.\par
Mais, le crime et la violence s’étant mis, avec le temps, de la partie, celui qui fut le plus fort ne voulut rien faire, et jouir des fruits du travail du plus faible, en se rebellant entièrement contre les ordres du Créateur ; et cette corruption est venue à un si grand excès, qu’aujourd’hui les hommes sont entièrement partagés en deux classes, savoir l’une qui ne fait rien et jouit de tous les plaisirs, et l’autre qui, travaillant depuis le matin jusqu’au soir, se trouve à peine en possession du nécessaire, et en est même souvent privée entièrement.\par
C’est de cette disposition que l’argent a pris son premier degré de dérogeance à son usage naturel : l’équivalence où il doit être avec toutes les autres denrées, pour être prêt d’en former l’échange à tout moment, a aussitôt reçu une grande atteinte. Un homme voluptueux, qui a à peine assez de temps de toute sa vie pour satisfaire à ses plaisirs, s’est moqué de tenir sa maison et ses magasins remplis de grains et d’autres fruits de la terre, pour être vendus au prix courant en temps et saison : ce soin, cette attente et cette inquiétude ne se sont pas accommodés avec son genre de vie ; la moitié moins d’argent comptant, même le quart, font mieux son affaire, et ses voluptés en sont servies avec plus de secret et plus de diligence.\par
Aussi cette main-basse que l’on fait, dans ces occasions, de toutes sortes de denrées, dérange-t-elle d’une terrible façon l’équilibre qui doit être entre l’or et l’argent, et toutes sortes de choses. L’âpreté que l’on a pour recouvrer l’un, et la profusion que l’on fait de l’autre, élèvent le premier jusqu’aux nuées, et abaissent l’autre jusqu’aux abîmes. Voilà donc l’esclave du commerce devenu son tyran ; mais ce n’est là que la moindre partie de sa vexation. Cette facilité qu’offre l’argent pour servir tous les crimes lui fait redoubler ses appointements, à proportion que la corruption s’empare des cœurs ; et il est certain que presque tous les forfaits seraient bannis d’un État, si l’on en pouvait faire autant de ce fatal métal : le peu de service qu’il rend au commerce, ainsi qu’on a fait voir en ce qui a précédé, ne vaut pas la centième partie du mal qu’il lui cause.\par
On ne parle point ni des voleurs ni des brigands, à qui l’argent seul sert de moyen certain pour enlever par violence tout le vaillant d’un homme, sans autre droit ni titre qu’une force majeure, qui les met par là non seulement en pouvoir de le ravir, mais même de le mettre à couvert et hors de toutes recherches.\par
Si toutes les facultés se terminaient aux denrées nécessaires à la vie, les brigands perdraient ces deux facilités pour voler ; ils ne pourraient enlever qu’une petite quantité de biens à la fois, pour laquelle même emporter il leur faudrait un grand nombre de chevaux et de voitures impossibles à cacher, parce que tout serait facile à reconnaître, et par conséquent aisé à découvrir.\par
Le premier législateur de l’antiquité avait si bien reconnu ce désordre, que la monnaie qu’il introduisit dans sa république était un métal si commun et d’un si grand volume, que ce prétendu précis de toutes les denrées avait un corps presque aussi étendu que les choses qu’il représentait : ainsi les voleurs, les banqueroutiers, et tous ceux qui ont besoin de secret et d’obscurité pour perpétrer les crimes, n’en étaient pas beaucoup mieux servis.\par
Mais il n’est pas encore temps de finir l’usage que le crime fait du seul argent, et dont il serait empêché par les autres genres de biens, s’ils n’avaient point cette malheureuse représentation : les banqueroutiers qui déconcertent entièrement le commerce, mettant tout le monde dans la défiance, et empêchant que l’on ne puisse trafiquer par crédit et par billets, ne pourraient presque plus voler aussi impunément tout le monde, qu’ils font journellement. On sait que leur jeu et leur manœuvre sont de se servir d’une réputation bien ou mal acquise, pour acheter de tous côtés à crédit, à tel prix que l’on y veut mettre, parce qu’ils sont bien assurés qu’ils n’en débourseront jamais rien ; puis, qu’ils revendent sur-le-champ, argent comptant, la moitié ou les deux tiers moins, et continuent cette fraude jusqu’à l’échéance de leurs billets, qu’ils font cession entière de biens, sous prétexte de prétendues pertes dont il les faut croire, attendu que la conviction du contraire est un procès éternel, encore plus ruineux envers ceux qui perdent, que la banqueroute même.\par
Et cette fraude est ce qu’il y a de moins désolant par rapport à tout le corps de l’État, attendu que la cherté que cela met à l’argent par ces crues d’usage, quoique criminel, le portant jusqu’au ciel, ainsi qu’on l’a dit, fait descendre en même temps l’autre côté de la balance, savoir celui des denrées, jusqu’aux abîmes ; l’un prend le prix des pierres précieuses, et l’autre n’est plus que de la poussière, par la prodigalité que l’on en fait, afin de parvenir à des desseins coupables. Et, bien que ces démarches ne se rencontrent qu’en quelques particuliers, elles ne laissent pas d’être contagieuses à toute la masse, parce que toutes choses ayant une solidarité d’intérêt, tant meubles qu’immeubles, la moindre atteinte qui arrive à une partie, soit en bien ou en mal, devient aussitôt commune à tout le reste.\par
Les blés ne peuvent hausser ni baisser considérablement en un marché, sans que cette disposition ne gagne aussitôt tous les lieux circonvoisins ; et sa continuation de trois ou quatre semaines seulement la fait pénétrer d’un bout du royaume à l’autre, de quelque étendue qu’il soit, et même plus loin.\par
Enfin, la gangrène à l’extrémité des membres du corps humain fait périr bientôt tout le sujet, quoique toutes les parties d’abord très éloignées du mal paraissent très saines et en fort bon état ; mais c’est ce qu’on expliquera mieux dans le chapitre suivant, qui sera celui des richesses, en montrant ce qu’elles doivent être pour rendre un pays opulent, surtout lorsqu’il est fourni de denrées par la nature.\par
Il n’est pas encore temps de finir le récit des ravages de l’argent, et de montrer que lui seul fait plus de dégât dans les contrées où l’on n’a pas soin de le renfermer dans ses véritables bornes, que toutes les nations barbares qui ont inondé la terre, exerçant toutes sortes de violences dont les histoires sont remplies.\par
Jusqu’ici, quelque grands que soient les désordres par lui causés que l’on vient de décrire, comme le sont tous crimes défendus par les lois, et qu’elles punissent même sévèrement lorsque la justice en peut être faite, la déclamation ou la description ne pouvait guère se terminer qu’à des vœux pour en voir la cessation, quoique néanmoins quelques-uns de ces crimes, comme les banqueroutes, tirent leur principe de plus loin, savoir d’une nécessité causée par un précédent déconcertement d’État, qui n’est point du tout l’effet d’un brigandage, ou de voleurs de grands chemins. Cette malheureuse idolâtrie de l’argent, source de tous les maux, n’aurait pas ses temples si remplis d’adorateurs, s’il n’y en avait point d’autres que des sujets exposés sans quartier à la rigueur des lois.\par
Voici bien un autre cortège, savoir ceux qui ont soin de faire payer les tributs des princes : la rigoureuse poursuite, et les recherches qu’on en a faites dans bien des occasions, sans parler de la voix publique, purgent cet énoncé de tout soupçon de calomnie, ou de discours séditieux. C’est au contraire le plus grand service qu’on puisse rendre aux princes, de faire voir la surprise qu’eux et leurs ministres souffrent, quoique bien intentionnés, dans cette grande préférence que ceux qui se couvrent de leur autorité donnent à l’argent sur les autres denrées ; bien que l’un ou l’autre soit indifférent au souverain, comme il l’est pareillement à tout ce qui est à leur solde, et surtout à leurs gens de guerre, qui n’ont pas sitôt reçu leur montre, qu’ils la convertissent à leur nourriture et aux besoins de la vie, en sorte qu’il leur serait égal de les recevoir immédiatement sans le ministère d’argent, comme cela se pratique en beaucoup d’endroits.\par
On éclaircira et on traitera davantage de cette vérité dans un chapitre particulier, où l’on montrera qu’il y a tel prince qui ne procure pas une pinte de vin à aucun de ses soldats, qu’on n’en ait anéanti jusqu’à vingt et même cent qu’il aurait reçues, si on n’avait pas immolé cette quantité à la volonté déterminée d’avoir de l’argent à quelque prix que ce fût, et non du vin ; et ainsi du reste.\par
Ce sont donc ceux qui surprennent leur autorité, qui inspirent que l’argent qu’ils font payer au prince n’est considérable que par sa quantité, et nullement par la manière dont il est levé sur les peuples. Et, bien que les souverains ne le reçoivent que pour fournir le moyen à ceux à qui ils le distribuent de se procurer les. besoins de la vie, ils osent prétendre qu’il n’est d’aucune considération que ces médiateurs aient abîmé ou anéanti pour vingt fois davantage de ces mêmes besoins, en faisant ce fatal recouvrement, que le maître ou ceux qui sont à sa solde n’en pourront avoir avec l’argent qui en provient, et qui leur est distribué.\par
Voilà un crime effroyable de ce métal, qui, bien loin d’être poursuivi par les prévôts comme les voleurs de grands chemins, est tous les jours couronné de lauriers, quoiqu’il ne fasse pas moins d’horreur au peuple, et que les maux qu’il cause excèdent tous ceux que l’on pourrait recevoir des plus fameux brigands, qui auraient une pleine liberté d’exercer les dernières violences.\par
Des contrées entières autrefois en valeur, présentement incultes des fruits les plus précieux, entièrement à l’abandon sans en pouvoir trouver les frais de la culture ; et surtout les liqueurs, pendant que les pays voisins ne boivent que de l’eau, et les achètent un prix exorbitant pour les extrêmes nécessités, ce qui ne va pas à la centième partie de la consommation possible, et leur fait souffrir le même sort pour d’autres denrées principales et singulières, qu’ils donneraient en contre-échange ; toutes ces choses, dis-je, qui sont autant de témoins vivants, quoique muets, montrent que ce n’est point exagération que cette préférence de crime et de désordre que fon donne à ces pourvoyeurs d’argent, sur tous les autres genres de violences et de vexations.\par
En effet, si les tributs s’exigeaient en essence sur chaque fruit et chaque denrée, comme on a fait uniquement très longtemps, et qu’il se pratique même en quantité d’endroits, puisqu’enfin toute réception d’impôt n’est que pour parvenir à ce recouvrement de denrées, et que ce cruel médiateur, savoir l’argent, en abîme une si grande quantité par son fatal ministère ; si, dis-je, cette exigence se faisait {\itshape réellement}, l’horreur de pareils effets aurait absolument empêché leur introduction, ou au moins l’aurait fait rejeter au plus vite à la première expérience. — Aurait-on pu, de sens rassis, mettre une ordonnance sur le papier, qui portât que quiconque recueillera sur sa terre trente setiers de blé, en paiera quarante pour l’impôt ; et un autre, dont la levée va à deux cents, ne contribuera que de quatre, et même moins, suivant son crédit ? — Comme une pareille demande, ainsi que l’exécution, aurait une vue et un visage effroyable, il les a fallu masquer, et c’est ce que l’argent fait merveilleusement bien ; il dérobe toute l’horreur d’une pareille mesure aux personnes élevées qui pourraient y donner ordre, parce que n’ayant qu’une idée confuse du détail, qui ne s’apprend que par la pratique, c’est-à-dire la vie privée, ce qui est bien éloigné de leur situation, ils ignorent tout à fait que qui que ce soit ne peut payer un sou, ni de tribut ni d’autres redevances, que par la vente des denrées qu’il possède ; et qu’ainsi la demande d’argent a des limites de rigueur, données par la nature, qui ne peuvent être violées sans produire un monstre effroyable.\par
En effet, si le manque de succès s’en tenait à un simple refus, on pourrait dire qu’il n’y aurait que du temps et du papier perdus ; mais il s’en faut beaucoup que les choses en demeurent là ; l’impossibilité morale et naturelle, qui n’arrête pas ceux qui sont chargés de pareilles exactions, force la nature pour se faire obéir ; et les préciputs qui doivent être pris avant le tribut, et même toutes sortes d’exigences, savoir les frais de la culture, sont d’abord immolés, ainsi que les ustensiles et instruments pour y parvenir ; et la certitude où cela met d’un abandon de toute la terre à l’avenir, c’est-à-dire mille de perte pour un de profit, n’est d’aucune considération pour des gens en qui domine l’intérêt du moment présent, soit qu’ils soient poussés par la nécessité d’agir de la sorte, faute de quoi ils seraient sujets eux-mêmes à pareil dommage, ce qui n’est que trop connu, ou soit que leur fortune singulière ne leur soit promise qu’à ce prix, ce qui est pareillement fort ordinaire ; enfin, dans l’un ou l’autre cas l’intérêt, dis-je, de ce moment acheté à si haut prix aux dépens du bien public, prévaut à toutes ces suites funestes, quelque nombreuses et quelque effroyables qu’elles soient, qui sont inséparables de cette conduite. Et puis, quand tous ces moyens sont à bout, un homme est criminel parce qu’il n’a pu faire l’impossible et donner ce qu’il n’a point ; on le traîne en prison, et on l’y tient des mois entiers par surcroît de perte de biens, savoir celle de son temps et de son travail, qui est son unique revenu, ainsi que celui de l’État et du prince.\par
Voilà le beau ménage de l’argent dans les tributs, qui ne diffère guère, s’il ne le surpasse, de celui de brigands, puisqu’au moins dans ce dernier, ce qui est enlevé de force demeure dans l’État, et qu’il n’y a que la justice de blessée, au lieu que dans l’autre manière le tout est anéanti.\par
En quoi le prince et les personnes mêmes qui, sur deux cents setiers de récolte, n’en veulent payer que quatre, pour laisser contribuer un misérable de trente sur vingt, prennent tout à fait le change, bâtissant absolument leur ruine, comme on fera voir, dans un chapitre particulier des véritables richesses, où l’on montrera que ces personnes puissantes y auraient gagné si elles avaient voulu contribuer aux impôts de cinquante setiers sur les deux cents mentionnés, et feront même un profit considérable quand elles en voudront user de la sorte et ne pas abîmer un malheureux dont le maintien fait l’opulence des riches, quoique ce soit la chose qu’ils conçoivent le moins, qu’il ne peut être détruit sans rendre sa perte commune à tout l’État.\par
Dans les impôts qu’on tire sur les liqueurs dans certains États, l’argent sert de manteau pour le moins à d’aussi grandes absurdités ; sous cette couverture, on suppose et on exige l’impossible, sans que les suites funestes d’une pareille conduite puissent presque jamais faire revenir les auteurs de démarches effroyables.\par
On pense tranquillement, en cet article de liqueurs, que l’argent croît dans une vigne ou dans la futaille, et non pas que l’on ne peut recouvrer ce métal que par la vente de cette denrée, vente qui est loin encore de représenter un bénéfice jusqu’à concurrence de tout ce que produit la nature, puisque sur le prix qui en provient, il y en a une partie qu’on doit regarder comme sacrée, et sur laquelle on ne saurait rien prendre sans crime, savoir celle qu’il a fallu pour couvrir les frais sans lesquels il n’y aurait rien du tout pour qui que ce soit au monde.\par
Il faut bien que cela soit, encore une fois, et que l’on suppose ce prodige, quand on demande tranquillement et sans prétendre déroger aux lois de la sagesse, de la prudence et de la politique la plus consommée, la valeur de quarante muids de vin sur une pièce de vigne qui n’en a produit que trente, et celle de trois cents pintes de vin sur une futaille qui n’en contient que deux cents ; en sorte que l’abandon entier qu’on en peut faire ne puisse point acquitter le marchand, et qu’il faut que sa personne et ses autres biens répondent du surplus, ce qui n’est pas absolument sans exemple en quelques contrées de l’Europe, et est un mal contre lequel on n’a point trouvé d’autre remède que de renoncer à la culture de la denrée en question, afin d’en être quitte par la perte de ce seul genre de biens, ce qui va dans plusieurs contrées à des centaines de millions par an ; et par-dessus cela, le mal se recommuniquant à toutes les autres espèces par une solidarité d’intérêts qu’elles ont entre elles, fait que cette même destinée gagne à peu près tous les autres genres de biens ; et voilà d’où procède ce grand déchet et cette épouvantable diminution arrivée à toutes choses, tant meubles qu’immeubles, dans ces mêmes pays. L’argent y a transgressé ses bornes naturelles d’une façon effroyable ; il a pris un prix de préférence sur toutes les autres denrées avec lesquelles il doit être seulement en concurrence pour conserver l’harmonie d’un État, c’est-à-dire une opulence générale, ce qui fait que, bien loin de servir à faciliter le trafic et l’échange des besoins de la vie, il en devient le tyran et le vautour, s’en faisant immoler tous les jours des quantités effroyables par un pur anéantissement, pour procurer très peu de ce métal par rapport à ce qu’il en coûte à tout le corps de l’État, à des entrepreneurs qui le possèdent moins innocemment que des voleurs de grands chemins, bien qu’ils ne pensent rien moins, attendu que les désastres que cette acquisition cause l’emportent de vingt fois sur les autres, quelque grands et quelque violents qu’ils soient.
\section[{Chapitre IV.}]{Chapitre IV.}
\noindent On a dit en général, au commencement de ces Mémoires, en quoi consistait la véritable richesse, savoir en une jouissance entière, non seulement des besoins de la vie, mais même de tout le superflu et de tout ce qui peut faire plaisir à la sensualité, sur laquelle la corruption du cœur invente et raffine tous les jours ; le tout néanmoins, dans toutes sortes d’états, à proportion que l’excès du nécessaire met en pouvoir de se procurer ce qui ne l’est pas à beaucoup près.\par
C’est ce qui fait que dans l’enfance ou l’innocence du monde, que l’homme était riche par la seule jouissance des simples besoins, il n’y avait de l’emploi que pour trois ou quatre professions ; ce qui se pratique encore eh quantité de pays mal partagés par la nature du côté du terroir ou du côté de l’esprit.\par
Mais aujourd’hui, dans les contrées où des dispositions contraires ont porté les choses dans l’excès en cet article d’opulence et de volupté, il y en a plus de deux cents, sans celles qui s’inventent tous les jours.\par
Il est donc à propos d’en faire un détail plus particulier, et de montrer que si c’est une richesse que cette ample possession de tout ce que l’esprit peut découvrir au-delà du nécessaire, c’est la situation la plus périlleuse et qui a le plus besoin de ménagement ; autrement il arrive que ce qui a été institué pour faire jouir du superflu ne sert, quand les mesures sont mal prises, qu’à priver du nécessaire, jetant en un instant un État du faîte de l’opulence au dernier degré de misère.\par
Les deux cents professions qui entrent aujourd’hui dans la composition d’un État poli et opulent, ce qui commence aux boulangers et finit aux comédiens, ne sont, pour la plupart, d’abord appelées les unes après les autres que par la volupté ; mais elles ne sont pas sitôt introduites, ou n’ont pas pris racine en quelque sorte, que faisant après cela partie de la substance d’un État, elles n’en peuvent plus être disjointes ou séparées, sans altérer aussitôt tout le corps. Elles sont toutes, et jusqu’à la moindre ou la moins nécessaire, comme l’empereur Auguste, de qui on disait fort justement qu’il n’aurait pas dû naître, ou n’aurait pas dû mourir.\par
Pour prouver ce raisonnement, il faut convenir d’un principe, qui est que toutes les professions, quelles qu’elles soient dans une contrée, travaillent les unes pour les autres, et se maintiennent réciproquement, non seulement pour la fourniture de leurs besoins, mais même pour leur propre existence.\par
Aucun n’achète la denrée de son voisin ou le fruit de son travail qu’à une condition de rigueur, quoique tacite et non exprimée, savoir que le vendeur en fera autant de celle de l’acheteur, ou immédiatement, comme il arrive quelquefois, ou par la circulation de plusieurs mains ou professions interposées, ce qui revient toujours au même ; sans quoi il se détruit la terre sous les pieds, puisque non seulement il le fera périr par cette cessation, mais même il causera sa perte personnelle, le mettant par là hors d’état de retourner chez lui à l’emplette, ce qui lui fera faire banqueroute et fermer sa boutique.\par
Il faut donc que ce commerce continue sans interruption, et même à un prix qui est de rigueur, quoique ce soit ce qu’on conçoive le moins, c’est-à-dire à un taux qui rende le marchand hors de perte, en sorte qu’il puisse continuer son métier avec profit ; autrement, c’est comme s’il ne vendait point du tout ; et périssant, il en arrivera comme dans ces vaisseaux accrochés, dont l’un met le feu aux poudres, ce qui les fait sauter tous deux.\par
Cependant, par un aveuglement effroyable, il n’y a point de négociant, quel qu’il soit, qui ne travaille de tout son pouvoir à déconcerter cette harmonie ; ce n’est qu’à la pointe de l’épée, soit en vendant, soit en achetant, qu’elle se maintient ; et l’opulence publique, qui fournit la pâture à tous les sujets, ne subsiste que par une Providence supérieure, qui la soutient comme elle fait fructifier les productions de la terre, n’y ayant pas un moment ni un seul marché où il ne faille qu’elle agisse, puisqu’il n’y a pas une seule rencontre où on ne lui fasse la guerre.\par
Tant que les choses demeurent dans cet équilibre, il n’y a point d’autre ressource pour s’enrichir, en quelque état que l’on soit, que de forcer de travail et d’habileté sur son voisin, non pour le tromper en tâchant d’avoir sa denrée à vil prix, mais pour le devancer en adresse.\par
Et cette émulation devenant générale par le désespoir de s’enrichir autrement, tous les arts se perfectionnent, et l’opulence est portée au plus haut point où elle puisse être.\par
L’argent, à qui ce chapitre avait donné du repos, bien loin d’être le tyran de la richesse, et d’abîmer toutes les denrées comme il fait dans la situation contraire, n’est que le très humble valet du commerce, et à peine trouve-t-il quelqu’un qui lui veuille donner retraite : quand il se présente en trop grande quantité à la fois, il n’y a point de denrée pour si déplorée qu’elle soit, pourvu qu’elle soit de mise, soit meuble ou immeuble, à laquelle on ne donne la préférence.\par
Comme il n’est et ne doit être que le gage de la tradition future, quand elle ne s’effectue pas sur-le-champ, et qu’il ne réside ou n’apparaît pas assez de solvabilité dans l’acheteur, pour la garantie par sa parole ou par son billet, sans quoi on préférerait cette voie au service de ce métal ; ne se rencontrant presque personne qui ait besoin de cette caution, par la valeur soutenue de toutes les denrées personnelles, cela les met hors de cette nécessité ; et c’est alors une conséquence indubitable que ce métal soit remercié presque par tout le monde.\par
Ainsi, étant absolument inutile au commerce, il est obligé, pour ne pas demeurer à rien faire, d’offrir son service au ménage et à la magnificence, et d’avoir recours à l’orfèvre et aux autres ouvrages ; ce qui n’est encore que le moindre désordre, car il est dans l’attente qu’on ait besoin de lui, auquel cas il est toujours prêt à bien faire, encore que ce secours ne puisse être imploré sans que l’État soit malade, et d’une si épouvantable indisposition, que, si elle était longue, le remède serait de moindre durée que le mal, dont on connaît l’extrémité par la recherche ou la cherté où l’or et l’argent se trouvent.\par
Dans l’autre situation, savoir celle de l’opulence, il est la dernière des denrées ; et dans la disette, il est non seulement la première, mais même presque l’unique ; dans le premier état, il n’y a que les indigents qui lui fassent la cour, et à qui il soit absolument nécessaire, étant même seuls au désespoir d’être dans cette servitude, et faisant tous leurs efforts pour en sortir ; et dans l’autre, les plus riches en ont à peine autant qu’il leur en faut, ce qui réduit toutes les autres conditions dans la dernière extrémité.\par
Cette disposition, qui est une maladie très dangereuse dans un État, n’est causée que par le déconcertement du prix des denrées, qui doit être toujours proportionné, n’y ayant que cette intelligence qui les puisse faire vivre ensemble, pour se donner à tous moments, et recevoir réciproquement la naissance les unes des autres.\par
Mais, comme leur dissension, et par conséquent la misère, n’est pas une chose fort inconnue dans l’Europe, il faut examiner qui a le premier commencé la querelle, et par où le désordre s’est introduit.\par
On a dit, dans ces Mémoires, que ces deux cents professions qui composent la perfection des États les plus polis et les mieux partagés par la nature, sont tous enfants des fruits de la terre ; que le plus ou le moins qu’elle est en état d’en produire avec abondance, et de faire consommer, sans quoi l’excroissance devient inutile et même à perte, est ce qui leur donne naissance, en commençant par le plus nécessaire, comme le boulanger et le tailleur, et finissant par le comédien, qui est le dernier ouvrage du luxe et la plus haute marque d’un excès du superflu, puisqu’il ne consiste qu’à flatter les oreilles, et réjouir l’esprit par un simple récit de fictions que l’on sait bien n’avoir jamais eu de réalité ; en sorte qu’on est si fort hors de crainte de manquer du nécessaire, que l’on achète avec plaisir la représentation du mensonge, comme il arrive dans ces occasions.\par
Ainsi quand l’état contraire, c’est-à-dire la misère, vient à s’introduire et à vouloir prendre la place de cet état florissant, c’est par cette profession que l’on commence la réforme, comme c’était par elle que l’on avait fini l’acquisition du superflu.\par
Cependant, comme ce n’est pas de son consentement, puisque ce congé envoie ces rois de théâtre personnellement à l’hôpital, et que ce retranchement ne s’en tient pas singulièrement à ces gens-là, faisant bien d’autres progrès toujours par degrés, cela ne peut arriver sans déconcerter tout un pays où plutôt toutes les professions, par les raisons qu’on a marquées.\par
Ils sont donc à plaindre, tant par rapport à eux qu’aux autres conditions que cela dérange et anéantit pareillement par contrecoup, attendu encore une fois qu’il en est d’un genre de métier comme de l’empereur Auguste, qu’il ne doit jamais être reçu, ou qu’il ne le faut jamais congédier ; l’ouvrier du superflu achetant son nécessaire de celui qui lui donnait sa vie à gagner, et soutenant par là le prix des denrées du laboureur, ce qui seul le peut faire payer son maître, et mettre celui-là en pouvoir d’acheter de cet ouvrier.\par
Mais si quelque chose diminue la pitié qu’on pourrait avoir d’eux, ou plutôt pour entrer dans la discussion de la cause de leur congé, on peut assurer que ce sont eux-mêmes qui se le procurent, et qu’ils se creusent tous le tombeau où ils sont enterrés.\par
On a dit, comme c’est la vérité, que ce sont les fruits de la terre, et principalement les blés, qui mettent toutes les professions sur pied : or, leur production n’est ni l’effet du hasard, ni un présent gratuit de la nature ; c’est une suite d’un travail continuel, et de frais achetés à prix d’argent, cette manne primitive et nécessaire n’étant abondante qu’à proportion qu’on est libéral pour n’y rien épargner, refusant entièrement tout à qui ne lui veut rien donner.\par
Or, il y a une attention à faire, qui est que les propriétaires des fonds, quoique paraissant les mieux partagés de la fortune, comme les maîtres absolus de tous les moyens de subsistance, ne sont au contraire que les commissionnaires et les facteurs de toutes les autres professions, jusqu’aux comédiens, et comptent avec elles tous les jours de clerc à maître ; et si un cordonnier ne peut vivre sans pain, qu’il ne recueille pas assurément sur des fonds qu’il ne possède point, un possesseur de terre ne saurait marcher sans souliers, et ainsi des autres.\par
Ces propriétaires, dis-je, donnent à chaque moment un mémoire des frais déboursés pour cultiver les fonds dont les métiers d’industrie sont soutenus et nourris : si leur dépense est allouée, comme il arrive lorsque les blés sont à un prix qui puisse supporter leurs frais avec des appointements honnêtes pour le facteur, le ménage continue, et chacun vit tranquillement dans sa profession, sans que qui que ce soit songe à prendre congé de l’autre.\par
Mais si par malheur le contraire arrive, et que l’abaissement du prix des grains (ce qui n’est pas présentement inconnu dans l’Europe) ne puisse atteindre aux frais de la culture, lesquels une fois contractés ne baissent jamais tout à coup comme font les blés, ne pouvant alors dédommager le pourvoyeur de sa dépense faite, ainsi que satisfaire au paiement de ses appointements ; celui-ci n’est non plus en état de continuer à nourrir tout un peuple, que les boulangers d’une ville qu’on obligerait de tenir leurs boutiques fournies, ayant le prix du pain au-dessous de celui des grains.\par
Voilà la cause du désordre et le principe de la querelle qui, augmentant toujours à la longue comme une pelote de neige ou comme un chancre, forme une extrême misère au milieu de l’abondance de toutes choses.\par
Un comédien se réjouit, ainsi que tous les autres, c’est-à-dire tous les métiers, d’avoir par une grâce spéciale du ciel, à ce qu’il croit, le pain à très grand marché, et que pour un sou il en recouvre autant qu’il en peut consommer en toute sa journée ; s’il lui en fallait pour deux sous, il ne serait pas dans cette joie.\par
Mais il ne voit pas, le malheureux qu’il est, ainsi que l’on a dit, qu’il se creuse son tombeau, et que le facteur et le propriétaire de fonds n’étant plus payé de ses frais et de ses appointements par son fermier, avec qui il ne forme qu’un intérêt, est obligé de se retrancher, et que commençant par le superflu, le comédien se trouve à la tête, et cessera par là de gagner un écu par jour parce qu’il a voulu et s’est réjoui de gagner un sou sur son pain.\par
Ce qu’il y a de merveilleux est qu’après cela l’un et l’autre, tant le comédien que celui qui allait au spectacle, jouent à qui pis faire, et à qui s’entre-détruira le plus tôt, en pensant se sauver réciproquement. Comme les biens ne viennent pas tout d’un coup, ainsi que leur jouissance, et que tout se fait par degrés, on peut dire qu’ils en usent de même dans leur décadence, s’en retournant pareillement par gradation.\par
Un homme qui allait autrefois tous les jours à la comédie dans le temps de son opulence, c’est-à-dire que ses fermiers, par la vente de leurs denrées aux comédiens mêmes, le payaient ponctuellement, y trouvant de la diminution par quelque cause violente, et telle qu’on a marqué ci-devant, savoir celles qui anéantissent cent fois autant de biens qu’elles font recevoir d’argent sur-le-champ à l’entrepreneur ; expérimentant, dis-je, ce déchet, se retranche à n’y aller plus que trois fois la semaine, pour compenser par la diminution de sa dépense celle qui lui arrive dans sa recette.\par
Le comédien, de son côté, qui est atteint du même mal, en fait tout autant de sa part, et s’il mangeait de la viande et même de la volaille tous les jours, il retranche pareillement son ordinaire, et se réduit à ne faire semblablement bonne chère que la moitié du temps ; par où, outre l’avilissement du prix des grains, le fermier de celui qui allait à la comédie, et qui est marchand de bestiaux, reçoit un surcroît de difficulté de payer son maître, et celui-ci de faire subsister le comédien ; et l’extravagance est de mettre ce déconcertement sur le compte du manque d’espèces, comme si l’on était au Pérou où, prenant naissance, elles sont le seul et unique principe de subsistance.\par
Et cette manœuvre continue jusqu’à ce qu’ils aient pris réciproquement tout à fait congé l’un de l’autre, ce qui est absolument la ruine d’un État, et d’un prince plus que de qui que ce soit, comme on l’expliquera dans le chapitre de l’intérêt des souverains.\par
C’est le même raisonnement de toutes les autres professions, qui ne sont toutes misérables que par la même conduite et les mêmes circonstances.\par
Mais ce qu’il y a de plus étonnant, est que l’avilissement du prix des grains, qui tient certainement la première place dans la désolation publique, est regardé au contraire comme le conservateur de l’utilité générale.\par
L’on ne se croit pouvoir garantir des horreurs de la disette qu’en se jetant dans la situation tout opposée, qui n’est pas moins préjudiciable à un État, puisqu’il est constant que toutes les extrémités, ou plutôt tous les excès, sont également dommageables, quoique toujours diamétralement contraires.\par
En effet, vouloir que les grains soient à si bas prix qu’ils ne puissent atteindre aux frais de la culture ni faire payer le propriétaire, en sorte qu’il ne soit point en état de donner du travail aux ouvriers qui n’ont d’autre moyen de subsister, c’est comme si on bannissait l’entier usage des liqueurs, même pour faire revenir un homme d’une faiblesse, parce qu’on en a vu quantité qui en avaient tant pris qu’ils en avaient perdu la raison, et même assez souvent la vie.\par
Mais c’est assez parler de richesses ; il faut venir présentement à la misère, quoique l’explication de l’une fasse le portrait de l’autre.
\section[{Chapitre V.}]{Chapitre V.}
\noindent Tout le monde sait ce que c’est que d’être misérable, puisque chacun travaille depuis le matin jusqu’au soir pour ne le point devenir, à moins que les passions ne l’aveuglent, ou pour cesser de l’être, s’il est assez malheureux pour se trouver dans cette situation.\par
Tous ont donc cette disposition en particulier, mais pas un n’a jamais étendu ses vues jusqu’au général, bien qu’on ne puisse nullement être riche d’une façon permanente, et le prince plus que les autres, que par l’opulence publique ; et que jamais qui que ce soit ne jouira aisément et longtemps de pain ou de vin, de viande, d’habits, de la magnificence la plus superflue, tant qu’il n’y en aura pas dans le pays, et même avec abondance : autrement ses fonds deviendront à rien, et son argent s’en ira sans pouvoir revenir.\par
Aucun n’est son propre ouvrier de toutes ces choses en général ; personne même, quelque riche qu’il soit, n’a point de domaine assez étendu pour qu’elles croissent toutes à beaucoup près sur ses fonds.\par
Il n’y a pareillement qui que ce soit qui, en possédant singulièrement et uniquement la denrée la plus précieuse pour la valeur, ne serait très misérable, s’il ne pouvait échanger ce qu’il a de trop pour recouvrer ce qui lui manque, en tirant ceux avec qui il traite de la pareille et fâcheuse obligation de consommer dix fois plus d’une chose qu’il ne leur est nécessaire, et d’être obligés de se passer de toutes les autres.\par
Comme la richesse donc n’est que ce mélange continuel, tant d’homme à homme, de métier à métier, que de contrée à contrée, et même de royaume à royaume ; c’est un aveuglement effroyable d’aller chercher la cause de la misère ailleurs que dans la cessation d’un pareil commerce, arrivée par le dérangement de proportion dans les prix, qui n’est pas moins essentielle à la prospérité de tous les États, qu’au maintien même de leur existence.\par
Tous entretiennent nuit et jour cette richesse par leur intérêt particulier, et forment en même temps, quoique ce soit ce à quoi ils songent le moins, le bien général dont, malgré qu’ils en aient, ils doivent toujours attendre leur utilité singulière.\par
Il faut une police pour faire observer la concorde et les lois de la justice parmi un si grand nombre d’hommes, qui ne cherchent qu’à les détruire, et qu’à se tromper et à se surprendre depuis le matin jusqu’au soir, et qui aspirent continuellement à fonder leur opulence sur la ruine de leurs voisins. Mais c’est à la nature seule à y mettre cet ordre, et à y entretenir la paix ; toute autre autorité gâte tout en voulant s’en mêler, quelque bien intentionnée qu’elle soit. La nature même, jalouse de ses opérations, se venge aussitôt par un déconcertement général, du moment qu’elle voit que, par un mélange étranger, on se défie de ses lumières et de la sagesse de ses opérations. Sa première intention est que tous les hommes vivent commodément de leur travail, ou de celui de leurs ancêtres ; en un mot, elle a établi qu’il faut que chaque métier nourrisse son maître, ou qu’il doit fermer sa boutique, et s’en procurer un autre : comme elle ne peut pas aimer les hommes moins qu’elle ne fait les bêtes, et qu’elle ne met point au monde une seule de ces dernières qu’elle ne l’assure de sa pitance en même temps, elle agit pareillement à l’égard des hommes partout où l’on s’en rapporte à elle.\par
Ainsi, afin que ce dessein soit effectué, il est nécessaire que chacun, tant en vendant qu’en achetant, trouve également son compte, c’est-à-dire que le profit soit justement partagé entre l’une et l’autre de ces deux situations. Cependant on ne chicane tant, comme l’on voit dans toutes sortes de marchés avant que de les conclure, qu’afin de donner atteinte à cette règle de justice : chaque commerçant, soit en gros ou en détail, voudrait que le profit du marché, au lieu d’être partagé comme cela doit être, fût pour lui seul, en dût-il coûter tous les biens et même la vie à son compatriote. Car de songer que c’est la ruine d’un État, de même que si le trafic se faisait avec de faux poids ou de fausses mesures, c’est de quoi qui que ce soit ne s’embarrassa jamais l’esprit, quoiqu’on puisse fort bien appliquer à cette conduite la maxime de l’Évangile, qui porte que, {\itshape de la même règle qu’on mesure les autres, on sera soi-même mesuré} ; et qu’il arrive que parce qu’on a voulu avoir la denrée de son voisin à perte, on sera obligé de donner la sienne de la même façon, par les causes que l’on a marquées.\par
La nature donc, ou la Providence, peut seule faire observer cette justice, pourvu encore une fois que qui que ce soit autre ne s’en mêle ; et voici comme elle s’en acquitte. Elle établit d’abord une égale nécessité de vendre et d’acheter dans toutes sortes de trafics, de façon que le seul désir de profit soit l’âme de tous les marchés, tant dans le vendeur que dans l’acheteur ; et c’est à l’aide de cet équilibre ou de cette balance, que l’un et l’autre sont également forcés d’entendre raison, et de s’y soumettre.\par
La moindre dérogeance, sans qu’il importe dans lequel des deux, gâte aussitôt tout ; et pourvu que l’un s’en aperçoive, il fait aussitôt capituler l’autre, et le veut avoir à discrétion ; et s’il ne lui tire pas l’âme du corps, ce n’est pas manque de bonne volonté, puisqu’il ne tiendrait pas à lui qu’il n’en usât comme dans les villes pressées par un long siège, où l’on achète le pain cent fois le prix ordinaire parce qu’il y va de la vie.\par
Tant, encore une fois, qu’on laisse faire la nature, on ne doit rien craindre de pareil ; aussi n’est-ce que parce qu’on la déconcerte, et qu’on dérange tous les jours ses opérations, que le malheur arrive.\par
On a dit, et on le répète encore, qu’afin que cette heureuse situation subsiste, il faut que toutes choses et toutes les denrées soient continuellement en équilibre, et conservent un prix de proportion par rapport entre elles, et aux frais qu’il a fallu faire pour les établir. Or, on sait que du moment que ce qui est en équilibre, comme dans une balance, reçoit le moindre surcroît en un des côtés, incontinent l’autre est emporté aussi haut que s’il n’y avait rien du tout.\par
Il en arrive de même dans toutes sortes de commerces : c’est tout ce que peut faire une marchandise, que de se défendre de l’oppression de l’autre, quand même il n’arriverait aucun secours étranger à son ennemie ; mais, du moment que cela advient, comme il n’est que trop connu, on peut dire aussitôt que tout est perdu, tant celui qui profite du malheur d’autrui que le sujet qui le souffre.\par
On éprouve ce sort de deux manières, savoir quand le marchand, ou sa denrée, se trouve atteint de quelque coup violent et imprévu, ce qui est égal et produit le même effet.\par
Voici comme la chose se passe, lorsque c’est le marchand, soit vendeur ou acheteur : on a dit que pour maintenir cet équilibre, unique conservateur de l’opulence générale, il faut qu’il y ait toujours une parité égale de ventes et d’achats, et une semblable obligation ou nécessité de faire l’un ou l’autre, sans quoi tout est perdu. Or, du moment qu’un nombre considérable d’acheteurs ou de vendeurs sont mis dans la nécessité d’acheter moins ou de vendre plus vite, pour satisfaire à quelque demande inopinée, ou s’abstenir de dépenser par la même raison, voilà aussitôt la denrée à rebut, ou par manque d’acheteurs, ou parce qu’il faut la jeter à la tête ; ce qui n’arrive jamais sans ruiner le marchand, parce qu’alors les gens avec qui on contracte, s’éjouissant du malheur de leur voisin, croient avoir trouvé le jeu de s’enrichir de sa ruine, ne voyant pas, comme on a dit, que c’est leur propre tombeau qu’ils construisent. Et il suffit que cette destinée arrive à une partie pour empoisonner tout le reste ; parce que cette parcelle de déconcertement est comme un levain contagieux qui corrompt toute la masse d’un État, par la solidarité d’intérêt que toutes choses ont les unes avec les autres, ainsi que l’on a montré.\par
Si c’est la denrée personnellement qui reçoive une atteinte particulière, et qui, étant donnée précédemment à un prix courant avec profit du marchand, ait besoin d’une hausse par celle qu’elle a reçue inopinément, comme d’un nouveau tribut, pour rendre le vendeur hors de perte ; et que l’acheteur toutefois n’en veuille point entendre parler, la nécessité de vendre où est le marchand pour subsister chaque jour, l’oblige de sacrifier sa ruine future au temps courant. L’acheteur ne songe à rien moins qu’à faire réflexion que tout vendeur n’est que le commissionnaire de l’acheteur, et qu’il doit compter avec lui de clerc à maître, comme un facteur avec un négociant, lui allouant tous ses frais justement déboursés, et lui payant le prix de son travail ; autrement plus de travail, et par conséquent plus de profit pour le maître.\par
Cette justice qui, étant de droit naturel, doit être observée dans le commerce singulier des moindres denrées, à faute de quoi elles se détruisent les unes les autres, est d’obligation indispensable dans le trafic des grains avec tout le reste, parce que donnant naissance à tous les besoins de la vie, en quelque nombre qu’ils soient, ils les jouent tous but à but ; mais il faut que ce soit à armes égales : autrement, par les raisons marquées, l’une a bientôt terrassé l’autre, ce qui est la mort incontinent de tous les deux, comme il n’est que trop connu, et que l’on a fait voir.\par
Cependant par un malheur effroyable, c’est où le déconcertement se rencontre le plus ordinaire, bien qu’il n’en soit pas dans cet article comme dans les autres qui se trouvent presque tous ouvrages de la main des hommes, et par conséquent plus sujets à leurs lois.\par
Mais dans celui-ci la nature y ayant la principale e presque l’unique part, la prévoyance et la sagesse pour en faire la dispensation est son unique affaire, et un ministère étranger ne s’en saurait mêler en nul endroit du monde, sans tout gâter, comme l’on a déjà dit.\par
Elle aime également tous les hommes, et les veut pareillement sans distinction faire subsister. Or, comme dans cette manne de grains elle n’est pas toujours aussi libérale dans une contrée qu’elle l’est dans une autre, et qu’elle les donne avec profusion dans un pays et même dans un royaume, pendant qu’elle en prive un autre presque tout à fait, elle entend que par un secours mutuel il s’en fasse une compensation pour l’utilité réciproque ; et que par un mélange de ces deux extrémités de cherté extraordinaire ou d’avilissement de grains, il en résulte un tout qui forme l’opulence publique, qui n’est autre chose que le maintien de cet équilibre si essentiel, ou plutôt l’unique principe de la richesse, quoique très inconnu aux personnes qui n’ont que de la spéculation.\par
C’est sur quoi elle ne connaît ni différents États ni divers souverains, ne s’embarrassant pas non plus s’ils sont amis ou ennemis, ni s’ils se font la guerre, pourvu qu’ils ne la lui déclarent pas ; ce qui arrivant, quoique par une pure ignorance, elle ne tarde guère à punir la rébellion que l’on fait à ses lois, comme on n’en a que trop fait expérience.\par
Et cela est si vrai, que dans l’empire romain, où presque toute la terre connue ne reconnaissait qu’une domination, et où par conséquent cette diversité de souverainetés ne mettait aucun prince dans ce prétendu et fatal intérêt de se révolter contre les lois de la nature à l’égard des grains, la différence d’un sort contraire à celui tant de fois éprouvé dans l’Europe depuis ces derniers temps, que l’on n’a pas voulu s’en rapporter à elle, est attestée authentiquement par Sénèque le Philosophe, dans ses écrits. Il marque en termes formels que jamais la nature de son temps, quoiqu’il fût fort âgé, ni dans l’antiquité, dont il avait une parfaite connaissance, n’avait refusé, même dans sa plus grande colère, le nécessaire aux hommes pour leur subsistance : s’il avait vécu dans ces derniers temps, il n’aurait pas assurément parlé de la sorte.\par
Les peuples barbares, qui n’ont d’autres lois ni d’autres livres que cette même nature, que l’on a connus dans ces derniers siècles et que l’on découvre même tous les jours, sont encore une preuve vivante et aussi certaine de cette vérité. La nature, leur conductrice, ne leur fait pas à la vérité, dans quelques particuliers, des repas aussi magnifiques ni aussi délicats que dans les pays polis et par conséquent rebelles ; mais en général il s’en faut beaucoup qu’elle leur en procure d’aussi mauvais, en sorte que, tout compensé, il y a à dire du tout au tout entre ces deux dispositions.\par
On s’est étendu sur cet article, parce que la dérogeance à cette loi, qui devrait être sacrée, est la première et la principale cause de la misère publique, attendu que l’observation en est plus ignorée.\par
L’équilibre entre toutes les denrées, unique conservateur de l’opulence générale, en reçoit les plus cruelles atteintes, en sorte que si on voit un royaume tout rempli de biens, pendant que les peuples en manquent tout à fait, il n’en faut point aller chercher la cause ailleurs : celui-ci périt, parce que ses caves sont pleines de vin, et qu’il manque du reste ; cet autre se trouve dans la même disposition à l’égard de ses grains ; et enfin tout le reste vivant d’industrie, languit également, ne pouvant recouvrer de pain et de liqueurs par le fruit de son travail, dont le défaut jette également les possesseurs de ces mannes dans la même misère, de ne pouvoir en échanger une partie contre leurs autres besoins, comme des habits, des souliers et le reste.\par
Si on demande à chacun de ces particuliers la raison de leur misère, ils répondent tranquillement qu’ils ne peuvent rien vendre à moins que ce ne soit à perte, ne prenant garde qu’ils ne sont dans cette malheureuse situation que parce qu’ils prétendent exiger cette règle des autres et ne la pas recevoir pour eux.\par
Un cordonnier veut vendre ses souliers quatre francs, si le prix a été une fois à ce taux ; il n’en démordra jamais d’un sou, à moins que ce ne soit pour faire banqueroute, et veut néanmoins avoir le blé du laboureur pour le prix que l’abondance, jointe à une défense de l’envoyer au-dehors, le force de le donner, c’est-à-dire pour moins qu’il ne lui a coûté à faire venir ; et ainsi de tous les autres ; sans que ce malheureux cordonnier prenne jamais garde qu’il se bâtit sa ruine, parce que ce laboureur est par là mis hors d’état de payer son maître, et celui-ci par conséquent hors de pouvoir d’acheter des souliers du cordonnier : ainsi, en vue de deux ou trois sous par jour que ce dernier gagne sur le pain de sa famille, il se met à l’hôpital lui et tous les siens.\par
Or, ce serait une pure extravagance de prétendre lui faire entendre raison là-dessus, en lui représentant que le prix de quatre francs avait été contracté par ces souliers, parce que les grains étaient à un taux proportionné, en sorte que l’un et l’autre des commerçants pouvaient trafiquer avec profit ; mais que présentement l’un ayant baissé, il faut que l’autre en fasse de même.\par
Une journée qu’il a devant soi de moindre obligation de vendre, que le laboureur qui est poussé par l’impôt ou par le maître, fait qu’il se moque de ces raisons, et tout son chagrin est de n’avoir pas encore le grain à meilleur marché ; et il est assez sot pour en bénir Dieu, qui n’est point assurément auteur de cette situation, parce qu’il ne l’est jamais du mal, qu’il ne fait que permettre ; mais ce sont ceux qui lui procurent par ignorance une si fatale félicité.\par
Quoique cette erreur à l’égard des grains fût plus que suffisante pour déconcerter l’équilibre, unique conservateur du commerce et par conséquent de l’opulence publique, elle reçoit encore une grande aide dans les atteintes particulières que l’on donne tous les jours singulièrement tant aux personnes qu’aux denrées, sur lesquelles les liqueurs en quelques pays en ont assurément pris plus que leur part, puisque c’est là, plus que partout ailleurs, où ces deux extrémités d’excès et de disette exercent plus violemment leur empire.\par
En sorte qu’une si grande combinaison de causes désolantes se rencontrant ensemble, bien que ce fût assez d’une seule pour ruiner tout un royaume, savoir tant à l’égard des grains et des liqueurs qu’autres denrées marquées, on ne doit pas s’étonner de voir habiter ensemble deux choses si contraires, c’est-à-dire une si grande abondance et une si extrême misère.\par
Mais, comme si ce n’était pas assez pour tout abîmer, il en vient encore en surtaux une dernière, dictée en quelque façon par l’injustice même, puisque c’est une dérogeance continuelle à l’équité dans la répartition des impôts.\par
Un homme riche croit avoir tout gagné quand, au lieu d’en prendre sa part par rapport à son opulence, il en accable tout à fait un malheureux, bâtissant sa ruine entière sans s’en apercevoir. Il déclare par là qu’il prétend être seul habitant du monde, et unique possesseur des fonds et de l’argent ; ce qui le jette dans la même situation des premiers habitants de la terre, à proportion que cette conduite a un malheureux succès, et il possède tout sans pouvoir jouir de rien.\par
Il y a là-dessus une attention à faire, à laquelle presque qui que ce soit n’a jamais réfléchi, qui est que, l’opulence consistant dans le maintien de toutes les professions d’un royaume poli et magnifique, qui se soutiennent et se font marcher réciproquement comme les pièces d’une horloge ; toutes, à beaucoup près, ne sont pas dans la même assurance et à l’épreuve de semblables atteintes.\par
Celles qui sont accueillies de longue main, ainsi que les particuliers qui les exercent, ne se trouvent pas absolument déconcertées par la survenue de quelque orage, quand il n’est pas de la dernière violence.\par
Quelques-uns, et même plusieurs, trouvent dans le passé des ressources qui aident au présent et même à l’avenir ; mais il n’en va pas de même à beaucoup près d’une infinité d’autres, c’est-à-dire des malheureux pour qui la misère, tenant continuellement le couteau à la gorge, s’empêcher de périr est tout ce qu’ils peuvent faire en travaillant nuit et jour : il n’y a continuellement qu’un filet de distance entre leur subsistance, même assez frugale, et leur destruction entière. Tout roule assez souvent sur un écu, lequel, par un renouvellement continuel, leur en produit pour l’ordinaire la consommation de cent pendant le cours de l’année. Que s’ils en sont privés par un coup inopiné, adieu les cent écus de consommation pour tout l’État ; ce qui se rencontrant en une infinité de sujets, on voit par là la perte qui en revient à la masse, laquelle seule, malgré l’erreur des riches, est ce qui leur doit procurer leur opulence au sou la livre du débit qui se fait, pendant que cet écu enlevé à un homme puissant n’aurait jamais été qu’un écu, tant à l’égard du particulier que de tout le corps de l’État.\par
On ne doit donc pas s’étonner que le pays où l’assemblage de tant de dérangements se rencontre, soit et paraisse misérable dans l’abondance de toutes choses, et qu’il soit comme un Tantale qui périt de soif au milieu des eaux.\par
Ce n’est point assurément par la faute de la nature, qui a fait plus que son devoir ; c’est parce que non seulement on ne s’en est pas rapporté à ses opérations, mais que même on les a combattues à toute outrance. On a regardé ses présents comme du fumier ; l’idée et l’usage criminel qu’on s’est fait de l’argent est cause qu’on lui a sacrifié pour cent fois autant de denrées les plus nécessaires à la vie que l’on recevait de ce fatal métal, qui n’étant introduit (ainsi qu’on a marqué) que pour faciliter le commerce et l’échange, est devenu le bourreau de toutes choses, parce qu’aucune n’a le pouvoir comme lui de servir et de couvrir les crimes en acquérant ou en dépensant.\par
Cet état de misère ayant donc fait un Dieu de ce qui n’était qu’un esclave dans la situation contraire, savoir dans la richesse, il faut voir avec quelle tyrannie il exerce sa puissance, et quel honteux hommage il fait rendre à sa divinité.\par
Premièrement, il lui faut faire satisfaction du passé ; et l’outrage qu’il prétend avoir reçu de la concurrence, et même de la préférence que l’on avait donnée à un morceau de papier et même à la simple parole, sur un métal si précieux, doit être solennellement expié par le feu, où tous ses concurrents doivent être jetés à fort peu près, avec promesse de ne s’en plus servir à l’avenir. Ceci n’est point un jeu, mais une vérité certaine connue de tous les négociants.\par
L’âme qui vivifie ces billets ou cet argent en papier, est la solvabilité connue du tireur, qui ne roule absolument que sur la valeur courante de ce qu’il possède, soit meubles ou immeubles : or, l’un et l’autre étant écrasés à tous moments par des coups inopinés, non seulement cette monnaie qui faisait vingt et trente fois plus de commerce que l’argent, est mise au billon, mais même toutes les fabriques en sont anéanties, et il faut de ce métal en personne partout, ou bien c’est une nécessité de périr.\par
On peut bien supposer qu’une si grande survenue de fonctions à une chose qui était auparavant presque entièrement inutile, au moins pour la subsistance honnête et nécessaire de la vie, la met en état de se bien faire valoir, et de ne passer entre les mains de qui que ce soit qu’à bonnes enseignes.\par
C’est aussi à quoi l’argent ne manque pas : au lieu que précédemment il ne trouvait personne qui voulût de son service pour plus que pour ses dépens, non seulement il se fait doubler et tripler ses appointements antérieurs, mais même il veut souvent avoir tout le vaillant d’un homme pour entrer chez lui, bien que quelque temps auparavant il se fût cru très redevable de n’avoir que le simple couvert. Or, cette hausse de gages ou intérêts effroyable est la mort et la ruine d’un État, comme elle le serait d’un particulier, n’y ayant nulle différence, quoique nul homme n’y fasse réflexion.\par
Dans les temps d’opulence, il n’était pas sitôt admis en un lieu, que l’on songeait à l’en déloger ; et il était accoutumé, sans s’étonner, à faire quelquefois plus de cent logis dans une même journée, c’est-à-dire cent fois autant de consommation, et par conséquent de revenu, qu’il en produit dans les temps de misère ; sans parler de ses consorts, savoir le papier et le crédit qui en faisaient vingt fois plus que lui, et qui perdent leur vertu du moment qu’il n’y a plus que l’argent qui en ait ; cependant on a l’aveuglement de publier, contre vérité, qu’il n’y a plus d’espèces.\par
Mais dans l’autre situation, il marche à pas de tortue, et la grande survenue de besogne ne sert qu’à le faire aller plus lentement, devenant paralytique partout où il met le pied ; et il faut des machines épouvantables pour l’en déloger, et encore le plus souvent c’est peine et temps perdus.\par
Mille raisons, dont la moindre autrefois aurait été suffisante pour le faire mettre dehors, sont inutiles le plus souvent pour en obtenir le moindre mouvement ; ce qui ne diffère guère d’une banqueroute générale, mettant tout le monde sur le qui-vive, et faisant prendre à toute heure des lettres d’atermoiement.\par
La vie, que le possesseur croit uniquement attachée à sa garde, fait qu’il en défend la possession, comme il en userait à l’égard de sa propre personne si on venait pour l’assassiner. On se retranche à moins dépenser, qui est un rengrégement de mal qui augmente la misère, et par conséquent la rareté de l’argent.\par
On sait qu’alors les plus grandes violences, et même les crimes, sont excusables ; on en use ainsi, et on croit le pouvoir faire innocemment dans ces temps fâcheux à l’égard de la garde de l’argent.\par
Dans un pays opulent par lui-même, il ne doit pas naturellement former plus de la millième partie des facultés, en lui supposant toute sa valeur ordinaire ; mais dans ce déconcertement, lui seul est et s’appelle richesse ; tout le reste n’est que de la poussière.\par
Il y avait peu de fausses divinités dans l’antiquité auxquelles on sacrifiât généralement toutes choses : on immolait aux unes des bêtes, aux autres des fruits et des liqueurs, et dans le plus grand aveuglement, la vie de quelque malheureux. Mais l’argent en use bien plus tyranniquement ; on brûle continuellement à son autel non toutes ces denrées, dont il est en quelque manière rebuté, mais il lui faut des immeubles si l’on veut captiver sa bienveillance, encore faut-il que ce soient les plus spécieux, les plus grandes terres : les dignités, autrefois du plus grand prix, et même les contrées entières, ne lui sont pas trop bonnes ou plutôt ne font qu’aiguiser son appétit ; et pour les victimes d’hommes, jamais tous les fléaux, dans leur plus forte union et leur plus grande colère, n’en détruisirent un si grand nombre que cette idole d’argent s’en fait immoler. Car premièrement ces marques de l’ire du ciel n’ont qu’une courte durée, après quoi un pays désolé se rétablit même quelquefois mieux que jamais ; mais ce dieu dévorant ne s’attache jamais à son objet, comme le feu naturel, que pour le dévorer. Les premières matières redoublent son ardeur pour consumer le reste, et l’anéantissement de biens effroyables qu’il cause, incommodant les plus riches, fait que la quote-part de ce déchet sur les misérables est la suppression de leur nécessaire, dont qui que ce soit ne peut être privé sans le dépérissement entier du sujet, ce qui n’est que trop connu. Après cela les hommes ne sont-ils pas, sans comparaison, comme les bêtes et surtout les chevaux ? Qui ferait travailler continuellement un cheval sans lui donner que le quart de sa nourriture nécessaire, n’en verrait-il pas incontinent la fin ? Or, des hommes à qui il faut une peine continuelle, et suer sang et eau pour subsister, sans autre aliment que du pain et de l’eau, au milieu d’un pays d’abondance, peuvent-ils espérer une longue vie, ou plutôt ne périssent-ils pas tous à la moitié de leur course, sans compter ceux que la misère de leurs parents empêche de sortir de l’enfance, étant comme étouffés au berceau, ce dieu ou ce vautour d’argent les dévorant à tout âge et en toutes sortes d’états ?\par
Voilà la description, la cause et les effets de la misère, lorsqu’elle paraît dans un pays qui devrait être riche par la destination de la nature, et qui le serait même si on lui laissait achever son ouvrage, comme elle l’a commencé ; elle est même si bienfaisante, qu’elle est toujours disposée à réparer le désordre au moindre signe qu’on lui fera, mais ce ne peut être qu’en quittant le faux culte de ce métal son ennemi, ou pour mieux dire celui des hommes.\par
Il ne faut pas que l’esclave devienne le maître, ou plutôt le tyran et l’idole ; c’est à la nature qui produit ses faveurs à les départir, autrement elle prend son congé, ce qui ne diffère point d’un bouleversement général ; et les particuliers qui croient faire leur fortune, et la font même apparemment dans une déroute si universelle, en péchant comme l’on dit en eau trouble, ne montent si haut qu’afin que leur chute les blesse davantage.\par
La nature qui les voit courir devant elle, sans faire semblant de les apercevoir, ne les oubliera pas à la fin dans sa vengeance ; le crédit qu’elle leur fait leur sera cher vendu, puisqu’ils ne seront jamais que des misérables, lorsqu’ils croiront pouvoir seuls être riches.\par
L’intérêt que tous les hommes ont en particulier de combattre une pareille situation, et d’en sortir lorsqu’ils s’y trouvent malheureusement enveloppés, est augmenté dans les princes à proportion de leur élévation, qui n’est absolument autre au sou la livre que celle de tous leurs sujets en général ; et c’est ce que l’on fera voir dans le chapitre suivant.
\section[{Chapitre VI.}]{Chapitre VI.}
\noindent Les princes dans les États desquels se passe ce dérangement, ou plutôt ce bouleversement de la nature de l’argent, qui met tout en combustion et en quelque manière rez-pierre rez-terre, sont constamment les plus malheureux.\par
Comme cela ne se peut opérer et ne s’opère même que par des intérêts indirects, qui n’ont pas un droit naturel à la chose, les sujets se mettent peu en peine de ce que doit coûter à tout un corps d’État un bien qu’ils n’auraient pu jamais acquérir d’une façon légitime.\par
Mais il s’en faut beaucoup que l’on doive faire le même raisonnement des souverains : non seulement ils n’ont pas besoin de crime pour acquérir et subsister, leur maintien étant de droit divin et humain, mais même toutes les pertes que les particuliers souffrent, ou plutôt tout le corps d’État, pour former par une infinité d’anéantissements ces précis criminels, retombent sur leur propre personne.\par
Ils sont les premiers propriétaires et les possesseurs éminents, en termes de philosophie, de tous les fonds, et sont riches ou pauvres à proportion qu’ils sont en valeur.\par
C’est de la part qu’on leur fait des fruits qu’ils soutiennent leur grandeur et entretiennent leurs armées, et non pas de la destruction de toutes ces choses, comme l’on a malheureusement pratiqué en quelques contrées.\par
Ainsi un écu, à leur égard, ne vaut jamais qu’autant qu’eux ou ceux qui sont à leur solde s’en peuvent procurer de pain, de vin ou d’autres denrées ; et sans les incommodités du transport, ils seraient tout disposés à donner la préférence à ces choses en essence, pour lesquelles seules ils veulent avoir de l’argent, et savent bien pareillement que leurs sujets ne leur en peuvent donner que par le débit de ces mêmes denrées.\par
Le crime donc et les anéantissements de fruits ne leur étant pas nécessaires pour recevoir de l’argent, ni n’en voulant point faire non plus un usage criminel, il s’en faut beaucoup que ce métal soit ou doive être une idole chez eux, comme il est chez des sujets qui n’ont point d’autre ressource que le crime pour finir leur misère, et à qui encore une fois les horreurs générales sont fort indifférentes quand elles font leur fortune particulière.\par
€e n’est donc ni leur intérêt ni leur volonté que les terres demeurent en friche, les fruits les plus précieux à l’abandon, par l’avilissement où ils se trouvent dans certaines contrées, pendant que d’autres en manquent tout à fait, qui souffrent le même sort à l’égard d’autres denrées singulières, quelles eussent données en contre-échange, par une compensation réciproque qui, de deux extrémités très défectueuses, aurait formé deux situations parfaites, s’il n’y avait eu que les intérêts des particuliers et ceux du prince à ménager.\par
Mais les sujets qui ne peuvent vivre et s’enrichir que de précis, mettent tous ces biens dans un alambic et en font évaporer en fumée dix-neuf parts sur vingt ; et de cette vingtième, en donnant une partie au prince, ils croient non seulement s’être bien acquittés de leur devoir, mais même que ce sont eux qui font subsister son État, et que sans ce précieux secours, tout serait perdu.\par
On se met un bandeau devant les yeux, pour supposer que la garantie ou le ministère personnel de gens qui n’ont rien absolument d’eux-mêmes, est d’une nécessité indispensable pour faire payer ceux qui possèdent tout, et que ce cruel service ne peut jamais être acheté à un assez haut prix.\par
Et ce qui renchérit encore par là-dessus, et fait en quelque manière honte aux lumières de l’homme, est que, n’étant pas douteux que le prince ne veuille avoir de l’argent que pour avoir des denrées, comme pareillement que ses sujets ne les lui peuvent fournir que par la vente des produits dont ils sont propriétaires, ainsi que l’on a dit tant de fois, on souffre néanmoins tranquillement, et on regarde même avec admiration des moyens lesquels, pour parvenir à ce but, abîment vingt fois autant de toutes choses qu’ils en mettent à profit.\par
On regarde comme une vision creuse ou une fable ce que l’on vient de marquer, savoir qu’un souverain n’a du bien qu’autant que ses sujets en possèdent, et qu’ils ne lui feront jamais part de ce qui n’est point en leurs mains, ou n’est ni consommé ni vendu, étant défendu par la nature de donner ce que l’on n’a point, ou qui est anéanti, comme il arrive à tout ce qui ne peut être vendu, ou qui l’est avec perte du marchand.\par
S’ils ont beaucoup de blés par la culture de quantité de terres, rendue possible par un prix de grains qui supporte les charges et les frais, le prince assurément aura de quoi donner du pain à quantité de troupes. De même du vin, des habits, de la viande, des chevaux, des cordages, des bois de charpente, des métaux dont on construit toutes sortes d’armes, et enfin toutes les espèces dont on lève et entretient toutes les armées de terre et de mer, lesquelles choses ne reçoivent leur naissance, leurs bornes et leur durée, que du degré de pouvoir plus ou moins, que le pays a non seulement de les produire, mais de les consommer, qui est seul ce qui fait tirer ces biens des entrailles de la terre, parce qu’il faut que les particuliers en absorbent pour leur usage dix fois plus que le souverain, si l’on veut que cette redevance soit de durée ; et si le prince a besoin d’une quantité de denrées, comme des matières dont on construit les vaisseaux et armements de mer, dans un degré qui excède la proportion de consommation dans ses sujets, en sorte qu’il lui en faille davantage qu’une partie de leur usage ordinaire, cela se remplace par le change qu’il fait et peut faire d’autres choses qu’il reçoit en plus haut degré qu’il ne lui en faut ; et il prendra, par exemple, toute la fonte d’un ouvrier qui ne travaillera que pour le prince seul, parce que lui seul lui paiera toute sa dépense à l’aide de ce qu’il a d’excédent d’autres redevances qu’il ne peut consommer : tout de même qu’un particulier qui n’a que du blé, comme c’est en très grande quantité, échange le surplus de son nécessaire contre tout le reste de ses besoins ou de ses désirs.\par
Car enfin, quelque justice qu’il y ait dans les tributs dus aux princes, il serait impossible aux peuples de s’en acquitter s’ils ne trouvaient leur subsistance dans les moyens que l’on prend ou qu’on leur fait prendre pour y satisfaire ; et il faut même que cette subsistance précède toutes sortes de paiements, par une justice qu’on doit jusqu’aux bêtes, et dont Dieu fait mention dans la première loi qu’il donna aux hommes.\par
Le maître d’un cheval de voiture lui donne sa nourriture, avant que de prendre le profit qu’il tire de son service, ou bien il le perdra absolument ; ce qui ne manquera pas de le ruiner, sans que personne le plaigne ni doute de la cause de sa désolation, qu’il s’est attirée par son imprudence.\par
Qu’un prince en use de même lorsqu’il est maître d’un pays naturellement fécond, et que le peuple est laborieux, et rien ne lui manquera.\par
La supposition ou la pratique du contraire sont un outrage à la religion, à l’humanité, à la justice, à la politique, et à la raison la plus grossière.\par
Pourquoi donc, dans une contrée naturellement très fertile, voit-on un souverain qui n’a pas des armées aussi nombreuses et aussi bien entretenues qu’il serait à souhaiter, et que ses besoins sembleraient exiger ? C’est parce qu’il n’a pas assez de pain, de vin, de viande, et enfin de tout le reste à départir.\par
Et pourquoi ce défaut ? C’est que les terres de son royaume, qui produiraient amplement toutes ces denrées, sont en friche et très mal cultivées.\par
Et pourquoi enfin ce désordre ? C’est parce qu’on a lié la bouche, non seulement aux bêtes, mais aux hommes, contre le précepte divin, pendant qu’ils travaillaient dans le champ.\par
On leur a refusé leur vie et leur subsistance, et ils ont abandonné le travail. Qui a fait ce beau ménage ? Ce sont les sacrificateurs et les prêtres de cette idole, l’argent.\par
Il n’a qu’une concurrence à l’égard du prince avec les autres denrées, et il ne doit être que leur esclave ou leur porteur de procuration pour la garantie de la tradition future de l’échange, tant envers le prince qu’entre les particuliers, qui n’ont qu’un seul et même intérêt ; mais il s’en faut beaucoup que les prêtres de cette idole le regardent de même œil.\par
Toutes ces sources d’armées et de flottes, ou plutôt de maintien de l’opulence publique, ne sont que des victimes qu’il faut brûler nuit et jour à cet autel ; et non content des fruits, il faut que les fonds prennent une semblable route et soient immolés à ce dieu, comme il n’est que trop public en quelques contrées de l’Europe.\par
Il y a donc de l’argent {\itshape bienfaisant}, soumis aux ordres de sa vocation dans le monde et toujours prêt à rendre service au commerce, sans qu’il soit besoin de lui faire la moindre violence, pourvu que l’on ne le dérange pas, et que devant être à la suite de la consommation, ainsi qu’un valet à celle de son maître, on ne le veuille pas faire passer devant, ou plutôt en former un vautour qui la dévore complètement.\par
Tant qu’il demeure dans ces bornes, non seulement il ne la déconcerte pas, mais même la fomente et la fait fleurir ; et, bien loin de refuser son secours, et que l’on puisse jamais en avoir disette, la célérité avec laquelle il marche fait qu’on le peut voir en un moment dans cent lieux différents ; et quand cela ne suffit pas, il souffre tranquillement la concurrence, et même la préférence que l’on donne à un morceau de papier ou de parchemin sur lui, n’y ayant aussi presque aucunes denrées qui ne le remplacent avec équivalence par le prix soutenu de leur valeur.\par
Mais il y a de l’argent {\itshape criminel}, parce qu’il a voulu être un dieu au lieu d’un esclave, qui, après avoir déclaré la guerre aux particuliers ou plutôt à tout le genre humain, s’adresse enfin au Trône, et ne lui fait pas plus de quartier qu’à tout le reste, en lui refusant une partie des besoins dont il met tous les jours une quantité effroyable en poudre, étant même impossible que les choses soient autrement.\par
Et le cruel est que, comme l’ignorance a fait admettre et souffrir sa tyrannie, elle redouble ses efforts pour empêcher toute sorte de fin à ces désordres, et fait chercher dans le redoublement du mal le remède des maux qu’il a causés.\par
Cet argent criminel, ou plutôt ses fauteurs, ont la hardiesse et l’effronterie d’alléguer, lorsque la désolation publique est dans son dernier période, qui est leur unique ouvrage, que c’est qu’il n’y a plus d’espèces et qu’elles ont passé dans les pays étrangers.\par
Mais c’est justement le contraire, et il y en a trop si l’on n’en corrompait pas l’usage par les manières décrites dans ce Mémoire ; lequel étant rétabli comme cela se peut en un moment, on ne verra rien d’approchant de ce qui paraît aujourd’hui. Si quelques particuliers ne sont pas si magnifiques, tout le reste ne sera pas si misérable ; et par une juste compensation, on sera vingt fois plus riche en général, et par conséquent le prince, que l’on ne l’est dans la situation opposée qui subsiste, et que l’on combat.\par
De croire que le remède du mal puisse jamais naître des auteurs mêmes, c’est s’abuser grossièrement. La corruption du cœur ne permettra jamais que l’on balance dans le choix entre une misère innocente et une opulence criminelle, surtout lorsque l’une et l’autre se trouvent en compromis en un si haut degré, et que ce genre de richesse est bien éloigné de craindre aucune persécution de la part de personnes qui soient à appréhender. La préférence est donnée au dernier tous les jours à moindre prix ; ainsi l’on peut supposer ce qu’on en peut attendre en pareille occasion.\par
La perfection et le comble sont les raisons et les discours qui se répandent lorsqu’il est question de parler du remède ; on ne touche de rien moins que d’un renversement entier d’État, quand on parle de voir s’il n’y aurait pas moyen de faire cesser le plus grand bouleversement qui fut jamais.\par
Et l’on n’a point de honte de soutenir, par un redoublement d’outrage à la raison, que l’on ne peut discontinuer de laisser les terres du milieu d’un royaume en friche, et les fruits excrus au néant, pendant que les peuples voisins en manquent tout à fait, jusqu’à ce qu’une guerre étrangère, qui se passe à deux cents lieues de ces contrées, soit finie ; bien qu’au contraire son sort, bon ou mauvais, dépende absolument des mesures justes ou mal concertées qu’on prend au-dedans d’un État : or, il est aisé de juger sur ce compte quel succès on peut attendre de dispositions telles qu’on les vient de décrire, quand par malheur elles se rencontrent, et que les ennemis en prennent de toutes contraires, qui sont celles de toutes les nations du monde.\par
Outre que toutes les choses que l’on anéantit sont seules le soutien de la guerre, et qu’elles y ont constamment la principale part par une ample fourniture aux décisions de la fortune, la parfaite connaissance que des ennemis peuvent avoir que cette unique ressource des armées sera plus ou moins de durée chez les nations opposées, par rapport à la situation où ils se trouvent à l’égard de ces mêmes provisions, est uniquement ce qui les porte à entendre à la paix, qui doit être l’objet de toutes les guerres, quelque saintes et quelque justes qu’elles soient.\par
Il ne faut qu’un moment pour changer tout à coup cette malheureuse situation, décrite dans le Mémoire des mauvais effets de l’argent criminel, en un état très heureux.\par
Il n’est pas question d’agir, il est nécessaire seulement de cesser d’agir avec une très grande violence que l’on fait à la nature, qui tend toujours à la liberté et à la perfection.\par
Comme il n’y a que de la surprise à l’égard de ces désordres, tant dans les princes que leurs ministres, qui ont toujours bien été intentionnés, leur simple changement de volonté sera la fin de tout le mal, et le commencement d’une opulence générale, et de celle du souverain par conséquent.\par
Ils n’ont qu’à souffrir que chaque particulier soit personnellement le fermier du prince à son égard, et que le prix de ce bail n’excède pas la valeur de la ferme ; ce qui arrivant, et ce qui n’est pas inconnu, un fermier ne peut que prendre la fuite et laisser la terre en friche, par où le prince perd pour le moins autant que lui.\par
Bien loin qu’après qu’un malheureux alambic a fait évaporer une quantité effroyable de biens et de denrées pour former ce fatal précis au maître, l’impôt perdu par le prince sur les biens anéantis soit remplacé par ceux qui ont causé ce dépérissement, ce qui ne serait pas même à leur pouvoir ; c’est justement le contraire, puisqu’ils ne paient pas même leur quote-part d’une juste contribution par rapport à ce qui reste de biens en essence en leurs mains, par cette malheureuse coutume, que la quantité de facultés est une sauvegarde contre les impôts dus au prince, qui ne doivent être exigés ou payés que par ceux qui s’en trouvent et en doivent être accablés.\par
Ainsi l’on voit la perte effroyable qui résulte à un souverain de cette conduite ; mais ce n’est pas tout, ou plutôt ce n’est que la moindre partie du désastre qu’il souffre ; et pour le vérifier, il faut rappeler ce qu’on a dit ci-devant, savoir qu’un écu chez un pauvre ou un très menu commerçant fait cent fois plus d’effet, ou plutôt de revenu que chez un riche, par le renouvellement continuel et journalier que souffre cette modique somme chez l’un ; ce qui n’arrive pas à l’égard de l’autre, dans les coffres duquel des quantités bien plus grandes d’argent demeurent des mois et des années entières oiseuses, et par conséquent inutiles, soit par corruption de cœur aveuglé par l’avarice, ou dans l’attente d’un marché plus considérable.\par
Or, sur cette garde, le roi et le corps de l’État ne retirent aucune utilité, et ce sont autant de larcins que l’on fait à l’un et à l’autre.\par
Mais cette somme, comme de mille écus, départie à mille menues gens, aurait fait cent mille mains dans un moindre temps qu’elle n’a résidé dans les coffres de ce riche, ce qui n’aurait pu arriver qu’en faisant par conséquent pour cent mille écus de consommation ; le prince en aurait eu et reçu la dixième partie pour sa part, c’est-à-dire qu’il eût reçu la valeur de dix mille écus sur une somme à l’égard de laquelle il ne reçoit pas un denier par le dérangement de l’usage que l’on en fait, et que l’on augmente et fomente tous les jours, en lui persuadant faussement que c’est pour son utilité particulière que l’on ruine également lui et ses peuples.\par
Si donc les riches entendaient leurs intérêts, ils déchargeraient entièrement les misérables de leurs impôts, ce qui en formerait sur-le-champ autant de gens opulents ; et ce qui ne se pouvant sans un grand surcroît de consommation, laquelle se répand sur toute la masse d’un État, dédommagerait au triple les riches de leurs premières avances, étant la même chose qu’un maître qui prête du grain à son fermier pour ensemencer sa terre, sans quoi il en perdrait la récolte ; et la pratique du contraire par le passé coûte de compte fait à ces puissances six fois ce qu’ils ont prétendu gagner, en renvoyant tous les impôts sur les misérables.\par
Ainsi l’on voit, par tout ce Mémoire, de quelle force on donne le change au prince lorsqu’on lui fait concevoir que son intérêt consiste à entretenir des médiateurs entre son peuple et lui pour le paiement des impôts, qui mettent tout dans l’alambic pour former ces précis criminels ; mais, comme c’est par une des plus hautes violences que la nature ait jamais reçues, le remède est d’autant plus aisé dans les contrées où ce déconcertement se rencontre, qu’il n’est pas question, encore une fois, d’agir pour procurer une très grande richesse, mais de cesser seulement d’agir, ce qui n’exige qu’un instant.\par
Et aussitôt cette même nature mise en liberté, rentrant dans tous ses droits, rétablira le commerce et la proportion de prix entre toutes les denrées, ce qui leur faisant s’entredonner naissance et continuellement par une vicissitude perpétuelle, il s’en formera une masse générale d’opulence, où chacun puisera à proportion de son travail ou de son domaine ; et ce qui allant toujours en augmentant, jusqu’à ce que la terre d’où partent toutes ces sources ne puisse plus fournir, on peut supposer quelle abondance de richesses on verrait si toutes choses, tant le terroir que le reste, étaient autant en valeur qu’il serait possible à la nature de les y mettre, puisqu’il n’y a point de contrée si inculte et si stérile qu’on le suppose, qu’il ne soit aisé de rendre très abondante, si le prix des fruits que l’on y recueillerait ne manquait point de garantie par rapport aux frais qu’il aurait fallu faire pour y parvenir. Et cela n’arriverait même jamais si une infinité d’hommes qui, par indigence, ne consomment presque rien, soit dans leur nourriture, soit dans leurs habits, étaient mis en état, comme cela serait facile, de se pouvoir fournir amplement de toutes leurs nécessités et même du superflu.\par
On peut dire même que l’on a des exemples en Europe de ce secours mutuel que se sont donné tant ces hommes dénués de tout, que ces terres mal partagées par la nature : leur alliance est un peu et même fort difficile à contracter ; les commencements en sont très rebutants ; il faut que le travail et la frugalité marchent longtemps du même pied à un très haut degré ; mais enfin l’un et l’autre viennent à bout de tout, et surpassent même assez souvent en richesse des contrées et des peuples beaucoup plus favorisés du ciel : les barbets vivent commodément dans les rochers des Alpes ; et l’Espagne manque presque de tout dans un pays très fertile et très fécond, qui est le plus souvent inculte en quantité d’endroits.\par
Mais, comme c’est là un chef-d’œuvre de la nature, il faut qu’elle agisse dans toute sa perfection, c’est-à-dire dans toute sa liberté, pour produire de pareils ouvrages : le degré de dérogeance que l’on apporte à l’un, savoir à cette liberté, est aussitôt puni d’une pareille diminution dans l’autre.\par
Ainsi l’on peut voir, pour finir cet ouvrage, quelle effroyable méprise est de se défier de la libéralité ou de la prudence d’une déesse qui sait procurer des richesses immenses, dans les pays les plus stériles, aux hommes qui veulent bien s’en rapporter à elle pour la fructification de leur travail, pendant qu’elle laisse dans la dernière misère ceux qu’elle avait beaucoup mieux partagés d’abord, mais qui ne lui marquent leur reconnaissance qu’en la voulant réduire dans l’esclavage ; de quoi ils ne viennent jamais à bout, que pour se rendre eux-mêmes plus malheureux que des esclaves.\par
Cependant elle est si bienfaisante et elle aime si fort les hommes, qu’au premier repentir elle oublie toutes les indignités passées, et les comble par conséquent en un moment de toutes les faveurs, ainsi que l’on a dit.\par
Il n’est question que de lui donner la liberté, ce qui n’exige pas un plus long temps que dans les affranchissements d’esclaves de l’ancienne Rome, c’est-à-dire un moment, et aussitôt toutes choses reprenant leur proportion de prix, ce qui est absolument nécessaire pour la consommation, c’est-à-dire l’opulence générale, il en résultera une richesse immense.\par
Le laboureur et le vigneron ne cultiveront plus la terre à perte, et ne seront point par là obligés de la laisser en friche ; et comme ils sont l’un et l’autre les nourriciers de tout le genre humain, ils ne se verront point obligés de déclarer à la plupart des hommes, comme ils font présentement en quelques contrées de l’Europe, qu’il n’y a plus de pain et de vin pour eux, parce qu’ils n’ont pas voulu ou pu payer les frais ordinaires, ou survenus par accident, aux commissionnaires ; ce qu’il ne faut jamais attendre de leur libéralité ou de leur prudence, quand ils devraient tous mourir de faim l’un après l’autre. Ce qui prouve que tout impôt singulier sur une seule denrée est mortel à tout l’État, parce que, tout y étant solidaire, les autres au lieu de partager le fardeau le lui laissent tout entier, ce qui les ruine toutes par contrecoup, et manque d’intelligence ; au lieu que les impôts personnels, par rapport aux facultés générales de chaque sujet, se répandent et se partagent sur toute la masse, et font l’impartition de la charge au sou la livre sur chaque denrée, qui est absolument nécessaire pour le commun maintien, et qu’il ne faut jamais attendre de la prudence et de la raison des particuliers qui ne cherchent qu’à se détruire, surtout dans une contrée où la désolation générale est en possession de former les plus grandes fortunes.\par
L’argent alors, par cette survenue innombrable de concurrents, qui seront les denrées mêmes rétablies dans leurs justes valeurs, sera rembarré dans ses bornes naturelles ; de tyran et de maître, il ne sera plus qu’un esclave dont le service même se trouvera le plus souvent inutile ; et dans cette hausse prodigieuse de mouvement qui lui surviendrait à la suite de la consommation, une course ou deux, ou davantage, chez le prince, suivies sur-le-champ d’un retour aussi prompt, seraient imperceptibles, et ne laisseraient pas d’être un doublement de tribut qui, bien loin d’incommoder les peuples, ne serait que l’effet de leur crue d’opulence, toutes sortes de redevances tirant leur degré d’excès ou de médiocrité, non de leur quotité singulière et absolue, mais des facultés de ceux qui paient ; et ces fréquentes visions d’argent, auparavant caché ou paralytique, feraient dire qu’il y en aurait beaucoup à ces mêmes ignorants qui publient que la misère moderne vient du manque d’espèces.\par
Comme tout ceci ne se peut, aux pays où ce déconcertement se rencontre, que par une cessation de manières pour lesquelles, quoique très ruineuses, on croyait mériter de fort grands établissements, on n’aura aucune peine à comprendre que, bien loin que de pareilles mesures fussent un sujet de mérite et l’effet d’un grand savoir, on leur est au contraire uniquement redevable ; tant le prince que ses peuples, d’une extrême misère, laquelle cessera aussitôt que la cause (qui ne pend qu’à un filet du côté de la nature) sera ôtée.\par
Mais il s’en faut beaucoup que ce soit la même chose du côté de la volonté ou plutôt du cœur, qu’un mort ressuscité, au témoignage de l’Écriture sainte, ne changerait pas lorsqu’il est une fois corrompu.\par
Voilà le principe pitoyable de l’allégation, que l’on ne peut sans risquer un bouleversement d’État cesser de ruiner meubles et immeubles depuis le matin jusqu’au soir, pour ne reconnaître d’autre Dieu ni d’autre bien que l’argent, qui n’en doit pas faire la millième partie dans un royaume rempli de denrées propres à tous les besoins de la vie, et qui n’est principe de richesses qu’au Pérou, parce qu’il y est uniquement le fruit du pays, qui bien loin par là d’être digne d’envie, ne nourrit ses habitants que très misérablement au milieu des piles de ce métal, pendant que des contrées qui le connaissent à peine ne manquent d’aucun de leurs besoins ; pourvu, s’entend, que la liberté ou plutôt la nature fasse la dispensation de ses présents, puisque la production a été son ouvrage.\par
Car enfin, pour faire un précis salutaire de ces Mémoires, dont l’objet a été de combattre les précis criminels, on peut dire avec certitude que l’opulence générale, tant à l’égard du prince que de ses peuples dans un pays abondant, est un composé général et perpétuel où chaque particulier doit travailler à tout moment, par un apport et un remport à la masse toujours pareil, le péril étant égal de quelque côté qu’arrivé la diminution ; ce qui étant observé exactement, il en résulte une composition parfaite où l’on trouve tout, parce qu’on y apporte tout. Mais, du moment que quelqu’un veut déroger à cette règle de la justice, pour prendre plus ou apporter moins que sa part, la défiance alors arrivant, ainsi que le déconcertement de proportion des prix, la masse se corrompt, et les particuliers, qui n’y trouvent plus leur subsistance, sont obligés d’y pourvoir par des mesures singulières, qui sont très désolantes et presque toujours criminelles, ou plutôt l’un et l’autre à la fois.\par
Chacun périt, ainsi qu’on a marqué, par l’excès d’une denrée et la disette d’une autre, ce qui jette tous les sujets réciproquement dans la misère, pendant que la compensation mutuelle de ces extrémités les avait rendus très heureux.\par
Il en arrive comme si quelque prince abusant de son autorité, ce qui n’est pas inconnu dans les persécutions de l’Église naissante ; comme si, dis-je, un souverain, pour tourmenter et faire périr divers sujets d’une façon grotesque, en faisait enchaîner dix ou douze à cent pas les uns des autres, et que l’un étant tout nu, quoiqu’il fît grand froid, il eût une quantité effroyable de viande et de pain auprès de lui, et plus dix fois qu’il n’en pourrait consommer avant que de périr, ce qui ne serait pas fort éloigné, parce qu’il manquerait de tout le reste, et surtout de liqueurs dont il n’aurait pas une goutte à sa portée ; — qu’un autre, pendant enchaîné dans l’éloignement marqué, aurait une vingtaine d’habits autour de lui, et plus trois fois qu’il n’en pourrait user en plusieurs années, sans aucuns aliments pour soutenir sa vie, et défense de lui en fournir ; — tandis qu’à pareille distance un troisième enfin, et ainsi de suite, se trouverait environné, de plusieurs muids de liqueurs, mais sans nuls habits ni aliments : — il serait vrai de dire après leur dépérissement qui serait immanquable, si la violence se continuait jusqu’au bout, qu’ils seraient tous morts de faim, de froid et de soif, manque de liqueurs, de pain, de viande et d’habits ; cependant il serait très certain que, tout pris en général, non seulement ils ne manquaient ni d’aliments ni d’habits, mais que même ils pouvaient, sans la force majeure, être bien habillés et faire bonne chère.\par
Et si quelqu’un au fort de leur mal, avant leur dépérissement entier, implorait la clémence du prince pour les faire déchaîner, ce qui se pourrait en un instant, et ce qui ne manquerait pas sur-le-champ de les rendre heureux par un échange réciproque, à quoi ils ne tarderaient pas un moment, le prince repartait, ou ceux qui le feraient parler, que le temps n’est pas propre, et que cela pourrait porter un grand préjudice ; qu’en tout cas il faudrait attendre qu’un démêlé qu’il a à deux cents lieues de la contrée où ces malheureux seraient en souffrance, fût terminé ; ne jugerait-on pas aussitôt que l’on voudrait ajouter l’injure et la raillerie à la persécution ?\par
Il peut y avoir des pays sur la terre où il se passe non pas à peu près, mais à un plus haut degré, des exemples d’une pareille conduite, et en faveur desquels on allègue de pareils raisonnements pour son maintien, ou pour tarder le remède lorsqu’on parle de l’apporter, comme cela se peut pareillement en un moment.\par
Mais n’y ayant que de la surprise, et nulle mauvaise volonté dans les maîtres du théâtre où une pareille scène se peut représenter aujourd’hui, on en doit avec certitude espérer la cessation, qui sera sur-le-champ un triplement d’opulence publique, dont il est autant impossible que le prince n’ait pas sa part, qu’il n’est pas présumable que l’état contraire et désolant qui subsiste n’apporte pas une diminution effroyable dans ses revenus, tant présents que possibles.\par
Et dire que cela ne se peut pas en deux heures de travail et quinze jours d’exécution, est proférer la même extravagance que l’on vient de mettre dans la bouche des auteurs de la violence que l’on a ci-dessus décrite ou supposée.\par
Tout se réduit enfin dans quatre mots souvent répétés, savoir que les peuples ne peuvent être riches ni payer le prince que par la vente de leurs denrées. Or, si l’on peut en deux heures de travail, ou plutôt de cessation de travail, doubler cette même vente de denrées, comme on ne saurait le contester sans renoncer à la raison et au sens commun, il est d’une pareille certitude que l’on peut en deux heures doubler leur richesse, et par conséquent les revenus du prince, bien qu’on ait en quelques contrées de l’Europe justement pris le contrepied pour parvenir aux mêmes intentions, ce qui a produit la désolation publique. Ainsi, par le principe naturel que, des causes contraires sortent toujours des effets opposés, les conséquences promises, et marquées dans ce raisonnement ou ces Mémoires, ne peuvent trouver de contredisants parmi les personnes qui voudront bien se laisser convaincre que l’autorité ni la faveur ne dispensent pas qui que ce soit d’obéir aux lois de la justice et de la raison.\par
Au reste, l’on croit s’être acquitté de la preuve, promise à la tête de ces Mémoires, de l’erreur qui règne chez la plupart des hommes, dans l’idée qu’ils se font des richesses, de l’argent et des tributs ; puisque dans le premier, ils cherchent l’opulence dans sa propre destruction, et font cacher l’argent en le voulant avoir contre les lois de la nature ; tout comme pour recevoir les tributs, on se sert de moyens qui mettent les peuples hors de pouvoir y satisfaire, en leur causant une perte de biens dix et vingt fois plus forte que la somme que l’on a intention de recevoir ; ce qui fait que souvent, le dommage étant certain, le paiement de l’impôt qui le cause ne peut pas s’ensuivre, étant devenu impossible, en sorte que la ruine est tout à fait gratuite : or, de nier que la cessation d’une pareille manœuvre soit une richesse immense pour les peuples et pour le prince, c’est ne pas convenir qu’un torrent retenu sur le bord d’une pente par une forte digue coulera en bas dès que la barrière qui servait d’obstacle à son cours sera enlevée.\par


\begin{raggedleft}FIN DE LA DISSERTATION SUR LA NATURE DES RICHESSES.\end{raggedleft}
\chapterclose

 


% at least one empty page at end (for booklet couv)
\ifbooklet
  \pagestyle{empty}
  \clearpage
  % 2 empty pages maybe needed for 4e cover
  \ifnum\modulo{\value{page}}{4}=0 \hbox{}\newpage\hbox{}\newpage\fi
  \ifnum\modulo{\value{page}}{4}=1 \hbox{}\newpage\hbox{}\newpage\fi


  \hbox{}\newpage
  \ifodd\value{page}\hbox{}\newpage\fi
  {\centering\color{rubric}\bfseries\noindent\large
    Hurlus ? Qu’est-ce.\par
    \bigskip
  }
  \noindent Des bouquinistes électroniques, pour du texte libre à participation libre,
  téléchargeable gratuitement sur \href{https://hurlus.fr}{\dotuline{hurlus.fr}}.\par
  \bigskip
  \noindent Cette brochure a été produite par des éditeurs bénévoles.
  Elle n’est pas faîte pour être possédée, mais pour être lue, et puis donnée.
  Que circule le texte !
  En page de garde, on peut ajouter une date, un lieu, un nom ; pour suivre le voyage des idées.
  \par

  Ce texte a été choisi parce qu’une personne l’a aimé,
  ou haï, elle a en tous cas pensé qu’il partipait à la formation de notre présent ;
  sans le souci de plaire, vendre, ou militer pour une cause.
  \par

  L’édition électronique est soigneuse, tant sur la technique
  que sur l’établissement du texte ; mais sans aucune prétention scolaire, au contraire.
  Le but est de s’adresser à tous, sans distinction de science ou de diplôme.
  Au plus direct ! (possible)
  \par

  Cet exemplaire en papier a été tiré sur une imprimante personnelle
   ou une photocopieuse. Tout le monde peut le faire.
  Il suffit de
  télécharger un fichier sur \href{https://hurlus.fr}{\dotuline{hurlus.fr}},
  d’imprimer, et agrafer ; puis de lire et donner.\par

  \bigskip

  \noindent PS : Les hurlus furent aussi des rebelles protestants qui cassaient les statues dans les églises catholiques. En 1566 démarra la révolte des gueux dans le pays de Lille. L’insurrection enflamma la région jusqu’à Anvers où les gueux de mer bloquèrent les bateaux espagnols.
  Ce fut une rare guerre de libération dont naquit un pays toujours libre : les Pays-Bas.
  En plat pays francophone, par contre, restèrent des bandes de huguenots, les hurlus, progressivement réprimés par la très catholique Espagne.
  Cette mémoire d’une défaite est éteinte, rallumons-la. Sortons les livres du culte universitaire, cherchons les idoles de l’époque, pour les briser.
\fi

\ifdev % autotext in dev mode
\fontname\font — \textsc{Les règles du jeu}\par
(\hyperref[utopie]{\underline{Lien}})\par
\noindent \initialiv{A}{lors là}\blindtext\par
\noindent \initialiv{À}{ la bonheur des dames}\blindtext\par
\noindent \initialiv{É}{tonnez-le}\blindtext\par
\noindent \initialiv{Q}{ualitativement}\blindtext\par
\noindent \initialiv{V}{aloriser}\blindtext\par
\Blindtext
\phantomsection
\label{utopie}
\Blinddocument
\fi
\end{document}
