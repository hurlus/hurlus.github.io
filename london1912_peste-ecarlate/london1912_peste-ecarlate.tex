%%%%%%%%%%%%%%%%%%%%%%%%%%%%%%%%%
% LaTeX model https://hurlus.fr %
%%%%%%%%%%%%%%%%%%%%%%%%%%%%%%%%%

% Needed before document class
\RequirePackage{pdftexcmds} % needed for tests expressions
\RequirePackage{fix-cm} % correct units

% Define mode
\def\mode{a4}

\newif\ifaiv % a4
\newif\ifav % a5
\newif\ifbooklet % booklet
\newif\ifcover % cover for booklet

\ifnum \strcmp{\mode}{cover}=0
  \covertrue
\else\ifnum \strcmp{\mode}{booklet}=0
  \booklettrue
\else\ifnum \strcmp{\mode}{a5}=0
  \avtrue
\else
  \aivtrue
\fi\fi\fi

\ifbooklet % do not enclose with {}
  \documentclass[french,twoside]{book} % ,notitlepage
  \usepackage[%
    papersize={105mm, 297mm},
    inner=12mm,
    outer=12mm,
    top=20mm,
    bottom=15mm,
    marginparsep=0pt,
  ]{geometry}
  \usepackage[fontsize=9.5pt]{scrextend} % for Roboto
\else\ifav
  \documentclass[french,twoside]{book} % ,notitlepage
  \usepackage[%
    a5paper,
    inner=25mm,
    outer=15mm,
    top=15mm,
    bottom=15mm,
    marginparsep=0pt,
  ]{geometry}
  \usepackage[fontsize=12pt]{scrextend}
\else% A4 2 cols
  \documentclass[twocolumn]{report}
  \usepackage[%
    a4paper,
    inner=15mm,
    outer=10mm,
    top=25mm,
    bottom=18mm,
    marginparsep=0pt,
  ]{geometry}
  \setlength{\columnsep}{20mm}
  \usepackage[fontsize=9.5pt]{scrextend}
\fi\fi

%%%%%%%%%%%%%%
% Alignments %
%%%%%%%%%%%%%%
% before teinte macros

\setlength{\arrayrulewidth}{0.2pt}
\setlength{\columnseprule}{\arrayrulewidth} % twocol
\setlength{\parskip}{0pt} % 1pt allow better vertical justification
\setlength{\parindent}{1.5em}

%%%%%%%%%%
% Colors %
%%%%%%%%%%
% before Teinte macros

\usepackage[dvipsnames]{xcolor}
\definecolor{rubric}{HTML}{800000} % the tonic 0c71c3
\def\columnseprulecolor{\color{rubric}}
\colorlet{borderline}{rubric!30!} % definecolor need exact code
\definecolor{shadecolor}{gray}{0.95}
\definecolor{bghi}{gray}{0.5}

%%%%%%%%%%%%%%%%%
% Teinte macros %
%%%%%%%%%%%%%%%%%
%%%%%%%%%%%%%%%%%%%%%%%%%%%%%%%%%%%%%%%%%%%%%%%%%%%
% <TEI> generic (LaTeX names generated by Teinte) %
%%%%%%%%%%%%%%%%%%%%%%%%%%%%%%%%%%%%%%%%%%%%%%%%%%%
% This template is inserted in a specific design
% It is XeLaTeX and otf fonts

\makeatletter % <@@@

\usepackage{alphalph} % for alph couter z, aa, ab…
\usepackage{blindtext} % generate text for testing
\usepackage[strict]{changepage} % for modulo 4
\usepackage{contour} % rounding words
\usepackage[nodayofweek]{datetime}
\usepackage{enumitem} % <list>
\usepackage{epigraph} % <epigraph>
\usepackage{etoolbox} % patch commands
\usepackage{fancyvrb}
\usepackage{fancyhdr}
\usepackage{float}
\usepackage{fontspec} % XeLaTeX mandatory for fonts
\usepackage{footnote} % used to capture notes in minipage (ex: quote)
\usepackage{framed} % bordering correct with footnote hack
\usepackage{graphicx}
\usepackage{lettrine} % drop caps
\usepackage{lipsum} % generate text for testing
\usepackage{manyfoot} % for parallel footnote numerotation
\usepackage[framemethod=tikz,]{mdframed} % maybe used for frame with footnotes inside
\usepackage[defaultlines=2,all]{nowidow} % at least 2 lines by par (works well!)
\usepackage{pdftexcmds} % needed for tests expressions
\usepackage{poetry} % <l>, bad for theater
\usepackage{polyglossia} % bug Warning: "Failed to patch part"
\usepackage[%
  indentfirst=false,
  vskip=1em,
  noorphanfirst=true,
  noorphanafter=true,
  leftmargin=\parindent,
  rightmargin=0pt,
]{quoting}
\usepackage{ragged2e}
\usepackage{setspace} % \setstretch for <quote>
\usepackage{tabularx} % <table>
\usepackage[explicit]{titlesec} % wear titles, !NO implicit
\usepackage{tikz} % ornaments
\usepackage{tocloft} % styling tocs
\usepackage[fit]{truncate} % used im runing titles
\usepackage{unicode-math}
\usepackage[normalem]{ulem} % breakable \uline, normalem is absolutely necessary to keep \emph
\usepackage{xcolor} % named colors
\usepackage{xparse} % @ifundefined
\XeTeXdefaultencoding "iso-8859-1" % bad encoding of xstring
\usepackage{xstring} % string tests
\XeTeXdefaultencoding "utf-8"


% TOTEST
% \usepackage{hypcap} % links in caption ?
% \usepackage{marginnote}
% TESTED
% \usepackage{background} % doesn’t work with xetek
% \usepackage{bookmark} % prefers the hyperref hack \phantomsection
% \usepackage[color, leftbars]{changebar} % 2 cols doc, impossible to keep bar left
% \usepackage{DejaVuSans} % override too much, was for for symbols
% \usepackage[utf8x]{inputenc} % inputenc package ignored with utf8 based engines
% \usepackage[sfdefault,medium]{inter} % no small caps
% \usepackage{firamath} % choose firasans instead, firamath unavailable in Ubuntu 21-04
% \usepackage{flushend} % bad for last notes, supposed flush end of columns
% \usepackage[stable]{footmisc} % BAD for complex notes https://texfaq.org/FAQ-ftnsect
% \usepackage{helvet} % not for XeLaTeX
% \usepackage{multicol} % not compatible with too much packages (longtable, framed, memoir…)
% \usepackage[default,oldstyle,scale=0.95]{opensans} % no small caps
% \usepackage{sectsty} % \chapterfont OBSOLETE
% \usepackage{soul} % \ul for underline, OBSOLETE with XeTeX
% \usepackage[breakable]{tcolorbox} % text styling gone, footnote hack not kept with breakable
% \usepackage{verse} % not enough control on indent, poetry is better

\defaultfontfeatures{
  % Mapping=tex-text, % no effect seen
  Scale=MatchLowercase,
  Ligatures={TeX,Common},
}
\newfontfamily\zhfont{Noto Sans CJK SC}

% Metadata inserted by a program, from the TEI source, for title page and runing heads
\title{\textbf{ La peste écarlate }\par
}
\date{1912}
\author{Jack London}
\def\elbibl{Jack London. 1912. \emph{La peste écarlate}}
\def\elabstract{%
 
\labelblock{Préface Hurlue}

 \noindent Jack London doit sa célèbrité aux phénomènes \emph{Croc-Blanc} et \emph{L’Appel de la Forêt}. En arrière plan de ces romans d’aventure, il est l’auteur de belles nouvelles dont en voici une narrant les ravages d’une épidémie et l’effondrement d’un monde : \emph{La Peste Écarlate}.\par
 Que feriez-vous, que penseriez-vous si, notre confort et nos valeurs disparaissaient à jamais ?\par
 London livre dans cette nouvelle, parue en 1912, un exercice de qualité qui procure à cette nouvelle une fraîcheur peu gâtée par le temps.\par
 Ce livre est à mettre en miroir avec le célèbre \emph{Robinson Crusoé} voire avec, le trop récent pour être aux Hurlus, \emph{Sa majesté des Mouches} pour penser l’Individu et la Société.
 \vfill

}
\def\elsource{\href{https://fr.wikisource.org/wiki/La\_Peste\_\%C3 \%A9carlate,\_trad.\_Postif\_et\_Gruyer,\_1924/La\_Peste\_\%C3 \%89carlate/1}{\dotuline{Wikisource}}\footnote{\href{https://fr.wikisource.org/wiki/La\_Peste\_\%C3 \%A9carlate,\_trad.\_Postif\_et\_Gruyer,\_1924/La\_Peste\_\%C3 \%89carlate/1}{\url{https://fr.wikisource.org/wiki/La\_Peste\_\%C3 \%A9carlate,\_trad.\_Postif\_et\_Gruyer,\_1924/La\_Peste\_\%C3 \%89carlate/1}}}}
\def\eltitlepage{%
{\centering\parindent0pt
  {\LARGE\addfontfeature{LetterSpace=25}\bfseries Jack London\par}\bigskip
  {\Large 1912\par}\bigskip
  {\LARGE
\bigskip\textbf{La peste écarlate}\par

  }
}

}

% Default metas
\newcommand{\colorprovide}[2]{\@ifundefinedcolor{#1}{\colorlet{#1}{#2}}{}}
\colorprovide{rubric}{red}
\colorprovide{silver}{lightgray}
\@ifundefined{syms}{\newfontfamily\syms{DejaVu Sans}}{}
\newif\ifdev
\@ifundefined{elbibl}{% No meta defined, maybe dev mode
  \newcommand{\elbibl}{Titre court ?}
  \newcommand{\elbook}{Titre du livre source ?}
  \newcommand{\elabstract}{Résumé\par}
  \newcommand{\elurl}{http://oeuvres.github.io/elbook/2}
  \author{Éric Lœchien}
  \title{Un titre de test assez long pour vérifier le comportement d’une maquette}
  \date{1566}
  \devtrue
}{}
\let\eltitle\@title
\let\elauthor\@author
\let\eldate\@date




% generic typo commands
\newcommand{\astermono}{\medskip\centerline{\color{rubric}\large\selectfont{\syms ✻}}\medskip\par}%
\newcommand{\astertri}{\medskip\par\centerline{\color{rubric}\large\selectfont{\syms ✻\,✻\,✻}}\medskip\par}%
\newcommand{\asterism}{\bigskip\par\noindent\parbox{\linewidth}{\centering\color{rubric}\large{\syms ✻}\\{\syms ✻}\hskip 0.75em{\syms ✻}}\bigskip\par}%

% lists
\newlength{\listmod}
\setlength{\listmod}{\parindent}
\setlist{
  itemindent=!,
  listparindent=\listmod,
  labelsep=0.2\listmod,
  parsep=0pt,
  % topsep=0.2em, % default topsep is best
}
\setlist[itemize]{
  label=—,
  leftmargin=0pt,
  labelindent=1.2em,
  labelwidth=0pt,
}
\setlist[enumerate]{
  label={\arabic*°},
  labelindent=0.8\listmod,
  leftmargin=\listmod,
  labelwidth=0pt,
}
% list for big items
\newlist{decbig}{enumerate}{1}
\setlist[decbig]{
  label={\bf\color{rubric}\arabic*.},
  labelindent=0.8\listmod,
  leftmargin=\listmod,
  labelwidth=0pt,
}
\newlist{listalpha}{enumerate}{1}
\setlist[listalpha]{
  label={\bf\color{rubric}\alph*.},
  leftmargin=0pt,
  labelindent=0.8\listmod,
  labelwidth=0pt,
}
\newcommand{\listhead}[1]{\hspace{-1\listmod}\emph{#1}}

\renewcommand{\hrulefill}{%
  \leavevmode\leaders\hrule height 0.2pt\hfill\kern\z@}

% General typo
\DeclareTextFontCommand{\textlarge}{\large}
\DeclareTextFontCommand{\textsmall}{\small}

% commands, inlines
\newcommand{\anchor}[1]{\Hy@raisedlink{\hypertarget{#1}{}}} % link to top of an anchor (not baseline)
\newcommand\abbr[1]{#1}
\newcommand{\autour}[1]{\tikz[baseline=(X.base)]\node [draw=rubric,thin,rectangle,inner sep=1.5pt, rounded corners=3pt] (X) {\color{rubric}#1};}
\newcommand\corr[1]{#1}
\newcommand{\ed}[1]{ {\color{silver}\sffamily\footnotesize (#1)} } % <milestone ed="1688"/>
\newcommand\expan[1]{#1}
\newcommand\foreign[1]{\emph{#1}}
\newcommand\gap[1]{#1}
\renewcommand{\LettrineFontHook}{\color{rubric}}
\newcommand{\initial}[2]{\lettrine[lines=2, loversize=0.3, lhang=0.3]{#1}{#2}}
\newcommand{\initialiv}[2]{%
  \let\oldLFH\LettrineFontHook
  % \renewcommand{\LettrineFontHook}{\color{rubric}\ttfamily}
  \IfSubStr{QJ’}{#1}{
    \lettrine[lines=4, lhang=0.2, loversize=-0.1, lraise=0.2]{\smash{#1}}{#2}
  }{\IfSubStr{É}{#1}{
    \lettrine[lines=4, lhang=0.2, loversize=-0, lraise=0]{\smash{#1}}{#2}
  }{\IfSubStr{ÀÂ}{#1}{
    \lettrine[lines=4, lhang=0.2, loversize=-0, lraise=0, slope=0.6em]{\smash{#1}}{#2}
  }{\IfSubStr{A}{#1}{
    \lettrine[lines=4, lhang=0.2, loversize=0.2, slope=0.6em]{\smash{#1}}{#2}
  }{\IfSubStr{V}{#1}{
    \lettrine[lines=4, lhang=0.2, loversize=0.2, slope=-0.5em]{\smash{#1}}{#2}
  }{
    \lettrine[lines=4, lhang=0.2, loversize=0.2]{\smash{#1}}{#2}
  }}}}}
  \let\LettrineFontHook\oldLFH
}
\newcommand{\labelchar}[1]{\textbf{\color{rubric} #1}}
\newcommand{\milestone}[1]{\autour{\footnotesize\color{rubric} #1}} % <milestone n="4"/>
\newcommand\name[1]{#1}
\newcommand\orig[1]{#1}
\newcommand\orgName[1]{#1}
\newcommand\persName[1]{#1}
\newcommand\placeName[1]{#1}
\newcommand{\pn}[1]{\IfSubStr{-—–¶}{#1}% <p n="3"/>
  {\noindent{\bfseries\color{rubric}   ¶  }}
  {{\footnotesize\autour{#1}}}}
\newcommand\reg{}
% \newcommand\ref{} % already defined
\newcommand\sic[1]{#1}
\newcommand\surname[1]{\textsc{#1}}
\newcommand\term[1]{\textbf{#1}}
\newcommand\zh[1]{{\zhfont #1}}


\def\mednobreak{\ifdim\lastskip<\medskipamount
  \removelastskip\nopagebreak\medskip\fi}
\def\bignobreak{\ifdim\lastskip<\bigskipamount
  \removelastskip\nopagebreak\bigskip\fi}

% commands, blocks
\newcommand{\byline}[1]{\bigskip{\RaggedLeft{#1}\par}\bigskip}
\newcommand{\bibl}[1]{{\RaggedLeft{#1}\par\bigskip}}
\newcommand{\biblitem}[1]{{\noindent\hangindent=\parindent   #1\par}}
\newcommand{\castItem}[1]{{\noindent\hangindent=\parindent #1\par}}
\newcommand{\dateline}[1]{\medskip{\RaggedLeft{#1}\par}\bigskip}
\newcommand{\labelblock}[1]{\medbreak{\noindent\color{rubric}\bfseries #1}\par\mednobreak}
\newcommand{\salute}[1]{\bigbreak{#1}\par\medbreak}
\newcommand{\signed}[1]{\medskip{\raggedleft #1\par}\bigbreak} % supposed bottom
\newcommand{\speaker}[1]{\medskip{\centering #1\par\nopagebreak}} % supposed bottom
\newcommand{\spl}[1]{\noindent\hangindent=2\parindent  #1\par} % sp/l


% environments for blocks (some may become commands)
\newenvironment{borderbox}{}{} % framing content
\newenvironment{citbibl}{\ifvmode\hfill\fi}{\ifvmode\par\fi }
\newenvironment{docAuthor}{\ifvmode\vskip4pt\fontsize{16pt}{18pt}\selectfont\fi\itshape}{\ifvmode\par\fi }
\newenvironment{docDate}{}{\ifvmode\par\fi }
\newenvironment{docImprint}{\vskip6pt}{\ifvmode\par\fi }
\newenvironment{docTitle}{\vskip6pt\bfseries\fontsize{18pt}{22pt}\selectfont}{\par }
\newenvironment{msHead}{\vskip6pt}{\par}
\newenvironment{msItem}{\vskip6pt}{\par}
\newenvironment{titlePart}{}{\par }


% environments for block containers
\newenvironment{argument}{\itshape\parindent0pt}{\bigskip}
\newenvironment{biblfree}{}{\ifvmode\par\fi }
\newenvironment{bibitemlist}[1]{%
  \list{\@biblabel{\@arabic\c@enumiv}}%
  {%
    \settowidth\labelwidth{\@biblabel{#1}}%
    \leftmargin\labelwidth
    \advance\leftmargin\labelsep
    \@openbib@code
    \usecounter{enumiv}%
    \let\p@enumiv\@empty
    \renewcommand\theenumiv{\@arabic\c@enumiv}%
  }
  \sloppy
  \clubpenalty4000
  \@clubpenalty \clubpenalty
  \widowpenalty4000%
  \sfcode`\.\@m
}%
{\def\@noitemerr
  {\@latex@warning{Empty `bibitemlist' environment}}%
\endlist}
\newenvironment{quoteblock}% may be used for ornaments
  {\begin{quoting}}
  {\end{quoting}}
% \newenvironment{epigraph}{\parindent0pt\raggedleft\it}{\bigskip}
 % epigraph pack
\setlength{\epigraphrule}{0pt}
\setlength{\epigraphwidth}{0.8\textwidth}
% \renewcommand{\epigraphflush}{center} ? dont work

% table () is preceded and finished by custom command
\newcommand{\tableopen}[1]{%
  \ifnum\strcmp{#1}{wide}=0{%
    \begin{center}
  }
  \else\ifnum\strcmp{#1}{long}=0{%
    \begin{center}
  }
  \else{%
    \begin{center}
  }
  \fi\fi
}
\newcommand{\tableclose}[1]{%
  \ifnum\strcmp{#1}{wide}=0{%
    \end{center}
  }
  \else\ifnum\strcmp{#1}{long}=0{%
    \end{center}
  }
  \else{%
    \end{center}
  }
  \fi\fi
}


% text structure
\newcommand\chapteropen{} % before chapter title
\newcommand\chaptercont{} % after title, argument, epigraph…
\newcommand\chapterclose{} % maybe useful for multicol settings
\setcounter{secnumdepth}{-2} % no counters for hierarchy titles
\setcounter{tocdepth}{5} % deep toc
\renewcommand\tableofcontents{\@starttoc{toc}}
% toclof format
% \renewcommand{\@tocrmarg}{0.1em} % Useless command?
% \renewcommand{\@pnumwidth}{0.5em} % {1.75em}
\renewcommand{\@cftmaketoctitle}{}
\setlength{\cftbeforesecskip}{\z@ \@plus.2\p@}
\renewcommand{\cftchapfont}{}
\renewcommand{\cftchapdotsep}{\cftdotsep}
\renewcommand{\cftchapleader}{\normalfont\cftdotfill{\cftchapdotsep}}
\renewcommand{\cftchappagefont}{\bfseries}
\setlength{\cftbeforechapskip}{0em \@plus\p@}
% \renewcommand{\cftsecfont}{\small\relax}
\renewcommand{\cftsecpagefont}{\normalfont}
% \renewcommand{\cftsubsecfont}{\small\relax}
\renewcommand{\cftsecdotsep}{\cftdotsep}
\renewcommand{\cftsecpagefont}{\normalfont}
\renewcommand{\cftsecleader}{\normalfont\cftdotfill{\cftsecdotsep}}
\setlength{\cftsecindent}{1em}
\setlength{\cftsubsecindent}{2em}
\setlength{\cftsubsubsecindent}{3em}
\setlength{\cftchapnumwidth}{1em}
\setlength{\cftsecnumwidth}{1em}
\setlength{\cftsubsecnumwidth}{1em}
\setlength{\cftsubsubsecnumwidth}{1em}

% footnotes
\newif\ifheading
\newcommand*{\fnmarkscale}{\ifheading 0.70 \else 1 \fi}
\renewcommand\footnoterule{\vspace*{0.3cm}\hrule height \arrayrulewidth width 3cm \vspace*{0.3cm}}
\setlength\footnotesep{1.5\footnotesep} % footnote separator
\renewcommand\@makefntext[1]{\parindent 1.5em \noindent \hb@xt@1.8em{\hss{\normalfont\@thefnmark . }}#1} % no superscipt in foot
\patchcmd{\@footnotetext}{\footnotesize}{\footnotesize\sffamily}{}{} % before scrextend, hyperref
\DeclareNewFootnote{A}[alph] % for editor notes
\renewcommand*{\thefootnoteA}{\alphalph{\value{footnoteA}}} % z, aa, ab…

% poem
\setlength{\poembotskip}{0pt}
\setlength{\poemtopskip}{0pt}
\setlength{\poemindent}{0pt}
\poemlinenumsfalse

%   see https://tex.stackexchange.com/a/34449/5049
\def\truncdiv#1#2{((#1-(#2-1)/2)/#2)}
\def\moduloop#1#2{(#1-\truncdiv{#1}{#2}*#2)}
\def\modulo#1#2{\number\numexpr\moduloop{#1}{#2}\relax}

% orphans and widows, nowidow package in test
% from memoir package
\clubpenalty=9996
\widowpenalty=9999
\brokenpenalty=4991
\predisplaypenalty=10000
\postdisplaypenalty=1549
\displaywidowpenalty=1602
\hyphenpenalty=400
% report h or v overfull ?
\hbadness=4000
\vbadness=4000
% good to avoid lines too wide
\emergencystretch 3em
\pretolerance=750
\tolerance=2000
\def\Gin@extensions{.pdf,.png,.jpg,.mps,.tif}

\PassOptionsToPackage{hyphens}{url} % before hyperref and biblatex, which load url package
\usepackage{hyperref} % supposed to be the last one, :o) except for the ones to follow
\hypersetup{
  % pdftex, % no effect
  pdftitle={\elbibl},
  % pdfauthor={Your name here},
  % pdfsubject={Your subject here},
  % pdfkeywords={keyword1, keyword2},
  bookmarksnumbered=true,
  bookmarksopen=true,
  bookmarksopenlevel=1,
  pdfstartview=Fit,
  breaklinks=true, % avoid long links, overrided by url package
  pdfpagemode=UseOutlines,    % pdf toc
  hyperfootnotes=true,
  colorlinks=false,
  pdfborder=0 0 0,
  % pdfpagelayout=TwoPageRight,
  % linktocpage=true, % NO, toc, link only on page no
}
\urlstyle{same} % after hyperref



\makeatother % /@@@>
%%%%%%%%%%%%%%
% </TEI> end %
%%%%%%%%%%%%%%


%%%%%%%%%%%%%
% footnotes %
%%%%%%%%%%%%%
\renewcommand{\thefootnote}{\bfseries\textcolor{rubric}{\arabic{footnote}}} % color for footnote marks

%%%%%%%%%
% Fonts %
%%%%%%%%%
\usepackage[]{roboto} % SmallCaps, Regular is a bit bold
% \linespread{0.90} % too compact, keep font natural
\newfontfamily\fontrun[]{Roboto Condensed Light} % condensed runing heads
\ifav
  \setmainfont[
    ItalicFont={Roboto Light Italic},
  ]{Roboto}
\else\ifbooklet
  \setmainfont[
    ItalicFont={Roboto Light Italic},
  ]{Roboto}
\else
\setmainfont[
  ItalicFont={Roboto Italic},
]{Roboto Light}
\fi\fi
\renewcommand{\LettrineFontHook}{\bfseries\color{rubric}}
% \renewenvironment{labelblock}{\begin{center}\bfseries\color{rubric}}{\end{center}}

%%%%%%%%
% MISC %
%%%%%%%%

\setdefaultlanguage[frenchpart=false]{french} % bug on part


\newenvironment{quotebar}{%
    \def\FrameCommand{{\color{rubric!10!}\vrule width 0.5em} \hspace{0.9em}}%
    \def\OuterFrameSep{0pt} % séparateur vertical
    \MakeFramed {\advance\hsize-\width \FrameRestore}
  }%
  {%
    \endMakeFramed
  }
\renewenvironment{quoteblock}% may be used for ornaments
  {%
    \savenotes
    \setstretch{0.9}
    \normalfont
    \begin{quotebar}
  }
  {%
    \end{quotebar}
    \spewnotes
  }


\renewcommand{\headrulewidth}{\arrayrulewidth}
\renewcommand{\headrule}{{\color{rubric}\hrule}}

% delicate tuning, image has produce line-height problems in title on 2 lines
\titleformat{name=\chapter} % command
  [display] % shape
  {\vspace{1.5em}\centering} % format
  {} % label
  {0pt} % separator between n
  {}
[{\color{rubric}\huge\textbf{#1}}\bigskip] % after code
% \titlespacing{command}{left spacing}{before spacing}{after spacing}[right]
\titlespacing*{\chapter}{0pt}{-2em}{0pt}[0pt]

\titleformat{name=\section}
  [display]{}{}{}{}
  [\vbox{\color{rubric}\large\raggedleft\textbf{#1}}]
\titlespacing{\section}{0pt}{0pt plus 4pt minus 2pt}{\baselineskip}

\titleformat{name=\subsection}
  [block]
  {}
  {} % \thesection
  {} % separator \arrayrulewidth
  {}
[\vbox{\large\textbf{#1}}]
% \titlespacing{\subsection}{0pt}{0pt plus 4pt minus 2pt}{\baselineskip}

\ifaiv
  \fancypagestyle{main}{%
    \fancyhf{}
    \setlength{\headheight}{1.5em}
    \fancyhead{} % reset head
    \fancyfoot{} % reset foot
    \fancyhead[L]{\truncate{0.45\headwidth}{\fontrun\elbibl}} % book ref
    \fancyhead[R]{\truncate{0.45\headwidth}{ \fontrun\nouppercase\leftmark}} % Chapter title
    \fancyhead[C]{\thepage}
  }
  \fancypagestyle{plain}{% apply to chapter
    \fancyhf{}% clear all header and footer fields
    \setlength{\headheight}{1.5em}
    \fancyhead[L]{\truncate{0.9\headwidth}{\fontrun\elbibl}}
    \fancyhead[R]{\thepage}
  }
\else
  \fancypagestyle{main}{%
    \fancyhf{}
    \setlength{\headheight}{1.5em}
    \fancyhead{} % reset head
    \fancyfoot{} % reset foot
    \fancyhead[RE]{\truncate{0.9\headwidth}{\fontrun\elbibl}} % book ref
    \fancyhead[LO]{\truncate{0.9\headwidth}{\fontrun\nouppercase\leftmark}} % Chapter title, \nouppercase needed
    \fancyhead[RO,LE]{\thepage}
  }
  \fancypagestyle{plain}{% apply to chapter
    \fancyhf{}% clear all header and footer fields
    \setlength{\headheight}{1.5em}
    \fancyhead[L]{\truncate{0.9\headwidth}{\fontrun\elbibl}}
    \fancyhead[R]{\thepage}
  }
\fi

\ifav % a5 only
  \titleclass{\section}{top}
\fi

\newcommand\chapo{{%
  \vspace*{-3em}
  \centering\parindent0pt % no vskip ()
  \eltitlepage
  \bigskip
  {\color{rubric}\hline\par}
  \bigskip
  {\Large TEXTE LIBRE À PARTICIPATIONS LIBRES\par}
  \centerline{\small\color{rubric} {\href{https://hurlus.fr}{\dotuline{hurlus.fr}}}, tiré le \today}\par
  \bigskip
}}

\newcommand\cover{{%
  \thispagestyle{empty}
  \centering\parindent0pt
  \eltitlepage
  \vfill\null
  {\color{rubric}\setlength{\arrayrulewidth}{2pt}\hline\par}
  \vfill\null
  {\Large TEXTE LIBRE À PARTICIPATIONS LIBRES\par}
  \centerline{\href{https://hurlus.fr}{\dotuline{hurlus.fr}}, tiré le \today}\par
}}

\begin{document}
\pagestyle{empty}
\ifbooklet{
  \cover\newpage
  \thispagestyle{empty}\hbox{}\newpage
  \cover\newpage\noindent Les voyages de la brochure\par
  \bigskip
  \begin{tabularx}{\textwidth}{l|X|X}
    \textbf{Date} & \textbf{Lieu}& \textbf{Nom/pseudo} \\ \hline
    \rule{0pt}{25cm} &  &   \\
  \end{tabularx}
  \newpage
  \addtocounter{page}{-4}
}\fi

\thispagestyle{empty}
\ifaiv
  \twocolumn[\chapo]
\else
  \chapo
\fi
{\it\elabstract}
\bigskip
\makeatletter\@starttoc{toc}\makeatother % toc without new page
\bigskip

\pagestyle{main} % after style
\setcounter{footnote}{0}
\setcounter{footnoteA}{0}
  
\chapteropen

\chapter[{I. Sur l’antique voie ferrée}]{I. Sur l’antique voie ferrée}
\renewcommand{\leftmark}{I. Sur l’antique voie ferrée}


\chaptercont
\noindent Le chemin, à peine tracé, suivait ce qui avait été jadis le remblai d’une voie ferrée, que depuis bien des années aucun train n’avait parcourue. À droite et à gauche, la forêt, qui escaladait et gonflait les pentes du remblai, l’enveloppait d’une vague verdoyante d’arbres et d’arbustes. Le chemin n’était qu’une simple piste, à peine assez large pour laisser passer deux hommes de front. C’était quelque chose comme un sentier d’animaux sauvages.\par
Çà et là, un morceau de fer rouillé apparaissait, indiquant que, sous les buissons, rails et traverses subsistaient. On voyait, à un endroit, un arbre surgir qui, en croissant, avait soulevé en l’air avec lui tout un rail, qui se montrait à nu. La lourde traverse avait suivi le rail, auquel elle était rivée encore par un écrou. On apercevait au-dessous les pierres du ballast, à demi recouvertes par des feuilles mortes. Ainsi, rail et traverse, bizarrement enlacés l’un dans l’autre, pointaient vers le ciel, fantomatiques. Si antique que fût la voie ferrée, on reconnaissait sans peine, à son étroitesse, qu’elle avait été à voie unique.\par
Un vieillard et un jeune garçon suivaient le sentier.\par
Ils avançaient lentement, car le vieillard était chargé d’ans. Un début de paralysie faisait trembloter ses membres et ses gestes, et il peinait en s’appuyant sur son bâton.\par
Un bonnet grossier de peau de chèvre protégeait sa tête contre le soleil. De dessous ce bonnet pendait une maigre frange de cheveux blancs, sales et souillés. Une sorte de visière, ingénieusement faite d’une large feuille courbe, gardait les yeux d’une trop vive lumière. Et, sous cette visière, les regards baissés du bonhomme suivaient attentivement le mouvement de ses pieds sur le sentier.\par
Sa barbe, qui descendait en masse, tout emmêlée, jusqu’à sa ceinture, aurait dû être, comme les cheveux, d’une blancheur de neige. Mais, comme eux, elle témoignait d’une grande négligence et d’une grande misère.\par
Un sordide vêtement de peau de chèvre, d’une seule pièce, pendait de la poitrine et des épaules du vieillard, dont les bras et les jambes, péniblement décharnés, et la peau flétrie, témoignaient d’un âge avancé. Les écorchures et les cicatrices qui les couvraient, et le ton bruni de l’épiderme, indiquaient de leur côté que, depuis longtemps, l’homme était exposé aux heurts de la nature et des éléments.\par
Le jeune garçon marchait devant, réglant l’ardeur robuste de ses jarrets sur les pas lents du vieillard qui le suivait. Lui aussi n’avait pour tout vêtement qu’une peau de bête. Un morceau de peau d’ours, aux bords déchiquetés, avec un trou en son milieu, par où il avait passé la tête.\par
Il semblait avoir douze ans au plus, et portait, coquettement juchée sur l’oreille, une queue de porc, fraîchement coupée.\par
Dans une de ses mains il tenait un arc, de taille moyenne, et une flèche. Sur son dos était un carquois rempli de flèches. D’un fourreau, pendu à son cou par une courroie, émergeait le manche noueux d’un couteau de chasse. Il était aussi noir qu’une mûre et sa souple allure ressemblait à celle d’un chat. Ses yeux bleus, d’un bleu profond, étaient vifs et perçants comme des vrilles, et leur azur formait un étrange contraste avec la peau brûlée par le soleil qui les encadrait.\par
Ces yeux semblaient épier sans trêve tous les objets ambiants. Et les narines dilatées du jeune garçon ne palpitaient pas moins, en un perpétuel affût du monde extérieur dont elles recueillaient avidement tous les messages. Son ouïe paraissait aussi subtile, et à ce point était-elle exercée qu’elle opérait automatiquement, sans même une tension de l’oreille.\par
Tout naturellement et sans effort, celle-ci percevait, dans le calme apparent qui régnait, les sons les plus légers, les départageait entre eux et les classait ; que ce fût le frôlement du vent sur les feuilles, le bourdonnement d’une abeille ou d’un moucheron, ou le bruit sourd et lointain de la mer, qui n’arrivait que comme un faible murmure, ou l’imperceptible grattement des pattes d’un petit rongeur, dégageant la terre à l’entrée de son trou.\par
Soudain, le corps du jeune garçon s’alerta et se tendit. Simultanément, le son, la vue et l’odeur l’avaient averti. Il tendit la main vers le vieux, et l’en toucha, et tous deux se tinrent cois.\par
Devant eux, sur la pente du remblai et vers son sommet, quelque chose avait craqué. Et le regard rapide du jeune garçon se fixa sur les buissons dont le faîte s’agitait.\par
Alors un grand ours, un ours grizzly, surgit bruyamment, en pleine vue, et lui aussi s’arrêta net, à l’aspect des deux humains.\par
L’ours n’aimait pas les hommes. Il grogna grincheusement. Lentement, et prêt à tout événement, le jeune garçon ajusta la flèche sur son arc et en tendit la corde, sans quitter la bête du regard. Le vieux, sous la feuille qui lui servait de visière, épiait le danger et, pas plus que son compagnon, ne bougeait.\par
Pendant quelques instants, l’ours et les deux humains se dévisagèrent mutuellement. Puis, comme la bête trahissait, par ses grognements, une irritation croissante, le jeune garçon fit signe au vieillard, d’un léger signe de tête, qu’il convenait de laisser le sentier libre et de descendre la pente du remblai. Ainsi agirent-ils tous deux, le vieux allant devant, l’enfant le suivant à reculons, l’arc toujours bandé, et prêt à tirer.\par
Une fois en bas, ils attendirent, jusqu’à ce qu’un grand bruit de feuilles et de branches froissées, sur l’autre face du remblai, les eût avertis que l’ours s’en était allé. Ils regrimpèrent vers le sommet et le jeune garçon dit, avec un ricanement prudemment étouffé :\par
— C’en était un gros, grand-père !\par
Le vieillard fit un signe affirmatif. Il secoua tristement la tête et répondit, d’une voix de fausset, pareille à celle d’un enfant :\par
— Ils deviennent de jour en jour plus nombreux. Qui aurait jamais pensé, autrefois, que je vivrais assez pour voir le temps où il y aurait danger pour sa vie à circuler sur le territoire de la station balnéaire de Cliff-House ? Au temps dont je te parle, Edwin, alors que j’étais moi-même un enfant, hommes, femmes, petits garçons et petites filles, et bébés, accouraient ici, par dizaines de mille, à la belle saison. Et il n’y avait pas d’ours alors, dans le pays, je te l’assure bien. Ou du moins, ils étaient si rares qu’on les mettait dans des cages et que l’on donnait de l’argent pour les voir.\par
— De l’argent, grand-père ? Qu’est cela ?\par
Avant que le vieillard eût répondu, Edwin, se frappant la tête, s’était souvenu. Il avait insinué sa main dans une sorte de poche, ménagée sous sa peau d’ours, et en avait tiré triomphalement un dollar en argent, tout bossué et terni.\par
Les yeux du bonhomme s’illuminèrent, tandis qu’il se penchait sur la pièce.\par
— Ma vue est mauvaise, marmotta-t-il. Toi, regarde, Edwin, si tu peux déchiffrer la date qui est inscrite.\par
L’enfant se mit à rire et s’exclama, tout hilare :\par
— Tu es étonnant, grand-père ! Toujours tu veux me faire croire que ces petits signes, qui sont là-dessus, veulent dire quelque chose.\par
Le vieux gémit profondément et amena le petit disque métallique à deux ou trois pouces de ses propres yeux :\par
— 2012 ! finit-il par s’exclamer.\par
Puis il s’abandonna à un cocasse caquetage :\par
— 2012 ! C’était l’année où Morgan V fut élu Président des États-Unis, par l’Assemblée des Magnats. Ce dut être une des dernières pièces frappées, car la Mort Écarlate survint en 2013. Seigneur ! Seigneur ! Quand j’y songe ! Il y a de cela soixante ans. Et je suis aujourd’hui le dernier survivant qui ait connu ce temps-là ! Cette pièce, Edwin, où l’as-tu trouvée ?\par
Edwin, qui avait écouté son grand-père avec la condescendance bienveillante que l’on doit aux radotages de ceux qui sont faibles d’esprit, répondit aussitôt :\par
— C’est Hou-Hou qui me l’a donnée ! Il l’a trouvé en gardant ses chèvres, près de San José, au printemps dernier. Hou-Hou dit que c’est de l’argent… Mais, grand-père, n’as-tu point faim ? Nous remettons-nous en marche ?\par
Le bonhomme, ayant rendu le dollar à Edwin, serra plus fort le bâton dans son poing et se hâta vers le sentier, ses vieux yeux brillants de gourmandise.\par
— Espérons, murmura-t-il, que Bec-de-Lièvre aura trouvé un crabe… Peut-être deux ! C’est bon à manger, l’intérieur des crabes. Très bon à manger, quand on n’a plus de dents, et lorsqu’on a des petits-fils qui aiment bien, comme vous, leur grand-père et se font un devoir de lui en attraper ! Lorsque j’étais enfant…\par
Mais Edwin, ayant vu quelque chose, s’était arrêté et, le doigt sur ses lèvres, avait fait signe à l’ancêtre qu’il se tût. Il ajusta une flèche sur la corde de son arc et s’avança, en se dissimulant dans une vieille conduite d’eau, à moitié éclatée, qui avait en crevant fait rompre un rail. Sous la vigne vierge et les plantes rampantes qui le recouvraient, on apercevait le gros tube rouillé.\par
Le jeune garçon arriva ainsi en face d’un lapin, assis sur son derrière, près d’un buisson, et qui le regarda, hésitant et tout tremblant.\par
La distance était bien encore de cinquante pieds. Mais la flèche fila droit au but, avec la vitesse de l’éclair, et le lapin, transpercé, poussa un cri de douleur. Puis il se traîna, en couinant, jusqu’au buisson, pour s’y réfugier.\par
Le jeune garçon était, comme la flèche, un éclair. Un éclair de peau brune et de fourrure flottante. Tandis qu’il bondissait vers le lapin, ses muscles se détendaient comme des ressorts d’acier, puissants et souples dans ses membres maigres. Il se saisit de l’animal blessé, l’acheva en lui cognant la tête contre un tronc d’arbre qui se trouvait à sa portée, puis, étant revenu vers le vieux, il le lui donna, pour qu’il s’en chargeât.\par
— C’est bon, le lapin, très bon… marmotta l’ancêtre. Mais en tant que friandise délicieuse au goût, je préfère le crabe. Quand j’étais enfant…\par
Edwin, impatienté de la vaine loquacité du vieux, l’interrompit.\par
— Pourquoi, dit-il en lui coupant la parole, tant de phrases à propos de tout, qui ne signifient rien ?\par
Il s’exprima moins correctement, mais tel était le sens approximatif de ses paroles. Son parler était guttural et impétueux, et le langage qu’il employait s’apparentait nettement à celui du vieux, qui était lui-même un dérivé, tant soit peu corrompu, de l’anglais.\par
Edwin reprit :\par
— Cela m’agace d’entendre, à chaque instant, des mots que je ne comprends pas. Pourquoi, par exemple, grand-père, appelles-tu le crabe une « friandise » ? Un crabe, c’est un crabe, et rien de plus. Que veut dire ce sobriquet ?\par
Le vieillard soupira et ne répondit pas, et tous deux continuèrent à marcher en silence. Le bruit du ressac se faisait de plus en plus fort et, comme ils émergeaient tous deux de la forêt, la mer soudain apparut, au-delà de grandes dunes de sable.\par
Quelques chèvres broutaient, parmi ces dunes, une herbe rare, gardées par un autre jeune garçon, vêtu de peaux de bêtes, et par un chien, qui n’était plus qu’une faible réminiscence du chien et semblait bien plutôt un loup. Au premier plan s’élevait la fumée d’un feu, que surveillait un troisième garçon, non moins hirsute d’aspect que les deux autres. Autour de lui se tenaient accroupis plusieurs chiens-loups, pareils à celui qui gardait les chèvres.\par
À une centaine de yards de la côte, on voyait un groupe de rochers déchiquetés et au grondement des vagues qui les battaient se mêlait une sorte d’aboiement profond. C’était le mugissement d’énormes lions marins, qui s’y traînaient, les uns pour s’y étendre au soleil, les autres pour se battre entre eux.\par
Le vieillard se dirigea vers le feu, en accélérant le pas et en reniflant l’air avidement.\par
— Des moules ! s’exclama-t-il, extasié, de sa petite voix chevrotante, quand il fut arrivé. Des moules ! Et qu’est ceci, Hou-Hou ? N’est-ce pas un crabe ? Mon Dieu ! mes enfants, comme vous êtes bons pour votre grand-père !\par
Hou-Hou, qui semblait être à peu près du même âge qu’Edwin, répondit, avec une grimace qui voulait être un sourire :\par
— Mange, grand-père, tout ce que tu veux. Les moules ou les crabes. Il y en a quatre.\par
L’enthousiasme paralytique du vieillard faisait peine à voir. Il s’assit sur le sable, aussi rapidement que le lui permirent ses membres raides, et tira des charbons ardents une grosse moule de rocher. La chaleur avait écarté les deux coquilles et la chair de la moule apparaissait, de couleur saumon et cuite à point.\par
Entre le pouce et l’index, avec une hâte fébrile, le vieillard se saisit de ce succulent morceau et le porta vivement à sa bouche. Mais la moule était brûlante et, l’instant d’après, il la recrachait violemment, en poussant des hurlements de douleur. Des larmes se prirent à couler le long de ses joues.\par
Les jeunes garçons étaient de vrais petits sauvages, et sauvage était leur cruelle gaîté. Ils éclatèrent de rire devant la déconvenue cuisante du vieillard, qu’ils trouvèrent fort divertissante. Hou-Hou en faisait en l’air d’interminables cabrioles, tandis qu’Edwin se roulait, en pouffant, sur le sol. Attiré par le bruit, le petit chevrier accourut et partagea bientôt leur hilarité.\par
— Fais-les refroidir, Edwin… Fais-les refroidir… supplia le vieillard, dans sa souffrance, et sans même essuyer les larmes qui continuaient à ruisseler de ses yeux. Fais aussi refroidir un crabe, Edwin… Tu sais comme ton grand-père aime les crabes.\par
Un grand grésillement s’éleva du feu, qui faisait s’ouvrir et éclater, dans une vapeur humide, toutes les coquilles des moules. Ces mollusques étaient, pour la plupart, de forte taille et mesuraient de trois à six pouces de long. Les gamins les tirèrent hors du feu, à l’aide de petits bâtons, et les alignèrent sur une vieille souche de bois flotté, pour qu’ils y refroidissent.\par
Le vieux gémissait :\par
— De mon temps, on ne se moquait pas ainsi des anciens… on les respectait…\par
Les jeunes garçons ne prêtèrent nulle attention aux plaintes et aux récriminations de l’ancêtre. Mais le vieux, cette fois, fut plus prudent et ne se brûla point la bouche. Tous s’étaient mis à manger, en faisant grand bruit avec leur langue et en claquant des lèvres.\par
Le troisième gamin, qui s’appelait Bec-de-Lièvre et avait envie de rire encore un peu, déposa sournoisement une pincée de sable sur une des moules, qu’il tendit ensuite au vieillard. Celui-ci, l’ayant portée à sa bouche, le sable écorcha ses gencives et ses muqueuses, et il en fit une horrible grimace.\par
Le rire alors reprit, tumultueux. Le vieux ne se rendait pas compte que c’était un mauvais tour qu’on lui avait joué. Il bredouillait lamentablement et crachait à force, jusqu’à ce qu’Edwin, pris de pitié, lui tendît une gourde d’eau fraîche, dont il se rinça la bouche.\par
— Voyons, Hou-Hou, où sont les crabes ? demanda Edwin, Grand-père, aujourd’hui, est en appétit…\par
En entendant parler de crabes, les yeux du vieux s’éclairèrent de gourmandise, et Hou-Hou lui en tendit un, qui était fort gros. La carapace était au complet avec toutes ses pattes, mais elle était vide. De ses mains tremblantes, avec de petits cris d’impatience, le vieillard brisa une des pattes et n’y trouva rien que du néant.\par
Il gémit :\par
— Un crabe, Hou-Hou ! donne-moi un vrai crabe…\par
Hou-Hou répondit :\par
— On s’est moqué de toi, grand-père. Il n’y a pas de crabe. Je n’en ai pas trouvé un seul.\par
Le désappointement se peignit sur le visage ridé de l’ancêtre et il se reprit à pleurer abondamment, tandis que les gamins ne se tenaient pas de joie.\par
Subrepticement, Hou-Hou remplaça la carcasse vide, que le vieux avait déposée par terre devant lui, par un crabe plein, dont il avait fait craquer pattes et carapace, et dont la chair blanche émettait un fumet délicieux. Les narines du vieillard en furent divinement chatouillées et il abaissa son regard, tout étonné.\par
Sa maussade humeur se mua instantanément en gaîté. Il renifla, renifla, puis, avec un ronron de béatitude, il commença à manger. Et, tout en mâchant des gencives, il marmottait un mot qui n’avait aucun sens pour ses auditeurs :\par
— Mayonnaise… Mayonnaise…\par
Il fit claquer sa langue et continua :\par
— De la mayonnaise ! Voilà qui serait bon… Et dire que voici plus de soixante ans qu’on n’en a vue ! Deux générations ont grandi sans connaître son merveilleux parfum. Dans tous les restaurants, autrefois, on en servait avec le crabe !\par
Quand il fut rassasié, le vieux soupira, s’essuya les mains sur ses cuisses nues, et son regard se perdit sur la mer. Puis, dans le bien-être d’un estomac bien garni, il se mit à fouiller au tréfonds de sa mémoire.\par
— Savez-vous, mes enfants, savez-vous bien que j’ai vu ce rivage grouillant de vie ? Hommes, femmes et enfants, s’y pressaient tous les dimanches. Il n’y avait pas d’ours pour les dévorer, mais là-haut, sur la falaise, un magnifique restaurant, où l’on pouvait trouver tout ce qu’on désirait manger. Quatre millions d’hommes vivaient alors à San Francisco. Et maintenant, dans toute cette contrée, il n’en reste pas quarante au total. La mer aussi était pleine de bateaux, de bateaux qui passaient et repassaient la Porte d’Or. Et il y avait dans l’air quantité de dirigeables et d’avions. Ils pouvaient franchir une distance de deux cents milles à l’heure.\par
« Oui, c’était la vitesse minima qu’exigeaient les contrats de la Compagnie Aérienne qui assurait le service postal entre New-York et San Francisco. Il y avait un homme, un Français, qui avait offert la vitesse de trois cents milles. Hum ! hum ! Ceci avait paru beaucoup, et trop risqué, aux gens rétrogrades. N’importe, le Français tenait le bon bout et il aurait mené son affaire à bien, n’eût été la Grande Peste. Au temps où j’étais enfant, il existait encore des gens qui se souvenaient d’avoir vu les premiers aéroplanes. Moi, j’ai vu les derniers. Il y a de cela soixante ans…\par
Les gamins l’écoutaient monologuer, d’un air distrait. Ils ne comprenaient pas les trois quarts des choses dont il parlait, et ils étaient las de l’entendre ainsi rabâcher. D’autant qu’au cours de ses rêveries à haute voix, il employait un anglais plus pur, qui n’avait qu’un lointain rapport avec le jargon grossier dont ils se servaient et dont il usait vis-à-vis d’eux.\par
Il continua :\par
— Les crabes, par contre, en ce temps-là, étaient plus rares, car on les pêchait partout, et c’était un mets très apprécié. La pêche en était autorisée un mois seulement, chaque année. Aujourd’hui, on peut les capturer d’un bout de l’an à l’autre bout. Cela, jadis, aurait paru merveilleux !\par
À ce moment, une vive agitation qui se produisit parmi les chèvres paissant sur les dunes fit se lever les trois jeunes garçons. Les chiens accroupis autour du feu coururent rejoindre leur camarade, qui était resté à côté des chèvres et qui grognait furieusement. Tout le troupeau rappliqua vers ses protecteurs humains.\par
Une demi-douzaine de formes grises et efflanquées glissaient furtivement sur le sable, et tenaient tête aux chiens, dont le poil se hérissait.\par
Edwin lança vers elles une flèche qui manqua son but. Mais Bec-de-Lièvre, armé d’une fronde toute semblable à celle qui dut servir à David dans sa lutte contre Goliath, fit tourbillonner une pierre, dont le vol rapide siffla à travers l’air. La pierre tomba en plein parmi les loups, qui disparurent vers les noires profondeurs de la forêt d’eucalyptus.\par
Leur fuite fit rire les trois gamins. Satisfaits, ils revinrent s’étendre sur le sable, près de l’ancêtre, qui geignait lourdement. Il avait trop mangé et sa digestion était pénible. Et, tout en tenant sur son ventre ses deux mains, aux doigts entrelacés, il poursuivait ses lamentations :\par
— « \emph{Le travail de l’homme est éphémère et s’évanouit comme l’écume de la mer…} » Oui, c’est bien cela. L’homme a, sur cette planète, domestiqué les animaux utiles, détruit ceux qui étaient nuisibles. Il a défriché la terre et l’a dépouillée de sa végétation sauvage. Puis, un jour, il disparaît, et le flot de la vie primitive est revenu sur lui-même, balayant l’œuvre humaine. Les mauvaises herbes et la forêt ont derechef envahi les champs, les bêtes de proie sont revenues sur les troupeaux, et maintenant il y a des loups sur la plage de Cliff-House !\par
Il parut effrayé à cette pensée, s’arrêta, puis reprit :\par
— Si quatre millions d’hommes ont disparu, en un seul pays, si les loups féroces errent aujourd’hui à cette place et si vous, progéniture barbare de tant de génie éteint, vous en êtes réduits à vous défendre, à l’aide d’armes préhistoriques, contre les crocs des envahisseurs à quatre pattes, c’est à cause de la Mort Écarlate !\par
— Écarlate… Écarlate… murmura Bec-de-Lièvre à l’oreille d’Edwin… Grand-père répète souvent ce mot. Sais-tu ce qu’il signifie ?\par
Le vieux avait entendu la question et déclama de sa voix aigrelette :\par
— « \emph{L’écarlate de l’érable, à la saison d’automne, me fait tressaillir comme une sonnerie de clairon qui passe…} » a dit un poète.\par
Edwin expliqua à Bec-de-Lièvre :\par
— L’écarlate, c’est rouge… Tu ne sais pas cela, parce que tu as été élevé dans la Tribu du Chauffeur. Aucun de ses membres n’a jamais rien su. L’écarlate est rouge… Je le sais, moi.\par
Bec-de-Lièvre protesta :\par
— Si l’écarlate est rouge, pourquoi ne pas dire rouge ? À quoi bon compliquer les choses par des mots que l’on ne comprend pas ? Rouge est rouge. Et voilà tout.\par
— Rouge n’est pas le mot propre, rétorqua le vieux. La peste n’était pas rouge, elle était écarlate. Le corps et la figure de celui qui en était atteint devenaient écarlates, dans l’espace d’une heure. Je le sais, je l’ai vu. C’est écarlate qu’il faut dire.\par
Mais Bec-de-Lièvre n’était pas convaincu ; il s’obstina :\par
— Rouge est assez bon pour moi. Papa n’emploie pas d’autre mot. Il dit que tout le monde mourut de la Mort Rouge.\par
Le Vieux s’irrita.\par
— Ton père, comme l’a dit Edwin, est un homme du commun, né d’un homme du commun. Il n’a jamais eu aucune éducation. Ton grand-père était un chauffeur, un domestique. Ta grand-mère, il est vrai, était de bonne souche. C’était une lady, mais ses enfants, ni ses petits-enfants, ne lui ressemblèrent. Avant la Mort Écarlate elle était la femme de Van Warden, un des douze Magnats de l’Industrie, qui gouvernaient l’Amérique. Il valait plus d’un milliard de dollars — tu entends bien, Edwin, plus d’un milliard de petites pièces pareilles à celles que tu as dans ta poche. Puis vint la Mort Écarlate. Et cette femme devint la femme de Bill, le chauffeur. Il avait l’habitude de la battre. Je l’ai vu de mes propres yeux. Voilà, Bec-de-Lièvre, qui fut ta grand-mère.\par
Hou-Hou, durant cette discussion, paresseusement allongé sur le sable, s’amusait à y creuser du pied une tranchée.\par
Tout à coup il poussa un cri. Son orteil s’était heurté à un corps dur, auquel il s’était écorché. Il se redressa et se mit à examiner le trou qu’il avait creusé.\par
Les deux autres garçons se joignirent à lui, et rapidement tous trois continuèrent à fouiller, enlevant le sable avec leurs mains. Trois squelettes apparurent. Deux étaient des squelettes d’adultes et le troisième avait appartenu à un adolescent.\par
L’ancêtre vint, sur ses genoux, jusqu’au trou au-dessus duquel il se pencha.\par
— Ce sont, annonça-t-il, des victimes de la Peste Écarlate. Voilà comme on mourait, n’importe où. Ceci fut sans doute une famille qui fuyait la contagion et qui est tombée ici, sur la grève de Cliff-House. Ils… Mais que fais-tu là, Edwin ?\par
Edwin, avec la pointe de son couteau de chasse, avait commencé à faire sauter les dents de la mâchoire d’un des squelettes.\par
— Seigneur, que fais-tu là ? répéta le vieux, tout effaré.\par
— C’est pour en fabriquer un collier… répondit le gamin.\par
Les deux autres garçons imitèrent Edwin, grattant ou cognant, de la pointe ou du dos de leurs couteaux. Le vieux gémissait :\par
— Vous êtes des sauvages, de vrais sauvages. La mode vient déjà de porter des parures de dents humaines. La prochaine génération se percera le nez et les oreilles, et se parera d’os d’animaux et de coquillages. Aucun doute là-dessus. La race humaine est condamnée à s’enfoncer de plus en plus dans la nuit primitive, avant de reprendre un jour sa réascension sanglante vers la civilisation. Le sol, aujourd’hui, est trop vaste pour les quelques hommes qui y survivent. Mais ces hommes croîtront et multiplieront et, dans quelques générations, ils trouveront la terre trop étroite et commenceront à s’entre-tuer. Cela, c’est fatal. Alors ils porteront à la taille les scalps de leurs ennemis, comme toi, Edwin, qui es le plus gentil de mes petits-enfants, tu commences déjà à porter sur l’oreille cette horrible queue de cochon. Crois-moi, mon petit, jette-la, jette-la au loin !\par
— Quelle tapette ! grogna Bec-de-Lièvre !\par
L’extraction des dents des trois squelettes était terminée et les trois jeunes garçons se mirent en devoir de se les partager équitablement. Ils étaient vifs et brusques, dans leurs gestes et dans leurs paroles, et la discussion fut chaude. Ils s’exprimaient par monosyllabes, en phrases courtes et hachées.\par
Puis, satisfaits de leur trouvaille, ils s’assirent en rond, autour de l’ancêtre, et, tout en jouant avec les petits bouts d’ivoire, Bec-de-Lièvre demanda :\par
— Veux-tu, vieux, nous parler un peu de la Mort Rouge ?\par
— De la Mort Écarlate… rectifia Edwin.
\chapterclose


\chapteropen

\chapter[{II. Au temps où San-Francisco comptait, quatre millions d’hommes}]{II. Au temps où San-Francisco comptait \\
quatre millions d’hommes}
\renewcommand{\leftmark}{II. Au temps où San-Francisco comptait \\
quatre millions d’hommes}


\chaptercont
\noindent Le bonhomme parut flatté de la demande. Il éclaircit sa gorge en toussant et commença :\par
— Il y a seulement vingt ou trente ans, on me demandait souvent de conter mon histoire. Aujourd’hui, la jeunesse se désintéresse de plus en plus du passé…\par
— Tâche seulement, observa Bec-de-Lièvre, de parler clairement, si tu veux que nous te comprenions. Pas de phrases compliquées et de mots savants !\par
Edwin poussa du coude Bec-de-Lièvre.\par
— Voyons, tais-toi, dit-il. Sinon grand-père va se fâcher. Il ne parlera pas et nous ne saurons rien. Ce n’est pas de sa faute s’il s’exprime mal.\par
Le vieux, en effet, était prêt déjà à s’irriter et à entreprendre un grand discours, tant sur le manque de respect des enfants actuels que sur le triste sort de l’humanité, retournée à la barbarie des premiers âges du monde.\par
— Vas-y, grand-père… insinua Hou-Hou, d’un ton conciliant.\par
Le vieux se décida.\par
— En ce temps-là, dit-il, le monde était très peuplé. Rien qu’à San Francisco, on comptait quatre millions d’habitants…\par
— Un million, qu’est-ce que c’est ? interrompit Edwin.\par
Le vieux le regarda de côté et expliqua, avec bonté :\par
— Tu ne sais pas compter plus loin que dix, je ne l’ignore pas. Mais je vais te faire comprendre. Lève en l’air tes deux mains. Sur les deux, tu as, en tout, dix doigts. Bon. Je ramasse maintenant ce grain de sable. Tends la main, Hou-Hou.\par
Il laissa tomber le grain de sable dans la paume de Hou-Hou et poursuivit :\par
— Ce grain de sable représente les dix doigts d’Edwin. J’y ajoute un autre grain. Voilà dix autres doigts en plus. Et j’ajoute encore un troisième, un quatrième, un cinquième grain, et ainsi de suite jusqu’à dix. Cela fait dix fois dix doigts d’Edwin. C’est ce que j’appelle une centaine. Tous trois, rappelez-vous bien ce mot : une centaine. Je prends maintenant ce petit caillou et je le mets dans la main de Bec-de-Lièvre. Il représente dix grains de sable ou dix dizaines de doigts, c’est-à-dire cent doigts. Je mets dix cailloux. Ils représentent mille doigts. Je continue et prends une coquille de moule, qui représente dix cailloux, c’est-à-dire cent grains de sable ou mille doigts…\par
Laborieusement, de la sorte, l’ancêtre, par répétitions successives, réussit tant bien que mal à édifier dans l’esprit des jeunes garçons une conception approximative des nombres. À mesure que les chiffres montaient, il mettait dans les mains des enfants des objets différents, qui les symbolisaient. Quand il en fut aux millions, il les figura par les dents arrachées aux squelettes. Puis il multiplia les dents par des carapaces de crabes, pour exprimer les milliards. Il s’arrêta là, car ses auditeurs donnaient manifestement des signes de fatigue.\par
Il reprit :\par
— Il y avait donc quatre millions d’hommes à San Francisco. Soit quatre dents…\par
Le regard des jeunes garçons se porta des dents aux cailloux, puis des cailloux aux grains de sable, et des grains de sable aux doigts levés d’Edwin. Après quoi, ils parcoururent en sens inverse la série ascendante des symboles, en s’efforçant de concevoir les sommes inouïes qu’ils représentaient.\par
— Quatre millions d’hommes, cela faisait un nombre considérable, hasarda enfin Edwin.\par
— Tu y es, mon enfant ! approuva le vieux. Tu peux faire encore une autre comparaison avec les grains de sable de ce rivage. Suppose que chacun de ces grains était un homme, une femme, ou un enfant. Voilà ! Ces quatre millions d’hommes vivaient à San Francisco, qui était une grande ville, sur cette même baie où nous sommes. Et les habitants s’étendaient au-delà de la ville, sur tout le contour de la baie et au bord de la mer, et dans les terres, parmi plaines et collines. Cela faisait au total sept millions d’habitants. Sept dents !\par
De nouveau, les yeux des jeunes garçons coururent sur les dents, sur les cailloux, sur les grains de sable et sur les doigts levés.\par
— Le monde tout entier fourmillait d’hommes. Le grand recensement de l’an 2010 avait donné huit milliards pour la population de l’univers. Huit milliards ou huit coquilles de crabes… Ce temps ne ressemblait guère à celui où nous vivons. L’humanité était étonnamment experte à se procurer de la nourriture. Et plus elle avait à manger, plus elle croissait en nombre. Si bien que huit milliards d’hommes vivaient sur la terre quand la Mort Écarlate commença ses ravages. J’étais, à ce moment, un jeune homme. J’avais vingt-sept ans. J’habitais Berkeley, qui est sur la baie de San Francisco, du côté qui fait face à la ville. Tu te souviens, Edwin, de ces grandes maisons de pierre que nous avons rencontrées un jour, dans cette direction… Par là… Voilà où j’habitais, dans une de ces maisons de pierre. J’étais professeur de littérature anglaise.\par
Une forte partie de ce discours dépassait l’entendement des gamins. Mais ils s’efforçaient à saisir, de leur mieux, quoique obscurément, ce récit du passé.\par
— Qu’est-ce que tu faisais, dans ces maisons ? questionna Bec-de-Lièvre.\par
— Ton père, tu t’en souviens, t’a appris un jour à nager…\par
Bec-de-Lièvre fit un signe affirmatif.\par
— Eh ! bien, à l’Université de Californie (c’est ainsi que s’appelaient ces maisons), on apprenait aux jeunes gens et aux jeunes filles toutes sortes de choses. On leur apprenait à penser et à s’instruire l’esprit. Tout comme je viens de vous enseigner, à l’aide du sable, des cailloux, des dents et des coquilles, à calculer combien d’habitants vivaient alors sur la terre. Il y avait beaucoup à enseigner. Les jeunes gens étaient appelés des « étudiants ». Il y avait de vastes salles, où moi et les autres professeurs, nous leur faisions la leçon. Je parlais, à la fois, à quarante ou cinquante auditeurs, tout comme je vous parle aujourd’hui, à vous trois. Je leur parlais des livres écrits par les hommes qui avaient vécu avant eux ; parfois aussi de ceux écrits à cette époque même.\par
— Et c’est là tout ce que tu faisais ? interrogea Hou-Hou. Parler, parler, parler, et rien d’autre. Qui donc chassait pour la viande ? Qui tirait le lait des chèvres ? Qui pêchait le poisson ?\par
— Bravo, Hou-Hou ! La question que tu me poses est tout à fait sensée. Eh bien, la nourriture, comme je te l’ai déjà dit, était pourtant très abondante. Car nous étions des hommes très sages. Quelques-uns s’occupaient spécialement de cette nourriture et les autres, pendant ce temps, vaquaient à d’autres occupations. Moi, je parlais, je parlais constamment. Et, en échange, on me donnait mon manger. Un manger copieux et délicat. Oh ! oui, délicat ! Jamais, depuis soixante ans, je n’en ai goûté de semblable, et sûrement je n’en goûterai jamais plus. J’ai souvent songé que l’œuvre la plus magnifique de notre ancienne civilisation était cette abondance de nourriture, sa variété infinie et son raffinement incroyable. Oh ! mes enfants ! La vie, oui, valait alors la peine d’être vécue, quand nous avions de si bonnes choses à manger !\par
Les gamins continuaient à écouter attentivement. Et tout ce qu’ils ne comprenaient pas ils le mettaient au compte du radotage sénile du vieillard.\par
— Nous appelions, en théorie, ceux qui produisaient la nourriture des hommes libres. Il n’en était rien et leur liberté n’était qu’un mot. La classe dirigeante possédait la terre et les machines. C’est pour elle que peinaient les producteurs, et du fruit de leur travail nous leur laissions juste assez pour qu’ils puissent travailler et produire toujours davantage.\par
— Quand j’ai été chercher de la nourriture dans la forêt, déclara Bec-de-Lièvre, si quelqu’un prétendait me l’enlever et se l’approprier, je le tuerais !\par
Le vieux éclata de rire,\par
— Mais puisque la terre, la forêt, les machines, tout nous appartenait, à nous qui étions la classe dirigeante, comment le travailleur aurait-il pu refuser de produire pour nous ? Il serait lui-même mort de faim. Voilà pourquoi il préférait besogner, assurer notre manger, nous faire nos vêtements et nous fournir mille et une coquilles de moules, Hou-Hou ! mille délices et plaisantes satisfactions. Ha ! ha ! ha ! Or donc, en ce temps, j’étais le professeur Smith – James Howard Smith. Mon cours était très fréquenté. Ce qui veut dire que beaucoup de jeunes gens, beaucoup de jeunes filles aimaient à m’entendre parler des livres écrits par d’autres hommes. J’étais très heureux. Ma nourriture était excellente. J’avais les mains douces, car elles ne se livraient à aucun dur travail. Mon corps était propre et bien entretenu, et mes habits on ne peut plus souples et agréables à porter.\par
Ici l’ancêtre laissa tomber sur sa peau de bique, toute galeuse, un regard de dégoût.\par
— Tels n’étaient point nos vêtements. Même nos travailleurs-esclaves en portaient de meilleurs. Et nos soins corporels étaient grands. Nous nous lavions la figure et les mains plusieurs fois par jour. Hein ? qu’en dites-vous, vous autres, qui ne vous lavez jamais, sinon quand vous tombez dans l’eau ou quand vous vous exercez à nager ?\par
— Toi non plus, tu ne te laves jamais ! riposta Hou-Hou.\par
— Je le sais, je le sais bien. Je suis aujourd’hui un vieux dégoûtant. Mais les temps ont changé. Personne ne se lave maintenant. On n’en a plus les moyens. Voici soixante ans que je n’ai vu un morceau de savon. Vous ne savez pas ce que c’est que du savon ? Je ne perdrai pas mon temps à vous l’apprendre, puisque c’est l’histoire de la Mort Écarlate que je suis en train de vous raconter… Vous connaissez ce qu’est une maladie. Autrefois on disait une « infection ». Il était admis que les maladies provenaient de germes malfaisants. J’ai dit « germe ». Retenez bien ce mot. Un germe est quelque chose de tout petit. De plus petit encore que les tiques qui s’accrochent, au printemps, au poil des chiens et à leur chair, lorsqu’ils courent dans la forêt. Oui, un germe est beaucoup plus petit, si petit qu’on ne peut le voir.\par
Hou-Hou s’esclaffa :\par
— Tu es drôle, grand-père, tu nous parles de choses que l’on ne peut pas voir. Mais alors comment sait-on qu’elles existent ? Ça n’a pas de bon sens.\par
— Bien, très bien ! Hou-Hou, excellente question que la tienne. Apprends donc que pour voir ces choses, et bien d’autres encore, nous possédions des instruments appelés « microscopes ». Microscopes, entends-tu bien ?… Microscopes et « ultramicroscopes ». Grâce à ces instruments que nous approchions de nos yeux, les objets nous apparaissaient plus grands qu’ils ne sont en réalité. Et nous percevions ainsi ceux même dont nous ignorions l’existence. Les meilleurs de ces ultramicroscopes grossissaient un germe quarante mille fois. Quarante mille, c’est-à-dire quarante coquilles de moules, qui représentent elles-mêmes mille doigts… Puis, à l’aide d’un second instrument que nous appelions le cinématographe, oui « ci-né-ma-to-gra-phe », ces germes, déjà grossis quarante mille fois, nous apparaissaient grandis des milliers et des milliers de fois encore. Prenez un grain de sable, mes enfants ! Partagez-le en dix. Puis prenez un de ces dix morceaux et brisez-le encore en dix. Puis un de ces dix partagés derechef en dix. Puis de ces dix en dix toujours. Continuez ainsi toute la journée et peut-être au coucher du soleil, aurez-vous atteint à la petitesse d’un de ces germes.\par
Les jeunes garçons paraissaient incrédules. Bec-de-Lièvre poussait des reniflements moqueurs et Hou-Hou ricanait sous cape. Edwin les fit taire et le vieux reprit :\par
— La tique des bois suce le sang des chiens. Mais le germe, grâce à sa petitesse extrême, pénètre discrètement dans le sang du corps et s’y multiplie à l’infini. Dans le corps d’un seul homme, il y avait, en ce temps-là, un milliard de germes. Un milliard… une carapace de crabes, s’il vous plaît ! Ces germes nous les appelions des microbes. Des « microbes ». Parfaitement. Et quand un homme en avait un milliard dans le sang, on disait qu’il était « infecté », qu’il était malade, si vous préférez. Ces microbes étaient de plusieurs espèces. Celles-ci étaient innombrables comme les grains de sable de ce rivage. Nous ne les connaissions pas toutes. Nous savions très peu de choses de ce monde invisible. Nous connaissions bien le \emph{bacillus anthracis} et encore le \emph{micrococcus}, le \emph{bacterium termo} et le \emph{bacterium lactis}. C’est celui-ci, soit dit en passant, qui continue à faire tourner le lait de chèvre, pour en faire du fromage. Tu me suis bien, Bec-de-Lièvre. Que dirai-je des \emph{schizomicètes}, dont la famille n’en finit pas ? J’en passe et des meilleurs…\par
Ici le vieillard se noya dans une longue dissertation sur les germes et sur leur nature. Il se servait de mots d’une telle longueur et de phrases si compliquées que les gamins se regardèrent en faisant la grimace et que, reportant leurs yeux sur l’immense océan, ils laissèrent l’ex-professeur Smith pérorer tout à son aise.\par
À la fin, Edwin lui tira le bras et suggéra :\par
— Et la Mort Écarlate, grand-père ?\par
L’ancêtre sursauta et, de sa chaire de l’Université de Berkeley, où il s’imaginait pontifier encore, devant un tout autre auditoire, il revint brusquement à la réalité de sa situation présente.\par
— Oui, oui, Edwin, dit-il, j’avais oublié. Parfois la mémoire du passé remonte en moi, si puissamment, que je me prends à oublier que je suis un très vieil homme sale, vêtu d’une peau de bique, errant avec mes petits-fils sauvages, eux-mêmes bergers dans un monde primitif et solitaire. « \emph{Le travail de l’homme est éphémère et s’évanouit comme l’écume de la mer}… » Ainsi s’est évanouie notre grandiose et colossale civilisation. Et je suis aujourd’hui l’ancêtre, je suis un vieillard très las, j’appartiens à la tribu actuelle des Santa-Rosa. C’est dans cette tribu que je me suis marié. Mes fils et mes filles se sont mariés à leur tour, soit dans la Tribu des Chauffeurs, soit dans celle des Sacramentos, ou dans celle encore des Palo-Altos. Toi, Bec-de-Lièvre, tu appartiens aux Chauffeurs. Toi, Edwin, aux Sacramentos. Toi, Hou-Hou, aux Palo-Altos. Et vous êtes tous trois mes petits-fils… Mais je voulais vous parler de la Mort Écarlate. Où en étais-je donc de mon récit ?\par
— Tu nous parlais des germes, répondit vivement Edwin, de ces toutes petites choses que l’on ne peut voir et qui rendent les hommes malades.\par
— Oui, c’est bien là que j’en étais. Aux premiers âges du monde, lorsqu’il y avait très peu d’hommes sur la terre, il n’existait que peu de ces germes et, par suite, peu de maladies. Mais, à mesure que les hommes devenaient plus nombreux et se rassemblaient dans les grandes villes, pour y vivre tous ensemble, pressés et serrés, de nouvelles espèces de germes pénétraient dans leur corps, et des maladies inconnues apparurent, qui étaient de plus en plus terribles. C’est ainsi que, bien avant mon temps, à l’époque que l’on nomme le Moyen Âge, il y eut la Peste Noire qui balaya l’Europe. Puis vint la Tuberculose, la Peste Bubonique. En Afrique, il y eut la Maladie du Sommeil. Les bactériologistes s’attaquaient à toutes ces maladies et les détruisaient. Comme vous, enfants, vous éloignez les loups de vos chèvres ou écrasez les moustiques qui s’abattent sur vous. Les bactériologistes…\par
— Comment dis-tu, grand-père ?… interrompit Edwin.\par
— « Bac-té-rio-lo-gis-tes »… Ta tâche, Edwin, est de garder les chèvres. Tu les surveilles tout le jour et tu connais beaucoup de choses les concernant. Un bac-té-rio-lo-giste est celui qui surveille les germes, les étudie et, quand il le faut, se bat avec eux et les détruit, comme tu le fais des loups. Mais, pas plus que toi, ils ne réussissent toujours. C’est ainsi qu’il y avait un mal affreux, appelé la « Lèpre ». Un siècle — cent ans — avant ma naissance, les bactériologistes ont découvert le germe de la Lèpre. Ils le connaissaient tout à fait bien. Ils l’ont dessiné, et j’ai vu ces dessins. Ils n’ont pas trouvé pourtant le moyen de le tuer. En 1894, survint la Peste Pantoblast. Elle éclata dans un pays nommé le Brésil, et fit périr des milliers de gens. Les bactériologistes en découvrirent le germe, réussirent à le tuer, et la Peste Pantoblast n’alla pas plus loin. Ils fabriquèrent ce qu’on appelait un « sérum », un liquide qu’ils introduisaient dans le corps humain et qui détruisait le germe du pantoblast, sans tuer l’homme. En 1947, ç’avait été un mal étrange, qui s’attaquait aux enfants âgés de dix mois et au-dessous, et qui les rendait incapables de mouvoir leurs mains ni leurs pieds, de manger et de faire quoi que ce fût. Les bactériologistes furent onze ans avant de trouver ce germe bizarre, de le pouvoir tuer et de sauver les bébés. En dépit de ces maladies et de leurs ravages, le monde continuait à croître, et toujours davantage les hommes se massaient dans les grandes villes. Dès 1929, un illustre savant, nommé Soldervetzsky, avait annoncé qu’une grande maladie, mille fois plus mortelle que toutes celles qui l’avaient précédée, arriverait un jour, qui tuerait les hommes par milliers et par milliards. Car la fécondité des alliances, ainsi disait-il, est sans fin…\par
Ici Bec-de-Lièvre se mit sur ses pieds et, avec une moue méprisante, déclara :\par
— Tu radotes, grand-père ! Veux-tu, oui ou non, nous parler de la Mort Rouge ? Si tu ne veux pas, il faut le dire, et nous regagnerons le campement !\par
Le vieux, froissé de se voir ainsi interpellé, se remit à pleurer silencieusement. De grosses larmes roulèrent lentement dans les rides de ses joues. Sa mine douloureuse trahissait toute la décrépitude physique et morale de ses quatre-vingts ans.\par
— Voyons, Bec-de-Lièvre, rassieds-toi, dit Edwin. Grand-père parle bien. Et il va justement arriver à la Mort Écarlate. Il va tout de suite nous la raconter… N’est-ce pas grand-père ? Un peu de patience, Bec-de-Lièvre.\par
\bigbreak
\chapterclose


\chapteropen

\chapter[{III. La peste écarlate}]{III. La peste écarlate}
\renewcommand{\leftmark}{III. La peste écarlate}


\chaptercont
\noindent Le vieillard essuya ses larmes, de ses doigts crasseux. Puis il reprit son récit, d’une voix chevrotante, qui devint plus ferme, à mesure qu’il s’animait au cours de son récit.\par
— Ce fut pendant l’été de 2013 que se déclara la Peste Écarlate…\par
Bec-de-Lièvre manifesta bruyamment sa joie, en battant des mains.\par
— … J’avais vingt-sept ans. Des télégrammes…\par
Bec-de-Lièvre fronça le sourcil.\par
— Des quoi ? demanda-t-il. Encore des mots qu’on ne comprend pas…\par
Edwin le fit taire et l’ancêtre continua :\par
— En ce temps-là, les hommes parlaient entre eux, à travers l’espace, à des milliers et des milliers de milliers de milles de distance. C’est ainsi que la nouvelle arriva à San Francisco qu’un mal inconnu s’était déclaré à New-York. Dans cette ville, la plus magnifique de toute l’Amérique, vivaient dix-sept millions de personnes. Tout d’abord, on ne s’alarma pas outre mesure. Il n’y avait eu que quelques morts. Les décès cependant avaient été très prompts, paraît-il. Un des premiers signes de cette maladie était que la figure et tout le corps de celui qui en était atteint devenaient rouges.\par
« Au cours des vingt-quatre heures qui suivirent, on apprit qu’un cas s’était déclaré à Chicago, une autre grande ville. Et, le même jour, la nouvelle fut publiée que Londres, la plus grande ville du monde après New-York et Chicago, luttait secrètement contre ce mal, depuis deux semaines déjà. Les nouvelles en avaient été censurées… je veux dire que l’on avait empêché qu’elles se répandissent dans le reste du monde.\par
« Cela semblait grave, évidemment. Mais nous autres, en Californie, et il en était partout de même, nous n’en fûmes pas affolés. Il n’y avait personne qui ne fût assuré que les bactériologistes trouveraient le moyen d’annihiler ce nouveau germe, tout comme ils l’avaient fait, dans le passé pour d’autres germes.\par
« Ce qui était pourtant inquiétant, c’était la prodigieuse rapidité avec laquelle ce germe détruisait les humains, et aussi que quiconque était atteint mourait infailliblement. Pas une guérison. On avait déjà connu la Fièvre Jaune, une vieille maladie qui, elle non plus, n’était pas tendre. Le soir vous étiez attablé avec une personne en bonne santé et, le lendemain, si vous étiez assez tôt levé, vous pouviez voir passer sous vos fenêtres le corbillard qui emportait votre convive de la veille.\par
« La Peste nouvelle était plus expéditive encore. Elle tuait beaucoup plus vite. Souvent une heure ne s’écoulait pas entre les premiers signes de la maladie et la mort. Parfois on traînait pendant plusieurs heures. Mais parfois aussi, dix ou quinze minutes après les premiers symptômes, tout était terminé.\par
« Le cœur, tout d’abord, accélérait ses battements et la température du corps s’élevait. Puis une éruption, d’un rouge violent, s’étendait comme un érésipèle, sur la figure et sur le corps. Beaucoup de gens ne se rendaient pas compte de l’accélération du cœur ni de la hausse de leur température. Ils n’étaient avertis qu’au moment où l’éruption se manifestait.\par
« Des convulsions accompagnaient d’ordinaire cette première phase de la maladie. Mais elles ne semblaient pas graves et, après leur passage, celui qui les avait surmontées redevenait soudain très calme. C’était maintenant une sorte d’engourdissement qui l’envahissait. Il montait du pied et du talon, puis gagnait les jambes, les genoux, les cuisses et le ventre, et montait toujours. Au moment même où il atteignait le cœur, c’était la mort.\par
« Aucun malaise, ni délire n’accompagnaient cet engourdissement progressif. L’esprit restait clair et net, jusqu’à l’instant où le cœur se paralysait et cessait de battre. Et ce qui était non moins surprenant, c’était, après la mort, la rapidité de décomposition de la victime. Tandis que vous la regardiez, sa chair semblait se désagréger, se dissoudre en bouillie.\par
« Ce fut une des raisons de la rapidité de la contagion. Les milliards de germes du cadavre se retrouvaient en liberté instantanément. Dans ces conditions, toute lutte de la science était vaine. Les bactériologistes périssaient dans leurs laboratoires, à l’instant même où ils commençaient l’étude de la Peste Écarlate. Ces savants étaient des héros. Dès qu’ils tombaient, d’autres se levaient pour prendre leur place.\par
« Un savant anglais réussit, à Londres, le premier, à isoler le germe. La nouvelle en fut télégraphiée partout et chacun se mit à espérer. Mais Trask (c’était le nom de ce savant) mourut dans les trente heures qui suivirent. Le fameux germe était trouvé cependant, et tous les laboratoires luttèrent d’ardeur, afin de découvrir le germe contraire qui tuerait celui de la Peste Écarlate. Tant d’efforts échouèrent. »\par
Bec-de-Lièvre, ici, interrompit :\par
— Les hommes de votre temps étaient fous, grand-père ! Ces germes étaient invisibles, avez-vous dit ? et ils prétendaient les combattre avec d’autres germes, invisibles eux aussi… C’est bien pour cela qu’ils sont morts… Lutter contre ce qu’on ne sait pas, à l’aide de ce qu’on ignore ! En voilà des sornettes !\par
L’ancêtre, aussitôt, rouvrit la fontaine de ses pleurs. Edwin se hâta de le consoler et de morigéner Bec-de-Lièvre.\par
— Écoute-moi un peu ! dit-il à celui-ci. Tu crois bien toi à des tas de choses que tu ne peux voir…\par
Et, comme Bec-de-Lièvre secouait la tête :\par
— Parfaitement, poursuivit-il. Tu crois aux morts qui marchent. Et tu n’en as jamais vu se promener…\par
Bec-de-Lièvre protesta :\par
— Si ! Si ! J’en ai vu errer, l’hiver dernier, lorsque j’étais avec papa à la chasse aux loups.\par
— Je l’admets… concéda Edwin. Mais tu ne nieras pas que tu craches toujours dans l’eau, chaque fois que tu traverses une rivière ou un torrent ?\par
— Soit ! C’est pour éloigner de moi le Mauvais Sort.\par
— Tu crois donc au Mauvais Sort ?\par
— Certainement.\par
Edwin conclut victorieusement :\par
— Peux-tu me dire où tu l’as jamais vu le Mauvais Sort ? Nulle part n’est-ce pas ? Tu es donc tout pareil à grand-père avec ses germes. Tu crois à des choses que tu ne vois pas… Continue, grand-père.\par
Bec-de-Lièvre, fort mortifié par ce raisonnement topique, demeura penaud et ne répondit rien. L’aïeul reprit la parole. Maintes fois encore il fut interrompu par les questions des enfants et par leurs disputes, tandis qu’ils se jetaient de l’un à l’autre leurs doutes et leurs objections, s’efforçant de suivre l’aïeul dans ce monde évanoui, qui leur était inconnu. Mais, afin d’alléger ce récit, nous ne ferons pas comme les enfants et ne le couperons plus de leurs réflexions.\par
— La Mort Écarlate, contait l’aïeul, fit un jour son apparition à San Francisco. Le premier décès, je m’en souviens encore, survint un lundi matin. Le lendemain mardi, les hommes tombaient comme des mouches à San Francisco et à Oakland.\par
« On mourait partout. Dans son lit, à son travail, en marchant dans la rue. Le jeudi, je fus, pour la première fois, témoin d’une de ces morts foudroyantes. Miss Collbran, une étudiante de mes élèves, était assise devant moi, dans la salle du cours. Tandis que je parlais, je remarquai soudain que son visage devenait écarlate.\par
« Je m’arrêtai de parler et me mis à la fixer. Tous les autres élèves firent comme moi. Car nous savions dès lors que le terrible fléau venait de s’introduire parmi nous. Les jeunes femmes, épouvantées, se prirent à crier et se précipitèrent hors de la salle. Puis les jeunes gens sortirent à leur tour, sauf deux.\par
« Miss Collbran fut saisie de quelques menues convulsions, qui ne durèrent pas plus d’une minute. Un des jeunes gens lui porta un verre d’eau. Elle le prit, en but quelques gouttes et s’écria :\par
— Mes pieds ! Je ne sens plus mes pieds !\par
« Un instant après elle ajouta :\par
— Je n’ai plus de pieds… ou du moins j’ignore si je les ai encore… Mes genoux maintenant sont froids ! Je ne sens plus mes genoux.\par
« Elle s’était étendue sur le parquet, un petit tas de livres et de cahiers sous la tête. Nous ne pouvions rien faire pour elle. L’engourdissement et le froid gagnaient la ceinture, puis le cœur. Et, quand il eut atteint le cœur, elle mourut.\par
« J’avais observé l’heure à l’horloge. En quinze minutes elle était morte. Là, dans ma propre classe. Morte ! C’était, à l’instant d’avant, une jeune femme pleine de vie et de santé, une robuste et belle fille. Et quinze minutes, oui, pas plus, s’étaient écoulées entre le premier symptôme du mal et le dénouement.\par
« Tandis que, durant ce quart d’heure, j’étais demeuré dans ma classe avec la moribonde, l’alarme avait été donnée dans l’Université. Partout les étudiants, nombreux de plus d’un millier, avaient fui les salles de cours et les laboratoires. Quand je sortis, afin d’aller présenter mon rapport au Président de la Faculté, je trouvai devant moi le désert. Seuls quelques traînards traversaient encore les cours intérieures pour s’enfuir chez eux. Certains couraient.\par
« Je trouvai le Président Hoag dans son bureau, seul et pensif. Il me parut plus vieux et plus blanchi, avec les rides de sa figure qui se marquaient d’une façon anormale.\par
« Quand il m’aperçut, il parut revenir à lui, se leva, et se dirigea en titubant vers la porte de son bureau qui était opposée à celle par où j’étais entré. Il sortit, fit claquer cette porte derrière lui, il en ferma à clef la serrure.\par
« Il savait, vous comprenez bien, que j’avais été exposé à la contagion, et il prenait peur. À travers la porte il me cria de m’en aller. Je fis ainsi et jamais je n’oublierai la sensation terrible que j’éprouvai en retraversant les cours et les corridors déserts. Ce n’était pas que je craignisse. J’avais été exposé et déjà je me considérais comme mort.\par
« Mais devant cet arrêt soudain de l’existence, dont j’avais été témoin autour de moi, il me semblait que j’assistais à la fin du monde. Cette Université avait été ma vie, ma raison d’être. Mon père y avait été professeur avant moi, et son père avant lui. Moi, j’y avais fait toute ma carrière, à laquelle, en naissant, j’étais prédestiné. Depuis un siècle et demi cette immense maison avait toujours marché sans arrêt aucun, comme une machine merveilleuse. Et maintenant, tout à coup, elle avait cessé de vivre. Le flambeau, trois fois sacré, de mon autel s’était éteint. J’étais anéanti d’horreur, d’une horreur inexprimable.\par
« Je rentrai chez moi. Dès qu’elle me vit, ma gouvernante se mit à hurler et prit la fuite. Je sonnai la femme de chambre. Personne ne vint. Elle était partie, elle aussi. Je fis le tour de la maison et trouvai, dans la cuisine, la cuisinière qui préparait sa valise. Elle poussa de grands cris à mon aspect et se sauva en laissant tomber la valise, avec tous ses effets personnels. Elle traversa la propriété en courant et en criant toujours. Aujourd’hui encore j’ai ses cris dans l’oreille.\par
« Ce n’était pas l’usage, mes enfants, vous le comprenez comme moi, d’agir ainsi, en temps ordinaire, avec les malades. Non ! on ne s’affolait pas de la sorte. On envoyait chercher les docteurs et les infirmières, qui vous appliquaient très calmement un traitement approprié. Ici le cas était différent. Le mal tuait sans manquer son coup. Il n’y eut pas un seul exemple de guérison.\par
« Je me trouvai seul dans la maison, qui était fort vaste. J’y attendais le retour de mon frère, lorsque résonna la sonnerie du téléphone. En ce temps-là, je vous l’ai dit, les hommes pouvaient à distance communiquer entre eux, à l’aide de fils qui couraient en l’air ou dans le sol, ou même sans fils. C’était mon frère qui me parlait. Il me disait qu’il ne rentrerait pas à la maison, de peur de se contaminer à mon contact, et qu’il avait conduit mes deux sœurs chez le professeur Bacon, mon collègue. Il me conseillait de demeurer tranquille au logis, jusqu’à ce que je connusse si, oui ou non, j’avais gagné la Peste.\par
« Je ne disconvins pas qu’il eût raison et restai chez moi. Comme j’avais faim, j’essayai, pour la première fois dans ma vie, de me faire un peu de cuisine. La Peste ne se déclarait pas. Par le téléphone, je pouvais causer avec qui je voulais et connaître les nouvelles du dehors. Je pouvais également communiquer avec le monde extérieur par le truchement des journaux. Je donnai l’ordre qu’on m’en lançât des paquets, par-dessus la grille d’entrée de la propriété.\par
« Je sus ainsi que New-York et Chicago étaient en plein chaos. Il en était de même dans toutes les grandes villes. Le tiers des policemen de New-York avait déjà succombé. Le Chef de la Police et le Maire étaient morts. Tout ordre social, toute loi avaient disparu. Les corps restaient étendus dans les rues, là où ils étaient tombés, sans sépulture. Les trains et les navires, qui transportaient coutumièrement, jusqu’aux grandes villes, les vivres et toutes choses nécessaires à la vie ne fonctionnaient plus, et les populaces affamées pillaient les boutiques et les entrepôts.\par
« Partout régnaient le meurtre, le vol et l’ivresse. Des millions de personnes avaient déjà déserté New-York, comme les autres villes. Les riches, d’abord, étaient partis, dans leurs autos, leurs avions et leurs dirigeables. Les masses avaient suivi, à pied, ou en véhicules de louage ou volés, portant la Peste avec elles à travers les campagnes, pillant et affamant sur leur passage les petites villes, les villages et les fermes qu’elles rencontraient.\par
« L’homme qui, de New-York, expédiait ces nouvelles à travers l’Amérique, l’opérateur du télégraphe sans fil, était seul, avec son instrument au faîte d’une tour élevée. Il annonçait que les quelques habitants demeurés dans la ville, une centaine de mille environ, étaient comme fous, de terreur et d’ivresse, et que, tout autour de lui, s’élevaient de grands feux dévastateurs. Cet homme, resté par devoir à son poste, quelque obscur journaliste sans doute, fut, comme les savants penchés sur leurs éprouvettes, un héros.\par
« Depuis vingt-quatre heures, annonçait-il, pas un aéroplane, pas un transatlantique n’était plus arrivé d’Europe ; plus même un message. Le dernier qui lui fût parvenu venait de Berlin, une ville d’un pays nommé l’Allemagne. Il disait qu’un illustre bactériologiste, nommé Hoffmeyer, avait découvert enfin le sérum de la Peste Écarlate. Ce fut la dernière nouvelle qui nous parvint d’Europe.\par
« Ce qui est en tous cas certain, c’est que cette découverte était venue trop tard, pour l’Europe comme pour nous. Sans quoi les derniers survivants américains n’auraient pas manqué de voir arriver un jour, de l’Ancien Monde, quelques explorateurs curieux, désireux de se rendre compte de ce que nous étions devenus. Il paraissait évident que le fléau avait fait une semblable extermination de l’humanité, dans l’un comme dans l’autre hémisphère, et que quelques vingtaines d’hommes, là-bas comme ici, avaient seuls survécu.\par
« Durant un jour encore, les sans-fils de New-York nous parvinrent. Puis ils firent défaut. L’homme qui les expédiait, perché sur sa tour, était mort sans doute de la Peste Écarlate, à moins qu’il n’eût été consumé par cet immense incendie que lui-même avait décrit et qui dévastait tout autour de lui.\par
« Ce qui s’était produit à New-York avait eu lieu de façon identique à San Francisco et dans sa banlieue. Dès le mardi, les gens mouraient si rapidement que les survivants ne pouvaient plus prendre soin des cadavres, qui gisaient partout. Au cours de la nuit suivante ce fut la panique, et l’exode commença vers les campagnes.\par
« Imaginez-vous, mes enfants, des troupes d’hommes plus nombreuses que des bandes de saumons que vous avez vues souvent remonter le fleuve Sacramento, des troupes d’hommes que dégorgeaient les villes, qui, comme des bandes de fous, se déversaient sur les campagnes, dans un inutile effort pour fuir la mort qui s’attachait à leurs pas.\par
« Car ils emportaient les germes avec eux, ces germes invisibles, mes chers enfants, dont je vous parlais tout à l’heure. Même les aéroplanes des riches, qui fuyaient vers les montagnes et vers les déserts, espérant y trouver la sécurité, les transportaient sur leurs ailes.\par
« Des centaines de ces aéroplanes s’enfuirent vers Hawaï. Ils y trouvèrent la Peste déjà installée. Cela encore nous l’apprîmes par les dépêches des sans-fils, jusqu’au moment où il ne resta plus d’opérateurs dans les postes pour recevoir et expédier les messages. Il y avait de la stupeur dans ce manque progressif de communications avec le reste du monde. Il semblait que le monde lui-même cessait d’exister, qu’il s’évanouissait et disparaissait.\par
« Voilà soixante ans qu’il a cessé d’exister pour moi. Je sais qu’il doit y avoir des territoires qui furent New-York, l’Europe, l’Asie et l’Afrique. Mais jamais plus, depuis soixante ans, je n’en ai entendu parler. Ce fut un écroulement total, absolu. Dix mille années de culture et de civilisation s’évaporèrent comme l’écume, en un clin d’œil. »
\chapterclose


\chapteropen

\chapter[{IV. Dans les tourbillons de flammes}]{IV. Dans les tourbillons de flammes}
\renewcommand{\leftmark}{IV. Dans les tourbillons de flammes}


\chaptercont
\noindent « Je vous parlais tout à l’heure des aéroplanes des riches, qui emportaient la Peste sur leurs ailes, en sorte que ces riches mouraient comme les autres hommes. Un seul d’entre eux survécut, à ma connaissance, et c’est lui qui épousa Mary, ma fille bien-aimée.\par
« Il vint à la Tribu des Santa Rosa, huit années après le désastre. Il avait alors dix-neuf ans, et il dut en attendre douze, avant de pouvoir se marier. Car il n’y avait aucune femme qui fût libre et la plupart des jeunes filles, tant soit peu avancées en âge, étaient fiancées déjà. C’est pourquoi il lui fallut attendre que ma fille Marie eut atteint ses seize ans. Court-Toujours, un de ses fils, votre cousin, a été pris, l’an dernier, par le lion de la montagne, vous vous souvenez…\par
« L’homme en question, qui devint mon gendre, avait onze ans, au moment de la Peste. Il se nommait Mungerson. Son père était un des Magnats de l’Industrie. C’était un homme riche et puissant. Sur son grand avion, le \emph{Condor}, toute la famille s’était envolée vers les solitudes de la Colombie Britannique, qui se trouve très loin vers le nord.\par
« Une panne survint, et l’avion s’abattit sur le Mont Shasta. Vous avez entendu parler de cette montagne, qui est vers le nord… La Peste Écarlate s’y déclara dans la famille et seul survécut ce garçon de onze ans.\par
« Huit ans durant, il vécut seul, errant sur la terre déserte et tentant en vain de rencontrer un être de son espèce. À force de marcher vers le Sud, il trouva un jour la Tribu des Santa Rosa et se raccrocha à nous…\par
« Mais je m’aperçois que je vais trop vite dans mon récit et que j’anticipe sur les événements.\par
« Je reviens au moment où débutait cet immense exode des grandes villes et où, isolé chez moi, je communiquais encore par téléphone avec mon frère. Je lui disais qu’il n’y avait en moi aucun symptôme de la Peste et que le mieux que nous avions à faire était de nous réunir, pour nous isoler dans un local sûr. Nous convînmes finalement de nous retrouver dans le bâtiment de l’Université qui était affecté à l’École de Chimie. Là nous emporterions avec nous une réserve de provisions. Puis, nous étant solidement barricadés, nous empêcherions, fût-ce par la force des armes, qui que ce soit de nous imposer sa présence et nous attendrions les événements.\par
« Ce plan arrêté, mon frère me supplia de demeurer encore vingt-quatre heures au moins dans ma maison, afin que la certitude que j’étais indemne fût absolue. J’y consentis et il me promit de venir me chercher le lendemain.\par
« Nous étions en train de causer des détails de notre approvisionnement et comment nous organiserions la défense de l’École de Chimie, lorsque le téléphone mourut. Il mourut tandis que nous parlions. Le soir, il n’y eut plus de lumière électrique et je restai seul dans ma maison, au milieu des ténèbres.\par
« Comme l’impression de tous journaux avait cessé, j’ignorais tout ce qui se passait dehors. J’entendais seulement le bruit des émeutes, les détonations des coups de revolvers, et j’apercevais dans le ciel la lueur d’un grand incendie, dans la direction d’Oakland. Ce fut une nuit d’angoisse et je ne pus fermer l’œil un instant.\par
« Au cours de cette nuit, un individu, j’ignore exactement dans quelles conditions, fut tué sur le trottoir qui faisait face à celui de ma maison. J’entendis soudain les détonations rapides d’un pistolet automatique et, quelques minutes après, le malheureux homme, se traînant blessé jusqu’à ma porte, y sonnait en gémissant et en implorant du secours.\par
« M’armant moi-même de deux pistolets automatiques, je descendis et allai vers lui. Je l’examinai à la lumière d’une allumette, à travers la grille, et je constatai que, tandis qu’il se mourait de ses blessures, il était atteint également de la Peste Écarlate. Je rentrai rapidement chez moi et, pendant une demi-heure encore, je l’entendis se plaindre et crier au secours.\par
« Le matin venu, je vis mon frère arriver. J’avais placé dans un sac à main tous les menus objets de valeur que je désirais emporter avec moi. Mais, ayant regardé mon frère au visage, je compris qu’il ne me suivrait pas, il avait la Peste.\par
« Il me tendit sa main, pour y serrer la mienne. Je reculai avec effroi. Je lui commandai :\par
— Regarde-toi dans la glace.\par
« Il fit ainsi et, devant les flammes rouges qui lui incendiaient le visage et qui augmentaient d’intensité, à mesure qu’il se regardait, il se laissa tomber sur une chaise dans un spasme nerveux.\par
— Mon Dieu ! dit-il, je suis atteint ! Frère, ne m’approche pas… Je suis un homme mort.\par
« Alors les convulsions le saisirent. Il ne mourut qu’au bout de deux heures et, jusqu’au dernier moment, il garda sa pleine connaissance, envahi par la paralysie qui montait lentement jusqu’à son cœur.\par
« Quand il fut mort, je pris mon sac à main et je me mis en route vers l’École de Chimie. Le spectacle des rues était terrifiant. On y trébuchait partout sur les cadavres. Quelques-unes des victimes de la Peste n’étaient point encore mortes. On les voyait agoniser. Les incendies s’étendaient. Ce n’étaient encore que des feux isolés à Berkeley, mais la flamme balayait Oakland et San Francisco. La fumée obscurcissait le ciel et le plein milieu du jour ressemblait à un sombre crépuscule. Parfois, quand le vent sautait et poussait d’un côté ou d’autre ces fumées, le soleil perçait obscurément la brume et l’on y voyait poindre son globe, qui était d’un rouge terne. En vérité, mes enfants, c’était tout l’aspect de la fin du monde.\par
« Çà et là, de nombreuses automobiles étaient en panne, par suite du manque d’essence dans les garages et des fournitures nécessaires. Je me souviens notamment d’une de ces voitures, où un homme et une femme, renversés en arrière sur leurs sièges, étaient morts. À côté, deux autres femmes et un enfant étaient descendus sur le trottoir, et attendaient ils ne savaient quoi.\par
« Partout s’offraient aux regards des spectacles douloureux du même genre. Des hommes se coulaient furtivement le long des maisons, silencieux et pareils à des fantômes. Des femmes, au teint livide, portaient des bébés dans leur bras ; les pères conduisaient par la main les enfants plus grands, qui pouvaient marcher. Seuls, par couples ou en famille, tous les habitants fuyaient la Cité de la Mort. Les uns s’étaient chargés de provisions. D’autres portaient des couvertures. La plupart ne portaient rien.\par
« Je passai devant une épicerie… Une épicerie, c’était, mes enfants, un endroit où l’on vendait coutumièrement de la nourriture. L’homme à qui elle appartenait, et que je connaissais bien, était une tête dure, point méchante, mais obstinée. Il défendait furieusement l’accès de sa boutique. La porte et la devanture avaient été défoncées. Lui, retranché derrière son comptoir, déchargeait ses revolvers sur les pillards qui prétendaient entrer. Plusieurs cadavres étaient déjà couchés sur le parquet.\par
« Tandis que j’observais à distance, je vis un des pillards, qui avait été repoussé, briser la devanture d’un magasin voisin, où se vendaient des chaussures, et, après s’être servi, mettre le feu. Je n’allai au secours ni du marchand de chaussures ni de l’épicier. Le temps n’était plus où l’on se dévouait pour les autres. Chacun luttait pour soi.\par
« Tandis que j’allais rapidement, descendant une rue en pente, j’assistai à une autre tragédie. Deux vagues ouvriers avaient attaqué un homme et une femme bien mis, qui marchaient avec leurs enfants, et qu’ils prétendaient dévaliser. Celui que l’on assaillait ne m’était pas étranger, quoique je ne lui eusse jamais été présenté. C’était un poète connu dont, depuis longtemps, j’admirais les vers. J’hésitais à lui prêter main forte, lorsqu’un coup de revolver éclata, et je le vis s’effondrer sur le sol. Sa femme poussait des cris affreux. Une des deux brutes l’assomma d’un coup de poing. Je lançai des menaces aux bandits. Sur quoi ils déchargèrent leurs revolvers dans ma direction et je me hâtai de fuir, en tournant au premier coin.\par
« Mais là je fus arrêté par l’incendie. À droite et à gauche, les maisons brûlaient et la rue était pleine de flammes et de fumée. Quelque part, dans les rouges ténèbres, on entendait la voix perçante d’une femme, qui implorait du secours. Je ne m’inquiétai pas d’elle. Parmi tant de scènes semblables et tant d’appels déchirants, le cœur de l’homme le meilleur devenait dur comme la pierre.\par
« Revenant sur mes pas, je vis que les deux ouvriers assassins étaient partis. Le poète et sa femme étaient étendus morts sur le trottoir. C’était un spectacle horrible. Les deux enfants avaient disparu. Où avaient-ils été ? Je ne saurais le dire. Et je comprenais maintenant pourquoi ceux qui fuyaient se glissaient si furtivement le long des maisons, avec leurs pâles figures.\par
« En plein cœur de notre civilisation, dans ses bas-fonds et dans ses ghettos du travail, nous avions laissé croître une race de barbares, qui maintenant se retournaient contre nous, dans nos malheurs, comme des animaux sauvages, cherchant à nous dévorer.\par
« Ces brutes, d’ailleurs, se détruisaient aussi bien entre elles. Elles se brûlaient le corps avec des boissons fortes et s’abandonnaient à mille atrocités, se battant et s’entre-tuant en une immense démence.\par
« Je repris mon chemin et rencontrai une autre bande d’ouvriers, d’une meilleure trempe, qui s’étaient groupés, leurs femmes et leurs enfants au milieu d’eux, les vieux et les malades portés sur des civières, et qui se frayaient ainsi un chemin hors de la ville, accompagnés d’un camion de provisions, tiré par des chevaux.\par
« Je ne pus m’empêcher d’admirer l’ordre de leur marche, quoiqu’ils me tirassent dessus, lorsqu’ils passèrent près de moi. Un des leurs me cria qu’ils tuaient sur leur chemin tous les détrousseurs et tous les voleurs qu’ils rencontraient, seule façon de se défendre efficacement eux-mêmes.\par
« Alors se passa une scène que je devais voir se renouveler plus d’une fois. Un des hommes du groupe se révéla soudainement marqué d’un signe infaillible de la Peste. Tous ceux qui se trouvaient près de lui s’écartèrent aussitôt. Et lui, sans s’irriter, sortit du rang et les laissa continuer leur route.\par
« Une femme, sa femme très probablement, qui conduisait par la main un petit garçon, prétendit ne point le quitter. Mais l’homme lui commanda de poursuivre, tandis que les autres hommes, se saisissant d’elle, l’empêchaient de s’éloigner et l’entraînaient. J’ai vu cela et j’ai vu le mari, dont la figure flamboyait d’écarlate, se retirer sous une porte de la rue. Puis j’entendis détoner son revolver et il tomba mort sur le sol.\par
« Après avoir été contraint par l’incendie de rebrousser chemin à deux reprises, je réussis à gagner l’Université.\par
« En pénétrant dans la grande cour, je me heurtai à un groupe d’universitaires qui se dirigeaient eux aussi vers l’École de Chimie, tous chefs de famille, et qu’accompagnaient leurs proches, y compris les nurses et les domestiques.\par
« Le Professeur Badminton me salua et j’eus quelque peine à le reconnaître. Il avait passé à travers les flammes d’un incendie et sa barbe avait roussi. Il portait autour de la tête un bandage taché de sang et ses vêtements étaient tout souillés. Il me conta qu’il avait été cruellement malmené par des rôdeurs et que, la nuit précédente, son frère avait été tué, tandis que tous deux défendaient leurs biens.\par
« À mi-route dans la cour, il désigna soudain, de la main, le visage de Mistress Swinton. L’infaillible signe de la Peste y était marqué. Aussitôt toutes les femmes présentes se mirent à crier et coururent loin d’elle. Ses deux enfants, accompagnés chacun par une nurse, se sauvèrent aussi, et les nurses avec eux. Mais son mari, le Docteur Swinton, resta avec elle.\par
— Continuez votre chemin, Smith, me dit-il. Prenez soin des enfants, moi je demeurerai avec ma femme. Je n’ignore point que c’est déjà comme si elle était morte. Mais je ne puis l’abandonner. Lorsqu’elle aura expiré, et si je ne suis pas contaminé, j’irai vous retrouver dans l’École de Chimie. Surveillez mon arrivée et laissez-moi entrer.\par
« Je le quittai, penché sur sa femme, adoucissant par sa présence ses derniers moments, et je courus pour rejoindre notre groupe.\par
« Nous fûmes les derniers admis dans l’École. Les portes se refermèrent sur nous et, de nos carabines, nous veillâmes à écarter dès lors quiconque se présenterait. Le Docteur Swinton, lui-même, lorsqu’il se présenta, une heure après, ne fut point admis.\par
« Des places avaient été prévues dans ce refuge, pour une soixantaine de personnes. Mais chacun de ceux qui s’y étaient donné rendez-vous avait amené avec lui ses parents et ses amis, et des familles entières. En sorte que nous nous trouvions être plus de quatre cents. Les locaux étaient heureusement fort vastes et tout ce monde y était à l’aise. En outre, l’École étant complètement isolée, il n’y avait pas à y craindre les incendies qui faisaient rage par toute la ville.\par
« Nous avions réuni d’importantes provisions de bouche, qu’un comité fut chargé de répartir quotidiennement entre chaque famille ou chaque groupe, qui constituaient autant de tables. D’autres comités furent formés, pour des objets divers. Je fis partie du Comité de Défense.\par
« Le premier jour, aucun rôdeur ni pilleur n’approcha. Ils étaient nombreux cependant et nous apercevions, des fenêtres, la fumée de leurs feux de campements, qui étaient installés tout autour de l’École. L’ivrognerie régnait parmi ces bandits et nous les entendions, à tout moment, chanter des obscénités et hurler comme des fous. Tandis que le monde s’écroulait autour d’eux, dans l’asphyxie d’une atmosphère saturée de fumée, ils lâchaient la bride à leur bestialité, s’enivraient et s’entre-tuaient. Peut-être, au fond, avaient-ils raison ? Ils ne faisaient rien que de devancer la mort. Le bon et le méchant, le fort et le faible, celui qui aimait la vie et celui qui la maudissait, tous pareillement y passaient.\par
« Après vingt-quatre heures écoulées, nous constatâmes avec satisfaction qu’aucun symptôme de Peste ne s’était manifesté parmi nous et, pour avoir de l’eau, nous entreprîmes d’aménager un puits. Vous avez tous vu des débris de ces énormes tuyaux de fonte qui, au temps dont je vous parle, portaient l’eau aux habitants des villes. L’incendie en avait déjà fait éclater la plupart et les vastes réservoirs qui les alimentaient étaient taris. C’est pourquoi nous défonçâmes le dallage cimenté de la grande cour de l’École et creusâmes un puits. Il y avait avec nous beaucoup de jeunes hommes, des étudiants pour la plupart, et nous travaillâmes nuit et jour. Nos craintes étaient justifiées. Trois heures avant que notre puits fût terminé, le peu d’eau qui nous arrivait encore fit défaut.\par
« Une seconde période de vingt-quatre heures s’écoula et la Peste n’avait toujours pas fait son apparition parmi nous. Nous pensions que nous étions sauvés. Nous ignorions alors le nombre exact de jours de l’incubation du mal. Nous estimions, étant donné la rapidité avec laquelle il tuait, dès qu’il s’était manifesté, que son développement interne était non moins prompt. Aussi, après ces deux jours nous pouvions croire, de bonne foi, que la contagion nous avait épargnés. Mais le troisième jour nous apporta une cruelle désillusion.\par
« Durant la nuit qui le précéda et que je n’ai jamais oubliée, j’effectuai ma ronde de garde, de huit heures du soir à minuit. Des toits de l’École j’assistais à un spectacle inouï. Comme un volcan en activité, San Francisco lançait ses flammes et sa fumée. L’éruption grandissait d’heure en heure, enveloppant le ciel et la terre de sa lueur ardente. Son flamboiement était tel que toute la fumée en était maintenant illuminée et qu’on pouvait lire, à cet embrasement, les plus menus caractères d’imprimerie.\par
« Oakland, San Léonardo, Haywards réunissaient leurs brasiers et, vers le nord, de nouveaux feux surgissaient jusqu’à la Pointe de Richmond. Le monde s’abîmait dans un linceul de flammes. Les grandes poudrières de la Pointe Pinole sautèrent, en explosions successives et rapides, qui furent terribles. Quoique solidement construite, l’École en fut ébranlée de la base au faîte, comme par un tremblement de terre, et toutes ses vitres furent brisées.\par
« Je quittai alors les toits et, par les longs corridors, j’allai de chambre en chambre, expliquer ce qui s’était passé et rassurer les femmes alarmées.\par
« Une heure après, un grand vacarme s’éleva parmi les campements de pillards. On entendait des cris variés, cris menaçants et cris de protestation, entremêlés de coups de revolvers. Nous pensâmes immédiatement et avec raison que cette bataille avait eu pour cause la prétention de ces gens qui étaient sains, de chasser ceux qui étaient atteints par le fléau.\par
« Plusieurs de ceux qui avaient été ainsi renvoyés vinrent se présenter aux portes de l’École. Nous leur notifiâmes d’avoir à passer leur chemin. En réponse, ils nous accablèrent d’injures et nous tirèrent dessus. Le Professeur Merryweather, qui se trouvait à une des fenêtres du rez-de-chaussée, reçut, juste entre les deux yeux, une balle de pistolet qui le tua net.\par
« Nous ripostâmes par une fusillade et les agresseurs s’enfuirent, sauf trois dont une femme. La Peste les avait marqués déjà pour la mort, en sorte qu’ils ne craignaient point d’exposer leur vie. La face écarlate dans le reflet rouge du ciel, pareils à des démons impudiques, ils continuaient à nous injurier et à tirer sur nous.\par
« Moi-même je tuai l’un d’un coup de feu. Après quoi, l’autre homme et la femme s’étendirent sur le trottoir, en dessous de nos fenêtres, et nous dûmes assister à leur agonie.\par
« Notre situation devenait fort dangereuse. Par les fenêtres, démunies de vitres par les explosions, les germes de la Peste émanés de ces deux cadavres allaient entrer librement. Le Comité Sanitaire fut invité à prendre les mesures qui s’imposaient et il répondit noblement à sa tâche. Deux hommes furent désignés pour sortir de l’École et emporter les cadavres. C’était, pour eux, le sacrifice probable de leur vie. Car, leur besogne accomplie, ils ne devaient plus réintégrer notre refuge.\par
« Un des professeurs, qui était célibataire, et un étudiant, se présentèrent comme volontaires. Ils nous firent leurs adieux et nous quittèrent. Ceux-là aussi furent des héros ! Ils donnèrent leur vie pour que quatre cents autres personnes pussent vivre. Ils sortirent, restèrent un moment debout près des deux corps, en nous regardant, pensifs, puis ils agitèrent leurs mains en un dernier adieu et ils partirent lentement vers la ville en flammes, en traînant chacun un des deux morts.\par
« Tant de précautions furent superflues. Le lendemain matin, la Peste fit parmi nous sa première victime : une petite nurse attachée à la famille du professeur Stout. L’heure n’était point de faire du sentiment. Espérant qu’elle était la seule atteinte, nous lui intimâmes l’ordre de s’en aller et la poussâmes dehors. Elle obéit et s’éloigna à pas lents, en se tordant les mains de désespoir et en sanglotant lamentablement. Nous n’étions pas sans ressentir toute la brutalité de notre acte. Mais qu’y faire ? Pour sauver la masse il fallait sacrifier l’individu.\par
« Nous n’étions pas au bout. Dans un des laboratoires de l’École, trois familles avaient conjointement élu domicile. Au cours de l’après-midi, nous trouvâmes parmi elles quatre cadavres et, à des degrés divers, sept cas de peste.\par
« De cet instant, l’horreur s’installa dans la maison. Abandonnant les corps là où ils étaient tombés, nous contraignîmes les survivants de ces familles à s’isoler dans une autre pièce. Les trois familles étaient contaminées et, dès que le symptôme de la Peste apparaissait, nous enfermions les victimes dans une chambre d’isolement. Et les gens devaient s’y rendre d’eux-mêmes, sans que nous eussions à les toucher. Cela soulevait le cœur.\par
« Mais la Peste continuait à gagner. Toutes les chambres isolées s’emplissaient successivement de morts et de mourants. Ceux qui étaient sains encore, abandonnant le premier étage, se retirèrent au second. Puis ils montèrent au troisième, devant cette marée de la mort qui, chambre par chambre, étage par étage, submergeait tout l’édifice.\par
« L’École devint bientôt un charnier et, au cours de la nuit suivante, les survivants l’abandonnèrent, n’emportant rien d’autre avec eux que des armes, des munitions et une lourde provision de conserves.\par
« Nous campâmes d’abord dans la grande cour et, tandis que les uns montaient la garde autour des provisions, les autres partaient en exploration dans la ville, à la recherche de chevaux et de voitures, ou charrettes, d’automobiles, ou de tout autre véhicule qui nous permettrait d’emporter avec nous le plus de vivres possible. Puis, comme nous l’avions vu faire aux bandes d’ouvriers, nous tenterions de nous frayer un chemin vers les campagnes.\par
« J’étais un de ceux qui furent envoyés en éclaireurs et le Docteur Hoyle, se souvenant que son automobile personnelle était demeurée dans son garage, me pria d’aller la quérir.\par
« Nous marchions deux par deux et Dombey, un jeune étudiant, m’accompagnait. Il nous fallait parcourir un demi mille environ à travers la ville, afin d’arriver à l’ancien domicile du Docteur Hoyle. Dans ce quartier, les maisons étaient séparées les unes des autres par des jardins, des arbres et des pelouses, et le feu, comme pour se jouer, avait détruit au hasard.\par
« Tantôt toute une suite de maisons, incendiées par les flammèches qu’y avait secouées le vent, avait brûlé. Plus loin, d’autres maisons étaient demeurées complètement intactes.\par
« Là, comme ailleurs, les pillards étaient à l’œuvre. Dombey et moi, nous tenions à la main, bien en vue, nos pistolets automatiques, et nous avions la mine si décidée et si mal commode que pas un de ceux que nous rencontrâmes ne se risqua à nous attaquer.\par
« La maison du Docteur Hoyle ne paraissait pas avoir été touchée encore par l’incendie. Mais la fumée s’en échappa, au moment juste où nous pénétrions dans le jardin.\par
« Le bandit qui avait allumé le feu, après avoir descendu l’escalier en titubant, ivre et des bouteilles de whisky, dont émergeaient les goulots, emplissant toutes les poches de ses vêtements, sortait du corridor intérieur et apparaissait sur le perron. Mon premier mouvement fut de décharger sur lui mon pistolet. Je ne le fis pas, et j’ai toujours regretté depuis de m’être abstenu.\par
« Flageolant et se parlant à lui-même, les yeux injectés de sang, deux entailles à vif dans son visage broussailleux et qui provenaient, sans nul doute de quelque verre brisé sur lequel il avait chu, cet individu était bien le spécimen le plus répugnant de la dégradation humaine.\par
« Comme il traversait la pelouse, afin de gagner la rue, il nous croisa et feignit de s’appuyer contre un arbre, pour nous laisser passer. Mais, juste au moment où nous nous trouvions en face de lui, il tira soudain son pistolet, visa et tua Dombey, d’une balle en pleine tête. C’était un meurtre gratuit, car nous ne le menacions pas et, l’instant d’après, je l’abattais moi-même. Mais c’était trop tard. Dombey était mort du coup, sans articuler un cri, et je doute qu’il se soit absolument rendu compte de ce qui lui arrivait.\par
« Abandonnant les deux corps, je courus jusqu’à l’arrière face de la maison en feu, vers le garage où je trouvai effectivement l’automobile du Docteur Hoyle. Le réservoir était plein d’essence et je n’eus qu’à mettre la voiture en marche. Je revins avec elle, à toute vitesse, à travers la ville en ruines, jusqu’au campement des survivants.\par
« Les autres escouades revinrent à leur tour. Elles avaient été moins heureuses que moi. Le professeur Fairmead seul avait déniché un poney des Shetland. Mais la pauvre bête, attachée dans son écurie et abandonnée depuis plusieurs jours, était si faible, par défaut de nourriture et d’eau, qu’elle était incapable de porter aucun fardeau. Quelques-uns d’entre nous proposèrent de lui rendre la liberté, mais j’insistai pour que nous emmenions l’animal, afin qu’en cas de besoin il pût nous servir de nourriture.\par
« Nous étions quarante-sept quand nous nous mîmes en route. Parmi nous, beaucoup de femmes et d’enfants. Dans l’automobile prit place tout d’abord le Président de la Faculté, un vieillard que ces événements terribles avaient complètement brisé. Avec lui montèrent plusieurs jeunes enfants et la mère, très âgée, du professeur Fairmead. Wathope, un jeune professeur d’anglais, qui était grièvement blessé à la jambe, prit le volant.\par
Le reste de notre troupe allait à pied, le Professeur Fairmead tenant le poney par la bride. »
\chapterclose


\chapteropen

\chapter[{V. Quand le monde fut vide}]{V. Quand le monde fut vide}
\renewcommand{\leftmark}{V. Quand le monde fut vide}


\chaptercont
\noindent « Ce jour où nous étions aurait dû être un jour splendide d’été. Mais les tourbillons de fumée de ce monde en feu continuaient à voiler le ciel d’un épais rideau, où le soleil sinistre n’était plus qu’un disque mort et rouge, sanguinolent. De ce soleil de sang nous avions pris, depuis plusieurs jours, l’accoutumance. Mais la fumée nous mordait les narines et les yeux, que nous avions entièrement pourpres et qui pleuraient.\par
« Nous dirigeâmes notre marche vers le sud-est, à travers les milles sans fin des collines basses et verdoyantes de la banlieue de la ville, où se succédaient sans interruption de charmantes ou superbes résidences.\par
« Nous n’avancions que péniblement, les femmes surtout et les enfants traînaient la patte. Alors, voyez-vous, mes chers petits enfants, nous avions tous, tant que nous étions, désappris plus ou moins à marcher. Nous avions trop de véhicules à notre disposition. Depuis la Peste, j’ai réappris à marcher. Mais alors j’étais comme les autres.\par
« Nous allions donc lentement, réglant nos pas les uns sur les autres, afin de maintenir la cohésion de notre troupe. Les pillards étaient devenus moins nombreux. Une bonne quantité de ces bêtes de proie humaines avaient succombé ; mais ceux qui restaient étaient encore pour nous une perpétuelle menace.\par
« De toutes les belles résidences abandonnées, devant lesquelles nous passions, un grand nombre était demeuré intact. Nous ne manquions pas d’aller visiter leurs garages, à la recherche de quelque autre automobile ou d’essence. Mais sans succès. Tout ce qui pouvait être utile avait déjà été emporté.\par
« Au cours de ces recherches, Calgan, un aimable jeune homme, perdit la vie. Il fut tué par un pillard, embusqué derrière un buisson. Cette mort fut le dernier accident de ce genre qui nous advint. Il y eut bien encore une espèce de brute qui ouvrit délibérément le feu sur notre groupe. Mais il tirait si stupidement, dans l’aveuglement de sa rage folle, que nous l’abattîmes avant qu’il ne nous eût fait aucun mal.\par
« À Fruitvale, un des plus beaux endroits de cette banlieue, la Peste Écarlate frappa encore l’un de nous. Sa victime fut le Professeur Fairmead. Dès qu’il s’aperçut qu’il était atteint, il nous fit comprendre par signes que sa mère, qui se trouvait dans l’auto, n’en devait pas être informée. Puis, s’écartant de nous, il alla s’asseoir, désespéré, sur les marches de la véranda d’une superbe villa, qui se trouvait là. J’étais à l’arrière de notre groupe et lui envoyai de la main un dernier adieu.\par
« Au cours de la journée, cinq des nôtres eurent le même sort. Nous n’en poursuivîmes pas moins notre route et, le soir, à plusieurs milles au-delà de Fruitvale, nous campâmes. Dix de nous périrent dans la nuit et, chaque fois, nous dûmes lever le camp pour nous écarter de ces morts. Nous n’étions plus que trente au matin.\par
« Pendant la première étape, la femme du Président de la Faculté, qui allait à pied, fut atteinte. Son malheureux mari, en la voyant s’éloigner, voulut à toute force descendre de l’automobile et rester avec elle. Nous fîmes tout ce qui était possible pour le dissuader de cette résolution. Finalement nous cédâmes à sa volonté.\par
« La seconde nuit de notre voyage, nous étions, quand nous campâmes, en pleine campagne. Nous avions eu onze morts dans la journée. Nous en eûmes trois autres pendant la nuit. En sorte que le lendemain matin nous n’étions plus que onze présents. Car Wathope, le professeur à la jambe blessée, s’était enfui avec l’auto, emmenant avec lui sa mère et sa sœur, et emportant presque toutes nos provisions.\par
« Ce fut au cours de cette journée qu’étant assis pour me reposer au bord de la route, j’aperçus le dernier aéroplane. La fumée dans la campagne était beaucoup moins épaisse et je le vis qui semblait tourniquer dans le ciel, complètement désemparé, à deux cents pieds de haut environ. Que lui était-il advenu ? Je ne saurais le dire. Mais, au bout d’un moment, je le vis qui baissait de plus en plus. Puis le réservoir à essence du moteur prit feu et explosa, et l’avion, après avoir, un instant encore, vacillé sur ses ailes, tomba perpendiculairement sur le sol, comme un bloc de plomb.\par
« Depuis ce jour, je n’ai jamais revu un aéroplane. Bien souvent, pendant les années qui suivirent, j’examinai le ciel, espérant, contre toute espérance, en voir un apparaître et que, quelque part dans le vaste monde, un îlot de l’ancienne civilisation avait survécu. Il n’en était rien, et ce qui s’était passé en Californie était arrivé de même partout l’univers.\par
« À Niles, le lendemain, nous n’étions plus que trois, et nous trouvâmes, au milieu de la route, Wathope et son automobile. L’automobile était en pièces et, sur les couvertures qu’ils avaient étendues par terre, gisaient morts, Wathope, sa mère et sa sœur.\par
« La nuit, je dormis lourdement. Ces marches forcées m’avaient anéanti de fatigue. À mon réveil, j’étais seul au monde. Canfield et Parsons, mes deux compagnons, étaient morts de la Peste. Des quatre cents personnes qui s’étaient réfugiées avec moi dans l’École de Chimie, et des quarante-sept qui vivaient encore au début de notre exode, je demeurais, moi, unique, avec le poney des Shetland.\par
« Pourquoi ? Je ne tenterai pas de l’expliquer. Je ne fus pas contaminé. Voilà tout. J’avais eu une chance contre un million. Je devrais dire contre plusieurs millions. Car telle fut la proportion de ceux qui, comme moi, survécurent.\par
« Durant deux jours, je campai sous un délicieux bocage, loin de tout cadavre. Là, quoique fort déprimé et pensant que mon tour de mourir allait venir d’un instant à l’autre, je me refis quelques forces. Il en fut de même du poney.\par
« Le troisième jour, commençant à me persuader que j’étais décidément immunisé, je chargeai sur le poney la petite provision de conserves qui me restait et repris ma route dans un monde désolé. Je ne rencontrai pas un seul être vivant, homme, femme ou enfant. Rien que des morts parsemés sur mon chemin.\par
« Les aliments naturels ne manquaient point. La terre, à cette époque, n’était point comme aujourd’hui. Débarrassée des arbres en surcroît et des taillis inutiles, elle était partout bien cultivée. Il y avait autour de moi de quoi nourrir des millions de bouches. Et cette nourriture, mûre à point, se perdait. Je récoltais à ma volonté, dans les champs et dans les vergers, fruits et légumes, et toutes sortes de baies. Dans les fermes délaissées, je trouvais des œufs fraîchement pondus et j’attrapais des poulets. Dans les armoires, je mettais la main sur de nombreuses conserves.\par
« Ce qui advint des animaux domestiques est tout à fait étrange. Ils retournaient à l’état sauvage et s’entre-dévoraient. Les poules, poulets et canards furent les premiers détruits. Les cochons, au contraire, s’adaptèrent merveilleusement à leur vie nouvelle, ainsi que les chats et les chiens. Ceux-ci devinrent rapidement un véritable fléau, tellement ils étaient nombreux. Ils dévoraient les cadavres et n’arrêtaient pas d’aboyer ni de hurler, la nuit comme le jour.\par
« Tout d’abord, ils demeurèrent solitaires, soupçonneux envers ceux de leurs frères qu’ils rencontraient, et prompts à engager la lutte avec eux. Au bout de quelques temps, ils se rassemblèrent et coururent en bandes. Cet animal est naturellement sociable et, la compagnie de l’homme lui manquant, il se rabattit sur ses semblables.\par
« Il y avait, avant les derniers jours du monde, de très nombreuses espèces de chiens : chiens à poil ras et chiens à belle et chaude fourrure ; chiens tout petits, si petits qu’ils auraient à peine pu faire une bouchée pour d’autres molosses, aussi robustes que les lions des montagnes. Tous les roquets et petits chiens, trop faibles pour la lutte, furent tués rapidement par leurs frères. Les très grandes espèces ne s’adaptèrent pas davantage à la vie sauvage. Il ne subsista finalement que les chiens de taille moyenne, mieux constitués dans leurs organismes et plus souples aux conditions nouvelles qui leur étaient imposées. Ce sont les chiens-loups, que vous connaissez bien et qui courent aujourd’hui la campagne. »\par
— Et les chats, interrogea Hou-hou, pourquoi ne courent-ils pas par bandes, comme les chiens. Pourquoi, grand-père ?\par
— Le chat, répondit l’aïeul, n’est pas un animal sociable. Je me souviens d’un grand écrivain, qui vécut jadis au XIX\textsuperscript{e} siècle, et qui l’a proclamé : le chat est un solitaire. Avant que l’homme ne l’attirât à lui et ne le domestiquât, au cours de la longue civilisation passée, il vivait seul. Cette civilisation écroulée, il a repris de lui-même sa liberté et son isolement.\par
« Le cheval retourna, lui aussi, à l’état sauvage. Toutes les belles espèces, que l’homme possédait et élevait jadis, ont dégénéré et se sont fondues dans un type unique, le mustang-horse, que vous connaissez. De même les vaches, les moutons et, parmi les oiseaux domestiques, les pigeons. Quant aux poules et aux poulets, ceux et celles qui ont survécu ne ressemblent plus en rien aux volatiles qui peuplaient autrefois nos basses-cours.\par
« Mais je reprends le fil de mon histoire. Je marchais donc à travers un monde désert. À mesure que le temps passait, je commençais à soupirer de plus en plus après des êtres humains. Mais je n’en rencontrais aucun et me sentais de plus en plus seul. Je traversai la Vallée de Livermore, puis les montagnes qui la séparent des hautes altitudes de la Vallée de San Joachim. Vous n’avez jamais, mes enfants, vu cette vallée. Elle est immense et magnifique, et peuplée aujourd’hui de chevaux sauvages, qui y vivent par grands troupeaux, de milliers et de dizaines de milliers de têtes.\par
« J’y suis retourné, voilà trente ans environ, et elle était telle que je vous le dis. Vous pensez, mes enfants, que les chevaux sauvages sont nombreux dans les vallées de la côte que vous fréquentez habituellement. Eh bien ce n’est rien, en comparaison des immenses troupeaux de la Vallée de San Joachim. Et veuillez observer que les vaches une fois redevenues sauvages, établirent leurs colonies dans des vallées moins hautes et plus tempérés, où elles pouvaient davantage se protéger du froid.\par
« À mesure que je m’éloignais des grands centres urbains, je trouvais plus de villages et de petites villes intacts. Les pilleurs et les incendiaires étaient venus moins nombreux jusque-là. Mais toutes ces agglomérations étaient emplies de cadavres de pestiférés et je passais soigneusement au large.\par
« Près de Lathrop, afin de tromper ma solitude, je recueillis une paire de chiens coolies, qui semblaient fort embarrassés de leur liberté retrouvée et qui revinrent d’eux-mêmes, avec joie, à l’obéissance de l’homme. Ces bêtes m’ont accompagné ensuite, durant bien des années, et leurs instincts étaient les mêmes que ceux des chiens que vous possédez. Mais, en soixante ans, ceux-ci ont perdu presque toute leur éducation ancestrale et ils ressemblent bien plutôt à des loups domestiqués. »\par
Ici Bec-de-Lièvre se leva, jeta un regard vers les chèvres, afin de s’assurer si rien de mal ne leur était advenu. Puis il observa la position du soleil, qui commençait à décliner sur l’horizon, et témoigna quelque impatience de l’abondance extrême des détails où s’attardait le vieillard. Edwin se joignit à lui pour solliciter de l’ancêtre un peu plus de rapidité dans son récit.\par
« Je n’ai plus grand’chose à vous dire, reprit le vieux. Accompagné de mes deux chiens, de mon poney, qui me servait de bête de charge, et d’un cheval que j’avais réussi à capturer et sur lequel j’étais monté, je dépassai la Vallée de San Joachim et atteignis, dans la Sierra, une autre vallée non moins magnifique, appelée Yosémite.\par
« Là, dans le Grand Hôtel, je trouvai une énorme provision de conserves de toutes sortes. Le gibier abondait dans les pâturages environnants et la rivière torrentueuse, qui bondissait au fond de la vallée, était pleine de truites.\par
« Je demeurai trois ans à cet endroit, dans une solitude absolue, dont seul peut comprendre la poignante mélancolie l’homme qui a connu la grandeur et le charme de la civilisation. Puis un moment arriva où je ne pus supporter davantage cet isolement. Je sentais que je devenais fou. Comme le chien, j’étais un animal sociable et je ne pouvais vivre sans la compagnie d’autres êtres de mon espèce.\par
« Le raisonnement me persuada que, puisque j’avais survécu à la Peste Écarlate, il y avait chance pour que quelques autres hommes eussent échappé comme moi. Je pensai en outre que, depuis trois ans, tout mauvais germe avait dû disparaître et que la terre était, sans nul doute, redevenue habitable.
\chapterclose


\chapteropen

\chapter[{VI. Vesta Van Warden et le chauffeur}]{VI. Vesta Van Warden et le chauffeur}
\renewcommand{\leftmark}{VI. Vesta Van Warden et le chauffeur}


\chaptercont
\noindent « Monté sur mon cheval, et toujours flanqué de mes deux chiens et de mon poney, je me mis en route. Je traversai à nouveau la Vallée de San Joachim et, abandonnant les montagnes, je redescendis vers la vallée de Livermore.\par
« La transformation qui s’était opérée dans les choses depuis ces trois années, était surprenante. C’est à peine si je pouvais reconnaître le pays. Hier merveilleusement cultivé, il avait été envahi par un océan de végétation sauvage et vigoureuse, qui avait submergé le travail des anciens agriculteurs.\par
« Comprenez bien, mes enfants, que le blé, les divers légumes, les arbres des vergers, qui nous donnaient leurs fruits, entretenus de tous temps et couvés par l’homme, étaient tendres et doux. Les mauvaises herbes au contraire et les arbustes épineux, auxquels l’homme avait de tout temps fait la guerre, étaient d’une race plus dure et plus résistante. Si bien que le jour où la main de l’homme se retira, cette seconde végétation prit le dessus et étouffa la première.\par
« Je rencontrai également un grand nombre de coyotes, qui s’étaient multipliés à foison, puis de petites troupes de loups qui allaient deux par deux, ou trois par trois, et qui, comme moi, redescendaient des montagnes vers les anciens territoires d’où ils avaient jadis été chassés.\par
« C’est au lac Temescal, non loin de ce qui avait été autrefois la ville d’Oakland, que je retrouvai les premiers êtres humains encore en vie.\par
« Ah ! mes enfants, comment pourrais-je vous dire mon émotion, quand, à califourchon sur mon cheval et dévalant de la colline qui domine le lac, j’aperçus la fumée d’un feu de campement, qui s’élevait parmi les arbres ? Mon cœur cessa presque de battre et il me sembla que ma raison s’égarait. Puis je perçus le vagissement d’un bébé, d’un bébé humain. Des chiens aboyèrent, auxquels répondirent les miens. J’avais cru longtemps que j’étais le seul survivant sur la terre, de l’immense désastre. Et voilà que j’apercevais une fumée, que j’entendais crier un bébé.\par
« Je ne tardai pas à voir au bord du lac, là, devant mes yeux, à moins de cent yards, un homme se dresser. Ce n’était point un être chétif ni malade. Non. Il semblait en excellente santé et, se tenant debout sur un rocher qui surplombait l’eau du lac, il pêchait.\par
« J’arrêtai mon cheval et appelai. L’homme, qui s’était retourné, ne me répondit pas. J’agitai la main, pour lui souhaiter le bonjour. Il ne répondit pas non plus à mon geste. Alors je pris ma figure dans mes mains et je l’y cachai. Je n’osais plus relever la tête et les yeux. Il me semblait que j’avais été la proie d’une hallucination et qu’au moment où je voudrais la fixer à nouveau, elle aurait disparu. Je craignais de détruire cette vision qui m’était si chère. Tant que je ne l’aurais pas fait fuir sous mon regard, elle subsisterait en ma pensée.\par
« Je demeurai donc ainsi, immobile, jusqu’au moment où je fus tiré de mon rêve par des grognements de chiens et par la voix de l’homme, qui me parlait. Savez-vous ce que cette voix disait ?… Non, n’est-ce pas ?… Eh bien, elle me disait :\par
— D’où diable viens-tu ?\par
« Oui, telles étaient les paroles textuelles que j’entendais prononcer, Bec-de-Lièvre, en guise de bienvenue, sur les bords du lac Temescal, il y a cinquante-sept ans exactement. Et jamais mots ne me semblèrent plus doux. Je rouvris les yeux.\par
« Devant moi je vis un homme de haute taille, au regard sombre et dur, à la mâchoire lourde, au front oblique. Je me laissai glisser, plutôt que je descendis de cheval, et tout ce que je sais, c’est que, la minute d’après, je pressais ses mains dans les miennes en pleurant. Je l’aurais embrassé.\par
« Il ne répondit point à mes effusions, me jeta un coup d’œil soupçonneux et s’éloigna de moi. Je courus après lui, me cramponnai à ses mains et sanglotai de plus belle. »\par
À ce souvenir, la voix de l’aïeul parut se briser et des larmes coulèrent derechef le long de ses joues. Tandis que les gamins l’observaient en ricanant, il reprit :\par
— Je voulais le serrer dans mes bras, le couvrir de baisers. Et lui ne voulait pas. C’était une brute, une brute parfaite. L’être le plus antipathique qui se puisse imaginer. Il s’appelait… Comment donc s’appelait-il ? Je ne me souviens plus de son vrai nom. Mais on le dénommait le Chauffeur. C’était le nom de son ancienne profession, et il l’avait conservé. Et voilà pourquoi la tribu qu’il a fondée s’appelle la Tribu des Chauffeurs.\par
« C’était un vilain individu, violent et injuste. Je n’ai jamais pu comprendre pourquoi la Peste Écarlate l’avait épargné. Il semblait, à le voir, qu’en dépit de nos risibles leçons de philosophie il n’y eût pas de justice dans l’Univers. Maintenant qu’il ne pouvait plus parler automobiles, moteurs et essence, il était incapable de plus rien dire, sinon de se vanter de tous les tours pendables qu’il avait joués à ses anciens patrons, comment il les escroquait et volait. Il ne tarissait pas sur ce chapitre, et se rengorgeait de ses méfaits. Et pourtant il fut épargné, tandis que des millions et des milliards d’hommes, meilleurs que lui, furent détruits.\par
« Je le suivis jusqu’à son campement. Là je fis connaissance avec sa femme.\par
« Voilà surtout qui était stupéfiant et pitoyable ! Je reconnus cette femme. C’était Vesta Van Warden, l’ancienne jeune épouse du banquier John Van Warden. Oui, elle-même, qui, vêtue de haillons et pleine de cicatrices, les mains calleuses, déformées par les plus durs travaux, était penchée au-dessus du feu du campement et cuisinait le dîner comme un simple marmiton ! Vesta Van Warden, née dans la pompe opulente du plus puissant baron de la finance que le monde eût jamais connu !\par
« Son père, Philip Saxon avait été, jusqu’à sa mort, président des Magnats de l’Industrie. Nul doute que s’il avait eu un fils, ce fils ne lui eût tout naturellement succédé, comme un rejeton royal hérite de la couronne. Mais son seul enfant avait été cette fille, fleur parfaite de la grâce et de la culture de notre vieille civilisation. En l’épousant, John Van Warden, riche à millions, reçut de Philip Saxon l’investiture de son titre et de sa fonction. Il y ajouta le titre de premier Ministre du Contrôle International des peuples, et il avait en fait, plusieurs années durant, gouverné le monde. Vesta avait-elle réellement aimé son mari ?… Avait-elle eu pour lui cette folle passion que chantent les poètes ?… Je me permets d’en douter. Ce fut avant tout un mariage politique, comme il s’en pratiquait dans les anciennes Cours.\par
« Et c’est cette femme qui faisait cuire le ragoût de poisson, dans un vieux pot encrassé de suie ! Et l’âcre fumée, qui tourbillonnait au vent, irritait et rendait rouges ses yeux admirables !\par
« Triste était son histoire. Comme le Chauffeur et comme moi-même, elle était une des très rares survivantes de la Peste. Sur une des collines qui dominent la baie de San Francisco, Van Warden avait édifié un superbe palais d’été, entouré d’un parc immense. Van Warden y envoya sa fille dès qu’éclata le fléau. Des gardiens en armes défendaient à quiconque l’entrée de la propriété et rien n’y pénétrait, vivres ou lettres, qui n’eût été au préalable rigoureusement désinfecté.\par
« La Peste cependant était entrée, tuant les gardiens à leur poste, les domestiques dans leur travail et balayant toute l’armée des intendants et des serviteurs – de ceux du moins qui n’avaient pas pris la fuite pour aller mourir ailleurs. Si bien qu’à la fin Vesta se trouva être la seule en vie dans le charnier de son palais.\par
« Le Chauffeur était l’un des anciens domestiques qui s’étaient enfuis. Il revint dans la propriété, deux mois après, et y découvrit la jeune femme, dans un petit pavillon du parc, où elle s’était installée.\par
« Effrayée à la vue de cette brute, elle se sauva, en se dissimulant parmi les arbres. Elle marcha à l’aventure, tout le jour et toute la nuit, elle dont les tendres pieds et le corps délicat n’avaient jamais connu la meurtrissure des cailloux et la blessure sanglante des épines. Le Chauffeur la poursuivit et il la rejoignit vers l’aube.\par
« Il commença à la frapper. Vous me comprenez bien, n’est-ce pas ? Il frappait de ses énormes poings la frêle jeune femme. Il voulait que, désormais, elle lui obéît en tout. Il prétendait qu’elle fût désormais son esclave. C’était elle qui ramasserait le bois pour faire le feu, pour s’occuper de la cuisine et des plus viles besognes. Elle qui, de sa vie, n’avait connu le moindre travail manuel. Et elle obéit. Elle subit son amour et se fit sa domestique. Tandis que lui, un vrai sauvage, se reposait tout le jour, en donnant des ordres à son esclave, en surveillant leur exécution. En dehors du soin de chasser la viande et de pêcher le poisson, ce fainéant se tournait les pouces, du matin au soir, du soir au matin. »\par
Bec-de-Lièvre approuva et déclara aux autres garçons :\par
— C’est bien là le portrait du Chauffeur. Je l’ai connu avant sa mort. C’était un homme peu ordinaire. Il fabriquait, pour se distraire, des mécaniques qui marchaient toutes seules. Mon père à moi avait épousé sa fille. Il les battait tous les deux, et moi aussi, qui étais tout gosse. Tout le monde devait lui obéir, c’était une ignoble brute. Comme il était en train de mourir et comme je m’étais approché trop près de lui, il empoigna un long bâton qu’il avait toujours à portée de sa main et faillit m’ouvrir le crâne.\par
À ce souvenir rétrospectif, Bec-de-Lièvre frotta sa tête ronde, comme s’il y avait encore mal, tandis que les autres enfants faisaient cercle autour de lui et que le vieillard, les yeux levés au ciel, se marmottait à lui-même on ne sait quelles mystérieuses paroles, au sujet de Vesta Van Warden, la squaw qui fonda la Tribu des Chauffeurs. Il poursuivit :\par
— Vous ne pouvez saisir, mes enfants, toute l’horreur de la situation. Le Chauffeur était hier encore ce qu’on appelait un domestique. Oui, un do-mes-ti-que. C’est-à-dire qu’il passait sa vie à obéir, à baisser la tête et à faire des courbettes devant celle qui était devenue maintenant son esclave. Elle était une reine de la vie, par sa naissance et par son mariage. Dans le creux de sa petite main blanche et rose, elle tenait le sort de millions d’hommes et elle commandait à des centaines d’autres domestiques, tout pareils, au point de vue social, au Chauffeur. Durant les jours qui avaient précédé la Peste Écarlate, le plus léger contact avec un être de cette sorte eût été pour elle une ineffaçable souillure. Oui il en était ainsi autrefois.\par
« Je me souviens avoir vu un jour Mistress Goldwyn, la femme d’un autre Magnat ; au moment où elle montait sur la plate-forme d’embarquement de son grand dirigeable. Elle laissa choir son ombrelle. Celle-ci fut ramassée par un domestique, qui s’oublia jusqu’à lui présenter directement l’objet ; oui, à elle-même, la femme toute-puissante. Elle recula, comme si elle avait eu en face d’elle un lépreux, et fit un signe à son secrétaire, qui ne la quittait pas, afin qu’il prît l’ombrelle et la lui remit. Elle ordonna en outre que fût relevé le nom de l’audacieux domestique et qu’on le renvoyât sur-le-champ. Vesta Van Warden était une femme de ce genre. Et le Chauffeur la battit, jusqu’à ce qu’elle consentît à être sa servante.\par
« Bill… Voilà que son nom me revient… Bill, le Chauffeur, était un affreux coquin, un être vil entre tous, dépourvu de toute culture et de toute galanterie envers les femmes. Et c’est à lui que revint la merveille des femmes, Vesta Van Warden ! Ce sont là des choses raffinées qui vous échappent, mes petits enfants. Car vous êtes vous-mêmes de petits sauvages, des natures des primaires. Vesta à cet homme ! C’était scandaleux…\par
« Pourquoi, aussi bien, ne m’était-elle pas échue ? Je m’en fusse parfaitement accommodé. J’étais, moi, un homme cultivé, bien éduqué et honorable, professeur d’une grande université. Il n’y a pas, je vous l’ai dit, de justice sur cette terre.\par
« Au temps de sa grandeur, elle était tellement au-dessus de moi qu’elle n’eût même pas daigné s’apercevoir que j’existais. Mais, après la Peste Écarlate, j’eusse été pour elle un excellent parti. Au lieu de cela, voyez dans quel abîme de dégradation elle tomba ! Et elle m’eût aimé, aimé, oui, j’en suis persuadé. Car le cataclysme effroyable qui nous réunit me permit de la connaître de près, d’interroger ses beaux yeux, de converser avec elle, de prendre sa main dans la mienne, de l’aimer et de savoir qu’elle aussi éprouvait pour moi les sentiments les plus tendres. Elle me préférait au chauffeur, c’était visible. Pourquoi la Peste, qui avait détruit tant de millions d’hommes, avait-elle justement épargné celui-là ?\par
« Un après-midi, tandis que le Chauffeur était parti à la pêche et que j’étais demeuré seul avec elle, elle me conjura de le tuer. Elle m’en supplia, avec des larmes dans les yeux. Mais le bandit était robuste et redoutable, et je n’osai pas tenter l’entreprise. Je lui offris, quelques jours après, mon cheval, mon poney et mes chiens, s’il consentait à me céder Vesta. Il me rit au nez et refusa avec insolence.\par
« Il me répondit que, dans les temps anciens, il avait été un domestique, de la boue que foulaient aux pieds les hommes comme moi et les femmes comme elle. Maintenant la roue avait tourné. Il possédait la plus belle femme du monde, elle lui préparait sa nourriture et soignait les enfants qu’il lui avait faits.\par
— Tu as eu ton heure, camarade ! me dit-il. J’ai la mienne, aujourd’hui. Et elle me convient fort, par ma foi ! Le passé est fini, bien fini, et je ne tiens pas à y revenir.\par
« C’est ainsi qu’il me parla. Mais pas avec les mêmes mots. Car c’était un homme horriblement vulgaire et il ne pouvait rien dire sans proférer les plus épouvantables jurons. Il ajouta que, s’il me surprenait à cligner de l’œil vers sa femme, il me tordrait le cou et la battrait, jusqu’à ce qu’elle en reste sur le carreau. Que pouvais-je faire ? J’avais peur, car il était le plus fort.\par
« Dès le premier soir où je découvris le campement du Chauffeur, Vesta et moi, nous avions eu une longue conversation, touchant bien des choses aimées de l’ancien monde évanoui. Nous avions causé livres et poésie. Le Chauffeur nous écoutait, en grimaçant et en ricanant. Cela l’ennuyait et l’irritait d’entendre parler de ce qu’il ignorait et ne pouvait comprendre.\par
« Il nous interrompit à la fin et déclara :\par
— Je te présente, professeur Smith, Vesta Van Warden, qui fut jadis la femme de Van Warden, le Magnat. Cette beauté arrogante et majestueuse est maintenant ma squaw. Elle va, devant toi, me retirer mes mocassins. Femme, fais vite ! Montre à Mister Smith comme je t’ai bien dressée.\par
« Je vis la malheureuse grincer des dents et une flamme de colère lui monter au visage\par
« Le Chauffeur dégagea son bras et recula son poing noueux, prêt à frapper. La crainte s’empara de moi et je me levai, pour m’éloigner, afin de n’être pas témoin. Mais le bourreau se mit à rire et me menaça moi-même d’une volée en règle, si je ne demeurais point à admirer la scène.\par
« Contraint et forcé, je me rassis donc près du feu du campement, au bord du lac Temescal, et je vis Vesta Van Warden s’agenouiller devant cette brute humaine, grimaçante et poilue, et tirer l’un après l’autre les deux mocassins du gorille !\par
« Non, non, vous ne pouvez comprendre, mes chers enfants, vous que la sauvagerie environne et qui n’avez rien connu du passé…\par
« Le Chauffeur semblait la couver des yeux, tandis qu’elle peinait à cette besogne immonde.\par
— Elle est, dit-il, rompue à la bride et au licol, professeur Smith. Un peu têtue parfois. Oui, un peu têtue. Mais un bon coup de poing ou une forte gifle sur la joue la rendent rapidement aussi gentille et douce qu’un agneau.\par
« Un beau jour, le Chauffeur me parla comme suit :\par
— Tout est à refaire ici-bas, professeur. C’est à nous de multiplier et de repeupler la terre. Tu n’as pas de femme et je ne suis point disposé à te prêter la mienne. Ce n’est pas ici le Paradis Terrestre. Mais je suis bon bougre. Écoute-moi, professeur Smith !\par
« Il me montra du doigt leur dernier enfant à peine âgé d’un an.\par
— C’est une fille, continua-t-il. Je te la donne pour femme. Seulement il te faudra attendre qu’elle ait un peu grandi. Riche idée, n’est-ce pas ? Ici nous sommes tous égaux et, s’il y avait une hiérarchie, c’est moi qui serais le plus gros crapaud de la mare. Mais je ne suis pas intraitable, oh non ! Je te fais donc l’honneur, professeur Smith, le très grand honneur de t’accorder comme fiancée ma fille et celle de Vesta Van Warden… Tout de même, dommage, n’est-ce pas ? que Van Warden ne soit pas ici, dans un coin, pour être témoin !\par
\bigbreak
\chapterclose


\chapteropen

\chapter[{VII. Pour repeupler la terre}]{VII. Pour repeupler la terre}
\renewcommand{\leftmark}{VII. Pour repeupler la terre}


\chaptercont
\noindent « Je demeurai, l’âme angoissée, durant un mois environ, au campement du Chauffeur. Jusqu’au jour où, las sans doute de me voir et irrité de la mauvaise influence qu’à son jugement j’exerçais sur Vesta, il jugea bon de se débarrasser de moi.\par
« Dans ce but, il me conta, d’un air détaché, que, l’année précédente, comme il errait parmi les collines de Contra Costa, il avait aperçu une fumée.\par
« Je tressautai. Cela signifiait que, de ce côté, il existait d’autres créatures humaines ! Et il m’avait caché, pendant un mois, cette inestimable et précieuse nouvelle !\par
« Je me mis en route aussitôt, avec mes deux chiens et mes deux chevaux, à travers les collines de Contra Costa, vers les Détroits de Carquinez.\par
« Je n’aperçus, du faîte des collines aucune fumée. Mais à leur pied, à Port Costa, je découvris un petit bateau en acier, amarré à la rive. J’y embarquai avec mes animaux. Un vieux bout de toile, qui me tomba sous la main, me servit de voile, et une brise du sud me poussa jusqu’aux ruines de Vallejo.\par
« Là, dans les faubourgs de la ville, je rencontrai les traces certaines d’un campement, récemment abandonné. De nombreuses coquilles de peignes m’expliquaient pourquoi ceux qui les avaient laissées derrière eux étaient venus jusqu’aux Détroits.\par
« Il s’agissait, comme je l’appris par la suite, de la Tribu des Santa Rosa, et je suivis ses traces par l’ancien sentier qui longeait le chemin de fer, à travers les marais salants qui s’étendent jusqu’à la vallée de Sonoma.\par
« Je découvris le campement des Santa Rosa dans l’ancienne briqueterie de Glen Ellen. Il y avait en tout dix-huit personnes. Deux d’entre elles étaient des vieillards : un nommé Jones, ex-banquier, et un certain Harrisson, usurier en retraite, qui avait pris pour femme l’ex-intendante de l’Hospice des Fous de Napa, qu’il avait rencontrée. De tous les habitants de la ville de Napa et des petites villes et villages de cette populeuse vallée, cette femme était la seule survivante.\par
« Puis venaient trois jeunes hommes : Cardiff et Hole, anciens fermiers, et Wainwright, un homme du commun, ancien journalier.\par
« Tous trois avaient, en errant, trouvé femme. Hole, un rustre illettré, était tombé sur Mistress Isadora, qui était, avec Vesta Van Warden, la plus belle femme de Californie qui eût échappé à la Peste Écarlate. C’était une cantatrice admirable, célèbre dans l’Univers entier, et elle se trouvait en tournée à San Francisco, lorsqu’éclata le fléau. Elle me conta, des heures durant, ses aventures, jusqu’au moment où elle fut enfin recueillie, et sans nul doute sauvée de la mort, par Hole, dans la Forêt de Mendocino. Elle devint – et elle n’avait rien de mieux à faire – la femme de cet homme qui, sous sa rude écorce et en dépit de son ignorance, était honnête et bon. Aussi était-elle bien plus heureuse en sa compagnie que Vesta Van Warden dans celle du Chauffeur.\par
« Les femmes de Cardiff et de Wainwright étaient des filles du peuple, solides et bien constituées, et accoutumées aux travaux manuels, le type voulu pour la nouvelle existence qu’elles étaient appelées à vivre.\par
« Ajoutez, pour parfaire le compte, deux idiots échappés de l’Hospice de Napa, six jeunes enfants, nés depuis la formation de la colonie, et finalement Bertha.\par
« Bertha était une brave, une excellente femme, Bec-de-Lièvre, en dépit des sarcasmes que ton père lui décochait constamment. Je la pris pour femme et m’en trouvai bien. Puis, ce fut votre grand’mère, Edwin et Bec-de-Lièvre, et la tienne aussi, Hou-Hou. Ta grand-mère maternelle, Bec-de-Lièvre, car ton père, qui était lui-même le fils aîné de Vesta Van Warden, et du Chauffeur, convola avec Vera, notre fille aînée.\par
« Je devins donc le dix-neuvième membre de la Tribu des Santa Rosa. Elle s’augmenta, après moi, de deux autres membres. Le premier fut Mongerson. C’était un descendant des Magnats. Je vous ai déjà parlé de lui. Après avoir fui en avion, il erra, pendant huit années, parmi les solitudes de la Colombie, avant de s’en revenir vers le sud, et de nous rejoindre. Il attendit douze ans encore, avant que Mary, ma deuxième fille, fût nubile et qu’il pût l’épouser.\par
« Le second fut Johnson, qui fonda la Tribu d’Utah. Il arrivait de la province d’Utah, un pays très éloigné d’ici, au-delà des grands déserts, vers l’Est. Ce n’est que vingt-sept ans après la Peste Écarlate qu’il atteignit la Californie.\par
« Dans tout le pays d’Utah, nous dit-il, il n’y avait eu, à sa connaissance, que trois survivants. Tous trois de sexe mâle. Pendant de nombreuses années, ces trois hommes chassèrent et vécurent ensemble, jusqu’à ce qu’enfin, las de leur solitude et désireux de procréer, pour que la race humaine ne disparût point de notre planète, ils firent route vers l’ouest, espérant trouver des femmes vivantes en Californie.\par
« Seul Johnson sortit indemne du Grand Désert, où ses deux compagnons avaient péri. Il avait quarante-six ans quand il se joignit à nous. Il épousa la troisième fille de Hole et d’Isadora, et son fils aîné convola avec ta tante, Bec-de-Lièvre, ta tante qui était elle-même la troisième fille du Chauffeur et de Vesta Van Warden.\par
« Johnson était un homme plein de force et d’initiative. Il se sépara des Santa Rosa, pour faire bande à part et aller former, à San José, une tribu nouvelle, la Tribu d’Utah. Ce n’est encore qu’une toute petite tribu de sept membres. Johnson est mort aujourd’hui ; mais ses descendants ont hérité de son intelligence et de son énergie. Nul doute qu’eux et leurs enfants ne soient appelés à jouer un rôle important dans la recivilisation de l’univers.\par
« Je ne connais, en dehors de ces trois tribus, que deux autres groupements humains : la Tribu de Los Angelitos et celle des Carmelitos ; celle-ci fut fondée par un homme, nommé Lopez, descendant des anciens Mexicains, qui était très sombre de peau et qui avait été vacher dans le ranch, et par une femme, ancienne servante au Grand Hôtel del Monte. Nous ne nous rencontrâmes avec eux qu’au bout de sept ans, comme ils étaient venus en exploration jusqu’en cette région. Ils habitaient, beaucoup plus vers le sud, un beau pays où il fait excessivement chaud.\par
« Je ne crois pas, mes enfants, qu’il existe à l’heure actuelle, sur la terre, plus de trois à quatre cents habitants. Depuis que Johnson a traversé le Grand Désert, en venant d’Utah, aucun signe de vie, aucune nouvelle ne nous sont venus de l’Est, ni de nulle part.\par
« Le monde magnifique et puissant que j’ai connu, aux jours de mon enfance et à ceux de ma jeunesse, a disparu. Il s’est anéanti. Je suis, à cette heure, le dernier survivant de la Peste Écarlate et seul je connais les merveilles du passé lointain. L’homme qui fut jadis le maître de la planète, maître de la terre, de la mer et du ciel, l’homme, qui fut un vrai Dieu, est retourné à son primitif état de sauvagerie et cherche sa vie le long des cours d’eau.\par
« Mais il multiplie rapidement. Ta sœur, Bec-de-Lièvre, a déjà quatre enfants. Nous préparons la voie à un autre saut vers la civilisation, voie lointaine encore, assurément, très lointaine même, mais inéluctable. Dans une centaine de générations, nos descendants, trop nombreux en ce pays, traverseront les Sierras et, génération par génération, se répandront vers l’Est, sur le grand continent américain.\par
« Beaucoup de temps s’écoulera d’ici là. Nous sommes redescendus très bas, désespérément bas. Si seulement un homme de science, physicien ou chimiste, avait survécu ! Quelle aide précieuse il nous apporterait ! Mais cela ne devait pas être et de la science nous avons tout oublié.\par
« Le Chauffeur s’était remis à travailler le fer. C’est lui qui a construit cette forge que nous utilisons aujourd’hui. C’était, malheureusement, un paresseux qui borna là son effort et qui, quand il mourut, emporta avec lui tout ce qu’il connaissait de la mécanique et de l’art de travailler les métaux. Moi, je n’entendais rien à ces choses. J’étais un lettré, sans plus, un humaniste. Et les autres survivants étaient dénués de toute instruction. Le Chauffeur avait réussi encore deux opérations, que nous connaissons par lui : la fabrication, par fermentation, de l’alcool et des boissons fortes, et la culture du tabac. Il en profitait pour s’enivrer, et c’est dans un accès d’ivresse qu’il a tué Vesta Van Warden. De cela je suis fermement persuadé, quoiqu’il ait toujours prétendu qu’elle s’était noyée en tombant dans le lac Temescal.\par
« Et maintenant, mes chers petits enfants, laissez-moi vous donner quelques bons conseils, dont vous aurez intérêt à faire votre profit dans la vie.\par
« Méfiez-vous, tout d’abord, des charlatans et des sorciers, qui se disent médecins. Ce sont des gens dangereux au premier chef, qui avilissent et déshonorent, dans notre petit monde, ce qui était autrefois la plus noble des professions.\par
« Je vois, autour de moi, la superstition en leur pouvoir, faire chaque jour des progrès nouveaux. Et ce mal ira toujours en empirant, tellement l’homme s’est dégradé. Ces prétendus docteurs sont, je vous l’assure, de fieffés voleurs, des mécréants issus de l’Enfer, qui n’ont qu’un but, mettre sur vous leur emprise et tirer de vous tout ce que vous possédez.\par
« Voyez, par exemple, ce jeune homme aux yeux de travers, connu parmi nous sous le nom du « Loucheur ». Il vend à tout le monde des charmes et sortilèges contre les maladies, et ne revient jamais bredouille de ses tournées. Il va même jusqu’à promettre le beau temps, en échange de bonne viande et de bonnes fourrures. À ceux qui se permettent de le contredire et de se proclamer ouvertement ses ennemis, il envoie ce qu’il appelle le bâton de la mort.\par
« Moi, l’ancien Professeur Smith, James Howard Smith, j’affirme qu’il se vante et qu’il ment effrontément. Je le lui ai dit en pleine figure. Pourquoi ne m’a-t-il pas, en réponse, envoyé le bâton de la mort ? Parce qu’il sait bien qu’avec moi ses jongleries ne prennent pas. Mais toi, Bec-de-Lièvre, tu es tellement enfoncé dans cette superstition que si, cette nuit, en t’éveillant, tu trouvais à côté de toi le bâton de la mort, tu en mourrais sans aucun doute. Et tu mourrais, non pas parce que ce bâton a un pouvoir quelconque, mais parce que tu n’es qu’un petit sauvage, à l’esprit crédule et obscurci !\par
« Il faut détruire tous ces exploiteurs de la crédulité publique, et puis aussi retrouver ces inventions utiles que nous avons perdues. C’est pourquoi, c’est pour vous aider dans ce travail, que je dois vous dire certaines choses que vous, mes enfants, répéterez à votre tour à vos enfants, quand vous en aurez.\par
« Vous devrez leur répéter que l’eau, lorsqu’elle est chauffée par le feu, se transforme en une substance merveilleuse qu’on nomme vapeur, que cette vapeur est plus forte et puissante que dix mille hommes réunis, et que, convenablement maniée et dirigée, elle est susceptible d’accomplir toutes les besognes de l’homme.\par
« Il y a encore d’autres choses fort utiles à savoir. L’électricité, qui produit dans le ciel les éclairs, est aussi une servante de l’homme. Elle a été jadis son esclave et elle le redeviendra un jour.\par
« L’alphabet est une invention toute différente, mais non moins précieuse. Sa connaissance me permet de lire dans les livres et de comprendre le sens d’une foule de petits signes qui y sont imprimés, tandis que vous, mes petits enfants sauvages, vous ne connaissez que l’écriture grossière des images figurées, qui représentent les divers objets.\par
« Dans la grotte de la Colline du Télégraphe, qui est fort sèche et que vous connaissez bien, et vers laquelle vous me voyez souvent me diriger, sur cette falaise, j’ai réuni beaucoup de livres, retrouvés par moi, et qui contiennent un résumé de la sagesse humaine. J’y ai placé aussi un alphabet, avec clef explicative, qui permet de lire et de comprendre son rapport avec l’écriture des images. Un jour viendra où les hommes, moins occupés des besoins de leur vie matérielle, réapprendront à lire. Alors, si aucun accident n’a détruit ma grotte et son contenu, ils sauront que le Professeur James Howard Smith a vécu jadis et a sauvé pour eux le legs spirituel des Anciens.\par
« Ce que l’homme futur ne manquera pas aussi de retrouver, j’en suis assuré, c’est la formule de préparation de la poudre à fusil. C’est cette poudre noirâtre qui nous permettait autrefois de tuer à longue distance. Certaines matières, que l’on retire du sol, mélangées en proportions convenables, donnent la poudre à fusil. Cela doit être expliqué dans mes livres. Mais je suis trop vieux et je manquerais, du reste, des ustensiles nécessaires pour réussir dans cette fabrication. Je le regrette. Car mon premier coup de fusil serait pour débarrasser la terre du Loucheur, de ce charlatan qui fait fleurir déjà la superstition et commence à empoisonner d’elle l’humanité qui renaît. »\par
Mais Hou-Hou protesta :\par
— Le Loucheur, dit-il, est un grand savant ! Dès que je serai homme, j’irai le trouver. Je lui donnerai toutes mes chèvres, toute la viande et toutes les fourrures que je pourrai me procurer, afin qu’il m’enseigne ses secrets et m’apprenne à devenir comme lui un docteur. Alors je serai craint et respecté, comme il l’est lui-même. Tout le monde s’aplatira à mes pieds, dans la boue.\par
Le vieillard secoua gravement la tête et murmura :\par
— Il est étrange d’entendre les mêmes idées absurdes et têtues, que formulaient les anciens hommes, tomber des lèvres d’un petit sauvage, sale et vêtu de peaux de bêtes. L’univers a été anéanti, bouleversé, et l’homme demeure toujours identique…\par
Bec-de-Lièvre intervint dans la discussion et se mit à gourmander superbement Hou-Hou :\par
— Tu ne m’en feras pas accroire, je te préviens ! dit-il. Si je te paie un jour, pour envoyer à quelqu’un le bâton de la mort, et s’il ne fonctionne pas, je te défoncerai la tête, Hou-Hou ! Oui, je te défoncerai la tête ! Tu m’entends bien ?\par
— Moi, dit Edwin doucement, je veux ne jamais oublier ce que grand-père nous a dit de la poudre à fusil. Quand j’aurai trouvé le moyen de la fabriquer, c’est moi qui vous ferai marcher tous. Toi, Bec-de-Lièvre, tu chasseras pour moi et tu me rapporteras ma viande. Et toi, Hou-Hou, quand tu seras Docteur, tu enverras le bâton de la mort où je voudrai, et chacun me craindra. Si Bec-de-Lièvre essaye de te défoncer la tête, c’est à moi qu’il aura affaire, et je le tuerai avec ma poudre. Grand-père n’est pas si sot que vous croyez. Je mettrai ses leçons à profit et je vous dominerai tous.\par
L’aïeul secoua la tête avec tristesse.\par
— La même histoire, dit-il en se parlant à lui-même, recommencera. Les hommes se multiplieront, puis ils se battront entre eux. Rien ne pourra l’empêcher. Quand ils auront retrouvé la poudre, c’est par milliers, puis par millions, qu’ils s’entre-tueront. Et c’est ainsi, par le feu et par le sang, qu’une nouvelle civilisation se formera. Peut-être lui faudra-t-il, pour atteindre son apogée, vingt mille, quarante mille, cinquante mille ans. Les trois types éternels de domination, le prêtre, le soldat, le roi y reparaîtront d’eux-mêmes. La sagesse des temps écoulés, qui sera celle des temps futurs, est sortie de la bouche de ces gamins. La masse peinera et travaillera comme par le passé. Et, sur un tas de carcasses sanglantes, croîtra toujours l’étonnante et merveilleuse beauté de la civilisation. Quand bien même je détruirais tous les livres de la grotte, le résultat serait le même. L’histoire du monde n’en reprendrait pas moins son cours éternel !\par
Bec-de-Lièvre se leva. Il regarda le soleil qui baissait de plus en plus et jeta un coup d’œil sur ses chèvres, qui continuaient à brouter paisiblement.\par
— Le vieux nous assomme, à ronchonner comme il fait. Il est aux trois quarts gâteux. Il est temps de s’en retourner au campement.\par
Aidé de Hou-Hou et des chiens, Bec-de-Lièvre rassembla les chèvres et les poussa, par la piste de la voie ferrée, vers la forêt profonde où ils disparurent.\par
Edwin, sa queue de cochon sur l’oreille, était resté seul avec l’aïeul, qui continuait à se parler à lui-même. Il observait avec amusement un petit troupeau de chevaux sauvages, qui étaient venus s’ébattre sur le sable de la grève. Il y en avait une vingtaine environ, des jeunes poulains de l’année pour la plupart, et plusieurs juments, que conduisait un superbe étalon. La bête ardente se tenait face à la mer, dans l’écume du ressac, le cou renversé et la tête levée, les yeux étincelants d’un sauvage éclair, et reniflant des naseaux l’air salé.\par
— Qu’est cela ? demanda le vieillard, en sortant enfin de sa rêverie.\par
— Ce sont des chevaux, répondit Edwin. C’est la première fois que j’en vois venir jusqu’ici. Les lions des montagnes, qui y deviennent de plus en plus nombreux, les chassent vers la mer.\par
Le soleil allait disparaître à l’horizon. Dans le ciel où roulaient des gros nuages, son disque enflammé dardait un éventail de rayons rouges. Au-delà des dunes du rivage pâle et désolé, où piaffaient les chevaux et venaient mourir les vagues, les lions marins se traînaient toujours sur les noirs récifs, ou s’ébattaient dans les flots, avec des meuglements de bataille ou d’amour, le vieux chant des premiers âges du monde.\par
— Viens, grand-père, dit Edwin, en tirant le vieillard par le bras.\par
Et tous deux, silhouettes hirsutes, vêtues de peaux, tournant le dos au rivage, emboîtèrent le pas aux chèvres, vers la forêt, sur la piste de la voie ferrée.\par
\bigbreak
\bigbreak
\chapterclose

 


% at least one empty page at end (for booklet couv)
\ifbooklet
  \pagestyle{empty}
  \clearpage
  % 2 empty pages maybe needed for 4e cover
  \ifnum\modulo{\value{page}}{4}=0 \hbox{}\newpage\hbox{}\newpage\fi
  \ifnum\modulo{\value{page}}{4}=1 \hbox{}\newpage\hbox{}\newpage\fi


  \hbox{}\newpage
  \ifodd\value{page}\hbox{}\newpage\fi
  {\centering\color{rubric}\bfseries\noindent\large
    Hurlus ? Qu’est-ce.\par
    \bigskip
  }
  \noindent Des bouquinistes électroniques, pour du texte libre à participations libres,
  téléchargeable gratuitement sur \href{https://hurlus.fr}{\dotuline{hurlus.fr}}.\par
  \bigskip
  \noindent Cette brochure a été produite par des éditeurs bénévoles.
  Elle n’est pas faite pour être possédée, mais pour être lue, et puis donnée.
  Que circule le texte !
  En page de garde, on peut ajouter une date, un lieu, un nom ;
  comme une fiche de bibliothèque en papier,
  pour suivre le voyage du texte. Qui sait, un jour, vous la retrouverez ?
  \par

  Ce texte a été choisi parce qu’une personne l’a aimé,
  ou haï, elle a pensé qu’il partipait à la formation de notre présent ;
  sans le souci de plaire, vendre, ou militer pour une cause.
  \par

  L’édition électronique est soigneuse, tant sur la technique
  que sur l’établissement du texte ; mais sans aucune prétention scolaire, au contraire.
  Le but est de s’adresser à tous, sans distinction de science ou de diplôme.
  Au plus direct ! (possible)
  \par

  Cet exemplaire en papier a été tiré sur une imprimante personnelle
   ou une photocopieuse. Tout le monde peut le faire.
  Il suffit de
  télécharger un fichier sur \href{https://hurlus.fr}{\dotuline{hurlus.fr}},
  d’imprimer, et agrafer ; puis de lire et donner.\par

  \bigskip

  \noindent PS : Les hurlus furent aussi des rebelles protestants qui cassaient les statues dans les églises catholiques. En 1566 démarra la révolte des gueux dans le pays de Lille. L’insurrection enflamma la région jusqu’à Anvers où les gueux de mer bloquèrent les bateaux espagnols.
  Ce fut une rare guerre de libération dont naquit un pays toujours libre : les Pays-Bas.
  En plat pays francophone, par contre, restèrent des bandes de huguenots, les hurlus, progressivement réprimés par la très catholique Espagne.
  Cette mémoire d’une défaite est éteinte, rallumons-la. Sortons les livres du culte universitaire, débusquons les idoles de l’époque, pour les démonter.
\fi

\end{document}
