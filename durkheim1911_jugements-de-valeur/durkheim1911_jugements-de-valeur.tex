%%%%%%%%%%%%%%%%%%%%%%%%%%%%%%%%%
% LaTeX model https://hurlus.fr %
%%%%%%%%%%%%%%%%%%%%%%%%%%%%%%%%%

% Needed before document class
\RequirePackage{pdftexcmds} % needed for tests expressions
\RequirePackage{fix-cm} % correct units

% Define mode
\def\mode{a4}

\newif\ifaiv % a4
\newif\ifav % a5
\newif\ifbooklet % booklet
\newif\ifcover % cover for booklet

\ifnum \strcmp{\mode}{cover}=0
  \covertrue
\else\ifnum \strcmp{\mode}{booklet}=0
  \booklettrue
\else\ifnum \strcmp{\mode}{a5}=0
  \avtrue
\else
  \aivtrue
\fi\fi\fi

\ifbooklet % do not enclose with {}
  \documentclass[french,twoside]{book} % ,notitlepage
  \usepackage[%
    papersize={105mm, 297mm},
    inner=12mm,
    outer=12mm,
    top=20mm,
    bottom=15mm,
    marginparsep=0pt,
  ]{geometry}
  \usepackage[fontsize=9.5pt]{scrextend} % for Roboto
\else\ifav
  \documentclass[french,twoside]{book} % ,notitlepage
  \usepackage[%
    a5paper,
    inner=25mm,
    outer=15mm,
    top=15mm,
    bottom=15mm,
    marginparsep=0pt,
  ]{geometry}
  \usepackage[fontsize=12pt]{scrextend}
\else% A4 2 cols
  \documentclass[twocolumn]{report}
  \usepackage[%
    a4paper,
    inner=15mm,
    outer=10mm,
    top=25mm,
    bottom=18mm,
    marginparsep=0pt,
  ]{geometry}
  \setlength{\columnsep}{20mm}
  \usepackage[fontsize=9.5pt]{scrextend}
\fi\fi

%%%%%%%%%%%%%%
% Alignments %
%%%%%%%%%%%%%%
% before teinte macros

\setlength{\arrayrulewidth}{0.2pt}
\setlength{\columnseprule}{\arrayrulewidth} % twocol
\setlength{\parskip}{0pt} % classical para with no margin
\setlength{\parindent}{1.5em}

%%%%%%%%%%
% Colors %
%%%%%%%%%%
% before Teinte macros

\usepackage[dvipsnames]{xcolor}
\definecolor{rubric}{HTML}{800000} % the tonic 0c71c3
\def\columnseprulecolor{\color{rubric}}
\colorlet{borderline}{rubric!30!} % definecolor need exact code
\definecolor{shadecolor}{gray}{0.95}
\definecolor{bghi}{gray}{0.5}

%%%%%%%%%%%%%%%%%
% Teinte macros %
%%%%%%%%%%%%%%%%%
%%%%%%%%%%%%%%%%%%%%%%%%%%%%%%%%%%%%%%%%%%%%%%%%%%%
% <TEI> generic (LaTeX names generated by Teinte) %
%%%%%%%%%%%%%%%%%%%%%%%%%%%%%%%%%%%%%%%%%%%%%%%%%%%
% This template is inserted in a specific design
% It is XeLaTeX and otf fonts

\makeatletter % <@@@


\usepackage{blindtext} % generate text for testing
\usepackage[strict]{changepage} % for modulo 4
\usepackage{contour} % rounding words
\usepackage[nodayofweek]{datetime}
% \usepackage{DejaVuSans} % seems buggy for sffont font for symbols
\usepackage{enumitem} % <list>
\usepackage{etoolbox} % patch commands
\usepackage{fancyvrb}
\usepackage{fancyhdr}
\usepackage{float}
\usepackage{fontspec} % XeLaTeX mandatory for fonts
\usepackage{footnote} % used to capture notes in minipage (ex: quote)
\usepackage{framed} % bordering correct with footnote hack
\usepackage{graphicx}
\usepackage{lettrine} % drop caps
\usepackage{lipsum} % generate text for testing
\usepackage[framemethod=tikz,]{mdframed} % maybe used for frame with footnotes inside
\usepackage{pdftexcmds} % needed for tests expressions
\usepackage{polyglossia} % non-break space french punct, bug Warning: "Failed to patch part"
\usepackage[%
  indentfirst=false,
  vskip=1em,
  noorphanfirst=true,
  noorphanafter=true,
  leftmargin=\parindent,
  rightmargin=0pt,
]{quoting}
\usepackage{ragged2e}
\usepackage{setspace} % \setstretch for <quote>
\usepackage{tabularx} % <table>
\usepackage[explicit]{titlesec} % wear titles, !NO implicit
\usepackage{tikz} % ornaments
\usepackage{tocloft} % styling tocs
\usepackage[fit]{truncate} % used im runing titles
\usepackage{unicode-math}
\usepackage[normalem]{ulem} % breakable \uline, normalem is absolutely necessary to keep \emph
\usepackage{verse} % <l>
\usepackage{xcolor} % named colors
\usepackage{xparse} % @ifundefined
\XeTeXdefaultencoding "iso-8859-1" % bad encoding of xstring
\usepackage{xstring} % string tests
\XeTeXdefaultencoding "utf-8"
\PassOptionsToPackage{hyphens}{url} % before hyperref, which load url package

% TOTEST
% \usepackage{hypcap} % links in caption ?
% \usepackage{marginnote}
% TESTED
% \usepackage{background} % doesn’t work with xetek
% \usepackage{bookmark} % prefers the hyperref hack \phantomsection
% \usepackage[color, leftbars]{changebar} % 2 cols doc, impossible to keep bar left
% \usepackage[utf8x]{inputenc} % inputenc package ignored with utf8 based engines
% \usepackage[sfdefault,medium]{inter} % no small caps
% \usepackage{firamath} % choose firasans instead, firamath unavailable in Ubuntu 21-04
% \usepackage{flushend} % bad for last notes, supposed flush end of columns
% \usepackage[stable]{footmisc} % BAD for complex notes https://texfaq.org/FAQ-ftnsect
% \usepackage{helvet} % not for XeLaTeX
% \usepackage{multicol} % not compatible with too much packages (longtable, framed, memoir…)
% \usepackage[default,oldstyle,scale=0.95]{opensans} % no small caps
% \usepackage{sectsty} % \chapterfont OBSOLETE
% \usepackage{soul} % \ul for underline, OBSOLETE with XeTeX
% \usepackage[breakable]{tcolorbox} % text styling gone, footnote hack not kept with breakable


% Metadata inserted by a program, from the TEI source, for title page and runing heads
\title{\textbf{ Jugements de valeur et jugements de réalité }}
\date{1911}
\author{Émile Durkheim}
\def\elbibl{Émile Durkheim. 1911. \emph{Jugements de valeur et jugements de réalité}}
\def\elsource{Émile Durkheim, « Jugements de valeur et jugements de réalité », Communication faite au Congrès international de Philosophie de Bologne le 6 avril 1911, in \emph{{\itshape Revue de Métaphysique et de Morale}}, 3 juillet 1911.}

% Default metas
\newcommand{\colorprovide}[2]{\@ifundefinedcolor{#1}{\colorlet{#1}{#2}}{}}
\colorprovide{rubric}{red}
\colorprovide{silver}{lightgray}
\@ifundefined{syms}{\newfontfamily\syms{DejaVu Sans}}{}
\newif\ifdev
\@ifundefined{elbibl}{% No meta defined, maybe dev mode
  \newcommand{\elbibl}{Titre court ?}
  \newcommand{\elbook}{Titre du livre source ?}
  \newcommand{\elabstract}{Résumé\par}
  \newcommand{\elurl}{http://oeuvres.github.io/elbook/2}
  \author{Éric Lœchien}
  \title{Un titre de test assez long pour vérifier le comportement d’une maquette}
  \date{1566}
  \devtrue
}{}
\let\eltitle\@title
\let\elauthor\@author
\let\eldate\@date


\defaultfontfeatures{
  % Mapping=tex-text, % no effect seen
  Scale=MatchLowercase,
  Ligatures={TeX,Common},
}


% generic typo commands
\newcommand{\astermono}{\medskip\centerline{\color{rubric}\large\selectfont{\syms ✻}}\medskip\par}%
\newcommand{\astertri}{\medskip\par\centerline{\color{rubric}\large\selectfont{\syms ✻\,✻\,✻}}\medskip\par}%
\newcommand{\asterism}{\bigskip\par\noindent\parbox{\linewidth}{\centering\color{rubric}\large{\syms ✻}\\{\syms ✻}\hskip 0.75em{\syms ✻}}\bigskip\par}%

% lists
\newlength{\listmod}
\setlength{\listmod}{\parindent}
\setlist{
  itemindent=!,
  listparindent=\listmod,
  labelsep=0.2\listmod,
  parsep=0pt,
  % topsep=0.2em, % default topsep is best
}
\setlist[itemize]{
  label=—,
  leftmargin=0pt,
  labelindent=1.2em,
  labelwidth=0pt,
}
\setlist[enumerate]{
  label={\bf\color{rubric}\arabic*.},
  labelindent=0.8\listmod,
  leftmargin=\listmod,
  labelwidth=0pt,
}
\newlist{listalpha}{enumerate}{1}
\setlist[listalpha]{
  label={\bf\color{rubric}\alph*.},
  leftmargin=0pt,
  labelindent=0.8\listmod,
  labelwidth=0pt,
}
\newcommand{\listhead}[1]{\hspace{-1\listmod}\emph{#1}}

\renewcommand{\hrulefill}{%
  \leavevmode\leaders\hrule height 0.2pt\hfill\kern\z@}

% General typo
\DeclareTextFontCommand{\textlarge}{\large}
\DeclareTextFontCommand{\textsmall}{\small}

% commands, inlines
\newcommand{\anchor}[1]{\Hy@raisedlink{\hypertarget{#1}{}}} % link to top of an anchor (not baseline)
\newcommand\abbr[1]{#1}
\newcommand{\autour}[1]{\tikz[baseline=(X.base)]\node [draw=rubric,thin,rectangle,inner sep=1.5pt, rounded corners=3pt] (X) {\color{rubric}#1};}
\newcommand\corr[1]{#1}
\newcommand{\ed}[1]{ {\color{silver}\sffamily\footnotesize (#1)} } % <milestone ed="1688"/>
\newcommand\expan[1]{#1}
\newcommand\foreign[1]{\emph{#1}}
\newcommand\gap[1]{#1}
\renewcommand{\LettrineFontHook}{\color{rubric}}
\newcommand{\initial}[2]{\lettrine[lines=2, loversize=0.3, lhang=0.3]{#1}{#2}}
\newcommand{\initialiv}[2]{%
  \let\oldLFH\LettrineFontHook
  % \renewcommand{\LettrineFontHook}{\color{rubric}\ttfamily}
  \IfSubStr{QJ’}{#1}{
    \lettrine[lines=4, lhang=0.2, loversize=-0.1, lraise=0.2]{\smash{#1}}{#2}
  }{\IfSubStr{É}{#1}{
    \lettrine[lines=4, lhang=0.2, loversize=-0, lraise=0]{\smash{#1}}{#2}
  }{\IfSubStr{ÀÂ}{#1}{
    \lettrine[lines=4, lhang=0.2, loversize=-0, lraise=0, slope=0.6em]{\smash{#1}}{#2}
  }{\IfSubStr{A}{#1}{
    \lettrine[lines=4, lhang=0.2, loversize=0.2, slope=0.6em]{\smash{#1}}{#2}
  }{\IfSubStr{V}{#1}{
    \lettrine[lines=4, lhang=0.2, loversize=0.2, slope=-0.5em]{\smash{#1}}{#2}
  }{
    \lettrine[lines=4, lhang=0.2, loversize=0.2]{\smash{#1}}{#2}
  }}}}}
  \let\LettrineFontHook\oldLFH
}
\newcommand{\labelchar}[1]{\textbf{\color{rubric} #1}}
\newcommand{\milestone}[1]{\autour{\footnotesize\color{rubric} #1}} % <milestone n="4"/>
\newcommand\name[1]{#1}
\newcommand\orig[1]{#1}
\newcommand\orgName[1]{#1}
\newcommand\persName[1]{#1}
\newcommand\placeName[1]{#1}
\newcommand{\pn}[1]{\IfSubStr{-—–¶}{#1}% <p n="3"/>
  {\noindent{\bfseries\color{rubric}   ¶  }}
  {{\footnotesize\autour{ #1}  }}}
\newcommand\reg{}
% \newcommand\ref{} % already defined
\newcommand\sic[1]{#1}
\newcommand\surname[1]{\textsc{#1}}
\newcommand\term[1]{\textbf{#1}}

\def\mednobreak{\ifdim\lastskip<\medskipamount
  \removelastskip\nopagebreak\medskip\fi}
\def\bignobreak{\ifdim\lastskip<\bigskipamount
  \removelastskip\nopagebreak\bigskip\fi}

% commands, blocks
\newcommand{\byline}[1]{\bigskip{\RaggedLeft{#1}\par}\bigskip}
\newcommand{\bibl}[1]{{\RaggedLeft{#1}\par\bigskip}}
\newcommand{\biblitem}[1]{{\noindent\hangindent=\parindent   #1\par}}
\newcommand{\dateline}[1]{\medskip{\RaggedLeft{#1}\par}\bigskip}
\newcommand{\labelblock}[1]{\medbreak{\noindent\color{rubric}\bfseries #1}\par\mednobreak}
\newcommand{\salute}[1]{\bigbreak{#1}\par\medbreak}
\newcommand{\signed}[1]{\bigbreak\filbreak{\raggedleft #1\par}\medskip}

% environments for blocks (some may become commands)
\newenvironment{borderbox}{}{} % framing content
\newenvironment{citbibl}{\ifvmode\hfill\fi}{\ifvmode\par\fi }
\newenvironment{docAuthor}{\ifvmode\vskip4pt\fontsize{16pt}{18pt}\selectfont\fi\itshape}{\ifvmode\par\fi }
\newenvironment{docDate}{}{\ifvmode\par\fi }
\newenvironment{docImprint}{\vskip6pt}{\ifvmode\par\fi }
\newenvironment{docTitle}{\vskip6pt\bfseries\fontsize{18pt}{22pt}\selectfont}{\par }
\newenvironment{msHead}{\vskip6pt}{\par}
\newenvironment{msItem}{\vskip6pt}{\par}
\newenvironment{titlePart}{}{\par }


% environments for block containers
\newenvironment{argument}{\itshape\parindent0pt}{\vskip1.5em}
\newenvironment{biblfree}{}{\ifvmode\par\fi }
\newenvironment{bibitemlist}[1]{%
  \list{\@biblabel{\@arabic\c@enumiv}}%
  {%
    \settowidth\labelwidth{\@biblabel{#1}}%
    \leftmargin\labelwidth
    \advance\leftmargin\labelsep
    \@openbib@code
    \usecounter{enumiv}%
    \let\p@enumiv\@empty
    \renewcommand\theenumiv{\@arabic\c@enumiv}%
  }
  \sloppy
  \clubpenalty4000
  \@clubpenalty \clubpenalty
  \widowpenalty4000%
  \sfcode`\.\@m
}%
{\def\@noitemerr
  {\@latex@warning{Empty `bibitemlist' environment}}%
\endlist}
\newenvironment{quoteblock}% may be used for ornaments
  {\begin{quoting}}
  {\end{quoting}}

% table () is preceded and finished by custom command
\newcommand{\tableopen}[1]{%
  \ifnum\strcmp{#1}{wide}=0{%
    \begin{center}
  }
  \else\ifnum\strcmp{#1}{long}=0{%
    \begin{center}
  }
  \else{%
    \begin{center}
  }
  \fi\fi
}
\newcommand{\tableclose}[1]{%
  \ifnum\strcmp{#1}{wide}=0{%
    \end{center}
  }
  \else\ifnum\strcmp{#1}{long}=0{%
    \end{center}
  }
  \else{%
    \end{center}
  }
  \fi\fi
}


% text structure
\newcommand\chapteropen{} % before chapter title
\newcommand\chaptercont{} % after title, argument, epigraph…
\newcommand\chapterclose{} % maybe useful for multicol settings
\setcounter{secnumdepth}{-2} % no counters for hierarchy titles
\setcounter{tocdepth}{5} % deep toc
\markright{\@title} % ???
\markboth{\@title}{\@author} % ???
\renewcommand\tableofcontents{\@starttoc{toc}}
% toclof format
% \renewcommand{\@tocrmarg}{0.1em} % Useless command?
% \renewcommand{\@pnumwidth}{0.5em} % {1.75em}
\renewcommand{\@cftmaketoctitle}{}
\setlength{\cftbeforesecskip}{\z@ \@plus.2\p@}
\renewcommand{\cftchapfont}{}
\renewcommand{\cftchapdotsep}{\cftdotsep}
\renewcommand{\cftchapleader}{\normalfont\cftdotfill{\cftchapdotsep}}
\renewcommand{\cftchappagefont}{\bfseries}
\setlength{\cftbeforechapskip}{0em \@plus\p@}
% \renewcommand{\cftsecfont}{\small\relax}
\renewcommand{\cftsecpagefont}{\normalfont}
% \renewcommand{\cftsubsecfont}{\small\relax}
\renewcommand{\cftsecdotsep}{\cftdotsep}
\renewcommand{\cftsecpagefont}{\normalfont}
\renewcommand{\cftsecleader}{\normalfont\cftdotfill{\cftsecdotsep}}
\setlength{\cftsecindent}{1em}
\setlength{\cftsubsecindent}{2em}
\setlength{\cftsubsubsecindent}{3em}
\setlength{\cftchapnumwidth}{1em}
\setlength{\cftsecnumwidth}{1em}
\setlength{\cftsubsecnumwidth}{1em}
\setlength{\cftsubsubsecnumwidth}{1em}

% footnotes
\newif\ifheading
\newcommand*{\fnmarkscale}{\ifheading 0.70 \else 1 \fi}
\renewcommand\footnoterule{\vspace*{0.3cm}\hrule height \arrayrulewidth width 3cm \vspace*{0.3cm}}
\setlength\footnotesep{1.5\footnotesep} % footnote separator
\renewcommand\@makefntext[1]{\parindent 1.5em \noindent \hb@xt@1.8em{\hss{\normalfont\@thefnmark . }}#1} % no superscipt in foot
\patchcmd{\@footnotetext}{\footnotesize}{\footnotesize\sffamily}{}{} % before scrextend, hyperref


%   see https://tex.stackexchange.com/a/34449/5049
\def\truncdiv#1#2{((#1-(#2-1)/2)/#2)}
\def\moduloop#1#2{(#1-\truncdiv{#1}{#2}*#2)}
\def\modulo#1#2{\number\numexpr\moduloop{#1}{#2}\relax}

% orphans and widows
\clubpenalty=9996
\widowpenalty=9999
\brokenpenalty=4991
\predisplaypenalty=10000
\postdisplaypenalty=1549
\displaywidowpenalty=1602
\hyphenpenalty=400
% Copied from Rahtz but not understood
\def\@pnumwidth{1.55em}
\def\@tocrmarg {2.55em}
\def\@dotsep{4.5}
\emergencystretch 3em
\hbadness=4000
\pretolerance=750
\tolerance=2000
\vbadness=4000
\def\Gin@extensions{.pdf,.png,.jpg,.mps,.tif}
% \renewcommand{\@cite}[1]{#1} % biblio

\usepackage{hyperref} % supposed to be the last one, :o) except for the ones to follow
\urlstyle{same} % after hyperref
\hypersetup{
  % pdftex, % no effect
  pdftitle={\elbibl},
  % pdfauthor={Your name here},
  % pdfsubject={Your subject here},
  % pdfkeywords={keyword1, keyword2},
  bookmarksnumbered=true,
  bookmarksopen=true,
  bookmarksopenlevel=1,
  pdfstartview=Fit,
  breaklinks=true, % avoid long links
  pdfpagemode=UseOutlines,    % pdf toc
  hyperfootnotes=true,
  colorlinks=false,
  pdfborder=0 0 0,
  % pdfpagelayout=TwoPageRight,
  % linktocpage=true, % NO, toc, link only on page no
}

\makeatother % /@@@>
%%%%%%%%%%%%%%
% </TEI> end %
%%%%%%%%%%%%%%


%%%%%%%%%%%%%
% footnotes %
%%%%%%%%%%%%%
\renewcommand{\thefootnote}{\bfseries\textcolor{rubric}{\arabic{footnote}}} % color for footnote marks

%%%%%%%%%
% Fonts %
%%%%%%%%%
\usepackage[]{roboto} % SmallCaps, Regular is a bit bold
% \linespread{0.90} % too compact, keep font natural
\newfontfamily\fontrun[]{Roboto Condensed Light} % condensed runing heads
\ifav
  \setmainfont[
    ItalicFont={Roboto Light Italic},
  ]{Roboto}
\else\ifbooklet
  \setmainfont[
    ItalicFont={Roboto Light Italic},
  ]{Roboto}
\else
\setmainfont[
  ItalicFont={Roboto Italic},
]{Roboto Light}
\fi\fi
\renewcommand{\LettrineFontHook}{\bfseries\color{rubric}}
% \renewenvironment{labelblock}{\begin{center}\bfseries\color{rubric}}{\end{center}}

%%%%%%%%
% MISC %
%%%%%%%%

\setdefaultlanguage[frenchpart=false]{french} % bug on part


\newenvironment{quotebar}{%
    \def\FrameCommand{{\color{rubric!10!}\vrule width 0.5em} \hspace{0.9em}}%
    \def\OuterFrameSep{\itemsep} % séparateur vertical
    \MakeFramed {\advance\hsize-\width \FrameRestore}
  }%
  {%
    \endMakeFramed
  }
\renewenvironment{quoteblock}% may be used for ornaments
  {%
    \savenotes
    \setstretch{0.9}
    \normalfont
    \begin{quotebar}
  }
  {%
    \end{quotebar}
    \spewnotes
  }


\renewcommand{\headrulewidth}{\arrayrulewidth}
\renewcommand{\headrule}{{\color{rubric}\hrule}}

% delicate tuning, image has produce line-height problems in title on 2 lines
\titleformat{name=\chapter} % command
  [display] % shape
  {\vspace{1.5em}\centering} % format
  {} % label
  {0pt} % separator between n
  {}
[{\color{rubric}\huge\textbf{#1}}\bigskip] % after code
% \titlespacing{command}{left spacing}{before spacing}{after spacing}[right]
\titlespacing*{\chapter}{0pt}{-2em}{0pt}[0pt]

\titleformat{name=\section}
  [block]{}{}{}{}
  [\vbox{\color{rubric}\large\raggedleft\textbf{#1}}]
\titlespacing{\section}{0pt}{0pt plus 4pt minus 2pt}{\baselineskip}

\titleformat{name=\subsection}
  [block]
  {}
  {} % \thesection
  {} % separator \arrayrulewidth
  {}
[\vbox{\large\textbf{#1}}]
% \titlespacing{\subsection}{0pt}{0pt plus 4pt minus 2pt}{\baselineskip}

\ifaiv
  \fancypagestyle{main}{%
    \fancyhf{}
    \setlength{\headheight}{1.5em}
    \fancyhead{} % reset head
    \fancyfoot{} % reset foot
    \fancyhead[L]{\truncate{0.45\headwidth}{\fontrun\elbibl}} % book ref
    \fancyhead[R]{\truncate{0.45\headwidth}{ \fontrun\nouppercase\leftmark}} % Chapter title
    \fancyhead[C]{\thepage}
  }
  \fancypagestyle{plain}{% apply to chapter
    \fancyhf{}% clear all header and footer fields
    \setlength{\headheight}{1.5em}
    \fancyhead[L]{\truncate{0.9\headwidth}{\fontrun\elbibl}}
    \fancyhead[R]{\thepage}
  }
\else
  \fancypagestyle{main}{%
    \fancyhf{}
    \setlength{\headheight}{1.5em}
    \fancyhead{} % reset head
    \fancyfoot{} % reset foot
    \fancyhead[RE]{\truncate{0.9\headwidth}{\fontrun\elbibl}} % book ref
    \fancyhead[LO]{\truncate{0.9\headwidth}{\fontrun\nouppercase\leftmark}} % Chapter title, \nouppercase needed
    \fancyhead[RO,LE]{\thepage}
  }
  \fancypagestyle{plain}{% apply to chapter
    \fancyhf{}% clear all header and footer fields
    \setlength{\headheight}{1.5em}
    \fancyhead[L]{\truncate{0.9\headwidth}{\fontrun\elbibl}}
    \fancyhead[R]{\thepage}
  }
\fi

\ifav % a5 only
  \titleclass{\section}{top}
\fi

\newcommand\chapo{{%
  \vspace*{-3em}
  \centering % no vskip ()
  {\Large\addfontfeature{LetterSpace=25}\bfseries{\elauthor}}\par
  \smallskip
  {\large\eldate}\par
  \bigskip
  {\Large\selectfont{\eltitle}}\par
  \bigskip
  {\color{rubric}\hline\par}
  \bigskip
  {\Large TEXTE LIBRE À PARTICPATION LIBRE\par}
  \centerline{\small\color{rubric} {hurlus.fr, tiré le \today}}\par
  \bigskip
}}

\newcommand\cover{{%
  \thispagestyle{empty}
  \centering
  {\LARGE\bfseries{\elauthor}}\par
  \bigskip
  {\Large\eldate}\par
  \bigskip
  \bigskip
  {\LARGE\selectfont{\eltitle}}\par
  \vfill\null
  {\color{rubric}\setlength{\arrayrulewidth}{2pt}\hline\par}
  \vfill\null
  {\Large TEXTE LIBRE À PARTICPATION LIBRE\par}
  \centerline{{\href{https://hurlus.fr}{\dotuline{hurlus.fr}}, tiré le \today}}\par
}}

\begin{document}
\pagestyle{empty}
\ifbooklet{
  \cover\newpage
  \thispagestyle{empty}\hbox{}\newpage
  \cover\newpage\noindent Les voyages de la brochure\par
  \bigskip
  \begin{tabularx}{\textwidth}{l|X|X}
    \textbf{Date} & \textbf{Lieu}& \textbf{Nom/pseudo} \\ \hline
    \rule{0pt}{25cm} &  &   \\
  \end{tabularx}
  \newpage
  \addtocounter{page}{-4}
}\fi

\thispagestyle{empty}
\ifaiv
  \twocolumn[\chapo]
\else
  \chapo
\fi
{\it\elabstract}
\bigskip
\makeatletter\@starttoc{toc}\makeatother % toc without new page
\bigskip

\pagestyle{main} % after style

  \section[{« Jugements de valeur et jugements de réalité »}]{« Jugements de valeur et jugements de réalité »}\renewcommand{\leftmark}{« Jugements de valeur et jugements de réalité »}

\noindent En soumettant au Congrès ce thème de discussion, je me suis proposé un double but : d’abord, montrer sur un exemple particulier comment la sociologie peut aider à résoudre un problème philosophique ; ensuite, dissiper certains préjugés dont la sociologie, dite positive, est trop souvent l’objet.\par
Quand nous disons que les corps sont pesants, que le volume des gaz varie en raison inverse de la pression qu’ils subissent, nous formulons des jugements qui se bornent à exprimer des faits donnés. Ils énoncent ce qui est et, pour cette raison, on les appelle jugements d’existence ou de réalité.\par
D’autres jugements ont pour objet de dire non ce que sont les choses, mais ce qu’elles valent par rapport à un sujet conscient, le prix que ce dernier y attache : on leur donne le nom de jugements de valeur. On étend même parfois cette dénomination à tout jugement qui énonce une estimation, quelle qu’elle puisse être. Mais cette extension peut donner lieu à des confusions qu’il importe de prévenir.\par
Quand je dis : j’aime la chasse, je préfère la bière au vin, la vie active au repos, etc., j’émets des jugements qui peuvent paraître exprimer des estimations, mais qui sont, au fond, de simples jugements de réalité. Ils disent uniquement de quelle façon nous nous comportons vis-à-vis de certains objets ; que nous aimons ceux-ci, que nous préférons ceux-là. Ces préférences sont des faits aussi bien que la pesanteur des corps ou que l’élasticité des gaz. De semblables jugements n’ont donc pas pour fonction d’attribuer aux choses une valeur qui leur appartienne, mais seulement d’affirmer des états déterminés du sujet. Aussi les prédilections qui sont ainsi exprimées sont-elles incommunicables. Ceux qui les éprouvent peuvent bien dire qu’ils les éprouvent ou, tout au moins, qu’ils croient les éprouver ; mais ils ne peuvent les transmettre à autrui. Elles tiennent à leurs personnes et n’en peuvent être détachées.\par
Il en va tout autrement quand je dis : cet homme a une haute valeur morale ; ce tableau a une grande valeur esthétique ; ce bijou vaut tant. Dans tous ces cas, j’attribue aux êtres ou aux choses dont il s’agit un caractère objectif, tout à fait indépendant de la manière dont je le sens au moment où je me prononce. Personnellement, je puis n’attacher aux bijoux aucun prix ; leur valeur n’en reste pas moins ce qu’elle est au moment considéré. Je puis, comme homme, n’avoir qu’une médiocre moralité ; cela ne m’empêche pas de reconnaître la valeur morale là où elle est. Je puis être, par tempérament, peu sensible aux joies de l’art ; ce n’est pas une raison pour que je nie qu’il y ait des valeurs esthétiques. Toutes ces valeurs existent donc, en un sens, en dehors de moi. Aussi, quand nous sommes en désaccord avec autrui sur la manière de les concevoir et de les estimer, tentons-nous de lui communiquer nos convictions. Nous ne nous contentons pas de les affirmer ; nous cherchons à les démontrer en donnant, à l’appui de nos affirmations, des raisons d’ordre impersonnel. Nous admettons donc implicitement que ces jugements correspondent à quelque réalité objective sur laquelle l’entente peut et doit se faire. Ce sont ces réalités {\itshape sui generis} qui constituent des valeurs, et les jugements de valeur sont ceux qui se rapportent à ces réalités.\par
Nous voudrions rechercher comment ces sortes de jugements sont possibles. On voit, par ce qui précède, comment se pose la question. D’une part, toute valeur suppose l’appréciation d’un sujet, en rapport défini avec une sensibilité déterminée. Ce qui a de la valeur est bon à quelque titre ; ce qui est bon est désirable ; tout désir est un état intérieur. Et pourtant les valeurs dont il vient d’être question ont la même objectivité que des choses. Comment ces deux caractères, qui, au premier abord, semblent contradictoires, peuvent-ils se concilier ? Comment un état de sentiment peut-il être indépendant du sujet qui l’éprouve ?\par
Deux solutions contraires ont été données à ce problème.\par
\subsection[{I}]{I}
\noindent Pour nombre de penseurs, qui se recrutent, d’ailleurs, dans des milieux assez hétérogènes, la différence entre ces deux espèces de jugements est purement apparente. La valeur, dit-on, tient essentiellement à quelque caractère constitutif de la chose à laquelle elle est attribuée, et le jugement de valeur ne ferait qu’exprimer la manière dont ce caractère agit sur le sujet qui juge. Si cette action est favorable, la valeur est positive ; elle est négative, dans le cas contraire. Si la vie a de la valeur pour l’homme, c’est que l’homme est un être vivant et qu’il est dans la nature du vivant de vivre. Si le blé a de la valeur, c’est qu’il sert à l’alimentation et entretient la vie. Si la justice est une vertu, c’est parce qu’elle respecte les nécessités vitales ; l’homicide est un crime pour la raison opposée. En somme, la valeur d’une chose serait simplement la constatation des effets qu’elle produit en raison de ses propriétés intrinsèques.\par
Mais quel est le sujet par rapport auquel la valeur des choses est et doit être estimée ?\par
Sera-ce l’individu ? Comment expliquer alors qu’il puisse exister un système de valeurs objectives, reconnues par tous les hommes, au moins par tous les hommes d’une même civilisation ? Ce qui fait la valeur de ce point de vue, c’est l’effet de la chose sur la sensibilité : or, on sait combien est grande la diversité des sensibilités individuelles. Ce qui plaît aux uns répugne aux autres. La vie elle-même n’est pas voulue par tous, puisqu’il y a des hommes qui s’en défont, soit par dégoût, soit par devoir. Surtout, quel désaccord dans la manière de l’entendre. Celui-ci la veut intense ; celui-là met sa joie à la réduire et à la simplifier. Cette objection a été trop souvent faite aux morales utilitaires pour qu’il y ait lieu de la développer ; nous remarquons seulement qu’elle s’applique également à toute théorie qui prétend expliquer, par des causes purement psychologiques, les valeurs économiques, esthétiques ou spéculatives. Dira-t-on qu’il y a un type moyen qui se retrouve dans la plupart des individus et que l’estimation objective des choses exprime la façon dont elles agissent sur l’individu moyen ? Mais l’écart est énorme entre la manière dont les valeurs sont, en fait, estimées par l’individu ordinaire et cette échelle objective des valeurs humaines sur laquelle doivent, en principe, se régler nos jugements. La conscience morale moyenne est médiocre ; elle ne sent que faiblement les devoirs même usuels et, par suite, les valeurs morales correspondantes ; il en est même pour lesquelles elle est frappée d’une sorte de cécité. Ce n’est donc pas elle qui peut nous fournir l’étalon de la moralité. À plus forte raison en est-il ainsi des valeurs esthétiques qui sont lettre morte pour le plus grand nombre. Pour ce qui concerne les valeurs économiques, la distance, dans certains cas, est peut-être moins considérable. Cependant, ce n’est évidemment pas la manière dont les propriétés physiques du diamant ou de la perle agissent sur la généralité de nos contemporains qui peut servir à en déterminer la valeur actuelle.\par
Il y a, d’ailleurs, une autre raison qui ne permet pas de confondre l’estimation objective et l’estimation moyenne c’est que les réactions de l’individu moyen restent des réactions individuelles. Parce qu’un état se trouve dans un grand nombre de sujets, il n’est pas, pour cela, objectif. De ce que nous sommes plusieurs à apprécier une chose de la même manière, il ne suit pas que cette appréciation nous soit imposée par quelque réalité extérieure. Cette rencontre peut être due à des causes toutes subjectives, notamment à une suffisante homogénéité des tempéraments individuels. Entre ces deux propositions : J’aime {\itshape ceci} et {\itshape Nous sommes un certain nombre à aimer ceci}, il n’y a pas de différence essentielle.\par
On a cru pouvoir échapper à ces difficultés en substituant la société à l’individu. Tout comme dans la thèse précédente, on maintient que la valeur tient essentiellement à quelque élément intégrant de la chose. Mais c’est la manière dont la chose affecterait le sujet collectif, et non plus le sujet individuel, qui en ferait la valeur. L’estimation serait objective par cela seul qu’elle serait collective.\par
Cette explication a sur la précédente d’incontestables avantages. En effet, le jugement social est objectif par rapport aux jugements individuels ; l’échelle des valeurs se trouve ainsi soustraite aux appréciations subjectives et variables des individus : ceux-ci trouvent en dehors d’eux une classification tout établie, qui n’est pas leur œuvre, qui exprime tout autre chose que leurs sentiments personnels et à laquelle ils sont tenus de se conformer. Car l’opinion publique tient de ses origines une autorité morale en vertu de laquelle elle s’impose aux particuliers. Elle résiste aux efforts qui sont faits pour lui faire violence ; elle réagit contre les dissidents tout comme le monde extérieur réagit douloureusement contre ceux qui tentent de se rebeller contre lui. Elle blâme ceux qui jugent des choses morales d’après des principes différents de ceux qu’elle prescrit ; elle ridiculise ceux qui s’inspirent d’une autre esthétique que la sienne. Quiconque essaie d’avoir une chose à un prix inférieur à sa valeur se heurte à des résistances comparables à celles que nous opposent les corps quand nous méconnaissons leur nature. Ainsi peut s’expliquer l’espèce de nécessité que nous subissons et dont nous avons conscience quand nous émettons des jugements de valeurs. Nous sentons bien que nous ne sommes pas maîtres de nos appréciations ; que nous sommes liés et contraints. C’est la conscience publique qui nous lie. Il est vrai que cet aspect des jugements de valeurs n’est pas le seul ; il en est un autre qui est presque l’opposé du premier. Ces mêmes valeurs qui, par certains côtés, nous font l’effet de réalités qui s’imposent à nous, nous apparaissent en même temps comme des choses désirables que nous aimons et voulons spontanément. Mais c’est que la société, en même temps qu’elle est la législatrice à laquelle nous devons le respect, est la créatrice et la dépositaire de tous ces biens de la civilisation auxquels nous sommes attachés de toutes les forces de notre âme. Elle est bonne et secourable en même temps qu’impérative. Tout ce qui accroît sa vitalité relève la nôtre. Il n’est donc pas surprenant que nous tenions à tout ce à quoi elle tient.\par
Mais, ainsi comprise, une théorie sociologique des valeurs soulève à son tour de graves difficultés qui, d’ailleurs, ne lui sont pas spéciales ; car elles peuvent être également objectées à la théorie psychologique dont il était précédemment question.\par
Il existe des types différents de valeurs. Autre chose est la valeur économique, autre chose les valeurs morales, religieuses, esthétiques, spéculatives. Les tentatives si souvent faites en vue de réduire les unes aux autres les idées de bien, de beau, de vrai et d’utile sont toujours restées vaines. Or, si ce qui fait la valeur, c’est uniquement la manière dont les choses affectent le fonctionnement de la vie sociale, la diversité des valeurs devient difficilement explicable. Si c’est la même cause qui est partout agissante, d’où vient que les effets sont spécifiquement différents ?\par
D’autre part, si vraiment la valeur des choses se mesurait d’après le degré de leur utilité sociale (ou individuelle), le système des valeurs humaines devrait être révisé et bouleversé de fond en comble ; car la place qui y est faite aux valeurs de luxe serait, de ce point de vue, incompréhensible et injustifiable. Par définition, ce qui est superflu n’est pas, ou est moins utile que ce qui est nécessaire. Ce qui est surérogatoire peut manquer sans gêner gravement le jeu des fonctions vitales. En un mot, les valeurs de luxe sont dispendieuses par nature ; elles coûtent plus qu’elles ne rapportent. Aussi se rencontre-t-il des doctrinaires qui les regardent d’un œil défiant et qui s’efforcent de les réduire à la portion congrue. Mais, en fait, il n’en est pas qui aient plus de prix aux yeux des hommes. L’art tout entier est chose de luxe ; l’activité esthétique ne se subordonne à aucune fin utile ; elle se déploie pour le seul plaisir de se déployer. De même, la pure spéculation, c’est la pensée affranchie de toute fin utilitaire et s’exerçant dans le seul but de s’exercer. Qui peut contester pourtant que, de tout temps, l’humanité a mis les valeurs artistiques et spéculatives bien au-dessus des valeurs économiques ? Tout comme la vie intellectuelle, la vie morale a son esthétique qui lui est propre. Les vertus les plus hautes ne consistent pas dans l’accomplissement régulier et strict des actes le plus immédiatement nécessaires au bon ordre social ; mais elles sont faites de mouvements libres et spontanés, de sacrifices que rien ne nécessite et qui même sont parfois contraires aux préceptes d’une sage économie. Il y a des vertus qui sont des folies, et c’est leur folie qui fait leur grandeur. Spencer a pu démontrer que la philanthropie est souvent contraire à l’intérêt bien entendu de la société ; sa démonstration n’empêchera pas les hommes de mettre très haut dans leur estime la vertu qu’il condamne. La vie économique elle-même ne s’astreint pas étroitement à la règle de l’économie. Si les choses de luxe sont celles qui coûtent le plus cher, ce n’est pas seulement parce qu’en général elles sont les plus rares ; c’est aussi parce qu’elles sont les plus estimées. C’est que la vie, telle que l’ont conçue les hommes de tous les temps, ne consiste pas simplement à établir exactement le budget de l’organisme individuel ou social, à répondre, avec le moins de frais possible, aux excitations venues du dehors, à bien proportionner les dépenses aux réparations. Vivre, c’est, avant tout, agir, agir sans compter, pour le plaisir d’agir. Et si, de toute évidence, on ne peut se passer d’économie, s’il faut amasser pour pouvoir dépenser, c’est pourtant la dépense qui est le but ; et la dépense, c’est l’action.\par
Mais allons plus loin et remontons jusqu’au principe fondamental sur lequel reposent toutes ces théories. Toutes supposent également que la valeur est dans les choses et exprime leur nature. Or ce postulat est contraire aux faits. Il y a nombre de cas où il n’existe, pour ainsi dire, aucun rapport entre les propriétés de l’objet et de la valeur qui lui est attribuée.\par
Une idole est une chose très sainte et la sainteté est la valeur la plus élevée que les hommes aient jamais reconnue. Or une idole n’est très souvent qu’une masse de pierres ou une pièce de bois qui, par elle-même, est dénuée de toute espèce de valeur. Il n’est pas d’être, si humble soit-il, pas d’objet vulgaire qui, à un moment donné de l’histoire, n’ait inspiré des sentiments de respect religieux. On a adoré les animaux les plus inutiles ou les plus inoffensifs, les plus pauvres en vertus de toute sorte. La conception courante d’après laquelle les choses auxquelles s’est adressé le culte ont toujours été celles qui frappaient le plus l’imagination des hommes est contredite par l’histoire. La valeur incomparable qui leur était attribuée ne tenait donc pas à leurs caractères intrinsèques. Il n’est pas de foi un peu vive, si laïque soit-elle, qui n’ait ses fétiches où la même disproportion éclate. Un drapeau n’est qu’un morceau d’étoffe ; le soldat, cependant, se fait tuer pour sauver son drapeau. La vie morale n’est pas moins riche en contrastes de ce genre. Entre l’homme et l’animal il n’y a, au point de vue anatomique, physiologique et psychologique, que des différences de degrés ; et pourtant l’homme a une éminente dignité morale, l’animal n’en a aucune. Sous le rapport des valeurs, il y a donc entre eux un abîme. Les hommes sont inégaux en force physique comme en talents ; et cependant nous tendons à leur reconnaître à tous une égale valeur morale. Sans doute, l’égalitarisme moral est une limite idéale qui ne sera jamais atteinte, mais nous nous en rapprochons toujours davantage. Un timbre-poste n’est qu’un mince carré de papier dépourvu, le plus souvent, de tout caractère artistique ; il peut néanmoins valoir une fortune. Ce n’est évidemment pas la nature interne de la perle ou du diamant, des fourrures ou des dentelles qui fait que la valeur de ces différents objets de toilette varie avec les caprices de la mode.
\subsection[{II}]{II}
\noindent Mais si la valeur n’est pas dans les choses, si elle ne tient pas essentiellement à quelque caractère de la réalité empirique, ne s’ensuit-il pas qu’elle a sa source en dehors du donné et de l’expérience ? Telle est, en effet, la thèse qu’ont soutenue, plus ou moins explicitement, toute une lignée de penseurs dont la doctrine, par-delà Ritschl, remonte jusqu’au moralisme kantien. On accorde à l’homme une faculté {\itshape sui generis} de dépasser l’expérience, de se représenter autre chose que ce qui est, en un mot de poser des idéaux. Cette faculté représentative on la conçoit, ici sous une forme plus intellectualiste, là plus sentimentale, mais toujours comme nettement distincte de celle que la science met en œuvre. Il y aurait donc une manière de penser le réel, et une autre, très différente, pour l’idéal ; et c’est par rapport aux idéaux ainsi posés que serait estimée la valeur des choses. On dit qu’elles ont de la valeur quand elles expriment, reflètent, à un titre quelconque, un aspect de l’idéal, et qu’elles ont plus ou moins de valeur selon l’idéal qu’elles incarnent et selon ce qu’elles en recèlent.\par
\par
Ainsi, tandis que, dans les théories précédentes, les jugements de valeur nous étaient présentés comme une autre forme des jugements de réalité, ici, l’hétérogénéité des uns et des autres est radicale : les objets sur lesquels ils portent sont différents comme les facultés qu’ils supposent. Les objections que nous faisions à la première explication ne sauraient donc s’appliquer à celle-ci. On comprend sans peine que la valeur soit, dans une certaine mesure, indépendante de la nature des choses, si elle dépend de causes qui sont extérieures à ces dernières. En même temps, la place privilégiée qui a toujours été faite aux valeurs de luxe devient facile à justifier. C’est que l’idéal n’est pas au service du réel ; il est là pour lui-même ; ce ne sont donc pas les intérêts de la réalité qui peuvent lui servir de mesure.\par
Seulement, la valeur qui est ainsi attribuée à l’idéal, si elle explique le reste, ne s’explique pas elle-même. On la postule, mais on n’en rend pas compte et on ne peut pas en rendre compte. Comment, en effet, serait-ce possible ? Si l’idéal ne dépend pas du réel, il ne saurait y avoir dans le réel les causes et les conditions qui le rendent intelligible. Mais, en dehors du réel, où trouver la matière nécessaire à une explication quelconque ? Il y a, au fond, quelque chose de profondément empiriste dans un idéalisme ainsi entendu. Sans doute, c’est un fait que les hommes aiment une beauté, une bonté, une vérité qui ne sont jamais réalisées d’une manière adéquate dans les faits. Mais cela même n’est qu’un fait que l’on érige, sans raison, en une sorte d’absolu au-delà duquel on s’interdit de remonter. Encore faudrait-il faire voir d’où vient que nous avons, à la fois, le besoin et le moyen de dépasser le réel, de surajouter au monde sensible un monde différent dont les meilleurs d’entre nous font leur véritable patrie.\par
À cette question, l’hypothèse théologique apporte un semblant de réponse. On suppose que le monde des idéaux est réel, qu’il existe objectivement, mais d’une existence supra-expérimentale, et que la réalité empirique dont nous faisons partie en vient et en dépend. Nous serions donc attachés à l’idéal comme à la source même de notre être. Mais, outre les difficultés connues que soulève cette conception, quand on hypostasie ainsi l’idéal, du même coup on l’immobilise et on se retire tout moyen d’en expliquer l’infinie variabilité. Nous savons aujourd’hui que non seulement l’idéal varie selon les groupes humains, mais qu’il doit varier ; celui des Romains n’était pas le nôtre et ne devait pas être le nôtre, et l’échelle des valeurs change parallèlement. Ces variations ne sont pas le produit de l’aveuglement humain ; elles sont fondées dans la nature des choses. Comment les expliquer, si l’idéal exprime une réalité une et inconcussible ? Il faudrait donc admettre que Dieu, lui aussi, varie dans l’espace comme dans le temps, et à quoi pourrait tenir cette surprenante diversité ? Le devenir divin ne serait intelligible que si Dieu lui-même avait pour tâche de réaliser un idéal qui le dépasse, et le problème, alors, ne serait que déplacé.\par
De quel droit, d’ailleurs, met-on l’idéal en dehors de la nature et de la science ? C’est dans la nature qu’il se manifeste ; il faut donc bien qu’il dépende de causes naturelles. Pour qu’il soit autre chose qu’un simple possible, conçu par les esprits, il faut qu’il soit voulu et, par suite, qu’il ait une force capable de mouvoir nos volontés. Ce sont elles qui, seules, peuvent en faire une réalité vivante. Mais puisque cette force vient finalement se traduire en mouvements musculaires, elle ne saurait différer essentiellement des autres forces de l’univers. Pourquoi donc serait-il impossible de l’analyser, de la résoudre en ses éléments, de chercher les causes qui ont déterminé la synthèse dont elle est la résultante ? Il est même des cas où il est impossible de la mesurer. Chaque groupe humain, à chaque moment de son histoire, a, pour la dignité humaine, un sentiment de respect d’une intensité donnée. C’est ce sentiment, variable suivant les peuples et les époques, qui est à la racine de l’idéal moral des sociétés contemporaines. Or, suivant qu’il est plus ou moins intense, le nombre des attentats Contre la personne est plus ou moins élevé. De même, le nombre des adultères, des divorces, des séparations de corps exprime la force relative avec laquelle l’idéal conjugal s’impose aux consciences particulières. Sans doute, ces mesures sont grossières ; mais est-il des forces physiques qui puissent être mesurées autrement que d’une manière grossièrement approximative ? Sous ce rapport encore, il ne peut y avoir entre les unes et les autres que des différences de degrés.\par
Mais il y a surtout un ordre de valeurs qui ne sauraient être détachées de l’expérience sans perdre toute signification : ce sont les valeurs économiques. Tout le monde sent bien qu’elles n’expriment rien de l’au-delà et n’impliquent aucune faculté supra-expérimentale. Il est vrai que, pour cette raison, Kant se refuse à y voir des valeurs véritables : il tend à réserver cette qualification aux seules choses morales\footnote{ Il dit que les choses économiques ont un prix (einen Preis, einen Marktpreis), non une valeur interne (einen inneren Werth). V. {\itshape édit. Hartenstein}, tome VII, pp. 270-271 et 614.}. Mais cette exclusion est injustifiée. Certes, il y a des types différents de valeurs, mais ce sont des espèces d’un même genre. Toutes correspondent à une estimation des choses, quoique l’estimation soit faite, suivant les cas, de points de vue différents. Le progrès qu’a fait, dans les temps récents, la théorie de la valeur est précisément d’avoir bien établi la généralité et l’unité de la notion. Mais alors, si toutes les sortes de valeurs sont patentes, et si certaines d’entre elles tiennent aussi étroitement à notre vie empirique, les autres n’en sauraient être indépendantes.
\subsection[{III}]{III}
\noindent En résumé, s’il est vrai que la valeur des choses ne peut être et n’a jamais été estimée que par rapport à certaines notions idéales, celles-ci ont besoin d’être expliquées. Pour comprendre comment des jugements de valeur sont possibles, il ne suffit pas de postuler un certain nombre d’idéaux ; il faut en rendre compte, il faut faire voir d’où ils viennent, comment ils se relient à l’expérience tout en la dépassant et en quoi consiste leur objectivité.\par
Puisqu’ils varient avec les groupes humains ainsi que les systèmes de valeurs correspondants, ne s’ensuit-il pas que les uns et les autres doivent être d’origine collective ? Il est vrai que nous avons précédemment exposé une théorie sociologique des valeurs dont nous avons montré l’insuffisance ; mais c’est qu’elle reposait sur une conception de la vie sociale qui en méconnaissait la nature véritable. La société y était présentée comme un système d’organes et de fonctions qui tend à se maintenir contre les causes de destruction qui l’assaillent du dehors, comme un corps vivant dont toute la vie consiste à répondre d’une manière appropriée aux excitations venues du milieu externe. Or, en fait, elle est, de plus, le foyer d’une vie morale interne dont on n’a pas toujours reconnu la puissance et l’originalité.\par
Quand les consciences individuelles, au lieu de rester séparées les unes des autres, entrent étroitement en rapports, agissent activement les unes sur les autres, il se dégage de leur synthèse une vie psychique d’un genre nouveau. Elle se distingue d’abord, de celle que mène l’individu solitaire, par sa particulière intensité. Les sentiments qui naissent et se développent au sein des groupes ont une énergie à laquelle n’atteignent pas les sentiments purement individuels. L’homme qui les éprouve a l’impression qu’il est dominé par des forces qu’il ne reconnaît pas comme siennes, qui le mènent, dont il n’est pas le maître, et tout le milieu dans lequel il est plongé lui semble sillonné par des forces du même genre. Il se sent comme transporté dans un monde différent de celui où s’écoule son existence privée. La vie n’y est pas seulement intense ; elle est qualitativement différente. Entraîné par la collectivité, l’individu se désintéresse de lui-même, s’oublie, se donne tout entier aux fins communes. Le pôle de sa conduite est déplacé et reporté hors de lui. En même temps, les forces qui sont ainsi soulevées, précisément parce qu’elles sont théoriques, ne se laissent pas facilement canaliser, compasser, ajuster à des fins étroitement déterminées ; elles éprouvent le besoin de se répandre pour se répandre, par jeu, sans but, sous forme, ici, de violences stupidement destructrices, là, de folies héroïques. C’est une activité de luxe, en un sens, parce que c’est une activité très riche. Pour toutes ces raisons, elle s’oppose à la vie que nous traînons quotidiennement, comme le supérieur s’oppose à l’inférieur, l’idéal à la réalité.\par
C’est, en effet, dans les moments d’effervescence de ce genre que se sont, de tout temps, constitués les grands idéaux sur lesquels reposent les civilisations. Les périodes créatrices ou novatrices sont précisément celles où, sous l’influence de circonstances diverses, les hommes sont amenés à se rapprocher plus intimement, où les réunions, les assemblées sont plus fréquentes, les relations plus suivies, les échanges d’idées plus actifs : c’est la grande crise chrétienne, c’est le mouvement d’enthousiasme collectif, qui, aux \textsc{xii}\textsuperscript{e} et \textsc{xiii}\textsuperscript{e} siècles, entraîne vers Paris la population studieuse de l’Europe et donne naissance à la scolastique, c’est la Réforme et la Renaissance, c’est l’époque révolutionnaire, ce sont les grandes agitations socialistes du \textsc{xix}\textsuperscript{e} siècle. A ces moments, il est vrai, cette vie plus haute est vécue avec une telle intensité et d’une manière tellement exclusive qu’elle tient presque toute la place dans les consciences, qu’elle en chasse plus ou moins complètement les préoccupations égoïstes et vulgaires. L’idéal tend alors à ne faire qu’un avec le réel ; c’est pourquoi les hommes ont l’impression que les temps sont tout proches où il deviendra la réalité elle-même et où le royaume de Dieu se réalisera sur cette terre. Mais l’illusion n’est jamais durable parce que cette exaltation elle-même ne peut pas durer : elle est trop épuisante. Une fois le moment critique passé, la trame sociale se relâche, le commerce intellectuel et sentimental se ralentit, les individus retombent à leur niveau ordinaire. Alors, tout ce qui a été dit, fait, pensé, senti pendant la période de tourmente féconde ne survit plus que sous forme de souvenir, de souvenir prestigieux, sans doute, tout comme la réalité qu’il rappelle, mais avec laquelle il a cessé de se confondre. Ce n’est plus qu’une idée, un ensemble d’idées. Cette fois, l’opposition est tranchée. Il y a, d’un côté, ce qui est donné dans les sensations et les perceptions et, de l’autre, ce qui est pensé sous formes d’idéaux. Certes, ces idéaux s’étioleraient vite, s’ils n’étaient périodiquement revivifiés. C’est à quoi servent les fêtes, les cérémonies publiques, ou religieuses, ou laïques, les prédications de toute sorte, celles de l’Église ou celles de l’école, les représentations dramatiques, les manifestations artistiques, en un mot tout ce qui peut rapprocher les hommes et les faire communier dans une même vie intellectuelle et morale. Ce sont comme des renaissances partielles et affaiblies de l’effervescence des époques créatrices. Mais tous ces moyens n’ont eux-mêmes qu’une action temporaire. Pendant un temps, l’idéal reprend la fraîcheur et la vie de l’actualité, il se rapproche à nouveau du réel, mais il ne tarde pas à s’en différencier de nouveau.\par
Si donc l’homme conçoit des idéaux, si même il ne peut se passer d’en concevoir et de s’y attacher, c’est qu’il est un être social. C’est la société qui le pousse ou l’oblige à se hausser ainsi au-dessus de lui-même, et c’est elle aussi qui lui en fournit les moyens. Par cela seul qu’elle prend conscience de soi, elle enlève l’individu à lui-même et elle l’entraîne dans un cercle de vie supérieure. Elle ne peut pas se constituer sans créer de l’idéal. Ces idéaux, ce sont tout simplement les idées dans lesquelles vient se peindre et se résumer la vie sociale, telle qu’elle est aux points culminants de son développement. On diminue la société quand on ne voit en elle qu’un corps organisé en vue de certaines fonctions vitales. Dans ce corps vit une âme : c’est l’ensemble des idéaux collectifs. Mais ces idéaux ne sont pas des abstraits, de froides représentations intellectuelles, dénuées de toute efficace. Ils sont essentiellement moteurs ; car derrière eux, il y a des forces réelles et agissantes : ce sont les forces collectives, forces naturelles, par conséquent, quoique toutes morales, et comparables à celles qui jouent dans le reste de l’univers. L’idéal lui-même est une force de ce genre ; la science en peut donc être faite. Voilà comment il se fait que l’idéal peut s’incorporer au réel : c’est qu’il en vient tout en le dépassant. Les éléments dont il est fait sont empruntés à la réalité, mais ils sont combinés d’une manière nouvelle. C’est la nouveauté de la combinaison qui fait la nouveauté du résultat. Abandonné à lui-même, jamais l’individu n’aurait pu tirer de soi les matériaux nécessaires pour une telle construction. Livré à ses seules forces, comment aurait-il pu avoir et l’idée et le pouvoir de se dépasser soi-même ? Son expérience personnelle peut bien lui permettre de distinguer des fins à venir et désirables et d’autres qui sont déjà réalisées. Mais l’idéal, ce n’est pas seulement quelque chose qui manque et qu’on souhaite. Ce n’est pas un simple futur vers lequel on aspire. Il est de sa façon ; il a sa réalité. On le conçoit planant, impersonnel, par-dessus les volontés particulières qu’il meut. S’il était le produit de la raison individuelle, d’où lui pourrait venir cette impersonnalité ? Invoquera-t-on l’impersonnalité de la raison humaine ? Mais c’est reculer le problème ; ce n’est pas le résoudre. Car cette impersonnalité n’est elle-même qu’un fait, à peine différent du premier, et dont il faut rendre compte. Si les raisons communient à ce point, n’est-ce pas qu’elles viennent d’une même source, qu’elles participent d’une raison commune ?\par
Ainsi, pour expliquer les jugements de valeur, il n’est nécessaire ni de les ramener à des jugements de réalité en faisant évanouir la notion de valeur, ni de les rapporter à je ne sais quelle faculté par laquelle l’homme entrerait en relation avec un monde transcendant. La valeur vient bien du rapport des choses avec les différents aspects de l’idéal ; mais l’idéal n’est pas une échappée vers un au-delà mystérieux ; il est dans la nature et de la nature. La pensée distincte a prise sur lui comme sur le reste de l’univers physique ou moral. Non certes qu’elle puisse jamais l’épuiser, pas plus qu’elle n’épuise aucune réalité ; mais elle peut s’y appliquer avec l’espérance de s’en saisir progressivement, sans qu’on puisse assigner par avance aucune limite à ses progrès indéfinis. De ce point de vue, on est mieux en état de comprendre comment la valeur des choses peut être indépendante de leur nature. Les idéaux collectifs ne peuvent se constituer et prendre conscience d’eux-mêmes qu’à condition de se fixer sur des choses qui puissent être vues par tous, comprises de tous, représentées à tous les esprits : dessins figurés, emblèmes de toute sorte, formules écrites ou parlées, êtres animés, ou inanimés. Et sans doute il arrive que, par certaines de leurs propriétés, ces objets aient une sorte d’affinité pour l’idéal et l’appellent à eux naturellement. C’est alors que les caractères intrinsèques de la chose peuvent paraître — à tort d’ailleurs — la cause génératrice de la valeur. Mais l’idéal peut aussi s’incorporer à une chose quelconque : il se pose où il veut. Toute sorte de circonstances contingentes peuvent déterminer la manière dont il se fixe. Alors cette chose, si vulgaire soit-elle, est mise hors de pair. Voilà comment un chiffon de toile peut s’auréoler de sainteté, comment un mince morceau de papier peut devenir une chose très précieuse. Deux êtres peuvent être très différents et très inégaux sous bien des rapports : s’ils incarnent un même idéal, ils apparaissent comme équivalents ; c’est que l’idéal qu’ils symbolisent apparaît alors comme ce qu’il y a de plus essentiel en eux et rejette au second plan tous les aspects d’eux-mêmes par où ils divergent l’un de l’autre. C’est ainsi que la pensée collective métamorphose tout ce qu’elle touche. Elle mêle les règnes, elle confond les contraires, elle renverse ce qu’on pourrait regarder comme la hiérarchie naturelle des êtres, elle nivelle les différences, elle différencie les semblables, en un mot elle substitue au monde que nous révèlent les sens un monde tout différent qui n’est autre chose que l’ombre projetée par les idéaux qu’elle construit.
\subsection[{IV}]{IV}
\noindent Comment faut-il donc concevoir le rapport des jugements de valeur aux jugements de réalité ?\par
De ce qui précède il résulte qu’il n’existe pas entre eux de différences de nature. Un jugement de valeur exprime la relation d’une chose avec un idéal. Or l’idéal est donné comme la chose, quoique d’une autre manière ; il est, lui aussi, une réalité à sa façon. La relation exprimée unit donc deux termes donnés, tout comme dans un jugement d’existence. Dira-t-on que les jugements de valeur mettent en jeu les idéaux ? Mais il n’en est pas autrement des jugements de réalité. Car les concepts sont également des constructions de l’esprit, partant, des idéaux ; et il ne serait pas difficile de montrer que ce sont même des idéaux collectifs, puisqu’ils ne peuvent se constituer que dans et par le langage, qui est, au plus haut point, une chose collective. Les éléments du jugement sont donc les mêmes de part et d’autre. Ce n’est pas à dire toutefois que le premier de ces jugements se ramène au second ou réciproquement. S’ils se ressemblent, c’est qu’ils sont l’œuvre d’une seule et même faculté. Il n’y a pas une manière de penser et de juger pour poser des existences et une autre pour estimer des valeurs. Tout jugement a nécessairement une base dans le donné : même ceux qui se rapportent à l’avenir empruntent leurs matériaux soit au présent soit au passé. D’autre part, tout jugement met en œuvre des idéaux. Il n’y a donc et il doit n’y avoir qu’une seule faculté de juger.\par
Cependant, la différence que nous avons signalée chemin faisant ne laisse pas de subsister. Si tout jugement met en œuvre des idéaux, ceux-ci sont d’espèces différentes. Il en est dont le rôle est uniquement d’exprimer les réalités auxquelles ils s’appliquent, de les exprimer telles qu’elles sont. Ce sont les concepts proprement dits. Il en est d’autres, au contraire, dont la fonction est de transfigurer les réalités auxquelles ils sont rapportés. Ce sont les idéaux de valeur. Dans les premiers cas, c’est l’idéal qui sert de symbole à la chose de manière à la rendre assimilable à la pensée. Dans le second, c’est la chose qui sert de symbole à l’idéal et qui le rend représentable aux différents esprits. Naturellement, les jugements diffèrent selon les idéaux qu’ils emploient. Les premiers se bornent à analyser la réalité et à la traduire aussi fidèlement que possible. Les seconds, au contraire, disent l’aspect nouveau dont elle s’enrichit sous l’action de l’idéal. Et sans doute, cet aspect est réel, lui aussi, mais à un autre titre et d’une autre manière que les propriétés inhérentes à l’objet. La preuve en est qu’une même chose peut ou perdre la valeur qu’elle a, ou en acquérir une différente sans changer de nature : il suffit que l’idéal change. Le jugement de valeur ajoute donc au donné, en un sens, quoique ce qu’il ajoute soit emprunté à un donné d’une autre sorte. Et ainsi la faculté de juger fonctionne différemment selon les circonstances, mais sans que ces différences altèrent l’unité fondamentale de la fonction.\par
On a parfois reproché à la sociologie positive une sorte de fétichisme empiriste pour le fait et une indifférence systématique pour l’idéal. On voit combien le reproche est injustifié. Les principaux phénomènes sociaux, religion, morale, droit, économie, esthétique, ne sont autre chose que des systèmes de valeurs, partant, des idéaux. La sociologie se place donc d’emblée dans l’idéal ; elle n’y parvient pas lentement, au terme de ses recherches ; elle en part. L’idéal est son domaine propre. Seulement (et c’est par là qu’on pourrait la qualifier de positive si, accolé à un nom de science, cet adjectif ne faisait pléonasme), elle ne traite que l’idéal que pour en faire la science. Non pas qu’elle entreprenne de le construire ; tout au contraire, elle le prend comme une donnée, comme un objet d’étude, et elle essaie de l’analyser et de l’expliquer. Dans la faculté d’idéal, elle voit une faculté naturelle dont elle cherche les causes et les conditions, en vue, si c’est possible, d’aider les hommes à en régler le fonctionnement. En définitive, la tâche du sociologue doit être de faire rentrer l’idéal, sous toutes ses formes, dans la nature, mais en lui laissant tous ses attributs distinctifs. Et si l’entreprise ne lui paraît pas impossible, c’est que la société remplit toutes les conditions nécessaires pour rendre compte de ces caractères opposés. Elle aussi vient de la nature, tout en la dominant. C’est que, non seulement toutes les forces de l’univers viennent aboutir en elle, mais de plus, elles y sont synthétisées de manière à donner naissance à un produit qui dépasse en richesse, en complexité et en puissance d’action tout ce qui a servi à le former. En un mot, elle est la nature, mais parvenue au plus haut point de son développement et concentrant toutes ses énergies pour se dépasser en quelque sorte elle-même.
 


% at least one empty page at end (for booklet couv)
\ifbooklet
  \pagestyle{empty}
  \clearpage
  % 2 empty pages maybe needed for 4e cover
  \ifnum\modulo{\value{page}}{4}=0 \hbox{}\newpage\hbox{}\newpage\fi
  \ifnum\modulo{\value{page}}{4}=1 \hbox{}\newpage\hbox{}\newpage\fi


  \hbox{}\newpage
  \ifodd\value{page}\hbox{}\newpage\fi
  {\centering\color{rubric}\bfseries\noindent\large
    Hurlus ? Qu’est-ce.\par
    \bigskip
  }
  \noindent Des bouquinistes électroniques, pour du texte libre à participation libre,
  téléchargeable gratuitement sur \href{https://hurlus.fr}{\dotuline{hurlus.fr}}.\par
  \bigskip
  \noindent Cette brochure a été produite par des éditeurs bénévoles.
  Elle n’est pas faîte pour être possédée, mais pour être lue, et puis donnée.
  Que circule le texte !
  En page de garde, on peut ajouter une date, un lieu, un nom ; pour suivre le voyage des idées.
  \par

  Ce texte a été choisi parce qu’une personne l’a aimé,
  ou haï, elle a en tous cas pensé qu’il partipait à la formation de notre présent ;
  sans le souci de plaire, vendre, ou militer pour une cause.
  \par

  L’édition électronique est soigneuse, tant sur la technique
  que sur l’établissement du texte ; mais sans aucune prétention scolaire, au contraire.
  Le but est de s’adresser à tous, sans distinction de science ou de diplôme.
  Au plus direct ! (possible)
  \par

  Cet exemplaire en papier a été tiré sur une imprimante personnelle
   ou une photocopieuse. Tout le monde peut le faire.
  Il suffit de
  télécharger un fichier sur \href{https://hurlus.fr}{\dotuline{hurlus.fr}},
  d’imprimer, et agrafer ; puis de lire et donner.\par

  \bigskip

  \noindent PS : Les hurlus furent aussi des rebelles protestants qui cassaient les statues dans les églises catholiques. En 1566 démarra la révolte des gueux dans le pays de Lille. L’insurrection enflamma la région jusqu’à Anvers où les gueux de mer bloquèrent les bateaux espagnols.
  Ce fut une rare guerre de libération dont naquit un pays toujours libre : les Pays-Bas.
  En plat pays francophone, par contre, restèrent des bandes de huguenots, les hurlus, progressivement réprimés par la très catholique Espagne.
  Cette mémoire d’une défaite est éteinte, rallumons-la. Sortons les livres du culte universitaire, cherchons les idoles de l’époque, pour les briser.
\fi

\ifdev % autotext in dev mode
\fontname\font — \textsc{Les règles du jeu}\par
(\hyperref[utopie]{\underline{Lien}})\par
\noindent \initialiv{A}{lors là}\blindtext\par
\noindent \initialiv{À}{ la bonheur des dames}\blindtext\par
\noindent \initialiv{É}{tonnez-le}\blindtext\par
\noindent \initialiv{Q}{ualitativement}\blindtext\par
\noindent \initialiv{V}{aloriser}\blindtext\par
\Blindtext
\phantomsection
\label{utopie}
\Blinddocument
\fi
\end{document}
