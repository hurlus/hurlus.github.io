%%%%%%%%%%%%%%%%%%%%%%%%%%%%%%%%%
% LaTeX model https://hurlus.fr %
%%%%%%%%%%%%%%%%%%%%%%%%%%%%%%%%%

% Needed before document class
\RequirePackage{pdftexcmds} % needed for tests expressions
\RequirePackage{fix-cm} % correct units

% Define mode
\def\mode{a4}

\newif\ifaiv % a4
\newif\ifav % a5
\newif\ifbooklet % booklet
\newif\ifcover % cover for booklet

\ifnum \strcmp{\mode}{cover}=0
  \covertrue
\else\ifnum \strcmp{\mode}{booklet}=0
  \booklettrue
\else\ifnum \strcmp{\mode}{a5}=0
  \avtrue
\else
  \aivtrue
\fi\fi\fi

\ifbooklet % do not enclose with {}
  \documentclass[french,twoside]{book} % ,notitlepage
  \usepackage[%
    papersize={105mm, 297mm},
    inner=12mm,
    outer=12mm,
    top=20mm,
    bottom=15mm,
    marginparsep=0pt,
  ]{geometry}
  \usepackage[fontsize=9.5pt]{scrextend} % for Roboto
\else\ifav
  \documentclass[french,twoside]{book} % ,notitlepage
  \usepackage[%
    a5paper,
    inner=25mm,
    outer=15mm,
    top=15mm,
    bottom=15mm,
    marginparsep=0pt,
  ]{geometry}
  \usepackage[fontsize=12pt]{scrextend}
\else% A4 2 cols
  \documentclass[twocolumn]{report}
  \usepackage[%
    a4paper,
    inner=15mm,
    outer=10mm,
    top=25mm,
    bottom=18mm,
    marginparsep=0pt,
  ]{geometry}
  \setlength{\columnsep}{20mm}
  \usepackage[fontsize=9.5pt]{scrextend}
\fi\fi

%%%%%%%%%%%%%%
% Alignments %
%%%%%%%%%%%%%%
% before teinte macros

\setlength{\arrayrulewidth}{0.2pt}
\setlength{\columnseprule}{\arrayrulewidth} % twocol
\setlength{\parskip}{0pt} % classical para with no margin
\setlength{\parindent}{1.5em}

%%%%%%%%%%
% Colors %
%%%%%%%%%%
% before Teinte macros

\usepackage[dvipsnames]{xcolor}
\definecolor{rubric}{HTML}{0c71c3} % the tonic
\def\columnseprulecolor{\color{rubric}}
\colorlet{borderline}{rubric!30!} % definecolor need exact code
\definecolor{shadecolor}{gray}{0.95}
\definecolor{bghi}{gray}{0.5}

%%%%%%%%%%%%%%%%%
% Teinte macros %
%%%%%%%%%%%%%%%%%
%%%%%%%%%%%%%%%%%%%%%%%%%%%%%%%%%%%%%%%%%%%%%%%%%%%
% <TEI> generic (LaTeX names generated by Teinte) %
%%%%%%%%%%%%%%%%%%%%%%%%%%%%%%%%%%%%%%%%%%%%%%%%%%%
% This template is inserted in a specific design
% It is XeLaTeX and otf fonts

\makeatletter % <@@@


\usepackage{blindtext} % generate text for testing
\usepackage{contour} % rounding words
\usepackage[nodayofweek]{datetime}
\usepackage{DejaVuSans} % font for symbols
\usepackage{enumitem} % <list>
\usepackage{etoolbox} % patch commands
\usepackage{fancyvrb}
\usepackage{fancyhdr}
\usepackage{fontspec} % XeLaTeX mandatory for fonts
\usepackage{footnote} % used to capture notes in minipage (ex: quote)
\usepackage{framed} % bordering correct with footnote hack
\usepackage{graphicx}
\usepackage{lettrine} % drop caps
\usepackage{lipsum} % generate text for testing
\usepackage[framemethod=tikz,]{mdframed} % maybe used for frame with footnotes inside
\usepackage{pdftexcmds} % needed for tests expressions
\usepackage{polyglossia} % non-break space french punct, bug Warning: "Failed to patch part"
\usepackage[%
  indentfirst=false,
  vskip=1em,
  noorphanfirst=true,
  noorphanafter=true,
  leftmargin=\parindent,
  rightmargin=0pt,
]{quoting}
\usepackage{ragged2e}
\usepackage{setspace}
\usepackage{tabularx} % <table>
\usepackage[explicit]{titlesec} % wear titles, !NO implicit
\usepackage{tikz} % ornaments
\usepackage{tocloft} % styling tocs
\usepackage[fit]{truncate} % used im runing titles
\usepackage{unicode-math}
\usepackage[normalem]{ulem} % breakable \uline, normalem is absolutely necessary to keep \emph
\usepackage{verse} % <l>
\usepackage{xcolor} % named colors
\usepackage{xparse} % @ifundefined
\XeTeXdefaultencoding "iso-8859-1" % bad encoding of xstring
\usepackage{xstring} % string tests
\XeTeXdefaultencoding "utf-8"
\PassOptionsToPackage{hyphens}{url} % before hyperref, which load url package
\usepackage{hyperref} % supposed to be the last one, :o) except for the ones to follow
\urlstyle{same} % after hyperref

% TOTEST
% \usepackage{hypcap} % links in caption ?
% \usepackage{marginnote}
% TESTED
% \usepackage{background} % doesn’t work with xetek
% \usepackage{bookmark} % prefers the hyperref hack \phantomsection
% \usepackage[color, leftbars]{changebar} % 2 cols doc, impossible to keep bar left
% \usepackage[utf8x]{inputenc} % inputenc package ignored with utf8 based engines
% \usepackage[sfdefault,medium]{inter} % no small caps
% \usepackage{firamath} % choose firasans instead, firamath unavailable in Ubuntu 21-04
% \usepackage{flushend} % bad for last notes, supposed flush end of columns
% \usepackage[stable]{footmisc} % BAD for complex notes https://texfaq.org/FAQ-ftnsect
% \usepackage{helvet} % not for XeLaTeX
% \usepackage{multicol} % not compatible with too much packages (longtable, framed, memoir…)
% \usepackage[default,oldstyle,scale=0.95]{opensans} % no small caps
% \usepackage{sectsty} % \chapterfont OBSOLETE
% \usepackage{soul} % \ul for underline, OBSOLETE with XeTeX
% \usepackage[breakable]{tcolorbox} % text styling gone, footnote hack not kept with breakable



% Metadata inserted by a program, from the TEI source, for title page and runing heads
\title{\textbf{ État français, loi du 4 octobre 1941, dite Charte du Travail }}
\date{1941}
\author{Lois}
\def\elbibl{Lois. 1941. \emph{État français, loi du 4 octobre 1941, dite Charte du Travail}}
\def\elsource{ \href{https://www.legifrance.gouv.fr/jorf/id/JORFTEXT000000572823}{\dotuline{Légifrance}}\footnote{\href{https://www.legifrance.gouv.fr/jorf/id/JORFTEXT000000572823}{\url{https://www.legifrance.gouv.fr/jorf/id/JORFTEXT000000572823}}} }

% Default metas
\newcommand{\colorprovide}[2]{\@ifundefinedcolor{#1}{\colorlet{#1}{#2}}{}}
\colorprovide{rubric}{red}
\colorprovide{silver}{Gray}
\@ifundefined{syms}{\newfontfamily\syms{DejaVu Sans}}{}
\newif\ifdev
\@ifundefined{elbibl}{% No meta defined, maybe dev mode
  \newcommand{\elbibl}{Titre court ?}
  \newcommand{\elbook}{Titre du livre source ?}
  \newcommand{\elabstract}{Résumé\par}
  \newcommand{\elurl}{http://oeuvres.github.io/elbook/2}
  \author{Éric Lœchien}
  \title{Un titre de test assez long pour vérifier le comportement d’une maquette}
  \date{1566}
  \devtrue
}{}
\let\eltitle\@title
\let\elauthor\@author
\let\eldate\@date


\defaultfontfeatures{
  % Mapping=tex-text, % no effect seen
  Scale=MatchLowercase,
  Ligatures={TeX,Common},
}

\@ifundefined{\columnseprulecolor}{%
    \patchcmd\@outputdblcol{% find
      \normalcolor\vrule
    }{% and replace by
      \columnseprulecolor\vrule
    }{% success
    }{% failure
      \@latex@warning{Patching \string\@outputdblcol\space failed}%
    }
}{}

\hypersetup{
  % pdftex, % no effect
  pdftitle={\elbibl},
  % pdfauthor={Your name here},
  % pdfsubject={Your subject here},
  % pdfkeywords={keyword1, keyword2},
  bookmarksnumbered=true,
  bookmarksopen=true,
  bookmarksopenlevel=1,
  pdfstartview=Fit,
  breaklinks=true, % avoid long links
  pdfpagemode=UseOutlines,    % pdf toc
  hyperfootnotes=true,
  colorlinks=false,
  pdfborder=0 0 0,
  % pdfpagelayout=TwoPageRight,
  % linktocpage=true, % NO, toc, link only on page no
}


% generic typo commands
\newcommand{\astermono}{\medskip\centerline{\color{rubric}\large\selectfont{\syms ✻}}\medskip\par}%
\newcommand{\astertri}{\medskip\par\centerline{\color{rubric}\large\selectfont{\syms ✻\,✻\,✻}}\medskip\par}%
\newcommand{\asterism}{\bigskip\par\noindent\parbox{\linewidth}{\centering\color{rubric}\large{\syms ✻}\\{\syms ✻}\hskip 0.75em{\syms ✻}}\bigskip\par}%

% lists
\newlength{\listmod}
\setlength{\listmod}{\parindent}
\setlist{
  itemindent=!,
  listparindent=\listmod,
  labelsep=0.2\listmod,
  parsep=0pt,
  % topsep=0.2em, % default topsep is best
}
\setlist[itemize]{
  label=—,
  leftmargin=0pt,
  labelindent=1.2em,
  labelwidth=0pt,
}
\setlist[enumerate]{
  label={\bf\color{rubric}\arabic*.},
  labelindent=0.8\listmod,
  leftmargin=\listmod,
  labelwidth=0pt,
}
\newlist{listalpha}{enumerate}{1}
\setlist[listalpha]{
  label={\bf\color{rubric}\alph*.},
  leftmargin=0pt,
  labelindent=0.8\listmod,
  labelwidth=0pt,
}
\newcommand{\listhead}[1]{\hspace{-1\listmod}\emph{#1}}

\renewcommand{\hrulefill}{%
  \leavevmode\leaders\hrule height 0.2pt\hfill\kern\z@}

% General typo
\DeclareTextFontCommand{\textlarge}{\large}
\DeclareTextFontCommand{\textsmall}{\small}


% commands, inlines
\newcommand{\anchor}[1]{\Hy@raisedlink{\hypertarget{#1}{}}} % link to top of an anchor (not baseline)
\newcommand\abbr[1]{#1}
\newcommand{\autour}[1]{\tikz[baseline=(X.base)]\node [draw=rubric,thin,rectangle,inner sep=1.5pt, rounded corners=3pt] (X) {\color{rubric}#1};}
\newcommand\corr[1]{#1}
\newcommand{\ed}[1]{ {\color{silver}\sffamily\footnotesize (#1)} } % <milestone ed="1688"/>
\newcommand\expan[1]{#1}
\newcommand\foreign[1]{\emph{#1}}
\newcommand\gap[1]{#1}
\renewcommand{\LettrineFontHook}{\color{rubric}}
\newcommand{\initial}[2]{\lettrine[lines=2, loversize=0.3, lhang=0.3]{#1}{#2}}
\newcommand{\initialiv}[2]{%
  \let\oldLFH\LettrineFontHook
  % \renewcommand{\LettrineFontHook}{\color{rubric}\ttfamily}
  \IfSubStr{QJ’}{#1}{
    \lettrine[lines=4, lhang=0.2, loversize=-0.1, lraise=0.2]{\smash{#1}}{#2}
  }{\IfSubStr{É}{#1}{
    \lettrine[lines=4, lhang=0.2, loversize=-0, lraise=0]{\smash{#1}}{#2}
  }{\IfSubStr{ÀÂ}{#1}{
    \lettrine[lines=4, lhang=0.2, loversize=-0, lraise=0, slope=0.6em]{\smash{#1}}{#2}
  }{\IfSubStr{A}{#1}{
    \lettrine[lines=4, lhang=0.2, loversize=0.2, slope=0.6em]{\smash{#1}}{#2}
  }{\IfSubStr{V}{#1}{
    \lettrine[lines=4, lhang=0.2, loversize=0.2, slope=-0.5em]{\smash{#1}}{#2}
  }{
    \lettrine[lines=4, lhang=0.2, loversize=0.2]{\smash{#1}}{#2}
  }}}}}
  \let\LettrineFontHook\oldLFH
}
\newcommand{\labelchar}[1]{\textbf{\color{rubric} #1}}
\newcommand{\milestone}[1]{\autour{\footnotesize\color{rubric} #1}} % <milestone n="4"/>
\newcommand\name[1]{#1}
\newcommand\orig[1]{#1}
\newcommand\orgName[1]{#1}
\newcommand\persName[1]{#1}
\newcommand\placeName[1]{#1}
\newcommand{\pn}[1]{\IfSubStr{-—–¶}{#1}% <p n="3"/>
  {\noindent{\bfseries\color{rubric}   ¶  }}
  {{\footnotesize\autour{ #1}  }}}
\newcommand\reg{}
% \newcommand\ref{} % already defined
\newcommand\sic[1]{#1}
\newcommand\surname[1]{\textsc{#1}}
\newcommand\term[1]{\textbf{#1}}

\def\mednobreak{\ifdim\lastskip<\medskipamount
  \removelastskip\nopagebreak\medskip\fi}
\def\bignobreak{\ifdim\lastskip<\bigskipamount
  \removelastskip\nopagebreak\bigskip\fi}

% commands, blocks
\newcommand{\byline}[1]{\bigskip{\RaggedLeft{#1}\par}\bigskip}
\newcommand{\bibl}[1]{{\RaggedLeft{#1}\par\bigskip}}
\newcommand{\biblitem}[1]{{\noindent\hangindent=\parindent   #1\par}}
\newcommand{\dateline}[1]{\medskip{\RaggedLeft{#1}\par}\bigskip}
\newcommand{\labelblock}[1]{\medbreak{\noindent\color{rubric}\bfseries #1}\par\mednobreak}
\newcommand{\salute}[1]{\bigbreak{#1}\par\medbreak}
\newcommand{\signed}[1]{\bigbreak\filbreak{\raggedleft #1\par}\medskip}

% environments for blocks (some may become commands)
\newenvironment{borderbox}{}{} % framing content
\newenvironment{citbibl}{\ifvmode\hfill\fi}{\ifvmode\par\fi }
\newenvironment{docAuthor}{\ifvmode\vskip4pt\fontsize{16pt}{18pt}\selectfont\fi\itshape}{\ifvmode\par\fi }
\newenvironment{docDate}{}{\ifvmode\par\fi }
\newenvironment{docImprint}{\vskip6pt}{\ifvmode\par\fi }
\newenvironment{docTitle}{\vskip6pt\bfseries\fontsize{18pt}{22pt}\selectfont}{\par }
\newenvironment{msHead}{\vskip6pt}{\par}
\newenvironment{msItem}{\vskip6pt}{\par}
\newenvironment{titlePart}{}{\par }


% environments for block containers
\newenvironment{argument}{\itshape\parindent0pt}{\vskip1.5em}
\newenvironment{biblfree}{}{\ifvmode\par\fi }
\newenvironment{bibitemlist}[1]{%
  \list{\@biblabel{\@arabic\c@enumiv}}%
  {%
    \settowidth\labelwidth{\@biblabel{#1}}%
    \leftmargin\labelwidth
    \advance\leftmargin\labelsep
    \@openbib@code
    \usecounter{enumiv}%
    \let\p@enumiv\@empty
    \renewcommand\theenumiv{\@arabic\c@enumiv}%
  }
  \sloppy
  \clubpenalty4000
  \@clubpenalty \clubpenalty
  \widowpenalty4000%
  \sfcode`\.\@m
}%
{\def\@noitemerr
  {\@latex@warning{Empty `bibitemlist' environment}}%
\endlist}
\newenvironment{quoteblock}% may be used for ornaments
  {\begin{quoting}}
  {\end{quoting}}

% table () is preceded and finished by custom command
\newcommand{\tableopen}[1]{%
  \ifnum\strcmp{#1}{wide}=0{%
    \begin{center}
  }
  \else\ifnum\strcmp{#1}{long}=0{%
    \begin{center}
  }
  \else{%
    \begin{center}
  }
  \fi\fi
}
\newcommand{\tableclose}[1]{%
  \ifnum\strcmp{#1}{wide}=0{%
    \end{center}
  }
  \else\ifnum\strcmp{#1}{long}=0{%
    \end{center}
  }
  \else{%
    \end{center}
  }
  \fi\fi
}


% text structure
\newcommand\chapteropen{} % before chapter title
\newcommand\chaptercont{} % after title, argument, epigraph…
\newcommand\chapterclose{} % maybe useful for multicol settings
\setcounter{secnumdepth}{-2} % no counters for hierarchy titles
\setcounter{tocdepth}{5} % deep toc
\markright{\@title} % ???
\markboth{\@title}{\@author} % ???
\renewcommand\tableofcontents{\@starttoc{toc}}
% toclof format
% \renewcommand{\@tocrmarg}{0.1em} % Useless command?
% \renewcommand{\@pnumwidth}{0.5em} % {1.75em}
\renewcommand{\@cftmaketoctitle}{}
\setlength{\cftbeforesecskip}{\z@ \@plus.2\p@}
\renewcommand{\cftchapfont}{}
\renewcommand{\cftchapdotsep}{\cftdotsep}
\renewcommand{\cftchapleader}{\normalfont\cftdotfill{\cftchapdotsep}}
\renewcommand{\cftchappagefont}{\bfseries}
\setlength{\cftbeforechapskip}{0em \@plus\p@}
% \renewcommand{\cftsecfont}{\small\relax}
\renewcommand{\cftsecpagefont}{\normalfont}
% \renewcommand{\cftsubsecfont}{\small\relax}
\renewcommand{\cftsecdotsep}{\cftdotsep}
\renewcommand{\cftsecpagefont}{\normalfont}
\renewcommand{\cftsecleader}{\normalfont\cftdotfill{\cftsecdotsep}}
\setlength{\cftsecindent}{1em}
\setlength{\cftsubsecindent}{2em}
\setlength{\cftsubsubsecindent}{3em}
\setlength{\cftchapnumwidth}{1em}
\setlength{\cftsecnumwidth}{1em}
\setlength{\cftsubsecnumwidth}{1em}
\setlength{\cftsubsubsecnumwidth}{1em}

% footnotes
\newif\ifheading
\newcommand*{\fnmarkscale}{\ifheading 0.70 \else 1 \fi}
\renewcommand\footnoterule{\vspace*{0.3cm}\hrule height \arrayrulewidth width 3cm \vspace*{0.3cm}}
\setlength\footnotesep{1.5\footnotesep} % footnote separator
\renewcommand\@makefntext[1]{\parindent 1.5em \noindent \hb@xt@1.8em{\hss{\normalfont\@thefnmark . }}#1} % no superscipt in foot


% orphans and widows
\clubpenalty=9996
\widowpenalty=9999
\brokenpenalty=4991
\predisplaypenalty=10000
\postdisplaypenalty=1549
\displaywidowpenalty=1602
\hyphenpenalty=400
% Copied from Rahtz but not understood
\def\@pnumwidth{1.55em}
\def\@tocrmarg {2.55em}
\def\@dotsep{4.5}
\emergencystretch 3em
\hbadness=4000
\pretolerance=750
\tolerance=2000
\vbadness=4000
\def\Gin@extensions{.pdf,.png,.jpg,.mps,.tif}
% \renewcommand{\@cite}[1]{#1} % biblio

\makeatother % /@@@>
%%%%%%%%%%%%%%
% </TEI> end %
%%%%%%%%%%%%%%


%%%%%%%%%%%%%
% footnotes %
%%%%%%%%%%%%%
\renewcommand{\thefootnote}{\bfseries\textcolor{rubric}{\arabic{footnote}}} % color for footnote marks

%%%%%%%%%
% Fonts %
%%%%%%%%%
\usepackage[]{roboto} % SmallCaps, Regular is a bit bold
% \linespread{0.90} % too compact, keep font natural
\newfontfamily\fontrun[]{Roboto Condensed Light} % condensed runing heads
\ifav
  \setmainfont[
    ItalicFont={Roboto Light Italic},
  ]{Roboto}
\else\ifbooklet
  \setmainfont[
    ItalicFont={Roboto Light Italic},
  ]{Roboto}
\else
\setmainfont[
  ItalicFont={Roboto Italic},
]{Roboto Light}
\fi\fi
\renewcommand{\LettrineFontHook}{\bfseries\color{rubric}}
% \renewenvironment{labelblock}{\begin{center}\bfseries\color{rubric}}{\end{center}}

%%%%%%%%
% MISC %
%%%%%%%%

\setdefaultlanguage[frenchpart=false]{french} % bug on part


\newenvironment{quotebar}{%
    \def\FrameCommand{{\color{rubric!10!}\vrule width 0.5em} \hspace{0.9em}}%
    \def\OuterFrameSep{\itemsep} % séparateur vertical
    \MakeFramed {\advance\hsize-\width \FrameRestore}
  }%
  {%
    \endMakeFramed
  }
\renewenvironment{quoteblock}% may be used for ornaments
  {%
    \savenotes
    \setstretch{0.9}
    \normalfont
    \begin{quotebar}
  }
  {%
    \end{quotebar}
    \spewnotes
  }


\renewcommand{\headrulewidth}{\arrayrulewidth}
\renewcommand{\headrule}{{\color{rubric}\hrule}}

% delicate tuning, image has produce line-height problems in title on 2 lines
\titleformat{name=\chapter} % command
  [display] % shape
  {\vspace{1.5em}\centering} % format
  {} % label
  {0pt} % separator between n
  {}
[{\color{rubric}\huge\textbf{#1}}\bigskip] % after code
% \titlespacing{command}{left spacing}{before spacing}{after spacing}[right]
\titlespacing*{\chapter}{0pt}{-2em}{0pt}[0pt]

\titleformat{name=\section}
  [block]{}{}{}{}
  [\vbox{\color{rubric}\large\raggedleft\textbf{#1}}]
\titlespacing{\section}{0pt}{0pt plus 4pt minus 2pt}{\baselineskip}

\titleformat{name=\subsection}
  [block]
  {}
  {} % \thesection
  {} % separator \arrayrulewidth
  {}
[\vbox{\large\textbf{#1}}]
% \titlespacing{\subsection}{0pt}{0pt plus 4pt minus 2pt}{\baselineskip}

\ifaiv
  \fancypagestyle{main}{%
    \fancyhf{}
    \setlength{\headheight}{1.5em}
    \fancyhead{} % reset head
    \fancyfoot{} % reset foot
    \fancyhead[L]{\truncate{0.45\headwidth}{\fontrun\elbibl}} % book ref
    \fancyhead[R]{\truncate{0.45\headwidth}{ \fontrun\nouppercase\leftmark}} % Chapter title
    \fancyhead[C]{\thepage}
  }
  \fancypagestyle{plain}{% apply to chapter
    \fancyhf{}% clear all header and footer fields
    \setlength{\headheight}{1.5em}
    \fancyhead[L]{\truncate{0.9\headwidth}{\fontrun\elbibl}}
    \fancyhead[R]{\thepage}
  }
\else
  \fancypagestyle{main}{%
    \fancyhf{}
    \setlength{\headheight}{1.5em}
    \fancyhead{} % reset head
    \fancyfoot{} % reset foot
    \fancyhead[RE]{\truncate{0.9\headwidth}{\fontrun\elbibl}} % book ref
    \fancyhead[LO]{\truncate{0.9\headwidth}{\fontrun\nouppercase\leftmark}} % Chapter title, \nouppercase needed
    \fancyhead[RO,LE]{\thepage}
  }
  \fancypagestyle{plain}{% apply to chapter
    \fancyhf{}% clear all header and footer fields
    \setlength{\headheight}{1.5em}
    \fancyhead[L]{\truncate{0.9\headwidth}{\fontrun\elbibl}}
    \fancyhead[R]{\thepage}
  }
\fi

\ifav % a5 only
  \titleclass{\section}{top}
\fi

\newcommand\chapo{{%
  \vspace*{-3em}
  \centering % no vskip ()
  {\Large\addfontfeature{LetterSpace=25}\bfseries{\elauthor}}\par
  \smallskip
  {\large\eldate}\par
  \bigskip
  {\Large\selectfont{\eltitle}}\par
  \bigskip
  {\color{rubric}\hline\par}
  \bigskip
  {\Large LIVRE LIBRE À PRIX LIBRE, DEMANDEZ AU COMPTOIR\par}
  \centerline{\small\color{rubric} {hurlus.fr, tiré le \today}}\par
  \bigskip
}}


\begin{document}
\pagestyle{empty}
\ifbooklet{
  \thispagestyle{empty}
  \centering
  {\LARGE\bfseries{\elauthor}}\par
  \bigskip
  {\Large\eldate}\par
  \bigskip
  \bigskip
  {\LARGE\selectfont{\eltitle}}\par
  \vfill\null
  {\color{rubric}\setlength{\arrayrulewidth}{2pt}\hline\par}
  \vfill\null
  {\Large LIVRE LIBRE À PRIX LIBRE, DEMANDEZ AU COMPTOIR\par}
  \centerline{\small{hurlus.fr, tiré le \today}}\par
  \newpage\null\thispagestyle{empty}\newpage
  \addtocounter{page}{-2}
}\fi

\thispagestyle{empty}
\ifaiv
  \twocolumn[\chapo]
\else
  \chapo
\fi
{\it\elabstract}
\bigskip
\makeatletter\@starttoc{toc}\makeatother % toc without new page
\bigskip

\pagestyle{main} % after style

  \section[Rapport au maréchal de France, chef de l’État français]{Rapport au maréchal de France, chef de l’État français}\renewcommand{\leftmark}{Rapport au maréchal de France, chef de l’État français}

\noindent L’élaboration d’une « Charte du travail », la détermination de rapports harmonieux et justes entre les patrons, les ouvriers, les techniciens, les artisans ont été, depuis quinze mois, l’une de vos plus constantes préoccupations.\par
A tout instant – dans vos audiences, au sein des conseils de gouvernement, dans vos messages et dans vos discours – vous avez rappelé votre désir d’imprégner d’un esprit social et novateur les grandes règles de l’organisation française du travail.\par
Le projet que nous avons l’honneur de vous soumettre est le résultat d’un travail considérable. Il s’appuie sur les vœux émis dans les cahiers nombreux que vous ont adressé, le 1\textsuperscript{er} mai dernier, les provinces françaises. Il tient compte de l’abondante documentation que vous avez recueillie et que vous avez bien voulu nous transmettre. Il s’inspire, enfin, largement des avis qui vous ont été soumis au cours des trois sessions du comité d’organisation professionnelle créé le 28 février dernier.\par
Ce comité a pleinement compris l’orientation qu’il convenait de donner au monde du travail. Il l’a montré, en soulignant, par une déclaration solennelle, sa volonté de rompre définitivement avec le vieux système de la lutte des classes.\par
C’est dans cet esprit qu’il a travaillé. C’est dans cet esprit que nous avons rédigé le projet de charte.\par
La charte précise les grandes règles qui régiront désormais les rapports des travailleurs, aussi bien dans l’exercice de leur métier que dans le développement de leur vie matérielle et morale. Elle s’adresse à l’industrie et au commerce, aux petites, aux moyennes et aux grandes entreprises.\par
Elle n’a pas la prétention d’apporter par elle-même des satisfactions directes, mais elle crée des institutions aptes à engendrer une atmosphère plus propice à la justice pour tous et à la prospérité pour chacun.\par
Nous tenons cependant à souligner deux réalisations concrètes, dont les grandes lignes sont édictées par la charte.\par
Elle fixe, tout d’abord, les principes du mode de détermination de salaires, mettant ainsi un terme à la plus grande source d’injustices et de discordes intestines du passé dans le monde du travail.\par
S’inspirant des directions, que vous avez données, récemment encore, tendant à instituer une participation au bénéfice des collaborateurs des entreprises, elle décide ensuite\par
que des prélèvements effectués sur ces bénéfices serviront à la création d’un fonds commun destiné à améliorer la sécurité et le bien-être de ces collaborateurs.\par
La charte renforce ainsi davantage encore la solidarité déjà si réelle entre les travailleurs et leurs industries.\par
Il est vain de penser que des ouvriers puissent être heureux au sein d’une industrie en détresse, la prospérité des entreprises conditionne le bien-être de leurs membres. La pierre angulaire de la charte réside dans la création des comités mixtes sociaux, au sein desquels se trouveront réunis tous les membres d’une même profession.\par
Le comité social sera, pour la profession d’aujourd’hui – pour la corporation de demain – le véritable animateur de la vie professionnelle.\par
Lien de tous ceux qui concourent à une même production, il recevra, de surcroît, la mission d’assurer la gestion sociale de la profession.\par
Il aura sa maison commune, où tout homme appartenant à une entreprise de la profession sera sûr, quel que soit son rang, de trouver encouragement, aide et protection.\par
L’expérience a montré que partout où des hommes de bonne foi se réunissent pour une explication loyale et franche, les oppositions s’atténuent, les malentendus se dissipent, l’accord s’établit, dans l’estime d’abord, dans l’amitié ensuite.\par
C’est en utilisant les bases de l’organisation professionnelle existante que sera réalisée – dans un esprit nouveau – la jonction de tous ceux que la vie sociale appelle à collaborer.\par
Les syndicats ont donc leur place dans cet ordre nouveau. Ils auront la double mission de discipliner les libres réactions de leurs adhérents et de participer à la formation des comités sociaux.\par
Mais les syndicats ne seront plus les syndicats de tendance du passé. S’ils demeurent voués dans chaque profession à la représentation d’une même catégorie sociale (patrons, ouvriers, cadres), ils seront désormais obligatoires pour être forts, uniques, pour être francs. Leur activité sera désormais strictement limitée au domaine de leur profession. Ils vivront et fonctionneront sous l’autorité des comités sociaux et en s’inspirant de leurs doctrines qui ne sauraient être elles-mêmes que celles du gouvernement.\par
Dotée de sa charte sociale, la famille professionnelle apparaîtra comme un corps vivant. Elle respectera les lois de l’État. L’État la respectera.\par
Elle servira de base à la création des futures corporations qui restent le grand espoir de l’avenir français. Seul le souci de ménager les étapes et de construire avec fruit n’a permis jusqu’ici de réaliser les corporations que partiellement.\par
Ces corporations ne se réaliseront que dans une heureuse articulation des liens sociaux et des intérêts économiques d’un même groupe de professions. L’interpénétration de l’économique et du social est une œuvre de longue haleine. Mais la charte du travail définit déjà les liens sociaux. Elle repose, de surcroît, sur une division de notre activité économique en grandes familles professionnelles, au sein desquelles se créeront les sections nécessaires – notamment les sections artisanales – dont l’ensemble fournira une première et utile ébauche de l’œuvre corporative.\par
La charte du travail ne peut, par elle-même, atteindre les buts qu’elle se propose, sans définir en tête de ses articles l’élément spirituel qu’elle contient.\par
Cet élément spirituel, monsieur le Maréchal, c’est le vôtre. C’est celui que vous avez communiqué à la France et dont l’œuvre de révolution nationale tire sa justification la plus sûre.\par
Cet élément spirituel, c’est l’aspiration vers un ordre nouveau où seront assurés :\par

\begin{itemize}[itemsep=0pt,]
\item la primauté de la Nation et du bien commun professionnel sur les intérêts particuliers ;
\item la collaboration confiante, loyale et permanente de tous les membres de la profession en vue de réaliser la paix sociale et la prospérité des entreprises ;
\item le respect d’une hiérarchie fondée sur le travail, le talent et le mérite ;
\item le développement progressif des réalisations sociales destinées à satisfaire les intérêts et les aspirations légitimes des travailleurs.
\end{itemize}

\noindent La paix sociale est le but suprême. Les institutions du passé ne peuvent être maintenues que dans la mesure où elles expriment le génie libre et divers de la nation. L’avenir est encore riche, chez nous, d’idées, d’efforts, de sacrifices. C’est vers cet avenir que nous nous tournons résolument, sûrs de l’assentiment des patrons, des ouvriers, des techniciens, des artisans, désormais convaincus que l’intérêt personnel ne trouvera sa sauvegarde que dans l’intérêt collectif.\par
C’est dans cet esprit que nous avons l’honneur de vous soumettre cette charte, que le pays attend, que le monde du travail a longuement souhaitée et qui, par son ampleur comme par sa nouveauté, prendra logiquement sa place dans la série des textes constitutionnels de la France nouvelle.
\section[Titre Ier. Principes généraux]{Titre I\textsuperscript{er.} \\
Principes généraux}\renewcommand{\leftmark}{Titre I\textsuperscript{er.} \\
Principes généraux}

\noindent Art. 1\textsuperscript{er} : Les activités professionnelles sont réparties entre un nombre déterminé de familles industrielles ou commerciales.\par
Ces familles, et les professions qui les composent, sont organisées dans les conditions générales fixées par la présente loi en vue de gérer en commun les intérêts professionnels de leurs membres de toute catégories et d’apporter leur concours à l’économie nationale, selon les directions des pouvoirs publics.\par
\bigbreak
\noindent \labelchar{Art. 2.} Dans le cadre de cette organisation, toutes les personnes participant à une activité professionnelle jouissent de droits et assument des devoirs, des obligations et des responsabilités.\par
Elles sont soumises aux lois et règlements professionnels généraux, ainsi qu’aux décisions corporatives.\par
Elles participent obligatoirement aux dépenses de fonctionnement des groupements dont elles relèvent.\par
Elles ont le devoir de pratiquer loyalement, à l’égard des autres membres de la profession, la collaboration et la solidarité qui sont les principes essentiels sur lesquels repose l’institution corporative.\par
En contrepartie, elles bénéficient du statut et des institutions professionnelles, participent à l’activité de l’organisme auquel elles adhèrent directement, et sont représentées dans les assemblées nationales constitutionnelles.\par
Elles possèdent la propriété d’une qualification professionnelle correspondant à leurs aptitudes, qui donne aux salariés, en échange du travail correspondant, le droit au salaire et avantages attachés à cette qualification, conformément aux règlements de la profession.\par
Les employeurs jouissent dans leur entreprise de l’autorité qui correspond aux responsabilités sociales, techniques et financières qu’ils assument.\par
La fonction patronale impose le devoir de gérer l’entreprise pour le bien commun de tous ses membres.\par
\bigbreak
\noindent \labelchar{Art. 3.} Dans le cadre de la législation en vigueur, les professions organisées s’efforcent d’assurer à leurs membres la sécurité du travail et contribuent à leur mieux-être et à celui des personnes à leur charge, par la création et la gestion d’institutions sociales de toute nature.\par
\bigbreak
\noindent \labelchar{Art. 4.} L’organisation professionnelle est appelée à connaître de tous les aspects sociaux et économiques de l’activité professionnelle. Toutefois, en raison des circonstances et sauf exceptions prévues à l’article 39, les questions d’ordre économiques resteront, jusqu’à ce qu’il en soit autrement décidé, dans les attributions des comités provisoires d’organisation créés en application de la loi du 26 août 1940.\par
\bigbreak
\noindent \labelchar{Art. 5.} Le lock-out et la grève sont et restent interdits.
\section[Titre II. Classification des industries]{Titre II. \\
Classification des industries}\renewcommand{\leftmark}{Titre II. \\
Classification des industries}


\labelblock{Commerces et professions}

\noindent \labelchar{Art. 6.} L’organisation prévue par la présente loiest à la fois sociale et professionnelle ; les activités auxquelles elle s’applique font, en conséquence, l’objet d’une double classification.\par
Pour les questions d’ordre social, les établissements industriels et commerciaux sont répartis entre un nombre déterminé de familles professionnelles.\par
Une organisation distincte est réalisée pour chacune de ces familles et, éventuellement, dans le cadre de la famille, par industrie ou par profession.\par
Pour les questions d’ordre professionnel, chaque profession est rattachée à l’une des familles professionnelles choisie en raison de sa compétence particulière à l’égard de la profession considérée, à charge par cette famille de constituer les organismes qualifiés pour traiter les problèmes des professions qui lui sont attachées.\par
\bigbreak
\noindent \labelchar{Art. 7.} Sont exclus du champ d’activité de la présente loi :\par

\begin{itemize}[itemsep=0pt,]
\item les fonctionnaires définis par l’article 2 de la loi du 14 septembre 1941 portant statut général des fonctionnaires civils de l’État et des établissements publics de l’État ;
\item les membres des ordres et le personnel des professions régis par des statuts, chartes et mesures législatives particulières, sous réserve que ces textes auront été publiés postérieurement au 15 juillet 1940.
\end{itemize}

\noindent Un règlement d’administration publique déterminera dans quelles conditions celles des dispositions de la présente loi qui ne sont pas incompatibles avec la loi du 14 septembre 1941, relative au droit d’association du personnel non-fonctionnaire des services publics exploités en régie, devront être appliquées à ce personnel.\par
Les agents des services publics industriels autres que ceux visés par la loi précitée du 14 septembre 1941 sont soumis aux dispositions de la présente loi. Toutefois, un régime particulier pourra être établi pour certains d’entr’eux par des lois spéciales.\par
\bigbreak
\noindent \labelchar{Art. 8.} Seront approuvés par décrets les tableaux fixant :\par

\begin{itemize}[itemsep=0pt,]
\item la nomenclature des familles professionnelles ;
\item la répartition des industries et commerces entre familles professionnelles ;
\item le rattachement des professions aux familles professionnelles ;
\item la correspondance entre les familles professionnelles et les comités provisoires d’organisation institués en application de la loi du 16 août 1940.
\end{itemize}

\section[Titre III.]{Titre III.}\renewcommand{\leftmark}{Titre III.}

\subsection[Chapitre Ier. Les syndicats]{Chapitre Ier. \\
Les syndicats}
\noindent \labelchar{Art. 9.} Les membres des professions sont groupés en syndicats professionnels.\par
Dans une même circonscription, pour une même profession, industrie ou famille professionnelle, et une même catégorie de membres, il sera formé un syndicat professionnel unique.\par
Les conditions dans lesquelles seront formés les nouveaux syndicats uniques en partant des organismes existants seront fixées par décrets.\par
\bigbreak
\noindent \labelchar{Art. 10.} Les syndicats professionnels sont constitués par catégories distinctes de membres.\par
Sont considérés comme pouvant former une catégorie distincte :\par

\begin{itemize}[itemsep=0pt,]
\item les employeurs ;
\item les ouvriers ;
\item les employés ;
\item les agents de maîtrise ;
\item les ingénieurs, cadres administratifs et commerciaux.
\end{itemize}

\noindent Les catégories similaires peuvent être fusionnées, notamment lorsque les effectifs de l’une d’elles sont insuffisantes pour constituer un organisme distinct.\par
Est considéré comme appartenant à la catégorie des employeurs le personnel de direction ayant reçu délégation de la signature sociale d’un patron ou d’une société.\par
Parmi les membres des sociétés coopératives, le président et le directeur général sont considérés comme appartenant à la catégorie des employeurs ; les autres membres entrent dans la catégorie ressortissant à leur fonction professionnelle.\par
\bigbreak
\noindent \labelchar{Art. 11.} Constitués pour rassembler directement les membres des professions au premier degré, les syndicats professionnels ont un caractère local.\par
Leur circonscription territoriale, qui reste néanmoins variable suivant les régions et les professions, sera déterminée dans chaque cas par les commissions prévues à l’article 77, étant entendu :\par

\begin{itemize}[itemsep=0pt,]
\item qu’un syndicat englobera en principe le personnel de plusieurs entreprises ;
\item qu’il n’y aura pas nécessairement similitude entre les circonscriptions des syndicats des différentes catégories.
\end{itemize}

\bigbreak
\noindent \labelchar{Art. 12.} Toutes les personnes, quels que soient leur âge et leur nationalité, exerçant une activité professionnelle, sont inscrites d’office au syndicat professionnel de leur catégorie, de leur circonscription et de leur profession, sous la responsabilité de ce syndicat, à moins qu’elles ne justifient de leur inscription dans l’u n des organismes prévus au chapitre II du titre IV.\par
Tout membre d’un syndicat peut être exclu par décis on du comité social régional de la profession ou du groupe de professions, après avis du bureau du syndicat, soit pour violation grave ou répétée de la législation du travail ou des règlements corporatifs, soit pour activité contraire à l’intérêt général du pays, soit pour des motifs d’ordre public.\par
Il pourra être fait appel des décisions du comité social régional devant le comité social national qui statue en dernier ressort.\par
Les personnes exclues d’un syndicat ne participent plus à l’activité de cet organisme, mais restent soumises aux obligations et devoirs corporatifs.\par

\labelblock{Intégration de l’artisanat dans l’organisation syndicale}

\noindent \labelchar{Art. 13.} Les artisans constituent, en principe, une section spéciale des syndicats professionnels.\par
Pour établir une correspondance entre les chambres de métiers et les organisations syndicales, les artisans sont répartis au sein des chambres de métiers, en sections professionnelles ; ces sections correspondent aux professions ou groupes de professions ayant donné lieu à la formation de syndicats professionnels.\par
Une représentation répondant à leur importance dans la profession ou le groupe de professions est assurée aux artisans dans les conseils syndicaux et organismes corporatifs des différents échelons.\par

\labelblock{Attributions, administrations et fonctionnements des syndicats}

\noindent \labelchar{Art. 14.} Les attributions des syndicats professionnels sont :\par

\begin{itemize}[itemsep=0pt,]
\item l’encadrement et la représentation de leurs ressortissants :
\item la transmission ou l’exécution des décisions corporatives ;
\item l’étude des questions professionnelles en vue de la présentation de suggestions corporatives ;
\item la recherche éventuelle des solutions à appliquer aux problèmes intéressant leurs propres membres dans leur circonscription territoriale.
\end{itemize}

\noindent Elles excluent strictement toute activité politique ou confessionnelle.\par
\bigbreak
\noindent \labelchar{Art. 15.} Les syndicats professionnels peuvent, sans autorisation, acquérir à titre onéreux, posséder et administrer les locaux et biens mobiliers destinés à leur fonctionnement administratif et à la réunion de leurs membres.\par
Ils disposent des fonds provenant des cotisations de leurs membres dans la limite nécessaire à leur fonctionnement et gèrent ces fonds.\par
Ils peuvent ester en justice.\par
\bigbreak
\noindent \labelchar{Art. 16.} Le syndicat professionnel est dirigé par un conseil d’administration dont la composition et le mode de désignation seront fixés par décrets.\par
Le conseil d’administration élit son bureau composé, en principe, de quatre membres. Ne peuvent être membres des conseils d’administration que les personnes de nationalité française d’origine, âgées de vingt-cinq ans au moins, n’ayant encouru aucune condamnation pour crime ou délit infamant, justifiant de tous leurs droits civils et exerçant la profession depuis cinq années au moins, dont deux ans dans la circonscription du syndicat. Une même personne ne peut exercer plus de deux mandats successifs, sauf dérogation accordée dans des conditions qui seront fixées par les décrets prévus à l’alinéa 1 du présent article.\par
Le renouvellement des conseils et bureaux s’opèrent toujours par fraction.\par
\bigbreak
\noindent \labelchar{Art. 17.} Les statuts et le règlement des syndicats professionnels doivent être approuvés par le comité social national de la profession ou du groupe de professions, à moins qu’ils ne soient conformes à un modèle type qui sera établi par un décret en Conseil d’État.\par
Le conseil d’administration délibère à la majoritédes membres présents. Les votes ont lieu au scrutin secret.\par
\bigbreak
\noindent \labelchar{Art. 18.} Les dépenses de fonctionnement des organismes professionnels sont couvertes par une contribution du comité social correspondant et par une cotisation des membres participants.
\subsection[Chapitre II. Les unions et les fédérations]{Chapitre II. \\
Les unions et les fédérations}
\noindent \labelchar{Art. 19.} Il est institué par profession ou groupe de professions, et par catégorie distincte, des unions et des fédérations professionnelles.\par
Les unions rassemblent, sur le plan régional, des représentants des conseils des syndicats professionnels.\par
Les fédérations rassemblent, sur le plan national, des représentants des unions régionales.\par
Certains sièges peuvent être réservés à des personnes ayant une action sociale sur le plan national, et ayant ou dirigeant des entreprises dans plusieurs régions. Les titulaires de ces sièges seront désignés par arrêté du secrétaire d’État au Travail, sur proposition du comité social national de la profession.\par
Pour une même famille professionnelle ou une même profession, et pour une même catégorie de membres, il ne peut être formé qu’une seule union par région et une seule fédération.\par
Les unions et fédérations élisent leurs conseils d’administration qui désignent à leur tour leurs bureaux.\par
Un décret fixera les conditions de désignation des membres des unions et fédérations, la composition de ces organismes et celle de leur conseil d’administration et bureau.\par
Les membres des unions et fédérations doivent répondre aux conditions fixées à l’article 16.\par
\bigbreak
\noindent \labelchar{Art. 20.} Les unions et les fédérations assurent la coordination de l’organisation syndicale. Leur activité s’exerce sous l’égide et selon les directions des comités sociaux fonctionnant à leur échelon. Elles ont la capacité définie à l’article 15 pour les syndicats.\par
\bigbreak
\noindent \labelchar{Art. 21.} Le statut et le règlement intérieur des unions professionnelles doivent être approuvés par le comité social national compétent.\par
Pour les fédérations, ces documents sont approuvés par le secrétaire d’État au Travail, après avis du ou des secrétaires d’État dont relève la famille ou la profession intéressée.\par
\bigbreak
\noindent \labelchar{Art. 22.} Les dispositions prévues à l’article 18 pour les syndicats sont applicables aux unions et fédérations professionnelles.
\section[Titre IV. Les comités sociaux et les corporations]{Titre IV. \\
Les comités sociaux et les corporations}\renewcommand{\leftmark}{Titre IV. \\
Les comités sociaux et les corporations}

\subsection[Chapitre Ier. Les comités sociaux d’entreprises]{Chapitre I\textsuperscript{er}. \\
Les comités sociaux d’entreprises}
\noindent \labelchar{Art. 23.} La collaboration entre employeurs et salariés est obligatoirement organisée dans les établissements dont l’effectif est au moins égal à cent ouvriers ou employés, au sein de comités sociaux ou d’établissements » qui rassemblent le chef d’entreprise et des représentants de toutes les catégories du personnel.\par
\bigbreak
\noindent \labelchar{Art. 24.} Les comités sociaux d’établissements réalisent au premier degré la collaboration sociale et professionnelle entre la direction et le personnel.\par
Leurs attributions excluent toute immixtion dans la conduite et la gestion de l’entreprise et dans les questions débordant le cadre de cette entreprise ; sous ces réserves, elles s’exercent dans le sens le plus large, notamment en vue :\par

\begin{itemize}[itemsep=0pt,]
\item d’aider la direction à résoudre toutes les questions relatives au travail et à la vie du personnel dans l’établissement ;
\item de provoquer un échange d’informations mutuel sur toutes les questions intéressant la vie sociale du personnel et des familles ;
\item de réaliser les mesures d’entraide sociale dans le cadre d’activité du comité social local correspondant.
\end{itemize}

\noindent Leur mode de fonctionnement est laissé à leur propre initiative.\par
Ils sont placés sous l’autorité corporative et le contrôle du comité social local de la profession.\par
\bigbreak
\noindent \labelchar{Art. 25.} Pour les entreprises comportant des établissements multiples de faible effectif, il pourra être constitué des comités sociaux d’entreprises réunissant le personnel de ces établissements existant dans une même région.\par
\bigbreak
\noindent \labelchar{Art. 26.} Les premiers comités sociaux d’établissements seront constitués par les représentants des différentes catégories de personnel de l’établissement en accord avec le chef de l’établissement.\par
Le comité social local donne son agrément à la composition du comité social d’établissement ; il arbitre les litiges qui peuvent naître à l’occasion de sa constitution.
\subsection[Chapitre II. Les comités sociaux par famille professionnelle ou profession]{Chapitre II. \\
Les comités sociaux par famille professionnelle ou profession}
\noindent \labelchar{Art. 27.} Il est créé dans chaque famille professionnelle ou profession et à chacun des échelons local, régional et national, un organisme corporatif à compétence sociale et professionnelle qui prend respectivement le titre de comité social local, régional et national.\par
\bigbreak
\noindent \labelchar{Art. 28.} Le comité social local comprend douze membres au moins et vingt-quatre au plus, pris dans les bureaux des syndicats professionnels existants, pour la famille ou la profession, dans la circonscription.\par
Les membres sont répartis en trois groupes égaux formés par :\par

\begin{itemize}[itemsep=0pt,]
\item la catégorie « employeurs » ;
\item la catégorie « ouvriers » et « employés », dans une proportion correspondant à la prédominance industrielle ou commerciale de la famille ou de la profession considérée ;
\item les autres catégories.
\end{itemize}

\noindent Le comité social désigne trois présidents constituant son bureau, choisis chacun dans l’un des groupes définis ci-dessus et présidant à tour de rôle par période de huit mois.\par
\bigbreak
\noindent \labelchar{Art. 29.} Les comités sociaux régionaux et nationaux sont formés, comme les comités locaux, sur le mode tripartite ; leur bureau est constitué et fonctionne dans les mêmes conditions que celles qui sont prévues pour les comités locaux.\par
Les membres des comités sociaux régionaux sont désignés par catégorie par les comités sociaux locaux. Les membres des comités sociaux nationaux sont désignés par catégorie par les comités sociaux régionaux. Un certain nombre d’entre eux sont obligatoirement choisis parmi les membres des bureaux des organismes professionnels de l’échelon correspondant.\par
Les effectifs des comités régionaux et nationaux et les conditions de désignation des membres des comités sociaux aux différents échelons local, régional et national seront fixés par décrets contresignés par le secrétaire d’État au Travail.\par
\bigbreak
\noindent \labelchar{Art. 30.} Le comité social se constitue en commissions mixtes, d’importance et de compositions variables, pour traiter les différentes catégories de questions qui entrent dans ses attributions.\par
Il peut s’adjoindre pour leur confier, sous sa responsabilité, un rôle d’étude ou d’action, des commissions mixtes constituées en totalité ou en partie hors de son sein.\par
Les membres de ces commissions sont choisis dans les conseils des syndicats, unions ou fédérations ou, en dehors de ces organismes, parmi les personnes qualifiées par leur activité ou leurs compétences sociales.\par
Le comité social peut être, à tout moment, convoqué par le président en exercice ou sur la demande de l’un des autres présidents.\par
Chaque comité social établit son statut et son règlement intérieur ; ces documents doivent être approuvés par le comité institué à l’échelon supérieur.\par
Les statuts et règlements des comités nationaux sont approuvés par arrêtés du secrétaire d’État au Travail, après avis du secrétaire d’État dont relève la profession ou la famille professionnelle.\par
Les comités sociaux siègent à la maison commune créée par l’article 50.\par

\labelblock{Attributions des comités sociaux}

\noindent \labelchar{Art. 31.} Les attributions des comités sociaux sont d’ordre professionnel et social ; elles excluent toute activité politique ou confessionnelle.\par
Dans l’ordre professionnel, elles comportent notamment :\par

\begin{itemize}[itemsep=0pt,]
\item les questions de salaire et de conventions collectives ;
\item les questions de formation professionnelle : apprentissage, perfectionnement, reclassement, écoles de cadres, etc.
\item l’élaboration des règlements relatifs à l’embauchage et au licenciement ;
\item l’étude et l’application des mesures relatives à l’hygiène et à la sécurité au travail.
\end{itemize}

\noindent Les questions d’appointements, de salaires ou autres, intéressant particulièrement une catégorie, pourront être discutées paritairement entre les représentants de cette catégorie et celle des employeurs.\par
\bigbreak
\noindent \labelchar{Art. 32.} En outre, pour chacune des professions qui lui est organiquement rattachée dans les conditions prévues à l’article 6, le comité social étudie, met au point ou applique les dispositions relatives à la pratique et à la propriété du métier, à la qualification professionnelle et à la promotion ouvrière.\par
Les commissions chargées de traiter les questions qui font l’objet du présent article comprennent, le cas échéant, des artisans.\par
\bigbreak
\noindent \labelchar{Art. 33.} Dans l’ordre social et familial, les comités sociaux étudient et réalisent toutes les mesures propres à mettre en œuvre les devoirs des corporations à l’égard de leurs membres, telles que :\par

\begin{itemize}[itemsep=0pt,]
\item la sécurité de l’emploi par la lutte systématique contre le chômage et les mesures de prévoyance en faveur des chômeurs ;
\item la généralisation et la gestion d’assurances et de retraites ;
\item l’entraide et l’assistance ;
\item l’aide familiale, sous les formes morale, matérielle et intellectuelle ;
\item l’amélioration des conditions d’existence : habitations, jardins, sports, loisirs et distractions, arts, culture générale, etc.
\end{itemize}

\bigbreak
\noindent Art.34 : Pour assurer le contrôle de l’application des lois et règlements professionnels, et de leurs décisions de toute nature, les comités sociaux font appel à des commissaires corporatifs assermentés.\par
Ces commissaires sont habilités à contrôler les co nditions de travail dans tous les établissements relevant du comité social. Ils recueillent les doléances et suggestions des différentes catégories de membres.\par
Ils signalent directement aux intéressés, afin qu’il y soit remédié sur le champ, toutes les infractions qu’ils constatent. Ils rendent compte à leur comité de toutes leurs activités et attirent son attention sur les cas qu’ils n’ont pu résoudre.\par
Le contrôle ainsi assuré au titre des organismes corporatifs est indépendant de celui qui demeure exercé par les services des secrétariat d’État compétents et, notamment, par l’Inspection du travail.\par

\labelblock{Pouvoirs et prérogatives des comités sociaux}

\noindent \labelchar{Art. 35.} Le comité social représente légalement, dans sa circonscription, la profession ou la famille professionnelle pour laquelle il a été constitué, devant les pouvoirs publics, les juridictions et les organismes de toute nature, publics ou privés.\par
Ses décisions ont un caractère réglementaire et sont obligatoires, sauf opposition du comité social de l’échelon supérieur ou des pouvoirs publics.\par
Il jouit de la personnalité civile.\par
Il a le droit d’ester en justice et d’acquérir sans autorisation tous biens meubles et immeubles et faire tous les actes, créer et gérer oust les organismes et institutions nécessaires à son activité.\par
Les institutions sociales de toute nature, créées par des particuliers ou des collectivités dans l’intérêt du personnel d’une entreprise ou d’une profession, ou des familles de ce personnel, sont obligatoirement gérées par le comité social d’entreprise, local ou régional, désigné par le comité social de la profession considérée.\par

\labelblock{Attributions relatives des comités aux différents échelons}

\noindent \labelchar{Art. 36.} Le comité national assume la haute direction sociale de la famille professionnelle ou de la profession.\par
Il favorise les initiatives régionales et locales.\par
Il coordonne et régularise l’activité des comités régionaux\par
Il centralise les éléments d’étude et d’information, les exploite et assure leur diffusion.\par
Il élabore, adapte ou entérine les clauses générales des conventions collectives, les tableaux des qualifications professionnelles et les règles de cette qualification, ainsi que celles de la promotion ouvrière, les coefficients applicables aux qualifications pour la détermination des salaires et enfin les règles générales d’embauchage et de licenciement.\par
Il arrête ou approuve les règlements professionnel généraux, notamment ceux touchant à l’hygiène et à la sécurité du travail.\par
Il conduit et oriente l’action sociale de la famille ou de la profession et gère les institutions et caisses auxquelles il estime devoir donner un caractère national.\par
Le comité régional assure le même rôle dans le cadre des directions et instructions du comité national.\par
Il coordonne l’activité des comités locaux, centralise les renseignements qui leur sont demandés et leur diffuse la documentation qu’il reçoit.\par
Il adapte en tant que de besoin au cadre régional les règlements, conventions et décisions de toute nature.\par
Il gère les institutions et caisses ayant un caractère régional.\par
Le comité local applique, dans sa circonscription, les règlements, conventions et décisions de toute nature, en leur apportant les adaptations nécessaires.\par
Il gère les institutions et œuvres qui fonctionnent localement.\par
Il coordonne et contrôle l’activité des comités d’établissements.\par
Il assure et contrôle l’orientation sociale des établissements dans lesquels il n’a pas été constitué de comité social.\par

\labelblock{Liaison des comités sociaux avec les pouvoirs publics}

\noindent \labelchar{Art. 37.} Les pouvoirs publics sont représentés, dans chaque comité social national, par un\par
commissaire du Gouvernement désigné par arrêté du secrétaire d’État au Travail et après avis du secrétaire d’État dont relève la profession ou la famille professionnelle intéressée.\par
D’autre part, les membres des bureaux des comités sociaux sont accrédités, pour assurer les relations officielles nécessaires à l’activité de leur organisme, auprès des représentants des pouvoirs publics dans leur circonscription.
\subsection[Chapitre III. Associations professionnelles mixtes et corporations]{Chapitre III. \\
Associations professionnelles mixtes et corporations}
\noindent \labelchar{Art. 38.} Dans les professions qui ont déjà réalisé ou qui se proposent d’instituer des organisations professionnelles de caractère mixte, ces organisations seront maintenues ou créées sous réserve de l’agrément des pouvoirs publics. Leurs membres ne peuvent faire partie des syndicats professionnels ou groupement syndicaux.\par
Après la publication de la présente loi, ne pourront être créés que les organismes résultant de l’accord de la moitié des membres de haquec catégorie de la profession ou d’une décision des syndicats intéressés.\par
Les groupements mixtes sont assimilés aux comités ociaux et en tiennent lieu dans les entreprises où ils réunissent la moitié des effectifs.\par
Sur le plan local ou régional, ils tiennent lieu de comité social ou forment une annexe de ce comité social, suivant qu’ils rassemblent la moitié ou moins de la moitié des effectifs des différentes catégories des membres des professions.\par
Dans le cas où un groupement mixte tient lieu de comité social, une annexe de ce comité peut être formée par les syndicats ou unions dans les conditions générales fixées par la présente loi.\par
\bigbreak
\noindent \labelchar{Art. 39.} Les professions qui se proposent par accord de la moitié des membres de chaque catégorie ou par suite d’une décision des syndicats intéressés de réaliser une organisation habilitée à connaître à la fois des questions économiques et sociales pourront recevoir les pouvoirs et prérogatives nécessaires à leur fonctionnement corporatif.\par
Chacune de ces professions établira une charte corporative particulière qui sera soumise à l’agrément des pouvoirs publics.\par
Ces chartes devront prévoir, dans l’ordre social et professionnel, des dispositions au moins équivalentes à celles qui constituent les attributions prévues aux articles 31 à 33 pour les comités sociaux.\par
Il pourra être organisé dans les mêmes conditions des unions de corporations ou des organismes intercorporatifs.\par
\bigbreak
\noindent \labelchar{Art. 40.} Les décisions d’agrément des organismes prévues aux articles 38 et 39 feront l’objet de décrets contresignés par le vice-président du Conseil et les secrétaires d’État intéressés pris sur avis d’une commission ainsi composée :\par

\begin{itemize}[itemsep=0pt,]
\item un représentant du vice-président du Conseil ;
\item un représentant du ministre d’État chargé de la coordination des institutions nouvelles ;
\item un représentant du secrétaire d’État à l’Économie nationale et aux Finances ;
\item un représentant du secrétaire d’État à l’Intérieur ;
\item un représentant du secrétaire d’État au Travail ;
\item un représentant du ou des secrétaires d’État dont relèvent les activités intéressées.
\end{itemize}

\noindent Les conditions de fonctionnement de la commission seront fixées par arrêté du vice-président du Conseil.
\subsection[Chapitre IV. L’organisation interprofessionnelle]{Chapitre IV. \\
L’organisation interprofessionnelle}
\noindent \labelchar{Art. 41.} Les questions interprofessionnelles sont exclusivement traitées par les bureaux des comités sociaux de famille professionnelle existant à un même échelon, soit au cours des réunions occasionnelles de la totalité ou d’une partie de ces bureaux, soit d’une manière régulière par la réunion de ces bureaux constitués en comité social interprofessionnel.\par
Il est formé un comité social interprofessionnel dans chaque région, réunissant les bureaux des comités sociaux régionaux ; il siège au chef-lieu de la région, soit dans la maison commune de l’une des familles professionnelles, soit dans la maison des corporations.\par
Les comités sociaux interprofessionnels locaux seront créés progressivement par arrêtés du secrétaire d’État au Travail, pris sur proposition des comités interprofessionnels régionaux, après avis du ou des secrétaires d’État dont relève la famille ou la profession intéressée.\par
\bigbreak
\noindent \labelchar{Art. 42.} Le comité social interprofessionnel est dirigé par un bureau élu formé comme il est prévu à l’article 28. Il jouit de la personnalité civile.\par
\bigbreak
\noindent \labelchar{Art. 43.} Les comités sociaux interprofessionnels réalisent la liaison entre les comités de famille professionnelle et sont compétents dans la limite générale des attributions des comités sociaux, pour les questions communes aux différentes familles.\par
Ils peuvent être consultés par les pouvoirs publics sur les questions générales, professionnelles ou sociales et notamment la détermination du coût de la vie et les problèmes d’utilisation de la main d’œuvre.\par
Des attributions particulières pourront être confiées à certains comités sociaux interprofessionnels, par arrêtés du secrétaire d’État au Travail pris après avis des secrétaires d’État intéressés.
\subsection[Chapitre V. Dispositions communes aux organismes à caractères corporatifs]{Chapitre V. \\
Dispositions communes aux organismes à caractères corporatifs}
\noindent \labelchar{Art. 44.} Dans chaque famille professionnelle ou profession, les dépenses nécessitées par le fonctionnement administratif des différents organismes sont couvertes par une contribution professionnelle imposée aux membres de toutes catégories.\par
Les ressources ainsi obtenues sont réparties entre les comités sociaux de chaque échelon, à charge par ces comités de reverser aux organismes qui leur sont rattachés les fonds ou compléments de fonds nécessaires à leur fonctionnement.\par
La répartition d’ensemble des recettes et des dépenses corporatives, qui permet de fixer le montant des contributions et de partager les ressources entre les différents organismes, est assurée par le comité social national qui soumet son budget général annuel à l’approbation du secrétaire d’État au travail.\par
La perception des contributions est assurée sous la responsabilité de l’employeur, qui doit, en ce qui concerne la part des salariés, effectuer directement les retenues sur les salaires et traitements.\par
\bigbreak
\noindent \labelchar{Art. 45.} Les cotisations destinées à la participation aux dépenses de fonctionnement et aux institutions, œuvres et caisses diverses, sont indépendantes de la contribution professionnelle. Elles sont perçues par les organismes intéressés.\par
Pour la gestion de leurs différentes caisses, les comités sociaux se constituent en conseils d’administration fonctionnant conformément à des statuts spéciaux approuvés par le secrétaire d’État au travail.\par

\labelblock{Le patrimoine corporatif commun}

\noindent \labelchar{Art. 46.} Chaque famille professionnelle constitue un patrimoine corporatif commun exclusivement destiné à concourir à l’amélioration des conditions d’existence des membres de la profession.\par
Ce patrimoine, qui est la propriété de l’ensemble des membres de la profession, est géré par les comités sociaux des trois échelons local, régional et national, entre lesquels il est réparti par le comité national.\par
\bigbreak
\noindent \labelchar{Art. 47.} Le patrimoine corporatif est constitué initialement par les apports résultant des dévolutions de biens prévues aux articles 72 à 75.\par
Il est ensuite normalement alimenté par un prélèvement sur les bénéfices des entreprises de la profession et par des dons et legs.\par
La définition des bénéfices, la fixation du prélèvement, et les modalités de son recouvrement, qui sera effectué comme en matière d’impôts sur les bénéfices industriels et commerciaux, seront déterminées par décrets.\par
\bigbreak
\noindent \labelchar{Art. 48.} La gestion du patrimoine commun est assurée dans les conditions fixées par un règlement particulier qu’établit le comité social national. Le règlement est approuvé par le secrétaire d’État à l’Économie nationale et aux Finances, le secrétaire d’État au Travail et le ou les secrétaires d’État dont relève la famille ou la profession intéressée.\par
Ce règlement fixe notamment les limites, inférieure et supérieure, entre lesquelles le montant du patrimoine doit être maintenu.\par
Le patrimoine ne peut, en aucun cas, être utilisé pour ouvrir des dépenses de fonctionnement administratif.\par
Il ne peut, d’autre part, servir à couvrir en totalité les charges des institutions sociales ou autres dont les ressources doivent toujours comporter, au moins pour une partie, le produit des cotisations des adhérents.\par

\labelblock{Le contrôle financier}

\noindent \labelchar{Art. 49.} Sans préjudice des mesures de contrôle réglementaires effectuées par les différents services ministériels, les organismes corporatifs assurent eux-mêmes le contrôle des comptabilités des organismes professionnels.\par
Ils disposent, à cet effet, d’un service commun composé de commissaires comptables assermentés, dont la mise sur pied et les conditions de fonctionnement seront fixées par décret.\par

\labelblock{La maison commune}

\noindent \labelchar{Art. 50.} Afin de faciliter le fonctionnement des comités sociaux et d’affirmer la solidarité corporative, il est créé une maison commune par famille professionnelle.\par
La maison commune est, dans chaque circonscription, le siège du comité social.\par
\bigbreak
\noindent \labelchar{Art. 51.} Le comité social est, suivant le cas, locataire ou propriétaire de la maison commune. La propriété de la maison peut résulter, soit d’une acquisition, soit d’un don ou legs, soit d’une dévolution par les pouvoirs publics.\par
L’acquisition d’une maison commune par un comité social, que ce soit à titre onéreux, par don ou legs ou par dévolution, n’entraîne ni droit de mutation ni frais d’aucune sorte.\par
\bigbreak
\noindent \labelchar{Art. 52.} La maison commune est ouverte à tous les membres des professions rattachées. Elle ne peut être utilisée qu’aux seuls fins corporatives et il est interdit d’y exercer toute activité politique ou commerciale.\par
Sa gestion est assurée par une commission tripartite particulière, composée de membres pris parmi les plus anciens dans le comité social ou les comités sociaux intéressés.\par
\bigbreak
\noindent \labelchar{Art. 53.} Différentes familles professionnelles peuvent utiliser, pour installer leur maison commune, des locaux situés dans un même immeuble. Les comités sociaux interprofessionnels peuvent utiliser une maison commune particulière, qui devient la maison des corporations.
\subsection[Chapitre VI. Les attributions corporatives générales]{Chapitre VI. \\
Les attributions corporatives générales}

\labelblock{Les salaires}

\noindent \labelchar{Art. 54.} Tous les membres des professions n’appartenant pas à la catégorie des employeurs reçoivent, en contrepartie du travail qu’ils fournissent, une rémunération différente suivant le lieu de leur emploi, leur qualification professionnelle et les conditions spéciales dans lesquelles ils exercent leur activité.\par
Le salaire est, en conséquence, déterminé d’après les principes généraux ci-après :\par
1° un salaire minimum vital est perçu par tous les salariés exerçant leur activité normale. Il correspond à la rémunération de celui qui n’a ni charges de famille ni qualification professionnelle ; il varie suivant les lieux d’emploi et le coût local de la vie ;\par
2° la rémunération professionnelle est un complément au salaire minimum vital. Elle correspond à la qualification professionnelle du bénéficiaire et est différente suivant les professions et le lieu d’emploi ;\par
3° des suppléments peuvent s’ajouter éventuellement au salaire tel qu’il est obtenu par l’addition des deux éléments ci-dessus pour tenir compte des aptitudes personnelles de l’intéressé, de son rendement, notamment quand il s’agit de travail exécuté « aux pièces », et des conditions particulières dans lesquelles le travail est effectué ;\par
4° au salaire ainsi défini s’ajoutent les allocations ou suppléments de salaires pour charges familiales résultant, soit de la législation générale sur la famille, soit des dispositions particulières prises par la profession.\par
Le supplément familial de salaire accordé par les professions peut se traduire par des avantages en nature.\par
\bigbreak
\noindent \labelchar{Art. 55.} Le salaire minimum vital, fixé par le Gouvernement, est arrêté par région, département ou localité, sur proposition du comité supérieur des salaires fonctionnant au secrétariat d’État au travail.\par
Les conditions d’institution et de fonctionnement de ce comité seront fixées par décret.\par
\bigbreak
\noindent \labelchar{Art. 56.} Le supplément de salaire correspondant à la rémunération professionnelle est fixé sous la forme d’un coefficient applicable au salaire minimum vital.\par
Le barème de base des coefficients applicable aux différentes qualifications professionnelles est arrêté, pour chaque profession, par le comité social national de la profession.\par
Le barème peut être adapté par les comités sociaux des différents échelons, sous le contrôle du comité social national.\par
\bigbreak
\noindent \labelchar{Art. 57.} Des accords pourront intervenir entre les secrétariats d’État intéressés et les professions organisées en vue de la délégation à ces dernières d’attributions d’ordre social telles qu’assurances, retraites, allocations de chômage, etc. ressortissant actuellement aux pouvoirs publics.\par
\bigbreak
\noindent \labelchar{Art. 58.} Les familles professionnelles peuvent réaliser entre elles des ententes et constituer des organismes de compensation pour assurer l’équilibre des charges qu’elles seront appelées à supporter pour l’application des mesures qui précèdent. Ces ententes seront soumises à l’agrément des pouvoirs publics.\par
L’État participera éventuellement aux charges ci-dessus visées en vue d’aider au fonctionnement initial des novelles institutions ou à l’occasion d’événements exceptionnels.\par

\labelblock{La formation professionnelle}

\noindent \labelchar{Art. 59.} Les questions de formation profession professionnelle : apprentissage, perfectionnement, reclassement et promotion ouvrière sont essentiellement d’ordre corporatif.\par
Une loi fixera le rôle respectif des organismes professionnels et des pouvoirs publics dans cette matière, ainsi que les conditions dans lesquelles sera assurée la coordination entre ces organismes et les secrétariats d’État compétents.
\section[Titre V. La juridiction du travail]{Titre V. \\
La juridiction du travail}\renewcommand{\leftmark}{Titre V. \\
La juridiction du travail}


\labelblock{Principes généraux}

\noindent \labelchar{Art. 60.} Tous les organismes professionnels, aux différents échelons, doivent s’efforcer de prévenir et de concilier les différends qui peuvent surgir à l’occasion de l’application de la législation et de la réglementation sociale des professions.\par
\bigbreak
\noindent \labelchar{Art. 61.} Dans le cas où, malgré l’intervention des organismes professionnels, les différends n’ont pu être évités, ni conciliés, ils sont :\par

\begin{itemize}[itemsep=0pt,]
\item portés devant les conseils de prud’hommes ou, à leur défaut, devant les justices de paix, s’il s’agit de différends individuels ;
\item soumis à l’arbitrage ou portés devant les tribunaux du travail, s’il s’agit de différends collectifs.
\end{itemize}

\noindent Les tribunaux de travail peuvent, en outre, être saisis des infractions à la réglementation qui sera établie en application de la présente loi.\par

\labelblock{L’arbitrage}

\noindent \labelchar{Art. 62.} Lorsque les différends du travail sont soumis à l’arbitrage, le comité social régional saisi du différend désigne, dans un délai de quarante-huit heures à partir du moment où il a été saisi, trois arbitres choisis sur une liste établie annuellement par le comité social national de chaque branche d’activité. Si le comité social régional n’a pas désigné les arbitres, le tribunal du travail, saisi à la requête, soit du commissaire du Gouvernement, soit de la partie la plus diligente, procède lui-même à la désignation. En cas de conflit sur le plan national, les arbitres doivent être désignés dans les mêmes conditions par le comité social national.\par

\labelblock{Les tribunaux du travail}

\noindent \labelchar{Art. 63.} Il est institué, dans le ressort de chaque cour d’appel, un tribunal régional du travail, composé :\par

\begin{itemize}[itemsep=0pt,]
\item de deux magistrats, dont l’un exerce les fonctions de président, désignés par ordonnance du premier président ;
\item et de trois membres du comité social régional compétent, désignés comme il est prévu à l’article 28.
\end{itemize}

\noindent Les recours contre les décisions des tribunaux régionaux du travail sont portés devant le tribunal national du travail, qui statue en dernier ressort.\par
Le tribunal national du travail est composé de trois magistrats, dont l’un exerce les fonctions de président, désignés par le garde des ceaux, S ministre secrétaire d’État à la Justice, et de quatre membres du comité social national compétent désignés par les secrétaires d’État au Travail et à la production industrielle.\par
Des fonctionnaires du corps de l’inspection du travail, désignés par le secrétaire d’État au Travail, exerceront les fonctions de commissaire du Gouvernement auprès du tribunal national et des tribunaux régionaux.\par
\bigbreak
\noindent \labelchar{Art. 64.} Un règlement d’administration publique, établi par le secrétaire d’État au travail et par le garde des Sceaux, ministre secrétaire d’État à la justice, déterminera les conditions d’application des diverses dispositions du présent titre.
\section[Titre VI]{Titre VI}\renewcommand{\leftmark}{Titre VI}

\subsection[Chapitre Ier. Dispositions communes]{Chapitre I\textsuperscript{er}. \\
Dispositions communes}
\noindent \labelchar{Art. 65.} Dans l’intérêt de la profession, les membres des organismes professionnels institués par la présente loi, appartenant à une catégorie de salariés, bénéficient de toutes les facilités nécessaires à l’exercice de leur mandat.\par
Des garanties de stabilité d’emploi sont prévues en leur faveur dans les règlements et statuts particuliers des professions.\par
\bigbreak
\noindent \labelchar{Art. 66.} Lorsqu’un des organismes professionnels prévus par la présente loi s’avère incapable de remplir la mission qui lui est impartie, ou refuse, soit de prendre une décision, soit d’appliquer un règlement, compromettant ainsi l’intérêt de ses ressortissants\par
ou celui de l’État, il est procédé, par arrêté du secrétaire d’État au Travail, sur avis des secrétaires d’État compétents, à la suspension de l’organisme intéressé et à la désignation d’une délégation provisoire de gestion qui recueille tous ses pouvoirs.\par
\bigbreak
\noindent \labelchar{Art. 67.} Les groupements professionnels formés en violation des dispositions qui précèdent, et ceux dont l’activité serait contraireà l’intérêt national ou étrangère à l’objet qui leur est assigné, seront dissous par décret.\par
La dévolution des biens de ces groupements sera réglée conformément aux dispositions des articles 72 à 75. Les dirigeants e t les membres des groupements dissous seront passibles d’une amende de 500 à 10 000 F et d’un emprisonnement de six mois à cinq ans ou de l’une de ces deux peines seulement.\par
\bigbreak
\noindent \labelchar{Art. 68.} Les infractions aux règlements et décisions qui sont relevées par les organismes corporatifs ou leurs représentants assermentés donnent lieu, soit à des sanctions corporatives, soit à des poursuites devant le tribunal du travail.\par
Les sanctions corporatives comportent :\par

\begin{itemize}[itemsep=0pt,]
\item les amendes au profit du patrimoine corporatif ;
\item l’exclusion des organismes professionnels ;
\item l’exclusion temporaire de la profession.
\end{itemize}

\noindent Elles sont prononcées par le bureau de l’organisme compétent, dans les limites fixées par les barèmes établis par les comités nationaux.\par
Les poursuites devant les tribunaux du travail sont intentées à la demande des organismes professionnels compétents.
\subsection[Chapitre II. Dispositions transitoires]{Chapitre II. \\
Dispositions transitoires}
\noindent \labelchar{Art. 69.} L’application de la présente loi sera entreprise dès l’achèvement des travaux des commissions prévus à l’article 77 et sera poursuivie progressivement, au fur et à mesure de la publication des textes législatifs et réglementaires complémentaires.\par
Dans le cadre général des lois, décrets et règlements relatifs à l’organisation professionnelle, les familles professionnelles, professions ou groupes de professions, établiront les règlements particuliers qui définiront leur propre organisation.\par
\bigbreak
\noindent \labelchar{Art. 70.} Les premières désignations des membres des conseils d’administration des organismes professionnels seront faites par arrêtés du ministre d’État chargé de la coordination des institutions nouvelles, du secrétaire d’État au Travail et du ou des secrétaires d’État dont relèvent les professions considérées, compte tenu des propositions des commissions prévues à l’article 77 ci-après.\par
\bigbreak
\noindent \labelchar{Art. 71.} Pendant un délai de deux ans à partir de la publication de la présente loi, les biens affectés à l’usage exclusif d’institutions sociales, visés au dernier alinéa de l’article 35, et qui n’auront pas fait l’objet d’un e dévolution dans les conditions fixées au présent chapitre, ne pourront être changés d’affectation, sauf dérogation accordée par arrêté du secrétaire d’État au Travail pris sur avis du ou des secrétaires d’États compétents.\par
\bigbreak
\noindent \labelchar{Art. 72.} La constitution des syndicats, comités et groupements prévus dans la nouvelle organisation professionnelle entraînera la dissolution des anciens syndicats et groupements syndicaux et professionnels de toute nature.\par
Les dévolutions de biens consécutives à ces dissolutions seront prononcées au profit des nouveaux organismes syndicaux et des comités locaux, en fonction de leurs attributions respectives, en conservant dans toute la mesure du possible ces biens aux mêmes professions, dans les mêmes entreprises, localités ou régions.\par
Les syndicats et les groupements de syndicats existant à la date de la publication de la présente loi continueront leur activité jusqu’à ce qu’il soit statué par décret sur leur dissolution ou leur intégration dans la nouvelle organisation professionnelle. Toutefois, pendant cette période, leur capacité civile sera limitée aux actes de simple administration.\par
\bigbreak
\noindent \labelchar{Art. 73.} Il sera procédé par les soins de l’administration de l’enregistrement, des domaines et du timbre à un inventaire des biens des syndicats et groupements de syndicats visés à l’article précédent, à la date de publication de la présente loi.\par
A cet effet, dans la huitaine qui suivra cette date, le préfet notifiera à ladite administration la liste de ces organismes ayant leur siège dans le département.\par
\bigbreak
\noindent \labelchar{Art. 74.} Les dévolutions de biens prévues au présent chapitre seront prononcées par décrets contresignés par le secrétaire d’État au Travail et le ou les autres secrétaires d’État intéressés, pris sur proposition d’un comité central institué à la vice-présidence du Conseil.\par
Ce comité aura qualité pour proposer, le cas échéant, la liquidation des biens qui ne peuvent être attribués directement.\par
Sa compétence s’étendra aux biens des syndicats ou groupements syndicaux communistes dissous par le décret du 26 septembre 1939 et qui n’auraient pas encore fait l’objet d’une attribution définitive.\par
Il recueillera les avis des comités sociaux nationaux des familles professionnelles et professions intéressées.\par
\bigbreak
\noindent \labelchar{Art. 75.} Toutes les opérations prévues par les deux articles précédents auront lieu sans droit de mutation et sans frais d’aucune sorte.\par
Un règlement d’administration publique déterminera les conditions d’application des quatre articles précédents.\par
\bigbreak
\noindent \labelchar{Art. 76.} Les lois spéciales ayant pour objet, aux termes de l’article 7, d’établir un régime particulier pour les agents des services publics industriels autres que ceux visés par la loi du 14 septembre 1941 devront intervenir avant le 1\textsuperscript{er} mars 1942.\par
Jusqu’à cette date les dispositions des articles 6 9 à 75 ne seront pas applicables en ce qui concerne lesdits agents.\par
\bigbreak
\noindent \labelchar{Art. 77.} Il sera institué, pour chaque famille professionnelle, une commission provisoire d’organisation chargée d’étudier et de proposer :\par

\begin{itemize}[itemsep=0pt,]
\item les limites des circonscriptions à attribuer dans c haque cas aux organismes syndicaux et corporatifs, locaux ou régionaux ;
\item les conditions de regroupement, au sein des nouveaux organismes, des éléments appartenant aux syndicats, unions, fédérations appelés à fusionner en application de la présente loi ;
\item la composition nominative des conseils d’administration des organismes corporatifs à mettre sur pied.
\end{itemize}

\noindent Des arrêtés du ministre d’État chargé de la coordination des institutions nouvelles et du secrétaire d’État intéressés, fixeront la composition des commissions provisoires d’organisation et les conditions de leur fonctionnement.\par
\bigbreak
\noindent \labelchar{Art. 78.} Une liaison sera établie entre les comités provisoires d’organisation, créés en application de la loi du 16 août 1940, et les comités sociaux institués par la présente loi, afin de réaliser l’harmonie et l’adaptation réciproque des mesures sociales et économiques.\par
Cette liaison sera assurée, d’une part par des délégués des comités d’organisation économique qui siégeront dans les comités sociaux régionaux et nationaux, avec voix consultative, d’autre part par un représentant des comités sociaux nationaux siégeant dans les comités d’organisation intéressés.\par
\bigbreak
\noindent \labelchar{Art. 79.} Les conditions dans lesquelles la présente loi ou certaines de ses dispositions pourront éventuellement être rendues applicables à l’Algérie, aux colonies ou aux territoires placés sous mandat français seront fixées par décrets.\par
\bigbreak
\noindent \labelchar{Art. 80.} Sont abrogées toutes dispositions contraires au présent décret, qui sera publié au \emph{Journal officiel} et exécuté comme loi de l’État.
\section[Signatures]{Signatures}\renewcommand{\leftmark}{Signatures}

\noindent \emph{Fait à Vichy, le 4 octobre 1941}\par
Par le Maréchal de France,\par
Chef de l’État français :\par
PHILIPPE PETAIN\par
L’amiral de la Flotte,\par
Vice-président du Conseil, ministre de la Défense nationale,\par
Ministre secrétaire d’État aux Affaires étrangères et à la Marine :\par
A. DARLAN\par
Le ministre d’État : Henry MOYSSET\par
Le ministre d’État : Lucien ROMIER\par
Le général d’armée, ministre secrétaire d’État à la Guerre : Général HUNTZINGER\par
Le ministre secrétaire d’État à l’Intérieur : Pierre PUCHEU\par
Le garde des sceaux, ministre secrétaire d’État à la Justice : Joseph BERTHELEMY\par
Le ministre secrétaire d’État à l’Agriculture : Pierre CAZIOT\par
Le ministre secrétaire d’État à l’Économie nationale aux aux Finances : Yves BOUTHILLIER\par
Le secrétaire d’État à la Production industrielle : François LEHIDEUX\par
Le secrétaire d’État au Travail : René BELIN\par
Le secrétaire d’État à l’Éducation nationale et à la Jeunesse : Jérôme CARCOPINO\par
Le secrétaire d’État à l’Aviation : Général BERGERET\par
Le secrétaire d’État au Ravitaillement : Paul CHARBIN\par
Le secrétaire d’État à la Famille et à la Santé : Serge HUARD\par
Le secrétaire d’État aux Colonies : amiral PLATON\par
Le secrétaire d’État aux Communications : Jean BERTHELOT\par
Le secrétaire d’État à la vice-présidence du Conseil : BENOIST-MECHIN
 


% at least one empty page at end (for booklet couv)
\ifbooklet
  \newpage\null\thispagestyle{empty}\newpage
\fi

\ifdev % autotext in dev mode
\fontname\font — \textsc{Les règles du jeu}\par
(\hyperref[utopie]{\underline{Lien}})\par
\noindent \initialiv{A}{lors là}\blindtext\par
\noindent \initialiv{À}{ la bonheur des dames}\blindtext\par
\noindent \initialiv{É}{tonnez-le}\blindtext\par
\noindent \initialiv{Q}{ualitativement}\blindtext\par
\noindent \initialiv{V}{aloriser}\blindtext\par
\Blindtext
\phantomsection
\label{utopie}
\Blinddocument
\fi
\end{document}
