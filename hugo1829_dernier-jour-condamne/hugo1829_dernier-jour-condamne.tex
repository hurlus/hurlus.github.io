%%%%%%%%%%%%%%%%%%%%%%%%%%%%%%%%%
% LaTeX model https://hurlus.fr %
%%%%%%%%%%%%%%%%%%%%%%%%%%%%%%%%%

% Needed before document class
\RequirePackage{pdftexcmds} % needed for tests expressions
\RequirePackage{fix-cm} % correct units

% Define mode
\def\mode{a4}

\newif\ifaiv % a4
\newif\ifav % a5
\newif\ifbooklet % booklet
\newif\ifcover % cover for booklet

\ifnum \strcmp{\mode}{cover}=0
  \covertrue
\else\ifnum \strcmp{\mode}{booklet}=0
  \booklettrue
\else\ifnum \strcmp{\mode}{a5}=0
  \avtrue
\else
  \aivtrue
\fi\fi\fi

\ifbooklet % do not enclose with {}
  \documentclass[french,twoside]{book} % ,notitlepage
  \usepackage[%
    papersize={105mm, 297mm},
    inner=12mm,
    outer=12mm,
    top=20mm,
    bottom=15mm,
    marginparsep=0pt,
  ]{geometry}
  \usepackage[fontsize=9.5pt]{scrextend} % for Roboto
\else\ifav
  \documentclass[french,twoside]{book} % ,notitlepage
  \usepackage[%
    a5paper,
    inner=25mm,
    outer=15mm,
    top=15mm,
    bottom=15mm,
    marginparsep=0pt,
  ]{geometry}
  \usepackage[fontsize=12pt]{scrextend}
\else% A4 2 cols
  \documentclass[twocolumn]{report}
  \usepackage[%
    a4paper,
    inner=15mm,
    outer=10mm,
    top=25mm,
    bottom=18mm,
    marginparsep=0pt,
  ]{geometry}
  \setlength{\columnsep}{20mm}
  \usepackage[fontsize=9.5pt]{scrextend}
\fi\fi

%%%%%%%%%%%%%%
% Alignments %
%%%%%%%%%%%%%%
% before teinte macros

\setlength{\arrayrulewidth}{0.2pt}
\setlength{\columnseprule}{\arrayrulewidth} % twocol
\setlength{\parskip}{0pt} % classical para with no margin
\setlength{\parindent}{1.5em}

%%%%%%%%%%
% Colors %
%%%%%%%%%%
% before Teinte macros

\usepackage[dvipsnames]{xcolor}
\definecolor{rubric}{HTML}{800000} % the tonic 0c71c3
\def\columnseprulecolor{\color{rubric}}
\colorlet{borderline}{rubric!30!} % definecolor need exact code
\definecolor{shadecolor}{gray}{0.95}
\definecolor{bghi}{gray}{0.5}

%%%%%%%%%%%%%%%%%
% Teinte macros %
%%%%%%%%%%%%%%%%%
%%%%%%%%%%%%%%%%%%%%%%%%%%%%%%%%%%%%%%%%%%%%%%%%%%%
% <TEI> generic (LaTeX names generated by Teinte) %
%%%%%%%%%%%%%%%%%%%%%%%%%%%%%%%%%%%%%%%%%%%%%%%%%%%
% This template is inserted in a specific design
% It is XeLaTeX and otf fonts

\makeatletter % <@@@


\usepackage{blindtext} % generate text for testing
\usepackage[strict]{changepage} % for modulo 4
\usepackage{contour} % rounding words
\usepackage[nodayofweek]{datetime}
% \usepackage{DejaVuSans} % seems buggy for sffont font for symbols
\usepackage{enumitem} % <list>
\usepackage{etoolbox} % patch commands
\usepackage{fancyvrb}
\usepackage{fancyhdr}
\usepackage{float}
\usepackage{fontspec} % XeLaTeX mandatory for fonts
\usepackage{footnote} % used to capture notes in minipage (ex: quote)
\usepackage{framed} % bordering correct with footnote hack
\usepackage{graphicx}
\usepackage{lettrine} % drop caps
\usepackage{lipsum} % generate text for testing
\usepackage[framemethod=tikz,]{mdframed} % maybe used for frame with footnotes inside
\usepackage{pdftexcmds} % needed for tests expressions
\usepackage{polyglossia} % non-break space french punct, bug Warning: "Failed to patch part"
\usepackage[%
  indentfirst=false,
  vskip=1em,
  noorphanfirst=true,
  noorphanafter=true,
  leftmargin=\parindent,
  rightmargin=0pt,
]{quoting}
\usepackage{ragged2e}
\usepackage{setspace} % \setstretch for <quote>
\usepackage{tabularx} % <table>
\usepackage[explicit]{titlesec} % wear titles, !NO implicit
\usepackage{tikz} % ornaments
\usepackage{tocloft} % styling tocs
\usepackage[fit]{truncate} % used im runing titles
\usepackage{unicode-math}
\usepackage[normalem]{ulem} % breakable \uline, normalem is absolutely necessary to keep \emph
\usepackage{verse} % <l>
\usepackage{xcolor} % named colors
\usepackage{xparse} % @ifundefined
\XeTeXdefaultencoding "iso-8859-1" % bad encoding of xstring
\usepackage{xstring} % string tests
\XeTeXdefaultencoding "utf-8"
\PassOptionsToPackage{hyphens}{url} % before hyperref, which load url package

% TOTEST
% \usepackage{hypcap} % links in caption ?
% \usepackage{marginnote}
% TESTED
% \usepackage{background} % doesn’t work with xetek
% \usepackage{bookmark} % prefers the hyperref hack \phantomsection
% \usepackage[color, leftbars]{changebar} % 2 cols doc, impossible to keep bar left
% \usepackage[utf8x]{inputenc} % inputenc package ignored with utf8 based engines
% \usepackage[sfdefault,medium]{inter} % no small caps
% \usepackage{firamath} % choose firasans instead, firamath unavailable in Ubuntu 21-04
% \usepackage{flushend} % bad for last notes, supposed flush end of columns
% \usepackage[stable]{footmisc} % BAD for complex notes https://texfaq.org/FAQ-ftnsect
% \usepackage{helvet} % not for XeLaTeX
% \usepackage{multicol} % not compatible with too much packages (longtable, framed, memoir…)
% \usepackage[default,oldstyle,scale=0.95]{opensans} % no small caps
% \usepackage{sectsty} % \chapterfont OBSOLETE
% \usepackage{soul} % \ul for underline, OBSOLETE with XeTeX
% \usepackage[breakable]{tcolorbox} % text styling gone, footnote hack not kept with breakable


% Metadata inserted by a program, from the TEI source, for title page and runing heads
\title{\textbf{ Le dernier jour d’un condamné }}
\date{1829}
\author{Victor Hugo}
\def\elbibl{Victor Hugo. 1829. \emph{Le dernier jour d’un condamné}}
\def\elsource{ \href{http://gallica.bnf.fr/ark:/12148/bpt6k37491h}{\dotuline{http://gallica.bnf.fr/ark:/12148/bpt6k37491h}}\footnote{\href{http://gallica.bnf.fr/ark:/12148/bpt6k37491h}{\url{http://gallica.bnf.fr/ark:/12148/bpt6k37491h}}}  \href{http://efele.net/ebooks/livres/000205}{\dotuline{http://efele.net/ebooks/livres/000205}}\footnote{\href{http://efele.net/ebooks/livres/000205}{\url{http://efele.net/ebooks/livres/000205}}} }

% Default metas
\newcommand{\colorprovide}[2]{\@ifundefinedcolor{#1}{\colorlet{#1}{#2}}{}}
\colorprovide{rubric}{red}
\colorprovide{silver}{lightgray}
\@ifundefined{syms}{\newfontfamily\syms{DejaVu Sans}}{}
\newif\ifdev
\@ifundefined{elbibl}{% No meta defined, maybe dev mode
  \newcommand{\elbibl}{Titre court ?}
  \newcommand{\elbook}{Titre du livre source ?}
  \newcommand{\elabstract}{Résumé\par}
  \newcommand{\elurl}{http://oeuvres.github.io/elbook/2}
  \author{Éric Lœchien}
  \title{Un titre de test assez long pour vérifier le comportement d’une maquette}
  \date{1566}
  \devtrue
}{}
\let\eltitle\@title
\let\elauthor\@author
\let\eldate\@date


\defaultfontfeatures{
  % Mapping=tex-text, % no effect seen
  Scale=MatchLowercase,
  Ligatures={TeX,Common},
}


% generic typo commands
\newcommand{\astermono}{\medskip\centerline{\color{rubric}\large\selectfont{\syms ✻}}\medskip\par}%
\newcommand{\astertri}{\medskip\par\centerline{\color{rubric}\large\selectfont{\syms ✻\,✻\,✻}}\medskip\par}%
\newcommand{\asterism}{\bigskip\par\noindent\parbox{\linewidth}{\centering\color{rubric}\large{\syms ✻}\\{\syms ✻}\hskip 0.75em{\syms ✻}}\bigskip\par}%

% lists
\newlength{\listmod}
\setlength{\listmod}{\parindent}
\setlist{
  itemindent=!,
  listparindent=\listmod,
  labelsep=0.2\listmod,
  parsep=0pt,
  % topsep=0.2em, % default topsep is best
}
\setlist[itemize]{
  label=—,
  leftmargin=0pt,
  labelindent=1.2em,
  labelwidth=0pt,
}
\setlist[enumerate]{
  label={\bf\color{rubric}\arabic*.},
  labelindent=0.8\listmod,
  leftmargin=\listmod,
  labelwidth=0pt,
}
\newlist{listalpha}{enumerate}{1}
\setlist[listalpha]{
  label={\bf\color{rubric}\alph*.},
  leftmargin=0pt,
  labelindent=0.8\listmod,
  labelwidth=0pt,
}
\newcommand{\listhead}[1]{\hspace{-1\listmod}\emph{#1}}

\renewcommand{\hrulefill}{%
  \leavevmode\leaders\hrule height 0.2pt\hfill\kern\z@}

% General typo
\DeclareTextFontCommand{\textlarge}{\large}
\DeclareTextFontCommand{\textsmall}{\small}

% commands, inlines
\newcommand{\anchor}[1]{\Hy@raisedlink{\hypertarget{#1}{}}} % link to top of an anchor (not baseline)
\newcommand\abbr[1]{#1}
\newcommand{\autour}[1]{\tikz[baseline=(X.base)]\node [draw=rubric,thin,rectangle,inner sep=1.5pt, rounded corners=3pt] (X) {\color{rubric}#1};}
\newcommand\corr[1]{#1}
\newcommand{\ed}[1]{ {\color{silver}\sffamily\footnotesize (#1)} } % <milestone ed="1688"/>
\newcommand\expan[1]{#1}
\newcommand\foreign[1]{\emph{#1}}
\newcommand\gap[1]{#1}
\renewcommand{\LettrineFontHook}{\color{rubric}}
\newcommand{\initial}[2]{\lettrine[lines=2, loversize=0.3, lhang=0.3]{#1}{#2}}
\newcommand{\initialiv}[2]{%
  \let\oldLFH\LettrineFontHook
  % \renewcommand{\LettrineFontHook}{\color{rubric}\ttfamily}
  \IfSubStr{QJ’}{#1}{
    \lettrine[lines=4, lhang=0.2, loversize=-0.1, lraise=0.2]{\smash{#1}}{#2}
  }{\IfSubStr{É}{#1}{
    \lettrine[lines=4, lhang=0.2, loversize=-0, lraise=0]{\smash{#1}}{#2}
  }{\IfSubStr{ÀÂ}{#1}{
    \lettrine[lines=4, lhang=0.2, loversize=-0, lraise=0, slope=0.6em]{\smash{#1}}{#2}
  }{\IfSubStr{A}{#1}{
    \lettrine[lines=4, lhang=0.2, loversize=0.2, slope=0.6em]{\smash{#1}}{#2}
  }{\IfSubStr{V}{#1}{
    \lettrine[lines=4, lhang=0.2, loversize=0.2, slope=-0.5em]{\smash{#1}}{#2}
  }{
    \lettrine[lines=4, lhang=0.2, loversize=0.2]{\smash{#1}}{#2}
  }}}}}
  \let\LettrineFontHook\oldLFH
}
\newcommand{\labelchar}[1]{\textbf{\color{rubric} #1}}
\newcommand{\milestone}[1]{\autour{\footnotesize\color{rubric} #1}} % <milestone n="4"/>
\newcommand\name[1]{#1}
\newcommand\orig[1]{#1}
\newcommand\orgName[1]{#1}
\newcommand\persName[1]{#1}
\newcommand\placeName[1]{#1}
\newcommand{\pn}[1]{\IfSubStr{-—–¶}{#1}% <p n="3"/>
  {\noindent{\bfseries\color{rubric}   ¶  }}
  {{\footnotesize\autour{ #1}  }}}
\newcommand\reg{}
% \newcommand\ref{} % already defined
\newcommand\sic[1]{#1}
\newcommand\surname[1]{\textsc{#1}}
\newcommand\term[1]{\textbf{#1}}

\def\mednobreak{\ifdim\lastskip<\medskipamount
  \removelastskip\nopagebreak\medskip\fi}
\def\bignobreak{\ifdim\lastskip<\bigskipamount
  \removelastskip\nopagebreak\bigskip\fi}

% commands, blocks
\newcommand{\byline}[1]{\bigskip{\RaggedLeft{#1}\par}\bigskip}
\newcommand{\bibl}[1]{{\RaggedLeft{#1}\par\bigskip}}
\newcommand{\biblitem}[1]{{\noindent\hangindent=\parindent   #1\par}}
\newcommand{\dateline}[1]{\medskip{\RaggedLeft{#1}\par}\bigskip}
\newcommand{\labelblock}[1]{\medbreak{\noindent\color{rubric}\bfseries #1}\par\mednobreak}
\newcommand{\salute}[1]{\bigbreak{#1}\par\medbreak}
\newcommand{\signed}[1]{\bigbreak\filbreak{\raggedleft #1\par}\medskip}

% environments for blocks (some may become commands)
\newenvironment{borderbox}{}{} % framing content
\newenvironment{citbibl}{\ifvmode\hfill\fi}{\ifvmode\par\fi }
\newenvironment{docAuthor}{\ifvmode\vskip4pt\fontsize{16pt}{18pt}\selectfont\fi\itshape}{\ifvmode\par\fi }
\newenvironment{docDate}{}{\ifvmode\par\fi }
\newenvironment{docImprint}{\vskip6pt}{\ifvmode\par\fi }
\newenvironment{docTitle}{\vskip6pt\bfseries\fontsize{18pt}{22pt}\selectfont}{\par }
\newenvironment{msHead}{\vskip6pt}{\par}
\newenvironment{msItem}{\vskip6pt}{\par}
\newenvironment{titlePart}{}{\par }


% environments for block containers
\newenvironment{argument}{\itshape\parindent0pt}{\vskip1.5em}
\newenvironment{biblfree}{}{\ifvmode\par\fi }
\newenvironment{bibitemlist}[1]{%
  \list{\@biblabel{\@arabic\c@enumiv}}%
  {%
    \settowidth\labelwidth{\@biblabel{#1}}%
    \leftmargin\labelwidth
    \advance\leftmargin\labelsep
    \@openbib@code
    \usecounter{enumiv}%
    \let\p@enumiv\@empty
    \renewcommand\theenumiv{\@arabic\c@enumiv}%
  }
  \sloppy
  \clubpenalty4000
  \@clubpenalty \clubpenalty
  \widowpenalty4000%
  \sfcode`\.\@m
}%
{\def\@noitemerr
  {\@latex@warning{Empty `bibitemlist' environment}}%
\endlist}
\newenvironment{quoteblock}% may be used for ornaments
  {\begin{quoting}}
  {\end{quoting}}

% table () is preceded and finished by custom command
\newcommand{\tableopen}[1]{%
  \ifnum\strcmp{#1}{wide}=0{%
    \begin{center}
  }
  \else\ifnum\strcmp{#1}{long}=0{%
    \begin{center}
  }
  \else{%
    \begin{center}
  }
  \fi\fi
}
\newcommand{\tableclose}[1]{%
  \ifnum\strcmp{#1}{wide}=0{%
    \end{center}
  }
  \else\ifnum\strcmp{#1}{long}=0{%
    \end{center}
  }
  \else{%
    \end{center}
  }
  \fi\fi
}


% text structure
\newcommand\chapteropen{} % before chapter title
\newcommand\chaptercont{} % after title, argument, epigraph…
\newcommand\chapterclose{} % maybe useful for multicol settings
\setcounter{secnumdepth}{-2} % no counters for hierarchy titles
\setcounter{tocdepth}{5} % deep toc
\markright{\@title} % ???
\markboth{\@title}{\@author} % ???
\renewcommand\tableofcontents{\@starttoc{toc}}
% toclof format
% \renewcommand{\@tocrmarg}{0.1em} % Useless command?
% \renewcommand{\@pnumwidth}{0.5em} % {1.75em}
\renewcommand{\@cftmaketoctitle}{}
\setlength{\cftbeforesecskip}{\z@ \@plus.2\p@}
\renewcommand{\cftchapfont}{}
\renewcommand{\cftchapdotsep}{\cftdotsep}
\renewcommand{\cftchapleader}{\normalfont\cftdotfill{\cftchapdotsep}}
\renewcommand{\cftchappagefont}{\bfseries}
\setlength{\cftbeforechapskip}{0em \@plus\p@}
% \renewcommand{\cftsecfont}{\small\relax}
\renewcommand{\cftsecpagefont}{\normalfont}
% \renewcommand{\cftsubsecfont}{\small\relax}
\renewcommand{\cftsecdotsep}{\cftdotsep}
\renewcommand{\cftsecpagefont}{\normalfont}
\renewcommand{\cftsecleader}{\normalfont\cftdotfill{\cftsecdotsep}}
\setlength{\cftsecindent}{1em}
\setlength{\cftsubsecindent}{2em}
\setlength{\cftsubsubsecindent}{3em}
\setlength{\cftchapnumwidth}{1em}
\setlength{\cftsecnumwidth}{1em}
\setlength{\cftsubsecnumwidth}{1em}
\setlength{\cftsubsubsecnumwidth}{1em}

% footnotes
\newif\ifheading
\newcommand*{\fnmarkscale}{\ifheading 0.70 \else 1 \fi}
\renewcommand\footnoterule{\vspace*{0.3cm}\hrule height \arrayrulewidth width 3cm \vspace*{0.3cm}}
\setlength\footnotesep{1.5\footnotesep} % footnote separator
\renewcommand\@makefntext[1]{\parindent 1.5em \noindent \hb@xt@1.8em{\hss{\normalfont\@thefnmark . }}#1} % no superscipt in foot
\patchcmd{\@footnotetext}{\footnotesize}{\footnotesize\sffamily}{}{} % before scrextend, hyperref


%   see https://tex.stackexchange.com/a/34449/5049
\def\truncdiv#1#2{((#1-(#2-1)/2)/#2)}
\def\moduloop#1#2{(#1-\truncdiv{#1}{#2}*#2)}
\def\modulo#1#2{\number\numexpr\moduloop{#1}{#2}\relax}

% orphans and widows
\clubpenalty=9996
\widowpenalty=9999
\brokenpenalty=4991
\predisplaypenalty=10000
\postdisplaypenalty=1549
\displaywidowpenalty=1602
\hyphenpenalty=400
% Copied from Rahtz but not understood
\def\@pnumwidth{1.55em}
\def\@tocrmarg {2.55em}
\def\@dotsep{4.5}
\emergencystretch 3em
\hbadness=4000
\pretolerance=750
\tolerance=2000
\vbadness=4000
\def\Gin@extensions{.pdf,.png,.jpg,.mps,.tif}
% \renewcommand{\@cite}[1]{#1} % biblio

\usepackage{hyperref} % supposed to be the last one, :o) except for the ones to follow
\urlstyle{same} % after hyperref
\hypersetup{
  % pdftex, % no effect
  pdftitle={\elbibl},
  % pdfauthor={Your name here},
  % pdfsubject={Your subject here},
  % pdfkeywords={keyword1, keyword2},
  bookmarksnumbered=true,
  bookmarksopen=true,
  bookmarksopenlevel=1,
  pdfstartview=Fit,
  breaklinks=true, % avoid long links
  pdfpagemode=UseOutlines,    % pdf toc
  hyperfootnotes=true,
  colorlinks=false,
  pdfborder=0 0 0,
  % pdfpagelayout=TwoPageRight,
  % linktocpage=true, % NO, toc, link only on page no
}

\makeatother % /@@@>
%%%%%%%%%%%%%%
% </TEI> end %
%%%%%%%%%%%%%%


%%%%%%%%%%%%%
% footnotes %
%%%%%%%%%%%%%
\renewcommand{\thefootnote}{\bfseries\textcolor{rubric}{\arabic{footnote}}} % color for footnote marks

%%%%%%%%%
% Fonts %
%%%%%%%%%
\usepackage[]{roboto} % SmallCaps, Regular is a bit bold
% \linespread{0.90} % too compact, keep font natural
\newfontfamily\fontrun[]{Roboto Condensed Light} % condensed runing heads
\ifav
  \setmainfont[
    ItalicFont={Roboto Light Italic},
  ]{Roboto}
\else\ifbooklet
  \setmainfont[
    ItalicFont={Roboto Light Italic},
  ]{Roboto}
\else
\setmainfont[
  ItalicFont={Roboto Italic},
]{Roboto Light}
\fi\fi
\renewcommand{\LettrineFontHook}{\bfseries\color{rubric}}
% \renewenvironment{labelblock}{\begin{center}\bfseries\color{rubric}}{\end{center}}

%%%%%%%%
% MISC %
%%%%%%%%

\setdefaultlanguage[frenchpart=false]{french} % bug on part


\newenvironment{quotebar}{%
    \def\FrameCommand{{\color{rubric!10!}\vrule width 0.5em} \hspace{0.9em}}%
    \def\OuterFrameSep{\itemsep} % séparateur vertical
    \MakeFramed {\advance\hsize-\width \FrameRestore}
  }%
  {%
    \endMakeFramed
  }
\renewenvironment{quoteblock}% may be used for ornaments
  {%
    \savenotes
    \setstretch{0.9}
    \normalfont
    \begin{quotebar}
  }
  {%
    \end{quotebar}
    \spewnotes
  }


\renewcommand{\headrulewidth}{\arrayrulewidth}
\renewcommand{\headrule}{{\color{rubric}\hrule}}

% delicate tuning, image has produce line-height problems in title on 2 lines
\titleformat{name=\chapter} % command
  [display] % shape
  {\vspace{1.5em}\centering} % format
  {} % label
  {0pt} % separator between n
  {}
[{\color{rubric}\huge\textbf{#1}}\bigskip] % after code
% \titlespacing{command}{left spacing}{before spacing}{after spacing}[right]
\titlespacing*{\chapter}{0pt}{-2em}{0pt}[0pt]

\titleformat{name=\section}
  [block]{}{}{}{}
  [\vbox{\color{rubric}\large\raggedleft\textbf{#1}}]
\titlespacing{\section}{0pt}{0pt plus 4pt minus 2pt}{\baselineskip}

\titleformat{name=\subsection}
  [block]
  {}
  {} % \thesection
  {} % separator \arrayrulewidth
  {}
[\vbox{\large\textbf{#1}}]
% \titlespacing{\subsection}{0pt}{0pt plus 4pt minus 2pt}{\baselineskip}

\ifaiv
  \fancypagestyle{main}{%
    \fancyhf{}
    \setlength{\headheight}{1.5em}
    \fancyhead{} % reset head
    \fancyfoot{} % reset foot
    \fancyhead[L]{\truncate{0.45\headwidth}{\fontrun\elbibl}} % book ref
    \fancyhead[R]{\truncate{0.45\headwidth}{ \fontrun\nouppercase\leftmark}} % Chapter title
    \fancyhead[C]{\thepage}
  }
  \fancypagestyle{plain}{% apply to chapter
    \fancyhf{}% clear all header and footer fields
    \setlength{\headheight}{1.5em}
    \fancyhead[L]{\truncate{0.9\headwidth}{\fontrun\elbibl}}
    \fancyhead[R]{\thepage}
  }
\else
  \fancypagestyle{main}{%
    \fancyhf{}
    \setlength{\headheight}{1.5em}
    \fancyhead{} % reset head
    \fancyfoot{} % reset foot
    \fancyhead[RE]{\truncate{0.9\headwidth}{\fontrun\elbibl}} % book ref
    \fancyhead[LO]{\truncate{0.9\headwidth}{\fontrun\nouppercase\leftmark}} % Chapter title, \nouppercase needed
    \fancyhead[RO,LE]{\thepage}
  }
  \fancypagestyle{plain}{% apply to chapter
    \fancyhf{}% clear all header and footer fields
    \setlength{\headheight}{1.5em}
    \fancyhead[L]{\truncate{0.9\headwidth}{\fontrun\elbibl}}
    \fancyhead[R]{\thepage}
  }
\fi

\ifav % a5 only
  \titleclass{\section}{top}
\fi

\newcommand\chapo{{%
  \vspace*{-3em}
  \centering % no vskip ()
  {\Large\addfontfeature{LetterSpace=25}\bfseries{\elauthor}}\par
  \smallskip
  {\large\eldate}\par
  \bigskip
  {\Large\selectfont{\eltitle}}\par
  \bigskip
  {\color{rubric}\hline\par}
  \bigskip
  {\Large TEXTE LIBRE À PARTICPATION LIBRE\par}
  \centerline{\small\color{rubric} {hurlus.fr, tiré le \today}}\par
  \bigskip
}}

\newcommand\cover{{%
  \thispagestyle{empty}
  \centering
  {\LARGE\bfseries{\elauthor}}\par
  \bigskip
  {\Large\eldate}\par
  \bigskip
  \bigskip
  {\LARGE\selectfont{\eltitle}}\par
  \vfill\null
  {\color{rubric}\setlength{\arrayrulewidth}{2pt}\hline\par}
  \vfill\null
  {\Large TEXTE LIBRE À PARTICPATION LIBRE\par}
  \centerline{{\href{https://hurlus.fr}{\dotuline{hurlus.fr}}, tiré le \today}}\par
}}

\begin{document}
\pagestyle{empty}
\ifbooklet{
  \cover\newpage
  \thispagestyle{empty}\hbox{}\newpage
  \cover\newpage\noindent Les voyages de la brochure\par
  \bigskip
  \begin{tabularx}{\textwidth}{l|X|X}
    \textbf{Date} & \textbf{Lieu}& \textbf{Nom/pseudo} \\ \hline
    \rule{0pt}{25cm} &  &   \\
  \end{tabularx}
  \newpage
  \addtocounter{page}{-4}
}\fi

\thispagestyle{empty}
\ifaiv
  \twocolumn[\chapo]
\else
  \chapo
\fi
{\it\elabstract}
\bigskip
\makeatletter\@starttoc{toc}\makeatother % toc without new page
\bigskip

\pagestyle{main} % after style

   \noindent Il n’y avait en tête des premières éditions de cet ouvrage, publié d’abord sans nom d’auteur, que les quelques lignes qu’on va lire :\par

\begin{quoteblock}
 \noindent « Il y a deux manières de se rendre compte de l’existence de ce livre. Ou il y a eu, en effet, une liasse de papiers jaunes et inégaux sur lesquels on a trouvé, enregistrées une à une, les dernières pensées d’un misérable ; ou il s’est rencontré un homme, un rêveur occupé à observer la nature au profit de l’art, un philosophe, un poëte, que sais-je ? dont cette idée a été la fantaisie, qui l’a prise ou plutôt s’est laissé prendre par elle, et n’a pu s’en débarrasser qu’en la jetant dans un livre.\par
 « De ces deux explications, le lecteur choisira celle qu’il voudra. »
 \end{quoteblock}

\noindent Comme on le voit, à l’époque où ce livre fut publié, l’auteur ne jugea pas à propos de dire dès lors toute sa  pensée. Il aima mieux attendre qu’elle fût comprise et voir si elle le serait. Elle l’a été. L’auteur aujourd’hui peut démasquer l’idée politique, l’idée sociale, qu’il avait voulu populariser sous cette innocente et candide forme littéraire. Il déclare donc, ou plutôt il avoue hautement que \emph{le Dernier Jour d’un condamné} n’est autre chose qu’un plaidoyer, direct ou indirect, comme on voudra, pour l’abolition de la peine de mort. Ce qu’il a eu dessein de faire, ce qu’il voudrait que la postérité vît dans son œuvre, si jamais elle s’occupe de si peu, ce n’est pas la défense spéciale, et toujours facile, et toujours transitoire, de tel ou tel criminel choisi, de tel ou tel accusé d’élection ; c’est la plaidoirie générale et permanente pour tous les accusés présents et à venir ; c’est le grand point de droit de l’humanité allégué et plaidé à toute voix devant la société, qui est la grande cour de cassation ; c’est cette suprême fin de non-recevoir, \emph{abhorrescere a sanguine}, construite à tout jamais en avant de tous les procès criminels ; c’est la sombre et fatale question qui palpite obscurément au fond de toutes les causes capitales sous les triples épaisseurs de pathos dont l’enveloppe la rhétorique sanglante des gens du roi ; c’est la question de vie et de mort, dis-je, déshabillée, dénudée, dépouillée des entortillages sonores du parquet, brutalement mise au jour, et posée où il faut qu’on la voie, et où il faut qu’elle soit, où elle est réellement, dans son vrai milieu, dans son milieu horrible, non au tribunal, mais à l’échafaud, non chez le juge, mais chez le bourreau.\par
Voilà ce qu’il a voulu faire. Si l’avenir lui décernait un jour la gloire de l’avoir fait, ce qu’il n’ose espérer, il ne voudrait pas d’autre couronne.\par
 Il le déclare donc, et il le répète, il occupe, au nom de tous les accusés possibles, innocents ou coupables, devant toutes les cours, tous les prétoires, tous les jurys, toutes les justices. Ce livre est adressé à quiconque juge. Et pour que le plaidoyer soit aussi vaste que la cause, il a dû, et c’est pour cela que \emph{le Dernier Jour d’un condamné} est ainsi fait, élaguer de toutes parts dans son sujet le contingent, l’accident, le particulier, le spécial, le relatif, le modifiable, l’épisode, l’anecdote, l’événement, le nom propre, et se borner (si c’est là se borner) à plaider la cause d’un condamné quelconque, exécuté un jour quelconque, pour un crime quelconque. Heureux si, sans autre outil que sa pensée, il a fouillé assez avant pour faire saigner un cœur sous l’\emph{æs triplex} du magistrat ! heureux s’il a rendu pitoyables ceux qui se croient justes ! heureux si, à force de creuser dans le juge, il a réussi quelquefois à y retrouver un homme !\par
Il y a trois ans, quand ce livre parut, quelques personnes imaginèrent que cela valait la peine d’en contester l’idée à l’auteur. Les uns supposèrent un livre anglais, les autres un livre américain. Singulière manie de chercher à mille lieues les origines des choses, et de faire couler des sources du Nil le ruisseau qui lave votre rue ! Hélas ! il n’y a en ceci ni livre anglais, ni livre américain, ni livre chinois. L’auteur a pris l’idée du \emph{Dernier Jour d’un condamné}, non dans un livre, il n’a pas l’habitude d’aller chercher ses idées si loin, mais là où vous pouviez tous la prendre, où vous l’avez prise peut-être (car qui n’a fait ou rêvé dans son esprit \emph{le Dernier Jour d’un condamné ?}), tout bonnement sur la place publique, sur la place de Grève.  C’est là qu’un jour en passant il a ramassé cette idée fatale, gisante dans une mare de sang sous les rouges moignons de la guillotine.\par
Depuis, chaque fois qu’au gré des funèbres jeudis de la cour de cassation, il arrivait un de ces jours où le cri d’un arrêt de mort se fait dans Paris, chaque fois que l’auteur entendait passer sous ses fenêtres ces hurleurs enroués qui ameutent des spectateurs pour la Grève, chaque fois la douloureuse idée lui revenait, s’emparait de lui, lui emplissait la tête de gendarmes, de bourreaux et de foule, lui expliquait heure par heure les dernières souffrances du misérable agonisant, — en ce moment on le confesse, en ce moment on lui coupe les cheveux, en ce moment on lui lie les mains, — le sommait, lui pauvre poëte, de dire tout cela à la société, qui fait ses affaires pendant que cette chose monstrueuse s’accomplit, le pressait, le poussait, le secouait, lui arrachait ses vers de l’esprit, s’il était en train d’en faire, et les tuait à peine ébauchés, barrait tous ses travaux, se mettait en travers de tout, l’investissait, l’obsédait, l’assiégeait. C’était un supplice, un supplice qui commençait avec le jour, et qui durait, comme celui du misérable qu’on torturait au même moment, jusqu’à \emph{quatre heures}. Alors seulement, une fois le \emph{ponens caput expiravit} crié par la voix sinistre de l’horloge, l’auteur respirait et retrouvait quelque liberté d’esprit. Un jour enfin, c’était, à ce qu’il croit, le lendemain de l’exécution d’Ulbach, il se mit à écrire ce livre. Depuis lors il a été soulagé. Quand un de ces crimes publics, qu’on nomme exécutions judiciaires, a été commis, sa conscience lui a dit qu’il n’en était plus solidaire ; et il n’a plus senti à son  front cette goutte de sang qui rejaillit de la Grève sur la tête de tous les membres de la communauté sociale.\par
Toutefois, cela ne suffit pas. Se laver les mains est bien, empêcher le sang de couler serait mieux.\par
Aussi ne connaîtrait-il pas de but plus élevé, plus saint, plus auguste que celui-là : concourir à l’abolition de la peine de mort. Aussi est-ce du fond du cœur qu’il adhère aux vœux et aux efforts des hommes généreux de toutes les nations qui travaillent depuis plusieurs années à jeter bas l’arbre patibulaire, le seul arbre que les révolutions ne déracinent pas. C’est avec joie qu’il vient à son tour, lui chétif, donner son coup de cognée, et élargir de son mieux l’entaille que Beccaria a faite, il y a soixante-six ans, au vieux gibet dressé depuis tant de siècles sur la chrétienté.\par
Nous venons de dire que l’échafaud est le seul édifice que les révolutions ne démolissent pas. Il est rare, en effet, que les révolutions soient sobres de sang humain, et, venues qu’elles sont pour émonder, pour ébrancher, pour étêter la société, la peine de mort est une des serpes dont elles se dessaisissent le plus malaisément.\par
Nous l’avouerons cependant, si jamais révolution nous parut digne et capable d’abolir la peine de mort, c’est la révolution de juillet. Il semble, en effet, qu’il appartenait au mouvement populaire le plus clément des temps modernes de raturer la pénalité barbare de Louis XI, de Richelieu et de Robespierre, et d’inscrire au front de la loi l’inviolabilité de la vie humaine. 1830 méritait de briser le couperet de 93.\par
Nous l’avons espéré un moment. En août 1830, il y avait tant de générosité dans l’air, un tel esprit de douceur et de  civilisation flottait dans les masses, on se sentait le cœur si bien épanoui par l’approche d’un bel avenir, qu’il nous sembla que la peine de mort était abolie de droit, d’emblée, d’un consentement tacite et unanime, comme le reste des choses mauvaises qui nous avaient gênés. Le peuple venait de faire un feu de joie des guenilles de l’ancien régime. Celle-là était la guenille sanglante. Nous la crûmes dans le tas. Nous la crûmes brûlée comme les autres. Et pendant quelques semaines, confiant et crédule, nous eûmes foi pour l’avenir à l’inviolabilité de la vie, comme à l’inviolabilité de la liberté.\par
Et en effet deux mois s’étaient à peine écoulés qu’une tentative fut faite pour résoudre en réalité légale l’utopie sublime de César Bonesana.\par
Malheureusement, cette tentative fut gauche, maladroite, presque hypocrite, et faite dans un autre intérêt que l’intérêt général.\par
Au mois d’octobre 1830, on se le rappelle, quelques jours après avoir écarté par l’ordre du jour la proposition d’ensevelir Napoléon sous la colonne, la Chambre tout entière se mit à pleurer et à bramer. La question de la peine de mort fut remise sur le tapis, nous allons dire quelques lignes plus bas à quelle occasion ; et alors il sembla que toutes ces entrailles de législateurs étaient prises d’une subite et merveilleuse miséricorde. Ce fut à qui parlerait, à qui gémirait, à qui lèverait les mains au ciel. La peine de mort, grand Dieu ! quelle horreur ! Tel vieux procureur général, blanchi dans la robe rouge, qui avait mangé toute sa vie le pain trempé de sang des réquisitoires, se composa tout à coup un air piteux et attesta les dieux qu’il était  indigné de la guillotine. Pendant deux jours la tribune ne désemplit pas de harangueurs en pleureuses. Ce fut une lamentation, une myriologie, un concert de psaumes lugubres, un \emph{Super flumina Babylonis}, un \emph{Stabat mater dolorosa}, une grande symphonie en ut, avec chœurs, exécutée par tout cet orchestre d’orateurs qui garnit les premiers bancs de la Chambre, et rend de si beaux sons dans les grands jours. Tel vint avec sa basse, tel avec son fausset. Rien n’y manqua. La chose fut on ne peut plus pathétique et pitoyable. La séance de nuit surtout fut tendre, paterne et déchirante comme un cinquième acte de Lachaussée. Le bon public, qui n’y comprenait rien, avait les larmes aux yeux\footnote{ \noindent Nous ne prétendons pas envelopper dans le même dédain \emph{tout} ce qui a été dit à cette occasion à la Chambre. Il s’est bien prononcé çà et là quelques belles et dignes paroles. Nous avons applaudi, comme tout le monde, au discours grave et simple de M. de Lafayette, et, dans une autre nuance, à la remarquable improvisation de M. Villemain.
 }.\par
De quoi s’agissait-il donc ? d’abolir la peine de mort ?\par
Oui et non.\par
Voici le fait :\par
Quatre hommes du monde, quatre hommes comme il faut, de ces hommes qu’on a pu rencontrer dans un salon, et avec qui peut-être on a échangé quelques paroles polies ; quatre de ces hommes, dis-je, avaient tenté, dans les hautes régions politiques, un de ces coups hardis que Bâcon appelle \emph{crimes}, et que Machiavel appelle \emph{entreprises}. Or, crime ou entreprises, la loi, brutale pour tout, punit cela de mort. Et les quatre malheureux étaient là, prisonniers, captifs de la loi, gardés par trois cents cocardes tricolores sous les belles ogives de Vincennes. Que faire et comment faire ? Vous comprenez qu’il est impossible d’envoyer à la Grève,  dans une charrette, ignoblement liés avec de grosses cordes, dos à dos avec ce fonctionnaire qu’il ne faut pas seulement nommer, quatre hommes comme vous et moi, quatre \emph{hommes du monde ?} Encore s’il y avait une guillotine en acajou !\par
Eh ! il n’y a qu’à abolir la peine de mort !\par
Et là-dessus, la Chambre se met en besogne.\par
Remarquez, messieurs, qu’hier encore, vous traitiez cette abolition d’utopie, de théorie, de rêve, de folie, de poésie. Remarquez que ce n’est pas la première fois qu’on cherche à appeler votre attention sur la charrette, sur les grosses cordes et sur l’horrible machine écarlate, et qu’il est étrange que ce hideux attirail vous saute ainsi aux yeux tout à coup.\par
Bah ! c’est bien de cela qu’il s’agit ! Ce n’est pas à cause de vous, peuple, que nous abolissons la peine de mort, mais à cause de nous, députés, qui pouvons être ministres. Nous ne voulons pas que la mécanique de Guillotin morde les hautes classes. Nous la brisons. Tant mieux si cela arrange tout le monde, mais nous n’avons songé qu’à nous. Ucalégon brûle. Éteignons le feu. Vite, supprimons le bourreau, biffons le code.\par
Et c’est ainsi qu’un alliage d’égoïsme altère et dénature les plus belles combinaisons sociales. C’est la veine noire dans le marbre blanc ; elle circule partout, et apparaît à tout moment à l’improviste sous le ciseau. Votre statue est à refaire.\par
Certes, il n’est pas besoin que nous le déclarions ici, nous ne sommes pas de ceux qui réclamaient les têtes des quatre ministres. Une fois ces infortunés arrêtés, la colère  indignée que nous avait inspirée leur attentat s’est changée, chez nous comme chez tout le monde, en une profonde pitié. Nous avons songé aux préjugés d’éducation de quelques-uns d’entre eux, au cerveau peu développé de leur chef, relaps fanatique et obstiné des conspirations de 1804, blanchi avant l’âge sous l’ombre humide des prisons d’état, aux nécessités fatales de leur position commune, à l’impossibilité d’enrayer sur cette pente rapide où la monarchie s’était lancée elle-même à toute bride le 8 août 1829, à l’influence trop peu calculée par nous jusqu’alors de la personne royale, surtout à la dignité que l’un d’entre eux répandait comme un manteau de pourpre sur leur malheur. Nous sommes de ceux qui leur souhaitaient bien sincèrement la vie sauve, et qui étaient prêts à se dévouer pour cela. Si jamais, par impossible, leur échafaud eût été dressé un jour en Grève, nous ne doutons pas, et si c’est une illusion nous voulons la conserver, nous ne doutons pas qu’il n’y eût eu une émeute pour le renverser, et celui qui écrit ces lignes eût été de cette sainte émeute. Car, il faut bien le dire aussi, dans les crises sociales, de tous les échafauds, l’échafaud politique est le plus abominable, le plus funeste, le plus vénéneux, le plus nécessaire à extirper. Cette espèce de guillotine-là prend racine dans le pavé, et en peu de temps repousse de bouture sur tous les points du sol.\par
En temps de révolution, prenez garde à la première tête qui tombe. Elle met le peuple en appétit.\par
Nous étions donc personnellement d’accord avec ceux qui voulaient épargner les quatre ministres, et d’accord de toutes manières, par les raisons sentimentales comme par  les raisons politiques. Seulement, nous eussions mieux aimé que la Chambre choisît une autre occasion pour proposer l’abolition de la peine de mort.\par
Si on l’avait proposée, cette souhaitable abolition, non à propos de quatre ministres tombés des Tuileries à Vincennes, mais à propos du premier voleur de grands chemins venu, à propos d’un de ces misérables que vous regardez à peine quand ils passent près de vous dans la rue, auxquels vous ne parlez pas, dont vous évitez instinctivement le coudoiement poudreux ; malheureux dont l’enfance déguenillée a couru pieds nus dans la boue des carrefours, grelottant l’hiver au rebord des quais, se chauffant au soupirail des cuisines de M. Véfour chez qui vous dînez, déterrant çà et là une croûte de pain dans un tas d’ordures et l’essuyant avant de la manger, grattant tout le jour le ruisseau avec un clou pour y trouver un liard, n’ayant d’autre amusement que le spectacle gratis de la fête du roi et les exécutions en Grève, cet autre spectacle gratis ; pauvres diables, que la faim pousse au vol, et le vol au reste ; enfants déshérités d’une société marâtre, que la maison de force prend à douze ans, le bagne à dix-huit, l’échafaud à quarante ; infortunés qu’avec une école et un atelier vous auriez pu rendre bons, moraux, utiles, et dont vous ne savez que faire, les versant, comme un fardeau inutile, tantôt dans la rouge fourmilière de Toulon, tantôt dans le muet enclos de Clamart, leur retranchant la vie après leur avoir ôté la liberté ; si c’eût été à propos d’un de ces hommes que vous eussiez proposé d’abolir la peine de mort, oh ! alors, votre séance eût été vraiment digne, grande, sainte, majestueuse, vénérable. Depuis les augustes pères de Trente, invitant les  hérétiques au concile au nom des entrailles de Dieu, \emph{per viscera Dei}, parce qu’on espère leur conversion, \emph{quoniam sancta synodis sperat hæreticorum conversionem,} jamais assemblée d’hommes n’aurait présenté au monde spectacle plus sublime, plus illustre et plus miséricordieux. Il a toujours appartenu à ceux qui sont vraiment forts et vraiment grands d’avoir souci du faible et du petit. Un conseil de brahmines serait beau prenant en main la cause du paria. Et ici la cause du paria, c’était la cause du peuple. En abolissant la peine de mort, à cause de lui et sans attendre que vous fussiez intéressés dans la question, vous faisiez plus qu’une œuvre politique, vous faisiez une œuvre sociale.\par
Tandis que vous n’avez pas même fait une œuvre politique en essayant de l’abolir, non pour l’abolir, mais pour sauver quatre malheureux ministres pris la main dans le sac des coups d’état !\par
Qu’est-il arrivé ? c’est que, comme vous n’étiez pas sincères, on a été défiant. Quand le peuple a vu qu’on voulait lui donner le change, il s’est fâché contre toute la question en masse, et, chose remarquable ! il a pris fait et cause pour cette peine de mort dont il supporte pourtant tout le poids. C’est votre maladresse qui l’a amené là. En abordant la question de biais et sans franchise, vous l’avez compromise pour longtemps. Vous jouiez une comédie. On l’a sifflée.\par
Cette farce pourtant, quelques esprits avaient eu la bonté de la prendre au sérieux. Immédiatement après la fameuse séance, ordre avait été donné aux procureurs généraux, par un garde des sceaux honnête homme, de suspendre indéfiniment toutes exécutions capitales. C’était en  apparence un grand pas. Les adversaires de la peine de mort respirèrent. Mais leur illusion fut de courte durée.\par
Le procès des ministres fut mené à fin. Je ne sais quel arrêt fut rendu. Les quatre vies furent épargnées. Ham fut choisi comme juste milieu entre la mort et la liberté. Ces divers arrangements une fois faits, toute peur s’évanouit dans l’esprit des hommes d’état dirigeants, et, avec la peur, l’humanité s’en alla. Il ne fut plus question d’abolir le supplice capital ; et une fois qu’on n’eut plus besoin d’elle, l’utopie redevint utopie, la théorie, théorie, la poésie, poésie.\par
Il y avait pourtant toujours dans les prisons quelques malheureux condamnés vulgaires qui se promenaient dans les préaux depuis cinq ou six mois, respirant l’air, tranquilles désormais, sûrs de vivre, prenant leur sursis pour leur grâce. Mais attendez.\par
Le bourreau, à vrai dire, avait eu grand’peur. Le jour où il avait entendu nos faiseurs de lois parler humanité, philanthropie, progrès, il s’était cru perdu. Il s’était caché, le misérable, il s’était blotti sous sa guillotine, mal à l’aise au soleil de juillet comme un oiseau de nuit en plein jour, tâchant de se faire oublier, se bouchant les oreilles et n’osant souffler. On ne le voyait plus depuis six mois. Il ne donnait plus signe de vie. Peu à peu cependant il s’était rassuré dans ses ténèbres. Il avait écouté du côté des Chambres et n’avait plus entendu prononcer son nom. Plus de ces grands mots sonores dont il avait eu si grande frayeur. Plus de commentaires déclamatoires du \emph{Traité des Délits et des Peines.} On s’occupait de toute autre chose, de quelque grave intérêt social, d’un chemin vicinal, d’une  subvention pour l’Opéra-Comique, ou d’une saignée de cent mille francs sur un budget apoplectique de quinze cents millions. Personne ne songeait plus à lui, coupe-tête. Ce que voyant, l’homme se tranquillise, il met sa tête hors de son trou, et regarde de tous côtés ; il fait un pas, puis deux, comme je ne sais plus quelle souris de La Fontaine, puis il se hasarde à sortir tout à fait de dessous son échafaudage, puis il saute dessus, le raccommode, le restaure, le fourbit, le caresse, le fait jouer, le fait reluire, se remet à suifer la vieille mécanique rouillée que l’oisiveté détraquait ; tout à coup il se retourne, saisit au hasard par les cheveux, dans la première prison venue, un de ces infortunés qui comptaient sur la vie, le tire à lui, le dépouille, l’attache, le boucle, et voilà les exécutions qui recommencent.\par
Tout cela est affreux, mais c’est de l’histoire.\par
Oui, il y a eu un sursis de six mois accordé à de malheureux captifs, dont on a gratuitement aggravé la peine de cette façon en les faisant reprendre à la vie ; puis, sans raison, sans nécessité, sans trop savoir pourquoi, \emph{pour le plaisir}, on a un beau matin révoqué le sursis, et l’on a remis froidement toutes ces créatures humaines en coupe réglée. Eh ! mon Dieu ! je vous le demande, qu’est-ce que cela nous faisait à tous que ces hommes vécussent ? Est-ce qu’il n’y a pas en France assez d’air à respirer pour tout le monde ?\par
Pour qu’un jour un misérable commis de la chancellerie, à qui cela était égal, se soit levé de sa chaise en disant : — Allons ! personne ne songe plus à l’abolition de la peine de mort. Il est temps de se remettre à guillotiner !  — il faut qu’il se soit passé dans le cœur de cet homme-là quelque chose de bien monstrueux.\par
Du reste, disons-le, jamais les exécutions n’ont été accompagnées de circonstances plus atroces que depuis cette révocation du sursis de juillet. Jamais l’anecdote de la Grève n’a été plus révoltante et n’a mieux prouvé l’exécration de la peine de mort. Ce redoublement d’horreur est le juste châtiment des hommes qui ont remis le code du sang en vigueur. Qu’ils soient punis par leur œuvre. C’est bien fait.\par
Il faut citer ici deux ou trois exemples de ce que certaines exécutions ont eu d’épouvantable et d’impie. Il faut donner mal aux nerfs aux femmes des procureurs du roi. Une femme, c’est quelquefois une conscience.\par
Dans le midi, vers la fin du mois de septembre dernier, nous n’avons pas bien présents à l’esprit le lieu, le jour, ni le nom du condamné, mais nous les retrouverons si l’on conteste le fait, et nous croyons que c’est à Pamiers ; vers la fin de septembre donc, on vient trouver un homme dans sa prison, où il jouait tranquillement aux cartes ; on lui signifie qu’il faut mourir dans deux heures, ce qui le fait trembler de tous ses membres, car, depuis six mois qu’on l’oubliait, il ne comptait plus sur la mort ; on le rase, on le tond, on le garrotte, on le confesse ; puis on le brouette entre quatre gendarmes, et à travers la foule, au lieu de l’exécution. Jusqu’ici rien que de simple. C’est comme cela que cela se fait. Arrivé à l’échafaud, le bourreau le prend au prêtre, l’emporte, le ficelle sur la bascule, \emph{l’enfourne,} je me sers ici du mot d’argot, puis il lâche le couperet. Le lourd triangle de fer se détache avec peine, tombe en  cahotant dans ses rainures, et, voici l’horrible qui commence, entaille l’homme sans le tuer. L’homme pousse un cri affreux. Le bourreau, déconcerté, relève le couperet et le laisse retomber. Le couperet mord le cou du patient une seconde fois, mais ne le tranche pas. Le patient hurle, la foule aussi. Le bourreau rehisse encore le couperet, espérant mieux du troisième coup. Point. Le troisième coup fait jaillir un troisième ruisseau de sang de la nuque du condamné, mais ne fait pas tomber la tête. Abrégeons. Le couteau remonta et retomba cinq fois, cinq fois il entama le condamné, cinq fois le condamné hurla sous le coup et secoua sa tête vivante en criant grâce ! Le peuple indigné prit des pierres et se mit dans sa justice à lapider le bourreau. Le bourreau s’enfuit sous la guillotine et s’y tapit derrière les chevaux des gendarmes. Mais vous n’êtes pas au bout. Le supplicié, se voyant seul sur l’échafaud, s’était redressé sur la planche, et là, debout, effroyable, ruisselant de sang, soutenant sa tête à demi coupée qui pendait sur son épaule, il demandait avec de faibles cris qu’on vînt le détacher. La foule, pleine de pitié, était sur le point de forcer les gendarmes et de venir à l’aide du malheureux qui avait subi cinq fois son arrêt de mort. C’est en ce moment-là qu’un valet de bourreau, jeune homme de vingt ans, monte sur l’échafaud, dit au patient de se tourner pour qu’il le délie, et, profitant de la posture du mourant qui se livrait à lui sans défiance, saute sur son dos et se met à lui couper péniblement ce qui lui restait de cou avec je ne sais quel couteau de boucher. Cela s’est fait. Cela s’est vu. Oui.\par
Aux termes de la loi, un juge a dû assister à cette  exécution ! D’un signe il pouvait tout arrêter. Que faisait-il donc au fond de sa voiture, cet homme, pendant qu’on massacrait un homme ? Que faisait-il, ce punisseur d’assassins, pendant qu’on assassinait en plein jour, sous ses yeux, sous le souffle de ses chevaux, sous la vitre de sa portière ?\par
Et le juge n’a pas été mis en jugement ! et le bourreau n’a pas été mis en jugement ! Et aucun tribunal ne s’est enquis de cette monstrueuse extermination de toutes les lois sur la personne sacrée d’une créature de Dieu !\par
Au dix-septième siècle, à l’époque de barbarie du code criminel, sous Richelieu, sous Christophe Fouquet, quand M. de Chalais fut mis à mort devant le Bouffay de Nantes par un soldat maladroit qui, au lieu d’un coup d’épée, lui donna trente-quatre coups\footnote{ \noindent La Porte dit vingt-deux, mais Aubery dit trente-quatre. M. de Chalais cria jusqu’au vingtième.
 } d’une doloire de tonnelier, du moins cela parut-il irrégulier au parlement de Paris ; il y eut enquête et procès, et si Richelieu ne fut pas puni, si Christophe Fouquet ne fut pas puni, le soldat le fut. Injustice sans doute, mais au fond de laquelle il y avait de la justice.\par
Ici, rien. La chose a eu lieu après juillet, dans un temps de douces mœurs et de progrès, un an après la célèbre lamentation de la Chambre sur la peine de mort. Eh bien ! le fait a passé absolument inaperçu. Les journaux de Paris l’ont publié comme une anecdote. Personne n’a été inquiété. On a su seulement que la guillotine avait été disloquée exprès par quelqu’un \emph{qui voulait nuire à l’exécuteur  des hautes oeuvres.} C’était un valet du bourreau, chassé par son maître, qui, pour se venger, lui avait fait cette malice.\par
Ce n’était qu’une espièglerie. Continuons.\par
A Dijon, il y a trois mois, on a mené au supplice une femme. (Une femme !) Cette fois encore, le couteau du docteur Guillotin a mal fait son service. La tête n’a pas été tout à fait coupée. Alors les valets de l’exécuteur se sont attelés aux pieds de la femme, et à travers les hurlements de la malheureuse, et à force de tiraillements et de soubresauts, ils lui ont séparé la tête du corps par arrachement.\par
A Paris, nous revenons au temps des exécutions secrètes. Comme on n’ose plus décapiter en Grève depuis juillet, comme on a peur, comme on est lâche, voici ce qu’on fait. On a pris dernièrement à Bicétre un homme, un condamné à mort, un nommé Désandrieux, je crois ; on l’a mis dans une espèce de panier traîné sur deux roues, clos de toutes parts, cadenassé et verrouillé ; puis, un gendarme en tête, un gendarme en queue, à petit bruit et sans foule, on a été déposer le paquet à la barrière déserte de Saint-Jacques. Arrivés là, il était huit heures du matin, à peine jour, il y avait une guillotine toute fraîche dressée, pour public, quelque douzaine de petits garçons groupés sur les tas de pierres voisins autour de la machine inattendue ; vite, on a tiré l’homme du panier, et, sans lui donner le temps de respirer, furtivement, sournoisement, honteusement, on lui a escamoté sa tête. Cela s’appelle un acte public et solennel de haute justice. Infâme dérision !\par
Comment donc les gens du roi comprennent-ils le mot civilisation ? Où en sommes-nous ? La justice ravalée aux  stratagèmes et aux supercheries ! la loi aux expédients ! monstrueux !\par
C’est donc une chose bien redoutable qu’un condamné à mort, pour que la société le prenne en traître de cette façon !\par
Soyons juste pourtant, l’exécution n’a pas été tout à fait secrète. Le matin on a crié et vendu comme de coutume l’arrêt de mort dans les carrefours de Paris. Il paraît qu’il y a des gens qui vivent de cette vente. Vous entendez ? du crime d’un infortuné, de son châtiment, de ses tortures, de son agonie, on fait une denrée, un papier qu’on vend un sou. Concevez-vous rien de plus hideux que ce sou vertdegrisé dans le sang ? Qui est-ce donc qui le ramasse ?\par
Voilà assez de faits. En voilà trop. Est-ce que tout cela n’est pas horrible ?\par
Qu’avez-vous à alléguer pour la peine de mort ?\par
Nous faisons cette question sérieusement ; nous la faisons pour qu’on y réponde ; nous la faisons aux criminalistes, et non aux lettrés bavards. Nous savons qu’il y a des gens qui prennent l’excellence de la peine de mort pour texte à paradoxe comme tout autre thème. Il y en a d’autres qui n’aiment la peine de mort que parce qu’ils haïssent tel ou tel qui l’attaque. C’est pour eux une question quasi-littéraire, une question de personnes, une question de noms propres. Ceux-là sont les envieux, qui ne font pas plus faute aux bons jurisconsultes qu’aux grands artistes. Les Joseph Grippa ne manquent pas plus aux Filangieri que les Torregiani aux Michel-Ange et les Scudéry aux Corneille.\par
 Ce n’est pas à eux que nous nous adressons, mais aux hommes de loi proprement dits, aux dialecticiens, aux raisonneurs, à ceux qui aiment la peine de mort pour la peine de mort, pour sa beauté, pour sa bonté, pour sa grâce.\par
Voyons, qu’ils donnent leurs raisons.\par
Ceux qui jugent et qui condamnent disent la peine de mort nécessaire. D’abord, — parce qu’il importe de retrancher de la communauté sociale un membre qui lui a déjà nui et qui pourrait lui nuire encore. — S’il ne s’agissait que de cela, la prison perpétuelle suffirait. A quoi bon la mort ? Vous objectez qu’on peut s’échapper d’une prison ? faites mieux votre ronde. Si vous ne croyez pas à la solidité des barreaux de fer, comment osez-vous avoir des ménageries ?\par
Pas de bourreau où le geôlier suffit.\par
Mais, reprend-on, — il faut que la société se venge, que la société punisse. — Ni l’un, ni l’autre. Se venger est de l’individu, punir est de Dieu.\par
La société est entre deux. Le châtiment est au-dessus d’elle, la vengeance au-dessous. Rien de si grand et de si petit ne lui sied. Elle ne doit pas « punir pour se venger » ; elle doit \emph{corriger pour améliorer}. Transformez de cette façon la formule des criminalistes, nous la comprenons et nous y adhérons.\par
Reste la troisième et dernière raison, la théorie de l’exemple. — Il faut faire des exemples ! il faut épouvanter par le spectacle du sort réservé aux criminels ceux qui seraient tentés de les imiter ! — Voilà bien à peu près textuellement la phrase éternelle dont tous les réquisitoires des cinq cents parquets de France ne sont que des variations  plus ou moins sonores. Eh bien ! nous nions d’abord qu’il y ait exemple. Nous nions que le spectacle des supplices produise l’effet qu’on en attend. Loin d’édifier le peuple, il le démoralise et ruine en lui toute sensibilité, partant toute vertu. Les preuves abondent, et encombreraient notre raisonnement si nous voulions en citer. Nous signalerons pourtant un fait entre mille, parce qu’il est le plus récent. Au moment où nous écrivons, il n’a que dix jours de date. Il est du 5 mars, dernier jour du carnaval. A Saint-Pol, immédiatement après l’exécution d’un incendiaire nommé Louis Camus, une troupe de masques est venue danser autour de l’échafaud encore fumant. Faites donc des exemples ! le mardi gras vous rit au nez.\par
Que si, malgré l’expérience, vous tenez à votre théorie routinière de l’exemple, alors rendez-nous le seizième siècle, soyez vraiment formidables, rendez-nous la variété des supplices, rendez-nous Farinacci, rendez-nous les tourmenteurs-jurés, rendez-nous le gibet, la roue, le bûcher, l’estrapade, l’essorillement, l’écartèlement, la fosse à enfouir vif, la cuve à bouillir vif ; rendez-nous, dans tous les carrefours de Paris, comme une boutique de plus ouverte parmi les autres, le hideux étal du bourreau, sans cesse garni de chair fraîche. Rendez-nous Montfaucon, ses seize piliers de pierre, ses brutes assises, ses caves à ossements, ses poutres, ses crocs, ses chaînes, ses brochettes de squelettes, son éminence de plâtre tachetée de corbeaux, ses potences succursales, et l’odeur de cadavre que par le vent du nord-est il répand à larges bouffées sur tout le faubourg du Temple. Rendez-nous dans sa permanence et dans sa puissance ce gigantesque appentis du bourreau de Paris. A la bonne  heure ! voilà de l’exemple en grand. Voilà de la peine de mort bien comprise. Voilà un système de supplices qui a quelque proportion. Voilà qui est horrible, mais qui est terrible.\par
Ou bien faites comme en Angleterre. En Angleterre, pays de commerce, on prend un contrebandier sur la côte de Douvres, on le pend \emph{pour l’exemple, pour l’exemple} on le laisse accroché au gibet ; mais, comme les intempéries de l’air pourraient détériorer le cadavre, on l’enveloppe soigneusement d’une toile enduite de goudron, afin d’avoir à le renouveler moins souvent. O terre d’économie ! goudronner les pendus !\par
Cela pourtant a encore quelque logique. C’est la façon la plus humaine de comprendre la théorie de l’exemple.\par
Mais vous, est-ce bien sérieusement que vous croyez faire un exemple quand vous égorgillez misérablement un pauvre homme dans le recoin le plus désert des boulevards extérieurs ? En Grève, en plein jour, passe encore ; mais à la barrière Saint-Jacques ! mais à huit heures du matin ! Qui est-ce qui passe là ? Qui est-ce qui va là ? Qui est-ce qui sait que vous tuez un homme là ? Qui est-ce qui se doute que vous faites un exemple là ? Un exemple pour qui ? Pour les arbres du boulevard, apparemment.\par
Ne voyez-vous donc pas que vos exécutions publiques se font en tapinois ? Ne voyez-vous donc pas que vous vous cachez ? Que vous avez peur et honte de votre œuvre ? Que vous balbutiez ridiculement votre \emph{discite justitiam moniti ?} Qu’au fond vous êtes ébranlés, interdits, inquiets, peu certains d’avoir raison, gagnés par le doute général, coupant  des têtes par routine et sans trop savoir ce que vous faites ? Ne sentez-vous pas au fond du cœur que vous avez tout au moins perdu le sentiment moral et social de la mission de sang que vos prédécesseurs, les vieux parlementaires, accomplissaient avec une conscience si tranquille ? La nuit, ne retournez-vous pas plus souvent qu’eux la tête sur votre oreiller ? D’autres avant vous ont ordonné des exécutions capitales, mais ils s’estimaient dans le droit, dans le juste, dans le bien. Jouvenel des Ursins se croyait un juge ; Élie de Thorrette se croyait un juge ; Laubardemont, La Reynie et Laffemas eux-mêmes se croyaient des juges ; vous, dans votre for intérieur, vous n’êtes pas bien sûrs de ne pas être des assassins !\par
Vous quittez la Grève pour la barrière Saint-Jacques, la foule pour la solitude, le jour pour le crépuscule. Vous ne faites plus fermement ce que vous faites. Vous vous cachez, vous dis-je !\par
Toutes les raisons pour la peine de mort, les voilà donc démolies. Voilà tous les syllogismes de parquet mis à néant. Tous ces copeaux de réquisitoires, les voilà balayés et réduits en cendres. Le moindre attouchement de la logique dissout tous les mauvais raisonnements.\par
Que les gens du roi ne viennent donc plus nous demander des têtes, à nous jurés, à nous hommes, en nous adjurant d’une voix caressante au nom de la société à protéger, de la vindicte publique à assurer, des exemples à faire. Rhétorique, ampoule, et néant que tout cela ! un coup d’épingle dans ces hyperboles, et vous les désenflez. Au fond de ce doucereux verbiage, vous ne trouvez que dureté de cœur, cruauté, barbarie, envie de prouver son zèle, nécessité de  gagner ses honoraires. Taisez-vous, mandarins ! Sous la patte de velours du juge on sent les ongles du bourreau.\par
Il est difficile de songer de sang-froid à ce que c’est qu’un procureur royal criminel. C’est un homme qui gagne sa vie à envoyer les autres à l’échafaud. C’est le pourvoyeur titulaire des places de Grève. Du reste, c’est un monsieur qui a des prétentions au style et aux lettres, qui est beau parleur ou croit l’être, qui récite au besoin un vers latin ou deux avant de conclure à la mort, qui cherche à faire de l’effet, qui intéresse son amour-propre, ô misère ! là où d’autres ont leur vie engagée, qui a ses modèles à lui, ses types désespérants à atteindre, ses classiques, son Bellart, son Marchangy, comme tel poëte a Racine et tel autre Boileau. Dans le débat, il tire du côté de la guillotine, c’est son rôle, c’est son état. Son réquisitoire, c’est son œuvre littéraire, il le fleurit de métaphores, il le parfume de citations, il faut que cela soit beau à l’audience, que cela plaise aux dames. Il a son bagage de lieux communs encore très neufs pour la province, ses élégances d’élocution, ses recherches, ses raffinements d’écrivain. Il hait le mot propre presque autant que nos poëtes tragiques de l’école de Delille. N’ayez pas peur qu’il appelle les choses par leur nom. Fi donc ! Il a pour toute idée dont la nudité vous révolterait des déguisements complets d’épithètes et d’adjectifs. Il rend M. Samson présentable. Il gaze le couperet. Il estompe la bascule. Il entortille le panier rouge dans une périphrase. On ne sait plus ce que c’est. C’est douceâtre et décent. Vous le représentez-vous, la nuit, dans son cabinet, élaborant à loisir et de son mieux cette harangue qui fera dresser un échafaud dans six semaines ? Le voyez-vous  suant sang et eau pour emboîter la tête d’un accusé dans le plus fatal article du code ? Le voyez-vous scier avec une loi mal faite le cou d’un misérable ? Remarquez-vous comme il fait infuser dans un gâchis de tropes et de synecdoches deux ou trois textes vénéneux pour en exprimer et en extraire à grand’peine la mort d’un homme ? N’est-il pas vrai que, tandis qu’il écrit, sous sa table, dans l’ombre, il a probablement le bourreau accroupi à ses pieds, et qu’il arrête de temps en temps sa plume pour lui dire, comme le maître à son chien : — Paix là ! paix là ! tu vas avoir ton os !\par
Du reste, dans la vie privée, cet homme du roi peut être un honnête homme, bon père, bon fils, bon mari, bon ami, comme disent toutes les épitaphes du Père-Lachaise.\par
Espérons que le jour est prochain où la loi abolira ces fonctions funèbres. L’air seul de notre civilisation doit dans un temps donné user la peine de mort.\par
On est parfois tenté de croire que les défenseurs de la peine de mort n’ont pas bien réfléchi à ce que c’est. Mais pesez donc un peu à la balance de quelque crime que ce soit ce droit exorbitant que la société s’arroge d’ôter ce qu’elle n’a pas donné, cette peine, la plus irréparable des peines irréparables !\par
De deux choses l’une :\par
Ou l’homme que vous frappez est sans famille, sans parents, sans adhérents dans ce monde. Et dans ce cas, il n’a reçu ni éducation, ni instruction, ni soins pour son esprit, ni soins pour son cœur ; et alors de quel droit tuez-vous ce misérable orphelin ? Vous le punissez de ce que  son enfance a rampé sur le sol sans tige et sans tuteur ! Vous lui imputez à forfait l’isolement où vous l’avez laissé ! De son malheur vous faites son crime ! Personne ne lui a appris à savoir ce qu’il faisait. Cet homme ignore. Sa faute est à sa destinée, non à lui. Vous frappez un innocent.\par
Ou cet homme a une famille ; et alors croyez-vous que le coup dont vous l’égorgez ne blesse que lui seul ? que son père, que sa mère, que ses enfants, n’en saigneront pas ? Non. En le tuant, vous décapitez toute sa famille. Et ici encore vous frappez des innocents.\par
Gauche et aveugle pénalité, qui, de quelque côté qu’elle se tourne, frappe l’innocent !\par
Cet homme, ce coupable qui a une famille, séquestrez-le. Dans sa prison, il pourra travailler encore pour les siens. Mais comment les fera-t-il vivre du fond de son tombeau ? Et songez-vous sans frissonner à ce que deviendront ces petits garçons, ces petites filles, auxquelles vous ôtez leur père, c’est-à-dire leur pain ? Est-ce que vous comptez sur cette famille pour approvisionner dans quinze ans, eux le bagne, elles le musico ? Oh ! les pauvres innocents !\par
Aux colonies, quand un arrêt de mort tue un esclave, il y a mille francs d’indemnité pour le propriétaire de l’homme. Quoi ! vous dédommagez le maître, et vous n’indemnisez pas la famille ! Ici aussi ne prenez-vous pas un homme à ceux qui le possèdent ? N’est-il pas, à un titre bien autrement sacré que l’esclave vis-à-vis du maître, la propriété de son père, le bien de sa femme, la chose de ses enfants ?\par
Nous avons déjà convaincu votre loi d’assassinat. La voici convaincue de vol.\par
 Autre chose encore. L’âme de cet homme, y songez-vous ? Savez-vous dans quel état elle se trouve ? Osez-vous bien l’expédier si lestement ? Autrefois du moins, quelque foi circulait dans le peuple ; au moment suprême, le souffle religieux qui était dans l’air pouvait amollir le plus endurci ; un patient était en même temps un pénitent ; la religion lui ouvrait un monde au moment où la société lui en fermait un autre ; toute âme avait conscience de Dieu ; l’échafaud n’était qu’une frontière du ciel. Mais quelle espérance mettez-vous sur l’échafaud maintenant que la grosse foule ne croit plus ? maintenant que toutes les religions sont attaquées du dry-rot, comme ces vieux vaisseaux qui pourrissent dans nos ports, et qui jadis peut-être ont découvert des mondes ? maintenant que les petits enfants se moquent de Dieu ? De quel droit lancez-vous dans quelque chose dont vous doutez vous-mêmes les âmes obscures de vos condamnés, ces âmes telles que Voltaire et M. Pigault-Lebrun les ont faites ? Vous les livrez à votre aumônier de prison, excellent vieillard sans doute ; mais croit-il et fait-il croire ? Ne grossoie-t-il pas comme une corvée son œuvre sublime ? Est-ce que vous le prenez pour un prêtre, ce bonhomme qui coudoie le bourreau dans la charrette ? Un écrivain plein d’âme et de talent l’a dit avant nous : \emph{C’est une horrible chose de conserver le bourreau après avoir ôté le confesseur !}\par
Ce ne sont là, sans doute, que des « raisons sentimentales, » comme disent quelques dédaigneux qui ne prennent leur logique que dans leur tête. A nos yeux, ce sont les meilleures. Nous préférons souvent les raisons du sentiment aux raisons de la raison. D’ailleurs les deux séries se  tiennent toujours, ne l’oublions pas. Le \emph{Traité des Délits} est greffé sur l’\emph{Esprit des Lois}. Montesquieu a engendré Beccaria.\par
La raison est pour nous, le sentiment est pour nous, l’expérience est aussi pour nous. Dans les états modèles, où la peine de mort est abolie, la masse des crimes capitaux suit d’année en année une baisse progressive. Pesez ceci.\par
Nous ne demandons cependant pas pour le moment une brusque et complète abolition de la peine de mort, comme celle où s’était si étourdiment engagée la Chambre des députés. Nous désirons, au contraire, tous les essais, toutes les précautions, tous les tâtonnements de la prudence. D’ailleurs, nous ne voulons pas seulement l’abolition de la peine de mort, nous voulons un remaniement complet de la pénalité sous toutes ses formes, du haut en bas, depuis le verrou jusqu’au couperet, et le temps est un des ingrédients qui doivent entrer dans une pareille œuvre pour qu’elle soit bien faite. Nous comptons développer ailleurs, sur cette matière, le système d’idées que nous croyons applicable. Mais, indépendamment des abolitions partielles pour le cas de fausse monnaie, d’incendie, de vols qualifiés, etc., nous demandons que dès à présent, dans toutes les affaires capitales, le président soit tenu de poser au jury cette question : \emph{L’accusé a-t-il agi par passion ou par intérêt ?} et que, dans le cas où le jury répondrait : \emph{L’accusé a agi par passion}, il n’y ait pas condamnation à mort. Ceci nous épargnerait du moins quelques exécutions révoltantes. Ulbach et Debacker seraient sauvés. On ne guillotinerait plus Othello.\par
 Au reste, qu’on ne s’y trompe pas, cette question de la peine de mort mûrit tous les jours. Avant peu, la société entière la résoudra comme nous.\par
Que les criminalistes les plus entêtés y fassent attention, depuis un siècle la peine de mort va s’amoindrissant. Elle se fait presque douce. Signe de décrépitude. Signe de faiblesse. Signe de mort prochaine. La torture a disparu. La roue a disparu. La potence a disparu. Chose étrange ! la guillotine est un progrès.\par
M. Guillotin était un philanthrope.\par
Oui, l’horrible Thémis dentue et vorace de Farinace et de Vouglans, de Delancre et d’Isaac Loisel, de d’Oppède et de Machault, dépérit. Elle maigrit. Elle se meurt.\par
Voici déjà la Grève qui n’en veut plus. La Grève se réhabilite. La vieille buveuse de sang s’est bien conduite en juillet. Elle veut mener désormais meilleure vie et rester digne de sa dernière belle action. Elle qui s’était prostituée depuis trois siècles à tous les échafauds, la pudeur la prend. Elle a honte de son ancien métier. Elle veut perdre son vilain nom. Elle répudie le bourreau. Elle lave son pavé.\par
A l’heure qu’il est, la peine de mort est déjà hors de Paris. Or, disons-le bien ici, sortir de Paris c’est sortir de la civilisation.\par
Tous les symptômes sont pour nous. Il semble aussi qu’elle se rebute et qu’elle rechigne, cette hideuse machine, ou plutôt ce monstre fait de bois et de fer qui est à Guillotin ce que Galatée est à Pygmalion. Vues d’un certain côté, les effroyables exécutions que nous avons détaillées plus haut sont d’excellents signes. La guillotine hésite. Elle en  est à manquer son coup. Tout le vieil échafaudage de la peine de mort se détraque.\par
L’infâme machine partira de France, nous y comptons, et, s’il plaît à Dieu, elle partira en boitant, car nous tâcherons de lui porter de rudes coups.\par
Qu’elle aille demander l’hospitalité ailleurs, à quelque peuple barbare, non à la Turquie, qui se civilise, non aux sauvages, qui ne voudraient pas d’elle\footnote{ \noindent Le « parlement » d’Otahiti vient d’abolir la peine de mort.
 } ; mais qu’elle descende quelques échelons encore de l’échelle de la civilisation, qu’elle aille en Espagne ou en Russie.\par
L’édifice social du passé reposait sur trois colonnes, le prêtre, le roi, le bourreau. Il y a déjà longtemps qu’une voix a dit : \emph{Les dieux s’en vont !} Dernièrement une autre voix s’est élevée et a crié : \emph{Les rois s’en vont !} Il est temps maintenant qu’une troisième voix s’élève et dise : \emph{Le bourreau s’en va !}\par
Ainsi l’ancienne société sera tombée pierre à pierre ; ainsi la providence aura complété l’écroulement du passé.\par
A ceux qui ont regretté les dieux, on a pu dire ; Dieu reste. A ceux qui regrettent les rois, on peut dire : La patrie reste. A ceux qui regretteraient le bourreau, on n’a rien à dire.\par
Et l’ordre ne disparaîtra pas avec le bourreau ; ne le croyez point. La voûte de la société future ne croulera pas pour n’avoir point cette clef hideuse. La civilisation n’est autre chose qu’une série de transformations successives. A quoi donc allez-vous assister ? à la transformation de la pénalité. La douce loi du Christ pénétrera enfin le code et  rayonnera à travers. On regardera le crime comme une maladie, et cette maladie aura ses médecins qui remplaceront vos juges, ses hôpitaux qui remplaceront vos bagnes. La liberté et la santé se ressembleront. On versera le baume et l’huile où l’on appliquait le fer et le feu. On traitera par la charité ce mal qu’on traitait par la colère. Ce sera simple et sublime. La croix substituée au gibet. Voilà tout.\par

\dateline{15 mars 1832.}
 \section[{Une comédie a propos d’une tragédie}]{Une comédie a propos d’une tragédie\protect\footnotemark }\phantomsection
\label{comedie}\renewcommand{\leftmark}{Une comédie a propos d’une tragédie}

\footnotetext{ \noindent Nous avons cru devoir réimprimer ici l’espèce de préface en dialogue qu’on va lire, et qui accompagnait la quatrième édition du \emph{Dernier Jour d’un condamné.} Il faut se rappeler, en la lisant, au milieu de quelles objections politiques, morales et littéraires les premières éditions de ce livre furent publiées. \emph{(Édition de} 1832).
 }
  \par\textit{PERSONNAGES.}
\begin{itemize} PERSONNAGES.
\item MADAME DE BLINVAL.
\item LE CHEVALIER.
\item ERGASTE.
\item UN POËTE ÉLÉGIAQUE.
\item UN PHILOSOPHE.
\item UN GROS MONSIEUR.
\item UN MONSIEUR MAIGRE.
\item DES FEMMES.
\item UN LAQUAIS.\end{itemize}  \textit{— Un salon —}\par
 
\labelblock{{\name UN POËTE ÉLÉGIAQUE}, lisant.}


\astertri


\astertri


\begin{verse}
Le lendemain, des pas traversaient la forêt,\\
Un chien le long du fleuve en aboyant errait ;\\
Et quand la bachelette en larmes\\
Revint s’asseoir, le cœur rempli d’alarmes,\\
Sur la tant vieille tour de l’antique châtel,\\
Elle entendit les flots gémir, la triste Isaure,\\
Mais plus n’entendit la mandore\\
Du gentil ménestrel !\\
\end{verse}

\labelblock{{\name TOUT L’AUDITOIRE}.}

\noindent Bravo ! charmant ! ravissant !\par
\textit{On bat des mains.}\par

\labelblock{{\name MADAME DE BLINVAL}.}

\noindent Il y a dans cette fin un mystère indéfinissable qui tire les larmes des yeux.\par

\labelblock{{\name LE POËTE ÉLÉGIAQUE}, modestement.}

\noindent La catastrophe est voilée.\par
 
\labelblock{{\name LE CHEVALIER}, hochant la tête.}

\noindent \emph{Mandore, ménestrel}, c’est du romantique ça !\par

\labelblock{{\name LE POËTE ÉLÉGIAQUE}.}

\noindent Oui, monsieur, mais du romantique raisonnable, du vrai romantique. Que voulez-vous ? il faut bien faire quelques concessions.\par

\labelblock{{\name LE CHEVALIER}.}

\noindent Des concessions ! des concessions ! c’est comme cela qu’on perd le goût. Je donnerais tous les vers romantiques seulement pour ce quatrain :\par


\begin{verse}
De par le Pinde et par Cythère,\\
Gentil-Bernard est averti\\
Que l’Art d’Aimer doit samedi\\
Venir souper chez l’Art de Plaire.\\
\end{verse}

\noindent Voilà la vraie poésie ! L’\emph{Art d’Aimer qui soupe samedi chez l’Art de Plaire !} à la bonne heure ! Mais aujourd’hui c’est \emph{la mandore, le ménestrel.} On ne fait plus de \emph{poésies fugitives.} Si j’étais poëte, je ferais des \emph{poésies fugitives ;} mais je ne suis pas poëte, moi.\par

\labelblock{{\name LE POËTE ÉLÉGIAQUE}.}

\noindent Cependant, les élégies…\par

\labelblock{{\name LE CHEVALIER}.}

\noindent \emph{Poésies fugitives}, monsieur. \textit{Bas à M\textsuperscript{me} de Blinval :} Et puis, \emph{châtel} n’est pas français ; on dit \emph{castel}.\par

\labelblock{{\name QUELQU’UN}, au poëte élégiaque.}

\noindent Une observation, monsieur. Vous dites l’\emph{antique} châtel, pourquoi pas le \emph{gothique} ?\par
 
\labelblock{{\name LE POËTE ÉLÉGIAQUE}.}

\noindent \emph{Gothique} ne se dit pas en vers.\par

\labelblock{{\name LE QUELQU’UN}.}

\noindent Ah ! c’est différent.\par

\labelblock{{\name LE POËTE ÉLÉGIAQUE}, poursuivant.}

\noindent Voyez-vous bien, monsieur, il faut se borner. Je ne suis pas de ceux qui veulent désorganiser le vers français, et nous ramener à l’époque des Ronsard et des Brébeuf. Je suis romantique, mais modéré. C’est comme pour les émotions. Je les veux douces, rêveuses, mélancoliques, mais jamais de sang, jamais d’horreurs. Voiler les catastrophes. Je sais qu’il y a des gens, des fous, des imaginations en délire qui… Tenez, mesdames, avez-vous lu le nouveau roman ?\par

\labelblock{{\name LES DAMES}.}

\noindent Quel roman ?\par

\labelblock{{\name LE POËTE ÉLÉGIAQUE}.}

\noindent \emph{Le Dernier Jour}…\par

\labelblock{{\name UN GROS MONSIEUR}.}

\noindent Assez, monsieur ! je sais ce que vous voulez dire. Le titre seul me fait mal aux nerfs.\par

\labelblock{{\name MADAME DE BLINVAL}.}

\noindent Et à moi aussi. C’est un livre affreux. Je l’ai là.\par

\labelblock{{\name LES DAMES}.}

\noindent Voyons, voyons.\par
\textit{On se passe le livre de main en main.}\par
 
\labelblock{{\name QUELQU’UN}, lisant.}

\noindent \emph{Le Dernier Jour d’un}…\par

\labelblock{{\name LE GROS MONSIEUR}.}

\noindent Grâce, madame !\par

\labelblock{{\name MADAME DE BLINVAL}.}

\noindent En effet, c’est un livre abominable, un livre qui donne le cauchemar, un livre qui rend malade.\par

\labelblock{{\name UNE FEMME}, bas.}

\noindent Il faudra que je lise cela.\par

\labelblock{{\name LE GROS MONSIEUR}.}

\noindent Il faut convenir que les mœurs vont se dépravant de jour en jour. Mon Dieu, l’horrible idée ! développer, creuser, analyser, l’une après l’autre, et sans en passer une seule, toutes les souffrances physiques, toutes les tortures morales que doit éprouver un homme condamné à mort, le jour de l’exécution ! Cela n’est-il pas atroce ? Comprenez-vous, mesdames, qu’il se soit trouvé un écrivain pour cette idée, et un public pour cet écrivain ?\par

\labelblock{{\name LE CHEVALIER}.}

\noindent Voilà en effet qui est souverainement impertinent.\par

\labelblock{{\name MADAME DE BLINVAL}.}

\noindent Qu’est-ce que c’est que l’auteur ?\par

\labelblock{{\name LE GROS MONSIEUR}.}

\noindent Il n’y avait pas de nom à la première édition.\par

\labelblock{{\name LE POËTE ÉLÉGIAQUE}.}

\noindent C’est le même qui a déjà fait deux autres romans,  ma foi, j’ai oublié les titres. Le premier commence à la Morgue et finit à la Grève. A chaque chapitre, il y a un ogre qui mange un enfant.\par

\labelblock{{\name LE GROS MONSIEUR}.}

\noindent Vous avez lu cela, monsieur ?\par

\labelblock{{\name LE POËTE ÉLÉGIAQUE}.}

\noindent Oui, monsieur ; la scène se passe en Islande.\par

\labelblock{{\name LE GROS MONSIEUR}.}

\noindent En Islande, c’est épouvantable !\par

\labelblock{{\name LE POËTE ÉLÉGIAQUE}.}

\noindent Il a fait en outre des odes, des ballades, je ne sais quoi, où il y a des monstres qui ont des \emph{corps bleus.}\par

\labelblock{{\name LE CHEVALIER}, riant.}

\noindent Corbleu ! cela doit faire un furieux vers.\par

\labelblock{{\name LE POËTE ÉLÉGIAQUE}.}

\noindent Il a publié aussi un drame, — on appelle cela un drame, — où l’on trouve ce beau vers :\par

Demain vingt-cinq juin mil six cent cinquante-sept.\\


\labelblock{{\name QUELQU’UN}.}

\noindent Ah, ce vers !\par

\labelblock{{\name LE POËTE ÉLÉGIAQUE}.}

\noindent Cela peut s’écrire en chiffres, voyez-vous, mesdames :\par

Demain, 25 juin 1657.\\

\textit{Il rit. On rit.}\par

\labelblock{{\name LE CHEVALIER}.}

\noindent C’est une chose particulière que la poésie d’à présent.\par
 
\labelblock{{\name LE GROS MONSIEUR}.}

\noindent Ah çà ! il ne sait pas versifier, cet homme-là ! Comment donc s’appelle-t-il déjà ?\par

\labelblock{{\name LE POËTE ÉLÉGIAQUE}.}

\noindent Il a un nom aussi difficile à retenir qu’à prononcer. Il y a du goth, du visigoth, de l’ostrogoth dedans.\par
\textit{Il rit.}\par

\labelblock{{\name MADAME DE BLINVAL}.}

\noindent C’est un vilain homme.\par

\labelblock{{\name LE GROS MONSIEUR}.}

\noindent Un abominable homme.\par

\labelblock{{\name UNE JEUNE FEMME}.}

\noindent Quelqu’un qui le connaît m’a dit…\par

\labelblock{{\name LE GROS MONSIEUR}.}

\noindent Vous connaissez quelqu’un qui le connaît ?\par

\labelblock{{\name LA JEUNE FEMME}.}

\noindent Oui, et qui dit que c’est un homme doux, simple, qui vit dans la retraite, et passe ses journées à jouer avec ses petits enfants.\par

\labelblock{{\name LE POËTE}.}

\noindent Et ses nuits à rêver des œuvres de ténèbres. — C’est singulier ; voilà un vers que j’ai fait tout naturellement. Mais c’est qu’il y est, le vers :\par

Et ses nuits à rêver des œuvres de ténèbres.\\

\noindent Avec une bonne césure. Il n’y a plus que l’autre rime à trouver. Pardieu ! \emph{funèbres.}\par

\labelblock{{\name MADAME DE BLINVAL}.}


\emph{Quidquid tentabat dicere, versus erat.}\\

 
\labelblock{{\name LE GROS MONSIEUR}.}

\noindent Vous disiez donc que l’auteur en question a des petits enfants. Impossible, madame. Quand on a fait cet ouvrage-là ! un roman atroce !\par

\labelblock{{\name QUELQU’UN}.}

\noindent Mais, ce roman, dans quel but l’a-t-il fait ?\par

\labelblock{{\name LE POËTE ÉLÉGIAQUE}.}

\noindent Est-ce que je sais, moi ?\par

\labelblock{{\name UN PHILOSOPHE}.}

\noindent A ce qu’il paraît, dans le but de concourir à l’abolition de la peine de mort.\par

\labelblock{{\name LE GROS MONSIEUR}.}

\noindent Une horreur, vous dis-je !\par

\labelblock{{\name LE CHEVALIER}.}

\noindent Ah çà ! c’est donc un duel avec le bourreau ?\par

\labelblock{{\name LE POËTE ÉLÉGIAQUE}.}

\noindent Il en veut terriblement à la guillotine.\par

\labelblock{{\name UN MONSIEUR MAIGRE}.}

\noindent Je vois cela d’ici ; des déclamations.\par

\labelblock{{\name LE GROS MONSIEUR}.}

\noindent Point. Il y a à peine deux pages sur ce texte de la peine de mort. Tout le reste, ce sont des sensations.\par

\labelblock{{\name LE PHILOSOPHE}.}

\noindent Voilà le tort. Le sujet méritait le raisonnement. Un drame, un roman ne prouve rien. Et puis, j’ai lu le livre, et il est mauvais.\par
 
\labelblock{{\name LE POËTE ÉLÉGIAQUE}.}

\noindent Détestable ! Est-ce que c’est là de l’art ? C’est passer les bornes, c’est casser les vitres. Encore, ce criminel, si je le connaissais ? mais point. Qu’a-t-il fait ? on n’en sait rien. C’est peut-être un fort mauvais drôle. On n’a pas le droit de m’intéresser à quelqu’un que je ne connais pas.\par

\labelblock{{\name LE GROS MONSIEUR}.}

\noindent On n’a pas le droit de faire éprouver à son lecteur des souffrances physiques. Quand je vois des tragédies, on se tue, eh bien ! cela ne me fait rien. Mais ce roman, il vous fait dresser les cheveux sur la tête, il vous fait venir la chair de poule, il vous donne de mauvais rêves. J’ai été deux jours au lit pour l’avoir lu.\par

\labelblock{{\name LE PHILOSOPHE}.}

\noindent Ajoutez à cela que c’est un livre froid et compassé.\par

\labelblock{{\name LE POËTE}.}

\noindent Un livre !… un livre !…\par

\labelblock{{\name LE PHILOSOPHE}.}

\noindent Oui. — Et comme vous disiez tout à l’heure, monsieur, ce n’est point là de véritable esthétique. Je ne m’intéresse pas à une abstraction, à une entité pure. Je ne vois point là une personnalité qui s’adéquate avec la mienne. Et puis le style n’est ni simple ni clair. Il sent l’archaïsme. C’est bien là ce que vous disiez, n’est-ce pas ?\par

\labelblock{{\name LE POËTE}.}

\noindent Sans doute, sans doute. Il ne faut pas de personnalités.\par
 
\labelblock{{\name LE PHILOSOPHE}.}

\noindent Le condamné n’est pas intéressant.\par

\labelblock{{\name LE POËTE}.}

\noindent Comment intéresserait-il ? il a un crime et pas de remords. J’eusse fait le contraire. J’eusse conté l’histoire de mon condamné. Né de parents honnêtes. Une bonne éducation. De l’amour. De la jalousie. Un crime qui n’en soit pas un. Et puis des remords, des remords, beaucoup de remords. Mais les lois humaines sont implacables ; il faut qu’il meure. Et là j’aurais traité ma question de la peine de mort. A la bonne heure !\par

\labelblock{{\name MADAME DE BLINVAL}.}

\noindent Ah ! Ah !\par

\labelblock{{\name LE PHILOSOPHE}.}

\noindent Pardon. Le livre, comme l’entend monsieur, ne prouverait rien. La particularité ne régit pas la généralité.\par

\labelblock{{\name LE POËTE}.}

\noindent Eh bien ! mieux encore ; pourquoi n’avoir pas choisi pour héros, par exemple… Malesherbes, le vertueux Malesherbes ? son dernier jour, son supplice ? Oh ! alors, beau et noble spectacle ! J’eusse pleuré, j’eusse frémi, j’eusse voulu monter sur l’échafaud avec lui.\par

\labelblock{{\name LE PHILOSOPHE}.}

\noindent Pas moi.\par

\labelblock{{\name LE CHEVALIER}.}

\noindent Ni moi. C’était un révolutionnaire, au fond, que votre M. de Malesherbes.\par
 
\labelblock{{\name LE PHILOSOPHE}.}

\noindent L’échafaud de Malesherbes ne prouve rien contre la peine de mort en général.\par

\labelblock{{\name LE GROS MONSIEUR}.}

\noindent La peine de mort ! à quoi bon s’occuper de cela ? Qu’est-ce que cela vous fait, la peine de mort ? Il faut que cet auteur soit bien mal né, de venir nous donner le cauchemar à ce sujet avec son livre !\par

\labelblock{{\name MADAME DE BLINVAL}.}

\noindent Ah ! oui, un bien mauvais cœur !\par

\labelblock{{\name LE GROS MONSIEUR}.}

\noindent Il nous force à regarder dans les prisons, dans les bagnes, dans Bicêtre. C’est fort désagréable. On sait bien que ce sont des cloaques ; mais qu’importe à la société ?\par

\labelblock{{\name MADAME DE BLINVAL}.}

\noindent Ceux qui ont fait les lois n’étaient pas des enfants.\par

\labelblock{{\name LE PHILOSOPHE}.}

\noindent Ah, cependant, en présentant les choses avec vérité…\par

\labelblock{{\name LE MONSIEUR MAIGRE}.}

\noindent Eh ! c’est justement ce qui manque, la vérité. Que voulez-vous qu’un poëte sache sur de pareilles matières ? Il faudrait être au moins procureur du roi. Tenez, j’ai lu dans une citation qu’un journal fait de ce livre, que le condamné ne dit rien quand on lui lit son arrêt de mort ; eh bien, moi, j’ai vu un condamné qui, dans ce moment-là, a poussé un grand cri. — Vous voyez.\par
 
\labelblock{{\name LE PHILOSOPHE}.}

\noindent Permettez…\par

\labelblock{{\name LE MONSIEUR MAIGRE}.}

\noindent Tenez, messieurs, la guillotine, la Grève, c’est de mauvais goût ; — et la preuve, c’est qu’il paraît que c’est un livre qui corrompt le goût, et vous rend incapable d’émotions pures, fraîches, naïves. Quand donc se lèveront les défenseurs de la saine littérature ? Je voudrais être, et mes réquisitoires m’en donneraient peut-être le droit, membre de l’académie française… — Voilà justement monsieur Ergaste, qui en est. Que pense-t-il du \emph{Dernier Jour d’un condamné ?}\par

\labelblock{{\name ERGASTE}.}

\noindent Ma foi, monsieur, je ne l’ai lu ni ne le lirai. Je dînais hier chez M\textsuperscript{me} de Sénange, et la marquise de Morival en a parlé au duc de Melcourt. On dit qu’il y a des personnalités contre la magistrature, et surtout contre le président d’Alimont. L’abbé de Floricour aussi était indigné. Il paraît qu’il y a un chapitre contre la religion, et un chapitre contre la monarchie. Si j’étais procureur du roi !…\par

\labelblock{{\name LE CHEVALIER}.}

\noindent Ah bien oui, procureur du roi ! et la charte ! et la liberté de la presse ! Cependant un poëte qui veut supprimer la peine de mort, vous conviendrez que c’est odieux. Ah ! ah ! dans l’ancien régime, quelqu’un qui se serait permis de publier un roman contre la torture !… — Mais depuis la prise de la Bastille on peut tout écrire. Les livres font un mal affreux.\par
 
\labelblock{{\name LE GROS MONSIEUR}.}

\noindent Affreux. — On était tranquille, on ne pensait à rien. Il se coupait bien de temps en temps en France une tête par-ci par-là, deux tout au plus par semaine. Tout cela sans bruit, sans scandale. Ils ne disaient rien. Personne n’y songeait. Pas du tout, voilà un livre… — un livre qui vous donne un mal de tête horrible !\par

\labelblock{{\name LE MONSIEUR MAIGRE}.}

\noindent Le moyen qu’un juré condamne après l’avoir lu !\par

\labelblock{{\name ERGASTE}.}

\noindent Cela trouble les consciences.\par

\labelblock{{\name MADAME DE BLINVAL}.}

\noindent Ah ! les livres ! les livres ! Qui eût dit cela d’un roman ?\par

\labelblock{{\name LE POËTE}.}

\noindent Il est certain que les livres sont bien souvent un poison subversif de l’ordre social.\par

\labelblock{{\name LE MONSIEUR MAIGRE}.}

\noindent Sans compter la langue, que messieurs les romantiques révolutionnent aussi.\par

\labelblock{{\name LE POËTE}.}

\noindent Distinguons, monsieur ; il y a romantiques et romantiques.\par

\labelblock{{\name LE MONSIEUR MAIGRE}.}

\noindent Le mauvais goût, le mauvais goût.\par

\labelblock{{\name ERGASTE}.}

\noindent Vous avez raison. Le mauvais goût.\par
 
\labelblock{{\name LE MONSIEUR MAIGRE}.}

\noindent Il n’y a rien à répondre à cela.\par

\labelblock{{\name LE PHILOSOPHE}, appuyé au fauteuil d’une dame.}

\noindent Ils disent là des choses qu’on ne dit même plus rue Mouffetard.\par

\labelblock{{\name ERGASTE}.}

\noindent Ah ! l’abominable livre !\par

\labelblock{{\name MADAME DE BLINVAL}.}

\noindent Hé ! ne le jetez pas au feu. Il est à la loueuse.\par

\labelblock{{\name LE CHEVALIER}.}

\noindent Parlez-moi de notre temps. Comme tout s’est dépravé depuis, le goût et les mœurs ! Vous souvient-il de notre temps, madame de Blinval ?\par

\labelblock{{\name MADAME DE BLINVAL}.}

\noindent Non, monsieur, il ne m’en souvient pas.\par

\labelblock{{\name LE CHEVALIER}.}

\noindent Nous étions le peuple le plus doux, le plus gai, le plus spirituel. Toujours de belles fêtes, de jolis vers. C’était charmant. Y a-t-il rien de plus galant que le madrigal de M. de La Harpe sur le grand bal que M\textsuperscript{me} la maréchale de Mailly donna en mil sept cent… l’année de l’exécution de Damiens ?\par

\labelblock{{\name LE GROS MONSIEUR}, soupirant.}

\noindent Heureux temps ! Maintenant les mœurs sont horribles, et les livres aussi. C’est le beau vers de Boileau :\par

Et la chute des arts suit la décadence des mœurs.\\

 
\labelblock{{\name LE PHILOSOPHE}, bas au poëte.}

\noindent Soupe-t-on, dans cette maison ?\par

\labelblock{{\name LE POËTE ÉLÉGIAQUE}.}

\noindent Oui, tout à l’heure.\par

\labelblock{{\name LE MONSIEUR MAIGRE}.}

\noindent Maintenant on veut abolir la peine de mort, et pour cela on fait des romans cruels, immoraux et de mauvais goût, \emph{le Dernier Jour d’un condamné}, que sais-je ?\par

\labelblock{{\name LE GROS MONSIEUR}.}

\noindent Tenez, mon cher, ne parlons plus de ce livre atroce ; et, puisque je vous rencontre, dites-moi, que faites-vous de cet homme dont nous avons rejeté le pourvoi depuis trois semaines ?\par

\labelblock{{\name LE MONSIEUR MAIGRE.}}

\noindent Ah ! un peu de patience ! je suis en congé ici. Laissez-moi respirer. A mon retour. Si cela tarde trop pourtant, j’écrirai à mon substitut…\par

\labelblock{{\name UN LAQUAIS}, entrant.}

\noindent Madame est servie.
 \section[{I}]{I}\phantomsection
\label{ch1}\renewcommand{\leftmark}{I}


\dateline{Bicêtre}
\noindent Condamné à mort !\par
Voilà cinq semaines que j’habite avec cette pensée, toujours seul avec elle, toujours glacé de sa présence, toujours courbé sous son poids !\par
Autrefois, car il me semble qu’il y a plutôt des années que des semaines, j’étais un homme comme un autre homme. Chaque jour, chaque heure, chaque  minute avait son idée. Mon esprit, jeune et riche, était plein de fantaisies. Il s’amusait à me les dérouler les unes après les autres, sans ordre et sans fin, brodant d’inépuisables arabesques cette rude et mince étoffe de la vie. C’étaient des jeunes filles, de splendides chapes d’évêque, des batailles gagnées, des théâtres pleins de bruit et de lumière, et puis encore des jeunes filles et de sombres promenades la nuit sous les larges bras des marronniers. C’était toujours fête dans mon imagination. Je pouvais penser à ce que je voulais, j’étais libre.\par
Maintenant je suis captif. Mon corps est aux fers dans un cachot, mon esprit est en prison dans une idée. Une horrible, une sanglante, une implacable idée ! Je n’ai plus qu’une pensée, qu’une conviction, qu’une certitude : condamné à mort !\par
Quoi que je fasse, elle est toujours là, cette pensée infernale, comme un spectre de plomb à mes côtés, seule et jalouse, chassant toute distraction, face à face avec moi misérable, et me secouant de ses deux mains de glace quand je veux détourner la tête ou fermer les yeux. Elle se glisse sous toutes les formes où mon esprit voudrait la fuir, se mêle comme un refrain horrible à toutes les paroles qu’on m’adresse, se colle avec moi aux grilles hideuses de mon cachot, m’obsède éveillé, épie mon sommeil convulsif, et reparaît dans mes rêves sous la forme d’un couteau.\par
Je viens de m’éveiller en sursaut, poursuivi par elle et me disant : — Ah ! ce n’est qu’un rêve ! — Eh bien ! avant même que mes yeux lourds aient eu le temps de  s’entr’ouvrir assez pour voir cette fatale pensée écrite dans l’horrible réalité qui m’entoure, sur la dalle mouillée et suante de ma cellule, dans les rayons pâles de ma lampe de nuit, dans la trame grossière de la toile de mes vêtements, sur la sombre figure du soldat de garde dont la giberne reluit à travers la grille du cachot, il me semble que déjà une voix a murmuré à mon oreille : — Condamné à mort !
 \section[{II}]{II}\phantomsection
\label{ch2}\renewcommand{\leftmark}{II}

\noindent C’était par une belle matinée d’août.\par
Il y avait trois jours que mon procès était entamé ; trois jours que mon nom et mon crime ralliaient chaque matin une nuée de spectateurs, qui venaient s’abattre sur les bancs de la salle d’audience comme des corbeaux autour d’un cadavre ; trois jours que toute cette fantasmagorie des juges, des témoins, des avocats, des procureurs du roi, passait et repassait devant moi, tantôt grotesque, tantôt sanglante, toujours sombre et fatale. Les deux premières nuits, d’inquiétude et de terreur, je n’en avais pu dormir ; la troisième, j’en avais dormi d’ennui et de fatigue. A minuit, j’avais laissé les jurés délibérant. On m’avait ramené sur la paille de mon cachot, et j’étais tombé sur-le-champ dans un sommeil profond, dans un sommeil d’oubli. C’étaient les premières heures de repos depuis bien des jours.\par
J’étais encore au plus profond de ce profond  sommeil lorsqu’on vint me réveiller. Cette fois il ne suffit point du pas lourd et des souliers ferrés du guichetier, du cliquetis de son nœud de clefs, du grincement rauque des verrous ; il fallut pour me tirer de ma léthargie sa rude voix à mon oreille et sa main rude sur mon bras. — Levez-vous donc ! — J’ouvris les yeux, je me dressai effaré sur mon séant. En ce moment, par l’étroite et haute fenêtre de ma cellule, je vis au plafond du corridor voisin, seul ciel qu’il me fût donné d’entrevoir, ce reflet jaune où des yeux habitués aux ténèbres d’une prison savent si bien reconnaître le soleil. J’aime le soleil.\par
— Il fait beau, dis-je au guichetier.\par
Il resta un moment sans me répondre, comme ne sachant si cela valait la peine de dépenser une parole ; puis avec quelque effort il murmura brusquement :\par
— C’est possible.\par
Je demeurais immobile, l’esprit à demi endormi, la bouche souriante, l’œil fixé sur cette douce réverbération dorée qui diaprait le plafond.\par
— Voilà une belle journée, répétai-je.\par
— Oui, me répondit l’homme, on vous attend.\par
Ce peu de mots, comme le fil qui rompt le vol de l’insecte, me rejeta violemment dans la réalité. Je revis soudain, comme dans la lumière d’un éclair, la sombre salle des assises, le fer à cheval des juges chargé de haillons ensanglantés, les trois rangs de témoins aux faces stupides, les deux gendarmes aux deux bouts de mon banc, et les robes noires s’agiter, et les têtes de la foule fourmiller au fond dans l’ombre, et s’arrêter  sur moi le regard fixe de ces douze jurés, qui avaient veillé pendant que je dormais !\par
Je me levai ; mes dents claquaient, mes mains tremblaient et ne savaient où trouver mes vêtements, mes jambes étaient faibles. Au premier pas que je fis, je trébuchai comme un portefaix trop chargé. Cependant je suivis le geôlier.\par
Les deux gendarmes m’attendaient au seuil de la cellule. On me remit les menottes. Cela avait une petite serrure compliquée qu’ils fermèrent avec soin. Je laissai faire ; c’était une machine sur une machine.\par
Nous traversâmes une cour intérieure. L’air vif du matin me ranima. Je levai la tête. Le ciel était bleu, et les rayons chauds du soleil, découpés par les longues cheminées, traçaient de grands angles de lumière au faîte des murs hauts et sombres de la prison. Il faisait beau en effet.\par
Nous montâmes un escalier tournant en vis ; nous passâmes un corridor, puis un autre, puis un troisième ; puis une porte basse s’ouvrit. Un air chaud, mêlé de bruit, vint me frapper au visage ; c’était le souffle de la foule dans la salle des assises. J’entrai.\par
Il y eut à mon apparition une rumeur d’armes et de voix. Les banquettes se déplacèrent bruyamment, les cloisons craquèrent ; et, pendant que je traversais la longue salle entre deux masses de peuple murées de soldats, il me semblait que j’étais le centre auquel se rattachaient les fils qui faisaient mouvoir toutes ces faces béantes et penchées.\par
En cet instant je m’aperçus que j’étais sans fers ;  mais je ne pus me rappeler où ni quand on me les avait ôtés.\par
Alors il se fit un grand silence. J’étais parvenu à ma place. Au moment où le tumulte cessa dans la foule, il cessa aussi dans mes idées. Je compris tout à coup clairement ce que je n’avais fait qu’entrevoir confusément jusqu’alors, que le moment décisif était venu, et que j’étais là pour entendre ma sentence.\par
L’explique qui pourra, de la manière dont cette idée me vint elle ne me causa pas de terreur. Les fenêtres étaient ouvertes ; l’air et le bruit de la ville arrivaient librement du dehors ; la salle était claire comme pour une noce ; les gais rayons du soleil traçaient çà et là la figure lumineuse des croisées, tantôt allongée sur le plancher, tantôt développée sur les tables, tantôt brisée à l’angle des murs ; et de ces losanges éclatants aux fenêtres chaque rayon découpait dans l’air un grand prisme de poussière d’or.\par
Les juges, au fond de la salle, avaient l’air satisfait, probablement de la joie d’avoir bientôt fini. Le visage du président, doucement éclairé par le reflet d’une vitre, avait quelque chose de calme et de bon ; et un jeune assesseur causait presque gaiement en chiffonnant son rabat avec une jolie dame en chapeau rose, placée par faveur derrière lui.\par
Les jurés seuls paraissaient blêmes et abattus, mais c’était apparemment de fatigue d’avoir veillé toute la nuit. Quelques-uns bâillaient. Rien, dans leur contenance, n’annonçait des hommes qui viennent de porter une sentence de mort ; et sur les figures de ces bons  bourgeois je ne devinais qu’une grande envie de dormir.\par
En face de moi une fenêtre était toute grande ouverte. J’entendais rire sur le quai des marchandes de fleurs ; et, au bord de la croisée, une jolie petite plante jaune, toute pénétrée d’un rayon de soleil, jouait avec le vent dans une fente de la pierre.\par
Comment une idée sinistre aurait-elle pu poindre parmi tant de gracieuses sensations ? Inondé d’air et de soleil, il me fut impossible de penser à autre chose qu’à la liberté ; l’espérance vint rayonner en moi comme le jour autour de moi ; et, confiant, j’attendis ma sentence comme on attend la délivrance et la vie.\par
Cependant mon avocat arriva. On l’attendait. Il venait de déjeuner copieusement et de bon appétit. Parvenu à sa place, il se pencha vers moi avec un sourire.\par
— J’espère, me dit-il.\par
— N’est-ce pas ? répondis-je, léger et souriant aussi.\par
— Oui, reprit-il ; je ne sais rien encore de leur déclaration, mais ils auront sans doute écarté la préméditation, et alors ce ne sera que les travaux forcés à perpétuité.\par
— Que dites-vous là, monsieur ? répliquai-je indigné ; plutôt cent fois la mort !\par
Oui, la mort ! — Et d’ailleurs, me répétait je ne sais quelle voix intérieure, qu’est-ce que je risque à dire cela ? A-t-on jamais prononcé sentence de mort autrement qu’à minuit, aux flambeaux, dans une salle  sombre et noire, et par une froide nuit de pluie d’hiver ? Mais au mois d’août, à huit heures du matin, un si beau jour, ces bons jurés, c’est impossible ! Et mes yeux revenaient se fixer sur la jolie fleur jaune au soleil.\par
Tout à coup le président, qui n’attendait que l’avocat, m’invita à me lever. La troupe porta les armes ; comme par un mouvement électrique, toute l’assemblée fut debout au même instant. Une figure insignifiante et nulle, placée à une table au-dessous du tribunal, c’était, je pense, le greffier, prit la parole, et lut le verdict que les jurés avaient prononcé en mon absence. Une sueur froide sortit de tous mes membres ; je m’appuyai au mur pour ne pas tomber.\par
— Avocat, avez-vous quelque chose à dire sur l’application de la peine ? demanda le président.\par
J’aurais eu, moi, tout à dire, mais rien ne me vint. Ma langue resta collée à mon palais.\par
Le défenseur se leva.\par
Je compris qu’il cherchait à atténuer la déclaration du jury, et à mettre dessous, au lieu de la peine qu’elle provoquait, l’autre peine, celle que j’avais été si blessé de lui voir espérer.\par
Il fallut que l’indignation fût bien forte, pour se faire jour à travers les mille émotions qui se disputaient ma pensée. Je voulus répéter à haute voix ce que je lui avais déjà dit : Plutôt cent fois la mort ! Mais l’haleine me manqua, et je ne pus que l’arrêter rudement par le bras, en criant avec une force convulsive : Non !\par
 Le procureur général combattit l’avocat, et je l’écoutai avec une satisfaction stupide. Puis les juges sortirent, puis ils rentrèrent, et le président me lut mon arrêt.\par
— Condamné à mort ! dit la foule ; et, tandis qu’on m’emmenait, tout ce peuple se rua sur mes pas avec le fracas d’un édifice qui se démolit. Moi je marchais, ivre et stupéfait. Une révolution venait de se faire en moi. Jusqu’à l’arrêt de mort, je m’étais senti respirer, palpiter, vivre dans le même milieu que les autres hommes ; maintenant je distinguais clairement comme une clôture entre le monde et moi. Rien ne m’apparaissait plus sous le même aspect qu’auparavant. Ces larges fenêtres lumineuses, ce beau soleil, ce ciel pur, cette jolie fleur, tout cela était blanc et pâle, de la couleur d’un linceul. Ces hommes, ces femmes, ces enfants qui se pressaient sur mon passage, je leur trouvais des airs de fantômes.\par
Au bas de l’escalier, une noire et sale voiture grillée m’attendait. Au moment d’y monter, je regardai au hasard dans la place. — Un condamné à mort ! criaient les passants en courant vers la voiture. A travers le nuage qui me semblait s’être interposé entre les choses et moi, je distinguai deux jeunes filles qui me suivaient avec des yeux avides. — Bon, dit la plus jeune en battant des mains, ce sera dans six semaines !
 \section[{III}]{III}\phantomsection
\label{ch3}\renewcommand{\leftmark}{III}

\noindent Condamné à mort !\par
Eh bien, pourquoi non ? \emph{Les hommes}, je me rappelle l’avoir lu dans je ne sais quel livre où il n’y avait que cela de bon, \emph{les hommes sont tous condamnés à mort avec des sursis indéfinis}. Qu’y a-t-il donc de si changé à ma situation ?\par
Depuis l’heure où mon arrêt m’a été prononcé, combien sont morts qui s’arrangeaient pour une longue vie ! Combien m’ont devancé qui, jeunes, libres et sains, comptaient bien aller voir tel jour tomber ma tête en place de Grève ! Combien d’ici là peut-être qui marchent et respirent au grand air, entrent et sortent à leur gré, et qui me devanceront encore !\par
Et puis, qu’est-ce que la vie a donc de si regrettable pour moi ? En vérité, le jour sombre et le pain noir du cachot, la portion de bouillon maigre puisée au baquet des galériens, être rudoyé, moi qui suis  raffiné par l’éducation, être brutalisé des guichetiers et des gardes-chiourme, ne pas voir un être humain qui me croie digne d’une parole et à qui je le rende, sans cesse tressaillir et de ce que j’ai fait et de ce qu’on me fera ; voilà à peu près les seuls biens que puisse m’enlever le bourreau.\par
Ah ! n’importe, c’est horrible !
 \section[{IV}]{IV}\phantomsection
\label{ch4}\renewcommand{\leftmark}{IV}

\noindent La voiture noire me transporta ici, dans ce hideux Bicêtre.\par
Vu de loin, cet édifice a quelque majesté. Il se déroule à l’horizon, au front d’une colline, et à distance garde quelque chose de son ancienne splendeur, un air de château de roi. Mais à mesure que vous approchez, le palais devient masure. Les pignons dégradés blessent l’œil. Je ne sais quoi de honteux et d’appauvri salit ces royales façades ; on dirait que les murs ont une lèpre. Plus de vitres, plus de glaces aux fenêtres ; mais de massifs barreaux de fer entre-croisés, auxquels se colle çà et là quelque hâve figure d’un galérien ou d’un fou.\par
C’est la vie vue de près.
 \section[{V}]{V}\phantomsection
\label{ch5}\renewcommand{\leftmark}{V}

\noindent A peine arrivé, des mains de fer s’emparèrent de moi. On multiplia les précautions ; point de couteau, point de fourchette pour mes repas ; la \emph{camisole de force}, une espèce de sac de toile à voilure, emprisonna mes bras ; on répondait de ma vie. Je m’étais pourvu en cassation. On pouvait avoir pour six ou sept semaines cette affaire onéreuse, et il importait de me conserver sain et sauf à la place de Grève.\par
Les premiers jours on me traita avec une douceur qui m’était horrible. Les égards d’un guichetier sentent l’échafaud. Par bonheur, au bout de peu de jours, l’habitude reprit le dessus ; ils me confondirent avec les autres prisonniers dans une commune brutalité, et n’eurent plus de ces distinctions inaccoutumées de politesse qui me remettaient sans cesse le bourreau sous les yeux. Ce ne fut pas la seule amélioration. Ma jeunesse, ma docilité, les soins de l’aumônier de la prison, et surtout quelques mots en latin que j’adressai au concierge, qui ne les comprit pas, m’ouvrirent la promenade une fois par semaine avec les autres détenus, et firent disparaître la camisole où j’étais paralysé. Après bien des hésitations, on m’a aussi donné de l’encre, du papier, des plumes, et une lampe de nuit.\par
 Tous les dimanches, après la messe, on me lâche dans le préau, à l’heure de la récréation. Là, je cause avec les détenus ; il le faut bien. Ils sont bonnes gens, les misérables. Ils me content leurs tours, ce serait à faire horreur ; mais je sais qu’ils se vantent. Ils m’apprennent à parler argot, à \emph{rouscailler bigorne}, comme ils disent. C’est toute une langue entée sur la langue générale comme une espèce d’excroissance hideuse, comme une verrue. Quelquefois une énergie singulière, un pittoresque effrayant : \emph{il y a du raisiné sur le trimar }(du sang sur le chemin), \emph{épouser la veuve} (être pendu), comme si la corde du gibet était veuve de tous les pendus. La tête d’un voleur a deux noms : \emph{la sorbonne}, quand elle médite, raisonne et conseille le crime ; \emph{la tronche,} quand le bourreau la coupe. Quelquefois de l’esprit de vaudeville : un \emph{cachemire d’osier} (une hotte de chiffonnier), \emph{la menteuse} (la langue) ; et puis partout, à chaque instant, des mots bizarres, mystérieux, laids et sordides, venus on ne sait d’où : \emph{le taule} (le bourreau), \emph{la cône} (la mort), \emph{la placarde} (la place des exécutions). On dirait des crapauds et des araignées. Quand on entend parler cette langue, cela fait l’effet de quelque chose de sale et de poudreux, d’une liasse de haillons que l’on secouerait devant vous.\par
Du moins ces hommes-là me plaignent, ils sont les seuls. Les geôliers, les guichetiers, les porte-clefs, — je ne leur en veux pas, — causent et rient, et parlent de moi, devant moi, comme d’une chose.
 \section[{VI}]{VI}\phantomsection
\label{ch6}\renewcommand{\leftmark}{VI}

\noindent Je me suis dit :\par
— Puisque j’ai le moyen d’écrire, pourquoi ne le ferais-je pas ? Mais quoi écrire ? Pris entre quatre murailles de pierre nue et froide, sans liberté pour mes pas, sans horizon pour mes yeux, pour unique distraction machinalement occupé tout le jour à suivre la marche lente de ce carré blanchâtre que le judas de ma porte découpe vis-à-vis sur le mur sombre, et, comme je le disais tout à l’heure, seul à seul avec une idée, une idée de crime et de châtiment, de meurtre et de mort ! est-ce que je puis avoir quelque chose à dire, moi qui n’ai plus rien à faire dans ce monde ? Et que trouverai-je dans ce cerveau flétri et vide qui vaille la peine d’être écrit ?\par
Pourquoi non ? Si tout, autour de moi, est monotone et décoloré, n’y a-t-il pas en moi une tempête, une lutte, une tragédie ? Cette idée fixe qui me possède ne se présente-t-elle pas à moi à chaque heure, à chaque instant, sous une nouvelle forme, toujours  plus hideuse et plus ensanglantée à mesure que le terme approche ? Pourquoi n’essayerais-je pas de me dire à moi-même tout ce que j’éprouve de violent et d’inconnu dans la situation abandonnée où me voilà ? Certes, la matière est riche ; et, si abrégée que soit ma vie, il y aura bien encore dans les angoisses, dans les terreurs, dans les tortures qui la rempliront, de cette heure à la dernière, de quoi user cette plume et tarir cet encrier. — D’ailleurs ces angoisses, le seul moyen d’en moins souffrir, c’est de les observer, et les peindre m’en distraira.\par
Et puis, ce que j’écrirai ainsi ne sera peut-être pas inutile. Ce journal de mes souffrances, heure par heure, minute par minute, supplice par supplice, si j’ai la force de le mener jusqu’au moment où il me sera \emph{physiquement} impossible de continuer, cette histoire, nécessairement inachevée, mais aussi complète que possible, de mes sensations, ne portera-t-elle point avec elle un grand et profond enseignement ? N’y aurait-il pas dans ce procès-verbal de la pensée agonisante, dans cette progression toujours croissante de douleurs, dans cette espèce d’autopsie intellectuelle d’un condamné, plus d’une leçon pour ceux qui condamnent ? Peut-être cette lecture leur rendra-t-elle la main moins légère, quand il s’agira quelque autre fois de jeter une tête qui pense, une tête d’homme, dans ce qu’ils appellent la balance de la justice ? Peut-être n’ont-ils jamais réfléchi, les malheureux, à cette lente succession de tortures que renferme la formule expéditive d’un arrêt de mort ? Se sont-ils  jamais seulement arrêtés à cette idée poignante que dans l’homme qu’ils retranchent il y a une intelligence, une intelligence qui avait compté sur la vie, une âme qui ne s’est point disposée pour la mort ? Non. Ils ne voient dans tout cela que la chute verticale d’un couteau triangulaire, et pensent sans doute que, pour le condamné, il n’y a rien avant, rien après.\par
Ces feuilles les détromperont. Publiées peut-être un jour, elles arrêteront quelques moments leur esprit sur les souffrances de l’esprit ; car ce sont celles-là qu’ils ne soupçonnent pas. Ils sont triomphants de pouvoir tuer sans presque faire souffrir le corps. Eh ! c’est bien de cela qu’il s’agit ! Qu’est-ce que la douleur physique près de la douleur morale ! Horreur et pitié, des lois faites ainsi ! Un jour viendra, et peut-être ces mémoires, derniers confidents d’un misérable, y auront-ils contribué… — \par
A moins qu’après ma mort le vent ne joue dans le préau avec ces morceaux de papier souillés de boue, ou qu’ils n’aillent pourrir à la pluie, collés en étoiles à la vitre cassée d’un guichetier.
 \section[{VII}]{VII}\phantomsection
\label{ch7}\renewcommand{\leftmark}{VII}

\noindent Que ce que j’écris ici puisse être un jour utile à d’autres, que cela arrête le juge prêt à juger, que cela sauve des malheureux, innocents ou coupables, de l’agonie à laquelle je suis condamné, pourquoi ? à quoi bon ? qu’importe ? Quand ma tête aura été coupée, qu’est-ce que cela me fait qu’on en coupe d’autres ? Est-ce que vraiment j’ai pu penser ces folies ? Jeter bas l’échafaud après que j’y aurai monté ! je vous demande un peu ce qui m’en reviendra.\par
Quoi ! le soleil, le printemps, les champs pleins de fleurs, les oiseaux qui s’éveillent le matin, les nuages, les arbres, la nature, la liberté, la vie, tout cela n’est plus à moi ?\par
Ah ! c’est moi qu’il faudrait sauver ! — Est-il bien vrai que cela ne se peut, qu’il faudra mourir demain, aujourd’hui peut-être, que cela est ainsi ? O Dieu ! l’horrible idée à se briser la tête au mur de son cachot !
 \section[{VIII}]{VIII}\phantomsection
\label{ch8}\renewcommand{\leftmark}{VIII}

\noindent Comptons ce qui me reste.\par
Trois jours de délai après l’arrêt prononcé pour le pourvoi en cassation.\par
Huit jours d’oubli au parquet de la cour d’assises, après quoi les \emph{pièces,} comme ils disent, sont envoyées au ministre.\par
Quinze jours d’attente chez le ministre, qui ne sait seulement pas qu’elles existent, et qui, cependant, est supposé les transmettre, après examen, à la cour de cassation.\par
Là, classement, numérotage, enregistrement ; car la guillotine est encombrée, et chacun ne doit passer qu’à son tour.\par
Quinze jours pour veiller à ce qu’il ne vous soit pas fait de passe-droit.\par
Enfin la cour s’assemble, d’ordinaire un jeudi, rejette vingt pourvois en masse, et renvoie le tout au ministre, qui renvoie au procureur général, qui renvoie au bourreau. Trois jours.\par
 Le matin du quatrième jour, le substitut du procureur général se dit, en mettant sa cravate : — Il faut pourtant que cette affaire finisse. — Alors, si le substitut du greffier n’a pas quelque déjeuner d’amis qui l’en empêche, l’ordre d’exécution est minuté, rédigé, mis au net, expédié, et le lendemain dès l’aube on entend dans la place de Grève clouer une charpente, et dans les carrefours hurler à pleine voix des crieurs enroués.\par
En tout six semaines. La petite fille avait raison.\par
Or, voilà cinq semaines au moins, six peut-être, je n’ose compter, que je suis dans ce cabanon de Bicêtre, et il me semble qu’il y a trois jours, c’était jeudi.
 \section[{IX}]{IX}\phantomsection
\label{ch9}\renewcommand{\leftmark}{IX}

\noindent Je viens de faire mon testament.\par
A quoi bon ? Je suis condamné aux frais, et tout ce que j’ai y suffira à peine. La guillotine, c’est fort cher.\par
Je laisse une mère, je laisse une femme, je laisse un enfant.\par
Une petite fille de trois ans, douce, rose, frêle, avec de grands yeux noirs et de longs cheveux châtains.\par
Elle avait deux ans et un mois quand je l’ai vue pour la dernière fois.\par
Ainsi, après ma mort, trois femmes sans fils, sans mari, sans père ; trois orphelines de différente espèce ; trois veuves du fait de la loi.\par
J’admets que je sois justement puni ; ces innocentes, qu’ont-elles fait ? N’importe ; on les déshonore, on les ruine ; c’est la justice.\par
Ce n’est pas que ma pauvre vieille mère m’inquiète ; elle a soixante-quatre ans, elle mourra du coup. Ou  si elle va quelques jours encore, pourvu que jusqu’au dernier moment elle ait un peu de cendre chaude dans sa chaufferette, elle ne dira rien.\par
Ma femme ne m’inquiète pas non plus ; elle est déjà d’une mauvaise santé et d’un esprit faible, elle mourra aussi.\par
A moins qu’elle ne devienne folle. On dit que cela fait vivre ; mais du moins, l’intelligence ne souffre pas ; elle dort, elle est comme morte.\par
Mais ma fille, mon enfant, ma pauvre petite Marie, qui rit, qui joue, qui chante à cette heure, et ne pense à rien, c’est celle-là qui me fait mal !
 \section[{X}]{X}\phantomsection
\label{ch10}\renewcommand{\leftmark}{X}

\noindent Voici ce que c’est que mon cachot :\par
Huit pieds carrés ; quatre murailles de pierre de taille qui s’appuient à angle droit sur un pavé de dalles exhaussé d’un degré au-dessus du corridor extérieur.\par
A droite de la porte, en entrant, une espèce d’enfoncement qui fait la dérision d’une alcôve. On y jette une botte de paille où le prisonnier est censé reposer et dormir, vêtu d’un pantalon de toile et d’une veste de coutil, hiver comme été.\par
Au-dessus de ma tête, en guise de ciel, une noire voûte en \emph{ogive} — c’est ainsi que cela s’appelle — à laquelle d’épaisses toiles d’araignée pendent comme des haillons.\par
Du reste, pas de fenêtres, pas même de soupirail ; une porte où le fer cache le bois.\par
Je me trompe ; au centre de la porte, vers le haut, une ouverture de neuf pouces carrés, coupée  d’une grille en croix, et que le guichetier peut fermer la nuit.\par
Au dehors, un assez long corridor, éclairé, aéré au moyen de soupiraux étroits au haut du mur, et divisé en compartiments de maçonnerie qui communiquent entre eux par une série de portes cintrées et basses ; chacun de ces compartiments sert en quelque sorte d’antichambre à un cachot pareil au mien. C’est dans ces cachots que l’on met les forçats condamnés par le directeur de la prison à des peines de discipline. Les trois premiers cabanons sont réservés aux condamnés à mort, parce qu’étant plus voisins de la geôle, ils sont plus commodes pour le geôlier.\par
Ces cachots sont tout ce qui reste de l’ancien château de Bicêtre tel qu’il fut bâti, dans le quinzième siècle, par le cardinal de Winchester, le même qui fit brûler Jeanne d’Arc. J’ai entendu dire cela à des \emph{curieux} qui sont venus me voir l’autre jour dans ma loge, et qui me regardaient à distance comme une bête de la ménagerie. Le guichetier a eu cent sous.\par
J’oubliais de dire qu’il y a nuit et jour un factionnaire de garde à la porte de mon cachot, et que mes yeux ne peuvent se lever vers la lucarne carrée sans rencontrer ses deux yeux fixes toujours ouverts.\par
Du reste, on suppose qu’il y a de l’air et du jour dans cette boîte de pierre.
 \section[{XI}]{XI}\phantomsection
\label{ch11}\renewcommand{\leftmark}{XI}

\noindent Puisque le jour ne paraît pas encore, que faire de la nuit ? Il m’est venu une idée. Je me suis levé et j’ai promené ma lampe sur les quatre murs de ma cellule. Ils sont couverts d’écritures, de dessins, de figures bizarres, de noms qui se mêlent et s’effacent les uns les autres. Il semble que chaque condamné ait voulu laisser trace, ici du moins. C’est du crayon, de la craie, du charbon, des lettres noires, blanches, grises, souvent de profondes entailles dans la pierre, çà et là des caractères rouillés qu’on dirait écrits avec du sang. Certes, si j’avais l’esprit plus libre, je prendrais intérêt à ce livre étrange qui se développe page à page à mes yeux sur chaque pierre de ce cachot. J’aimerais à recomposer un tout de ces fragments de pensée, épars sur la dalle ; à retrouver chaque homme sous chaque nom ; à rendre le sens et la vie à ces inscriptions mutilées, à ces phrases démembrées, à ces mots tronqués, corps sans tête, comme ceux qui les ont écrits.\par
 A la hauteur de mon chevet, il y a deux cœurs enflammés, percés d’une flèche, et au-dessus : \emph{Amour pour la vie.} Le malheureux ne prenait pas un long engagement.\par
A côté, une espèce de chapeau à trois cornes avec une petite figure grossièrement dessinée au-dessus, et ces mots : \emph{Vive l’empereur !} 1824.\par
Encore des cœurs enflammés, avec cette inscription, caractéristique dans une prison : \emph{J’aime et j’adore Mathieu Danvin}. J{\scshape acques}.\par
Sur le mur opposé on lit ce mot : \emph{Papavoine.} Le \emph{P }majuscule est brodé d’arabesques et enjolivé avec soin.\par
Un couplet d’une chanson obscène.\par
Un bonnet de liberté sculpté assez profondément dans la pierre, avec ceci dessous : — \emph{Bories.} — \emph{La République}. C’était un des quatre sous-officiers de la Rochelle. Pauvre jeune homme ! Que leurs prétendues nécessités politiques sont hideuses ! pour une idée, pour une rêverie, pour une abstraction, cette horrible réalité qu’on appelle la guillotine ! Et moi qui me plaignais, moi, misérable qui ai commis un véritable crime, qui ai versé du sang !\par
Je n’irai pas plus loin dans ma recherche. — Je viens de voir, crayonnée en blanc au coin du mur, une image épouvantable, la figure de cet échafaud qui, à l’heure qu’il est, se dresse peut-être pour moi. — La lampe a failli me tomber des mains.
 \section[{XII}]{XII}\phantomsection
\label{ch12}\renewcommand{\leftmark}{XII}

\noindent Je suis revenu m’asseoir précipitamment sur ma paille, la tête dans les genoux. Puis mon effroi d’enfant s’est dissipé, et une étrange curiosité m’a repris de continuer la lecture de mon mur.\par
A côté du nom de Papavoine j’ai arraché une énorme toile d’araignée, tout épaissie par la poussière et tendue à l’angle de la muraille. Sous cette toile il y avait quatre ou cinq noms parfaitement lisibles, parmi d’autres dont il ne reste rien qu’une tache sur le mur. — D{\scshape autun}, 1815. — P{\scshape oulain}, 1818. — J{\scshape ean} M{\scshape artin}, 1821. — C{\scshape astaing}, 1823. J’ai lu ces noms, et les lugubres souvenirs me sont venus. Dautun, celui qui a coupé son frère en quartiers, et qui allait la nuit dans Paris jetant la tête dans une fontaine, et le tronc dans un égout ; Poulain, celui qui a assassiné sa femme ; Jean Martin, celui qui a tiré un coup de pistolet à son père au moment où le vieillard ouvrait une fenêtre ; Castaing, ce médecin qui a empoisonné son ami, et qui, le soignant dans cette dernière maladie qu’il lui avait faite, au  lieu de remède lui redonnait du poison ; et auprès de ceux-là, Papavoine, l’horrible fou qui tuait les enfants à coups de couteau sur la tête !\par
Voilà, me disais-je, et un frisson de fièvre me montait dans les reins, voilà quels ont été avant moi les hôtes de cette cellule. C’est ici, sur la même dalle où je suis, qu’ils ont pensé leurs dernières pensées, ces hommes de meurtre et de sang ! c’est autour de ce mur, dans ce carré étroit, que leurs derniers pas ont tourné comme ceux d’une bête fauve. Ils se sont succédé à de courts intervalles ; il paraît que ce cachot ne désemplit pas. Ils ont laissé la place chaude, et c’est à moi qu’ils l’ont laissée. J’irai à mon tour les rejoindre au cimetière de Clamart, où l’herbe pousse si bien !\par
Je ne suis ni visionnaire, ni superstitieux, il est probable que ces idées me donnaient un accès de fièvre ; mais, pendant que je rêvais ainsi, il m’a semblé tout à coup que ces noms fatals étaient écrits avec du feu sur le mur noir ; un tintement de plus en plus précipité a éclaté dans mes oreilles ; une lueur rousse a rempli mes yeux ; et puis il m’a paru que le cachot était plein d’hommes, d’hommes étranges qui portaient leur tête dans leur main gauche, et la portaient par la bouche, parce qu’il n’y avait pas de chevelure. Tous me montraient le poing, excepté le parricide.\par
J’ai fermé les yeux avec horreur, alors j’ai tout vu plus distinctement.\par
Rêve, vision ou réalité, je serais devenu fou, si une impression brusque ne m’eût réveillé à temps.  J’étais près de tomber à la renverse lorsque j’ai senti se traîner sur mon pied nu un ventre froid et des pattes velues ; c’était l’araignée que j’avais dérangée et qui s’enfuyait.\par
Cela m’a dépossédé. — O les épouvantables spectres ! — Non, c’était une fumée, une imagination de mon cerveau vide et convulsif. Chimère à la Macbeth ! Les morts sont morts, ceux-là surtout. Ils sont bien cadenassés dans le sépulcre. Ce n’est pas là une prison dont on s’évade. Comment se fait-il donc que j’aie eu peur ainsi ?\par
La porte du tombeau ne s’ouvre pas en dedans.
 \section[{XIII}]{XIII}\phantomsection
\label{ch13}\renewcommand{\leftmark}{XIII}

\noindent J’ai vu, ces jours passés, une chose hideuse.\par
Il était à peine jour, et la prison était pleine de bruit. On entendait ouvrir et fermer les lourdes portes, grincer les verrous et les cadenas de fer, carillonner les trousseaux de clefs entre-choqués à la ceinture des geôliers, trembler les escaliers du haut en bas sous des pas précipités, et des voix s’appeler et se répondre des deux bouts des longs corridors. Mes voisins de cachot, les forçats en punition, étaient plus gais qu’à l’ordinaire. Tout Bicêtre semblait rire, chanter, courir, danser.\par
Moi, seul muet dans ce vacarme, seul immobile dans ce tumulte, étonné et attentif, j’écoutais.\par
Un geôlier passa.\par
Je me hasardai à l’appeler et à lui demander si c’était fête dans la prison.\par
— Fête si l’on veut ! me répondit-il. C’est aujourd’hui qu’on ferre les forçats qui doivent partir  demain pour Toulon. Voulez-vous voir ? cela vous amusera.\par
C’était en effet, pour un reclus solitaire, une bonne fortune qu’un spectacle, si odieux qu’il fût. J’acceptai l’amusement.\par
Le guichetier prit les précautions d’usage pour s’assurer de moi, puis me conduisit dans une petite cellule vide, et absolument démeublée, qui avait une fenêtre grillée, mais une véritable fenêtre à hauteur d’appui, et à travers laquelle on apercevait réellement le ciel.\par
— Tenez, me dit-il, d’ici vous verrez et vous entendrez. Vous serez seul dans votre loge, comme le roi.\par
Puis il sortit et referma sur moi serrures, cadenas et verrous.\par
La fenêtre donnait sur une cour carrée assez vaste, et autour de laquelle s’élevait des quatre côtés, comme une muraille, un grand bâtiment de pierre de taille à six étages. Rien de plus dégradé, de plus nu, de plus misérable à l’œil que cette quadruple façade percée d’une multitude de fenêtres grillées auxquelles se tenaient collés, du bas en haut, une foule de visages maigres et blêmes, pressés les uns au-dessus des autres, comme les pierres d’un mur, et tous pour ainsi dire encadrés dans les entre-croisements des barreaux de fer. C’étaient les prisonniers, spectateurs de la cérémonie en attendant leur jour d’être acteurs. On eût dit des âmes en peine aux soupiraux du purgatoire qui donnent sur l’enfer.\par
 Tous regardaient en silence la cour vide encore. Ils attendaient. Parmi ces figures éteintes et mornes, çà et là brillaient quelques yeux perçants et vifs comme des points de feu.\par
Le carré de prisons qui enveloppe la cour ne se referme pas sur lui-même. Un des quatre pans de l’édifice (celui qui regarde le levant) est coupé vers son milieu, et ne se rattache au pan voisin que par une grille de fer. Cette grille s’ouvre sur une seconde cour, plus petite que la première, et, comme elle, bloquée de murs et de pignons noirâtres.\par
Tout autour de la cour principale, des bancs de pierre s’adossent à la muraille. Au milieu se dresse une tige de fer courbée, destinée à porter une lanterne.\par
Midi sonna. Une grande porte cochère, cachée sous un enfoncement, s’ouvrit brusquement. Une charrette, escortée d’espèces de soldats sales et honteux, en uniformes bleus, à épaulettes rouges et à bandoulières jaunes, entra lourdement dans la cour avec un bruit de ferraille. C’était la chiourme et les chaînes.\par
Au même instant, comme si ce bruit réveillait tout le bruit de la prison, les spectateurs des fenêtres, jusqu’alors silencieux et immobiles, éclatèrent en cris de joie, en chansons, en menaces, en imprécations mêlées d’éclats de rire poignants à entendre. On eût cru voir des masques de démons. Sur chaque visage parut une grimace, tous les poings sortirent des barreaux, toutes les voix hurlèrent, tous les yeux flamboyèrent, et je fus épouvanté de voir tant d’étincelles reparaître dans cette cendre.\par
 Cependant les argousins, parmi lesquels on distinguait, à leurs vêtements propres et à leur effroi, quelques curieux venus de Paris, les argousins se mirent tranquillement à leur besogne. L’un d’eux monta sur la charrette, et jeta à ses camarades les chaînes, les colliers de voyage, et les liasses de pantalons de toile. Alors ils se dépecèrent le travail ; les uns allèrent étendre dans un coin de la cour les longues chaînes qu’ils nommaient dans leur argot \emph{les ficelles ;} les autres déployèrent sur le pavé \emph{les taffetas}, les chemises et les pantalons ; tandis que les plus sagaces examinaient un à un, sous l’œil de leur capitaine, petit vieillard trapu, les carcans de fer, qu’ils éprouvaient ensuite en les faisant étinceler sur le pavé. Le tout aux acclamations railleuses des prisonniers, dont la voix n’était dominée que par les rires bruyants des forçats pour qui cela se préparait, et qu’on voyait relégués aux croisées de la vieille prison qui donne sur la petite cour.\par
Quand ces apprêts furent terminés, un monsieur brodé en argent, qu’on appelait \emph{monsieur l’inspecteur}, donna un ordre au directeur de la prison ; et un moment après voilà que deux ou trois portes basses vomirent presque en même temps, et comme par bouffées, dans la cour, des nuées d’hommes hideux, hurlants et déguenillés. C’étaient les forçats.\par
A leur entrée, redoublement de joie aux fenêtres. Quelques-uns d’entre eux, les grands noms du bagne, furent salués d’acclamations et d’applaudissements qu’ils recevaient avec une sorte de modestie fière. La plupart avaient des espèces de chapeaux tressés de  leurs propres mains, avec la paille du cachot, et toujours d’une forme étrange, afin que dans les villes où l’on passerait le chapeau fît remarquer la tête. Ceux-là étaient plus applaudis encore. Un, surtout, excita des transports d’enthousiasme ; un jeune homme de dix-sept ans, qui avait un visage de jeune fille. Il sortait du cachot, où il était au secret depuis huit jours ; de sa botte de paille il s’était fait un vêtement qui l’enveloppait de la tête aux pieds, et il entra dans la cour en faisant la roue sur lui-même avec l’agilité d’un serpent. C’était un baladin condamné pour vol. Il y eut une rage de battements de mains et de cris de joie. Les galériens y répondaient, et c’était une chose effrayante que cet échange de gaietés entre les forçats en titre et les forçats aspirants. La société avait beau être là, représentée par les geôliers et les curieux épouvantés, le crime la narguait en face, et de ce châtiment horrible faisait une fête de famille.\par
A mesure qu’ils arrivaient, on les poussait, entre deux haies de gardes-chiourme, dans la petite cour grillée, où la visite des médecins les attendait. C’est là que tous tentaient un dernier effort pour éviter le voyage, alléguant quelque excuse de santé, les yeux malades, la jambe boiteuse, la main mutilée. Mais presque toujours on les trouvait bons pour le bagne ; et alors chacun se résignait avec insouciance, oubliant en peu de minutes sa prétendue infirmité de toute la vie.\par
La grille de la petite cour se rouvrit. Un gardien fit l’appel par ordre alphabétique ; et alors ils sortirent  un à un, et chaque forçat s’alla ranger debout dans un coin de la grande cour, près d’un compagnon donné par le hasard de sa lettre initiale. Ainsi chacun se voit réduit à lui-même ; chacun porte sa chaîne pour soi, côte à côte avec un inconnu ; et si par hasard un forçat a un ami, la chaîne l’en sépare. Dernière des misères.\par
Quand il y en eut à peu près une trentaine de sortis, on referma la grille. Un argousin les aligna avec son bâton, jeta devant chacun d’eux une chemise, une veste et un pantalon de grosse toile, puis fit un signe, et tous commencèrent à se déshabiller. Un incident inattendu vint, comme à point nommé, changer cette humiliation en torture.\par
Jusqu’alors le temps avait été assez beau, et, si la bise d’octobre refroidissait l’air, de temps en temps aussi elle ouvrait çà et là dans les brumes grises du ciel une crevasse par où tombait un rayon de soleil. Mais à peine les forçats se furent-ils dépouillés de leurs haillons de prison, au moment où ils s’offraient nus et debout à la visite soupçonneuse des gardiens, et aux regards curieux des étrangers qui tournaient autour d’eux pour examiner leurs épaules, le ciel devint noir, une froide averse d’automne éclata brusquement, et se déchargea à torrents dans la cour carrée, sur les têtes découvertes, sur les membres nus des galériens, sur leurs misérables sayons étalés sur le pavé.\par
En un clin d’œil le préau se vida de tout ce qui n’était pas argousin ou galérien. Les curieux de Paris allèrent s’abriter sous les auvents des portes.\par
Cependant la pluie tombait à flots. On ne voyait  plus dans la cour que les forçats nus et ruisselants sur le pavé noyé. Un silence morne avait succédé à leurs bruyantes bravades. Ils grelottaient, leurs dents claquaient ; leurs jambes maigries, leurs genoux noueux s’entre-choquaient ; et c’était pitié de les voir appliquer sur leurs membres bleus ces chemises trempées, ces vestes, ces pantalons dégouttants de pluie. La nudité eût été meilleure.\par
Un seul, un vieux, avait conservé quelque gaieté. Il s’écria, en s’essuyant avec sa chemise mouillée, que \emph{cela n’était pas dans le programme ;} puis se prit à rire en montrant le poing au ciel.\par
Quand ils eurent revêtu les habits de route, on les mena par bandes de vingt ou trente à l’autre coin du préau, où les cordons allongés à terre les attendaient. Ces cordons sont de longues et fortes chaînes coupées transversalement de deux en deux pieds par d’autres chaînes plus courtes, à l’extrémité desquelles se rattache un carcan carré, qui s’ouvre au moyen d’une charnière pratiquée à l’un des angles et se ferme à l’angle opposé par un boulon de fer, rivé pour tout le voyage sur le cou du galérien. Quand ces cordons sont développés à terre, ils figurent assez bien la grande arête d’un poisson.\par
On fit asseoir les galériens dans la boue, sur les pavés inondés ; on leur essaya les colliers ; puis deux forgerons de la chiourme, armés d’enclumes portatives, les leur rivèrent à froid à grands coups de masses de fer. C’est un moment affreux, où les plus hardis pâlissent. Chaque coup de marteau, asséné sur l’enclume  appuyée à leur dos, fait rebondir le menton du patient ; le moindre mouvement d’avant en arrière lui ferait sauter le crâne comme une coquille de noix.\par
Après cette opération, ils devinrent sombres. On n’entendait plus que le grelottement des chaînes, et par intervalles un cri et le bruit sourd du bâton des gardes-chiourme sur les membres des récalcitrants. Il y en eut qui pleurèrent ; les vieux frissonnaient et se mordaient les lèvres. Je regardai avec terreur tous ces profils sinistres dans leurs cadres de fer.\par
Ainsi, après la visite des médecins, la visite des geôliers ; après la visite des geôliers, le ferrage. Trois actes à ce spectacle.\par
Un rayon de soleil reparut. On eût dit qu’il mettait le feu à tous ces cerveaux. Les forçats se levèrent à la fois, comme par un mouvement convulsif. Les cinq cordons se rattachèrent par les mains, et tout à coup se formèrent en ronde immense autour de la branche de la lanterne. Ils tournaient à fatiguer les yeux. Ils chantaient une chanson du bagne, une romance d’argot, sur un air tantôt plaintif, tantôt furieux et gai ; on entendait par intervalles des cris grêles, des éclats de rire déchirés et haletants se mêler aux mystérieuses paroles ; puis des acclamations furibondes ; et les chaînes qui s’entre-choquaient en cadence servaient d’orchestre à ce chant plus rauque que leur bruit. Si je cherchais une image du sabbat, je ne la voudrais ni meilleure ni pire.\par
On apporta dans le préau un large baquet. Les gardes-chiourme rompirent la danse des forçats à coups  de bâton, et les conduisirent à ce baquet, dans lequel on voyait nager je ne sais quelles herbes dans je ne sais quel liquide fumant et sale. Ils mangèrent.\par
Puis, ayant mangé, ils jetèrent sur le pavé ce qui restait de leur soupe et de leur pain bis, et se remirent à danser et à chanter. Il paraît qu’on leur laisse cette liberté le jour du ferrage et la nuit qui le suit.\par
J’observais ce spectacle étrange avec une curiosité si avide, si palpitante, si attentive, que je m’étais oublié moi-même. Un profond sentiment de pitié me remuait jusqu’aux entrailles, et leurs rires me faisaient pleurer.\par
Tout à coup, à travers la rêverie profonde où j’étais tombé, je vis la ronde hurlante s’arrêter et se taire. Puis tous les yeux se tournèrent vers la fenêtre que j’occupais. — Le condamné ! le condamné ! crièrent-ils tous en me montrant du doigt ; et les explosions de joie redoublèrent.\par
Je restai pétrifié.\par
J’ignore d’où ils me connaissaient et comment ils m’avaient reconnu.\par
— Bonjour ! bonsoir ! me crièrent-ils avec leur ricanement atroce. Un des plus jeunes, condamné aux galères perpétuelles, face luisante et plombée, me regarda d’un air d’envie en disant : — Il est heureux ! il sera \emph{rogné !} Adieu, camarade !\par
Je ne puis dire ce qui se passait en moi. J’étais leur camarade en effet. La Grève est sœur de Toulon. J’étais même placé plus bas qu’eux ; ils me faisaient honneur. Je frissonnai.\par
 Oui, leur camarade ! Et quelques jours plus tard, j’aurais pu aussi, moi, être un spectacle pour eux.\par
J’étais demeuré à la fenêtre, immobile, perclus, paralysé. Mais quand je vis les cinq cordons s’avancer, se ruer vers moi avec des paroles d’une infernale cordialité ; quand j’entendis le tumultueux fracas de leurs chaînes, de leurs clameurs, de leurs pas, au pied du mur, il me sembla que cette nuée de démons escaladait ma misérable cellule ; je poussai un cri, je me jetai sur la porte d’une violence à la briser ; mais pas moyen de fuir ; les verrous étaient tirés en dehors. Je heurtai, j’appelai avec rage. Puis il me sembla entendre de plus près encore les effrayantes voix des forçats. Je crus voir leurs têtes hideuses paraître déjà au bord de ma fenêtre, je poussai un second cri d’angoisse, et je tombai évanoui.
 \section[{XIV}]{XIV}\phantomsection
\label{ch14}\renewcommand{\leftmark}{XIV}

\noindent Quand je revins à moi, il était nuit. J’étais couché dans un grabat ; une lanterne qui vacillait au plafond me fit voir d’autres grabats alignés des deux côtés du mien. Je compris qu’on m’avait transporté à l’infirmerie.\par
Je restai quelques instants éveillé, mais sans pensée et sans souvenir, tout entier au bonheur d’être dans un lit. Certes, en d’autres temps, ce lit d’hôpital et de prison m’eût fait reculer de dégoût et de pitié ; mais je n’étais plus le même homme. Les draps étaient gris et rudes au toucher, la couverture maigre et trouée ; on sentait la paillasse à travers le matelas ; qu’importe ! mes membres pouvaient se déroidir à l’aise entre ces draps grossiers ; sous cette couverture, si mince qu’elle fût, je sentais se dissiper peu à peu cet horrible froid de la moelle des os dont j’avais pris l’habitude. — Je me rendormis.\par
Un grand bruit me réveilla ; il faisait petit jour. Ce bruit venait du dehors ; mon lit était à côté de la  fenêtre, je me levai sur mon séant pour voir ce que c’était.\par
La fenêtre donnait sur la grande cour de Bicêtre. Cette cour était pleine de monde ; deux haies de vétérans avaient peine à maintenir libre, au milieu de cette foule, un étroit chemin qui traversait la cour. Entre ce double rang de soldats cheminaient lentement, cahotées à chaque pavé, cinq longues charrettes chargées d’hommes ; c’étaient les forçats qui partaient.\par
Ces charrettes étaient découvertes. Chaque cordon en occupait une. Les forçats étaient assis de côté sur chacun des bords, adossés les uns aux autres, séparés par la chaîne commune, qui se développait dans la longueur du chariot, et sur l’extrémité de laquelle un argousin debout, fusil chargé, tenait le pied. On entendait bruire leurs fers, et, à chaque secousse de la voiture, on voyait sauter leurs têtes et ballotter leurs jambes pendantes.\par
Une pluie fine et pénétrante glaçait l’air, et collait sur leurs genoux leurs pantalons de toile, de gris devenus noirs. Leurs longues barbes, leurs cheveux courts ruisselaient ; leurs visages étaient violets ; on les voyait grelotter, et leurs dents grinçaient de rage et de froid. Du reste, pas de mouvements possibles. Une fois rivé à cette chaîne, on n’est plus qu’une fraction de ce tout hideux qu’on appelle le cordon, et qui se meut comme un seul homme. L’intelligence doit abdiquer, le carcan du bagne la condamne à mort ; et quant à l’animal lui-même, il ne doit plus avoir de besoins et d’appétits qu’à heures fixes. Ainsi, immobiles,  la plupart demi-nus, têtes découvertes et pieds pendants, ils commençaient leur voyage de vingt-cinq jours, chargés sur les mêmes charrettes, vêtus des mêmes vêtements pour le soleil à plomb de juillet et pour les froides pluies de novembre. On dirait que les hommes veulent mettre le ciel de moitié dans leur office de bourreaux.\par
Il s’était établi entre la foule et les charrettes je ne sais quel horrible dialogue ; injures d’un côté, bravades de l’autre, imprécations des deux parts ; mais, à un signe du capitaine, je vis les coups de bâton pleuvoir au hasard dans les charrettes, sur les épaules ou sur les têtes, et tout rentra dans cette espèce de calme extérieur qu’on appelle l’\emph{ordre}. Mais les yeux étaient pleins de vengeance, et les poings des misérables se crispaient sur leurs genoux.\par
Les cinq charrettes, escortées de gendarmes à cheval et d’argousins à pied, disparurent successivement sous la haute porte cintrée de Bicêtre ; une sixième les suivit, dans laquelle ballottaient pêle-mêle les chaudières, les gamelles de cuivre et les chaînes de rechange. Quelques gardes-chiourme qui s’étaient attardés à la cantine sortirent en courant pour rejoindre leur escouade. La foule s’écoula. Tout ce spectacle s’évanouit comme une fantasmagorie. On entendit s’affaiblir par degrés dans l’air le bruit lourd des roues et des pieds des chevaux sur la route pavée de Fontainebleau, le claquement des fouets, le cliquetis des chaînes, et les hurlements du peuple qui souhaitait malheur au voyage des galériens.\par
 Et c’est là pour eux le commencement !\par
Que me disait-il donc, l’avocat ? Les galères ! Ah ! oui, plutôt mille fois la mort, plutôt l’échafaud que le bagne, plutôt le néant que l’enfer ; plutôt livrer mon cou au couteau de Guillotin qu’au carcan de la chiourme ! Les galères, juste ciel !
 \section[{XV}]{XV}\phantomsection
\label{ch15}\renewcommand{\leftmark}{XV}

\noindent Malheureusement je n’étais pas malade. Le lendemain il fallut sortir de l’infirmerie. Le cachot me reprit.\par
Pas malade ! en effet, je suis jeune, sain et fort. Le sang coule librement dans mes veines ; tous mes membres obéissent à tous mes caprices ; je suis robuste de corps et d’esprit, constitué pour une longue vie ; oui, tout cela est vrai ; et cependant j’ai une maladie, une maladie mortelle, une maladie faite de la main des hommes.\par
Depuis que je suis sorti de l’infirmerie, il m’est venu une idée poignante, une idée à me rendre fou, c’est que j’aurais peut-être pu m’évader si l’on m’y avait laissé. Ces médecins, ces sœurs de charité, semblaient prendre intérêt à moi. Mourir si jeune et d’une telle mort ! On eût dit qu’ils me plaignaient, tant ils étaient empressés autour de mon chevet. Bah ! curiosité ! Et puis, ces gens qui guérissent vous guérissent bien d’une fièvre, mais non d’une sentence de mort. Et pourtant cela leur serait si facile ! une porte ouverte ! Qu’est-ce que cela leur ferait ?\par
 Plus de chance maintenant ! mon pourvoi sera rejeté, parce que tout est en règle ; les témoins ont bien témoigné, les plaideurs ont bien plaidé, les juges ont bien jugé. Je n’y compte pas, à moins que… Non, folie ! plus d’espérance ! Le pourvoi, c’est une corde qui vous tient suspendu au-dessus de l’abîme, et qu’on entend craquer à chaque instant jusqu’à ce qu’elle se casse. C’est comme si le couteau de la guillotine mettait six semaines à tomber.\par
Si j’avais ma grâce ? — Avoir ma grâce ! Et par qui ? et pourquoi ? et comment ? Il est impossible qu’on me fasse grâce. L’exemple ! comme ils disent.\par
Je n’ai plus que trois pas à faire : Bicêtre, la Conciergerie, la Grève.
 \section[{XVI}]{XVI}\phantomsection
\label{ch16}\renewcommand{\leftmark}{XVI}

\noindent Pendant le peu d’heures que j’ai passées à l’infirmerie, je m’étais assis près d’une fenêtre, au soleil, — il avait reparu — ou du moins recevant du soleil tout ce que les grilles de la croisée m’en laissaient.\par
J’étais là, ma tête pesante et embrasée dans mes deux mains, qui en avaient plus qu’elles n’en pouvaient porter, mes coudes sur mes genoux, les pieds sur les barreaux de ma chaise ; car l’abattement fait que je me courbe et me replie sur moi-même comme si je n’avais plus ni os dans les membres ni muscles dans la chair.\par
L’odeur étouffée de la prison me suffoquait plus que jamais, j’avais encore dans l’oreille tout ce bruit de chaînes des galériens, j’éprouvais une grande lassitude de Bicêtre. Il me semblait que le bon Dieu devrait bien avoir pitié de moi et m’envoyer au moins un petit oiseau pour chanter là, en face, au bord du toit.\par
Je ne sais si ce fut le bon Dieu ou le démon qui  m’exauça ; mais presque au même moment j’entendis s’élever sous ma fenêtre une voix, non celle d’un oiseau, mais bien mieux, la voix pure, fraîche, veloutée d’une jeune fille de quinze ans. Je levai la tête comme en sursaut, j’écoutai avidement la chanson qu’elle chantait. C’était un air lent et langoureux, une espèce de roucoulement triste et lamentable ; voici les paroles :\par


\begin{verse}
C’est dans la rue du Mail\\
Où j’ai été coltigé,\\
Maluré,\\
Par trois coquins de railles,\\
Lirlonfa malurette,\\
Sur mes sique’ ont foncé,\\
Lirlonfa maluré.\\
\end{verse}

\noindent Je ne saurais dire combien fut amer mon désappointement. La voix continua :\par


\begin{verse}
Sur mes sique’ ont foncé,\\
Maluré.\\
Ils m’ont mis la tartouve,\\
Lirlonfa malurette,\\
Grand Meudon est aboulé,\\
Lirlonfa maluré.\\
Dans mon trimin rencontre\\
Lirlonfa malurette,\\
Un peigre du quartier,\\
Lirlonfa maluré.\\!Un peigre du quartier,\\
Maluré.\\
— Va-t’en dire à ma largue,\\
Lirlonfa malurette,\\! Que je suis enfourraillé,\\
Lirlonfa maluré.\\
Ma largue tout en colère,\\
Lirlonfa malurette,\\
M’dit : Qu’as-tu donc morfillé ?\\
Lirlonfa maluré.\\!M’dit : Qu’as-tu donc morfillé ?\\
Maluré.\\
 — J’ai fait suer un chêne,\\
Lirlonfa malurette,\\
Son auberg j’ai enganté,\\
Lirlonfa maluré,\\
Son auberg et sa toquante,\\
Lirlonfa malurette,\\
Et ses attachas de cés,\\
Lirlonfa maluré.\\!Et ses attachas de cés,\\
Maluré. — \\
Ma largu’ part pour Versailles,\\
Lirlonfa malurette.\\
Aux pieds d’sa majesté,\\
Lirlonfa maluré.\\
Elle lui fonce un babillard,\\
Lirlonfa malurette,\\
Pour m’faire défourrailler,\\
Lirlonfa maluré,\\!Pour m’faire défourrailler,\\
Maluré.\\
 — Ah l si j’en défourraille,\\
Lirlonfa malurette,\\
Ma largue j’enti ferai,\\
Lirlonfa maluré.\\
J’ii ferai porter fontange,\\
Lirlonfa malurette,\\! Et souliers galuchés,\\
Lirlonfa maluré.\\!Et souliers galuchés,\\
Maluré.\\
Mais grand dabe qui s’fâche,\\
Lirlonfla malurette,\\
Dit : — Par mon caloquet,\\
Lirlonfa maluré,\\
J’ii ferai danser une danse,\\
Lirlonfa malurette,\\
Où il n’y a pas de plancher,\\
Lirlonfa maluré. — \\!
\end{verse}

\noindent Je n’en ai pas entendu et n’aurais pu en entendre davantage. Le sens à demi compris et à demi caché de cette horrible complainte ; cette lutte du brigand avec le guet, ce voleur qu’il rencontre et qu’il dépêche à sa femme, cet épouvantable message : J’ai assassiné un homme et je suis arrêté, \emph{j’ai fait suer un chêne et je suis enfourraillé ;} cette femme qui court à Versailles avec un placet, et cette \emph{Majesté} qui s’indigne et menace le coupable de lui faire danser \emph{la danse où il n’y a pas de plancher ;} et tout cela chanté sur l’air le plus doux et par la plus douce voix qui ait jamais endormi l’oreille humaine !… J’en suis resté navré, glacé, anéanti. C’était une chose repoussante que toutes ces monstrueuses paroles sortant de cette bouche vermeille et fraîche. On eût dit la bave d’une limace sur une rose.\par
Je ne saurais rendre ce que j’éprouvais ; j’étais à la fois blessé et caressé. Le patois de la caverne et du bagne, cette langue ensanglantée et grotesque, ce  hideux argot, marié à une voix de jeune fille, gracieuse transition de la voix d’enfant à la voix de femme ! tous ces mots difformes et mal faits, chantés, cadencés, perlés !\par
Ah ! qu’une prison est quelque chose d’infâme ! il y a un venin qui y salit tout. Tout s’y flétrit, même la chanson d’une fille de quinze ans ! Vous y trouvez un oiseau, il a de la boue sur son aile ; vous y cueillez une jolie fleur, vous la respirez ; elle pue.
 \section[{XVII}]{XVII}\phantomsection
\label{ch17}\renewcommand{\leftmark}{XVII}

\noindent Oh ! si je m’évadais, comme je courrais à travers champs !\par
Non, il ne faudrait pas courir. Cela fait regarder et soupçonner. Au contraire, marcher lentement, tête levée, en chantant. Tâcher d’avoir quelque vieux sarrau bleu à dessins rouges. Cela déguise bien. Tous les maraîchers des environs en portent.\par
Je sais auprès d’Arcueil un fourré d’arbres à côté d’un marais, où, étant au collège, je venais avec mes camarades pêcher des grenouilles tous les jeudis. C’est là que je me cacherais jusqu’au soir.\par
La nuit tombée, je reprendrais ma course. J’irais à Vincennes. Non, la rivière m’empêcherait. J’irais à Arpajon. — Il aurait mieux valu prendre du côté de Saint-Germain, et aller au Havre, et m’embarquer pour l’Angleterre. — N’importe ! j’arrive à Longjumeau. Un gendarme passe ; il me demande mon passe-port… Je suis perdu !\par
 Ah ! malheureux rêveur, brise donc d’abord le mur épais de trois pieds qui t’emprisonne ! La mort ! la mort !\par
Quand je pense que je suis venu tout enfant, ici, à Bicêtre, voir le grand puits et les fous !
 \section[{XVIII}]{XVIII}\phantomsection
\label{ch18}\renewcommand{\leftmark}{XVIII}

\noindent Pendant que j’écrivais tout ceci, ma lampe a pâli, le jour est venu, l’horloge de la chapelle a sonné six heures. —\par
Qu’est-ce que cela veut dire ? Le guichetier de garde vient d’entrer dans mon cachot, il a ôté sa casquette, m’a salué, s’est excusé de me déranger, et m’a demandé, en adoucissant de son mieux sa rude voix, ce que je désirais à déjeuner ?…\par
Il m’a pris un frisson. — Est-ce que ce serait pour aujourd’hui ?
 \section[{XIX}]{XIX}\phantomsection
\label{ch19}\renewcommand{\leftmark}{XIX}

\noindent C’est pour aujourd’hui !\par
Le directeur de la prison lui-même vient de me rendre visite. Il m’a demandé en quoi il pourrait m’être agréable ou utile ; a exprimé le désir que je n’eusse pas à me plaindre de lui ou de ses subordonnés ; s’est informé avec intérêt de ma santé et de la façon dont j’avais passé la nuit ; en me quittant, il m’a appelé \emph{monsieur !}\par
C’est pour aujourd’hui !
 \section[{XX}]{XX}\phantomsection
\label{ch20}\renewcommand{\leftmark}{XX}

\noindent Il ne croit pas, ce geôlier, que j’aie à me plaindre de lui et de ses sous-geôliers. Il a raison. Ce serait mal à moi de me plaindre ; ils ont fait leur métier, ils m’ont bien gardé ; et puis ils ont été polis à l’arrivée et au départ. Ne dois-je pas être content ?\par
Ce bon geôlier, avec son sourire bénin, ses paroles caressantes, son œil qui flatte et qui espionne, ses grosses et larges mains, c’est la prison incarnée, c’est Bicêtre qui s’est fait homme. Tout est prison autour de moi ; je retrouve la prison sous toutes les formes, sous la forme humaine comme sous la forme de grille ou de verrou. Ce mur, c’est de la prison en pierre ; cette porte, c’est de la prison en bois ; ces guichetiers, c’est de la prison en chair et en os. La prison est une espèce d’être horrible, complet, indivisible, moitié maison, moitié homme. Je suis sa proie ; elle me couve, elle m’enlace de tous ses replis. Elle m’enferme dans ses murailles de granit, me cadenasse sous ses serrures de fer, et me surveille avec ses yeux de geôlier.\par
Ah ! misérable ! que vais-je devenir ? qu’est-ce qu’ils vont faire de moi ?
 \section[{XXI}]{XXI}\phantomsection
\label{ch21}\renewcommand{\leftmark}{XXI}

\noindent Je suis calme maintenant. Tout est fini, bien fini. Je suis sorti de l’horrible anxiété où m’avait jeté la visite du directeur. Car, je l’avoue, j’espérais encore. — Maintenant, Dieu merci, je n’espère plus.\par
Voici ce qui vient de se passer :\par
Au moment où six heures et demie sonnaient, — non, c’était l’avant-quart, — la porte de mon cachot s’est rouverte. Un vieillard à tête blanche, vêtu d’une redingote brune, est entré. Il a entr’ouvert sa redingote. J’ai vu une soutane, un rabat. C’était un prêtre.\par
Ce prêtre n’était pas l’aumônier de la prison. Cela était sinistre.\par
Il s’est assis en face de moi avec un sourire bienveillant ; puis a secoué la tête et levé les yeux au ciel, c’est-à-dire à la voûte du cachot. Je l’ai compris.\par
— Mon fils, m’a-t-il dit, êtes-vous préparé ?\par
Je lui ai répondu d’une voix faible :\par
— Je ne suis pas préparé, mais je suis prêt.\par
Cependant ma vue s’est troublée, une sueur glacée est sortie à la fois de tous mes membres, j’ai senti mes tempes se gonfler, et j’avais les oreilles pleines de bourdonnements.\par
 Pendant que je vacillais sur ma chaise comme endormi, le bon vieillard parlait. C’est du moins ce qui m’a semblé, et je crois me souvenir que j’ai vu ses lèvres remuer, ses mains s’agiter, ses yeux reluire.\par
La porte s’est rouverte une seconde fois. Le bruit des verrous nous a arrachés, moi à ma stupeur, lui à son discours. Une espèce de monsieur, en habit noir, accompagné du directeur de la prison, s’est présenté, et m’a salué profondément. Cet homme avait sur le visage quelque chose de la tristesse officielle des employés des pompes funèbres. Il tenait un rouleau de papier à la main.\par
— Monsieur, m’a-t-il dit avec un sourire de courtoisie, je suis huissier près la cour royale de Paris. J’ai l’honneur de vous apporter un message de la part de monsieur le procureur général.\par
La première secousse était passée. Toute ma présence d’esprit m’était revenue.\par
— C’est monsieur le procureur général, lui ai-je répondu, qui a demandé si instamment ma tête ? Bien de l’honneur pour moi qu’il m’écrive. J’espère que ma mort lui va faire grand plaisir ; car il me serait dur de penser qu’il l’a sollicitée avec tant d’ardeur et qu’elle lui était indifférente.\par
J’ai dit tout cela, et j’ai repris d’une voix ferme :\par
— Lisez, monsieur !\par
Il s’est mis à me lire un long texte, en chantant à la fin de chaque ligne et en hésitant au milieu de chaque mot. C’était le rejet de mon pourvoi.\par
— L’arrêt sera exécuté aujourd’hui en place de  Grève, a-t-il ajouté quand il a eu terminé, sans lever les yeux de dessus son papier timbré. Nous partons à sept heures et demie précises pour la Conciergerie. Mon cher monsieur, aurez-vous l’extrême bonté de me suivre ?\par
Depuis quelques instants je ne l’écoutais plus. Le directeur causait avec le prêtre ; lui, avait l’œil fixé sur son papier ; je regardais la porte, qui était restée entrouverte… — Ah ! misérable ! quatre fusiliers dans le corridor !\par
L’huissier a répété sa question, en me regardant cette fois.\par
— Quand vous voudrez, lui ai-je répondu. A votre aise !\par
Il m’a salué en disant :\par
— J’aurai l’honneur de venir vous chercher dans une demi-heure.\par
Alors ils m’ont laissé seul.\par
Un moyen de fuir, mon Dieu ! un moyen quelconque ! Il faut que je m’évade ! il le faut ! sur-le-champ ! par les portes, par les fenêtres, par la charpente du toit ! quand même je devrais laisser de ma chair après les poutres !\par
O rage ! démons ! malédiction ! Il faudrait des mois pour percer ce mur avec de bons outils, et je n’ai ni un clou, ni une heure !
 \section[{XXII}]{XXII}\phantomsection
\label{ch22}\renewcommand{\leftmark}{XXII}


\dateline{De la Conciergerie.}
\noindent Me voici \emph{transféré}, comme dit le procès-verbal.\par
Mais le voyage vaut la peine d’être conté.\par
Sept heures et demie sonnaient lorsque l’huissier s’est présenté de nouveau au seuil de mon cachot. — Monsieur, m’a-t-il dit, je vous attends. — Hélas ! lui et d’autres !\par
Je me suis levé, j’ai fait un pas ; il m’a semblé que je n’en pourrais faire un second, tant ma tête était lourde et mes jambes faibles. Cependant je me suis remis et j’ai continué d’une allure assez ferme. Avant de sortir du cabanon, j’y ai promené un dernier coup d’œil. — Je l’aimais, mon cachot. — Puis, je l’ai laissé vide et ouvert ; ce qui donne à un cachot un air singulier.\par
Au reste, il ne le sera pas longtemps. Ce soir on y attend quelqu’un, disaient les porte-clefs, un condamné que la cour d’assises est en train de faire à l’heure qu’il est.\par
 Au détour du corridor, l’aumônier nous a rejoints. Il venait de déjeuner.\par
Au sortir de la geôle, le directeur m’a pris affectueusement la main, et a renforcé mon escorte de quatre vétérans.\par
Devant la porte de l’infirmerie, un vieillard moribond m’a crié : Au revoir !\par
Nous sommes arrivés dans la cour. J’ai respiré ; cela m’a fait du bien.\par
Nous n’avons pas marché longtemps à l’air. Une voiture attelée de chevaux de poste stationnait dans la première cour ; c’est la même voiture qui m’avait amené ; une espèce de cabriolet oblong, divisé en deux sections par une grille transversale de fil de fer si épaisse qu’on la dirait tricotée. Les deux sections ont chacune une porte, l’une devant, l’autre derrière la carriole. Le tout si sale, si noir, si poudreux, que le corbillard des pauvres est un carrosse du sacre en comparaison.\par
Avant de m’ensevelir dans cette tombe à deux roues, j’ai jeté un regard dans la cour, un de ces regards désespérés devant lesquels il semble que les murs devraient crouler. La cour, espèce de petite place plantée d’arbres, était plus encombrée encore de spectateurs que pour les galériens. Déjà la foule !\par
Comme le jour du départ de la chaîne, il tombait une pluie de la saison, une pluie fine et glacée qui tombe encore à l’heure où j’écris, qui tombera sans doute toute la journée, qui durera plus que moi.\par
Les chemins étaient effondrés, la cour pleine de  fange et d’eau. J’ai eu plaisir à voir cette foule dans cette boue.\par
Nous sommes montés, l’huissier et un gendarme, dans le compartiment de devant ; le prêtre, moi et un gendarme dans l’autre. Quatre gendarmes à cheval autour de la voiture. Ainsi, sans le postillon, huit hommes pour un homme.\par
Pendant que je montais, il y avait une vieille aux yeux gris qui disait : — J’aime encore mieux cela que la chaîne.\par
Je conçois. C’est un spectacle qu’on embrasse plus aisément d’un coup d’œil, c’est plus tôt vu. C’est tout aussi beau et plus commode. Rien ne vous distrait. Il n’y a qu’un homme, et sur cet homme seul autant de misère que sur tous les forçats à la fois. Seulement cela est moins éparpillé ; c’est une liqueur concentrée, bien plus savoureuse.\par
La voiture s’est ébranlée. Elle a fait un bruit sourd en passant sous la voûte de la grande porte, puis a débouché dans l’avenue, et les lourds battants de Bicêtre se sont refermés derrière elle. Je me sentais emporter avec stupeur, comme un homme tombé en léthargie qui ne peut ni remuer ni crier et qui entend qu’on l’enterre. J’écoutais vaguement les paquets de sonnettes pendus au cou des chevaux de poste sonner en cadence et comme par hoquets, les roues ferrées bruire sur le pavé ou cogner la caisse en changeant d’ornières, le galop sonore des gendarmes autour de la carriole, le fouet claquant du postillon. Tout cela me semblait comme un tourbillon qui m’emportait.\par
 A travers le grillage d’un judas percé en face de moi, mes yeux s’étaient fixés machinalement sur l’inscription gravée en grosses lettres au-dessus de la grande porte de Bicêtre : H{\scshape ospice de la} V{\scshape ieillesse}.\par
— Tiens, me disais-je, il paraît qu’il y a des gens qui vieillissent là.\par
Et, comme on fait entre la veille et le sommeil, je retournais cette idée en tous sens dans mon esprit engourdi de douleur. Tout à coup la carriole, en passant de l’avenue dans la grande route, a changé le point de vue de la lucarne. Les tours de Notre-Dame sont venues s’y encadrer, bleues et à demi effacées dans la brume de Paris. Sur-le-champ le point de vue de mon esprit a changé aussi. J’étais devenu machine comme la voiture. A l’idée de Bicêtre a succédé l’idée des tours de Notre-Dame. — Ceux qui seront sur la tour où est le drapeau verront bien, me suis-je dit en souriant stupidement.\par
Je crois que c’est à ce moment-là que le prêtre s’est remis à me parler. Je l’ai laissé dire patiemment. J’avais déjà dans l’oreille le bruit des roues, le galop des chevaux, le fouet du postillon. C’était un bruit de plus.\par
J’écoutais en silence cette chute de paroles monotones qui assoupissaient ma pensée comme le murmure d’une fontaine, et qui passaient devant moi, toujours diverses et toujours les mêmes, comme les ormeaux tortus de la grande route, lorsque la voix brève et saccadée de l’huissier, placé sur le devant, est venue subitement me secouer.\par
 — Eh bien ! monsieur l’abbé, disait-il avec un accent presque gai, qu’est-ce que vous savez de nouveau ?\par
C’est vers le prêtre qu’il se retournait en parlant ainsi.\par
L’aumônier, qui me parlait sans relâche, et que la voiture assourdissait, n’a pas répondu.\par
— Hé ! hé ! a repris l’huissier en haussant la voix pour avoir le dessus sur le bruit des roues ; infernale voiture !\par
Infernale ! En effet.\par
Il a continué :\par
— Sans doute, c’est le cahot ; on ne s’entend pas. Qu’est-ce que je voulais donc dire ? Faites-moi le plaisir de m’apprendre ce que je voulais dire, monsieur l’abbé ? — Ah ! savez-vous la grande nouvelle de Paris, aujourd’hui ?\par
J’ai tressailli, comme s’il parlait de moi.\par
— Non, a dit le prêtre, qui avait enfin entendu, je n’ai pas eu le temps de lire les journaux ce matin. Je verrai cela ce soir. Quand je suis occupé comme cela toute la journée, je recommande au portier de me garder mes journaux, et je les lis en rentrant.\par
— Bah ! a repris l’huissier, il est impossible que vous ne sachiez pas cela. La nouvelle de Paris ! la nouvelle de ce matin !\par
J’ai pris la parole.\par
— Je crois la savoir.\par
L’huissier m’a regardé.\par
— Vous ! vraiment ! En ce cas, qu’en dites-vous ?\par
 — Vous êtes curieux ! lui ai-je dit.\par
— Pourquoi, monsieur ? a répliqué l’huissier. Chacun a son opinion politique. Je vous estime trop pour croire que vous n’avez pas la vôtre. Quant à moi, je suis tout à fait d’avis du rétablissement de la garde nationale. J’étais sergent de ma compagnie, et, ma foi, c’était fort agréable.\par
Je l’ai interrompu.\par
— Je ne croyais pas que ce fût de cela qu’il s’agissait.\par
— Et de quoi donc ? vous disiez savoir la nouvelle…\par
— Je parlais d’une autre, dont Paris s’occupe aussi aujourd’hui.\par
L’imbécile n’a pas compris ; sa curiosité s’est éveillée.\par
— Une autre nouvelle ? Où diable avez-vous pu apprendre des nouvelles ? Laquelle, de grâce, mon cher monsieur ? Savez-vous ce que c’est, monsieur l’abbé ? êtes-vous plus au courant que moi ? Mettez-moi au fait, je vous prie. De quoi s’agit-il ? — Voyez-vous, j’aime les nouvelles. Je les conte à monsieur le président, et cela l’amuse.\par
Et mille billevesées. Il se tournait tour à tour vers le prêtre et vers moi, et je ne répondais qu’en haussant les épaules.\par
— Eh bien ! m’a-t-il dit, à quoi pensez-vous donc ?\par
— Je pense, ai-je répondu, que je ne penserai plus ce soir.\par
 — Ah ! c’est cela ! a-t-il répliqué. Allons, vous êtes trop triste ! M. Castaing causait.\par
Puis, après un silence :\par
— J’ai conduit M. Papavoine ; il avait sa casquette de loutre et fumait son cigare. Quant aux jeunes gens de la Rochelle, ils ne parlaient qu’entre eux. Mais ils parlaient.\par
Il a fait encore une pause, et a poursuivi :\par
— Des fous ! des enthousiastes ! Ils avaient l’air de mépriser tout le monde. Pour ce qui est de vous, je vous trouve vraiment bien pensif, jeune homme.\par
— Jeune homme ! lui ai-je dit, je suis plus vieux que vous ; chaque quart d’heure qui s’écoule me vieillit d’une année.\par
Il s’est retourné, m’a regardé quelques minutes avec un étonnement inepte, puis s’est mis à ricaner lourdement.\par
— Allons, vous voulez rire, plus vieux que moi ! je serais votre grand-père.\par
— Je ne veux pas rire, lui ai-je répondu gravement.\par
Il a ouvert sa tabatière.\par
— Tenez, cher monsieur, ne vous fâchez pas ; une prise de tabac, et ne me gardez pas rancune.\par
— N’ayez pas peur ; je n’aurai pas longtemps à vous la garder.\par
En ce moment sa tabatière, qu’il me tendait, a rencontré le grillage qui nous séparait. Un cahot a fait qu’elle l’a heurté assez violemment et est tombée tout ouverte sous les pieds du gendarme.\par
 — Maudit grillage ! s’est écrié l’huissier.\par
Il s’est tourné vers moi.\par
— Eh bien ! ne suis-je pas malheureux ? tout mon tabac est perdu !\par
— Je perds plus que vous, ai-je répondu en souriant.\par
Il a essayé de ramasser son tabac, en grommelant entre ses dents :\par
— Plus que moi ! cela est facile à dire. Pas de tabac jusqu’à Paris ! c’est terrible !\par
L’aumônier alors lui a adressé quelques paroles de consolation, et je ne sais si j’étais préoccupé, mais il m’a semblé que c’était la suite de l’exhortation dont j’avais eu le commencement. Peu à peu la conversation s’est engagée entre le prêtre et l’huissier ; je les ai laissés parler de leur côté, et je me suis mis à penser du mien.\par
En abordant la barrière, j’étais toujours préoccupé sans doute, mais Paris m’a paru faire un plus grand bruit qu’à l’ordinaire.\par
La voiture s’est arrêtée un moment devant l’octroi. Les douaniers de ville l’ont inspectée. Si c’eût été un mouton ou un bœuf qu’on eût mené à la boucherie, il aurait fallu leur jeter une bourse d’argent ; mais une tête humaine ne paye pas de droit. Nous avons passé.\par
Le boulevard franchi, la carriole s’est enfoncée au grand trot dans ces vieilles rues tortueuses du faubourg Saint-Marceau et de la Cité, qui serpentent et s’entrecoupent comme les mille chemins d’une fourmilière. Sur le pavé de ces rues étroites le roulement de la voiture  est devenu si bruyant et si rapide, que je n’entendais plus rien du bruit extérieur. Quand je jetais les yeux par la petite lucarne carrée, il me semblait que le flot des passants s’arrêtait pour regarder la voiture, et que des bandes d’enfants couraient sur sa trace. Il m’a semblé aussi voir de temps en temps dans les carrefours çà et là un homme ou une vieille en haillons, quelquefois les deux ensemble, tenant en main une liasse de feuilles imprimées que les passants se disputaient, en ouvrant la bouche comme pour un grand cri.\par
Huit heures et demie sonnaient à l’horloge du Palais au moment où nous sommes arrivés dans la cour de la Conciergerie. La vue de ce grand escalier, de cette noire chapelle, de ces guichets sinistres, m’a glacé. Quand la voiture s’est arrêtée, j’ai cru que les battements de mon cœur allaient s’arrêter aussi.\par
J’ai recueilli mes forces ; la porte s’est ouverte avec la rapidité de l’éclair ; j’ai sauté à bas du cachot roulant, et je me suis enfoncé à grands pas sous la voûte entre deux haies de soldats. Il s’était déjà formé une foule sur mon passage.
 \section[{XXIII}]{XXIII}\phantomsection
\label{ch23}\renewcommand{\leftmark}{XXIII}

\noindent Tant que j’ai marché dans les galeries publiques du Palais de Justice, je me suis senti presque libre et à l’aise ; mais toute ma résolution m’a abandonné quand on a ouvert devant moi des portes basses, des escaliers secrets, des couloirs intérieurs, de longs corridors étouffés et sourds, où il n’entre que ceux qui condamnent ou ceux qui sont condamnés.\par
L’huissier m’accompagnait toujours. Le prêtre m’avait quitté pour revenir dans deux heures ; il avait ses affaires.\par
On m’a conduit au cabinet du directeur, entre les mains duquel l’huissier m’a remis. C’était un échange. Le directeur l’a prié d’attendre un instant, lui annonçant qu’il allait avoir du \emph{gibier} à lui remettre, afin qu’il le conduisît sur-le-champ à Bicêtre par le retour de la carriole. Sans doute le condamné d’aujourd’hui, celui qui doit coucher ce soir sur la botte de paille que je n’ai pas eu le temps d’user.\par
— C’est bon, a dit l’huissier au directeur, je vais  attendre un moment ; nous ferons les deux procès-verbaux à la fois, cela s’arrange bien.\par
En attendant, on m’a déposé dans un petit cabinet attenant à celui du directeur. Là on m’a laissé seul, bien verrouillé.\par
Je ne sais à quoi je pensais, ni depuis combien de temps j’étais là, quand un brusque et violent éclat de rire à mon oreille m’a réveillé de ma rêverie.\par
J’ai levé les yeux en tressaillant. Je n’étais plus seul dans la cellule. Un homme s’y trouvait avec moi, un homme d’environ cinquante-cinq ans, de moyenne taille ; ridé, voûté, grisonnant ; à membres trapus ; avec un regard louche dans des yeux gris, un rire amer sur le visage ; sale, en guenilles, demi-nu, repoussant à voir.\par
Il paraît que la porte s’était ouverte, l’avait vomi, puis s’était refermée sans que je m’en fusse aperçu. Si la mort pouvait venir ainsi !\par
Nous nous sommes regardés quelques secondes fixement, l’homme et moi ; lui, prolongeant son rire qui ressemblait à un râle ; moi, demi-étonné, demi-effrayé.\par
— Qui êtes-vous ? lui ai-je dit enfin.\par
— Drôle de demande ! a-t-il répondu. Un friauche.\par
— Un friauche ! Qu’est-ce que cela veut dire ?\par
Cette question a redoublé sa gaieté.\par
— Cela veut dire, s’est-il écrié au milieu d’un éclat de rire, que le taule jouera au panier avec ma sorbonne dans six semaines, comme il va faire avec ta tronche dans six heures. — Ha ! ha ! il paraît que tu comprends maintenant.\par
 En effet, j’étais pâle, et mes cheveux se dressaient. C’était l’autre condamné, le condamné du jour, celui qu’on attendait à Bicêtre, mon héritier.\par
Il a continué :\par
— Que veux-tu ? voilà mon histoire à moi. Je suis fils d’un bon peigre ; c’est dommage que Charlot\footnote{ \noindent Le bourreau.
 } ait pris la peine un jour de lui attacher sa cravate. C’était quand régnait la potence, par la grâce de Dieu. A six ans, je n’avais plus ni père ni mère ; l’été, je faisais la roue dans la poussière au bord des routes, pour qu’on me jetât un sou par la portière des chaises de poste ; l’hiver, j’allais pieds nus dans la boue en soufflant dans mes doigts tout rouges ; on voyait mes cuisses à travers mon pantalon. A neuf ans, j’ai commencé à me servir de mes louches\footnote{ \noindent Mes mains.
 }, de temps en temps je vidais une fouillouse\footnote{ \noindent Une poche.
 }, je filais une pelure\footnote{ \noindent Je volais un manteau.
 } ; à dix ans, j’étais un marlou\footnote{ \noindent Un filou.
 }. Puis j’ai fait des connaissances ; à dix-sept, j’étais un grinche\footnote{ \noindent Un voleur.
 }. Je forçais une boutanche, je faussais une tournante\footnote{ \noindent Je forçais une boutique, je faussais une clef.
 }. On m’a pris. J’avais l’âge, on m’a envoyé ramer dans la petite marine\footnote{ \noindent Aux galères.
 }. Le bagne, c’est dur ; coucher sur une planche, boire de l’eau claire, manger du pain noir, traîner un imbécile de boulet qui ne sert à rien ; des  coups de bâton et des coups de soleil. Avec cela on est tondu, et moi qui avais de beaux cheveux châtains !… N’importe ! j’ai fait mon temps. Quinze ans, cela s’arrache ! J’avais trente-deux ans. Un beau matin on me donna une feuille de route et soixante-six francs que je m’étais amassés dans mes quinze ans de galères, en travaillant seize heures par jour, trente jours par mois, et douze mois par année. C’est égal, je voulais être honnête homme avec mes soixante-six francs, et j’avais de plus beaux sentiments sous mes guenilles qu’il n’y en a sous une serpillière de ratichon\footnote{ \noindent Une soutane d’abbé.
 }. Mais que les diables soient avec le passe-port ! il était jaune, et on avait écrit dessus \emph{forçat libéré.} Il fallait montrer cela partout où je passais et le présenter tous les huit jours au maire du village où l’on me forçait de tapiquer\footnote{ \noindent Habiter.
 }. La belle recommandation ! un galérien ! Je faisais peur, et les petits enfants se sauvaient, et l’on fermait les portes. Personne ne voulait me donner d’ouvrage. Je mangeai mes soixante-six francs. Et puis il fallut vivre. Je montrai mes bras bons au travail, on ferma les portes. J’offris ma journée pour quinze sous, pour dix sous, pour cinq sous. Point. Que faire ? Un jour, j’avais faim, je donnai un coup de coude dans le carreau d’un boulanger ; j’empoignai un pain, et le boulanger m’empoigna ; je ne mangeai pas le pain, et j’eus les galères à perpétuité, avec trois lettres de feu sur l’épaule. — Je te montrerai, si tu veux. — On appelle cette justice-là \emph{la  récidive}. Me voilà donc cheval de retour\footnote{ \noindent Ramené au bagne.
 }. On me remit à Toulon ; cette fois avec les bonnets verts\footnote{ \noindent Les condamnés à perpétuité.
 }. Il fallait m’évader. Pour cela, je n’avais que trois murs à percer, deux chaînes à couper, et j’avais un clou. Je m’évadai. On tira le canon d’alerte ; car, nous autres, nous sommes comme les cardinaux de Rome, habillés de rouge, et on tire le canon quand nous partons. Leur poudre alla aux moineaux. Cette fois, pas de passe-port jaune, mais pas d’argent non plus. Je rencontrai des camarades qui avaient aussi fait leur temps ou cassé leur ficelle. Leur coire\footnote{ \noindent Leur chef.
 } me proposa d’être des leurs ; on faisait la grande soûlasse sur le trimar\footnote{ \noindent On assassinait sur les grands chemins.
 }. J’acceptai, et je me mis à tuer pour vivre. C’était tantôt une diligence, tantôt une chaise de poste, tantôt un marchand de bœufs à cheval. On prenait l’argent, on laissait aller au hasard la bête ou la voiture, et l’on enterrait l’homme sous un arbre, en ayant soin que les pieds ne sortissent pas ; et puis on dansait sur la fosse, pour que la terre ne parût pas fraîchement remuée. J’ai vieilli comme cela, gîtant dans les broussailles, dormant aux belles étoiles, traqué de bois en bois, mais du moins libre et à moi. Tout a une fin, et autant celle-là qu’une autre. Les marchands de lacets\footnote{ \noindent Les gendarmes.
 }, une belle nuit, nous ont pris au collet. Mes fanandels\footnote{ \noindent Camarades.
 } se sont sauvés ; mais moi, le plus vieux, je suis resté sous  la griffe de ces chais à chapeaux galonnés. On m’a amené ici. J’avais déjà passé par tous les échelons de l’échelle, excepté un. Avoir volé un mouchoir ou tué un homme, c’était tout un pour moi désormais ; il y avait encore une récidive à m’appliquer. Je n’avais plus qu’à passer par le faucheur\footnote{ \noindent Le bourreau.
 }. Mon affaire a été courte. Ma foi, je commençais à vieillir et à n’être plus bon à rien. Mon père a épousé la veuve\footnote{ \noindent A été pendu.
 }, moi je me retire à l’abbaye de Mont’-à-Regret\footnote{ \noindent La guillotine.
 }. — Voilà, camarade.\par
J’étais resté stupide en l’écoutant. Il s’est remis à rire plus haut encore qu’en commençant, et a voulu me prendre la main. J’ai reculé avec horreur.\par
— L’ami, m’a-t-il dit, tu n’as pas l’air brave. Ne va pas faire le sinvre devant la carline\footnote{ \noindent Le poltron devant la mort.
 }. Vois-tu, il y a un mauvais moment à passer sur la placarde\footnote{ \noindent Place de Grève.
 } ; mais cela est sitôt fait ! Je voudrais être là pour te montrer la culbute. Mille dieux ! j’ai envie de ne pas me pourvoir, si l’on veut me faucher aujourd’hui avec toi. Le même prêtre nous servira à tous deux ; ça m’est égal d’avoir tes restes. Tu vois que je suis un bon garçon. Hein ! dis, veux-tu ? d’amitié !\par
Il a encore fait un pas pour s’approcher de moi.\par
— Monsieur, lui ai-je répondu en le repoussant, je vous remercie.\par
Nouveaux éclats de rire à ma réponse.\par
 — Ah ! ah ! monsieur, vousailles\footnote{ \noindent Vous.
 } êtes un marquis ! c’est un marquis !\par
Je l’ai interrompu :\par
— Mon ami, j’ai besoin de me recueillir, laissez-moi.\par
La gravité de ma parole l’a rendu pensif tout à coup. Il a remué sa tête grise et presque chauve ; puis, creusant avec ses ongles sa poitrine velue, qui s’offrait nue sous sa chemise ouverte :\par
— Je comprends, a-t-il murmuré entre ses dents ; au fait, le sanglier\footnote{ \noindent Le prêtre.
 } !…\par
Puis, après quelques minutes de silence :\par
— Tenez, m’a-t-il dit presque timidement, vous êtes un marquis, c’est fort bien ; mais vous avez là une belle redingote qui ne vous servira plus à grand’chose ! le taule la prendra. Donnez-la-moi, je la vendrai pour avoir du tabac.\par
J’ai ôté ma redingote et je la lui ai donnée. Il s’est mis à battre des mains avec une joie d’enfant. Puis, voyant que j’étais en chemise et que je grelottais :\par
— Vous avez froid, monsieur, mettez ceci ; il pleut, et vous seriez mouillé ; et puis il faut être décemment sur la charrette.\par
En parlant ainsi, il ôtait sa grosse veste de laine grise et la passait dans mes bras. Je le laissais faire.\par
Alors j’ai été m’appuyer contre le mur, et je ne saurais dire quel effet me faisait cet homme. Il s’était  mis à examiner la redingote que je lui avais donnée, et poussait à chaque instant des cris de joie.\par
— Les poches sont toutes neuves ! le collet n’est pas usé ! — j’en aurai au moins quinze francs. Quel bonheur ! du tabac pour mes six semaines !\par
La porte s’est rouverte. On venait nous chercher tous deux ; moi, pour me conduire à la chambre où les condamnés attendent l’heure ; lui, pour le mener à Bicêtre. Il s’est placé en riant au milieu du piquet qui devait l’emmener, et il disait aux gendarmes :\par
— Ah çà ! ne vous trompez pas ; nous avons changé de pelure, monsieur et moi ; mais ne me prenez pas à sa place. Diable ! cela ne m’arrangerait pas, maintenant que j’ai de quoi avoir du tabac !
 \section[{XXIV}]{XXIV}\phantomsection
\label{ch24}\renewcommand{\leftmark}{XXIV}

\noindent Ce vieux scélérat, il m’a pris ma redingote, car je ne la lui ai pas donnée, et puis il m’a laissé cette guenille, sa veste infâme. De qui vais-je avoir l’air ?\par
Je ne lui ai pas laissé prendre ma redingote par insouciance ou par charité. Non ; mais parce qu’il était plus fort que moi. Si j’avais refusé, il m’aurait battu avec ses gros poings.\par
Ah bien oui, charité ! j’étais plein de mauvais sentiments. J’aurais voulu pouvoir l’étrangler de mes mains, le vieux voleur ! pouvoir le piler sous mes pieds !\par
Je me sens le cœur plein de rage et d’amertume. Je crois que la poche au fiel a crevé. La mort rend méchant.
 \section[{XXV}]{XXV}\phantomsection
\label{ch25}\renewcommand{\leftmark}{XXV}

\noindent Ils m’ont amené dans une cellule où il n’y a que les quatre murs, avec beaucoup de barreaux à la fenêtre et beaucoup de verrous à la porte, cela va sans dire.\par
J’ai demandé une table, une chaise, et ce qu’il faut pour écrire. On m’a apporté tout cela.\par
Puis j’ai demandé un lit. Le guichetier m’a regardé de ce regard étonné qui semble dire : — A quoi bon ?\par
Cependant ils ont dressé un lit de sangle dans le coin. Mais en même temps un gendarme est venu s’installer dans ce qu’ils appellent \emph{ma chambre.} Est-ce qu’ils ont peur que je ne m’étrangle avec le matelas ?
 \section[{XXVI}]{XXVI}\phantomsection
\label{ch26}\renewcommand{\leftmark}{XXVI}

\noindent Il est dix heures.\par
O ma pauvre petite fille ! encore six heures, et je serai mort ! je serai quelque chose d’immonde qui traînera sur la table froide des amphithéâtres ; une tête qu’on moulera d’un côté, un tronc qu’on disséquera de l’autre ; puis de ce qui restera, on en mettra plein une bière, et le tout ira à Clamart.\par
Voilà ce qu’ils vont faire de ton père, ces hommes dont aucun ne me hait, qui tous me plaignent et tous pourraient me sauver. Ils vont me tuer. Comprends-tu cela, Marie ? me tuer de sang-froid, en cérémonie, pour le bien de la chose ! Ah ! grand Dieu !\par
Pauvre petite ! ton père, qui t’aimait tant, ton père qui baisait ton petit cou blanc et parfumé, qui passait la main sans cesse dans les boucles de tes cheveux comme sur de la soie, qui prenait ton joli visage rond dans sa main, qui te faisait sauter sur ses genoux, et le soir joignait tes deux petites mains pour prier Dieu !\par
Qui est-ce qui te fera tout cela maintenant ? Qui est-ce qui t’aimera ? Tous les enfants de ton âge auront  des pères, excepté toi. Comment te déshabitueras-tu, mon enfant, du jour de l’an, des étrennes, des beaux joujoux, des bonbons et des baisers ? — Comment te déshabitueras-tu, malheureuse orpheline, de boire et de manger ?\par
Oh ! si ces jurés l’avaient vue, au moins, ma jolie petite Marie, ils auraient compris qu’il ne faut pas tuer le père d’un enfant de trois ans.\par
Et quand elle sera grande, si elle va jusque-là, que deviendra-t-elle ? Son père sera un des souvenirs du peuple de Paris. Elle rougira de moi et de mon nom ; elle sera méprisée, repoussée, vile à cause de moi, de moi qui l’aime de toutes les tendresses de mon cœur. O ma petite Marie bien-aimée ! Est-il bien vrai que tu auras honte et horreur de moi ?\par
Misérable ! quel crime j’ai commis, et quel crime je fais commettre à la société !\par
Oh ! est-il bien vrai que je vais mourir avant la fin du jour ? Est-il bien vrai que c’est moi ? Ce bruit sourd de cris que j’entends au dehors, ce flot de peuple joyeux qui déjà se hâte sur les quais, ces gendarmes qui s’apprêtent dans leurs casernes, ce prêtre en robe noire, cet autre homme aux mains rouges, c’est pour moi ! c’est moi qui vais mourir ! moi, le même qui est ici, qui vit, qui se meut, qui respire, qui est assis à cette table, laquelle ressemble à une autre table, et pourrait bien être ailleurs ; moi, enfin, ce moi que je touche et que je sens, et dont le vêtement fait les plis que voilà !
 \section[{XXVIII}]{XXVIII}\phantomsection
\label{ch27}\renewcommand{\leftmark}{XXVIII}

\noindent Encore si je savais comment cela est fait et de quelle façon on meurt là-dessus ; mais, c’est horrible, je ne le sais pas.\par
Le nom de la chose est effroyable, et je ne comprends point comment j’ai pu jusqu’à présent l’écrire et le prononcer.\par
La combinaison de ces dix lettres, leur aspect, leur physionomie est bien faite pour réveiller une idée épouvantable, et le médecin de malheur qui a inventé la chose avait un nom prédestiné.\par
L’image que j’y attache, à ce mot hideux, est vague, indéterminée, et d’autant plus sinistre. Chaque syllabe est comme une pièce de la machine. J’en construis et j’en démolis sans cesse dans mon esprit la monstrueuse charpente.\par
Je n’ose faire une question là-dessus, mais il est affreux de ne savoir ce que c’est, ni comment s’y prendre. Il paraît qu’il y a une bascule et qu’on vous couche sur le ventre… — Ah ! mes cheveux blanchiront avant que ma tête ne tombe !
 \section[{XXVIII}]{XXVIII}\phantomsection
\label{ch28}\renewcommand{\leftmark}{XXVIII}

\noindent Je l’ai cependant entrevue une fois.\par
Je passais sur la place de Grève, en voiture, un jour, vers onze heures du matin. Tout à coup la voiture s’arrêta.\par
Il y avait foule sur la place. Je mis la tête à la portière. Une populace encombrait la Grève et le quai, et des femmes, des hommes, des enfants étaient debout sur le parapet. Au-dessus des têtes, on voyait une espèce d’estrade en bois rouge que trois hommes échafaudaient.\par
Un condamné devait être exécuté le jour même, et l’on bâtissait la machine.\par
Je détournai la tête avant d’avoir vu. A côté de la voiture, il y avait une femme qui disait à un enfant :\par
— Tiens, regarde ! le couteau coule mal, ils vont graisser la rainure avec un bout de chandelle.\par
C’est probablement là qu’ils en sont aujourd’hui. Onze heures viennent de sonner. Ils graissent sans doute la rainure.\par
Ah ! cette fois, malheureux, je ne détournerai pas la tête.
 \section[{XXIX}]{XXIX}\phantomsection
\label{ch29}\renewcommand{\leftmark}{XXIX}

\noindent O ma grâce ! ma grâce ! on me fera peut-être grâce. Le roi ne m’en veut pas. Qu’on aille chercher mon avocat ! vite l’avocat ! Je veux bien des galères. Cinq ans de galères, et que tout soit dit, — ou vingt ans, — ou à perpétuité avec le fer rouge. Mais grâce de la vie !\par
Un forçat, cela marche encore, cela va et vient, cela voit le soleil.
 \section[{XXX}]{XXX}\phantomsection
\label{ch30}\renewcommand{\leftmark}{XXX}

\noindent Le prêtre est revenu.\par
Il a des cheveux blancs, l’air très doux, une bonne et respectable figure ; c’est en effet un homme excellent et charitable. Ce matin, je l’ai vu vider sa bourse dans les mains des prisonniers. D’où vient que sa voix n’a rien qui émeuve et qui soit ému ? D’où vient qu’il ne m’a rien dit encore qui m’ait pris par l’intelligence ou par le cœur ?\par
Ce matin, j’étais égaré. J’ai à peine entendu ce qu’il m’a dit. Cependant ses paroles m’ont semblé inutiles, et je suis resté indifférent ; elles ont glissé comme cette pluie froide sur cette vitre glacée.\par
Cependant, quand il est rentré tout à l’heure près de moi, sa vue m’a fait du bien. C’est parmi tous ces hommes le seul qui soit encore homme pour moi, me suis-je dit. Et il m’a pris une ardente soif de bonnes et consolantes paroles.\par
Nous nous sommes assis, lui sur la chaise, moi sur le lit. Il m’a dit : — Mon fils… — Ce mot m’a ouvert le cœur. Il a continué :\par
— Mon fils, croyez-vous en Dieu ?\par
— Oui, mon père, lui ai-je répondu.\par
 — Croyez-vous en la sainte église catholique, apostolique et romaine ?\par
— Volontiers, lui ai-je dit.\par
— Mon fils, a-t-il repris, vous avez l’air de douter.\par
Alors il s’est mis à parler. Il a parlé longtemps ; il a dit beaucoup de paroles ; puis, quand il a cru avoir fini, il s’est levé et m’a regardé pour la première fois depuis le commencement de son discours, en m’interrogeant :\par
— Eh bien ?\par
Je proteste que je l’avais écouté avec avidité d’abord, puis avec attention, puis avec dévouement.\par
Je me suis levé aussi.\par
— Monsieur, lui ai-je répondu, laissez-moi seul, je vous prie.\par
Il m’a demandé :\par
— Quand reviendrai-je ?\par
— Je vous le ferai savoir.\par
Alors il est sorti sans rien dire, mais en hochant la tête, comme se disant à lui-même :\par
— Un impie !\par
Non, si bas que je sois tombé, je ne suis pas un impie, et Dieu m’est témoin que je crois en lui. Mais que m’a-t-il dit, ce vieillard ? rien de senti, rien d’attendri, rien de pleuré, rien d’arraché de l’âme, rien qui vînt de son cœur pour aller au mien, rien qui fût de lui à moi. Au contraire, je ne sais quoi de vague, d’inaccentué, d’applicable à tout et à tous ; emphatique où il eût été besoin de profondeur, plat où il eût fallu être simple ; une espèce de sermon sentimental et d’élégie  théologique. Çà et là, une citation latine en latin. Saint Augustin, saint Grégoire, que sais-je ? Et puis, il avait l’air de réciter une leçon déjà vingt fois récitée, de repasser un thème, oblitéré dans sa mémoire à force d’être su. Pas un regard dans l’œil, pas un accent dans la voix, pas un geste dans les mains.\par
Et comment en serait-il autrement ? Ce prêtre est l’aumônier en titre de la prison. Son état est de consoler et d’exhorter, et il vit de cela. Les forçats, les patients sont du ressort de son éloquence. Il les confesse et les assiste, parce qu’il a sa place à faire. Il a vieilli à mener des hommes mourir. Depuis longtemps il est habitué à ce qui fait frissonner les autres ; ses cheveux, bien poudrés à blanc, ne se dressent plus ; le bagne et l’échafaud sont de tous les jours pour lui. Il est blasé. Probablement il a son cahier ; telle page les galériens, telle page les condamnés à mort. On l’avertit la veille qu’il y aura quelqu’un à consoler le lendemain à telle heure ; il demande ce que c’est, galérien ou supplicié, et relit la page ; et puis il vient. De cette façon, il advient que ceux qui vont à Toulon et ceux qui vont à la Grève sont un lieu commun pour lui, et qu’il est un lieu commun pour eux.\par
Oh ! qu’on m’aille donc, au lieu de cela, chercher quelque jeune vicaire, quelque vieux curé, au hasard, dans la première paroisse venue ; qu’on le prenne au coin de son feu, lisant son livre et ne s’attendant à rien, et qu’on lui dise :\par
— Il y a un homme qui va mourir, et il faut que ce soit vous qui le consoliez. Il faut que vous soyez là  quand on lui liera les mains, là quand on lui coupera les cheveux ; que vous montiez dans sa charrette avec votre crucifix pour lui cacher le bourreau ; que vous soyez cahoté avec lui par le pavé jusqu’à la Grève ; que vous traversiez avec lui l’horrible foule buveuse de sang ; que vous l’embrassiez au pied de l’échafaud, et que vous restiez jusqu’à ce que la tête soit ici et le corps là.\par
Alors, qu’on me l’amène, tout palpitant, tout frissonnant de la tête aux pieds ; qu’on me jette entre ses bras, à ses genoux ; et il pleurera, et nous pleurerons, et il sera éloquent, et je serai consolé, et mon cœur se dégonflera dans le sien, et il prendra mon âme, et je prendrai son Dieu.\par
Mais, ce bon vieillard, qu’est-il pour moi ? que suis-je pour lui ? un individu de l’espèce malheureuse, une ombre comme il en a déjà tant vu, une unité à ajouter au chiffre des exécutions.\par
J’ai peut-être tort de le repousser ainsi ; c’est lui qui est bon et moi qui suis mauvais. Hélas ! ce n’est pas ma faute. C’est mon souffle de condamné qui gâte et flétrit tout.\par
On vient de m’apporter de la nourriture ; ils ont cru que je devais avoir besoin. Une table délicate et recherchée, un poulet, il me semble, et autre chose encore. Eh bien ! j’ai essayé de manger ; mais, à la première bouchée, tout est tombé de ma bouche, tant cela m’a paru amer et fétide !
 \section[{XXXI}]{XXXI}\phantomsection
\label{ch31}\renewcommand{\leftmark}{XXXI}

\noindent Il vient d’entrer un monsieur, le chapeau sur la tête, qui m’a à peine regardé, puis a ouvert un pied-de-roi et s’est mis à mesurer de bas en haut les pierres du mur, parlant d’une voix très haute pour dire tantôt : \emph{C’est cela ;} tantôt : \emph{Ce n’est pas cela.}\par
J’ai demandé au gendarme qui c’était. Il paraît que c’est une espèce de sous-architecte employé à la prison.\par
De son côté, sa curiosité s’est éveillée sur mon compte. Il a échangé quelques demi-mots avec le porte-clefs qui l’accompagnait ; puis a fixé un instant les yeux sur moi, a secoué la tête d’un air insouciant, et s’est remis à parler à haute voix et à prendre des mesures.\par
Sa besogne finie, il s’est approché de moi en me disant avec sa voix éclatante :\par
— Mon bon ami, dans six mois cette prison sera beaucoup mieux.\par
Et son geste semblait ajouter :\par
 — Vous n’en jouirez pas, c’est dommage.\par
Il souriait presque. J’ai cru voir le moment où il allait me railler doucement, comme on plaisante une jeune mariée le soir de ses noces.\par
Mon gendarme, vieux soldat à chevrons, s’est chargé de la réponse.\par
— Monsieur, lui a-t-il dit, on ne parle pas si haut dans la chambre d’un mort.\par
L’architecte s’en est allé.\par
Moi, j’étais là, comme une des pierres qu’il mesurait.
 \section[{XXXII}]{XXXII}\phantomsection
\label{ch32}\renewcommand{\leftmark}{XXXII}

\noindent Et puis, il m’est arrivé une chose ridicule.\par
On est venu relever mon bon vieux gendarme, auquel, ingrat égoïste que je suis, je n’ai seulement pas serré la main. Un autre l’a remplacé, homme à front déprimé, des yeux de bœuf, une figure inepte.\par
Au reste, je n’y avais fait aucune attention. Je tournais le dos à la porte, assis devant la table ; je tâchais de rafraîchir mon front avec ma main, et mes pensées troublaient mon esprit.\par
Un léger coup, frappé sur mon épaule, m’a fait tourner la tête. C’était le nouveau gendarme, avec qui j’étais seul.\par
Voici à peu près de quelle façon il m’a adressé la parole.\par
— Criminel, avez-vous bon cœur ?\par
— Non, lui ai-je dit.\par
La brusquerie de ma réponse a paru le déconcerter. Cependant il a repris en hésitant :\par
— On n’est pas méchant pour le plaisir de l’être.\par
— Pourquoi non ? ai-je répliqué. Si vous n’avez que cela à me dire, laissez-moi. Où voulez-vous en venir ?\par
 — Pardon, mon criminel, a-t-il répondu. Deux mots seulement. Voici. Si vous pouviez faire le bonheur d’un pauvre homme, et que cela ne vous coûtât rien, est-ce que vous ne le feriez pas ?\par
J’ai haussé les épaules.\par
— Est-ce que vous arrivez de Charenton ? Vous choisissez un singulier vase pour y puiser du bonheur. Moi, faire le bonheur de quelqu’un !\par
Il a baissé la voix et pris un air mystérieux, ce qui n’allait pas à sa figure idiote.\par
— Oui, criminel, oui bonheur, oui fortune. Tout cela me sera venu de vous. Voici. Je suis un pauvre gendarme. Le service est lourd, la paye est légère ; mon cheval est à moi et me ruine. Or, je mets à la loterie pour contre-balancer. Il faut bien avoir une industrie. Jusqu’ici il ne m’a manqué pour gagner que d’avoir de bons numéros. J’en cherche partout de sûrs ; je tombe toujours à côté. Je mets le 76 ; il sort le 77. J’ai beau les nourrir, ils ne viennent pas… — Un peu de patience, s’il vous plaît ; je suis à la fin. — Or, voici une belle occasion pour moi. Il paraît, pardon, criminel, que vous passez aujourd’hui. Il est certain que les morts qu’on fait périr comme cela voient la loterie d’avance. Promettez-moi de venir demain soir, qu’est-ce que cela vous fait ? me donner trois numéros, trois bons. Hein ? — Je n’ai pas peur des revenants, soyez tranquille. — Voici mon adresse : Caserne Popincourt, escalier A, n\textsuperscript{o} 26, au fond du corridor. Vous me reconnaîtrez bien, n’est-ce pas ? — Venez même ce soir, si cela vous est plus commode.\par
 J’aurais dédaigné de lui répondre, à cet imbécile, si une espérance folle ne m’avait traversé l’esprit. Dans la position désespérée où je suis, on croit par moments qu’on briserait une chaîne avec un cheveu.\par
— Écoute, lui ai-je dit en faisant le comédien autant que le peut faire celui qui va mourir, je puis en effet te rendre plus riche que le roi, te faire gagner des millions. — A une condition.\par
Il ouvrait des yeux stupides.\par
— Laquelle ? laquelle ? tout pour vous plaire, mon criminel.\par
— Au lieu de trois numéros, je t’en promets quatre. Change d’habits avec moi.\par
— Si ce n’est que cela ! s’est-il écrié en défaisant les premières agrafes de son uniforme.\par
Je m’étais levé de ma chaise. J’observais tous ses mouvements, mon cœur palpitait. Je voyais déjà les portes s’ouvrir devant l’uniforme de gendarme, et la place, et la rue, et le Palais de Justice derrière moi !\par
Mais il s’est retourné d’un air indécis.\par
— Ah çà ! ce n’est pas pour sortir d’ici ?\par
J’ai compris que tout était perdu. Cependant j’ai tenté un dernier effort, bien inutile et bien insensé !\par
— Si fait, lui ai-je dit, mais ta fortune est faite…\par
Il m’a interrompu.\par
— Ah bien non ! tiens ! et mes numéros ! pour qu’ils soient bons, il faut que vous soyez mort.\par
Je me suis rassis, muet et plus désespéré de toute l’espérance que j’avais eue.
 \section[{XXXIII}]{XXXIII}\phantomsection
\label{ch33}\renewcommand{\leftmark}{XXXIII}

\noindent J’ai fermé les yeux, et j’ai mis les mains dessus, et j’ai tâché d’oublier le présent dans le passé. Tandis que je rêve, les souvenirs de mon enfance et de ma jeunesse me reviennent un à un, doux, calmes, riants, comme des îles de fleurs sur ce gouffre de pensées noires et confuses qui tourbillonnent dans mon cerveau.\par
Je me revois enfant, écolier rieur et frais, jouant, courant, criant avec mes frères dans la grande allée verte de ce jardin sauvage où ont coulé mes premières années, ancien enclos de religieuses que domine de sa tête de plomb le sombre dôme du Val-de-Grâce.\par
Et puis, quatre ans plus tard, m’y voilà encore, toujours enfant, mais déjà rêveur et passionné. Il y a une jeune fille dans le solitaire jardin.\par
La petite espagnole, avec ses grands yeux et ses grands cheveux, sa peau brune et dorée, ses lèvres  rouges et ses joues roses, l’andalouse de quatorze ans, Pepa.\par
Nos mères nous ont dit d’aller courir ensemble ; nous sommes venus nous promener.\par
On nous a dit de jouer, et nous causons, enfants du même âge, non du même sexe.\par
Pourtant, il n’y a encore qu’un an, nous courions, nous luttions ensemble. Je disputais à Pepita la plus belle pomme du pommier ; je la frappais pour un nid d’oiseau. Elle pleurait ; je disais : C’est bien fait ! et nous allions tous deux nous plaindre ensemble à nos mères, qui nous donnaient tort tout haut et raison tout bas.\par
Maintenant elle s’appuie sur mon bras, et je suis tout fier et tout ému. Nous marchons lentement, nous parlons bas. Elle laisse tomber son mouchoir ; je le lui ramasse. Nos mains tremblent en se touchant. Elle me parle des petits oiseaux, de l’étoile qu’on voit là-bas, du couchant vermeil derrière les arbres, ou bien de ses amies de pension, de sa robe et de ses rubans. Nous disons des choses innocentes, et nous rougissons tous deux. La petite fille est devenue jeune fille.\par
Ce soir-là, — c’était un soir d’été, nous étions sous les marronniers, au fond du jardin. Après un de ces longs silences qui remplissaient nos promenades, elle quitta tout à coup mon bras et me dit : Courons !\par
 Je la vois encore ; elle était tout en noir, en deuil de sa grand’mère. Il lui passa par la tête une idée d’enfant, Pepa redevint Pepita, elle me dit : Courons !\par
Et elle se mit à courir devant moi avec sa taille fine comme le corset d’une abeille et ses petits pieds qui relevaient sa robe jusqu’à mi-jambe. Je la poursuivis, elle fuyait ; le vent de sa course soulevait par moments sa pèlerine noire, et me laissait voir son dos brun et frais.\par
J’étais hors de moi. Je l’atteignis près du vieux puisard en ruine ; je la pris par la ceinture, du droit de victoire, et je la fis asseoir sur un banc de gazon ; elle ne résista pas. Elle était essoufflée et riait. Moi, j’étais sérieux, et je regardais ses prunelles noires à travers ses cils noirs.\par
— Asseyez-vous là, me dit-elle. Il fait encore grand jour, lisons quelque chose. Avez-vous un livre ?\par
J’avais sur moi le tome second des Voyages de Spallanzani. J’ouvris au hasard, je me rapprochai d’elle, elle appuya son épaule à mon épaule, et nous nous mîmes à lire chacun de notre côté, tout bas, la même page. Avant de tourner le feuillet, elle était toujours obligée de m’attendre. Mon esprit allait moins vite que le sien.\par
— Avez-vous fini ? me disait-elle, que j’avais à peine commencé.\par
Cependant nos têtes se touchaient, nos cheveux se mêlaient, nos haleines peu à peu se rapprochèrent, et nos bouches tout à coup.\par
Quand nous voulûmes continuer notre lecture, le ciel était étoilé.\par
— Oh ! maman, maman, dit-elle en rentrant, si tu savais comme nous avons couru !\par
 Moi, je gardais le silence.\par
— Tu ne dis rien, me dit ma mère, tu as l’air triste.\par
J’avais le paradis dans le cœur.\par
C’est une soirée que je me rappellerai toute ma vie.\par
Toute ma vie !
 \section[{XXXIV}]{XXXIV}\phantomsection
\label{ch34}\renewcommand{\leftmark}{XXXIV}

\noindent Une heure vient de sonner, je ne sais laquelle ; j’entends mal le marteau de l’horloge. Il me semble que j’ai un bruit d’orgue dans les oreilles ; ce sont mes dernières pensées qui bourdonnent.\par
A ce moment suprême où je me recueille dans mes souvenirs, j’y retrouve mon crime avec horreur ; mais je voudrais me repentir davantage encore. J’avais plus de remords avant ma condamnation ; depuis, il semble qu’il n’y ait plus de place que pour les pensées de mort. Pourtant, je voudrais bien me repentir beaucoup.\par
Quand j’ai rêvé une minute à ce qu’il y a de passé dans ma vie, et que j’en reviens au coup de hache qui doit la terminer tout à l’heure, je frissonne comme d’une chose nouvelle. Ma belle enfance ! ma belle jeunesse ! étoffe dorée dont l’extrémité est sanglante. Entre alors et à présent il y a une rivière de sang ; le sang de l’autre et le mien.\par
Si on lit un jour mon histoire, après tant d’années  d’innocence et de bonheur, on ne voudra pas croire à cette année exécrable, qui s’ouvre par un crime et se clôt par un supplice ; elle aura l’air dépareillée.\par
Et pourtant, misérables lois et misérables hommes, je n’étais pas un méchant !\par
Oh ! mourir dans quelques heures, et penser qu’il y a un an, à pareil jour, j’étais libre et pur, que je faisais mes promenades d’automne, que j’errais sous les arbres, et que je marchais dans les feuilles !
 \section[{XXXV}]{XXXV}\phantomsection
\label{ch35}\renewcommand{\leftmark}{XXXV}

\noindent En ce moment même, il y a tout auprès de moi, dans ces maisons qui font cercle autour du Palais et de la Grève, et partout dans Paris, des hommes qui vont et viennent, causent et rient, lisent le journal, pensent à leurs affaires ; des marchands qui vendent ; des jeunes filles qui préparent leurs robes de bal pour ce soir ; des mères qui jouent avec leurs enfants !
 \section[{XXXVI}]{XXXVI}\phantomsection
\label{ch36}\renewcommand{\leftmark}{XXXVI}

\noindent Je me souviens qu’un jour, étant enfant, j’allai voir le bourdon de Notre-Dame.\par
J’étais déjà étourdi d’avoir monté le sombre escalier en colimaçon, d’avoir parcouru la frêle galerie qui lie les deux tours, d’avoir eu Paris sous les pieds, quand j’entrai dans la cage de pierre et de charpente où pend le bourdon avec son battant, qui pèse un millier.\par
J’avançai en tremblant sur les planches mal jointes, regardant à distance cette cloche si fameuse parmi les enfants et le peuple de Paris, et ne remarquant pas sans effroi que les auvents couverts d’ardoises qui entourent le clocher de leurs plans inclinés étaient au niveau de mes pieds. Dans les intervalles, je voyais, en quelque sorte à vol d’oiseau, la place du Parvis-Notre-Dame, et les passants comme des fourmis.\par
Tout à coup l’énorme cloche tinta ; une vibration profonde remua l’air, fit osciller la lourde tour. Le plancher sautait sur les poutres. Le bruit faillit me renverser ; je chancelai, prêt à tomber, prêt à glisser  sur les auvents d’ardoises en pente. De terreur, je me couchai sur les planches, les serrant étroitement de mes deux bras, sans parole, sans haleine, avec ce formidable tintement dans les oreilles, et, sous les yeux, ce précipice, cette place profonde où se croisaient tant de passants paisibles et enviés.\par
Eh bien ! il me semble que je suis encore dans la tour du bourdon. C’est tout ensemble un étourdissement et un éblouissement. Il y a comme un bruit de cloche qui ébranle les cavités de mon cerveau, et autour de moi je n’aperçois plus cette vie plane et tranquille que j’ai quittée, et où les autres hommes cheminent encore, que de loin et à travers les crevasses d’un abîme.
 \section[{XXXVII}]{XXXVII}\phantomsection
\label{ch37}\renewcommand{\leftmark}{XXXVII}

\noindent L’hôtel de ville est un édifice sinistre.\par
Avec son toit aigu et roide, son clocheton bizarre, son grand cadran blanc, ses étages à petites colonnes, ses mille croisées, ses escaliers usés par les pas, ses deux arches à droite et à gauche, il est là, de plain-pied avec la Grève ; sombre, lugubre, la face toute rongée de vieillesse, et si noir, qu’il est noir au soleil.\par
Les jours d’exécution, il vomit des gendarmes de toutes ses portes, et regarde le condamné avec toutes ses fenêtres.\par
Et le soir, son cadran, qui a marqué l’heure, reste lumineux sur sa façade ténébreuse.
 \section[{XXXVIII}]{XXXVIII}\phantomsection
\label{ch38}\renewcommand{\leftmark}{XXXVIII}

\noindent Il est une heure et quart.\par
Voici ce que j’éprouve maintenant :\par
Une violente douleur de tête, les reins froids, le front brûlant. Chaque fois que je me lève ou que je me penche, il me semble qu’il y a un liquide qui flotte dans mon cerveau, et qui fait battre ma cervelle contre les parois du crâne.\par
J’ai des tressaillements convulsifs, et de temps en temps la plume tombe de mes mains comme par une secousse galvanique.\par
Les yeux me cuisent comme si j’étais dans la fumée.\par
J’ai mal dans les coudes.\par
Encore deux heures et quarante-cinq minutes, et je serai guéri.
 \section[{XXXIX}]{XXXIX}\phantomsection
\label{ch39}\renewcommand{\leftmark}{XXXIX}

\noindent Ils disent que ce n’est rien, qu’on ne souffre pas, que c’est une fin douce, que la mort de cette façon est bien simplifiée.\par
Eh ! qu’est-ce donc que cette agonie de six semaines et ce râle de tout un jour ? Qu’est-ce que les angoisses de cette journée irréparable, qui s’écoule si lentement et si vite ? Qu’est-ce que cette échelle de tortures qui aboutit à l’échafaud ?\par
Apparemment ce n’est pas là souffrir.\par
Ne sont-ce pas les mêmes convulsions, que le sang s’épuise goutte à goutte, ou que l’intelligence s’éteigne pensée à pensée ?\par
Et puis, on ne souffre pas, en sont-ils sûrs ? Qui le leur a dit ? Conte-t-on que jamais une tête coupée se soit dressée sanglante au bord du panier, et qu’elle ait crié au peuple : Cela ne fait pas de mal !\par
Y a-t-il des morts de leur façon qui soient venus  les remercier et leur dire : C’est bien inventé. Tenez-vous-en là. La mécanique est bonne.\par
Est-ce Robespierre ? Est-ce Louis XVI ?…\par
Non, rien ! moins qu’une minute, moins qu’une seconde, et la chose est faite. — Se sont-ils jamais mis, seulement en pensée, à la place de celui qui est là, au moment où le lourd tranchant qui tombe mord la chair, rompt les nerfs, brise les vertèbres… Mais quoi ! une demi-seconde ! la douleur est escamotée… Horreur !
 \section[{XL}]{XL}\phantomsection
\label{ch40}\renewcommand{\leftmark}{XL}

\noindent Il est singulier que je pense sans cesse au roi. J’ai beau faire, beau secouer la tête, j’ai une voix dans l’oreille qui me dit toujours :\par
— Il y a dans cette même ville, à cette même heure, et pas bien loin d’ici, dans un autre palais, un homme qui a aussi des gardes à toutes ses portes, un homme unique comme toi dans le peuple, avec cette différence qu’il est aussi haut que tu es bas. Sa vie entière, minute par minute, n’est que gloire, grandeur, délices, enivrement. Tout est autour de lui amour, respect, vénération. Les voix les plus hautes deviennent basses en lui parlant et les fronts les plus fiers ploient. Il n’a que de la soie et de l’or sous les yeux. A cette heure, il tient quelque conseil de ministres où tous sont de son avis ; ou bien songe à la chasse de demain, au bal de ce soir, sûr que la fête viendra à l’heure, et laissant à d’autres le travail de ses plaisirs. Eh bien ! cet homme est de chair et d’os comme toi !  — Et pour qu’à l’instant même l’horrible échafaud s’écroulât, pour que tout te fût rendu, vie, liberté, fortune, famille, il suffirait qu’il écrivît avec cette plume les sept lettres de son nom au bas d’un morceau de papier, ou même que son carrosse rencontrât ta charrette ! — Et il est bon, et il ne demanderait pas mieux peut-être, et il n’en sera rien !
 \section[{XLI}]{XLI}\phantomsection
\label{ch41}\renewcommand{\leftmark}{XLI}

\noindent Eh bien donc ! ayons courage avec la mort, prenons cette horrible idée à deux mains, et considérons-la en face. Demandons-lui compte de ce qu’elle est, sachons ce qu’elle nous veut, retournons-la en tous sens, épelons l’énigme, et regardons d’avance dans le tombeau.\par
Il me semble que, dès que mes yeux seront fermés, je verrai une grande clarté et des abîmes de lumière où mon esprit roulera sans fin. Il me semble que le ciel sera lumineux de sa propre essence, que les astres y feront des taches obscures, et qu’au lieu d’être comme pour les yeux vivants des paillettes d’or sur du velours noir, ils sembleront des points noirs sur du drap d’or.\par
Ou bien, misérable que je suis, ce sera peut-être un gouffre hideux, profond, dont les parois seront tapissées de ténèbres, et où je tomberai sans cesse en voyant des formes remuer dans l’ombre.\par
Ou bien, en m’éveillant après le coup, je me  trouverai peut-être sur quelque surface plane et humide, rampant dans l’obscurité et tournant sur moi-même comme une tête qui roule. Il me semble qu’il y aura un grand vent qui me poussera, et que je serai heurté çà et là par d’autres têtes roulantes. Il y aura par places des mares et des ruisseaux d’un liquide inconnu et tiède ; tout sera noir. Quand mes yeux, dans leur rotation, seront tournés en haut, ils ne verront qu’un ciel d’ombre, dont les couches épaisses pèseront sur eux, et au loin dans le fond de grandes arches de fumée plus noires que les ténèbres. Ils verront aussi voltiger dans la nuit de petites étincelles rouges, qui, en s’approchant, deviendront des oiseaux de feu. Et ce sera ainsi toute l’éternité.\par
Il se peut bien aussi qu’à certaines dates les morts de la Grève se rassemblent par de noires nuits d’hiver sur la place qui est à eux. Ce sera une foule pâle et sanglante, et je n’y manquerai pas. Il n’y aura pas de lune, et l’on parlera à voix basse. L’hôtel de ville sera là, avec sa façade vermoulue, son toit déchiqueté, et son cadran qui aura été sans pitié pour tous. Il y aura sur la place une guillotine de l’enfer, où un démon exécutera un bourreau ; ce sera à quatre heures du matin. A notre tour nous ferons foule autour.\par
Il est probable que cela est ainsi. Mais si ces morts-là reviennent, sous quelle forme reviennent-ils ? Que gardent-ils de leur corps incomplet et mutilé ? Que choisissent-ils ? Est-ce la tête ou le tronc qui est spectre ?\par
Hélas ! qu’est-ce que la mort fait avec notre âme ?  quelle nature lui laisse-t-elle ? qu’a-t-elle à lui prendre ou à lui donner ? où la met-elle ? lui prête-t-elle quelquefois des yeux de chair pour regarder sur la terre et pleurer ?\par
Ah ! un prêtre ! un prêtre qui sache cela ! Je veux un prêtre, et un crucifix à baiser !\par
Mon Dieu, toujours le même !
 \section[{XLII}]{XLII}\phantomsection
\label{ch42}\renewcommand{\leftmark}{XLII}

\noindent Je l’ai prié de me laisser dormir, et je me suis jeté sur le lit.\par
En effet, j’avais un flot de sang dans la tête, qui m’a fait dormir. C’est mon dernier sommeil, de cette espèce.\par
J’ai fait un rêve.\par
J’ai rêvé que c’était la nuit. Il me semblait que j’étais dans mon cabinet avec deux ou trois de mes amis, je ne sais plus lesquels.\par
Ma femme était couchée dans la chambre à coucher, à côté, et dormait avec son enfant.\par
Nous parlions à voix basse, mes amis et moi, et ce que nous disions nous effrayait.\par
Tout à coup il me sembla entendre un bruit quelque part dans les autres pièces de l’appartement ; un bruit faible, étrange, indéterminé.\par
Mes amis avaient entendu comme moi. Nous écoutâmes ; c’était comme une serrure qu’on ouvre sourdement, comme un verrou qu’on scie à petit bruit.\par
 Il y avait quelque chose qui nous glaçait ; nous avions peur. Nous pensâmes que peut-être c’étaient des voleurs qui s’étaient introduits chez moi, à cette heure si avancée de la nuit.\par
Nous résolûmes d’aller voir. Je me levai, je pris la bougie. Mes amis me suivaient, un à un.\par
Nous traversâmes la chambre à coucher, à côté. Ma femme dormait avec son enfant.\par
Puis nous arrivâmes dans le salon. Rien. Les portraits étaient immobiles dans leurs cadres d’or sur la tenture rouge. Il me sembla que la porte du salon à la salle à manger n’était point à sa place ordinaire.\par
Nous entrâmes dans la salle à manger ; nous en fîmes le tour. Je marchais le premier. La porte sur l’escalier était bien fermée, les fenêtres aussi. Arrivé près du poêle, je vis que l’armoire au linge était ouverte, et que la porte de cette armoire était tirée sur l’angle du mur, comme pour le cacher.\par
Cela me surprit. Nous pensâmes qu’il y avait quelqu’un derrière la porte.\par
Je portai la main à cette porte pour refermer l’armoire ; elle résista. Étonné, je tirai plus fort, elle céda brusquement, et nous découvrit une petite vieille, les mains pendantes, les yeux fermés, immobile, debout, et comme collée dans l’angle du mur.\par
Cela avait quelque chose de hideux, et mes cheveux se dressent d’y penser.\par
Je demandai à la vieille :\par
— Que faites-vous là ?\par
Elle ne répondit pas.\par
 Je lui demandai :\par
— Qui êtes-vous ?\par
Elle ne répondit pas, ne bougea pas, et resta les yeux fermés.\par
Mes amis dirent :\par
— C’est sans doute la complice de ceux qui sont entrés avec de mauvaises pensées ; ils se sont échappés en nous entendant venir ; elle n’aura pu fuir, et s’est cachée là.\par
Je l’ai interrogée de nouveau ; elle est demeurée sans voix, sans mouvement, sans regard.\par
Un de nous l’a poussée à terre, elle est tombée.\par
Elle est tombée tout d’une pièce, comme un morceau de bois, comme une chose morte.\par
Nous l’avons remuée du pied, puis deux de nous l’ont relevée et de nouveau appuyée au mur. Elle n’a donné aucun signe de vie. On lui a crié dans l’oreille, elle est restée muette comme si elle était sourde.\par
Cependant, nous perdions patience, et il y avait de la colère dans notre terreur. Un de nous m’a dit :\par
— Mettez-lui la bougie sous le menton.\par
Je lui ai mis la mèche enflammée sous le menton. Alors elle a ouvert un œil à demi, un œil vide, terne, affreux, et qui ne regardait pas.\par
J’ai ôté la flamme et j’ai dit :\par
— Ah ! enfin ! répondras-tu, vieille sorcière ? Qui es-tu ?\par
L’œil s’est refermé comme de lui-même.\par
— Pour le coup, c’est trop fort, ont dit les autres. Encore la bougie ! encore ! il faudra bien qu’elle parle.\par
 J’ai replacé la lumière sous le menton de la vieille.\par
Alors, elle a ouvert ses deux yeux lentement, nous a regardés tous les uns après les autres, puis, se baissant brusquement, a soufflé la bougie avec un souffle glacé. Au même moment j’ai senti trois dents aiguës s’imprimer sur ma main dans les ténèbres.\par
Je me suis réveillé, frissonnant et baigné d’une sueur froide.\par
Le bon aumônier était assis au pied de mon lit, et lisait des prières.\par
— Ai-je dormi longtemps ? lui ai-je demandé.\par
— Mon fils, m’a-t-il dit, vous avez dormi une heure. On vous a amené votre enfant. Elle est là dans la pièce voisine qui vous attend. Je n’ai pas voulu qu’on vous éveillât.\par
— Oh ! ai-je crié. Ma fille ! qu’on m’amène ma fille !
 \section[{XLIII}]{XLIII}\phantomsection
\label{ch43}\renewcommand{\leftmark}{XLIII}

\noindent Elle est fraîche, elle est rose, elle a de grands yeux, elle est belle !\par
On lui a mis une petite robe qui lui va bien.\par
Je l’ai prise, je l’ai enlevée dans mes bras, je l’ai assise sur mes genoux, je l’ai baisée sur ses cheveux.\par
Pourquoi pas avec sa mère ? — Sa mère est malade, sa grand’mère aussi. C’est bien.\par
Elle me regardait d’un air étonné. Caressée, embrassée, dévorée de baisers et se laissant faire, mais jetant de temps en temps un coup d’œil inquiet sur sa bonne, qui pleurait dans le coin.\par
Enfin j’ai pu parler.\par
— Marie ! ai-je dit, ma petite Marie !\par
Je la serrais violemment contre ma poitrine enflée de sanglots. Elle a poussé un petit cri.\par
— Oh ! vous me faites du mal, monsieur, m’a-t-elle dit.\par
\emph{Monsieur !} il y a bientôt un an qu’elle ne m’a vu, la pauvre enfant. Elle m’a oublié, visage, parole, accent ;  et puis, qui me reconnaîtrait avec cette barbe, ces habits et cette pâleur ? Quoi ! déjà effacé de cette mémoire, la seule où j’eusse voulu vivre ! Quoi ! déjà plus père ! être condamné à ne plus entendre ce mot, ce mot de la langue des enfants, si doux qu’il ne peut rester dans celle des hommes : \emph{papa !}\par
Et pourtant l’entendre de cette bouche, encore une fois, une seule fois, voilà tout ce que j’eusse demandé pour les quarante ans de vie qu’on me prend.\par
— Écoute, Marie, lui ai-je dit en joignant ses deux petites mains dans les miennes, est-ce que tu ne me connais point ?\par
Elle m’a regardé avec ses beaux yeux, et a répondu :\par
— Ah bien non !\par
— Regarde bien, ai-je répété. Comment, tu ne sais pas qui je suis ?\par
— Si, a-t-elle dit. Un monsieur.\par
Hélas ! n’aimer ardemment qu’un seul être au monde, l’aimer avec tout son amour, et l’avoir devant soi, qui vous voit et vous regarde, vous parle et vous répond, et ne vous connaît pas ! Ne vouloir de consolation que de lui, et qu’il soit le seul qui ne sache pas qu’il vous en faut parce que vous allez mourir !\par
— Marie, ai-je repris, as-tu un papa ?\par
— Oui, monsieur, a dit l’enfant.\par
— Eh bien, où est-il ?\par
Elle a levé ses grands yeux étonnés.\par
— Ah ! vous ne savez donc pas ? il est mort.\par
Puis elle a crié ; j’avais failli la laisser tomber.\par
 — Mort ! disais-je. Marie, sais-tu ce que c’est qu’être mort ?\par
— Oui, monsieur, a-t-elle répondu. Il est dans la terre et dans le ciel.\par
Elle a continué d’elle-même :\par
— Je prie le bon Dieu pour lui matin et soir sur les genoux de maman.\par
Je l’ai baisée au front.\par
— Marie, dis-moi ta prière.\par
— Je ne peux pas, monsieur. Une prière, cela ne se dit pas dans le jour. Venez ce soir dans ma maison ; je la dirai.\par
C’était assez de cela. Je l’ai interrompue.\par
— Marie, c’est moi qui suis ton papa.\par
— Ah ! m’a-t-elle dit.\par
J’ai ajouté : — Veux-tu que je sois ton papa ?\par
L’enfant s’est détournée.\par
— Non, mon papa était bien plus beau.\par
Je l’ai couverte de baisers et de larmes. Elle a cherché à se dégager de mes bras en criant :\par
— Vous me faites mal avec votre barbe.\par
Alors, je l’ai replacée sur mes genoux, en la couvant des yeux, et puis je l’ai questionnée.\par
— Marie, sais-tu lire ?\par
— Oui, a-t-elle répondu. Je sais bien lire. Maman me fait lire mes lettres.\par
— Voyons, lis un peu, lui ai-je dit en lui montrant un papier qu’elle tenait chiffonné dans une de ses petites mains.\par
Elle a hoché sa jolie tête.\par
 — Ah bien ! je ne sais lire que des fables.\par
— Essaie toujours. Voyons, lis.\par
Elle a déployé le papier, et s’est mise à épeler avec son doigt :\par
— A, R, \emph{ar}, R, Ê, T, \emph{rêt,} {\scshape arrêt}…\par
Je lui ai arraché cela des mains. C’est ma sentence de mort qu’elle me lisait. Sa bonne avait eu le papier pour un sou. Il me coûtait plus cher, à moi.\par
Il n’y a pas de paroles pour ce que j’éprouvais. Ma violence l’avait effrayée ; elle pleurait presque. Tout à coup elle m’a dit :\par
— Rendez-moi donc mon papier ; tiens ! c’est pour jouer.\par
Je l’ai remise à sa bonne.\par
— Emportez-la.\par
Et je suis retombé sur ma chaise, sombre, désert, désespéré. A présent ils devraient venir ; je ne tiens plus à rien ; la dernière fibre de mon cœur est brisée. Je suis bon pour ce qu’ils vont faire.
 \section[{XLIV}]{XLIV}\phantomsection
\label{ch44}\renewcommand{\leftmark}{XLIV}

\noindent Le prêtre est bon, le geôlier aussi. Je crois qu’ils ont versé une larme quand j’ai dit qu’on m’emportât mon enfant.\par
C’est fait. Maintenant il faut que je me roidisse en moi-même, et que je pense fermement au bourreau, à la charrette, aux gendarmes, à la foule sur le pont, à la foule sur le quai, à la foule aux fenêtres, et à ce qu’il y aura exprès pour moi sur cette lugubre place de Grève, qui pourrait être pavée des têtes qu’elle a vu tomber.\par
Je crois que j’ai encore une heure pour m’habituer à tout cela.
 \section[{XLV}]{XLV}\phantomsection
\label{ch45}\renewcommand{\leftmark}{XLV}

\noindent Tout ce peuple rira, battra des mains, applaudira. Et parmi tous ces hommes, libres et inconnus des geôliers, qui courent pleins de joie à une exécution, dans cette foule de têtes qui couvrira la place, il y aura plus d’une tête prédestinée qui suivra la mienne tôt ou tard dans le panier rouge. Plus d’un qui y vient pour moi y viendra pour soi.\par
Pour ces êtres fatals il y a sur un certain point de la place de Grève un lieu fatal, un centre d’attraction, un piège. Ils tournent autour jusqu’à ce qu’ils y soient.
 \section[{LXVI}]{LXVI}\phantomsection
\label{ch46}\renewcommand{\leftmark}{LXVI}

\noindent Ma petite Marie ! — On l’a remmenée jouer ; elle regarde la foule par la portière du fiacre, et ne pense déjà plus à ce \emph{monsieur.}\par
Peut-être aurais-je encore le temps d’écrire quelques pages pour elle, afin qu’elle les lise un jour, et qu’elle pleure dans quinze ans pour aujourd’hui.\par
Oui, il faut qu’elle sache par moi mon histoire, et pourquoi le nom que je lui laisse est sanglant.
 \section[{XLVII}]{XLVII}\phantomsection
\label{ch47}\renewcommand{\leftmark}{XLVII}

\begin{center}MON HISTOIRE.\end{center}
\noindent \emph{Note de l’éditeur.} — On n’a pu encore retrouver les feuillets qui se rattachaient à celui-ci. Peut-être, comme ceux qui suivent semblent l’indiquer, le condamné n’a-t-il pas eu le temps de les écrire. Il était tard quand cette pensée lui est venue.
 \section[{XLVIII}]{XLVIII}\phantomsection
\label{ch48}\renewcommand{\leftmark}{XLVIII}


\dateline{D’une chambre de l’hôtel de ville.}
\noindent De l’hôtel de ville !.. — Ainsi j’y suis. Le trajet exécrable est fait. La place est là, et au-dessous de la fenêtre l’horrible peuple qui aboie, et m’attend, et rit.\par
J’ai eu beau me roidir, beau me crisper, le cœur m’a failli. Quand j’ai vu au-dessus des têtes ces deux bras rouges avec leur triangle noir au bout, dressés entre les deux lanternes du quai, le cœur m’a failli. J’ai demandé à faire une dernière déclaration. On m’a déposé ici, et l’on est allé chercher quelque procureur du roi. Je l’attends, c’est toujours cela de gagné.\par
Voici.\par
Trois heures sonnaient, on est venu m’avertir qu’il était temps. J’ai tremblé, comme si j’eusse pensé à autre chose depuis six heures, depuis six semaines, depuis six mois. Cela m’a fait l’effet de quelque chose d’inattendu.\par
Ils m’ont fait traverser leurs corridors et descendre  leurs escaliers. Ils m’ont poussé entre deux guichets du rez-de-chaussée, salle sombre, étroite, voûtée, à peine éclairée d’un jour de pluie et de brouillard. Une chaise était au milieu. Ils m’ont dit de m’asseoir ; je me suis assis.\par
Il y avait près de la porte et le long des murs quelques personnes debout, outre le prêtre et les gendarmes, et il y avait aussi trois hommes.\par
Le premier, le plus grand, le plus vieux, était gras et avait la face rouge. Il portait une redingote et un chapeau à trois cornes déformé. C’était lui.\par
C’était le bourreau, le valet de la guillotine. Les deux autres étaient ses valets, à lui.\par
A peine assis, les deux autres se sont approchés de moi, par derrière, comme des chats ; puis tout à coup j’ai senti un froid d’acier dans mes cheveux, et les ciseaux ont grincé à mes oreilles.\par
Mes cheveux, coupés au hasard, tombaient par mèches sur mes épaules, et l’homme au chapeau à trois cornes les époussetait doucement avec sa grosse main.\par
Autour, on parlait à voix basse.\par
Il y avait un grand bruit au dehors, comme un frémissement qui ondulait dans l’air. J’ai cru d’abord que c’était la rivière ; mais, à des rires qui éclataient, j’ai reconnu que c’était la foule.\par
Un jeune homme, près de la fenêtre, qui écrivait, avec un crayon, sur un portefeuille, a demandé à un des guichetiers comment s’appelait ce qu’on faisait là.\par
— La toilette du condamné, a répondu l’autre.\par
J’ai compris que cela serait demain dans le journal.\par
 Tout à coup l’un des valets m’a enlevé ma veste, et l’autre a pris mes deux mains qui pendaient, les a ramenées derrière mon dos, et j’ai senti les nœuds d’une corde se rouler lentement autour de mes poignets rapprochés. En même temps, l’autre détachait ma cravate. Ma chemise de batiste, seul lambeau qui me restât du moi d’autrefois, l’a fait en quelque sorte hésiter un moment ; puis il s’est mis à en couper le col.\par
A cette précaution horrible, au saisissement de l’acier qui touchait mon cou, mes coudes ont tressailli, et j’ai laissé échapper un rugissement étouffé. La main de l’exécuteur a tremblé.\par
— Monsieur, m’a-t-il dit, pardon ! Est-ce que je vous ai fait mal ?\par
Ces bourreaux sont des hommes très-doux.\par
La foule hurlait plus haut au dehors.\par
Le gros homme au visage bourgeonné m’a offert à respirer un mouchoir imbibé de vinaigre.\par
— Merci, lui ai-je dit de la voix la plus forte que j’ai pu, c’est inutile ; je me trouve bien.\par
Alors l’un d’eux s’est baissé et m’a lié les deux pieds, au moyen d’une corde fine et lâche, qui ne me laissait à faire que de petits pas. Cette corde est venue se rattacher à celle de mes mains.\par
Puis le gros homme a jeté la veste sur mon dos, et a noué les manches ensemble sous mon menton. Ce qu’il y avait à faire là était fait.\par
Alors le prêtre s’est approché avec son crucifix.\par
— Allons, mon fils, m’a-t-il dit.\par
 Les valets m’ont pris sous les aisselles. Je me suis levé, j’ai marché. Mes pas étaient mous et fléchissaient comme si j’avais eu deux genoux à chaque jambe.\par
En ce moment la porte extérieure s’est ouverte à deux battants. Une clameur furieuse et l’air froid et la lumière blanche ont fait irruption jusqu’à moi dans l’ombre. Du fond du sombre guichet, j’ai vu brusquement tout à la fois, à travers la pluie, les mille têtes hurlantes du peuple entassées pêle-mêle sur la rampe du grand escalier du Palais ; à droite, de plain-pied avec le seuil, un rang de chevaux de gendarmes, dont la porte basse ne me découvrait que les pieds de devant et les poitrails ; en face, un détachement de soldats en bataille ; à gauche, l’arrière d’une charrette, auquel s’appuyait une roide échelle. Tableau hideux, bien encadré dans une porte de prison.\par
C’est pour ce moment redouté que j’avais gardé mon courage. J’ai fait trois pas, et j’ai paru sur le seuil du guichet.\par
— Le voilà ! le voilà ! a crié la foule. Il sort ! enfin !\par
Et les plus près de moi battaient des mains. Si fort qu’on aime un roi, ce serait moins de fête.\par
C’était une charrette ordinaire, avec un cheval étique, et un charretier en sarrau bleu à dessins rouges, comme ceux des maraîchers des environs de Bicêtre.\par
Le gros homme en chapeau à trois cornes est monté le premier.\par
— Bonjour, monsieur Samson ! criaient des enfants pendus à des grilles.\par
Un valet l’a suivi.\par
 — Bravo, Mardi ! ont crié de nouveau les enfants.\par
Ils se sont assis tous deux sur la banquette de devant.\par
C’était mon tour. J’ai monté d’une allure assez ferme.\par
— Il va bien ! a dit une femme à côté des gendarmes.\par
Cet atroce éloge m’a donné du courage. Le prêtre est venu se placer auprès de moi. On m’avait assis sur la banquette de derrière, le dos tourné au cheval. J’ai frémi de cette dernière attention.\par
Ils mettent de l’humanité là dedans.\par
J’ai voulu regarder autour de moi. Gendarmes devant, gendarmes derrière ; puis de la foule, de la foule, et de la foule ; une mer de têtes sur la place.\par
Un piquet de gendarmerie à cheval m’attendait à la porte de la grille du Palais.\par
L’officier a donné l’ordre. La charrette et son cortège se sont mis en mouvement, comme poussés en avant par un hurlement de la populace.\par
On a franchi la grille. Au moment où la charrette a tourné vers le Pont-au-Change, la place a éclaté en bruit, du pavé aux toits, et les ponts et les quais ont répondu à faire un tremblement de terre.\par
C’est là que le piquet qui attendait s’est rallié à l’escorte.\par
— Chapeaux bas ! chapeaux bas ! criaient mille bouches ensemble. — Comme pour le roi.\par
Alors j’ai ri horriblement aussi, moi, et j’ai dit au prêtre :\par
 — Eux les chapeaux, moi la tête.\par
On allait au pas.\par
Le quai aux Fleurs embaumait ; c’est jour de marché. Les marchandes ont quitté leurs bouquets pour moi.\par
Vis-à-vis, un peu avant la tour carrée qui fait le coin du Palais, il y a des cabarets, dont les entresols étaient pleins de spectateurs heureux de leurs belles places, surtout des femmes. La journée doit être bonne pour les cabaretiers.\par
On louait des tables, des chaises, des échafaudages, des charrettes. Tout pliait de spectateurs. Des marchands de sang humain criaient à tue-tête :\par
— Qui veut des places ?\par
Une rage m’a pris contre ce peuple. J’ai eu envie de leur crier :\par
— Qui veut la mienne ?\par
Cependant la charrette avançait. A chaque pas qu’elle faisait, la foule se démolissait derrière elle, et je la voyais de mes yeux égarés qui s’allait reformer plus loin sur d’autres points de mon passage.\par
En entrant sur le Pont-au-Change, j’ai par hasard jeté les yeux à ma droite en arrière. Mon regard s’est arrêté sur l’autre quai, au-dessus des maisons, à une tour noire, isolée, hérissée de sculptures, au sommet de laquelle je voyais deux monstres de pierre assis de profil. Je ne sais pourquoi j’ai demandé au prêtre ce que c’était que cette tour.\par
— Saint-Jacques-la-Boucherie, a répondu le bourreau.\par
 J’ignore comment cela se faisait ; dans la brume, et malgré la pluie fine et blanche qui rayait l’air comme un réseau de fils d’araignée, rien de ce qui se passait autour de moi ne m’a échappé. Chacun de ces détails m’apportait sa torture. Les mots manquent aux émotions.\par
Vers le milieu de ce Pont-au-Change, si large et si encombré que nous cheminions à grand’peine, l’horreur m’a pris violemment. J’ai craint de défaillir, dernière vanité ! Alors je me suis étourdi moi-même pour être aveugle et pour être sourd à tout, excepté au prêtre, dont j’entendais à peine les paroles, entrecoupées de rumeurs.\par
J’ai pris le crucifix et je l’ai baisé.\par
— Ayez pitié de moi, ai-je dit, ô mon Dieu ! — Et j’ai tâché de m’abîmer dans cette pensée.\par
Mais chaque cahot de la dure charrette me secouait. Puis tout à coup je me suis senti un grand froid. La pluie avait traversé mes vêtements, et mouillait la peau de ma tête à travers mes cheveux coupés et courts.\par
— Vous tremblez de froid, mon fils ? m’a demandé le prêtre.\par
— Oui, ai-je répondu.\par
Hélas ! pas seulement de froid.\par
Au détour du pont, des femmes m’ont plaint d’être si jeune.\par
Nous avons pris le fatal quai. Je commençais à ne plus voir, à ne plus entendre. Toutes ces voix, toutes ces têtes aux fenêtres, aux portes, aux grilles des  boutiques, aux branches des lanternes ; ces spectateurs avides et cruels ; cette foule où tous me connaissent et où je ne connais personne ; cette route pavée et murée de visages humains… J’étais ivre, stupide, insensé. C’est une chose insupportable que le poids de tant de regards appuyés sur vous.\par
Je vacillais donc sur le banc, ne prêtant même plus d’attention au prêtre et au crucifix.\par
Dans le tumulte qui m’enveloppait, je ne distinguais plus les cris de pitié des cris de joie, les rires des plaintes, les voix du bruit ; tout cela était une rumeur qui résonnait dans ma tête comme dans un écho de cuivre.\par
Mes yeux lisaient machinalement les enseignes des boutiques.\par
Une fois l’étrange curiosité me prit de tourner la tête et de regarder vers quoi j’avançais. C’était une dernière bravade de l’intelligence. Mais le corps ne voulut pas ; ma nuque resta paralysée et d’avance comme morte.\par
J’entrevis seulement de côté, à ma gauche, au delà de la rivière, la tour de Notre-Dame, qui, vue de là, cache l’autre. C’est celle où est le drapeau. Il y avait beaucoup de monde, et qui devait bien voir.\par
Et la charrette allait, allait, et les boutiques passaient, et les enseignes se succédaient, écrites, peintes, dorées, et la populace riait et trépignait dans la boue, et je me laissais aller, comme à leurs rêves ceux qui sont endormis.\par
Tout à coup la série des boutiques qui occupait  mes yeux s’est coupée à l’angle d’une place ; la voix de la foule est devenue plus vaste, plus glapissante, plus joyeuse encore ; la charrette s’est arrêtée subitement, et j’ai failli tomber la face sur les planches. Le prêtre m’a soutenu. — Courage ! a-t-il murmuré. — Alors on a apporté une échelle à l’arrière de la charrette ; il m’a donné le bras, je suis descendu, puis j’ai fait un pas, puis je me suis retourné pour en faire un autre, et je n’ai pu. Entre les deux lanternes du quai j’avais vu une chose sinistre.\par
Oh ! c’était la réalité !\par
Je me suis arrêté, comme chancelant déjà du coup.\par
— J’ai une dernière déclaration à faire ! ai-je crié faiblement.\par
On m’a monté ici.\par
J’ai demandé qu’on me laissât écrire mes dernières volontés. Ils m’ont délié les mains, mais la corde est ici, toute prête, et le reste est en bas.
 \section[{XLIX}]{XLIX}\phantomsection
\label{ch49}\renewcommand{\leftmark}{XLIX}

\noindent Un juge, un commissaire, un magistrat, je ne sais de quelle espèce, vient de venir. Je lui ai demandé ma grâce en joignant les deux mains et en me traînant sur les deux genoux. Il m’a répondu, en souriant fatalement, si c’est là tout ce que j’avais à lui dire.\par
— Ma grâce ! ma grâce ! ai-je répété, ou, par pitié, cinq minutes encore !\par
Qui sait ? elle viendra peut-être ! Cela est si horrible, à mon âge, de mourir ainsi ! Des grâces qui arrivent au dernier moment, on l’a vu souvent. Et à qui fera-t-on grâce, monsieur, si ce n’est à moi ?\par
Cet exécrable bourreau ! il s’est approché du juge pour lui dire que l’exécution devait être faite à une certaine heure, que cette heure approchait, qu’il était responsable, que d’ailleurs il pleut et que cela risque de se rouiller.\par
— Eh, par pitié ! une minute pour attendre ma grâce ! ou je me défends, je mords !\par
 Le juge et le bourreau sont sortis. Je suis seul. — Seul avec deux gendarmes.\par
Oh ! l’horrible peuple avec ses cris d’hyène ! — Qui sait si je ne lui échapperai pas ? si je ne serai pas sauvé ? si ma grâce ?… Il est impossible qu’on ne me fasse pas grâce !\par
Ah ! les misérables ! il me semble qu’on monte l’escalier…\par
\bigbreak

\begin{center}
\noindent \centerline{{\scshape quatre heures}}\par
\end{center}

  \section[{Notes du dernier jour d’un condamné}]{Notes du dernier jour d’un condamné}\phantomsection
\label{notes}\renewcommand{\leftmark}{Notes du dernier jour d’un condamné}


\labelblock{1829}

\noindent Nous donnons ci-jointe, pour les personnes curieuses de cette sorte de littérature, la chanson d’argot avec l’explication en regard, d’après une copie que nous avons trouvée dans les papiers du condamné, et dont ce fac-simile reproduit tout, orthographe et écriture. La signification des mots était écrite de la main du condamné ; il y a aussi dans le dernier couplet deux vers intercalés qui semblent de son écriture ; le reste de la complainte est d’une autre main. Il est probable que, frappé de cette chanson, mais ne se la rappelant qu’imparfaitement, il avait cherché à se la procurer, et que copie lui en avait été donnée par quelque calligraphe de la geôle.\par
La seule chose que ce fac-simile ne reproduise pas, c’est l’aspect du papier de la copie, qui est jaune, sordide et rompu à ses plis.\par
 
\labelblock{1881}

\noindent Le manuscrit original du \emph{Dernier Jour d’un condamné} porte en marge de la première page :\par
\bigbreak
\noindent Mardi 14 octobre 1828.\par
\bigbreak
\noindent Au bas de la dernière page :\par
\bigbreak
\noindent Nuit du 25 décembre 1828 au 26. — 3 heures du matin.
 


% at least one empty page at end (for booklet couv)
\ifbooklet
  \pagestyle{empty}
  \clearpage
  % 2 empty pages maybe needed for 4e cover
  \ifnum\modulo{\value{page}}{4}=0 \hbox{}\newpage\hbox{}\newpage\fi
  \ifnum\modulo{\value{page}}{4}=1 \hbox{}\newpage\hbox{}\newpage\fi


  \hbox{}\newpage
  \ifodd\value{page}\hbox{}\newpage\fi
  {\centering\color{rubric}\bfseries\noindent\large
    Hurlus ? Qu’est-ce.\par
    \bigskip
  }
  \noindent Des bouquinistes électroniques, pour du texte libre à participation libre,
  téléchargeable gratuitement sur \href{https://hurlus.fr}{\dotuline{hurlus.fr}}.\par
  \bigskip
  \noindent Cette brochure a été produite par des éditeurs bénévoles.
  Elle n’est pas faîte pour être possédée, mais pour être lue, et puis donnée.
  Que circule le texte !
  En page de garde, on peut ajouter une date, un lieu, un nom ; pour suivre le voyage des idées.
  \par

  Ce texte a été choisi parce qu’une personne l’a aimé,
  ou haï, elle a en tous cas pensé qu’il partipait à la formation de notre présent ;
  sans le souci de plaire, vendre, ou militer pour une cause.
  \par

  L’édition électronique est soigneuse, tant sur la technique
  que sur l’établissement du texte ; mais sans aucune prétention scolaire, au contraire.
  Le but est de s’adresser à tous, sans distinction de science ou de diplôme.
  Au plus direct ! (possible)
  \par

  Cet exemplaire en papier a été tiré sur une imprimante personnelle
   ou une photocopieuse. Tout le monde peut le faire.
  Il suffit de
  télécharger un fichier sur \href{https://hurlus.fr}{\dotuline{hurlus.fr}},
  d’imprimer, et agrafer ; puis de lire et donner.\par

  \bigskip

  \noindent PS : Les hurlus furent aussi des rebelles protestants qui cassaient les statues dans les églises catholiques. En 1566 démarra la révolte des gueux dans le pays de Lille. L’insurrection enflamma la région jusqu’à Anvers où les gueux de mer bloquèrent les bateaux espagnols.
  Ce fut une rare guerre de libération dont naquit un pays toujours libre : les Pays-Bas.
  En plat pays francophone, par contre, restèrent des bandes de huguenots, les hurlus, progressivement réprimés par la très catholique Espagne.
  Cette mémoire d’une défaite est éteinte, rallumons-la. Sortons les livres du culte universitaire, cherchons les idoles de l’époque, pour les briser.
\fi

\ifdev % autotext in dev mode
\fontname\font — \textsc{Les règles du jeu}\par
(\hyperref[utopie]{\underline{Lien}})\par
\noindent \initialiv{A}{lors là}\blindtext\par
\noindent \initialiv{À}{ la bonheur des dames}\blindtext\par
\noindent \initialiv{É}{tonnez-le}\blindtext\par
\noindent \initialiv{Q}{ualitativement}\blindtext\par
\noindent \initialiv{V}{aloriser}\blindtext\par
\Blindtext
\phantomsection
\label{utopie}
\Blinddocument
\fi
\end{document}
